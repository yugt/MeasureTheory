\frfilename{mt448.tex}
\versiondate{12.4.13}
\copyrightdate{1997}

\def\Gse{$G$-\vthsp$\sigma$-\vthsp equidecomposable}
\def\IMPLY#1#2{{\bf (#1)$\Rightarrow$(#2)}}
\def\psG{\preccurlyeq^{\sigma}_G}
\def\psH{\preccurlyeq^{\sigma}_H}
\def\pstildeG{\preccurlyeq^{\sigma}_{\tilde G}}
\def\ptG{\preccurlyeq^{\tau}_G}
\def\low#1{\lfloor#1\rfloor}
\def\high#1{\lceil#1\rceil}

\def\chaptername{Topological groups}
\def\sectionname{Polish group actions}

\newsection{448}

I devote this section to two quite separate theorems.   The first is
an interesting result about measures on
Polish spaces which are invariant under actions of Polish groups.   In
contrast to \S441, we no longer have a strong general
existence theorem for such measures, but instead have a natural
necessary and sufficient condition in terms of countable dissections:
there is an invariant probability measure on $X$ if and only if there is no
countable dissection of $X$ into Borel sets which can be rearranged, by
the action of the group, into two copies of $X$ (448P).

\cmmnt{The principal ideas needed here have already been set out in
\S395, and in many of the proofs I allow myself to direct you to the
corresponding arguments there rather than write the formulae out again.
I do not think you need read through \S395 before embarking on this
section;  I will try to give sufficiently detailed references so that
you can take them one paragraph at a time, and many of the arguments
referred to are in any case elementary.   But unless you are already
familiar with this topic, you will need a copy of \S395 to hand to fully
follow the proofs below.
}%end of comment

The second theorem concerns the representation of group actions on measure
algebras in terms of group actions on measure spaces.   If we have a
locally compact Polish group $G$ (so that we do have Haar measures), and
a Borel measurable action of $G$ on the measure algebra of a Radon measure
$\mu$ on a Polish space $X$, then it can be represented by a Borel
measurable action of $G$ on $X$ (448S).   The proof is mostly descriptive
set theory based on \S\S423-424, but it also uses some interesting facts
about $L^0$ spaces (448Q-448R).

\leader{448A}{Definitions}\cmmnt{ (Compare 395A.)}   Let $\frak A$ be a
Dedekind $\sigma$-complete Boolean algebra, and $G$ a subgroup of
$\Aut\frak A$.   For $a$, $b\in\frak A$ I will say that an isomorphism
$\phi:\frak A_a\to\frak A_b$ between the corresponding principal ideals
belongs to the {\bf countably full local semigroup generated by $G$} if
there are a countable partition of unity $\langle a_i\rangle_{i\in I}$
in $\frak A_a$ and a family
$\langle\pi_i\rangle_{i\in I}$ in $G$ such that $\phi c=\pi_ic$
whenever $i\in I$ and $c\Bsubseteq a_i$.   If such an isomorphism exists
I will say that $a$ and $b$ are {\bf \Gse}.

I write $a\psG b$ to mean that there is a $b'\Bsubseteq b$ such
that $a$ and $b'$ are \Gse.

\cmmnt{As in \S395, }I will say that a function $f$ with domain
$\frak A$ is {\bf $G$-invariant} if $f(\pi a)=f(a)$ whenever
$a\in\frak A$ and $\pi\in G$.

\cmmnt{I have expressed these definitions, and most of the work below,
in terms of abstract Dedekind $\sigma$-complete Boolean algebras.   The
applications I have in mind for this section are to
$\sigma$-algebras of sets.   If you have already worked through \S395,
the version here should come very easily;  but even if you have not, I
think that the extra abstraction clarifies some of the ideas.
}

\leader{448B}{}\cmmnt{ I begin with results corresponding to
395B-395D;  there is hardly any difference, except that we must now
occasionally pause to check that a partition of unity is countable.

\medskip

\noindent}{\bf Lemma} Let $\frak A$ be a Dedekind $\sigma$-complete
Boolean algebra and $G$ a subgroup of $\Aut\frak A$.   Write
$G^*_{\sigma}$ for the countably full local semigroup generated by $G$.

(a) If $a$, $b\in\frak A$ and $\phi:\frak A_a\to\frak A_b$ belongs to
$G^*_{\sigma}$, then $\phi^{-1}:\frak A_b\to\frak A_a$ also belongs to
$G^*_{\sigma}$.

(b) Suppose that $a$, $b$, $a'$, $b'\in\frak A$ and that
$\phi:\frak A_a\to\frak A_{a'}$, $\psi:\frak A_{b}\to\frak A_{b'}$
belong to $G^*_{\sigma}$.   Then $\psi\phi\in G^*_{\sigma}$;
its domain is $\frak A_c$ where $c=\phi^{-1}(b\Bcap a')$, and its
set of values is $\frak A_{c'}$ where $c'=\psi(b\Bcap a')$.

(c) If $a$, $b\in\frak A$ and $\phi:\frak A_a\to\frak A_b$ belongs to
$G^*_{\sigma}$, then $\phi\restrp\frak A_c\in G^*_{\sigma}$ for any
$c\Bsubseteq a$.

(d) Suppose that $a$, $b\in\frak A$ and that
$\psi:\frak A_a\to\frak A_b$ is an isomorphism such that there are a
countable partition of unity $\langle a_i\rangle_{i\in I}$ in
$\frak A_a$ and a family
$\langle\phi_i\rangle_{i\in I}$ in $G^*_{\sigma}$ such that
$\psi c=\phi_ic$ whenever $i\in I$ and $c\Bsubseteq a_i$.   Then
$\psi\in G^*_{\sigma}$.

\proof{{\bf (a)} As 395Bb.

\medskip

{\bf (b)} As 395Bc.

\medskip

{\bf (c)} As 395Bd.

\medskip

{\bf (d)} For each $i\in I$, let $\family{j}{J(i)}{a_{ij}}$,
$\family{j}{J(i)}{\pi_{ij}}$ witness that $\phi_i\in G^*_{\sigma}$;
then $\langle a_i\Bcap a_{ij}\rangle_{i\in I,j\in J(i)}$ and $\langle
\pi_{ij}\rangle_{i\in I,j\in J(i)}$ witness that $\psi\in G^*_{\sigma}$.
}%end of proof of 448B

\leader{448C}{Lemma} Let $\frak A$ be a Dedekind $\sigma$-complete
Boolean algebra and $G$ a subgroup of $\Aut\frak A$.   Write
$G^*_{\sigma}$ for the countably full local semigroup generated by $G$.

(a) For $a$, $b\in\frak A$, $a\psG b$ iff there is a
$\phi\in G^*_{\sigma}$ such that $a\in\dom\phi$ and
$\phi a\Bsubseteq b$.

(b)(i) $\psG$ is transitive and reflexive;

\quad(ii) if $a\psG b$ and $b\psG a$ then $a$ and $b$ are \Gse.

(c) $G$-$\sigma$-equidecomposability is an equivalence relation on
$\frak A$.

(d) If $\langle a_i\rangle_{i\in I}$ and $\langle b_i\rangle_{i\in I}$
are countable families in $\frak A$, of which
$\langle b_i\rangle_{i\in I}$ is disjoint, and $a_i\psG b_i$ for every
$i\in I$, then $\sup_{i\in I}a_i\psG\sup_{i\in I}b_i$.

\proof{ The arguments of 395C apply unchanged, calling on 448B in place
of 395B.
}%end of proof of 448C

\vleader{72pt}{448D}{Theorem} Let $\frak A$ be a Dedekind $\sigma$-complete
Boolean algebra and $G$ a subgroup of $\Aut\frak A$.   Then the
following are equiveridical:

(i) there is an $a\ne 1$ such that $a$ is \Gse\
with $1$;

(ii) there is a disjoint sequence $\sequencen{a_n}$ of non-zero elements
of $\frak A$ which are all \Gse;

(iii) there are non-zero \Gse\ $a$, $b$, $c\in\frak A$ such that
$a\Bcap b=0$ and $a\Bcup b\Bsubseteq c$;

(iv) there are \Gse\ $a$, $b\in\frak A$ such
that $a\Bsubset b$.

\proof{ As 395D.
}%end of proof of 448D

\leader{448E}{Definition}\cmmnt{ (Compare 395E.)}   Let $\frak A$ be a
Dedekind $\sigma$-complete Boolean algebra and $G$ a subgroup of
$\Aut\frak A$.   I will say that  $G$ is {\bf countably non-paradoxical}
if\cmmnt{ the statements of 448D are false;  that
is, if} one of the following equiveridical statements is true:

\inset{(i)
if $a$ is \Gse\ with $1$ then $a=1$;}

\inset{(ii) there is no disjoint sequence $\sequencen{a_n}$ of
non-zero elements of $\frak A$ which are all \Gse;}

\inset{(iii) there are no non-zero \Gse\ $a$, $b$, $c\in\frak A$ such
that $a\Bcap b=0$ and $a\Bcup b\Bsubseteq c$;}

\inset{(iv) if $a$, $b\in\frak A$ are \Gse\ and
$a\Bsubseteq b$ then $a=b$.}

%add: there are $a\in\frak A$, $\phi$ in the
%countably full subgroup generated
%by $\pi$ which is not recurrent on $a$.


\leader{448F}{}\cmmnt{ We now come to one of the points where we need
to find a new path because we are looking at algebras which need not be
Dedekind complete.   Provided the original group $G$ is {\it countable},
we can still follow the general line of \S395, as follows.

\medskip

\noindent}{\bf Lemma}\cmmnt{ (Compare 395G.)}   Let $\frak A$ be a
Dedekind $\sigma$-complete Boolean algebra and $G$ a countable subgroup
of $\Aut\frak A$.   Let $\frak C$ be the fixed-point subalgebra of $G$.

(a) For any $a\in\frak A$, $\upr(a,\frak C)$\cmmnt{ (313S)} is
defined, and is given by the formula

\Centerline{$\upr(a,\frak C)=\sup\{\pi a:\pi\in G\}$.}

(b) If $G^*_{\sigma}$ is the countably full local semigroup generated by
$G$, then $\phi(c\Bcap a)=c\Bcap\phi a$ whenever
$\phi\in G^*_{\sigma}$, $a\in\dom\phi$ and $c\in\frak C$.

(c) $\upr(\phi a,\frak C)=\upr(a,\frak C)$ whenever
$\phi\in G^*_{\sigma}$ and $a\in\dom\phi$;\cmmnt{  consequently,}
$\upr(a,\frak C)\Bsubseteq\upr(b,\frak C)$ whenever
$a\psG b$.

(d) If $a\psG b$ and $c\in\frak C$ then $a\Bcap c\psG b\Bcap c$.
So $a\Bcap c$ and $b\Bcap c$ are \Gse\ whenever $a$ and $b$ are
\Gse\ and $c\in\frak C$.

\proof{{\bf (a)} As remarked in 395Ga, $\frak C$ is order-closed.
Because $G$ is countable and $\frak A$ is Dedekind
$\sigma$-complete, $c^*=\sup\{\pi a:\pi\in G\}$ is defined in $\frak A$
If $\phi\in G$, then

\Centerline{$\phi c^*=\sup\{\phi\pi a:\pi\in G\}\Bsubseteq c^*$}

\noindent because $\phi$ is order-continuous and $\phi\pi\in G$ for every
$\pi\in G$.   Similarly $\phi^{-1}c^*\Bsubseteq c^*$ and
$c^*\Bsubseteq\phi c^*$.   Thus $\phi c^*=c^*$;  as $\phi$ is arbitrary,
$c^*\in\frak C$.

If $c\in\frak C$, then

$$\eqalign{a\Bsubseteq c
&\iff\pi a\Bsubseteq\pi c\text{ for every }\pi\in G\cr
&\iff\pi a\Bsubseteq c\text{ for every }\pi\in G\cr
&\iff c^*\Bsubseteq c,\cr}$$

\noindent so $c^*=\inf\{c:a\Bsubseteq c\in\frak C\}$, taking the infimum
in $\frak C$, as required in the definition of $\upr(a,\frak C)$.

\medskip

{\bf (b)} Suppose that $\familyiI{a_i}$, $\familyiI{\pi_i}$ witness that
$\phi\in G^*_{\sigma}$.   Then

\Centerline{$\phi(a\Bcap c)=\sup_{i\in I}\pi_i(a_i\Bcap a\Bcap c)
=\sup_{i\in I}\pi_i(a_i\Bcap a)\Bcap c=c\Bcap\phi a$.}

\medskip

{\bf (c)} For $c\in\frak C$,

\Centerline{$a\Bsubseteq c
\iff a\Bcap c=a
\iff\phi(a\Bcap c)=\phi a
\iff c\Bcap\phi a=\phi a
\iff\phi a\Bsubseteq c$.}

\medskip

{\bf (d)} There is a $\phi\in G^*_{\sigma}$ such that
$\phi a\Bsubseteq b$;  now

\Centerline{$a\Bcap c\psG\phi(a\Bcap c)=c\Bcap\phi a
\Bsubseteq b\Bcap c$.}
}%end of proof of 448F

\leader{448G}{}\cmmnt{ With this support, we can now continue with the
ideas of 395H-395L, adding at each step the hypothesis `$G$ is
countable' to compensate for the weakening of the hypotheses `$\frak A$
is Dedekind complete, $G$ is fully non-paradoxical' to `$\frak A$ is
Dedekind $\sigma$-complete, $G$ is countably non-paradoxical'.

\medskip

\noindent}{\bf Lemma}\cmmnt{ (Compare 395H.)}   Let $\frak A$ be a
Dedekind $\sigma$-complete Boolean algebra and $G$ a countable countably
non-paradoxical subgroup of $\Aut\frak A$.   Write $\frak C$ for the
fixed-point subalgebra of $G$.   Take any $a$, $b\in\frak A$.   Then
$c_0=\sup\{c:c\in\frak C,\,a\Bcap c\psG b\}$ is defined in $\frak A$ and
belongs to $\frak C$;  $a\Bcap c_0\psG b$ and $b\Bsetminus c_0\psG a$.

\proof{ Let $\sequencen{\pi_n}$ be a sequence running over $G$.   Define
$\sequencen{a_n}$, $\sequencen{b_n}$ inductively,  setting

\Centerline{$a_n=(a\Bsetminus\sup_{i<n}a_i)
   \Bcap\pi_n^{-1}(b\Bsetminus\sup_{i<n}b_i)$,
\quad$b_n=\pi_na_n$.}

\noindent Then $\sequencen{a_n}$ is a disjoint sequence in $\frak A_a$
and $\sequencen{b_n}$ is a
disjoint sequence in $\frak A_b$, and $\sup_{n\in\Bbb N}a_n$ is \Gse\
with $\sup_{n\in\Bbb N}b_n$.
Set

\Centerline{$a'=a\Bsetminus\sup_{n\in\Bbb N}a_n$,
\quad$b'=b\Bsetminus\sup_{n\in\Bbb N}b_n$,
\quad$c_0=1\Bsetminus\upr(a',\frak C)\Bsubseteq\sup_{n\in\Bbb N}a_n$.}

\noindent Then

\Centerline{$a\Bcap c_0\Bsubseteq\sup_{n\in\Bbb N}a_n\psG b$.}

Now $b'\Bsubseteq c_0$.   \Prf\Quer\ Otherwise, because
$c_0=1\Bsetminus\sup_{n\in\Bbb N}\pi_na'$, there must be an
$n\in\Bbb N$ such that $b'\Bcap\pi_na'\ne 0$.   But in this case
$d=a'\Bcap\pi_n^{-1}b'\ne 0$, and we have

\Centerline{$d\Bsubseteq(a\Bsetminus\sup_{i<n}a_i)
  \Bcap\pi_n^{-1}(b\Bsetminus\sup_{i<n}b_i)$,}

\noindent so that $d\Bsubseteq a_n$, which is absurd.\ \Bang\QeD\
Consequently

\Centerline{$b\Bsetminus c_0
\Bsubseteq b\Bsetminus b'=\sup_{n\in\Bbb N}b_n\psG a$.}

Now take any $c\in\frak C$ such that $a\Bcap c\psG b$, and consider
$c'=c\Bsetminus c_0$.   Then $b'\Bcap c'=0$, that is, $b\Bcap
c'=\sup_{n\in\Bbb N}b_n\Bcap c'$, which is \Gse\ with
$\sup_{n\in\Bbb N}a_n\Bcap c'=(a\Bsetminus a')\Bcap c'$.   But now

\Centerline{$a\Bcap c'=a\Bcap c_0\Bcap c'
\psG b\Bcap c'\psG(a\Bcap c')\Bsetminus(a'\Bcap c')$;}

\noindent because $G$ is countably non-paradoxical, $a'\Bcap c'$ must be
$0$, that is, $c'\Bsubseteq c_0$ and $c\Bsubseteq c_0$.
So $c_0$ has the required properties.
}%end of proof of 448G

\leader{448H}{Lemma}\cmmnt{ (Compare 395I.)}   Let $\frak A$ be a
Dedekind $\sigma$-complete Boolean
algebra, not $\{0\}$, and $G$ a countable countably non-paradoxical
subgroup of $\Aut\frak A$.   Let $\frak C$ be the fixed-point subalgebra
of $G$.   Suppose that $a$, $b\in\frak A$ and that $\upr(a,\frak C)=1$.
Then there are non-negative $u$, $v\in L^0(\frak C)$ such that

$$\eqalign{\Bvalue{u\ge n}
&=\max\{c:c\in\frak C,
\text{ there is a disjoint family }\langle d_i\rangle_{i<n}\cr
&\qquad\qquad\qquad\qquad
\text{such that }a\Bcap c\psG d_i\Bsubseteq b\text{ for every }i<n\},\cr}$$

$$\eqalign{\Bvalue{v\le n}
&=\max\{c:c\in\frak C,
\text{ there is a family }\langle d_i\rangle_{i<n}\cr
&\qquad\qquad\qquad\qquad\text{ such that }d_i\psG a
\text{ for every }i<n\text{ and }b\Bcap c\Bsubseteq\sup_{i<n}d_i\}\cr}$$

\noindent for every $n\in\Bbb N$.   Moreover, we have

\quad(i) $\Bvalue{u\in\Bbb N}=\Bvalue{v\in\Bbb N}=1$,

\quad(ii) $\Bvalue{v>0}=\upr(b,\frak C)$,

\quad(iii) $u\le v\le u+\chi 1$.

\proof{ The argument of 395I applies unchanged, except that every $\ptG$
must be replaced with a $\psG$, and we use 448F and 448G in place of
395G and 395H.   $\frak C$ is Dedekind $\sigma$-complete because it is
order-closed in the Dedekind $\sigma$-complete algebra $\frak A$
(314Eb).
}%end of proof of 448H

\leader{448I}{Notation}\cmmnt{ (Compare 395J.)}   In the context of
448H, I will write $\low{b:a}$ for $u$, $\high{b:a}$ for $v$.

\leader{448J}{Lemma}\cmmnt{ (Compare 395K-395L.)}   Let $\frak A$ be a
Dedekind $\sigma$-complete Boolean algebra, not $\{0\}$, and $G$ a
countable countably non-paradoxical subgroup of
$\Aut\frak A$ with fixed-point subalgebra $\frak C$.   Suppose that $a$,
$a_1$, $a_2$, $b$, $b_1$, $b_2\in\frak A$ and that

\Centerline{$\upr(a,\frak C)=\upr(a_1,\frak C)=\upr(a_2,\frak C)=1$.}

\noindent Then

(a) $\low{0:a}=\high{0:a}=0$ and $\low{1:a}\ge\chi 1$.

(b) If $b_1\psG b_2$ then $\low{b_1:a}\le\low{b_2:a}$ and
$\high{b_1:a}\le\high{b_2:a}$.

(c) $\high{b_1\Bcup b_2:a}\le\high{b_1:a}+\high{b_2:a}$.

(d) If $b_1\Bcap b_2=0$, then
$\low{b_1:a}+\low{b_2:a}\le\low{b_1\Bcup b_2:a}$.

(e) If $c\in\frak C$ is such that $a\Bcap c$ is a
relative atom over $\frak C$, then
$c\Bsubseteq\Bvalue{\high{b:a}-\low{b:a}=0}$.

(f) $\low{b:a_2}\ge\low{b:a_1}\times\low{a_1:a_2}$,
$\high{b:a_2}\le\high{b:a_1}\times\high{a_1:a_2}$.

\proof{ As in 395K-395L.
}%end of proof of 448J.

\leader{448K}{}\cmmnt{ For the result corresponding to 395Mb, we again
need to find a new approach;  I deal with it by adding a further
hypothesis to the list which has already accreted.

\medskip

\noindent}{\bf Definition}  Let $\frak A$ be a Dedekind
$\sigma$-complete Boolean algebra and $G$ a
countable subgroup of $\Aut\frak A$
with fixed-point subalgebra $\frak C$.   I will
say that $G$ has the {\bf $\sigma$-refinement property} if for every
$a\in\frak A$ there is a $d\Bsubseteq a$ such that $d\psG a\Bsetminus d$
and $a'=a\Bsetminus\upr(d,\frak C)$ is a relative atom over
$\frak C$\cmmnt{, that is, every $b\Bsubseteq a'$ is expressible as
$a'\Bcap c$ for some $c\in\frak C$}.

\cmmnt{(If we replace $\psG$ with $\ptG$, as used in \S395, we see
that 395Ma could be read as `if $\frak A$ is a Dedekind complete Boolean
algebra, then any subgroup of $\Aut\frak A$ has the $\tau$-refinement
property'.)}

\leader{448L}{}\cmmnt{ I give the principal case in which the
`$\sigma$-refinement property' just defined arises.

\medskip

\noindent}{\bf Proposition} Let $\frak A$ be a Dedekind
$\sigma$-complete Boolean algebra with countable Maharam
type\cmmnt{ (definition:  331F)}.   Then any countable
subgroup of $\Aut\frak A$ has the $\sigma$-refinement property.

\proof{{\bf (a)} Let $E$ be a countable subset of $\frak A$ which
$\tau$-generates $\frak A$, and $\frak E$ the subalgebra of $\frak A$
generated by $E$;  then $\frak E$ is countable (331Gc), and the smallest
order-closed subset of $\frak A$ including $\frak E$ is a subalgebra of
$\frak A$ (313Fc), so must be $\frak A$ itself.

\medskip

{\bf (b)} Suppose that $b\in\frak A\setminus\{0\}$ and
$\pi\in\Aut\frak A$
are such that $b\Bcap\pi b=0$.   Then there is an $e\in\frak E$ such
that $b\Bcap e\Bsetminus\pi e\ne 0$.   \Prf\Quer\ Otherwise, set

\Centerline{$D=\{d:d\in\frak A,\,b\Bcap d\Bsetminus\pi d=0\}$.}

\noindent Then $\frak E\subseteq D$, but $b\notin D$.   So $D$ cannot be
order-closed.   {\bf case 1} If $D_0\subseteq D$ is a non-empty
upwards-directed set with supremum $d_0\notin D$, then $b\Bcap
d_0\Bsetminus\pi d_0\ne 0$, so there is a $d\in D_0$ such that $b\Bcap
d\Bsetminus\pi d_0\ne 0$;  but now $d\notin D$, which is impossible.
{\bf case 2} If $D_0\subseteq D$ is a non-empty
downwards-directed subset of $D$ with infimum $d_0\notin D$, then
$b\Bcap d_0\Bsetminus\pi d_0\ne 0$.   But $\pi$ is
order-continuous, so there is a $d\in D_0$ such that $b\Bcap
d_0\Bsetminus\pi d\ne 0$;  and now $d\notin D$, which is impossible.
Thus in either case we have a contradiction.\ \Bang\Qed

\medskip

{\bf (c)} Now let $G$ be a countable subgroup of $\Aut\frak A$, with
fixed-point subalgebra $\frak C$, and let $\sequencen{(\pi_n,e_n)}$ be a
sequence running over $G\times\frak E$.  Take any $a\in\frak A$.   For
$k\in\Bbb N$ set

\Centerline{$a_k=a\Bcap e_k\Bcap\pi_k^{-1}(a\Bsetminus e_k)$,}

\Centerline{$a'_k=a_k\Bsetminus\sup_{j<k}\upr(a'_j,\frak C)$.}

\noindent Then

\Centerline{$a'_k\Bcap\pi_ka'_k\Bsubseteq a_k\Bcap\pi_ka_k=0$}

\noindent for every $k\in\Bbb N$, and whenever $j<k$ in $\Bbb N$ we have

\Centerline{$a'_j\Bcap a'_k=0$,
\quad$\pi_ja'_j\Bcap a'_k=0$,}

\Centerline{$a'_j\Bcap\pi_ka'_k
=\pi_k(\pi_k^{-1}a'_j\Bcap a'_k)=0$,
\quad$\pi_ja'_j\Bcap\pi_ka'_k
=\pi_k(a'_k\Bcap\pi_k^{-1}\pi_ja'_j)=0$.}

\noindent So, setting $d=\sup_{k\in\Bbb N}a'_k$ and $d'=\sup_{k\in\Bbb
N}\pi_ka'_k$, $d$ and $d'$ are disjoint and \Gse\ and included in $a$,
and $d\psG a\Bsetminus d$.

Consider $a'=a\Bsetminus\upr(d,\frak C)$.   Since
$a'_k=a_k\Bsetminus\sup_{j<k}\upr(a'_j,\frak C)$ for each $k$,

\Centerline{$\upr(d,\frak C)
=\sup_{k\in\Bbb N}\upr(a'_k,\frak C)=\sup_{k\in\Bbb N}\upr(a_k,\frak C)$.}

\noindent\Quer\ Suppose, if possible, that $a'$ is not a relative atom
over $\frak C$;  that is, that there is a $b\Bsubseteq a'$ such that
$b\ne a'\Bcap c$ for any $c\in\frak C$.   Then, in particular,
$b\ne a'\Bcap\upr(b,\frak C)$, and there is a $\pi\in G$ such that
$b'=a'\Bcap\pi b\Bsetminus b\ne 0$.   Then
$b'\Bcup\pi^{-1}b'\Bsubseteq a$, while $b'\Bcap\pi^{-1}b'=0$,
so $\pi b'\Bcap b'=0$.   By (b), there
is an $e\in\frak E$ such that $b''=b'\Bcap e\Bsetminus\pi e\ne 0$.   Let
$k$ be such that $\pi^{-1}=\pi_k$ and $e=e_k$, so that

\Centerline{$b''=b'\Bcap e_k\Bsetminus\pi_k^{-1}e_k
\Bsubseteq a\Bcap e_k\Bcap\pi_k^{-1}(a\Bsetminus e_k)=a_k$.}

\noindent(Because $\pi^{-1}b'\Bsubseteq a$, $b'\Bsubseteq\pi_k^{-1}a$.)
Since also

\Centerline{$b''\Bcap\upr(a'_j,\frak C)
\Bsubseteq a'\Bcap\upr(d,\frak C)=0$}

\noindent for every $j$, $b''\Bsubseteq a'_k\Bsubseteq d$, which is
impossible, because $b''\Bsubseteq a'$.\ \Bang

Thus $a'$ is a relative atom over $\frak C$, as required.
}%end of proof of 448L


\leader{448M}{Lemma} Let $\frak A$ be a Dedekind $\sigma$-complete
Boolean algebra, not $\{0\}$, and $G$ a countable countably
non-paradoxical subgroup of $\Aut\frak A$ with fixed-point subalgebra
$\frak C$.   If $G$ has the $\sigma$-refinement property, then for any
$\epsilon>0$ there is an $a^*\in\frak A$ such that $\upr(a^*,\frak C)=1$
and $\high{b:a^*}\le\low{b:a^*}+\epsilon\low{1:a^*}$ for every
$b\in\frak A$.

\proof{ As part (b) of the proof of 395M.
}%end of proof of 448M

\leader{448N}{Theorem}\cmmnt{ (Compare 395N.)}  Let $\frak A$ be a
Dedekind $\sigma$-complete Boolean algebra and $G$ a countable countably
non-paradoxical subgroup of $\Aut\frak A$ with
fixed-point subalgebra $\frak C$.   Suppose that $G$ has the
$\sigma$-refinement
property\cmmnt{ of 448K}.   Then there is a function
$\theta:\frak A\to L^{\infty}(\frak C)$ such that

(i) $\theta$ is additive, non-negative and sequentially
order-continuous;

(ii) $\theta a=0$ iff $a=0$, $\theta 1=\chi 1$;

(iii) $\theta(a\Bcap c)=\theta a\times\chi c$ for every $a\in\frak A$,
$c\in\frak C$;  in particular, $\theta c=\chi c$ for every
$c\in\frak C$;

(iv) if $a$, $b\in\frak A$ are \Gse, then $\theta a=\theta b$;  in
particular, $\theta$ is $G$-invariant.

\proof{ The arguments of the proof of 395N apply here also, though we
have to take things in a slightly different order.   As in 395N, set

$$\theta_a(b)=\bover{\high{b:a}}{\low{1:a}}\in L^0(\frak C)$$

\noindent whenever $\upr(a,\frak C)=1$ and $b\in\frak A$.   This time,
turn immediately to part (c) of the proof to see that if $e_n$ is chosen
(using 448M) such that $\upr(e_n,\frak C)=1$ and
$\high{b:e_n}\le\low{b:e_n}+2^{-n}\low{1:e_n}$ for every $b\in\frak A$,
then $\theta_{e_n}b\le\theta_ab+2^{-n}\high{1:a}$ whenever
$\upr(a,\frak C)=1$ and $b\in\frak A$.   So we can write

\Centerline{$\theta b=\inf_{n\in\Bbb N}\theta_{e_n}b
=\inf_{\upr(a,\frak C)=1}\theta_ab$}

\noindent for every $b\in\frak A$, and we have a function
$\theta:\frak A\to L^0$ as before.   The rest of the proof is unchanged,
except that
we have a simplification in (h), since we need consider only the case
$\kappa=\omega$.
}%end of proof of 448N

\leader{448O}{}\cmmnt{ This concludes the adaptations we need from
\S395.   I now return to the specific problem addressed in the present
section.   The first step is a variation on 448N.

\medskip

\noindent}{\bf Theorem} Let $\frak A$ be a Dedekind $\sigma$-complete
Boolean algebra, not $\{0\}$, and $G$ a countable subgroup of
$\Aut\frak A$ with the $\sigma$-refinement property.   Let $\frak C$ be
the fixed-point subalgebra of $G$.   Then the following are equiveridical:

(i) there are a Dedekind $\sigma$-complete Boolean algebra $\frak D$,
not $\{0\}$, and a $G$-invariant sequentially order-continuous
non-negative additive function $\theta:\frak A\to L^{\infty}(\frak D)$
such that $\theta 1=\chi 1$;

(ii) if $a\in\frak A$ and $1\psG a$, then
$\upr(1\Bsetminus a,\frak C)\ne 1$;

(iii) if $a\in\frak A$ and $1\psG a$, then $1\not\psG 1\Bsetminus a$.

\proof{{\bf (a)(i)$\Rightarrow$(iii)} Take
$\theta:\frak A\to L^{\infty}(\frak D)$ as in (i).   If $1\psG a$, then
there is a
$b\Bsubseteq a$ which is \Gse\ with $1$, so that $\theta b=\chi 1$, just
as in 395N(v)/448N(iv).   But this means that $\theta a=\chi 1$;  so
that $\theta(1\Bsetminus a)\ne\chi 1$ and $1\not\psG 1\setminus a$.

\medskip

{\bf (b)not-(ii)$\Rightarrow$not-(iii)} Suppose that (ii) is false;
that there is an $a\in\frak A$ such that $1\psG a$ and
$\upr(1\Bsetminus a,\frak C)=1$.   Let $G^*_{\sigma}$ be the countably
full local
semigroup generated by $G$;  then there is a $\psi\in G^*_{\sigma}$ such
that $\psi 1\Bsubseteq a$.   Set $b_0=1\Bsetminus\psi 1$ and
$b_n=\psi^nb_0$ for every $n\ge 1$;  then

\Centerline{$b_0\Bcap b_n\Bsubseteq b_0\Bcap\psi 1=0$}

\noindent for every $n\ge 1$, so

\Centerline{$b_m\Bcap b_n=\psi^m(b_0\Bcap b_{n-m})=0$}

\noindent whenever $m<n$.

Let $\sequence{i}{\pi_i}$ be a sequence running over $G$;  then

\Centerline{$\sup_{i\in\Bbb N}\pi_ib_0=\upr(b_0,\frak C)
\Bsupseteq\upr(1\Bsetminus a,\frak C)=1$.}

\noindent Set

\Centerline{$a_j=\pi_jb_0\Bsetminus\sup_{i<j}\pi_ib_0$}

\noindent for every $j\in\Bbb N$, so that $\sequence{j}{a_j}$ is a
partition of unity in $\frak A$.   Define $\psi_1$,
$\psi_2:\frak A\to\frak A$ by setting

\Centerline{$\psi_1d=\sup_{i\in\Bbb N}\psi^{2i}\pi_i^{-1}(d\Bcap a_i)$,
\quad$\psi_2d=\sup_{i\in\Bbb N}\psi^{2i+1}\pi_i^{-1}(d\Bcap a_i)$}

\noindent for every $d\in\frak A$.   Because
$\psi^{2i}\pi_i^{-1}a_i\Bsubseteq b_{2i}$ for every $i$,
$\sequence{i}{\psi^{2i}\pi_i^{-1}a_i}$ is disjoint, so $\psi_1\in
G^*_{\sigma}$ (448Bd);  similarly, $\psi_2\in G^*_{\sigma}$.   Thus

\Centerline{$1\psG\psi_11\Bsubseteq\sup_{i\in\Bbb N}b_{2i}$,}

\Centerline{$1\psG\psi_21\Bsubseteq\sup_{i\in\Bbb N}b_{2i+1}
\Bsubseteq 1\Bsetminus\psi_11$}

\noindent and (iii) is false.

\medskip

{\bf (c)} For the rest of this proof I will suppose that (ii) is
true and seek to prove (i).

Let $\Cal I$ be the $\sigma$-ideal of $\frak A$ generated by
$\{1\Bsetminus a:a\in\frak A,\,1\psG a\}$.   Then $1\notin\Cal I$.
\Prf\Quer\   Otherwise, there is a sequence $\sequencen{a_n}$ such that
$1\psG a_n$ for every $n$ and $\sup_{n\in\Bbb N}1\Bsetminus a_n=1$.
Choose $\psi_n\in G^*_{\sigma}$ such that $\psi_n1\Bsubseteq a_n$, and
set $c_n=\upr(1\Bsetminus\psi_n1,\frak C)$ for each $n$, so that
$\sup_{n\in\Bbb N}c_n=1$.   Set $c'_n=c_n\Bsetminus\sup_{i<n}c_i$ for
each $n$, and write

\Centerline{$\psi d=\sup_{n\in\Bbb N}\psi_n(d\Bcap c'_n)$}

\noindent for each $d\in\frak A$.
Because every $c'_n$ belongs to $\frak C$,
$\sequencen{\psi_nc'_n}=\sequencen{c'_n}$ is disjoint, and
$\psi\in G^*_{\sigma}$.

By (ii), $c=\upr(1\Bsetminus\psi 1,\frak C)$ is not $1$;  let $n$
be such that $c'=c'_n\Bsetminus c\ne 0$.   Because $c'\subseteq c'_n$,
$c'\Bsetminus\psi_n1=c'\Bsetminus\psi 1$;  because $c'\in\frak C$,

$$\eqalign{0
&\ne c'\Bcap c'_n
\Bsubseteq c'\Bcap\upr(1\Bsetminus\psi_n1,\frak C)
=\upr(c'\Bsetminus\psi_n1,\frak C)\cr
&=\upr(c'\Bsetminus\psi 1,\frak C)
=c'\Bcap\upr(1\Bsetminus\psi 1,\frak C)
=0\cr}$$

\noindent which is absurd.\ \Bang\Qed

Let $\frak B$ be the quotient Boolean algebra $\frak A/\Cal I$;  then
$\frak B$ is Dedekind $\sigma$-complete and the canonical homomorphism
$a\mapsto a^{\ssbullet}:\frak A\to\frak B$ is sequentially
order-continuous (314C, 313Qb).

\medskip

{\bf (d)} Next, $\pi b\in\Cal I$ whenever $b\in\Cal I$ and $\pi\in G$.
\Prf\ The sets $\{a:1\psG a\}$ and $\{1\Bsetminus a:1\psG a\}$ are both
invariant under the action of $G$, so $\Cal I$ also must be
invariant.\ \QeD\   We
can therefore define, for each $\pi\in G$, a Boolean automorphism
$\tilde\pi:\frak B\to\frak B$, setting $\tilde\pi a^{\ssbullet}=(\pi
a)^{\ssbullet}$ for every $a\in\frak A$.    Because
$(\pi\phi)\ssptilde=\tilde\pi\tilde\phi$ for all $\pi$, $\phi\in G$,
$\tilde G=\{\tilde\pi:\pi\in G\}$ is a subgroup of $\Aut\frak B$;  of
course it is countable.   Let $\frak D$ be the fixed-point subalgebra of
$\tilde G$ in $\frak B$.   Because $\frak B$ is not $\{0\}$, nor is
$\frak D$.

\medskip

{\bf (e)} $\tilde G$ is countably non-paradoxical.   \Prf\ Suppose that
$b$ is $\tilde G$-$\sigma$-equidecomposable with $1$ in $\frak B$.   Let
$\sequencen{b_n}$ be a partition of unity in $\frak B$ and
$\sequencen{\pi_n}$ a sequence in $G$ such that
$\sequencen{\tilde\pi_nb_n}$ is disjoint and has supremum $b$.
For each $n\in\Bbb N$, let
$a_n\in\frak A$ be such that $a_n^{\ssbullet}=b_n$.   We have

\Centerline{$(a_m\Bcap a_n)^{\ssbullet}=(\pi_m a_m\Bcap\pi_n
a_n)^{\ssbullet}=0$}

\noindent whenever $m\ne n$, so

\Centerline{$d=\sup_{m\ne n}(a_m\Bcap a_n)
  \Bcup\sup_{m\ne n}(a_m\Bcap\pi_m^{-1}\pi_na_n)$}

\noindent belongs to $\Cal I$, while $\sequencen{a_n\Bsetminus d}$,
$\sequencen{\pi_n(a_n\Bsetminus d)}$ are disjoint.

Because $\sup_{n\in\Bbb N}b_n=1$ in $\frak B$,
$d'=1\Bsetminus\sup_{n\in\Bbb N}a_n\in\Cal I$.   Because $\Cal I$ is
a $\sigma$-ideal and $G$ is countable,

\Centerline{$c^*=\upr(d\Bcup d',\frak C)=\sup_{\pi\in G}\pi(d\Bcup d')$}

\noindent belongs to $\Cal I$, while
$\{c^*\}\cup\{a_n\Bsetminus c^*:n\in\Bbb N\}$ is a partition of unity in
$\frak A$.

Define $\psi\in G^*_{\sigma}$ by setting

\Centerline{$\psi a
=\sup_{n\in\Bbb N}\pi_n(a\Bcap a_n\Bsetminus c^*)\Bcup(a\Bcap c^*)$}

\noindent for every $a\in\frak A$.   Then $1\Bsetminus\psi 1\in\Cal I$,
by the definition of $\Cal I$, so $(\psi 1)^{\ssbullet}=1$ in $\frak B$.   But

$$\eqalign{(\psi 1)^{\ssbullet}
&=\sup_{n\in\Bbb N}(\pi_n(a_n\Bsetminus c^*))^{\ssbullet}
=\sup_{n\in\Bbb N}(\pi_na_n)^{\ssbullet}\cr
&=\sup_{n\in\Bbb N}\tilde \pi_nb_n
=b.\cr}$$

\noindent So $b=1$.   As $b$ is arbitrary, $\tilde G$ is countably
non-paradoxical.\ \Qed

\medskip

{\bf (f)} $\tilde G$ has the $\sigma$-refinement property.   \Prf\ Let
$b\in\frak
B$.   Then there is an $a\in\frak A$ such that $a^{\ssbullet}=b$.
Because $G$ is supposed to have the $\sigma$-refinement property, there
is a
$d\Bsubseteq a$ such that $d\psG a\Bsetminus d$ and
$a\Bsetminus\upr(d,\frak C)$ is a relative atom over $\frak C$.
Set $e=d^{\ssbullet}\Bsubseteq b$.

We know that there are a partition of unity $\sequencen{d_n}$ in $\frak
A_d$ and a sequence $\sequencen{\pi_n}$ in $G$ such that
$\pi_nd_n\Bsubseteq a\Bsetminus d$ for every $n$ and
$\sequencen{\pi_nd_n}$ is disjoint.   Now $\sequencen{d_n^{\ssbullet}}$
is a partition of unity in $\frak B_e$,
$\tilde\pi_nd_n^{\ssbullet}\Bsubseteq b\setminus e$ for every $n$, and
$\sequencen{\tilde\pi_nd_n^{\ssbullet}}$ is disjoint;  so $e\pstildeG
b\Bsetminus e$.

Suppose that $b_0\Bsubseteq b\Bsetminus\upr(e,\frak D)$.   Then it is
expressible as $a_0^{\ssbullet}$ where $a_0\Bsubseteq a$ and

\Centerline{$(a_0\Bcap\pi d)^{\ssbullet}=b_0\Bcap\tilde\pi e=0$}

\noindent for every $\pi\in G$.   So if we set
$a_1=a_0\Bsetminus\sup_{\pi\in G}\pi d$, we shall have $a_1\Bsubseteq
a\Bsetminus\upr(d,\frak C)$ and $a_1^{\ssbullet}=b_0$.   Now
$a\Bsetminus\upr(d,\frak C)$ is supposed to be a relative atom over
$\frak C$, so $a_1=a\Bcap c$ for some $c\in\frak C$.   In this case,

\Centerline{$\tilde\pi c^{\ssbullet}=(\pi c)^{\ssbullet}=c^{\ssbullet}$}

\noindent for every $\pi\in G$, so $c^{\ssbullet}\in\frak D$, while
$b_0=b\Bcap c^{\ssbullet}$.   As $b_0$ is arbitrary,
$b\Bsetminus\upr(e,\frak D)$ is a relative atom over $\frak D$.

Thus $e$ has both the properties required by the definition 448K.   As
$b$ is arbitrary, $\tilde G$ has the $\sigma$-refinement property.\ \Qed

\medskip

{\bf (g)} 448N now tells us that there is a sequentially
order-continuous non-negative additive functional $\theta_0:\frak B\to
L^{\infty}(\frak D)$ such that $\theta_01=\chi 1$ and
$\theta_0(\tilde\pi b)=\theta_0b$ whenever $b\in\frak B$ and $\pi\in G$.
If we set $\theta a=\theta_0a^{\ssbullet}$ for $a\in\frak A$,
it is easy to see that
$\theta$ has all the properties required by (i) of this theorem.   Thus
(ii)$\Rightarrow$(i), and the proof is complete.
}%end of proof of 448O

\leader{448P}{}\cmmnt{ At last we come to Polish spaces.

\medskip

\noindent}{\bf Theorem}\cmmnt{ ({\smc Nadkarni 90},
{\smc Becker \& Kechris 96})}  Let $G$ be a Polish group acting on a non-empty Polish
space $(X,\frak T)$ with a Borel measurable action $\action$.   For
Borel sets $E$, $F\subseteq X$ say that $E\psG F$ if there are a
countable partition $\familyiI{E_i}$ of $E$ into Borel sets, and a
family $\familyiI{g_i}$ in $G$, such that $g_i\action E_i\subseteq F$
for every $i$ and $\familyiI{g_i\action E_i}$ is disjoint.
Then the following are equiveridical:

(i) there is a $G$-invariant Radon probability measure $\mu$ on $X$;

(ii) if $F\subseteq X$ is a Borel set such that $X\psG F$, then
$\bigcap_{n\in\Bbb N}g_n\action F\ne\emptyset$ for any sequence
$\sequencen{g_n}$ in $G$;

(iii) there are no disjoint Borel sets $E$, $F\subseteq X$ such that
$X\psG E$ and $X\psG F$.

\proof{{\bf (a)} Let us start with the easy parts.

\medskip

\quad{\bf (i)$\Rightarrow$(ii)} Let $\mu$ be a $G$-invariant Radon
probability measure on $X$, and suppose that $X\psG F$.   Let
$\familyiI{E_i}$ be a countable partition of $X$ into Borel sets and
$\familyiI{h_i}$ a family in $G$ such that $\familyiI{h_i\action E_i}$
is disjoint and $h_i\action E_i\subseteq F$ for every $i$.   Then

\Centerline{$\mu F
\ge\sum_{i\in I}\mu(h_i\action E_i)
=\sum_{i\in I}\mu E_i
=\mu X$,}

\noindent so $F$ is conegligible.   Consequently
$\bigcap_{n\in\Bbb N}g_n\action F$ must be conegligible and cannot be
empty, for any sequence $\sequencen{g_n}$ in $G$.   As $F$ and
$\sequencen{g_n}$ are arbitrary, (ii) is true.

\medskip

\quad{\bf (ii)$\Rightarrow$(iii)} Assume (ii).   \Quer\ If there are
disjoint $E$, $F$ such that $X\psG E$ and $X\psG F$, then we have a
countable partition $\familyiI{E_i}$ of $X$ into Borel sets and a family
$\familyiI{g_i}$ in $G$ such that $g_i\action E_i\subseteq E$ for every
$i\in I$.   But there is an $x\in\bigcap_{i\in I}g_i^{-1}\action F$, by
(ii).   In this case there is a $j\in I$ such that $x\in E_j$ and
$g_j\action x\in E\cap F$, which is impossible.\ \BanG\  So (iii) must
be true.

\medskip

{\bf (b)} For the rest of the proof, therefore, I shall assume (iii) and
seek to prove (i).

Let $\frak T$ be the topology of $X$, and $\Cal B=\Cal B(X)$ its Borel
$\sigma$-algebra.
For $g\in G$ define $\pi_g:\Cal B\to\Cal B$ by writing $\pi_gE=g\action
E$ for every $E\in\Cal B$.   Then $\pi_{gh}=\pi_g\pi_h$ for all $g$,
$h\in G$, so $\tilde G=\{\pi_g:g\in G\}$ is a subgroup of $\Aut\Cal B$.
Observe that for $E$, $F\in\Cal B$, $E\psG F$, in the sense here, iff
$E\pstildeG F$ in the sense of 448A.

By the Becker-Kechris theorem (424H), there is a Polish topology
$\frak T_1$ on $X$, giving rise to the same Borel $\sigma$-algebra
$\Cal B$ as
the original topology, for which the action of $G$ is continuous.   Let
$\Cal U$ be a countable base for $\frak T_1$.
(We are going to have three Polish topologies on $X$ in this
proof, so watch carefully.)

\medskip

{\bf (c)} For the time being (down to the end of (f) below) let us
suppose that $G$, and therefore $\tilde G$, are countable.   In this
case, because $\Cal B$ is countably generated,
$\tilde G$ has the $\sigma$-refinement property, by 448L.
We can therefore apply 448O to see that (iii) implies that

\inset{(i)$'$ there are a Dedekind $\sigma$-complete Boolean algebra
$\frak D$, not $\{0\}$, and a $\tilde G$-invariant sequentially
order-continuous non-negative additive functional
$\theta:\Cal B\to L^{\infty}(\frak D)$ such that $\theta X=\chi 1$.}

Express
$\frak D$ as $\Sigma/\Cal J$ where $\Sigma$ is a $\sigma$-algebra of
subsets of
a set $Z$ and $\Cal J$ is a $\sigma$-ideal of $\Sigma$ (314N).  Then we
can identify $L^{\infty}(\frak D)$ with the quotient $\eusm L^{\infty}/W$,
where $\eusm L^{\infty}$ is the space of bounded $\Sigma$-measurable
real-valued functions on $Z$ and $W$ is the set
$\{f:f\in \eusm L^{\infty},\,\{z:f(z)\ne 0\}\in\Cal J\}$ (363Hb).   For
each $E\in\Cal B$, let $f_E\in\eusm L^{\infty}$ be a representative of
$\theta E\in L^{\infty}(\frak D)$;  because $\theta(\pi E)=\theta E$
whenever $E\in\Cal B$ and $\pi\in\tilde G$, we may suppose that
$f_{\pi E}=f_E$ whenever $E\in\Cal B$ and $\pi\in\tilde G$.

Let $\frak B$ be the subalgebra of $\Cal B$ generated by
$\{\pi U:U\in\Cal U,\,\pi\in\tilde G\}$.   Then $\frak B$ is countable
and $\pi E\in\frak B$ for every $E\in\frak B$, $\pi\in \tilde G$.   By
4A3I, there is yet
another Polish topology $\frak S$ on $X$ which is zero-dimensional
and
such that every member of $\frak B$ is open-and-closed for $\frak S$.
Of course $\frak S\supseteq\Cal U$, so $\Cal B$ is still the algebra of
$\frak S$-Borel sets (423Fb).   Let $\Cal W$ be a countable base
for $\frak S$ consisting of sets which are open-and-closed for
$\frak S$, and let $\frak B_1$ be the subalgebra of $\Cal B$ generated
by $\Cal W\cup\frak B$;  then $\frak B_1$ is countable and consists of
open-and-closed sets for $\frak S$.   Let $\sequencen{W_n}$ be a
sequence running over $\Cal W$.   Let $\rho$ be a complete metric on $X$
defining the topology $\frak S$, and for $m$, $n\in\Bbb N$ set

\Centerline{$W_{mn}=\bigcup\{W_i:i\le n,\,
   \diam_{\rho}(W_i)\le 2^{-m}\}$;}

\noindent then for each $m\in\Bbb N$, $\sequencen{W_{mn}}$ is a
non-decreasing sequence in $\frak B_1$ with union $X$.

\wheader{448P}{4}{2}{2}{72pt}

{\bf (d)} Consider the subsets of $Z$ of the following types:

\Centerline{$P_E=\{z:f_E(z)<0\}$, where $E\in\frak B_1$,}

\Centerline{$Q_{EF}=\{z:f_{E\cup F}(z)\ne f_E(z)+f_F(z)\}$, where $E$,
$F\in\frak B_1$ and $E\cap F=\emptyset$,}

\Centerline{$R=\{z:f_X(z)\ne 1\}$,}

\Centerline{$S_m=\{z:\sup_{n\in\Bbb N}f_{W_{mn}}(z)\ne 1\}$,
 where $m\in\Bbb N$.}

\noindent Because

\Centerline{$f_E^{\ssbullet}=\theta E\ge 0$ for every $E\in\Cal B$,}

\Centerline{$f_{E\cup F}^{\ssbullet}=\theta(E\cup F)=\theta E+\theta F
=f_E^{\ssbullet}+f_F^{\ssbullet}$ whenever $E\cap F=\emptyset$,}

\Centerline{$f_X^{\ssbullet}=\theta X=\chi 1$,}

\Centerline{$\sup_{n\in\Bbb N}f_{W_{mn}}^{\ssbullet}
=\sup_{n\in\Bbb N}\theta W_{mn}
=\theta(\bigcup_{n\in\Bbb N}W_{mn})
=\theta X=\chi 1$}

\noindent for every $m\in\Bbb N$, all the sets $P_E$, $Q_{EF}$, $R$ and
$S_m$ belong to $\Cal J$.   Since $\frak D\ne\{0\}$, $Z\notin\Cal J$;
so there is a $z_0\in Z$ not belonging to $R$ or $P_E$ or $Q_{EF}$ or
$S_m$ whenever $m\in\Bbb N$ and $E$, $F\in\frak B_1$ are disjoint.

Set $\nu E=f_E(z_0)$ for every $E\in\frak B_1$.   If $E$,
$F\in\frak B_1$ are disjoint, then $\nu(E\cup F)=\nu E+\nu F$ because
$z_0\notin Q_{EF}$;  thus $\nu:\frak B_1\to\Bbb R$ is additive.   If
$E\in\frak B_1$ then $\nu E\ge 0$ because $z_0\notin P_E$, so $\nu$ is
non-negative.   $\nu X=1$ because $z_0\notin R$.   For each
$m\in\Bbb N$, $\sup_{n\in\Bbb N}\nu W_{mn}=1$ because $z_0\notin S_m$.

\medskip

{\bf (e)} For any $\epsilon>0$ there is an $\frak S$-compact set
$K\subseteq X$ such that $\nu E\ge 1-\epsilon$ whenever $E\in\frak B_1$
and $E\supseteq K$.   \Prf\ For each $m\in\Bbb N$ we have a
$k(m)\in\Bbb N$ such that $\nu W_{m,k(m)}\ge 1-2^{-m-1}\epsilon$.   Set
$K=\bigcap_{m\in\Bbb N}W_{m,k(m)}$.   Because every $W_{m,k(m)}$ is
$\frak S$-closed, $K$ is $\frak S$-closed, therefore $\rho$-complete;
because every $W_{m,k(m)}$ is a finite union of sets of diameter at most
$2^{-m}$, $K$ is $\rho$-totally bounded, therefore $\frak S$-compact
(4A2Je).   \Quer\ Suppose, if possible, that $E\in\frak B_1$ is
such that $K\subseteq E$ and $\nu E<1-\epsilon$.   For every
$m\in\Bbb N$,

\Centerline{$\nu(\bigcap_{i\le m}W_{i,k(i)})
\ge 1-\sum_{i=0}^m\nu(X\setminus W_{i,k(i)})
\ge 1-\sum_{i=0}^m2^{-i-1}\epsilon
>1-\epsilon
>\nu E$}

\noindent because $\nu$ is non-negative and finitely additive.   So
$\bigcap_{i\le m}W_{i,k(i)}\setminus E$ must be non-empty.   There is
therefore an ultrafilter $\Cal F$ on $X$ containing
$W_{i,k(i)}\setminus E$ for every $i\in\Bbb N$.   Now for each $i$ there
must be a $j\le k(i)$ such that $\diam W_j\le 2^{-i}$ and
$W_j\in\Cal F$, so $\Cal F$ is
a $\rho$-Cauchy filter, and $\frak S$-converges to $x$ say.   Because
every $W_{i,k(i)}$ is $\frak S$-closed,
$x\in\bigcap_{i\in\Bbb N}W_{i,k(i)}=K$;  because $E\in\frak S$,
$x\notin E$;  but $K$ is
supposed to be included in $E$.\ \Bang

Thus $\inf\{\nu E:K\subseteq E\in\frak B_1\}\ge 1-\epsilon$.   As
$\epsilon$ is arbitrary, we have the result.\ \Qed

\medskip

{\bf (f)} By 416O, there is an $\frak S$-Radon measure $\mu$ on
$X$ extending $\nu$.   Because $\mu$ is just the completion of its
restriction to $\Cal B$, it is also $\frak T$-Radon and
$\frak T_1$-Radon (433Cb).

Now $\mu$ is $G$-invariant.   \Prf\ Take any $g\in G$.   Set
$\mu_gE=\mu(g\action E)$ whenever $E\subseteq X$ and $g\action
E\in\dom\mu$.   The map $x\mapsto g\action x$ is a homeomorphism for
$\frak T_1$, so $\mu_g$ also is a $\frak T_1$-Radon measure.   (Setting
$\phi(x)=g^{-1}\action x$, $\mu_g$ is the image measure $\mu\phi^{-1}$.)
Again because $\frak T$ and $\frak T_1$ have the same Borel
$\sigma$-algebras,
$\mu_g$ is $\frak T$-Radon.   If $E\in\frak B$, then $E$ and
$g\action E$ belong to $\frak B\subseteq\frak B_1$, so

$$\eqalignno{\mu_gE
&=\mu(g\action E)
=\nu(g\action E)
=\nu(\pi_gE)
=f_{\pi_gE}(z_0)
=f_E(z_0)\cr
\noalign{\noindent (because $f_{\pi_gE}=f_E$, as declared in (c) above)}
&=\nu E
=\mu E.\cr}$$

\noindent In particular, $\mu_gE=\mu E$ for every $E$ in the algebra
generated by $\Cal U$.   But $\mu_g$ and $\mu$ are both $\frak T$-Radon
measures, and $\Cal U$ is a base for $\frak T$, so $\mu_g=\mu$ (415H(iv)).
As $g$ is arbitrary, $\mu$ is $G$-invariant.\ \Qed

Thus we have found a $G$-invariant Radon probability measure, and (i) is
true.

\medskip

{\bf (g)} Thus (iii)$\Rightarrow$(i) if $G$ is countable.   Now let us
consider the general case.   Because $G$ is a Polish group, it has a
countable dense subgroup $H$.   (Take $H$ to be the subgroup generated
by any countable dense subset of $G$.)   Of course there can be no
disjoint $E$, $F\in\Cal B$ such that $X\psH E$ and $X\psH F$, so there
must be an $H$-invariant Radon probability
measure $\mu$ on $X$, by the arguments of
(b)-(f).   ($H$ need not be a Polish group in its subspace topology.
But if we give it its discrete topology, then $x\mapsto h\action x$ is
still a $\frak T_1$-homeomorphism for every $h\in H$, so the action of
$H$ on $X$ is still continuous if $H$ is given its discrete topology and
$X$ is given $\frak T_1$.)

Now $\mu$ is $G$-invariant.   \Prf\ For any $g\in G$, let $\mu_g$ be the
Radon probability measure defined by setting $\mu_gE=\mu(g\action E)$
whenever this is defined.   (As in (f) above, this formula does define a
probability measure which is Radon for either $\frak T$ or $\frak T_1$.)
Let $f:X\to\Bbb R$ be any bounded $\frak T_1$-continuous function.
Then

\Centerline{$\int f d\mu_g=\int f(g^{-1}\action x)\mu(dx)$}

\noindent (applying 235G with
$\phi(x)=g^{-1}\action x$).   Now there is
a sequence $\sequencen{h_n}$ in $H$ converging to $g$.   In this case,
because $G$ is a topological group, $g^{-1}=\lim_{n\to\infty}h_n^{-1}$.
Because the action of $G$ on $X$ is $\frak T_1$-continuous,
$g^{-1}\action x=\lim_{n\to\infty}h^{-1}_n\action x$, for $\frak T_1$,
for every $x\in X$.   Because $f$ is $\frak T_1$-continuous,
$f(g^{-1}\action x)=\lim_{n\to\infty}f(h_n^{-1}\action x)$ in $\Bbb R$
for every $x\in X$.   By Lebesgue's Dominated Convergence Theorem,

\Centerline{$\int fd\mu_g
=\int f(g^{-1}\action x)\mu(dx)
=\lim_{n\to\infty}f(h_n^{-1}\action x)\mu(dx)
=\lim_{n\to\infty}\int fd\mu_{h_n}
=\int fd\mu$}

\noindent because $\mu$ is $H$-invariant, so $\mu_{h_n}=\mu$ for every
$n$.   As $f$ is arbitrary, $\mu_g=\mu$, by 415I.   As $g$ is arbitrary,
$\mu$ is $G$-invariant.\ \Qed

Thus (iii)$\Rightarrow$(i) in all cases, and the proof is complete.
}%end of proof of 448P

\leader{448Q}{}\dvAnew{2011}\cmmnt{ I turn now to Mackey's theorem.
I pave the
way with a couple of lemmas which are of independent interest.

\medskip

\noindent}{\bf Lemma}
Let $(X,\Sigma,\mu)$ be a $\sigma$-finite measure
space with countable Maharam type.   Write $L^0(\Sigma)$ for the
set of $\Sigma$-measurable functions from $X$ to $\Bbb R$.
Then there is a function $T:L^0(\mu)\to L^0(\Sigma)$ such that

($\alpha$) $u=(Tu)^{\ssbullet}$ for every $u\in L^0$,

($\beta$) $(u,x)\mapsto(Tu)(x):L^0\times X\to\Bbb R$ is
$(\Cal B\tensorhat\Sigma)$-measurable,

\noindent where $\Cal B=\Cal B(L^0)$ is the Borel $\sigma$-algebra of
$L^0$ with its topology of convergence in measure.

\proof{{\bf (a)} Consider first the case in which $\mu$ is a probability
measure.

\medskip

\quad{\bf (i)} Let $\sequencen{E_n}$ be a sequence in $\Sigma$ such that
the measure algebra $\frak A$ of $\mu$ is $\tau$-generated by
$\{E_n^{\ssbullet}:n\in\Bbb N\}$.
For $n\in\Bbb N$ let $\Sigma_n$ be
the finite subalgebra of $\Sigma$ generated by $\{E_i:i\le n\}$, and
for $n\in\Bbb N$, $u\in L^{\infty}=L^{\infty}(\mu)$ and $x\in X$ set

$$\eqalign{(S_nu)(x)
&=\Bover1{\mu E}\int_Eu\text{ if }E\text{ is the atom of
}\Sigma_n\text{ containing }x\text{ and }\mu E>0,\cr
&=0\text{ if the atom of }\Sigma_n\text{ containing }x\text{ is
negligible}.\cr}$$

\noindent Then $(u,x)\mapsto(S_nu)(x)$ is
$(\Cal B\tensorhat\Sigma)$-measurable, because
$u\mapsto\int_Eu:L^{\infty}\to\Bbb R$ is
continuous (for the topology of convergence in measure) for every
$E\in\Sigma$.   So if we set
$Su=\limsup_{n\to\infty}S_nu$ for $u\in L^{\infty}$,
$(u,x)\mapsto(Su)(x)$ will be $(\Cal B\tensorhat\Sigma)$-measurable.

On the other hand, if $f\in\eusm L^{\infty}$, $S_nf^{\ssbullet}$ is a
conditional expectation of $f$ on $\Sigma_n$ for each $n$.
So L\'evy's martingale theorem (275I) tells us that
if $f\in\eusm L^{\infty}$ then $\sequencen{S_nf^{\ssbullet}}$ converges
a.e.\ to a conditional expectation $g$ of $f$ on the $\sigma$-algebra
$\Sigma_{\infty}$ generated by $\bigcup_{n\in\Bbb N}\Sigma_n$.   But we
chose $\sequencen{E_n}$ to generate $\frak A$, so
$\frak A=\{E^{\ssbullet}:E\in\Sigma_{\infty}\}$.   If now $E\in\Sigma$,
there is an $F\in\Sigma_{\infty}$ such that $E\symmdiff F$ is negligible,
so

\Centerline{$\int_Eg=\int_Fg=\int_Ff=\int_Ef$.}

\noindent As $E$ is arbitrary,

\Centerline{$f\eae g\eae\limsup_{n\to\infty}S_nf^{\ssbullet}
=Sf^{\ssbullet}$.}

\noindent Turning this round, $(Su)^{\ssbullet}=u$ for every
$u\in L^{\infty}$.

\medskip

\quad{\bf (ii)} Now define $R:L^0\to L^{\infty}$ by setting

\Centerline{$Rf^{\ssbullet}=(\arctan f)^{\ssbullet}$}

\noindent for $f\in\eusm L^0$ (see 241I).   Then $R$ is continuous for
the topology of convergence in measure (245Dd), so
$(u,x)\mapsto(SRu)(x):L^0\times X\to\Bbb R$ is
$(\Cal B\tensorhat\Sigma)$-measurable.   Note that if $u\in L^0$, then
$-\bover{\pi}2<(S_nRu)(x)<\bover{\pi}2$ for every $x$ and $n$, so
$-\bover{\pi}2\le(SRu)(x)\le\bover{\pi}2$ for every $x$;  also, if
$u=f^{\ssbullet}$, then $-\bover{\pi}2<\arctan f(x)<\bover{\pi}2$
whenever $f(x)$ is defined,
so $-\bover{\pi}2<(SRu)(x)<\bover{\pi}2$ for almost every $x$.
If now we set

$$\eqalign{\tan_0t
&=\tan t\text{ for }-\Bover{\pi}2<t<\Bover{\pi}2,\cr
&=0\text{ for }t=\pm\Bover{\pi}2,\cr}$$

\noindent and $Tu=\tan_0SRu$, we shall have $Tu\in L^0(\Sigma)$ and
$(Tu)^{\ssbullet}=u$ for every
$u\in L^0$, while $(u,x)\mapsto(Tu)(x)$ is
$(\Cal B\tensorhat\Sigma)$-measurable.

\medskip

{\bf (b)} For the general case, if $\mu X=0$ the result is trivial,
as we can just set
$(Tu)(x)=0$ for all $u$ and $x$.   So suppose otherwise.   Let
$\nu$ be a probability measure with the same domain and the same negligible
sets as $\mu$ (215B(vii)).   Then the measure algebra of $\nu$, regarded as a
Boolean algebra, is the same as that of $\mu$, so
$\nu$ also has countable Maharam type;  similarly,
$L^0=L^0(\nu)$.   Moreover, the topology of convergence in measure on $L^0$
is the same, whichever measure we take to define it (245Xm, 367T).
So we can apply (a) to $(X,\Sigma,\nu)$.
}%end of proof of 448Q

\leader{448R}{Lemma}\dvAnew{2011}
Let $(X,\Sigma,\mu)$ be a $\sigma$-finite measure
space with countable Maharam type.

(a) $L^0=L^0(\mu)$, with its topology of convergence in measure,
is a Polish space.

(b) Let $\frak A$ be the measure algebra
of $\mu$, and $\frak A^f$ the set $\{a:a\in\frak A$, $\bar\mu a<\infty\}$.
Then the Borel $\sigma$-algebra $\Cal B=\Cal B(L^0)$
is the $\sigma$-algebra
of subsets of $L^0$ generated by sets of the form
$\{u:\bar\mu(a\Bcap\Bvalue{u\in F})>\alpha\}$,
where $a\in\frak A^f$, $F\subseteq\Bbb R$ is Borel, and
$\alpha\in\Bbb R$.

\proof{{\bf (a)} By 245Eb, $L^0$ is metrizable, and complete when
regarded as a linear topological space;  so by 4A4Bj there is a metric on
$L^0$, defining its topology, under which $L^0$ is complete.
By 367Rb, $L^0$ is separable, so it is a Polish space.

\medskip

{\bf (b)} Write $\Upsilon$ for the $\sigma$-algebra
of subsets of $L^0$ generated by sets of the form
$\{u:\bar\mu(a\Bcap\Bvalue{u\in F})>\alpha\}$,
where $a\in\frak A^f$, $F\subseteq\Bbb R$ is Borel, and
$\alpha\in\Bbb R$.

\medskip

\quad{\bf (i)}
If $a\in\frak A^f$, $\alpha\in\Bbb R$ and $H\subseteq\Bbb R$ is open, then
$U=\{u:\bar\mu(a\Bcap\Bvalue{u\in H})>\alpha\}$ is open in $L^0$.
\Prf\ If $u\in U$, there are a compact set
$K\subseteq H$ and a $\delta>0$ such
that $\bar\mu(a\Bcap\Bvalue{u\in K})>\alpha+\delta$.   Now there is an
$\eta\in\ocint{0,1}$
such that $|\alpha-\beta|>\eta$ whenever $\alpha\in K$ and
$\beta\in\Bbb R\setminus H$.   In this case,
$V=\{v:v\in L^0$, $\bar\mu(a\Bcap\Bvalue{|u-v|>\eta})\le\delta\}$
is a neighbourhood of $u$ in $L^0$ (367L).   If $v\in V$, then

\Centerline{$\Bvalue{v\in H}
\Bsupseteq\Bvalue{u\in K}\Bcap\Bvalue{|u-v|\le\eta}$,}

$$\eqalign{\bar\mu(a\Bcap\Bvalue{v\in H})
&\ge\bar\mu(a\Bcap\Bvalue{u\in K})-\bar\mu(a\Bcap\Bvalue{|u-v|>\eta})\cr
&\ge\bar\mu(a\Bcap\Bvalue{u\in K})-\delta
>\alpha.\cr}$$

\noindent Thus $V\subseteq U$ and $U$ is a neighbourhood of $u$;  as $u$ is
arbitrary, $U$ is open.\ \Qed

\medskip

\quad{\bf (ii)} Thus $u\mapsto\bar\mu(a\Bcap\Bvalue{u\in H})$ is
$\Cal B$-measurable for every $a\in\frak A^f$ and open $H\subseteq\Bbb R$.
Now the set

\Centerline{$\{F:F\subseteq\Bbb R$ is Borel,
$u\mapsto\bar\mu(a\Bcap\Bvalue{u\in F})$ is $\Cal B$-measurable
for every $a\in\frak A^f\}$}

\noindent is a Dynkin class containing all open sets, so is the Borel
$\sigma$-algebra of $\Bbb R$ (136B), and
$u\mapsto\bar\mu(a\Bcap\Bvalue{u\in F})$ is
$\Cal B$-measurable for every $a\in\frak A^f$ and Borel
$F\subseteq\Bbb R$.   Thus $\Upsilon\subseteq\Cal B$.

\medskip

\quad{\bf (iii)} In the other direction, we know that $\frak A$ is
separable, by 331O;  let $\sequence{k}{c_k}$ run over a dense
subset of $\frak A$.   We also know that there is a sequence
$\sequencen{a_n}$ in $\frak A^f$ with supremum $1$.   Set

\Centerline{$E_{nkqq'}=\{u:u\in L^0$,
$\bar\mu(a_n\Bcap c_k\Bcap\Bvalue{u>q})>q'\}\in\Upsilon$}

\noindent for $n$, $k\in\Bbb N$ and $q$, $q'\in\Bbb Q$.   If $u$,
$v\in L^0$ are different, there are $n$, $k$, $q$ and $q'$ such that
$E_{nkqq'}$ contains one of $u$, $v$ and not the other.   \Prf\ Choose
$q\in\Bbb Q$ such that
$\Bvalue{u>q}\ne\Bvalue{v>q}$.   Suppose for the moment that
$c=\Bvalue{u>q}\Bsetminus\Bvalue{v>q}\ne 0$.   Let $n\in\Bbb N$ be such
that $\bar\mu(a_n\Bcap c)>0$.   Let $k\in\Bbb N$ be such that
$\bar\mu(a_n\Bcap(c\Bsymmdiff c_k))<\bar\mu(a_n\Bcap c)$.   Then

$$\eqalign{\bar\mu(a_n\Bcap c_k\Bcap\Bvalue{v>q})
&\le\bar\mu(a_n\Bcap c_k\Bsetminus c)\cr
&<\bar\mu(a_n\Bcap c)-\bar\mu(a_n\Bcap c\Bsetminus c_k)
\le\bar\mu(a_n\Bcap c_k\Bcap\Bvalue{u>q}),\cr}$$

\noindent so there is a $q'\in\Bbb Q$ such that $u\in E_{nkqq'}$ and
$v\notin E_{nkqq'}$.   Similarly, if
$\Bvalue{v>q}\notBsubseteq\Bvalue{u>q}$ there are $n$, $k\in\Bbb N$ and
$q'\in\Bbb Q$ such that $v\in E_{nkqq'}$ and
$u\notin E_{nkqq'}$.\ \Qed

By 423S, the $\sigma$-algebra
generated by $\{E_{nkqq'}:n$, $k\in\Bbb N$, $q$, $q'\in\Bbb Q\}$
is the whole of
$\Cal B$, and $\Upsilon$ must be equal to $\Cal B$, as claimed.
}%end of proof of 448R

\leader{448S}{Mackey's theorem}\dvAnew{2011} ({\smc Mackey 62})
Let $G$ be a locally compact Polish group,
$(X,\Sigma)$ a standard Borel space
and $\mu$ a $\sigma$-finite measure with domain $\Sigma$.
Let $(\frak A,\bar\mu)$ be the measure algebra of $\mu$
with its measure-algebra topology.   Let $\varaction$
be a Borel measurable action of $G$ on $\frak A$ such that
$a\mapsto g\varaction a$ is a Boolean automorphism for every $g\in G$.
Then we have a $(\Cal B(G)\tensorhat\Sigma,\Sigma)$-measurable action
$\action$ of $G$ on $X$ such that

\Centerline{$g\varaction E^{\ssbullet}=(g\action E)^{\ssbullet}$}

\noindent for every $g\in G$ and $E\in\Sigma$,
writing $g\action E$ for $\{g\action x:x\in E\}$ as usual.

\proof{{\bf (a)} To begin with (down to the end of (j) below) suppose
that $X=\Bbb R$,
with $\Sigma=\Cal B(\Bbb R)$ its Borel $\sigma$-algebra, and that $\mu$ is
totally finite.   The first thing to note is
that for every $g\in G$ the automorphism $a\mapsto g\varaction a$ can be
represented by a Borel automorphism $f_g:\Bbb R\to\Bbb R$ such that
$g\varaction E^{\ssbullet}=f_g^{-1}[E]^{\ssbullet}$ for every
$E\in\Cal B(\Bbb R)$ (425Ac).   Of course $f_g$ belongs to the
space $L^0(\Sigma)$ of $\Sigma$-measurable
functions from $\Bbb R$ to itself, so we can speak of its
equivalence class
$f_g^{\ssbullet}\in L^0(\mu)$.   If we give $L^0(\mu)$ its topology of
convergence in measure, it is a Polish space (448Ra).

The function $g\mapsto f_g^{\ssbullet}:G\to L^0(\mu)$ is Borel measurable.
\Prf\ If $E\in\Cal B(\Bbb R)$, $a\in\frak A$
and $\alpha\in\Bbb R$, then, setting $b=E^{\ssbullet}$,

\Centerline{$\Bvalue{f_g^{\ssbullet}\in E}=f_g^{-1}[E]^{\ssbullet}
=g\varaction b$}

\noindent for every $g\in G$, so

\Centerline{$\{g:\bar\mu(a\Bcap\Bvalue{f_g^{\ssbullet}\in E})>\alpha\}
=\{g:\bar\mu(a\Bcap(g\varaction b))>\alpha\}$}

\noindent is a Borel set in $G$, because $\varaction$ is Borel
measurable and $\{c:\bar\mu(a\Bcap c)>\alpha\}$ is open in $\frak A$.
Thus $\Cal B(G)$ contains the inverse images of the sets generating
$\Cal B(L^0(\mu))$ described in 448Rb,
and therefore the inverse image of every set in $\Cal B(L^0(\mu))$,
as required.\ \Qed

\medskip

{\bf (b)} By 448Q, there is a function
$T:L^0(\mu)\to L^0(\Sigma)$
such that $(Tu)^{\ssbullet}=u$ for every $u\in L^0(\mu)$ and
$(u,x)\mapsto (Tu)(x)$ is
$(\Cal B(L^0(\mu))\tensorhat\Cal B(\Bbb R))$-measurable.
Define $\phi:G\times\Bbb R\to\Bbb R$ by setting

\Centerline{$\phi(g,x)=(Tf_g^{\ssbullet})(x)$}

\noindent for $g\in G$ and $x\in\Bbb R$.   Then $\phi$ is a composition of
the Borel measurable functions $(g,x)\mapsto(f_g^{\ssbullet},x)$ and
$(u,x)\mapsto(Tu)(x)$, so is Borel measurable;  and if $g\in G$ then
$\phi(g,x)=f_g(x)$ for
$\mu$-almost every $x$, because $f_g\eae Tf_g^{\ssbullet}$.   Now

\Centerline{$g\varaction E^{\ssbullet}
=(f_g^{-1}[E])^{\ssbullet}
=\{x:\phi(g,x)\in E\}^{\ssbullet}$}

\noindent for every $g\in G$ and $E\in\Cal B(\Bbb R)$.

\medskip

{\bf (c)} Let $\lambda$ be a Haar measure on $G$.   Because $G$ is a Polish
space and $\lambda$ is a Radon measure on $G$, $\lambda$ is
$\sigma$-finite (411Ge) and $L^0(\lambda)$, with its
topology of convergence in measure, is a Polish space (448Ra again).
For $x\in\Bbb R$, set
$\phi_x(g)=\phi(g,x)$ for $g\in G$;  then $\phi_x:G\to\Bbb R$ is Borel
measurable.    Set $\theta(x)=\phi_x^{\ssbullet}$ in $L^0(\lambda)$.
Then $\theta:\Bbb R\to L^0(\lambda)$ is Borel measurable.
\Prf\ Again I use the
characterization of the Borel $\sigma$-algebra of $L^0(\lambda)$ in
448Rb.   Let $(\frak C,\bar\lambda)$ be the measure algebra of
$\lambda$.
If $c\in\frak C$, $\bar\lambda c<\infty$, $E\subseteq\Bbb R$ is Borel, and
$\alpha\in\Bbb R$, take a Borel set $F\subseteq G$ such that
$c=F^{\ssbullet}$;  then

$$\eqalign{\{x:\bar\lambda(c\Bcap\Bvalue{\theta(x)\in E})>\alpha\}
&=\{x:\lambda(F\cap\phi_x^{-1}[E])>\alpha\}\cr
&=\{x:\lambda\{g:g\in F,\,\phi_x(g)\in E\}>\alpha\}\cr
&=\{x:\lambda\{g:g\in F,\,\phi(g,x)\in E\}>\alpha\}\cr
&=\{x:\lambda W^{-1}[\{x\}]>\alpha\}\cr}$$

\noindent where $W=\{(g,x):g\in F$, $x\in\Bbb R$, $\phi(g,x)\in E\}$ is a
Borel subset of $G\times\Bbb R$.   But this means that
$W\in\Cal B(G)\tensorhat\Cal B(\Bbb R)$ (4A3Ga) and
$x\mapsto\lambda W^{-1}[\{x\}]$ is Borel measurable (252P),
so $\{x:\bar\lambda(c\Bcap\Bvalue{\theta(x)\in E})>\alpha\}$ is a Borel
subset of $\Bbb R$.   Thus the inverse image of every set in the generating
family for the Borel $\sigma$-algebra of $L^0(\lambda)$
is a Borel set, and we have a Borel measurable function.\ \Qed

Let $\nu$ be the totally finite Borel measure on $L^0(\lambda)$ defined
by setting
$\nu F=\mu\theta^{-1}[F]$ for every Borel set $F\subseteq L^0(\lambda)$.

\medskip

{\bf (d)} If $E\subseteq\Bbb R$ is Borel, there is a set
$A\subseteq L^0(\lambda)$ such that $E\symmdiff\theta^{-1}[A]$ is
$\mu$-negligible.   \Prf\ Let $\hat\mu$,
$\hat\nu$ be the completions of $\mu$ and $\nu$, so that $\theta$ is \imp\
for $\hat\mu$ and $\hat\nu$ (234Ba).
Because $E$ is a Borel subset of $\Bbb R$,
$\theta[E]$ is an analytic subset
of $L^0(\lambda)$ (423Gb), therefore Souslin-F (423Eb);
accordingly $\hat\nu$ measures $\theta[E]$
(431B).   Let $\Tau_0$ be the
$\sigma$-algebra of subsets of $L^0(\lambda)$ generated by the Souslin-F
subsets of $L^0(\lambda)$.   By 423O, there is a $\Tau_0$-measurable
function $\theta':\theta[E]\to E$ such that $\theta\theta'$ is the identity
on $\theta[E]$.   Now $\Tau_0$ is included in the domain $\hat\Tau$ of
$\hat\nu$, so $\theta'$ is $\hat\Tau$-measurable, and there is a
Borel set $F_0\subseteq\theta[E]$ such that
$\theta'\restr F_0$ is Borel measurable and
$\theta[E]\setminus F_0$ is $\hat\nu$-negligible (212Fa).
Since $\theta'$ is
surely injective, $E_0=\theta'[F_0]$ is a Borel subset of $E$ (423Ib)
and $\theta\restr E_0$ is a bijection from $E_0$ to $F_0$ with inverse
$\theta'$.   Note that

\Centerline{$\mu(E\setminus\theta^{-1}[F_0])
\le\hat\mu(\theta^{-1}[\theta[E]]\setminus\theta^{-1}[F_0])
=\hat\nu(\theta[E]\setminus F_0)
=0$.}

Define $\psi:\Bbb R\to\Bbb R$ by setting

$$\eqalign{\psi(x)
&=\theta'\theta(x)\text{ if }x\in\theta^{-1}[F_0],\cr
&=x\text{ otherwise}.\cr}$$

\noindent Then $\psi$ is a Borel measurable
function and $\theta\psi=\theta$, that is,
$\phi_x^{\ssbullet}=\phi_{\psi(x)}^{\ssbullet}$ for every $x\in\Bbb R$.
Consequently

\Centerline{$\{(g,x):g\in G$, $x\in\Bbb R$,
$\phi(g,x)\ne\phi(g,\psi(x))\}$}

\noindent has $\lambda$-negligible horizontal sections.
Since it is a Borel set, it must have many $\mu$-negligible vertical
sections;  let $g_0\in G$ be such that
$\{x:\phi(g_0,x)\ne\phi(g_0,\psi(x))\}$ is $\mu$-negligible.
By (b), we also have $\phi(g_0,x)=f_{g_0}(x)$ for $\mu$-almost every $x$.
So the Borel set
$H=\{x:f_{g_0}(x)=\phi(g_0,x)=\phi(g_0,\psi(x))\}$
is $\mu$-conegligible.

Set $A=\theta[E\cap\theta^{-1}[F_0]\cap H]$.
Of course $A\subseteq F_0$.   If $x\in\theta^{-1}[A]\setminus E$,
then there is a
$y\in E\cap\theta^{-1}[F_0]\cap H$ such that $\theta(y)=\theta(x)$;  now
$\psi(y)=\psi(x)$ and $x\ne y$, so

\Centerline{$\phi(g_0,\psi(x))=\phi(g_0,\psi(y))=f_{g_0}(y)\ne f_{g_0}(x)$}

\noindent and $x\notin H$.   Thus
$\theta^{-1}[A]\setminus E\subseteq\Bbb R\setminus H$ is $\mu$-negligible.
On the other hand,
$E\setminus\theta^{-1}[A]\subseteq E\setminus(\theta^{-1}[F_0]\cap H)$
is also $\mu$-negligible.   So
$E\symmdiff\theta^{-1}[A]$ is $\mu$-negligible, as required.\ \Qed

\medskip

{\bf (e)} There is a $\mu$-conegligible Borel set
$H\subseteq\Bbb R$ such that
$\theta\restr H$ is injective.   \Prf\ Let $\sequencen{E_n}$ be a sequence
of Borel sets in $\Bbb R$ such that whenever $x$, $y\in\Bbb R$ are
distinct there is an $n$ such that $x\in E_n$ and $y\notin E_n$.
For each $n\in\Bbb N$
let $A_n\subseteq L^0(\lambda)$ be such that
$E_n\symmdiff\theta^{-1}[A_n]$ is
$\mu$-negligible;  let $H$ be a $\mu$-conegligible Borel set disjoint from
$\bigcup_{n\in\Bbb N}(E_n\symmdiff\theta^{-1}[A_n])$.   If $x$, $y\in H$
are distinct, there is an $n\in\Bbb N$ such that $x\in E_n$ and
$y\notin E_n$;  now $x\in\theta^{-1}[A_n]$ and $y\notin\theta^{-1}[A_n]$,
so $\theta(x)\ne\theta(y)$.\ \Qed

\medskip

{\bf (f)} Of course $\theta[H]$ is now a Borel subset of $L^0(\lambda)$,
and must be
$\hat\nu$-conegligible.   Let $\frak B$ be the measure algebra of $\nu$,
and $\pi:\frak B\to\frak A$ the measure-preserving homomorphism defined by
setting $\pi F^{\ssbullet}=\theta^{-1}[F]^{\ssbullet}$ for every Borel set
$F$.   If $E\subseteq\Bbb R$ is Borel, then
$E^{\ssbullet}=\pi(\theta[E\cap H])^{\ssbullet}$ belongs to $\pi[\frak B]$,
so $\pi$ is surjective and is an isomorphism.

\medskip

{\bf (g)} Recall that
we have a continuous action $\action_l$ of $G$ on $L^0(\lambda)$
defined as in 443G.   If $g\in G$, then

\Centerline{$g\action_l\theta(x)=\theta(\phi(g^{-1},x))
=\theta(f_{g^{-1}}(x))$}

\noindent for $\mu$-almost
every $x\in\Bbb R$.   \Prf\ Consider the set
$\{(h,x):\phi(g^{-1}h,x)=\phi(h,\phi(g^{-1},x))\}
\subseteq G\times\Bbb R$.
Because $h\mapsto g^{-1}h$ is continuous, it is Borel measurable,
so $(h,x)\mapsto\phi(g^{-1}h,x)$ is Borel measurable;
the same is true of $(h,x)\mapsto\phi(h,\phi(g^{-1},x))$, so
$\{(h,x):\phi(g^{-1}h,x)=\phi(h,\phi(g^{-1},x))\}$ is a Borel set.
For given $h\in G$ and $E\in\Cal B(\Bbb R)$, set
$F=\{x:\phi(h,x)\in E\}$;   then

$$\eqalign{\{x:\phi(g^{-1}h,x)\in E\}^{\ssbullet}
&=(g^{-1}h)\varaction E^{\ssbullet}
=g^{-1}\varaction(h\varaction E^{\ssbullet})
=g^{-1}\varaction\{x:\phi(h,x)\in E\}^{\ssbullet}\cr
&=g^{-1}\varaction F^{\ssbullet}
=\{x:\phi(g^{-1},x)\in F\}^{\ssbullet}
=\{x:\phi(h,\phi(g^{-1},x))\in E\}^{\ssbullet}\cr}$$

\noindent so
$\{x:\phi(g^{-1}h,x)\in E\}\symmdiff\{x:\phi(h,\phi(g^{-1},x))\in E\}$
is $\mu$-negligible.   As $E$ is arbitrary,
$\phi(g^{-1}h,x)=\phi(h,\phi(g^{-1},x))$ for $\mu$-almost
every $x$.

This is true for every $h\in G$.   So there is a
$\mu$-conegligible Borel set $H'\subseteq\Bbb R$ such that
if $x\in H'$ then
$\phi(g^{-1}h,x)=\phi(h,\phi(g^{-1},x))$ for $\lambda$-almost every $h$.
But this means that if $x\in H'$ then

\Centerline{$(g\action_l\phi_x)(h)
=\phi_x(g^{-1}h)
=\phi(g^{-1}h,x)
=\phi(h,\phi(g^{-1},x))
=\phi_{\phi(g^{-1},x)}(h)$}

\noindent for $\lambda$-almost every $h$, and

\Centerline{$g\action_l\theta(x)
=g\action_l\phi_x^{\ssbullet}
=(g\action_l\phi_x)^{\ssbullet}
=\phi_{\phi(g^{-1},x)}^{\ssbullet}
=\theta(\phi(g^{-1},x))$.}

\noindent Thus $g\action_l\theta(x)=\theta(\phi(g^{-1},x))$ for almost
every $x$.   And of course we already know from (b) that
$\phi(g^{-1},x)=f_{g^{-1}}(x)$ for almost every $x$.\ \Qed

\medskip

{\bf (h)} We have a function $\varaction_l:G\times\frak B\to\frak B$
defined by setting

\Centerline{$g\smallcirc_lF^{\ssbullet}=(g\action_lF)^{\ssbullet}$}

\noindent for every Borel set $F\subseteq L^0(\lambda)$ and $g\in G$,
writing
$g\action_lF=\{g\action_lu:u\in F\}$ as in 4A5Bc.   \Prf\ Take any
$g\in G$.   By (g) just above, applied to $g^{-1}$,
$g^{-1}\action_l\theta(x)=\theta(f_g(x))$
for $\mu$-almost every $x$.   Because the shift
operator $u\mapsto g\action_lu:L^0(\lambda)\to L^0(\lambda)$ is a
homeomorphism, it is a
Borel automorphism, and $g\action_lF$ is a Borel set for every Borel set
$F\subseteq L^0(\lambda)$.   If $\nu F=0$, then

$$\eqalign{\nu(g\action_lF)
&=\mu\{x:\theta(x)\in g\action_lF\}
=\mu\{x:g^{-1}\action_l\theta(x)\in F\}\cr
&=\mu\{x:\theta(f_g(x))\in F\}
=\mu(f_g^{-1}[\theta^{-1}[F]])
=0\cr}$$

\noindent because $\theta^{-1}[F]$ is $\mu$-negligible and $f_g$
represents an automorphism of the measure algebra $\frak A$ of $\mu$.
It follows that
$(g\action_lF_0)\symmdiff(g\action_lF_1)=g\action_l(F_0\symmdiff F_1)$ is
$\nu$-negligible and
$(g\action_lF_0)^{\ssbullet}=(g\action_lF_1)^{\ssbullet}$ whenever
$F_0^{\ssbullet}=F_1^{\ssbullet}$, which is what we need to know.\ \Qed

\medskip

{\bf (i)} For any $b\in\frak B$ and $g\in G$,
$g\varaction\pi b=\pi(g\varaction_lb)$.   \Prf\ Let
$F\subseteq L^0(\lambda)$ be
a Borel set such that $b=F^{\ssbullet}$, and set $E=\theta^{-1}[F]$,
$a=E^{\ssbullet}=\pi b$.   Then
$g\varaction_lb=(g\action_lF)^{\ssbullet}$, so

$$\eqalign{\pi(g\varaction_lb)
&=\{x:\theta(x)\in g\action_lF\}^{\ssbullet}
=\{x:g^{-1}\action_l\theta(x)\in F\}^{\ssbullet}
=\{x:\theta(f_g(x))\in F\}^{\ssbullet}\cr
&=\{x:f_g(x)\in E\}^{\ssbullet}
=(f_g^{-1}[E])^{\ssbullet}
=g\varaction a
=g\varaction\pi b,\cr}$$

\noindent as required.\ \Qed

\medskip

{\bf (j)} Now observe that because $\lambda$ is a Haar measure,
$\lambda G>0$, so $L^0(\lambda)\ne\{0\}$, $L^0(\lambda)$ is uncountable and
$\#(L^0(\lambda))=\frak c=\#(\Bbb R)$ (423K).   By
425Ad, there is a Borel isomorphism
$\tilde\theta:\Bbb R\to L^0(\lambda)$ which represents $\pi$.   Set

\Centerline{$g\action x=\tilde\theta^{-1}(g\action_l\tilde\theta(x))$}

\noindent for $g\in G$ and $x\in\Bbb R$.   Then
$\action:G\times\Bbb R\to\Bbb R$ is a composition of the Borel measurable
functions $(g,x)\mapsto(g,\tilde\theta(x))$, $(g,u)\mapsto g\action_lu$ and
$u\mapsto\tilde\theta^{-1}(u)$, so is Borel measurable.
Because $\Sigma=\Cal B(\Bbb R)$ and
$\Cal B(G\times\Bbb R)=\Cal B(G)\tensorhat\Cal B(\Bbb R)$ (4A3Ga again),
$\action$ is $(\Cal B(G)\tensorhat\Sigma,\Sigma)$-measurable.   If $g$,
$h\in G$ and $x\in\Bbb R$,

\Centerline{$gh\action x
=\tilde\theta^{-1}(gh\action_l\tilde\theta(x))
=\tilde\theta^{-1}(g\action_l(h\action_l\tilde\theta(x)))
=\tilde\theta^{-1}(g\action_l\tilde\theta(h\action x))
=g\action(h\action x)$,}

\noindent and if $e$ is the identity of $G$ and $x\in\Bbb R$,

\Centerline{$e\action x
=\tilde\theta^{-1}(e\action_l\tilde\theta(x))
=\tilde\theta^{-1}\tilde\theta(x)
=x$.}

\noindent Thus $\action$ is an action of $G$ on $\Bbb R$.   If
$g\in G$ and $E\in\Cal B(\Bbb R)$, set $F=\tilde\theta[E]$.   Then
$F^{\ssbullet}=\pi^{-1}[E^{\ssbullet}]$, so

\Centerline{$(\tilde\theta^{-1}[g\action_lF])^{\ssbullet}
=\pi((g\action_lF)^{\ssbullet})
=\pi(g\varaction_lF^{\ssbullet})
=g\varaction\pi F^{\ssbullet}
=g\varaction E^{\ssbullet}$.}

\noindent As

\Centerline{$g\action E
=\{g\action x:x\in E\}
=\{\tilde\theta^{-1}(g\action_l\tilde\theta(x)):x\in E\}
=\{\tilde\theta^{-1}(g\action_l u):u\in F\}
=\tilde\theta^{-1}[g\action_l F]$,}

\noindent we see that

\Centerline{$g\varaction E^{\ssbullet}=(g\action E)^{\ssbullet}$}

\noindent as required in the statement of this theorem.

\medskip

{\bf (k)} Thus the result is true if
$(X,\Sigma)=(\Bbb R,\Cal B(\Bbb R))$ and $\mu$ is totally finite.   As for
non-totally-finite $\mu$, there will always be a totally finite measure
$\mu_1$ with the same domain and the same null ideal (215B again), in which
case the measure algebra of $\mu_1$ will have the same Boolean algebra
$\frak A$, though with a different measure $\bar\mu_1$.   However the
measure-algebra topology of $\frak A$ is unchanged (324H), so
$\varaction$ is still Borel measurable, and we can use the Borel
measurable action of $G$ on $\Bbb R$ found by the method of (a)-(j) above.
Since we are assuming that $(X,\Sigma)$ is a standard Borel space,
this covers all the cases in which $X$ is uncountable, by 424C-424D.

\medskip

{\bf (l)} We are left with the case of countable $X$.   This is of course
essentially trivial.
$\Sigma=\Cal PX$ and $\mu$ is a point-supported measure.   Let $Y$ be the
set of atoms of $\mu$, that is, the set $\{x:\mu\{x\}>0\}$.   Then we can
identify the measure algebra $\frak A=\Cal PX/\Cal P(X\setminus Y)$
with $\Cal PY$, in which case the equivalence class $E^{\ssbullet}$ of any
$E\subseteq X$ becomes identified with $E\cap Y$.
As in part (a) of the proof above, we can represent each automorphism
$a\mapsto g\varaction a:\frak A\to\frak A$ by a
permutation $f_g:X\to X$, and we must have
$f_g^{-1}[Y]=Y$.   Try

$$\eqalign{g\action x
&=f_g^{-1}(x)\text{ if }x\in Y,\cr
&=x\text{ if }x\in X\setminus Y\cr}$$

\noindent for every $g\in G$.   If $g$, $h\in G$ and $x\in Y$,

$$\eqalign{\{gh\action x\}
&=\{f_{gh}^{-1}(x)\}
=gh\varaction\{x\}
=g\varaction(h\varaction\{x\})
=g\varaction f_h^{-1}[\{x\}]\cr
&=f_g^{-1}[f_h^{-1}[\{x\}]]
=f_g^{-1}[\{f_h^{-1}(x)\}]
=f_g^{-1}\{h\action x\}
=\{g\action(h\action x)\}\cr}$$

\noindent and $gh\action x=g\action(h\action x)$;  if $x\in X\setminus Y$,
then $gh\action x=x=g\action(h\action x)$.
Of course $e\action x=x$ for every $x\in X$.
So $\action$ is an action of
$G$ on $X$.   To see that it is
$(\Cal B(G)\tensorhat\Sigma,\Sigma)$-measurable, note that the
measure-algebra topology of $\frak A\cong\Cal PY$ is the discrete topology.
If $y\in Y$, then $\{g:g\action y=z\}=\{g:g\varaction\{y\}=\{z\}\}$ is a
Borel set for every $z\in X$;  if $x\in X\setminus Y$, then
$\{g:g\action x=z\}$ is either $G$ or $\emptyset$ for every $z\in X$.
So

\Centerline{$\{(g,x):g\action x\in W\}
=\bigcup_{z\in W,x\in X}\{g:g\action x=z\}\times\{x\}$}

\noindent belongs to $\Cal B(G)\tensorhat\Sigma$
for every subset $W$ of $X$, and
$\action$ is $(\Cal B(G)\tensorhat\Sigma,\Sigma)$-measurable.
Finally, if $g\in G$,

\Centerline{$(g\action E)^{\ssbullet}
=(g\action E)\cap Y
=Y\cap f_g^{-1}[E]
=(f_g^{-1}[E])^{\ssbullet}
=g\varaction E^{\ssbullet}$}

\noindent for every $E\subseteq X$.   So again we have a suitable action of
$G$ on $X$.

This completes the proof.
}%end of proof of 448S

\leader{448T}{Corollary}\dvAnew{2011} Let $G$ be a $\sigma$-compact
locally compact Hausdorff group, $X$ a Polish space,
$\mu$ a $\sigma$-finite Borel measure on $X$,
and $(\frak A,\bar\mu)$ the measure algebra of $\mu$,
with its measure-algebra topology.   Let $\varaction$
be a continuous action of $G$ on $\frak A$ such that
$a\mapsto g\varaction a$ is a Boolean automorphism for every $g\in G$.
Then we have a Borel measurable action
$\action$ of $G$ on $X$ such that

\Centerline{$g\varaction E^{\ssbullet}=(g\action E)^{\ssbullet}$}

\noindent for every $g\in G$ and $E\in\Cal B(X)$.

\proof{ We know that $\frak A$ is separable (331O again)
and metrizable (323Gb);   let $\sequencen{a_n}$ run over a
topologically dense subset of $\frak A$ and $\sequencen{U_n}$ over a base
for its topology.   For each $(m,n)$ such that $a_m\in U_n$,
$V_{mn}=\{g:g\varaction a_m\in U_n\}$ is a neighbourhood of the identity
$e$ of $G$.   By 4A5S, there is a compact normal subgroup $H$ of $G$ such
that $H\subseteq\bigcap\{V_{mn}:m$, $n\in\Bbb N$, $a_m\in U_n\}$
and $G/H$ is Polish.
Now we have a continuous action $\bar{\varaction}$ of $G/H$ on $\frak A$
such that $g^{\ssbullet}\bar{\varaction} a=g\varaction a$ for every
$g\in G$ and $a\in\frak A$.   \Prf\ If $g$, $h\in G$ are such that
$g^{\ssbullet}=h^{\ssbullet}$, $m\in\Bbb N$, and $a_m\in U_n$, then
$g^{-1}h\in V_{mn}$ so $g^{-1}h\varaction a_m\in U_n$.
As $n$ is arbitrary, $g^{-1}h\varaction a_m=a_m$;  as
$a\mapsto g^{-1}h\varaction a$ is continuous, and $m$ is arbitrary,
$g^{-1}h\varaction a=a$ and $h\varaction a=g\varaction a$ for every
$a\in\frak A$.   This shows that the given formula defines a function
$\bar{\varaction}$ from $(G/H)\times\frak A$ to $\frak A$.   It is easy to
check that $\bar{\varaction}$ is an action of $G/H$ on $\frak A$.

Now suppose that
$v\in G/H$, $a\in\frak A$ and $U$ is a neighbourhood of
$v\bar{\varaction}a$
in $\frak A$.   Let $g\in G$ be such that $g^{\ssbullet}=v$;  then
$g\varaction a=v\bar{\varaction} a$, so there are
open sets $V\subseteq G$ and $U'\subseteq\frak A$ such that $g\in V$,
$a\in U'$ and $h\varaction b\in U$ whenever $h\in V$ and $b\in U'$.
By 4A5Ja, $W=\{h^{\ssbullet}:h\in V\}$ is open in $G/H$;  now
$v\in W$ and $w\bar{\varaction} b\in U$ whenever $w\in W$ and
$b\in U'$.   As $v$, $a$ and $U$ are arbitrary, $\bar{\varaction}$ is
continuous.\ \Qed

There is therefore a Borel measurable action
$\bar{\action}:(G/H)\times X\to X$ such that
$v\bar{\varaction} E^{\ssbullet}=(v\bar{\action}E)^{\ssbullet}$ whenever
$v\in G/H$ and $E\in\Cal B(X)$ (448S).
Set $g\action x=g^{\ssbullet}\bar{\action}x$ for $g\in G$ and $x\in X$.
It is elementary to check that $\action$ is an action of $G$ on $X$.
Also it is Borel measurable, because $(g,x)\mapsto(g^{\ssbullet},x)$ is
continuous, therefore Borel measurable, and
$(g^{\ssbullet},x)\mapsto g^{\ssbullet}\bar{\action}x$ is Borel measurable.
If $g\in G$ and $E\in\Cal B(X)$, then

\Centerline{$g\varaction E^{\ssbullet}
=g^{\ssbullet}\bar{\varaction} E^{\ssbullet}
=(g^{\ssbullet}\bar{\action}E)^{\ssbullet}
=(g\action E)^{\ssbullet}$,}

\noindent so $\action$ is an action of the kind we seek.
}%end of proof of 448T

\exercises{
\leader{448X}{Basic exercises (a)}
%\spheader 448Xa
Show that the results in 448Fb and 448Fd remain true if $G$ is not
assumed to be countable.
%448F

\spheader 448Xb
In part (c) of the proof of 448O, show that $\Cal I$ is just the set of
those $d\in\frak A$ such that $d\Bsubseteq\upr(1\Bsetminus a,\frak C)$
for some $a$ such that $1\psG a$.
%448O

\spheader 448Xc   Show that, in part (c) of the proof of 448P, we can
if we wish take $Z=X$ and $\Sigma=\Cal B$.
%448P

\sqheader 448Xd Let $(X,\Sigma)$ be a standard Borel space and
$\Sigma_0$ a countable subalgebra of $\Sigma$.   Show that there is a
sequence $\sequencen{\sequence{i}{E_{ni}}}$ of partitions of unity in
$\Sigma$ such that whenever $\nu:\Sigma\to\Bbb R$ is a finitely additive
functional and $\nu X=\sum_{i=0}^{\infty}\nu E_{ni}$ for every $n\in\Bbb
N$, then $\nu\restr\Sigma_0$ is countably additive.
%448P

\sqheader 448Xe  Set $X=[0,1]\setminus\Bbb Q$, $G=\Bbb Q$ and define
$\action:G\times X\to X$ by requiring that $g\action x-g-x\in\Bbb Z$ for
$g\in G$ and $x\in X$.   Show that this is a Borel measurable action and that Lebesgue
measure on $X$ is $G$-invariant.   Find a metric on $X$, inducing its
topology, for which all the maps $x\mapsto g\action x$ are isometries.
%448P

\sqheader 448Xf Show that a Polish group carries Haar measures iff it is
locally compact.   \Hint{443E.}

\spheader 448Xg Give $\Bbb Z^{\Bbb N}$ its usual (product) topology and
abelian group structure.   Show that it is a Polish group,
and has no Haar measure.
%448P

\sqheader 448Xh Let $(X,\rho)$ be a metric space, $G$ a group and
$\action$ an action of $G$ on $X$ such that $x\mapsto g\action x$ is an
isometry for every $g\in G$.   (i) Show that if $\mu$ is a $G$-invariant
quasi-Radon probability measure on $X$ then $\{g\action x:g\in G\}$ is
totally bounded for every $x$ in the support of $\mu$.   (ii) Show that
if the action is transitive and there is a non-zero $G$-invariant
quasi-Radon measure on $X$, then $X$ is covered by totally bounded open
sets.   (iii) Suppose that $X$ has measure-free weight (see \S438;  for
instance, $X$ could be separable).   Show that if the action is
transitive and there is a $G$-invariant topological
probability measure on $X$ then $X$ is totally bounded.
%448P

\spheader 448Xi Let $(\frak A,\bar\mu)$ be a measure algebra, with the
operation $\Bsymmdiff$ and the measure-algebra topology.   (i) Show that
$\frak A$ is a topological group.   (ii) Show that if $\bar\mu$ is
$\sigma$-finite and $\frak A$ has countable Maharam type, it is a Polish
group.   (iii) Show that if $(\frak A,\bar\mu)$ is semi-finite and not
purely atomic, then $\frak A$ has no Haar measure.
%448Xh 448P

\spheader 448Xj Let $(\frak A,\bar\mu)$ be the measure algebra of
Lebesgue measure on $[0,1]$, with its measure metric, and
$G=\Aut_{\bar\mu}\frak A$ the group of measure-preserving automorphisms
on $\frak A$.   (i) Show that if we give $G$ the topology induced by the
topology of pointwise convergence on
the isometry group of $\frak A$, then it is a
Polish group.   \Hint{441Xp(iv).}  (ii) Show that if $\nu$ is a
$G$-invariant topological probability measure on $\frak A$, then
$\nu\{0,1\}=1$.
%448Xh 448P

\leader{448Y}{Further exercises (a)}
%\spheader 448Ya
Let $\frak A$ be a Dedekind $\sigma$-complete Boolean algebra, and $G$ a
subgroup of $\Aut\frak A$;  let $G^*_{\sigma}$ be the countably full
local semigroup generated by $G$, and write $H$ for the union of all the
full local semigroups generated by countable subgroups of $G$ (following
the definition in 395A as written, without troubling about whether
$\frak A$ is Dedekind complete).   (i) Show that
$G^*_{\sigma}\subseteq H$.   (ii) Find an example in which
$H\ne G^*_{\sigma}$.   (iii) Show
that if $\frak A$ is Dedekind complete then $G^*_{\sigma}=H$.   (iv)
Show that if $\frak A$ is ccc then $G^*_{\sigma}=H$ is the full local
semigroup generated by $G$.
%448A %mt44bits

\spheader 448Yb In 448N, show that $\theta$ is uniquely defined.
%448N

\spheader 448Yc Let $(X,\Sigma)$ be a standard Borel space, $Y$ any
set, $\Tau$ a $\sigma$-algebra of subsets of $Y$ and $\Cal J$ a
$\sigma$-ideal of subsets of $\Tau$.
Let $\theta:\Sigma\to L^{\infty}(\Tau/\Cal J)$ be a non-negative,
sequentially
order-continuous additive function.   Show that there is a
non-negative, sequentially order-continuous additive function
$\phi:\Sigma\to L^{\infty}(\Tau)$ such that (identifying
$L^{\infty}(\Tau/\Cal J)$ with a quotient space of $L^{\infty}(\Tau)$)
$\theta E=(\phi E)^{\ssbullet}$ for every $E\in\Sigma$.
%448P
}%end of exercises

%\query:  Mackey's theorem for semigroup actions?

\cmmnt{\Notesheader{448} The keys to the first part of the section
are in 448F, 448G and 448L.   Even though we no longer have
a Dedekind complete algebra, the fact that we are working with countable
groups means that the suprema we actually need are defined.   The final
step, however, uses yet another idea.   In a standard Borel space, given
a finitely additive functional on the $\sigma$-algebra, we can sometimes
confirm an adequate approximation to countable additivity by looking at
only countably many sequences
(448Xd).   This enables us to pass from a $G$-invariant map
$\theta:\frak A\to L^{\infty}(\frak D)$ to a $G$-invariant Radon measure
(parts (d)-(f) of the proof of 448P), without needing to know
anything about the algebra $\frak D$ except that it is Dedekind
$\sigma$-complete.   In particular (and in contrast to the corresponding
step in 395P) we do not need to suppose that $\frak D$ is a measurable
algebra.   I do not know whether there is a useful ergodicity condition
which could be added to the hypotheses of 448O to ensure that $\frak D$
there becomes $\{0,1\}$.

448P was proved in the case $G=\Bbb Z$ by {\smc Nadkarni 90};  the
extension to general Borel actions by Polish groups is due to
{\smc Becker \& Kechris 96}.   (See {\smc Nadkarni 90} for notes on the
history of the problem, and {\smc Kechris 95} for the basic general
theory of Polish groups and Borel actions.)
It is a remarkable result, but its application is limited by the
difficulty of determining whether either condition (ii) or condition
(iii) is satisfied.   Much commoner situations are those like
448Xe-448Xj, %448Xe 448Xf 448Xg 448Xh 448Xi 448Xj
where either there is no invariant measure or we can find one easily.

The second main theorem makes no reference to the first.   But it has
something in common.   It is an example of the power of descriptive set
theory to dramatically extend a result on group actions, which is
comparatively straightforward when the group in question is $\Bbb Z$, to
arbitrary Polish groups.   Nadkarni's theorem is not obvious, but it is a
lot easier than the general result here.   Mackey's theorem for countable
groups also requires a little care, but is essentially covered (in usefully
greater generality) by 344C.   The descriptive set theory
the theorem here relies on does
not go as deep as the Becker-Kechris theorem, but in exchange it calls on a
kind of lifting theorem quite different from those in Chapter 34.
Looked at from the standpoint of Chapter 34, 448Q is a
rank impossibility (see 341Xg);
but the point is that we have abandoned the ordinary
algebraic requirements on a lifting and replaced them by a strong
measurability property.

Of course a lifting was used in 344C as well, but in a quite different way.
There the hypotheses were adjusted to give a slightly more general context
in which we could be sure that individual homomorphisms from the measure
algebra to itself were representable by functions from the measure space to
itself;  and I relied indirectly on the lifting theorem 341K
to set up the functions.   For the context of the present section,
this step was done in 425A with no mention of
liftings, but using the classification of standard Borel spaces in 424D.
In view of 424Yf, it is plain that we do not get much extra generality by
using the argument through 341K.   The real difference in 344B-344C is that
we can deal with semigroups of homomorphisms as well as groups of
automorphisms.

The proof of Mackey's theorem is based on there being a Haar measure on
$G$, so that we can use Fubini's theorem
(three times, in parts (c), (d) and (g) of the
proof).   There are non-locally-compact groups $G$ for which a
corresponding result is true ({\smc Kwiatowska \& Solecki 11});
it remains quite unclear when to expect this.
}%end of notes

\discrpage


