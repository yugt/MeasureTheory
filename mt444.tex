\frfilename{mt444.tex}
\versiondate{23.7.07/4.6.13}
\copyrightdate{1999}

\def\esssup{\mathop{\text{ess sup}}}
\def\ssplrarrow{^{\scriptscriptstyle\leftrightarrow}}

\def\chaptername{Topological groups}
\def\sectionname{Convolutions}

\newsection{444}

In this section, I look again at the ideas of \S\S255 and 257, seeking
the appropriate generalizations to topological groups other than
$\Bbb R$.   Following {\smc Hewitt \& Ross 63}, I begin with
convolutions of measures (444A-444E) %444A 444B 444C 444D 444E
before proceeding to convolutions of functions
(444O-444V); %444O 444P 444Q 444R 444T 444U 444S 444V
in between, I mention the convolution of a function and a measure
(444G-444M) %444G 444H 444I 444J 444K 444L 444M
and a general result concerning continuous group actions on
quasi-Radon measure spaces (444F).

While I continue to give the results in terms of
real-valued functions, the applications
of the ideas here in the next section will be to complex-valued functions;
so you may wish to keep the complex case in mind\cmmnt{ (444Xx)}.

\leader{444A}{Convolution of measures:  Proposition} If $X$ is a
topological group and $\lambda$ and $\nu$ are two totally finite
quasi-Radon measures on $X$, we have a quasi-Radon measure $\lambda*\nu$
on $X$ defined by saying that

$$\eqalign{(\lambda*\nu)(E)
&=(\lambda\times\nu)\{(x,y):xy\in E\}\cr
&=\int\nu(x^{-1}E)\lambda(dx)
=\int\lambda(Ey^{-1})\nu(dy)\cr}$$

\noindent for every $E\in\dom(\lambda*\nu)$, where $\lambda\times\nu$ is
the quasi-Radon product measure on $X\times X$.

\proof{ Set $\phi(x,y)=xy$ for $x$, $y\in X$.   Then $\phi$ is continuous,
while $X$, being a topological group, is regular (4A5Ha);  so
so 418Hb tells us that there is a unique quasi-Radon measure $\lambda*\nu$
on $X$ such that $\phi$ is \imp\ for $\lambda\times\nu$ and $\lambda*\nu$,
that is, $(\lambda*\nu)(E)=(\lambda\times\nu)\{(x,y):xy\in E\}$ whenever
$E$ is measured by $\lambda*\nu$.

As for the other formulae, Fubini's theorem (417Ha) tells us that

$$\eqalign{(\lambda*\nu)(E)
&=(\lambda\times\nu)\phi^{-1}[E]\cr
&=\int\nu(\phi^{-1}[E])[\{x\}]\lambda(dx)
   =\int\nu(x^{-1}E)\lambda(dx)\cr
&=\int\lambda(\phi^{-1}[E])^{-1}[\{y\}]\nu(dy)
   =\int\lambda(Ey^{-1})\nu(dy)\cr}$$

\noindent for any $E\in\dom(\lambda*\nu)$.
}%end of proof of 444A

\leader{444B}{Proposition} If $X$ is a topological group,
$\lambda_1*(\lambda_2*\lambda_3)=(\lambda_1*\lambda_2)*\lambda_3$ for
all totally finite quasi-Radon measures $\lambda_1$, $\lambda_2$ and
$\lambda_3$ on $X$.

\proof{ If $E\subseteq X$ is Borel, then

$$\eqalign{(\lambda_1*(\lambda_2*\lambda_3))(E)
&=\int(\lambda_2*\lambda_3)(x^{-1}E)\lambda_1(dx)\cr
&=\int(\lambda_2\times\lambda_3)\{(y,z):xyz\in E\}\lambda_1(dx)\cr
&=(\lambda_1\times(\lambda_2\times\lambda_3))\{(x,(y,z)):xyz\in E\}\cr
&=((\lambda_1\times\lambda_2)\times\lambda_3)\{((x,y),z):xyz\in E\}\cr
&=\int(\lambda_1\times\lambda_2)\{(x,y):xyz\in E\}\lambda_3(dz)\cr
&=((\lambda_1*\lambda_2)*\lambda_3)(E).\cr}$$

\noindent (For the central identification between
$\lambda_1\times(\lambda_2\times\lambda_3)$ and
$(\lambda_1\times\lambda_2)\times\lambda_3$, observe that as both are
quasi-Radon measures it is enough to check that they agree on sets of
the type $G_1\times(G_2\times G_3)\cong (G_1\times G_2)\times G_3$ where
$G_1$, $G_2$ and $G_3$ are open, as in 417J.)

So $\lambda_1*(\lambda_2*\lambda_3)$ and
$(\lambda_1*\lambda_2)*\lambda_3$ agree on the Borel sets and must be
identical.
}%end of proof of 444B

\leader{444C}{Theorem} Let $X$ be a topological group and $\lambda$,
$\nu$ two totally finite quasi-Radon measures on $X$.
Then

\Centerline{$\int fd(\lambda*\nu)
=\int f(xy)(\lambda\times\nu)d(x,y)
=\iint f(xy)\lambda(dx)\nu(dy)
=\iint f(xy)\nu(dy)\lambda(dx)$}

\noindent for any $(\lambda*\nu)$-integrable real-valued function $f$.
In particular, $(\lambda*\nu)(X)=\lambda X\cdot\nu X$.

%what about [-\infty,\infty]-valued functions?  see 447

\proof{ If $f$ is of the form $\chi E$, so that

\Centerline{$\int fd(\lambda*\nu)=(\lambda*\nu)(E)$,
\quad$\int f(xy)(\lambda\times\nu)d(x,y)
=(\lambda\times\nu)\{(x,y):xy\in E\}$,}

\Centerline{$\iint f(xy)\lambda(dx)\nu(dy)
=\int\lambda(Ey^{-1})\nu(dy)$,
\quad$\iint f(xy)\nu(dy)\lambda(dx)
=\int\nu(x^{-1}E)\lambda(dx)$}

\noindent this is covered by the result in 444A.   Now it is easy to run
through the standard progression to the cases of (i) simple functions
(ii) non-negative Borel measurable functions defined everywhere (iii)
functions defined, and zero, almost everywhere (iv) non-negative
integrable functions and (v) arbitrary integrable functions.
}%end of proof of 444C

\leader{444D}{Proposition} Let $X$ be an abelian topological group.
Then $\lambda*\nu=\nu*\lambda$ for all totally finite quasi-Radon
measures $\lambda$, $\mu$ on $X$.

\proof{ For any Borel set $E\subseteq X$,

$$\eqalign{(\lambda*\nu)(E)
&=(\lambda\times\nu)\{(x,y):xy\in E\}
=(\nu\times\lambda)\{(y,x):xy\in E\}\cr
&=(\nu\times\lambda)\{(y,x):yx\in E\}
=(\nu*\lambda)(E).\cr}$$
}%end of proof of 444D

\leader{444E}{The Banach algebra of $\tau$-additive
measures (a)} Let $X$ be a topological group.   \cmmnt{Recall from
437Ab that we have a band $C_b(X)^{\sim}_{\tau}$ in the $L$-space
$C_b(X)^{\sim}$
consisting of those order-bounded linear functionals $f:C_b(X)\to\Bbb R$
such that $|f|$
is smooth (equivalently, $f^+$ and $f^-$ are both smooth);  that is,
such that $|f|$, $f^+$ and $f^-$ can be represented by totally finite
quasi-Radon measures on $X$.}
\cmmnt{Because $X$ is completely regular (4A5Ha again),}
$C_b(X)^{\sim}_{\tau}$ can be identified with
the band $M_{\tau}$ of signed $\tau$-additive Borel measures on $X$,
that is, the set of those countably additive functionals $\nu$
defined on the Borel $\sigma$-algebra of $X$ such that $|\nu|$
is $\tau$-additive\cmmnt{ (437G)}.

\spheader 444Eb For any $\tau$-additive totally finite Borel measures
$\lambda$, $\nu$ on $X$ we can define their convolution $\lambda*\nu$ by
\cmmnt{the formulae of 444A, that is,}

\Centerline{$(\lambda*\nu)(E)
=\int\nu(x^{-1}E)\lambda(dx)
=\int\lambda(Ey^{-1})\nu(dy)$}

\noindent for any Borel set $E\subseteq X$\cmmnt{, if we note that the
completions $\hat\lambda$, $\hat\nu$ of $\lambda$ and $\nu$ are
quasi-Radon measures (415Cb), so that $\lambda*\nu$, as defined by these
formulae, is just the restriction of $\hat\lambda*\hat\nu$, as defined
in 444A, to the Borel $\sigma$-algebra}.   Now\cmmnt{ the formulae
make it
obvious that} the map $*$ is bilinear in the sense that

\Centerline{$(\lambda_1+\lambda_2)*\nu=\lambda_1*\nu+\lambda_2*\nu$,}

\Centerline{$\lambda*(\nu_1+\nu_2)=\lambda*\nu_1+\lambda*\nu_2$,}

\Centerline{$(\alpha\lambda)*\nu=\lambda*(\alpha\nu)
=\alpha(\lambda*\nu)$}

\noindent for all totally finite $\tau$-additive Borel measures
$\lambda$, $\lambda_1$, $\lambda_2$, $\nu$, $\nu_1$, $\nu_2$ and all
$\alpha\ge 0$.   Consequently\cmmnt{, regarding elements of $M_{\tau}$
as functionals on the Borel $\sigma$-algebra,} we have a bilinear operator
$*:M_{\tau}\times M_{\tau}\to M_{\tau}$ defined by saying that

\Centerline{$(\lambda_1-\lambda_2)*(\nu_1-\nu_2)
=\lambda_1*\nu_1-\lambda_1*\nu_2-\lambda_2*\nu_1+\lambda_2*\nu_2$}

\noindent for all $\lambda_1$, $\lambda_2$, $\nu_1$, $\nu_2\in
M_{\tau}^+$.

\spheader 444Ec
\cmmnt{We see from 444B that} $*$ is associative\cmmnt{ on
$M_{\tau}^+$, so it will be associative on the whole of $M_{\tau}$}.
\cmmnt{Observe
that $*$ is positive in the sense that $\lambda*\nu\ge 0$ if $\lambda$,
$\nu\ge 0$;  so that}
\dvro{$|\lambda*\nu|
\le|\lambda|*|\nu|$}
{
$$\eqalign{|\lambda*\nu|
&=|\lambda^+*\nu^+-\lambda^+*\nu^--\lambda^-*\nu^+
    +\lambda^-*\nu^-|\cr
&\le\lambda^+*\nu^++\lambda^+*\nu^-+\lambda^-*\nu^++\lambda^-*\nu^-\cr
&=|\lambda|*|\nu|\cr}$$

\noindent} for any $\lambda$, $\nu\in M_{\tau}$.

\spheader 444Ed
If $\lambda$, $\nu\in M_{\tau}^+$ then\cmmnt{ $\|\lambda\|=\lambda X$
and $\|\nu\|=\nu X$ (362Ba), so}

\dvro{\Centerline{$\|\lambda*\nu\|
=\|\lambda\|\|\nu\|$.}}
{\Centerline{$\|\lambda*\nu\|
=(\lambda*\nu)(X)
=(\lambda\times\nu)(X\times X)
=\lambda X\cdot\nu X
=\|\lambda\|\|\nu\|$.}}

\noindent Generally, for any $\lambda$, $\nu\in M_{\tau}$,

\Centerline{$\|\lambda*\nu\|
\cmmnt{\mskip5mu =\||\lambda*\nu|\|}
\le\cmmnt{\||\lambda|*|\nu|\|
=\||\lambda|\|\||\nu|\|
=}\|\lambda\|\|\nu\|$.}

\noindent\cmmnt{Thus }$M_{\tau}$ is a Banach algebra under the operation
$*$\cmmnt{, as well as being an $L$-space}.   If $X$ is abelian
then $M_{\tau}$ will be a commutative algebra\cmmnt{, by 444D}.

\leader{444F}{}\cmmnt{ In preparation for the next construction I give
a general result extending ideas already touched on in 443C and 443G.

\medskip

\noindent}{\bf Theorem} Let $X$ be a topological space, $G$ a
topological group and $\action$ a continuous action of $G$ on $X$.   For
$A\subseteq X$, $a\in G$ write $a\action A=\{a\action x:x\in A\}$.   Let
$\nu$ be a   measure on $X$.

(a) If $f:X\to[0,\infty]$ is lower semi-continuous, then
$a\mapsto\int a\action f\,d\nu:G\to[0,\infty]$ is lower semi-continuous.
\cmmnt{ (See 4A5Cc for the definition of $a\action f$.)}   In
particular, if $V\subseteq X$ is open, then
$a\mapsto\nu(a\action V):G\to[0,\infty]$ is lower semi-continuous.

(b) If $f:X\to\Bbb R$ is continuous, then
$a\mapsto(a\action f)^{\ssbullet}:G\to L^0$ is continuous, if
$L^0=L^0(\nu)$ is given the topology of convergence in measure.

(c) If $\nu$ is $\sigma$-finite and $E\subseteq X$ is a Borel set, then
$a\mapsto(a\action E)^{\ssbullet}:G\to\frak A$ is Borel
measurable, if
the measure algebra $\frak A$ of $\nu$ is given its measure-algebra
topology.

(d) If $\nu$ is $\sigma$-finite and $f:X\to\Bbb R$ is Borel measurable,
then $a\mapsto (a\action f)^{\ssbullet}:G\to L^0$ is Borel measurable.

(e) If $\nu$ is $\sigma$-finite, then

\quad(i)
$a\mapsto\nu(a\action E):G\to[0,\infty]$ is Borel measurable for any
Borel set $E\subseteq X$;

\quad (ii)
if $f:X\to\Bbb R$ is Borel measurable, then $Q=\{a:\int a\action f\,d\nu$
is defined in $[-\infty,\infty]\}$ is a Borel set, and
$a\mapsto\int a\action f\,d\nu:Q\to[-\infty,\infty]$ is Borel measurable.

\proof{{\bf (a)(i)} Note first that if $f:X\to[0,\infty]$ is Borel
measurable, then, for each $a\in G$, $a\action f$ is the composition of
$f$ with the continuous function $x\mapsto a^{-1}\action x$, so is Borel
measurable, and if $f$ is finite-valued then $(a\action f)^{\ssbullet}$
is defined in $L^0=L^0(\nu)$.

\medskip

\quad{\bf (ii)} Suppose that $f:X\to[0,\infty]$ is lower
semi-continuous, $\gamma\in\coint{0,\infty}$, $a\in G$ and
$\int a\action f>\gamma$.   Let
$\Cal U$ be the set of open neighbourhoods of $a$ in $G$.   For
$U\in\Cal U$, $x\in X$ set

\Centerline{$\phi_U(x)=\sup\{\inf_{b\in U,y\in V}(b\action f)(y):V$ is
an open neighbourhood of $x$ in $X\}$.}

\noindent Then $\phi_U$ is lower semi-continuous.   \Prf\ If
$\phi_U(x)>\alpha$, there is an open neighbourhood $V$ of $x$ such that
$\inf_{b\in U,y\in V}(b\action f)(y)\penalty-100>\alpha$;
now $\phi_U(y)>\alpha$ for every $y\in V$.\ \QeD\   If $U'\subseteq U$ then
$\phi_{U'}\ge\phi_U$,
so $\{\phi_U:U\in\Cal U\}$ is upwards-directed.   Also
$\sup_{U\in\Cal U}\phi_U=a\action f$ in $[0,\infty]^X$.   \Prf\ Of
course $\phi_U(x)\le (a\action f)(x)$ for every $x$.   If $x\in X$ and
$(a\action f)(x)>\alpha$, then
$\{y:f(y)>\alpha\}$ is an open set containing $a^{-1}\action x$, so
(because $\action$ is continuous) there are a $U\in\Cal U$ and an open
neighbourhood $V$ of $x$ such that $f(b^{-1}\action y)>\alpha$ whenever
$b\in U$ and $y\in V$;  in which case $\phi_U(x)\ge\alpha$.   As $\alpha$
is arbitrary, $\sup_{U\in\Cal U}\phi_U(x)=(a\action f)(x)$.\ \Qed

By 414Ba, $\int a\action f\,d\nu=\sup_{U\in\Cal U}\int\phi_Ud\nu$, and
there is a $U\in\Cal U$ such that $\int\phi_Ud\nu>\gamma$.   Now suppose
that $b\in U$;  then $\phi_U(x)\le (b\action f)(x)$ for every $x$, so
$\int b\action f\,d\nu>\gamma$.
This shows that $\{a:\int a\action f\,d\nu>\gamma\}$ is an open set in
$G$, so that $a\mapsto\int a\action f\,d\nu$ is lower semi-continuous.

\medskip

\quad{\bf (iii)} If $V\subseteq X$ is open, then $\chi V$ is lower
semi-continuous, and $\chi(a\action V)=a\action(\chi V)$ for every
$a\in G$.   So $a\mapsto\nu(a\action V)=\int a\action(\chi V)d\nu$ is
lower semi-continuous.

Thus (a) is true.

\medskip

{\bf (b)} Take any $a\in G$, $E\in\dom\nu$ such that $\nu E<\infty$ and
$\epsilon>0$.   Let $\Cal U$ be the family of open neighbourhoods of $a$
in $G$, and for $U\in\Cal U$ set

\Centerline{$H_U
=\interior\{x:|(b\action f)(x)-(a\action f)(x)|\le\epsilon$ whenever
$b\in U\}$.}

\noindent Then $\{H_U:U\in\Cal U\}$ is upwards-directed.   Also,
it has union $X$.   \Prf\ If $x\in X$ then, because
$(b,y)\mapsto f(b^{-1}\action y)$ is continuous, there are a
$U\in\Cal U$ and an open neighbourhood $V$ of $x$ such that
$|(b\action f)(y)-(a\action f)(x)|\le\bover12\epsilon$ whenever
$b\in U$ and $y\in V$.
But now $|(b\action f)(y)-(a\action f)(y)|\le\epsilon$ whenever
$b\in U$ and
$y\in V$, so that $H_U$ includes $V$, which contains $x$.\ \Qed

So there is a $U\in\Cal U$ such that
$\nu(E\setminus H_U)\le\epsilon$ (414Ea).   In this case, for any
$b\in U$, we must have

\Centerline{$\int_E\min(1,|b\action f-a\action f|)d\nu
\le\epsilon(1+\nu E)$.}

\noindent As $E$ and $\epsilon$ are arbitrary,
$b\mapsto (b\action f)^{\ssbullet}$ is continuous at $a$;  as $a$ is
arbitrary, it is continuous everywhere.   Thus (b) is true.

\medskip

{\bf (c)(i)} Let us start by supposing that $E$ is an open set and that
$\nu$ is totally finite.   In this case the function
$a\mapsto\nu(a\action E)$
is lower semi-continuous, by (a) above, therefore Borel measurable.
Now let $W\subseteq\frak A$ be an open set, and write
$H=\{a:a\in G,\,(a\action E)^{\ssbullet}\in W\}$.   For $m$,
$k\in\Bbb N$ set
$H_{mk}=\{a:2^{-m}k\le\nu(a\action E)<2^{-m}(k+1)\}$, so that $H_{mk}$
is Borel, and
$U_{mk}=G\setminus\overline{H_{mk}\setminus H}$, so that $U_{mk}$ is
open.   Let $H'$ be $\bigcup_{m,k\in\Bbb N}H_{mk}\cap U_{mk}$;  then
$H'$ is Borel and $H'\subseteq H$.   In fact $H'=H$.   \Prf\ If
$a\in H$, then $W$ is an open
set containing $(a\action E)^{\ssbullet}$.   Let $\delta>0$ be such that
$a\in W$ whenever
$\bar\nu(a\Bsymmdiff(a\action E)^{\ssbullet})\le\delta$,
where $\bar\nu$ is the measure on $\frak A$;
let $m$, $k\in\Bbb N$ be such that $2^{-m}\le\bover14\delta$ and
$2^{-m}k\le\nu(a\action E)<2^{-m}(k+1)$.   If we take $\nuprime$ to be
the indefinite-integral measure over $\nu$ defined by $\chi(a\action E)$,
then $\nuprime$ is a quasi-Radon measure (415Ob), so (by (a) again)
$U=\{b:\nuprime(b\action E)>2^{-m}(k-1)\}$ is an open set, and of course
it
contains $a$.   If $b\in U\cap H_{mk}$, then

$$\eqalign{\nu((b\action E)\symmdiff(a\action E))
&=\nu(b\action E)+\nu(a\action E)-2\nu((b\action E)\cap(a\action E))\cr
&\le 2^{-m}(k+1)+2^{-m}(k+1)-2\cdot 2^{-m}(k-1)
\le 4\cdot 2^{-m}
\le\delta,\cr}$$

\noindent so $(b\action E)^{\ssbullet}\in W$ and $b\in H$.   This shows
that $U\cap(H_{mk}\setminus H)=\emptyset$ and $U\subseteq U_{mk}$ and
$a\in U\cap H_{mk}\subseteq H'$.   As $a$ is arbitrary, $H=H'$.\ \QeD\
Thus $H$ is a
Borel subset of $G$.   As $W$ is arbitrary, the map
$a\mapsto(a\action E)^{\ssbullet}$ is Borel measurable.

\medskip

\quad{\bf (ii)} To extend this to a general $\sigma$-finite quasi-Radon
measure $\nu$, still
supposing that $E$ is open, let $h:X\to\Bbb R$ be a strictly positive
integrable function (215B(viii)) and $\nuprime$ the corresponding
indefinite-integral measure.   As in (i), this $\nuprime$ also is a
quasi-Radon measure.   Since $\nu$
and $\nuprime$ have the same domains and the same null ideals,
the Boolean algebra $\frak A$ is still the underlying algebra of the
measure algebra of $\nuprime$;  by 324H, the topologies on $\frak A$
induced by the measures $\bar\nu$, $\bar\nuprime$ are the same.   So we
can apply (i) to the measure $\nuprime$ to see that
$a\mapsto(a\action E)^{\ssbullet}:G\to\frak A$ is still Borel
measurable.

\medskip

\quad{\bf (iii)} Next, suppose that $E$ is expressible as
$\bigcup_{i\le n}V_{2i}\setminus V_{2i+1}$ where each $V_i$ is open.
Then $a\mapsto(a\action E)^{\ssbullet}$ is Borel measurable.   \Prf\ Set
$X'=X\times\{0,\ldots,2n+1\}$, with the product topology (giving
$\{0,\ldots,2n+1\}$ its discrete topology) and define a measure
$\nuprime$ on $X$ and an action of $G$ on $X'$ by setting

\Centerline{$\nuprime F=\sum_{i=0}^{2n+1}\nu\{x:(x,i)\in F\}$}

\noindent whenever $F\subseteq X'$ is such that
$\{x:(x,i)\in F\}\in\dom\nu$ for every $i\le 2n+1$,

\Centerline{$a\action(x,i)=(a\action x,i)$}

\noindent whenever $a\in G$, $x\in X$ and $i\le 2n+1$.   Then
$V=\{(x,i):i\le 2n+1,\,x\in V_i\}$ is an open set in $X'$, while
$\nuprime$
is a $\sigma$-finite quasi-Radon measure, as is easily checked;  so, by
(ii), the map
$a\mapsto(a\action V)^{\ssbullet}:G\to\frak A'$ is Borel measurable,
where $\frak A'$ is the measure algebra of $\nuprime$.   On the other
hand, we can identify $\frak A'$ with the simple power $\frak A^{2n+2}$
(322Lb), and the map

\Centerline{$\langle c_i\rangle_{i\le 2n+1}
\mapsto\sup_{i\le n}c_{2i}\Bsetminus c_{2i+1}:\frak A^{2n+2}\to\frak A$}

\noindent is continuous, by 323B.   So the map

\Centerline{$a\mapsto(a\action E)^{\ssbullet}
=\sup_{i\le n}(a\action V_{2i})^{\ssbullet}
  \Bsetminus(a\action V_{2i+1})^{\ssbullet}$}

\noindent is the composition of a Borel measurable function with a
continuous function, and is Borel measurable.\ \Qed

\medskip

\quad{\bf (iv)} Now the family $\Cal E$ of all those Borel sets
$E\subseteq X$ such that $a\mapsto(a\action E)^{\ssbullet}$ is Borel
measurable is closed under unions and intersections of monotonic
sequences.   \Prf\ ($\alpha$) If $\sequencen{E_n}$ is a non-decreasing
sequence in $\Cal E$ with union $E$, then

\Centerline{$(a\action E)^{\ssbullet}
=\sup_{n\in\Bbb N}(a\action E_n)^{\ssbullet}
=\lim_{n\to\infty}(a\action E_n)^{\ssbullet}$}

\noindent (323Ea) for every $a\in G$.   So
$a\mapsto(a\action E)^{\ssbullet}$ is
the pointwise limit of a sequence of Borel measurable functions into a
metrizable space (323Gb, because $(\frak A,\bar\nu)$ is
$\sigma$-finite), and is Borel measurable, by 418Ba.   Thus
$E\in\Cal E$.   ($\beta$) If $\sequencen{E_n}$ is a non-increasing
sequence in $\Cal E$ the same argument applies, since this time

\Centerline{$(a\action E)^{\ssbullet}
=\inf_{n\in\Bbb N}(a\action E_n)^{\ssbullet}
=\lim_{n\to\infty}(a\action E_n)^{\ssbullet}$}

\noindent (323Eb) for every $a\in G$.\ \Qed

Since $\Cal E$ contains all sets of the form
$\bigcup_{i\le n}V_i\cap F_i$ where every $V_i$ is open and every $F_i$
is closed, by (iii), $\Cal E$ must be the whole Borel $\sigma$-algebra,
by 4A3Cg(ii).

This completes the proof of (c).

\medskip

{\bf (d)(i)} We need the following extension of (c):  if
$\sequencen{E_n}$ is any sequence of Borel sets in $X$, then
$a\mapsto\sequencen{(a\action E_n)^{\ssbullet}}:
  G\mapsto\frak A^{\Bbb N}$ is Borel measurable.   \Prf\ I repeat the
idea of (c-iii) above.
On $X'=X\times\Bbb N$ define a measure $\nuprime$ by setting

\Centerline{$\nuprime F=\sum_{n=0}^{\infty}\nu\{x:(x,n)\in F\}$}

\noindent whenever $F\subseteq X'$ is such that
$\{x:(x,n)\in F\}\in\dom\nu$ for every $n\in\Bbb N$.   As before, it is
easy to check
that $\nuprime$ is a $\sigma$-finite quasi-Radon measure, if we give
$\Bbb N$ its discrete
topology and $X'$ the product topology.   As before, set
$a\action(x,n)=(a\action x,n)$ for $a\in G$, $x\in X$ and $n\in\Bbb N$,
to obtain a continuous action of $G$ on $X'$.   Applying (c) to this
action, the map $a\mapsto(a\action E)^{\ssbullet}:G\to\frak A'$ is
Borel measurable, where $\frak A'$ is the measure algebra of $\nuprime$
and $E=\{(x,n):n\in\Bbb N,\,x\in E_n\}$.   But we can identify $\frak A'$
(as Boolean algebra) with $\frak A^{\Bbb N}$, by 322L, as before;  so
that if we re-interpret $a\mapsto(a\action E)^{\ssbullet}:G\to\frak A'$
as $a\mapsto\sequencen{(a\action E_n)^{\ssbullet}}:G\to\frak A^{\Bbb N}$
it is still Borel measurable.   (As in (c-ii), this time using
323L, the measure-algebra topology of $\frak A'$ matches the product
topology on $\frak A^{\Bbb N}$.)\ \Qed

\medskip

\quad{\bf (ii)} Now suppose that $f:X\to[0,1]$ is Borel measurable.
Define $\sequencen{E_n}$ inductively by the formula

\Centerline{$E_n=\{x:x\in X,\,
(f-\sum_{i<n}2^{-i-1}\chi E_i)(x)\ge 2^{-n-1}\}$.}

\noindent Then every $E_n$ is a Borel set and
$f=\sum_{n=0}^{\infty}2^{-n-1}\chi E_n$.   Next, observe that the
function

\Centerline{$\sequencen{c_n}\mapsto\sum_{n=0}^{\infty}2^{-n-1}\chi c_n:
\frak A^{\Bbb N}\to L^0$}

\noindent is continuous, because each of the maps
$c\mapsto 2^{-n-1}\chi c$ is continuous (367R), addition is continuous
(245Da, 367Ma) and the
series is uniformly summable.   Accordingly we may think of the map
$a\mapsto (a\action f)^{\ssbullet}$ as the composition of the continuous
function
$\sequencen{c_n}\mapsto\sum_{n=0}^{\infty}2^{-n-1}\chi c_i$ with the
Borel measurable function
$a\mapsto\sequencen{(a\action E_n)^{\ssbullet}}$, and it is Borel
measurable.

\wheader{444F}{4}{2}{2}{30pt}

\quad{\bf (iii)} For general Borel measurable $f:X\to\Bbb R$, set
$q(t)=\bover12(1+\bover{t}{|t|+1})$, so that $q:\Bbb R\to\ooint{0,1}$ is
a homeomorphism, and set $p=q^{-1}:\ooint{0,1}\to\Bbb R$.   Then the
function $\bar p:P\to L^0$ is continuous, where
$P=\{u:u\in L^0,\,\Bvalue{u\in\ooint{0,1}\,}=1\}$ (367S).   But now

\Centerline{$a\action f=p\frsmallcirc q\frsmallcirc(a\action f)
=p\frsmallcirc(a\action(q\frsmallcirc f))$}

\noindent for every $a$, so that $a\mapsto (a\action f)^{\ssbullet}$ is
the composition of the Borel map
$a\mapsto(a\action(q\frsmallcirc f))^{\ssbullet}$ with the
continuous map $\bar p$, and is Borel measurable.

\medskip

{\bf (e)(i)} We need only recall that $\bar\nu:\frak A\to\Bbb R$ is
lower semi-continuous (323Cb), so that (applying (c) above)

\Centerline{$a\mapsto\nu(a\action E)=\bar\nu(a\action E)^{\ssbullet}$}

\noindent is a composition of Borel measurable functions and is Borel
measurable.   (Of course there are
much more direct arguments, using fragments of the proof above.)

\medskip

\quad{\bf (ii)} The point is that the maps $u\mapsto u^+$,
$u\mapsto u^-:L^0\to(L^0)^+$ are continuous (245Db, 367S), while
$u\mapsto\int u:(L^0)^+\to[0,\infty]$ is lower semi-continuous
(369Mb), therefore Borel measurable.   Accordingly
$a\mapsto\int (a\action f)^+=\int((a\action f)^{\ssbullet})^+$ and
$a\mapsto\int (a\action f)^-$ are Borel measurable functions from $G$ to
$[0,\infty]$, so that

\Centerline{$Q
=\{a:\min(\int (a\action f)^+,\int (a\action f)^-)<\infty\}$}

\noindent is a Borel set, and

\Centerline{$a\mapsto\int a\action f
=\int (a\action f)^+-\int (a\action f)^-:
Q\to[-\infty,\infty]$}

\noindent is Borel measurable.
}%end of proof of 444F

\vleader{48pt}{444G}{Corollary} Let $X$ be a topological group and $\nu$ a
$\sigma$-finite quasi-Radon measure on $X$.

(a) If $f:X\to\Bbb R$ is a Borel measurable function, then
$\{x:\int f(y^{-1}x)\nu(dy)$ is defined in $[-\infty,\infty]\}$ is a
Borel set in
$X$ and $x\mapsto\int f(y^{-1}x)\nu(dy)$ is Borel measurable.

(b) If $f$, $g:X\to\Bbb R$ are Borel measurable functions, then
$\{x:\int f(xy^{-1})g(y)\nu(dy)$ is defined in $[-\infty,\infty]\}$ is a
Borel set and $x\mapsto\int f(xy^{-1})g(y)\nu(dy)$ is Borel measurable.

(c) If $\nu$ is totally finite and $f:X\to\Bbb R$ is a bounded
continuous function, then $x\mapsto\int f(y^{-1}x)\nu(dy):X\to\Bbb R$ is
continuous.

\proof{{\bf (a)} Set $\Reverse{f}(x)=f(x^{-1})$ for $x\in X$ (4A5C(c-ii));
then $\Reverse{f}$ is Borel
measurable.   Let $\action_l$ be the left action of $X$ on itself.
Then, in the language of 444Fe,
$Q=\{x:\int x\action_l\Reverse{f}\,d\nu$ is defined in $[-\infty,\infty]\}$
is a Borel set, and $x\mapsto\int x\action_l\Reverse{f}\,d\nu$ is Borel
measurable.   But 

\Centerline{$(x\action_l\Reverse{f})(y)
=\Reverse{f}(x^{-1}y)=f(y^{-1}x)$}

\noindent for all $x$, $y$, so
$\int x\action_l\Reverse{f}\,d\nu=\int f(y^{-1}x)\nu(dy)$ if either
integral is defined.

\medskip

{\bf (b)(i)} Set $\Reverse{\nu}E=\nu E^{-1}$ when this is defined, writing
$E^{-1}=\{x^{-1}:x\in E\}$ for $E\subseteq X$;  that is, $\Reverse{\nu}$ is
the image measure $\nu\phi^{-1}$, where $\phi(x)=x^{-1}$ for $x\in X$.
Because $\phi$ is a homeomorphism, $\Reverse{\nu}$ is a quasi-Radon measure.
By 235J,
$\int h\,d\Reverse{\nu}=\int h(x^{-1})\nu(dx)$ for any real-valued function
$h$ for which either integral is defined in $[-\infty,\infty]$.

\medskip

\quad{\bf (ii)} Now suppose that $f$ and $g$ are non-negative Borel
measurable functions from $X$ to $\Bbb R$.   
Then $\Reverse{g}$ also is a non-negative Borel
measurable function.   We know from 444Fd that
$x\mapsto (x^{-1}\action_lf)^{\ssbullet}:X\to L^0(\Reverse{\nu})$
is a Borel measurable function;  now multiplication in $L^0$ is
continuous (245Dc), so the map
$x\mapsto((x^{-1}\action_lf)\times\Reverse{g})^{\ssbullet}$
is Borel measurable;  since integration is lower semi-continuous on
$(L^0)^+$,
$x\mapsto\int(x^{-1}\action_lf)\times\Reverse{g}\,d\Reverse{\nu}:
X\to[0,\infty]$ is Borel measurable.   But

\Centerline{$\int(x^{-1}\action_lf)\times\Reverse{g}\,d\Reverse{\nu}
=\int f(xy)g(y^{-1})\Reverse{\nu}(dy)
=\int f(xy^{-1})g(y)\nu(dy)$}

\noindent whenever any of the integrals is defined in
$[-\infty,\infty]$, so this is the function we needed to know about.

\medskip

\quad{\bf (iii)} For general Borel measurable functions $f$ and $g$, we
have

$$\eqalign{\int f(xy^{-1})g(y)\nu(dy)
&=\bigl(\int f^+(xy^{-1})g^+(y)\nu(dy)
+\int f^-(xy^{-1})g^-(y)\nu(dy)\bigr)\cr
&\qquad\qquad-\bigl(\int f^+(xy^{-1})g^-(y)\nu(dy)
+\int f^-(xy^{-1})g^+(y)\nu(dy)\bigr)\cr}$$

\noindent exactly when the subtraction can be done in
$[-\infty,\infty]$, so that $x\mapsto\int f(xy^{-1})g(y)\nu(dy)$ is a
difference of Borel measurable functions and is Borel measurable, with
Borel measurable domain.

\medskip

{\bf (c)} Continuing the argument of (a), if $f$ is bounded and
continuous and $\nu$ is
totally finite then all the functions $x\action_l\Reverse{f}$ are bounded
and continuous, so $\int x\action_l\Reverse{f}\,d\nu$ is defined and finite
for every $x$.   Next, the function
$x\mapsto(x\action_l\Reverse{f})^{\ssbullet}:X\to L^0(\nu)$ is
continuous for the topology of convergence in measure (444Fb), which
agrees with the norm topology of $L^1(\nu)$ on $\|\,\|_{\infty}$-bounded
sets (246Jb).   It follows that

\Centerline{$x\mapsto\int (x\action_l\Reverse{f})^{\ssbullet}
=\int f(y^{-1}x)\nu(dy)$}

\noindent is continuous.
}%end of proof of 444G

\leader{444H}{Convolutions of measures and functions}\cmmnt{ I
introduce some notation which I shall use for the rest of the section.}
Let $X$ be a topological group.   If $f$ is a real-valued function
defined on a subset of $X$, and $\nu$ is a measure on $X$, set

\Centerline{$(\nu*f)(x)=\int f(y^{-1}x)\nu(dy)$}

\noindent whenever the integral is defined in $\Bbb R$.

\leader{444I}{Proposition} Let $X$ be a topological group and $\lambda$,
$\nu$ two totally finite quasi-Radon measures on $X$.

(a) For any Borel measurable function $f:X\to\Bbb R$, $\nu*f$ is a Borel
measurable function with a Borel domain.

(b) $\nu*f\in C_b(X)$ for every $f\in C_b(X)$.

(c) For any real-valued function $f$ defined on a subset of $X$,
$(\lambda*(\nu*f))(x)=((\lambda*\nu)*f)(x)$ whenever the right-hand side
is defined.

\proof{{\bf (a)} This follows at once from 444Ga.

\medskip

{\bf (b)} This is just a restatement of 444Gc.

\medskip

{\bf (c)} If $((\lambda*\nu)*f)(x)$ is defined, then

$$\eqalignno{((\lambda*\nu)*f)(x)
&=\int f(t^{-1}x)(\lambda*\nu)(dt)
=\iint f((yz)^{-1}x)\nu(dz)\lambda(dy)\cr
\displaycause{444C}
&=\iint f(z^{-1}y^{-1}x)\nu(dz)\lambda(dy)\cr
&=\int(\nu*f)(y^{-1}x)\lambda(dy)
=(\lambda*(\nu*f))(x).\cr}$$
}%end of proof of 444I

\leader{444J}{Convolutions of functions and measures}
Let $X$ be a topological group carrying Haar measures;  let $\Delta$ be
its left modular function\cmmnt{ (442I)}.   If $f$ is a real-valued
function defined on a subset of $X$, and $\nu$ is a measure on $X$, set

\Centerline{$(f*\nu)(x)=\int f(xy^{-1})\Delta(y^{-1})\nu(dy)$}

\noindent whenever the integral is defined in $\Bbb R$.
\cmmnt{From 444Gb we see that if $\nu$ is a $\sigma$-finite
quasi-Radon measure and
$f$ is Borel measurable, then $f*\nu$ is a Borel measurable function
with a Borel domain.}   If $f$ is
non-negative and $\nu$-integrable, write $f\nu$ for the corresponding
indefinite-integral measure over $\nu$\cmmnt{ (234J\formerly{2{}34B})}.

\leader{444K}{Proposition} Let $X$ be a topological group with a left
Haar measure $\mu$.   Let $\nu$ be a totally finite
quasi-Radon measure on $X$.   Then for any non-negative $\mu$-integrable
real-valued function $f$, $f\mu$ is a quasi-Radon measure;  moreover,
$\nu*f$ and $f*\nu$ are $\mu$-integrable, and we have

\Centerline{$(\nu*f)\mu=\nu*f\mu$,
\quad$(f*\nu)\mu=f\mu*\nu$.}

\noindent In particular,
$\int\nu*f\,d\mu=\int f*\nu\,d\mu=\nu X\cdot\int fd\mu$.

\proof{{\bf (a)} $f\mu$ is a quasi-Radon measure by 415Ob.

\medskip

{\bf (b)} Suppose first that $f$ is Borel measurable and
defined everywhere on $X$, as well as being non-negative and
$\mu$-integrable.

\medskip

\quad{\bf (i)} Let $E\subseteq X$ be a Borel set such that $\mu E<\infty$.
The function
$(x,y)\mapsto f(x^{-1}y)\chi E(y):X\times X\to\coint{0,\infty}$ is Borel
measurable, so

$$\eqalignno{(\nu*f\mu)(E)
&=\int(f\mu)(x^{-1}E)\nu(dx)\cr
\noalign{\noindent (444A)}
&=\iint f(y)\chi(x^{-1}E)(y)\mu(dy)\nu(dx)\cr
\noalign{\noindent (by the definition of $f\mu$, 234I\formerly{2{}34A})}
&=\iint f(x^{-1}y)\chi(x^{-1}E)(x^{-1}y)\mu(dy)\nu(dx)\cr
\noalign{\noindent (441J)}
&=\iint f(x^{-1}y)\chi E(y)\mu(dy)\nu(dx)
=\iint f(x^{-1}y)\chi E(y)\nu(dx)\mu(dy)\cr}$$

\noindent (by Fubini's theorem, 417Ha, because
$(x,y)\mapsto f(x^{-1}y)\chi E(y)$ is non-negative and Borel measurable
and zero outside $X\times E$).   So
$\int f(x^{-1}y)\nu(dx)$ must be finite for $\mu$-almost every $y\in E$.
Because $\mu$ is complete and locally determined and inner regular with
respect to the Borel sets of finite measure,
$(\nu*f)(y)=\int f(x^{-1}y)\nu(dx)$ is defined in $\Bbb R$ for $\mu$-almost
every $y\in X$.   So we have an indefinite-integral measure $(\nu*f)\mu$.
Next, we have

$$\eqalign{\infty
&>\nu X\int fd\mu
=(\nu*f\mu)(X)
\ge(\nu*f\mu)(E)\cr
&=\iint f(x^{-1}y)\chi E(y)\nu(dx)\mu(dy)
=\iint(\nu*f)(y)\chi E(y)\mu(dy)
=((\nu*f)\mu)(E)\cr}$$

\noindent for every Borel set $E$ such that $\mu E$ is finite.   Again
because $\mu$ is inner regular with respect to the Borel sets of finite
measure, $\nu*f$ is $\mu$-integrable and $(\nu*f)\mu$ is totally finite.
Since $\nu*f\mu$ and $(\nu*f)(\mu)$ are totally finite quasi-Radon measures
agreeing on open sets of finite measure for $\mu$, and $\mu$ is locally
finite (442Aa), 415H(iv) assures us that $\nu*f\mu=(\nu*f)(\mu)$.

\medskip

\quad{\bf (ii)} Now consider $f*\nu$.   This time, if
$E\subseteq X$ is Borel and $\mu E<\infty$,

$$\eqalignno{(f\mu*\nu)(E)
&=\int(f\mu)(Ey^{-1})\nu(dy)
=\iint_{Ey^{-1}}f(x)\mu(dx)\nu(dy)\cr
&=\int\Delta(y^{-1})\int_Ef(xy^{-1})\mu(dx)\nu(dy)\cr
\noalign{\noindent (by 442Kc)}
&=\int_E\int\Delta(y^{-1})f(xy^{-1})\nu(dy)\mu(dx).\cr}$$

\noindent Once again, we see that $\int\Delta(y^{-1})f(xy^{-1})\nu(dy)$ is
defined for $\mu$-almost every $x\in E$;  as $E$ is arbitrary,
$(f*\nu)(x)$ is defined in $\Bbb R$ for $\mu$-almost every $x\in X$;
and 444Gb tells us that $f*\nu$ is Borel measurable.   As before,

\Centerline{$\int_Ef*\nu\,d\mu\le(f\mu*\nu)(X)
=\int fd\mu\cdot\nu X<\infty$}

\noindent for Borel sets $E$ with $\mu E<\infty$, so that $f*\nu$ is
$\mu$-integrable;  as before, the quasi-Radon measures $(f*\nu)\mu$ and
$f\mu*\nu$ agree on open sets of finite $\mu$-measure, so coincide.

\medskip

{\bf (c)} Next, suppose that $f$ is defined and zero $\mu$-a.e.   In
this case there is a Borel set $E$ such that $\mu E=0$ and $f(x)$ is
defined and equal to zero for every $x\in X\setminus E$ (443J(b-ii)).
Set $g=\chi E$.   Then $g\mu$ is the zero measure, so
$(\nu*g)\mu=\nu*g\mu$, $(g*\nu)\mu=g\mu*\nu$ are all zero;  that is,
there is some $\mu$-conegligible set $F$ such that

\Centerline{$0=(\nu*g)(x)=\int\chi E(y^{-1}x)\nu(dy)
=\nu(xE^{-1})$,}

\Centerline{$0=(g*\nu)(x)=\int\chi E(xy^{-1})\Delta(y^{-1})\nu(dy)
=\int_{E^{-1}x}\Delta(y^{-1})\nu(dy)$}

\noindent for every $x\in F$.   But now, for $x\in F$, we must have
$\nu(xE^{-1})=\nu(E^{-1}x)=0$ (since $\Delta$ is strictly positive), so
that

\Centerline{$(\nu*f)(x)=\int f(y^{-1}x)\nu(dy)
=\int_{xE^{-1}}f(y^{-1}x)\nu(dy)=0$,}

\noindent because if $y\notin xE^{-1}$ then $y^{-1}x\notin E$ and
$f(y^{-1}x)=0$.   Similarly,

\Centerline{$(f*\nu)(x)=\int f(xy^{-1})\Delta(y^{-1})\nu(dy)
=\int_{E^{-1}x}f(xy^{-1})\Delta(y^{-1})\nu(dy)=0$.}

\noindent Thus $\nu*f$ and $f*\nu$ are defined, and zero, $\mu$-almost
everywhere.

\medskip

{\bf (d)} For an arbitrary non-negative $\mu$-integrable function $f$,
we can express it in the form $g+h$ where $g$ is a non-negative
$\mu$-integrable Borel measurable function defined everywhere, and $h$
is zero almost everywhere.   In this case, $\nu*h^+$, $\nu*h^-$,
$h^+*\nu$ and $h^-*\nu$ are defined, and zero, $\mu$-a.e., so
$\nu*f\eae\nu*g$ and $f*\nu\eae g*\nu$.   We therefore have

\Centerline{$(\nu*f)\mu=(\nu*g)\mu=\nu*g\mu=\nu*f\mu$,
\quad$(f*\nu)\mu=(g*\nu)\mu=g\mu*\nu=f\mu*\nu$,}

\noindent as required.

\medskip

{\bf (e)} Finally, we have

\Centerline{$\int\nu*fd\mu=((\nu*f)\mu)(X)=(\nu*f\mu)(X)
=\nu X\cdot(f\mu)(X)=\nu X\cdot\int fd\mu$,}

\Centerline{$\int f*\nu\,d\mu=((f*\nu)\mu)(X)=(f\mu*\nu)(X)
=(f\mu)(X)\cdot\nu X=\nu X\cdot\int fd\mu$.}
}%end of proof of 444K

\leader{444L}{Corollary} Let $X$ be a topological group carrying Haar
measures.   Suppose that $\nu$ is a non-zero quasi-Radon measure on $X$
and $E\subseteq X$ is a Haar measurable set such that $\nu(xE)=0$ for every
$x\in X$.   Then $E$ is Haar negligible.

\proof{ Let $\mu$ be a left Haar measure on $X$.   There is a non-zero
totally finite quasi-Radon measure $\nuprime$ on $X$ such that
$\nuprime(xE)=0$
for every $x\in X$.   \Prf\ Take any $F$ such that $0<\nu F<\infty$, and
set $\nuprime H=\nu(H\cap F)$ whenever this is defined.\ \QeD\   Let $G$
be
any Borel set such that $\mu G<\infty$, and set $f=\chi(G\cap E^{-1})$.
Then $f$ is $\mu$-integrable, and

\Centerline{$(\nuprime*f)(x)
=\int\chi(G\cap E^{-1})(y^{-1}x)\nuprime(dy)
=\nuprime(xG^{-1}\cap xE)=0$}

\noindent for every $x\in X$.   By 444K,
$\nuprime*f\mu=(\nuprime*f)\mu$ is the zero
measure, and $(f\mu)(X)=0$, that is, $\mu(G\cap E^{-1})=0$.   As $G$ is
arbitrary, $\mu E^{-1}=0$ and $E$ is Haar negligible (442H).
}%end of proof of 444L

\leader{444M}{Proposition} Let $X$ be a topological group and $\mu$ a
left Haar measure on $X$.   Let $\nu$ be a quasi-Radon
measure on $X$ and $p\in[1,\infty]$.

(a) Suppose that $\nu X<\infty$.   Then
we have a bounded positive linear operator
$u\mapsto\nu*u:L^p(\mu)\to L^p(\mu)$, of norm at most $\nu X$, defined
by saying that $\nu*f^{\ssbullet}=(\nu*f)^{\ssbullet}$ for every
$f\in\eusm L^p(\mu)$.

(b) Set $\gamma=\int\Delta(y)^{(1-p)/p}\nu(dy)$ if $p<\infty$,
$\int\Delta(y)^{-1}\nu(dy)$ if $p=\infty$, where $\Delta$ is the left
modular function of $X$.   Suppose that $\gamma<\infty$.   Then we have
a bounded positive linear operator
$u\mapsto u*\nu:L^p(\mu)\to L^p(\mu)$, of norm at most $\gamma$, defined
by saying that $f^{\ssbullet}*\nu=(f*\nu)^{\ssbullet}$ for every
$f\in\eusm L^p(\mu)$.

\proof{ I will write $\eusm L^p$, $L^p$ for $\eusm L^p(\mu)$, $L^p(\mu)$.
Note that if $f_1$, $f_2\in\eusm L^0(\mu)$ and $f_1=f_2\,\,\mu$-a.e., then
444L tells us that $\nu*|f_1-f_2|$ and $|f_1-f_2|*\nu$ are both zero
$\mu$-almost everywhere, so that $\nu*f_1\eae\nu*f_2$ and
$f_1*\nu\eae f_2*\nu$, in the sense that there is a $\mu$-conegligible set
$F$ such that $(\nu*f_1)\restr F=(\nu*f_2)\restr F$ and
$(f_1*\nu)\restr F=(f_2*\nu)\restr F$.   In particular, if we are told that
$\nu*f_1$ belongs to $\eusm L^p$, and that
$f_1^{\ssbullet}=f_2^{\ssbullet}$ in $L^0(\mu)$, then we can be sure that
$\nu*f_2\in\eusm L^p$ and $(\nu*f_2)^{\ssbullet}=(\nu*f_1)^{\ssbullet}$;
and similarly for $f_1*\nu$, $f_2*\nu$.

\wheader{444M}{6}{2}{2}{48pt}

{\bf (a)} If $\nu X=0$ the result is trivial.   Multiplying $\nu$ by a
positive scalar does not affect the inequalities we need, so we may suppose
that $\nu X=1$.   If $f\ge 0$ is $\mu$-integrable, then 444K tells us that
$\nu*f$ is $\mu$-integrable and that

$$\eqalign{\|\nu*f\|_1&=\int_X(\nu*f)(x)\mu(dx)=((\nu*f)\mu)(X)\cr
&=(\nu*f\mu)(X)=\nu X\cdot(f\mu)(X)=\|f\|_1,\cr}$$

\noindent using 444C or 444A for the penultimate equality.   Since
evidently $\nu*(f+g)=\nu*f+\nu*g$, $\nu*(\alpha f)=\alpha\nu*f$ at any
point where
the right-hand sides of the equations are defined in $\Bbb R$,
we have a positive
linear operator $T_1:L^1\to L^1$ defined by saying that
$T_1g^{\ssbullet}=(\nu*g)^{\ssbullet}$ for every $\mu$-integrable Borel
measurable function $g$, with $\|T_1\|=1$.

Similarly, if $h:X\to\Bbb R$ is a bounded Borel measurable function,
then $\nu*h$ also is a Borel
measurable function, by 444Ga.   Of course it is bounded, since

\Centerline{$|(\nu*h)(x)|=|\int h(y^{-1}x)\nu(dy)|
\le\sup_{y\in X}|h(y)|$}

\noindent for every $x$.   So we have a positive linear operator
$T_{\infty}:L^{\infty}\to L^{\infty}$ defined by saying that
$T_{\infty}h^{\ssbullet}=(\nu*h)^{\ssbullet}$ for every bounded Borel
measurable function $h$.   Moreover, if $u\in L^{\infty}$, there is a
Borel measurable $h$ such that $h^{\ssbullet}=u$ and
$\sup_{y\in X}|h(y)|=\|u\|_{\infty}$, so that

\Centerline{$\|T_{\infty}u\|_{\infty}
\le\sup_{x\in X}|(\nu*h)(x)|\le\|u\|_{\infty}$;}

\noindent thus $\|T_{\infty}\|\le 1$.

Since $T_1$ and $T_{\infty}$ agree on $L^1\cap L^{\infty}$, they have a
common extension to a linear operator
$T:L^1+L^{\infty}\to L^1+L^{\infty}$.   By 371Gd, $\|Tu\|_p\le\|u\|_p$
whenever $p\in[1,\infty]$ and $u\in L^p$.   (Strictly speaking, I am
relying on the standard identifications of $L^1$, $L^{\infty}$ and $L^p$
with the corresponding subspaces of $L^0(\frak A)$, where
$(\frak A,\bar\mu)$ is the measure algebra of $\mu$.   Of course the
argument for 371Gd applies equally well in $L^0(\mu)$.)   Now suppose
that $f\in\eusm L^p$.   Then it is expressible as $g+h$ where
$g\in\eusm L^1$ and $h:X\to\Bbb R$ is a bounded Borel measurable
function, so we shall have

\Centerline{$\nu*f=\nu*g+\nu*h$ wherever the right-hand side is
defined;}

\noindent accordingly $\nu*f$ is defined $\mu$-a.e.\ and is measurable, and

$$\eqalign{\|\nu*f\|_p
&=\|(\nu*g)^{\ssbullet}+(\nu*h)^{\ssbullet}\|_p
=\|T_1g^{\ssbullet}+T_{\infty}h^{\ssbullet}\|_p\cr
&=\|Tf^{\ssbullet}\|_p
\le\|f^{\ssbullet}\|_p
=\|f\|_p,\cr}$$

\noindent as required.

\medskip

{\bf (b)(i)} As in (a), the case $\nu X=0$ is trivial.   Otherwise, because
$\Delta$ is strictly positive, $\gamma>0$;  again considering a scalar
multiple of $\nu$ if necessary, we may suppose that $\gamma=1$.   Note that
$\nu$ is surely $\sigma$-finite.

\medskip

\quad{\bf (ii)} If $p=1$, then $\nu X=\gamma=1$.   If $f\in\eusm L^1$ is
non-negative, then, by 444K, as in (a) above,

\Centerline{$\|f*\nu\|_1=((f*\nu)\mu)(X)
=(f\mu*\nu)(X)=(f\mu)(X)\cdot\nu X=\|f\|_1$.}

\noindent For general $\mu$-integrable $f$,

\Centerline{$\|f*\nu\|_1\le\|f^+*\nu\|_1+\|f^-*\nu\|_1
=\|f^+\|_1+\|f^-\|_1=\|f\|_1$.}

\medskip

\quad{\bf (iii)} If $p=\infty$, then directly from the formula
$(f*\nu)(x)=\int f(xy^{-1})\Delta(y^{-1})\nu(dy)$ we see that if
$f:X\to\Bbb R$ is a bounded Borel measurable function then

\Centerline{$|(f*\nu)(x)|
\le\int\Delta(y)^{-1}\nu(dy)\cdot\sup_{y\in X}|f(y)|
=\sup_{y\in X}|f(y)|$}

\noindent for every $x$.   Since changing $f$ on a $\mu$-negligible set
changes $f*\nu$ on a $\mu$-negligible set, we can argue as in (a) above
to see that $f^{\ssbullet}\mapsto(f*\nu)^{\ssbullet}$ defines a linear
operator from $L^{\infty}$ to itself of norm at most $1$.

\medskip

\quad{\bf (iv)} Now suppose that $1<p<\infty$.   Set
$q=\bover{p}{p-1}$, so that $\int\Delta(y^{-1})^{1/q}\nu(dy)=\gamma=1$.
Suppose for the moment that $f\in\eusm L^p$ is a non-negative Borel
measurable function, and let $h:X\to\Bbb R$
be another non-negative Borel measurable function such that
$\int h^qd\mu\le 1$.   In this case

$$\eqalignno{\int(f*\nu)\times h\,d\mu
&=\iint h(x)f(xy^{-1})\Delta(y^{-1})\nu(dy)\mu(dx)\cr
&=\iint h(x)f(xy^{-1})\Delta(y^{-1})\mu(dx)\nu(dy)\cr
\noalign{\noindent (by 417Ha, because
$(x,y)\mapsto h(x)f(xy^{-1})\Delta(y^{-1})$ is Borel measurable and
$\{x:h(x)\ne 0\}$
is a countable union of sets of finite measure for $\mu$, while $\nu$ is
$\sigma$-finite)}
&=\iint h(xy)f(x)\mu(dx)\nu(dy)\cr}$$

\noindent by 442Kc, as usual, at least if the last integral is finite.
But, for any $y\in X$,

$$\eqalign{\int h(xy)f(x)\mu(dx)
&\le\|f\|_p\bigl(\int|h(xy)|^q\mu(dx)\bigr)^{1/q}\cr
&=\|f\|_p\bigl(\Delta(y^{-1})\int|h(x)|^q\mu(dx)\bigr)^{1/q}
\le\|f\|_p\Delta(y^{-1})^{1/q}.\cr}$$

\noindent So

\Centerline{$\int(f*\nu)\times h\,d\mu
=\iint h(xy)f(x)\mu(dx)\nu(dy)
\le\int\|f\|_p\Delta(y^{-1})^{1/q}\nu(dy)
=\|f\|_p$.}

\noindent Because $\mu$ (being a quasi-Radon measure)
is semi-finite, this means that
$f*\nu\in\eusm L^p$ and that $\|f*\nu\|_p\le\|f\|_p$ (366D-366E,
or 244Xe and 244Fa).
(Once again, we need to know that every member of $L^q$ can be
represented by a Borel measurable function;  this is a consequence of
443J or 412Xd.)

For general Borel measurable $f:X\to\Bbb R$ such that
$\int|f|^pd\mu<\infty$, we know that from 444G that $f*\nu$ is Borel
measurable, while
$|f*\nu|\le|f|*\nu$ (and $f*\nu$ is defined wherever $|f|*\nu$ is
finite), so that

\Centerline{$\|f*\nu\|_p\le\||f|*\nu\|_p\le\||f|\|_p
=\|f\|_p$.}

\noindent Finally, if $f\in\eusm L^p$ is arbitrary, then there is a
Borel measurable $g:X\to\Bbb R$ such that $f\eae g$, so that
$f*\nu\eae g*\nu$ and

\Centerline{$\|f*\nu\|_p=\|g*\nu\|_p\le\|g\|_p=\|f\|_p$.}

\noindent It follows at once that we have a bounded linear operator
$f^{\ssbullet}\mapsto(f*\nu)^{\ssbullet}:L^p\to L^p$, of norm at most
$1=\gamma$.
}%end of proof of 444M

\leader{444N}{}\cmmnt{ The following lemma on exchanging the order of
repeated integrals will be fundamental to the formulae in the rest of
the section.

\medskip

\noindent}{\bf Lemma} Let $X$ be a topological group and $\mu$ a left
Haar measure on $X$.   Suppose that $f$, $g$,
$h\in\eusm L^0(\mu)$\cmmnt{ (the space of measurable real-valued
functions defined $\mu$-a.e.\ in $X$)} are non-negative.   Then, writing
$\int\ldots d(x,y)$ to denote integration with respect to the
quasi-Radon product measure $\mu\times\mu$,

\Centerline{$\iint f(x)g(y)h(xy)dxdy=\iint f(x)g(y)h(xy)dydx
=\int f(x)g(y)h(xy)d(x,y)$}

\noindent in $[0,\infty]$.

\proof{ Following the standard pattern in results of this type, I deal
with successively more complicated functions $f$, $g$ and $h$.
Evidently the situation is symmetric, so that it will be enough if I can
show that $\iint f(x)g(y)h(xy)dxdy=\int f(x)g(y)h(xy)d(x,y)$.

\medskip

{\bf (a)} Suppose first that $f=\chi F$, $g=\chi G$ and $h=\chi H$,
where $F$, $G$, $H$ are Borel subsets of $X$.   In this case

\Centerline{$\iint f(x)g(y)h(xy)dxdy
=\sup_{U,V\in\Sigma^f}\int_V\int_Uf(x)g(y)h(xy)dxdy$,}

\noindent where $\Sigma^f$ is the ideal of measurable sets of finite
measure for $\mu$.   \Prf\ For $y\in X$, $n\in\Bbb N$ and $U\in\Sigma^f$
write

\Centerline{$q(y)=\int f(x)h(xy)dx=\mu(F\cap Hy^{-1})$,
\quad$q_U(y)=\int_Uf(x)h(xy)dx=\mu(U\cap F\cap Hy^{-1})$,}

\Centerline{$q^{(n)}(y)=\min(n,q(y))$,
\quad$q^{(n)}_U(y)=\min(n,q_U(y))$.}

\noindent Then every $q_U$ is continuous, by 443C (with a little help
from 323Cc), while
$\sup_{U\in\Sigma^f}q_U(y)=q(y)$ for every $y$, because $\mu$ is
semi-finite.   Because $\mu$ is $\tau$-additive and effectively locally
finite,
$(q^{(n)})^{\ssbullet}=\sup_{U\in\Sigma^f}(q_U^{(n)})^{\ssbullet}$ in
$L^0(\mu)$ for every $n$ (414Ab);  because $\Sigma^f$ is
upwards-directed,

$$\eqalign{\int q(y)g(y)dy
&=\sup_{n\in\Bbb N}\int q^{(n)}(y)g(y)dy\cr
&=\sup_{n\in\Bbb N,U\in\Sigma^f}\int q^{(n)}_U(y)g(y)dy
=\sup_{U\in\Sigma^f}\int q_U(y)g(y)dy,\cr}$$

\noindent that is,

\Centerline{$\iint f(x)g(y)h(xy)dxdy
=\sup_{U\in\Sigma_f}\int\int_Uf(x)g(y)h(xy)dxdy$.}

\noindent On the other hand, for any $U\in\Sigma^f$, we surely have

\Centerline{$\int\int_Uf(x)g(y)h(xy)dxdy
=\sup_{V\in\Sigma^f}\int_V\int_Uf(x)g(y)h(xy)dxdy$,}

\noindent again because $\mu$ is semi-finite.   Putting these together,
we have the result.\ \Qed

Looking at the other side of the equation,
$\int f(x)g(y)h(xy)d(x,y)=(\mu\times\mu)W$, where
$W=(F\times G)\cap\{(x,y):xy\in H\}$ is a Borel set;  so that

$$\eqalign{\iint f(x)g(y)h(xy)dxdy
&=\sup_{U,V\in\Sigma^f}(\mu\times\mu)((U\times V)\cap W)\cr
&=\sup_{U,V\in\Sigma^f}\int_{U\times V}f(x)g(y)h(xy)d(x,y)\cr}$$

\noindent (417C(iii)).   But now we can apply 417Ha to see that, for any
$U$, $V\in\Sigma^f$,

\Centerline{$\int_{U\times V}f(x)g(y)h(xy)d(x,y)
=\int_V\int_Uf(x)g(y)h(xy)dxdy$.}

\noindent Taking the supremum over $U$ and $V$, we get

\Centerline{$\int f(x)g(y)h(xy)d(x,y)
=\iint f(x)g(y)h(xy)dxdy$.}

\medskip

{\bf (b)} Clearly both sides of our equation

\Centerline{$\int f(x)g(y)h(xy)d(x,y)
=\iint f(x)g(y)h(xy)dxdy$}

\noindent are additive in $f$, $g$ and $h$ separately (subtraction, of
course, will be another matter, as I am allowing $\infty$ to appear
without restriction);  and also behave identically if $f$ or $g$ or $h$
is multiplied by a non-negative scalar.   So the identity will be valid
if $f$, $g$ and $h$ are all finite sums of non-negative multiples of
indicator functions of Borel sets.   Moreover, by repeated use of
B.Levi's theorem, we see that if $\sequencen{f_n}$, $\sequencen{g_n}$
and $\sequencen{h_n}$ are non-decreasing sequences of such functions
with suprema $f$, $g$ and $h$, then

$$\eqalign{\int f(x)g(y)h(xy)d(x,y)
&=\sup_{n\in\Bbb N}\int f_n(x)g_n(y)h_n(xy)d(x,y)\cr
&=\sup_{n\in\Bbb N}\iint f_n(x)g_n(y)h_n(xy)dxdy
=\iint f(x)g(y)h(xy)dxdy.\cr}$$

\noindent So the identity is valid for all non-negative Borel functions
$f$, $g$ and $h$.

\medskip

{\bf (c)} Finally, suppose only that $f$, $g$ and $h$ are non-negative,
measurable and defined almost everywhere.   In this case, by 443J(b-iv),
there are Borel measurable functions $f_0$, $g_0$ and $h_0$,
non-negative, defined everywhere on $X$ and equal almost everywhere to
$f$, $g$ and $h$ respectively.   Let $E$ be the conegligible set

\Centerline{$\{x:x\in\dom f\cap\dom h\cap\dom g,\,
f(x)=f_0(x),\,g(x)=g_0(x),\,h(x)=h_0(x)\}$.}

We find that $\int f(x)h(xy)dx=\int f_0(x)h_0(xy)dx$ for every $y\in X$.
\Prf\ $E\cap Ey^{-1}$ is conegligible (see 443A), and
$f(x)h(xy)=f_0(x)h_0(xy)$ for every $x\in E\cap Ey^{-1}$.\ \QeD\
Consequently

\Centerline{$\iint f(x)g(y)h(xy)dxdy
=\iint f_0(x)g_0(y)h_0(xy)dxdy$.}

Secondly, $f(x)g(y)h(xy)=f_0(x)g_0(y)h_0(xy)\,\,(\mu\times\mu)$-a.e.
\Prf\ Set $W=\{(x,y):x\in E,\,y\in E,\,xy\in E\}$.   Fubini's theorem,
applied to $(U\times V)\setminus W$ where $U$, $V\in\Sigma^f$, shows
that $W$ is conegligible;  but of course
$f(x)g(y)h(xy)=f_0(x)g_0(y)h_0(xy)$ whenever $(x,y)\in W$.\ \QeD\
Accordingly

\Centerline{$\int f(x)g(y)h(xy)d(x,y)
=\int f_0(x)g_0(y)h_0(xy)d(x,y)$.}

Combining this with the result of (b), applied to $f_0$, $g_0$ and
$h_0$, we see that once again

\Centerline{$\int f(x)g(y)h(xy)d(x,y)
=\iint f(x)g(y)h(xy)dxdy$,}

\noindent as required.
}%end of proof of 444N

\leader{444O}{Convolutions of functions:  Theorem} Let $X$ be a
topological group and $\mu$ a left Haar measure on $X$.   For $f$,
$g\in\eusm L^0=\eusm L^0(\mu)$,  write
$(f*g)(x)=\int f(y)g(y^{-1}x)dy$ whenever this is defined in
$\Bbb R$, taking the integral with respect to $\mu$.

(a) Writing $\Delta$ for the left modular function of $X$,

$$\eqalign{(f*g)(x)
&=\int f(y)g(y^{-1}x)dy
=\int f(xy)g(y^{-1})dy\cr
&=\int\Delta(y^{-1})f(y^{-1})g(yx)dy
=\int\Delta(y^{-1})f(xy^{-1})g(y)dy\cr}$$

\noindent whenever any of these integrals is defined in $\Bbb R$.

(b) If $f\eae f_1$ and $g\eae g_1$, then $f*g=f_1*g_1$.

(c)(i) $|(f*g)(x)|\le(|f|*|g|)(x)$ whenever either is defined in $\Bbb R$.

\quad(ii)

\Centerline{$((f_1+f_2)*g)(x)=(f_1*g)(x)+(f_2*g)(x)$,}

\Centerline{$(f*(g_1+g_2))(x)=(f*g_1)(x)+(f*g_2)(x)$,}

\Centerline{$((\alpha f)*g)(x)=(f*(\alpha g))(x)=\alpha(f*g)(x)$}

\noindent whenever the right-hand expressions are defined in $\Bbb R$.

(d) If $f$, $g$ and $h$ belong to $\eusm L^0$ and any of

\Centerline{$\int(|f|*|g|)(x)|h|(x)dx$,
\quad$\iint|f(x)g(y)h(xy)|dxdy$,}

\Centerline{$\iint|f(x)g(y)h(xy)|dydx$,
\quad$\int|f(x)g(y)h(xy)|d(x,y)$}

\noindent is defined in $\coint{0,\infty}$ (writing $\int\ldots d(x,y)$ for
integration with respect to the quasi-Radon product measure
$\mu\times\mu$ on $X\times X$), then

\Centerline{$\int(f*g)(x)h(x)dx$,
\quad$\iint f(x)g(y)h(xy)dxdy$,}

\Centerline{$\iint f(x)g(y)h(xy)dydx$,
\quad$\int f(x)g(y)h(xy)d(x,y)$}

\noindent are all defined, finite and equal, provided that in the
expression $(f*g)(x)h(x)$ we interpret the product as $0$ when $h(x)=0$
and $(f*g)(x)$ is undefined.

(e) If $f$, $g$ and $h$ belong to $\eusm L^0$, $f*g$ and $g*h$ are
defined a.e.\ and $x\in X$ is such that either $(|f|*(|g|*|h|))(x)$ or
$((|f|*|g|)*|h|)(x)$ is defined in $\Bbb R$, then $(f*(g*h))(x)$ and
$((f*g)*h)(x)$ are defined and equal.

(f) If $a\in X$ and $f$, $g\in\eusm L^0$,

\Centerline{$a\action_l(f*g)=(a\action_lf)*g$,
\quad$a\action_r(f*g)=f*(a\action_rg)$,}

\Centerline{$(a\action_rf)*g=\Delta(a^{-1})f*(a^{-1}\action_lg)$,}

\Centerline{$\Reverse{f}*\Reverse{g}=(g*f)\ssplrarrow$.}

(g) If $X$ is abelian then $f*g=g*f$ for all $f$ and $g$.

\proof{{\bf (a)} Use 441J and 442Kb to see that the
four formulae for $f*g$ coincide.

\medskip

{\bf (b)} Setting

\Centerline{$E=\{y:y\in\dom f\cap\dom f_1\cap\dom g\cap\dom g_1$,
$f(y)=f_1(y)$, $g(y)=g_1(y)\}$,}

\noindent $E$ is conegligible.   If $x\in X$, then
$f(y)g(y^{-1}x)=f_1(y)g_1(y^{-1}x)$ for every $y\in E\cap xE^{-1}$,
which is also conegligible, by 443A;  so $(f*g)(x)=(f_1*g_1)(x)$ if
either is defined.

\medskip

{\bf (c)} These are all elementary.

\medskip

{\bf (d)} First consider non-negative $f$, $g$ and $h$.   The point is
that, if any of the integrals is defined and finite,

$$\eqalignno{\int(f*g)(x)h(x)dx
&=\iint\Delta(x^{-1})f(x^{-1})g(xy)h(y)dxdy\cr
&=\iint\Delta(x^{-1})f(x^{-1})g(xy)h(y)dydx\cr
\noalign{\noindent (by 444N, recalling that
$x\mapsto\Delta(x^{-1})f(x^{-1})$ will belong to $\eusm L^0$ if $f$
does, by 442J and 442H)}
&=\iint f(x)g(x^{-1}y)h(y)dydx
=\iint f(x)g(y)h(xy)dydx\cr}$$

\noindent (substituting $xy$ for $y$ in the inner integral, as permitted
by 441J).   (The `and finite' at the beginning of the last sentence is
there because I have changed the rules since the last paragraph, and
$f*g$ is not allowed to take the
value $\infty$.   So we have to take care that

\Centerline{$\{y:h(y)>0,\,\int f(x)g(x^{-1}y)dx=\infty\}$}

\noindent is negligible.)   Now applying 444N again, we get

$$\eqalign{\int(f*g)(x)h(x)dx
&=\iint f(x)g(y)h(xy)dxdy\cr
&=\iint f(x)g(y)h(xy)dydx
=\int f(x)g(y)h(xy)d(x,y)\cr}$$

\noindent if any of these integrals is finite.

For the general case, the hypothesis on $|f|$, $|g|$ and $|h|$ is
sufficient to ensure that the four expressions are equal for any
combination of $f^{\pm}$, $g^{\pm}$ and $h^{\pm}$;  adding and
subtracting these combinations appropriately, we get the result.

\medskip

{\bf (e)} The point is that, for non-negative $f$, $g$ and $h$,

$$\eqalignno{((f*g)*h)(x)
&=\int(f*g)(z)h(z^{-1}x)dz
=\int(f*g)(z)h'(z)dz\cr
\noalign{\noindent (setting $h'(z)=h(z^{-1}x)$)}
&=\iint f(y)g(z)h'(yz)dzdy\cr
\noalign{\noindent (using (d);  to see that $h'$ is measurable,
refer to 443A as usual)}
&=\iint f(y)g(z)h(z^{-1}y^{-1}x)dzdy\cr
&=\int f(y)(g*h)(y^{-1}x)dy
=(f*(g*h))(x)\cr}$$

\noindent at least as long as one of the expressions here is finite.
(Note that, as in 255J, we need to suppose that $f*g$ and $g*h$ are
defined a.e.\ when moving from
$\int(f*g)(z)h(z^{-1}x)dz$ to $\iint f(y)g(z)h(z^{-1}y^{-1})dydz$ and
from $\iint f(y)g(z)h(z^{-1}y^{-1}x)dzdy$ to
$\int f(y)(g*h)(y^{-1}x)dy$, since in part (d) I am more tolerant
of infinities in the repeated integrals than I was in the definition of
$f*g$.)   Once again, subject to the inner integrals implicit in the
formulae $f*(g*h)$ and $(f*g)*h$ being adequately defined, we can
use addition and subtraction to obtain the result for general $f$, $g$
and $h$.

\medskip

{\bf (f)} These are immediate from the formulae in (a), using 442K if
necessary.

\medskip

{\bf (g)} If $X$ is abelian, then $\Delta(y)=1$ for every $y$, so

\Centerline{$(g*f)(x)
=\int g(y)f(y^{-1}x)dy
=\int g(y)\Delta(y^{-1})f(xy^{-1})dy
=(f*g)(x)$}

\noindent if either $(f*g)(x)$ or $(g*f)(x)$ is defined in $\Bbb R$.
}%end of proof of 444O

\leader{444P}{Proposition} Let $X$ be a topological group and $\mu$ a
left Haar measure on $X$.

(a) If $f\in\eusm L^1(\mu)^+$ and $g\in\eusm L^0(\mu)$ then
$f*g$\cmmnt{, as
defined in 444O,} is equal to $(f\mu)*g$\cmmnt{ as defined in 444H}.

(b) If $f\in\eusm L^0(\mu)$ and $g\in\eusm L^1(\mu)^+$ then
$f*g=f*(g\mu)$.

\proof{ Again, these are immediate from the formulae above:

\Centerline{$(f*g)(x)
=\int g(y^{-1}x)f(y)\mu(dy)
=\int g(y^{-1}x)(f\mu)(dy)
=(f\mu*g)(x)$,}

\Centerline{$(f*g)(x)
=\int f(xy^{-1})\Delta(y^{-1})g(y)\mu(dy)
=\int f(xy^{-1})\Delta(y^{-1})(g\mu)(dy)
=(f*g\mu)(x)$}

\noindent whenever these are defined, using 235K, as usual, to calculate
$\int\ldots d(f\mu)$, $\int\ldots d(g\mu)$.   (Note that as we assume throughout
that $f$ and $g$ are defined $\mu$-almost everywhere, all the functions
$y\mapsto g(y^{-1}x)$, $y\mapsto f(xy^{-1})$ are also defined
$\mu$-a.e., by the results set out in 443A.)
}%end of proof of 444P

\leader{444Q}{Proposition} Let $X$ be a topological group and $\mu$ a
left Haar measure on $X$.

(a) Let $f$, $g$ be non-negative
$\mu$-integrable functions.   Then\cmmnt{, defining $f*g$ as in 444O,
we have}
$f*g\in\eusm L^1\cmmnt{=\eusm L^1(\mu)}$ and

\Centerline{$(f\mu)*(g\mu)=(f*g)\mu$.}

(b) For any $f$, $g\in\eusm L^1$, $f*g\in\eusm L^1$ and

\Centerline{$\int f*g\,d\mu=\int fd\mu\int g\,d\mu$,
\quad$\|f*g\|_1\le\|f\|_1\|g\|_1$.}

\proof{{\bf (a)} Putting 444K and 444P together,
$f\mu*g\mu=(f\mu*g)\mu$, so that $f*g=f\mu*g$ is $\mu$-integrable, and

\Centerline{$(f*g)\mu=(f\mu*g)\mu=f\mu*g\mu$.}

\medskip

{\bf (b)} Taking $h=\chi X$ in 444Od, we get
$\int f*g\,d\mu=\int fd\mu\int g\,d\mu$.   Now

\Centerline{$\|f*g\|_1=\int|f*g|\le\int|f|*|g|=\int|f|
\int|g|
=\|f\|_1\|g\|_1$.}

}%end of proof of 444Q

\leader{444R}{Proposition} Let $X$ be a topological group and $\mu$ a
left Haar measure on $X$.   Take any $p\in[1,\infty]$.

(a) If $f\in\eusm L^1(\mu)$ and $g\in\eusm L^p(\mu)$, then
$f*g\in\eusm L^p(\mu)$ and $\|f*g\|_p\le\|f\|_1\|g\|_p$.

(b) $f*\Reverse{g}=(g*\Reverse{f})\ssplrarrow$ for all $f$,
$g\in\eusm L^0$.   If $X$
is unimodular then $\|\Reverse{f}\|_p=\|f\|_p$ for every $f\in\eusm L^0$.

(c) Set $q=\infty$ if $p=1$, $p/(p-1)$ if $1<p<\infty$, $1$ if
$p=\infty$.   If $f\in\eusm L^p(\mu)$ and $g\in\eusm L^q(\mu)$, then
$f*\Reverse{g}$ is defined everywhere in $X$ and is continuous, and
$\|f*\Reverse{g}\|_{\infty}\le\|f\|_p\|g\|_q$.   If $X$ is unimodular, then
$f*g\in C_b(X)$ and $\|f*g\|_{\infty}\le\|f\|_p\|g\|_q$ for every
$f\in\eusm L^p(\mu)$, $g\in\eusm L^q(\mu)$.

\medskip

\noindent{\bf Remark} In the formulae above, interpret $\|g\|_{\infty}$
as $\|g^{\ssbullet}\|_{\infty}=\esssup|g|$ for
$g\in\eusm L^{\infty}=\eusm L^{\infty}(\mu)$,
and as $\infty$ for $g\in\eusm L^0\setminus\eusm L^{\infty}$.
\cmmnt{Because $\mu$ is strictly positive, this agrees with the usual
definition $\|g\|_{\infty}=\sup_{x\in X}|g(x)|$ when $g$ is continuous
and defined everywhere on $X$.}

\proof{{\bf (a)} If $f\ge 0$, then $f*g=(f\mu)*g$ belongs to
$\eusm L^p(\mu)$, and

\Centerline{$\|f*g\|_p=\|f\mu*g\|_p\le(f\mu)(X)\|g\|_p=\|f\|_1\|g\|_p$,}

\noindent by 444Pa and 444Ma.   Generally, $f*g\eae f^+*g-f^-*g$ belongs to
$\eusm L^p(\mu)$ and

\Centerline{$\|f*g\|_p\le\|f^+*g\|_p+\|f^-*g\|_p
\le(\|f^+\|_1+\|f^-\|_1)\|g\|_p=\|f\|_1\|g\|_p$.}

\medskip

{\bf (b)(i)} By 443A once more, $\Reverse{f}\in\eusm L^0$ whenever
$f\in\eusm L^0$.   For $x\in X$,

\Centerline{$(f*\Reverse{g})(x)
=\int f(y)\Reverse{g}(y^{-1}x)dy
=\int\Reverse{f}(y^{-1})g(x^{-1}y)dy
=(g*\Reverse{f})(x^{-1})
=(g*\Reverse{f})\ssplrarrow(x)$}

\noindent if any of these are defined.

\medskip

\quad{\bf (ii)} If $X$ is unimodular, then, for any $f\in\eusm L^0$,

\Centerline{$\|\Reverse{f}\|_p^p
=\int|f(x^{-1})|^pdx
=\int\Delta(x^{-1})|f(x)|^pdx
=\|f\|_p^p$;}

\noindent while $\esssup|\Reverse{f}|=\esssup|f|$ because $E^{-1}$ is
conegligible whenever $E\subseteq X$ is conegligible.

\medskip

{\bf (c)(i)} For any $x\in X$,

\Centerline{$(f*\Reverse{g})(x)
=\int f(y)g(x^{-1}y)dy
=\int f\times(x\action_lg)$}

\noindent in the language of 443G and 444O.   By 443Gb,
$x\action_lg\in\eusm L^q$, so
$(f*\Reverse{g})(x)=\int f\times(x\action_lg)$ is defined.

\medskip

\quad{\bf (ii)} If $p>1$, so that $q<\infty$, then
$x\mapsto(x\action_lg)^{\ssbullet}:X\to L^q$ is continuous (443Gf), so

\Centerline{$x\mapsto(f*\Reverse{g})(x)
=\int f^{\ssbullet}\times(x\action_lg)^{\ssbullet}$}

\noindent is continuous, because $f^{\ssbullet}\in L^p\cong(L^q)^*$.
If $p=1$, then

\Centerline{$(f*\Reverse{g})(x)=\int f(xy)g(y)dy
=\int(x^{-1}\action_lf)\times g$}

\noindent for every $x$;  since
$x\mapsto(x^{-1}\action_lf)^{\ssbullet}:X\to L^1$ is
continuous, so is $f*\Reverse{g}$.

\medskip

\quad{\bf (iii)} If $X$ is unimodular then $f*g=f*\Tilde{\Tilde g}$ is
continuous, because $\Reverse{g}\in\eusm L^q$ by (b), and
$\|f*g\|_{\infty}\le\|f\|_p\|\Reverse{g}\|_q=\|f\|_p\|g\|_q$.
}%end of proof of 444R

\leader{444S}{Remarks} Let $X$ be a topological group and $\mu$ a left Haar
measure on $X$.

\spheader 444Sa From 444Ob and 444Ra we see that we have a
bilinear operator $(u,v)\mapsto u*v:L^1(\mu)\times L^p(\mu)\to L^p(\mu)$
defined by saying that $f^{\ssbullet}*g^{\ssbullet}=(f*g)^{\ssbullet}$ for
every $f\in\eusm L^1(\mu)$ and $g\in\eusm L^p(\mu)$.
\cmmnt{Indeed, 444Ob tells us that} $*$ can\cmmnt{ actually} be
regarded as a function from $L^1\times L^p$ to $\eusm L^p$.
Putting 443Ge together with 444Oe and 444Of, we have

\Centerline{$u*(v*w)=(u*v)*w$,}

\Centerline{$a\action_l(u*w)=(a\action_lu)*w$,
\quad$a\action_r(u*w)=u*(a\action_rw)$,}

\Centerline{$(a\action_ru)*w=\Delta(a^{-1})u*(a^{-1}\action_lw)$}

\noindent whenever $u$, $v\in L^1$, $w\in L^p$ and $a\in X$.

Similarly, if the group is unimodular, and $\bover1p+\bover1q=1$, the
map $*:\eusm L^p\times\eusm L^q\to C_b(X)$\cmmnt{ (444Rc)}
factors through a map from $L^p\times L^q$ to $C_b(X)$.

\spheader 444Sb In particular, $*:L^1\times L^1\to L^1$ is
associative;  evidently it is bilinear;  and $\|u*v\|_1\le\|u\|_1\|v\|_1$
for all $u$, $v\in L^1$.   So $L^1$ is a\cmmnt{ normed algebra;
since $L^1$ is $\|\,\|_1$-complete, it is a} Banach
algebra.   By 444Qb, $\int u*v=\int u\int v$ for all $u$, $v\in L^1$.
$L^1$ is commutative if $X$ is abelian\cmmnt{ (444Og)}.

\spheader 444Sc
Let $\Cal B$ be the Borel $\sigma$-algebra of $X$ and
$M_{\tau}$ the Banach algebra of signed $\tau$-additive Borel
measures on $X$, as in 444E.   If, for $f\in\eusm L^1=\eusm L^1(\mu)$
and $E\in\Cal B$, we
write $(f\mu\restr\Cal B)(E)=\int_Efd\mu$, then
$f\mu\restr\Cal B\in M_{\tau}$\cmmnt{;  for $f\ge 0$, this is because the
indefinite-integral measure $f\mu$ is a quasi-Radon measure, and in general
it is because $f\mu\restr\Cal B=f^+\mu\restr\Cal B-f^-\mu\restr\Cal B$}.
For $f$, $g\in\eusm L^1$, we have

\Centerline{$f^{\ssbullet}=g^{\ssbullet}\text{ in }L^1
\Longrightarrow f\eae g
\Longrightarrow f\mu\restr\Cal B=g\mu\restr\Cal B$,}

\noindent so we have an operator $T:L^1\to M_{\tau}$ defined by setting
$T(f^{\ssbullet})=f\mu\restr\Cal B$ for $f\in\eusm L^1$.\cmmnt{   It
is easy to check that

\Centerline{$(f+g)\mu\restr\Cal B=f\mu\restr\Cal B+g\mu\restr\Cal B$,
\quad$(\alpha f)\mu\restr\Cal B=\alpha(f\mu\restr\Cal B)$,}

\Centerline{$(|f|\mu\restr\Cal B)(E)=\int_E|f|
=\sup_{F\in\Cal B,F\subseteq E}\int_Ff-\int_{E\setminus F}f
=|f\mu\restr\Cal B|(E)$}

\noindent for $f$, $g\in\eusm L^1$, $\alpha\in\Bbb R$ and $E\in\Cal B$, so
that} $T$ is a Riesz homomorphism;  moreover,\cmmnt{

\Centerline{$\|f\mu\restr\Cal B\|
=|f\mu\restr\Cal B|(X)
=(|f|\mu\restr\Cal B)(X)
=\int|f|d\mu
=\|f^{\ssbullet}\|_1$}

\noindent for $f\in\eusm L^1$, so that} $T$ is
norm-preserving.\cmmnt{   Finally, for non-negative $f$, $g\in\eusm L^1$,
we have

$$\eqalignno{Tf^{\ssbullet}*Tg^{\ssbullet}
&=(f\mu\restr\Cal B)*(g\mu\restr\Cal B)
=(f\mu*g\mu)\restr\Cal B\cr
\displaycause{444Eb, since the completions of $f\mu\restr\Cal B$,
$g\mu\restr\Cal B$ are the quasi-Radon indefinite-integral measures $f\mu$,
$g\mu$}
&=(f*g)\mu\restr\Cal B\cr
\displaycause{444Qa}
&=T(f*g)^{\ssbullet}
=T(f^{\ssbullet}*g^{\ssbullet}).\cr}$$

\noindent Thus $Tu*Tv=T(u*v)$ for non-negative $u$, $v\in L^1$;  by
linearity,} $Tu*Tv=T(u*v)$ for all $u$, $v\in L^1$, and $T$ is an embedding
of $L^1$ as a subalgebra of $M_{\tau}$.

\leader{444T}{Proposition} Let $X$ be a topological group and $\mu$ a
left Haar measure on $X$.   Then for any $p\in\coint{1,\infty}$,
$f\in\eusm L^p(\mu)$ and $\epsilon>0$ there is a neighbourhood $U$ of
the identity $e$ in $X$ such that
$\|\nu*f-f\|_p\le\epsilon$ and $\|f*\nu-f\|_p\le\epsilon$ whenever $\nu$
is a quasi-Radon measure on $X$ such that $\nu U=\nu X=1$.

\proof{{\bf (a)} To begin with, suppose that $f$ is non-negative,
continuous and bounded, and that $G=\{x:f(x)>0\}$ has finite measure;
set $M=\sup_{x\in X}f(x)$.    Write $\Cal U$ for the family of
neighbourhoods of $e$.   Take $\delta>0$, $\eta\in\ocint{0,1}$ such that

\Centerline{$(2\delta+(1+\delta)^p-1)^{1/p}\|f\|_p\le\epsilon$,}

\Centerline{$(1-\eta)^p\int((f-\eta\chi X)^+)^pd\mu-M^p\eta
\ge(1-\delta)\int f^pd\mu$,
\quad$(1-\eta)^{(1-p)/p}\le 1+\delta$.}

For each $U\in\Cal U$, set

\Centerline{$H_U=\interior\{x:f(y)\ge f(x)-\eta$ for every
$y\in xU^{-1}\cup U^{-1}x\}$.}

\noindent Then $H_U$ is open and for every $x\in X$ there is a $U\in\Cal
U$ such that $|f(y)-f(x)|\le\bover12\eta$ whenever
$y\in xUU^{-1}\cup U^{-1}xU$, so that $x\in\interior xU\subseteq H_U$.
Thus $\{H_U:U\in\Cal U\}$
is an upwards-directed family of open sets with union $X$, and there is
a $U\in\Cal U$ such that $\mu(G\setminus H_U)\le\eta$;  moreover,
because $\Delta$ is continuous, we can suppose that
$\Delta(y^{-1})\ge 1-\eta$ for every $y\in U$.

Now suppose that $\nu$ is a quasi-Radon measure on $X$ such that
$\nu U=\nu X=1$.   Then, for any $x\in H_U$,

\Centerline{$(\nu*f)(x)
=\int f(y^{-1}x)\nu(dy)
=\int_U f(y^{-1}x)\nu(dy)
\ge f(x)-\eta$}

\noindent because $x\in H_U$ and $y^{-1}x\in U^{-1}x$ whenever $y\in U$.
Similarly,

\Centerline{$(f*\nu)(x)
=\int_U f(xy^{-1})\Delta(y^{-1})\nu(dy)
\ge (f(x)-\eta)(1-\eta)$}

\noindent for every $x\in H_U$.   Now this means that, setting
$h_1=\nu*f$, $h_2=f*\nu$ we have
$(f\wedge h_i)(x)\ge (f(x)-\eta)(1-\eta)$ for every $x\in H_U$ and
both $i$.   Accordingly

$$\eqalign{\int(f\wedge h_i)^pd\mu
&\ge(1-\eta)^p\int_{G\cap H_U}((f-\eta\chi X)^+)^pd\mu\cr
&\ge(1-\eta)^p\int((f-\eta\chi X)^+)^pd\mu-\int_{G\setminus H_U}f^p\cr
&\ge(1-\eta)^p\int((f-\eta\chi X)^+)^pd\mu-M^p\eta
\ge (1-\delta)\int f^pd\mu.\cr}$$

Now, just because $f$ and $h_i$ are non-negative, and $p\ge 1$,

\Centerline{$|f-h_i|^p+2(f\wedge h_i)^p\le f^p+h_i^p$.}

\noindent Also, writing

\Centerline{$\gamma=\int\Delta(y)^{(1-p)/p}\nu(dy)
=\int_U\Delta(y)^{(1-p)/p}\nu(dy)
\le(1-\eta)^{(1-p)/p}
\le 1+\delta$,}

\noindent we have

\Centerline{$\|h_1\|_p=\|\nu*f\|_p\le\|f\|_p$,
\quad $\|h_2\|_p=\|f*\nu\|_p\le\gamma\|f\|_p$}

\noindent (444M), so that $\int h_i^pd\nu\le(1+\delta)^p\int f^pd\mu$
for both $i$, and

\Centerline{$\int|f-h_i|^pd\mu
\le\int f^pd\mu+\int h_i^pd\mu-2\int(f\wedge
h_i)^pd\mu
\le(2\delta+(1+\delta)^p-1)\int f^pd\mu$}

\noindent for both $i$.   But this means that

\Centerline{$\max(\|f-f*\nu\|_p,\|f-\nu*f\|_p)
\le(2\delta+(1+\delta)^p-1)^{1/p}\|f\|_p\le\epsilon$.}

\noindent As $\nu$ is arbitrary, we have found a suitable $U$.

\medskip

{\bf (b)} For any continuous bounded function $f$ such that
$\mu\{x:f(x)\ne 0\}<\infty$, we can find neighbourhoods $U_1$, $U_2$ of
$e$ such that

\Centerline{$\|f^+-\nu*f^+\|_p\le\bover12\epsilon$,
\quad$\|f^+-f^+*\nu\|_p\le\bover12\epsilon$}

\noindent whenever $\nu U_1=\nu X=1$,

\Centerline{$\|f^--\nu*f^-\|_p\le\bover12\epsilon$,
\quad$\|f^--f^-*\nu\|_p\le\bover12\epsilon$}

\noindent whenever $\nu U_2=\nu X=1$.   So we shall have

\Centerline{$\|f-\nu*f\|_p\le\epsilon$,
\quad$\|f-f*\nu\|_p\le\epsilon$}

\noindent whenever $\nu(U_1\cap U_2)=\nu X=1$.

\medskip

{\bf (c)} For general $f\in\eusm L^p(\mu)$, there is a bounded
continuous function $g:X\to\Bbb R$ such that $\mu\{x:g(x)\ne 0\}<\infty$
and $\|f-g\|_p\le\bover14\epsilon$ (415Pa).   Now there is a
neighbourhood $U_1$ of $e$ such that

\Centerline{$\|g-\nu*g\|_p\le\bover14\epsilon$,
\quad$\|g-g*\nu\|_p\le\bover14\epsilon$}

\noindent whenever $\nu U_1=\nu X=1$.   There is also a neighbourhood
$U_2$ of $e$ such that $\Delta(y^{-1})^{(1-p)/p}\le 2$ for every
$y\in U_2$, so that

\Centerline{$\|g*\nu-f*\nu\|_p\le 2\|g-f\|_p\le\bover12\epsilon$}

\noindent whenever $\nu U_2=\nu X=1$.   Since we have

\Centerline{$\|\nu*g-\nu*f\|_p\le\|g-f\|_p\le\bover14\epsilon$}

\noindent whenever $\nu X=1$, we get $\|f-\nu*f\|_p\le\epsilon$,
$\|f=f*\nu\|_p\le\epsilon$ whenever $\nu(U_1\cap U_2)=\nu X=1$.

This completes the proof.
}%end of proof of 444T

\vleader{60pt}{444U}{Corollary} Let $X$ be a topological group and $\mu$ a left
Haar measure on $X$.   For any Haar measurable $E\subseteq X$ such that
$0<\mu E<\infty$, and any $f\in\bigcup_{1\le p<\infty}\eusm L^p(\mu)$,
write

\Centerline{$f_E(x)=\Bover1{\mu E}\int_{xE}fd\mu$,
\quad$f'_E(x)=\Bover1{\mu(Ex)}\int_{Ex}fd\mu$}

\noindent for $x\in X$.  Then, for any $p\in\coint{1,\infty}$,
$f\in\eusm L^p$ and
$\epsilon>0$, there is a neighbourhood $U$ of the identity in $X$ such
that $\|f_E-f\|_p\le\epsilon$ and $\|f'_E-f\|_p\le\epsilon$ whenever
$E\subseteq U$ is a non-negligible Haar measurable set.

\proof{ Take $\delta\in\ooint{0,1}$ such that
$\delta(1-\delta)^{(1-p)/p}\|f\|_p\le\bover12\epsilon$.
By 444T, there is a neighbourhood $U$ of the identity such that
$\|f-f*\nu\|_p\le\bover12\epsilon$, $\|f-\nu*f\|_p\le\epsilon$ whenever
$\nu$ is a quasi-Radon measure on $X$ such that $\nu U=\nu X=1$.
Shrinking $U$ if necessary, we may suppose also that $U=U^{-1}$, that
$\mu U<\infty$
and that $|\Delta(x)-1|\le\delta$ for every $x\in U$, where $\Delta$ is
the left modular function of $X$.   If $E\subseteq U$ and $\mu E>0$,
consider the quasi-Radon measures $\nu$, $\nuprime$, $\Reverse{\nu}$ and
$\Reverse{\nu}'$ on $X$ defined by setting

\Centerline{$\nu F
 =\Bover1{\mu E^{-1}}\int_{E\cap F}\Delta(x^{-1})\mu(dx)$,
\quad$\nuprime F=\Bover{\mu(E\cap F)}{\mu E}$,
\quad$\Reverse{\nu} F=\nu F^{-1}$,
\quad$\Reverse{\nu}' F=\nuprime F^{-1}$}

\noindent whenever these are defined.   (They are quasi-Radon measures
because $\nu$ and $\nuprime$ are totally finite indefinite-integral
measures over $\mu$ and
the map $x\mapsto x^{-1}$ is a homeomorphism.)   Because
$E\subseteq U=U^{-1}$, we have

\Centerline{$\Reverse{\nu} U=\nu U^{-1}
=\Bover1{\mu E^{-1}}\int_E\Delta(x^{-1})\mu(dx)=1=\Reverse{\nu} X$}

\noindent by 442Ka, while

\Centerline{$\Reverse{\nu}' X=\Reverse{\nu}' U=\nuprime U^{-1}=1$.}

\noindent Now consider
$f*\Reverse{\nu}$ and $\Reverse{\nu}'*f$.   For any $x\in X$,

$$\eqalignno{(f*\Reverse{\nu})(x)
&=\int f(xy^{-1})\Delta(y^{-1})\Reverse{\nu}(dy)
=\int f(xy)\Delta(y)\nu(dy)\cr
\noalign{\noindent (because $\Reverse{\nu}$ is the image of $\nu$ under the
map $y\mapsto y^{-1}$)}
&=\Bover1{\mu E^{-1}}\int\chi E(y)\Delta(y^{-1})f(xy)\Delta(y)\mu(dy)\cr
\noalign{\noindent (noting that $\nu$ is an indefinite-integral measure
over $\mu$, and using 235K)}
&=\Bover1{\mu E^{-1}}\int\chi E(x^{-1}y)f(y)\mu(dy)
=\Bover1{\mu E^{-1}}\int_{xE}f(y)\mu(dy)
=\Bover{\mu E}{\mu E^{-1}}f_E(x),\cr
(\Reverse{\nu}'*f)(x)
&=\int f(y^{-1}x)\Reverse{\nu}'(dy)
=\int f(yx)\nuprime(dy)\cr
\noalign{\noindent (because $\Reverse{\nu}'$ is the image of $\nuprime$
under
the map $y\mapsto y^{-1}$)}
&=\Bover1{\mu E}\int\chi E(y)f(yx)\mu(dy)
=\Bover{\Delta(x^{-1})}{\mu E}\int\chi E(yx^{-1})f(y)\mu(dy)\cr
\noalign{\noindent (by 442Kc)}
&=\Bover1{\mu(Ex)}\int_{Ex}f(y)\mu(dy)
=f'_E(x).\cr}$$

So

\Centerline{$\|f-f_E\|_p
\le\|f-f*\Reverse{\nu}\|_p
 +\bigl|\Bover{\mu E^{-1}}{\mu E}-1\bigr|\|f*\Reverse{\nu}\|_p$.}

\noindent Now $\mu E^{-1}=\int_E\Delta(y^{-1})\mu(dy)$, so that

\Centerline{$|\mu E-\mu E^{-1}|\le\int_E|\Delta(y^{-1})-1|\mu(dy)
\le\delta\mu E$,
\quad$|\Bover{\mu E^{-1}}{\mu E}-1|\le\delta$;}

\noindent also

\Centerline{$\|f*\Reverse{\nu}\|_p
\le\|f\|_p\int\Delta(y)^{(1-p)/p}\nu(dy)
\le\|f\|_p(1-\delta)^{(1-p)/p}$}

\noindent by 444Mb, so

\Centerline{$\bigl|\Bover{\mu E^{-1}}{\mu E}-1\bigr|\|f*\Reverse{\nu}\|_p
\le\delta(1-\delta)^{(1-p)/p}\|f\|_p\le\Bover12\epsilon$,}

\noindent and $\|f-f_E\|_p\le\epsilon$.

On the other hand,

\Centerline{$\|f-f'_E\|_p=\|f-\Reverse{\nu}'*f\|_p\le\epsilon$;}

\noindent as $E$ is arbitrary, we have found a suitable $U$.
}%end of proof of 444U

 \leader{444V}{}\cmmnt{ So far I have not emphasized the special
properties of compact groups.   But of course they are the centre of the
subject, and for the sake of a fundamental theorem in \S446 I give the
following result.

\wheader{444V}{4}{2}{2}{72pt}

\noindent}{\bf Theorem} Let $X$ be a compact topological group and $\mu$
a left Haar measure on $X$.

(a) For any $u$, $v\in L^2=L^2(\mu)$ we can interpret their convolution
$u*v$ either as a member of the space $C(X)$ of continuous real-valued
functions on $X$, or as a member of the space $L^2$.

(b) If $w\in L^2$, then $u\mapsto u*w$ is a compact linear operator
whether regarded as a map from $L^2$ to $C(X)$ or as a map from $L^2$ to
itself.

(c) If $w\in L^2$ and $w=\Reverse{w}$\cmmnt{ (as defined in 443Af)}, then
$u\mapsto u*w:L^2\to L^2$ is a self-adjoint operator.

\proof{{\bf (a)} Being compact, $X$ is unimodular (442Ic).   As noted in
444Sa, $*$ can be regarded as a bilinear operator from $L^2\times L^2$ to
$C_b(X)=C(X)$.   Because $\mu X$ must be finite, we now have a natural
map $f\mapsto f^{\ssbullet}$ from $C(X)$ to $L^2$, so that we can think
of $u*v$ as a member of $L^2$ for $u$, $v\in L^2$.

\medskip

{\bf (b)(i)} Evidently $u\mapsto u*w:L^2\to C(X)$ is linear,
for any $w\in L^2$.

\medskip

\quad{\bf (ii)} Let $B$ be the unit ball of $L^2$, and give it the
topology induced by the weak topology $\frak T_s(L^2,L^2)$, so that $B$
is compact (4A4Ka).   Let $\action_l$ be the left action of $X$ on $L^2$
as in 443G and 444S.

If $f$, $g\in\eusm L^2$ and $a\in X$, then

\Centerline{$(f*g)(a)
=\int f(x)g(x^{-1}a)dx
=\int f(x)\Reverse{g}(a^{-1}x)dx
=\int f\times a\action_l\Reverse{g}$.}

\noindent (Note that because $X$ is unimodular, $\Reverse{g}$ and
$a\action_l\Reverse{g}$ are square-integrable whenever $g$ is.)   So
if $u$, $w\in L^2$ and $a\in X$,
$(u*w)(a)=\innerprod{u}{a\action_l\Reverse{w}}$.
It follows that, for any $w\in L^2$, the function
$(a,u)\mapsto(u*w)(a):X\times B\to\Bbb R$ is continuous.   \Prf\ We know
that $\action_l:X\times L^2\to L^2$ is continuous when $L^2$ is given
its norm topology (443Gf).   Now $(u,v)\mapsto\innerprod{u}{v}$ is
continuous, so $(a,u)\mapsto(u*w)(a)=\innerprod{u}{a\action_l\Reverse{w}}$
must be continuous.\ \Qed

Because $X$ is compact, this means that $u\mapsto u*w:B\to C(X)$ is
continuous when $C(X)$ is given its norm topology and $B$ is given the
weak topology (4A2G(g-ii)).   Because $B$ is compact in the weak
topology, $\{u*w:u\in B\}$ is compact in $C(X)$.   But this implies that
$u\mapsto u*w$ is a compact linear operator (definition:  3A5La).

\medskip

\quad{\bf (iii)} Again
because $X$ is compact, $\mu$ is totally finite, so,
for $f\in C(X)$, $\|f\|_2\le\|f\|_{\infty}\sqrt{\mu X}$, and the natural
map $f\mapsto f^{\ssbullet}:C(X)\to L^2$ is a bounded linear operator.
Consequently the map $u\mapsto(u*w)^{\ssbullet}:L^2\to L^2$ is a compact
operator, by 4A4La.

\medskip

{\bf (c)} Now suppose that $w=\Reverse{w}$.   In this case
$\innerprod{u*w}{v}=\innerprod{u}{v*w}$ for all $u$, $v\in L^2$.   \Prf\
Express $u$, $v$ and $w$ as $f^{\ssbullet}$, $g^{\ssbullet}$ and
$h^{\ssbullet}$ where $f$, $g$ and $h$ are square-integrable Borel
measurable functions defined everywhere on $X$.   We have

$$\eqalignno{\innerprod{u*w}{v}
&=\int(f*h)(x)g(x)dx
=\iint f(y)h(y^{-1}x)g(x)dydx\cr
&=\iint f(y)h(y^{-1}x)g(x)dxdy\cr
\noalign{\noindent (because $(x,y)\mapsto f(y)h(y^{-1}x)g(x)$ is Borel
measurable, $\mu$ is totally finite and
\discrversionA{\break}{}$\iint|f(y)h(y^{-1}x)g(x)|dydx
=\innerprod{|u|*|w|}{|v|}$ is finite)}
&=\iint f(y)\Reverse{h}(x^{-1}y)g(x)dxdy
=\int f(y)(g*\Reverse{h})(y)dy\cr
&=\innerprod{u}{v*\Reverse{w}}
=\innerprod{u}{v*w}.\text{ \Qed}\cr}$$

\noindent As $u$ and $v$ are arbitrary, this shows that
$u\mapsto u*w:L^2\to L^2$ is self-adjoint.
}%end of proof of 444V

\exercises{\leader{444X}{Basic exercises $\pmb{>}$(a)}
%\spheader 444Xa
Let $X$ be a Hausdorff topological group.    Show that if $\lambda$ and
$\nu$ are totally finite Radon measures on $X$ then $\lambda*\nu$ is the
image measure $(\lambda\times\nu)\phi^{-1}$, where $\phi(x,y)=xy$ for
$x$, $y\in X$, and in particular is a Radon measure.
%444A

\sqheader 444Xb Let $X$ be a topological group and $\lambda$, $\nu$ two
totally finite quasi-Radon measures on $X$.   Writing $\supp\lambda$ for
the support of $\lambda$, show that
$\supp(\lambda*\nu)=\overline{(\supp\lambda)(\supp\nu)}$.
%444A

\spheader 444Xc Let $X$ be a topological group and
$M^+_{\text{qR}}$ the family of totally finite quasi-Radon measures on
$X$.   Show that $(\lambda,\nu)\mapsto\lambda*\nu:
M^+_{\text{qR}}\times M^+_{\text{qR}}\to M^+_{\text{qR}}$ is continuous
for the narrow topology on $M^+_{\text{qR}}$.   \Hint{437Ma, 437N.}
%444A

\spheader 444Xd Let $X$ be a Hausdorff topological group.   Show that
$X$ is abelian iff its Banach algebra of signed $\tau$-additive Borel
measures is commutative.
%444E

\spheader 444Xe Let $X$ be a topological group, and $M_{\tau}$ its
Banach algebra of signed $\tau$-additive Borel measures.   (i) Show that
we have actions $\action_l$, $\action_r$ of $X$ on $M_{\tau}$ defined by
writing $(a\action_l\nu)(E)=\nu(aE)$, $(a\action_r\nu)(E)=\nu(Ea^{-1})$.
(ii) Show that $(a\action_l\lambda)*\nu=a\action_l(\lambda*\nu)$,
$\lambda*(a\action_r\nu)=a\action_r(\lambda*\nu)$ for all $a\in X$ and
$\lambda$, $\nu\in M_{\tau}$.
%444E

\spheader 444Xf Let $X$ be a compact Hausdorff topological group, and $B$ a
norm-bounded subset of the Banach algebra $M_{\tau}$ of signed
$\tau$-additive Borel measures on $X$.   Show that
$(\lambda,\nu)\mapsto\lambda*\nu:B\times B\to M_{\tau}$ is continuous
for the vague topology on $M_{\tau}$.   \Hint{437Md.}
%444E 444Xc

\spheader 444Xg Let $X$ be a topological group, and $\nu$ a totally
finite quasi-Radon measure on $X$.   Show that for any
Borel sets $E$, $F\subseteq X$, the function
$(g,h)\mapsto\nu(gE\cap Fh)$ is Borel measurable.   \Hint{for Borel sets
$W\subseteq X\times X$,
set $\nuprime W=\nu\{x:(x,x)\in W\}$.   Consider the action of
$X\times X$ on
itself defined by writing $(g,h)\action(x,y)=(gx,yh^{-1})$.}
%444F

\spheader 444Xh Let $X$ be a topological group and $f$ a real-valued
function defined on a subset of $X$.   (i) Show that
$a\action_r(\nu*f)=\nu*(a\action_rf)$ (definition:  4A5Cc) whenever
$a\in X$ and $\nu$ is a
measure on $X$.   (ii) Show that if $X$ carries Haar measures, then
$a\action_l(f*\nu)=(a\action_lf)*\nu$ whenever $a\in X$ and $\nu$ is a
measure on $X$.
%444J

\spheader 444Xi Let $X$ be a topological group carrying Haar measures,
$f:X\to\Bbb R$ a bounded continuous function and $\nu$ a totally finite
quasi-Radon measure on $X$.   Show that $f*\nu$ is continuous.
%444J 444Fe 444Ib

\spheader 444Xj Let $X$ be a topological group carrying Haar measures,
$f$ a real-valued function defined on a subset of $X$, and $\lambda$,
$\nu$ totally finite quasi-Radon measures on $X$.   Show that
$((f*\nu)*\lambda)(x)=(f*(\nu*\lambda))(x)$ whenever the right-hand side
is defined.   (See also 444Yj.)
%444Ic 444J

\spheader 444Xk Let $X$ be an abelian topological group carrying Haar
measures.   Show that $f*\nu=\nu*f$ for every measure $\nu$ on $X$ and
every real-valued function $f$ defined on a subset of $X$.
%444J

\sqheader 444Xl Let $X$ be a topological group and $\mu$ a left Haar
measure on $X$.   (i) Let $\nu$ be a totally finite quasi-Radon measure
on $X$ such that $x\mapsto\nu(xF)$ is continuous for every closed set
$F\subseteq X$.   Show that $\nu$ is truly continuous with respect to
$\mu$.   \Hint{if $\mu F=0$, apply 444K to $\nu*\chi F^{-1}$ to see that
$\nu(xF)=0$ for $\mu$-almost every $x$.}
(ii) Let $\nu$ be a totally finite Radon measure on $X$ such that
$x\mapsto\nu(xK)$ is continuous for every compact set $K\subseteq X$.
Show that $\nu$ is truly continuous with respect to $\mu$.
%444K

\spheader 444Xm Let $X$ be a topological group carrying Haar measures,
$E\subseteq X$ a Haar negligible set and $\nu$ a $\sigma$-finite
quasi-Radon measure on $X$.   Show that $\nu(xE)=\nu(Ex)=0$ for
Haar-a.e.\ $x\in X$.
%444K

\spheader 444Xn Let $X$ be a topological group carrying Haar measures,
and $\nu$ a non-zero totally finite quasi-Radon measure on $X$ such that
$\nu(xE)=0$
whenever $x\in X$ and $\nu E=0$.   (i) Show that $\nu$ is strictly
positive, so that $X$ is ccc.
(ii) Show that a subset of $X$ is $\nu$-negligible iff it is Haar
negligible.
%444L

\spheader 444Xo Use the method of part (b) of the proof of 444M to prove
part (a) there.
%444M

\sqheader 444Xp Let $X$ be the group $S^1\times S^1$, with the topology
defined by giving the first coordinate the usual topology of $S^1$ and
the second coordinate its discrete topology, so that $X$ is a locally
compact abelian group.   Let $\mu$ be a Haar measure on $X$.   (i) Find
a Borel measurable function $f:X\times X\to\{0,1\}$ such that
$\iint f(x,y)\mu(dx)\mu(dy)\ne\iint f(x,y)\mu(dy)\mu(dx)$.   (ii) Let
$\nu$ be the Radon measure on $X$ defined by setting
$\nu E=\#(\{s:(s,s^{-1})\in E\})$ if this is finite, $\infty$ otherwise.
Define $g:X\to\{0,1\}$ by setting $g(s,t)=1$ if $s=t$, $0$ otherwise.
Show that $\iint g(xy)\nu(dy)\mu(dx)=\infty$,
$\iint g(xy)\mu(dx)\nu(dy)=0$.   (iii) Find a closed set $F\subseteq X$
such that $x\mapsto\nu(xF)$ is not Haar measurable.
%444N  mt44bits

\sqheader 444Xq Let $X$ be a Hausdorff topological group and for
$a\in X$ write $\delta_a$ for the Dirac measure on $X$ concentrated at $a$.
(i) Show that $\delta_a*\delta_b=\delta_{ab}$ for all
$a$, $b\in X$.   (ii) Show that, in the notation of 444Xe,
$\delta_a*\hat\nu$ is the completion of $a^{-1}\action_l\nu$ and
$\hat\nu*\delta_a$ is the completion of $a\action_r\nu$ for every
$a\in X$ and every totally finite $\tau$-additive Borel measure $\nu$ on
$X$ with completion $\hat\nu$.   (iii) Show that
$\delta_a*f=a\action_lf$ for every $a\in X$ and every real-valued
function $f$ defined on a subset of $X$.   (iv) Show that if $X$ carries
Haar measures, and has left modular function $\Delta$,
$f*\delta_a=\Delta(a^{-1})a^{-1}\action_rf$ for every $a\in X$ and
every real-valued function $f$ defined on a subset of $X$.   (v) Use
these formulae to relate 444Of to 444B.
%444O

\spheader 444Xr Let $X$ be a topological group and $\mu$ a left Haar
measure on $X$.   Show that if $f$, $g\in\eusm L^0(X)$ then
$(f*g)\ssplrarrow=\Reverse{g}*\Reverse{f}$.
%444O

\spheader 444Xs Let $X$ be a locally compact Hausdorff topological group
and $\mu$ a left Haar measure on $X$.   Show that if $f$, $g:X\to\Bbb R$
are continuous functions with compact support, then $f*g$ is a
continuous function with compact support.
%444R 444Xb

\spheader 444Xt In 444Rc, show that $f*\Reverse{g}$ is uniformly continuous
for the bilateral uniformity.   \Hint{in 443Gf, $x\mapsto x\action_lu$
is uniformly continuous.}
%444R

\spheader 444Xu Let $X$ be a topological group with a totally finite
Haar measure $\mu$.   Show that (i)
$\innerprod{u*w}{v}=\innerprod{u}{v*\Reverse{w}}$ for any $u$, $v$,
$w\in L^2=L^2(\mu)$, where $\Reverse{w}$ and $u*v$ are defined as in 443Af
and 444V (ii) the map $u\mapsto u*w:L^2\to L^2$ is a compact linear
operator for any $w\in L^2$.   \Hint{for (ii), use 443L.}
%444V

\spheader 444Xv Let $X$ be a topological group with a Haar probability
measure $\mu$.   Show that $L^2(\mu)$ with convolution is a Banach
algebra.
%444V

\sqheader 444Xw(i) Let $X_1$, $X_2$ be topological groups with totally
finite quasi-Radon measures $\lambda_i$, $\nu_i$ on $X_i$ for each $i$.
Let $\lambda=\lambda_1\times\lambda_2$, $\nu=\nu_1\times\nu_2$ be the
quasi-Radon product measure on the topological group
$X=X_1\times X_2$.   Show that
$\lambda*\nu=(\lambda_1*\nu_1)\times(\lambda_2*\nu_2)$.
(ii) Let $\familyiI{X_i}$ be a family of topological groups, and
$\lambda_i$, $\nu_i$ quasi-Radon probability measures
on $X_i$ for each $i$.   Let $\lambda=\prod_{i\in I}\lambda_i$,
$\nu=\prod_{i\in I}\nu_i$ be the
quasi-Radon product measures on the topological group
$\prod_{i\in I}X_i$.   Show that
$\lambda*\nu=\prod_{i\in I}\lambda_i*\nu_i$.
%444A out of order query

\sqheader 444Xx Show that 444C, 444O, 444P, 444Qb and
444R-444U %444R 444S 444T 444U
remain valid if we work with complex-valued, rather than real-valued,
functions, and with $\eusm L^p_{\Bbb C}$ and $L^p_{\Bbb C}$
rather than $\eusm L^p$ and $L^p$.

\spheader 444Xy\dvAnew{2010} Let $X$ be a topological group with a left
Haar measure $\mu$ and left modular function $\Delta$.   Write
$\pmb{\Delta}\in L^0=L^0(\mu)$ for the equivalence class of the function
$\Delta$.   For $u\in L^0$ write $u^*$ for
$\Reverse{u}\times\Reverse{\pmb{\Delta}}$.   Show that (i) $(u^*)^*=u$ for
every $u\in L^0$ (ii) $u\mapsto u^*:L^0\to L^0$ is a Riesz space
automorphism (iii) $u^*\in L^1$ for every
$u\in L^1=L^1(\mu)$ (iv) $u\mapsto u^*:L^1\to L^1$ is an $L$-space
automorphism (v) $u^**v^*=(v*u)^*$ for all $u$, $v\in L^1$ (v) defining
$T:L^1\to M_{\tau}$ as in 444Sc, show that $Tu^*=\Reverse{Tu}$
(that is, $(Tu^*)(E)=(Tu)(E^{-1})$ for Borel sets $E$) for every
$u\in L^1$.

\leader{444Y}{Further exercises (a)}
%\spheader 444Ya
Find a subgroup $X$ of $\{0,1\}^{\Bbb N}$ and
quasi-Radon probability measures $\lambda$, $\nu$ on $X$ and a set
$A\subseteq X$ such that $(\lambda*\nu)^*(A)=1$ but
$(\lambda\times\nu)\{(x,y):x,\,y\in X,\,x+y\in A\}=0$.
%444A  %mt44bits

\spheader 444Yb Let $X$ be a {\bf topological semigroup}, that is, a
semigroup with a topology such that multiplication is continuous.
(i) For totally finite
$\tau$-additive Borel measures $\lambda$, $\nu$ on $X$, show that there
is a $\tau$-additive Borel measure $\lambda*\nu$ defined by saying that
$(\lambda*\nu)(E)=(\lambda\times\nu)\{(x,y):xy\in E\}$ for every Borel
set $E\subseteq X$.   (ii) Show that in this context
$(\lambda_1*\lambda_2)*\lambda_3=\lambda_1*(\lambda_2*\lambda_3)$.
(iii) Show that $\int fd(\lambda*\nu)=\int f(xy)\lambda(dx)\nu(dy)$
whenever $f$ is $(\lambda*\nu)$-integrable.   (iv) Show that if the
topology is Hausdorff and $\lambda$ and $\nu$ are tight (that is, inner
regular with respect to the compact sets) so is $\lambda*\nu$.   (v)
Show that we have
a Banach algebra of signed $\tau$-additive Borel measures on $X$, as in
444E.
%444E

\spheader 444Yc Let $X$ be a topological group, and write
$M_{\tau}^{(\Bbb C)}$ for the complexification of the $L$-space
$M_{\tau}$ of 444E, as described in 354Yl.   Show that
$M_{\tau}^{(\Bbb C)}$, with the natural extension of the convolution
operator of 444E, is a complex Banach algebra, and that we still have
$|\lambda*\nu|\le|\lambda|*|\nu|$ for $\lambda$,
$\nu\in M_{\tau}^{(\Bbb C)}$.
%444E

\spheader 444Yd Find a locally compact Hausdorff topological group $X$,
a Radon probability measure $\nu$ on $X$ and an open set $G\subseteq X$
such that $\{(xGx^{-1})^{\ssbullet}:x\in X\}$ is not a separable subset
of the measure algebra of $\nu$.
%444G  %mt44bits

\spheader 444Ye Let $X$ be a metrizable group.
We say that a subset $A$ of $X$ is {\bf Haar null}
%or {\bf shy} watch it:  maybe "shy" demands a shy Borel superset
if there are a universally Radon-measurable set $E\supseteq A$ and a
non-zero Radon measure
$\nu$ on $X$ such that $\nu(xEy)=0$ for every $x$, $y\in X$.   (i) Show
that the family of Haar null sets is a translation-invariant
$\sigma$-ideal of subsets of $X$.
\Hint{if $\sequencen{E_n}$ is a sequence of universally
Radon-measurable Haar null sets, we can find Radon
probability measures $\nu_n$ concentrated on compact sets
near the identity such that
$\nu_n(xE_ny)=0$ for every $x$, $y$ and $n$;  now construct an infinite
convolution product $\nu=\nu_0*\nu_1*\ldots$ from the
probability product of the $\nu_n$ and show that
$\nu(xE_ny)=0$ for every $x$, $y$ and $n$.}   (ii) Show that if $X$ and $Y$
are Polish groups, $\phi:X\to Y$ is a surjective continuous
homomorphism and
$B\subseteq Y$ is Haar null, then $\phi^{-1}[B]$ is Haar null in $X$.
(iii) Show that if $X$ is a
locally compact Polish group then a subset of $X$ is Haar null iff it is
Haar negligible in the sense of 442H.   (See {\smc Solecki 01}.)
%444L

\spheader 444Yf Suppose that the continuum hypothesis is true.   Let $\nu$
be Cantor measure on $\Bbb R$ (256Hc).   Show that there is a set
$A\subseteq\Bbb R$ such that $\nu(x+A)=0$ for every $x\in\Bbb R$, but
$A$ is not Haar negligible.
%444L

\spheader 444Yg Let $X$ be a topological group and $\mu$ a left Haar
measure on $X$.   Let $\tau$ be a $\Cal T$-invariant extended Fatou norm
on $L^0(\mu)$ (\S374).   Show that if $\nu$ is any totally finite
quasi-Radon measure on $X$, then we have a linear operator
$f^{\ssbullet}\mapsto(\nu*f)^{\ssbullet}$ from $L^{\tau}$ to itself, of
norm at most $\nu X$.
%444M

\spheader 444Yh Let $X$ be a topological group with a left Haar measure
$\mu$, $M_{\tau}$ the Banach algebra of signed $\tau$-additive Borel
measures on $X$, and $p\in[1,\infty]$.   (i) Show that we have a
multiplicative linear operator $T$ from $M_{\tau}$ to the Banach algebra
$\eurm B(L^p(\mu);L^p(\mu))$ defined by writing
$(T\nu)(f^{\ssbullet})=(\hat\nu*f)^{\ssbullet}$ whenever $\nu$ is a
totally finite $\tau$-additive Borel measure on $X$ with completion
$\hat\nu$ and $f\in\eusm L^p(\mu)$.   \Hint{Use 444K and 444B to show
that $(\lambda*\nu)*f\eae\lambda*(\nu*f)$ for enough $\lambda$, $\nu$
and $f$.   See also 444Yj.}   (ii) Show that $\|T\nu\|\le\|\nu\|$ for
every $\nu\in M_{\tau}^+$.
(iii) Give an example in which $\|T\nu\|<\|\nu\|$.
%444M

\spheader 444Yi Let $X$ be a unimodular topological group with left Haar
measure $\mu$.   Suppose that $p$, $q$, $r\in[1,\infty]$ are such that
$\bover1p+\bover1q=1+\bover1r$, interpreting $\bover1{\infty}$ as $0$.
Show that if $f\in\eusm L^p(\mu)$ and $g\in\eusm L^q(\mu)$ then
$f*g\in\eusm L^r(\mu)$ and $\|f*g\|_r\le\|f\|_p\|g\|_q$.   \Hint{255Yl.
Take care to justify any changes in order of integration.}
%444O, 444Xt

\spheader 444Yj Let $X$ be a topological group carrying Haar measures.
Investigate conditions under which the associative laws

\Centerline{$\lambda*(\nu*f)=(\lambda*\nu)*f$,
\quad$\lambda*(f*\nu)=(\lambda*f)*\nu$,
\quad$f*(\lambda*\nu)=(f*\lambda)*\nu$,}

\Centerline{$f*(g*\nu)=(f*g)*\nu$,
\quad$f*(\nu*g)=(f*\nu)*g$,
\quad$\nu*(f*g)=(\nu*f)*g$}

\noindent will be valid, where $\lambda$ and $\nu$ are quasi-Radon
measures on $X$ and $f$, $g$ are real-valued functions.   Relate your
results to 444Xq.
%444O

\spheader 444Yk Let $X$ be a topological group with a left Haar measure
$\mu$ and left modular function $\Delta$.   (i) Suppose that
$f\in\eusm L^0(\mu)$.   Show that the following are equiveridical:
($\alpha$) $f(yx)=\Delta(y)f(xy)$ for $(\mu\times\mu)$-almost every
$x$, $y\in X$;
($\beta$) $(a\action_cf)^{\ssbullet}=\Delta(a^{-1})f^{\ssbullet}$
for every $a\in X$.
(ii) Show that in this case $f(x)=0$ for almost every $x$ such that
$\Delta(x)\ne 1$.
(iii) Suppose that $f\in\eusm L^1(\mu)$.   Show that the following are
equiveridical:
($\alpha$) $f(yx)=\Delta(y)f(xy)$ for $(\mu\times\mu)$-almost every
$x$, $y\in X$;
($\beta$) $(f*g)^{\ssbullet}=(g*f)^{\ssbullet}$ for every
$g\in\eusm L^1(\mu)$.
%444O

\spheader 444Yl Let $X$ be a topological group and $\mu$ a left Haar
measure on $X$.   Let $\tau$ be a $\Cal T$-invariant extended Fatou
norm on $L^0(\mu)$ such that $\tau\restr L^{\tau}$ is an
order-continuous norm.   For a totally finite quasi-Radon measure $\nu$
on $X$, let $T_{\nu}:L^{\tau}\to L^{\tau}$ be the corresponding linear
operator (444Yg).   Show that for any $u\in L^{\tau}$ and $\epsilon>0$
there is a neighbourhood $U$ of the identity in $X$ such that
$\tau(T_{\nu}u-u)\le\epsilon$ whenever $\nu U=\nu X=1$.
%444T

\spheader 444Ym Let $X$ be a topological group with a left Haar measure
$\mu$.   For $u\in L^2=L^2(\mu)$, set $A_u=\{a\action_lu:a\in X\}$ (443G)
in $L^2$, and $D=\{v*u:v\in L^1(\mu),\,v\ge 0,\,\int v=1\}$.   (i) Show
that the closed convex hull of $A_u$ in $L^2$ is the closure of $D$.
\Hint{($\alpha$) use 444Od to show that if $w\in L^2$ and
$\innerprod{w'}{w}\ge\gamma$ for every $w'\in A_u$, then
$\innerprod{w'}{w}\ge\gamma$ for every $w'\in D$ ($\beta$) use 444U to
show that $A_u\subseteq\overline{D}$.}   (ii) Show that the closed linear
subspace $W_u$ generated by $A_u$ is the closure of $\{v*u:v\in L^1\}$.
(iii) Show that if $w\in L^2$ and $w\in A_u^{\perp}$, that is,
$\innerprod{u'}{w}=0$ for every $u'\in A$, then
$W_w\subseteq W_u^{\perp}$.
(iv) Show that if $X$ is $\sigma$-compact, then $W_u$ is
separable.   \Hint{$A_u$ is $\sigma$-compact, by 443Gf.}   (v) Set
$C=\{f^{\ssbullet}:f\in\eusm L^2\cap C(X)\}$.   Show that $C\cap W_u$ is
dense in $W_u$.   \Hint{$v*u\in C$ for many $v$, by 444Rc.}   (vi) Show
that if $X$ is $\sigma$-compact, then $W_u$ has an orthonormal basis in
$C$.   (vii) Show that $L^2$ has an orthonormal basis in $C$.   \Hint{if
$X$ is $\sigma$-compact, take a maximal orthogonal family of subspaces
$W_u$, find a suitable orthonormal basis of each, and use (iii) to see
that these assemble to form a basis of $L^2$.   For a general locally
compact Hausdorff group, start with a $\sigma$-compact open subgroup,
and then deal with its cosets.   For a general topological group with a
Haar measure, use 443L.}   (Compare 416Yg.)
%444S

\spheader 444Yn Let $X$ be a topological group with a left Haar measure
$\mu$.   Let $\lambda$ be the quasi-Radon product measure on
$X\times X$.   Let $\eusm U$ be the set of those
$h\in\eusm L^1(\lambda)$ such
that $(X\times X)\setminus\{(x,y):(x,y)\in\dom h,\,h(x,y)=0\}$ can be
covered by a sequence of open sets of finite measure.
(i) Show that if $h\in\eusm U$, then $(x,y)\mapsto h(y,y^{-1}x)$ belongs
to $\eusm U$.   \Hint{443Xa.}
(ii) Show that if
$h\in\eusm U$, then $(Th)(x)=\int h(y,y^{-1}x)\mu(dy)$ is defined for
almost every $x\in X$ and $Th$
is $\mu$-integrable, with $\|Th\|_1\le\|h\|_1$.   \Hint{255Xj.}
(iii) Show that if $h_1$, $h_2\in\eusm U$ are equal
$\lambda$-a.e.\ then $Th_1=Th_2\,\,\mu$-a.e.
(iv) Show that every
member of $L^1(\lambda)$ can be represented by a member of $\eusm U$.
\Hint{443Xk.}
(v) Show that if $f$, $g\in\eusm L^1(\mu)$ and both are
zero outside some countable union of open sets of finite measure, then
$T(f\otimes g)=f*g$, where $(f\otimes g)(x,y)=f(x)g(y)$.
(vi) Show that
if we set $\tilde T(h^{\ssbullet})=(Th)^{\ssbullet}$ for $h\in\eusm U$,
then $\tilde T:L^1(\lambda)\to L^1(\mu)$ is the unique continuous linear
operator such that $\tilde T(u\otimes v)=u*v$ for all $u$,
$v\in L^1(\mu)$, where $u*v$ is defined in 444S and
$\otimes:L^1(\mu)\times L^1(\mu)\to L^1(\lambda)$ is the canonical
bilinear operator (253E).
%444S

\spheader 444Yo In 444Yn, suppose that $\mu X=1$.   (i) Show that the
map $\tilde T$ belongs to the
class $\Cal T_{\bar\lambda,\bar\mu}$ of \S373.   (ii) Show that if
$p\in[1,\infty]$ then $\|Th\|_p\le\|h\|_p$ whenever
$h\in\Cal U\cap\eusm L^p(\lambda)$.
%444Yn, 444S

\spheader 444Yp Rewrite this section in terms of right Haar measures
instead of left Haar measures.
%444+

\spheader 444Yq\dvAnew{2010}
Let $X$ be a topological group and $M_{\text{qR}}^+$ the set of totally
finite quasi-Radon measures on $X$.   For $\nu\in M_{\text{qR}}^+$,
define $\Reverse{\nu}\in M_{\text{qR}}^+$ by
saying that $\Reverse{\nu}(E)=\nu E^{-1}$ whenever $E\subseteq X$ and
$\nu$ measures $E^{-1}$.   (i) Show that if $\lambda$,
$\nu\in M_{\text{qR}}^+$ then
$\Reverse{\lambda}*\Reverse{\nu}=(\nu*\lambda)\ssplrarrow$.
(ii) Taking $\action_l$, $\action_r$ to be the left and right actions of
$X$ on itself, and defining corresponding actions of $X$ on
$M_{\text{qR}}^+$ as in 441Yo, show that
$a\action_l(\lambda*\nu)=(a\action_l\lambda)*\nu$ and
$a\action_r(\lambda*\nu)=\lambda*(a\action_r\nu)$ for $\lambda$,
$\nu\in M_{\text{qR}}^+$ and $a\in X$.
}%end of exercises

\endnotes{
\Notesheader{444} The aim of this section and the next is to work
through ideas from the second half of Chapter 25, and Chapter 28, in
forms natural in the context of general topological groups.   (It is of
course possible to go farther;  see 444Yb.   It is the glory and
confusion of twentieth-century mathematics that it has no firm stopping
points.)   The move from $\Bbb R$ to an arbitrary topological group is a
large one, and I think it is worth examining the various aspects of this
leap as they affect the theorems here.   The most conspicuous change,
and the one which most greatly affects the forms of the results, is the
loss of commutativity.   We are forced to re-examine every formula to
determine exactly which manipulations can still be justified.
Multiplications must be written the right way round, and inversions
especially must be watched.   But while there are undoubtedly some
surprises, we find that in fact (provided we take care over the
definitions) the most important results survive.   Of course I wrote the
earlier results out with a view to what I expected to do here, but no
dramatic manoeuvers are needed to turn the fundamental results 255G,
255H, 255J, 257B, 257E, 257F into the new versions 444Od, 444Qb, 444Oe,
444C, 444B, 444Qa.   (The changed order of presentation is an indication
of the high
connectivity of the web here, not of any new pattern.)   In fact what
makes the biggest difference is not commutativity, as such, but
unimodularity.   In groups which are not unimodular we do have new
phenomena, as in 444Mb and 444Of,
and these lead to complications in the proofs of such results as 444U,
even though the result there is exactly what one would expect.

In this section I ignore right Haar measures entirely.   I do not even
put them in the exercises.   If you wish to take this theory farther,
you may some day have to work out the formulae appropriate to
right Haar measures.   (You can check your results in
{\smc Hewitt \& Ross 63}, 20.32.)   But for the moment, I think that
they are likely
to be just a source of confusion.   There is one point which you may
have noticed.   The theory of groups is essentially symmetric.   In the
definition of `group' there is no distinction between left and right.
In the formulae defining group actions, we do have such a distinction,
because they must reflect the fact that we write $g\action x$ rather
than $x\action g$.
With $\action_l$ and $\action_r$, for instance (444Of), if we want them
to be actions in the standard sense we have to put
an \hbox{$^{-1}$} into the definition of $\action_l$ but not into the
definition of $\action_r$.   But we still expect that, for instance,
$\lambda*\nu$ and $\nu*\lambda$ will be related in some transparent way.
However there is an exception to this rule in the definitions of $\nu*f$
and $f*\nu$ (444H, 444J).   The modular function appears in the latter,
so in fact the definition applies only in a more restricted class of
groups.   In abelian groups we assume that $f*\nu$ and $\nu*f$ will be
equal, and they are (444Xk), but strictly speaking, on the definitions
here, we can write
$f*\nu=\nu*f$ only for abelian topological groups carrying Haar
measures.

From the point of view of the proofs in this section, the principal
change is that the Haar measures here are no longer assumed to be
$\sigma$-finite.   I am well aware that non-$\sigma$-finite measures are
a minority interest, especially in harmonic analysis, but I do think it
interesting that $\sigma$-finiteness is not relevant to the main
results, and the techniques required to demonstrate this are very much
in the spirit of this treatise (see, in particular, the proof of 444N,
and the repeated applications of 443Jb).
The basic difficulty is that we can no longer exchange repeated
integrals, even of non-negative Borel measurable functions, quite
automatically.   Let me emphasize that the result in 444N is really
rather special.   If we try to generalize it to other measures or other
types of function we encounter the usual obstacles (444Xp).

A difficulty of a different kind arises in the proof of 444Fc.   Here I
wish to show that the function
$g\mapsto(g\action E)^{\ssbullet}:G\to\frak A$ is Borel measurable for
every Borel
measurable set $E$.   The first step is to deal with open sets $E$, and
it would be nice if we could then apply the Monotone Class Theorem.
But the difficulty is that even though the map
$(a,b)\mapsto a\Bsetminus b:\frak A\times\frak A\to\frak A$ is
continuous, it does not quite follow that the map

\Centerline{$g\mapsto(g\action(E\setminus F))^{\ssbullet}
=(g\action E)^{\ssbullet}\Bsetminus(g\action F)^{\ssbullet}$}

\noindent is Borel measurable whenever
$g\mapsto(g\action E)^{\ssbullet}$ and
$g\mapsto(g\action F)^{\ssbullet}$ are, because the map
$g\mapsto((g\action E)^{\ssbullet},(g\action F)^{\ssbullet}):
G\to\frak A\times\frak A$ might conceivably fail to be Borel measurable,
if the metric space $\frak A$ is not separable, that is, if the Maharam
type of the measure $\nu$ is uncountable.   Of course the difficulty is
easily resolved by an extra twist in the argument.

I use different techniques for the two parts of 444M as an excuse to
recall the ideas of \S371;  in fact part (a) is slightly easier than
part (b) when proved by the method of the latter (444Xo).

444U is a kind of density theorem.   Compared with the density theorems
in \S\S223 and 261, it is a `mean' rather than `pointwise' density
theorem;  if $E$ is concentrated near the identity, then
$f_E^{\ssbullet}$ approximates $f^{\ssbullet}$ in $L^p$, but there is no
suggestion that we can be sure that $f_E(x)\bumpeq f(x)$ for any
particular $x$s unless we know much more about the set $E$.   In fact
this is to be expected from the form of the results concerning Lebesgue
measure.   The sets $E$ considered in Volume 2 are generally intervals or
balls, and even in such a general form as 223Ya we need a notion of
scalar multiplication separate from the group operation.
}%end of notes

\discrpage

