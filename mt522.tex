\frfilename{mt522.tex}
\versiondate{31.3.10}
\copyrightdate{2003}

\def\chaptername{Cardinal functions of measure theory}
\def\sectionname{Cicho\'n's diagram}

\def\finint{\text{\tt finint}}
\def\disj{\text{\tt disj}}

\newsection{522}

In this section I describe some extraordinary relationships between the
cardinals associated with the ideals of meager and negligible sets in
the real line.   I concentrate on the strikingly symmetric pattern of
Cicho\'n's diagram (522B);  most of the section is taken up with proofs
of the facts encapsulated in this diagram.   I include a handful of
results characterizing some of the most important cardinals here (522C,
522M, 522S), notes on the Martin cardinals associated with the diagram
(522T) and the Freese-Nation number of $\Cal P\Bbb N$ (522U), and a
brief discussion of cofinalities (522V).

\leader{522A}{Notation} In this section, I will use the symbols $\Cal M$
and $\Cal N$ for the ideals of meager and negligible subsets of
$\Bbb R$ respectively.   Associated with these we have the eight
cardinals $\add\Cal M$,
$\cov\Cal M$, $\non\Cal M$, $\cf\Cal M$, $\add\Cal N$, $\cov\Cal N$,
$\non\Cal N$ and $\cf\Cal N$.   In addition we have two
cardinals associated with the partially ordered set $\NN$:  the {\bf
bounding number} $\frak b=\add_{\omega}\NN$\cmmnt{ (see 513H for the
definition of $\add_{\omega}$, and 522C for alternative descriptions of
$\frak b$)} and the {\bf dominating number}
$\frak d=\cf\NN$\cmmnt{;  and finally I should include $\frak c$
itself as an eleventh cardinal in the list to be examined here}.
I use the notions of Galois-Tukey connection and Tukey function, and the
associated relations $\prGT$, $\equivGT$ and $\prT$, as described in
\S\S512-513.

\leader{522B}{Cicho\'n's diagram} The diagram itself is the following:

\def\tmphrule{\hskip0.4em\raise
2.5pt\hbox{\leaders\hrule\hskip1.7em\hfil}\hskip0.4em}
\def\tmpstrut{\vrule height10.5pt depth5.5pt width0pt}

$$\vbox{\offinterlineskip
\halign{\hfil#\hfil&\hfil#\hfil&\hfil#\hfil&\hfil#\hfil
  &\hfil#\hfil&\hfil#\hfil&\hfil#\hfil
  &\hfil#\hfil&\hfil#\hfil&\hfil#\hfil&\hfil#\hfil\cr
&\tmpstrut&$\cov\Cal N$&\tmphrule&$\non\Cal M$&\tmphrule
  &$\cf\Cal M$&\tmphrule&$\cf\Cal N$&\tmphrule&$\frak c$\cr
&\tmpstrut&\vrule&&\vrule&&\vrule&&\vrule\cr
&\tmpstrut&\vrule&&$\frak b$&\tmphrule&$\frak d$&&\vrule\cr
&\tmpstrut&\vrule&&\vrule&&\vrule&&\vrule\cr
$\tmpstrut\omega_1$&\tmphrule&$\add\Cal N$&\tmphrule
  &$\add\Cal M$&\tmphrule&$\cov\Cal M$&\tmphrule&$\non\Cal N$\cr
}}$$

\noindent The cardinals here increase from bottom left to
top right;  that is,

\Centerline{$\omega_1\le\add\Cal N\le\add\Cal M\le\frak b\le\frak d
\le\cf\Cal M\le\cf\Cal N\le\frak c$,}

\noindent etc.   In addition, we have two equalities:

\Centerline{$\add\Cal M=\min(\frak b,\cov\Cal M)$,
\quad$\cf\Cal M=\max(\frak d,\non\Cal M)$.}

\cmmnt{\noindent In the rest of this section I will prove all the
inequalities
declared here, seeking to demonstrate reasons for the remarkable
symmetry of the diagram.   I will make heavy use of the ideas of \S512.
Of course many of the elementary results can be proved directly without
difficulty;  but for the most interesting part of the argument
(522K-522Q below) Tukey functions seem to be the right way to proceed.

I start with the easiest results.   It will be helpful to have
descriptions of $\frak b$ and $\frak d$ in terms of other partially
ordered sets.
}%end of comment

\leader{522C}{Lemma} (i) On $\NN$ define a relation $\le^*$ by saying that
$f\le^*g$ if
if $\{n:f(n)>g(n)\}$ is finite.   Then $\le^*$ is a pre-order on $\NN$;
$\add(\NN,\le^*)=\frak b$ and $\cf(\NN,\le^*)=\frak d$.

(ii) On $\NN$ define a relation $\preceq$ by saying that $f\preceq g$ if
either $f\le g$ or $\{n:g(n)\le f(n)\}$ is finite.
Then $\preceq$ is a partial order on $\NN$, $\add(\NN,\preceq)=\frak b$
and $\cf(\NN,\preceq)=\frak d$.

\proof{{\bf (a)} The checks that $\le^*$ is a pre-order and that $\preceq$
is a partial order are elementary.   Write $\iota$ for the identity map
from $\NN$ to itself.

\medskip

{\bf (b)} For $f\in\NN$ and $A\subseteq\NN$ say that $f\le^{\strprime}A$ if
there is a $g\in A$ such that $f\le g$ (see 512F).   Now
$(\NN,\le^{\strprime},[\NN]^{\le\omega})\prGT(\NN,\le^*,\NN)$.   \Prf\ For
$g\in\NN$, set
$\psi(g)=\{h:g\le^*h\le^*g\}\in[\NN]^{\omega}$.   If $f\le^*g$,
then $f\le f\vee g\in\psi(g)$, so $f\le^{\strprime}\psi(g)$.   Thus
$(\iota,\psi)$ is a Galois-Tukey connection and
$(\NN,\le^{\strprime},[\NN]^{\le\omega})\prGT(\NN,\le^*,\NN)$.\ \Qed

\medskip

{\bf (c)} $(\NN,\le^*,\NN)\prGT(\NN,\preceq,\NN)$.   \Prf\ If $f$,
$g\in\NN$ and $f\preceq g$, then $f\le^*g$;  so $(\iota,\iota)$
is a Galois-Tukey connection from
$(\NN,\le^*,\NN)$ to $(\NN,\preceq,\NN)$.\ \Qed

\medskip

{\bf (d)} If $A\subseteq\NN$ is countable, there is a $\psi(A)\in\NN$
such that $g\preceq\psi(A)$ for every $g\in A$.   \Prf\ If $A$ is empty,
this is trivial.   Otherwise, let $\sequencen{g_n}$ be a sequence
running over $A$, and set $\psi(A)(i)=1+\max_{n\le i}g_n(i)$ for every
$i\in\Bbb N$.\ \Qed

It follows that
$(\NN,\preceq,\NN)\prGT(\NN,\le^{\strprime},[\NN]^{\le\omega})$.
\Prf\ If $A\in[\NN]^{\le\omega}$ and
$f\le^{\strprime}A$, then there is some $g\in A$ such that $f\le g$, so
that $f\preceq\psi(A)$.   Thus $(\iota,\psi)$ is a Galois-Tukey
connection
from $(\NN,\preceq,\NN)$ to $(\NN,\le^{\strprime},[\NN]^{\le\omega})$.\
\Qed

\medskip

{\bf (e)} By 512D,

\Centerline{$\add(\NN,\le^{\strprime},[\NN]^{\le\omega})
=\add(\NN,\le^*,\NN)=\add(\NN,\preceq,\NN)$,}

\Centerline{$\cov(\NN,\le^{\strprime},[\NN]^{\le\omega})
=\cov(\NN,\le^*,\NN)=\cov(\NN,\preceq,\NN)$.}

\noindent But by 513Ia we have

\Centerline{$\add(\NN,\le^{\strprime},[\NN]^{\le\omega})
=\add_{\omega}(\NN)=\frak b$,}

\noindent so $\frak b=\add(\NN,\le^*)=\add(\NN,\preceq)$.   In
the other direction,

$$\eqalignno{\frak d
&=\cf\NN=\cov(\NN,\le,\NN)
\le\max(\omega,\cov(\NN,\le^{\strprime},[\NN]^{\le\omega}))\cr
\displaycause{512Gf}
&\le\max(\omega,\cov(\NN,\le,\NN))\cr
\displaycause{512Gc}
&=\max(\omega,\frak d)=\frak d.\cr}$$

\noindent So

\Centerline{$\frak d=\cov(\NN,\le^{\strprime},[\NN]^{\le\omega})
=\cf(\NN,\le^*)=\cf(\NN,\preceq)$,}

\noindent and the proof is complete.
}%end of proof of 522C



\leader{522D}{Proposition} $\frak b\le\frak d$.

\proof{ Use 511He and 522C.
}%end of proof of 522D

\leader{522E}{Proposition} $\add\Cal N\le\cov\Cal N$,
$\add\Cal M\le\cov\Cal M$, $\non\Cal M\le\cf\Cal M$ and
$\non\Cal N\le\cf\Cal N$.

\proof{ We need only observe that both $\Cal M$ and $\Cal N$ are proper
ideals of $\Cal P\Bbb R$ with union $\Bbb R$, and use 511Jc.}

\leader{522F}{Proposition} $\omega_1\le\add\Cal N$ and
$\cf\Cal N\le\frak c$.

\proof{ Of course $\omega_1\le\add\Cal N$ because $\Cal N$ is a
$\sigma$-ideal of sets.    As for $\cf\Cal N$, we know that the family
of negligible Borel sets is cofinal with $\Cal N$ (134Fb) and has at
most $\frak c$ members (4A3Fa), so $\cf\Cal N\le\frak c$.
}%end of proof of 522F

\leader{522G}{Proposition}\cmmnt{ ({\smc Rothberger 38a})}
$\cov\Cal N\le\non\Cal M$ and $\cov\Cal M\le\non\Cal N$.

\proof{ The point is just that there is a comeager negligible set
$E\subseteq\Bbb R$.   \Prf\ Enumerate $\Bbb Q$ as $\sequencen{q_n}$, and
set

\Centerline{$E
=\bigcap_{n\in\Bbb N}\bigcup_{m\ge n}\ooint{q_n-2^{-n},q_n+2^{-n}}$.
\Qed}

\noindent Because $x\mapsto a+x$ and $x\mapsto a-x$ are
measure-preserving homeomorphisms, $a+E$ is negligible and $a-E$ is
comeager for every $a\in\Bbb R$.   Let $A\subseteq\Bbb R$ be a
non-meager set of size $\non\Cal M$.   Then
$A\cap(a-E)\ne\emptyset$ for every $a\in\Bbb R$, that is,
$\{x+E:x\in A\}$ covers $\Bbb R$;  so $\cov\Cal N\le\#(A)=\non\Cal M$.

For the other inequality, note that $F=\Bbb R\setminus E$ is
conegligible and meager;  so the same argument shows that
$\cov\Cal M\le\non N$.
}%end of proof of 522G

\leader{522H}{Proposition} $\add\Cal M\le\frak b$ and
$\frak d\le\cf\Cal M$.

\proof{{\bf (a)} For $f\in\NN$ and $B\subseteq\NN$ say that
$f\le^{\strprime}B$ if
there is a $g\in B$ such that $f\le g$.   I seek functions
$\phi:\NN\to\Cal M$ and
$\psi:\Cal M\to[\NN]^{\le\omega}$ which will form a Galois-Tukey
connection from $(\NN,\le^{\strprime},[\NN]^{\le\omega})$ to
$(\Cal M,\subseteq,\Cal M)$.
Start by choosing a countable base $\Cal U$ for the topology of
$\Bbb R$, not containing $\emptyset$, and enumerate it as
$\sequence{k}{U_k}$.   For each $k\in\Bbb N$ let $\sequence{l}{V_{kl}}$
be a disjoint sequence of non-empty open subsets of $U_k$;  finally,
enumerate $V_{kl}\cap\Bbb Q$ as $\sequence{i}{x_{kli}}$ for each $k$,
$l\in\Bbb N$.

\medskip

{\bf (b)} Fix $f\in\NN$ for the moment.   Set
$E_k(f)=\{x_{kli}:l\in\Bbb N,\,i\le f(l)\}\subseteq U_k$ for each
$k\in\Bbb N$.
This is nowhere dense because if $G$ is a non-empty open set, either
$G\cap\bigcup_{l\in\Bbb N}V_{kl}=\emptyset$ and
$G\cap E_k(f)=\emptyset$, or there is an $l$ such that $G\cap V_{kl}$ is
non-empty, in which case $G\cap V_{kl}\cap E_k(f)$ is finite and
$G\setminus\overline{E_k(f)}\supseteq G\cap V_{kl}\setminus E_k(f)$ is
non-empty.

Now choose $\sequencen{k_n}$, $\sequencen{l_n}$ inductively as follows.
$k_0=0$.   Given $\langle k_i\rangle_{i\le n}$, $\bigcup_{i\le
n}E_{k_i}(f)$ is
nowhere dense, so there is an $l_n\in\Bbb N$ such that
$\overline{U}_{l_n}\subseteq U_n\setminus\bigcup_{i\le n}E_{k_i}(f)$.
Now if $U_n\subseteq\overline{\bigcup_{i\le n}U_{l_i}}$, set $k_n=0$;
otherwise, take $k_n$ such that
$U_{k_n}\subseteq U_n\setminus\bigcup_{i\le n}U_{l_i}$, and continue.
At the end of the induction, set
$\phi(f)=\overline{\bigcup_{n\in\Bbb N}E_{k_n}}$.

The construction ensures that $\overline{U}_{l_n}\cap E_{k_m}=\emptyset$
for all $m$ and $n$, so that $U_{l_n}$ is always a non-empty open subset
of $U_n\setminus\phi(f)$;
accordingly $\phi(f)$ is nowhere dense.   If  $G\subseteq\Bbb R$ is a
non-empty open set meeting $\phi(f)$, there
is a $k\in\Bbb N$ such that $E_k(f)\subseteq G\cap\phi(f)$.   \Prf\ Let
$n\ge 1$ be such that $U_n\subseteq G$ and
$U_n\cap\phi(f)\ne\emptyset$.   Then there is an $i\in\Bbb N$ such that
$U_n\cap E_{k_i}(f)\ne\emptyset$;
as $E_{k_i}(f)\cap\overline{U}_{l_j}$ is empty for every $j$,
$U_n\not\subseteq\overline{\bigcup_{j\le n}U_{l_j}}$ and

\Centerline{$E_{k_n}(f)\subseteq U_{k_n}\subseteq U_n\subseteq G$,}

\noindent so we can take $k=k_n$.\ \Qed

\medskip

{\bf (c)} In the other direction, given $M\in\Cal M$, choose a sequence
$\sequencen{F_n(M)}$ of closed nowhere dense closed sets covering $M$.
For $k$, $l$, $n\in\Bbb N$ set $g_{nk}(l)=\min\{j:x_{klj}\notin F_n(M)\}$,
and set $\psi(M)=\{g_{nk}:n,\,k\in\Bbb N\}$.

\medskip

{\bf (d)} Now $(\phi,\psi)$ is a Galois-Tukey connection from
$(\NN,\le^{\strprime},[\NN]^{\le\omega})$ to
$(\Cal M,\subseteq,\Cal M)$.   \Prf\ Suppose that $f\in\NN$ and
$M\in\Cal M$ are such that $\phi(f)\subseteq M$.   Because $\phi(f)$ is
closed and not empty and included in $\bigcup_{n\in\Bbb N}F_n(M)$,
Baire's theorem (3A3G or 4A2Ma) tells us that there are $n\in\Bbb N$ and
an open set $G$ such that $\emptyset\ne G\cap\phi(f)\subseteq F_n(M)$.
By the last remark in (b), there is a $k\in\Bbb N$ such that
$E_k(f)\subseteq G\cap\phi(f)$.
But this means that, for any $l\in\Bbb N$, $x_{kli}\in G\cap\phi(f)$
for every
$i\le f(l)$, while if $j=g_{nk}(l)$ then $x_{klj}\notin G\cap\phi(f)$.   So
$f(l)\le g_{nk}(l)$ for every $l$, and $f\le^{\strprime}\psi(M)$.\ \Qed

\medskip

{\bf (e)} It follows at once that

$$\eqalignno{\add\Cal M
&=\add(\Cal M,\subseteq,\Cal M)\cr
\displaycause{512Ea}
&\le\add(\NN,\le^{\strprime},[\NN]^{\le\omega})\cr
\displaycause{512Db}
&=\add_{\omega}\NN\cr
\displaycause{513Ia again}
&=\frak b.\cr}$$

\noindent On the other hand,

$$\eqalignno{\frak d
&=\cf\NN=\cov(\NN,\le,\NN)\cr
\displaycause{512Ea}
&=\cov(\NN,\le^{\strprime},[\NN]^{\le\omega})\cr
\displaycause{512Gf, since $\omega_1\le\add\Cal M\le\frak b\le\frak d$}
&\le\cov(\Cal M,\subseteq,\Cal M)\cr
\displaycause{512Da}
&=\cf\Cal M\cr}$$

\noindent (512Ea again).
}%end of proof of 522H

\leader{522I}{Proposition} $\frak b\le\non\Cal M$ and
$\cov\Cal M\le\frak d$.

\proof{ Let $\preceq$ be the partial order on $\NN$ described in
522C(ii).
Then $([0,1]\setminus\Bbb Q,\in,\Cal M)\prGT(\NN,\preceq,\NN)$.   \Prf\
Let $\phi:[0,1]\setminus\Bbb Q\to\NN$ be a homeomorphism (4A2Ub).   For
$f\in\NN$, set $K_f=\{g:g\le f\}$;  then $K_f$ is compact, so
$\phi^{-1}[K_f]$ is compact.   Because $\phi^{-1}[K_f]$ is disjoint from
$\Bbb Q$, it is nowhere dense.   Set

\Centerline{$\psi(f)
=\bigcup\{\phi^{-1}[K_g]:g\in\NN$, $\{n:g(n)\ne f(n)\}$ is finite$\}\}$.}

\noindent Because there are only countably many functions eventually
equal to $f$, $\psi(f)\in\Cal M$.

Suppose that $x\in[0,1]\setminus\Bbb Q$ and $f\in\NN$ are such that
$\phi(x)\preceq f$.   Set $g=\phi(x)\vee f$;  then $g(n)=f(n)$ for all
but
finitely many $n$, and $\phi(x)\le g$, so
$x\in\phi^{-1}[K_g]\subseteq\psi(f)$.   This shows that $(\phi,\psi)$ is
a Galois-Tukey connection from $([0,1]\setminus\Bbb Q,\in,\Cal M)$ to
$(\NN,\preceq,\NN)$, so that
$([0,1]\setminus\Bbb Q,\in,\Cal M)\prGT(\NN,\preceq,\NN)$.\ \Qed

It follows (using 522C(ii)) that

$$\eqalign{\frak b
&=\add(\NN,\preceq,\NN)
\le\add([0,1]\setminus\Bbb Q,\in,\Cal M)\cr
&=\min\{\#(A):A\subseteq[0,1]\setminus\Bbb Q,\,A\notin\Cal M\}
\le\non\Cal M,\cr}$$

\Centerline{$\frak d=\cov(\NN,\preceq,\NN)
\ge\cov([0,1]\setminus\Bbb Q,\in,\Cal M)
=\min\{\Cal A:\Cal A\subseteq\Cal M$,
$[0,1]\setminus\Bbb Q\subseteq\bigcup\Cal A\}$.}

\noindent But if we take $\Cal A\subseteq\Cal M$ such that
$[0,1]\setminus\Bbb Q\subseteq\bigcup\Cal A$ and $\#(\Cal A)\le\frak d$,
and set

\Centerline{$\Cal A'
=\{A+n:A\in\Cal A$, $n\in\Bbb Z\}\cup\{\{q\}:q\in\Bbb Q\}$,}

\noindent then $\Cal A'\subseteq\Cal M$ is a cover of $\Bbb R$ and

\Centerline{$\cov\Cal M
\le\#(\Cal A')\le\max(\#(\Cal A),\omega)\le\frak d$.}
}%end of proof of 522I

\leader{522J}{Theorem}\cmmnt{ (see {\smc Truss 77} and
{\smc Miller 81})} $\add\Cal M=\min(\frak b,\cov\Cal M)$ and
$\cf\Cal M=\max(\frak d,\non\Cal M)$.

\proof{ My aim this time is to prove that

\Centerline{$(\Cal M,\subseteq,\Cal M)
\prGT(\Bbb R,\in,\Cal M)^{\perp}
\ltimes(\NN,\le^{\strprime},[\NN]^{\le\omega})$,}

\noindent defining $\le^{\strprime}$ as in the proof of 522H and
$\ltimes$ as in 512I.

\medskip

{\bf (a)} Let $\sequencen{q_n}$ be a sequence running over $\Bbb Q$ with
cofinal repetitions.   For
$f\in\NN$, set

\Centerline{$E_f
=\Bbb R\setminus\bigcap_{n\in\Bbb N}\bigcup_{m\ge n}
  \ooint{q_m-2^{-f(m)},q_m+2^{-f(m)}}$,}

\noindent so that $E_f$ is a meager set disjoint from $\Bbb Q$.
Observe that if $\sequencen{H_n}$ is any sequence of closed sets
disjoint from $\Bbb Q$, then there is an $f\in\NN$ such that
$\bigcup_{n\in\Bbb N}H_n\subseteq E_f$.   \Prf\ For each $n\in\Bbb N$,
let $f(n)$ be such that
$\ooint{q_n-2^{-f(n)},q_n+2^{-f(n)}}$ does not meet
$\bigcup_{m\le n}H_m$.\ \Qed

For $M\in\Cal M$, choose a sequence $\sequencen{F_n(M)}$ of nowhere
dense closed sets covering $M$.   For $x\in\Bbb R$, if
$\Bbb Q\cap(\bigcup_{n\in\Bbb N}F_n(M)-x)$ is not empty, set
$p_M(x)(n)=0$ for every $n\in\Bbb N$;  otherwise, take $p_M(x)=f$ for
some $f\in\NN$ such that $E_f\supseteq\bigcup_{n\in\Bbb N}F_n(M)-x$.
Now set $\phi(M)=(\bigcup_{n\in\Bbb N}F_n(M)+\Bbb Q,p_M)$.   This
defines $\phi:\Cal M\to\Cal M\times(\NN)^{\Bbb R}$.

\medskip

{\bf (b)} In the other direction, define
$\psi:\Bbb R\times[\NN]^{\le\omega}\to\Cal M$ by setting
$\psi(x,B)=\bigcup_{f\in B}(x+E_f)$ for $x\in\Bbb R$ and
$B\in[\NN]^{\le\omega}$.   Now $(\phi,\psi)$ is a Galois-Tukey
connection from $(\Cal M,\subseteq,\Cal M)$ to
$(\Bbb R,\in,\Cal M)^{\perp}
 \ltimes(\NN,\le^{\strprime},[\NN]^{\le\omega})$.
\Prf\ $(\Bbb R,\in,\Cal M)^{\perp}=(\Cal M,\not\ni,\Bbb R)$, so

\Centerline{$(\Bbb R,\in,\Cal M)^{\perp}
  \ltimes(\NN,\le^{\strprime},[\NN]^{\le\omega})
=(\Cal M\times(\NN)^{\Bbb R},T,\Bbb R\times[\NN]^{\le\omega})$,}

\noindent where $((M,p),(x,B))\in T$ iff $x\notin M$ and $p(x)\le g$ for
some $g\in B$.   Now suppose that $M\in\Cal M$ and
$(x,B)\in\Bbb R\times[\NN]^{\le\omega}$ are such that
$(\phi(M),(x,B))\in T$.   Then
$x\notin\bigcup_{n\in\Bbb N}F_n(M)+\Bbb Q$, so
$\Bbb Q\cap(\bigcup_{n\in\Bbb N}F_n(M)-x)=\emptyset$, while
$p_M(x)\le g$ for some $g\in B$.   But this means that

\Centerline{$E_g\supseteq E_{p_M(x)}\supseteq\bigcup_{n\in\Bbb N}F_n(M)-x
\supseteq M-x$,
\quad$M\subseteq E_g+x\subseteq\psi(x,B)$.}

\noindent As $M$ and $(x,B)$ are arbitrary, $(\phi,\psi)$
is a Galois-Tukey connection, as claimed.\ \Qed

\medskip

{\bf (c)} It follows that

$$\eqalignno{\cf\Cal M
&=\cov(\Cal M,\subseteq,\Cal M)\cr
\displaycause{512Ea}
&\le\cov((\Bbb R,\in,\Cal M)^{\perp}
  \ltimes(\NN,\le^{\strprime},[\NN]^{\le\omega}))\cr
&=\max(\cov(\Bbb R,\in,\Cal M)^{\perp},
  \cov(\NN,\le^{\strprime},[\NN]^{\le\omega}))\cr
\displaycause{512Jb}
&=\max(\add(\Bbb R,\in,\Cal M),\frak d)
=\max(\non\Cal M,\frak d)\cr}$$

\noindent by 512Ed and the calculation in part (e) of the proof of 522H.
On the other hand

$$\eqalignno{\min(\cov\Cal M,\frak b)
&=\min(\cov\Cal M,\add(\NN,\le^{\strprime},[\NN]^{\le\omega}))\cr
&=\min(\cov(\Bbb R,\in,\Cal M),
  \add(\NN,\le^{\strprime},[\NN]^{\le\omega}))\cr
&=\min(\add(\Bbb R,\in,\Cal M)^{\perp},
  \add(\NN,\le^{\strprime},[\NN]^{\le\omega}))\cr
&=\add((\Bbb R,\in,\Cal M)^{\perp}
  \ltimes\cov(\NN,\le^{\strprime},[\NN]^{\le\omega}))\cr
\displaycause{512Jc}
&\le\add(\Cal M,\subseteq,\Cal M)\cr
\displaycause{(b) above and 512Db}
&=\add\Cal M\cr}$$

\noindent(512Ea).   Since we already know from 522E and 522H that
$\add\Cal M\le\min(\frak b,\cov\Cal M)$ and that
$\max(\frak d,\non\Cal M)\penalty-100\le\cf\Cal M$, we have the result.
}%end of proof of 522J

\leader{522K}{Localization}\cmmnt{ The last step in proving the facts
announced in 522B depends on the following construction.}   Let $I$ be
any set.   Write $\Cal S_I$ for the family of sets
$S\subseteq\Bbb N\times I$ such that each vertical section
$S[\{n\}]$ has at most $2^n$ members.    For $f\in I^{\Bbb N}$ and
$S\subseteq\Bbb N\times I$ say that $f\subseteq^*S$ if
$\{n:n\in\Bbb N,\,(n,f(n))\notin S\}$ is finite\cmmnt{;  that is,
$f\setminus S$ is finite, if we identify $f$ with its graph}.
I will say that
the supported relation $(I^{\Bbb N},\subseteq^*,\Cal S_I)$ is the {\bf
$I$-localization relation}.   \cmmnt{By far the most important
case (and the only one needed in this section) is when $I$ is
countably infinite;  when $I=\Bbb N$ I will generally write $\Cal S$ rather
than $\Cal S_{\Bbb N}$.}

\cmmnt{Members of $\Cal S_I$, or similar sets, are sometimes called
{\bf slaloms}.   The particular formula `$\#(S[\{n\}])\le 2^n$' is
convenient
for the results of this section, but it is worth knowing that any other
function diverging to $\infty$ will give rise to equivalent partially
ordered sets.}

\leader{*522L}{Lemma} Let $I$ be an infinite set.
For any $\alpha\in\NN$ write

\Centerline{$\Cal S_I^{(\alpha)}=\{S:S\subseteq \Bbb N\times I,\,
  \#(S[\{n\}])\le\alpha(n)$ for every $n\in\Bbb N\}$.}

\noindent Then $(I^{\Bbb N},\subseteq^*,\Cal S_I^{(\alpha)})
\equivGT(I^{\Bbb N},\subseteq^*,\Cal S_I^{(\beta)})$ whenever $\alpha$,
$\beta\in\NN$ and
$\lim_{n\to\infty}\alpha(n)=
\penalty-100\lim_{n\to\infty}\beta(n)
\penalty-100=\infty$.

\proof{ Let $g\in\NN$ be a strictly increasing sequence such that
$\beta(n)\le\alpha(i)$ whenever $n\in\Bbb N$ and $i\ge g(n)$, and let
$h_n:I\to I^{g(n+1)\setminus g(n)}$ be a bijection for each
$n$.   Define $\phi:I^{\Bbb N}\to I^{\Bbb N}$ by setting
$\phi(f)(n)=h_n^{-1}(f\restr g(n+1)\setminus g(n))$ for $f\in I^{\Bbb N}$
and $n\in\Bbb N$.   Define
$\psi:\Cal S_I^{(\beta)}\to\Cal P(\Bbb N\times I)$ by setting
$\psi(S)=\bigcup_{(n,i)\in S}h_n(i)$, identifying each
$h_n(i)\in I^{g(n+1)\setminus g(n)}
\subseteq(g(n+1)\setminus g(n))\times I$ with a subset of
$\Bbb N\times I$.   Now for $g(n)\le j<g(n+1)$,
$\psi(S)[\{j\}]=\{h_n(i)(j):i\in S[\{n\}]\}$ has at most
$\beta(n)\le\alpha(j)$ members, while $\psi(S)[\{j\}]=\emptyset$ for
$j<g(0)$, so $\psi(S)\in\Cal S_I^{(\alpha)}$ for every
$S\in\Cal S_I^{(\beta)}$.

If $f\in I^{\Bbb N}$ and $S\in\Cal S_I^{(\beta)}$ and
$\phi(f)\subseteq^*S$, then
there is an $n_0\in\Bbb N$ such that $\phi(f)(n)\in S[\{n\}]$ for every
$n\ge n_0$.   So

\Centerline{$f\restr g(n+1)\setminus g(n)
=h_n(\phi(f)(n))\subseteq\psi(S)$}

\noindent for every $n\ge n_0$, $(m,f(m))\in\psi(S)$ for every
$m\ge g(n_0)$ and $f\subseteq^*\psi(S)$.   This means that $(\phi,\psi)$
is a Galois-Tukey connection from $(\NN,\subseteq^*,\Cal S_I^{(\alpha)})$
to $(\NN,\subseteq^*,\Cal S_I^{(\beta)})$.   Similarly,
$(\NN,\subseteq^*,\Cal S_I^{(\beta)})
\prGT(\NN,\subseteq^*,\Cal S_I^{(\alpha)})$
and the two supported relations are equivalent.
}%end of proof

\leader{522M}{Proposition} Let $(\NN,\subseteq^*,\Cal S)$ be the
$\Bbb N$-localization relation.   Then
$(\NN,\subseteq^*,\Cal S)\penalty-100\equivGT(\Cal N,\subseteq,\Cal N)$.

\proof{{\bf (a)} Let $\langle G_{ij}\rangle_{i,j\in\Bbb N}$ be a
stochastically independent family of open subsets of $[0,1]$ such that the
Lebesgue measure $\mu G_{ij}$ of $G_{ij}$ is $2^{-i}$ for all $i$,
$j\in\Bbb N$.   For $f\in\NN$, set
$\phi(f)=\bigcap_{n\in\Bbb N}\bigcup_{m\ge n}G_{m,f(m)}$.
Then $\phi(f)$ is negligible.

For each $E\in\Cal N$, choose a non-empty compact self-supporting set
$K_E\subseteq[0,1]\setminus E$ (416Dc).   Let $\sequencen{W_{En}}$
enumerate a
base for the relative topology on $K_E$ not containing $\emptyset$;
because $K_E$ is self-supporting, no $W_{En}$ is negligible.   Set

\Centerline{$I_{Eni}=\{j:j\in\Bbb N$, $W_{En}\cap G_{ij}=\emptyset\}$}

\noindent for $n$, $i\in\Bbb N$.   Then

\Centerline{$\sum_{i=0}^{\infty}2^{-i}\#(I_{Eni})
=\sum\{\mu G_{ij}:i$, $j\in\Bbb N$, $G_{ij}\cap W_{En}=\emptyset\}$}

\noindent is finite, by the Borel-Cantelli lemma (273K).   For each $n$,
let $k(E,n)\in\Bbb N$ be such that
$2^{-i}\#(I_{Eni})\le 2^{-n-1}$ for $i\ge k(E,n)$, and set

\Centerline{$\psi(E)
=\bigcup_{n\in\Bbb N}\{(i,j):i,\,j\in\Bbb N,\,i\ge k(E,n),
  \,j\in I_{Eni}\}$.}

\noindent Then

$$\#(\{j:(i,j)\in\psi(E)\})
\le\sum_{n\in\Bbb N,k(E,n)\le i}\#(I_{Eni})
\le\sum_{n\in\Bbb N,k(E,n)\le i}2^{-n-1}2^i
\le 2^i$$

\noindent for every $i\in\Bbb N$, so $\psi(E)\in\Cal S$.

Now $(\phi,\psi)$ is a Galois-Tukey connection from
$(\NN,\subseteq^*,\Cal S)$ to $(\Cal N,\subseteq,\Cal N)$.   \Prf\
Suppose that $f\in\NN$ and $E\in\Cal N$ are such that
$\phi(f)\subseteq E$.   Then
$K_E\cap\bigcap_{n\in\Bbb N}\bigcup_{m\ge n}G_{m,f(m)}=\emptyset$.
By Baire's theorem, there is some $m\in\Bbb N$ such that
$\bigcup_{i\ge m}G_{i,f(i)}\cap K_E$ is not dense in $K_E$, that is,
there is an $n\in\Bbb N$ such that
$W_{En}\cap\bigcup_{i\ge m}G_{i,f(i)}=\emptyset$ and $f(i)\in I_{Eni}$
for every $i\ge m$.   But this means that $(i,f(i))\in\psi(E)$ for every
$i\ge\max(m,k(E,n))$, so that $f\subseteq^*\psi(E)$.   As $f$ and $E$
are arbitrary, we have the result.\ \Qed

Thus $(\NN,\subseteq^*,\Cal S)\prGT(\Cal N,\subseteq,\Cal N)$.

\medskip

{\bf (b)} Let $\Cal H$ be the family of finite unions of bounded open
intervals in $\Bbb R$ with rational endpoints.   Then $\Cal H$ is
countable.   For each $n\in\Bbb N$, let $\sequence{i}{H_{ni}}$ be an
enumeration of $\{H:H\in\Cal H,\,\mu H\le 4^{-n}\}$.   Now for each
$E\in\Cal N$ there is an $f\in\NN$ such that
$E\subseteq\bigcap_{n\in\Bbb N}\bigcup_{m\ge n}H_{m,f(m)}$.   \Prf\ For
each $n\in\Bbb N$, let $\sequence{i}{J_{ni}}$ be a sequence of open
intervals with rational endpoints
such that $E\subseteq\bigcup_{i\in\Bbb N}J_{ni}$ and
$\sum_{i=0}^{\infty}\mu J_{ni}\le 2^{-n-1}$.   Re-enumerating
$\langle J_{ni}\rangle_{n\in\Bbb N,i\in\Bbb N}$ as $\sequence{i}{J_i}$,
we have a sequence of open intervals with rational endpoints such that
$\sum_{i=0}^{\infty}\mu J_i\le 1$ and $E\subseteq\bigcup_{i\ge n}J_i$
for every $n$.   Let $\sequencen{k(n)}$ be a strictly increasing
sequence such that $k(0)=0$ and
$\sum_{i=k(n)}^{\infty}\mu J_i\le 4^{-n}$ for every $n\in\Bbb N$.   Then
$V_n=\bigcup_{k(n)\le i<k(n+1)}J_i$ belongs to $\Cal H$ and has measure
at most $4^{-n}$ for each $n$, so we can define $f\in\NN$ by saying that
$H_{n,f(n)}=V_n$ for each $n$, and we shall have an appropriate
function.\ \Qed

We can therefore find a function $\phi:\Cal N\to\NN$ such that
$E\subseteq\bigcap_{n\in\Bbb N}\bigcup_{m\ge n}H_{m,\phi(E)(m)}$ for
every $E\in\Cal N$.   In the reverse direction, define

\Centerline{$\psi(S)
=\bigcap_{n\in\Bbb N}\bigcup\{H_{mi}:m\ge n,\,(m,i)\in S\}$}

\noindent for $S\in\Cal S$;  because

\Centerline{$\sum_{(m,i)\in S}\mu H_{mi}
\le\sum_{m=0}^{\infty}2^m4^{-m}<\infty$,}

\noindent $\psi(S)\in\Cal N$.

Now $(\phi,\psi)$ is a Galois-Tukey connection from
$(\Cal N,\subseteq,\Cal N)$ to $(\NN,\subseteq^*,\Cal S)$.   \Prf\ If
$E\in\Cal N$ and $S\in\Cal S$ are such that $\phi(E)\subseteq^*S$, then

\Centerline{$E
\subseteq\bigcap_{n\in\Bbb N}\bigcup_{m\ge n}H_{m,\phi(E)(m)}
\subseteq\bigcap_{n\in\Bbb N}\bigcup_{m\ge n,(m,i)\in S}H_{mi}
=\psi(S)$.  \Qed}

\noindent So $(\Cal N,\subseteq,\Cal N)\prGT(\NN,\subseteq^*,\Cal S)$
and the proof is complete.
}%end of proof of 522M

\leader{522N}{Lemma} Let $X$ be a topological
space with a countable $\pi$-base.   Then there is for each $n\in\Bbb N$
a countable family $\Cal U_n$ of open subsets of $X$ such
that $\bigcap\Cal V\ne\emptyset$ for every
$\Cal V\in[\Cal U_n]^{\le n}$ and every dense open subset of $X$
includes some member of $\Cal U_n$.

\proof{ Induce on $n$.   Start by taking $\Cal U$ to be a countable
$\pi$-base for the topology of $X$ which is closed under finite unions.
Set $\Cal U_0=\{\emptyset\}$.   For the inductive step to $n+1$, let
$\sequence{i}{H_i}$ be a sequence running over $\Cal U_n$, and set

\Centerline{$\Cal J_i
=\{J:J\subseteq i,\,\bigcap_{j\in J}H_j\ne\emptyset\}$}

\noindent for $i\in\Bbb N$,

\Centerline{$\Cal U_{n+1}
=\{U\cup H_i:i\in\Bbb N,\,U\in\Cal U$,
$U\cap\bigcap_{j\in J}H_j\ne\emptyset$ whenever $J\in\Cal J_i\}$.}

\noindent Then $\Cal U_{n+1}$ is a countable family of open sets.   If
$G\subseteq X$ is a dense open set, let $i\in\Bbb N$ be such that
$H_i\subseteq G$.   Then
$\Cal J_i$ is finite, so we can find a $U\in\Cal U$ such that
$U\subseteq G$ and $U\cap\bigcap_{j\in J}U_j\ne\emptyset$ for every
$J\in\Cal J_i$;  then $U\cup H_i$ belongs to $\Cal U_{n+1}$ and is
included in $G$.
If $\Cal V\subseteq\Cal U_{n+1}$ and $\#(\Cal V)\le n+1$, then if
$\Cal V$ is empty we certainly have $\bigcap\Cal V\ne\emptyset$.
Otherwise, express $\Cal V$ as $\{U_k\cup H_{i(k)}:k\le n\}$ where
$U_k\cap\bigcap_{j\in J}H_j\ne\emptyset$ whenever $J\in\Cal J_{i(k)}$;
do this in such a way that
$i(k)\le i(n)$ for every $k<n$.   By the inductive hypothesis,
$\bigcap_{k<n}H_{i(k)}\ne\emptyset$;  if $i(k)=i(n)$ for some $k<n$,
then of course $\bigcap_{k\le n}H_{i(k)}\ne\emptyset$;  otherwise,
$U_n\cap\bigcap_{k<n}H_{i(k)}\ne\emptyset$.   In either case,
$\bigcap\Cal V$ is non-empty.   So $\Cal U_{n+1}$ has the required
properties and the induction continues.
}%end of proof of 522N
% for `endowments' see mt52bits

\leader{522O}{Proposition} Let $(\NN,\subseteq^*,\Cal S)$ be the
$\Bbb N$-localization relation.   Then
$(\Cal M,\subseteq,\Cal M)\penalty-100\prGT(\NN,\subseteq^*,\Cal S)$.

\proof{ Let $\sequencen{U_n}$ enumerate
a $\pi$-base for the topology of
$\Bbb R$ not containing $\emptyset$.   By 522N, there is for each
$n\in\Bbb N$ a countable family $\Cal V_n$
of open subsets of $U_n$ such that $\bigcap\Cal V\ne\emptyset$ for every
$\Cal V\in[\Cal V_n]^{\le 2^n}$ and every dense open subset of $U_n$
includes some member of $\Cal V_n$.   Enumerate $\Cal V_n$
as $\sequence{m}{U_{nm}}$.

For each $M\in\Cal M$, let $\sequencen{F_n(M)}$ be a non-decreasing
sequence of nowhere dense sets covering $M$, and let $\phi(M)\in\NN$ be
such that $F_n(M)\cap U_{n,\phi(M)(n)}=\emptyset$ for every $n$.
In the other direction, for $S\in\Cal S$ set

\Centerline{$\psi(S)
=\Bbb R\setminus\bigcap_{n\in\Bbb N}
  \bigcup_{m\ge n}(U_m\cap\bigcap_{i\in S[\{m\}]}U_{mi})$;}

\noindent then because $\bigcap_{i\in S[\{m\}]}U_{mi}$ is non-empty for
every $n$, $\bigcup_{m\ge n}(U_m\cap\bigcap_{i\in S[\{m\}]}U_{mi})$
is a dense open set for every $n$, and $\psi(S)$ is meager.

Now $(\phi,\psi)$ is a Galois-Tukey connection from
$(\Cal M,\subseteq,\Cal M)$ to $(\NN,\subseteq^*,\Cal S)$.   \Prf\
Suppose that $M\in\Cal M$ and $S\in\Cal S$ are such that
$\phi(M)\subseteq^*S$.   Let $n\in\Bbb N$ be such that
$\phi(M)(k)\in S[\{k\}]$ for every $k\ge n$.   Then

\Centerline{$F_m(M)\cap\bigcap_{i\in S[\{k\}]}U_{ki}
\subseteq F_k(M)\cap U_{k,\phi(M)(k)}=\emptyset$}

\noindent whenever $k\ge\max(m,n)$, so

\Centerline{$F_m(M)
\subseteq\Bbb R\setminus\bigcup_{k\ge\max(m,n)}\bigcap_{i\in
S[\{k\}]}U_{ki}
\subseteq\psi(S)$}

\noindent for every $m$, and $M\subseteq\psi(S)$.\ \Qed

So we have the result.
}%end of proof of 522O

\leader{522P}{Corollary} $\Cal M\prT\Cal N$.

\proof{ Putting 522M and 522O and 512Cb together, we see that
$(\Cal M,\subseteq,\Cal M)
\prGT(\Cal N,\penalty-100\subseteq,\penalty-100\Cal N)$, that is,
$\Cal M\prT\Cal N$.
}%end of proof of 522P

\leader{522Q}{Theorem}\cmmnt{ ({\smc Bartoszy\'nski 84},
{\smc Raisonnier \& Stern 85})} $\add\Cal N\le\add\Cal M$ and
$\cf\Cal M\le\cf\Cal N$.

\proof{ 522P, 513Ee.
}%end of proof of 522Q

\leader{522R}{The exactness of Cicho\'n's diagram} The list of
inequalities displayed in Cicho\'n's diagram is complete in the
following sense:  it is known that all assignments of the values
$\omega_1$, $\omega_2$ to the eleven cardinals of the diagram which are
allowed by the diagram together with the two equalities
$\add\Cal M=\min(\frak b,\cov\Cal M)$,
$\cf\Cal M=\max(\frak d,\non\Cal M)$ are relatively consistent with the
axioms of ZFC.   \cmmnt{So, for instance, it is possible to have

\Centerline{$\omega_1=\add\Cal N=\cov\Cal N=\add\Cal M=\frak b
=\non\Cal M$,}

\Centerline{$\cov\Cal M=\frak d=\cf\Cal M=\non\Cal N=\cf\Cal N
=\frak c=\omega_2$.}

\noindent In \S\S552 and 554 below I will describe forcing constructions
demonstrating a few of these combinations;
for the rest, I refer you to {\smc Bartoszy\'nski \& Judah 95},
\S\S5.2, 7.5 and 7.6.   I remark also that
the forcing methods so far known are not all effective beyond
$\omega_2$, so that if we allow $\frak c=\omega_3$ then some puzzles
remain.}%end of comment

\leader{522S}{The cardinal \dvrocolon{$\cov\Cal M$}}\cmmnt{ All the
cardinals in Cicho\'n's diagram appear in many different ways in
set-theoretic real analysis.   But $\add\Cal N$, the additivity of
Lebesgue measure, the bounding number $\frak b$, and $\cov\Cal M$, the
Nov\'ak number of $\Bbb R$, seem
to be particularly important.   The additivity of measure will play a
large role in the next section.   Here I will give two striking
characterizations of $\cov\Cal M$.

\medskip

\noindent}{\bf Theorem} (a) $n(\Bbb R)=\cov\Cal M=\frakmctbl$.

(b) \cmmnt{({\smc Bartoszy\'nski 87})} $\cov\Cal M$ is the least
cardinal of any set $A\subseteq\NN$ such
that for every $g\in\NN$ there is an $f\in A$ such that $f(n)\ne g(n)$
for every $n\in\Bbb N$.

\proof{{\bf (a)} Because $\Bbb R$ is a Baire space, the Nov\'ak number
$n(\Bbb R)$ is equal to $\cov\Cal M$.   By 517P(d-ii) or
517P(d-iii), $n(\Bbb R)=\frakmctbl$.

\medskip

{\bf (b)} Let $\kappa$ be the smallest cardinal of any $A\subseteq\NN$
such that for every $g\in\NN$ there is an $f\in A$ such that
$f\cap g=\emptyset$, identifying the functions $f$ and $g$ with their
graphs in $\Bbb N\times\Bbb N$.

\medskip

\quad{\bf (i)} Suppose that $A\subseteq\NN$ and that $\#(A)<\cov\Cal M$.
Set $P=\bigcup_{n\in\Bbb N}\BbbN^n$, ordered by extension of functions.
Then $P$ is a non-empty countable partially ordered set.   For each
$f\in A$ set
$Q_f=\{p:p\in P,\,p\cap f\ne\emptyset\}$;  then $Q_f$ is cofinal with
$P$.   Set $\Cal Q=\{Q_f:f\in A\}$.   Then

\Centerline{$\#(\Cal Q)\le\#(A)<\cov\Cal M=\frakmctbl
\le\frak m^{\uparrow}(P)$,}

\noindent so there is an upwards-linked $R\subseteq P$ meeting every
member of $\Cal Q$.   Now $g_0=\bigcup R\subseteq\Bbb N\times\Bbb N$ is
a function;   taking $g\in\NN$ to be any extension of $g_0$ to the whole
of $\Bbb N$, $g\cap f\ne\emptyset$ for every $f\in A$.

As $A$ is arbitrary, this shows that $\kappa\ge\cov\Cal M$.

In particular, $\kappa\ge\omega_1$, as can also be seen by elementary
arguments.

\medskip

\quad{\bf (ii)} Let $\sequencen{K_n}$ be any sequence of non-empty
countable sets, and write $F$ for the set of all functions $f$ such that
$\dom f$ is an infinite subset of $\Bbb N$ and $f(n)\in K_n$ for every
$n\in\dom f$.   Then if $A\in[F]^{<\kappa}$ there is a
$g\in\prod_{n\in\Bbb N}K_n$ such that $f\cap g\ne\emptyset$ for every
$f\in A$.   \Prf\ For each $n\in\Bbb N$, let $F_n$ be the countably
infinite set $\bigcup\{\prod_{n\in I}K_n:I\in[\Bbb N]^{n+1}\}$.
For $f\in F$ and $n\in\Bbb N$ take any $n+1$-element subset of $f$ and call
it $p_f(n)$, so that $p_f(n)\in F_n$.   Now each $F_n$ is countably
infinite, and

\Centerline{$A'=\{p_f:f\in A\}\subseteq\prod_{n\in\Bbb N}F_n\cong\NN$}

\noindent has cardinal less than $\kappa$, so there is a
$\phi\in\prod_{n\in\Bbb N}F_n$ such that $\phi\cap p_f\ne\emptyset$ for
every $f\in A$.

Now choose $\sequence{k}{i_k}$ inductively so that
$i_k\in\dom\phi(k)\setminus\{i_j:j<k\}$ for each $k\in\Bbb N$, and take
$g\in\prod_{n\in\Bbb N}K_n$ such that $g(i_k)=\phi(k)(i_k)$ for every
$k$.   Then for any $f\in A$ there is a $k\in\Bbb N$ such that
$\phi(k)=p_f(k)\subseteq f$, so that $g(i_k)=f(i_k)$ and $f\cap
g\ne\emptyset$,
as required.\ \Qed

\medskip

\quad{\bf (iii)} If $A\subseteq\NN$ and $f_0\in\NN$ and $\#(A)<\kappa$,
then there is a function $g\in\NN$ such that
$g(n+1)\ge f_0(g(n))$ for every $n$ and
$\{n:f(g(n))\le g(n+1)\}$ is infinite for every $f\in A$.   \Prf\ For
$f\in A$ set

\Centerline{$f^*(0)=0$,
\quad$\tilde f(n)=\max_{i\le n}f(i)$,
\quad$f^*(n+1)=n+\tilde f(\tilde f(f^*(n)))$}

\noindent for each $n$, so that $f\le\tilde f$, $\tilde f$ and $f^*$ are
non-decreasing, and $f^*$ is unbounded.   Consider
$B=\{f^*\restr\Bbb N\setminus n:f\in A,\,n\in\Bbb N\}$;  then
$\#(B)\le\max(\#(A),\omega)<\kappa$, so by (ii) (or otherwise) there is
an $h\in\NN$ meeting every member of $B$.   Now $h\cap f^*$ is infinite
for every $f\in A$.   Set

\Centerline{$g(0)=1+h(0)$,
\quad$g(n+1)=1+\max_{i\le n+1}h(i)+\max_{i\le n}f_0(g(i))$}

\noindent for $n\in\Bbb N$, so that
$h(n)<g(n)$ and $f_0(g(n))\le g(n+1)$ for every
$n$, and $g$ is non-decreasing.

\Quer\ Suppose, if possible, that $f\in A$ is such that
$\{n:f(g(n))\le g(n+1)\}$ is finite.
Let $n_0\in\Bbb N$ be such that $f(g(n))\ge g(n+1)$ for every $n\ge n_0$.
If $i\ge n_0$ then

\Centerline{$\tilde f(g(i))\ge f(g(i))\ge g(i+1)$,}

\noindent so if $i\ge n_0$ and $j\in\Bbb N$ are such that $f^*(j)\ge g(i)$,
then

\Centerline{$f^*(j+1)
=\tilde f(\tilde f(f^*(j)))
\ge\tilde f(\tilde f(g(i)))
\ge\tilde f(g(i+1))
\ge g(i+2)$}

\noindent because $\tilde f$ is non-decreasing.   But $f^*$ is also
unbounded;
taking $k$ such that $f^*(k)\ge g(n_0)$, we have $f^*(k+i)\ge g(n_0+2i)$
for every $i\in\Bbb N$;  because both $f^*$ and $g$ are
non-decreasing, this means that $f^*(n)\ge g(n)$ whenever
$n\ge\max(k,2k-n_0)$.   But there must be such an $n$ with
$f^*(n)=h(n)<g(n)$, so this is impossible.\ \Bang

Thus $g$ has the required property.\ \Qed

\medskip

\quad{\bf (iv)} Now suppose that $P$ is a countable partially ordered
set, $\Cal Q$ is a family of cofinal subsets of $P$ with
$\#(\Cal Q)<\kappa$, and $p_0\in P$.   Let $\langle p_i\rangle_{i\ge 1}$
be such that $\sequence{i}{p_i}$ runs over $P$ with cofinal repetitions.
Let $f\in\Bbb N$ be a  strictly increasing function such that whenever
$n\in\Bbb N$ and $i<n$ then there is a $j\in f(n)\setminus n$ such that
$p_i\le p_j$.   For each $Q\in\Cal Q$ let $f_Q\in\NN$ be a strictly
increasing function such that whenever $n\in\Bbb N$ and $i<n$ there is a
$j\in f_Q(n)\setminus n$ such that $p_i\le p_j\in Q$.   By (iii), we can
find a strictly increasing $g\in\NN$ such that
$g(n+1)\ge f(g(n))$ for every $n$ and $I_Q=\{n:g(n+1)\ge f_Q(g(n))\}$ is
infinite for every $Q\in\Cal Q$.

For each $n\in\Bbb N$, set $J_n=g(n+1)\setminus g(n)$, and let $\Phi_n$
be the set of functions $h:g(n)\to J_n$ such that
$p_i\le p_{h(i)}$ for every $i\in J_n$;  because
$g(n+1)\ge f(g(n))$ this is non-empty.   For $Q\in\Cal Q$ and $n\in I_Q$
let $\phi_Q(n)\in\Phi_n$ be such that $p_{\phi_Q(n)(i)}\in Q$ for every
$i<g(n)$;  such a function exists because $g(n+1)\ge f_Q(g(n))$.   Now
all the $\Phi_n$ are countable (indeed finite), so (ii) tells us that
there is a $\phi\in\prod_{n\in\Bbb N}\Phi_n$ such that $\phi\cap\phi_Q$
is non-empty for every $Q\in\Cal Q$.

Define $\sequencen{i_n}$ by setting $i_0=0$ and $i_{n+1}=\phi(n)(i_n)$
for $n\in\Bbb N$;  because $\dom\phi(n)=g(n)$ and $\phi(n)(i)<g(n+1)$
whenever $i<g(n)$, $i_n$ is well-defined for each $n$.   Because
$\phi(n)\in\Phi_n$ for each $n$, $p_{i_n}\le p_{i_{n+1}}$ for each $n$.
If $Q\in\Cal Q$ there is some $n$ such that $\phi(n)=\phi_Q(n)$, so that

\Centerline{$p_{i_{n+1}}=p_{\phi(n)(i_n)}=p_{\phi_Q(n)(i_n)}\in Q$.}

\noindent But this means that $R=\{p_{i_k}:k\in\Bbb N\}$ is an
upwards-linked (indeed, totally ordered) subset of $P$ meeting every
member of $\Cal Q$ and containing $p_0$.   As $p_0$ and $\Cal Q$ are
arbitrary, $\frak m^{\uparrow}(P)\ge\kappa$.   As $P$ is arbitrary,
$\frakmctbl\ge\kappa$ and $\kappa=\frakmctbl=\cov\Cal M$, as claimed.
}%end of proof of 522S

\leader{522T}{Martin numbers} Following the identification of
$\cov\Cal M$ with $\frakmctbl$, we can amalgamate the diagrams in 522B
and 517Ob, as follows:

$$\vbox{\offinterlineskip
\halign{\hfil#\hfil&\hfil#\hfil&\hfil#\hfil&\hfil#\hfil
  &\hfil#\hfil&\hfil#\hfil&\hfil#\hfil&\hfil#\hfil&\hfil#\hfil
  &\hfil#\hfil&\hfil#\hfil&\hfil#\hfil&\hfil#\hfil\cr
&\tmpstrut&&&$\cov\Cal N$&\tmphrule&$\non\Cal M$
  &\tmphrule&$\cf\Cal M$&\tmphrule&$\cf\Cal N$&\tmphrule&$\frak c$\cr
&\tmpstrut&&&\vrule&&\vrule&&\vrule&&\vrule\cr
&\tmpstrut&&&\vrule&&$\frak b$&\tmphrule&$\frak d$&&\vrule\cr
&\tmpstrut&&&\vrule&&\vrule&&\vrule&&\vrule\cr
&\tmpstrut&&&$\add\Cal N$&\tmphrule&$\add\Cal M$
  &\tmphrule&$\cov\Cal M$&\tmphrule&$\non\Cal N$\cr
&\tmpstrut&&&\vrule&&\vrule\cr
&\tmpstrut&&&$\frak m_{\sigma\text{-linked}}$
  &\tmphrule&$\frak p$\cr
&\tmpstrut&&&\vrule&&\vrule\cr
$\tmpstrut\omega_1$&\tmphrule&$\frak m$
  &\tmphrule&$\frak m_{\text{K}}$
  &\tmphrule&$\frak m_{\text{pc}\omega_1}$\cr
}}$$

\proof{ The two new inequalities to be proved are
$\frak m_{\sigma\text{-linked}}\le\add\Cal N$ and
$\frak p\le\add\Cal M$.

\medskip

{\bf (a)} Let $\Cal S^{\infty}$ be the `localization poset'

\Centerline{$\{p:p\subseteq\Bbb N\times\Bbb N$, $\#(p[\{n\}])\le 2^n$
for every $n$, $\sup_{n\in\Bbb N}\#(p[\{n\}])<\infty\}$,}

\noindent ordered by $\subseteq$.   For $p\in\Cal S^{\infty}$ set
$\|p\|=\max_{n\in\Bbb N}\#(p[\{n\}])$.   Then $\Cal S^{\infty}$ is
$\sigma$-linked upwards.   \Prf\ If $p$, $q\in\Cal S^{\infty}$,
$\|p\|\le n$, $\|q\|\le n$ and $p[\{i\}]=q[\{i\}]$ for every $i\le n$,
then $p\cup q\in\Cal S^{\infty}$.   So for any $n\in\Bbb N$ and
$\langle J_i\rangle_{i\le n}\in\prod_{i\le n}[\Bbb N]^{\le 2^i}$ we have an
upwards-linked set

\Centerline{$\{p:p\in\Cal S^{\infty}$, $\|p\|\le n$,
$p[\{i\}]=J_i$ for every $i\le n\}$;}

\noindent as there are only countably many such families
$\langle J_i\rangle_{i\le n}$,
$\Cal S^{\infty}$ is $\sigma$-linked upwards.\ \Qed

Accordingly $\frak m_{\sigma\text{-linked}}
\le\frak m^{\uparrow}(\Cal S^{\infty})$.   Next,
$\frak m^{\uparrow}(\Cal S^{\infty})\le\add(\NN,\subseteq^*,\Cal S)$, where
$(\NN,\subseteq^*,\Cal S)$ is the $\Bbb N$-localization relation.
\Prf\ Suppose that
$A\subseteq\NN$ and $\#(A)<\frak m^{\uparrow}(\Cal S^{\infty})$.   For
each $f\in A$, set $Q_f=\{p:p\in\Cal S^{\infty}$, $f\subseteq^*p\}$.
If $p\in\Cal S^{\infty}$ and $\|p\|=n$, then
$p\subseteq p\cup\{(i,f(i)):i\ge n\}\in Q_f$;  so $Q_f$ is cofinal with
$\Cal S^{\infty}$.   As
$\#(\{Q_f:f\in A\})<\frak m^{\uparrow}(\Cal S^{\infty})$, there is an
upwards-directed $R\subseteq\Cal S^{\infty}$ meeting $Q_f$ for every
$f\in A$.   Set $S=\bigcup R$.   For each $n\in\Bbb N$,
$\{p[\{n\}]:p\in R\}$ is an upwards-directed family of subsets of
$\Bbb N$, all of size at most $2^n$, with union $S[\{n\}]$.   So
$\#(S[\{n\}])\le 2^n$;  as $n$ is arbitrary, $S\in\Cal S$.   If
$f\in A$, there is a
$p\in R\cap Q_f$, and now $f\subseteq^*p\subseteq S$.
As $A$ is arbitrary, we have the result.\ \Qed

Now

$$\eqalignno{\frak m_{\sigma\text{-linked}}
&\le\frak m^{\uparrow}(\Cal S^{\infty})
\le\add(\NN,\subseteq^*,\Cal S)
=\add(\Cal N,\subseteq,\Cal N)\cr
\displaycause{522M, 512Db}
&=\add\Cal N,\cr}$$

\noindent as required.

\medskip

{\bf (b)(i)} Let $\Cal U$ be a countable base for the topology of
$\Bbb R$, not containing $\emptyset$.   Consider the set $P$ of pairs
$(\sigma,F)$ where $\sigma\in\bigcup_{n\in\Bbb N}\Cal U^n$ and
$F\subseteq\Bbb R$ is nowhere dense, together with the relation $\le$
where $(\sigma,F)\le(\sigma',F')$ if $\sigma'$ extends $\sigma$,
$F'\supseteq F$ and $F\cap\sigma'(i)=\emptyset$ whenever
$i\in\dom\sigma'\setminus\dom\sigma$.   Then $\le$ is a partial order on
$P$.   \Prf\ If $(\sigma,F)\le(\sigma',F')\le(\sigma'',F'')$ then we
surely have $\sigma\subseteq\sigma'\subseteq\sigma''$ and
$F\subseteq F'\subseteq F''$.   If
$i\in\dom\sigma''\setminus\dom\sigma$, then either
$i\in\dom\sigma'\setminus\dom\sigma$ and $\sigma''(i)=\sigma'(i)$ must
be disjoint from $F$, or $i\in\dom\sigma''\setminus\dom\sigma'$ and
$\sigma''(i)$ must be disjoint from $F'\supseteq F$.   Thus in either
case $F\cap\sigma''(i)=\emptyset$;  as $i$ is arbitrary,
$(\sigma,F)\le(\sigma'',F'')$.   Thus $\le$ is transitive.   Evidently
it is also reflexive and anti-symmetric, so it is a partial order.\ \Qed

\medskip

\quad{\bf (ii)} $(P,\le)$ is $\sigma$-centered upwards.   \Prf\ If
$(\sigma,F_0),\ldots,(\sigma,F_k)$ are members of $P$ with a common
first member, then they have a common upper bound
$(\sigma,\bigcup_{i\le k}F_i)$ in $P$.   So for any $n\in\Bbb N$ and
$\sigma\in\Cal U^n$ the set
$\{(\sigma,F):F\subseteq\Bbb R$ is nowhere dense$\}$ is upwards-centered
in $P$;  as $\bigcup_{n\in\Bbb N}\Cal U^n$ is countable, $P$ is
$\sigma$-centered upwards.\ \Qed

\medskip

\quad{\bf (iii)} For each $V\in\Cal U$ and $n\in\Bbb N$ set

\Centerline{$Q_{nV}=\{(\sigma,F):(\sigma,F)\in P$,
$V\cap\bigcup_{n\le i<\dom\sigma}\sigma(i)\ne\emptyset\}$.}

\noindent Then $Q_{nV}$ is cofinal with $P$.   \Prf\ If
$(\sigma,F)\in P$, set $m=\max(n,\dom\sigma)+1$, and take $U\in\Cal U$
such that $U\subseteq V\setminus F$.   Setting

$$\eqalignno{\sigma'(i)
&=\sigma(i)\text{ for }i<\dom\sigma,\cr
&=U\text{ for }\dom\sigma\le i<m,\cr}$$

\noindent we find that $(\sigma,F)\le(\sigma',F)\in Q_{nV}$.\ \Qed

For each nowhere dense set $H\subseteq\Bbb R$,

\Centerline{$Q'_H=\{(\sigma,F):(\sigma,F)\in P$, $H\subseteq F\}$}

\noindent is cofinal with $P$.   \Prf\ For any $(\sigma,F)\in P$, we
have $(\sigma,F)\le(\sigma,F\cup H)\in Q'_H$.\ \Qed

\medskip

\quad{\bf (iv)} Now suppose that $\Cal A\subseteq\Cal M$ and
$\#(\Cal A)<\frak p$.   Then each member of $\Cal A$ is covered by a
sequence of nowhere dense sets, so there is a family $\Cal H$ of nowhere
dense sets with the same union as $\Cal A$ and with
$\#(\Cal H)\le\max(\omega,\#(\Cal A))$.   In this case

\Centerline{$\Cal Q
=\{Q_{nV}:n\in\Bbb N$, $V\in\Cal U\}\cup\{Q'_H:H\in\Cal H\}$}

\noindent is a family of cofinal subsets of $P$ and

\Centerline{$\#(\Cal Q)\le\max(\omega,\#(\Cal A))<\frak p
\le\frak m^{\uparrow}(P)$.}

\noindent There is therefore an upwards-directed $R\subseteq P$ meeting
every member of $\Cal Q$.   If $(\sigma,F)$ and $(\sigma',F')$ belong to
$R$, they must be upwards-compatible in $P$, and in particular $\sigma$
and $\sigma'$ have a common extension;  we therefore have a function
$\phi=\bigcup_{(\sigma,F)\in R}\sigma$ from a subset of $\Bbb N$ to
$\Cal U$.   If $n\in\Bbb N$ and $V\in\Cal U$, then there is a
$(\sigma,F)\in R\cap Q_{nV}$, so that there is some $i\ge n$ such that
$\phi(i)=\sigma(i)$ meets $V$.   As $V$ is arbitrary, the open set
$W_n=\bigcup_{i\in\dom\phi,i\ge n}\phi(i)$ is dense;  as $n$ is arbitrary,
$M=\Bbb R\setminus\bigcap_{n\in\Bbb N}W_n$ is meager.   Now
$H\subseteq M$ for every $H\in\Cal H$.   \Prf\ There is a
$(\sigma,F)\in R\cap Q'_H$.   Set $n=\dom\sigma$.   If
$i\in\dom\phi\setminus n$, there is a $(\sigma',F')\in R$ such that
$i\in \dom\sigma'$;  because $R$ is upwards-directed, we may suppose
that $(\sigma,F)\le(\sigma',F')$.   But in this case
$\phi(i)=\sigma'(i)$ must be disjoint from $F$ and therefore from $H$.
As $i$ is arbitrary, $H\cap W_n=\emptyset$ and $H\subseteq M$.\ \Qed

As $H$ is arbitrary, $\bigcup\Cal A=\bigcup\Cal H\subseteq M$.   As
$\Cal A$ is arbitrary, $\add\Cal M\ge\frak p$, as claimed.
}%end of proof of 522T

\cmmnt{\medskip

\noindent{\bf Remark} In fact $\frak m^{\uparrow}(\Cal S^{\infty})$ is
exactly equal to $\add\Cal N$;  see 528N.
}%end of comment

\leader{*522U}{\dvrocolon{FN($\Cal P\Bbb N$)}}\cmmnt{ For any cardinal
which is known to lie between $\omega_1$ and $\frak c$, it is natural,
and often profitable, to try to locate it on Cicho\'n's diagram.   For
the Freese-Nation number of $\Cal P\Bbb N$, which appeared more than once
in \S518, we have the following results.

\medskip

\noindent}{\bf Proposition}\cmmnt{ ({\smc Fuchino Koppelberg \&
Shelah 96}, {\smc Fuchino Geschke \& Soukup 01})}
(a) $\FN(\Cal P\Bbb N)\ge\frak b$.

(b) $\FN(\Cal P\Bbb N)\ge\cov\Cal N$.

(c) If $\FN(\Cal P\Bbb N)=\omega_1$ then $\shr\Cal M=\omega_1$, so

\Centerline{$\frak m=\frak m_{\text{K}}=\frak m_{\text{pc}\omega_1}
=\frak m_{\sigma\text{-linked}}=\frak p=\add\Cal N=\add\Cal M
=\frak b=\cov\Cal N=\non\Cal M=\omega_1$.}

(d) If $\FN(\Cal P\Bbb N)=\omega_1$ and $\kappa\ge\frakmctbl$ is such
that $\cff[\kappa]^{\le\omega}\le\kappa\le\frak c$, then 
$\kappa=\frak c$.
So if $\FN(\Cal P\Bbb N)=\omega_1$ and $\frakmctbl<\omega_{\omega}$,
then

\Centerline{$\frakmctbl=\non\Cal N=\frak d=\cf\Cal M=\cf\Cal N
=\frak c$.}

(e) There is a set $A\subseteq\Bbb R$ with cardinal $\frakmctbl$ such that
every meager set meets $A$ in a set with cardinal less than
$\FN^*(\Cal P\Bbb N)$.

\proof{{\bf (a)} Let $\le^*$ and $\preceq$
be the pre-order and partial order on $\NN$ described in
522C, so that $\frak b=\add(\NN,\preceq)$.   Write $\kappa$ for
$\FN(\Cal P\Bbb N)$;  by
518D, $\kappa=\FN(\NN,\le)$ and we have a Freese-Nation function
$\phi:\NN\to[\NN]^{<\kappa}$ for
$\le$.   For $f\in\NN$, set $\psi(f)=\bigcup\{\phi(g):g\le^*f\le^*g\}$;
then $\#(\psi(f))\le\kappa$.

\Quer\ Suppose, if possible, that $\kappa<\frak b$.   Choose
a family $\langle f_{\xi}\rangle_{\xi\le\kappa}$ in $\NN$
inductively, as follows.   Given
$\ofamily{\eta}{\xi}{f_{\eta}}$ where $\xi\le\kappa$,
$\bigcup_{\eta<\xi}\psi(f_{\eta})$ has cardinal at most $\kappa<\frak b$,
so has a $\preceq$-upper bound $f'_{\xi}$;  now set
$f_{\xi}(i)=f'_{\xi}(i)+1$ for every $i$, and continue.

Now choose $\ofamily{\xi}{\kappa}{h_{\xi}}$ in $\phi(f_{\kappa})$
as follows.   For each $\xi<\kappa$, $f_{\xi}\preceq f_{\kappa}$, so if we
set $g_{\xi}=f_{\xi}\wedge f_{\kappa}$ then
$g_{\xi}\le^*f_{\xi}\le^*g_{\xi}$ and
$g_{\xi}\le f_{\kappa}$.   There is therefore an
$h_{\xi}\in\phi(g_{\xi})\cap\phi(f_{\kappa})$ such that
$g_{\xi}\le h_{\xi}\le f_{\kappa}$.   Now if $\eta<\xi<\kappa$,
$h_{\eta}\phi(g_{\eta})\subseteq\in\psi(f_{\eta})$ so
$h_{\eta}\preceq f'_{\xi}$.   Accordingly


\Centerline{$\{i:h_{\xi}(i)\le h_{\eta}(i)\}
\subseteq\{i:f'_{\xi}(i)<h_{\eta}(i)\}\cup\{i:g_{\xi}(i)<f_{\xi}(i)\}$}

\noindent is finite and $h_{\xi}\ne h_{\eta}$.   But this means that
$\{h_{\xi}:\xi<\kappa\}$ has cardinal $\kappa$ and
$\#(\phi(f_{\kappa}))=\kappa$, contrary to the choice of $\phi$.\ \Bang

Thus $\frak b\le\kappa=\FN(\Cal P\Bbb N)$, as claimed.
In particular, $\FN(\Cal P\Bbb N)$ is uncountable.

\medskip

{\bf (b)(i)} We need to know the following fact:  if $\Cal E$ is a
family of non-negligible Lebesgue measurable subsets of $\Bbb R$, and
$\#(\Cal E)<\cov\Cal N$, there is a countable set meeting every member of
$\Cal E$.   \Prf\ For each $E\in\Cal E$, $\Bbb R\setminus(\Bbb Q+E)$ is
negligible (439Eb), so there is an
$x\in\Bbb R\cap\bigcap_{E\in\Cal E}\Bbb Q+E$;  now $\Bbb Q+x$ is countable
and meets every member of $\Cal E$.\ \Qed

\medskip

\quad{\bf (ii)} Set $\kappa=\FN(\Cal P\Bbb N)$.   If $\Cal C$ is the
family of closed sets in $\Bbb R$, then
$(\Cal C,\subseteq)\cong(\frak T,\supseteq)$, so
$\FN(\Cal C)=\FN(\frak T)=\kappa$ (518D).
Let $f:\Cal C\to[\Cal C]^{<\kappa}$ be a
Freese-Nation function.

\medskip

\quad{\bf (iii)} \Quer\ If $\kappa<\cov\Cal N$, write $K$ for the set of
infinite successor cardinals $\lambda\le\kappa$, and for
$\lambda\in K$ set
$D_{\lambda}=\{x:x\in\Bbb R$, $\#(f(\{x\}))<\lambda\}$.   As
$\Bbb R=\bigcup_{\lambda\in K}D_{\lambda}$, there must be
some $\lambda\in K$ such that $D_{\lambda}$ cannot be covered
by $\kappa$ negligible sets.   Choose
$\langle M_{\xi}\rangle_{\xi\le\lambda}$ and
$\langle H_{\xi n}\rangle_{\xi<\lambda,n\in\Bbb N}$ inductively, as
follows.   $M_{\xi}=\emptyset$.   Given that $M_{\xi}\subseteq\Cal C$
and $\#(M_{\xi})\le\kappa$, (i)
tells us that there is a countable set $A_{\xi}\subseteq\Bbb R$
meeting every non-negligible member of $M_{\xi}$;  let
$\sequencen{H_{\xi n}}$ be a sequence of closed subsets of
$\Bbb R\setminus A_{\xi}$ such that $\bigcup_{n\in\Bbb N}H_{\xi n}$ is
conegligible.   Now set

\Centerline{$M_{\xi+1}
=M_{\xi}\cup\{H_{\xi n}:n\in\Bbb N\}\cup\bigcup_{F\in M_{\xi}}f(F)
\in[\Cal C]^{\le\kappa}$.}

\noindent At non-zero limit ordinals $\xi\le\lambda$, set
$M_{\xi}=\bigcup_{\eta<\xi}M_{\eta}$.

By the choice of $\lambda$, there is an $x\in D_{\lambda}$ which does not
belong to any negligible set belonging to $M_{\lambda}$,
nor to any of the sets
$\Bbb R\setminus\bigcup_{n\in\Bbb N}H_{\xi n}$ for $\xi<\lambda$.   Now
$\#(M_{\lambda}\cap f(\{x\}))<\lambda$;  because $\lambda$ is regular,
there is a $\xi<\lambda$ such that
$M_{\lambda}\cap f(\{x\})\subseteq M_{\xi}$.
Let $n\in\Bbb N$ be such that $x\in H_{\xi n}$.   Then there must be an
$F\in f(\{x\})\cap f(H_{\xi n})$ such that $x\in F\subseteq H_{\xi n}$.
In this case, $F\in M_{\xi+2}\subseteq M_{\lambda}$, so in fact
$F\in M_{\xi}$.   Because $x\in F$, $F$ cannot be negligible, so
$A_{\xi}\cap F\ne\emptyset$;  but $H_{\xi n}$ was chosen to be disjoint
from $A_{\xi}$.\ \Bang

\medskip

\quad{\bf (iv)} Thus $\kappa\ge\cov\Cal N$, as claimed.
\medskip

{\bf (c)} Let $A\subseteq\Bbb R$ be a non-meager set.

\medskip

\quad{\bf (i)} By 518D, $\FN(\frak T)=\omega_1$, where $\frak T$ is the
topology of $\Bbb R$.   Let $f:\frak T\to[\frak T]^{\le\omega}$ be a
Freese-Nation function.   There is a set $M$ such that

\inset{($\alpha$) whenever $G\in M\cap\frak T$ then $f(G)\subseteq M$;

($\beta$) whenever $t\in M\cap\Bbb R$ then $\Bbb R\setminus\{t\}\in M$;

($\gamma$) whenever $\Cal G\subseteq M$ is a countable family of dense
open subsets of $\Bbb R$, $M\cap A\cap\bigcap\Cal G$ is non-empty;

($\delta$) $\#(M)\le\omega_1$.}

\noindent\Prf\ Build a non-decreasing family
$\ofamily{\xi}{\omega_1}{M_{\xi}}$ of countable sets as follows.
$M_0=\emptyset$.   Given that $M_{\xi}$ is countable, let $M_{\xi+1}$
be a countable set including $M_{\xi}$ such that

\inset{($\alpha$) whenever $G\in M_{\xi}\cap\frak T$ then
$f(G)\subseteq M_{\xi+1}$;

($\beta$) whenever $t\in M_{\xi}\cap\Bbb R$ then
$\Bbb R\setminus\{t\}\in M_{\xi+1}$;

($\gamma$) $M_{\xi+1}\cap A\cap\bigcap\{G:G\in M_{\xi}$ is a dense open
subset of $\Bbb R\}$ is not empty.}

\noindent For limit ordinals $\xi>0$, set
$M_{\xi}=\bigcup_{\eta<\xi}M_{\eta}$.   At the end of the construction,
set $M=\bigcup_{\xi<\omega_1}M_{\xi}$.\ \Qed

\medskip

\quad{\bf (ii)} If $H\subseteq\Bbb R$ is an open set, there is a
countable family $\Cal G\subseteq M\cap\frak T$ such that
$M\cap\Bbb R\cap\bigcap\Cal G\subseteq H\subseteq\bigcap\Cal G$.   \Prf\
Set $\Cal G=\{G:G\in f(H)\cap M$, $H\subseteq G\}$;  then certainly
$H\subseteq\bigcap\Cal G$ and $\Cal G$ is countable.   If
$t\in M\cap\Bbb R\setminus H$, then $H\subseteq\Bbb R\setminus\{t\}$ so
there is a $G\in f(H)\cap f(\Bbb R\setminus\{t\})$ such that
$H\subseteq G\subseteq\Bbb R\setminus\{t\}$;  since
$\Bbb R\setminus\{t\}\in M$,
$G\in M$;  and $t\notin G$.   As $t$ is arbitrary,
$M\cap\Bbb R\cap\bigcap\Cal G\subseteq H$.\ \Qed

\medskip

\quad{\bf (iii)} Now consider $B=A\cap M$.   Then $\#(B)\le\omega_1$.
\Quer\ If $B$ is meager, let $\sequencen{H_n}$ be a sequence of dense
open sets such that $B\cap\bigcap_{n\in\Bbb N}H_n=\emptyset$.   For each
$n\in\Bbb N$, let $\Cal G_n$ be a countable family of dense open sets
belonging to $M$ such that
$M\cap\Bbb R\cap\bigcap\Cal G_n\subseteq H_n$ (using (ii)).   Set
$\Cal G=\bigcup_{n\in\Bbb N}\Cal G_n$;  then $\Cal G\subseteq M$ is a
countable family of dense open sets, so there is a
$t\in M\cap A\cap\bigcap\Cal G$, by condition ($\gamma$) in the
specification of $M$.   But now
$t\in M\cap A\cap\bigcap\Cal G_n\subseteq H_n$ for each $n$, so
$t\in B\cap\bigcap_{n\in\Bbb N}H_n$, which is impossible.\ \Bang

Thus $A$ has a non-meager subset of size at most $\omega_1$;  as $A$ is
arbitrary, $\shr\Cal M=\omega_1$.

\medskip

{\bf (d)(i)} Again let $\frak T$ be the topology of $\Bbb R$ and
$f:\frak T\to[\frak T]^{\le\omega}$ a
Freese-Nation function.   This time, we can find a set $M$ such that

\inset{($\dagger$) for every $g\in\NN$ there is an $h\in M\cap\NN$ such
that $g(n)\ne h(n)$ for every $n\in\Bbb N$;

($\alpha$) whenever $G\in M\cap\frak T$ then
$f(G)\subseteq M$;

($\beta$) $M\cap[M]^{\le\omega}$ is cofinal with $[M]^{\le\omega}$;

($\gamma$) whenever $D\in M$ is countable, then there is a double
sequence $\langle G_{ij}\rangle_{i,j\in\Bbb N}$ belonging to
$M$ such that every $G_{ij}$ belongs to $\frak T$,
$\sequence{j}{G_{ij}}$ is disjoint for each $i\in\Bbb N$ and whenever
$G\in D$ is an open subset of $\Bbb R$ with infinite complement, there
is an $i\in\Bbb N$ such that $G_{ij}\setminus G$ is non-empty for every
$j\in\Bbb N$;

($\delta$) whenever $\langle G_{ij}\rangle_{i,j\in\Bbb N}\in M$ is a
double sequence of open subsets of $\Bbb R$, and $h\in M\cap\NN$, then
$\bigcup_{i\in\Bbb N}G_{i,h(i)}\in M$;

($\epsilon$) $\#(M)\le\kappa$.}

\noindent\Prf\ Build a non-decreasing family
$\ofamily{\xi}{\omega_1}{M_{\xi}}$ of sets of size $\kappa$ as
follows.   Start with a set $M_0\subseteq\NN$ such that $\#(M_0)=\kappa$
and for every $g\in\NN$ there is an $h\in M_0$ such that $g(n)\ne h(n)$
for every $n\in\Bbb N$ (using 522Sb).
Given that $\#(M_{\xi})=\kappa$, let $M_{\xi+1}\supseteq M_{\xi}$ be
such that

\inset{($\alpha$) whenever $G\in M_{\xi}\cap\frak T$ then
$f(G)\subseteq M_{\xi+1}$;

($\beta$) $M_{\xi+1}\cap[M_{\xi}]^{\le\omega}$ is cofinal with
$[M_{\xi}]^{\le\omega}$;

($\gamma$) whenever $D\in M_{\xi}$ is countable, then there is a double
sequence $\langle G_{ij}\rangle_{i,j\in\Bbb N}$ belonging to
$M$ such that every $G_{ij}$ is an open set,
$\sequence{j}{G_{ij}}$ is disjoint for each $i\in\Bbb N$ and whenever
$G\in D$ is an open subset of $\Bbb R$ with infinite complement, there
is an $i\in\Bbb N$ such that $G_{ij}\setminus G$ is non-empty for every
$j\in\Bbb N$;

($\delta$) whenever $\langle G_{ij}\rangle_{i,j\in\Bbb N}\in M_{\xi}$ is
a double sequence of open subsets of $\Bbb R$, and
$h\in M_{\xi}\cap\NN$, then
$\bigcup_{i\in\Bbb N}G_{i,h(i)}\in M_{\xi+1}$;

($\epsilon$) $\#(M_{\xi+1})=\kappa$.}

\noindent For limit ordinals $\xi>0$, set
$M_{\xi}=\bigcup_{\eta<\xi}M_{\eta}$.   At the end of the construction,
set $M=\bigcup_{\xi<\omega_1}M_{\xi}$.   Then

\Centerline{$M\cap[M]^{\le\omega}
=\bigcup_{\xi<\omega_1}M_{\xi+1}\cap[M_{\xi}]^{\le\omega}$}

\noindent is cofinal with
$\bigcup_{\xi<\omega_1}[M_{\xi}]^{\le\omega}=[M]^{\le\omega}$, and it is
easy to see that the other conditions are satisfied.\ \Qed

\medskip

\quad{\bf (ii)} \Quer\ Now suppose, if possible, that there is a
$t\in\Bbb R$ such that $\Bbb R\setminus I\notin M$ for any finite set
$I$ containing $t$.   Set $\Cal G=f(\Bbb R\setminus\{t\})\cap M$.
Then $\Cal G$ is a countable subset of
$M$ and $\Bbb R\setminus G$ is infinite for every $G\in\Cal G$.   Let
$D\in M$ be a countable set including $\Cal G$.   Then we have a double
sequence $\langle G_{ij}\rangle_{i,j\in\Bbb N}\in M$ such that
$\sequence{j}{G_{ij}}$ is a disjoint sequence of open sets for each $i\in\Bbb N$ and whenever
$G\in D$ is an open subset of $\Bbb R$ with infinite complement, there
is an $i\in\Bbb N$ such that $G_{ij}\setminus G$ is non-empty for every
$j\in\Bbb N$.   In particular, this last clause is true for every
$G\in\Cal G$.   For each $i\in\Bbb N$ choose $g(i)\in\Bbb N$ such that
$t\notin G_{ij}$ for any $j\ne g(i)$;   let $h\in M\cap\NN$ be such that
$h(i)\ne g(i)$ for every $i$, and set
$H=\bigcup_{i\in\Bbb N}G_{i,h(i)}\in M$;  note that $t\notin H$.   Now
there is a
$G\in f(H)\cap f(\Bbb R\setminus\{t\})$ such that
$H\subseteq G$ and $t\notin G$.   As $f(H)\subseteq M$,
$G\in M$, so $G\in\Cal G$.   But this means that $G_{i,h(i)}\subseteq G$ for every
$i\in\Bbb N$;  and we chose $\langle G_{ij}\rangle_{i,j\in\Bbb N}$ so
that this could not be so.\ \Bang

Thus $\Cal I=\{I:I\in[\Bbb R]^{<\omega}$, $\Bbb R\setminus I\in M\}$
covers $\Bbb R$.   As $\#(\Cal I)\le\#(M)\le\kappa$,
$\#(\Bbb R)\le\kappa$ and $\kappa=\frak c$, as claimed.

\medskip

\quad{\bf (iii)} Finally, if $\frakmctbl<\omega_{\omega}$, then we can
take $\kappa=\frakmctbl$, by 5A1E(e-iv),
and get $\frakmctbl=\ldots=\frak c$.

\medskip

{\bf (e)} Because $\FN(\frak T)=\FN(\Cal P\Bbb N)$, 518E tells us that
there is a set $A\subseteq\Bbb R$, with cardinal $n(\Bbb R)=\frakmctbl$, 
such that
$\#(A\cap F)<\FN^*(\frak T)=\FN^*(\Cal P\Bbb N)$ for every nowhere dense
set $F\subseteq\Bbb R$.   As $\FN^*(\Cal P\Bbb N)$ certainly has
uncountable cofinality, $A$ meets every meager set in a set of size less
than $\FN^*(\Cal P\Bbb N)$.
}%end of proof of 522U
%elementary submodel arguments

%is FN(PN)\ge\non\Cal M?

\leader{522V}{\dvrocolon{Cofinalities}}\cmmnt{ For any cardinal
associated
with a mathematical structure, we can ask whether there are any
limitations on what that cardinal can be.   The commonest form of such
limitations, when they appear, is a restriction on the possible
cofinalities of the cardinal.   I run through the known results
concerning the cardinals of Cicho\'n's diagram.   Most are elementary,
but part (f) requires a substantial argument.

\medskip

\noindent}{\bf Proposition} (a) $\cf\frak c\ge\frak p$.

(b) $\add\Cal N$, $\add\Cal M$ and $\frak b$ are regular.

(c) $\cf(\cf\Cal N)\ge\add\Cal N$,
$\cf(\cf\Cal M)\ge\add\Cal M$ and $\cf\frak d\ge\frak b$.

(d) $\cf(\non\Cal N)\ge\add\Cal N$, $\cf(\non\Cal M)\ge\add\Cal M$.

(e) If $\cf\Cal M=\frakmctbl$ then $\cf(\cf\Cal M)\ge\non\Cal M$;  if
$\cf\Cal N=\cov\Cal N$, then $\cf(\cf\Cal N)\ge\non\Cal N$.

(f)\cmmnt{ ({\smc Bartoszy\'nski \& Judah 89})}
$\cf(\frakmctbl)\ge\add\Cal N$.

\proof{{\bf (a)} If $\omega\le\kappa<\frak p$ then $2^{\kappa}=\frak c$,
by 517Rb, so $\cf\frak c>\kappa$ by 5A1Ed.

\medskip

{\bf (b)} Use 513C(a-i);  to see that $\frak b$ is regular, use
its characterization as the additivity of a partially ordered set in
522C(ii).

\medskip

{\bf (c)} Use 513C(a-ii);  this time, we need to know that $\frak d$ is
the cofinality of a partially ordered set for which $\frak b$ is the
additivity.

\medskip

{\bf (d)-(e)} 513Cb.

\medskip

{\bf (f)(i)} Write $\Cal M_1$ for the ideal of meager subsets of
$\NN$, where $\NN$ is given its usual topology.   Let $\Cal S^{(0)}$ be the
family of subsets $S$ of $\Bbb N\times\Bbb N$ such that
$\#(S[\{n\}])\le 2^n$ for every $n$ and
$\lim_{n\to\infty}2^{-n}\#(S[\{n\}])=0$.

I will write \finint\ and \disj\ for the relations
$\{(A,B):A\cap B$ is finite$\}$, $\{(A,B):A\cap B=\emptyset\}$.
Following the same mild abuse of notation as in 512Aa and elsewhere, I
will write
$(\Cal S^{(0)},\finint,\NN)$ and $(\NN,\disj,\NN)$ for the supported
relations $(\Cal S^{(0)},R_1,\NN)$ and $(\NN,R_2,\NN)$, where

\Centerline{$R_1
=\{(S,f):S\in\Cal S^{(0)}$, $f\in\NN$, $\{n:(n,f(n))\in S\}$ is finite$\}$,}

\Centerline{$R_2
=\{(f,g):f,\,g\in\NN$, $f(n)\ne g(n)$ for every $n\}$.}

\medskip

\quad{\bf (ii)($\alpha$)}
$(\NN,\in,\Cal M_1)\prGT(\Cal S^{(0)},\finint,\NN)$.
\Prf\ For $f\in\NN$, set $\phi(f)=f$ (identifying $f$ with its graph, as
usual);  for $g\in\NN$, set
$\psi(g)=\{h:h\in\NN,\,h\cap g$ is finite$\}$.   Then
$\phi(f)\in\Cal S^{(0)}$ for every $f\in\NN$, and $\psi(g)\in\Cal M_1$ for
every $g\in\NN$, because all the sets $\{h:h\cap g\subseteq n\}$ are
nowhere dense.   If $f$, $g\in\NN$ and $(\phi(f),g)\in\finint$, then
$f\cap g$ is finite so $f\in\psi(g)$;  thus $(\phi,\psi)$ is a
Galois-Tukey connection from
$(\NN,\in,\Cal M_1)$ to $(\Cal S^{(0)},\finint,\NN)$.\ \Qed

\medskip

\qquad\grheadb\ $(\Cal S^{(0)},\finint,\NN)\prGT(\NN,\disj,\NN)$.   \Prf\
Let $\sequencen{I_n}$ be a partition of $\Bbb N$ such that $\#(I_n)=2^n$
for each $n$.   For $n\in\Bbb N$, let $\theta_n:\BbbN^{I_n}\to\Bbb N$ be
a bijection.   For $S\in\Cal S^{(0)}$, choose $\phi(S)\in\NN$ such that
whenever $(n,i)\in S$ then $\phi(S)\cap\theta_n^{-1}(i)\ne\emptyset$,
where once again both the function $\phi(S)$ and the function
$\theta_n^{-1}(i)$ are identified with their graphs;  this is possible
because on each set $I_n$ there are at most $2^n$ functions with domain
$I_n$ that $\phi(S)$ has to meet.   For $g\in\NN$, define $\psi(g)\in\NN$ by
saying that $\psi(g)(n)=\theta_n(g\restr I_n)$ for every $n$.

Now suppose that $S\in\Cal S^{(0)}$ and $g\in\NN$ are such that
$S\cap\psi(g)$ is infinite.   Then there is certainly an $n$ such that
$(n,\psi(g)(n))\in S$.   In this case,

\Centerline{$\emptyset\ne\phi(S)\cap\theta_n^{-1}(\psi(g)(n))
=\phi(S)\cap g\restr I_n$,}

\noindent so $\phi(S)\cap g$ is non-empty.   Turning this round, if
$(\phi(S),g)\in\disj$ then $(S,\psi(g))\in\finint$;  that is,
$(\phi,\psi)$ is a Galois-Tukey connection from
$(\Cal S^{(0)},\finint,\NN)$ to $(\NN,\disj,\NN)$.\ \Qed

\medskip

\qquad\grheadc\ $\cov(\Cal S^{(0)},\finint,\NN)=\frakmctbl$.   \Prf\

$$\eqalignno{\frakmctbl
&=n(\NN)=\cov\Cal M_1\cr
\displaycause{517Pd}
&=\cov(\NN,\in,\Cal M_1)\le\cov(\Cal S^{(0)},\finint,\NN)\cr
\displaycause{512Da and ($\alpha$) above}
&\le\cov(\NN,\disj,\NN)\cr
\displaycause{($\beta$) above}
&=\frakmctbl\cr}$$

\noindent (522Sb).\ \Qed

\medskip

\quad{\bf (iii)} Suppose that $\kappa<\add\Cal N$ and that
$\ofamily{\xi}{\kappa}{S_{\xi}}$ is any family in $\Cal S^{(0)}$.   Then
there is an $S^*\in\Cal S^{(0)}$ such that
$S_{\xi}\setminus S^*$ is finite for every $\xi<\kappa$.   \Prf\ For
$\xi<\kappa$, $n\in\Bbb N$ let $f_{\xi}(n)\in\Bbb N$ be such that
$\#(S_{\xi}[\{i\}])\le 2^{i-2n}$ for every $i\ge f_{\xi}(n)$.   Because
$\kappa<\add\Cal N\le\frak b$, there is an $f\in\NN$ such that
$\{n:f_{\xi}(n)>f(n)\}$ is finite for every $\xi<\kappa$ (522C(ii));
of course we may suppose that $f(0)=0$ and that $f$ is strictly
increasing
and that $f(n)\ge 2n$ for every $n$.   Set $J_n=f(n+1)\setminus f(n)$
for each $n$.   For each $\xi<\kappa$, let $m_{\xi}$ be such that
$f_{\xi}(n)\le f(n)$ for every $n\ge m_{\xi}$;   set
$S'_{\xi}=\{(i,j):(i,j)\in S_{\xi}$,
$i\ge f(m_{\xi})\}$.   Then $S_{\xi}\setminus S'_{\xi}$ is finite and
$\#(S'_{\xi}[\{i\}])\le 2^{i-2n}$ whenever $i\in J_n$.

For each $n\in\Bbb N$, let $\Cal K_n$ be the family of those sets
$K\subseteq J_n\times\Bbb N$ such that $\#(K[\{i\}])\le 2^{i-2n}$ for
every $i\in J_n$, and $\theta_n:\Cal K_n\to\Bbb N$ a bijection;  set
$h_{\xi}(n)=\theta_n(S'_{\xi}\cap(J_n\times\Bbb N))$ for each
$\xi<\kappa$.

Let $(\NN,\subseteq^*,\Cal S)$ be the $\Bbb N$-localization relation.
By 522M, $\add(\NN,\subseteq^*,\Cal S)=\add\Cal N$ is greater than
$\kappa$, so there is an $S\in\Cal S$ such that $h_{\xi}\subseteq^*S$
for every $\xi<\kappa$.   Set
$S^*=\bigcup_{(n,j)\in S}\theta_n^{-1}(j)$.   For any $n\in\Bbb N$ and
$i\in J_n$, $\#(\theta_n^{-1}(j)[\{i\}])\le 2^{i-2n}$ for every
$j\in\Bbb N$, so that
$S^*[\{i\}]=\bigcup_{(n,j)\in S}\theta_n^{-1}(j)[\{i\}]$ has cardinal at
most $2^{i-n}$.   This means that $S^*\in\Cal S^{(0)}$.

Take any $\xi<\kappa$.   As $h_{\xi}\subseteq^*S$, there is some
$m\in\Bbb N$ such that $(n,h_{\xi}(n))\in S$, that is,
$(n,\theta_n(S'_{\xi}\cap(J_n\times\Bbb N)))\in S$, for every $n\ge m$.
But this means that $S'_{\xi}\cap(J_n\times\Bbb N)\subseteq S^*$ for
every $n\ge m$, so $S'_{\xi}\setminus S^*$ is finite;  it follows at
once that $S_{\xi}\setminus S^*$ is finite.   Thus we have a suitable
$S^*$.\ \Qed

\medskip

\quad{\bf (iv)} \Quer\ Now suppose, if possible, that
$\cf(\frakmctbl)=\kappa<\add\Cal N$.   By (ii-$\gamma$), there is a
set $A\subseteq\NN$ of
size $\frakmctbl$ such that for every $S\in\Cal S^{(0)}$ there is an
$f\in A$ such that $S\cap f$ is finite.   Express $A$ as
$\bigcup_{\xi<\kappa}A_{\xi}$ where $\#(A_{\xi})<\frakmctbl$ for every
$\xi<\kappa$.   Then for each $\xi<\kappa$ we can find an
$S_{\xi}\in\Cal S^{(0)}$ such that $S_{\xi}\cap f$ is infinite for every
$f\in A_{\xi}$.   By (iii), there is an $S^*\in\Cal S^{(0)}$ such that
$S_{\xi}\setminus S^*$ is finite for every $\xi<\kappa$.   But this
means that $S^*\cap f$ must be infinite for every $f\in A_{\xi}$ and
every $\xi<\kappa$;  which contradicts the choice of $A$.\ \Bang

So we are forced to conclude that $\cf(\frakmctbl)\ge\add\Cal N$, as
stated.
}%end of proof of 522V

\leader{522W}{Other spaces}\cmmnt{ All the theorems above refer to the
specific
$\sigma$-ideals $\Cal M$ and $\Cal N$ of subsets of $\Bbb R$ or the
specific partially ordered set $\NN$.   Of course the structures
involved appear in many other guises.   In particular, we have the
following results.

\medskip

} {\bf (a)(i)} Let $(X,\Sigma,\mu)$ be an atomless countably
separated\cmmnt{ (definition:  343D)} $\sigma$-finite
perfect\cmmnt{ (definition:  342K)} measure space of non-zero measure,
and $\Cal N(\mu)$ the null ideal of $\mu$.   Then
$(X,\Cal N(\mu))$ is isomorphic to $(\Bbb R,\Cal N)$;  in particular,
$\add\Cal N(\mu)=\add\Cal N$, $\cov\Cal N(\mu)=\cov\Cal N$,
$\non\Cal N(\mu)=\non\Cal N$ and $\cf\Cal N(\mu)=\cf\Cal N$.
\prooflet{\Prf\
The first thing to note is that because $\mu$ is $\sigma$-finite there
is a probability measure $\nu$ on $X$ with the same measurable sets and
the same negligible sets as $\mu$ (215B(vii));  and of course $\nu$ is
still atomless, countably separated and perfect.   Next, the completion
$\hat\nu$ of $\nu$ is again atomless, countably separated and perfect
(212Gd, 343H(vi), 451G(c-i))
and has the same negligible sets as $\nu$ (212Eb).   In the same way,
starting from Lebesgue measure instead of $\mu$, we have a complete
atomless countably separated perfect probability measure $\lambda$ on
$\Bbb R$ with the same negligible sets as Lebesgue measure.   But now
$(X,\hat\nu)$ and $(\Bbb R,\lambda)$ are isomorphic (344I), so that
$(X,\Cal N(\mu))$ and $(\Bbb R,\Cal N)$ are
isomorphic.\ \Qed}%end of prooflet

\medskip

\quad{\bf (ii)}\cmmnt{ The most important examples of spaces
satisfying the conditions of (i) are Lebesgue measure on the unit
interval and the usual measure on $\{0,1\}^{\Bbb N}$.   But the ideas go
much farther.}   On a Hausdorff space with a countable
network\cmmnt{ (e.g., any separable metrizable space, or any analytic
Hausdorff space), any topological measure is countably separated (433B).
So} any non-zero atomless Radon measure\cmmnt{ on such a space} will
have a null ideal isomorphic to
$\Cal N$.   \cmmnt{(The measure will be $\sigma$-finite because it is
a locally finite measure on a Lindel\"of space, and perfect by 416Wa.)}

\cmmnt{\medskip

\quad{\bf (iii)} As we shall see in \S523, there are
many more measure spaces $(X,\mu)$ for which $\Cal N(\mu)$ is close
enough to $\Cal N$ to have the same additivity and cofinality, and even
uniformity and covering number match in a number of interesting cases.}

\spheader 522Wb{\bf(i)}\cmmnt{ Similarly, the structure}
$(\Bbb R,\Cal M)$ is
duplicated in any non-empty Polish space $X$ without isolated points, in
the sense that $(X,\Cal B(X),\Cal M(X))\cong(\Bbb R,\Cal B,\Cal M)$,
where
$\Cal B$ and $\Cal B(X)$ are the Borel $\sigma$-algebras of $\Bbb R$ and
$X$ respectively, and $\Cal M(X)$ is the ideal of meager subsets of $X$.
\prooflet{\Prf\ Note first that $\NN$, with its usual topology, has an
uncountable
nowhere dense closed set;  e.g., $\{f:f(2n)=0$ for every $n\}$.   Now we
know that $X$ has a dense G$_{\delta}$ set $X_1$ homeomorphic to $\NN$
(5A4Ie), and $X_1$ must also have an uncountable nowhere dense closed set
$F_1$;  since $X_1\setminus F_1$ is again a non-empty Polish space
without isolated points (4A2Qd), it too has a dense G$_{\delta}$ set
$X_2$ homeomorphic to $\NN$, and $X_2$ is a dense G$_{\delta}$ set in
$X$ with
uncountable complement.   Similarly, $\Bbb R$ has a dense G$_{\delta}$
subset $H$ which is homeomorphic to $\NN$ and has uncountable
complement.

Let $\Cal M(X_2)$, $\Cal M(H)$ be the ideals of meager subsets of $X_2$
and $H$ when they are given their subspace topologies.   Because $X_2$
is dense, a closed subset of $X$ is nowhere dense in $X$ iff its
intersection with $X_2$ is nowhere dense in $X_2$;  accordingly
$\Cal M(X_2)$ is precisely $\{M\cap X_2:M\in\Cal M(X)\}$.
Similarly, $\Cal M(H)=\{M\cap H:M\in\Cal M\}$.

Consider the complements $X\setminus X_2$, $\Bbb R\setminus H$.   These
are uncountable Borel subsets of Polish spaces.   They are therefore
Borel isomorphic (424G, 424Cb);  let
$\phi:X\setminus X_2\to\Bbb R\setminus H$ be a Borel isomorphism.
Next, $X_2$ and $H$ are homeomorphic to $\NN$, therefore to each other;
let $\psi:X_2\to H$ be a homeomorphism.   Finally, set
$\theta=\psi\cup\phi$, so that $\theta:X\to\Bbb R$ is a Borel
isomorphism.   For $M\subseteq X$,

$$\eqalignno{M\in\Cal M(X)
&\iff M\cap X_2\in\Cal M(X)\cr
\displaycause{because $X\setminus X_2$ is meager}
&\iff M\cap X_2\in\Cal M(X_2)
\iff\psi[M\cap X_2]\in\Cal M(H)\cr
&\iff\theta[M]\cap H\in\Cal M(H)
\iff\theta[M]\in\Cal M.\cr}$$

\noindent So $\theta$ is an isomorphism between the structures
$(X,\Cal B(X),\Cal M(X))$ and $(\Bbb R,\Cal B,\Cal M)$.\ \Qed
}%end of prooflet

\medskip

\quad{\bf (ii)} Again, the most important special cases here are
$X=[0,1]$, $X=\{0,1\}^{\Bbb N}$ and $X=\NN$.

\exercises{\leader{522X}{Basic exercises $\pmb{>}$(a)}
%\sqheader 522Xa
Let $\Cal K$ be the $\sigma$-ideal of subsets of $\NN$
generated by the compact sets.   Show that $(\Cal K,\subseteq)$ is Tukey
equivalent to the pre-ordered sets of 522C, so that
$\add\Cal K=\frak b$ and $\cf\Cal K=\frak d$.
%522C

\spheader 522Xb Let $(X,\Sigma,\mu)$ be an atomless semi-finite measure
space with $\mu X>0$.   Show that $\#(X)\ge\non\Cal N$.
\Hint{343Cb.}
%522G

\spheader 522Xc Show that
$(\Bbb R,\in,\Cal M)\equivGT(\Bbb R,\in,\Cal N)^{\perp}$.   \Hint{there
is a comeager negligible set.}   Use this to prove 522G.
%522G

\sqheader 522Xd Show that there are just 23 assignments of values to
the cardinals of Cicho\'n's diagram which are allowed by the results in
522D-522Q and have $\frak c=\omega_2$.
%522R mt52bits

\spheader 522Xe Let $(X,\Sigma,\mu)$ be a complete locally determined
space, and suppose that $\kappa$ is the least cardinal of any cover of a
non-negligible measurable set by negligible sets.   Let $A\subseteq X$ be
such that both $A$ and $X\setminus A$ can be expressed as the union of
fewer than $\kappa$ measurable sets.   Show that $A\in\Sigma$.
%522+

\spheader 522Xf Show that if $\cov\Cal N>\omega_1$ then every
$\pmb{\Delta}^1_2$ (= PCA-\&-CPCA) set in a Polish space is universally
measurable.  \Hint{423Rb, 521Xc.}
%522+

\spheader 522Xg Let $Z$ be the Stone space of the measure algebra $\frak A$ of
Lebesgue measure.   Show that the Nov\'ak number $n(Z)$ of $Z$ and
the Martin number $\frak m(\frak A)$ of $\frak A$ are both equal to
$\cov\Cal N$.
%522+

\spheader 522Xh\dvAnew{2014} (O.Kalenda) 
Set $P=\Bbb N\times\NN$ with the product of the
usual order on $\Bbb N$ and the partial ordering $\preceq$ of 522C(ii).
Show that $\Bbb N\prT P\prT\NN$, $P\not\prT\Bbb N$ and $\NN\not\prT P$.
%522C out of order query

\leader{522Y}{Further exercises (a)}
%\spheader 522Ya
Show that if $\add\Cal N=\cf\Cal N$ then
$(\Bbb R,\Cal M)$
and $(\Bbb R,\Cal N)$ are isomorphic, in the sense that there is a
permutation $f:\Bbb R\to\Bbb R$ such that $A\subseteq\Bbb R$ is meager iff
$f[A]$ is Lebesgue negligible.
%522Q

\spheader 522Yb Show that if $\cov\Cal N>\omega_1$ then
$\cov\Cal N\ge\frak m_{\text{pc}\omega_1}$.   \Hint{525Td.}
%522T

\spheader 522Yc Let $P$ and $Q$ be partially ordered sets such that
$Q$ has no greatest member, $\sim$ an
equivalence relation on $P$, and $\pi:P\to Q$ a surjective
function such that, for $p_0$, $p_1\in P$, $\pi(p_0)\le\pi(p_1)$ iff
there is
a $p\sim p_0$ such that $p\le p_1$.   Suppose that $\kappa$ is a cardinal
such that no $\sim$-equivalence class has cardinal greater than $\kappa$.
Show that $\add(Q)\le\max(\FN(P),\kappa)$.
%522Ua

\spheader 522Yd Suppose that $\FN(\Cal P\Bbb N)=\omega_1$.   Show that
whenever $A\subseteq\Bbb R$ is non-meager there is a set
$B\in[A]^{\omega_1}$ such that every uncountable subset of $B$ is
non-meager.
%522Uc

\spheader 522Ye Suppose that $\FN(\Cal P\Bbb N)=\frak p$ and that
$\kappa\ge\frakmctbl$ is such that $\cff[\kappa]^{<\frak p}\le\kappa$.
Show that $\kappa\ge\frak c$.
%522Ud

\spheader 522Yf (S.Geschke)
Show that if $\FN^*(\Cal P\Bbb N)\le\frakmctbl$ then
$\non\Cal M\le\FN^*(\Cal P\Bbb N)$.   \Hint{proof of 522Uc.}
%522Uc mt52bits

\spheader 522Yg Let $\Cal S^{(0)}$ be the family described in the proof of
522Vf.   For any sets $A$, $B$ say that $A\subseteq^*B$ if $A\setminus B$
is finite, and define $\le^*$ as in 522C.   Show that
$(\Cal S^{(0)},\subseteq^*,\Cal S^{(0)})
\prGT(\NN,\le^*,\penalty-50\NN)\penalty-100\ltimes
(\NN,\subseteq^*,\Cal S^{(0)})$.
%522V

\spheader 522Yh Suppose that we have supported relations $(A,R,B)$ and
$(A,S,A)$ such that $R\frsmallcirc S\subseteq R$, that is,
$(a,b)\in R$ whenever $(a,a')\in S$ and $(a',b)\in R$.
Show that if $\omega\le\cov(A,R,B)<\infty$ then
$\cf(\cov(A,R,B))\ge\add(A,S,A)$.
%522V

\spheader 522Yi
Let $X$ be any topological space with countable
$\pi$-weight and write $\Cal M(X)$ for the family of meager subsets of
$X$.   Show that there is a Tukey function from $\Cal M(X)$ to $\Cal M$,
and that if the category algebra of $X$ is not purely atomic then
$\Cal M(X)$ and $\Cal M$ are Tukey equivalent.
%522W

\spheader 522Yj\dvAnew{2013}
Show that a cardinal $\kappa$ is less than $\non\Cal M$ iff whenever
$A\subseteq\NN$ and $\#(A)\le\kappa$ then there is a $g\in\NN$ such that
$\{n:f(n)=g(n)\}$ is finite for every $f\in A$.
\Hint{{\smc Bartoszy\'nski \& Judah 95}, 2.4.7.}
%522S out of order query mt52bits
}%end of exercises

\endnotes{
\Notesheader{522} All the significant ideas of this section may be found
in {\smc Bartoszy\'nski \& Judah 95}, with a good deal more.

For many years it appeared that `measure' and
`category' on the real line, or at least the structures
$(\Bbb R,\Cal B,\Cal N)$ and $(\Bbb R,\Cal B,\Cal M)$ where $\Cal B$ is
the Borel $\sigma$-algebra of $\Bbb R$, were in a symmetric duality.
It was perfectly well understood that the algebras
$\frak A=\Cal B/\Cal B\cap\Cal N$ and $\frak G=\Cal B/\Cal B\cap\Cal M$
-- what in this book I call the `Lebesgue measure algebra' and the
`category algebra of $\Bbb R$' -- are very different, but their
complexities seemed to be balanced, and such results as 522G
encouraged us to suppose that anything provable in ZFC relating measure
to category ought to respect the symmetry.   It therefore came as a
surprise to most of us when Bartoszy\'nski and Raisonnier \& Stern
(independently, but both drawing inspiration from ideas of {\smc Shelah
84}, themselves responding to a difficulty noted in {\smc Solovay 70})
showed that $\add\Cal N\le\add\Cal M$ in all models of set theory.   (It
was already known that $\add\Cal N$ could be strictly less than
$\add\Cal M$.)

The diagram in its present form emphasizes a new dual symmetry,
corresponding to the duality of Galois-Tukey connections (512Ab).   No
doubt this also is only part of the true picture.   It gives no hint,
for instance, of a striking difference between $\cov\Cal M$ and
$\cov\Cal N$.   While $\cov\Cal M=\frakmctbl$ must have uncountable
cofinality (522Vf), $\cov\Cal N$ can be $\omega_{\omega}$
({\smc Shelah 00}).

I have hardly mentioned shrinking numbers here.   This is because while
$\shr\Cal M$ and $\shr\Cal N$ can be located in Cicho\'n's diagram (we
have $\non\Cal M\le\shr\Cal M\le\cf\Cal M$ and
$\non\Cal N\le\shr\Cal N\le\cf\Cal N$, by 511Jc), they are not known to
be connected organically with the rest of the diagram.   I will return to
them in a more general context in 523M.   I have also not said where the
$\pi$-weight of Lebesgue measure (see 511Gb) fits in;  this is in fact
equal to $\cf\Cal N$, as will appear in 524P.

In 522T I give two classic `Martin's axiom' arguments.   They are
typical in that the structure of the proof is to establish that there is
a suitable
partially ordered set for which a `generic' upwards-directed subset will
provide an object to witness the truth of some assertion.   `Generic',
in this context, means `meeting sufficiently many cofinal sets'.   If
there were any more definite method of finding the object sought, we
would use it;  these constructions are always even more ethereal than
those which depend on unscrupulous use of the axiom of choice.
`Really' they are names in a suitable forcing language, since (as a
rule) we can lift Martin numbers above $\omega_1$ only by entering a
universe created by forcing.   But in this chapter, at least, I will
try to avoid such considerations, and use arguments which are
expressible in the ordinary language of ZFC, even though their
non-trivial applications depend on assumptions beyond ZFC.

Of the partially ordered sets $\Cal S^{\infty}$ and $P$ in the proof of
522T, the former comes readily to hand as soon as we cast the problem in
terms of the supported relation $(\NN,\subseteq^*,\Cal S)$;  we need
only realize that we can express members of $\Cal S$ as limits of
upwards-directed subsets of a subfamily in which there is some room to
manoeuvre, so that we have enough cofinal sets.   The latter is more
interesting.   It belongs to one of the standard types in that the
partially ordered set is made up of pairs $(\sigma,F)$ in which $\sigma$
is the `working part', from which the desired meager set

\Centerline{$M=\Bbb R\setminus\bigcap_{n\in\Bbb N}
  \bigcup_{\sigma\in R,i\ge n}\sigma(i)$}

\noindent will be constructed, and $F$ is a `side condition', designed
to ensure that the partial order of $P$ interacts correctly with the
problem.   In such cases, there is generally a not-quite-trivial step to
be made in proving that the ordering is transitive ((b-i) of the proof
of 522T).   Note that we have two classes of cofinal set to declare in
(b-iii) of the proof here;  the $Q_{nV}$ are there to ensure that $M$ is
meager, and the $Q'_H$ to ensure that it includes every member of
$\Cal H$.   And a final element which must appear in every proof of this
kind, is the check that the partial order found is of the correct
type, $\sigma$-linked in (a) and $\sigma$-centered in (b).

In 522U I suggest that it is natural to try to locate any newly defined
cardinal among those displayed in Cicho\'n's diagram.   Of course there
is no presumption that it will be possible to do this tidily, or that we
can expect any final structure to be low-dimensional;  the picture in
522T is already neater than we are entitled to expect, and the
complications in 522U (and 522Yd-522Yf) %522Yd 522Ye 522Yg 522Yf
are a warning that our luck may be running out.   However, we can
surprisingly often find relationships like the ones between
$\FN(\Cal P\Bbb N)$, $\frak b$, $\shr\Cal M$ and $\frakmctbl$ here,
which is one of my reasons for using this approach.   It is very
remarkable that under fairly weak assumptions on cardinal arithmetic
(the hypothesis
`$\frakmctbl<\omega_{\omega}$' in 522Ud is much stronger than is
necessary, since in `ordinary' models of set theory we have
$\cff[\kappa]^{\le\omega}=\kappa$ whenever $\cf\kappa>\omega$ -- see
5A6Bc and 5A6C),
%Jech 03, \S36
the axiom `$\FN(\Cal P\Bbb N)=\omega_1$' splits Cicho\'n's diagram neatly
into two halves.   For an explanation of why it was worth looking for
such a split, see {\smc Fuchino Geschke \& Soukup 01}.

For the sake of exactness and simplicity, I have maintained rigorously
the convention that $\Cal M$ and $\Cal N$ are the ideals of meager and
negligible sets in $\Bbb R$ with Lebesgue measure.   But from the point
of view of the diagram, they are `really' representatives of
classes of ideals defined on non-empty Polish spaces without isolated
points, on the one hand, and on atomless countably separated
$\sigma$-finite perfect measure spaces of non-zero measure on the other
(522W).   The most natural expressions of the duality between the
supported relations $(\Bbb R,\in,\Cal M)$ and $(\Bbb R,\in,\Cal N)$
(522G, 522Xc) depend, of course, on the fact that both structures are
invariant under translation;  but even this is duplicated in $\BbbR^r$
and in infinite compact metrizable groups like $\{0,1\}^{\Bbb N}$.

At some stage I ought to mention a point concerning the language of this
chapter.   It is natural to think of such expressions as $\add\Cal N$ as
names for
objects which exist in some ideal universe.   Starting from such a
position, the sentence `it is possible that $\add\Cal N<\add\Cal M$' has
to be interpreted as
`there is a possible mathematical universe in which
$\add\Cal N<\add\Cal M$'.   But this can make sense only if
`$\add\Cal N$' can refer to different objects in
different universes, and has a meaning independent of any particular
incarnation.   I think that in fact we have to start again,
and say that the expression $\add\Cal N$ is not a name
for an object, but an abbreviation of a definition.   We can then speak
of the interpretations of that definition in different worlds.   In fact
we have to go much
farther back than the names for cardinals in this section.
$\Cal P\Bbb N$ and $\Bbb R$ also have to be considered primarily as
definitions.   The set $\Bbb N$
itself has a relatively privileged position;  but even here it is
perhaps safest to regard the symbol $\Bbb N$ as a name for a formula in
the language of set theory
rather than anything else.   Fortunately, one can do mathematics without
aiming at perfect consistency or logical purity, and I will make no
attempt to disinfect
my own language beyond what seems to be demanded by the ideas I am
trying to express at each moment;  but you should be aware that there
are possibilities for confusion here, and that at some point you will
need to find your own way of balancing among them.   My own practice,
when the path does not seem clear, is to re-read {\smc Kunen 80}.
}%end of notes

\discrpage

