\frfilename{mt2a4.tex} 
\versiondate{4.3.14} 
\copyrightdate{1996} 
 
\def\chaptername{Appendix} 
\def\sectionname{Normed spaces} 
 
\newsection{2A4} 
 
In Chapter 24 I discuss the spaces $L^p$, for 
$1\le p\le\infty$, and describe their most basic properties.   These 
spaces form a cluster of fundamental examples for the general theory of 
`normed spaces', the basis of functional analysis.   This is not the 
book from which you should learn that theory, but once again it may save 
you trouble if I briefly outline those parts of the general theory which 
are essential if you are to make sense of the ideas here. 
 
\leader{2A4A}{The real and complex fields} While the most important 
parts of the theory, from the point 
of view of measure theory, are most effectively dealt with in terms of 
{\it real} linear spaces, there are many applications in which {\it 
complex} 
linear spaces are essential.   I will therefore use the phrase 
 
\Centerline{`$U$ is a linear space over $\RoverC$'} 
 
\noindent to mean that $U$ is either a linear space over the field 
$\Bbb R$ or a linear space over the field $\Bbb C$;   it being 
understood that 
in any particular context all linear spaces considered will be 
over the same field.   In the same way, I will write 
`$\alpha\in\RoverC$' to mean that $\alpha$ belongs to whichever is the 
current underlying field. 
 
\leader{2A4B}{Definitions  (a)} A {\bf normed space} is a linear space 
$U$ over $\RoverC$ together with a {\bf norm}, that is, a functional 
$\|\,\|:U\to\coint{0,\infty}$ such that 
 
\qquad $\|u+v\|\le\|u\|+\|v\|$ for all $u$, $v\in U$, 
 
\qquad $\|\alpha u\|=|\alpha|\|u\|$ for $u\in U$, $\alpha\in\RoverC$, 
 
\qquad $\|u\|=0$ only when $u=0$, the zero vector of $U$. 
 
\cmmnt{\noindent (Observe that if $u=0$ (the zero vector) then $0u=u$ 
(where this $0$ is the zero scalar) so that $\|u\|=|0|\|u\|=0$.) 
} 
 
\header{2A4Bb}{\bf (b)} If $U$ is a normed space, then we have a metric 
$\rho$ on $U$ defined by saying that $\rho(u,v)=\|u-v\|$ for $u$, 
$v\in U$.   \prooflet{\Prf\ $\rho(u,v)\in\coint{0,\infty}$ for all $u$, 
$v$ because $\|u\|\in\coint{0,\infty}$ for every $u$. 
$\rho(u,v)=\rho(v,u)$ for all $u$, $v$ because 
$\|v-u\|=|-1|\|u-v\|=\|u-v\|$ for all $u$, $v$. 
If $u$, $v$, $w\in U$ then 
 
\Centerline{$\rho(u,w)=\|u-w\|=\|(u-v)+(v-w)\|\le\|u-v\|+\|v-w\| 
=\rho(u,v)+\rho(v,w)$.} 
 
\noindent If $\rho(u,v)=0$ then $\|u-v\|=0$ so $u-v=0$ and $u=v$.\ \Qed} 
 
\cmmnt{We therefore have a corresponding topology, with open and 
closed sets, closures, convergent sequences and so on.} 
 
\spheader 2A4Bc If $U$ is a normed space, a set $A\subseteq U$ is {\bf 
bounded}\cmmnt{ (for the norm)} if $\{\|u\|:u\in A\}$ is bounded in 
$\Bbb R$\cmmnt{;  that is, there is some $M\ge 0$ such that 
$\|u\|\le M$ for every $u\in A$}. 
 
\leader{2A4C}{Linear subspaces (a)} If $U$ is any normed space and $V$ 
is a linear subspace of $U$, then $V$ is also a normed space, if we take 
the norm of $V$ to be just the restriction to $V$ of the norm of 
$U$\cmmnt{;  the verification is trivial}. 
 
\spheader 2A4Cb If $V$ is a linear subspace of $U$, so is its
closure\cmmnt{ $\overline{V}$}.   
\prooflet{\Prf\ Take $u$, $u'\in\overline{V}$ and 
$\alpha\in\RoverC$.   If $\epsilon>0$, set 
$\delta=\epsilon/(2+|\alpha|)>0$;  then there are $v$, $v'\in V$ such 
that $\|u-v\|\le\delta$ and $\|u'-v'\|\le\delta$.   Now $v+v'$, 
$\alpha v\in V$ and 
 
\Centerline{$\|(u+u')-(v+v')\|\le\|u-v\|+\|u'-v'\|\le\epsilon$, 
\quad$\|\alpha u-\alpha v\|\le|\alpha|\|u-v\|\le\epsilon$.} 
 
\noindent As $\epsilon$ is arbitrary, $u+u'$ and $\alpha u$ belong to 
$\overline{V}$;  as $u$, $u'$ and $\alpha$ are arbitrary, and $0$ surely 
belongs to $V\subseteq\overline{V}$, $\overline{V}$ is a linear subspace 
of $U$.\ \Qed} 
 
\leader{2A4D}{Banach spaces (a)} If $U$ is a normed space, a sequence 
$\sequencen{u_n}$ in $U$ is {\bf Cauchy} if $\|u_m-u_n\|\to 0$ as $m$, 
$n\to\infty$\cmmnt{, that is, for every 
$\epsilon>0$ there is an $n_0\in\Bbb N$ such that 
$\|u_m-u_n\|\le\epsilon$ for all $m$, $n\ge n_0$}. 
 
\spheader 2A4Db A normed space $U$ is {\bf complete} if every 
Cauchy sequence has a limit;  a 
complete normed space is called a {\bf Banach space}. 
 
\leader{2A4E}{}\cmmnt{ It is helpful to know the following result. 
 
\medskip 
 
\noindent}{\bf Lemma}  Let $U$ be a normed 
space such that $\sequencen{u_n}$ is convergent\cmmnt{ (that is, has a 
limit)} in $U$ whenever $\sequencen{u_n}$ is a sequence in $U$ such that 
$\|u_{n+1}-u_n\|\le 4^{-n}$ for every $n\in\Bbb N$.   Then $U$ is 
complete. 
 
\proof{ Let $\sequencen{u_n}$ be any Cauchy sequence in $U$.   For 
each $k\in\Bbb N$, let $n_k\in\Bbb N$ be such that 
$\|u_{m}-u_n\|\le 4^{-k}$ whenever $m$, $n\ge n_k$.   Set $v_k=u_{n_k}$ 
for each $k$. 
Then $\|v_{k+1}-v_k\|\le 4^{-k}$ (whether $n_k\le n_{k+1}$ or 
$n_{k+1}\le n_k$).   So $\sequence{k}{v_k}$ has a limit $v\in U$.   I 
seek to show that $v$ is the required limit of $\sequencen{u_n}$. 
Given $\epsilon>0$, let $l\in\Bbb N$ be such that $\|v_k-v\|\le\epsilon$ 
for every $k\ge l$;  let $k\ge l$ be such that $4^{-k}\le\epsilon$; 
then if $n\ge n_k$, 
 
\Centerline{$\|u_n-v\|=\|(u_n-v_k)+(v_k-v)\| 
\le\|u_n-v_k\|+\|v_k-v\| 
\le\|u_n-u_{n_k}\|+\epsilon 
\le 2\epsilon$.} 
 
\noindent As $\epsilon$ is arbitrary, $v$ is a limit of 
$\sequencen{u_n}$.   As $\sequencen{u_n}$ is arbitrary, $U$ is complete. 
}%end of proof of 2A4E 
 
\leader{2A4F}{Bounded linear operators (a)} Let $U$, $V$ be two normed 
spaces.   A linear operator $T:U\to V$ 
is {\bf bounded} if $\{\|Tu\|:u\in U,\,\|u\|\le 1\}$ is bounded. 
\cmmnt{({\bf Warning!} in this context, we do not ask for the whole 
set of values $T[U]$ to be bounded;  a `bounded linear operator' need 
not be what we ordinarily call a `bounded function'.)} 
Write $\eurm B(U;V)$ for the space of all bounded linear operators from 
$U$ to $V$, and for $T\in \eurm B(U;V)$ write 
$\|T\|= \sup\{\|Tu\|:u\in U,\,\|u\|\le 1\}$. 
 
\header{2A4Fb}{\bf (b)} 
\cmmnt{A useful fact:}  $\|Tu\|\le\|T\|\|u\|$ whenever
$T\in\eurm B(U;V)$ and $u\in U$.   \prooflet{\Prf\ If $|\alpha|>\|u\|$ then 
 
\Centerline{$\|\Bover1{\alpha}u\|=\Bover1{|\alpha|}\|u\|\le 1$,} 
 
\noindent so 
 
\Centerline{$\|Tu\|=\|\alpha T(\Bover1{\alpha}u)\| 
=|\alpha|\|T(\Bover1{\alpha}u)\|\le|\alpha|\|T\|$;} 
 
\noindent as $\alpha$ is arbitrary, $\|Tu\|\le\|T\|\|u\|$.  \Qed} 
 
\spheader 2A4Fc A linear operator $T:U\to V$ is bounded iff it is 
continuous for the norm topologies on $U$ and $V$.   \prooflet{\Prf\ 
(i) If $T$ is bounded, $u_0\in U$ and $\epsilon>0$, then 
 
\Centerline{$\|Tu-Tu_0\|=\|T(u-u_0)\|\le\|T\|\|u-u_0\|\le\epsilon$} 
 
\noindent whenever $\|u-u_0\|\le\Bover{\epsilon}{1+\|T\|}$;  by 2A3H, 
$T$ is continuous.   (ii) If $T$ is continuous, then there is some 
$\delta>0$ such that $\|Tu\|=\|Tu-T0\|\le 1$ whenever 
$\|u\|=\|u-0\|\le\delta$.   If now $\|u\|\le 1$, 
 
\Centerline{$\|Tu\|=\Bover1{\delta}\|T(\delta u)\|\le\Bover1{\delta}$,} 
 
\noindent so $T$ is a bounded operator.\ \Qed} 
 
\spheader 2A4Fd\dvAnew{2013} If $U$, $V$ and $W$ are normed spaces,  
$S\in\eurm B(U;V)$ and $T\in\eurm B(V;W)$ then $TS\in\eurm B(U;W)$ and 
$\|TS\|\le\|T\|\|S\|$. 
\prooflet{\Prf\ I am rather supposing that you are aware, but 
in any case you will find it easy to check, that $TS:U\to W$ is a linear 
operator.   Now if $u\in U$ and $\|u\|\le 1$, 
 
\Centerline{$\|TSu\|=\|T(Su)\|\le\|T\|\|Su\|\le\|T\|\|S\|$} 
 
\noindent (using (b) for the middle inequality), 
so $TS$ is bounded and $\|TS\|\le\|T\|\|S\|$.\ \Qed} 
 
\leader{2A4G}{Theorem} $\eurm B(U;V)$ is a linear space over $\RoverC$, 
and $\|\,\|$ is a norm on $\eurm B(U;V)$. 
 
\proof{ As in 2A4Fd, it is easy to check, that if $S:U\to V$ and 
$T:U\to V$ are linear operators, and $\alpha\in\RoverC$, then we have 
linear operators $S+T$ and $\alpha T$ from $U$ to $V$ defined by the 
formulae 
 
\Centerline{$(S+T)(u)=Su+Tu$,\quad$(\alpha T)(u)=\alpha(Tu)$} 
 
\noindent for every $u\in U$;  moreover, that under these definitions of 
addition and scalar multiplication the space of all linear operators 
from $U$ to $V$ is a linear space.   Now we see that whenever $S$, $T\in 
\eurm B(U;V)$, $\alpha\in\RoverC$, $u\in U$ and $\|u\|\le 1$, 
 
\Centerline{$\|(S+T)(u)\|=\|Su+Tu\|\le\|Su\|+\|Tu\|\le\|S\|+\|T\|$,} 
 
\Centerline{$\|(\alpha 
T)u\|=\|\alpha(Tu)\|=|\alpha|\|Tu\|\le|\alpha|\|T\|$;} 
 
\noindent so that $S+T$ and $\alpha T$ belong to $\eurm B(U;V)$, with 
$\|S+T\|\le\|S\|+\|T\|$ and $\|\alpha T\|\le|\alpha|\|T\|$.   This shows 
that $\eurm B(U;V)$ is a linear subspace of the space of all linear 
operators and is therefore a linear space over $\RoverC$ in its own 
right.   To 
check that the given formula for $\|T\|$ defines a norm, most of the 
work has just been done;   I suppose I should remark, for the sake of 
form, that $\|T\|\in\coint{0,\infty}$ for every $T$;  if $\alpha=0$, 
then of course $\|\alpha T\|=0=|\alpha|\|T\|$;  for other $\alpha$, 
 
\Centerline{$|\alpha|\|T\|=|\alpha|\|\alpha^{-1}\alpha T\| 
\le|\alpha||\alpha^{-1}|\|\alpha T\|=\|\alpha T\|\le|\alpha|\|T\|$,} 
 
\noindent so $\|\alpha T\|=|\alpha|\|T\|$.    Finally, if $\|T\|=0$ then 
$\|Tu\|\le\|T\|\|u\|=0$ for every $u\in U$, so $Tu=0$ for every $u$ and 
$T$ is the zero operator (in the space of all linear operators, and 
therefore in its subspace $\eurm B(U;V)$). 
}%end of proof of 2A4G 
 
\leader{2A4H}{Dual spaces} The most important case of $\eurm B(U;V)$ is 
when $V$ is the scalar 
field $\RoverC$ itself\cmmnt{ (of course we can think of $\RoverC$ as 
a normed space over itself, writing $\|\alpha\|=|\alpha|$ for each 
scalar $\alpha$)}.   In this case we call $\eurm B(U;\RoverC)$ the 
{\bf dual} of $U$; it is commonly denoted $U'$ or $U^*$;   I use the 
latter. 
 
\leader{2A4I}{Extensions of bounded operators:  Theorem} Let $U$ be a 
normed space and $V\subseteq U$ a dense linear subspace.   Let $W$ be a 
Banach space and $T_0:V\to 
W$ a bounded linear operator;  then there is a unique bounded linear 
operator $T:U\to W$ extending $T_0$, and $\|T\|=\|T_0\|$. 
 
\proof{{\bf (a)} For any $u\in U$, there is a sequence $\sequencen{v_n}$ 
in $V$ converging to $u$.   Now 
 
\Centerline{$\|T_0v_m-T_0v_n\|=\|T_0(v_m-v_n)\| 
\le\|T_0\|\|v_m-v_n\|\le\|T_0\|(\|v_m-u\|+\|u-v_n\|) 
\to 0$} 
 
\noindent as $m$, $n\to\infty$, so $\sequencen{T_0v_n}$ is Cauchy and 
$w=\lim_{n\to\infty}T_0v_n$ is defined in $W$.   If $\sequencen{v'_n}$ 
is another sequence in $V$ converging to $u$, then 
 
$$\eqalign{\|w-T_0v'_n\| 
&\le\|w-T_0v_n\|+\|T_0(v_n-v'_n)\|\cr 
&\le\|w-T_0v_n\|+\|T_0\|(\|v_n-u\|+\|u-v'_n\|)\to 0\cr}$$ 
 
\noindent as $n\to\infty$, so $w$ is also the limit of 
$\sequencen{T_0v'_n}$. 
 
\medskip 
 
{\bf (b)} We may therefore define $T:U\to W$ by setting 
$Tu=\lim_{n\to\infty}T_0v_n$ whenever $\sequencen{v_n}$ is a sequence in 
$V$ converging to $u$.   If $v\in V$, then we can set $v_n=v$ for every 
$n$ to see that $Tv=T_0v$;  thus $T$ extends $T_0$.   If $u$, $u'\in U$ 
and $\alpha\in\RoverC$, take sequences $\sequencen{v_n}$, 
$\sequencen{v'_n}$ in $V$ converging to $u$, $u'$ respectively;  in this 
case 
 
\Centerline{$\|(u+u')-(v_n+v'_n)\|\le\|u-v_n\|+\|u'-v'_n\|\to 0$, 
\quad$\|\alpha u-\alpha u_n\|=|\alpha|\|u-u_n\|\to 0$} 
 
\noindent as $n\to\infty$, so that 
$T(u+u')=\lim_{n\to\infty}T_0(v_n+v'_n)$, $T(\alpha 
u)=\lim_{n\to\infty}T_0(\alpha v_n)$, and 
 
$$\eqalign{\|T(u+u')-Tu-Tu'\| 
&\le\|T(u+u')-T_0(v_n+v'_n)\|+\|T_0v_n-Tu\|+\|T_0v'_n-Tu'\|\cr 
&\to 0,\cr}$$ 
 
\Centerline{$\|T(\alpha u)-\alpha Tu\| 
\le\|T(\alpha u)-T_0(\alpha v_n)\|+|\alpha|\|T_0v_n-Tu\|\to 0$} 
 
\noindent as $n\to\infty$.   This means that $\|T(u+u')-Tu-Tu'\|=0$, 
$\|T(\alpha u)-\alpha Tu\|=0$ so $T(u+u')=Tu+Tu'$, $T(\alpha u)=\alpha 
Tu$;  as $u$, $u'$ and $\alpha$ are arbitrary, $T$ is linear. 
 
\medskip 
 
{\bf (c)} For any $u\in U$, let $\sequencen{v_n}$ be a sequence in $V$ 
converging to $u$.   Then 
 
$$\eqalign{\|Tu\| 
&\le\|T_0v_n\|+\|Tu-T_0v_n\| 
\le\|T_0\|\|v_n\|+\|Tu-T_0v_n\|\cr 
&\le\|T_0\|(\|u\|+\|v_n-u\|)+\|Tu-T_0v_n\| 
\to\|T_0\|\|u\|\cr}$$ 
 
\noindent as $n\to\infty$, so $\|Tu\|\le\|T_0\|\|u\|$.   As $u$ is 
arbitrary, $T$ is bounded and $\|T\|\le\|T_0\|$.  Of course 
$\|T\|\ge\|T_0\|$ just because $T$ extends $T_0$. 
 
\medskip 
 
{\bf (d)} Finally, let $\tilde T$ be any other bounded linear operator 
from $U$ to $W$ extending $T$.   If $u\in U$, there is a sequence 
$\sequencen{v_n}$ in $V$ converging to $u$;  now 
 
\Centerline{$\|\tilde Tu-Tu\|\le\|\tilde T(u-v_n)\|+\|T(v_n-u)\| 
\le(\|\tilde T\|+\|T\|)\|u-v_n\|\to 0$} 
 
\noindent as $n\to\infty$, so $\|\tilde Tu-Tu\|=0$ and $\tilde Tu=Tu$. 
As $u$ is arbitrary, $\tilde T=T$.   Thus $T$ is unique. 
}%end of proof of 2A4I 
 
\leader{2A4J}{Normed algebras (a)} A {\bf normed algebra} is a normed 
space $(U,\|\,\|)$ together with a multiplication, a binary operator 
$\times$ on $U$, such that 
 
\Centerline{$u\times(v\times w)=(u\times v)\times w$,} 
 
\Centerline{$u\times(v+w)=(u\times v)+(u\times w)$, 
\quad$(u+v)\times w=(u\times w)+(v\times w)$,} 
 
\Centerline{$(\alpha u)\times v=u\times(\alpha v)=\alpha(u\times v)$,} 
 
\Centerline{$\|u\times v\|\le\|u\|\|v\|$} 
 
\noindent for all $u$, $v$, $w\in U$ and $\alpha\in\RoverC$. 
 
\spheader 2A4Jb A {\bf Banach algebra} is a normed algebra which is a 
Banach space.   A normed algebra\cmmnt{ $U$} is {\bf commutative} if its multiplication is commutative\cmmnt{, that is,  
$u\times v=v\times u$ for all $u$, $v\in U$}. 
 
\leader{*2A4K}{Definition} A  
normed space $U$ is {\bf uniformly convex} if 
for every $\epsilon>0$ there is a $\delta>0$ such that  
$\|u+v\|\le 2-\delta$ whenever $u$, $v\in U$, $\|u\|=\|v\|=1$ and  
$\|u-v\|\ge\epsilon$. 
 
\discrpage 
 
