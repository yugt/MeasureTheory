\frfilename{mt4a2.tex}
\versiondate{21.4.13}
\copyrightdate{2002}

\def\Cech{{\smc \v{C}ech 66}}
\def\Csaszar{{\smc Cs\'asz\'ar 78}}
\def\Engelking{{\smc Engelking 89}}
\def\Gaal{{\smc Gaal 64}}
\def\Kuratowski{{\smc Kuratowski 66}}
\def\Kechris{{\smc Kechris 95}}

\def\glitem#1{{\it #1}\enskip}

\def\chaptername{Appendix}
\def\sectionname{General topology}

\newsection{4A2}

Even more than in previous volumes, naturally enough, the work of this
volume depends on results from general topology.   We have now reached
the point where some of the facts I rely on are becoming hard
to find as explicitly stated theorems in standard textbooks.   I find
myself therefore writing out rather a lot of proofs.   You should not
suppose that the results to which I attach proofs, rather than
references, are particularly deep;  on the contrary, in many cases I am
merely spelling out solutions to classic exercises.

The style of `general' topology, as it has evolved over the last hundred
years, is to develop a language capable of squeezing the utmost from
every step of argument.   While this does sometimes lead to absurdly
obscure formulations, it remains a natural, and often profitable,
response to the remarkably dense network of related ideas in this area.
I therefore follow the spirit of the subject in giving the results I
need in the full generality achievable within the terminology I use.
For the convenience of anyone coming to the theory for the first time,
I repeat some of them in the forms in which they are actually applied.
I should remark, however, that in some cases materially stronger results
can be proved with little extra effort;  as always, this appendix is to
be thought of not as a substitute for a thorough study of the subject,
but as a guide connecting standard approaches to the general theory with
the special needs of this volume.

\vleader{72pt}{4A2A}{Definitions}\cmmnt{ I begin the section with a
glossary
of terms not defined elsewhere.

\medskip}

\glitem{Baire space} A topological space $X$ is a {\bf Baire space} if
$\bigcap_{n\in\Bbb N}G_n$ is dense in $X$ whenever $\sequencen{G_n}$ is a
sequence of dense open subsets of $X$.

\glitem{Base of neighbourhoods} If $X$ is a topological space and
$x\in X$,
a {\bf base of neighbourhoods} of $x$ is a family $\Cal V$ of
neighbourhoods of $x$ such that
every neighbourhood of $x$ includes some member of $\Cal V$.
%Engelking 1.1

\glitem{boundary} If $X$ is a topological space and $A\subseteq X$, the
{\bf boundary} of $A$ is
$\partial A=\overline{A}\setminus\interior A
=\overline{A}\cap\overline{X\setminus A}$.
%Engelking 1.3

\glitem{\cadlag}\dvAnew{2008} If $X$ is a Hausdorff space, a function
$f:\coint{0,\infty}\to X$ is {\bf c\`adl\`ag}  (`continue \`a
droit, limite \`a gauche') (or {\bf RCLL}
(`right continuous, left limits'), an {\bf $\pmb{R}$-function},)
%Rogers & Williams 94
%also  corlol  (continuous on right, limits on left)
if $\lim_{s\downarrow t}f(s)=f(t)$
for every $t\ge 0$ and $\lim_{s\uparrow t}f(s)$ is defined in
$X$ for every $t>0$.

\glitem{\callal}\dvAnew{2008}  If $X$ is a Hausdorff space, a function
$f:\coint{0,\infty}\to X$ is {\bf\callal}  (`continue \`a l'une,
limite \`a l'autre') if $f(0)=\lim_{s\downarrow 0}f(s)$ and, for every
$t>0$, $\lim_{s\downarrow t}f(s)$ and $\lim_{s\uparrow t}f(s)$
are defined in $X$, and at least one of them is equal to $f(t)$.

\glitem{\v{C}ech-complete} A completely regular Hausdorff topological
space
$X$ is {\bf \v{C}ech-complete} if it is homeomorphic to a G$_{\delta}$
subset of a compact Hausdorff space.
%4@34 %4@35
%Engelking 3.9

   %>component

\glitem{closed interval} Let $X$ be a totally ordered set.   A {\bf
closed interval} in $X$ is an interval of one of the forms $\emptyset$,
$[x,y]$,
$\ocint{-\infty,y}$, $\coint{x,\infty}$ or $X=\ooint{-\infty,\infty}$
where $x$, $y\in X$\cmmnt{ (see the definition of `interval' below)}.

\glitem{coarser topology} If $\frak S$ and $\frak T$ are two topologies
on a set $X$, we say that $\frak S$ is {\bf coarser} than $\frak T$ if
$\frak S\subseteq\frak T$.   (Equality allowed.)

\glitem{compact support} Let $X$ be a topological space and
$f:X\to\Bbb R$
a function.   I say that $f$ has {\bf compact support} if
$\overline{\{x:x\in X,\,f(x)\ne 0\}}$ is compact in $X$.

\glitem{countably compact} A topological space $X$ is {\bf countably
compact} if every countable open cover of $X$ has a finite subcover.
\cmmnt{({\bf Warning!} some authors reserve the term for Hausdorff
spaces.)}
%Engelking 3.10
A subset of a topological space is countably compact if it is countably
compact in its subspace topology.

\glitem{countably paracompact} A topological space $X$ is {\bf countably
paracompact} if given any countable open cover $\Cal G$ of $X$ there is
a locally finite family $\Cal H$ of open sets which refines $\Cal G$ and
covers $X$.
\cmmnt{({\bf Warning!} some authors reserve the term for Hausdorff
spaces.)}
%Engelking 5.2

\glitem{countably tight} A topological space $X$ is {\bf countably
tight}
(or has {\bf countable tightness}) if whenever $A\subseteq X$ and
$x\in\overline{A}$ there is a countable set $B\subseteq A$ such that
$x\in\overline{B}$.
%4@25
%Engelking 1.7.13

\leaveitout{\glitem{development} If $X$ is a topological space, a
{\bf development} for the topology of $X$ is a sequence
$\sequencen{\Cal G_n}$ of open covers of $X$ such that whenever
$H\subseteq X$ is open and $x\in H$, there is an $n\in\Bbb N$ such that
$\bigcup\{G:x\in G\in\Cal G_n\}\subseteq H\}$.
}%end of leaveitout

\glitem{direct sum, disjoint union} Let $\familyiI{(X_i,\frak T_i)}$ be
a family of topological spaces, and set
$X=\{(x,i):i\in I$, $x\in X_i\}$.
The {\bf disjoint union topology} on $X$ is $\frak T=\{G:G\subseteq X$,
$\{x:(x,i)\in G\}\in\frak T_i$ for every $i\in I\}$;  $(X,\frak T)$ is
the {\bf (direct) sum} of $\familyiI{(X_i,\frak T_i)}$.

If $X$ is a set, $\familyiI{X_i}$ a partition of $X$, and
$\frak T_i$ a topology on $X_i$ for every $i\in I$, then the {\bf
disjoint union topology} on $X$ is
$\{G:G\subseteq X$, $G\cap X_i\in\frak T_i$ for every $i\in I\}$.
%Engelking \S2.2)
%4@14

\glitem{dyadic} A Hausdorff space is {\bf dyadic} if it is a continuous
image of $\{0,1\}^I$ for some set $I$.
%Engelking exercises

\glitem{equicontinuous} If $X$ is a topological space, $(Y,\Cal W)$ a
uniform space, and $F$ a set of functions from $X$ to $Y$, then $F$ is {\bf
equicontinuous} if for every $x\in X$ and $W\in\Cal W$ the set
$\{y:(f(x),f(y))\in W$ for every $f\in F\}$ is a neighbourhood of $x$.
%Kelley

\glitem{finer topology} If $\frak S$ and $\frak T$ are two topologies on
a set $X$, we say that $\frak S$ is {\bf finer} than $\frak T$ if
$\frak S\supseteq\frak T$.   (Equality allowed.)

\glitem{first-countable} A topological space $X$ is
{\bf first-countable} if every point has a countable base of
neighbourhoods.
%4@39
%Engelking 1.1

\glitem{half-open} Let $X$ be a totally ordered set.   A {\bf half-open
interval} in $X$ is a set of one of the forms $\coint{x,y}$,
$\ocint{x,y}$ where $x$, $y\in X$ and $x<y$\cmmnt{ (see the definition
of `interval' below)}.
%this excludes \emptyset

\glitem{hereditarily Lindel\"of} A topological space is {\bf
hereditarily Lindel\"of} if every subspace is Lindel\"of.
%4@39, %4@33, %4@25

\glitem{hereditarily metacompact} A topological space is {\bf
hereditarily metacompact} if every subspace is metacompact.
%4@38

\glitem{indiscrete} If $X$ is any set, the {\bf indiscrete} topology on
$X$ is the topology $\{\emptyset,X\}$.
%4@77

\glitem{interval} Let $(P,\le)$ be a partially ordered set.
An {\bf interval} in $P$ is a set of one of the forms
$[p,q]=\{r:p\le r\le q\}$, $\coint{p,q}=\{r:p\le r<q\}$,
$\ocint{p,q}=\{r:p<r\le q\}$, $\ooint{p,q}=\{r:p<r<q\}$,
$\coint{p,\infty}=\{r:p\le r\}$,
$\ocint{-\infty,q}=\{r:r\le q\}$, $\ooint{p,\infty}=\{r:p<r\}$,
$\ooint{-\infty,q}=\{r:r<q\}$, $\ooint{-\infty,\infty}=P$, where $p$,
$q\in P$.   \cmmnt{Note that every interval is
order-convex, but even in a totally ordered set not every order-convex set
need be an interval in this sense;  an interval always has end-points, if
we allow $\pm\infty$.}

\glitem{irreducible} If $X$ and $Y$ are topological spaces, a continuous
surjection $f:X\to Y$ is {\bf irreducible} if $f[F]\ne Y$ for any closed
proper subset $F$ of $X$.
%Engelking 3.1.C
%what one actually seems to need is that $\overline{f[F]}\ne Y$ for
%closed $F\subset X$

\glitem{isolated} If $X$ is a topological space, a family
$\Cal A$ of subsets of $X$ is {\bf isolated}
%or `relatively open'
if $A\cap\overline{\bigcup(\Cal A\setminus\{A\})}$ is empty for every
$A\in\Cal A$\cmmnt{;  that is, if $\Cal A$ is disjoint and every
member of $\Cal A$ is a relatively open set in $\bigcup\Cal A$}.

\glitem{Lindel\"of} A topological space is {\bf Lindel\"of} if every
open
cover has a countable subcover.    \cmmnt{({\bf Warning!} some authors
reserve the term for regular spaces.)}
%4@12
%Engelking 3.8 requires `regular'

\glitem{Lipschitz} If $(X,\rho)$ and $(Y,\sigma)$ are metric spaces, a
function $f:X\to Y$ is {\bf $\gamma$-Lipschitz}, where $\gamma\ge 0$, if
$\sigma(f(x),f(y))\le\gamma\rho(x,y)$ for all $x$, $y\in X$.   $f:X\to Y$
is {\bf Lipschitz} if it is $\gamma$-Lipschitz for some $\gamma\ge 0$.
%4@71

\glitem{locally finite} If $X$ is a topological space, a family $\Cal A$
of
subsets of $X$ is {\bf locally finite} if for every $x\in X$ there is an
open set which contains $x$ and meets only finitely many members of
$\Cal A$.
%4@A2
%Engelking 1.1 (for indexed families)

\glitem{lower semi-continuous} If $X$ is a topological space and $T$ a
totally ordered set, a function $f:X\to T$ is
{\bf lower semi-continuous} if $\{x:f(x)>t\}$ is open for every
$t\in T$.  \cmmnt{(Cf.\ 225H, 3A3Cf.)}
%Engelking 1.7.14 (for real-valued functions)

%\glitem{manifold} If $r\ge 1$ is an integer, an {\bf $r$-dimensional
%manifold} is a Hausdorff space $X$ such that every point of $X$ belongs
%to an open set homeomorphic to the open ball
%$\{x:\|x\|<1\}\subseteq\BbbR^r$.

\glitem{metacompact} A topological space is
{\bf metacompact} if every open
cover has a point-finite refinement which is an open cover.
\cmmnt{({\bf Warning!} some authors reserve the term for Hausdorff
spaces.)}
%Engelking 5.3

\leaveitout{
\glitem{monotonically normal} Let $(X,\frak T)$ be a
topological space
and $Q$ the set of disjoint pairs $(E,F)$ of closed sets in $X$.   $X$ is
{\bf monotonically normal} if there is a function $W:Q\to\frak T$ such that
($\alpha$) $E\subseteq W(E,F)$ and $F\cap\overline{W(E,F)}=\emptyset$ for
all $(E,F)\in Q$ ($\beta$) if $(E,F)$ and $(E',F')\in Q$, $E\subseteq E'$
and $F'\subseteq F$ then $W(E,F)\subseteq W(E',F')$.
}%end of leaveitout

\leaveitout{\glitem{Moore space} A {\bf Moore space} is a regular
Hausdorff space with a development.
}

\glitem{neighbourhood} If $X$ is a topological space and $x\in X$, a
{\bf
neighbourhood} of $x$ is any subset of $X$ including an open set which
contains $x$.
%Engelking 1.1
%4@A2

\glitem{network} Let $(X,\frak T)$ be a topological space.   A
{\bf network} for $\frak T$ is a family $\Cal E\subseteq\Cal PX$ such
that whenever $x\in G\in\frak T$ there is an $E\in\Cal E$ such that
$x\in E\subseteq G$.
%Engelking 3.1

\glitem{normal} A topological space $X$ is {\bf normal} if for any
disjoint
closed sets $E$, $F\subseteq X$ there are disjoint open sets $G$, $H$
such that $E\subseteq G$ and $F\subseteq H$.   \cmmnt{({\bf Warning!}
some authors reserve the term for Hausdorff spaces.)}

\glitem{open interval} Let $X$ be a totally ordered set.   An
{\bf open interval} in $X$ is a set of one of the the forms
$\ooint{x,y}$, $\ooint{x,\infty}$, $\ooint{-\infty,x}$ or
$\ooint{-\infty,\infty}=X$ where $x$, $y\in X$\cmmnt{ (see
the definition of `interval' above)}.

\glitem{open map} If $(X,\frak T)$ and $(Y,\frak S)$ are topological
spaces, a function $f:X\to Y$ is {\bf open} if $f[G]\in\frak S$ for
every $G\in\frak T$.

\glitem{order-convex} Let $(P,\le)$ be a partially ordered set.   A
subset $C$ of $P$ is {\bf order-convex} if
$[p,q]=\{r:p\le r\le q\}$ is included in $C$ whenever $p$, $q\in C$.

\glitem{order topology}  Let $(X,\le)$ be a totally ordered set.   Its
{\bf order topology} is that generated by intervals of the form
$\ooint{x,\infty}\cmmnt{\mskip5mu =\{y:y>x\}}$,
$\ooint{-\infty,x}\cmmnt{\mskip5mu =\{y:y<x\}}$ as $x$ runs over $X$.
%Engelking 1.7.4

\glitem{paracompact} A topological space is {\bf paracompact} if every
open cover has a locally finite refinement which is an
open cover.   \cmmnt{({\bf Warning!}  some authors reserve the term
for Hausdorff spaces.)}
%Engelking 5.1

\glitem{perfect} A topological space is {\bf perfect} if
it is compact and has no isolated points.

\glitem{perfectly normal} A topological space is {\bf perfectly normal}
if it is normal and every closed set is a G$_{\delta}$ set.
\cmmnt{({\bf
Warning!} remember that some authors reserve the term `normal' for
Hausdorff spaces.)}

\glitem{point-countable, point-finite} A family $\Cal A$ of sets is
{\bf point-countable} if no
point belongs to more than countably many members of $\Cal A$.
%4@A2
Similarly, an indexed family $\familyiI{A_i}$ of sets is
{\bf point-finite} if $\{i:x\in A_i\}$ is finite for every $x$.
%4@38

\glitem{Polish} A topological space $X$ is {\bf Polish} if it is
separable
and its topology can be defined from a metric under which $X$ is
complete.

\glitem{pseudometrizable} A topological space $(X,\frak T)$ is
{\bf pseudometrizable} if $\frak T$ is defined by a single
pseudometric\cmmnt{ (2A3F)}.

\glitem{refine(ment)} If $\Cal A$ is a family of sets, a {\bf
refinement}
of $\Cal A$ is a family $\Cal B$ of sets such that every member of
$\Cal B$ is included in some member of $\Cal A$;  in this case I say
that $\Cal B$ {\bf refines} $\Cal A$.   \cmmnt{({\bf Warning!} I do
not suppose that $\bigcup\Cal B=\bigcup\Cal A$.)}
%4@A2

\glitem{relatively countably compact} If $X$ is a topological space, a
subset $A$ of $X$ is {\bf relatively countably compact} if every
sequence in $A$ has a cluster point in $X$.   \cmmnt{({\bf Warning!}
This is {\it not} the same as supposing that $A$ is included in a
countably compact subset of $X$.)}

\glitem{scattered} A topological space $X$ is {\bf scattered} if every
non-empty subset of $X$ has an isolated point (in its subspace
topology).

\glitem{second-countable} A topological space is
{\bf second-countable} if the topology has a
countable base\cmmnt{, that is, if its weight is at most $\omega$}.
%Engelking 1.1
%4@13

\glitem{semi-continuous} {\it see} lower semi-continuous, upper
semi-continuous.

\glitem{sequential} A topological space is {\bf sequential} if every
sequentially closed set in $X$ is closed.
%Engelking 1.6

\glitem{sequentially closed} If $X$ is a topological space, a subset $A$
of
$X$ is {\bf sequentially closed} if $x\in A$ whenever $\sequencen{x_n}$
is a sequence in $A$ converging to $x\in X$.

\glitem{sequentially compact} A topological space is
{\bf sequentially compact} if every sequence has a convergent sequence.
A subset of a
topological space is sequentially compact if it is sequentially compact
in its subspace topology.   \cmmnt{({\bf Warning!} some authors
reserve the term for Hausdorff spaces.)}
%4@16 %4@25

\glitem{sequentially continuous} If $X$ and $Y$ are topological spaces,
a function $f:X\to Y$ is {\bf sequentially continuous} if
$\sequencen{f(x_n)}\to f(x)$ in $Y$ whenever $\sequencen{x_n}\to x$ in
$X$.

\glitem{subbase} If $(X,\frak T)$ is a topological space, a {\bf
subbase}
for $\frak T$ is a family $\Cal U\subseteq\frak T$ which generates
$\frak T$, in the sense that $\frak T$ is the coarsest topology on $X$
including $\Cal U$.   \cmmnt{({\bf Warning!} most authors reserve the
term for families $\Cal U$ with union $X$.)}

\glitem{totally bounded} If $(X,\Cal W)$ is a uniform space, a subset
$A$ of $X$ is {\bf totally bounded} if for every $W\in\Cal W$ there is a
finite set $I\subseteq X$ such that $A\subseteq W[I]$.   If $(X,\rho)$
is a metric space, a subset of $X$ is totally bounded if it is totally
bounded for the associated uniformity\cmmnt{ (3A4B)}.

\glitem{uniform convergence} If $X$ is a set, $(Y,\sigma)$ is a metric
space and $\Cal A$ is a family
of subsets of $X$ then the {\bf topology of uniform convergence} on
members of $\Cal A$ is the topology on $Y^X$ generated by the
pseudometrics $(f,g)\mapsto\min(1,\sup_{x\in A}\sigma(f(x),g(x)))$ as
$A$ runs over
$\Cal A\setminus\{\emptyset\}$.   \cmmnt{(It is elementary to
verify that the formula here defines a pseudometric.)}

\glitem{upper semi-continuous} If $X$ is a topological space and $T$ is a
totally ordered set, a function
$f:X\to T$ is {\bf upper semi-continuous} if $\{x:f(x)<t\}$ is
open for every $t\in T$.
%Engelking 1.7.14

\glitem{weakly $\alpha$-favourable} A topological space $(X,\frak T)$ is
{\bf weakly $\alpha$-favourable} if there is a function
$\sigma:\bigcup_{n\in\Bbb N}(\frak T\setminus\{\emptyset\})^{n+1}
\to\frak T\setminus\{\emptyset\}$ such that (i)
$\sigma(G_0,\ldots,G_n)\subseteq G_n$ whenever $G_0,\ldots,G_n$ are
non-empty open sets (ii) whenever
$\sequencen{G_n}$ is a sequence in $\frak T\setminus\{\emptyset\}$ such
that $G_{n+1}\subseteq\sigma(G_0,\ldots,G_n)$ for every $n$, then
$\bigcap_{n\in\Bbb N}G_n$ is non-empty.
%Kechris uses `Choquet space'

\glitem{weight} If $X$ is a topological space, its {\bf weight} $w(X)$
is the smallest cardinal of any base for the topology.
%4@25
%Engelking 1.1

\glitem{$C_b$} If $X$ is a topological space, $C_b(X)$ is the space of
bounded continuous real-valued functions defined on $X$.

\glitem{F$_{\sigma}$} If $X$ is a topological space, an {\bf
F$_{\sigma}$}
set in $X$ is one expressible as the union of a sequence of closed sets.
%4@22
%Engelking 1.3

\glitem{G$_{\delta}$} If $X$ is a topological space, a {\bf
G$_{\delta}$}
set in $X$ is one expressible as the intersection of a
sequence of open sets.
%Engelking 1.3

\glitem{K$_{\sigma}$} If $X$ is a topological space, a
{\bf K$_{\sigma}$} set in $X$ is one expressible as the union of a
sequence of compact sets.
%4{}59

%\glitem{lim sup} If $\family{i}{I}{A_i}$ is a non-empty family of sets
%in a topological space and $\Cal F$ is a filter on $I$,
%I write $\limsup_{i\to\Cal F}A_i$ for
%$\bigcap_{J\in\Cal F}\overline{\bigcup_{i\in J}A_i}$.

$\Cal PX$ If $X$ is any set, the {\bf usual topology} on $\Cal PX$ is
that generated by the sets $\{a:a\subseteq X$, $a\cap J=K\}$ where
$J\subseteq X$ is finite and $K\subseteq J$.

\glitem{T$_0$} If $(X,\frak T)$ is a topological space, we say that it
is
{\bf T$_0$} if for any two distinct points of $X$ there is an open set
containing one but not the other.
%Engelking 1.5

\glitem{T$_1$} If $(X,\frak T)$ is a topological space, we say that it
is
{\bf T$_1$} if singleton sets are closed.

\glitem{$\pi$-base} If $(X,\frak T)$ is a topological space, a
{\bf $\pi$-base} for $\frak T$ is a set $\Cal U\subseteq\frak T$ such
that every non-empty open set includes a non-empty member of $\Cal U$.

\glitem{$\sigma$-compact} A topological space $X$ is
{\bf $\sigma$-compact}
if there is a sequence of compact subsets of $X$ covering $X$.

\glitem{$\sigma$-disjoint} A family of sets is
{\bf $\sigma$-disjoint} if it is expressible as
$\bigcup_{n\in\Bbb N}\Cal A_n$ where every $\Cal A_n$ is disjoint.

\glitem{$\sigma$-isolated} If $X$ is a topological space, a
family of subsets of $X$ is {\bf $\sigma$-isolated} if it is expressible
as $\bigcup_{n\in\Bbb N}\Cal A_n$ where every $\Cal A_n$ is an isolated
family.

\glitem{$\sigma$-metrically-discrete} If $(X,\rho)$ is a metric space, a
family of subsets of $X$ is {\bf $\sigma$-metrically-discrete}
if it is expressible as $\bigcup_{n\in\Bbb N}\Cal A_n$ where
$\rho(x,y)\ge 2^{-n}$ whenever $n\in\Bbb N$, $A$ and $B$ are distinct
members of $\Cal A_n$, $x\in A$ and $y\in B$.

\leader{4A2B}{Elementary facts about general topological spaces}
{\bf (a) Bases and networks} (i) Let $(X,\frak T)$ be a topological
space and $\Cal U$ a subbase for $\frak T$.   Then
$\{X\}\cup\{U_0\cap U_1\cap\ldots\cap U_n:U_0,\ldots,U_n\in\Cal U\}$ is
a base for $\frak T$.   \prooflet{(For this is a base for a topology, by
3A3Mc.)}

\quad (ii) Let $X$ and $Y$ be topological spaces, and $\Cal V$ a subbase
for the topology of $Y$.   Then a function $f:X\to Y$ is continuous iff
$f^{-1}[V]$ is open for every $V\in\Cal V$.   \prooflet{(\Engelking,
1.4.1(ii)).}
%4@18 %4@17 %4@23

\quad (iii) If $X$ and $Y$ are topological spaces, $\Cal E$ is a network
for the topology of $Y$, and $f:X\to Y$ is a function such that
$f^{-1}[E]$ is open for every $E\in\Cal E$, then $f$ is continuous.
\prooflet{(The topology generated by $\Cal E$ includes the given
topology on $Y$.)}

\quad (iv) If $X$ is a topological space and $\Cal U$ is a subbase for
the topology of $X$, then a filter $\Cal F$ on $X$ converges to $x\in X$
iff $\{U:x\in U\in\Cal U\}\subseteq\Cal F$.   \prooflet{(If the
condition is satisfied, $\Cal F\cup\{A:A\subseteq X,\,x\notin A\}$ is a
topology on $X$ including $\Cal U$.)}
%4@39

\quad (v) If $X$ and $Y$ are topological spaces with subbases
$\Cal U$, $\Cal V$ respectively, then
$\{U\times Y:U\in\Cal U\}\cup\{X\times V:V\in\Cal V\}$ is a subbase for
the product topology of $X\times Y$.   \prooflet{(\Kuratowski, \S15.I.)}

%A base for a topology $\frak T$ is just a network for
%$\frak T$ consisting of members of $\frak T$.

\quad (vi) If $\Cal U$ is a (sub-\nobreak)base for a topology on $X$,
and $Y\subseteq X$, then
$\{Y\cap U:U\in\Cal U\}$ is a (sub-\nobreak)base for the subspace
topology of $Y$.   \prooflet{(\Csaszar, 2.3.13(e)-(f).)}

\quad (vii) If $X$ is a topological space, $\Cal E$ is a network for the
topology of $X$, and $Y$ is a subset of $X$, then
$\{E\cap Y:E\in\Cal E\}$ is a network for the topology of $Y$.

\quad (viii) If $X$ is a topological space and $\Cal A$ is a
($\sigma$-\nobreak)isolated family of subsets of $X$, then
$\{A\cap Y:A\in\Cal A'\}$ is ($\sigma$-\nobreak)isolated whenever
$Y\subseteq X$ and $\Cal A'\subseteq\Cal A$.

\quad (ix) If a topological space $X$ has a $\sigma$-isolated
network, so has every subspace of $X$.

%partitions into open sets
\spheader 4A2Bb
If $\familyiI{H_i}$ is a partition of a topological space $X$ into open
sets and $F_i\subseteq H_i$ is closed\cmmnt{ (either in $X$ or in
$H_i$)} for each $i\in I$, then $F=\bigcup_{i\in I}F_i$ is closed in
$X$.
\prooflet{($X\setminus F=\bigcup_{i\in I}(H_i\setminus F_i)$.)}
%4@43

%filters
\spheader 4A2Bc If $X$ is a topological space, $A\subseteq X$ and
$x\in X$,
then $x\in\overline{A}$ iff there is an ultrafilter on $X$, containing
$A$, which converges to $x$.
\prooflet{($\{A\}\cup\{G:x\in G\subseteq X$, $G$ is open$\}$ has the
finite intersection property;  use 4A1Ia.)}
%4@22

%semi-continuity
\spheader 4A2Bd {\bf Semi-continuity} Let $X$ be a topological space.

\quad(i) A function $f:X\to\Bbb R$ is lower semi-continuous iff $-f$ is
upper semi-continuous.
\cmmnt{(\Cech, 18D.8.)}
A function $f:X\to\Bbb R$ is lower semi-continuous iff
$\Omega=\{(x,\alpha):x\in X$, $\alpha\ge f(x)\}$ is closed in
$X\times\Bbb R$.
%\query not \Cech not \Engelking
\prooflet{(If $f$ is lower semi-continuous and
$\alpha<\beta<f(x)$ then $\{y:f(y)>\beta\}\times\ooint{-\infty,\beta}$ is a
neighbourhood of $(x,\alpha)$;  so $\Omega$ is closed.   If $\Omega$ is
closed then for any $\gamma\in\Bbb R$ the set
$\{x:f(x)>\gamma\}=\{x:(x,\gamma)\notin\Omega\}$ is open;  so $f$ is lower
semi-continuous.)}

\quad(ii) If $T$ is a totally ordered set, $f:X\to T$ is lower
semi-continuous, $Y$ is another topological space,
and $g:Y\to X$ is continuous, then $fg:Y\to T$ is lower
semi-continuous.
\prooflet{($\{y:(fg)(y)>t\}=g^{-1}[\{x:f(x)>t\}]$.)}
%4@43, 4@44
In particular, if $f:X\to T$ is lower semi-continuous and
$Y\subseteq X$, then $f\restr Y$ is lower semi-continuous.
Similarly, if $f:X\to T$ is upper semi-continuous
and $g:Y\to X$ is continuous, then $fg:Y\to T$ is upper
semi-continuous.

\quad(iii) If $f$, $g:X\to\ocint{-\infty,\infty}$ are lower
semi-continuous so is $f+g:X\to\ocint{-\infty,\infty}$.
\cmmnt{(\Cech, 18D.8.)}
%\prooflet{($\{x:(f+g)(x)>\alpha\}
%=\bigcup_{\beta\in\Bbb R}
%\{x:f(x)>\beta\}\cap\{x:g(x)>\alpha-\beta\}$.)}

\quad(iv) If $f$, $g:X\to[0,\infty]$ are lower semi-continuous so
is $f\times g:X\to[0,\infty]$.
\cmmnt{(\Cech, 18D.8.)}
%\prooflet{($\{x:(f\times g)(x)>\alpha\}
%=\bigcup_{\beta>0}\{x:f(x)>\beta\}\cap\{x:g(x)>\Bover{\alpha}{\beta}\}$
%if $\alpha>0$.)}
%4@43

\quad(v) If $\Phi$ is any non-empty set of lower semi-continuous
functions from $X$ to $[-\infty,\infty]$, then
$x\mapsto\sup_{f\in\Phi}f(x):X\to[-\infty,\infty]$ is lower
semi-continuous.

\quad(vi) $f:X\to\Bbb R$ is continuous iff $f$ is both upper
semi-continuous and lower semi-continuous iff $f$ and $-f$ are both
lower semi-continuous.

\quad(vii)\dvAnew{2008}
If $f:X\to[-\infty,\infty]$ is lower semi-continuous, and
$\Cal F$ is a filter on $X$ converging to $y\in X$, then
$f(y)\le\liminf_{x\to\Cal F}f(x)$.
%\prooflet{\query.} %not Engelking
%4{}94C

\quad(viii)\dvAnew{2008}
If $X$ is compact and not empty, and $f:X\to[-\infty,\infty]$ is lower
semi-continuous then $K=\{x:f(x)=\inf_{y\in X}f(y)\}$ is non-empty and
compact.
\prooflet{\Prf\ Setting $\gamma=\inf_{y\in X}f(y)\in[-\infty,\infty]$,
$\{\{x:f(x)\le\alpha\}:\alpha>\gamma\}$ is a downwards-directed family of
non-empty closed sets, so its intersection $K$ is a non-empty closed set.\
\Qed}

\quad(ix)\dvAnew{2008}
If $f$, $g:X\to[0,\infty]$ are lower semi-continuous and
$f+g$ is continuous at $x\in X$ and finite there,
then $f$ and $g$ are continuous at $x$.
\prooflet{\Prf\ If $\epsilon>0$ there is a neighbourhood $G$ of $x$ such
that $(f+g)(y)\le(f+g)(x)+\epsilon$ for every $y\in G$ and
$g(y)\ge g(x)-\epsilon$ for every $y\in G$, so that
$f(y)\le f(x)+2\epsilon$ for every $y\in G$.\ \Qed}

% If $f:X\to[-\infty,\infty]$ is lower semi-continuous and
% $\Cal F$ is a convergent ultrafilter on $X$ then
% $f(\lim\Cal F)\le\lim f[[\Cal F]]
% If $f$, $g$ are lower semi-continuous so are $f\vee g$, $f\wedge g$

%separability
\spheader 4A2Be {\bf Separable spaces} (i)
If $\familyiI{A_i}$ is a countable family of separable subsets of a
topological space $X$ then $\bigcup_{i\in I}A_i$
and $\overline{\bigcup_{i\in I}A_i}$ are separable.
\prooflet{(If $D_i\subseteq A_i$ is countable and dense for each $i$,
$\bigcup_{i\in I}D_i$ is countable and dense in both
$\bigcup_{i\in I}A_i$ and its closure.)}
%4@18 tacitly

\quad (ii) If $\familyiI{X_i}$ is a family of separable topological
spaces and $\#(I)\le\frak c$, then $\prod_{i\in I}X_i$ is separable.
\prooflet{(\Engelking, 2.3.16.)}

\quad (iii) A continuous image of a separable topological space is
separable.   \prooflet{(\Engelking, 1.4.11.)}

\spheader 4A2Bf {\bf Open maps} (i) Let $\familyiI{X_i}$ be any family of
topological
spaces, with product $X$.   If $J\subseteq I$ is any set, and we write
$X_J$ for $\prod_{i\in J}X_i$, then the canonical map
$x\mapsto x\restr J:X\to X_J$ is open.
\prooflet{(\Engelking, p.\ 79.)}
%4@15

\quad(ii) Let $X$ and $Y$ be topological spaces and $f:X\to Y$ a
continuous open map.   Then
$\interior{f^{-1}[B]}=f^{-1}[\interior B]$ and
$\overline{f^{-1}[B]}=f^{-1}[\overline{B}]$ for every $B\subseteq Y$.
\prooflet{\Prf\ Because $f$ is continuous, $f^{-1}[\interior B]$ is an
open set included in
$f^{-1}[B]$, so is included in $\interior f^{-1}[B]$.   Because $f$ is
open, $f[\interior f^{-1}[B]]$ is an open set included in
$f[f^{-1}[B]]\subseteq B$, so
$f[\interior f^{-1}[B]]\subseteq\interior B$, that is,
$\interior f^{-1}[B]\subseteq f^{-1}[\interior B]$.
Now apply this to $Y\setminus B$ and take complements.\ \Qed}

\dvAnew{2013}
It follows that $f^{-1}[B]$ is nowhere dense in $X$ whenever $B\subseteq Y$
is nowhere dense in $Y$.   \prooflet{($\interior\overline{f^{-1}[B]}
=\interior f^{-1}[\overline{B}]=f^{-1}[\interior\overline{B}]=\emptyset$.)}
If $f$ is surjective and $B\subseteq Y$, then
$B$ is nowhere dense in $Y$ iff $f^{-1}[B]$ is nowhere dense in $X$.
\prooflet{(For $\interior\overline{f^{-1}[B]}
=f^{-1}[\interior\overline{B}]$ is empty iff $\interior\overline{B}$ is
empty.)}
%4@43

\quad(iii)\dvAnew{2008}
Let $X$ and $Y$ be topological spaces and $f:X\to Y$ a
continuous open map.   Then $H\mapsto f^{-1}[H]$ is an order-continuous
Boolean homomorphism from the regular open algebra of $Y$ to the regular
open algebra of $X$.   \prooflet{\Prf\ If $H\subseteq Y$ is a regular
open set,

\Centerline{$\interior\overline{f^{-1}[H]}=\interior f^{-1}[\overline{H}]
=f^{-1}[\interior\overline{H}]=f^{-1}[H]$}

\noindent by (ii), so $f^{-1}[H]$ is a regular open set in $X$.   If
$F\subseteq Y$ is nowhere dense, then
$f^{-1}[F]$ is nowhere dense in $X$, as noted in (ii) above.   By 314Ra,
$H\mapsto f^{-1}[H]=\interior\overline{f^{-1}[H]}$ is an order-continuous
Boolean
homomorphism from $\RO(Y)$ to $\RO(X)$.\ \QeD}   If $f$ is surjective, then
the homomorphism is injective\prooflet{ (because
$f^{-1}[H]\ne\emptyset$ whenever
$H\ne\emptyset$)}, and for $H\subseteq Y$, $H$ is a regular
open set in $Y$ iff $f^{-1}[H]$ is a regular open set in $X$\prooflet{
(because in this case $f^{-1}[H]=f^{-1}[\interior\overline{H}]$)}.
% 5{}27

\quad(iv)\dvAnew{2012}
If $X_0$, $Y_0$, $X_1$, $Y_1$ are topological spaces,
and $f_i:X_i\to Y_i$ is an open map for each $i$, then
$(x_0,x_1)\mapsto(f_0(x_0),f_1(x_1)):X_0\times X_1\to Y_0\times Y_1$ is
open.   \prooflet{(\Engelking, 2.3.29.)}

%product spaces
\spheader 4A2Bg Let $\familyiI{X_i}$ be a family of topological spaces
with product $X$.

\quad(i) If $A\subseteq X$ is determined by coordinates in
$J\subseteq I$\cmmnt{ in the sense of 254M}, then $\overline{A}$ and
$\interior A$ are also determined by coordinates in $J$.
\prooflet{\Prf\ Let
$\pi:X\to\prod_{i\in J}X_i$ be the canonical map.   Then
$A=\pi^{-1}[\pi[A]]$, so (f) tells us that
$\interior A=\pi^{-1}[\interior\pi[A]]$ and
$\overline{A}=\pi^{-1}[\overline{\pi[A]}]$;  but these are both
determined by coordinates in $J$.\ \Qed}

\quad(ii) If $F\subseteq X$ is closed, there is a smallest set
$J^*\subseteq I$ such that $F$ is determined by coordinates in
$J^*$.

\prooflet{\Prf\ Let $\Cal J$ be the family of all those sets
$J\subseteq I$ such that $F$ is determined by coordinates in $\Cal J$.
If $J_1$, $J_2\in\Cal J$, then $J_1\cap J_2\in\Cal J$ (254Ta).
Set $J^*=\bigcap\Cal J$.   \Quer\ Suppose, if possible, that $F$ is not
determined by coordinates in $J^*$.   Then there are $x\in F$,
$y\in X\setminus F$ such that $x\restr J^*=y\restr J^*$.   Because
$X\setminus F$ is open, there is a finite set $K\subseteq I$ such that
$z\notin F$ whenever $z\in X$ and $z\restr K=y\restr K$.   Because
$\Cal J$ is closed under finite intersections, there is a $J\in\Cal J$
such that $K\cap J=K\cap J^*$.   Define $z\in X$ by setting $z(i)=x(i)$
for $i\in J$, $z(i)=y(i)$ for $i\in I\setminus J$.   Then
$z\restr J=x\restr J$, so $z\in F$, but $z\restr K=y\restr K$, so
$z\notin F$.\ \Bang

Thus $J^*\in\Cal J$ and is the required smallest member of $\Cal J$.\
\Qed}%end of prooflet

%locally finite families
\spheader 4A2Bh Let $X$ be a topological space.

\quad(i) If $\Cal E$ is a locally finite family of closed subsets of
$X$, then $\bigcup\Cal E'$ is closed for every $\Cal E'\subseteq\Cal E$.
\prooflet{(\Engelking, 1.1.11.)}

\quad(ii) If $\familyiI{f_i}$ is a
family in $C(X)$ such that $\familyiI{\{x:f_i(x)\ne 0\}}$ is locally
finite, then we have a continuous function $f:X\to\Bbb R$ defined by
setting $f(x)=\sum_{i\in I}f_i(x)$ for every $x\in X$.
\prooflet{\Prf\ For any $x$, $\{i:f_i(x)\ne 0\}$ is finite, so $f$
is well-defined.   If $x_0\in X$ and $\epsilon>0$, there is a
neighbourhood $V$ of $x_0$ such that
$J=\{i:i\in I,\,f_i(x)\ne 0$ for some $x\in V\}$ is finite;  now there
is a neighbourhood $W$ of $x_0$,
included in $V$, such that
$\sum_{i\in J}|f_i(x)-\sum_{i\in J}f_i(x_0)|<\epsilon$ for every
$x\in W$, so that $|f(x)-f(x_0)|<\epsilon$ for every $x\in W$.   As
$x_0$ and $\epsilon$ are arbitrary, $f$ is continuous.\ \Qed}

%boundary
\spheader 4A2Bi Let $X$ be a topological space and $A$, $B$
two subsets of $X$.   Then the boundary $\partial(A*B)$ is included in
$\partial A\cup\partial B$, where $*$ is
any of $\cup$, $\cap$, $\setminus$, $\symmdiff$.  \prooflet{(Generally,
if $F\subseteq X$, $\{A:\partial A\subseteq F\}
=\{A:\overline{A}\setminus F\subseteq\interior A\}$ is a subalgebra of
$\Cal PX$.)}

%\leaveitout{\spheader 4A2Bi
%\quad(i) If $G\subseteq X$ is open then
%$G\cap\partial B\subseteq\partial(G\cap B)$.
%\prooflet{($G\cap\overline{B}\subseteq\overline{G\cap B}$,
%$G\cap\overline{X\setminus B}\subseteq\overline{G\setminus B}$.)}

%\quad(ii) \quad(iii) $A\cap\partial B\subseteq\partial
%A\cup\partial(A\cap B)$.
%4@74?
%}%end of leaveitout

%dense subspaces
\spheader 4A2Bj\dvAnew{2009}
Let $X$ be a topological space and $D$ a dense subset of
$X$, endowed with its subspace topology.

\quad(i) A set $A\subseteq D$ is nowhere dense in $D$ iff it is nowhere
dense in $X$.   \prooflet{\Prf\

$$\eqalignno{A\text{ is nowhere dense in }X
&\iff X\setminus\overline{A}\text{ is dense in }X\cr
\displaycause{writing $\overline{A}$ for the closure of $A$ in $X$}
&\iff D\setminus\overline{A}\text{ is dense in }X\cr
\displaycause{3A3Ea}
&\iff D\setminus\overline{A}\text{ is dense in }D\cr
&\iff D\setminus(D\cap\overline{A})\text{ is dense in }D\cr
&\iff A\text{ is nowhere dense in }D\cr}$$

\noindent because $D\cap\overline{A}=\overline{A}^{(D)}$
is the closure of $A$ in $D$.\ \Qed}

\quad(ii) A set $G\subseteq D$ is a regular open set in $D$ iff it is
expressible as $D\cap H$ for some regular open set $H\subseteq X$.
\prooflet{\Prf\ ($\alpha$) If $G$ is a regular open subset of $D$, set
$H=\interior\overline{G}$, taking both the closure and the interior in $X$.
Then $H$ is a regular open set in $X$.   Now $D\cap H$ is a relatively open
subset of $D$ included in $D\cap\overline{G}=\overline{G}^{(D)}$, so
$D\cap H\subseteq\interior_D\overline{G}^{(D)}=G$.   In the other
direction, $\overline{G}\cup\overline{D\setminus G}=\overline{D}=X$,
so $\overline{G}\supseteq X\setminus\overline{D\setminus G}$ and
$H\supseteq X\setminus\overline{D\setminus G}\supseteq G$.   So
$G=H\cap D$ is of the required form.   ($\beta$) If $H\subseteq X$ is a
regular open set such that $G=D\cap H$,
set $V=X\setminus\overline{H}$;  then $H=X\setminus\overline{V}$.
Now

\Centerline{$\overline{V\cap D}^{(D)}
=D\cap\overline{V\cap D}=D\cap\overline{V}=D\setminus H=D\setminus G$,}

\noindent so
$G=D\setminus\overline{V\cap D}^{(D)}$ is the complement of the closure of
an open set in $D$, and is a regular open set in $D$.\ \Qed}

%Moore spaces
%\leaveitout{\spheader 4A2B? If $X$ has a development, it has a
%$\sigma$-isolated network.   \prooflet{({\smc Fremlin 84}, A4O.)}
%}%end of leaveitout

\leader{4A2C}{G$_{\delta}$, F$_{\sigma}$, zero and cozero sets} Let
$X$ be a topological space.

\spheader 4A2Ca(i) The union of two G$_{\delta}$ sets in $X$ is a
G$_{\delta}$ set.
\prooflet{(\Engelking, p.\ 26;  \Kuratowski, \S5.V.)}
%4@41

\quad (ii) The intersection of countably many G$_{\delta}$ sets is a
G$_{\delta}$ set.
\prooflet{(\Engelking, p.\ 26;  \Kuratowski, \S5.V.)}
%4@41 %4@24

\quad (iii) If $Y$ is another topological space, $f:X\to Y$ is
continuous and $E\subseteq Y$ is G$_{\delta}$ in $Y$,
then $f^{-1}[E]$ is G$_{\delta}$ in $X$.
\prooflet{($f^{-1}[\bigcap_{n\in\Bbb N}H_n]
=\bigcap_{n\in\Bbb N}f^{-1}[H_n]$.)}
%4@39

\quad (iv) If $Y$ is a G$_{\delta}$ set in $X$ and $Z\subseteq Y$ is a
G$_{\delta}$ set for the subspace topology of $Y$, then $Z$ is a
G$_{\delta}$ set in $X$.
\prooflet{(\Kuratowski, \S5.V.)}
%4@19 %4@39

\quad (v) A set $E\subseteq X$ is an F$_{\sigma}$ set iff $X\setminus E$
is a G$_{\delta}$ set.
\cmmnt{(\Kuratowski, \S5.V.)}
%\prooflet{($\bigcup_{n\in\Bbb N}F_n
%=X\setminus\bigcap_{n\in\Bbb N}X\setminus F_n$.)}

\spheader 4A2Cb(i) A zero set is closed.   A cozero set is open.

\quad (ii) The union of two zero sets is a zero set.
\prooflet{(\Csaszar, 4.2.36.)}
%4@12 %4@19
The intersection of two cozero sets is a cozero set.
%4@22

\quad (iii) The intersection of a sequence of zero sets is a zero set.
\prooflet{(If $f_n:X\to\Bbb R$ is continuous for each $n$,
$x\mapsto\sum_{n=0}^{\infty}\min(2^{-n},|f_n(x)|)$ is continuous.)}
%4@12
The union of a sequence of cozero sets is a cozero set.
%4@22

\quad (iv) If $Y$ is another topological space, $f:X\to Y$ is continuous
and $L\subseteq Y$ is a zero set, then $f^{-1}[L]$ is a zero set.
If $f:X\to Y$ is continuous and $H\subseteq Y$ is a cozero set, then
$f^{-1}[H]$ is a cozero set.
\prooflet{(\Cech, 28B.3.)}
%4@43
If $K\subseteq X$ and $L\subseteq Y$ are zero sets then $K\times L$ is a
zero set in $X\times Y$.
\prooflet{($K\times L=\pi_1^{-1}[K]\cap\pi_2^{-1}[L]$.)}

\quad (v) If $H\subseteq X$ is a (co-\nobreak)zero set and
$Y\subseteq X$, then $H\cap Y$ is a (co-\nobreak)zero set in $Y$.
\prooflet{(Use (iv).)}
%4@12 4@43

\quad (vi) A cozero set is the union of
a non-decreasing sequence of zero sets.
\prooflet{(If $f:X\to\Bbb R$ is continuous,
$X\setminus f^{-1}[\{0\}])=\bigcup_{n\in\Bbb N}g_n^{-1}[\{0\}]$, where
$g_n(x)=\max(0,2^{-n}-|f(x)|)$.)}
%4@12 %4@21
In particular, a cozero set is an F$_{\sigma}$ set;  \cmmnt{taking
complements,} a zero set is a G$_{\delta}$ set.
%4@19

\quad (vii) If $\Cal G$ is a partition of $X$ into open sets, and
$H\subseteq X$ is such that $H\cap G$ is a cozero set in $G$ for every
$G\in\Cal G$, then $H$ is a cozero set in $X$.
\prooflet{(If $f_G:G\to\Bbb R$ is continuous for every $G\in\Cal G$,
then $f:X\to\Bbb R$ is continuous, where $f(x)=f_G(x)$ for
$x\in G\in\Cal G$.)}
Similarly, if $F\subseteq X$ is such that $F\cap G$ is a zero set in $G$
for every $G\in\Cal G$, then $F$ is a zero set in $X$.
%4@43

\vleader{72pt}{4A2D}{Weight} Let $X$ be a topological space.

\spheader 4A2Da(i)
$w(Y)\le w(X)$ for every subspace $Y$ of $X$\prooflet{ (4A2B(a-vi))}.
%4@25

\quad(ii) If $X=\prod_{i\in I}X_i$ then
$w(X)\le\max(\omega,\#(I),\sup_{i\in I}w(X_i))$.
\prooflet{(\Engelking, 2.3.13.)}

\spheader 4A2Db A disjoint family\cmmnt{ $\Cal G$} of non-empty open
sets in $X$ has cardinal at most $w(X)$.
\prooflet{(If $\Cal U$ is a base for the topology of $X$, then every
non-empty member of $\Cal G$ includes a
non-empty member of $\Cal U$, so we have an injective function from
$\Cal G$ to $\Cal U$.)}
%4@25 %4@38

\spheader 4A2Dc A point-countable family\cmmnt{ $\Cal G$} of open sets
in $X$ has cardinal at most $\max(\omega,w(X))$.
\prooflet{\Prf\ If $X=\emptyset$, this is trivial.   Otherwise, let
$\Cal U$ be a base for the topology of $X$ with $\#(\Cal U)=w(X)>0$.
Choose a function $f:\Cal G\to\Cal U$ such that
$\emptyset\ne f(G)\subseteq G$ whenever
$G\in\Cal G\setminus\{\emptyset\}$.   Then $\Cal G_U=\{G:f(G)=U\}$ is
countable for every $U\in\Cal U$, so there is an injection
$h_U:\Cal G_U\to\Bbb N$;  now
$G\mapsto(f(G),h_{f(G)}(G)):\Cal G\to\Cal U\times\Bbb N$ is injective,
so $\#(\Cal G)\le\#(\Cal U\times\Bbb N)=\max(\omega,w(X))$.\ \Qed}
%4@25

\spheader 4A2Dd If $X$ is a dyadic Hausdorff
space then $X$ is a continuous image of $\{0,1\}^{w(X)}$.
\prooflet{\Prf\ There are a set $I$
and a continuous surjection $f:\{0,1\}^I\to X$;
because any power of $\{0,1\}$ is compact, so is $X$.   If $w(X)$ is
finite, $\#(X)=w(X)\le\#(\{0,1\}^{w(X)})$ and the result is trivial;  so
we may suppose that $w(X)$ is infinite.    Let $\Cal U$ be
a base for the topology of $X$ with cardinality
$w(X)$.   Set $Z=\{0,1\}^I$ and let $\Cal E$ be the
algebra of subsets of $Z$ determined by coordinates in finite sets, so
that $\Cal E$ is an algebra of subsets of $Z$ and is a base for the
topology of $Z$.
For each pair $U$, $V$ of members of $\Cal U$ such that
$\overline{U}\subseteq V$, $f^{-1}[V]\subseteq Z$ is open;  the
set $\{E:E\in\Cal E$, $E\subseteq f^{-1}[V]\}$ is upwards-directed and
covers the compact set $f^{-1}[\overline{U}]$, so there
is an $E_{UV}\in\Cal E$ such that
$f^{-1}[\overline{U}]\subseteq E_{UV}\subseteq f^{-1}[V]$.   Let
$J\subseteq I$ be a set of cardinal at most $\max(\omega,w(X))$ such
that every $E_{UV}$ is determined by coordinates in $J$.   Fix any
$w\in\{0,1\}^{I\setminus J}$ and define $g:\{0,1\}^J\to X$ by setting
$g(z)=f(z,w)$ for every $z\in\{0,1\}^J$, identifying $Z$ with
$\{0,1\}^J\times\{0,1\}^{I\setminus J}$.   Then $g$ is continuous.
\Quer\ If $g$ is not surjective, set
$H=X\setminus g[\{0,1\}^J]$.   Take $x\in H$;  take $V\in\Cal U$ such
that $x\in V\subseteq H$;  take an open set $G$ such that
$x\in G\subseteq\overline{G}\subseteq V$
(this must be possible because $X$, being compact and Hausdorff, is
regular);  take $U\in\Cal U$ such that $x\in U\subseteq G$, so that
$x\in U\subseteq\overline{U}\subseteq V$.
Because $f$ is surjective, there is a
$(u,v)\in\{0,1\}^J\times\{0,1\}^{I\setminus J}$
such that $f(u,v)=x$.   Now $(u,v)\in f^{-1}[U]\subseteq E_{UV}$;  as
$E_{UV}$ is determined by coordinates in $J$,
$(u,w)\in E_{UV}\subseteq f^{-1}[V]$ and $g(u)=f(u,w)\in V$;  but $V$ is
supposed to be
disjoint from $g[\{0,1\}^J]$.\ \BanG\  So $g$ is surjective, and $X$ is
a continuous image of $\{0,1\}^J$.   Since
$\#(J)\le\max(\omega,w(X))=w(X)$, $\{0,1\}^J$ and
$X$ are continuous images of $\{0,1\}^{w(X)}$.\ \Qed}

\spheader 4A2De If $X$ is a dyadic
Hausdorff space then $X$ is separable iff it is
a continuous image of $\{0,1\}^{\frak c}$.
\prooflet{\Prf\ $\{0,1\}^{\frak c}$ is separable (4A2B(e-ii)), so any
continuous image of it is separable.   If $X$ is a separable dyadic
Hausdorff space, let $A\subseteq X$ be a countable
dense set.   If $G$, $G'\subseteq X$ are distinct regular open sets,
then $G\cap A\ne G\cap A'$.   Thus $X$ has at most $\frak c$ regular
open sets;  since $X$ is compact
and Hausdorff, therefore regular, its regular open sets form a base
(4A2F(b-ii)), and $w(X)\le\frak c$.   By (d), $X$ is a continuous image of
$\{0,1\}^{\max(\omega,w(X))}$ which is
in turn a continuous image of $\{0,1\}^{\frak c}$.\ \Qed}

\leader{4A2E}{The countable chain condition (a)}(i) Let $\familyiI{X_i}$
be a family of topological spaces.   If $\prod_{i\in J}X_i$ is ccc
for every finite $J\subseteq I$, then $\prod_{i\in I}X_i$ is ccc.
\prooflet{({\smc Kunen 80}, II.1.9;  {\smc Fremlin 84}, 12I.)}

\quad(ii) A separable topological space is ccc.   \prooflet{(If $D$ is a
countable dense set and $\Cal G$ is a disjoint family of non-empty open
sets, we have a surjection from a subset of $D$ onto $\Cal G$.)}

\quad(iii) The product of any family of separable topological spaces is
ccc.   \prooflet{\Prf\ By 4A2B(e-ii) and (ii) here, the product of
finitely many separable spaces is separable, therefore ccc;  so we can
apply (i).\ \Qed}

\spheader 4A2Eb Let $\familyiI{X_i}$ be a family of topological
spaces, and suppose that $X=\prod_{i\in I}X_i$ is ccc.   For
$J\subseteq I$ and $x\in X$ set
$X_J=\prod_{i\in J}X_i$, $\pi_J(x)=x\restr J$.

\quad(i) If $G\subseteq X$ is open, there is an open set $W\subseteq G$
determined by coordinates in a countable subset of $I$ such that
$G\subseteq\overline{W}$.
\prooflet{\Prf\ Let $\Cal W$ be the family of subsets of $X$ determined
by coordinates in countable sets.   Then $\Cal W$ is a $\sigma$-algebra
(254Mb) including the standard base
$\Cal U$ for the topology of $X$.   Let $\Cal U_0$ be a maximal disjoint
family in $\{U:U\in\Cal U,\,U\subseteq G\}$.   Then $\Cal U_0$ is
countable, so $W=\bigcup\Cal U_0$ belongs to $\Cal W$.   No member of
$\Cal U$ can be included in $G\setminus W$, so $G\setminus\overline{W}$
must be empty, and we have a suitable set.\ \QeD}   So
$\overline{G}=\overline{W}$
and $\interior\overline{G}$ are
determined by coordinates in a countable set\prooflet{ (4A2B(g-i))};
in particular, if $G$ is a regular open set, then it is determined by
coordinates in a countable set.

\quad(ii) If $f:X\to\Bbb R$ is continuous, there are a countable set
$J\subseteq I$ and a continuous function $g:X_J\to\Bbb R$ such that
$f=g\pi_J$.   \prooflet{\Prf\ For each
$q\in\Bbb Q$, set $F_q=\overline{\{x:f(x)<q\}}$.   By (i), $F_q$ is
determined by coordinates in a countable set.   Because
$\Bbb Q$ is countable, there is a countable $J\subseteq I$ such that
every $F_q$ is determined by coordinates in $J$.   Also
$\{x:f(x)<\alpha\}=\bigcup_{q\in\Bbb Q,q<\alpha}F_q$ is determined by
coordinates in $J$ for every $\alpha\in\Bbb R$, so $f(x)=f(y)$ whenever
$x\restr J=y\restr J$, and there is a $g:X_J\to\Bbb R$ such that
$f=g\pi_J$.   Now if $H\subseteq\Bbb R$ is
open, $g^{-1}[H]=\pi_J[f^{-1}[H]]$ is open (4A2B(f-i)),
so $g$ is continuous.\ \Qed}

\quad(iii)\dvAnew{2013}
If $A\subseteq X$ is nowhere dense there is a countable set
$J\subseteq I$ such that $\pi_J^{-1}[\pi_J[A]]$ is nowhere dense.
\prooflet{\Prf\
By (ii), there are a countable set $J$ and an open set
$W\subseteq X\setminus\overline{A}$ such that $W$ is determined by
coordinates in $J$ and $X\setminus\overline{A}\subseteq\overline{W}$;
now $W$ is dense in $X$ and $\pi_J^{-1}[\pi_J[A]]\subseteq X\setminus W$ is
nowhere dense.\ \Qed}

\leader{4A2F}{Separation axioms (a) Hausdorff spaces}
(i) A Hausdorff space is T$_1$.
\cmmnt{(\Cech, 27A.1.)}
Any subspace of a Hausdorff space is Hausdorff.
\prooflet{(\Engelking, 2.1.6;  \Cech, 27A.3;  \Kuratowski, \S5.VIII;
\Csaszar, 2.5.21.)}
%4@15 %4@16 %4@23

\quad (ii) If $X$ is a Hausdorff space and $\sequencen{x_n}$ is a
sequence in $X$,
then a point $x$ of $X$ is a cluster point of $\sequencen{x_n}$ iff
there is a non-principal ultrafilter $\Cal F$ on $\Bbb N$ such that
$x=\lim_{n\to\Cal F}x_n$.   \prooflet{(If $x$ is a cluster point of
$\sequencen{x_n}$, apply 4A1Ia to
$\{\{n:n\ge n_0,\,x_n\in G\}:n_0\in\Bbb N$, $G\subseteq X$ is open,
$x\in G\}$.)}


\quad(iii) A topological space $X$ is Hausdorff iff $\{(x,x):x\in X\}$ is
closed in $X\times X$.
%Engelking ex 2.3.C
\prooflet{(\Cech\ 27A.7;  \Kuratowski, I.15.IV.)}
%If $I$ is a set, $\Cal F$ is a filter on $I$, $X$ and $Y$ are Hausdorff
%spaces, $f:I\to X$ is a function, $g:X\to Y$ is a continuous function,
%and $\lim_{i\to\Cal F}f(i)=x$, then $\lim_{i\to\Cal F}gf(i)=g(x)$.

%If $\family{j}{J}{X_j}$ is a family of Hausdorff spaces, $I$ is a set,
%$\Cal F$ is a filter on $I$, and $f:I\to X$ is a function, then
%$\lim_{i\to\Cal F}f(i)=x$ iff $\lim_{i\to\Cal F}f(i)(j)=x(j)$ for every
%$j\in J$.   \cmmnt{(Cf.\ 3{}A3Ic.)}

\spheader 4A2Fb {\bf Regular spaces}
(i) A regular T$_1$ space is Hausdorff.
\prooflet{(\Cech, 27B.7;  \Gaal, p.\ 81.)}
%4@39
Any subspace of a regular space is regular.
\cmmnt{(\Engelking, 2.1.6;  \Kuratowski, \S14.I.)}

\quad(ii) If $X$ is a regular topological space, the regular open subsets
of $X$ form a base for the topology.   \prooflet{\Prf\ If $G$ is open and
$x\in G$, there is an open set $H$ such that
$x\in H\subseteq\overline{H}\subseteq G$;  now $\interior\overline{H}$ is a
regular open set containing $x$ and included in $G$.\ \Qed}

\spheader 4A2Fc {\bf Completely regular spaces}
In a completely regular space, the cozero sets form a base
for the topology.
\prooflet{(\Cech, 28B.5.)}
%4@22

\spheader 4A2Fd {\bf Normal spaces} (i) {\bf Urysohn's Lemma}  If
$X$ is normal and $E$, $F$ are disjoint closed
subsets of $X$, then there is a continuous function $f:X\to[0,1]$ such
that $f(x)=0$ for $x\in E$ and $f(x)=1$ for $x\in F$.
\prooflet{(\Engelking, 1.5.11;  \Kuratowski, \S14.IV.)}
%4@33

\quad (ii) A regular normal space is completely regular.
%4@A2Jb

\quad (iii) A normal T$_1$ space is Hausdorff
\prooflet{(\Gaal, p.\ 86)}
%4@39
and completely regular
\prooflet{(\Csaszar, 4.2.5;  \Gaal, p.\ 110)}.
%4@26

\quad (iv) If $X$ is normal and $E$, $F$ are disjoint closed sets in $X$
there is a zero set including $E$ and disjoint from $F$.
\prooflet{(Take a continuous function $f$ which is zero on $E$ and $1$
on $F$, and set $Z=\{x:f(x)=0\}$.)}
%4@36

\quad (v) In a normal space a closed G$_{\delta}$ set is a zero set.
\prooflet{(\Engelking, 1.5.12.)}
%4@33

\quad (vi) If $X$ is a normal space and $\familyiI{G_i}$ is a
point-finite cover of $X$ by open sets, there is a family
$\familyiI{H_i}$ of open
sets, still covering $X$, such that $\overline{H}_i\subseteq G_i$ for
every $i$.
\prooflet{(\Engelking, 1.5.18;  \Cech, 29C.1;  \Gaal, p.\ 89.)}
%for (vii)

\quad (vii) If $X$ is a normal space and $\familyiI{G_i}$ is a
point-finite cover of $X$ by open sets, there is a family
$\familyiI{H'_i}$ of cozero
sets, still covering $X$, such that $H'_i\subseteq G_i$ for every $i$.
\prooflet{(Take $\familyiI{H_i}$ from (vi), and apply (iv) to the
disjoint closed sets $X\setminus G_i$, $\overline{H}_i$ to find a
suitable cozero set $H'_i$ for each $i$.)}
%4@26

\quad (viii) If $X$ is a normal space and $\familyiI{G_i}$ is a
locally finite cover of $X$ by open sets, there is a family
$\familyiI{g_i}$ of continuous functions from $X$ to $[0,1]$ such that
$g_i\le\chi G_i$ for every $i\in I$ and $\sum_{i\in I}g_i(x)=1$ for
every $x\in X$.   \prooflet{(\Engelking, proof of 5.1.9.)}
%4@15

\quad (ix) {\bf Tietze's theorem}\dvAformerly{4{}A2F(d-viii)} Let 
$X$ be a normal space, $F$ a
closed subset of $X$ and $f:F\to\Bbb R$ a continuous function.   Then
there is a continuous function $g:X\to\Bbb R$ extending $f$.
\prooflet{(\Engelking, 2.1.8;  \Kuratowski, \S14.IV;  \Gaal, p.\ 203.)}
It follows that if $F\subseteq X$ is closed and $f:F\to[0,1]^I$ is a
continuous function from $F$ to any power of the unit interval, there is
a continuous function from $X$ to $[0,1]^I$ extending $f$.
\prooflet{ (Extend each of the functionals $x\mapsto f(x)(i)$ for
$i\in I$.)}

\spheader 4A2Fe {\bf Paracompact spaces} A Hausdorff paracompact space
is regular.
\prooflet{(\Engelking, 5.1.5.)}
A regular paracompact
space is normal.
\prooflet{(\Engelking, 5.1.5;  \Gaal, p.\ 160.)}

%4@36
\spheader 4A2Ff {\bf Countably paracompact spaces} A normal space $X$ is
countably paracompact iff whenever $\sequencen{F_n}$
is a non-increasing
sequence of closed subsets of $X$ with empty intersection, there is
a sequence $\sequencen{G_n}$ of open sets, also with empty intersection,
such that $F_n\subseteq G_n$ for every $n\in\Bbb N$.
\prooflet{(\Engelking, 5.2.2;  \Csaszar, 8.3.56(f).)}
%4@26

\spheader 4A2Fg {\bf Metacompact spaces} (i) A paracompact space is
metacompact.

\quad (ii) A closed subspace of a metacompact space is metacompact.
%4@26

\quad (iii) A normal metacompact space is countably paracompact.
\prooflet{(\Engelking, 5.2.6;  \Csaszar, 8.3.56(c).)}
%4@26

\spheader 4A2Fh {\bf Separating compact sets} (i) If $X$ is a Hausdorff
space and $K$ and $L$ are disjoint compact subsets of $X$, there are
disjoint open sets $G$, $H\subseteq X$ such that $K\subseteq G$ and
$L\subseteq H$.
\prooflet{(\Csaszar, 5.3.18.)}
%4@22 %4@16
If $\Tau$ is an algebra of subsets of $X$ including a subbase for the
topology of $X$, there is an open $V\in\Tau$ such that
$K\subseteq V\subseteq X\setminus L$.   \prooflet{\Prf\ By 4A2B(a-i),
$\Tau$ includes a base for the topology of $X$.   So
$\Cal E=\{U:U\in\Tau$ is open, $U\subseteq G\}$ has union $G$ and there
must be a finite $\Cal E_0\subseteq\Cal E$ covering $K$;  set
$V=\bigcup\Cal E_0$.\ \Qed}

\quad(ii) If $X$ is a regular space, $F\subseteq X$ is closed, and
$K\subseteq X\setminus F$ is compact, there are disjoint open sets $G$,
$H\subseteq X$ such that $K\subseteq G$ and $F\subseteq H$.
\prooflet{(\Engelking, 3.1.6.)}
%4@22

\quad(iii) If $X$ is a completely regular space,
$G\subseteq X$ is open and $K\subseteq G$ is compact, there is a
continuous function $f:X\to[0,1]$ such that $f(x)=1$ for $x\in K$ and
$f(x)=0$ for $x\in X\setminus G$.
\prooflet{\Prf\ For each $x\in K$ there is a continuous function
$f_x:X\to[0,1]$ such that $f_x(x)=1$ and $f_x(y)=0$ for
$y\in X\setminus G$.   Set $H_x=\{y:f_x(y)>\bover12\}$.   Then
$\bigcup_{x\in K}H_x\supseteq K$, so there is a finite set
$I\subseteq K$ such that $K\subseteq\bigcup_{x\in I}H_x$.   Set
$f(y)=\min(1,2\sum_{x\in I}f_x(y))$ for $y\in X$.\ \Qed}
%for (iv), (v)

\quad (iv) If $X$ is a completely regular Hausdorff space and $K$ and
$L$ are disjoint compact subsets of $X$, there are disjoint cozero sets
$G$, $H\subseteq X$ such that $K\subseteq G$ and $L\subseteq H$.
\prooflet{\Prf\ By (i), there are disjoint open sets $G'$, $H'$ such
that $K\subseteq G'$ and $L\subseteq H'$.   By (iii), there is a
continuous function $f:X\to[0,1]$ such that $f(x)=1$ for $x\in K$ and
$f(x)=0$ for $x\in X\setminus G'$;  set $G=\{x:f(x)\ne 0\}$, so that $G$
is a cozero set and $K\subseteq G\subseteq G'$.   Similarly there is a
cozero set $H$ including $L$ and included in $H'$.\ \Qed}
%4@22

\quad (v) If $X$ is a completely regular space and $K\subseteq X$ is a
compact G$_{\delta}$ set, then $K$ is a zero set.
\prooflet{\Prf\ Let $\sequencen{G_n}$ be a sequence of open sets with
intersection $K$.   For each $n\in\Bbb N$ there is a continuous function
$f_n:X\to[0,1]$ such that $f_n(x)=1$ for $x\in K$ and $f_n(x)=0$ for
$x\in X\setminus G_n$, by (iii).   Now
$K=\bigcap_{n\in\Bbb N}\{x:1-f_n(x)=0\}$ is a zero set, by 4A2C(b-iii).\
\Qed}
%4@19 %4@43

\quad (vi) If $\sequencen{X_n}$ is a sequence of topological spaces with
product $X$, $K\subseteq X$ is compact, $F\subseteq X$ is closed and
$K\cap F=\emptyset$, there is some $n\in\Bbb N$ such that
$x\restr n\ne y\restr n$ for any $x\in F$ and $y\in K$.

\prooflet{\Prf\ For $n\in\Bbb N$ and $x\in X$ set $\pi_n(x)=x\restr n$;
set $F_n=\overline{\pi_n^{-1}[\pi_n[F]]}$.   Since
$\sequencen{\pi_n^{-1}[\pi_n[F]]}$ is non-increasing, so is
$\sequencen{F_n}$.   If $x\in K$, there is an open set $G\subseteq X$,
determined by coordinates in a finite set, such that
$x\in G\subseteq X\setminus F$;  in this case there is an $n\in\Bbb N$ such
that $\pi_n^{-1}[\pi_n[G]]=G$ is disjoint from $F$, so that
$\pi_n[G]\cap\pi_n[F]=\emptyset$, $G$ does not meet
$\pi_n^{-1}[\pi_n[F]]$ and $x\notin F_n$.   As $x$ is arbitrary,
$\sequencen{K\cap F_n}$ is a non-increasing sequence of relatively closed
subsets of $K$ with empty intersection;  as $K$ is compact, there is an
$n$ such that $K\cap F_n=\emptyset$, so that
$K\cap\pi_n^{-1}[\pi_n[F]]=\emptyset$ and $x\restr n\ne y\restr n$
whenever $x\in F$ and $y\in K$.\ \Qed}

\quad(vii) If $X$ is a compact Hausdorff space, $f:X\to\Bbb R$ is
continuous, and $\Cal U$ is a subbase for $\frak T$, then there is a
countable set
$\Cal U_0\subseteq\Cal U$ such that $f(x)=f(y)$ whenever
$\{U:x\in U\in\Cal U_0\}=\{U:y\in U\in\Cal U_0\}$.   \prooflet{(Apply (i) to
sets of the form $K=\{x:f(x)\le\alpha\}$, $L=\{x:f(x)\ge\beta\}$.)}
%4@91Xi

\spheader 4A2Fi {\bf Perfectly normal spaces} A topological space $X$ is
perfectly normal iff every closed set is a zero set.
\prooflet{(\Engelking, 1.4.9.)}

Consequently, every open set in a perfectly normal space is a cozero set
(and\cmmnt{, of course,} an F$_{\sigma}$ set).

\spheader 4A2Fj {\bf Covers of compact sets} Let $X$ be a Hausdorff space,
$K$ a compact subset of $X$, and $\familyiI{G_i}$ a family of open subsets
of $X$ covering $K$.   Then there are a finite set $J\subseteq I$ and a
family $\family{i}{J}{K_i}$ of compact sets such that
$K=\bigcup_{i\in J}K_i$ and $K_i\subseteq G_i$ for every $i\in J$.
\prooflet{\Prf\ (i) Suppose first that $I=\{i,j\}$ has just two members.
Then $K\setminus G_j$ and $K\setminus G_i$ are disjoint compact sets.
By (h-i), there are disjoint open sets $H_i$, $H_j$ such that
$K\setminus G_j\subseteq H_i$ and $K\setminus G_i\subseteq H_j$;  setting
$K_i=K\setminus H_j$ and $K_j=K\setminus H_i$ we have a suitable pair
$K_i$, $K_j$.   (ii) Inducing on $\#(I)$ we get the result for finite $I$.
(iii) In general, there is certainly a finite $J\subseteq I$ such that
$K\subseteq\bigcup_{i\in J}G_i$, and we can apply the result to
$\family{i}{J}{G_i}$.\ \Qed}

\leaveitout{
\spheader 4A2F? {\bf Monotonically normal spaces} (i)
A monotonically normal space is normal, so a
T$_1$ monotonically normal space is regular and Hausdorff.

\quad(ii) Let $(X,\frak T)$ be a regular space and $R$ the set of pairs
$(x,G)$ such that $x\in G\in\frak T$.   $(X,\frak T)$ is monotonically
normal iff there is a function $W:R\to\frak T$ such that ($\dagger$)
$x\in W(x,G)\subseteq G$ whenever $(x,G)\in R$ ($\ddagger$) if $(x,G)\in R$
and $G\subseteq H\in\frak T$ then $W(x,G)\subseteq W(x,H)$
(*) if $x$,
$y\in X$ and $\overline{\{x\}}\ne\overline{\{y\}}$ then
$W(x,X\setminus\overline{\{y\}})$ and $W(y,X\setminus\overline{\{x\}})$ are
disjoint.   \Prf\ Let $Q$ be the set of disjoint pairs of closed subsets of
$X$.   ($\alpha$) Suppose that $(X,\frak T)$ is monotonically normal.   Let
$W_1:Q\to\frak T$ be a function as in the definition in 4A2A.   If
$(x,G)\in R$, then (because $X$ is regular) $\overline{\{x\}}$ is disjoint
from $X\setminus G$;  set

\Centerline{$W(x,G)
=W_1(\overline{\{x\}},X\setminus G)
 \setminus\overline{W(X\setminus G,\overline{\{x\}})}$.}

\noindent It is easy to check that this function $W$ satisfies the
conditions ($\dagger$), ($\ddagger$) and (*) here.   ($\beta$) Suppose that
$W:R\to\frak T$ satisfies the conditions here.   For disjoint closed sets
$E$, $F\subseteq X$ set $W_1(E,F)=\bigcup_{x\in E}W(x,X\setminus F)$.
Then $W_1$ witnesses that $X$ is monotonically normal, because
$W_1(E,F)$ will be disjoint from $W(y,E)$ for every $y\in F$.\ \Qed

Note that if $W$ is any function satisfying ($\dagger$), ($\ddagger$) and
(*), and $(x,G)$, $(y,H)\in R$ are such that
$W(x,G)\cap W(y,H)\ne\emptyset$, then either $x\in H$ or $y\in G$.

\quad(iii) If $X$ is a regular monotonically normal space, every subspace
of $X$ is monotonically normal.
\Prf\ Let $W$ be a function as in (ii), and
$Y$ a subspace of $X$.   Then $Y$ is regular.
If $y\in Y$ and $H$ is a relatively open subset of $Y$, write $\tilde H$
for $\bigcup\{G:G\subseteq X$ is open, $G\cap Y\subseteq H\}$ and
$W'(y,H)=Y\cap W(y,\tilde H)$.   Then $W'$ satisfies the conditions
($\dagger$), ($\ddagger$) and (*) of (ii), so $Y$ is monotonically
normal.\ \Qed

\quad(iv) A regular monotonically normal space is countably paracompact.
\Prf\ (See {\smc Rudin 84}.) Let $X$ be a regular monotonically normal
space and $W$ a function as in (ii) above.   Let $\sequencen{F_n}$ be a
non-decreasing sequence of closed sets in $X$ with empty intersection and
$F_0=X$.   For $x\in F_n\setminus F_{n+1}$ define $\sequence{k}{V_k(x)}$ by
setting $V_0(x)=W(x,X\setminus F_{n+1})$ and $V_{k+1}(x)=W(x,V_k(x))$ for
every $k$.   Let $P$ be the set of all families $\family{n}{I}{p_n}$ where
$I\subseteq\Bbb N$ is infinite and
$p_m\in F_m\cap V_{n-m}(p_n)\setminus F_{n+1}$ whenever $m$, $n\in K$ and
$m<n$;  for $\tbf{p}=\family{n}{I}{p_n}\in P$ set

\Centerline{$S_{\tbf{p}}
=\bigcup_{m\in\Bbb N}F_m\cap\bigcap_{n\in I,n>m}V_{n-m}(p_n)
  \setminus F_{m+1}$.}

\noindent Let $\preccurlyeq$ be a well-ordering of $P$.   For
$x\in F_m\setminus F_{m+1}$, set

$$\eqalignno{U(x)&=V_m(x)\cr
\noalign{\noindent if $x\notin\bigcup_{\tbf{p}\in P}S_{\tbf{p}}$}
&=V_m(x)\setminus\bigcup_{n\in I\cap m}\overline{V_0(p_n)}\cr}$$

\noindent if $\tbf{p}=\family{n}{I}{p_n}$ is the $\preccurlyeq$-first
member of $P$ such that $x\in S_{\tbf{p}}$.   Note that in this case
$\overline{V_0(p_n)}$ is disjoint from $F_m$ and does not contain $x$ for
every $n\in I\cap m$, so
$U(x)$ contains $x$.   Set $U'(x)=W(x,U(x))$ for every $x\in X$, and
$G_n=\bigcup_{x\in F_n}U'(x)$;  then
$G_n$ is an open set including $F_n$ for each $n$.

\Quer\ Suppose, if possible, that $\bigcap_{n\in\Bbb N}G_n$ is not empty;
take $z$ in the intersection.   For each $n$, there is an $x\in F_n$ such
that $z\in U'(x)$, so
$J=\{n:z\in\bigcup_{x\in F_n\setminus F_{n+1}}U'(x)\}$
is infinite;  for each $n\in J$, choose $q_n\in F_n\setminus F_{n+1}$ such
that $z\in U'(q_n)$.

If $m$, $n\in J$ and $m<n$, then $U'(q_m)\cap U'(q_n)$ is not empty, so
$W(q_m,U(q_m))\subseteq W(q_m,V_0(q_m))$ meets
$W(q_n,U(q_n))$.   But $q_n\in F_n$ so
$q_n\notin V_0(q_m)$;  it follows that $q_m$ must belong to
$U(q_n)\subseteq V_n(q_n)$.
(See the last sentence in (ii) above.)     As $m$ and $n$ are arbitrary,
$\tbf{q}=\family{n}{J}{q_n}$ belongs to $P$, and of course every $q_n$
belongs to $S_{\tbf{q}}$.

Let $\tbf{p}=\family{n}{I}{p_n}$ be the
$\preccurlyeq$-first member of $P$ such that $S_{\tbf{p}}$ contains any
$q_n$;   let $j\in J$ be such that $q_j\in S_{\tbf{p}}$.   If $j<m<n$,
$m\in J$ and $n\in I$, then $V_{m-j}(q_m)\cap V_{n-j}(p_n)$ contains $q_j$
so is not empty, and either $p_n\in V_{m-j-1}(q_m)$ or
$q_m\in V_{n-j-1}(p_n)$;  the former is impossible because
$p_n\in F_{m+1}$, so $q_m\in V_{n-m}(p_n)$.   As $n$ is arbitrary,
$q_m\in S_{\tbf{p}}$, and $\tbf{p}$ must be
the $\preccurlyeq$-least member of
$P$ such that $q_m\in S_{\tbf{p}}$.   And this is true for every $m\in J$
such that $m>j$.

Now take $n\in I$ and $m\in J$ such that $j<n<m$.   We have
$q_j\in V_{n-j}(p_n)\cap U(q_m)$.   But $U(q_m)$ is
disjoint from $V_0(p_n)$, so this is impossible.\ \Bang

Thus $\bigcap_{n\in\Bbb N}G_n$ is empty.   As $\sequencen{F_n}$ is
arbitrary, $X$ is countably paracompact.\ \Qed
}%end of leaveitout

\leader{4A2G}{Compact and locally compact spaces (a)} In any topological
space, the union of two compact subsets is compact.

\spheader 4A2Gb A compact Hausdorff space is normal.
\prooflet{(\Engelking, 3.1.9;  \Csaszar, 5.3.23;  \Gaal, p.\ 139.)}
%4@33 %4@16 %4@46

\spheader 4A2Gc(i) If $X$ is a compact Hausdorff space,
%normal will do
$Y\subseteq X$
is a zero set and $Z\subseteq Y$ is a zero set in $Y$, then $Z$ is a
zero set in $X$.
\prooflet{(By 4A2C(b-vi) and
4A2C(a-iv), $Z$ is a G$_{\delta}$ set in $X$;  now use 4A2F(d-v).)}
%4@19

\quad(ii) Let $X$ and $Y$ be compact Hausdorff spaces,
$f:X\to Y$ a
continuous open map and $Z\subseteq X$ a zero set in $X$.   Then $f[Z]$
is a zero set in $Y$.   \prooflet{\Prf\ Let $g:X\to\Bbb R$ be a
continuous function such that $Z=g^{-1}[\{0\}]$.   Set
$G_n=\{x:x\in X$, $|g(x)|<2^{-n}\}$ for each $n\in\Bbb N$.   If
$y\in\bigcap_{n\in\Bbb N}f[G_n]$, then $f^{-1}[\{y\}]$ is a compact set
meeting all the closed sets $\overline{G}_n$, so meets their
intersection, which is $Z$.   Thus $f[Z]=\bigcap_{n\in\Bbb N}f[G_n]$ is
a G$_{\delta}$ set.   By 4A2F(d-v), it is a zero set.\ \Qed}

\spheader 4A2Gd If $X$ is a Hausdorff space, $\Cal V$ is a
downwards-directed family
of compact neighbourhoods of a point $x$ of $X$ and
$\bigcap\Cal V=\{x\}$, then $\Cal V$ is a base of neighbourhoods of $x$.
\prooflet{\Prf\ Let $G$ be any open set containing $x$.   Fix any
$V_0\in\Cal V$.   Note that because $X$ is Hausdorff, every member of
$\Cal V$ is closed (3A3Dc).   So $\{V_0\cap V\setminus G:V\in\Cal V\}$
is a family of (relatively) closed subsets of $V_0$ with empty
intersection, cannot have the finite intersection property (3A3Da),
and there is a $V\in\Cal V$ such that $V_0\cap V\setminus G=\emptyset$.
Now there is a $V'\in\Cal V$ such that $V'\subseteq V_0\cap V$ and
$V'\subseteq G$.   As $G$ is arbitrary, $\Cal V$ is a base of
neighbourhoods of $x$.\ \Qed}
%4@47? %4@46

%continuous r-v functions
\spheader 4A2Ge Let $(X,\frak T)$ be a locally compact Hausdorff space.

\quad(i) If $K\subseteq X$ is a compact set and
$G\supseteq K$ is open, then there is a continuous $f:X\to[0,1]$ with
compact support such that $\chi K\le f\le\chi G$.
\prooflet{(Let $\Cal V$ be the family of relatively compact open subsets
of $X$.   Then $\Cal V$ is upwards-directed and covers $X$, so there is
a $V\in\Cal V$ including $K$.   By 3A3Bb, $\frak T$ is completely
regular;  now 4A2F(h-iii) tells us that there is a continuous
$f:X\to[0,1]$ such that $\chi K\le f\le\chi(G\cap V)$, so that $f$ has
compact support.)}
%4@34

\quad (ii) $\frak T$ is the coarsest topology on $X$ such that every
$\frak T$-continuous real-valued function with compact support is
continuous.
\prooflet{\Prf\ Let $\Phi$ be the set of continuous functions of compact
support for $\frak T$.   If $\frak S$ is a topology on $X$ such that
every member of $\Phi$ is continuous, and $x\in G\in\frak T$, then there
is an $f\in\Phi$ such that $f(x)=1$ and $f(y)=0$ for
$y\in X\setminus G$, by (i).   Now $f$ is $\frak S$-continuous, by
hypothesis, so
$H=\{y:f(y)>\bover12\}$ belongs to $\frak S$ and $x\in H\subseteq G$.
As $x$ is arbitrary, $G=\interior_{\frak S}G$ belongs to $\frak S$;  as
$G$ is arbitrary, $\frak T\subseteq\frak S$.\ \Qed}
%4@34

%countably compact spaces
\spheader 4A2Gf(i) A topological space $X$ is countably compact iff
every sequence in $X$ has a cluster point in $X$\cmmnt{, that is, $X$
is relatively countably compact in itself.}
\prooflet{(\Engelking, 3.10.3;  \Csaszar, 5.3.31(e);  \Gaal, p.\ 129.)}
%4@16 %4@25

\quad (ii) If $X$ is a countably compact topological space and
$\sequencen{F_n}$ is a sequence of
closed sets such that $\bigcap_{i\le n}F_i\ne\emptyset$ for every
$n\in\Bbb N$, then $\bigcap_{n\in\Bbb N}F_n\ne\emptyset$.
\prooflet{(\Engelking, 3.10.3;  \Csaszar, 5.3.31(c).)}
%4@35

\quad (iii) In any topological space, a relatively compact set is
relatively countably compact\prooflet{ (2A3Ob)}.

\quad (iv) Let $X$ and $Y$ be topological spaces and $f:X\to Y$ a
continuous function.   If $A\subseteq X$ is relatively countably compact
in $X$, then $f[A]$ is relatively countably compact in $Y$.
\prooflet{\Prf\ Let $\sequencen{y_n}$ be a sequence in $f[A]$.   Then
there is a sequence $\sequencen{x_n}$ in $A$ such that $f(x_n)=y_n$ for
every $n\in\Bbb N$.   Because $A$ is relatively countably compact,
$\sequencen{x_n}$ has a cluster point $x\in X$.   If $n_0\in\Bbb N$ and
$H$ is an open set containing $f(x)$, there is an $n\ge n_0$ such that
$x_n\in f^{-1}[H]$, so that $y_n\in H$.   Thus $f(x)$ is a cluster point
of $\sequencen{y_n}$;  as $\sequencen{y_n}$ is arbitrary, $f[A]$ is
relatively countably compact.\ \Qed}

\quad (v) A relatively countably compact set in $\Bbb R$ must be
bounded.
\prooflet{(If $A\subseteq\Bbb R$ is unbounded there is a sequence
$\sequencen{x_n}$ in $A$ such that $|x_n|\ge n$ for every $n$.)}
So if $X$ is a topological space, $A\subseteq X$ is relatively countably
compact and $f:X\to\Bbb R$ is continuous, then $f[A]$ is bounded.

\quad (vi) If $X$ and $Y$ are topological spaces and $f:X\to Y$ is
continuous, then $f[A]$ is countably compact whenever $A\subseteq X$ is
countably compact.
\prooflet{(\Engelking, 3.10.5.)}

%uniform convergence
\spheader 4A2Gg(i) Let $X$ and $Y$ be topological spaces and
$\phi:X\times Y\to\Bbb R$ a continuous function.   Define
$\theta:X\to C(Y)$ by setting $\theta(x)(y)=\phi(x,y)$ for $x\in X$,
$y\in Y$.   Then $\theta$ is continuous if we give $C(Y)$ the topology
of uniform convergence on compact subsets of $Y$.
\prooflet{(As noted in \Engelking, pp.\ 157-158, the topology of uniform
convergence on compact sets is the `compact-open' topology of $C(Y)$,
as defined in 441Yh, so
the result here is covered by \Engelking, 3.4.1.)}

\quad (ii) In particular, if $Y$ is compact then $\theta$ is continuous
if we give $C(Y)$ its usual norm topology.
%4@44

%convergent sequences
\spheader 4A2Gh(i) Suppose that $X$ is a compact space such that there
are no non-trivial convergent sequences in $X$\cmmnt{, that is, no
convergent sequences which are not eventually constant}.   If
$\sequencen{F_n}$ is a non-increasing sequence of infinite closed
subsets of $X$, then $F=\bigcap_{n\in\Bbb N}F_n$ is infinite.
\prooflet{\Prf\ Because $X$ is compact, $F$ cannot be empty (3A3Da).
Choose a sequence $\sequencen{x_n}$ such that
$x_n\in F_n\setminus\{x_i:i<n\}$ for each $n\in\Bbb N$.    If
$G\supseteq F$ is an open set, then
$\bigcap_{n\in\Bbb N}F_n\setminus G=\emptyset$, so there must be some
$n\in\Bbb N$ such that $F_n\subseteq G$, and $x_i\in G$ for $i\ge n$.
\Quer\ If $F=\{y_0,\ldots,y_k\}$, let $l\le k$ be the first point such
that whenever $G\supseteq\{y_0,\ldots,y_l\}$ is open, then
$\{i:x_i\notin G\}$ is finite.   Then there is an open set
$G'\supseteq\{y_j:j<l\}$ such that $I=\{i:x_i\notin G'\}$ is infinite.
But if $H$ is any open set containing $y_l$, then $\{i:x_i\notin G'\cup
H\}$ is finite, so $\{i:i\in I,\,x_i\notin H\}$ is finite.   Thus if we
re-enumerate $\familyiI{x_i}$ as $\sequencen{x'_n}$, $\sequencen{x'_n}$
converges to $y_l$ and is a non-trivial convergent sequence.\ \BanG\
Thus $F$ is infinite, as claimed.\ \Qed}


\quad (ii) If $X$ is an infinite scattered compact Hausdorff space it
has a non-trivial convergent sequence.
\prooflet{\Prf\ Let $\sequencen{x_n}$ be any sequence of distinct points
in $X$.   Set $F_n=\overline{\{x_i:i\ge n\}}$ for each $n$, so that
$F=\bigcap_{n\in\Bbb N}F_n$ is a non-empty set.   Because $X$ is
scattered, $F$ has an isolated point $z$ say;  let $G$ be an open set
such that $F\cap G=\{z\}$, and $H$ an open set such that
$z\in H\subseteq\overline{H}\subseteq G$ (3A3Bb).   In this case,
$I=\{i:x_i\in H\}$ must be infinite;
re-enumerate $\familyiI{x_i}$ as $\sequencen{x'_n}$.   \Quer\ If
$\sequencen{x'_n}$ does not converge to $z$, there is an open set $H'$
containing $z$ such that $\{n:x'_n\notin H'\}$ is infinite, that is,
$\{i:i\in I,\,x_i\notin H'\}$ is infinite.   In this case,
$F_n\cap\overline{H}\setminus H'$ is non-empty for every $n\in\Bbb N$,
but $F\cap\overline{H}\setminus H'=\emptyset$, which is impossible.\
\BanG\   Thus $\sequencen{x'_n}$ is a non-trivial convergent sequence in
$X$.\ \Qed}


\quad (iii) If $X$ is an extremally disconnected Hausdorff
space\cmmnt{ (definition: 3A3Af)}, it has no non-trivial convergent
sequence.   \prooflet{\Prf\Quer\ Suppose, if possible, that there is a
sequence $\sequencen{x_n}$ converging to $x\in X$ such that
$\{n:x_n\ne x\}$ is infinite.   Choose
$\sequence{i}{n_i}$ and $\sequence{i}{G_i}$ inductively, as follows.
Given that $x\notin\overline{G}_j$ for $j<i$, there is an $n_i$ such
that $x_{n_i}\ne x$ and $x_{n_i}\notin\overline{G}_j$ for every $j<i$;
now let $G_i$ be an open set such that $x_{n_i}\in G_i$ and
$x\notin\overline{G}_i$, and continue.

Since all the $n_i$ must be distinct, $\sequence{i}{x_{n_i}}\to x$.
But consider

\Centerline{$G
=\bigcup_{i\in\Bbb N}G_{2i}\setminus\bigcup_{j<2i}\overline{G}_j$,
\quad$H
=\bigcup_{i\in\Bbb N}G_{2i+1}\setminus\bigcup_{j\le 2i}\overline{G}_j$.}

\noindent Then $G$ and $H$ are disjoint open sets and $x_{n_{2i}}\in G$,
$x_{n_{2i+1}}\in H$ for every $i$.   So
$x\in\overline{G}\cap\overline{H}$.   But $\overline{G}$ is open
(because $X$ is extremally disconnected), and is disjoint from $H$, and
now $\overline{H}$ is disjoint from $\overline{G}$;  so they cannot both
contain $x$.\ \Bang\Qed}

\spheader 4A2Gi(i) If $X$ and $Y$ are compact Hausdorff spaces and
$f:X\to Y$ is a continuous surjection then there is a closed set
$K\subseteq X$ such that $f[K]=Y$ and $f\restr K$ is irreducible.
\prooflet{\Prf\
Let $\Cal E$ be the family of closed sets $F\subseteq X$
such that $f[F]=Y$.   If $\Cal F\subseteq\Cal E$ is non-empty and
downwards-directed, then for any $y\in Y$ the family
$\{F\cap f^{-1}[\{y\}]:F\in\Cal F\}$ is a downwards-directed family of
non-empty closed sets, so (because $X$ is compact) has non-empty
intersection;  this shows that $\bigcap\Cal F\in\Cal E$.   By Zorn's
Lemma, $\Cal E$ has a minimal element $K$ say.   Now $f[K]=Y$ but
$f[F]\ne Y$ for any closed proper subset of $K$,
so $f\restr K$ is irreducible.\ \Qed}

\quad(ii) If $X$ and $Y$ are compact Hausdorff spaces and $f:X\to Y$ is an
irreducible continuous surjection, then ($\alpha$) if $\Cal U$ is a
$\pi$-base for the topology of $Y$ then $\{f^{-1}[U]:U\in\Cal U\}$ is a
$\pi$-base for the topology of $X$ ($\beta$) if $Y$ has a countable
$\pi$-base so does $X$ ($\gamma$) if $x$ is an isolated point in $X$ then
$f(x)$ is an isolated point in $Y$ ($\delta$) if $Y$ has no isolated
points, nor does $X$.
\prooflet{\Prf\ ($\alpha$) If $G\subseteq X$ is a non-empty open set then
$f[X\setminus G]\ne Y$.   As $f[X\setminus G]$ is closed, there is a
non-empty $U\in\Cal U$ disjoint from $f[X\setminus G]$.   Now $f^{-1}[U]$
is a non-empty subset of $G$.   ($\beta$) Follows at once from
($\alpha$).   ($\gamma$) By ($\alpha$), with $\Cal U$ the family of all
open subsets of $Y$, there is a non-empty open set $U\subseteq Y$ such that
$f^{-1}[U]\subseteq\{x\}$, that is, $U=\{f(x)\}$.   ($\delta$) Follows at
once from ($\gamma$).\ \Qed}

\spheader 4A2Gj(i) Let $X$ be a non-empty compact Hausdorff space
without isolated points.   Then there are a closed set $F\subseteq X$
and a continuous surjection $f:F\to\{0,1\}^{\Bbb N}$.
\prooflet{\Prf\ For $\sigma\in\bigcup_{n\in\Bbb N}\{0,1\}^n$ choose
closed sets $V_{\sigma}\subseteq X$ inductively, as follows.
$V_{\emptyset}=X$.   Given that $V_{\sigma}$ is a closed set with
non-empty interior, there are distinct points $x$,
$y\in\interior V_{\sigma}$ (because $X$ has no isolated points);  let
$G$, $H$ be disjoint open subsets of $X$ such that $x\in G$ and
$y\in H$;  and let $V_{\sigma^{\smallfrown}\fraction{0}}$ and
$V_{\sigma^{\smallfrown}\fraction{1}}$ be closed sets such that

\Centerline{$x\in\interior V_{\sigma^{\smallfrown}\fraction{0}}
\subseteq G\cap\interior V_{\sigma}$,
\quad$y\in\interior V_{\sigma^{\smallfrown}\fraction{1}}
\subseteq H\cap\interior V_{\sigma}$.}

\noindent (This is possible because $X$ is regular.)   The construction
ensures that $V_{\tau}\subseteq V_{\sigma}$ whenever $\tau\in\{0,1\}^n$
extends $\sigma\in\{0,1\}^m$, and that
$V_{\tau}\cap V_{\sigma}=\emptyset$ whenever $\tau$,
$\sigma\in\{0,1\}^n$ are different.   Set
$F=\bigcap_{n\in\Bbb N}\bigcup_{\sigma\in\{0,1\}^n}V_{\sigma}$;  then
$F$ is a closed subset of $X$ and we have a continuous function
$f:F\to\{0,1\}^{\Bbb N}$ defined by saying that $f(x)(i)=\sigma(i)$
whenever $n\in\Bbb N$, $\sigma\in\{0,1\}^n$, $i<n$ and
$x\in V_{\sigma}$.   Finally, $f$ is surjective, because if
$z\in\{0,1\}^{\Bbb N}$ then $\sequencen{V_{z\restr n}}$ is a
non-increasing sequence of closed sets in the compact space $X$, so has
non-empty intersection $V$ say, and $f(x)=z$ for any $x\in V$.\ \Qed}
%for (ii) and (iii)

\quad (ii) If $X$ is a non-empty compact Hausdorff space without
isolated points, then $\#(X)\ge\frak c$.
\prooflet{(Use (i).)}
%4@19

\quad (iii) If $X$ is a compact Hausdorff space which is not scattered,
it has an infinite closed subset with a
countable $\pi$-base and no isolated points.
\prooflet{\Prf\ Because $X$ is not scattered, it has a non-empty subset
$A$ without isolated points.   Then $\overline{A}$ is compact and has no
isolated points;  by (i), there are a closed set
$F_0\subseteq\overline{A}$ and a continuous surjection
$f:F_0\to\{0,1\}^{\Bbb N}$.   By (i-i) above, there is a closed
$F\subseteq F_0$ such that $f[F]=\{0,1\}^{\Bbb N}$ and $f\restr F$ is
surjective.   Of course $F$ is infinite;  by (i-ii), it has a countable
$\pi$-base and no isolated points.\ \Qed}


\quad (iv) Let $X$ be a compact Hausdorff space.   Then there is a
continuous surjection from $X$ onto $[0,1]$ iff $X$ is not scattered.
\prooflet{\Prf\ ($\alpha$) Suppose that $f:X\to[0,1]$ is a continuous
surjection.   By (i-i) again, there is a closed set $F\subseteq X$ such
that $f[F]=[0,1]$ and $f\restr F$ is irreducible;  by (i-ii) $F$ has no
isolated points.   So $X$ is not scattered.   ($\beta$)
If $X$ is not scattered, let $A\subseteq X$ be a
non-empty set with no isolated points.   Then $\overline{A}$ is a
non-empty compact subset of $X$ with no isolated points, so there is a
continuous surjection $g:\overline{A}\to\{0,1\}^{\Bbb N}$ ((i) of this
subparagraph).
Now there is a continuous surjection $h:\{0,1\}^{\Bbb N}\to[0,1]$ (e.g.,
set $h(y)=\sum_{n=0}^{\infty}2^{-n-1}y(n)$ for $y\in\{0,1\}^{\Bbb N}$),
so we have a continuous surjection $hg:\overline{A}\to[0,1]$.   By
Tietze's theorem (4A2F(d-ix)), there is a continuous function
$f_0:X\to\Bbb R$ extending $hg$;  setting $f(x)=\med(0,f_0(x),1)$
for $x\in X$, we have a continuous surjection $f:X\to[0,1]$.\ \Qed
}%end of prooflet

\quad(v) A Hausdorff continuous image of a scattered compact Hausdorff
space is scattered.   \prooflet{(Immediate from (iv).)}

\quad(vi) If $X$ is an uncountable first-countable compact Hausdorff
space, it is not scattered.   \prooflet{\Prf\ Let $\Cal G$ be the family
of countable open subsets of $X$, and $G^*$ its union.  No finite subset
of $\Cal G$ can cover $X$, so $X\setminus G^*$ is
non-empty.   \Quer\ If $x$ is an isolated point of $X\setminus G^*$,
then $\{x\}\cup G^*$ is a neighbourhood of $x$;  let $\sequencen{U_n}$
run over a base of open neighbourhoods of $x$ with
$\overline{U}_0\subseteq\{x\}\cup G^*$.   For each $n\in\Bbb N$,
$F_n=\overline{U}_0\setminus U_n$ is a compact set included in $G^*$, so
is covered by finitely many members of $\Cal G$, and is countable.   But
this means that $U_0=\{x\}\cup\bigcup_{n\in\Bbb N}U_0\setminus U_n$ is
countable, and $x\in G^*$.\ \BanG\  Thus $X\setminus G^*$ is a non-empty
set with no isolated points, and $X$ is not scattered.\ \Qed

}%end of prooflet
It follows that there is a continuous surjection from $X$ onto
$[0,1]$\prooflet{, by (iv)}.
%4@39

\spheader 4A2Gk A locally compact Hausdorff space is \v{C}ech-complete.
\prooflet{(\Engelking, p.\ 196.)}

\spheader 4A2Gl If $X$ is a topological space, $f:X\to\Bbb R$ is lower
semi-continuous, and $K\subseteq X$ is compact and not empty, then there
is an $x_0\in K$ such that $f(x_0)=\inf_{x\in K}f(x)$.
\prooflet{(\Gaal, p.\ 209 Theorem 3.)}
Similarly, if $g:X\to\Bbb R$ is upper
semi-continuous, there is an $x_1\in K$ such that
$g(x_1)=\sup_{x\in K}g(x)$.
%4@84 %4@75

\spheader 4A2Gm\dvAnew{2012}
If $X$ is a Hausdorff space, $Y$ is a compact space and
$F\subseteq X\times Y$ is closed, then its projection
$\{x:(x,y)\in F\}$ is a closed subset of $X$.   \prooflet{(\Engelking,
3.1.16.)}

\spheader 4A2Gn\dvAnew{2012}
If $X$ is a locally compact topological space,
$Y$ is a topological space and $f:X\to Y$ is a continuous open surjection,
then $Y$ is locally compact.   \prooflet{(\Engelking, 3.3.15.)}

\leader{4A2H}{Lindel\"of spaces (a)}
If $X$ is a topological space, then a subset $Y$ of $X$ is
Lindel\"of\cmmnt{ (in its subspace topology)} iff for
every family $\Cal G$ of open subsets of $X$ covering $Y$ there is a
countable subfamily of $\Cal G$ still covering $Y$.
%4@22

\spheader 4A2Hb(i) A regular Lindel\"of space $X$ is
paracompact, therefore normal and completely regular.
\prooflet{(\Engelking, 3.8.11 \& 5.1.2.)}
%4@15 %4@22 %4@26 %4@39?

\quad(ii) If $X$ is a Lindel\"of space and $\Cal A$ is a locally finite
family of subsets of $X$ then $\Cal A$ is countable.   \prooflet{\Prf\
The family $\Cal G$ of open sets meeting only finitely many members of
$\Cal A$ is an open cover of $X$.   If $\Cal G_0\subseteq\Cal G$ is a
countable cover of $X$ then $\{A:A\in\Cal A,\,A$ meets some member of
$\Cal G_0\}=\Cal A\setminus\{\emptyset\}$ is countable.\ \Qed}

%hereditarily Lindelof spaces
\spheader 4A2Hc(i) A topological space $X$ is hereditarily Lindel\"of
iff for any family $\Cal G$ of open
subsets of $X$ there is a countable family $\Cal G_0\subseteq\Cal G$
such that $\bigcup\Cal G_0=\bigcup\Cal G$.
\prooflet{\Prf\ ($\alpha$) If $X$ is hereditarily Lindel\"of and
$\Cal G$
is a family of open subsets of $X$, then $\Cal G$ is an open cover of
$\bigcup\Cal G$, so has a countable subcover.   ($\beta$) If $X$ is not
hereditarily Lindel\"of, let $Y\subseteq X$ be a non-Lindel\"of
subspace, and $\Cal H$ a cover of $Y$ by relatively open sets which has
no countable subcover;  setting
$\Cal G=\{G:G\subseteq X$ is open, $G\cap Y\in\Cal H\}$, there can be no
countable $\Cal G_0\subseteq\Cal G$ with union $\bigcup\Cal G$.\ \Qed}
%4@39

\quad(ii) Let $X$ be a regular hereditarily Lindel\"of space.   Then
$X$ is perfectly normal.
\prooflet{\Prf\ Let $F\subseteq X$ be closed.   Let $\Cal G$ be the
family of open sets $G\subseteq X$ such that
$\overline{G}\cap F=\emptyset$;  because $X$ is regular,
$\bigcup\Cal G=X\setminus F$;  because $X$ is hereditarily Lindel\"of,
there is a sequence $\sequencen{G_n}$ in $G$ such that
$X\setminus F=\bigcup_{n\in\Bbb N}G_n$.   This means that
$F=\bigcap_{n\in\Bbb N}X\setminus\overline{G_n}$ is a G$_{\delta}$ set.
But $X$ is normal ((b) above), so is perfectly normal.\ \Qed}
%4@35

\spheader 4A2Hd Any $\sigma$-compact topological space is Lindel\"of.
\prooflet{(\Engelking, 3.8.5.)}

\leader{4A2I}{Stone-\v{C}ech compactifications (a)}
Let $X$ be a completely regular Hausdorff space.
Then there is a compact Hausdorff space $\beta X$, the
{\bf Stone-\v{C}ech compactification} of $X$, in which $X$ can be
embedded as a dense subspace.
%4@16 %4@22
If $Y$ is another compact Hausdorff space, then every continuous
function from $X$ to $Y$ has a unique continuous extension to a
continuous function from $\beta X$ to $Y$.
\prooflet{(\Engelking, 3.6.1;  \Csaszar, 6.4d;  \Cech, 41D.5.)}
%4@18

\spheader 4A2Ib Let $I$ be any set, and write $\beta I$ for its
Stone-\v{C}ech compactification when $I$ is given its discrete topology.
Let $Z$ be the Stone space of the Boolean algebra $\Cal PI$.

\quad (i) There is a canonical homeomorphism $\phi:\beta I\to Z$ defined by
saying that $\phi(i)(a)=\chi a(i)$ for every $i\in I$ and $a\subseteq I$.
\prooflet{\Prf\ Recall that $Z$ is the set of ring homomorphisms from
$\Cal PI$ onto $\Bbb Z_2$ (311E).   If $i\in I$, let $\hat{i}$ be the
corresponding member of $Z$ defined by setting $\hat{i}(a)=\chi a(i)$
for every $a\subseteq I$.   Then $Z$ is compact and Hausdorff (311I), and
$i\mapsto\hat{i}:I\to Z$ is continuous, so has a unique extension to
a continuous function $\phi:\beta I\to Z$.

If $G\subseteq Z$ is open and not empty, it includes a set of the form
$\widehat{a}=\{\theta:\theta\in Z,\,\theta(a)=1\}$ where $a\subseteq I$
is not empty;  if $i$ is any member of $a$,
$\hat{i}\in\widehat{a}\subseteq G$ so
$G\cap\phi[\beta I]\ne\emptyset$.   This shows that $\phi[\beta I]$ is
dense in $Z$;  as $\beta I$ is compact, $\phi[\beta I]$ is compact,
therefore closed, and is equal to $Z$.   Thus $\phi$ is surjective.

If $t$, $u$ are distinct points of $\beta I$, there is an open subset
$H$ of $\beta I$ such that $t\in H$ and $u\notin\overline{H}$.   Set
$a=H\cap I$.   Then $t\in\overline{a}$, the closure of $a$ regarded as a
subset of $\beta I$, so $\phi(t)\in\overline{\phi[a]}$ (3A3Cd).   But
$\phi[a]=\{\hat{i}:i\in a\}\subseteq\widehat{a}$, which is
open-and-closed, so $\phi(t)\in\widehat{a}$.   Similarly, setting
$b=I\setminus\overline{H}$, $\phi(u)\in\widehat{b}$;  since
$\widehat{a}\cap\widehat{b}=\widehat{a\cap b}$ is empty,
$\phi(t)\ne\phi(u)$.   This shows that $\phi$ is injective, therefore a
homeomorphism between $\beta I$ and $Z$ (3A3Dd).\ \Qed}\cmmnt{

Note that if $z:\Cal PI\to\Bbb Z_2$ is a Boolean homomorphism, then
$\{J:z(J)=1\}$ is an ultrafilter on $I$;  and conversely, if $\Cal F$ is an
ultrafilter on $I$, we have a Boolean homomorphism $z:\Cal PI\to\Bbb Z_2$
such that $\Cal F=z^{-1}[\{1\}]$.}
\dvro{We}{So we} can identify $\beta I$ with the
set of ultrafilters on $I$.   Under this identification, the canonical
embedding of $I$ in $\beta I$ corresponds to matching each member of $I$
with the corresponding principal ultrafilter on $I$.

\quad (ii) $C(\beta I)$ is isomorphic, as Banach lattice, to
$\ell^{\infty}(I)$.
\prooflet{\Prf\ By 363Ha, we
can identify $\ell^{\infty}(I)$, as Banach lattice, with
$L^{\infty}(\Cal PI)=C(Z)$.   But (i) tells us that we have a canonical
identification between $C(Z)$ and $C(\beta I)$.\ \Qed}

\quad (iii) We have a one-to-one correspondence between filters $\Cal F$
on $I$ and non-empty closed sets $F\subseteq\beta I$, got by matching
$\Cal F$ with $\bigcap\{\widehat{a}:a\in\Cal F\}$, or $F$ with
$\{a:a\subseteq I,\,F\subseteq\widehat{a}\}$, where
$\widehat{a}\subseteq\beta I$ is the open-and-closed set corresponding
to $a\subseteq I$.
\prooflet{\Prf\ The identification of $\beta I$ with $Z$
means that we can regard the map
$a\mapsto\widehat{a}$ as a Boolean isomorphism between $\Cal PI$ and the
algebra of open-and-closed subsets of $\beta I$ (311I).   For any filter
$\Cal F$ on $I$, set $H(\Cal F)=\bigcap\{\widehat{a}:a\in\Cal F\}$;
because $\{\widehat{a}:a\in\Cal F\}$ is a downwards-directed family of
non-empty closed sets in the compact Hausdorff space $\beta I$,
$H(\Cal F)$ is a non-empty closed set.   If $F\subseteq\beta I$ is a
non-empty closed set, then it is elementary to check that
$\Cal H(F)=\{a:F\subseteq\widehat{a}\}$ is a filter on $I$, and
evidently $H(\Cal H(F))\supseteq F$.   But if $t\in\beta I\setminus F$,
then (because $\{\widehat{a}:a\subseteq I\}$ is a base for the topology
of $\beta I$, see 311I again) there is an $a\subseteq I$ such that
$t\in\widehat{a}$ and $F\cap\widehat{a}=\emptyset$, that is,
$F\subseteq\widehat{I\setminus a}$;  so
$\widehat{I\setminus a}\in\Cal H(F)$ and
$H(\Cal H(F))\subseteq\widehat{I\setminus a}$ and
$t\notin H(\Cal H(F))$.   Thus $H(\Cal H(F))=F$ for every non-empty
closed set $F\subseteq\beta I$.

If $\Cal F_1$ and $\Cal F_2$ are filters on $I$ and
$a\in\Cal F_1\setminus\Cal F_2$, then
$\{\widehat{b\setminus a}:b\in\Cal F_2\}$ is a downwards-directed family
of non-empty closed sets in $\beta I$, so has non-empty intersection;
if $t\in\widehat{b\setminus a}=\widehat{b}\setminus\widehat{a}$ for
every $b\in\Cal F_2$, then $t\in H(\Cal F_2)\setminus H(\Cal F_1)$.
This shows that $\Cal F\mapsto H(\Cal F)$ is injective.   It follows
that $\Cal F\mapsto H(\Cal F)$, $F\mapsto\Cal H(F)$ are the two halves
of a bijection, as claimed.\ \Qed}

\quad (iv) $\beta I$ is extremally disconnected.
\prooflet{(Because $\Cal PI$ is Dedekind complete, $Z$
is extremally disconnected (314S).)}

\quad(v) There are no non-trivial convergent sequences in
$\beta I$.
\prooflet{(4A2G(h-iii).   Compare \Engelking, 3.6.15.)}


\vleader{48pt}{4A2J}{Uniform spaces}\cmmnt{ (See \S3A4.)}  Let
$(X,\Cal W)$ be a uniform space;  give $X$ the associated topology
$\frak T$\cmmnt{ (3A4Ab)}.

\spheader 4A2Ja $\Cal W$ is generated by a family of pseudometrics.
\prooflet{(\Engelking, 8.1.10;  {\smc Bourbaki 66}, IX.1.4;  \Csaszar,
4.2.32.)}   More precisely:  if $\sequencen{W_n}$ is any sequence in
$\Cal W$, there is a pseudometric $\rho$ on $X$ such that ($\alpha$)
$\{(x,y):\rho(x,y)\le\epsilon\}\in\Cal W$ for every $\epsilon>0$
($\beta$) whenever $n\in\Bbb N$ and $\rho(x,y)<2^{-n}$ then
$(x,y)\in W_n$\prooflet{ (\Engelking, 8.1.10)}.

It follows that $\frak T$ is completely regular, therefore
regular\prooflet{ (3A3Be)}.
$\frak T$ is defined by the bounded uniformly continuous functions, in
the sense that it is the coarsest topology\cmmnt{ $\frak S$} on $X$
such that these are all continuous.   \prooflet{\Prf\ Let $\Rho$ be the
family of
pseudometrics compatible with $\Cal W$ in the sense of ($\alpha$) just
above.   If $x\in G\in\frak T$, there is a $\rho\in\Rho$ such that
$\{y:\rho(x,y)<1\}\subseteq G$;  setting $f(y)=\rho(x,y)$, we see that
$f$ is uniformly continuous, therefore $\frak S$-continuous, and that
$x\in\interior_{\frak S}G$.   As $x$ is arbitrary, $G\in\frak S$;  as
$G$ is arbitrary, $\frak T\subseteq\frak S$;  but of course
$\frak S\subseteq\frak T$ just because uniformly continuous functions
are continuous.\ \Qed}
%4@49

\spheader 4A2Jb\dvAnew{2011}
If $\Cal W$ is countably generated and $\frak T$ is
Hausdorff, there is a metric $\rho$ on $X$ defining $\Cal W$ and
$\frak T$.  \prooflet{({\smc Engelking 89}, 8.1.21.)}
%Bourbaki Gen Top \query

\spheader 4A2Jc
%If $W\in\Cal W$, the interior of $W$\cmmnt{ (for the product topology
%on $X\times X$)} belongs to $\Cal W$.
%\prooflet{(\Engelking, 8.1.12.)}
If $W\in\Cal W$ and $x\in X$ then $x\in\interior W[\{x\}]$.
\prooflet{(\Engelking, 8.1.3.)}   If $A\subseteq X$ then
$\overline{A}=\bigcap_{W\in\Cal W}W[A]$.   \prooflet{(\Engelking,
8.1.4.)}
%4@A2Lc

\spheader 4A2Jd Any subset of a totally bounded set in $X$ is totally
bounded.
\prooflet{(\Engelking, 8.3.2;  \Csaszar, 3.2.70.)}
The closure of a totally bounded set is totally bounded.
\prooflet{\Prf\ If $A$ is totally bounded and $W\in\Cal W$, take
$W'\in\Cal W$ such that $W'\frsmallcirc W'\subseteq W$.   Then there is
a finite set
$I\subseteq X$ such that $A\subseteq W'[I]$.   In this case

\Centerline{$\overline{A}\subseteq W'[A]
\subseteq W'[W'[I]]=(W'\frsmallcirc W')[I]\subseteq W[I]$}

\noindent by (b).   As $W$ is arbitrary, $\overline{A}$ is totally
bounded.\ \Qed}

\spheader 4A2Je A subset of $X$ is compact iff it is
complete\cmmnt{ (definition: 3A4F)} (for its subspace uniformity) and
totally bounded.
\prooflet{(\Engelking, 8.3.16;  \Cech, 41A.8;  \Csaszar, 5.2.22;  \Gaal,
pp.\ 278-279.)}
%4@43  %4@33 %4@25
So if $X$ is complete, every closed totally bounded subset of $X$ is
compact, and the totally bounded sets are just the relatively compact
sets.
\prooflet{(A closed subspace of a complete space is complete.)}
%4@41 %4@48 %4@A4

\spheader 4A2Jf If $f:X\to\Bbb R$ is a continuous function with compact
support, it is uniformly continuous.
\prooflet{\Prf\ Set $K=\overline{\{x:f(x)\ne 0\}}$.
Let $\epsilon>0$.   For each $x\in X$, there is a $W_x\in\Cal W$ such
that $|f(y)-f(x)|\le\bover12\epsilon$ whenever $y\in W_x[\{x\}]$.   Let
$W'_x\in\Cal W$ be such that $W'_x\frsmallcirc W'_x\subseteq W_x$.   Set
$G_x=\interior W'_x[\{x\}]$;  then $x\in G_x$, by (b).   Because $K$ is
compact, there is a finite set $I\subseteq K$ such that
$K\subseteq\bigcup_{x\in I}G_x$.   Set
$W=(X\times X)\cap\bigcap_{x\in I}W'_x\in\Cal W$.   Take any
$(y,z)\in W\cap W^{-1}$.   If neither $y$ nor $z$ belongs to $K$, then
of course $|f(y)-f(z)|\le\epsilon$.   If $y\in K$, let $x\in I$ be such
that $y\in G_x$.   Then

\Centerline{$y\in W'_x[\{x\}]\subseteq W_x[\{x\}]$,
\quad$z\in W[W'_x[\{x\}]]\subseteq W'_x[W'_x[\{x\}]]
\subseteq W_x[\{x\}]$,}

\noindent so

\Centerline{$|f(y)-f(z)|\le|f(y)-f(x)|+|f(z)-f(x)|\le\epsilon$.}

\noindent The same idea works if $z\in K$.   So $|f(y)-f(z)|\le\epsilon$
for all $y$, $z\in W\cap W^{-1}$;  as $\epsilon$ is arbitrary, $f$ is
uniformly continuous.\ \Qed}
%4@A4

\spheader 4A2Jg\dvAformerly{4{}A2Jf}(i) 
If $(Y,\frak S)$ is a completely regular space,
there is a uniformity on $Y$ compatible with $\frak S$.
\prooflet{(\Engelking, 8.1.20.)}

\quad(ii) If $(Y,\frak S)$ is a compact
Hausdorff space, there is exactly one
uniformity on $Y$ compatible with $\frak S$;  it is induced by the set
of all those pseudometrics on $Y$ which are continuous as functions from
$Y\times Y$ to $\Bbb R$.
\prooflet{(\Engelking, 8.3.13;  \Gaal, p.\ 304.)}

\quad(iii) If $(Y,\frak S)$ is a compact Hausdorff space and $\Cal V$ is
the uniformity on $Y$ compatible with $\frak S$, then any continuous
function from $Y$ to $X$ is uniformly continuous.
%not Engelking, Kuratowski
\prooflet{(\Gaal, p.\ 305 Theorem 8.)}

\spheader 4A2Jh The set $U$ of uniformly continuous real-valued
functions on $X$ is a Riesz subspace of $\BbbR^X$ containing the
constant functions.   If a sequence in $U$ converges uniformly, the
limit function again belongs to $U$.
%not Kuratowski, Engelking
\prooflet{(\Csaszar, 3.2.64;  \Gaal, p.\ 237 Lemma 4.)}

\spheader 4A2Ji Let $(Y,\Cal V)$ be another uniform space.   If $\Cal F$
is a Cauchy filter on $X$ and $f:X\to Y$ is a uniformly continuous
function, then $f[[\Cal F]]$ is a Cauchy filter on $Y$.
\prooflet{(\Csaszar, 5.1.2.)}
%not Engelking ;  not \Gaal

\vleader{48pt}{4A2K}{First-countable, sequential and countably tight
spaces (a)}
%\spheader 4A2Ka
Let $X$ be a countably tight topological space.   If
$\ofamily{\xi}{\zeta}{F_{\xi}}$ is a non-decreasing family of closed
subsets of $X$ indexed by an ordinal $\zeta$, then
$E=\bigcup_{\xi<\zeta}F_{\xi}$ is an F$_{\sigma}$ set, and is
closed unless $\cf\zeta=\omega$.
\prooflet{\Prf\ If $\cf\zeta=0$, that is, $\zeta=0$, then $E=\emptyset$
is closed.   If $\cf\zeta=1$, that is, $\zeta=\xi+1$ for some ordinal
$\xi$, then $E=F_{\xi}$ is closed.   If $\cf\zeta=\omega$, there is a
sequence $\sequencen{\xi_n}$ in $\zeta$ with supremum $\zeta$, so that
$E=\bigcup_{n\in\Bbb N}F_{\xi_n}$ is F$_{\sigma}$.   If
$\cf\zeta>\omega$, take $x\in\overline{E}$.   Then there is a sequence
$\sequencen{x_n}$ in $E$ such that $x\in\overline{\{x_n:n\in\Bbb N\}}$.
For each $n$ there is a $\xi_n<\zeta$ such that $x_n\in F_{\xi_n}$, and
now $\xi=\sup_{n\in\Bbb N}\xi_n<\zeta$ and
$x\in\overline{F}_{\xi}=F_{\xi}\subseteq E$.   As
$x$ is arbitrary, $E$ is closed.\ \Qed}

\spheader 4A2Kb If $X$ is countably tight, any subspace of $X$ is
countably tight\prooflet{, just because if $A\subseteq Y\subseteq X$
then the closure of $A$ in $Y$ is the intersection of $Y$ with the
closure of $A$ in $X$}.   If $X$ is compact and countably tight, then
any Hausdorff continuous image of $X$ is countably tight.
\prooflet{ \Prf\ Let $f:X\to Y$ be a continuous surjection, where $Y$ is
Hausdorff, $B$ a subset of $Y$ and $y\in\overline{B}$.   Set
$A=f^{-1}[B]$.   Then $\overline{A}$ is compact, so $f[\overline{A}]$ is
compact, therefore closed;  because $f$ is surjective,
$y\in\overline{f[A]}\subseteq f[\overline{A}]$, and there is an
$x\in\overline{A}$ such that $f(x)=y$.   Now there is a countable set
$A_0\subseteq A$ such that $x\in\overline{A}_0$, in which case

\Centerline{$y=f(x)\in f[\overline{A}_0]\subseteq\overline{f[A_0]}$,}

\noindent while $f[A_0]$ is a countable subset of $B$.\ \Qed}

\spheader 4A2Kc If $X$ is a sequential space, it is countably tight.
\prooflet{\Prf\ Suppose that $A\subseteq X$ and $x\in\overline{A}$.
Set $B=\bigcup\{\overline{C}:C\subseteq A$ is countable$\}$.   If
$\sequencen{y_n}$ is a sequence in $B$ converging to $y\in X$, then for
each $n\in\Bbb N$ we can find a countable set $C_n\subseteq A$ such that
$y_n\in\overline{C}_n$, and now $C=\bigcup_{n\in\Bbb N}C_n$ is a
countable subset of $A$ such that $y\in\overline{C}\subseteq B$.   So
$B$ is sequentially closed, therefore closed, and $x\in B$.   As $A$ and
$x$ are arbitrary, $X$ is countably tight.\ \Qed}

\spheader 4A2Kd If $X$ is a sequential space, $Y$ is a topological space
and $f:X\to Y$ is sequentially continuous, then $f$ is continuous.
\prooflet{(\Engelking, 1.6.15.)}
%4@36

\spheader 4A2Ke First-countable spaces are sequential.
\prooflet{(\Engelking, 1.6.14.)}
%4@A2N

%any subspace of a first-countable space is first-countable

\spheader 4A2Kf Let $X$ be a locally compact Hausdorff space in which
every singleton set is G$_{\delta}$.   Then $X$ is first-countable.
\prooflet{\Prf\ If $\{x\}=\bigcap_{n\in\Bbb N}G_n$ where each $G_n$ is
open, then for each $n\in\Bbb N$ we can find a compact set $F_n$ such
that $x\in\interior F_n\subseteq G_n$.   By 4A2Gd,
$\{\bigcap_{i\le n}F_i:n\in\Bbb N\}$ is a base of neighbourhoods of
$x$.\ \Qed}
%4@39

\leader{4A2L}{(Pseudo-\nobreak)metrizable spaces}
\cmmnt{`Pseudometrizable'
spaces, as such, hardly appear in this volume, for the usual reasons;
they surface briefly in \S463.   It is perhaps worth noting, however,
that all the ideas, and very nearly all the results, in this paragraph
apply equally well to pseudometrics and pseudometrizable topologies.
If $X$ is a set and $\rho$ is a pseudometric on $X$,
set $U(x,\delta)=\{y:\rho(x,y)<\delta\}$ for $x\in X$ and $\delta>0$.}

\spheader 4A2La Any subspace of a (pseudo-\nobreak)metrizable space is
(pseudo-)\discretionary{}{}{}metrizable\cmmnt{ (2A3J)}.
A topological space is metrizable iff it is pseudometrizable and
Hausdorff\prooflet{ (2A3L)}.

\spheader 4A2Lb Metrizable spaces are
paracompact\prooflet{ (\Engelking, 5.1.3;  \Csaszar, 8.3.16;  \Cech,
30C.2; \Gaal, p.\ 155)},
%4@26
therefore hereditarily metacompact\cmmnt{ ((a) above and 4A2F(g-i))}.
%4@25 4@38

\spheader 4A2Lc A metrizable space is
perfectly normal\prooflet{ (\Engelking, 4.1.13;  \Csaszar, 8.4.5.)},
so every closed set is a zero set and every open set is a cozero set (in
particular, is F$_{\sigma}$).
%4@23 %4@11, %4@12
\leaveitout{A metrizable space is monotonically normal.    \Prf\ If
$(X,\rho)$ is a metric space, then for disjoint closed sets $E$,
$F\subseteq X$ set $W(E,F)=\{x:\rho(x,E)<\rho(x,F)\}$, writing
$\rho(x,E)=\inf_{y\in E}\rho(x,y)$ (with $\rho(x,\emptyset)=\infty$).
Now $W$ satisfies the conditions of the definition in 4A2A.\ \Qed
}%end of leaveitout

\spheader 4A2Ld If $X$ is a pseudometrizable space, it is
first-countable.
\prooflet{(If $\rho$ is a pseudometric defining the topology of $X$, and
$x$ is any point of $X$, then $\{\{y:\rho(y,x)<2^{-n}\}:n\in\Bbb N\}$ is
a base of neighbourhoods of $x$.)}   So $X$ is sequential and countably
tight\prooflet{ (4A2Ke, 4A2Kc)}, and if $Y$ is another topological space
and $f:X\to Y$ is sequentially continuous, then $f$ is
continuous\prooflet{ (4A2Kd)}.

\spheader 4A2Le {\bf Relative compactness} Let $X$ be a
pseudometrizable space and $A$ a subset of $X$.   Then the following are
equiveridical:  ($\alpha$) $A$ is relatively compact;  ($\beta$) $A$ is
relatively countably compact;  ($\gamma$) every sequence in $A$ has a
subsequence with a limit in $X$.
\prooflet{\Prf\ Fix a pseudometric $\rho$ defining the topology of $X$.
($\alpha$)$\Rightarrow$($\beta$) by 4A2G(f-iii).   If $\sequencen{x_n}$
is a sequence in $A$ with a cluster point $x\in X$, then we can choose
$\sequence{i}{n_i}$ inductively such that $\rho(x_{n_i},x)\le 2^{-i}$
and $n_{i+1}>n_i$ for every $i$;  now $\sequence{i}{x_{n_i}}\to x$;  it
follows that ($\beta$)$\Rightarrow$($\gamma$).   Now assume that
($\alpha$) is false.   Then there is an ultrafilter $\Cal F$ on $X$
containing $A$ which has no limit in $X$ (3A3Be, 3A3De).   If $\Cal F$
is a Cauchy
filter, choose $F_n\in\Cal F$ such that $\rho(x,y)\le 2^{-n}$ whenever
$x$, $y\in F_n$, and $x_n\in A\cap\bigcap_{i\le n}F_i$ for each $n$;
then it is easy to see that $\sequencen{x_n}$ is a sequence in $A$ with
no convergent subsequence.   If $\Cal F$ is not a Cauchy filter, let
$\epsilon>0$ be such that there is no $F\in\Cal F$ such that
$\rho(x,y)\le\epsilon$ for every $x$, $y\in F$.
Then $X\setminus U(x,\bover12\epsilon)\in\Cal F$ for every $x\in X$, so we can choose
$\sequencen{x_n}$ inductively such that
$x_n\in A\setminus\bigcup_{i<n}U(x_i,\bover12\epsilon)$ for every
$n\in\Bbb N$, and again we have a sequence $\sequencen{x_n}$ in $A$ with
no convergent subsequence in $X$.   Thus
not-($\alpha$)$\Rightarrow$ not-($\gamma$) and the proof is complete.\
\Qed}

\spheader 4A2Lf {\bf Compactness} If $X$ is a pseudometrizable space,
it is compact iff it is countably compact iff it is sequentially
compact.
\prooflet{((e) above, using 4A2G(f-i).   Compare \Engelking, 4.1.17, and
\Csaszar, 5.3.33 \& 5.3.47.)}

\spheader 4A2Lg(i) If $(X,\rho)$ is a metric space, its topology has a
base which is $\sigma$-metrically-discrete.
\prooflet{\Prf\ Enumerate $X$ as $\ofamily{\xi}{\kappa}{x_{\xi}}$ where
$\kappa$ is a cardinal.   Let $\sequencen{(q_n,q'_n)}$ be a sequence
running over $\{(q,q'):q,\,q'\in\Bbb Q,\,0<q<q'\}$ in such a way that
$q'_n-q_n\ge 2^{-n}$ for every $n\in\Bbb N$.   For $n\in\Bbb N$,
$\xi<\kappa$ set $G_{n\xi}
=\{x:\rho(x,x_{\xi})<q_n,\,\inf_{\eta<\xi}\rho(x,x_{\eta})>q'_n\}$
(interpreting $\inf\emptyset$ as $\infty$).   Then
$\Cal U=\langle G_{n\xi}\rangle_{\xi<\kappa,n\in\Bbb N}$ is a
$\sigma$-metrically-discrete family of open sets.   If $G\subseteq X$ is
open and $x\in G$,
let $\epsilon>0$ be such that $U(x,2\epsilon)\subseteq G$.   Let
$\xi<\kappa$ be minimal such that $\rho(x,x_{\xi})<\epsilon$, and let
$n\in\Bbb N$ be such that $\rho(x,x_{\xi})<q_n<q'_n<\epsilon$;  then
$x\in G_{n\xi}\subseteq G$.   As $x$ and $G$ are arbitrary, $\Cal U$ is
a base for the topology of $X$.\ \Qed}

\quad (ii) Consequently, any metrizable space has a
$\sigma$-disjoint base.
\cmmnt{(Compare \Engelking, 4.4.3;  \Csaszar, 8.4.5;
\Kuratowski, \S21.XVII.)}
%4@17 %4@25

\spheader 4A2Lh The product of a countable family of metrizable spaces
is metrizable.
\prooflet{(\Engelking, 4.2.2; \Csaszar, 7.3.27.)}

\spheader 4A2Li Let $X$ be a metrizable space and $\kappa\ge\omega$ a
cardinal.   Then $w(X)\le\kappa$ iff $X$ has a dense subset of cardinal
at most $\kappa$.
\prooflet{(\Engelking, 4.1.15.)}

\spheader 4A2Lj If $(X,\rho)$ is any metric space, then the balls
$B(x,\delta)=\{y:\rho(y,x)\le\delta\}$ are all closed
sets\cmmnt{ (cf.\ 1A2G)}.
In particular, in a normed space $(X,\|\,\|)$, the balls
$B(x,\delta)=\{y:\|y-x\|\le\delta\}$ are closed.
%4@46

\leader{4A2M}{Complete metric spaces (a)}
%\spheader 4A2Ma
{\bf Baire's theorem for complete metric spaces} Every
complete metric space is a Baire space.
\prooflet{(\Engelking, 4.3.36 \& 3.9.4;  \Kechris, 8.4;
\Csaszar, 9.2.1 \& 9.2.8;  \Gaal, p.\ 287.)}
So a non-empty complete metric space is not
meager\cmmnt{ (cf.\ 3A3Ha)}.
%4@41

\spheader 4A2Mb Let $\familyiI{(X_i,\rho_i)}$ be a countable family of
complete metric spaces.   Then there is a complete metric on
$X=\prod_{i\in I}X_i$ which defines the product topology on $X$.
\prooflet{(\Engelking, 4.3.12;  \Kuratowski, \S33.III.)}

\spheader 4A2Mc Let $(X,\rho)$ be a complete metric space, and
$E\subseteq X$ a G$_{\delta}$ set.   Then there is a complete metric on
$E$ which defines the subspace topology of $E$.
\prooflet{(\Engelking, 4.3.23;  \Kuratowski, \S33.VI;  \Kechris, 3.11.)}

\spheader 4A2Md Let $(X,\rho)$ be a complete metric space.   Then it is
\v{C}ech-complete.   \prooflet{(\Engelking, 4.3.26.)}

\spheader 4A2Me A non-empty complete metric space without isolated
points is uncountable.  %surely has cardinal \frakc
%4@39S
\prooflet{(If $x$ is not isolated, $\{x\}$ is nowhere dense.)}
%not Kechris not Engelking I think

\leader{4A2N}{Countable networks:  Proposition} (a) If $X$ is a
topological space with a
countable network, any subspace of $X$ has a countable network.

(b) Let $X$ be a space with a countable network.    Then $X$ is
hereditarily Lindel\"of.   If it is regular, it is perfectly normal.
%4@35

(c) If $X$ is a topological space, and $\sequencen{A_n}$ is a sequence
of subsets of $X$ each of which has a countable network (for its
subspace topology), then $A=\bigcup_{n\in\Bbb N}A_n$ has a countable
network.

(d) A continuous image of a space with a countable network has a
countable network.

(e) Let $\familyiI{X_i}$ be a countable family of topological spaces
with countable networks, with product $X$.   Then $X$ has a countable
network.

(f) If $X$ is a Hausdorff space with a countable network, there is a
countable family $\Cal G$ of open sets such that whenever $x$, $y$ are
distinct points in $X$ there are disjoint $G$, $H\in\Cal G$ such that
$x\in G$ and $y\in H$.
%4@23 %4@33

(g) If $X$ is a regular topological space with a countable network, it
has a countable network consisting of closed sets.

(h) A compact Hausdorff space with a countable network
is second-countable.

(i) If a topological space $X$ has a countable network, then any dense
set in $X$ includes a countable dense set;  in particular, $X$ is
separable.

(j) If a topological space $X$ has a countable network, then $C(X)$,
with the topology of pointwise convergence inherited from the product
topology of $\Bbb R^X$, has a countable network.

\proof{{\bf (a)} If $\Cal E$ is a countable network for the topology of
$X$, and $Y\subseteq X$, then $\{Y\cap E:E\in\Cal E\}$ is a countable
network for the topology of $Y$.

\medskip

{\bf (b)} By \Engelking, 3.8.12 $X$ is Lindel\"of.
Since any subspace of $X$ has a countable network ((a) above), it also
is Lindel\"of, and $X$ is hereditarily Lindel\"of.
By 4A2H(c-ii), if $X$ is regular, it is perfectly normal.

\medskip

{\bf (c)} If $\Cal E_n$ is a countable network for the topology of $A_n$
for each $n$, then $\bigcup_{n\in\Bbb N}\Cal E_n$ is a countable network
for the topology of $A$.

\medskip

{\bf (d)} Let $X$ be a topological space with a countable network
$\Cal E$, and $Y$ a continuous image of $X$.   Let $f:X\to Y$ be a
continuous
surjection.   Then $\{f[E]:E\in\Cal E\}$ is a network for the topology
of $Y$.   \Prf\ If $H\subseteq Y$ is open and $y\in H$, then $f^{-1}[H]$
is an open subset of $X$ and there is an $x\in X$ such that $f(x)=y$.
Now there must be an $E\in\Cal E$ such that $x\in E\subseteq f^{-1}[H]$,
so that $y\in f[E]\subseteq H$.\ \QeD\   But $\{f[E]:E\in\Cal E\}$ is
countable, so $Y$ has a countable network.

\medskip

{\bf (e)} For each $i\in I$ let $\Cal E_i$ be a countable network for
the topology of $X_i$.   For each finite $J\subseteq I$, let $\Cal C_J$
be the family of sets expressible as $\prod_{i\in I}E_i$ where
$E_i\in\Cal E_i$ for each $i\in J$ and $E_i=X_i$ for $i\in
I\setminus J$; then $\Cal C_J$ is countable because $\Cal
E_i$ is countable for
each $i\in J$.  Because the family $[I]^{<\omega}$ of finite subsets of
$I$ is countable (3A1Cd), $\Cal E=\bigcup\{\Cal C_J:J\in[I]^{<\omega}\}$
is countable.   But $\Cal E$ is a network for the topology of $X$.
\Prf\ If $G\subseteq X$ is open and $x\in G$, then there is a family
$\familyiI{G_i}$ such that every $G_i\subseteq X_i$ is open,
$J=\{i:G_i\ne X_i\}$ is finite, and $x\in\prod_{i\in I}G_i$.   For $i\in
J$, there is an $E_i\in\Cal E_i$ such that $x(i)\in E_i\subseteq G_i$;
set $E_i=X_i$ for $i\in I\setminus J$.   Then

\Centerline{$E=\prod_{i\in I}E_i\in\Cal C_J\subseteq\Cal E$}

\noindent and $x\in E\subseteq G$.\ \Qed

So $\Cal E$ is a countable network for the topology of $X$.

\medskip

{\bf (f)} By (b) and (e), $X\times X$ is hereditarily Lindel\"of.   In
particular, $W=\{(x,y):x\ne y\}$ is Lindel\"of.   Set

\Centerline{$\Cal V=\{G\times H:G$, $H\subseteq X$ are open, $G\cap
H=\emptyset\}$.}

\noindent Because $X$ is Hausdorff, $\Cal V$ is a cover of $W$.   So
there is a countable $\Cal V_0\subseteq\Cal V$ covering $W$.   Set

\Centerline{$\Cal G=\{G:G\times H\in\Cal V_0\}\cup\{H:G\times H\in\Cal
V_0\}$.}

\noindent Then $\Cal G$ is a countable family of open sets separating
the points of $X$.

\medskip

{\bf (g)} Let $\Cal E$ be a countable network
for the topology of $X$.   Set $\Cal E'=\{\overline{E}:E\in\Cal E\}$.
If $G\subseteq X$ is open and $x\in G$, then (because the topology is
regular) there is an open set $H$ such that
$x\in H\subseteq\overline{H}\subseteq G$.   Now there is an $E\in\Cal E$
such that $x\in E\subseteq H$, in
which case $\overline{E}\in\Cal E'$ and $x\in\overline{E}\subseteq G$.
So $\Cal E'$ is a countable network for $X$ consisting of closed
sets.

\medskip

{\bf (h)} \Engelking, 3.1.19.

\medskip

{\bf (i)} Let $D\subseteq X$ be dense, and $\Cal E$ a countable network
for the topology of $X$.   Let $D'\subseteq D$ be a countable set such
that $D'\cap E\ne\emptyset$ whenever $E\in\Cal E$ and
$D\cap E\ne\emptyset$.   If $G\subseteq X$ is open and not empty, there
is an $x\in D\cap G$;  now there is an $E\in\Cal E$ such that
$x\in E\subseteq G$, and as $x\in D\cap E$ there must be an
$x'\in D'\cap E$, so that $x'\in D'\cap G$.   As $G$ is arbitrary, $D'$
is dense in $X$.

Taking $D=X$, we see that $X$ has a countable dense subset.

\medskip

{\bf (j)} Let $\Cal E$ be a countable network for the topology of $X$
and $\Cal U$ a countable base for the topology of
$\Bbb R$\cmmnt{ (4A2Ua)}.   For
$E\in\Cal E$ and $U\in\Cal U$ set
$H(E,U)=\{f:f\in C(X),\,E\subseteq f^{-1}[U]\}$.   Then the set of
finite intersections of sets of the form $H(E,U)$ is a countable network
for the topology of pointwise convergence on $C(X)$.
(Compare 4A2Oe.)
}%end of proof of 4A2N

\leader{4A2O}{Second-countable spaces (a)} Let $(X,\frak T)$ be a
topological space and $\Cal U$ a countable subbase for $\frak T$.
Then $\frak T$ is second-countable.
\prooflet{
($\{X\}\cup\{U_0\cap U_1\cap\ldots\cap U_n:U_0,\ldots,U_n\in\Cal U\}$ is
countable and is a base for $\frak T$,
by 4A2B(a-i).)}
%4@24

\spheader 4A2Ob Any base of a second-countable space includes a
countable base.
\prooflet{(\Csaszar, 2.4.17.)}

\spheader 4A2Oc A second-countable space has a countable
network\cmmnt{ (because a base is also a network)}, so is separable
and hereditarily Lindel\"of\cmmnt{ (\Engelking, 1.3.8 \& 3.8.1, 4A2Nb,
4A2Ni)}.

\spheader 4A2Od The product of a countable family of second-countable
spaces is second-countable.
\prooflet{(\Engelking, 2.3.14.)}

\spheader 4A2Oe If $X$ is a second-countable space then $C(X)$, with the
topology of uniform convergence on compact sets, has a
countable network.   \prooflet{\Prf\ (See \Engelking, Ex.\ 3.4H.)
Let $\Cal U$ be a countable base for the topology of $X$
and $\Cal V$ a countable base for the topology of
$\Bbb R$\cmmnt{ (4A2Ua)}.   For $U\in\Cal U$, $V\in\Cal V$ set
$H(U,V)=\{f:f\in C(X)$, $U\subseteq f^{-1}[V]\}$.   Then the set of
finite intersections of sets of the form $H(U,V)$ is a countable network
for the topology of uniform convergence on compact subsets of
$X$.\ \Qed}

\leader{4A2P}{Separable metrizable spaces (a)}(i) A metrizable space is
second-countable iff it is separable.
\prooflet{(\Engelking, 4.1.16;  \Csaszar, 2.4.16;  \Gaal\ p.\ 120.)}

\quad(ii) A compact metrizable space is separable\prooflet{ (\Engelking,
4.1.18;
\Csaszar, 5.3.35; \Kuratowski, \S21.IX)}, so is second-countable and has
a countable network.

\quad(iii)
Any base of a separable metrizable space includes a countable
base\cmmnt{ (4A2Ob)},
%4@17 %4@14 %4@27
which is also a countable network,
%4@17
so the space is hereditarily Lindel\"of\cmmnt{ (4A2Nb)}.

\quad(iv)
Any subspace of a separable metrizable space is separable and
metrizable\cmmnt{ (4A2La, 4A2Na, 4A2Ni)}.

\quad(v)
A countable product of separable metrizable spaces is separable and
metrizable\cmmnt{ (4A2B(e-ii), 4A2Lh)}.
%4@15 %4@33 %4@25 %4@24 %4@18

\spheader 4A2Pb A topological space is separable and metrizable iff it
is second-countable, regular and Hausdorff.
\prooflet{(\Engelking, 4.2.9;  \Csaszar, 7.1.57;
\Kuratowski, \S22.II.)}
%4@19

\spheader 4A2Pc A Hausdorff continuous image of a compact metrizable space
is metrizable.
\prooflet{(It is a compact Hausdorff space, by 2A3N(b-ii), with a
countable network, by 4A2Nd, so is metrizable, by 4A2Nh.)}

\spheader 4A2Pd A metrizable space is separable iff it is ccc iff it is
Lindel\"of.
\prooflet{(\Engelking, 4.1.16.)}
%4@18

\spheader 4A2Pe If $X$ is a compact metrizable space, then $C(X)$ is
separable under its usual norm topology defined from the norm
$\|\,\,\|_{\infty}$.
\prooflet{(4A2Oe, or \Engelking, 3.4.16.)}

\vleader{48pt}{4A2Q}{Polish spaces:  Proposition}
(a) A countable discrete space is Polish.

(b) A compact metrizable space is Polish.
%4@35

(c) The product of a countable family of Polish spaces is Polish.
%4@24

(d) A G$_{\delta}$ subset of a Polish space is Polish\cmmnt{ in its
subspace topology};  in particular, a set which is either open or closed
is Polish.
%4@24

(e) The disjoint union of countably many Polish spaces is Polish.
%for (g)

(f) If $X$ is any set and $\sequencen{\frak T_n}$ is a sequence of
Polish topologies on $X$ such that $\frak T_m\cap\frak T_n$ is Hausdorff
for all $m$, $n\in\Bbb N$, then the
topology\cmmnt{ $\frak T_{\infty}$}
generated by $\bigcup_{n\in\Bbb N}\frak T_n$ is Polish.
%4@23 4@A3

(g) If $X$ is a Polish space, it is homeomorphic to a G$_{\delta}$ set
in a compact metrizable space.

(h) If $X$ is a locally compact Hausdorff space, it is Polish iff it has a
countable network iff it is metrizable and $\sigma$-compact.

\proof{{\bf (a)} Any set $X$ is complete under the discrete
metric $\rho$ defined by setting $\rho(x,y)=1$ whenever $x$, $y\in X$
are distinct.   This defines the discrete topology, and if $X$ is
countable it is separable, therefore Polish.

\medskip

{\bf (b)} By 4A2P(a-ii), it is separable;  by 4A2Je, any metric defining
the topology is complete.

\medskip

{\bf (c)} If $\familyiI{X_i}$ is a countable family of Polish spaces
with product $X$, then surely $X$ is separable (4A2B(e-ii));  and 4A2Mb
tells us that its topology is defined by a complete metric.

\medskip

{\bf (d)} If $X$ is Polish and $E$ is a G$_{\delta}$ set in $X$, then
$E$ is separable, by 4A2P(a-iv),
and its topology is defined by a complete metric, by 4A2Mc.   So $E$ is
Polish.   Any open set is of course a G$_{\delta}$ set, and any closed
set is a G$_{\delta}$ set by 4A2Lc.

\medskip

{\bf (e)} Let $\familyiI{X_i}$ be a countable disjoint family of Polish
spaces, and $X=\bigcup_{i\in I}X_i$.   For each $i\in I$ let $\rho_i$ be
a complete metric on $X_i$ defining the topology of $X_i$.   Define
$\rho:X\times X\to\coint{0,\infty}$ by setting
$\rho(x,y)=\min(1,\rho_i(x,y))$ if $i\in I$ and $x$, $y\in X_i$,
$\rho(x,y)=1$ otherwise.   It is easy to check that $\rho$ is a complete
metric on $X$ defining the disjoint union topology on $X$.   $X$ is
separable, by 4A2B(e-i), therefore Polish.

\medskip

{\bf (f)} This result is in \Kechris, 13.3;  but I spell out the proof
because it is an essential element of some measure-theoretic arguments.
On $X^{\Bbb N}$ take the product topology $\frak T$ of the topologies
$\frak T_n$.   This is Polish, by (c).   Consider the diagonal
$\Delta=\{x:x\in X^{\Bbb N}$, $x(m)=x(n)$ for all $m$, $n\in\Bbb N\}$.
This is closed in $X^{\Bbb N}$.   \Prf\ If
$x\in X^{\Bbb N}\setminus\Delta$, let $m$, $n\in\Bbb N$ be such that
$x(m)\ne x(n)$.   Because $\frak T_m\cap\frak T_n$ is Hausdorff, there
are disjoint $G$, $H\in\frak T_m\cap\frak T_n$ such that $x(m)\in G$ and
$x(n)\in H$.   Now $\{y:y\in X^{\Bbb N}$, $y(m)\in G$, $y(n)\in H\}$ is
an open set in $X^{\Bbb N}$ containing $x$ and disjoint from $\Delta$.
As $x$ is arbitrary, $\Delta$ is closed.\ \Qed

By (d), $\Delta$, with its subspace topology, is a Polish space.   Let
$f:X\to\Delta$ be the natural bijection, setting $f(t)=x$ if $x(n)=t$
for every $n$, and let $\frak S$ be the topology on $X$ which makes $f$
a homeomorphism.   The topology on $\Delta$ is generated by
$\{\{x:x\in\Delta$, $x(n)\in G\}:n\in\Bbb N$, $G\in\frak T_n\}$, so
$\frak S$ is generated by
$\{\{t:t\in X$, $t\in G\}:n\in\Bbb N$, $G\in\frak T_n\}
=\bigcup_{n\in\Bbb N}\frak T_n$.   Thus $\frak S=\frak T_{\infty}$ and
$\frak T_{\infty}$ is Polish.

\medskip

{\bf (g)} \Kechris, 4.14.

\medskip

{\bf (h)} If $X$ is Polish, then it is separable,
therefore Lindel\"of (4A2P(a-iii)).   Since the family
$\Cal G$ of relatively compact open subsets of $X$ covers $X$, there is a
countable $\Cal G_0\subseteq\Cal G$ covering $X$, and
$\{\overline{G}:G\in\Cal G_0\}$ witnesses that $X$ is $\sigma$-compact.
Also, of course, $X$ is metrizable.

If $X$ is metrizable and $\sigma$-compact, let $\sequencen{K_n}$ be a
sequence of compact sets covering $X$;  each $K_n$ has a countable network
(4A2P(a-ii)), so $X=\bigcup_{n\in\Bbb N}K_n$ has a countable network
(4A2Nc).

If $X$ has a countable network, let $Z=X\cup\{\infty\}$ be its one-point
compactification (3A3O).   This is compact and Hausdorff and has a
countable network, by 4A2Nc again, so is second-countable (4A2Nh)
and metrizable (4A2Pb) and Polish ((b) above).
So $X$ also, being an open set in $Z$, is Polish ((d) above).
}%end of proof of 4A2Q

\leader{4A2R}{Order topologies} Let $(X,\le)$ be a totally ordered set
and $\frak T$ its order topology.

(a) The set $\Cal U$ of open intervals in $X$\cmmnt{ (definition:
4A2A)} is a base for $\frak T$.

(b) $[x,y]$,
$\coint{x,\infty}$ and $\ocint{-\infty,x}$ are closed sets for all $x$,
$y\in X$.

(c) $\frak T$ is Hausdorff,
%and monotonically normal, therefore
normal and countably paracompact.
%4@39

(d) If $A\subseteq X$ then $\overline{A}$ is the set of
elements of $X$ expressible as either suprema or infima of
non-empty subsets of $A$.

(e) A subset of $X$ is closed iff it is order-closed.

(f) If $\sequencen{x_n}$ is a non-decreasing sequence in $X$
with supremum $x$, then $x=\lim_{n\to\infty}x_n$.
%4@16 %4@A2U

(g) A set $K\subseteq X$ is compact iff $\sup A$ and
$\inf A$ are defined in
$X$ and belong to $K$ for every non-empty $A\subseteq K$.

(h) $X$ is Dedekind complete iff $[x,y]$ is compact for
all $x$, $y\in X$.

(i) $X$ is compact iff it is either empty or Dedekind complete
with greatest and least elements.

(j) Any open set $G\subseteq X$ is
expressible as a union of disjoint open order-convex sets;  if
$X$ is Dedekind complete, these will be open intervals.
%4@16  5@37

(k) If $X$ is well-ordered it is locally compact.

(l) In $X\times X$, $\{(x,y):x<y\}$ is open and $\{(x,y):x\le y\}$ is
closed.
%4@39

(m) If $F\subseteq X$ and {\it either} $F$ is order-convex {\it or} $F$
is compact {\it or} $X$ is Dedekind complete and $F$ is closed, then
the subspace topology on $F$ is induced by the inherited order of $F$.
%4@19 5@33  also if  X  is Ded cpletion of  F

(n) If $X$ is ccc it is hereditarily Lindel\"of, therefore perfectly
normal.

(o) If $Y$ is another totally ordered set with its order topology, an
order-preserving function from $X$ to $Y$ is continuous iff it is
order-continuous.
%4@16

\proof{{\bf (a)} Put the definition of `order topology' (4A2A) together
with 4A2B(a-i).

\medskip

{\bf (b)} Their complements are either $X$, or members of $\Cal U$, or
unions of two members of $\Cal U$.

\medskip

{\bf (c)} Fix a well-ordering $\preccurlyeq$ of $X$.

\medskip

\quad{\bf (i)} If $x<y\in X$,
define $U_{xy}$, $U_{yx}$ as follows:  if $\ooint{x,y}$ is empty,
$U_{xy}=\ocint{-\infty,x}=\ooint{-\infty,y}$ and
$U_{yx}=\coint{y,\infty}=\ooint{x,\infty}$;  otherwise, let $z$ be the
$\preccurlyeq$-least member of $\ooint{x,y}$ and set
$U_{xy}=\ooint{-\infty,z}$, $U_{yx}=\ooint{z,\infty}$.

This construction ensures that if $x$, $y$ are any distinct points of
$X$, $U_{xy}$ and $U_{yx}$ are disjoint open sets containing $x$, $y$
respectively, so $\frak T$ is Hausdorff.

\medskip

\quad{\bf (ii)} Now suppose that $F\subseteq X$ is closed and that
$x\in X\setminus F$.
Then $V_{xF}=\interior(X\cap\bigcap_{y\in F}U_{xy})$ contains $x$.
\Prf\ There are $u$, $v\in X\cup\{-\infty,\infty\}$ such that
$x\in\ooint{u,v}\subseteq X\setminus F$.   If $\ooint{u,x}=\emptyset$,
set $u'=u$;  otherwise, let $u'$ be the $\preccurlyeq$-least member of
$\ooint{u,x}$.   Similarly,
if $\ooint{x,v}=\emptyset$, set $v'=v$;  otherwise, let $v'$ be the
$\preccurlyeq$-least member of $\ooint{x,v}$.   Then $u'<x<v'$.   Now
suppose that
$y\in F$ and $y>x$.   If $\ooint{x,v}=\emptyset$, then
$U_{xy}\supseteq\ocint{-\infty,x}=\ooint{-\infty,v'}$.   Otherwise,
$v'\in\ooint{x,v}\subseteq\ooint{x,y}$, so $U_{xy}=\ooint{-\infty,z}$
where $z$ is the $\preccurlyeq$-least member of $\ooint{x,y}$.   But
this means that $z\preccurlyeq v'$ and either $z=v'$ or
$z\notin\ooint{x,v}$;  in either case, $v'\le z$ and
$\ooint{-\infty,v'}\subseteq U_{xy}$.

Similarly, $\ooint{u',\infty}\subseteq U_{xy}$ whenever $y\in F$ and
$y<x$.   So $x\in\ooint{u',v'}\subseteq V_{xF}$.\ \Qed

\medskip

\leaveitout{
\quad{\bf (iii)} It follows that $X$ is regular.   \Prf\
If $F\subseteq X$ is open and $x\in X\setminus F$, then
$V_{xF}$ is an open set containing $x$.   But $\bigcup_{y\in F}U_{yx}$ is
an open set including $F$ and disjoint from $V_{xF}$.   So $x$ and $F$ can
be separated by open sets.\ \Qed

\medskip

\quad{\bf (iv)} For open $G\subseteq X$ and $x\in G$, set
$W(x,G)=V_{x,X\setminus G}$.   Then $W(x,G)$ is an open set containing $x$
and included in $G$.   If $x\in G\subseteq H$ then

\Centerline{$W(x,G)=\interior(X\cap\bigcap_{y\in X\setminus G}U_{xy})
\subseteq\interior(X\cap\bigcap_{y\in X\setminus H}U_{xy})
=W(x,H)$.}

\noindent If $x$, $y\in X$ are distinct then
$W(x,X\setminus\{y\})\subseteq U_{xy}$ is
disjoint from $W(y,X\setminus\{x\})\subseteq U_{yx}$.   So all the
conditions of 4A2F(?-ii) are satisfied and $X$ is monotonically normal.
By 4A2F?, it is normal and countably paracompact.
}%end of leaveitout

\quad{\bf (iii)} Let $E$ and $F$ be disjoint closed sets.   Set
$G=\bigcup_{x\in E}V_{xF}$, $H=\bigcup_{y\in F}V_{yE}$.   Then $G$ and
$H$ are open sets including $E$, $F$ respectively.   If $x\in E$ and
$y\in F$, then $V_{xF}\cap V_{yE}\subseteq U_{xy}\cap U_{yx}=\emptyset$,
so $G\cap H=\emptyset$.   As $E$ and $F$ are arbitrary, $\frak T$ is
normal.

\medskip

\quad{\bf (iv)} Let $\sequencen{F_n}$ be a non-increasing sequence of
closed sets with empty intersection.   Let $\Cal I$ be the family of
open intervals $I\subseteq X$ such that $I\cap F_n=\emptyset$ for some
$n$.   Because the $F_n$ are closed and have empty intersection,
$\Cal I$ covers $X$.   If $I$, $I'\in\Cal I$ are not
disjoint, $I\cup I'\in\Cal I$;  so we have an equivalence relation
$\sim$ on $X$ defined by saying that $x\sim y$ if there is some
$I\in\Cal I$ containing both $x$ and $y$.   The corresponding
equivalence classes are open and therefore closed, and are order-convex.
Let $\Cal G$ be the set of equivalence classes for $\sim$.

For each $G\in\Cal G$, fix $x_G\in G$.   Set
$G^+=G\cap\coint{x_G,\infty}$.   Then we have a
non-decreasing sequence $\sequencen{G^+_n}$ of closed sets, with union
$G^+$, such that $G^+_n\cap F_n=\emptyset$ for each $n$.
\Prf\ If there is some $m\in\Bbb N$ such that
$G^+\cap F_m=\emptyset$, set $G^+_n=\emptyset$ if
$G^+\cap F_n\ne\emptyset$, $G^+$ if $G^+\cap F_n=\emptyset$.   Otherwise,
given $x\in G^+$ and $n\in\Bbb N$, there is some $m$ such that $[x_G,x]$
does not meet $F_m$, and an $x'\in G^+\cap F_{\max(m,n)}$, so that
$x'\in F_n$ and $x'>x$.   We can therefore choose a strictly increasing
sequence $\sequence{k}{x_k}$ such that $x_0=x_G$ and
$x_{k+1}\in G^+\cap F_k$ for each $k$.
If $x$ is any upper bound of $\{x_k:k\in\Bbb N\}$ then $x\not\sim x_G$,
so $G^+=\bigcup_{k\in\Bbb N}[x_G,x_k]$.   Now, for each $n$, there is a
least $k(n)$ such that $[x_G,x_{k(n)}]\cap F_n\ne\emptyset$;  set
$G^+_n=\emptyset$ if $k(n)=0$, $[x_G,x_{k(n-1)}]$ otherwise.
As $F_{n+1}\subseteq F_n$, $k(n+1)\ge k(n)$
for each $n$.   Since each $[x_G,x_k]$ is disjoint from some
$F_n$, and therefore from all but finitely many $F_n$,
$\lim_{n\to\infty}k(n)=\infty$ and $G^+=\bigcup_{n\in\Bbb N}G^+_n$.\ \Qed

Similarly, $G^-=G\cap\ocint{-\infty,x_G}$ can be expressed as the union
of a non-decreasing sequence $\sequencen{G^-_n}$ of closed sets such that
$G^-_n\cap F_n=\emptyset$ for every $n$.   Now set
$F'_n=\bigcup_{G\in\Cal G}G^+_n\cup G^-_n$ for each $n$.   Because every
$G^+_n$ and $G^-_n$ is closed, and every $G$ is open-and-closed, $F'_n$
is closed.   So $\sequencen{F'_n}$ is a non-decreasing sequence of closed
sets with union $X$, and $F'_n$ is disjoint from $F_n$ for each $n$.
Accordingly $\sequencen{X\setminus F'_n}$ is a non-increasing sequence of
open sets with empty intersection enveloping the $F_n$.   As
$\sequencen{F_n}$ is arbitrary, $\frak T$ is countably paracompact (4A2Ff).
%should check Rudin 84 for the `monotonically normal' case, the
%proof may be more useful
%not sure about that, but if there is any other reason to talk about
%monotonically normal spaces then this would be the right thing to do

\medskip

{\bf (d)} Let $B$ be the set of such suprema and infima.   For
$x\in X$ set $A_x=A\cap\ocint{-\infty,x}$, $A'_x=A\cap\coint{x,\infty}$.
Then $x\in B$ iff either $x=\sup A_x$ or $x=\inf A'_x$, so

$$\eqalign{x\notin B
&\iff x\ne\sup A_x\text{ and }x\ne\inf A'_x\cr
&\iff\text{ there are }u<x,\,v>x\text{ such that }
  A_x\subseteq\ocint{-\infty,u}
  \text{ and }A'_x\subseteq\coint{v,\infty}\cr
&\iff\text{ there are }u,\,v\text{ such that }
  x\in\ooint{u,v}\subseteq X\setminus A\cr
&\iff x\notin\overline{A}.\cr}$$

\noindent Thus $B=\overline{A}$, as claimed.

\medskip

{\bf (e)} Because $X$ is totally ordered, all its subsets are both
upwards-directed and downwards-directed;  so we have only to join
the definition in 313Da to (d) above.

\medskip

{\bf (f)} If $x\in\ooint{u,v}$ then there is some $n\in\Bbb N$ such
that $x_n\ge u$, and now $x_i\in\ooint{u,v}$ for every $i\ge n$.

\medskip

{\bf (g)(i)} If $K$ is compact and $A\subseteq K$ is non-empty,
let $B$ be the set of upper bounds for $A$ in $X\cup\{-\infty,\infty\}$,
and set
$\Cal G=\{\ooint{-\infty,a}:a\in A\}\cup\{\ooint{b,\infty}:b\in B\}$.
Then no finite subfamily of $\Cal G$ can cover $K$;  and if
$c\in K\setminus\bigcup\Cal G$ then $c=\sup A$.   Similarly, any
non-empty subset of $K$ has an infimum in $X$ which belongs to $K$.

\medskip

\quad{\bf (ii)} Now suppose that $K$ satisfies the condition.   By (d)
above, it
is closed.   If it is empty it is certainly compact.   Otherwise,
$a_0=\inf K$ and $b_0=\sup K$ are defined in $X$ and belong to $K$.
Let $\Cal G$ be an open cover of $K$.   Set

\Centerline{$A=\{x:x\in X$,
$K\cap[a_0,x]$ is not covered by any finite
$\Cal G_0\subseteq\Cal G\}$.}

\noindent Note that $A$ is bounded below by $a_0$.   \Quer\ If
$b_0\in A$, then $c=\inf A$ is defined and belongs to $[a_0,b_0]$,
because $X$ is Dedekind complete.   If $c\notin K$ then
there are $u$, $v$ such that $c\in\ooint{u,v}\subseteq X\setminus K$;
if $c\in K$ then there are $u$, $v$ such that
$c\in\ooint{u,v}\subseteq G$ for some $G\in\Cal G$.   In either case,
$u\notin A$, so that
$K\cap\coint{a_0,v}\subseteq(K\cap[a_0,u])\cup\ooint{u,v}$
is covered by a finite subset of $\Cal G$, and $A$ does not meet
$\coint{a_0,v}$, that is, $A\subseteq\coint{v,\infty}$ and $v$ is a
lower bound of $A$.\ \BanG\   Thus $b_0\notin A$, and $K=K\cap[a_0,b_0]$
is covered by a finite subset of $\Cal G$.   As $\Cal G$ is arbitrary,
$K$ is compact.

\medskip

{\bf (h)} Use (g).

\medskip

{\bf (i)} Use (h).

\medskip

{\bf (j)} (Compare 2A2I.)
For $x$, $y\in G$ write $x\sim y$ if either $x\le y$ and
$[x,y]\subseteq G$ or $y\le x$ and $[y,x]\subseteq G$.   It is easy to
check that $\sim$ is an equivalence relation on $G$.   Let $\Cal C$ be
the set of equivalence classes under $\sim$.   Then $\Cal C$ is a
partition of $G$ into order-convex sets.
Now every $C\in\Cal C$ is open.   \Prf\ If $x\in C\in\Cal C$ then
there are $u$, $v\in X\cup\{-\infty,\infty\}$ such that
$x\in\ooint{u,v}\subseteq G$;  now $\ooint{u,v}\subseteq C$.\ \QeD\   So we
have our partition of $G$ into disjoint open order-convex sets.

If $X$ is Dedekind complete, then every member of $\Cal C$ is an open
interval.   \Prf\ Take $C\in\Cal C$.   Set

\Centerline{$A=\{u:u\in X\cup\{-\infty\},\,u<x$ for every $x\in C\}$,}

\Centerline{$B=\{v:v\in X\cup\{\infty\},\,x<v$ for every $x\in C\}$,}

\Centerline{$a=\sup A$,\quad$b=\inf B$;}

\noindent these are defined because $X$ is Dedekind complete.   If
$a<x<b$, there are $y$, $z\in C$ such that $y\le x\le z$, so that
$[y,x]\subseteq[y,z]\subseteq G$ and $y\sim x$ and $x\in C$;  thus
$\ooint{a,b}\subseteq C$.   If $x\in C$, there is an open interval
$\ooint{u,v}$ containing $x$ and included in $G$;  now
$\ooint{u,v}\subseteq C$, so $a\le u<x<v\le b$
and $x\in\ooint{a,b}$.   Thus $C=\ooint{a,b}$ is an open interval.\
\Qed

\medskip

{\bf (k)} Use (h).

\medskip

{\bf (l)} Write $W$ for $\{(x,y):x<y\}$.   If $x<y$, then either there
is a $z$ such that $x<z<y$, in which case
$\ooint{-\infty,z}\times\ooint{z,\infty}$ is an open set containing
$(x,y)$ and included in $W$, or $\ooint{x,y}=\emptyset$,
so $\ooint{-\infty,y}\times\ooint{x,\infty}$ is an open set containing
$(x,y)$ and included in $W$.   Thus $W$ is open.

Now $\{(x,y):x\le y\}=(X\times X)\setminus\{(x,y):y<x\}$ is closed.

\medskip

{\bf (m)} The subspace topology $\frak T_F$ on $F$ is generated by sets
of the form $F\cap\ooint{-\infty,x}$, $F\cap\ooint{x,\infty}$ where
$x\in X$ (4A2B(a-vi)), while the order topology $\frak S$ on $F$ is
generated by sets of the form $F\cap\ooint{-\infty,x}$,
$F\cap\ooint{x,\infty}$ where $x\in F$.   So
$\frak S\subseteq\frak T_F$.

Now suppose that one of the three conditions is satisfied, and that
$x\in X$.   If $x\in F$, or $F\cap\ooint{-\infty,x}$ is either $F$ or
$\emptyset$, then of course
$F\cap\ooint{-\infty,x}\in\frak S$.   Otherwise, $F$ meets both
$\ooint{-\infty,x}$ and $\coint{x,\infty}$ and does not contain $x$, so
is not order-convex.   In this case
$x'=\inf(F\cap\coint{x,\infty})$ is defined and belongs to $F$.   \Prf\
If $F$ is compact, this is covered by (g).   If $X$ is Dedekind complete
and $F$ is closed, then $x'$ is defined, and belongs to $F$ by (e).\
\QeD\   Now $F\cap\ooint{-\infty,x}=F\cap\ooint{-\infty,x'}\in\frak S$.

Similarly, $F\cap\ooint{x,\infty}\in\frak S$ for every $x\in X$.   But
this means that $\frak T_F\subseteq\frak S$, so the two topologies are
equal, as stated.

\medskip

{\bf (n)(i)} Let $\Cal G$ be a family of open subsets of $X$ with
union $H$.   Set

\Centerline{$\Cal A=\{A:A\subseteq\bigcup\Cal G_0$ for some countable
$\Cal G_0\subseteq\Cal G\}$.}

\noindent   (I seek to show that
$H\in\Cal A$.)   Of course $\bigcup\Cal A_0\in\Cal A$ for every
countable subset $\Cal A_0$ of $\Cal A$.

\medskip

\quad{\bf (ii)} Let $\Cal C$ be the family of order-convex members
of $\Cal A$, and $\Cal C^*$ the family
of maximal members of $\Cal C$.   If $C\in\Cal C$ is not included in any
member of $\Cal C^*$, then there is a $C'\in\Cal C$ such that
$C\subseteq C'$ and $\interior(C'\setminus C)\ne\emptyset$.   \Prf\
Since no member of $\Cal C$ including $C$ can be maximal, we can find
$C'\in\Cal C$ such that $C\subseteq C'$ and $\#(C'\setminus C)\ge 5$.
Because $C$ is order-convex, every point of $X\setminus C$ is either a
lower bound or an upper bound of $C$, and there must be three points
$x<y<z$ of $C'\setminus C$ on the same side of $C$.   In this case,

\Centerline{$y\in\ooint{x,z}
\subseteq\interior(C'\setminus C)$,}

\noindent so we have an appropriate $C'$.\ \Qed

\medskip

\quad{\bf (iii)} In fact, every member of $\Cal C$ is included in a
member of
$\Cal C^*$.   \Prf\Quer\ Suppose, if possible, otherwise.   Then we can
choose a strictly increasing family $\ofamily{\xi}{\omega_1}{C_{\xi}}$
in $\Cal C$  inductively, as follows.   Start from any non-empty
$C_0\in\Cal C$ not included in any member of $\Cal C^*$.   Given that
$C_0\subseteq C_{\xi}\in\Cal C$, then $C_{\xi}$ cannot be included in
any member of $\Cal C^*$, so by ($\beta$) above there is a
$C_{\xi+1}\in\Cal C$ such that
$\interior(C_{\xi+1}\setminus C_{\xi})$ is non-empty.   Given
$\ofamily{\eta}{\xi}{C_{\eta}}$ where $\xi<\omega_1$ is a non-zero
countable limit ordinal, set $C_{\xi}=\bigcup_{\eta<\xi}C_{\eta}$;  then
$C_{\xi}$ is order-convex, because $\{C_{\eta}:\eta<\xi\}$ is
upwards-directed, and belongs to $\Cal A$, because $\Cal A$ is closed
under countable unions, so $C_{\xi}\in\Cal C$ and the induction
proceeds.

Now, however,
$\ofamily{\xi}{\omega_1}{\interior(C_{\xi+1}\setminus C_{\xi})}$ is an
uncountable disjoint family of non-empty open sets, and $X$ is not ccc.\
\Bang\Qed

\medskip

\quad{\bf (iv)} Since $C\cup C'$ is order-convex whenever $C$,
$C'\in\Cal C$ and $C\cap C'\ne\emptyset$, $\Cal C^*$ is a disjoint
family.   Moreover, if $x\in H$, there is some open interval containing
$x$ and belonging to $\Cal C$, so $x\in\interior C$ for some
$C\in\Cal C^*$;  this shows that $\Cal C^*$ is an open cover of $H$.
Because $X$ is ccc, $\Cal C^*$ is countable, so
$H=\bigcup\Cal C^*\in\Cal A$.   Thus there is some countable
$\Cal G_0\subseteq\Cal G$ with union $H$;  as $\Cal G$ is arbitrary, $X$
is hereditarily Lindel\"of, by 4A2H(c-i).

\medskip

\quad{\bf (v)} By 4A2H(c-ii), $X$ is perfectly normal.

\medskip

{\bf (o)(i)} Suppose that $f$ is continuous.   If $A\subseteq X$
is a non-empty set with supremum $x$ in $X$, then $x\in\overline{A}$, by
(d), so $f(x)\in\overline{f[A]}$ (3A3Cc) and $f(x)$ is less than or
equal to any upper bound of $f[A]$;  but $f(x)$ is an upper bound of
$f[A]$, because $f$ is order-preserving, so $f(x)=\sup f[A]$.
Similarly, $f(\inf A)=\inf f[A]$ whenever $A$ is non-empty and has an
infimum, so $f$ is order-continuous.

\medskip

\quad{\bf (ii)} Now suppose that $f$ is order-continuous.   Take any
$y\in Y$ and consider $A=f^{-1}[\,\ooint{-\infty,y}\,]$,
$B=X\setminus A$.   If
$x\in A$ then $f(x)$ cannot be $\inf f[B]$ so $x$ cannot be $\inf B$ and
there is an $x'\in X$ such that $x<x'\le z$ for every $z\in B$;  in
which case
$x\in\ooint{-\infty,x'}\subseteq A$.   So $A$ is open.   Similarly,
$f^{-1}[\,\ooint{y,\infty}\,]$ is open.   By 4A2B(a-ii), $f$ is
continuous.
}%end of proof of 4A2R

\vleader{48pt}{4A2S}{Order topologies on ordinals (a)}
Let $\zeta$ be an ordinal with its order topology.

\quad(i) $\zeta$ is locally compact\cmmnt{ (4A2Rk)};  all the sets
$[0,\eta]=\ooint{-\infty,\eta+1}$, for $\eta<\zeta$, are open and
compact\cmmnt{ (4A2Rh)}.
If $\zeta$ is a successor ordinal, it is compact\cmmnt{, being of the
form $[0,\eta]$ where $\zeta=\eta+1$}.
%4@35

\quad(ii) For any $A\subseteq\zeta$,
$\overline{A}=\{\sup B:\emptyset\ne B\subseteq A,\,\sup B<\zeta\}$.
\prooflet{(4A2Rd, because $\inf B\in B\subseteq A$ for every non-empty
$B\subseteq A$.)}

\quad (iii) If $\xi\le\zeta$, then the subspace topology on
$\xi$ induced by the order topology of $\zeta$ is the order topology of
$\xi$.  \prooflet{(4A2Rm.)}

\spheader 4A2Sb Give $\omega_1$ its order topology.

\quad(i) $\omega_1$ is first-countable.   \prooflet{\Prf\ If
$\xi<\omega_1$ is either zero or a successor ordinal, then $\{\xi\}$ is
open so $\{\{\xi\}\}$ is a base of neighbourhoods of $\xi$.   If $\xi$
is a non-zero limit ordinal, there is a sequence $\sequencen{\xi_n}$ in
$\xi$ with supremum $\xi$, and $\{\ocint{\xi_n,\xi}:n\in\Bbb N\}$ is a
base of neighbourhoods of $\xi$.\ \Qed}
%4@25

\quad(ii) Singleton subsets of $\omega_1$ are zero sets.
\prooflet{(Assemble 4A2F(d-v), 4A2Rc and (i) above.)}
%4@19

\quad(iii) If $f:\omega_1\to\Bbb R$ is continuous, there is a
$\xi<\omega_1$ such
that $f(\eta)=f(\xi)$ for every $\eta\ge\xi$.
\prooflet{\Prf\Quer\ Otherwise, we may define a strictly increasing
family $\langle\zeta_{\xi}\rangle_{\xi<\omega_1}$ in $\omega_1$ by
saying that $\zeta_0=0$,

\Centerline{$\zeta_{\xi+1}=\min\{\eta:\eta>\zeta_{\xi},\,f(\eta)\ne
f(\zeta_{\xi})\}$}

\noindent for every $\xi<\omega_1$,

\Centerline{$\zeta_{\xi}=\sup\{\zeta_{\eta}:\eta<\xi\}$}

\noindent for non-zero countable limit ordinals $\xi$.   Now

\Centerline{$\omega_1
=\bigcup_{k\in\Bbb N}\{\xi:|f(\zeta_{\xi+1})-f(\zeta_{\xi})|
\ge 2^{-k}\}$,}

\noindent so there is a $k\in\Bbb N$ such that
$A=\{\xi:|f(\zeta_{\xi+1})-f(\zeta_{\xi})|\ge 2^{-k}\}$ is infinite.
Take a strictly increasing sequence $\langle\xi_n\rangle_{n\in\Bbb N}$
in $A$ and set
$\xi=\sup_{n\in\Bbb N}\xi_n=\sup_{n\in\Bbb N}(\xi_n+1)$.   Then
$\sequencen{\zeta_{\xi_n}}$ and
$\sequencen{\zeta_{\xi_n+1}}$ are strictly increasing sequences with
supremum $\zeta_{\xi}$, so both converge to $\zeta_{\xi}$ in the order
topology of $\omega_1$ (4A2Rf), and

\Centerline{$f(\zeta_{\xi})
=\lim_{n\to\infty}f(\zeta_{\xi_n})
=\lim_{n\to\infty}f(\zeta_{\xi_n+1})$.}

\noindent But this means that

\Centerline{$\lim_{n\to\infty}f(\zeta_{\xi_n})-f(\zeta_{\xi_n+1})=0$,}

\noindent which is impossible, because
$|f(\zeta_{\xi_n})-f(\zeta_{\xi_n+1})|\ge 2^{-k}$ for every
$n$.\ \Bang\Qed}%end of prooflet

\leader{4A2T}{Topologies on spaces of subsets}\cmmnt{ In \S\S446, 476 and
479 it will be useful to be able to discuss topologies on spaces of closed
sets.   In fact everything we really need can be expressed in terms of Fell
topologies ((a-ii) here), but it may help if I put these in the context of
two other constructions, Vietoris topologies and Hausdorff metrics (see (a)
and (g) below), which
may be more familiar to some readers.}    Let $X$ be a topological
space, and $\Cal C=\Cal C_X$ the family of closed subsets of $X$.

\spheader 4A2Ta(i) The {\bf Vietoris topology} on $\Cal C$ is the
topology generated by sets of the forms

\Centerline{$\{F:F\in\Cal C,\,F\cap G\ne\emptyset\}$,
\quad$\{F:F\in\Cal C,\,F\subseteq G\}$}

\noindent where $G\subseteq X$ is open.
%4@41Xr 4@76A 5@26I
%Engelking 2.7.20  Kuratowski I.17 ("exponential topology")

\quad(ii) The {\bf Fell topology} on
$\Cal C$ is the topology generated by sets of the forms

\Centerline{$\{F:F\in\Cal C,\,F\cap G\ne\emptyset\}$,
\quad$\{F:F\in\Cal C,\,F\cap K=\emptyset\}$}

\noindent where $G\subseteq X$ is open and $K\subseteq X$ is compact.
If $X$ is Hausdorff then the Fell topology is coarser than the
Vietoris topology.
If $X$ is compact and Hausdorff the two topologies agree.
%5@26I

\quad(iii) Suppose $X$ is metrizable, and
that $\rho$ is a metric on $X$ inducing its
topology.   For a non-empty subset $A$ of $X$, write
$\rho(x,A)=\inf_{y\in A}\rho(x,y)$ for every $x\in X$.
Note that\cmmnt{ $\rho(x,A)\le\rho(x,y)+\rho(y,A)$ for all $x$,
$y\in X$, so that} $x\mapsto\rho(x,A):X\to\Bbb R$ is $1$-Lipschitz.

For $E$, $F\in\Cal C\setminus\{\emptyset\}$, set

\Centerline{$\tilde\rho(E,F)
=\min(1,\max(\sup_{x\in E}\rho(x,F),\sup_{y\in F}\rho(y,E)))$.}

\noindent\cmmnt{If $E$, $F\in\Cal C\setminus\{\emptyset\}$ and $z\in X$, then
$\rho(z,F)\le\rho(z,E)+\sup_{x\in E}\rho(x,F)$;  from this it
is easy to
see that }$\tilde\rho$ is a metric on $\Cal C\setminus\{\emptyset\}$,
the {\bf Hausdorff metric}.   \cmmnt{Observe that}
$\tilde\rho(\{x\},\{y\})=\min(1,\rho(x,y))$ for all $x$, $y\in X$.
%4@76A

\cmmnt{\medskip

\noindent{\bf Remarks} The formula I give for $\tilde\rho$ has a somewhat
arbitrary feature `$\min(1,\ldots)$'.
Any number strictly greater than $0$ can be used in place of `$1$' here.
Many authors prefer to limit themselves to the family of non-empty closed
sets of finite diameter, rather than the whole of
$\Cal C\setminus\{\emptyset\}$;
this makes it more more natural to omit the truncation, and work with
$(E,F)\mapsto\max(\sup_{x\in E}\rho(x,F),\sup_{y\in F}\rho(y,E))$.
All such variations produce uniformly equivalent metrics.
A more radical approach redefines `metric' to allow functions which take
the value $\infty$;  but this seems a step too far.

Given that I am truncating my Hausdorff metrics by the value $1$,
there would be no extra problems if I defined $\tilde\rho(\emptyset,E)=1$
for every non-empty closed set $E$;
but I think I am following the majority in regarding Hausdorff distance
as defined only for non-empty sets.
}%end of comment

\spheader 4A2Tb(i) The Fell topology is T$_1$.
\prooflet{\Prf\ If $F\subseteq X$ is closed and $x\in X$, then
$\{E:E\in\Cal C$, $E\cap(X\setminus F)=\emptyset\}$ and
$\{E:E\in\Cal C$, $E\cap\{x\}\ne\emptyset\}$ are complements of open sets,
so are closed.   Now if $F\in\Cal C$ then

\Centerline{$\{F\}=\{E:E\subseteq F\}\cap\bigcap_{x\in F}\{E:x\in E\}$}

\noindent is closed.\ \Qed}

\quad(ii) The map $(E,F)\mapsto E\cup F:\Cal C\times\Cal C\to\Cal C$
is continuous for the Fell topology.
%5@26I
\prooflet{\Prf\ If $G\subseteq X$ is open and $K\subseteq X$ is compact,
then

\Centerline{$\{(E,F):(E\cup F)\cap G\ne\emptyset\}
=\{(E,F):E\cap G\ne\emptyset$ or $F\cap G\ne\emptyset\}$,}

\Centerline{$\{(E,F):(E\cup F)\cap K=\emptyset\}
=\{E:E\cap K=\emptyset\}\times\{F:F\cap K=\emptyset\}$}

\noindent are open in the product topology.   So the result follows by
4A2B(a-ii).\ \Qed}

%Similarly, $(E,F)\mapsto E\cup F:\Cal C\times\Cal C\to\Cal C$
%is continuous for the Vietoris topology.}

\quad(iii) $\Cal C$ is compact in the Fell topology.
\prooflet{\Prf\ Let $\frak F$ be an ultrafilter on $\Cal C$.
Set $F_0=\bigcap_{\Cal E\in\frak F}\overline{\bigcup\Cal E}$.
($\alpha$) If $G\subseteq X$ is open and $G\cap F_0\ne\emptyset$, set
$\Cal E=\{F:F\in\Cal C,\,F\cap G=\emptyset\}$.   Then
$\overline{\bigcup\Cal E}$ does not meet $G$, so does not include $F_0$,
and $\Cal E\notin\frak F$.   Accordingly
$\{F:F\cap G\ne\emptyset\}=\Cal C\setminus\Cal E$ belongs to $\frak F$.
($\beta$) If $K\subseteq X$ is compact and $F_0\cap K=\emptyset$, then
$\{K\cap\overline{\bigcup\Cal E}:\Cal E\in\frak F\}$ is a
downwards-directed family of relatively closed subsets of $K$ with empty
intersection so must contain the empty set, and there is an
$\Cal E\in\frak F$ such that $K\cap F=\emptyset$ for every $F\in\Cal E$,
that is, $\{F:F\cap K=\emptyset\}$ belongs to $\frak F$.   ($\gamma$) By
4A2B(a-iv), $\frak F\to F_0$.   As $\frak F$ is arbitrary, $\Cal C$ is
compact.\ \Qed}
%4@76F 4@76G

\spheader 4A2Tc If $X$ is Hausdorff,
$x\mapsto\{x\}$ is continuous for the Fell topology on $\Cal C$.
\prooflet{\Prf\ If $G\subseteq X$ is open, then
$\{x:\{x\}\cap G\ne\emptyset\}=G$ is open.   If $K\subseteq X$ is compact,
then $\{x:\{x\}\cap K=\emptyset\}=X\setminus K$ is open (3A3Dc).\ \Qed}

\spheader 4A2Td If $X$ and another topological space $Y$ are regular,
and $\Cal C_Y$, $\Cal C_{X\times Y}$ are the families of closed subsets
of $Y$ and $X\times Y$ respectively, then
$(E,F)\mapsto E\times F:\Cal C_X\times\Cal C_Y\to\Cal C_{X\times Y}$ is
continuous when each space is given its Fell topology.
%4@46G 4@76G 4@76K
\prooflet{\Prf\
(i) Suppose that $W\subseteq X\times Y$ is open, and consider $\Cal V_W
=\{(E,F):E\in\Cal C_X$, $F\in\Cal C_Y$, $(E\times F)\cap W\ne\emptyset\}$.
If $(E_0,F_0)\in\Cal V_W$, take $(x_0,y_0)\in (E_0\times F_0)\cap W$.   Let
$G\subseteq X$ and $H\subseteq Y$ be open sets such that
$(x_0,y_0)\in G\times H\subseteq W$.   Then $\{(E,F):E\in\Cal C_X$,
$F\in\Cal C_Y$, $E\cap G\ne\emptyset$, $F\cap H\ne\emptyset\}$ is an open
set in $\Cal C_X\times\Cal C_Y$ containing $(E_0,F_0)$ and included in
$\Cal V_W$.   As $(E_0,F_0)$ is arbitrary, $\Cal V_W$ is open in
$\Cal C_X\times\Cal C_Y$.   (ii) Suppose that $K\subseteq X\times Y$ is
compact, and consider $\Cal W_K
=\{(E,F):E\in\Cal C_X$, $F\in\Cal C_Y$, $(E\times F)\cap K=\emptyset\}$.
If $(E_0,F_0)\in\Cal W_K$, then set $K_1=K\cap(E_0\times Y)$ and
$K_2=K\cap(X\times F_0)$.   These are disjoint closed subsets of $K$, so
there are disjoint open subsets $U_1$, $U_2$ of $X\times Y$ including
$K_1$, $K_2$ respectively (4A2F(h-ii)).   Now $K'_1=K\setminus U_1$ and
$K'_2=K\setminus U_2$ are compact subsets of $K$ with union $K$.

Let $\pi_1$, $\pi_2$ be the projections from $X\times Y$
onto $X$, $Y$ respectively;  then
$\{(E,F):E\in\Cal C_X$, $E\cap\pi_1[K'_1]=F\cap\pi_2[K'_2]=\emptyset\}$
is an open set in $\Cal C_X\times\Cal C_Y$ containing $(E_0,F_0)$ and
included in $\Cal W_K$.   As $(E_0,F_0)$ is arbitrary, $\Cal W_K$ is open.
(iii) Putting these together with 4A2B(a-ii), we see that
$(E,F)\mapsto E\times F$ is continuous.\ \Qed}

\spheader 4A2Te Suppose that $X$ is locally compact and Hausdorff.

\quad(i) The
set $\{(E,F):E$, $F\in\Cal C$, $E\subseteq F\}$ is closed in
$\Cal C\times\Cal C$ for the product topology defined from the Fell
topology on $\Cal C$.
\prooflet{\Prf\ Suppose that $E_0$, $F_0\in\Cal C$ and
$E_0\not\subseteq F_0$.   Take $x\in E_0\setminus F_0$.
Because $X$ is locally compact and Hausdorff,
there is a relatively compact open set $G$ such that $x\in G$ and
$\overline{G}\cap F_0=\emptyset$.   Now
$\Cal V=\{E:E\cap G\ne\emptyset\}$ and
$\Cal W=\{F:F\cap\overline{G}=\emptyset\}$ are open sets in $\Cal C$
containing $E_0$, $F_0$ respectively, and $E\not\subseteq F$ for every
$E\in\Cal V$ and $F\in\Cal W$.   This shows that
$\{(E,F):E\not\subseteq F\}$ is open, so that its complement is closed.\
\Qed}
%4@46G
\cmmnt{

It follows that }$\{(x,F):x\in F\}\cmmnt{=\{(x,F):\{x\}\subseteq F\}}$ is
closed in $X\times\Cal C$ when $\Cal C$ is given its
Fell topology\cmmnt{, since $x\mapsto\{x\}$ is continuous,
by (c) above}.
%4@46

\quad(ii) The Fell topology on $\Cal C$ is Hausdorff.
%4@95N 4@95O
\prooflet{\Prf\ The set

\Centerline{$\{(E,E):E\in\Cal C\}
=\{(E,F):E\subseteq F$ and $F\subseteq E\}$}

\noindent is closed in $\Cal C\times\Cal C$, by (i).   So 4A2F(a-iii)
applies.\ \Qed}\cmmnt{

}  It follows that if $\familyiI{F_i}$ is a family in $\Cal C$,
and $\Cal F$
is an ultrafilter on $I$, then we have a well-defined limit
$\lim_{i\to\Cal F}F_i$ defined in $\Cal C$ for the Fell
topology\cmmnt{, because $\Cal C$ is compact ((b-iii) above)}.
%4@46G 4@46O

\quad(iii) If $\Cal L\subseteq\Cal C$ is compact, then $\bigcup\Cal L$ is a
closed subset of $X$.   \prooflet{\Prf\ Take
$x\in X\setminus\bigcup\Cal L$.   For every $C\in\Cal L$, there is a
relatively compact open set $G$ containing $x$ such that
$C\cap\overline{G}$ is empty;  now finitely many such open sets $G$ must
suffice for every $C\in\Cal L$, and the intersection of these $G$ is a
neighbourhood of $x$ not meeting $\bigcup\Cal L$.\ \QeD\   (Compare
4A2Gm.)}


\spheader 4A2Tf Suppose that $X$ is metrizable, locally compact and
separable.   Then the Fell topology on $\Cal C$ is metrizable.
%5@26I
\prooflet{\Prf\ $X$ is second-countable (4A2P(a-i));
let $\Cal U$ be a countable base for the topology of $X$ consisting of
relatively compact open sets (4A2Ob) and closed under finite unions.
Let $\Bbb V$ be the set of open sets in $\Cal C$ expressible in the form

\Centerline{$\{F:F\cap U\ne\emptyset$ for every $U\in\Cal U_0$,
$F\cap\overline{V}=\emptyset\}$}

\noindent where $\Cal U_0\subseteq\Cal U$ is finite and $V\in\Cal U$.
Then $\Bbb V$ is countable.   If $\Cal V\subseteq\Cal C$ is open and
$F_0\in\Cal V$, then there are a finite family $\Cal G$ of open sets in $X$
and a compact $K\subseteq X$ such that

\Centerline{$F_0\in\{F:F\cap G\ne\emptyset$ for every $G\in\Cal G$,
$F\cap K=\emptyset\}\subseteq\Cal V$.}

\noindent For each $G\in\Cal G$ there is a $U_G\in\Cal U$ such that
$U_G\subseteq G$ and $F\cap U_G\ne\emptyset$.   Next, each point of $K$
belongs to a member of $\Cal U$ with closure disjoint from $F_0$, so
(because $K$ is compact and
$\Cal U$ is closed under finite unions) there is a $V\in\Cal U$
such that $K\subseteq V$ and $F_0\cap\overline{V}=\emptyset$.   Now

\Centerline{$\Cal V'=\{F:F\cap U_G\ne\emptyset$ for every $G\in\Cal G$,
$F\cap\overline{V}=\emptyset\}$}

\noindent belongs to $\Bbb V$, contains $F_0$ and is included in $\Cal V$.
This shows that $\Bbb V$ is a base for the Fell topology, and the Fell
topology is second-countable.   Since we already know that it is compact
and Hausdorff, therefore regular, it is metrizable (4A2Pb).\ \Qed}

\spheader 4A2Tg Suppose that $X$ is metrizable, and
that $\rho$ is a metric
inducing the topology of $X$;  let $\tilde\rho$ be the corresponding
Hausdorff metric on $\Cal C\setminus\{\emptyset\}$.

\quad(i) The topology $\frak S_{\tilde\rho}$ defined by
$\tilde\rho$ is finer than the Fell topology $\frak S_F$
on $\Cal C\setminus\{\emptyset\}$.
\prooflet{\Prf\ Let $G\subseteq X$ be open, and consider the set
$\Cal V_G=\{F:F\in\Cal C$, $F\cap G\ne\emptyset\}$.   If $E\in\Cal V_G$,
take $x\in E\cap G$ and $\epsilon>0$ such that $U(x,\epsilon)\subseteq G$;  then
$\{F:\tilde\rho(F,E)<\epsilon\}\subseteq\Cal V_G$.   As $E$ is arbitrary,
$\Cal V_G$ is $\frak S_{\tilde\rho}$-open.   Next, suppose that
$K\subseteq X$ is compact, and consider the set
$\Cal W_K=\{F:F\in\Cal C\setminus\{\emptyset\}$, $F\cap K=\emptyset\}$.
If $E\in\Cal W_K$, the function $x\mapsto\rho(x,E):K\to\ooint{0,\infty}$
is continuous, so has a non-zero lower bound $\epsilon$ say;  now
$\{F:\tilde\rho(F,E)<\epsilon\}\subseteq\Cal W_K$.   As $E$ is arbitrary,
$\Cal W_K$ is $\frak S_{\tilde\rho}$-open.   So $\frak S_{\tilde\rho}$ is
finer than the topology $\frak S_F$ generated by the sets $\Cal V_G$ and
$\Cal W_K$.\ \Qed}

\quad(ii) If $X$ is compact, then $\frak S_{\tilde\rho}$ and
$\frak S_F$ are the same, and both are compact.
\prooflet{\Prf\ Suppose that $E\in\Cal V\in\frak S_{\tilde\rho}$.
Let $\epsilon\in\ooint{0,1}$ be such that
$F\in\Cal V$ whenever $F\in\Cal C\setminus\{\emptyset\}$ and
$\tilde\rho(E,F)<2\epsilon$.
Because $X$ is compact, $E$ is $\rho$-totally bounded (4A2Je) and there is
a finite set $I\subseteq E$ such that
$E\subseteq\bigcup_{x\in I}U(x,\epsilon)$.   Because $x\mapsto\rho(x,E)$ is
continuous, $K=\{x:\rho(x,E)\ge\epsilon\}$ is closed, therefore compact;
now

\Centerline{$\Cal W=\{F:F\in\Cal C$, $F\cap K=\emptyset$,
$F\cap U(x,\epsilon)\ne\emptyset$ for every $x\in I\}$}

\noindent is a neighbourhood of $E$ for $\frak S_F$ included in
$\Cal V$.   Thus $\Cal V$ is a neighbourhood of $E$ for
$\frak S_F$;  as $E$ and $\Cal V$ are arbitrary, $\frak S_F$ is finer than
$\frak S_{\tilde\rho}$.   So the two topologies are equal.

Observe finally that $\{\emptyset\}=\{F:F\in\Cal C$, $F\cap X=\emptyset\}$
is open for the Fell topology on $\Cal C$, so
$\Cal C\setminus\{\emptyset\}$ is closed, therefore compact, by (b-iii).
So $\frak S_{\tilde\rho}=\frak S_F$ is compact.\ \Qed}

\leader{4A2U}{Old friends (a)}
$\Bbb R$, with its usual topology, is metrizable\cmmnt{ (2A3Ff)} and
separable\cmmnt{ (the countable set $\Bbb Q$ is dense)}, so is
second-countable\cmmnt{ (4A2P(a-i))}.
Every subset of $\Bbb R$ is separable\cmmnt{ (4A2P(a-iv))};
\cmmnt{in particular,}
every dense subset of $\Bbb R$ has a countable subset which is still
dense.
%4@18 %4@33 %4@25 %4@39

\spheader 4A2Ub $\BbbN^{\Bbb N}$ is Polish in its usual
topology\cmmnt{ (4A2Qc)},
%4@23
so has a countable network\cmmnt{ (4A2P(a-iii) or 4A2Ne)},
%4@A3
and is hereditarily Lindel\"of\cmmnt{ (4A2Nb or 4A2Pd)}.
%4@22
Moreover, it is homeomorphic to
$[0,1]\setminus\Bbb Q$\prooflet{ (\Kuratowski, \S36.II;  \Kechris, 7.7)}.
%4@39

\spheader 4A2Uc The map
$x\mapsto\bover23\sum_{j=0}^{\infty}3^{-j}x(j)$\cmmnt{ (cf.\ 134Gb)}
is a homeomorphism between $\{0,1\}^{\Bbb N}$ and the Cantor set
$C\subseteq[0,1]$.   \prooflet{(It is a continuous bijection.)}
%4@39

\spheader 4A2Ud If $I$ is any set, then the map
$A\mapsto\chi A:\Cal PI\to\{0,1\}^I$ is a homeomorphism\cmmnt{ (for
the usual topologies on $\Cal PI$ and $\{0,1\}^I$, as described in 4A2A
and 3A3K)}.   So $\Cal PI$ is zero-dimensional, compact\cmmnt{ (3A3K)}
and Hausdorff.   If $I$ is countable, then $\Cal PI$ is metrizable,
therefore Polish\cmmnt{ (4A2Qb)}.

\spheader 4A2Ue Give the space $C(\coint{0,\infty})$ the topology
$\frak T_c$ of uniform convergence on compact sets.

\medskip

\quad{\bf (i)} $C(\coint{0,\infty})$ is a Polish locally convex
linear topological space.   \prooflet{\Prf\ $\frak T_c$ is determined by
the seminorms $f\mapsto\sup_{t\le n}|f(t)|$ for $n\in\Bbb N$, so it is a
metrizable linear space topology.   By 4A2Oe, it has a countable network,
so is separable.   Any function which is continuous on
every set $[0,n]$ is continuous on $\coint{0,\infty}$, so
$C(\coint{0,\infty})$ is complete under the metric
$(f,g)\mapsto\sum_{n=0}^{\infty}\min(2^{-n},\sup_{t\le n}|f(t)-g(t)|)$;  as
this metric defines $\frak T_c$, $\frak T_c$ is Polish.\ \Qed}

\medskip

\quad{\bf (ii)} Suppose that $A\subseteq C(\coint{0,\infty})$ is such that
$\{f(0):f\in A\}$ is bounded and for every $a\ge 0$
and $\epsilon>0$ there is
a $\delta>0$ such that $|f(s)-f(t)|\le\epsilon$ whenever
$f\in A$, $s$, $t\in[0,a]$
and $|s-t|\le\delta$.   Then $A$ is relatively compact for $\frak T_c$.
\prooflet{
\Prf\ Note first that $\{f(a):f\in A\}$ is bounded for every $a\ge 0$,
since if $\delta>0$ is such that $|f(s)-f(t)|\le 1$ whenever $f\in A$,
$s$, $t\in[0,a]$ and $|s-t|\le\delta$, then
$|f(a)|\le|f(0)|+\lceil\Bover{a}{\delta}\rceil$ for every $f\in A$.   So if
$\Cal F$ is an ultrafilter on $C(\coint{0,\infty})$ containing $A$,
$g(a)=\lim_{f\to\Cal F}f(a)$ is defined for every $a\ge 0$.   If $a\ge 0$
and $\epsilon>0$, let $\delta\in\ocint{0,1}$ be such that
$|f(s)-f(t)|\le\epsilon$ whenever $f\in A$,
$s$, $t\in[0,a+1]$ and $|s-t|\le\delta$;
then $|g(s)-g(a)|\le\epsilon$ whenever $|s-a|\le\delta$;  as $a$ and
$\epsilon$ are arbitrary, $g\in C(\coint{0,\infty})$.   If $a\ge 0$ and
$\epsilon>0$, let $\delta>0$ be such that
$|f(s)-f(t)|\le\bover13\epsilon$ whenever $f\in A$, $s$, $t\in[0,a]$
and $|s-t|\le\delta$.   Then

\Centerline{$A'
=\{f:f\in A$, $|f(i\delta)-g(i\delta)|\le\Bover13\epsilon$ for every
$i\le\lceil\Bover{a}{\delta}\rceil\}$}

\noindent belongs to $\Cal F$.   Now $|f(t)-g(t)|\le\epsilon$ for every
$f\in A'$ and $t\in[0,a]$.   As $a$ and $\epsilon$ are arbitrary,
$\Cal F\to g$ for $\frak T_c$.   As $\Cal F$ is arbitrary, $A$ is
relatively compact (3A3De).\ \Qed}

\discrpage

