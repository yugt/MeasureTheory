\frfilename{mt365.tex}
\versiondate{26.5.03}
\copyrightdate{1996}
     
\def\chaptername{Function spaces}
\def\sectionname{$L^1$}
     
\newsection{365}
     
Continuing my programme of developing the ideas of Chapter 24 at a
deeper level of abstraction, I arrive at last at $L^1$.   As usual, the
first step is to establish a definition which can be matched both with
the constructions of the previous sections and with the definition of
$L^1(\mu)$ in \S242 (365A-365C, 365F).   Next, I give what I regard as
the most characteristic internal properties of $L^1$ spaces, including
versions of the Radon-Nikod\'ym theorem (365E),
before turning to abstract versions of theorems in \S235 (365H, 365T)
and the duality between $L^1$ and $L^{\infty}$
(365L-365N).   As in \S\S361 and 363, the construction is associated
with universal mapping theorems (365I-365K) which define the Banach
lattice structure of $L^1$.   As in \S\S361, 363 and 364, homomorphisms
between measure algebras correspond to operators between their $L^1$
spaces;  but now the duality theory gives us two types of operators
(365O-365Q), of which one class can be thought of as abstract
conditional expectations (365R).   For localizable measure algebras, the
underlying algebra can be recovered from its $L^1$ space (365S), but the
measure cannot.
     
\leader{365A}{Definition} Let $(\frak A,\bar\mu)$ be a measure algebra.
For $u\in L^0(\frak A)$, write
     
\Centerline{$\|u\|_1
=\int_0^{\infty}\bar\mu\Bvalue{|u|>\alpha}\,d\alpha$,}
     
\noindent the integral being with respect to Lebesgue measure on
$\Bbb R$, and allowing $\infty$ as a value of the integral.
\cmmnt{(Because
the integrand is monotonic, it is certainly measurable.)}   Set
$L^1_{\bar\mu}=L^1(\frak A,\bar\mu)
=\{u:u\in L^0(\frak A),\,\|u\|_1<\infty\}$.
     
\cmmnt{It is convenient to note at once that if
$u\in L^1_{\bar\mu}$, then $\mu\Bvalue{|u|>\alpha}$ must be
finite for almost
every $\alpha>0$, and therefore for every $\alpha>0$, since it is a
non-increasing function of $\alpha$;  so that $\Bvalue{u>\alpha}$ also
belongs to the Boolean ring $\frak A^f=\{a:\bar\mu a<\infty\}$ for every
$\alpha>0$.
}
     
\leader{365B}{Theorem} Let $(X,\Sigma,\mu)$ be a measure space with
measure algebra $(\frak A,\bar\mu)$.   Then the canonical isomorphism
between $L^0(\mu)$ and $L^0(\frak A)$\cmmnt{ (364Ic)}
matches $L^1(\mu)\subseteq L^0(\mu)$\cmmnt{, defined in \S242,} with
$L^1(\frak A,\bar\mu)\subseteq L^0(\frak A)$, and the standard norm of
$L^1(\mu)$ with
$\|\,\|_1:L^1(\frak A,\bar\mu)\to\coint{0,\infty}$\cmmnt{, as
defined in 365A}.
     
\proof{ Take any $\Sigma$-measurable function $f:X\to\Bbb R$ (364B);
write $f^{\ssbullet}$
for its equivalence class in $L^0(\mu)$, and $u$ for the corresponding
element of $L^0(\frak A)$.   Then
$\Bvalue{|u|>\alpha}=\{x:|f(x)|>\alpha\}^{\ssbullet}$ in $\frak A$ for
every $\alpha\in\Bbb R$, and
     
\Centerline{$\|u\|_1
=\int_0^{\infty}\mu\{x:|f(x)|>\alpha\}\,d\alpha=\int|f|d\mu$}
     
\noindent by 252O.   In particular, $u\in L^1(\frak A,\bar\mu)$ iff
$f\in\eusm L^1(\mu)$ iff $f^{\ssbullet}\in L^1(\mu)$, and in this case
$\|u\|_1=\|f^{\ssbullet}\|_1$.
}%end of proof of 365B
     
\leader{365C}{}\cmmnt{ Accordingly we can apply everything we know
about $L^1(\mu)$ spaces to $L^1_{\bar\mu}$ spaces.   For instance:
     
\medskip
     
\noindent}{\bf Theorem} For any measure algebra $(\frak A,\bar\mu)$,
$L^1(\frak A,\bar\mu)$ is a solid linear subspace of
$L^0(\frak A)$, and
$\|\,\|_1$ is a norm on $L^1(\frak A,\bar\mu)$ under which $L^1(\frak
A,\bar\mu)$ is
an $L$-space.   Consequently $L^1(\frak A,\bar\mu)$ is a perfect Riesz
space with an order-continuous norm
which has the Levi property, and if $\sequencen{u_n}$ is a
non-decreasing norm-bounded sequence in $L^1(\frak A,\bar\mu)$ then it
converges for $\|\,\|_1$ to $\sup_{n\in\Bbb N}u_n$.
     
\proof{ $(\frak A,\bar\mu)$ is isomorphic to the measure algebra of some
measure space $(X,\Sigma,\mu)$ (321J).   $L^1(\mu)$ is a solid linear
subspace of $L^0(\mu)$ (242Cb), so $L^1_{\bar\mu}$ is a solid
linear subspace of $L^0(\frak A)$.   $L^1(\mu)$ is an $L$-space
(354M), so $L^1_{\bar\mu}$ also is.   The rest of the properties
claimed are general features of $L$-spaces (354N, 354E, 356P).
}%end of proof of 365C
     
\leader{365D}{Integration} Let $(\frak A,\bar\mu)$ be any measure
algebra.
     
\spheader 365Da If $u\in L^1=L^1(\frak A,\bar\mu)$, then $u^+$ and $u^-$
belong to $L^1$, and we may set
     
\Centerline{$\int u
=\|u^+\|_1-\|u^-\|_1
=\int_0^{\infty}\bar\mu\Bvalue{u>\alpha}\,d\alpha
   -\int_0^{\infty}\bar\mu\Bvalue{-u>\alpha}\,d\alpha$.}
     
\noindent Now $\int:L^1\to\Bbb R$ is an order-continuous positive linear
functional\cmmnt{ (356Pc), and under the translation of 365B matches
the integral on $L^1(\mu)$ as defined in 242Ab}.   \cmmnt{Note that if
$a\in\frak A^f$ then
     
\Centerline{$\int\chi a
=\int_0^{\infty}\bar\mu\Bvalue{\chi a>\alpha}d\alpha
=\int_0^1\bar\mu a\,d\alpha=\bar\mu a$,}
     
\noindent so that if $\bar\mu$ is totally finite then the integral here
agrees with that of 363L on $L^{\infty}(\frak A)$.
I will sometimes write $\int u\,d\bar\mu$ if it seems helpful to
indicate the measure.}
     
\spheader 365Db\cmmnt{ Of course} $\|u\|_1=\int|u|\ge|\int u|$ for
every $u\in L^1$.
     
\spheader 365Dc If $u\in L^1$, $a\in\frak A$ we may set
$\int_au=\int u\times\chi a$.   \cmmnt{(Compare 242Ac.)}   If
$\gamma>0$ and $0\ne a\Bsubseteq\Bvalue{u>\gamma}$ then\cmmnt{ there
is a $\delta>\gamma$ such that $a'=a\Bcap\Bvalue{u>\delta}\ne 0$, so
that}
     
\Centerline{$\int_au
\cmmnt{\mskip5mu
=\intop_0^{\infty}\bar\mu(a\Bcap\Bvalue{u>\alpha})d\alpha
\ge\intop_0^{\gamma}\bar\mu a\,d\alpha+\intop_{\gamma}^{\delta}\bar\mu a'}
>\gamma\bar\mu a$.}
     
\noindent In particular,\cmmnt{ setting $a=\Bvalue{u>\gamma}$,}
$\bar\mu\Bvalue{u>\gamma}$ must be finite.
     
\spheader 365Dd{\bf (i)} If $u\in L^1$ then $u\ge 0$
iff $\int_au\ge 0$ for every $a\in\frak A^f$\cmmnt{, writing
$\frak A^f=\{a:\bar\mu a<\infty\}$, as usual}.   \prooflet{\Prf\ If
$u\ge 0$
then $u\times\chi a\ge 0$, $\int_au\ge 0$ for every $a\in\frak A$.   If
$u\not\ge 0$, then $\Bvalue{u^->0}\ne 0$ and there is an $\alpha>0$ such
that $a=\Bvalue{u^->\alpha}\ne 0$.   But now
$\bar\mu a$ is finite ((c) above) and
     
\Centerline{$\int u\times\chi a=-\int u^-\times\chi a
=-\int\bar\mu(a\Bcap\Bvalue{u^-\ge\beta})d\beta
\le-\alpha\bar\mu a<0$,}
     
\noindent so $\int_au<0$.\ \QeD}
     
\medskip
     
\quad{\bf (ii)} If $u$, $v\in L^1$ and $\int_au=\int_av$ for every
$a\in\frak A^f$ then $u=v$\cmmnt{ (cf.\ 242Ce)}.
     
\medskip
     
\quad{\bf (iii)} If $u\ge 0$ in $L^1$ then
$\int u=\sup\{\int_au:a\in\frak A^f\}$.   \prooflet{\Prf\ Of course
$u\times\chi a\le u$ so $\int_au\le u$ for every $a\in\frak A$.   On the
other hand,
setting $a_n=\Bvalue{u>2^{-n}}$, $\sequencen{u\times\chi a_n}$ is a
non-decreasing sequence with supremum $u$, so
$\int u=\lim_{n\to\infty}\int_{a_n}u$, while $\bar\mu a_n$ is finite for
every $n$.\ \Qed}
     
\spheader 365De If $u\in L^1$, $u\ge 0$ and $\int u=0$ then
$u=0$\prooflet{ (put 365B and 122Rc together)}.   If
$u\in L^1$, $u\ge 0$ and $\int_au=0$ then $u\times\chi a=0$, that is,
$a\Bcap\Bvalue{u>0}=0$.
     
\spheader 365Df If $C\subseteq L^1$ is non-empty and
upwards-directed and $\sup_{v\in C}\int v$ is finite, then $\sup C$ is
defined in $L^1$ and
$\int\sup C=\sup_{v\in C}\int v$\prooflet{ (356Pc)}.
     
\spheader 365Dg\cmmnt{ It will occasionally be convenient to adapt the
conventions of \S133 to the new context;  so that} I may write
$\int u=\infty$ if $u\in L^0(\frak A)$, $u^-\in L^1$ and
$u^+\notin L^1$, while $\int u=-\infty$ if $u^+\in L^1$ and
$u^-\notin L^1$.
     
\spheader 365Dh On this convention,\cmmnt{ we can restate (f) as
follows:}  if
$C\subseteq(L^0)^+$ is non-empty and upwards-directed and has a supremum
$u$ in $L^0$, then $\int u=\sup_{v\in C}\int v$ in $[0,\infty]$.
\prooflet{\Prf\ For if $\sup_{v\in C}\int v$ is infinite, then surely
$\int u=\infty$;  while otherwise we can apply (f).\ \Qed}
 
\leader{365E}{The Radon-Nikod\'ym theorem again} (a) Let
$(\frak A,\bar\mu)$ be a semi-finite measure algebra and
$\nu:\frak A\to\Bbb R$
an additive functional.   Then the following are equiveridical:
     
\inset{(i) there is a $u\in L^1=L^1(\frak A,\bar\mu)$ such that
$\nu a=\int_au$ for every $a\in\frak A$;
     
(ii) $\nu$ is additive and continuous for the measure-algebra
topology on $\frak A$;
     
(iii) $\nu$ is completely additive.}
     
(b) Let $(\frak A,\bar\mu)$ be any measure algebra, and
$\nu:\frak A^f\to\Bbb R$ a function.   Then the following are
equiveridical:
     
\inset{(i) $\nu$ is additive and bounded and $\inf_{a\in A}|\nu a|=0$
whenever $A\subseteq\frak A^f$ is downwards-directed and has infimum
$0$;
     
(ii) there is a $u\in L^1$ such that
$\nu a=\int_au$ for every $a\in\frak A^f$.}
     
\proof{{\bf (a)} The equivalence of (ii) and (iii) is 327Bd.
The equivalence of (i) and (iii) is just a translation of 327D into
the new context.
     
\medskip
     
{\bf (b)(i)$\Rightarrow$(ii)}\grheada\ Set
$M=\sup_{a\in\frak A^f}|\nu a|<\infty$.
     
Let $D\subseteq\frak A^f$ be a maximal disjoint set.   For each $d\in
D$, write $\frak A_d$ for the principal ideal of $\frak A$ generated by
$d$, and $\bar\mu_d$ for the restriction of $\bar\mu$ to $\frak A_d$, so
that $(\frak A_d,\bar\mu_d)$ is a totally finite measure algebra.   Set
$\nu_d=\nu\restrp\frak A_d$;  then $\nu_d:\frak A_d\to\Bbb R$ is
completely additive.   By (a), there is a
$u_d\in L^1(\frak A_d,\bar\mu_d)$
such that $\int_au_d=\nu_da=\nu a$ for every $a\Bsubseteq d$.
     
Now $u_d^+\in L^0(\frak A_d)$ corresponds to a member $\tilde u_d^+$ of
$L^0(\frak A)^+$ defined by saying
     
$$\eqalign{\Bvalue{\tilde u_d^+>\alpha}&=\Bvalue{u_d^+>\alpha}
=\Bvalue{u_d>\alpha}\text{ if }\alpha\ge 0,\cr
&=1\text{ if }\alpha<0.\cr}$$
     
\noindent If $a\in\frak A$, then
     
\Centerline{$\int_a\tilde u_d^+d\bar\mu
=\int_0^{\infty}\bar\mu(a\Bcap\Bvalue{\tilde u_d^+>\alpha})d\alpha
=\int_0^{\infty}\bar\mu_d(a\Bcap\Bvalue{u_d^+>\alpha})d\alpha
=\int_{a\Bcap d}u_d^+d\bar\mu_d$;}
     
\noindent taking $a=1$, we see that
$\|\tilde u_d^+\|_1=\|u_d^+\|_1=\nu\Bvalue{u_d>0}$ is
finite, so that $\tilde u_d^+\in L^1$.
     
\medskip
     
\qquad\grheadb\ For any finite $I\subseteq D$, set
$v_I=\sum_{d\in I}\tilde u_d^+$.   Then
     
\Centerline{$\int v_I=\nu(\sup_{d\in I}\Bvalue{u_d>0})\le M$;}
     
\noindent consequently the upwards-directed set $A=\{v_I:I\subseteq D$
is finite$\}$ is bounded above in $L^1$, and we can set
$v=\sup A$ in $L^1$.
If $a\in\frak A$, then
$\int_av_I=\sum_{d\in I}\int_{a\Bcap d}u_d^+$ for each finite
$I\subseteq D$, so $\int_av=\sum_{d\in D}\int_{a\Bcap d}u_d^+$.
     
Applying the same arguments to $-\nu$, there is a $w\in L^1$
such that
     
\Centerline{$\int_aw=\sum_{d\in D}\int_{a\Bcap d}u_d^-$}
     
\noindent for every $a\in\frak A$.   Try $u=v-w$;  then
     
\Centerline{$\int_au
=\sum_{d\in D}\int_{a\Bcap d}u_d^+-\int_{a\Bcap d}u_d^- %
=\sum_{d\in D}\int_{a\Bcap d}u_d
=\sum_{d\in D}\nu(a\Bcap d)$}
     
\noindent for every $a\in\frak A$.
     
\medskip
     
\qquad\grheadc\ Now take any $a\in\frak A^f$.   For
$J\subseteq D$ set $a_J=\sup_{d\in J}a\Bcap d$.   Let $\epsilon>0$.
Then there is a finite $I\subseteq D$ such that
     
\Centerline{$|\int_au-\nu a_J|
=|\sum_{d\in D}\nu(a\Bcap d)-\sum_{d\in J}\nu(a\Bcap d)|
\le\epsilon$}
     
\noindent whenever $I\subseteq J\subseteq D$ and $J$ is finite.   But
now consider
     
\Centerline{$A=\{a\Bsetminus a_J:I\subseteq J\subseteq D,\,J$ is
finite$\}$.}
     
\noindent Then $\inf A=0$, so there is a finite $J$ such that
$I\subseteq J\subseteq D$ and
     
\Centerline{$|\nu a-\nu a_J|=|\nu(a\Bsetminus a_J)|\le\epsilon$.}
     
\noindent Consequently
     
\Centerline{$|\nu a-\int_au|
\le|\nu a-\nu a_J|+|\int_au-\nu a_J|\le 2\epsilon$.}
     
\noindent As $\epsilon$ is arbitrary, $\nu a=\int_au$.   As $a$ is
arbitrary, (ii) is proved.
     
\medskip
     
\quad{\bf (ii)$\Rightarrow$(i)} From where we now are, this is nearly
trivial.   Thinking of $\nu a$ as $\int u\times\chi a$, $\nu$ is surely
additive and bounded.   Also $|\nu a|\le\int|u|\times\chi a$.   If
$A\subseteq\frak A^f$ is non-empty, downwards-directed and has infimum
$0$, the same is true of $\{|u|\times\chi a:a\in A\}$, because $a\mapsto
|u|\times\chi a$ is order-continuous, so
     
\Centerline{$\inf_{a\in A}|\nu a|\le\inf_{a\in A}\int|u|\times\chi
a=\inf_{a\in A}\||u|\times\chi a\|_1=0$.}
     
}%end of proof of 365E
     
\leader{365F}{}\cmmnt{ It will be useful later to have spelt out the
following elementary facts.
     
\medskip
     
\noindent}{\bf Lemma} Let $(\frak A,\bar\mu)$ be a measure algebra.
Write $S^f$ for the intersection $S(\frak A)\cap L^1(\frak A,\bar\mu)$.
Then $S^f$ is a norm-dense and order-dense Riesz subspace of
$L^1(\frak A,\bar\mu)$, and can be identified with $S(\frak A^f)$.
The function $\chi:\frak A^f\to S^f\subseteq L^1(\frak A,\bar\mu)$ is an
injective
order-continuous additive lattice homomorphism.   If $u\ge 0$ in
$L^1(\frak A,\bar\mu)$,
there is a non-decreasing sequence $\sequencen{u_n}$ in $(S^f)^+$ such
that $u=\sup_{n\in\Bbb N}u_n=\lim_{n\to\infty}u_n$.
     
\proof{ As in 364K, we can think of $S(\frak A^f)$ as a Riesz
subspace of $S=S(\frak A)$, embedded in $L^0(\frak A)$.
If $u\in S$, it is
expressible as $\sum_{i=0}^n\alpha_i\chi a_i$ where
$a_0,\ldots,a_n\in\frak A$ are disjoint and no $\alpha_i$ is zero.
Now $|u|=\sum_{i=0}^n|\alpha_i|\chi a_i$, so $u\in L^1$ iff $\bar\mu
a_i<\infty$ for every $i$, that is, iff $u\in S(\frak A^f)$;  thus
$S^f=S(\frak A^f)$.
     
Now suppose that $u\ge 0$ in $L^1$.   By 364Jd, there is a
non-decreasing sequence $\sequencen{u_n}$ in $S(\frak A)^+$ such that
$u_0\ge 0$ and $u=\sup_{n\in\Bbb N}u_n$ in $L^0$.   Because $L^1$ is a
solid linear subspace of $L^0$, every $u_n$ belongs to $L^1$ and
therefore to $S^f$.   By 365C, $\sequencen{u_n}$ is norm-convergent to
$u$.   This shows also that $S^f$ is order-dense in $L^1$.
     
The map $\chi:\frak A^f\to S^f$ is an injective order-continuous
additive lattice homomorphism;  because $S^f$ is regularly embedded in
$L^1$ (352Ne), $\chi$ has the same properties when regarded as a map
into $L^1$.
     
For general $u\in L^1$, there are sequences in $S^f$ converging to $u^+$
and to $u^-$, so that their difference is a sequence in $S^f$ converging
to $u$, and $u$ belongs to the closure of $S^f$;  thus $S^f$ is
norm-dense in $L^1$.
}%end of proof of 365F
     
\cmmnt{\medskip
     
\noindent{\bf Remark} Of course $S^f$ here corresponds to the space of
(equivalence classes of) simple functions, as in 242Mb.
}
     
\leader{365G}{Semi-finite algebras:  Lemma} Let $(\frak A,\bar\mu)$ be a
measure algebra.
     
(a) $(\frak A,\bar\mu)$ is semi-finite iff $L^1=L^1(\frak A,\bar\mu)$ is
order-dense in $L^0=L^0(\frak A)$.
     
(b) In this case, writing
$S^f=S(\frak A)\cap L^1$\cmmnt{ (as in 365F)},
$\int u=\sup\{\int v:v\in S^f,\,0\le v\le u\}$ in $[0,\infty]$ for every
$u\in(L^0)^+$.
     
     
\proof{{\bf (a)} If $(\frak A,\bar\mu)$ is semi-finite then $S^f$ is
order-dense in $L^0$ (364K), so $L^1$ must also be.   If $L^1$ is
order-dense in $L^0$, then so is $S^f$, by 365F and 352Nc, so
($\frak A,\bar\mu)$ is semi-finite, by 364K in the other direction.
     
\medskip
     
{\bf (b)} Set $C=\{v:v\in S^f,\,0\le v\le u\}$.   Then $C$ is an
upwards-directed set with supremum $u$, because $S^f$ is order-dense in
$L^0$.   So $\int u=\sup_{v\in C}\int v$ by 365Dh.
}%end of proof of 365G
     
\leader{365H}{Measurable \dvrocolon{transformations}}\cmmnt{ We have a
generalization of the ideas of \S235 in this abstract context.
     
\medskip
     
\noindent}{\bf Theorem} Let $(\frak A,\bar\mu)$ and $(\frak B,\bar\nu)$
be measure algebras, and $\pi:\frak A\to\frak B$ a sequentially
order-continuous Boolean homomorphism.   Let
$T:L^0(\frak A)\to L^0(\frak B)$ be the sequentially order-continuous
Riesz homomorphism associated with $\pi$\cmmnt{ (364P)}.
     
(a) Suppose that $w\ge 0$ in $L^0(\frak B)$ is such that
$\int_{\pi a}w\,d\bar\nu=\bar\mu a$ whenever $a\in\frak A$ and
$\bar\mu a<\infty$.
Then for any $u\in L^1(\frak A,\bar\mu)$ and $a\in\frak A$,
$\int_{\pi a}Tu\times w\,d\bar\nu$ is defined and equal to
$\int_au\,d\bar\mu$.
     
(b) Suppose that $w'\ge 0$ in $L^0(\frak A)$ is such that
$\int_aw'd\bar\mu=\bar\nu(\pi a)$ for every $a\in\frak A$.   Then
$\int Tu\,d\bar\nu=\int u\times w'\,d\bar\mu$ whenever
$u\in L^0(\frak A)$ and
either integral is defined in $[-\infty,\infty]$.
     
\cmmnt{\medskip
     
\noindent{\bf Remark} I am using the convention of 365Dg concerning
`$\infty$' as the value of an integral.}
     
\proof{{\bf (a)} If $u\in S^f=L^1_{\bar\mu}\cap S(\frak A)$ then
$u$ is expressible as $\sum_{i=0}^n\alpha_i\chi a_i$ where
$a_0,\ldots,a_n$ have finite measure, so that
$Tu=\sum_{i=0}^n\alpha_i\chi(\pi a_i)$ and
     
\Centerline{$\int Tu\times w\,d\bar\nu
=\sum_{i=0}^n\alpha_i\int_{\pi a_i}w
=\sum_{i=0}^n\alpha_i\bar\mu a_i=\int u\,d\bar\mu$.}
     
\noindent If $u\ge 0$ in $L^1_{\bar\mu}$ there is a
non-decreasing sequence $\sequencen{u_n}$ in $S^f$ with supremum $u$, so
that $Tu=\sup_{n\in\Bbb N}Tu_n$ and
$w\times Tu=\sup_{n\in\Bbb N}w\times Tu_n$ in $L^0(\frak B)$, and
     
\Centerline{$\int Tu\times w=\sup_{n\in\Bbb N}\int Tu_n\times w
=\sup_{n\in\Bbb N}\int u_n=\int u$.}
     
\noindent (365Df tells us that in this context
$Tu\times w\in L^1_{\bar\nu}$.)   Finally, for general
$u\in L^1_{\bar\mu}$,
     
\Centerline{$\int Tu\times w=\int Tu^+\times w-\int Tu^-\times w
=\int u^+-\int u^-=\int u$.}
     
\medskip
     
{\bf (b)} The argument follows the same lines:  start with $u=\chi a$
for $a\in\frak A$, then with $u\in S(\frak A)$, then with $u\in
L^0(\frak A)^+$ and conclude with general $u\in L^0(\frak A)$.   The
point is that $T$ is a Riesz homomorphism, so that at the last step
     
$$\eqalign{\int Tu
&=\int(Tu)^+-\int(Tu)^- %
=\int T(u^+)-\int T(u^-)\cr
&=\int u^+\times w'-\int u^-\times w'
=\int(u\times w')^+-\int(u\times w')^- %
=\int u\times w'\cr}$$
     
\noindent whenever either side is defined in $[-\infty,\infty]$.
}%end of proof of 365H
     
\leader{365I}{Theorem} Let $(\frak A,\bar\mu)$ be a measure algebra and
$U$ a Banach space.   Let $\nu:\frak A^f\to U$ be a function.   Then the
following are equiveridical:
     
\inset{(i) there is a continuous linear operator $T$ from
$L^1=L^1(\frak A,\bar\mu)$ to $U$ such that $\nu a=T(\chi a)$ for every
$a\in\frak A^f$;
     
(ii)($\alpha$) $\nu$ is additive
     
\quad($\beta$) there is an $M\ge 0$ such
that $\|\nu a\|\le M\bar\mu a$ for every $a\in\frak A^f$.}
     
\noindent Moreover, in this case, $T$ is unique and $\|T\|$ is the
smallest number $M$ satisfying the condition in (ii-$\beta$).
     
\proof{{\bf (a)(i)$\Rightarrow$(ii)} If $T:L^1\to U$ is a continuous
linear operator, then $\chi a\in L^1$ for every
$a\in\frak A^f$, so
$\nu=T\chi$ is a function from $\frak A^f$ to $U$.   If $a$, 
$b\in\frak A^f$ and $a\Bcap b=0$, then $\chi(a\Bcup b)=\chi a+\chi b$ in
$L^0=L^0(\frak A)$ and therefore in $L^1$, so
     
\Centerline{$\nu(a\Bcup b)=T\chi(a\Bcup b)=T(\chi a+\chi b)=T(\chi
a)+T(\chi b)=\nu a+\nu b$.}
     
\noindent If $a\in\frak A^f$ then $\|\chi a\|_1=\bar\mu a$ (using the
formula in 365A, or otherwise), so
     
\Centerline{$\|\nu a\|=\|T(\chi a)\|\le\|T\|\|\chi a\|_1=\|T\|\bar\mu
a$.}
     
\medskip
     
{\bf (b)(ii)$\Rightarrow$(i)} Now suppose that $\nu:\frak A^f\to U$ is
additive and that $\|\nu a\|\le M\bar\mu a$ for every $a\in\frak A^f$.
Let $S^f=L^1\cap S(\frak A)$, as in 365F.   Then there is a linear
operator $T_0:S^f\to U$ such that $T_0(\chi a)=\nu a$ for every
$a\in\frak A^f$ (361F).
Next, $\|T_0u\|\le M\|u\|_1$ for every $u\in S^f$.   \Prf\ If
$u\in S^f\cong S(\frak A^f)$, then $u$ is expressible as
$\sum_{j=0}^m\beta_j\chi b_j$ where
$b_0,\ldots,b_m\in\frak A^f$ are disjoint (361Eb).   So
     
\Centerline{$\|T_0u\|=\|\sum_{j=0}^m\beta_j\nu b_j\|
\le M\sum_{j=0}^m|\beta_j|\bar\mu b_j=M\|u\|_1$.   \Qed}
     
\noindent There is therefore a continuous linear operator $T:L^1\to U$,
extending
$T_0$, and with $\|T\|\le\|T_0\|\le M$ (2A4I).   Of course we
still have $\nu=T\chi$.
     
\medskip
     
{\bf (c)} The argument in (b) shows that $T_0=T\restr S^f$ and $T$ are
uniquely defined from $\nu$.   We have also seen that if $T$, $\nu$
correspond to each other then
     
\Centerline{$\|\nu a\|\le\|T\|\bar\mu a$ for every $a\in\frak A^f$,}
     
\Centerline{$\|T\|\le M$ whenever $\|\nu a\|\le M\bar\mu a$ for every
$a\in\frak A^f$,}
     
\noindent so that
     
\Centerline{$\|T\|=\min\{M:M\ge 0,\,\|\nu a\|\le M\bar\mu a$ for every
$a\in\frak A^f\}$.}
}%end of proof of 365I
     
\leader{365J}{Corollary} Let $(X,\Sigma,\mu)$ be a measure space and $U$
any Banach space.   Set $\Sigma^f=\{E:E\in \Sigma,\,\mu E<\infty\}$.
Let $\nu:\Sigma^f\to U$ be a function.   Then the following are
equiveridical:
     
\inset{(i) there is a continuous linear operator $T:L^1(\mu)\to U$
such that $\nu E=T(\chi E)^{\ssbullet}$ for every $E\in\Sigma^f$;
     
(ii)($\alpha$) $\nu(E\cup F)=\nu E+\nu F$ whenever $E$, $F\in\Sigma^f$
and $E\cap F=0$ ($\beta$) there is an $M\ge 0$ auch that
$\|\nu E\|\le M\mu E$ for every $E\in\Sigma^f$.}
     
\noindent Moreover, in this case, $T$ is unique and $\|T\|$ is the
smallest number
$M$ satisfying the condition in (ii-$\beta$).
     
\proof{ This is a direct translation of 365I.   The only point to note
is that if $\nu$ satisfies the conditions of (ii), and $E$,
$F\in\Sigma^f$ are such that $E^{\ssbullet}=F^{\ssbullet}$ in the
measure algebra $(\frak A,\bar\mu)$ of $(X,\Sigma,\mu)$, then
$\mu(E\setminus F)=\mu(F\setminus E)=0$, so that
$\nu(E\setminus F)=\nu(F\setminus E)=0$ (using condition (ii-$\beta$))
and
     
\Centerline{$\nu E=\nu(E\cap F)+\nu(E\setminus F)
=\nu(E\cap F)+\nu(F\setminus E)=\nu F$.}
     
\noindent This means that we have a function $\bar\nu:\frak A^f\to U$,
where
     
\Centerline{$\frak A^f=\{a:a\in\frak A,\,\bar\mu a<\infty\}
=\{E^{\ssbullet}:E\in\Sigma^f\}$,}
     
\noindent defined by setting $\bar\nu E^{\ssbullet}=\nu E$ for every
$E\in\Sigma^f$.   Of course we now have
$\bar\nu(a\Bcup b)=\bar\nu a+\bar\nu b$ whenever $a$, $b\in\frak A^f$
and $a\Bcap b=0$ (since we
can express them as $a=E^{\ssbullet}$, $b=F^{\ssbullet}$ with
$E\cap F=\emptyset$), and $\|\bar\nu a\|\le M\bar\mu a$ for every
$a\in\frak A^f$.   Thus we have a one-to-one correspondence between
functions $\nu:\Sigma^f\to U$ satisfying the conditions (ii) here, and
functions $\bar\nu:\frak A^f\to U$ satisfying the conditions (ii) of
365I.   The
rest of the argument is covered by the identification between $L^1(\mu)$
and $L^1$ in 365B.
}%end of proof of 365J
     
\leader{365K}{Theorem} Let $(\frak A,\bar\mu)$ be a measure algebra, $U$
a Banach lattice, and $T$ a bounded linear operator from
$L^1=L^1(\frak A,\bar\mu)$ to $U$.   Let $\nu:\frak A^f\to U$ be the
corresponding additive function, as in 365I.
     
(a) $T$ is a positive linear operator iff $\nu a\ge 0$ in $U$ for every
$a\in\frak A^f$;  in this case, $T$ is order-continuous.
     
(b) If $U$ is Dedekind complete and
$T\in\eurm L^{\sim}(L^1;U)$, then
$|T|:L^1\to U$ corresponds to $|\nu|:\frak A^f\to U$, where
     
\Centerline{$|\nu|(a)=\sup\{\sum_{i=0}^n|\nu
a_i|:a_0,\ldots,a_n\Bsubseteq a$ are disjoint$\}$}
     
\noindent for every $a\in\frak A^f$.
     
(c) $T$ is a Riesz homomorphism iff $\nu$ is a lattice homomorphism.
     
\proof{ As in 365F, let $S^f$ be $L^1\cap S(\frak A)$, identified with
$S(\frak A^f)$.
     
\medskip
     
{\bf (a)(i)} If $T$ is a positive linear operator and $a\in\frak A^f$,
then $\chi a\ge 0$ in $L^1$, so $\nu a=T(\chi a)\ge 0$ in $U$.
     
\medskip
     
\quad{\bf (ii)} Now suppose that $\nu a\ge 0$ in $U$ for every
$a\in\frak A^f$, and let $u\ge 0$ in $L^1$, $\epsilon>0$ in $\Bbb R$.
Then there is a $v\in S^f$ such that $0\le v\le u$ and
$\|u-v\|_1\le\epsilon$ (365F).   Express $v$ as
$\sum_{i=0}^n\alpha_i\chi a_i$ where $a_i\in\frak A^f$, $\alpha_i\ge 0$
for each $i$.   Now
     
\Centerline{$\|Tu-Tv\|\le\|T\|\|u-v\|_1
\le\epsilon\|T\|$.}
     
\noindent On the other hand,
     
\Centerline{$Tv=\sum_{i=0}^n\alpha_i\nu a_i\in U^+$.}
     
\noindent As $U^+$ is norm-closed in $U$ (354Bc), and $\epsilon$ is
arbitrary,
$Tu\in U^+$.   As $u$ is arbitrary, $T$ is a positive linear operator.
     
\medskip
     
\quad{\bf (iii)} By 355Ka, $T$ is order-continuous.
     
\medskip
     
{\bf (b)} The point is that $|T\restr S^f|=|T|\restr S^f$.   \Prf\ (i)
Because the embedding $S^f\embedsinto L^1$ is positive, the map
$R\mapsto R\restr S^f$ is a positive linear operator from
$\eurm L^{\sim}(L^1;U)$
to $\eurm L^{\sim}(S^f;U)$ (see 355Bd).   So
$|T\restr S^f|\le|T|\restr S^f$.   (ii) There is a positive linear
operator $T_1:L^1\to U$
extending $|T\restr S^f|$, by 365J and (a) above, and now
$T_1\restr S^f$ dominates both $T\restr S^f$ and $-T\restr S^f$;  since
$(S^f)^+$ is dense in $(L^1)^+$, $T_1\ge T$ and $T_1\ge -T$, so that
$T_1\ge|T|$ and
     
\Centerline{$|T\restr S^f|=T_1\restr S^f\ge|T|\restr S^f$.  \Qed}
     
Now 361H tells us that
     
\Centerline{$|T|(\chi a)=|T\restr S^f|(\chi a)=|\nu|a$}
     
\noindent for every $a\in\frak A^f$.
     
\medskip
     
{\bf (c)(i)}  If $T$ is a lattice homomorphism, then so is $\nu=T\chi$,
because $\chi:\frak A^f\to S^f$ is a lattice homomorphism.
     
\medskip
     
\quad{\bf (ii)} Now suppose that $\chi$ is a lattice homomorphism.   In
this case $T\restr S^f$ is a Riesz homomorphism (361Gc), that is,
$|Tv|=T|v|$ for every $v\in S^f$.   Because $S^f$
is norm-dense in $L^1$ and the map $u\mapsto|u|$ is continuous both in
$L^1$
and in $U$ (354Bb), $|Tu|=T|u|$ for every $u\in L^1$, and $T$ is a Riesz
homomorphism.
}%end of proof of 365K
     
\leader{365L}{The duality between $L^1$ and $L^{\infty}$}  Let
$(\frak A,\bar\mu)$ be a measure algebra, and set
$L^1=L^1(\frak A,\bar\mu)$, $L^{\infty}=L^{\infty}(\frak A)$.   If we
identify $L^{\infty}$ with the solid linear subspace of
$L^0=L^0(\frak A)$ generated by $e=\chi 1_{\frak A}$ (364J), then we
have a bilinear operator $(u,v)\mapsto u\times v:
L^1\times L^{\infty}\to L^1$\prooflet{, because
$|u\times v|\le\|v\|_{\infty}|u|$ and
$L^1$ is a solid linear subspace of $L^0$}.
\cmmnt{Note that} $\|u\times v\|_1\le\|u\|_1\|v\|_{\infty}$,
so\cmmnt{ that}
the bilinear operator $(u,v)\mapsto u\times v$ has norm at most
$1$\cmmnt{ (253Ab, 253E)}.   Consequently we have a bilinear functional
$(u,v)\mapsto\int u\times v:L^1\times L^{\infty}\to\Bbb R$, which also
has norm at most $1$, corresponding to
linear operators $S:L^1\to (L^{\infty})^*$ and $T:L^{\infty}\to(L^1)^*$,
both of norm at most $1$, defined by the formula
     
\Centerline{$(Su)(v)=(Tv)(u)=\int u\times v$ for $u\in L^1$,
$v\in L^{\infty}$.}
     
\noindent\cmmnt{ Because $L^1$ and $L^{\infty}$ are
both Banach lattices, we have }$(L^1)^*=(L^1)^{\sim}$ and
$(L^{\infty})^*=(L^{\infty})^{\sim}$\cmmnt{ (356Dc)}.
\cmmnt{Because the norm of
$L^1$ is order-continuous,} $(L^1)^*=(L^1)^{\times}$\cmmnt{ (356Dd)}.
     
\leader{365M}{Theorem} Let $(\frak A,\bar\mu)$ be a measure algebra, and
set $L^1=L^1(\frak A,\bar\mu)$, $L^{\infty}=L^{\infty}(\frak A)$.   Let
$S:L^1\to(L^{\infty})^*=(L^{\infty})^{\sim}$,
$T:L^{\infty}\to (L^1)^*=(L^1)^{\sim}=(L^1)^{\times}$ be the canonical
maps defined by
the duality between $L^1$ and $L^{\infty}$\cmmnt{, as in 365L}.   Then
     
(a) $S$ and $T$ are order-continuous Riesz homomorphisms,
$S[L^1]\subseteq(L^{\infty})^{\times}$, $S$ is
norm-preserving and $T[L^{\infty}]$ is order-dense in $(L^1)^{\sim}$;
     
(b) $(\frak A,\bar\mu)$ is semi-finite iff $T$ is injective, and in this
case $T$ is norm-preserving, while $S$ is a normed Riesz space
isomorphism between $L^1$ and $(L^{\infty})^{\times}$;
     
(c) $(\frak A,\bar\mu)$ is localizable iff $T$ is bijective, and in this
case $T$ is a normed Riesz space isomorphism between $L^{\infty}$ and
$(L^1)^*=(L^1)^{\sim}=(L^1)^{\times}$.
     
\proof{{\bf (a)(i)} If $u\ge 0$ in $L^1$ and $v\ge 0$ in $L^{\infty}$
then $u\times v\ge 0$ and
     
\Centerline{$(Tv)(u)=\int u\times v\ge 0$.}
     
\noindent As $u$ is arbitrary, $Tv\ge 0$ in $(L^1)^{\times}$;  as $v$ is
arbitrary, $T$ is a positive linear operator.
     
If $v\in L^{\infty}$, set $a=\Bvalue{v>0}\in\frak A$.   (Remember that
we are identifying $L^0(\mu)$, as defined in \S241, with $L^0(\frak A)$,
as defined in \S364.)   Then $v^+=v\times\chi a$, so for any $u\ge 0$ in
$L^1$
     
\Centerline{$(Tv^+)(u)=\int u\times v\times\chi a
=(Tv)(u\times\chi a)\le(Tv)^+(u)$.}
     
\noindent As $u$ is arbitrary, $Tv^+\le(Tv)^+$.   On the other hand,
because $T$ is a positive linear operator, $Tv^+\ge Tv$ and $Tv^+\ge 0$,
so $Tv^+\ge(Tv)^+$.   Thus $Tv^+=(Tv)^+$.   As $v$ is arbitrary, $T$ is
a Riesz homomorphism (352G).
     
\medskip
     
\quad{\bf (ii)} Exactly the same arguments show that $S$ is a Riesz
homomorphism.
     
\medskip
     
\quad{\bf (iii)} Given $u\in L^1$, set $a=\Bvalue{u>0}$;  then
     
\Centerline{$\|Su\|\ge (Su)(\chi a-\chi(1\Bsetminus a))
=\int_au-\int_{1\Bsetminus a}u=\int|u|=\|u\|_1\ge\|Su\|$.}
     
\noindent So $S$ is norm-preserving.
     
\medskip
     
\quad{\bf (iv)} By 355Ka, $S$ is order-continuous.
     
\medskip
     
\quad{\bf (v)} If $A\subseteq L^{\infty}$ is a non-empty
downwards-directed set with infimum $0$, and $u\in(L^1)^+$, then
$\inf_{v\in A}u\times v=0$ for every $u\in(L^1)^+$, because $v\mapsto
u\times v:L^0\to L^0$ is
order-continuous.   So
     
\Centerline{$\inf_{v\in A}(Tv)(u)
=\inf_{v\in A}\int u\times v=\inf_{v\in A}\|u\times v\|_1=0$}
     
\noindent and the only possible non-negative lower bound for $T[A]$ in
$(L^1)^{\times}$ is $0$.   As $A$ is arbitrary, $T$ is order-continuous.
     
\medskip
     
\quad{\bf (vi)} The ideas of (v) show also that
$S[L^1]\subseteq(L^{\infty})^{\times}$.   \Prf\ If $u\in(L^1)^+$ and
$A\subseteq L^{\infty}$ is non-empty, downwards-directed and has infimum
$0$, then
     
\Centerline{$\inf_{v\in A}(Su)(v)=\inf_{v\in A}\int u\times v=0$.}
     
\noindent As $A$ is arbitrary, $Su$ is order-continuous.   For general
$u\in L^1$, $Su=Su^+-Su^-$ belongs to $(L^{\infty})^{\times}$.\ \Qed
     
\medskip
     
\quad{\bf (vii)} Now suppose that $h>0$ in $(L^1)^*=(L^1)^{\times}$.
By 365Ka, applied to $-h$, there must be an $a\in\frak A^f$ such that
$h(\chi a)>0$.   Set $\nu b=h(\chi(a\Bcap b))$ for
$b\in\frak A^f$.   Then $\nu$ is additive and non-negative and bounded
by $\|h\|\bar\mu a$.   If $A\subseteq\frak A^f$ is a non-empty
downwards-directed set with infimum $0$, then $C=\{\chi b:b\in A\}$ is
downwards-directed and has infimum $0$ in $L^0(\frak A)$ (364Jc), so
$\inf_{b\in A}\nu b=\inf_{u\in C}h(u)=0$.   By 365Eb, there is a
$v\in L^1$ such that
$\nu b=\int_bu$ for every $b\in\frak A^f$.   As $\int_bv\ge 0$ for every
$b\in\frak A^f$, $v\ge 0$ (365E(d-i)).   Setting $b=\Bvalue{v>\|h\|}$,
we have
     
\Centerline{$\int_bv\le h(\chi b)\le\|h\|\|\chi b\|_1=\|h\|\bar\mu b$;}
     
\noindent so $b=0$ (365Ec).   Accordingly $0\le v\le\|h\|\chi 1$ and
$v\in L^{\infty}$.   Consider $Tv\in(L^1)^{\times}$.   We have
$Tv\ge 0$ because $T$ is positive;  also
     
\Centerline{$(Tv)(\chi a)=\int_av=\nu a=h(\chi a)>0$,}
     
\noindent so $Tv>0$.   Next, for every $b\in\frak A^f$,
     
\Centerline{$(Tv)(\chi b)=\int_bv=h(\chi(a\Bcap b))\le h(\chi b)$.}
     
\noindent By 365Ka again, $h-Tv\ge 0$, that is, $Tv\le h$.   As $h$ is
arbitrary, $T[L^{\infty}]$ is quasi-order-dense in $(L^1)^*$, therefore
order-dense (353A).
     
\medskip
     
{\bf (b)(i)}  If $(\frak A,\bar\mu)$ is not semi-finite, let
$a\in\frak A$ be such that $\bar\mu a=\infty$ and $\bar\mu b=\infty$
whenever $0\ne b\Bsubseteq a$.   If $u\in L^1$, then
$\Bvalue{|u|>\bover1n}$ has finite
measure for every $n\ge 1$, so must be disjoint from $a$;  accordingly
     
\Centerline{$a\Bcap\Bvalue{|u|>0}=\sup_{n\ge
1}a\Bcap\Bvalue{|u|>\bover1n}=0$.}
     
\noindent This means that $\int u\times\chi a=0$ for every $u\in L^1$.
Accordingly $T(\chi a)=0$ and $T$ is not injective.
     
\medskip
     
\quad{\bf (ii)} If $(\frak A,\bar\mu)$ is semi-finite, take any $v\in
L^{\infty}$.   Then if $0\le\delta<\|v\|_{\infty}$,
$a=\Bvalue{|v|>\delta}\ne 0$.   Let $b\subseteq a$ be such that
$0<\bar\mu b<\infty$.   Then $\chi b\in L^1$, and
     
\Centerline{$\|Tv\|=\||Tv|\|=\|T|v|\|
\ge (T|v|)(\chi b)/\|\chi b\|_1\ge\delta$}
     
\noindent because $|v|\times\chi b\ge\delta\chi b$, so
     
\Centerline{$(T|v|)(\chi b)\ge\delta\bar\mu b=\delta\|\chi b\|_1$.}
     
\noindent   As $\delta$ is arbitrary, $\|Tv\|\ge\|v\|_{\infty}$.   But
we already know that $\|Tv\|\le\|v\|_{\infty}$, so the two are equal.
As $v$ is arbitrary, $T$ is norm-preserving (and, in particular, is
injective).
     
\medskip
     
\quad{\bf (iii)} Still supposing that $(\frak A,\bar\mu)$ is
semi-finite, $S[L^1]=(L^{\infty})^{\times}$.
\Prf\ Take any $h\in(L^{\infty})^{\times}$.    For $a\in\frak A$, set
$\nu a=h(\chi a^{\ssbullet})$.   By 363K, $\nu:\frak A\to\Bbb R$
is completely additive.   By 365Ea, there is a $u\in L^1$ such that
     
\Centerline{$(Su)(\chi a)=\int u\times\chi a=\int_au=\nu a
=h(\chi a)$}
     
\noindent for every $a\in\frak A$.   Because $Su$ and $h$ are both
linear functionals on $L^{\infty}$, they must agree on $S(\frak A)$;
because they are continuous and $S(\frak A)$ is dense in $L^{\infty}$
(363C), $Su=h$.   As $h$ is arbitrary, $S$ is surjective.\ \Qed
     
\medskip
     
{\bf (c)} Using (b), we know that if either $T$ is bijective or
$(\frak A,\bar\mu)$ is localizable, then $(\frak A,\bar\mu)$ is
semi-finite.   Given this, if $T$
is bijective, then it is a Riesz space isomorphism between $L^{\infty}$
and $(L^1)^{\sim}$, which is Dedekind complete (356B);  so 363Mb tells
us that $\frak A$ is Dedekind complete and $(\frak A,\bar\mu)$ is
localizable.   In the other direction, if $(\frak A,\bar\mu)$ is
localizable, then $L^{\infty}$ is Dedekind complete.   As $T$ is
injective, $T[L^{\infty}]$ is, in itself, Dedekind complete;  being an
order-dense Riesz subspace of $(L^1)^{\sim}$ (by (a) here) it must be
solid (353K);  as it contains $T(\chi 1)$, which is the standard order
unit of the $M$-space $(L^1)^{\sim}$, it is the whole of $(L^1)^{\sim}$,
and $T$ is bijective.
}%end of proof of 365M
     
\leader{365N}{Corollary} If $(\frak A,\bar\mu)$ is a localizable measure
algebra, $L^{\infty}(\frak A)$ is a perfect Riesz space.
     
\proof{ By 365M(b)-(c), we can identify $L^{\infty}$ with
$(L^1_{\bar\mu})^{\times}\cong(L^{\infty})^{\times\times}$.
}%end of proof of 365N
     
\leader{365O}{Theorem} Let $(\frak A,\bar\mu)$ and $(\frak B,\bar\nu)$
be measure algebras.   Let $\pi:\frak A^f\to\frak B^f$ be a
measure-preserving ring homomorphism.
     
(a) There is a unique order-continuous norm-preserving Riesz
homomorphism $T_{\pi}:L^1(\frak A,\bar\mu)\to L^1(\frak B,\bar\nu)$ such
that
$T_{\pi}(\chi a)=\chi(\pi a)$ whenever $a\in\frak A^f$.   We have
$T_{\pi}(u\times\chi a)=T_{\pi}u\times\chi(\pi a)$ whenever
$a\in\frak A^f$ and $u\in L^1(\frak A,\bar\mu)$.
     
(b) $\int T_{\pi}u=\int u$ and $\int_{\pi a}T_{\pi}u=\int_au$ for every
$u\in L^1(\frak A,\bar\mu)$ and $a\in\frak A^f$.
     
(c) $\Bvalue{T_{\pi}u>\alpha}=\pi\Bvalue{u>\alpha}$ for every
$u\in L^1(\frak A,\bar\mu)$ and $\alpha>0$.
     
(d) $T_{\pi}$ is surjective iff $\pi$ is.
     
(e) If $(\frak C,\bar\lambda)$ is another measure algebra and
$\theta:\frak B^f\to\frak C^f$ another measure-preserving ring
homomorphism, then $T_{\theta\pi}=T_{\theta}T_{\pi}:
L^1(\frak A,\bar\mu)\to L^1(\frak C,\bar\lambda)$.
     
\proof{ Throughout the
proof I will write $T$ for $T_{\pi}$ and $S^f$ for
$S(\frak A)\cap L^1_{\bar\mu}\cong S(\frak A^f)$ (see 365F).
     
\medskip
     
{\bf (a)(i)} We have a map $\psi:\frak A^f\to L^1_{\bar\nu}$ defined by
writing
$\psi a=\chi(\pi a)$ for $a\in\frak A^f$.   Because
     
\Centerline{$\chi\pi(a\Bcup b)=\chi(\pi a\Bcup\pi b)
=\chi\pi a+\chi\pi b,\quad\|\chi(\pi a)\|_1=\bar\nu(\pi a)=\bar\mu a$}
     
\noindent whenever $a$, $b\in\frak A^f$ and $a\Bcap b=0$, we get a
(unique) corresponding bounded linear operator
$T:L^1_{\bar\mu}\to L^1_{\bar\nu}$ such that $T\chi=\chi\pi$ on
$\frak A^f$ (365I).   Because
$\pi:\frak A^f\to\frak B^f$ and $\chi:\frak B^f\to L^1_{\bar\nu}$ are
lattice homomorphisms, so is $\psi$, and $T$ is a Riesz homomorphism
(365Kc).
     
\medskip
     
\quad{\bf (ii)}
If $u\in S^f$, express it as $\sum_{i=0}^n\alpha_i\chi a_i$ where
$a_0,\ldots,a_n$ are disjoint in $\frak A^f$.   Then
$Tu=\sum_{i=0}^n\alpha_i\chi(\pi a_i)$ and $\pi a_0,\ldots,\pi a_n$ are
disjoint in $\frak B^f$, so
     
\Centerline{$\|Tu\|_1
=\sum_{i=0}^n|\alpha_i|\bar\nu(\pi a_i)
=\sum_{i=0}^n|\alpha_i|\bar\mu a_i
=\|u\|_1$.}
     
\noindent Because $S^f$ is dense in $L^1_{\bar\mu}$ and
$u\mapsto\|u\|_1$ is continuous (in both $L^1_{\bar\mu}$ and
$L^1_{\bar\nu}$), $\|Tu\|_1=\|u\|_1$ for every $u\in L^1_{\bar\mu}$,
that is, $T$ is norm-preserving.   As noted in 365Ka, $T$ is
order-continuous.
     
\medskip
     
\quad{\bf (iii)} If $a$, $b\in\frak A^f$ then
     
\Centerline{$T(\chi a\times\chi b)=T(\chi(a\Bcap b))=\chi\pi(a\Bcap b)
=\chi(\pi a\Bcap\pi b)=\chi\pi a\times\chi\pi b
=\chi\pi a\times T(\chi b)$.}
     
\noindent Because $T$ is linear and $\times$ is bilinear,
$T(\chi a\times u)=\chi\pi a\times Tu$ for every $u\in S^f$.
Because the maps
$u\mapsto u\times\chi a:L^1_{\bar\mu}\to L^1_{\bar\mu}$,
$T:L^1_{\bar\mu}\to L^1_{\bar\nu}$ and
$v\mapsto v\times\chi\pi a:L^1_{\bar\nu}\to L^1_{\bar\nu}$ are all
continuous,
$Tu\times\chi\pi a=T(u\times\chi a)$ for every $u\in L^1_{\bar\mu}$.
     
\medskip
     
\quad{\bf (iv)} $T$ is unique because the formula $T(\chi a)=\chi\pi a$
defines $T$ on the norm-dense and order-dense subspace $S^f$.
     
\medskip
     
{\bf (b)} Because $T$ is positive,
     
\Centerline{$\int Tu=\|Tu^+\|_1-\|Tu^-\|_1
=\|u^+\|_1-\|u^-\|_1=\int u$.}
     
\noindent For $a\in\frak A^f$,
     
\Centerline{$\int_{\pi a}Tu=\int Tu\times\chi\pi a
=\int T(u\times\chi a)=\int u\times\chi a=\int_au$.}
     
\medskip
     
{\bf (c)}  If $u\in S^f$, express it as $\sum_{i=0}^n\alpha_i\chi a_i$
where $a_0,\ldots,a_n$ are disjoint;  then
     
\Centerline{$\pi\Bvalue{u>\alpha}
=\pi(\sup_{i\in I}a_i)
=\sup_{i\in I}\pi a_i
=\Bvalue{Tu>\alpha}$}
     
\noindent where $I=\{i:i\le n,\,\alpha_i>\alpha\}$.   For
$u\in(L^1_{\bar\mu})^+$,
take a sequence $\sequencen{u_n}$ in $S^f$ with supremum $u$;  then
$\sup_{n\in\Bbb N}Tu_n=Tu$, so
     
$$\eqalignno{\pi\Bvalue{u>\alpha}
&=\pi(\sup_{n\in\Bbb N}\Bvalue{u_n>\alpha})\cr
\noalign{\noindent (364L(a-ii);  $\Bvalue{u>\alpha}\in\frak A^f$ by 365A)}
&=\sup_{n\in\Bbb N}\pi\Bvalue{u_n>\alpha}\cr
\noalign{\noindent (because $\pi$ is order-continuous, see 361Ad)}
&=\sup_{n\in\Bbb N}\Bvalue{Tu_n>\alpha}
=\Bvalue{Tu>\alpha}\cr}$$
     
\noindent because $T$ is order-continuous.
For general $u\in L^1_{\bar\mu}$,
     
\Centerline{$\pi\Bvalue{u>\alpha}=\pi\Bvalue{u^+>\alpha}
=\Bvalue{T(u^+)>\alpha}
=\Bvalue{(Tu)^+>\alpha}
=\Bvalue{Tu>\alpha}$}
     
\noindent because $T$ is a Riesz homomorphism.
     
\medskip
     
{\bf (d)(i)} Suppose that $T$ is surjective and that $b\in\frak B^f$.
Then there is a $u\in L^1_{\bar\mu}$ such that $Tu=\chi b$.   Now
     
\Centerline{$b=\Bvalue{Tu>\bover12}
=\pi\Bvalue{u>\bover12}\in\pi[\frak A^f]$;}
     
\noindent as $b$ is arbitrary, $\pi$ is surjective.
     
\medskip
     
\quad{\bf (ii)} Suppose now that $\pi$ is surjective.   Then
$T[L^1_{\bar\mu}]$ is a linear subspace of $L^1_{\bar\nu}$ containing
$\chi b$ for every $b\in\frak B^f$, so includes $S(\frak B^f)$.   If
$v\in(L^1_{\bar\nu})^+$ there is a sequence $\sequencen{v_n}$ in
$S(\frak B^f)^+$ with supremum $v$.   For each $n$, choose $u_n$ such
that
$Tu_n=v_n$.   Setting $u'_n=\sup_{i\le n}u_i$, we get a
non-decreasing sequence $\sequencen{u'_n}$ such that $v_n\le Tu'_n\le v$
for every $n\in\Bbb N$.   So
     
\Centerline{$\sup_{n\in\Bbb N}\|u'_n\|_1=\sup_{n\in\Bbb
N}\|Tu'_n\|_1\le\|v\|_1<\infty$}
     
\noindent and $u=\sup_{n\in\Bbb N}u'_n$ is defined in $L^1_{\bar\mu}$,
with
     
\Centerline{$Tu=\sup_{n\in\Bbb N}Tu'_n=v$.}
     
\noindent Thus $(L^1_{\bar\nu})^+\subseteq T[L^1_{\bar\mu}]$;
consequently $L^1_{\bar\nu}\subseteq T[L^1_{\bar\mu}]$ and $T$ is
surjective.
     
\medskip
     
{\bf (e)} This is an immediate consequence of the `uniqueness'
assertion in (i), because for any $a\in\frak A^f$
     
\Centerline{$T_{\theta}T_{\pi}(\chi a)
=T_{\theta}\chi(\pi a)
=\chi(\theta\pi a)$,}
     
\noindent so that $T_{\theta}T_{\pi}:L^1_{\bar\mu}\to L^1_{\bar\lambda}$
is a bounded linear operator taking the right values at elements
$\chi a$, and must therefore be equal to $T_{\theta\pi}$.
}%end of proof of 365O
     
\leader{365P}{Theorem} Let $(\frak A,\bar\mu)$ and $(\frak B,\bar\nu)$
be measure algebras, and $\pi:\frak A^f\to\frak B$ an order-continuous
ring homomorphism.
     
(a) There is a unique positive linear operator
$P_{\pi}:L^1(\frak B,\bar\nu)\to L^1(\frak A,\bar\mu)$ such that
$\int_aP_{\pi}v=\int_{\pi a}v$ for every $v\in L^1(\frak B,\bar\nu)$ and
$a\in\frak A^f$.
     
(b) $P_{\pi}$ is order-continuous and norm-continuous, and
$\|P_{\pi}\|\le 1$.
     
(c) If $a\in\frak A^f$ and $v\in L^1(\frak B,\bar\nu)$ then
$P_{\pi}(v\times\chi\pi a)=P_{\pi}v\times\chi a$.
     
(d) If $\pi[\frak A^f]$ is order-dense in $\frak B$ then $P_{\pi}$ is a
norm-preserving
Riesz homomorphism;  in particular, $P_{\pi}$ is injective.
     
(e) If $(\frak B,\bar\nu)$ is semi-finite and $\pi$ is
injective, then $P_{\pi}$ is surjective, and there is for every
$u\in L^1(\frak A,\bar\mu)$ a $v\in L^1(\frak B,\bar\nu)$ such that
$P_{\pi}v=u$ and $\|v\|_1=\|u\|_1$.
     
(f) Suppose again that $(\frak B,\bar\nu)$ is semi-finite.   If
$(\frak C,\bar\lambda)$
is another measure algebra and $\theta:\frak B\to\frak C$ an
order-continuous Boolean homomorphism, then
$P_{\theta\pi}=P_{\pi}P_{\theta'}:
L^1(\frak C,\bar\lambda)\to L^1(\frak A,\bar\mu)$, where I write
$\theta'$ for the restriction of $\theta$ to $\frak B^f$.
     
\proof{ I write $P$ for $P_{\pi}$.
     
\medskip
     
{\bf (a)-(b)} For $v\in L^1_{\bar\nu}$, $a\in\frak A^f$ set
$\nu_v(a)=\int_{\pi a}v$.   Then $\nu_v:\frak A^f\to\Bbb R$ is additive,
bounded (by $\|v\|_1$) and if $A\subseteq\frak A^f$ is non-empty,
downwards-directed and has infimum $0$, then
     
\Centerline{$\inf_{a\in A}|\nu_v(a)|
\le\inf_{a\in A}\int|v|\times\chi\pi a
=0$}
     
\noindent because $a\mapsto\int|v|\times\chi\pi a$ is a composition of
order-continuous functions, therefore
order-continuous.   So 365Eb tells us that there is a $Pv\in
L^1_{\bar\mu}$ such that $\int_aPv=\nu_v(a)=\int_{\pi a}v$ for every
$a\in\frak A^f$.   By 365D(d-ii), this formula defines $Pv$ uniquely.
Consequently $P$ must be linear (since $Pv_1+Pv_2$, $\alpha Pv$ will
always have the properties defining $P(v_1+v_2)$, $P(\alpha v)$).
     
If $v\ge 0$ in $L^1_{\bar\nu}$, then $\int_aPv=\int_{\pi a}v\ge 0$ for
every $a\in\frak A^f$, so $Pv\ge 0$ (365D(d-i));  thus $P$ is positive.
It must therefore be norm-continuous and order-continuous (355C, 355Ka).
     
Again supposing that $v\ge 0$, we have
     
\Centerline{$\|Pv\|_1=\int Pv=\sup_{a\in\frak A^f}\int_aPv
=\sup_{a\in\frak A^f}\int_{\pi a}v\le\|v\|_1$}
     
\noindent (using 365D(d-iii)).   For general $v\in L^1_{\bar\nu}$,
     
\Centerline{$\|Pv\|_1=\||Pv|\|_1\le\|P|v|\|_1\le\|v\|_1$.}
     
\medskip
     
{\bf (c)} For any $c\in\frak A^f$,
     
\Centerline{$\int_cPv\times\chi a
=\int_{c\Bcap a}Pv
=\int_{\pi(c\Bcap a)}v
=\int_{\pi c}v\times\chi\pi a
=\int_cP(v\times\chi\pi a)$.}
     
\medskip
     
{\bf (d)} Now suppose that $\pi[\frak A^f]$ is order-dense.   Take any
$v$, $v'\in L^1_{\bar\nu}$ such that $v\wedge v'=0$.   \Quer\ Suppose,
if possible, that $u=Pv\wedge Pv'>0$.   Take $\alpha>0$ such that
$a=\Bvalue{u>\alpha}$ is non-zero.   Since
     
\Centerline{$\int_{\pi a}v=\int_aPv\ge\int_au>0$,}
     
\noindent $b=\pi a\Bcap\Bvalue{v>0}\ne 0$.   Let $c\in\frak A^f$ be such
that $0\ne \pi c\Bsubseteq b$;  then $\pi(a\Bcap c)=\pi c\ne 0$, so
$a\Bcap c\ne 0$, and
     
\Centerline{$0<\int_{a\Bcap c}u\le\int_{a\Bcap c}Pv'
\le\int_{\pi c}v'$.}
     
\noindent But $\pi c\Bsubseteq\Bvalue{v>0}$ and $v\wedge v'=0$ so
$\int_{\pi c}v'=0$.\ \Bang
     
So $Pv\wedge Pv'=0$.   As $v$, $v'$ are arbitrary, $P$ is a Riesz
homomorphism (352G).
     
Next, if $v\ge 0$ in $L^1_{\bar\nu}$,
     
\Centerline{$\int Pv=\sup_{a\in\frak A^f}\int_aPv=\sup_{a\in\frak
A^f}\int_{\pi a}v=\int v$}
     
\noindent because $\pi[\frak A^f]$ is upwards-directed and has supremum
$1$ in $\frak B$.   So, for general $v\in L^1_{\bar\nu}$,
     
\Centerline{$\|Pv\|_1=\int|Pv|=\int P|v|=\int|v|=\|v\|_1$,}
     
\noindent and $P$ is norm-preserving.
     
\medskip
     
{\bf (e)} Next suppose that $(\frak B,\bar\nu)$ is semi-finite and that
$\pi$ is injective.
     
\medskip
     
\quad{\bf (i)} If $u>0$ in $L^1_{\bar\mu}$, there is a $v>0$ in
$L^1_{\bar\nu}$ such that $Pv\le u$ and $\int Pv\ge\int v$.   \Prf\ Let
$\delta>0$ be such that
$a=\Bvalue{u>\delta}\ne 0$.   Then $\pi a\ne 0$.   Because $(\frak
B,\bar\nu)$ is semi-finite, there is a non-zero $b\in\frak B^f$ such
that $b\Bsubseteq \pi a$.   Set $u_1=P(\chi b)$.   Then $u_1\ge 0$,
$\int_au_1=\bar\nu b>0$ and
     
\Centerline{$\int_{1\Bsetminus a}u_1
=\sup_{c\in\frak A^f}\int_{c\Bsetminus a}u_1
=\sup_{c\in\frak A^f}\int_{\pi c\Bsetminus\pi a}\chi b
=0$.}
     
\noindent So $u_1\times\chi(1\Bsetminus a)=0$ and
$0\ne\Bvalue{u_1>0}\Bsubseteq a$.
Let $\gamma>0$ be such that $\Bvalue{u_1>\gamma}\ne\Bvalue{u_1>0}$, and
set $a_1=a\Bsetminus\Bvalue{u_1>\gamma}$,
$v=\Bover{\delta}{\gamma}\chi(b\Bcap\pi a_1)$.   Then
     
\Centerline{$Pv
=\Bover{\delta}{\gamma} P(\chi b\times\chi(\pi a_1))
=\Bover{\delta}{\gamma} P(\chi b)\times\chi a_1
=\Bover{\delta}{\gamma}u_1\times\chi a_1
\le\delta\chi a\le u$,}
     
\noindent because
     
\Centerline{$\Bvalue{u_1\times\chi a_1
>\gamma}\Bsubseteq\Bvalue{u_1>\gamma}\Bcap a_1=0$}
     
\noindent so
     
\Centerline{$u_1\times\chi a_1\le\gamma\chi\Bvalue{u_1>0}
\le\gamma\chi a$.}
     
\noindent Also $a_1\Bcap\Bvalue{u_1>0}\ne 0$, so $Pv$ and $v$ are
non-zero;  and
     
\Centerline{$\int Pv\ge\int_{a_1}Pv=\int_{\pi a_1}v=\int v$.
\Qed}
     
\medskip
     
\quad{\bf (ii)} Now take any $u\ge 0$ in $L^1_{\bar\mu}$, and set
$B=\{v:v\in L^1_{\bar\nu},\,v\ge 0,\,Pv\le u,\,\int v\le\int Pv\}$.
$B$ is not empty because it contains $0$.
If $C\subseteq B$ is non-empty and upwards-directed, then
$\sup_{v\in C}\int v\le\int u$ is finite, so $C$ has a supremum in
$L^1_{\bar\nu}$ (365Df).   Because $P$ is order-continuous,
$P(\sup C)=\sup P[C]\le u$;  also
     
\Centerline{$\int\sup C=\sup_{v\in C}\int v
\le\sup_{v\in C}\int Pv\le\int P(\sup C)$.}
     
\noindent Thus $\sup C\in B$.   As $C$ is arbitrary, $B$ satisfies the
conditions of Zorn's Lemma, and has a maximal element $v_0$ say.
     
\Quer\ Suppose, if possible, that $Pv_0\ne u$.   By ($\alpha$),
there is a $v_1>0$ such that $Pv_1\le u-Pv_0$, $\int v_1\le\int Pv_1$.
In this case,
$v_0<v_0+v_1\in B$, which is impossible.\ \Bang\ Thus
$Pv_0=u$;  also
     
\Centerline{$\|v_0\|_1=\int v_0\le\int Pv_0=\|Pv_0\|_1$.}
     
\medskip
     
\quad{\bf (iii)} Finally, take any $u\in L^1_{\bar\mu}$.   By (ii),
there are non-negative $v_1$, $v_2\in L^1_{\bar\nu}$ such that
$Pv_1=u^+$,
$Pv_2=u^-$, $\|v_1\|_1\le\|u^+\|_1$ and $\|v_2\|_1\le\|u^-\|_1$.
Setting $v=v_1-v_2$, we have $Pv=u$.   Also we must have
     
\Centerline{$\|v\|_1\le\|v_1\|_1+\|v_2\|_1\le\|u^+\|_1+\|u^-\|_1
=\|u\|_1\le\|P\|\|v\|_1=\|v\|_1$,}
     
\noindent so $\|v\|_1=\|u\|_1$, as required.
     
\medskip
     
{\bf (f)} As usual, this is a consequence of the
uniqueness of $P$.   However (because I do not assume that
$\pi[\frak A^f]\subseteq\frak B^f$) there is an extra refinement:  we
need to know
that $\int_bP_{\theta'}w=\int_{\theta b}w$ for every $b\in\frak B$ and
$w\in L^1_{\bar\lambda}$.   \Prf\ Because $\theta$ is order-continuous
and $(\frak B,\bar\nu)$ is semi-finite,
$\theta b=\sup\{\theta b':b'\in\frak B^f,\,b'\Bsubseteq b\}$, so if
$w\ge 0$ then
     
\Centerline{$\int_{\theta b}w
=\sup_{b'\in\frak B^f,b'\Bsubseteq b}\int_{\theta b'}w
=\sup_{b'\in\frak B^f,b'\Bsubseteq b}\int_{b'}P_{\theta'}w
=\int_bP_{\theta'}w$.}
     
\noindent Expressing $w$ as $w^+-w^-$, we see that the same is true for
every $w\in L^1_{\bar\nu}$.\ \Qed
     
Now we can say that $PP_{\theta'}$ is a positive linear operator
from $L^1_{\bar\lambda}$ to $L^1_{\bar\mu}$ such that
     
\Centerline{$\int_aPP_{\theta'}w
=\int_{\pi a}P_{\theta'}w
=\int_{\theta\pi a}w
=\int_aP_{\theta\pi}w$}
     
\noindent whenever $a\in\frak A^f$ and $w\in L^1_{\bar\lambda}$, and
must be equal to $P_{\theta\pi}$.
}%end of proof of 365P
     
\leader{365Q}{Proposition} Let $(\frak A,\bar\mu)$ and
$(\frak B,\bar\mu)$ be measure algebras and $\pi:\frak A^f\to\frak B^f$
a measure-preserving ring homomorphism.
     
(a) In the language of 365O-365P above,
$P_{\pi}T_{\pi}$ is the identity operator on $L^1(\frak A,\bar\mu)$.
     
(b) If $\pi$ is surjective\cmmnt{ (so that it is an isomorphism
between $\frak A^f$ and $\frak B^f$)} then
$P_{\pi}=T_{\pi}^{-1}=T_{\pi^{-1}}$ and
$T_{\pi}=P_{\pi}^{-1}=P_{\pi^{-1}}$.
     
\proof{{\bf (a)} If $u\in L^1_{\bar\mu}$, $a\in\frak A^f$ then
     
\Centerline{$\int_aP_{\pi}T_{\pi}u=\int_{\pi a}T_{\pi}u=\int_au$.}
     
\noindent So $u=P_{\pi}T_{\pi}u$, by 365D(d-ii).
     
\medskip
     
{\bf (b)} From 365Od, we know that $T_{\pi}$ is
surjective, while $P_{\pi}T_{\pi}$ is the identity, so that
$P_{\pi}=T_{\pi}^{-1}$, $T_{\pi}=P_{\pi}^{-1}$.
As for $T_{\pi^{-1}}$, 365Oe tells us that
$T_{\pi^{-1}}=T_{\pi}^{-1}$;  so
     
\Centerline{$P_{\pi^{-1}}=T_{\pi^{-1}}^{-1}=T_{\pi}$.}
     
}%end of proof of 365Q
     
\leader{365R}{Conditional expectations}\cmmnt{ It is a nearly
universal rule that any investigation of $L^1$ spaces must include a
look at conditional expectations.   In the present context, they take
the following form.
     
\medskip
     
}{\bf (a)} Let $(\frak A,\bar\mu)$ be a probability algebra
and $\frak B$ a closed subalgebra;  write $\bar\nu$ for the restriction
$\bar\mu\restrp\frak B$.   The identity map from $\frak B$ to $\frak A$
induces operators $T:L^1(\frak B,\bar\nu)\to L^1(\frak A,\bar\mu)$ and
$P:L^1(\frak A,\bar\mu)\to L^1(\frak B,\bar\nu)$.   If we take
$L^0(\frak A)$ to be
defined as the set of functions from $\Bbb R$ to $\frak A$ described in
364Aa, then $L^0(\frak B)$
becomes a subset of $L^0(\frak A)$ in the literal sense, and $T$ is
actually the identity operator associated with the subset
$L^1(\frak B,\bar\nu)\subseteq L^1(\frak A,\bar\mu)$;
$L^1(\frak B,\bar\nu)$ is a
norm-closed and order-closed Riesz subspace of $L^1(\frak A,\bar\mu)$.
$P$ is a positive linear operator, while $PT$ is the identity, so $P$ is
a projection from $L^1(\frak A,\bar\mu)$ onto $L^1(\frak B,\bar\nu)$.
$P$ is defined by the familiar formula
     
\Centerline{$\int_bPu=\int_bu$ for every
$u\in L^1(\frak A,\bar\mu)$, $b\in\frak B$,}
     
\noindent so is the conditional expectation operator in the sense of
242J.   
Observe that the formula in 365A now tells us that
$L^1(\frak B,\bar\nu)$ is just $L^1(\frak A,\bar\mu)\cap L^0(\frak B)$.
Translating 233K into this language, we see that
$P(u\times v)=Pu\times v$ whenever $u\in L^1(\frak A,\bar\mu)$,
$v\in L^0(\frak B)$ and $u\times v\in L^1(\frak A,\bar\mu)$\dvAnew{2012}.

     
\spheader 365Rb\cmmnt{ Just as in 233I-233J and 242K, we have a version of
Jensen's inequality.}  Let $h:\Bbb R\to\Bbb R$ be a convex
function and $\bar h:L^0(\frak A)\to L^0(\frak A)$ the corresponding map
(364H).   If
$u\in L^1(\frak A,\bar\mu)$, then $h(\int u)\le\int\bar h(u)$;  and if
$\bar h(u)\in L^1(\frak A,\bar\mu)$, then $\bar h(Pu)\le P(\bar h(u))$.
\prooflet{\Prf\ I repeat the proof of 233I-233J. For each $q\in\Bbb Q$,
take $\beta_q\in\Bbb R$ such that $h(t)\ge h_q(t)=h(q)+\beta_q(t-q)$ for
every $t\in\Bbb R$, so that
$h(t)=\sup_{q\in\Bbb Q}h_q(t)$ for every $t\in\Bbb R$, and
$\bar h(u)=\sup_{q\in\Bbb Q}\bar h_q(u)$ for every
$u\in L^0(\frak A)$.   (This is because
     
$$\eqalign{\Bvalue{\bar h(u)>\alpha}
&=\Bvalue{u\in h^{-1}[\,\ooint{\alpha,\infty}\,]}
=\Bvalue{u\in\bigcup_{q\in\Bbb Q}h_q^{-1}[\,\ooint{\alpha,\infty}\,]}\cr
&=\sup_{q\in\Bbb Q}\Bvalue{u\in h_q^{-1}[\,\ooint{\alpha,\infty}\,]}
=\sup_{q\in\Bbb Q}\Bvalue{\bar h_q(u)>\alpha}\cr}$$
     
\noindent for every $\alpha\in\Bbb R$.)   But setting $e=\chi 1$, we see
that $\bar h_q(u)=h(q)e+\beta_q(u-qe)$ for every $u\in L^0(\frak A)$, so
that
     
\Centerline{$\int\bar h_q(u)=h(q)+\beta_q(\int u-q)=h_q(\int u)$,}
     
\Centerline{$P(\bar h_q(u))=h(q)e+\beta_q(Pu-qe)=\bar h_q(Pu)$}
     
\noindent because $\int e=1$ and $Pe=e$.   Taking the
supremum over $q$, we get
     
\Centerline{$h(\int u)
=\sup_{q\in\Bbb Q}h_q(\int u)
=\sup_{q\in\Bbb Q}\int\bar h_q(u)
\le\int\bar h(u)$,}
     
\noindent and if $\bar h(u)\in L^1_{\bar\mu}$ then
     
\Centerline{$\bar h(Pu)
=\sup_{q\in\Bbb Q}\bar h_q(Pu)
=\sup_{q\in\Bbb Q}P(\bar h_q(u))
\le P(\bar h(u))$.   \Qed}}%end of prooflet
     
\cmmnt{Of course the result in this form can also be deduced from
233I-233J if we represent $(\frak A,\bar\mu)$ as the measure algebra of
a probability space $(X,\Sigma,\mu)$ and set
$\Tau=\{E:E\in\Sigma,\,E^{\ssbullet}\in\frak B\}$.}
     
\spheader 365Rc\cmmnt{ I note here a fact which is occasionally 
useful.}   If
$u\in L^1(\frak A,\bar\mu)$ is non-negative, then
$\Bvalue{Pu>0}=\upr(\Bvalue{u>0},\frak B)$, the upper envelope of
$\Bvalue{u>0}$ in $\frak B$ as defined in 313S.   \prooflet{\Prf\ We
have only to observe that, for $b\in\frak B$,
     
$$\eqalign{b\Bcap\Bvalue{Pu>0}=0
&\iff\chi b\times Pu=0
\iff\int_bPu=0\cr
&\iff\int_bu=0
\iff b\Bcap\Bvalue{u>0}=0.\cr}$$
     
\noindent Taking complements, $b\Bsupseteq\Bvalue{Pu>0}$ iff
$b\Bsupseteq\Bvalue{u>0}$.\ \Qed}
     
\spheader 365Rd\dvAnew{2012} Suppose now that $(\frak C,\bar\lambda)$ is
another probability algebra and $\pi:\frak A\to\frak C$ is a
measure-preserving Boolean homomorphism.   Then $\frak D=\pi[\frak B]$
is a closed subalgebra of $\frak C$\cmmnt{ (314F(a-i))}.   Let 
$Q:L^1(\frak C,\bar\lambda)\to L^1(\frak D,\bar\lambda\restrp\frak D)
\subseteq L^1(\frak C,\bar\lambda)$ 
be the conditional expectation associated with $\frak D$, and
$T_{\pi}:L^1(\frak A,\bar\mu)\to L^1(\frak C,\bar\lambda)$ 
the norm-preserving Riesz homomorphism defined by $\pi$.   
Then $T_{\pi}P=QT_{\pi}$.\prooflet{ \Prf\ Take $u\in L^1(\frak A,\bar\mu)$.
Then 

\Centerline{$\Bvalue{T_{\pi}Pu>\alpha}
=\pi\Bvalue{Pu>\alpha}\in\pi[\frak B]=\frak D$}

\noindent for every $\alpha\in\Bbb R$, so $T_{\pi}Pu\in L^0(\frak D)$.
If $d\in\frak D$, set $b=\pi^{-1}d\in\frak B$;  then

$$\eqalign{\int_dT_{\pi}Pu
&=\int T_{\pi}Pu\times\chi d
=\int T_{\pi}Pu\times T_{\pi}\chi b
=\int T_{\pi}(Pu\times\chi b)\cr
&=\int Pu\times\chi b
=\int_bPu
=\int_bu
=\int u\times\chi b\cr
&=\int T_{\pi}(u\times\chi b)
=\int T_{\pi}u\times T_{\pi}\chi b
=\int T_{\pi}u\times\chi d
=\int_d T_{\pi}u.\cr}$$

\noindent As $d$ is arbitrary, $T_{\pi}Pu$ satisfies the defining formula
for $QT_{\pi}u$ and $T_{\pi}Pu=QT_{\pi}u$;  as $u$ is arbitrary,
$T_{\pi}P=QT_{\pi}$.\ \Qed}
     
\leader{365S}{Recovering the algebra:  Proposition} (a) Let
$(\frak A,\bar\mu)$ be a localizable measure algebra.   Then $\frak A$
is isomorphic to the band algebra of $L^1(\frak A,\bar\mu)$.
     
(b) Let $\frak A$ be a Dedekind $\sigma$-complete Boolean algebra, and
$\bar\mu$, $\bar\nu$ two measures on $\frak A$ such that
$(\frak A,\bar\mu)$ and $(\frak A,\bar\nu)$ are both semi-finite measure
algebras.   Then $L^1(\frak A,\bar\mu)$ is isomorphic, as Banach
lattice, to $L^1(\frak A,\bar\nu)$.
     
\proof{{\bf (a)} Because $(\frak A,\bar\mu)$ is semi-finite,
$L^1_{\bar\mu}$ is order-dense in $L^0=L^0(\frak A)$ (365G).
Consequently, $L^1_{\bar\mu}$ and
$L^0$ have isomorphic band algebras (353D).   But the band algebra of
$L^0$ is just its algebra of projection bands (because
$\frak A$ and therefore $L^0$ are Dedekind complete, see 364M and 353I),
which is isomorphic to $\frak A$ (364O).
     
\medskip
     
{\bf (b)} Let $\pi:\frak A\to\frak A$ be the identity map.   Regarding
$\pi$ as an order-continuous Boolean homomorphism from
$\frak A^f_{\bar\mu}=\{a:\bar\mu a<\infty\}$ to $(\frak A,\bar\nu)$, we
have an associated positive linear
operator $P=P_{\pi}:L^1_{\bar\nu}\to L^1_{\bar\mu}$;
similarly, we have
$Q=P_{\pi^{-1}}:L^1_{\bar\mu}\to L^1_{\bar\nu}$, and both
$P$ and $Q$ have
norm at most $1$ (365Pb).   Now 365Pf assures us that $QP$ is the
identity operator on $L^1_{\bar\nu}$ and $PQ$ is the identity
operator on $L^1_{\bar\mu}$.   So $P$ and $Q$ are the two halves
of a Banach lattice isomorphism between $L^1_{\bar\mu}$ and
$L^1_{\bar\nu}$.
}%end of proof of 365S
     
\leader{365T}{}\cmmnt{ Having opened the question of varying measures
on a single Boolean algebra, this seems an appropriate moment
for a general description of how they interact.
     
\medskip
     
\noindent}{\bf Proposition} Let $\frak A$ be a Dedekind complete Boolean
algebra, and $\bar\mu:\frak A\to[0,\infty]$,
$\bar\nu:\frak A\to[0,\infty]$ two functions such that
$(\frak A,\bar\mu)$ and 
$(\frak A,\bar\nu)$ are both semi-finite\cmmnt{ (therefore localizable)} measure algebras.
     
(a) There is a
unique $u\in L^0=L^0(\frak A)$ such that\cmmnt{ (if we allow $\infty$
as a value of the integral)} $\int_au\,d\bar\mu=\bar\nu a$
for every $a\in\frak A$.
     
(b)\dvArevised{2012} For $v\in L^0(\frak A)$, 
$\int v\,d\bar\nu=\int u\times v\,d\bar\mu$ if either is defined in
$[-\infty,\infty]$.
     
(c) $u$ is strictly positive\cmmnt{ (i.e.,
$\Bvalue{u>0}=1$)} and, writing $\bover{1}{u}$ for the multiplicative
inverse of $u$, $\int_a\bover{1}{u}d\bar\nu=\bar\mu a$ for every
$a\in\frak A$.
     
\proof{{\bf (a)} Because $(\frak A,\bar\nu)$ is semi-finite, there is a
partition
of unity $D\subseteq \frak A$ such that $\bar\nu d<\infty$ for every
$d\in D$.   For each $d\in D$, the functional
$a\mapsto\bar\nu(a\Bcap d):\frak A\to\Bbb R$ is completely additive, so
there is a $u_d\in L^1_{\bar\mu}$ such that
$\int_au_dd\bar\mu=\bar\nu(a\Bcap d)$ for every $a\in\frak A$.   Because
$\int_au_dd\bar\mu\ge 0$ for every $a$, $u_d\ge 0$.   Because
$\int_{1\Bsetminus d}u_d=0$, $\Bvalue{u_d>0}\Bsubseteq d$.   Now
$u=\sup_{d\in D}u_d$ is defined in $L^0$.   \Prf\ (This is a special
case of 368K below.)
For $n\in\Bbb N$, set $c_n=\sup_{d\in D}\Bvalue{u_d>n}$.   If $d$,
$d'\in D$ are distinct, then $d\Bcap\Bvalue{u_{d'}>n}=0$, so
$d\Bcap c_n=\Bvalue{u_d>n}$.   Set $c=\inf_{n\in\Bbb N}c_n$.   If
$d\in D$, then
     
\Centerline{$d\Bcap c
=\inf_{n\in\Bbb N}d\Bcap c_n
=\inf_{n\in\Bbb N}\Bvalue{u_d>n}
=0$.}
     
\noindent But $c\Bsubseteq c_0\Bsubseteq\sup D$, so $c=0$.   By 364L(a-i),
$\{u_d:d\in D\}$ is bounded above in $L^0$, so has a supremum, because
$L^0$ is Dedekind complete, by 364M.\ \Qed
     
For finite
$I\subseteq D$ set $\tilde u_I=\sum_{d\in I}u_d=\sup_{d\in I}u_d$
(because $u_d\wedge u_c=0$ for distinct $c$, $d\in D$).   Then
$u=\sup\{\tilde u_I:I\subseteq D,\,I$ is
finite$\}$.
So, for any $a\in\frak A$,
     
$$\eqalignno{\int_au\,d\bar\mu
&=\sup_{I\subseteq D\text{ is finite}}\int_a\tilde u_Id\bar\mu\cr
\noalign{\noindent (365Dh)}
&=\sup_{I\subseteq D\text{ is finite}}\sum_{d\in
I}\int_au_dd\bar\mu
=\sup_{I\subseteq D\text{ is finite}}\sum_{d\in I}\bar\nu(a\Bcap
d)
=\bar\nu a.\cr}$$
     
Note that if $a\in\frak A$ is non-zero, then $\bar\nu a>0$, so
$a\Bcap\Bvalue{u>0}\ne 0$;  consequently $\Bvalue{u>0}=1$.
     
To see that $u$ is unique, observe that if $u'$ has the same property
then for any $d\in D$
     
\Centerline{$\int_au\times\chi d\,d\bar\mu
=\bar\nu(a\Bcap d)=\int_au'\times\chi d\,d\bar\mu$}
     
\noindent for every $a\in\frak A$, so that
$u\times\chi d=u'\times\chi d$;  because $\sup D=1$ in $\frak A$, $u$
must be equal to $u'$.
     
\medskip
     
{\bf (b)} Use 365Hb, with $\pi$ and $T$ the identity maps.
     
\medskip
     
{\bf (c)} In the same way, there is a $w\in L^0$ such that
$\int_aw\,d\bar\nu=\bar\mu a$ for every $a\in\frak A$.   To relate $u$
and $w$, observe that applying 365Hb we get
     
\Centerline{$\int w\times\chi a\times u\,d\bar\mu
=\int w\times\chi a\,d\bar\nu$}
     
\noindent for every $a\in\frak A$, that is, $\int_aw\times
u\,d\bar\mu=\bar\mu a$ for every $a$.   But from this we see that
$w\times u\times\chi b=\chi b$ at least when $\bar\mu b<\infty$, so that
$w\times u=\chi 1$ is the multiplicative identity of $L^0$, and
$w=\bover1{u}$.
}%end of proof of 365T

\leader{365U}{Uniform \dvrocolon{integrability}}\dvAnew{2009}\cmmnt{ 
Continuing the
programme in 365C, we can transcribe the ideas of  \S\S246, 247, 354 and
356 into the new context.   

\medskip

\noindent}{\bf Theorem} Let $(\frak A,\bar\mu)$ be a measure algebra.
Set $L^1=L^1(\frak A,\bar\mu)$.

(a) For a non-empty subset $A$ of $L^1$, the following are equiveridical:

\quad(i) $A$ is uniformly integrable in the sense of 354P;

\quad(ii) for every $\epsilon>0$ there are an $a\in\frak A^f$ and an 
$M\ge 0$ such that $\int(u-M\chi a)^+\le\epsilon$ for every $u\in\frak A$;

\quad(iii)($\alpha$) 
$\sup_{u\in A}|\int_au|$ is finite for every atom $a\in\frak A$,

\qquad($\beta$) for every $\epsilon>0$ there are $c\in\frak A^f$ and
$\delta>0$ such that $|\int_au|\le\epsilon$ whenever $u\in A$,
$a\in\frak A$ and $\bar\mu(a\Bcap c)\le\delta$;

\quad(iv)($\alpha$) 
$\sup_{u\in A}|\int_au|$ is finite for every atom $a\in\frak A$,

\qquad($\beta$) $\lim_{n\to\infty}\sup_{u\in A}|\int_{a_n}u|=0$ for
every disjoint sequence $\sequencen{a_n}$ in $\frak A$;

\quad(v) $A$ is relatively weakly compact in $L^1$.

(b) If $(\frak A,\bar\mu)$ is a probability algebra and $A\subseteq L^1$  
is uniformly integrable, then there is a solid convex norm-closed uniformly
integrable set $C\supseteq A$ such that $P[C]\subseteq C$ whenever
$P:L^1\to L^1$ is the conditional expectation operator associated with a
closed subalgebra of $\frak A$.

\proof{ 354Q, 354R, 356Q and 246D, with a little help from 246C and 246G.
}%end of proof of 365U
     
\exercises{\leader{365X}{Basic exercises (a)}
%\spheader 365Xa
Let $(\frak A,\bar\mu)$ be a measure algebra, and $u\in L^1_{\bar\mu}$.
Show that
     
\Centerline{$\int
u=\int_0^{\infty}\bar\mu\Bvalue{u>\alpha}\,d\alpha
  -\int_{-\infty}^0\bar\mu(1\Bsetminus\Bvalue{u>\alpha})\,d\alpha$.}
%365D
     
\sqheader 365Xb Let $(\frak A,\bar\mu)$ be any measure algebra, and
$u\in L^1_{\bar\mu}$.   (i) Show that
$\|u\|_1\le 2\sup_{a\in\frak A^f}|\int_au|$.   \Hint{246F.}   (ii) Show
that for any $\epsilon>0$ there is an $a\in\frak A^f$ such that
$|\int u-\int_bu|\le\epsilon$ whenever $a\Bsubseteq b\in\frak A$.
%365D
     
\sqheader 365Xc Let $U$ be an $L$-space.   If $\sequencen{u_n}$ is any
norm-bounded sequence in $U^+$, show that

\Centerline{$\liminf_{n\to\infty}u_n
=\sup_{n\in\Bbb N}\inf_{m\ge n}u_m$}
%\Centerline needed for Lulu version

\noindent is defined in $U$, and that
$\int\liminf_{n\to\infty}u_n\le\liminf_{n\to\infty}\int u_n$.
%365D
     
\spheader 365Xd Let $U$ be an $L$-space.   Let $\Cal F$ be a filter on
$U$ such that $\{u:u\ge 0,\,\|u\|\le k\}$ belongs to $\Cal F$ for some
$k\in\Bbb N$.   Show that $u_0=\sup_{F\in\Cal F}\inf F$ is defined in
$U$, and that $\int u_0\le\sup_{F\in\Cal F}\inf_{u\in F}\int u$.
%365D
     
\spheader 365Xe Let $(\frak A,\bar\mu)$ be a measure algebra and
$A\subseteq L^1_{\bar\mu}$ a non-empty set.   Show that $A$ is bounded
above in $L^1_{\bar\mu}$ iff
     
\Centerline{$\sup\{\sum_{i=0}^n\int_{a_i}u_i:a_0,\ldots,a_n$ is a
partition of unity in $\frak A$, $u_0,\ldots,u_n\in A\}$}
     
\noindent is finite, and that in this case the supremum is $\int\sup A$.
\Hint{given $u_0,\ldots,u_n\in A$, set $b_{ij}=\Bvalue{u_i\ge u_j}$,
$b_i=\sup_{j\ne i}b_{ij}$, $a_i=b_i\Bsetminus\sup_{j<i}b_j$, and show
that $\int\sup_{i\le n}u_i=\sum_{i=0}^n\int_{a_i}u_i$.}
%365D
     
\spheader 365Xf Let $(\frak A,\bar\mu)$ be any measure algebra and
$\nu:\frak A^f\to\Bbb R$ a bounded additive functional.   Show that the
following are equiveridical:  (i) there is a $u\in L^1_{\bar\mu}$ such
that
$\nu a=\int_au$ for every $a\in\frak A^f$;  (ii) for every $\epsilon>0$
there is a $\delta>0$ such that $|\nu a|\le\epsilon$ whenever $\bar\mu
a\le\delta$;  (iii) for every $\epsilon>0$, $c\in\frak A^f$ there is a
$\delta>0$ such that $\nu a\le\epsilon$ whenever $a\Bsubseteq c$ and
$\bar\mu a\le\delta$;  (iv) for every $\epsilon>0$ there are $c\in\frak
A^f$, $\delta>0$ such that $|\nu a|\le\epsilon$ whenever $a\in\frak A^f$
and $\bar\mu(a\Bcap c)\le\delta$;  (v) $\lim_{n\to\infty}\nu a_n=0$
whenever $\sequencen{a_n}$ is a
non-increasing sequence in $\frak A^f$ with infimum $0$.
%365E
     
\spheader 365Xg Let $(\frak A,\bar\mu)$ and $(\frak B,\bar\nu)$ be
measure algebras, and $\pi:\frak A\to\frak B$ a sequentially
order-continuous Boolean homomorphism.   Let
$T:L^0(\frak A)\to L^0(\frak B)$ be the Riesz homomorphism associated
with $\pi$ (364P).
Suppose that $w\ge 0$ in $L^0(\frak B)$ is such that
$\int_{\pi a}w\,d\bar\nu=\bar\mu a$ whenever $a\in\frak A$.   Show that
for any $u\in L^0(\frak A,\bar\mu)$,
$\int Tu\times w\,d\bar\nu=\int u\,d\bar\mu$ whenever either is defined
in $[-\infty,\infty]$.
%365H
     
\sqheader 365Xh Let $(\frak A,\bar\mu)$ be a measure algebra and
$a\in\frak A$;  write $\frak A_a$ for the principal ideal it generates.
Show that if $\pi$ is the identity embedding of $\frak A^f\cap\frak A_a$
into $\frak A^f$, then $T_{\pi}$, as defined in 365O, identifies
$L^1(\frak A_a,\bar\mu\restrp\frak A_a)$ with a band in
$L^1_{\bar\mu}$.
%365O
     
\sqheader 365Xi Let $(X,\Sigma,\mu)$ and $(Y,\Tau,\nu)$ be measure
spaces, with measure algebras $(\frak A,\bar\mu)$ and
$(\frak B,\bar\nu)$.   Let $\phi:X\to Y$ be an \imp\ function and
$\pi:\frak B\to\frak A$ the corresponding
measure-preserving homomorphism (324M).   Show that
$T_{\pi}:L^1_{\bar\nu}\to L^1_{\bar\mu}$ (365O) corresponds to the map
$g^{\ssbullet}\mapsto(g\phi)^{\ssbullet}:L^1(\nu)\to L^1(\mu)$ of
242Xd.
%365O
     
\spheader 365Xj Let $(\frak A,\bar\mu)$ and $(\frak B,\bar\nu)$
be measure algebras.   Let $\pi:\frak A^f\to\frak B^f$ be a
ring homomorphism such that, for some $\gamma>0$, $\bar\nu(\pi
a)\le\gamma\bar\mu a$ for every $a\in\frak A^f$.  (i) Show that there is
a unique order-continuous Riesz
homomorphism $T:L^1_{\bar\mu}\to L^1_{\bar\nu}$ such that
$T(\chi a)=\chi(\pi a)$ whenever $a\in\frak A^f$, and that
$\|T\|\le\gamma$.   (ii) Show that
$\Bvalue{Tu>\alpha}=\pi\Bvalue{u>\alpha}$ for every
$u\in L^1_{\bar\mu}$, $\alpha>0$.
(iii) Show that $T$ is surjective iff $\pi$ is, injective iff $\pi$ is.
(iv) Show that $T$ is norm-preserving iff $\bar\nu(\pi a)=\bar\mu a$ for
every $a\in\frak A^f$.
%365O
     
\spheader 365Xk Let $(\frak A,\bar\mu)$ and $(\frak B,\bar\nu)$ be
measure algebras, and $\pi:\frak A\to\frak B$ a measure-preserving
Boolean homomorphism.   Let $T:L^1_{\bar\mu}\to L^1_{\bar\nu}$ and
$P:L^1_{\bar\nu}\to L^1_{\bar\mu}$ be the operators corresponding to
$\pi\restrp\frak A^f$, as described in 365O-365P, and
$\tilde T:L^{\infty}(\frak A)\to L^{\infty}(\frak B)$ the operator
corresponding to $\pi$, as described in 363F.   (i) Show that
$T(u\times v)=Tu\times \tilde Tv$ for every $u\in L^1_{\bar\mu}$,
$v\in L^{\infty}(\frak A)$.
(ii) Show that if $\pi$ is order-continuous, then $\int Pv\times u=\int
v\times\tilde Tu$ for every $u\in L^{\infty}(\frak A)$, $v\in
L^1_{\bar\nu}$.
%365O
     
\sqheader 365Xl Let $(X,\Sigma,\mu)$ be a probability space, with
measure algebra $(\frak A,\bar\mu)$, and let $\Tau$ be a
$\sigma$-subalgebra of $\Sigma$.   Set $\nu=\mu\restrp\Tau$, $\frak
B=\{F^{\ssbullet}:F\in\Tau\}\subseteq\frak A$,
$\bar\nu=\bar\mu\restrp\frak B$, so that
$(\frak B,\bar\nu)$ is a measure algebra.   Let $\pi:\frak B\to\frak A$
be the identity homomorphism.    Show that
$T_{\pi}:L^1_{\bar\nu}\to L^1_{\bar\mu}$ (365O) corresponds to the
canonical
embedding of $L^1(\nu)$ in $L^1(\mu)$ described in 242Jb, while
$P_{\pi}:L^1_{\bar\mu}\to L^1_{\bar\nu}$ (365P)
corresponds to the conditional expectation operator described in
242Jd.
%365P
     
\spheader 365Xm Let $(\frak A,\bar\mu)$ be a semi-finite measure
algebra, and $(\widehat{\frak A},\hat\mu)$ its localization (322Q).
Show that the natural embedding of $\frak A$ in $\widehat{\frak A}$
induces a Banach lattice isomorphism between $L^1_{\bar\mu}$ and
$L^1_{\hat\mu}$, so that the band algebra of
$L^1_{\bar\mu}$ can be identified with the Dedekind completion
$\widehat{\frak A}$ of $\frak A$.
%365O, 365S
     
\spheader 365Xn Let $\frak A$ be a Dedekind $\sigma$-complete Boolean
algebra and $\bar\mu$, $\bar\nu$ two functions such that $(\frak
A,\bar\mu)$, $(\frak A,\bar\nu)$ are measure algebras.   Show that
$L^1_{\bar\mu}\subseteq L^1_{\bar\nu}$ (as subsets of $L^0(\frak A)$)
iff there is a $\gamma>0$ such that $\bar\nu a\le\gamma\bar\mu a$ for
every $a\in\frak A$.   \Hint{show that the identity operator from
$L^1_{\bar\mu}$ to $L^1_{\bar\nu}$ is bounded.}
%365S
     
\spheader 365Xo Let $(\frak A,\bar\mu)$ be a measure algebra,
$I_{\infty}$ the ideal of `purely infinite' elements of $\frak A$
together with $0$,
$\bar\mu_{sf}$ the measure on $\frak B=\frak A/I_{\infty}$ (322Xa).
Let $\pi:\frak A\to\frak B$ be the canonical map.   Show that $T_{\pi}$,
as defined in 365O, is a Banach lattice isomorphism between
$L^1_{\bar\mu}$ and $L^1(\frak B,\bar\mu_{sf})$.
%365O, 365Xm, 365S
     
\spheader 365Xp Let $(X,\Sigma,\mu)$ be a a semi-finite measure space.
Show that $L^1(\mu)$ is separable iff $\mu$ is $\sigma$-finite and has
countable Maharam type.
%365S

\spheader 365Xq Let $(\frak A,\bar\mu)$ and $(\frak B,\bar\nu)$ be
probability algebras, $\pi:\frak A\to\frak B$ a measure-preserving Boolean
homomorphism, and $T:L^0(\frak A)\to L^0(\frak B)$ the corresponding Riesz
homomorphism.   Let $\frak C$ be a closed subalgebra of $\frak A$ and
$P:L^1(\frak A,\bar\mu)\to L^1(\frak C,\bar\mu\restrp\frak C)
\subseteq L^1(\frak A,\bar\mu)$, 
$Q:L^1(\frak B,\bar\nu)\to L^1(\frak B,\bar\nu)$ the conditional
expectation operators defined from $\frak C\embedsinto\frak A$
and $\pi[\frak C]\embedsinto\frak B$.   Show that $TP=QT$.
%365R
     
\leader{365Y}{Further exercises (a)}
%\spheader 365Ya
Let $(\frak A,\bar\mu)$ be a semi-finite measure
algebra, not $\{0\}$.   Show that the topological density of
$L^1_{\bar\mu}$ (331Yf) is
$\max(\omega,\tau(\frak A),c(\frak A))$, where
$\tau(\frak A)$, $c(\frak A)$ are the Maharam type and cellularity of
$\frak A$.
%365Xp, 365S
     
}%end of exercises
     
\cmmnt{
\Notesheader{365} You should not suppose that $L^1$ spaces appear in the
second half of this chapter because they are of secondary importance.
Indeed I regard them as the most important of all function spaces.   I
have delayed the discussion of them for so long because it is here that
for the first time we need measure algebras in an essential way.
     
The actual definition of $L^1_{\bar\mu}$ which I give is designed
for speed rather than illumination;  I seek only a formula, visibly
independent of any particular representation of $(\frak A,\bar\mu)$ as
the measure algebra of a measure space, from which I can prove 365B.
365C-365D and 365Ea are now elementary.   In 365Eb I take a
page to describe a form of the Radon-Nikod\'ym theorem which is
applicable to arbitrary measure algebras, at the cost of dealing with
functionals on the ring $\frak A^f$ rather than on the whole algebra
$\frak A$.    This is less for the sake of applications than to
emphasize one of the central properties of $L^1$:  it depends only on
$\frak A^f$ and $\bar\mu\restrp\frak A^f$.   For alternative versions of
the condition 365Eb(i) see 365Xf.
     
The convergence theorems (B.Levi's theorem, Fatou's lemma and Lebesgue's
dominated convergence theorem) are so central to the theory of
integrable functions that it is natural to look for versions in the
language here.   Corresponding to B.Levi's theorem is the Levi property
of a norm in an $L$-space;  note how the abstract formulation makes it
natural to speak of general upwards-directed families rather than of
non-decreasing sequences, though the sequential form is so often used
that I have spelt it out (365C).   In the same way, the integral becomes
order-continuous rather than just sequentially order-continuous (365Da).
Corresponding to Fatou's lemma we have 365Xc-365Xd.   For abstract
versions of Lebesgue's theorem I will wait until \S367.
%367I 367Xf
     
In 365H I have deliberately followed the hypotheses of 235A and 235R.
Of course 365H can be deduced from these if we use the Stone
representations of $(\frak A,\bar\mu)$ and $(\frak B,\bar\nu)$, so that
$\pi$ can be represented by a function between the Stone spaces
(312Q).   But 365H is essentially simpler, because the technical
problems concerning measurability which took up so much of \S235 have
been swept under the carpet.   In the same way, 365Xg corresponds to
235E.   Here we have a fair example of the way in which the abstract
expression in terms of measure algebras can be tidier than the
expression in terms of measure spaces.   But in my view this is because
here, at least, some of the mathematics has been left out.
     
365I-365K %365I 365J 365K
correspond closely to 361F-361H %361F 361G 361H
and 363E.
365M is a re-run of 243G, but with the additional refinement that I
examine the action of $L^1$ on $L^{\infty}$ (the operator $S$) as well
as the action of $L^{\infty}$ on $L^1$ (the operator $T$).   Of course
365Mc is just the abstract version of 243Hb, and can easily be proved
from it.   Note that while the proof of 365M does not itself involve any
representation of $(\frak A,\bar\mu)$ as the measure algebra of a
measure space, (a-vii) and (b-iii) of the proof of 365M depend on the
Radon-Nikod\'ym theorem through 327D and 365E.   For a development of
the theory of $L^1(\frak A,\bar\mu)$ which does not (in a formal sense)
depend on measure spaces, see {\smc Fremlin 74a}, 63J.
     
Theorems 365O-365Q lie at the centre of my picture of $L^1$ spaces, and
are supposed to show their dual nature.   Starting from a semi-finite
measure algebra $(\frak A,\bar\mu)$ we have two essentially different
routes to the $L^1$-space:  we can either build it up from
indicator functions of elements of finite measure, so that it is
naturally embedded in $L^0(\frak A)$, or we can think of it as the
order-continuous dual of $L^{\infty}(\frak A)$.   The first is a
`covariant' construction (signalled by the formula
$T_{\theta\pi}=T_{\theta}T_{\pi}$ in 365Oe) and the second is
`contravariant' (so that $P_{\theta\pi}=P_{\pi}P_{\theta'}$ in
365Pf).   The first construction is the natural one if we are
seeking to copy the ideas of \S242, but the second arises inevitably if
we follow the ordinary paths of functional analysis and study dual
spaces whenever they appear.   The link between them is the
Radon-Nikod\'ym theorem.
     
I have deliberately written out 365O and 365P with different
hypotheses on the homomorphism $\pi$ in the hope of showing that the two
routes to $L^1$ really are different, and can be expected to tell us
different things about it.
I use the letter $P$ in 365P in order to echo the language of 242J:
in the most important context, in which $\frak A$ is actually a
subalgebra of $\frak B$ and $\pi$ is the identity map, $P$ is a kind of
conditional expectation operator (365R).   I note that in the proof of
365Pe I have returned to first principles, using some of the ideas of
the Radon-Nikod\'ym theorem (232E), but a different approach to the
exhaustion step (converting `for every $u>0$ there is a $v>0$ such
that $Pv\le u$' into `$P$ is surjective').   I chose the somewhat
cruder method in 232E (part (c) of the proof) in order to use the
weakest possible form of the axiom of choice.   In the present context
such scruples seem absurd.
     
I used the words `covariant' and `contravariant' above;  of course
this distinction depends on the side of the mirror on which we are
standing;  if our measure-preserving homomorphism is derived
(contravariantly) from an inverse-measure-preserving transformation,
then the $T$'s become contravariant (365Xi).   An important
component of this work, for me, is the fact that not all
measure-preserving homomorphisms between measure algebras can be
represented by inverse-measure-preserving functions (343Jb, 343M).
     
I have already remarked (in the notes to \S244) that the properties of
$L^1(\mu)$ are not much affected by peculiarities in a measure space
$(X,\Sigma,\mu)$.   In this section I offer an explanation:  unlike
$L^0$ or $L^{\infty}$, $L^1$ really
depends only on $\frak A^f$, the ring of elements of finite measure in
the measure algebra.   (See 365O-365Q, %365O 365P 365Q
365Xm and 365Xo.) %365Xm 365Xn 365Xo
Note that while the
algebra $\frak A$ is uniquely determined (given that $(\frak A,\bar\mu)$
is localizable, 365Sa), the measure $\bar\mu$ is not;  if $\frak A$ is
any algebra carrying two non-isomorphic semi-finite measures, the
corresponding $L^1$ spaces are still isomorphic (365Sb).   For instance,
the $L^1$-spaces of Lebesgue measure $\mu$ on $\Bbb R$, and the subspace
measure $\mu_{[0,1]}$ on $[0,1]$, are isomorphic, though their measure
algebras are not.
     
I make no attempt here to add to the results in \S\S246, 247, 354 and
356 concerning
uniform integrability and weak compactness.   Once we have left measure
spaces behind, these ideas belong to the theory of Banach lattices, and
there is little to relate them to the questions dealt with in this
section.   But see 373Xj and 373Xn below.
}%end of notes
     
\discrpage
     
