\frfilename{mt547.tex}
\versiondate{26.8.14}
\copyrightdate{2005}

\def\chaptername{Real-valued-measurable cardinals}
\def\sectionname{Disjoint refinements of sequences of sets}

\newsection{547}

I continue my account of results from {\smc Gitik \& Shelah 01} and
{\smc Burke n96}.   Given a family $\familyiI{A_i}$ of sets
in a measure space, when can we find a disjoint family $\familyiI{A'_i}$
such that $A'_i\subseteq A_i$ has the same outer measure as $A_i$ for
every $i$?   A partial result is in Theorem 547F.   Allied questions are:
when can we find a set $D$ such that $A_i\cap D$ and $A_i\setminus D$ have
the same outer measure as $A_i$ for every $i$?  (547G) or are just
non-negligible?  (547I).

\leader{547A}{Lemma} Let $(X,\Sigma,\mu)$ be a totally finite
measure space with Maharam type $\tau(\mu)$.   Suppose that
$A\subseteq X$ is such that $\min(\mu^*D,\mu^*(A\setminus D))<\mu^*A$ for
every $D\subseteq A$.

(a) There is a non-negligible set $B\subseteq A$ such that the completion
$\hat\mu_B$ of the subspace measure on $B$ measures every subset of $B$.

(b) If $\mu$ is atomless, there is an \am\ cardinal $\kappa$ such that
$\min(\kappa^{(+\omega)},2^{\kappa})\penalty-100\le\tau(\mu)$.

(c) If $\mu$ is purely atomic and singletons in $A$ are negligible,
there is a \2vm\ cardinal $\kappa\le\#(A)$.

\proof{{\bf (a)(i)}
Let $\langle(F_i,D_i)\rangle_{i\in I}$ be a maximal family such that

\Centerline{$F_i\in\Sigma$,
\quad$\mu^*(F_i\cap A\cap D_i)=\mu^*(F_i\cap A\setminus D_i)=\mu F_i>0$}

\noindent for every $i$, and $\langle F_i\rangle_{i\in I}$ is disjoint.
Because $\mu F_i>0$ for every $i$, $I$ must be countable.   Set
$E=\bigcup_{i\in I}F_i$.   \Quer\ If $B=A\setminus E$ is negligible, set
$D=\bigcup_{i\in I}F_i\cap A\cap D_i$;  then

$$\eqalignno{\mu^*A
&=\mu^*(A\cap E)
=\sum_{i\in I}\mu^*(A\cap F_i)\cr
\displaycause{put 132Ae and 132Af together}
&=\sum_{i\in I}\mu F_i
=\sum_{i\in I}\mu^*(F_i\cap A\cap D_i)
=\sum_{i\in I}\mu^*(F_i\cap D)
=\mu^*D,\cr}$$

\noindent and similarly $\mu^*A=\mu^*(A\setminus D)$,
contrary to hypothesis.\ \BanG\   So $\mu^*B>0$.

\medskip

\quad{\bf (ii)} If $D\subseteq B$ then $\hat\mu_B$ measures $D$.
\Prf\ Let $H_1\subseteq X\setminus E$, $H_2\subseteq X\setminus E$
be measurable envelopes of $D$ and $B\setminus D$ respectively, and set
$F=H_1\cap H_2$.   Then

$$\eqalign{\mu^*(F\cap B\cap D)
&=\mu^*(F\cap D)
=\mu(F\cap H_1)
=\mu F\cr
&=\mu(F\cap H_2)=\mu^*(F\cap B\setminus D)=\mu^*(F\cap A\setminus D).\cr}$$

\noindent But $F\cap F_i=\emptyset$ for every $i\in I$, so we must have
$\mu F=0$.   Now $B\setminus H_2\subseteq D\subseteq B\cap H_1$ and
$\mu_B((B\cap H_1)\setminus(B\setminus H_2))=0$, so
$\hat\mu_B$ measures $D$.\
\Qed

\medskip

{\bf (b)} If $\mu$ is atomless, the measure algebra of $\hat\mu_B$ can be
identified with the measure algebra of $\mu_B$ (322Da) and a principal
ideal in the measure algebra of $\mu$
(322I), so is atomless, and $\tau(\hat\mu_B)\le\tau(\mu)$ (331Hc).
Setting $\kappa=\add\hat\mu_B$, $\kappa$ is an \am\ cardinal (543B)
and

\Centerline{$\min(\kappa^{(+\omega)},2^{\kappa})\le\tau(\hat\mu_B)
\le\tau(\mu)$}

\noindent by the Gitik-Shelah theorem (543F).

\medskip

{\bf (c)} If $\mu$ is purely atomic, so are $\mu_B$ (214Xd, or use the
method of (b)) and $\hat\mu_B$
(212Gd).   Also $\hat\mu_B\{x\}=0$ for every $x\in B$.
So $\add\hat\mu_B$ is \2vm\ (543B(f-i)), and is at most
$\#(B)\le\#(A)$.
}%end of proof of 547A

\leader{547B}{Lemma} Let $(X,\Sigma,\mu)$
be an atomless totally finite measure space, 
$\sequencen{A_n}$ a sequence of
subsets of $X$, and $A\subseteq X$ a non-negligible set.   Then

{\it either} ($\alpha$) there is a set $C\subseteq A$ such
that $\mu^*C>0$ and $\mu^*(A_n\setminus C)=\mu^*A_n$ for every $n\in\Bbb N$

{\it or} ($\beta$) there is a \qm\ cardinal less than the Maharam type of $\mu$.

\proof{ Suppose that ($\alpha$) is false.
If there is an \am\ cardinal less than the Maharam type of $\mu$, we
can stop;  so let us suppose that there is no such cardinal.   Since
we are supposing that $X$ has a non-negligible subset, $\mu X$ cannot be
$0$, and we can suppose that it is actually $1$.

\medskip

{\bf (a)} To begin with (down to the end of (e)) let us suppose that $\mu^*A=1$.
For $\sigma\in S_2=\bigcup_{n\in\Bbb N}\{0,1\}^n$ choose
$D_{\sigma}$, $E_{\sigma}$ and $D'_{\sigma}$ inductively, as follows.
Start with $D_{\emptyset}=A$.   Given that $D_{\sigma}\subseteq A$ and
$\mu^*D_{\sigma}=1$, where $\sigma\in\{0,1\}^n$, let $E_{\sigma}$ be a
measurable envelope of $D_{\sigma}\cap A_n$, and set
$D'_{\sigma}=(D_{\sigma}\cap A_n)\cup(D_{\sigma}\setminus E_{\sigma})$;
then $\mu^*D'_{\sigma}=1$.   By 547Ab, we can choose
$D_{\sigma^{\smallfrown}\fraction{0}}\subseteq D'_{\sigma}$ such that

\Centerline{$\mu^*D_{\sigma^{\smallfrown}\fraction{0}}
=\mu^*(D'_{\sigma}\setminus D_{\sigma^{\smallfrown}\fraction{0}})=1$,}

\noindent and set $D_{\sigma^{\smallfrown}\fraction{1}}
=D_{\sigma}\setminus D_{\sigma^{\smallfrown}\fraction{0}}$.   Then

\Centerline{$\mu^*D_{\sigma^{\smallfrown}\fraction{0}}
=\mu^*D_{\sigma^{\smallfrown}\fraction{1}}=1$.}

\noindent Continue.

\medskip

{\bf (b)} Set $Z=\{0,1\}^{\Bbb N}$, and for $\sigma\in S_2$ set
$Z_{\sigma}=\{z:\sigma\subseteq z\in Z\}$.   Define
$f:A\to X\times Z$ by
setting $f(x)=(x,\alpha)$, where $\alpha$ is that member of $Z$ such
that $x\in D_{\alpha\restr n}$ for every $n$.   Then for any
non-negligible $C\subseteq A$ there are a non-negligible measurable set
$E\subseteq X$ and a $\sigma\in S_2$ such that
$f^{-1}[E\times Z_{\sigma}]\subseteq C$.   \Prf\
Because ($\alpha$) is supposed to be false, there is some $n\in\Bbb N$ such that
$\mu^*(A_n\setminus C)<\mu^*A_n$.   Let $F$ be a measurable envelope of
$A_n\setminus C$, so
that $A_n\setminus F\subseteq C$ and $\mu^*(A_n\setminus F)>0$.   As
$A_n\setminus F\subseteq A$, there is some
$\sigma\in\{0,1\}^n$ such that $\mu^*(D_{\sigma}\cap A_n\setminus F)>0$.
Set $E=E_{\sigma}\setminus F$.   Now
$\mu E=\mu^*(D_{\sigma}\cap A_n\setminus F)>0$ and

\Centerline{$f^{-1}[E\times Z_{\sigma^{\smallfrown}\fraction{0}}]
=D_{\sigma^{\smallfrown}\fraction{0}}\cap E
\subseteq D'_{\sigma}\cap E_{\sigma}\setminus F
\subseteq D_{\sigma}\cap A_n\setminus F
\subseteq C$.   \Qed}

\medskip

{\bf (c)(i)} Write $\Cal L$ for
$(\Sigma\tensorhat\Cal B(Z))\cap(\Cal N(\mu)\ltimes\Cal M(Z))$,
where, as in \S546, $\Cal B(Z)$ is the Borel $\sigma$-algebra of $Z$,
$\Cal N(\mu)$ is the null ideal of $\mu$ and $\Cal M(Z)$ is
the $\sigma$-ideal of meager subsets of $Z$.
Let $\Cal W$ be the family of subsets
of $X\times Z$ expressible as
$\bigcup_{\sigma\in S_2}F_{\sigma}\times Z_{\sigma}$
where $F_{\sigma}\in\Sigma$ for every $\sigma\in S_2$,
and $\Cal V$ the set of those
$V\in\Cal W$ such that $V[\{x\}]$ is dense for every $x\in X$.

\medskip

\quad{\bf (ii)}
If $V\in\Cal W$ and $f^{-1}[(X\times Z)\setminus V]\notin\Cal N(\mu)$ then
$V\notin\Cal V$.
\Prf\ By (b), there are a non-negligible
$E\in\Sigma$ and a $\sigma\in\bigcup_{n\in\Bbb N}\{0,1\}^n$ such
that $f^{-1}[E\times Z_{\sigma}]\subseteq f^{-1}[(X\times Z)\setminus V]$, that
is, $E\cap D_{\sigma}$ does not meet $f^{-1}[V]$.
Express $V$ as $\bigcup_{\tau\in S_2}(F_{\tau}\times Z_{\tau})$ where every
$F_{\tau}$ is measurable.   Setting

\Centerline{$E'
=\bigcup\{E\cap F_{\tau}:\tau\in S_2$, $E\cap F_{\tau}$ is negligible$\}$,}

\noindent there must be some $x\in D_{\sigma}\cap E\setminus E'$.   If
$\tau\in S_2$ and $x\in F_{\tau}$, then $E\cap F_{\tau}$ is
non-negligible and meets $D_{\tau}$ in $y$ say.   But this means that
$y\in E$ and $f(y)\in F_{\tau}\times Z_{\tau}$ belongs to $V$;
so $y\notin D_{\sigma}$ and $\tau\not\supseteq\sigma$.   Thus
$V[\{x\}]\cap Z_{\sigma}$ is empty and $V[\{x\}]$ is not dense and
$V\notin\Cal V$.\ \Qed

\medskip

\quad{\bf (iii)} If $W\in\Cal L$ then $f^{-1}[W]\in\Cal N(\mu)$.
\Prf\ By 527I, we can find
$W_0\in\Cal W$ and a sequence $\sequencen{V_n}$ in $\Cal V$ such that
$(W\symmdiff W_0)\cap\bigcap_{n\in\Bbb N}V_n=\emptyset$.   
By (ii) just above,
$f^{-1}[(X\times Z)\setminus V_n]$ is negligible for every $n$, so
$f^{-1}[W]\symmdiff f^{-1}[W_0]$ is negligible.   
Also $(X\times Z)\setminus V_n$
belongs to $\Cal L$ for every $n$.   Accordingly $W\symmdiff W_0$ and
$W_0$ belong to $\Cal L$, so

\Centerline{$f^{-1}[W_0]
\subseteq\{x:W_0[\{x\}]\ne\emptyset\}=\{x:W_0[\{x\}]\notin\Cal M(Z)\}$}

\noindent is negligible, and $f^{-1}[W]$ is negligible.\ \Qed

\medskip

{\bf (d)} It follows that
$\Sigma\tensorhat\Cal B(Z)/\Cal L\cong\Cal PA/\Cal PA\cap\Cal N(\mu)$.
\Prf\ Since $f^{-1}[W]\in\Cal N(\mu)$ for every $W\in\Cal L$,
we have a Boolean homomorphism
$\pi:\Sigma\tensorhat\Cal B(Z)/\Cal L\to\Cal PA/\Cal PA\cap\Cal N(\mu)$
defined by setting
$\pi W^{\ssbullet}=(f^{-1}[W])^{\ssbullet}$ for every $W\in\Sigma\tensorhat\Cal B(Z)$.
$\pi$ is sequentially order-continuous, and because
$\Sigma\tensorhat\Cal B(Z)/\Cal L$ is ccc (527O),
$\pi$ is order-continuous.  If
$W\in(\Sigma\tensorhat\Cal B(Z))\setminus\Cal L$, there are a
non-negligible measurable set $E\subseteq X$ and a $\sigma\in S_2$
such that $(E\times Z_{\sigma})\setminus W\in\Cal L$.   
(Take $W'\in\Cal W$ such that
$W\symmdiff W'\in\Cal L$, and $E$, $\sigma$ such that
$E\times Z_{\sigma}\subseteq W'$.)   Now
$\mu^*f^{-1}[E\times Z_{\sigma}]=\mu^*(E\cap D_{\sigma})>0$, while (by
(c)) $f^{-1}[(E\times Z_{\sigma})\setminus W]$ is negligible, so
$f^{-1}[W]$ is not negligible.   Thus $\pi W^{\ssbullet}\ne 0$ in
$\Cal PA/\Cal N(\mu)$;  as $W$ is arbitrary, $\pi$ is injective.
Finally, if $C\in\Cal PA\setminus\Cal N(\mu)$, (b) tells us that
$\pi(E\times Z_{\sigma})^{\ssbullet}\Bsubseteq C^{\ssbullet}$ for some
non-negligible
$E$ and some $\sigma$;  so that the range of $\pi$ is order-dense.
Because $\Sigma\tensorhat\Cal B(Z)/\Cal L$ is Dedekind complete (527O),
$\pi$ is a surjection (314F(a-i)), and
$\Sigma\tensorhat\Cal B(Z)/\Cal L\cong\Cal PA/\Cal PA\cap\Cal N(\mu)$.\ \Qed

\medskip

{\bf (e)} But this means that
$\Sigma\tensorhat\Cal B(Z)/\Cal L$ is a \pssqa.
By 527O and 546P, there is a \qm\
cardinal less than the Maharam type of $\mu$.
Thus the result is proved in the case in which $\mu^*A=1$.

\medskip

{\bf (f)} For the general case, let $H$ be a measurable envelope of $A$
and $\nu=(\mu H)^{-1}\mu_H$ the normalized subspace measure on $H$.
Then $A\cap H$ is of full outer measure for $\nu$.
If $C\subseteq A\cap H$ is such that $\nu^*C>0$, then

\Centerline{$\mu^*C=\mu_H^*C=\mu H\cdot\nu^*C>0$,}

\noindent and there is an $n\in\Bbb N$ such that

$$\eqalign{\mu^*(A_n\setminus H)+\mu H\cdot\nu^*(A_n\cap H)
&=\mu^*(A_n\setminus H)+\mu^*(A_n\cap H)
=\mu^*A_n\cr
&>\mu^*(A_n\setminus C)
=\mu^*(A_n\setminus H)+\mu^*(A_n\cap H\setminus C)\cr
&=\mu^*(A_n\setminus H)+\mu H\cdot\nu^*(A_n\cap H\setminus C),\cr}$$

\noindent that is, $\nu^*(A_n\cap H\setminus C)<\nu^*(A_n\cap H)$.   By (a)-(e),
there is a \qm\ cardinal less than the Maharam type of $\nu$.   But the measure
algebra of $\nu$ is isomorphic, up to a scalar multiple of the measure,
to a principal ideal of the measure algebra of $\mu$,
so there is a \qm\ cardinal less than the Maharam type of $\mu$.   Thus ($\beta$) is
true in this case also.
}%end of proof of 547B

\leader{547C}{Lemma} Let $(X,\Sigma,\mu)$ be a purely atomic totally
finite measure
space, $\sequencen{A_n}$ a sequence of subsets of $X$ and $A\subseteq X$ a
non-negligible set.   Then

{\it either} ($\alpha$) there is an $x\in A$ such that $\mu^*\{x\}>0$

{\it or} ($\beta$) there is a set $C\subseteq A$ such
that $\mu^*C>0$ and $\mu^*(A_n\setminus C)=\mu^*A_n$ for every
$n\in\Bbb N$

{\it or} ($\gamma$) there is a \2vm\ cardinal less than or equal to
$\#(A)$.

\proof{ Suppose that ($\alpha$) and ($\beta$) are false.
Because $\mu$ is purely atomic,
there is a countable family $\familyiI{E_i}$ of atoms for $\mu$ such that
$\bigcup_{i\in I}E_i$ is conegligible in $X$.   Consider the algebra
$\frak A=\Cal PA/\Cal PA\cap\Cal N(\mu)$, where $\Cal N(\mu)$ is the null
ideal of $\mu$.   For each $n\in\Bbb N$ and $i\in I$, set
$a_n=(A_n\cap A)^{\ssbullet}$ and $e_i=(A\cap E_i)^{\ssbullet}$
in $\frak A$.   Then $\{a_n\Bcap e_i:n\in\Bbb N$, $i\in I\}$ is order-dense
in $\frak A$.   \Prf\
If $c\in\frak A\setminus\{0\}$, then $c=C^{\ssbullet}$ for
some non-negligible $C\subseteq A$;  now there are an $n\in\Bbb N$ such
that $\mu^*(A_n\setminus C)<\mu^*A_n$ and an $E\in\Sigma$ such that
$A_n\setminus C\subseteq E$ and $\mu E<\mu^*A_n$.   In this case,
$A_n\setminus E$ is a non-negligible subset of $C$;  let $i\in I$ be such
that $\mu^*(E_i\cap A_n\setminus E)>0$.   As $E_i$ is a $\mu$-atom,
$E_i\cap E$ is negligible and $(A_n\cap E_i)\setminus C$ is negligible.
But this means that $a_n\Bcap e_i\Bsubseteq c$.   Also

\Centerline{$\mu^*(A\cap E_i\cap A_n)
\ge\mu^*(C\cap E_i\cap A_n)
\ge\mu^*(E_i\cap A_n\setminus E)>0$,}

\noindent so $a_n\Bcap e_i\ne 0$.\ \Qed

It follows that $\frak A$ is purely atomic.
\Prf\Quer\ Otherwise, there is some $B\subseteq A$ such that the principal
ideal $\frak A_b$ of $\frak A$ generated by $b=B^{\ssbullet}$ is non-zero
and atomless.   Now $\Cal PB/\Cal PB\cap\Cal N(\mu)\cong\frak A_b$ has
countable $\pi$-weight (514Ed), and in particular is ccc;  as it is
Dedekind $\sigma$-complete (314C), it is Dedekind complete (316Fa);
as it is atomless and not $\{0\}$, it is isomorphic to the category algebra
of $\{0,1\}^{\omega}$ (546H).   But it is also a \pssqa, contradicting
546I.\ \Bang\Qed

Let $d$ be an atom of $\frak A$ and $D\subseteq A$ a set such that
$D^{\ssbullet}=d$.   Then for every $C\subseteq D$, one of $C^{\ssbullet}$,
$(D\setminus C)^{\ssbullet}$ must be zero;  that is,
$\Cal PD\cap\Cal N(\mu)$ is a maximal ideal, and the completion
$\hat\mu_D$ of the subspace measure $\mu_D$ measures every subset of $D$.

At this point recall that we are supposing that every singleton subset of
$A$ is negligible, so that $\add\hat\mu_D\le\#(D)$.   By 543B(f-i),
$\add\hat\mu_D$ is \2vm, and ($\gamma$) is true.
}%end of proof of 547C

\leader{547D}{Lemma} Let $(X,\Sigma,\mu)$ be a totally finite measure
space in which singletons are negligible.   Suppose

\inset{{\it either} that $\mu$ is atomless and there is no
\qm\ cardinal less than the Maharam type of $\mu$

{\it or} that $\mu$ is purely atomic and there is no \2vm\ cardinal less
than or equal to $\#(X)$.}

\noindent Let $\sequencen{A_n}$ be a sequence of subsets of
$X$.   Then there is a set $A\subseteq A_0$ such that $\mu^*A=\mu^*A_0$
and $\mu^*(A_n\setminus A)=\mu^*A_n$ for every $n\in\Bbb N$.

\proof{ Let $\langle(C_i,E_i)\rangle_{i\in I}$ be a maximal
family such that

\Centerline{$C_i\subseteq A_0$,
\quad$\mu^*(A_n\setminus C_i)=\mu^*A_n$ for every $n\ge 1$,
\quad$\mu^*C_i>0$,}

\Centerline{$E_i$ is a measurable envelope of $C_i$}

\noindent for every $i\in I$, and $E_i\cap E_j=\emptyset$ for $i\ne j$.
Then $I$ must be countable.   Set

\Centerline{$F=X\setminus\bigcup_{i\in I}E_i$,
\quad $A=\bigcup_{i\in I}C_i$.}

\noindent If $n\ge 1$, then

$$\eqalign{\mu^*(A_n\setminus A)
&=\mu^*(A_n\cap F)+\sum_{i\in I}\mu^*(A_n\cap E_i\setminus A)\cr
&=\mu^*(A_n\cap F)+\sum_{i\in I}\mu^*(A_n\cap E_i\setminus C_i)\cr
&=\mu^*(A_n\cap F)+\sum_{i\in I}\mu^*(A_n\cap E_i)
=\mu^*A_n.\cr}$$

\noindent\Quer\ Suppose, if possible, that $\mu^*A<\mu^*A_0$.   Then

$$\eqalign{\mu^*A
&<\mu^*(A_0\cap F)+\sum_{i\in I}\mu^*(A_0\cap E_i)
\le\mu^*(A_0\cap F)+\sum_{i\in I}\mu E_i\cr
&=\mu^*(A_0\cap F)+\sum_{i\in I}\mu^*(A\cap E_i)
\le\mu^*(A_0\cap F)+\mu^*A,\cr}$$

\noindent so $\mu^*(A_0\cap F)>0$.   By 547B or 547C, there is a
$C\subseteq A_0\cap F$ such that $\mu^*C>0$ and
$\mu^*(A_n\setminus C)=\mu^*A_n$ for
every $n\ge 1$.   Let $E\subseteq F$ be a measurable envelope for $C$.
Then we ought to have added $(C,E)$ to
$\langle(C_i,E_i)\rangle_{i\in I}$.\ \Bang

Accordingly $\mu^*A=\mu^*A_0$ and we're home.
}%end of proof of 547D

\leader{547E}{Lemma} Let
$(X,\Sigma,\mu)$ be a totally finite measure space in which singletons
are negligible.
Suppose that there is no \qm\ cardinal less than or equal to $\#(X)$.
Then for any $A\subseteq X$ there is a disjoint
family $\ofamily{\xi}{\omega_1}{D_{\xi}}$ of subsets of $A$ such that
$\mu^*D_{\xi}=\mu^*A$ for every $\xi<\omega_1$.

\proof{{\bf (a)}
Let $\familyiI{(F_i,\ofamily{\xi}{\omega_1}{D_{i\xi}})}$ be a
maximal family such that

\inset{$F_i\in\Sigma$,

$\ofamily{\xi}{\omega_1}{D_{i\xi}}$ is a disjoint family of subsets of
$F_i\cap A$,

$\mu^*D_{i\xi}=\mu F_i>0$ for every $\xi<\omega_1$}

\noindent for each $i\in I$, and $\familyiI{F_i}$ is disjoint.   Then $I$ is
countable.   Set $E=\bigcup_{i\in I}F_i$ and $B=A\setminus E$.

\medskip

{\bf (b)}  $B$ is negligible.
\Prf\Quer\ Otherwise, consider the $\sigma$-ideal
$\Cal N(\mu_B)=\Cal PB\cap\Cal N(\mu)$, where $\Cal N(\mu)$ is the null
ideal of $\mu$.   This is a proper
$\sigma$-ideal containing singletons;  because its additivity
(being less than or equal to $\#(X)$) is not \qm, $\Cal N(\mu_B)$
cannot be $\omega_1$-saturated, and there is a disjoint family
$\ofamily{\xi}{\omega_1}{C_{\xi}}$ in $\Cal PB\setminus\Cal N(\mu)$.
For $\xi\le\zeta<\omega_1$ let $H_{\xi\zeta}\subseteq X\setminus E$
be a measurable envelope of
$\bigcup_{\xi\le\eta<\zeta}C_{\eta}$ and set
$a_{\xi\zeta}=H_{\xi\zeta}^{\ssbullet}$ in the measure algebra
$(\frak A,\bar\mu)$ of
$\mu$.   If $\xi\le\xi'\le\zeta'\le\zeta<\omega_1$ then
$\bigcup_{\xi'\le\eta<\zeta'}C_{\eta}
\subseteq H_{\xi\zeta}$, so
$\mu(H_{\xi'\zeta'}\setminus H_{\xi\zeta})=0$ and
$a_{\xi'\zeta'}\Bsubseteq a_{\xi\zeta}$.   For each $\xi<\omega_1$, set
$a_{\xi}=\sup_{\xi\le\zeta<\omega_1}a_{\xi\zeta}$, and let 
$h(\xi)<\omega_1$ be such that $a_{\xi}=a_{\xi,h(\xi)}$.   
Next, $\ofamily{\xi}{\omega_1}{a_{\xi}}$
is non-increasing;  set $a=\inf_{\xi<\omega_1}a_{\xi}$.   
Let $\alpha<\omega_1$ be such that $a=a_{\alpha}$.   Note that

\Centerline{$a=a_{\alpha}\Bsupseteq a_{\alpha\alpha}\ne 0$}

\noindent because $C_{\alpha}$ is non-negligible, and that
$a\Bcap E^{\ssbullet}=0$ because no $C_{\xi}$ meets $E$.

Let $F\subseteq X\setminus E$ be such that $a=F^{\ssbullet}$.   Now let
$\ofamily{\xi}{\omega_1}{\alpha_{\xi}}$ be a strictly increasing family in
$\omega_1$ such that $\alpha_0=0$ and $\alpha_{\xi+1}\ge h(\alpha_{\xi})$ for
each $\xi$.   Set
$D_{\xi}=F\cap\bigcup_{\alpha_{\xi}\le\eta<\alpha_{\xi+1}}C_{\eta}$
for each $\xi<\omega_1$.   Then $\ofamily{\xi}{\omega_1}{D_{\xi}}$
is a disjoint family of subsets of $F\cap A$.   Next, for each $\xi$,
$H_{\alpha_{\xi},\alpha_{\xi+1}}$ is a measurable envelope of
$\bigcup_{\alpha_{\xi}\le\eta<\alpha_{\xi+1}}C_{\eta}$, so

\Centerline{$\mu^*D_{\xi}
=\mu(F\cap H_{\alpha_{\xi},\alpha_{\xi+1}})
=\bar\mu(a\Bcap a_{\alpha_{\xi},\alpha_{\xi+1}})
=\bar\mu(a\Bcap a_{\alpha_{\xi}})
=\bar\mu a
=\mu F$.}

\noindent But this means that we ought to have added
$(F,\ofamily{\xi}{\omega_1}{D_{\xi}})$ to the family
$\familyiI{(F_i,\ofamily{\xi}{\omega_1}{D_{i\xi}})}$.\ \Bang\Qed

\medskip

{\bf (c)} So if we set
$D_{\xi}=\bigcup_{i\in I}D_{i\xi}$ for each $\xi<\omega_1$,
$\ofamily{\xi}{\omega_1}{D_{\xi}}$ is a disjoint family of subsets of $A$
and

\Centerline{$\mu^*A\ge\mu^*D_{\xi}=\sum_{i\in I}\mu^*(F_i\cap D_{\xi})
=\sum_{i\in I}\mu^* D_{i\xi}=\sum_{i\in I}\mu F_i=\mu E\ge\mu^*A$}

\noindent for every $\xi<\omega_1$, as required.
}%end of proof of 547E

\leader{547F}{Theorem} Let
$(X,\Sigma,\mu)$ be a totally finite measure space in which singletons
are negligible.   Suppose

\inset{
{\it either} that $\mu$ is atomless and there is no \qm\ cardinal less than
the Maharam type of $\mu$

{\it or} that $\mu$ is purely atomic and there is no \2vm\ cardinal less
than or equal to $\#(X)$

{\it or} that there is no \qm\ cardinal
less than or equal to $\#(X)$.}

\noindent
Then for any sequence $\sequencen{A_n}$ of subsets of $X$ there is a
disjoint sequence $\sequencen{A'_n}$ such that $A'_n\subseteq A_n$
and $\mu^*A'_n=\mu^*A_n$ for every $n\in\Bbb N$.

\proof{{\bf (a)} If $A\subseteq X$ and $\sequencen{B_n}$ is a sequence of
subsets of $X$ there is a $D\subseteq A$ such that $\mu^*D=\mu^*A$ and
$\mu^*(B_n\setminus D)=\mu^*B_n$ for every $n$.

\medskip

\Prf\ {\bf (i)} If either $\mu$ is atomless and there is no \qm\ cardinal less
than the Maharam type of $\mu$, or $\mu$ is purely atomic and there is no
\2vm\ cardinal less than or equal to $\#(X)$, this is just 547D.

\medskip

\quad{\bf (ii)} If there is no \qm\ cardinal less than or equal to $\#(X)$,
then 547E tells us that there is a
disjoint family $\ofamily{\xi}{\omega_1}{D_{\xi}}$ of subsets of $A$ such
that $\mu^*D_{\xi}=\mu^*A$ for every $\xi$.   Now, for each $n$, let
$\mu_{B_n}$ be
the subspace measure on $B_n$ and $(\mu_{B_n})_*$ the corresponding inner
measure (413D).
Then $\sum_{\xi<\omega_1}(\mu_{B_n})_*D_{\xi}\le\mu_{B_n}B_n$
is finite, so $(\mu_{B_n})_*(B_n\cap D_{\xi})=0$
for all but countably many $\xi$.
There is therefore some $\xi<\omega_1$ such that, taking $D=D_{\xi}$,
we have $(\mu_{B_n})_*(B_n\cap D)=0$ for every $n\in\Bbb N$.   Of course
$D\subseteq A$ and $\mu^*D=\mu^*A$, while

\Centerline{$\mu^*(B_n\setminus D)=\mu_{B_n}^*(B_n\setminus D)
=\mu_{B_n}B_n-(\mu_{B_n})_*(B_n\cap D)=\mu_{B_n}B_n=\mu^*B_n$}

\noindent for every $n$.\ \Qed

\medskip

{\bf (b)} Now choose inductively, for
$n\in\Bbb N$, sets $A'_n$ such that, for each $n$,

\Centerline{$A'_n\subseteq A_n\setminus\bigcup_{i<n}A'_i$,}

\Centerline{$\mu^*A'_n=\mu^*(A_n\setminus\bigcup_{i<n}A'_i)$,}

\Centerline{$\mu^*(A_m\setminus\bigcup_{i\le n}A'_i)
=\mu^*(A_m\setminus\bigcup_{i<n}A'_i)$ for every $m>n$.}

\noindent Then an easy induction shows that
$\mu^*(A_m\setminus \bigcup_{i\le n}A'_i)=\mu^*A_m$ whenever $n<m$, so that
$\mu^*A'_n=\mu^*A_n$ for each $n$.
}%end of proof of 547F

\leader{547G}{Corollary} Let $(X,\Sigma,\mu)$
be an atomless probability space such that there is no \qm\ cardinal less
than the Maharam type of $\mu$.
Let $\sequencen{A_n}$ be a sequence of subsets of
$X$.   Then there is a set $D\subseteq X$ such that
$\mu^*(A_n\cap D)=\mu^*(A_n\setminus D)=\mu^*A_n$ for every $n\in\Bbb N$.

\proof{ Take $\sequencen{A'_n}$ from 547F.  For each $n$,
we can choose $D_n\subseteq A'_n$ such that
$\mu^*D_n=\mu^*(A'_n\setminus D_n)=\mu^*A'_n$ (547Ab).   Set
$D=\bigcup_{n\in\Bbb N}D_n$;  then

\Centerline{$\mu^*(A_n\cap D)\ge\mu^*D_n=\mu^*A'_n=\mu^*A_n$,}

\Centerline{$\mu^*(A_n\setminus D)\ge\mu^*(A'_n\setminus D_n)
=\mu^*A'_n=\mu^*A_n$}

\noindent for every $n$, as required.
}%end of proof of 547G

\leader{547H}{}\cmmnt{ I do not know how far we can hope to extend 547G
to uncountable families in place of $\sequencen{A_n}$.   If in place of

\Centerline{$\mu^*(A_n\cap D)=\mu^*(A_n\setminus D)=\mu^*A_n$}

\noindent we ask rather for

\Centerline{$\min(\mu^*(A_n\cap D),\mu^*(A_n\setminus D))>0$}

\noindent we are led to rather different patterns, as follows.

\medskip

\noindent}{\bf Proposition} Let $(\frak A,\bar\mu)$ be a
probability algebra and $\kappa$ a cardinal.   Then the following are
equiveridical:

(i) $\kappa<\pi(\frak A_d)$ for every
$d\in\frak A^+=\frak A\setminus\{0\}$, writing
$\frak A_d$ for the principal ideal generated by $d$;

(ii) whenever $A\subseteq\frak A^+$ and $\#(A)\le\kappa$
there is a $b\in\frak A$ such that $a\Bcap b$ and
$a\Bsetminus b$ are both non-zero for every $a\in A$.

\proof{{\bf (a)} Suppose that (i) is false;  that there are a non-zero
$d\in\frak A$ and an order-dense set $A\subseteq\frak A_d^+$
such that $\#(A)\le\kappa$.   If $b\in\frak A$ and $b\Bcap d=0$ then
$a\Bcap b=0$ for every $a\in A$;  if $b\Bcap d\ne 0$ then there is an
$a\in A$ such that $a\Bsubseteq b\Bcap d$ and $a\Bsetminus b=0$.   So $A$
witnesses that (ii) is false.

For the rest of the proof, therefore, I suppose that (i) is true and seek
to prove (ii).

\medskip

{\bf (b)} We need an elementary calculation.   Let
$\ofamily{i}{n}{c_i}$ be a stochastically independent family in $\frak A$
such that $\bar\mu c_i=\gamma$ for every $i<n$, where $0<\gamma<1$.
Suppose that $a\in\frak A$ and that

\Centerline{$\sup_{i<n}\bar\mu(a\Bcap c_i)
=\beta\gamma\bar\mu a$}

\noindent where $\beta<1$.
Then $n\le\Bover1{(1-\beta)^2\gamma\bar\mu a}$.

\Prf\ In $L^2(\frak A,\bar\mu)$ set

\Centerline{$e_i=\sqrt{\bover{\gamma}{1-\gamma}}\chi(1\Bsetminus c_i)
  -\sqrt{\bover{1-\gamma}{\gamma}}\chi(1\Bsetminus c_i)$}

\noindent for each $i<n$.   An easy calculation shows that
$\ofamily{i}{n}{e_i}$ is orthonormal.   Next, for each $i$,

$$\eqalign{(e_i|\chi a)
&=\sqrt{\Bover{\gamma}{1-\gamma}}\bar\mu(a\Bsetminus c_i)
  -\sqrt{\Bover{1-\gamma}{\gamma}}\bar\mu(a\Bcap c_i)\cr
&\ge\sqrt{\Bover{\gamma}{1-\gamma}}(\bar\mu a-\beta\gamma\bar\mu a)
  -\sqrt{\Bover{1-\gamma}{\gamma}}\beta\gamma\bar\mu a\cr
&=\sqrt{\Bover1{\gamma(1-\gamma)}}
  (\gamma(\bar\mu a-\beta\gamma\bar\mu a)-(1-\gamma)\beta\gamma\bar\mu a)\cr
&=\gamma\bar\mu a\sqrt{\Bover1{\gamma(1-\gamma)}}(1-\beta)
\ge(1-\beta)\bar\mu a\sqrt{\gamma}.\cr}$$

\noindent By 4A4Ji,

\Centerline{$\bar\mu a=\|\chi a\|_2^2
\ge\sum_{i<n}|(e_i|\chi a)|^2\ge n\gamma(1-\beta)^2(\bar\mu a)^2$}

\noindent and

\Centerline{$n\le\Bover1{(1-\beta)^2\gamma\bar\mu a}$}

\noindent as claimed.\ \Qed

\medskip

{\bf (c)} If $\frak A$ has an atom, then $\kappa=0$ and there is nothing to
prove.   So we may suppose henceforth that $\frak A$ is atomless.   Set
$\lambda=\min\{\tau(\frak A_d):d\in\frak A^+\}$.   Then the
measure algebra $(\frak B_{\lambda},\bar\nu_{\lambda})$ can be embedded in
$(\frak A,\bar\mu)$ (332P).   In particular, we can find, for each $n$,
a stochastically independent family $\ofamily{\xi}{\lambda}{c_{n\xi}}$ of
elements of $\frak A$ with $\bar\mu c_{n\xi}=\bover1{n!}$ for every $\xi$.

Set $q(n)=4n((2n)!+(2n+1)!)$ for each $n\in\Bbb N$, and let
$(\lambda^{\Bbb N},\subseteq^*,\Cal S^{(q)}_{\lambda})$ be the
corresponding version of the
$\lambda$-localization relation as described in 522L.   For
$a\in\frak A^+$ let $n_a\in\Bbb N$ be such that
$n_a\bar\mu a\ge 1$ and set

$$\eqalign{S_a
&=\{(n,\xi):n\ge n_a\text{ and
either }\bar\mu(a\Bcap c_{2n,\xi})\le\Bover{\bar\mu a}{2(2n)!}
\text{ or }\bar\mu(a\Bcap c_{2n+1,\xi})\le\Bover{\bar\mu a}{2(2n+1)!}\}\cr
&\subseteq\Bbb N\times\lambda.\cr}$$

\noindent Then $S_a\in\Cal S^{(q)}_{\lambda}$.   \Prf\ If $n\ge n_a$ then
(b), with $\beta=\bover12$, tells us that

\Centerline{$\#(\{\xi:\bar\mu(a\Bcap c_{2n,\xi})
\le\Bover{\bar\mu a}{2(2n)!}\})
\le\Bover{4(2n)!}{\bar\mu a}
\le 4n(2n)!$,}

\Centerline{$\#(\{\xi:\bar\mu(a\Bcap c_{2n+1,\xi})
\le\Bover{\bar\mu a}{2(2n+1)!}\})
\le\Bover{4(2n+1)!}{\bar\mu a}
\le 4n(2n+1)!$,}

\noindent so $\#(S_a[\{n\}])\le 4n((2n)!+(2n+1)!)=q(n)$.\ \Qed

\medskip

{\bf (d)} Now observe that

$$\eqalignno{\min\{\pi(\frak A_d):d\in\frak A^+\}
&=\pi(\frak B_{\lambda})\cr
\displaycause{see 524Mc}
&=\ci(\frak B_{\lambda}^+)
=\cov(\frak B_{\lambda}^+,\Bsupseteqshort,\frak B_{\lambda}^+)
=\cov(\frak B_{\lambda}^+,\Bsupseteqshort^{\strprime},
[\frak B_{\lambda}^+]^{\le\omega})\cr
\displaycause{512Gf}
&=\cov(\lambda^{\Bbb N},\subseteq^*,\Cal S_{\lambda})\cr
\displaycause{where $(\lambda^{\Bbb N},\subseteq^*,\Cal S_{\lambda})$ is
the ordinary $\lambda$-localization relation, by 524H and 512Da}
&=\cov(\lambda^{\Bbb N},\subseteq^*,\Cal S^{(q)}_{\lambda})\cr}$$

\noindent by 522L.

\medskip

{\bf (e)} Let $A\subseteq\frak A^+$ be a set of size
less than  $\min\{\pi(\frak A_d):d\in\frak A^+\}$.   Then
$\#(A)<\cov(\lambda^{\Bbb N},\subseteq^*,\Cal S^{(q)}_{\lambda})$,
so there must be an
$f\in\lambda^{\Bbb N}$ such that $f\not\subseteq^*S_a$ for any $a\in A$.
Set

\Centerline{$b_{2n}=c_{2n,f(n)}$,
\quad$b_{2n+1}=c_{2n+1,f(n)}$,
\quad$b'_n=b_n\Bsetminus\sup_{i>n}b_i$ for $n\in\Bbb N$,}

\Centerline{$b=\sup_{n\in\Bbb N}b'_{2n}$.}

\noindent Then $a\Bcap b$ and $a\Bsetminus b$ are both non-zero
for every $a\in A$.   \Prf\ There is an
$n\ge n_a$ such that $(n,f(n))\notin S_a$, so that

\Centerline{$\bar\mu(a\Bcap b_{2n})
=\bar\mu(a\Bcap c_{2n,f(n)})
>\Bover{\bar\mu a}{2(2n)!}
\ge\Bover1{2n(2n)!}$,}

\Centerline{$\bar\mu(a\Bcap b_{2n+1})
=\bar\mu(a\Bcap c_{2n+1,f(n)})
>\Bover{\bar\mu a}{2(2n+1)!}
\ge\Bover1{2n(2n+1)!}$.}

\noindent But

$$\eqalignno{\bar\mu(b_{2n}\Bsetminus b'_{2n})
&\le\sum_{i=2n+1}^{\infty}\bar\mu b_i
=\sum_{i=2n+1}^{\infty}\Bover1{i!}\cr
&\le\Bover1{(2n)!}\sum_{j=1}^{\infty}\Bover1{(2n+1)^j}
=\Bover1{2n(2n)!}
<\bar\mu(a\Bcap b_{2n}),\cr
\bar\mu(b_{2n+1}\Bsetminus b'_{2n+1})
&\le\Bover1{(2n+1)(2n+1)!}<\bar\mu(a\Bcap b_{2n+1}),\cr}$$

\noindent so $a\Bcap b\Bsupseteq a\Bcap b'_{2n}$ and
$a\Bsetminus b\Bsupseteq a\Bcap b'_{2n+1}$ are both non-zero.\ \Qed

\medskip

{\bf (f)} As $A$ is arbitrary, (ii) is true.
}%end of proof of 547H

\leader{547I}{Proposition} Let $(X,\Sigma,\mu)$ be a strictly localizable
measure space with null ideal $\Cal N(\mu)$,
and $\kappa$ a cardinal such that

\doubleinset{(*)
whenever $\Cal E\in[\Sigma\setminus\Cal N(\mu)]^{\le\kappa}$ and
$F\in\Sigma\setminus\Cal N(\mu)$, there is a non-negligible measurable
$G\subseteq F$ such that $E\setminus G$ is non-negligible for every
$E\in\Cal E$.}

\noindent Then whenever $\ofamily{\xi}{\kappa}{A_{\xi}}$ is a family of
non-negligible subsets of $X$, there is a $G\in\Sigma$ such that
$A_{\xi}\cap G$ and $A_{\xi}\setminus G$ are non-negligible for every
$\xi<\kappa$.

\proof{{\bf (a)} Suppose to begin with that $\mu X=1$.   Let $\frak A$ be
the measure algebra of $\mu$.   Then (*) says just that
$\kappa<\pi(\frak A_d)$ for every $d\in\frak A^+$.   For each $\xi<\kappa$
let $E_{\xi}$ be a measurable envelope of $A_{\xi}$ and set
$a_{\xi}=E_{\xi}^{\ssbullet}$ in $\frak A$.   By 547H, there is a
$b\in\frak A$ such that $a_{\xi}\Bcap b$ and $a_{\xi}\Bsetminus b$ are
non-zero for every $\xi<\kappa$.   Let $G\in\Sigma$ be such that
$b=G^{\ssbullet}$;  then $E_{\xi}\cap G$ and $E_{\xi}\setminus G$ are
non-negligible for every $\xi$.   But this means that $A_{\xi}\cap G$ and
$A_{\xi}\setminus G$ are non-negligible for every $\xi$.

\medskip

{\bf (b)} If $\mu X=0$ the result is trivial.   For other totally
finite $\mu$, we get the result from (a)
if we replace $\mu$ by a suitable scalar
multiple.

\medskip

{\bf (c)} For the general case, let $\familyiI{X_i}$ be a decomposition of
$X$ and for $i\in I$ set
$J_i=\{\xi:\xi<\kappa$, $A_{\xi}\cap X_i$ is not negligible$\}$.   By (b),
applied to the subspace measure on $X_i$,
there is a measurable $G_i\subseteq X_i$ such that $A_{\xi}\cap G_i$ and
$A_{\xi}\cap X_i\setminus G_i$ are non-negligible for every $\xi\in J_i$.
Set $G=\bigcup_{i\in I}G_i$;  this works.
}%end of proof of 547I

\leader{547J}{Corollary} Let $(X,\Sigma,\mu)$ be an atomless
quasi-Radon measure space and
$\ofamily{\xi}{\omega_1}{A_{\xi}}$ a family of non-negligible subsets of
$X$.   Then there is a $D\subseteq X$ such that $A_{\xi}\cap D$ and
$A_{\xi}\setminus D$ are non-negligible for every $\xi<\omega_1$.

\proof{{\bf (a)} Suppose to begin with that $\mu$ is a \Mth\ probability
measure.   Let $\frak A$ be the measure algebra of $\mu$.   If
$\pi(\frak A)>\omega_1$ we can use 547I.   Otherwise, the $\pi$-weight
$\pi(\mu)$ of $\mu$ is $\omega_1$ (524Tb);
let $\ofamily{\xi}{\omega_1}{E_{\xi}}$ be a
coinitial family in $\Sigma\setminus\Cal N(\mu)$.   Then we can choose
$x_{\xi\eta}$ and $y_{\xi\eta}$, for $\xi$, $\eta<\omega_1$, so that

\inset{all the $x_{\xi\eta}$, $y_{\xi\eta}$ are different,

if $A_{\xi}\cap E_{\eta}\notin\Cal N(\mu)$ then $x_{\xi\eta}$ and
$y_{\xi\eta}$ belong to $A_{\xi}\cap E_{\eta}$.}

\noindent Set $D=\{x_{\xi\eta}:\xi$, $\eta<\omega_1\}$;  then
$\mu^*(A_{\xi}\cap D)=\mu^*A_{\xi}$ for every $\xi$.   \Prf\Quer\
Otherwise, let $E$ be a measurable envelope of $A_{\xi}$ and $F$ a
measurable envelope of $A_{\xi}\cap D$.   We have $\mu(E\setminus F)>0$, so
there is an $\eta<\omega_1$ such that $E_{\eta}\subseteq E\setminus F$, in
which case

\Centerline{$x_{\xi\eta}\in A_{\xi}\cap E_{\eta}\cap D\subseteq F$.\
\Bang\Qed}

\noindent Similarly, $\mu^*(A_{\xi}\setminus D)=\mu^*A_{\xi}$ for every
$\xi$, and we have a suitable set $D$.

\medskip

{\bf (b)} In general, $X$ has a decomposition into \Mth\ subspaces (as in
the proofs of 524J and 524P), so the full result follows as in
(b)-(c) of the proof of 547I.
}%end of proof of 547J

\exercises{\leader{547X}{Basic exercises (a)}
%\spheader 547Xa
Let $(X,\Sigma,\mu)$ be a semi-finite
measure space with the measurable envelope property (definition:  213Xl).
Suppose that $A\subseteq X$ is
such that $\min(\mu^*D,\mu^*(A\setminus D))<\mu^*A$ for every $D\subseteq A$.
Show that there is a non-negligible set $B\subseteq A$, of finite outer
measure, such that the completion $\hat\mu_B$ of the subspace measure on
$B$ measures every subset of $B$.
%547A

\spheader 547Xb
Let $(X,\Sigma,\mu)$ be a measure space.   Show that the following are
equiveridical:  (i) for every $A\subseteq X$ there is an $A'\subseteq A$
such that $\mu^*A'=\mu^*(A\setminus A')=\mu^*A$ (ii) whenever
$\familyiI{A_i}$ is a finite family of subsets of $X$, there is a
disjoint family $\familyiI{A'_i}$ such that $A'_i\subseteq A_i$ and
$\mu^*A'_i=\mu^*A_i$ for every $i\in I$.
%547Xa 547A

\spheader 547Xc Show that the following are equiveridical:  (i) there is no
\qm\ cardinal;  (ii) if
$(X,\Sigma,\mu)$ is a probability space such that $\mu\{x\}=0$ for every
$x\in X$ then there is a disjoint
family $\ofamily{\xi}{\omega_1}{D_{\xi}}$ of subsets of $X$ such that
$\mu^*D_{\xi}=1$ for every $\xi<\omega_1$.
%547E  X=B\cup\{\infty\}

\spheader 547Xd Let $(X,\Sigma,\mu)$ be a probability space and suppose
that $\non\Cal N(\mu)=\pi(\mu)=\kappa$.
Show that if $\ofamily{\xi}{\kappa}{A_{\xi}}$ is any family of
subsets of $X$, then there is a disjoint family
$\ofamily{\xi}{\kappa}{A'_{\xi}}$ of sets such that
$A'_{\xi}\subseteq A_{\xi}$ and $\mu^*A'_{\xi}=\mu^*A_{\xi}$ for every
$\xi<\kappa$.
%547F

\spheader 547Xe
Let $\ofamily{\xi}{\frak c}{E_{\xi}}$ be a family of Lebesgue measurable
subsets of $\Bbb R$.   Show that there is a disjoint family
$\ofamily{\xi}{\frak c}{A_{\xi}}$ of sets such that
$A_{\xi}\subseteq E_{\xi}$
and $\mu^*A_{\xi}=\mu E_{\xi}$ for every $\xi<\frak c$, where $\mu$ is
Lebesgue measure on $\Bbb R$.   \Hint{419I.}
%547F

\spheader 547Xf (M.R.Burke)
Let $\Cal N$ be the null ideal of Lebesgue measure $\mu$
on $\Bbb R$.   Show that if $2^{\non\Cal N}=\frak c$ then there is a family
$\ofamily{\xi}{\frak c}{A_{\xi}}$ of subsets of $\Bbb R$ such that for
every $D\subseteq\Bbb R$ there is some $\xi<\frak c$ such that
$\min(\mu^*(A_{\xi}\cap D),\mu^*(A_{\xi}\setminus D))<\mu^*A_{\xi}$.
%547F

\spheader 547Xg Let $\frak A$ be a Boolean algebra.   (i) Show that the
following are equiveridical:  ($\alpha$)
$\pi(\frak A_a)>\omega$ for every $a\in\frak A^+$,
where $\frak A_a$ is the principal ideal generated
by $a$  ($\beta$) for every sequence $\sequencen{a_n}$ in
$\frak A^+$ there is a disjoint sequence $\sequencen{b_n}$ in
$\frak A^+$ such that $b_n\Bsubseteq a_n$ for every $n$.
(ii) Show that if $\frak A$ has the $\sigma$-interpolation property then
we can add ($\gamma$) whenever $A\subseteq\frak A^+$ is countable,
there is a $b\in\frak A$ such that $a\Bcap b$ and
$a\Bsetminus b$ are both non-zero for every $a\in A$.
%547F

%\leader{547Y}{Further exercises (a)}
%\spheader 547Ya
}%end of exercises

\leader{547Z}{Problems (a)}
Suppose that $\ofamily{\xi}{\omega_1}{A_{\xi}}$ is a
family of subsets of $[0,1]$.   Must there be a set $D\subseteq[0,1]$ such that
$\mu^*(A_{\xi}\cap D)=\mu^*(A_{\xi}\setminus D)=\mu^*A_{\xi}$ for every
$\xi<\omega_1$, where $\mu$ is Lebesgue measure on $[0,1]$?

\spheader 547Zb Suppose that there is no \qm\ cardinal.   Let
$(X,\Sigma,\mu)$ be an atomless probability space
and $\ofamily{\xi}{\omega_1}{A_{\xi}}$ a
family of subsets of $X$.   Must there be a disjoint family
$\ofamily{\xi}{\omega_1}{A'_{\xi}}$ such that $A'_{\xi}\subseteq A_{\xi}$
and $\mu^*A'_{\xi}=\mu^*A_{\xi}$ for every $\xi<\omega_1$?

\endnotes{
\Notesheader{547} Of course the most important case of Theorem 547F is when
$X=[0,1]$ with Lebesgue measure, so that we have a result provable in ZFC,
whether or not there are \qm\ cardinals.
As far as I know there is no real
simplification available for this special case if we wish to avoid special
axioms.   In many models of set theory, of course,
there are other approaches, as in 547E and 547Xd;
and I note that it makes a difference that we start with
not-necessarily-measurable sets $A_n$ (547Xe).

The arguments here leave many obvious questions open.   The first group
concerns possible
extensions of 547F to uncountable families of sets, as in 547Z.
The methods of \S546, as I have written them out, do not
seem adequate in this context;  an answer to 546Zc might help.
I remark that {\smc Shelah 03} describes a model in which there is a set
$A\in\Cal P\Bbb R\setminus\Cal N$
such that $\Cal PA\cap\Cal N$ is $\omega_1$-saturated in
$\Cal PA$, where $\Cal N$ is the null ideal of Lebesgue measure.
Elsewhere we can ask, in 547B and 547F, whether the hypotheses involving
\qm\ cardinals could be rewritten with \am\ cardinals.   Only in 547E is it
clear that non-\am\ \qm\ cardinals are relevant (547Xc).

The questions tackled in this section can be re-phrased as questions about
structures $(\Cal PX/\Cal I,\frak A)$ where $\Cal I$ is a $\sigma$-ideal of
subsets of $X$ and $\frak A$ is a $\sigma$-subalgebra of the \pssqa\
$\Cal PX/\Cal I$;   a requirement of the form
`$\mu^*(A\cap D)=\mu^*A$' becomes (in the context of a totally finite
measure $\mu$) `$\upr(a\Bcap d,\frak A)=\upr(a,\frak A)$', where
$\upr(a,\frak A)$ is the upper envelope of $a$ in $\frak A$\cmmnt{
(313S)}.

I include 547H-547J %547H 547I 547J
to show that if we are less ambitious then there are
quite different, and rather easier, arguments available.   The condition
(*) of 547I is exact if we are looking for a measurable splitting set $G$.
But I am not at all sure that 547J is in the right form.
}%end of notes

\discrpage

