\frfilename{mt4a1.tex}
\versiondate{27.1.13}
\copyrightdate{2000}

\def\refquery{\medskip reference\query \medskip}
\def\JWII{{\smc Just \& Weese 97}}
\def\Kunen{{\smc Kunen 80}}
\def\Levy{{\smc Levy 79}}
\def\JechGH{{\smc Jech 78}}
\def\JechIC{{\smc Jech 03}}

\def\chaptername{Appendix}
\def\sectionname{Set theory}

\newsection{4A1}

For this volume, we need fragments from four topics in set theory and
one in Boolean algebra.   The
most important are the theory of closed cofinal sets and stationary
sets (4A1B-4A1C)
and infinitary combinatorics (4A1D-4A1H).  %4A1D 4A1E 4A1F 4A1G 4A1H
Rather more specialized, we have the theory of normal (ultra)filters
(4A1J-4A1L) %4A1J 4A1K 4A1L
and a mention of Ostaszewski's $\clubsuit$ (4A1M-4A1N), used for an
example in \S439.   I conclude with a simple result on the cardinality
of $\sigma$-algebras (4A1O).

\leader{4A1A}{Cardinals again (a)} An infinite cardinal $\kappa$ is
{\bf regular} if\cmmnt{ it is not the supremum of fewer than
$\kappa$
smaller cardinals, that is, if} $\cf\kappa=\kappa$.   Any infinite
successor cardinal is regular.\cmmnt{  (\Kunen, I.10.37;
{\smc Just \& Weese 96}, 11.18;  \JechGH, p.\ 40;
\Levy, IV.3.11.)}\cmmnt{ In particular,} $\omega_1=\omega^+$ is regular.
%4{}19

\spheader 4A1Ab If $\zeta$ is an ordinal and $X$ is a set then I say that a
family $\langle x_{\xi}\rangle_{\xi<\zeta}$ in $X$ {\bf runs over $X$
with cofinal repetitions} if $\{\xi:\xi<\zeta,\,x_{\xi}=x\}$ is
cofinal with $\zeta$ for every $x\in X$.
%4{}29
Now if $X$ is any non-empty set and $\kappa$ is a cardinal greater than
or equal to $\max(\omega,\#(X))$, there
is a family $\langle x_{\xi}\rangle_{\xi<\kappa}$ running over $X$ with
cofinal repetitions.  \prooflet{\Prf\ By 3A1Ca, there is a surjection
$\xi\mapsto(x_{\xi},\alpha_{\xi}):\kappa\to X\times\kappa$.\ \Qed}

%\#([X]^{<\omega})\le\max(\omega,\#(X))  is 3A1Cd

\spheader 4A1Ac {\bf The cardinal $\frak c$} (i) Every non-trivial
interval in $\Bbb R$ has cardinal $\frak c$.
\cmmnt{({\smc Enderton 77}, p.\ 131.)}
%4{}19

\quad(ii)\dvAnew{2009} If $\#(A)\le\frak c$ and $D$ is countable, then
$\#(A^D)\le\frak c$.   
\cmmnt{($\#(A^D)\le\#(\Cal P\Bbb N)^{\Bbb N})
=\#(\Cal P(\Bbb N\times\Bbb N))=\#(\Cal P\Bbb N)$.)}
%\refquery

\quad(iii) $\cf(2^{\kappa})>\kappa$ for every infinite
cardinal $\kappa$;  in particular, $\cf\frak c>\omega$.
\cmmnt{(\Kunen, I.10.40;  {\smc Just \& Weese 96},
11.24;  \JechGH, p.\ 46;  \JechIC, 5.11;  \Levy, V.5.2.)}
%4{}39 %5{}11

\spheader 4A1Ad {\bf The Continuum Hypothesis} This is the statement
`$\frak c=\omega_1$';  it is neither provable nor disprovable from the
ordinary axioms of mathematics\cmmnt{, including the Axiom of Choice}.
\cmmnt{As such, it belongs to Volume 5 rather than to the present
volume.   But I do at one point refer to one of its immediate
consequences.}   If the continuum hypothesis is true, then there is a
well-ordering $\preccurlyeq$ of $[0,1]$ such that
$([0,1],\preccurlyeq)$
has order type $\omega_1$\prooflet{ (because there is a bijection
$f:[0,1]\to\omega_1$, and we can set $s\preccurlyeq t$ if
$f(s)\le f(t)$)}.
%4{}63

\leader{4A1B}{Closed cofinal sets} Let $\alpha$ be an ordinal.
%we need the results for ordinals in 5{}A5F

\spheader 4A1Ba Note that a subset $F$ of $\alpha$ is closed in the
order topology iff $\sup A\in F$ whenever
$A\subseteq F$ is non-empty and $\sup A<\alpha$.
\cmmnt{(4A2S(a-ii).)}

\spheader 4A1Bb If $\alpha$ has uncountable cofinality, and
$A\subseteq\alpha$ has supremum $\alpha$, then
$A'=\{\xi:0<\xi<\alpha$, $\xi=\sup(A\cap\xi)\}$ is a closed cofinal
set in $\alpha$.
\prooflet{\Prf\ ($\alpha$)
For any $\eta<\alpha$ we can choose inductively a
strictly increasing sequence $\sequencen{\xi_n}$ in $A$ starting from
$\xi_0\ge\eta$;  now $\xi=\sup_{n\in\Bbb N}\xi_n\in A'$ and
$\xi\ge\eta$;  this shows that $A'$ is cofinal with $\alpha$.   ($\beta$)
If $\emptyset\ne B\subseteq A'$ and $\sup B=\xi<\alpha$, then
$A\cap\xi\supseteq A\cap\eta$ for every $\eta\in B$, so

\Centerline{$\xi=\sup B=\sup_{\eta\in B}\sup(A\cap\eta)
=\sup(\bigcup_{\eta\in B}A\cap\eta)\le\sup A\cap\xi\le\xi$}

\noindent and $\xi\in A'$.\ \Qed
%4{}29

}In particular, \cmmnt{taking $A=\alpha=\omega_1$,} the set of
non-zero
countable limit ordinals is a closed cofinal set in $\omega_1$.
%4{}19

\spheader 4A1Bc(i) If $\ofamily{\xi}{\alpha}{F_{\xi}}$ is a family of
subsets of $\alpha$,
the {\bf diagonal intersection} of $\ofamily{\xi}{\alpha}{F_{\xi}}$ is
$\{\xi:\xi<\alpha,\,\xi\in F_{\eta}$ for every $\eta<\xi\}$.

\quad(ii) If $\kappa$ is a regular uncountable cardinal and
$\ofamily{\xi}{\kappa}{F_{\xi}}$ is any family of closed cofinal sets
in $\kappa$, its diagonal intersection $F$ is again a closed cofinal set
in $\kappa$.
\prooflet{\Prf\ $F=\bigcap_{\xi<\kappa}(F_{\xi}\cup[0,\xi])$ is
certainly closed.   To see that it is cofinal, argue as follows.
Start from any $\zeta_0<\kappa$.   Given $\zeta_n<\kappa$, set

\Centerline{$\zeta_{n+1}
=\sup_{\xi<\zeta_n}(\min(F_{\xi}\setminus\zeta_n)+1)$;}

\noindent this is defined because every $F_{\xi}$ is cofinal with
$\kappa$, and is less than $\kappa$ because $\cf\kappa=\kappa$.   At
the
end of the induction, set $\zeta^*=\sup_{n\in\Bbb N}\zeta_n$;  then
$\zeta_0\le\zeta^*$ and $\zeta^*<\kappa$ because $\cf\kappa>\omega$.
If $\xi$, $\eta<\zeta^*$, there is an $n\in\Bbb N$ such that
$\max(\xi,\eta)<\zeta_n$, in which case
$F_{\xi}\cap(\zeta^*\setminus\eta)
\supseteq F_{\xi}\cap\zeta_{n+1}\setminus\zeta_n$ is non-empty.   As
$\eta$ is arbitrary and $F_{\xi}$ is closed, $\zeta^*\in F_{\xi}$;  as
$\xi$ is arbitrary, $\zeta^*\in F$;  as $\zeta_0$ is arbitrary, $F$ is
cofinal.\ \Qed}

%(\Kunen, II.6.14;  \JechGH, p.\ 58;  \JechIC, 8.4;
%\JWII, Theorem 21.7;
%\Levy, IV.4.20.)

\quad(iii) In particular, if $f:\kappa\to\kappa$ is any function, then
$\{\xi:\xi<\kappa,\,f(\eta)<\xi$ for every $\eta<\xi\}$ is a closed
cofinal set in $\kappa$\cmmnt{, being the diagonal intersection of
$\ofamily{\xi}{\kappa}{\kappa\setminus(f(\xi)+1)}$}.
%4{}23

\spheader 4A1Bd If $\alpha$ has uncountable cofinality, $\Cal F$ is a
non-empty family of closed cofinal sets
in $\alpha$ and $\#(\Cal F)<\cf\alpha$, then $\bigcap\Cal F$ is a
closed
cofinal set in $\alpha$.   \prooflet{\Prf\ Being the intersection of
closed sets it is surely closed.   Set
$\lambda=\max(\omega,\#(\Cal F))$ and let
$\ofamily{\xi}{\lambda}{F_{\xi}}$ run over $\Cal F$ with
cofinal repetitions.   Starting from any $\zeta_0<\alpha$, we can
choose $\langle\zeta_{\xi}\rangle_{1\le\xi\le\lambda}$ such that

\inset{-- if $\xi<\lambda$ then
$\zeta_{\xi}\le\zeta_{\xi+1}\in F_{\xi}$;

-- if $\xi\le\lambda$ is a non-zero limit ordinal,
$\zeta_{\xi}=\sup_{\eta<\xi}\zeta_{\eta}$.}

\noindent (Because $\lambda<\cf\alpha$, $\zeta_{\xi}<\alpha$ for
every $\xi$.)   Now $\zeta_0\le\zeta_{\lambda}<\alpha$, and if
$F\in\Cal F$, $\zeta<\zeta_{\lambda}$ there is a $\xi<\lambda$ such
that $F=F_{\xi}$ and $\zeta\le\zeta_{\xi}$, in which case
$\zeta\le\zeta_{\xi+1}\le\zeta_{\lambda}$ and $\zeta_{\xi+1}\in F$.
This shows that either $\zeta_{\lambda}\in F$ or
$\zeta_{\lambda}=\sup(F\cap\zeta_{\lambda})$, in which case again
$\zeta_{\lambda}\in F$.   As $F$ is arbitrary,
$\zeta_{\lambda}\in\bigcap\Cal F$;
as $\zeta_0$ is arbitrary, $\bigcap\Cal F$ is cofinal.\ \Qed
%4{}19 5{}A5

}In particular, the intersection of any sequence of closed cofinal sets
in $\omega_1$ is again a closed cofinal set in $\omega_1$.
%4{}19 %4{}11

\leader{4A1C}{Stationary sets (a)} Let $\kappa$ be a cardinal.   A
subset of $\kappa$ is
{\bf stationary} in $\kappa$ if it meets every closed cofinal set in
$\kappa$;  otherwise it is {\bf non-stationary}.
%4{}19

\spheader 4A1Cb If $\kappa$ is a cardinal of uncountable cofinality,
the intersection of any stationary subset of $\kappa$ with a closed
cofinal
set in $\kappa$ is again a stationary set\prooflet{ (because the
intersection
of two closed cofinal sets is a closed cofinal set)};  the family of
non-stationary subsets of $\kappa$ is a $\sigma$-ideal, the
{\bf non-stationary ideal} of $\kappa$.
\cmmnt{(\Kunen, II.6.9;  \JWII, Lemma 21.11;  \JechIC, p.\ 93;
\Levy, IV.4.35.)}
%4{}19

\spheader 4A1Cc {\bf Pressing-Down Lemma\cmmnt{ (Fodor's theorem)}} If
$\kappa$ is a regular uncountable cardinal, $A\subseteq\kappa$ is
stationary and $f:A\to\kappa$ is such that $f(\xi)<\xi$ for every
$\xi\in A$, then there is a stationary set $B\subseteq A$ such that
$f$ is constant on $B$.
\leaveitout{\Prf\Quer\ Otherwise, we have closed cofinal sets
$F_{\eta}\subseteq\omega_1$ such
that $f(\xi)\ne\eta$ for $\xi\in A\cap F_{\eta}$.   Set
$F=\{\xi:\xi<\omega_1,\,\xi\in F_{\eta}$ for every $\eta<\xi\}$;  then
$F$ is a closed cofinal set, so meets $A$.   \Bang\Qed}
\cmmnt{(\Kunen, II.6.15;  \JWII, Theorem 21.2;
\JechGH, Theorem 22;  \JechIC, 8.7;  \Levy, IV.4.40.)}
%4{}19

\spheader 4A1Cd There are disjoint stationary sets $A$,
$B\subseteq\omega_1$.
\cmmnt{(This is easily deduced from 419G or 438Cd, and is also a
special case
of very much stronger results.   See 541Ya in Volume 5, or
\Kunen, II.6.12;
\JWII, Corollary 23.4;
\JechGH, p.\ 59;   \JechIC, 8.8;  \Levy, IV.4.48.)}
%4{}39

\leader{4A1D}{$\Delta$-systems (a)} A family
$\family{\xi}{A}{I_{\xi}}$
of sets is a {\bf $\Delta$-system} with {\bf root} $I$ if
$I_{\xi}\cap I_{\eta}=I$ for all distinct $\xi$, $\eta\in A$.

\spheader 4A1Db {\bf $\Delta$-system Lemma} If $\#(A)$ is a regular
uncountable cardinal and $\langle I_{\xi}\rangle_{\xi\in A}$ is any
family of finite sets, there is a set $D\subseteq A$ such that
$\#(D)=\#(A)$ and $\family{\xi}{D}{I_{\xi}}$ is a $\Delta$-system.
\prooflet{(\Kunen, II.1.6;  \JWII, Theorem 16.3.  For the present
volume
we need only the case $\#(A)=\omega_1$, which is treated in
\JechGH, p.\ 225 and \JechIC, 9.18.)}
%4{}17 %4{}35  %5{}13 %5{}53
%5{}14 5{}15 for cardinals > omega_1;
% check refs for this case

\leader{4A1E}{Free sets (a)} Let $A$ be a set of cardinal at least
$\omega_2$, and $\langle J_{\xi}\rangle_{\xi\in A}$ a family of
countable sets.   Then there are distinct $\xi$, $\eta\in A$ such that
$\xi\notin J_{\eta}$ and $\eta\notin J_{\xi}$.
\prooflet{\Prf\ Let $K\subseteq A$ be a set of cardinal $\omega_1$,
and set
$L=K\cup\bigcup_{i\in K}J_i$.   Then $L$ has cardinal $\omega_1$, so
there is a $\xi\in A\setminus L$.   Now there is an $\eta\in
K\setminus
J_{\xi}$, and this pair $(\xi,\eta)$ serves.\ \Qed}
%for Losert's example, 4{}53

\spheader 4A1Eb If $\family{\xi}{A}{K_{\xi}}$ is a disjoint family of
sets indexed by an uncountable subset $A$ of $\omega_1$,
and $\langle J_{\eta}\rangle_{\eta<\omega_1}$ is a family
of countable sets, there is an uncountable $B\subseteq A$ such that
$K_{\xi}\cap J_{\eta}=\emptyset$ whenever $\eta$, $\xi\in B$ and
$\eta<\xi$.
\prooflet{\Prf\ Choose $\ofamily{\xi}{\omega_1}{\zeta_{\xi}}$
inductively in such a way that $\zeta_{\xi}\in A$ and
$K_{\zeta_{\xi}}\cap J_{\zeta_{\eta}}=\emptyset$,
$\zeta_{\xi}>\zeta_{\eta}$
for every $\eta<\xi$.   Set $B=\{\zeta_{\xi}:\xi<\omega_1\}$.\ \Qed}
%4{}35

\leader{4A1F}{Selecting subsequences (a)} Let $\familyiI{K_i}$ be a
countable family of
sets such that $\bigcap_{i\in J}K_i$ is infinite for every finite
subset $J$ of $I$.   Then there is an infinite set $K$ such that
$K\setminus K_i$ is finite and $K_i\setminus K$ is infinite for every
$i\in I$.
\prooflet{\Prf\ We can suppose that $I\subseteq\Bbb N$.   Choose
$\sequencen{k_n}$ inductively such that
$k_n\in\bigcap_{i\in I,i\le n}K_i\setminus\{k_i:i<n\}$ for every
$n\in\Bbb N$, and set $K=\{k_{2n}:n\in\Bbb N\}$.\ \Qed

}Consequently there is a family
$\ofamily{\xi}{\omega_1}{K_{\xi}}$ of
infinite subsets of $\Bbb N$ such that $K_{\xi}\setminus K_{\eta}$ is
finite if $\eta\le\xi$, infinite if $\xi<\eta$.
\prooflet{ (Choose the $K_{\xi}$ inductively.)}
%4{}66

\spheader 4A1Fb Let $\familyiI{\Cal J_i}$ be a countable family of
subsets of $[\Bbb N]^{\omega}$ such that
$\Cal J_i\cap\Cal PK\ne\emptyset$ for every
$K\in[\Bbb N]^{\omega}$ and $i\in I$.   Then there is an infinite
$K\subseteq\Bbb N$ such that for every $i\in I$ there is a
$J\in\Cal J_i$ such that $K\setminus J$ is finite.   \prooflet{\Prf\
The case $I=\emptyset$ is trivial;  suppose that $\sequence{n}{i_n}$ runs
over $I$.   Choose $K_n$, $k_n$ inductively, for $n\in\Bbb N$, by
taking

\Centerline{$K_0=\Bbb N$,\quad$k_n\in K_n$,
\quad$K_{n+1}\subseteq K_n\setminus\{k_n\}$,
\quad$K_{n+1}\in\Cal J_{i_n}$}

\noindent for every $n$;  set  $K=\{k_n:n\in\Bbb N\}$.\ \Qed}
%4{}37G

\leader{4A1G}{Ramsey's theorem} If $n\in\Bbb N$, $K$ is finite and
$h:[\Bbb N]^n\to K$ is any function, there is an infinite
$I\subseteq\Bbb N$ such that $h$ is constant on $[I]^n$.
\prooflet{({\smc Bollob\'as 79}, p.\ 105, Theorem 3;
\JWII, 15.3;
\JechGH, 29.1;  \JechIC, 9.1;
\Levy, IX.3.7.   For the present volume we need only the
case $n=\#(K)=2$.)}

\leader{4A1H}{\cmmnt{The Marriage Lemma \dvrocolon{again}}}\cmmnt{ In 449L
it will be useful to have an infinitary version of the Marriage Lemma
available.

\medskip

\noindent}{\bf Proposition} Let $X$ and $Y$ be sets, and
$R\subseteq X\times Y$ a set such that $R[\{x\}]$ is finite for every
$x\in X$ and
$\#(R[I])\ge\#(I)$ for every finite set $I\subseteq X$.   Then there
is an injective function $f:X\to Y$ such that $(x,f(x))\in R$ for every
$x\in X$.

\proof{ For each finite $J\subseteq X$ there is an injective function
$f_J:J\to Y$ such that $f_J\subseteq R$ (identifying $f_J$ with its
graph), by the ordinary Marriage Lemma (3A1K) applied to
$R\cap(J\times R[J])$.   Let $\Cal F$ be any ultrafilter on
$[X]^{<\omega}$ containing $\{J:I\subseteq J\in[X]^{<\omega}\}$ for
every $I\in[X]^{<\omega}$;  then for each $x\in X$ there must be an
$f(x)\in R[\{x\}]$ such that $\{J:f_J(x)=f(x)\}\in\Cal F$, because
$R[\{x\}]$ is finite.   Now $f\subseteq R$ is a function from $X$ to
$Y$
and must be injective, because for any $x$, $x'\in X$ there is a
$J\in[X]^{<\omega}$ such that $f$ and $f_J$ agree on $\{x,x'\}$.
}%end of proof of 4A1H

\leader{4A1I}{Filters (a)} Let $X$ be a non-empty set.
If $\Cal E\subseteq\Cal PX$ is non-empty
and has the finite intersection property,

\Centerline{$\Cal F=\{A:A\subseteq X,\,A\supseteq\bigcap\Cal E'$ for
some non-empty finite $\Cal E'\subseteq\Cal E\}$}

\noindent is the smallest filter on $X$ including $\Cal E$, the filter
{\bf generated} by $\Cal E$.

If $\Cal E\subseteq\Cal PX$ is non-empty and
downwards-directed, then it has the finite intersection property iff
it does not contain $\emptyset$;  in this case we say that $\Cal E$ is a
{\bf filter base};
$\Cal F=\{A:A\subseteq X,\,A\supseteq E$ for some $E\in\Cal E\}$, and
$\Cal E$ is a base for the filter $\Cal F$.

In general, if $\Cal E$ is a family of subsets of $X$,
then there is a filter on $X$ including
$\Cal E$ iff $\Cal E$ has the finite intersection property;  in this
case, there is an ultrafilter on $X$ including
$\Cal E$\prooflet{ (2A1O)}.

\spheader 4A1Ib If $\kappa$ is a cardinal and $\Cal F$ is a filter
then $\Cal F$ is {\bf $\kappa$-complete} if $\bigcap\Cal E\in\Cal F$
whenever $\Cal E\subseteq\Cal F$ and $0<\#(\Cal E)<\kappa$.
%4{}35
Every filter is $\omega$-complete.

\spheader 4A1Ic A filter $\Cal F$ on a regular uncountable cardinal
$\kappa$ is {\bf normal} if ($\alpha$) $\kappa\setminus\xi\in\Cal F$ for
every $\xi<\kappa$ ($\beta$) whenever
$\langle F_{\xi}\rangle_{\xi<\kappa}$ is a family in $\Cal F$, its
diagonal intersection belongs to $\Cal F$.

\leader{4A1J}{Lemma} A normal filter $\Cal F$ on a regular uncountable
cardinal $\kappa$ is $\kappa$-complete.

\proof{ If $\lambda<\kappa$ and $\ofamily{\xi}{\lambda}{F_{\xi}}$ is a
family in $\Cal F$, set $F_{\xi}=\kappa$ for $\lambda\le\xi<\kappa$,
and
let $F$ be the diagonal intersection of
$\ofamily{\xi}{\kappa}{F_{\xi}}$;  then
$\bigcap_{\xi<\lambda}F_{\xi}\supseteq F\setminus\lambda$ belongs to
$\Cal F$.
}%end of proof of 4A1J

\leader{4A1K}{Theorem} Let $X$ be a set and $\Cal F$ a non-principal
$\omega_1$-complete ultrafilter on $X$.   Let $\kappa$ be the least
cardinal of any non-empty set $\Cal E\subseteq\Cal F$ such that
$\bigcap\Cal E\notin\Cal F$.   Then $\kappa$ is a regular uncountable
cardinal, $\Cal F$ is $\kappa$-complete, and
there is a function $g:X\to\kappa$ such that $g[[\Cal F]]$ is a normal
ultrafilter on $\kappa$.

\proof{{\bf (a)} By the definition of $\kappa$, $\Cal F$ is
$\kappa$-complete.   Because $\Cal F$ is $\omega_1$-complete,
$\kappa>\omega$.   Let $H$ be the set of all functions $h:X\to\kappa$
such that $h^{-1}[\kappa\setminus\xi]\in\Cal F$ for every
$\xi<\kappa$.
Then $H$ is not empty.   \Prf\ Let $\ofamily{\xi}{\kappa}{E_{\xi}}$
be a
family in $\Cal F$ such that
$E=\bigcap_{\xi<\kappa}E_{\xi}\notin\Cal F$.   Because $\Cal F$ is an
ultrafilter, $X\setminus E\in\Cal F$.   Set $h(x)=0$ if
$x\in E$, $h(x)=\min\{\xi:x\notin E_{\xi}\}$ if $x\in X\setminus E$;
then

\Centerline{$h^{-1}[\kappa\setminus\xi]
\supseteq(X\setminus E)\cap\bigcap_{\eta<\xi}E_{\eta}\in\Cal F$}

\noindent for every $\xi<\kappa$, because $\Cal F$ is
$\kappa$-complete,
so $h\in H$.\ \Qed

\medskip

{\bf (b)} For $h$, $h'\in H$, say that $h\prec h'$ if
$\{x:h(x)<h'(x)\}\in\Cal F$.   Then there is a $g\in H$ such that
$h\not\prec g$ for any $h\in H$.   \Prf\Quer\ Otherwise, there is a
sequence $\sequencen{h_n}$ in $H$ such that $h_{n+1}\prec h_n$ for
every $n\in\Bbb N$.   In this case
$E_n=\{x:h_{n+1}(x)<h_n(x)\}\in\Cal F$ for every
$n$.   Because $\Cal F$ is $\omega_1$-complete, there is an
$x\in\bigcap_{n\in\Bbb N}E_n$;  but now $\sequencen{h_n(x)}$ is a
strictly decreasing sequence of ordinals, which is impossible.\
\Bang\Qed

\medskip

{\bf (c)} I should check that $\kappa$ is regular.   \Prf\ If
$\ofamily{\xi}{\lambda}{\alpha_{\xi}}$ is any family in $\kappa$ with
$\lambda<\kappa$, then $g^{-1}[\kappa\setminus\alpha_{\xi}]\in\Cal F$
for every $\xi$, so (because $\Cal F$ is $\kappa$-complete)

\Centerline{$g^{-1}[\kappa\setminus\sup_{\xi<\lambda}\alpha_{\xi}]
=\bigcap_{\xi<\lambda}g^{-1}[\kappa\setminus\alpha_{\xi}]\in\Cal F$,}

\noindent and $\sup_{\xi<\lambda}\alpha_{\xi}\ne\kappa$.\ \Qed

\medskip

{\bf (d)} The image filter $g[[\Cal F]]$ is an ultrafilter on $\kappa$,
by 2A1N.   Because $g\in H$, $g^{-1}[\kappa\setminus\xi]\in\Cal F$
and $\kappa\setminus\xi\in g[[\Cal F]]$ for any $\xi<\kappa$.   \Quer\
Suppose, if possible, that $g[[\Cal F]]$ is not normal.   Then there
is a family $\ofamily{\xi}{\kappa}{A_{\xi}}$ in $g[[\Cal F]]$ such that
its diagonal intersection $A$ does not belong to $g[[\Cal F]]$, that is,
$g^{-1}[A]\notin\Cal F$ and $X\setminus g^{-1}[A]\in\Cal F$.   Define
$h:X\to\kappa$ by setting

$$\eqalign{h(x)&=0\text{ if }g(x)\in A,\cr
&=\min\{\eta:\eta<g(x),\,g(x)\notin A_{\eta}\}
  \text{ if }g(x)\notin A.\cr}$$

\noindent Then

\Centerline{$h^{-1}[\kappa\setminus\xi]
\supseteq(X\setminus g^{-1}[A])\cap\bigcap_{\eta<\xi}g^{-1}[A_{\eta}]
\in\Cal F$}

\noindent for every $\xi<\kappa$.   Thus $h\in H$.   But also
$h(x)<g(x)$ for every $x\in X\setminus g^{-1}[A]$, so $h\prec g$,
contrary to the choice of $g$.\ \Bang

Thus $g[[\Cal F]]$ is a normal filter, and the theorem is proved.
}%end of proof of 4A1K

%4{}35

\leader{4A1L}{Theorem} Let $\kappa$ be a regular uncountable cardinal,
and $\Cal F$
a normal ultrafilter on $\kappa$.   If $S\subseteq[\kappa]^{<\omega}$,
there is a set $F\in\Cal F$ such that, for each $n\in\Bbb N$, $[F]^n$
is either a subset of $S$ or disjoint from $S$.

\proof{{\bf (a)} For each $n\in\Bbb N$ there is an
$F_n\in\Cal F$ such that either $[F_n]^n\subseteq S$ or
$[F_n]^n\cap S=\emptyset$.   \Prf\ Induce on $n$.   If $n=0$ we can take
$F_n=\kappa$, because $[\kappa]^0=\{\emptyset\}$.   For the inductive
step to $n+1$, set
$S_{\xi}=\{I:I\in[\kappa]^{<\omega}$, $I\cup\{\xi\}\in S\}$ for each
$\xi<\kappa$.   By the inductive hypothesis, there is for
each $\xi<\kappa$ a set $E_{\xi}\in\Cal F$ such that either
$[E_{\xi}]^n\subseteq S_{\xi}$ or $[E_{\xi}]^n\cap S_{\xi}=\emptyset$.
Let $E$ be the diagonal intersection of
$\ofamily{\xi}{\kappa}{E_{\xi}}$, so that $E\in\Cal F$.

Suppose that $A=\{\xi:[E_{\xi}]^n\subseteq S_{\xi}\}$ belongs to
$\Cal F$.   Then $E\cap A\in\Cal F$.   If $I\in[E\cap A]^{n+1}$, set
$\xi=\min I$.   Then $I\setminus\{\xi\}\subseteq E_{\xi}$, so that
$I\setminus\{\xi\}\in S_{\xi}$ and $I\in S$.   Thus
$[E\cap A]^{n+1}\subseteq S$.   Similarly, if $A\notin\Cal F$, then
$E\setminus A\in\Cal F$ and $[E\setminus A]^{n+1}\cap S=\emptyset$.
Thus we can take one of $E\cap A$, $E\setminus A$ for $F_{n+1}$, and
the induction continues.\ \Qed

\medskip

{\bf (b)} At the end of the induction, take $F=\bigcap_{n\in\Bbb N}F_n$;
this serves.
}%end of proof of 4A1L
%4{}35

\vleader{72pt}{4A1M}{Ostaszewski's $\clubsuit$} This is the statement

\inset{Let $\Omega$ be the family of non-zero countable limit
ordinals.   Then there is a family
$\langle\theta_{\xi}(n)\rangle_{\xi\in\Omega,n\in\Bbb N}$ such that
($\alpha$) for each
$\xi\in\Omega$, $\sequencen{\theta_{\xi}(n)}$ is a strictly increasing
sequence with supremum $\xi$ ($\beta$) for any uncountable
$A\subseteq\omega_1$ there is a $\xi\in\Omega$ such that
$\theta_{\xi}(n)\in A$ for every $n\in\Bbb N$.}

\cmmnt{\noindent This is an immediate consequence of Jensen's
$\diamondsuit$ (\JWII, Exercise 22.9), which is itself a consequence
of G\"odel's Axiom of Constructibility (\Kunen, \S II.7;  \JWII, \S22;
\JechGH, \S22;  \JechIC, 13.21).}
%4{}39

\leader{4A1N}{Lemma} Assume $\clubsuit$.   Then there is a family
$\langle C_{\xi}\rangle_{\xi<\omega_1}$ of sets such that (i)
$C_{\xi}\subseteq\xi$ for every $\xi<\omega_1$ (ii) $C_{\xi}\cap\eta$
is finite whenever $\eta<\xi<\omega_1$ (iii) for any uncountable sets
$A$, $B\subseteq\omega_1$ there is a $\xi<\omega_1$ such that
$A\cap C_{\xi}$ and $B\cap C_{\xi}$ are both infinite.

\proof{{\bf (a)} Let
$\langle\theta_{\xi}(n)\rangle_{\xi\in\Omega,n\in\Bbb N}$ be a family
as in 4A1M.   Let $f:\omega_1\to[\omega_1]^2$ be a surjection (3A1Cd).
For $\xi\in\Omega$, set

\Centerline{$C_{\xi}=\bigcup_{i\in\Bbb N}
f(\theta_{\xi}(i+1))\cap\xi\setminus\theta_{\xi}(i)$.}

\noindent Then $C_{\xi}\subseteq\xi$, and if $\eta<\xi$ there is some
$n\in\Bbb N$ such that $\theta_{\xi}(n)\ge\eta$, so that

\Centerline{$C_{\xi}\cap\eta
\subseteq\bigcup_{i\le n}f(\theta_{\xi}(i))$}

\noindent is finite.   For $\xi\in\omega_1\setminus\Omega$ set
$C_{\xi}=\emptyset$.   Then $\langle C_{\xi}\rangle_{\xi<\omega_1}$
satisfies (i) and (ii) above.

\medskip

{\bf (b)} Now suppose that $A$, $B\subseteq\omega_1$ are uncountable.
Choose $\langle\alpha_{\xi}\rangle_{\xi<\omega_1}$,
$\langle\beta_{\xi}\rangle_{\xi<\omega_1}$,
%$\langle\gamma_{\xi}\rangle_{\xi<\omega_1}$,
$\langle I_{\xi}\rangle_{\xi<\omega_1}$ inductively, as follows.
$\beta_{\xi}$ is to be the smallest ordinal such that
$\{\alpha_{\eta}:\eta<\xi\}\cup\bigcup_{\eta<\xi}I_{\eta}
\subseteq\beta_{\xi}$;  $I_{\xi}$ is to be a doubleton subset of
$\omega_1\setminus(\beta_{\xi}\cup\bigcup_{\eta\le\beta_{\xi}}f(\eta))$
meeting both $A$ and $B$;  and $\alpha_{\xi}<\omega_1$ is to be such
that $f(\alpha_{\xi})=I_{\xi}$.   Set
$D=\{\alpha_{\xi}:\xi<\omega_1\}$.
This construction ensures that $\ofamily{\xi}{\omega_1}{\alpha_{\xi}}$
and $\ofamily{\xi}{\omega_1}{\beta_{\xi}}$ are strictly increasing,
with
$\beta_{\xi}<\alpha_{\xi}<\beta_{\xi+1}$ for every $\xi$, so that
$f(\delta)$ meets both $A$ and $B$ for
every $\delta\in D$, while $f(\delta)\subseteq\delta'$ and
$f(\delta')\cap(\delta\cup f(\delta))=\emptyset$
whenever $\delta<\delta'$ in $D$.

By the choice of
$\langle\theta_{\xi}(n)\rangle_{\xi\in\Omega,n\in\Bbb N}$, there is a
$\xi\in\Omega$ such that $\theta_{\xi}(n)\in D$ for
every $n\in\Bbb N$.   But this means that

\Centerline{$f(\theta_{\xi}(i))\subseteq\theta_{\xi}(i+1)\subseteq\xi$,
\quad$f(\theta_{\xi}(i+1))\cap(\theta_{\xi}(i)\cup f(\theta_{\xi}(i)))
=\emptyset$}

\noindent for every $i\in\Bbb N$, so
$C_{\xi}=\bigcup_{i\ge 1}f(\theta_{\xi}(i))$ meets both $A$ and
$B$
in infinite sets.
}%end of proof
%4{}29N

\leader{4A1O}{The size of $\sigma$-algebras:  Proposition} Let $\frak A$
be a Boolean algebra, $B$ a subset of $\frak A$, and $\frak B$ the
$\sigma$-subalgebra of $\frak A$ generated by $B$\cmmnt{ (331E)}.
Then $\#(\frak B)\le\max(4,\#(B^{\Bbb N}))$.   In particular, if
$\#(B)\le\frak c$ then $\#(\frak B)\le\frak c$.

\proof{{\bf (a)} If $\#(B)\le 1$, this is trivial, since then
$\#(\frak B)\le 4$.   So we need consider only the case $\#(B)\ge 2$.

\medskip

{\bf (b)} Set $\kappa=\#(B^{\Bbb N})$;  then whenever
$\#(A)\le\kappa$,
that is, there is an injection from $A$ to $B^{\Bbb N}$, then

\Centerline{$\#(A^{\Bbb N})\le\#((B^{\Bbb N})^{\Bbb N})
=\#(B^{\Bbb N\times\Bbb N})=\#(B^{\Bbb N})=\kappa$.}

\noindent As we are supposing that $B$ has more than one element,
$\kappa\ge\#(\{0,1\}^{\Bbb N})=\frak c\ge\omega_1$.

\medskip

{\bf (c)} Define $\langle B_{\xi}\rangle_{\xi<\omega_1}$ inductively,
as follows.   $B_0=B\cup\{0\}$.   Given $\ofamily{\eta}{\xi}{B_{\eta}}$,
where $0<\xi<\omega_1$, set $B'_{\xi}=\bigcup_{\eta<\xi}B_{\eta}$ and

$$\eqalign{B_{\xi}
&=\{1\Bsetminus b:b\in B'_{\xi}\}\cr
&\qquad\cup\{\sup_{n\in\Bbb N}b_n:\sequencen{b_n}
  \text{ is a sequence in }B'_{\xi}
  \text{ with a supremum in }\frak A\};\cr}$$

\noindent continue.

An easy induction on $\xi$ (relying on 3A1Cc and (b) above) shows that
every $B_{\xi}$ has cardinal at most $\kappa$.   So
$C=\bigcup_{\xi<\omega_1}B_{\xi}$ has cardinal at most $\kappa$.

\medskip

{\bf (d)} Now $\ofamily{\xi}{\omega_1}{B_{\xi}}$ is a non-decreasing
family, so if $\sequencen{c_n}$ is any sequence in $C$ there is some
$\xi<\omega_1$ such that every $c_n$ belongs to
$B_{\xi}\subseteq B'_{\xi+1}$.   But this means that if
$\sup_{n\in\Bbb N}c_n$ is defined in $\frak A$, it belongs to
$B_{\xi+1}\subseteq C$.   At the same time,

\Centerline{$1\Bsetminus c_0\in B_{\xi+1}\subseteq C$.}

\noindent This shows that $C$ is closed under complementation and
countable suprema;  since it contains $0$, it is a $\sigma$-subalgebra
of $\frak A$;  since it includes $B$, it includes $\frak B$, and
$\#(\frak B)\le\#(C)\le\kappa$, as claimed.

\medskip

{\bf (d)} Finally, if $\#(B)\le\frak c$,
$\#(\frak B)\le\max(4,\#(B^{\Bbb N}))\le\frak c$ by 4A1A(c-ii).
}%end of proof of 4A1O

\leader{4A1P}{An incidental fact} If $I$ is a countable set and
$\epsilon>0$, there is a family
$\langle\epsilon_i\rangle_{i\in I}$ of strictly positive real numbers
such that $\sum_{i\in I}\epsilon_i\le\epsilon$.  \prooflet{\Prf\ Let
$f:I\to\Bbb N$ be an injection and set
$\epsilon_i=2^{-f(i)-1}\epsilon$.\ \Qed}
%4{}12 %4{}19  used in 3{}42

\exercises{
%\leader{4A1X}{Basic exercises (a)}

%\leader{4A1Y}{Further exercises (a)}

}%end of exercises

%\cmmnt{\Notesheader{4A1}

%}%end of notes

\discrpage

