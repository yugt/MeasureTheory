\frfilename{mt538.tex}
\versiondate{18.2.14}
\copyrightdate{2009}

\def\chaptername{Topologies and Measures III}
\def\sectionname{Filters and limits}

\def\FS{\mathop{\text{FS}}\nolimits}

\newsection{538}

A great many special types of filter have been studied.   In this section I
look at some which are particularly interesting from the point of view of
measure theory:  Ramsey ultrafilters, measure-converging filters and
filters with the Fatou property.   About half the section is directed
towards Benedikt's theorem (538M) on extensions of perfect probability
measures;  on the way we need to look at
measure-centering ultrafilters (538G-538K) %538G 538H 538I 538J 538K
and iterated
products of filters (538E, 538L).   The second major topic here is a study
of `medial limits' (538P-538S); %538P 538Q 538R 538S
these are Banach limits of a very special type.
In between, the measure-converging property (538N)
and the Fatou property (538O) offer some intriguing patterns.

%538A definitions - filter types
%538B R-K ordering
%538C ultrafilters and R-K
%538D \Cal F\ltimes\Cal G
%538E iteration
%538F Ramsey ufilters
%538H measure-centering ultrafilters
%538G characterizations
%538I extending perfect probabilities
%538J   properties of construction
%538K   more
%538L iterated product of Ramsey ufilters is measure-centering
%538M universal extension of perfect probability
%538N measure-converging filters.
%538O Fatou property
%538P medial functionals:  equivalent conditions
%538Q    definition
%538R    properties
%538S    existence

\leader{538A}{\dvrocolon{Filters}}\cmmnt{ For ease of reference, I begin
the section with a list of the special types of filter on $\Bbb N$
which we shall be looking at later.

\medskip

\noindent}{\bf  Definitions}
Let $\Cal F$ be a filter on $\Bbb N$.

\spheader 538Aa $\Cal F$ is {\bf free} if it\cmmnt{ contains every cofinite
subset of $\Bbb N$, that is,} includes the Fr\'echet filter.

\spheader 538Ab $\Cal F$ is a {\bf $p$-point filter} if it is free and
for every sequence
$\sequencen{A_n}$ in $\Cal F$ there is an $A\in\Cal F$ such that
$A\setminus A_n$ is finite for every $n\in\Bbb N$.
\cmmnt{(Compare 5A6Ga.)}

\spheader 538Ac $\Cal F$ is {\bf Ramsey} or {\bf selective}
if it is free and for every
$f:[\Bbb N]^2\to\{0,1\}$ there is an $A\in\Cal F$ such that $f$ is constant
on $[A]^2$.
%C&N p 212 gives defn for uniform ultrafilters on other cardinals

\spheader 538Ad $\Cal F$ is {\bf rapid} if it is free and for
every sequence $\sequencen{t_n}$
of real numbers which converges to $0$, there is an $A\in\Cal F$ such that
$\sum_{n\in A}|t_n|$ is finite.   Note that a free filter $\Cal F$ on
$\Bbb N$ is rapid iff for every $f\in\BbbN^{\Bbb N}$ there is an
$A\in\Cal F$ such that $\#(A\cap f(k))\le k$ for every $k\in\Bbb N$.
\prooflet{\Prf\ (i) If $\Cal F$ is rapid and $f\in\BbbN^{\Bbb N}$, let
$g\in\BbbN^{\Bbb N}$ be a strictly increasing sequence such that $f\le g$.
Set $t_i=2$ if $i<g(0)$, $\Bover1{k+1}$ if $g(k)\le i<g(k+1)$;  then there
is an $A\in\Cal F$ such that $\sum_{i\in A}t_i$ is finite;  as $\Cal F$ is
free, there is an $A\in\Cal F$ such that $\sum_{i\in A}t_i\le 1$, in which
case $\#(A\cap f(k))\le\#(A\cap g(k))\le k$ for every $k\in\Bbb N$.
(ii) If $\Cal F$ satisfies the condition and $\sequence{i}{t_i}\to 0$, take
a strictly increasing
$f\in\BbbN^{\Bbb N}$ such that $|t_i|\le 2^{-k}$ whenever $k\in\Bbb N$ and
$i\ge f(k)$;  let $A\in\Cal F$ be such that $\#(A\cap f(k))\le k$ for every
$k$;  then $\sum_{i\in A}|t_i|
\le\sum_{k=0}^{\infty}2^{-k}\#(A\cap f(k+1)\setminus f(k))$ is finite.\
\Qed}

\spheader 538Ae $\Cal F$ is {\bf nowhere dense} if for every sequence
$\sequencen{t_n}$ in $\Bbb R$ there is an $A\in\Cal F$ such that
$\{t_n:n\in A\}$ is nowhere dense.

\spheader 538Af $\Cal F$ is {\bf measure-centering} or has {\bf property M}
if whenever $\frak A$ is a Boolean algebra,
$\nu:\frak A\to\coint{0,\infty}$ is an additive functional, and
$\sequencen{a_n}$ is a sequence in $\frak A$ such that
$\inf_{n\in\Bbb N}\nu a_n>0$, there is an $A\in\Cal F$ such that
$\{a_n:n\in A\}$ is centered.

\spheader 538Ag $\Cal F$ is {\bf measure-converging} if
whenever $(X,\Sigma,\mu)$ is a probability space,
$\sequencen{E_n}$ is a sequence in
$\Sigma$, and $\lim_{n\to\infty}\mu E_n=1$, then
$\bigcup_{A\in\Cal F}\bigcap_{n\in A}E_n$ is conegligible.
%this is "measure-converging over the Frechet filter"

\spheader 538Ah $\Cal F$ has the {\bf Fatou property} if whenever
$(X,\Sigma,\mu)$ is a probability space, $\sequencen{E_n}$ is a sequence in
$\Sigma$, and $X=\bigcup_{A\in\Cal F}\bigcap_{n\in A}E_n$, then
$\lim_{n\to\Cal F}\mu E_n$ is defined and equal to $1$.

\spheader 538Ai For any countably infinite set $I$, I will say that a
filter $\Cal F$ on $I$ is free, or a $p$-point filter, or Ramsey, etc., if
it is isomorphic to such a filter on $\Bbb N$.   \cmmnt{Of course this
usage is
possible only because every property here is invariant under permutations
of $\Bbb N$.}   For `rapid' and `measure-converging'
filters, we need an appropriate translation of
`sequence converging to $0$';  the corresponding notion on an arbitrary
index set $I$ is a function $u\in\pmb{c}_0(I)$, that is,
a real-valued function $u$ on $I$ such that
$\{i:i\in I$, $|u(i)|\ge\epsilon\}$ is finite for every
$\epsilon>0$\cmmnt{;  if we give $I$
its discrete topology, $c_0(I)$ is $C_0(I)$ as defined in 436I}.

%rare=Q-point:  for any partition of \Bbb N into finite sets,
%\exists A\in\Cal F  meeting each member of the partition in at most
%one point.   Rare => rapid;  Ramsey <=> rare p-point ultrafilter.

\leader{538B}{}\cmmnt{ We need a number of basic ideas which can
profitably be examined in a rather more general context.   I start with a
fundamental pre-order on the class of all filters.

\medskip

\noindent}{\bf The Rudin-Keisler ordering} If
$\Cal F$, $\Cal G$ are filters on sets $I$, $J$ respectively, I will
say that
$\Cal F\leRK\Cal G$ if there is a function $f:J\to I$ such that

\Centerline{$\Cal F=f[[\Cal G]]=\{A:A\subseteq I$, $f^{-1}[A]\in\Cal G\}$,}

\noindent the filter on $I$ generated by
$\{f[B]:B\in\Cal G\}$.\cmmnt{  (I ought to remark that while this is a
standard idea for ultrafilters, in the case of general filters the
terminology is not well established.)}\cmmnt{   Of course}
$\leRK$ is reflexive and transitive.   If $\Cal F\leRK\Cal G$ and
$\Cal G$ is an ultrafilter, then $\Cal F$ is an
ultrafilter\cmmnt{ (2A1N)}.   If
$\Cal F$ is a principal ultrafilter then $\Cal F\leRK\Cal G$ for every
filter $\Cal G$.
%check Mildenberger
%C&N p 206 for ultrafilters

\leader{538C}{Lemma} (a) If $I$ is a set, $\Cal F$ is an ultrafilter on
$I$ and $f:I\to I$ is a function such that $f[[\Cal F]]=\Cal F$, then
$\{i:f(i)=i\}\in\Cal F$.
%C&N 9.2a

(b) If $I$ is a set, $\Cal F$ and $\Cal G$ are ultrafilters on $I$,
$\Cal F\leRK\Cal G$ and $\Cal G\leRK\Cal F$, then there is a permutation
$h:I\to I$ such that $h[[\Cal F]]=\Cal G$;  that is, $\Cal F$ and
$\Cal G$ are isomorphic.
%C&N 9.2b/9.3

\proof{{\bf (a)} It is enough to consider the case in which $I=\kappa$ is a
cardinal.

\medskip

\quad{\bf (i)} $\{\xi:\xi<\kappa$, $\xi\le f(\xi)\}\in\Cal F$.
\Prf\ Define $\sequencen{D_n}$, $\sequencen{E_n}$ by saying that

\Centerline{$D_0=\kappa$,
\quad$D_{n+1}=\{\xi:\xi\in D_n$, $f(\xi)\in D_n$, $f(\xi)<\xi\}$,
\quad$E_n=D_n\setminus D_{n+1}$}

\noindent for $n\in\Bbb N$.   If $\xi\in D_n$ then
$\xi>f(\xi)>\ldots>f^n(\xi)$, so $\bigcap_{n\in\Bbb N}D_n=\emptyset$ and
$\sequencen{E_n}$ is a partition of $\kappa$.   If $\xi\in E_{n+1}$ then
$f^{n+1}(\xi)<f^n(\xi)<\ldots<\xi$, $f^{n+1}(\xi)\le f^{n+2}(\xi)$, so
$f(\xi)\in E_n$.   Set $E=\bigcup_{n\ge 1}E_{2n}$,
$E'=\bigcup_{n\in\Bbb N}E_{2n+1}$;  then $f[E]\subseteq E'$ is disjoint
from $E$, so $E\notin\Cal F$.   Also $f[E']\subseteq E\cup E_0$ is disjoint
from $E'$, so $E'\notin\Cal F$.   Because $\Cal F$ is an ultrafilter,
$E_0\in\Cal F$, as claimed.\ \Qed

\medskip

\quad{\bf (ii)} If $A\subseteq I$ and $A\notin\Cal F$ then
$B=\bigcup_{n\in\Bbb N}(f^n)^{-1}[A]$ does not belong to $\Cal F$.   \Prf\
For $\xi\in B$ set $m(\xi)=\min\{n:n\in\Bbb N$, $f^n(\xi)\in A\}$.
If $m(\xi)>0$ then $m(f(\xi))=m(\xi)-1$.   So setting
$C=\{\xi:m(\xi)$ is even and not $0\}$, $C'=\{\xi:m(\xi)$ is odd$\}$ we
have $f[C]\cap C=\emptyset$, $f[C']\cap C'=\emptyset$ and
$B\subseteq A\cup C\cup C'$;  so $B\notin\Cal F$.\ \Qed

Turning this round, if $A\in\Cal F$ then
$\bigcup_{n\in\Bbb N}(f^n)^{-1}[\kappa\setminus A]\notin\Cal F$ and
$\bigcap_{n\in\Bbb N}(f^n)^{-1}[A]\in\Cal F$.

\medskip

\quad{\bf (iii)} For $\xi<\kappa$ set

\Centerline{$g(\xi)=\min\{\zeta:$ there is some $n\in\Bbb N$ such that
$f^n(\zeta)=\xi\}$.}

\noindent Then $g[[\Cal F]]=\Cal F$.   \Prf\ If $A\in\Cal F$ then
$\Cal F$ contains $\bigcap_{n\in\Bbb N}(f^n)^{-1}[A]\subseteq g^{-1}[A]$,
so $g^{-1}[A]\in\Cal F$.   Thus $\Cal F\subseteq g[[\Cal F]]$;  as $\Cal F$
is an ultrafilter, $\Cal F=g[[\Cal F]]$.\ \Qed

Now $g(\xi)\le\xi$ for every $\xi<\kappa$;  applying (i) to $g$,
we see that $G=\{\xi:g(\xi)=\xi\}\in\Cal F$.   But consider
$H=\{\xi:\xi<f(\xi)\}$.   Then $g(\eta)<\eta$ for every $\eta\in f[H]$, so
$f[H]\notin\Cal F$ and $H\notin\Cal F$.   Since we already know that
$\{\xi:\xi\le f(\xi)\}\in\Cal F$, we see that $\{\xi:f(\xi)=\xi\}$ belongs
to $\Cal F$, as claimed.

\medskip

{\bf (b)} Let $f$, $g:I\to I$ be such that $f[[\Cal F]]=\Cal G$ and
$g[[\Cal G]]=\Cal F$.   Then $(gf)[[\Cal F]]=g[[f[[\Cal F]]\,]]=\Cal F$, so
$J_0=\{i:g(f(i))=i\}\in\Cal F$, by (a).   Similarly, $J_1=\{i:f(g(i))=i\}$
belongs to $\Cal G$.   Set $J=J_0\cap f^{-1}[J_1]\in\Cal F$;
then $g(f(i))=i$ for
every $i\in J$ and $f(g(j))=j$ for every $j\in f[J]$, so $f\restr J$ and
$g\restr f[J]$ are inverse bijections between $J\in\Cal F$ and
$f[J]\in\Cal G$.   If $J$ is finite, then certainly
$\#(I\setminus J)=\#(I\setminus f[J])$ and there is an extension of
$f\restr J$ to a permutation of $I$.   If $J$ is infinite, let
$J'\subseteq J$ be a set such that $\#(J')=\#(J\setminus J')=\#(J)$ and
$J'\in\Cal F$;  then $\#(I\setminus J')=\#(I\setminus f[J'])=\#(I)$
so there is an extension of $f\restr J'$ to a permutation of $I$.

Thus in either case we have a permutation
$h:I\to I$ and a $K\in\Cal F$ such
that $K\subseteq J$ and $h\restr K=f\restr K$.   But now
$h[[\Cal F]]=\Cal G$ and $h$ is an isomorphism between $(I,\Cal F)$ and
$(I,\Cal G)$.
}%end of proof of 538C

\leader{538D}{Finite products of filters (a)}
Suppose that $\Cal F$, $\Cal G$ are filters on sets $I$, $J$ respectively.
I will write $\Cal F\ltimes\Cal G$ for

\Centerline{$\{A:A\subseteq I\times J$,
$\{i:A[\{i\}]\in\Cal G\}\in\Cal F\}$.}

\noindent\cmmnt{It is easy to check that }$\Cal F\ltimes\Cal G$
is a filter.
\cmmnt{ (Compare the skew product $\Cal I\ltimes\Cal J$ of ideals
defined in 527Ba.)}

\spheader 538Db
If $\Cal F$ and $\Cal G$ are ultrafilters, so is $\Cal F\ltimes\Cal G$.
\prooflet{\Prf\ If $A\subseteq I\times J$ and $A\notin\Cal F\ltimes\Cal G$,
then $\{i:A[\{i\}]\in\Cal G\}\notin\Cal F\}$ and

\Centerline{$\{i:((I\times J)\setminus A)[\{i\}]\in\Cal G\}
=\{i:i\in I,\,J\setminus A[\{i\}]\in\Cal G\}
=I\setminus\{i:A[\{i\}]\in\Cal G\}
\in\Cal F$,}

\noindent so $(I\times J)\setminus A\in\Cal F\ltimes\Cal G$.\ \Qed}

\spheader 538Dc If $\Cal F$, $\Cal G$ and $\Cal H$ are filters on $I$,
$J$, $K$ respectively, then the natural bijection between
$(I\times J)\times K$ and $I\times(J\times K)$ is an isomorphism between
$(\Cal F\ltimes\Cal G)\ltimes\Cal H$ and
$\Cal F\ltimes(\Cal G\ltimes\Cal H)$.   \prooflet{\Prf\ If
$A\subseteq I\times(J\times K)$ and $B=\{(i,j),k):(i,(j,k))\in A\}$, then

$$\eqalign{A\in\Cal F\ltimes(\Cal G\ltimes\Cal H)
&\iff\{i:A[\{i\}]\in\Cal G\times\Cal H\}\in\Cal F\cr
&\iff\{i:\{j:(A[\{i\}])[\{j\}]\in\Cal H\}\in\Cal G\}\in\Cal F\cr
&\iff\{(i,j):(A[\{i\}])[\{j\}]\in\Cal H\}\in\Cal F\ltimes\Cal G\cr
&\iff\{(i,j):B[\{(i,j)\}]\in\Cal H\}\in\Cal F\ltimes\Cal G\cr
&\iff B\in(\Cal F\ltimes\Cal G)\ltimes\Cal H.  \text{ \Qed}\cr}$$}

\spheader 538Dd It follows that we can define a product
$\Cal F_0\ltimes\ldots\ltimes\Cal F_n$ of any finite string
$\Cal F_0,\ldots,\Cal F_n$ of filters, and under the natural
identifications of the base sets we shall have
$(\Cal F_0\ltimes\ldots\ltimes\Cal F_n)
\ltimes(\Cal F_{n+1}\ltimes\ldots\ltimes\Cal F_m)$ identified with
$\Cal F_0\ltimes\ldots\ltimes\Cal F_m$ whenever
$\Cal F_0,\ldots,\Cal F_n,\ldots,\Cal F_m$ are filters.

\spheader 538De\cmmnt{ For any filters $\Cal F$ and $\Cal G$,
$\Cal F\leRK\Cal F\ltimes\Cal G$ and $\Cal G\leRK\Cal F\ltimes\Cal G$.
\prooflet{\Prf\ Taking the base sets to be $I$, $J$ respectively and
$f(i,j)=i$, $g(i,j)=j$ for $i\in I$ and $j\in J$, we have
$\Cal F=f[[\Cal F\ltimes\Cal G]]$ and $\Cal G=g[[\Cal F\ltimes\Cal G]]$.\
\Qed}

Inducing on $n$, we see that
$\Cal F_n\leRK\Cal F_0\ltimes\ldots\ltimes\Cal F_n$ whenever
$\Cal F_0,\ldots,\Cal F_n$ are filters;  consequently}
$\Cal F_m\leRK\Cal F_0\ltimes\ldots\ltimes\Cal F_n$ whenever
$\Cal F_0,\ldots,\Cal F_n$ are filters and $m\le n$.

\spheader 538Df If $\Cal F$, $\Cal F'$, $\Cal G$ and $\Cal G'$ are
filters, with $\Cal F\leRK\Cal F'$ and $\Cal G\leRK\Cal G'$, then
$\Cal F\ltimes\Cal G\leRK\Cal F'\ltimes\Cal G'$.   \prooflet{\Prf\ Let the
base sets of the filters be $I$, $I'$, $J$ and $J'$, and let $f:I'\to I$
and $g:J'\to J$ be such that $\Cal F=f[[\Cal F']]$ and
$\Cal G=g[[\Cal G']]$.   Set $h(i,j)=(f(i),g(j))$ for $i\in I$ and
$j\in J$.   If $A\subseteq I\times J$, then

$$\eqalign{h^{-1}[A]\in\Cal F'\ltimes\Cal G'
&\iff\{i:(h^{-1}[A])[\{i\}]\in\Cal G'\}\in\Cal F'\cr
&\iff\{i:g^{-1}[A[\{f(i)\}]]\in\Cal G'\}\in\Cal F'\cr
&\iff\{i:A[\{f(i)\}]\in\Cal G\}\in\Cal F'\cr
&\iff\{i:A[\{i\}]\in\Cal G\}\in\Cal F
\iff A\in\Cal F\ltimes\Cal G.\cr}$$

\noindent So $\Cal F\ltimes\Cal G=h[[\Cal F'\ltimes\Cal G']]$ and
$\Cal F\ltimes\Cal G\leRK\Cal F'\ltimes\Cal G'$.\ \Qed}

\cmmnt{Accordingly} $\Cal F_0\ltimes\ldots\ltimes\Cal F_n
\leRK\Cal G_0\ltimes\ldots\ltimes\Cal G_n$ whenever
$\Cal F_i\leRK\Cal G_i$ for every $i\le n$.

\spheader 538Dg It follows that if $\Cal F_0,\ldots,\Cal F_n$ are filters
and $k_0<\ldots<k_m\le n$, then
$\Cal F_{k_0}\ltimes\ldots\ltimes\Cal F_{k_m}
\leRK\Cal F_0\ltimes\ldots\ltimes\Cal F_n$.   \prooflet{\Prf\ Induce on $m$
to see that $\Cal F_{k_0}\ltimes\ldots\ltimes\Cal F_{k_m}
\leRK\Cal F_0\ltimes\ldots\ltimes\Cal F_{k_m}$.\ \Qed}

\vleader{72pt}{538E}{}\cmmnt{ There are many variations on the construction
here.   A fairly elaborate extension will be needed in 538L below.

\medskip

\noindent}{\bf Iterated products of filters (a)}\cmmnt{ First,
a scrap of notation for the rest of the first half of this section (down to
538M).}   Set $S=\bigcup_{i\in\Bbb N}\BbbN^i$.
If $i$, $j\in\Bbb N$, $\sigma\in\BbbN^i$ and
$\tau\in\BbbN^j$, define $\sigma^{\smallfrown}\tau\in\BbbN^{i+j}$ by
setting

$$\eqalign{(\sigma^{\smallfrown}\tau)(k)
&=\sigma(k)\text{ if }k<i,\cr
&=\tau(k-i)\text{ if }i\le k<i+j.\cr}$$

\noindent For $k\in\Bbb N$ write $\fraction{k}$ for the member of $\BbbN^1$
with value $k$.

Fix on a family
$\langle\theta(\xi,k)\rangle_{1\le\xi<\omega_1,k\in\Bbb N}$
such that each $\sequence{k}{\theta(\xi,k)}$ is a non-decreasing sequence
running over a cofinal subset of $\xi$.
\cmmnt{(You will
probably prefer to suppose that when $\xi=\eta+1$ is a successor ordinal,
then $\theta(\xi,k)=\eta$ for every $k\in\Bbb N$.)}

\spheader 538Eb Now suppose that $\zeta$ is a non-zero
countable ordinal.
Let $\langle\Cal F_{\xi}\rangle_{1\le\xi\le\zeta}$ be a family of
filters on
$\Bbb N$.   For $\xi\le\zeta$, define $\Cal G_{\xi}\subseteq\Cal PS$ as
follows.   Start by taking $\Cal G_0$ to be
the principal filter generated by
$\{\emptyset\}$.   For $1\le\xi\le\zeta$, set

\Centerline{$\Cal G_{\xi}
=\{A:A\subseteq S,\,\{k:k\in\Bbb N,\,
    \{\tau:\fraction{k}^{\smallfrown}\tau\in A\}
    \in\Cal G_{\theta(\xi,k)}\}\in\Cal F_{\xi}\}$.}

\noindent It is elementary to check that every $\Cal G_{\xi}$ is a
filter, and that if every $\Cal F_{\xi}$ is free, so is every
$\Cal G_{\xi}$.
Moreover, if every $\Cal F_{\xi}$ is an ultrafilter, so is every
$\Cal G_{\xi}$.

\spheader 538Ec \cmmnt{Continuing from (b), we find that}
$\Cal F_{\xi}\leRK\Cal G_{\xi}$ whenever $1\le\xi\le\zeta$
and $\Cal G_{\eta}\leRK\Cal G_{\xi}$
whenever $0\le\eta\le\xi\le\zeta$.   \prooflet{\Prf\ Induce on $\xi$.
(i) If $\xi\ge 1$, define $f:S\to\Bbb N$ by setting $f(\tau)=\tau(0)$ if
$\tau\ne\emptyset$,
$f(\emptyset)=0$.   Then, for $B\subseteq\Bbb N$,

$$\eqalign{f^{-1}[B]\in\Cal G_{\xi}
&\iff\{k:\{\tau:\fraction{k}^{\smallfrown}\tau\in f^{-1}[B]\}
   \in\Cal G_{\theta(\xi,k)}\}\in\Cal F_{\xi}\cr
&\iff B\in\Cal F_{\xi},\cr}$$

\noindent so $\Cal F_{\xi}=f[[\Cal G_{\xi}]]\leRK\Cal G_{\xi}$.
(ii) If $\eta=\xi\le\zeta$ then of course $\Cal G_{\eta}\leRK\Cal G_{\xi}$.
(iii) If $0\le\eta<\xi$ then there is a $k_0$ such that
$\eta\le\theta(\xi,k)$
for $k\ge k_0$.   For $k\ge k_0$,
$\Cal G_{\eta}\leRK\Cal G_{\theta(\xi,k)}$ by the inductive hypothesis;
let $g_k:S\to S$ be such that
$\Cal G_{\eta}=g_k[[\Cal G_{\theta(\xi,k)}]]$.   Now define $g:S\to S$
by setting

$$\eqalign{g(\tau)
&=g_k(\sigma)\text{ if }k\ge k_0
    \text{ and }\tau=\fraction{k}^{\smallfrown}\sigma,\cr
&=\emptyset\text{ otherwise}.\cr}$$

\noindent For $B\subseteq S$,

$$\eqalignno{g^{-1}[B]\in\Cal G_{\xi}
&\iff\{k:\{\sigma:\fraction{k}^{\smallfrown}\sigma\in g^{-1}[B]\}
  \in\Cal G_{\theta(\xi,k)}\}\in\Cal F_{\xi}\cr
&\iff\{k:k\ge k_0,\,
  \{\sigma:g(\fraction{k}^{\smallfrown}\sigma)\in B\}
  \in\Cal G_{\theta(\xi,k)}\}\in\Cal F_{\xi}\cr
\displaycause{because $\Cal F_{\xi}$ is free}
&\iff\{k:k\ge k_0,\,
  \{\sigma:g_k(\sigma)\in B\}
  \in\Cal G_{\theta(\xi,k)}\}\in\Cal F_{\xi}\cr
&\iff\{k:k\ge k_0,\,
  B\in\Cal G_{\eta}\}\in\Cal F_{\xi}
\iff B\in\Cal G_{\eta},\cr}$$

\noindent so $\Cal G_{\eta}=g[[\Cal G_{\xi}]]\leRK\Cal G_{\xi}$.\ \Qed}

\spheader 538Ed\cmmnt{ It follows that if
$1\le\xi_0<\ldots<\xi_n\le\zeta$ then
$\Cal F_{\xi_n}\ltimes\ldots\ltimes\Cal F_{\xi_0}\leRK\Cal G_{\xi_n}$.
\prooflet{\Prf\ Induce on the pair $(\xi_n,n)$.   If $\xi_n=1$ then $n=0$
and we just have $\Cal F_1\leRK\Cal G_1$, as in part (i) of the proof of
(c).   For the inductive step to
$\xi_n=\xi>1$, if $n=0$ then again we need only note that
$\Cal F_{\xi_0}=\Cal F_{\xi}\leRK\Cal G_{\xi}$.    If $n>0$, let
$k_0\ge 1$ be such that $\xi_{n-1}\le\theta(\xi,k)$ for every
$k\ge k_0$.   For $k\ge k_0$,

\Centerline{$\Cal F_{\xi_{n-1}}\ltimes\ldots\ltimes\Cal F_{\xi_0}
\leRK\Cal G_{\xi_{n-1}}\le\Cal G_{\theta(\xi,k)}$}

\noindent by the inductive hypothesis, so
we have a function $g_k:S\to\BbbN^n$ such that
$\Cal F_{\xi_{n-1}}\ltimes\ldots\ltimes\Cal F_{\xi_0}
=g_k[[\Cal G_{\theta(\xi,k)}]]$.   Define $g:S\to\BbbN^{n+1}$ by setting

$$\eqalign{g(\tau)
&=\fraction{k}^{\smallfrown}g_k(\sigma)\text{ if }k\ge k_0
    \text{ and }\tau=\fraction{k}^{\smallfrown}\sigma,\cr
&=\text{the constant function with value }0\text{ otherwise}.\cr}$$

\noindent Then, for $B\subseteq\BbbN^{n+1}$,

$$\eqalignno{g^{-1}[B]\in\Cal G_{\xi}
&\iff\{k:\{\sigma:g(\fraction{k}^{\smallfrown}\sigma)\in B\}
  \in\Cal G_{\theta(\xi,k)}\}\in\Cal F_{\xi}\cr
&\iff\{k:k\ge k_0,\,\{\sigma:\fraction{k}^{\smallfrown}g_k(\sigma)\in B\}
  \in\Cal G_{\theta(\xi,k)}\}\in\Cal F_{\xi}\cr
&\iff\{k:k\ge k_0,\,\{\sigma:g_k(\sigma)\in B_k\}
  \in\Cal G_{\theta(\xi,k)}\}\in\Cal F_{\xi}\cr
\displaycause{writing
$B_k=\{\sigma:\fraction{k}^{\smallfrown}\sigma\in B\}\subseteq\BbbN^n$
for $k\in\Bbb N$}
&\iff\{k:k\ge k_0,\,
  B_k\in\Cal F_{\xi_{n-1}}\ltimes\ldots\ltimes\Cal F_{\xi_0}\}
  \in\Cal F_{\xi}\cr
&\iff\{k:k\in\Bbb N,\,
  B_k\in\Cal F_{\xi_{n-1}}\ltimes\ldots\ltimes\Cal F_{\xi_0}\}
  \in\Cal F_{\xi}\cr
&\iff B\in\Cal F_{\xi_n}\ltimes\ldots\ltimes\Cal F_{\xi_0}.\cr}$$

\noindent Thus $g$ witnesses that
$\Cal F_{\xi_n}\ltimes\ldots\ltimes\Cal F_{\xi_0}\leRK\Cal G_{\xi_n}$, and
the induction proceeds.\ \Qed}

Consequently}
$\Cal F_{\xi_n}\ltimes\ldots\ltimes\Cal F_{\xi_0}\leRK\Cal G_{\zeta}$
whenever $1\le\xi_0<\ldots<\xi_n\le\zeta$.

\spheader 538Ee \cmmnt{The following special remark will be useful in
Theorem 538L.}
Suppose that we are given $A_{\xi}\in\Cal F_{\xi}$ for each
$\xi\in[1,\zeta]$.   Define $T\subseteq S$ and $\alpha:T\to[0,\zeta]$ as
follows.   Start by saying that $\emptyset\in T$ and
$\alpha(\emptyset)=\zeta$.   Having determined $T\cap\BbbN^n$ and
$\alpha\restr T\cap\BbbN^n$, where $n\in\Bbb N$, then for
$\tau\in\BbbN^{n+1}$ say that $\tau\in T$ iff $\tau$ is of the form
$\sigma^{\smallfrown}\fraction{k}$ where

\Centerline{$\sigma\in T\cap\BbbN^n$,
\quad$\alpha(\sigma)>0$,
\quad$k\in A_{\alpha(\sigma)}$,
\quad$\sigma(i)<k$ for every $i<n$,}

\noindent and in this case set $\alpha(\tau)=\theta(\alpha(\sigma),k)$.
Continue.   Observe that $\alpha(\tau)<\alpha(\sigma)$ whenever $\sigma$,
$\tau\in T$ and $\tau$ properly extends $\sigma$.

Suppose that $D\in\bigcap_{1\le\xi\le\zeta}\Cal F_{\xi}$.   Then
$T^*_D=\{\tau:\tau\in T\cap\bigcup_{n\in\Bbb N}D^n$, $\alpha(\tau)=0\}$
belongs to $\Cal G_{\zeta}$.
\prooflet{\Prf\ I aim to show by induction on
$\xi$ that if $\tau\in T\cap\bigcup_{n\in\Bbb N}D^n$ and
$\alpha(\tau)=\xi$ then
$\{\sigma:\tau^{\smallfrown}\sigma\in T_D^*\}$ belongs to $\Cal G_{\xi}$.
If $\xi=0$ then of course
$\{\sigma:\tau^{\smallfrown}\sigma\in T_D^*\}=\{\emptyset\}\in\Cal G_0$.
For the inductive step to $\xi>0$,

$$\eqalignno{\{k:
  \{\sigma:\tau^{\smallfrown}\fraction{k}^{\smallfrown}&\sigma\in T_D^*\}
       \in\Cal G_{\theta(\xi,k)}\}\cr
&\supseteq\{k:k\in D,\,\tau^{\smallfrown}\fraction{k}\in T,\,
    \alpha(\tau^{\smallfrown}\fraction{k})=\theta(\xi,k)\}\cr
\displaycause{by the inductive hypothesis}
&\supseteq\{k:k\in A_{\xi}\cap D,\,
    \tau(i)<k\text{ for every }i<\dom\tau\}\cr
&\in\Cal F_{\xi},\cr}$$

\noindent so $\{\sigma:\tau^{\smallfrown}\sigma\in T_D^*\}\in\Cal G_{\xi}$.
At the end of the induction, we can apply this to $\tau=\emptyset$
and $\xi=\zeta$.\ \Qed}

\leader{538F}{Ramsey \dvrocolon{filters}}\cmmnt{ There is an extensive
and fascinating theory of Ramsey filters;
see, for instance, {\smc Comfort
\& Negrepontis 74}.   Here, however, I will give only those fragments which
are directly relevant to the other work of this section.

\medskip

\noindent}{\bf Proposition} (a) A Ramsey filter on $\Bbb N$ is a
rapid $p$-point ultrafilter.

(b) If $\Cal F$ is a Ramsey ultrafilter on $\Bbb N$, $\Cal G$ is a
non-principal ultrafilter on $\Bbb N$, and $\Cal G\leRK\Cal F$,
then $\Cal F$ and $\Cal G$ are isomorphic and $\Cal G$
is a Ramsey ultrafilter.

(c) Let $\Cal F$ be a Ramsey filter on $\Bbb N$.
Suppose that $\sequencen{A_n}$ is any sequence in $\Cal F$.   Then
there is an $A\in\Cal F$ such that $n\in A_m$ whenever $m$, $n\in A$ and
$m<n$.

(d) Let $\Cal F$ be a Ramsey filter on $\Bbb N$.   Let
$\Cal S\subseteq[\Bbb N]^{<\omega}$ be such that $\emptyset\in\Cal S$ and
$\{n:I\cup\{n\}\in\Cal S\}\in\Cal F$ for every $I\in\Cal S$.   Then there
is an $A\in\Cal F$ such that $[A]^{<\omega}\subseteq\Cal S$.

(e) If $\frak F$ is a countable family of distinct Ramsey filters on
$\Bbb N$, there is a disjoint family $\family{\Cal F}{\frak F}{A_{\Cal F}}$
of subsets of $\Bbb N$ such that $A_{\Cal F}\in\Cal F$ for every
$\Cal F\in\frak F$.

(f) Let $\frak F$ be a countable
family of non-isomorphic Ramsey ultrafilters on $\Bbb N$, and
$\frak h:\Bbb N\to[\frak F]^{<\omega}$ a function.
Suppose that we are given
an $A_{\Cal F}\in\Cal F$ for each $\Cal F\in\frak F$.
Then there is an $A\in\bigcap\frak F$ such that whenever $i$, $j\in A$,
$\Cal F\in\frak h(i)$ and $i<j$, there is a $k\in A_{\Cal F}$ such that
$i<k<j$.

(g) If $\frakmctbl=\frak c$, there is a Ramsey ultrafilter on $\Bbb N$.

\woddheader{538F}{0}{0}{0}{54pt}

\proof{{\bf (a)} Let $\Cal F$ be a Ramsey filter on $\Bbb N$.

\medskip

\quad{\bf (i)} $\Cal F$ is an ultrafilter.   \Prf\ Let $A$ be any subset of
$\Bbb N$.   Define $f:[\Bbb N]^2\to\{0,1\}$ by setting $f(I)=1$ if
$\#(I\cap A)=1$, $0$ otherwise.   Then we have an $I\in\Cal F$ such that
$f$ is constant on $[I]^2$.   As $\Cal F$ is free, $\#(I)\ge 3$ and
the constant value of $f$ cannot be $1$.   So either $I\subseteq A$ and
$A\in\Cal F$, or $I\cap A=\emptyset$ and $\Bbb N\setminus A\in\Cal F$.
As $A$ is arbitrary, $\Cal F$ is an ultrafilter.\ \Qed

\medskip

\quad{\bf (ii)} $\Cal F$ is a $p$-point filter.   \Prf\ Let
$\sequencen{I_n}$ be a sequence in $\Cal F$.   Set
$K_n=(\Bbb N\setminus n)\cap\bigcap_{i<n}I_i$,
$J_n=K_n\setminus K_{n+1}$ for each $n$;
then $\sequencen{J_n}$ is a partition of $\Bbb N$.
Define $f:[\Bbb N]^2\to\{0,1\}$ by
setting $f(a)=0$ if there is an $n\in\Bbb N$ such that $a\subseteq J_n$,
$1$ otherwise.   Let $I\in\Cal F$ be such that $f$ is constant on $[I]^2$.

Since $\Bbb N\setminus J_n\in\Cal F$ for every $n$, there must be two
points in $I$ belonging to different $J_n$;  so that the constant value of
$f$ must be $1$, and no two points of $I$ belong to the same $J_n$.
In particular, $I\cap J_n$ is always finite, and
$I\setminus I_n\subseteq\bigcup_{i\le n}I\cap J_i$ is always finite.
As $\sequencen{I_n}$ is arbitrary, $\Cal F$ is a $p$-point filter.\ \Qed

\medskip

\quad{\bf (iii)} $\Cal F$ is rapid.   \Prf\ Let $\sequencen{t_n}$ be a
sequence converging to $0$.   For each $n$, set
$I_n=\{i:|t_i|\le 2^{-n}\}$;  as $\Cal F$ is free, $I_n\in\Cal F$.
Looking again at the proof of (ii) above, we see that the construction
there gives us an $I\in\Cal F$ such that $\#(I\setminus I_n)\le n+1$ for
every $n$.   We can therefore enumerate $I$ as $\sequencen{k_n}$ in such a
way that $k_{n+1}\in I_n$ for every $n$.   But this means that

\Centerline{$\sum_{i\in I}|t_i|=\sum_{n=0}^{\infty}|t_{k_n}|
\le|t_{k_0}|+\sum_{n=1}^{\infty}2^{-n+1}<\infty$.}

\noindent As $\sequencen{t_n}$ is arbitrary, $\Cal F$ is rapid.\ \Qed

\medskip

{\bf (b)} Let $f:\Bbb N\to\Bbb N$ be
such that $f[[\Cal F]]=\Cal G$.   For $K\in[\Bbb N]^2$, set
$h(K)=0$ if $f\restr K$ is constant, $1$ otherwise.
Then there is an $A\in\Cal F$ such that $h$ is constant on $[A]^2$,
that is, $f$ is either
constant or injective on $A$.   Since $f[A]\in\Cal G$, $f[A]$ is infinite,
so $f$ is injective on $A$.   Let $g:\Bbb N\to\Bbb N$ be any function
extending $(f\restr A)^{-1}$;  then $gf(n)=n$ for every $n\in A$, so

\Centerline{$(gf)[[\Cal F]]
=\{I:(gf)^{-1}[I]\in\Cal F\}
=\{I:A\cap(gf)^{-1}[I]\in\Cal F\}
=\{I:A\cap I\in\Cal F\}
=\Cal F$.}

\noindent But this means that $g[[\Cal G]]=\Cal F$ and $\Cal F\leRK\Cal G$.

By 538Cb, $\Cal F$ and $\Cal G$ are isomorphic, so $\Cal G$ also must be a
Ramsey ultrafilter.

\medskip

{\bf (c)} For $m<n$ in $\Bbb N$, set $h(\{m,n\})=1$ if $n\in A_m$, $0$
otherwise.   Then there is an $A\in\Cal F$ such that $h\restr[A]^2$ is
constant.   Setting $k=\min A$, $A$ meets $A_k\setminus(k+1)$, so $h$ takes
the value $1$ on $[A]^2$;   consequently
$n\in A_m$ whenever $m$, $n\in A$ and $m<n$.

\medskip

{\bf (d)} For $n\in\Bbb N$, set

\Centerline{$A_n
=\{i:I\cup\{i\}\in\Cal S$ whenever $I\subseteq n+1$ and $I\in\Cal S\}
\in\Cal F$.}

\noindent By (c), there is an $A\in\Cal F$ such that $n\in A_m$ whenever
$m$, $n\in A$ and $m<n$;  and we can suppose that $A\subseteq A_0$, so that
$\{n\}\in\Cal S$ for every $n\in A$.   Now an easy induction on $n$ shows
that $\Cal P(A\cap n)\subseteq\Cal S$ for every $n$, so
$[A]^{<\omega}\subseteq\Cal S$.

\medskip

{\bf (e)} Enumerate $\frak F$ as $\ofamily{n}{\#(\frak F)}{\Cal F_n}$.
For distinct $m$, $n<\#(\frak F)$ there is a
$B_{mn}\in\Cal F_m\setminus\Cal F_n$.
\Prf\ We know that there is a set in $\Cal F_m\symmdiff\Cal F_n$;  now
either this set or its complement will serve for $B_{mn}$.\ \QeD\  Because
every member of $\frak F$ is a $p$-point filter ((a) above), we can find
for each $n<\#(\frak F)$ a set $C_n\in\Cal F_n$ such that
$C_n\setminus(B_{nm}\setminus B_{mn})$ is finite for every $m<\#(\frak F)$
such that $m\ne n$.   Set $A_{\Cal F_n}=C_n\setminus\bigcup_{m<n}C_m$ for
$n<\#(\frak F)$;  then $\family{\Cal F}{\frak F}{A_{\Cal F}}$ is disjoint.
Since

\Centerline{$C_m\cap C_n
\subseteq(C_m\setminus B_{mn})\cup(C_n\cap B_{mn})$}

\noindent is finite whenever $m\ne n$, $C_n\setminus A_{\Cal F_n}$ is
finite and  $A_{\Cal F_n}\in\Cal F_n$ for each $n<\#(\frak F)$.

\medskip

{\bf (f)(i)} We can suppose that $\frak h(i)\subseteq\frak h(j)$ whenever
$i\le j$, and that $\frak F=\bigcup_{i\in\Bbb N}\frak h(i)$.
Let $g:\Bbb N\to\Bbb N$ be a strictly increasing function
such that $g(0)>0$ and whenever $i\in\Bbb N$ and
$\Cal F\in \frak h(i)$, there is a $k\in A_{\Cal F}$ such that $i<k<g(i)$.
Set $l_m=g^m(0)$ and
$J_m=l_{m+1}\setminus l_m$ for each $m$, so that $\sequence{m}{J_m}$ is a
partition of $\Bbb N$.   Let $\ofamily{\xi}{\omega_1}{a_{\xi}}$ be a family
of infinite subsets of $\Bbb N$, all containing $0$,
such that $a_{\xi}\cap a_{\eta}$ is finite
for all distinct $\xi$, $\eta<\omega_1$ (5A1Fa), and set
$M_{\xi}=\bigcup_{m\in a_{\xi}}J_m$ for each $\xi$;  then
$M_{\xi}\cap M_{\eta}$ is finite for all distinct $\xi$, $\eta<\omega_1$.
It follows that each member of $\frak F$ can contain at most one $M_{\xi}$,
and there is a $\xi<\omega_1$ such that $M_{\xi}$ does not belong to any
member of $\frak F$, that is, $M=\Bbb N\setminus M_{\xi}$ belongs to
$\bigcap\frak F$.

\medskip

\quad{\bf (ii)} Define $f:\Bbb N\to\Bbb N$ by setting
$f(n)=\max\{m:m\in a_{\xi}$, $l_m\le n\}$ for $n\in\Bbb N$.   For each
$\Cal F\in\frak F$, $f[[\Cal F]]$ is isomorphic to $\Cal F$, by (b).
It follows that if $\Cal F$, $\Cal F'$ are distinct
members of $\frak F$, $f[[\Cal F]]\ne f[[\Cal F']]$.   Because $\frak F$ is
countable,
there is a disjoint family $\family{\Cal F}{\frak F}{K_{\Cal F}}$ of sets
such that $K_{\Cal F}\in f[[\Cal F]]$ for every $\Cal F\in\frak F$
((e) above).   Set $L_{\Cal F}=f^{-1}[K_{\Cal F}]\in\Cal F$ for each
$\Cal F\in\frak F$.

\medskip

\quad{\bf (iii)} For $i<j$ in $\Bbb N$, set $h(\{i,j\})=1$ if
$j<g(i)$, $0$ otherwise.
$\Cal F\in\frak F$, there is an $L'_{\Cal F}\in\Cal F$ such that
$L'_{\Cal F}\subseteq L_{\Cal F}$ and $h$ is constant on
$[L'_{\Cal F}]^2$.   As $L'_{\Cal F}$ is infinite, the constant value
cannot be $1$ and must be $0$, that is, $g(i)\le j$ whenever $i$,
$j\in L'_{\Cal F}$ and $i<j$.

\medskip

\quad{\bf (iv)} Consider
$A=\bigcup_{\Cal F\in\frak F}L'_{\Cal F}\cap M$.   Then
$A\in\bigcap\frak F$.   Suppose that $i$, $j\in A$ and $i<j$;
then $g(i)\le j$.   \Prf\ Let
$\Cal F$, $\Cal F'\in\frak F$ be such that
$i\in L'_{\Cal F}$ and $j\in L'_{\Cal F'}$.

\medskip

\qquad{\bf case 1} If $\Cal F=\Cal F'$, then both $i$ and $j$ belong to
$L'_{\Cal F}$, so $g(i)\le j$ by (iii).

\medskip

\qquad{\bf case 2} If $\Cal F\ne\Cal F'$, then $i\in L_{\Cal F}$ and
$j\in L_{\Cal F'}$, so $f(i)\in K_{\Cal F}$ and $f(j)\in K_{\Cal F'}$ and
$f(i)\ne f(j)$.   Let $m$, $n\in\Bbb N$ be such that $i\in J_m$ and
$j\in J_n$;  since $j\notin M_{\xi}$,
$n\notin a_{\xi}$ and $f(j)<n$.   As $K_{\Cal F}$ and
$K_{\Cal F'}$ are disjoint, $f(i)<f(j)$.   It follows that $m<f(j)<n$, so

\Centerline{$g(i)\le g(l_{m+1})\le g(l_{f(j)})\le l_n\le j$}

\noindent and $g(i)\le j$ in this case also.\ \Qed

By the choice of $g$,
this means that if $\Cal F\in\frak h(i)$ there must be a
$k\in A_{\Cal F}$ such that $i<k<j$, as required.

\medskip

{\bf (g)(i)} Suppose that $\Cal E\subseteq\Cal P\Bbb N$ is a filter base,
containing $\Bbb N\setminus n$ for every $n\in\Bbb N$, and of size less
than $\frakmctbl$.   Let $f:[\Bbb N]^2\to\{0,1\}$ be a function.
Then there is an $F\subseteq\Bbb N$ such that $f$ is constant on $[F]^2$
and $F$ meets every member of $\Cal E$.   \Prf\ Set

\Centerline{$\Cal E^+=\{J:J\subseteq\Bbb N$, $J\cap E\ne\emptyset$ for
every $E\in\Cal E\}$,}

\Centerline{$S_n
=\{n\}\cup\{i:i\in\Bbb N\setminus\{n\}$, $f(\{i,n\})=1\}$,}

\Centerline{$S'_n
=\{n\}\cup\{i:i\in\Bbb N\setminus\{n\}$, $f(\{i,n\})=0\}$}

\noindent for $n\in\Bbb N$.

\medskip

\qquad{\bf case 1} Suppose that $\{n:n\in J$, $J\cap S_n\in\Cal E^+\}$
belongs to $\Cal E^+$ for every $J\in\Cal E^+$.   Set

\Centerline{$\Cal I
=\{I:I\in[\Bbb N]^{<\omega}$, $f(K)=1$ for every $K\in[I]^2$,
$\Bbb N\cap\bigcap_{i\in I}S_i\in\Cal E^+\}$.}

\noindent
If $I\in\Cal I$, $J=\Bbb N\cap\bigcap_{i\in I}S_i$ and $E\in\Cal E$, then
$J\in\Cal E^+$;  because $\Cal E$ is a filter base, $J\cap E\in\Cal E^+$;
by hypothesis, $\{n:n\in J\cap E$, $J\cap E\cap S_n\in\Cal E^+\}$ belongs
to $\Cal E^+$ and is not empty.   There is therefore some $n\in J\cap E$
such that $J\cap S_n\in\Cal E^+$, in which case
$I\cup\{n\}\in\Cal I$.

In particular, there is some $k\in\Bbb N$ such that $\{k\}\in\Cal I$.   Set

\Centerline{$C
=\{\alpha:\alpha\in\NN$, $\{\alpha(i):i<m\}\in\Cal I$ for every
$m\in\Bbb N\}$.}

\noindent Then $C$ is compact, and it is non-empty because the constant
function with value $k$ belongs to $C$.   Moreover, if $\alpha\in C$ and
$m\in\Bbb N$ and $E\in\Cal E$, there is an $n\in E$ such that
$\{\alpha(i):i<m\}\cup\{n\}\in\Cal I$, so there is a $\beta\in C$ such that
$\beta(i)=\alpha(i)$ for $i<m$ and $\beta(m)=n$.   Thus
$\{\beta:\beta\in C$, $E\cap\beta[\Bbb N]\ne\emptyset\}$ is a dense open
subset of $C$.   Writing $\Cal M(C)$ for the ideal of meager subsets of
$C$, $\cov\Cal M(C)$ is either $\infty$ (if $C$ has an isolated point)
or $\cov\Cal M(\Bbb R)=\frakmctbl$, by 522Wb and 522Sb;
in either case, it is greater than
$\#(\Cal E)$.   There is therefore some $\alpha\in C$ such that
$F=\alpha[\Bbb N]$ meets every member of $\Cal E$;  in this case,
$f$ is equal to $1$ everywhere in $[F]^2$, so we have an appropriate $F$.

\medskip

\qquad{\bf case 2} Otherwise, there is a $K\in\Cal E^+$ such that
$\{n:n\in K$, $K\cap S_n\in\Cal E^+\}$ does not belong to $\Cal E^+$.
Let $E_0\in\Cal E$ be disjoint from
$\{n:n\in K$, $K\cap S_n\in\Cal E^+\}$.
Set $\Cal G=\Cal E\cup\{K\cap E:E\in\Cal E\}$,
so that $\Cal G$ is a filter base and $\#(\Cal G)<\frakmctbl$.
If $n\in E_0$ then there is an
$E'_n\in\Cal E$ disjoint from $K\cap S_n$.
So if $J\in\Cal G^+$,
$J\cap S'_n\supseteq(J\cap K\cap E'_n)\setminus\{n\}$ belongs to
$\Cal G^+$ for every $n\in E_0$;  accordingly
$\{n:n\in J$, $J\cap S'_n\in\Cal G^+\}\supseteq J\cap E_0$
belongs to $\Cal G^+$.

We can therefore apply the argument of case 1 to $\Cal G$ and
the function $1-f$ to see that there is an $F\subseteq\Bbb N$, meeting
every member of $\Cal G\supseteq\Cal E$, such that $f=0$ on
$[F]^2$.\ \Qed

\medskip

\quad{\bf (ii)}
Enumerate the set of functions from $[\Bbb N]^2$ to $\{0,1\}$ as
$\ofamily{\xi}{\frak c}{f_{\xi}}$.   Choose a non-decreasing family
$\langle\Cal E_{\xi}\rangle_{\xi\le\frak c}$ inductively, as follows;
the inductive
hypothesis will be that $\Cal E_{\xi}\subseteq\Cal P\Bbb N$ is a
filter base with cardinal at most $\max(\omega,\#(\xi))$.
Start with $\Cal E_0=\{\Bbb N\setminus n:n\in\Bbb N\}$.
Given $\Cal E_{\xi}$, where $\xi<\frak c=\frakmctbl$,
use (i) to find a set $F_{\xi}$, meeting every member of $\Cal E_{\xi}$,
such that $f_{\xi}$ is constant on $[F_{\xi}]^2$;  take
$\Cal E_{\xi+1}=\Cal E_{\xi}\cup\{E\cap F_{\xi}:E\in\Cal E_{\xi}\}$.
Given $\ofamily{\eta}{\xi}{\Cal E_{\eta}}$, where $\xi\le\frak c$ is a
non-zero limit ordinal, set $\Cal E_{\xi}=\bigcup_{\eta<\xi}\Cal E_{\eta}$.

At the end of the induction, let $\Cal F$ be the filter generated by
$\Cal E_{\frakc}$;  then $\Cal F$ is a Ramsey filter.
}%end of proof of 538F

\vleader{60pt}{538G}{Measure-centering filters:  Theorem}
Let $\Cal F$ be a free filter on $\Bbb N$.   Write
$\nu_{\omega}$ for the usual measure on $\{0,1\}^{\Bbb N}$,
$\Tau_{\omega}$ for its domain and $(\frak B_{\omega},\bar\nu_{\omega})$
for its measure algebra.   Then the following are equiveridical:

(i) $\Cal F$ is measure-centering;

(ii) whenever $\sequencen{a_n}$ is a sequence in $\frak B_{\omega}$ such
that $\inf_{n\in\Bbb N}\bar\nu_{\omega}a_n>0$, there is an $A\in\Cal F$
such that $\{a_n:n\in A\}$ is centered;

(iii) whenever $\sequencen{F_n}$ is a sequence in $\Tau_{\omega}$ such that
$\inf_{n\in\Bbb N}\nu_{\omega}F_n>0$, there is an $A\in\Cal F$ such that
$\bigcap_{n\in A}F_n\ne\emptyset$;

\allowmorestretch{468}{
(iv) whenever $(X,\Sigma,\mu)$ is a perfect totally finite measure space
and $\sequencen{F_n}$ is a sequence in $\Sigma$,
$\mu^*(\bigcup_{A\in\Cal F}\bigcap_{n\in A}F_n)
\ge\liminf_{n\to\Cal F}\mu F_n$;
}

(v) whenever $\mu$ is a Radon probability measure on $\Cal P\Bbb N$, then
$\mu^*\Cal F\ge\liminf_{n\to\Cal F}\mu E_n$, where
$E_n=\{a:n\in a\subseteq\Bbb N\}$ for each $n$.

\proof{{\bf (i)$\Rightarrow$(ii)} is trivial.

\medskip

\allowmorestretch{468}{
{\bf not-(iv)$\Rightarrow$not-(ii)} Suppose there are
a perfect totally finite measure space $(X,\Sigma,\mu)$ and a sequence
$\sequencen{F_n}$ in $\Sigma$ such that
$\liminf_{n\in\Bbb N}\mu F_n
>\mu^*(\bigcup_{A\in\Cal F}\bigcap_{n\in A}F_n)$.
Let $F$ be a measurable envelope of
$\bigcup_{A\in\Cal F}\bigcap_{n\in A}F_n$.
Let $\Tau$ be the $\sigma$-subalgebra of
$\Sigma$ generated by $\{F\}\cup\{F_n:n\in\Bbb N\}$;
then $\mu\restrp\Tau$ is a compact measure (451F).   Let
$\nu$ be its normalization $\Bover1{\mu X}\mu\restrp\Tau$;
then $\nu$ is a compact probability measure.   We see that
$\liminf_{n\to\Cal F}\nu F_n>\nu F$;  take $\gamma$ such that
$\nu F<\gamma<\liminf_{n\to\Cal F}\nu F_n$, and set
$C=\{n:\nu F_n>\gamma\}$, so that $C\in\Cal F$.
}%end of allowmorestretch

Let $\Cal K$ be a compact class such that
$\nu$ is inner regular with respect to $\Cal K$.   For $n\in C$, let
$K_n\in\Cal K\cap\Tau$
be such that $K_n\subseteq F_n\setminus F$ and
$\nu K_n\ge\gamma-\nu F$;  for $n\in\Bbb N\setminus C$ set $K_n=X$.

The measure algebra $(\frak B,\bar\nu)$ of $\nu$ is a probability algebra
with countable Maharam type, so there is a measure-preserving Boolean
homomorphism $\pi:\frak B\to\frak B_{\omega}$ (332P or 333D).   Set
$a_n=\pi K_n^{\ssbullet}$ for each $n$.   Then

\Centerline{$\bar\nu_{\omega}a_n=\nu K_n\ge\gamma-\nu F>0$}

\noindent for every $n$.   On the other hand, if $A\in\Cal F$, then
$A\cap C\in\Cal F$ so
$\bigcap_{n\in A\cap C}K_n\subseteq\bigcap_{n\in A\cap C}F_n\setminus F$ is
empty.   As $K_n$ belongs to the compact class $\Cal K$ for every
$n\in A\cap C$, there must be a finite set $I\subseteq A\cap C$ such that
$\bigcap_{n\in I}K_n=\emptyset$, in which case
$\inf_{n\in I}a_n=\pi(\bigcap_{n\in I}K_n)^{\ssbullet}=0$.   This shows
that $\{a_n:n\in A\}$ is not centered.   So $\sequencen{a_n}$ witnesses
that (ii) is false.

\medskip

{\bf (iv)$\Rightarrow$(i)} Suppose that (iv) is true.   Take a Boolean
algebra $\frak A$, an additive functional $\nu:\frak A\to\coint{0,\infty}$
and a sequence $\sequencen{a_n}$ in $\frak A$ such that
$\inf_{n\in\Bbb N}\nu a_n>0$.   By 311E and 311H, we can suppose that
$\frak A$ is the algebra of open-and-closed subsets of a compact
zero-dimensional space $Z$.   In this case, there is a Radon measure
$\mu$ on $Z$ extending $\nu$ (416Qa).   Of course $\mu$ is perfect
(416Wa), and $\liminf_{n\to\Cal F}\mu a_n\ge\inf_{n\in\Bbb N}\nu a_n>0$,
so (iv) tells us that there is an $A\in\Cal F$ such that
$\bigcap_{n\in A}a_n\ne\emptyset$, in which case $\{a_n:n\in\Bbb N\}$ is
centered in $\frak A$.   As
$\frak A$, $\nu$ and $\sequencen{a_n}$ are arbitrary, $\Cal F$ is
measure-centering.

\medskip

{\bf (iv)$\Rightarrow$(v)} The point is simply that $\mu$ is perfect
(416Wa again) and that

\Centerline{$\bigcup_{A\in\Cal F}\bigcap_{n\in A}E_n
=\bigcup_{A\in\Cal F}\{a:A\subseteq a\subseteq\Bbb N\}
=\Cal F$.}

\medskip

{\bf (v)$\Rightarrow$(iii)} Suppose that (v) is true, and that
$\sequencen{F_n}$ is a sequence in $\Tau_{\omega}$ such that
$\inf_{n\in\Bbb N}\nu_{\omega}F_n>0$.   Define
$\phi:\{0,1\}^{\Bbb N}\to\Cal P\Bbb N$ by setting
$\phi(x)=\{n:x\in F_n\}$ for each $n$.   Then $\phi$ is almost continuous
(418J), so the image measure $\mu=\nu_{\omega}\phi^{-1}$ is a Radon
probability measure on $\Cal P\Bbb N$ (418I).
Defining $E_n$ as in (v), we have

\Centerline{$\mu E_n=\nu_{\omega}\phi^{-1}[E_n]=\nu_{\omega}F_n$}

\noindent for every $n\in\Bbb N$, so

\Centerline{$0<\inf_{n\in\Bbb N}\bar\nu_{\omega}a_n
\le\liminf_{n\to\Cal F}\mu E_n\le\mu^*\Cal F
=\nu_{\omega}^*\phi^{-1}[\Cal F]$}

\noindent (451Pc).   In particular, there must be an
$x\in\phi^{-1}[\Cal F]$, so that $A=\{n:x\in F_n\}$ belongs to $\Cal F$,
and $\bigcap_{n\in A}F_n$ is non-empty.

\medskip

{\bf (iii)$\Rightarrow$(ii)} Assume (iii).   Let $\sequencen{a_n}$ be a
sequence in $\frak B_{\omega}$ such that
$\inf_{n\in\Bbb N}\bar\nu_{\omega}a_n>0$.   Let
$\theta:\frak B_{\omega}\to\Tau_{\omega}$ be a lifting (341K), and set
$F_n=\theta a_n$ for each $n$.   Then $\nu_{\omega}F_n=\bar\nu_{\omega}a_n$
for every $n$, so (iii) tells us that there is an $A\in\Cal F$ such that
$\bigcap_{n\in A}F_n\ne\emptyset$.   In this case,
$\theta(\inf_{n\in I}a_n)=\bigcap_{n\in I}F_n\ne\emptyset$
for every non-empty finite $I\subseteq A$, so $\{a_n:n\in A\}$ is centered.
}%end of proof of 538G

\leader{538H}{Proposition}
(a) Any measure-centering filter on $\Bbb N$ is an ultrafilter.

(b) If $\Cal F$ is a measure-centering
ultrafilter on $\Bbb N$ and $\Cal G$ is a filter on
$\Bbb N$ such that $\Cal G\leRK\Cal F$, then $\Cal G$ is measure-centering.

(c) Every Ramsey ultrafilter on $\Bbb N$ is measure-centering.

(d)\cmmnt{ ({\smc Shelah 98b})} Every
measure-centering ultrafilter on $\Bbb N$ is a nowhere dense ultrafilter.

(e)\cmmnt{ ({\smc Benedikt 99})} If $\cov\Cal N=\frak c$, where $\Cal N$
is the Lebesgue null ideal, then there is a measure-centering ultrafilter
on $\Bbb N$.
%would  \cov\Cal N=\cf\Cal N  be enough?

\proof{{\bf (a)} Let $a$, $b$ be disjoint non-zero elements of
$\frak B_{\omega}$, where $(\frak B_{\omega},\bar\nu_{\omega})$ is the
measure algebra of the usual measure on $\{0,1\}^{\Bbb N}$, as in 538G.
Given $I\subseteq\Bbb N$, set $a_n=a$ if $n\in I$, $b$ if
$n\in\Bbb N\setminus I$.   Then $\inf_{n\in\Bbb N}\bar\nu_{\omega}a_n>0$,
so there is a $J\in\Cal F$ such that $\{a_n:n\in J\}$ is
centered, in which case either $J\subseteq I$ or $J\cap I=\emptyset$;  so
that one of $I$, $\Bbb N\setminus I$ must belong to $\Cal F$.

\medskip

{\bf (b)} Let $f:\Bbb N\to\Bbb N$ be such that $f[[\Cal F]]=\Cal G$.
Let $(\frak A,\bar\mu)$ be a totally finite measure algebra and
$\sequencen{a_n}$ a sequence in
$\frak A$ with $\inf_{n\in\Bbb N}\bar\mu a_n>0$.   Then
$\sequencen{a_{f(n)}}$ has the same property, so there is an
$A\in\Cal F$ such that $\{a_{f(n)}:n\in A\}$ is centered.   Now
$f[A]\in\Cal G$ and $\{a_m:m\in f[A]\}$ is centered.

\medskip

{\bf (c)} Let $\Cal F$ be a Ramsey ultrafilter
and $\sequencen{b_n}$ a sequence in $\frak B_{\omega}$
such that $\gamma=\inf_{n\in\Bbb N}\bar\nu_{\omega}b_n$ is greater than
$0$.
Set $b=\inf_{A\in\Cal F}\sup_{n\in A}b_n$;  then
$\bar\nu_{\omega}b\ge\gamma$.
Set $\Cal S=\{I:I\in[\Bbb N]^{<\omega}$, $b\Bcap\inf_{n\in I}b_n\ne 0\}$.
Then $\emptyset\in\Cal S$.   If $I\in\Cal S$, set
$c=b\Bcap\inf_{n\in I}b_n$ and $C=\{n:c\Bcap b_n=0\}$.   Then
$\sup_{n\in C}b_n$ does not meet $c$ so does not include $b$, and
$C\notin\Cal F$.   Accordingly

\Centerline{$\{n:I\cup\{n\}\in\Cal S\}=\Bbb N\setminus C\in\Cal F$.}

\noindent By 538Fd, there is an $A\in\Cal F$ such that
$[A]^{<\omega}\subseteq\Cal S$, in which case $\{b_n:n\in A\}$ is centered.
As $\sequencen{b_n}$ is arbitrary, $\Cal F$ is measure-centering.

\medskip

{\bf (d)} Let $\Cal F$ be a measure-centering ultrafilter, and
$\sequencen{t_n}$ a sequence in $\Bbb R$.   Let $F\subseteq\coint{0,1}$ be
a nowhere dense set with non-zero Lebesgue measure, and set
$H=\bigcup_{k\in\Bbb Z}F+k$, so that $H$ is nowhere dense in $\Bbb R$;  let
$\mu$ be Lebesgue measure on $[0,1]$.
For $n\in\Bbb N$ set

\Centerline{$E_n=\{x:x\in[0,1]$, $x+t_n\in H\}
=[0,1]\cap\bigcup_{k\in\Bbb Z}F-t_n+k$,}

\noindent so that $\mu E_n=\mu F>0$.   By 538G(iv), there is an
$A\in\Cal F$ such that $\bigcap_{n\in A}E_n$ is non-empty;  take
$x\in\bigcap_{n\in A}E_n$, so that $t_n\in H-x$ for every $n\in A$, and
$\{t_n:n\in A\}$ is nowhere dense.
As $\sequencen{t_n}$ is arbitrary, $\Cal F$ is a nowhere dense filter.

\medskip

{\bf (e)(i)} Let $\sequencen{a_n}$ be a sequence in
$\frak B_{\omega}$ such that $\inf_{n\in\Bbb N}\bar\nu_{\omega}a_n>0$, and
$\Cal C\subseteq\Cal P\Bbb N$ a filter base such that
$\#(\Cal C)<\cov\Cal N$.   Then there is an $A\subseteq\Bbb N$ such that
$A$ meets every member of $C$ and
$\{a_n:n\in A\}$ is centered.   \Prf\ Set
$\epsilon=\inf_{n\in\Bbb N}\bar\nu_{\omega}a_n$.
For $C\in\Cal C$ set $b_C=\sup_{n\in C}a_n$;  because $C\ne\emptyset$,
$\bar\nu_{\omega}b_C\ge\epsilon$.
Set $b=\inf_{C\in\Cal C}b_C$;  because $\Cal C$ is downwards-directed,
$\bar\nu_{\omega}b\ge\epsilon$ (321F) and $b\ne 0$.

Let $\theta:\frak B_{\omega}\to\Tau_{\omega}$ be a lifting (341K).   For
$C\in\Cal C$, set $F_C=\bigcup_{n\in C}\theta a_n$;  then

\Centerline{$F_C^{\ssbullet}=b_C\Bsupseteq b$,}

\noindent so $\theta b\setminus F_C$ is negligible.   Because
$b\ne 0$, $\theta b$ is not negligible;  because
$\#(\Cal C)<\cov\Cal N$, $\theta b\cap\bigcap_{C\in\Cal C}F_C$ is non-empty
(apply 522Wa to the subspace measure on $\theta b$).   Take any $x$ in the
intersection, and set $A=\{n:x\in\theta a_n\}$.   For every $C\in\Cal C$,
there is an $n\in C$ such that $x\in\theta a_n$, so $A\cap C\ne\emptyset$.
If $I\subseteq A$ is finite and not empty, then
$\theta(\inf_{n\in I}a_n)=\bigcap_{n\in I}\theta a_n$ contains $x$, so
$\inf_{n\in I}a_n\ne 0$;  thus $\{a_n:n\in A\}$ is centered.\ \Qed

\medskip

\quad{\bf (ii)} Since $\#(\frak B_{\omega})=\frak c$ (524Ma), we can
enumerate as $\ofamily{\xi}{\frak c}{\sequencen{a_{\xi n}}}$ the family of
all sequences $\sequencen{a_n}$ in $\frak B_{\omega}$ such that
$\inf_{n\in\Bbb N}\bar\nu_{\omega}a_n>0$.   Choose
$\ofamily{\xi}{\frak c}{\Cal C_{\xi}}$ inductively, as follows.
The inductive hypothesis will be that $\Cal C_{\xi}\subseteq\Cal P\Bbb N$
is a filter base and $\#(\Cal C_{\xi})\le\max(\omega,\#(\xi))$.
Start with $\Cal C_0=\{\Bbb N\setminus n:n\in\Bbb N\}$.
Given $\Cal C_{\xi}$, where $\xi<\frak c$, such that

\Centerline{$\#(\Cal C_{\xi})\le\max(\omega,\#(\xi))<\frak c=\cov\Cal N$,}

\noindent (i) tells us that there is an $A_{\xi}\subseteq\Bbb N$, meeting
every member of $\Cal C_{\xi}$, such
that $\{a_{\xi n}:n\in A_{\xi}\}$ is centered;  set

\Centerline{$\Cal C_{\xi+1}
=\Cal C_{\xi}\cup\{C\cap A_{\xi}:C\in\Cal C_{\xi}\}$.}

\noindent For a non-zero limit ordinal $\xi\le\frak c$, set
$\Cal C_{\xi}=\bigcup_{\eta<\xi}\Cal C_{\eta}$.
Let $\Cal F$ be
the filter generated by $\Cal C_{\frak c}$;  then $\Cal F$ is a free filter
satisfying 538G(ii), so is measure-centering.
}%end of proof of 538H

\leader{538I}{Theorem} Suppose that $\Cal F$ is a measure-centering
ultrafilter on $\Bbb N$, and that $(X,\Sigma,\mu)$ is a perfect probability
space.   Let $\Cal A$ be the family of all sets of the form
$\lim_{n\to\Cal F}E_n$ where $\sequencen{E_n}$ is a sequence in $\Sigma$.
Then there is a unique complete measure $\lambda$ on $X$ such that
$\lambda$ is inner regular with respect to $\Cal A$ and
$\lambda(\lim_{n\to\Cal F}E_n)=\lim_{n\to\Cal F}\mu E_n$
for every sequence $\sequencen{E_n}$ in $\Sigma$;  and $\lambda$ extends
$\mu$.

\medskip

\noindent{\bf Remark} By `$\lim_{n\to\Cal F}E_n$' I mean the limit in the
compact Hausdorff space $\Cal PX$, that is,

\Centerline{$\{x:\{n:x\in E_n\}\in\Cal F\}
=\bigcup_{A\in\Cal F}\bigcap_{n\in A}E_n
=\bigcap_{A\in\Cal F}\bigcup_{n\in A}E_n$.}

\proof{{\bf (a)}
$\Cal A$ is an algebra of subsets of $X$.   \Prf\ If $\sequencen{E_n}$,
$\sequencen{F_n}$ are sequences in $\Sigma$, then

\Centerline{$\lim_{n\to\Cal F}(E_n\cap F_n)
=(\lim_{n\to\Cal F}E_n)\cap(\lim_{n\to\Cal F}F_n)$,}

\Centerline{$\lim_{n\to\Cal F}(E_n\symmdiff F_n)
=(\lim_{n\to\Cal F}E_n)\symmdiff(\lim_{n\to\Cal F}F_n)$}

\noindent because $\Cal F$ is an ultrafilter.\ \QeD\   Of course
$\Sigma\subseteq\Cal A$, because if $E_n=E$ for every $n$ then
$\lim_{n\to\Cal F}E_n=E$.

\wheader{538I}{6}{2}{2}{72pt}

{\bf (b)} If $\sequencen{E_n}$ and $\sequencen{F_n}$ are sequences in
$\Sigma$ and $\lim_{n\to\Cal F}E_n=\lim_{n\to\Cal F}F_n$, then
$\lim_{n\to\Cal F}\mu E_n=\lim_{n\to\Cal F}\mu F_n$.   \Prf\

$$\eqalignno{|\lim_{n\to\Cal F}\mu E_n-\lim_{n\to\Cal F}\mu F_n|
&=\lim_{n\to\Cal F}|\mu E_n-\mu F_n|
\le\lim_{n\to\Cal F}\mu(E_n\symmdiff F_n)
\le\mu^*(\lim_{n\to\Cal F}E_n\symmdiff F_n)\cr
\displaycause{538G(iv)}
&=\mu^*(\lim_{n\to\Cal F}E_n\symmdiff\lim_{n\to\Cal F}F_n)
=\mu^*\emptyset
=0.  \text{ \Qed}\cr}$$

\medskip

{\bf (c)} We therefore have a functional $\phi:\Cal A\to[0,1]$ defined by
setting $\phi(\lim_{n\to\Cal F}E_n)=\lim_{n\to\Cal F}\mu E_n$ for every
sequence $\sequencen{E_n}$ in $\Sigma$.   Clearly $\phi$ extends $\mu$.
Also $\phi$ is additive.   \Prf\ If $\sequencen{E_n}$, $\sequencen{F_n}$
are sequences in $\Sigma$ such that $\lim_{n\to\Cal F}E_n$ and
$\lim_{n\to\Cal F}F_n$ are disjoint, then

$$\eqalign{\phi(\lim_{n\to\Cal F}E_n\cup\lim_{n\to\Cal F}F_n)
&=\phi(\lim_{n\to\Cal F}E_n\cup F_n)
=\lim_{n\to\Cal F}\mu(E_n\cup F_n)\cr
&=\lim_{n\to\Cal F}\mu E_n+\mu F_n-\mu(E_n\cap F_n)\cr
&=\lim_{n\to\Cal F}\mu E_n
  +\lim_{n\to\Cal F}\mu F_n
  -\lim_{n\to\Cal F}\mu(E_n\cap F_n)\cr
&=\phi(\lim_{n\to\Cal F}E_n)
  +\phi(\lim_{n\to\Cal F}F_n)
  -\phi(\lim_{n\to\Cal F}E_n\cap F_n)\cr
&=\phi(\lim_{n\to\Cal F}E_n)
  +\phi(\lim_{n\to\Cal F}F_n).  \text{ \Qed}\cr}$$

\medskip

{\bf (d)} Next, if
$\sequence{m}{A_m}$ is a non-increasing sequence in $\Cal A$, and
$0\le\gamma<\inf_{m\in\Bbb N}\phi A_m$,
there is an $A\in\Cal A$ such that
$A\subseteq\bigcap_{m\in\Bbb N}A_m$ and $\phi A\ge\gamma$.
\Prf\ We can suppose that $A_0=X$.   For each
$m\in\Bbb N$, let $\sequencen{E_{mn}}$ be a sequence in $\Sigma$ such that
$A_m=\lim_{n\to\Cal F}E_{mn}$, starting with $E_{0n}=X$ for every $n$.
For $m\in\Bbb N$, set $E'_{mn}=\bigcap_{i\le m}E_{in}$ for
$n\in\Bbb N$;  then

\Centerline{$A_m=\bigcap_{i\le m}A_i=\lim_{n\to\Cal F}E'_{mn}$;}

\noindent set $I_m=\{n:n\in\Bbb N$, $\mu E'_{mn}\ge\gamma\}$.  Since
$\lim_{n\to\Cal F}\mu E'_{mn}=\phi A_m>\gamma$, $I_m\in\Cal F$.
For $n\in\Bbb N$, set
$F_n=\bigcap\{E'_{mn}:m\in\Bbb N$, $\mu E'_{mn}\ge\gamma\}$;
set $A=\lim_{n\to\Cal F}F_n$.   Then $\mu F_n\ge\gamma$ for every $n$,
so $\phi A\ge\gamma$.   Also, for $m\in\Bbb N$,
$F_n\subseteq E'_{mn}$ whenever $n\in I_m$, so
$A\subseteq\lim_{n\to\Cal F}E'_{mn}=A_m$.\ \Qed

\medskip

{\bf (e)} In particular, $\inf_{m\in\Bbb N}\phi A_m$ must be $0$ whenever
$\sequence{m}{A_m}$ is a non-increasing sequence in $\Cal A$ with empty
intersection.   By 413J, there is a complete measure $\lambda$ on $X$
extending $\phi$ and inner regular with respect to $\Cal A_{\delta}$, the
set of intersections of sequences in $\Cal A$.   But
$\lambda C=\sup\{\lambda A:A\in\Cal A$, $A\subseteq C\}$ for every
$C\in\Cal A_{\delta}$.   \Prf\ Suppose that $0\le\gamma<\lambda C$.
There is a sequence $\sequence{m}{A_m}$ in
$\Cal A$ with intersection $C$;  because $\Cal A$ is an algebra of sets, we
can suppose that $\sequence{m}{A_m}$ is non-increasing.   Now

\Centerline{$\gamma<\lambda C=\inf_{m\in\Bbb N}\lambda A_m
=\inf_{m\in\Bbb N}\phi A_m$,}

\noindent so (d) tells us that there is an $A\in\Cal A$ such that
$A\subseteq C$ and $\gamma\le\phi A=\lambda A$.\ \QeD\  It follows at once
that $\lambda$ is inner regular with respect to $\Cal A$.

\medskip

{\bf (f)} If $E\in\Sigma$ and we set $E_n=E$ for every $n\in\Bbb N$, then
$E=\lim_{n\to\Cal F}E_n$ belongs to $\Cal A$ and

\Centerline{$\lambda E=\phi E=\lim_{n\to\Cal F}\mu E_n=\mu E$.}

\noindent So $\lambda$ extends $\mu$.   Finally, we see from 412L, as
usual, that $\lambda$ is uniquely defined.
}%end of proof of 538I

\medskip

\noindent{\bf Notation} In this context, I will call $\lambda$ the
{\bf $\Cal F$-extension} of $\mu$.

\leader{538J}{Proposition} Let $\Cal F$ be a measure-centering ultrafilter
on $\Bbb N$ and $(X,\Sigma,\mu)$ a
perfect probability space;  let $\lambda$ be the
$\Cal F$-extension of $\mu$ as defined in 538I.

(a) Let $(\frak A,\bar\mu)$ be the measure algebra of $\mu$,
$(\frak B,\bar\lambda)$ the measure algebra of $\lambda$, and
$(\frak C,\bar\nu)$ the probability algebra reduced power
$(\frak A,\bar\mu)^{\Bbb N}|\Cal F$
(328C).  Then we have a
measure-preserving isomorphism $\pi:\frak B\to\frak C$ defined by saying
that

\Centerline{$\pi((\lim_{n\to\Cal F}E_n)^{\ssbullet})
=\sequencen{E_n^{\ssbullet}}^{\ssbullet}$}

\noindent for every sequence $\sequencen{E_n}$ in $\Sigma$.

(b) Let $(X',\Sigma',\mu')$ be another
perfect probability space, and $\phi:X\to X'$ an \imp\ function.
Let $\lambda'$ be the $\Cal F$-extension of $\mu'$.
Then $\phi$ is \imp\ for $\lambda$ and $\lambda'$.

(c) Let $\Cal F'$ be a filter on $\Bbb N$ such that $\Cal F'\leRK\Cal F$,
and $\lambda'$ the
$\Cal F'$-extension of $\mu$.   Then $\lambda$ extends $\lambda'$.

\proof{{\bf (a)(i)} I had better check first that the formula for $\pi$
defines a function.   If $\sequencen{E_n}$, $\sequencen{F_n}$ are sequences
in $\Sigma$ such that
$(\lim_{n\to\Cal F}E_n)^{\ssbullet}=(\lim_{n\to\Cal F}F_n)^{\ssbullet}$ in
$\frak B$, then

$$\eqalign{0
&=\lambda(\lim_{n\to\Cal F}E_n\symmdiff\lim_{n\to\Cal F}F_n)
=\lim_{n\to\Cal F}\mu(E_n\symmdiff F_n)\cr
&=\lim_{n\to\Cal F}\bar\mu(E^{\ssbullet}_n\Bsymmdiff F_n^{\ssbullet})
=\bar\nu(\sequencen{E_n^{\ssbullet}}^{\ssbullet}
  \Bsymmdiff\sequencen{F_n^{\ssbullet}}^{\ssbullet}),\cr}$$

\noindent so $\sequencen{E_n^{\ssbullet}}^{\ssbullet}
=\sequencen{F_n^{\ssbullet}}^{\ssbullet}$ in $\frak C$.

\medskip

\quad{\bf (ii)} Setting $\frak B_0=\{E^{\ssbullet}:E\in\Cal A\}$, where
$\Cal A$ is as in 538I, it is now routine to check that
$\pi:\frak B_0\to\frak C$ is a surjective
measure-preserving Boolean homomorphism.   (Recall that $\frak C$ is, by
definition, the quotient of $\frak A^{\Bbb N}$ by the ideal
$\{\sequencen{a_n}:\lim_{n\to\Cal F}\bar\mu a_n=0\}$.)
But of course this means that
$\frak B_0$ is isomorphic to $\frak C$, therefore Dedekind complete.
Since $\lambda$ is inner regular with respect to $\Cal A$ (538I),
$\frak B_0$ is order-dense in $\frak B$, and must be the whole of
$\frak B$.

\wheader{538J}{6}{2}{2}{36pt}

{\bf (b)} Setting

\Centerline{$\Cal A
=\{\lim_{n\to\Cal F}E_n:E_n\in\Sigma\Forall n\in\Bbb N\}$,
\quad$\Cal A'
=\{\lim_{n\to\Cal F}F_n:F_n\in\Sigma'\Forall n\in\Bbb N\}$}

\noindent as in 538I, $\phi^{-1}[C]\in\Cal A$ and
$\lambda\phi^{-1}[C]=\lambda'C$ for every $C\in\Cal A'$.   \Prf\ Let
$\sequencen{F_n}$ be a sequence in $\Sigma'$ such that
$C=\lim_{n\to\Cal F}F_n$;  then

$$\eqalign{\lambda\phi^{-1}[C]
&=\lambda\phi^{-1}[\lim_{n\to\Cal F}F_n]
=\lambda(\lim_{n\to\Cal F}\phi^{-1}[F_n])\cr
&=\lim_{n\to\Cal F}\mu\phi^{-1}[F_n]
=\lim_{n\to\Cal F}\mu'F_n
=\lambda'C.  \text{ \Qed}\cr}$$

\noindent By 412K, $\phi$ is \imp\ for $\lambda$ and $\lambda'$.

\medskip

{\bf (c)} By 538Hb, $\Cal F'$ is measure-centering.
Let $f:\Bbb N\to\Bbb N$ be such that $\Cal F'=f[[\Cal F]]$.
Setting

\Centerline{$\Cal A
=\{\lim_{n\to\Cal F}E_n:E_n\in\Sigma\Forall n\in\Bbb N\}$,
\quad$\Cal A'
=\{\lim_{n\to\Cal F'}E_n:E_n\in\Sigma\Forall n\in\Bbb N\}$,}

\noindent $\Cal A'\subseteq\Cal A$ and $\lambda A=\lambda'A$ for every
$A\in\Cal A'$.   \Prf\ Let
$\sequencen{E_n}$ be a sequence in $\Sigma$ such that
$A=\lim_{n\to\Cal F'}E_n$;  then $A=\lim_{n\to\Cal F}E_{f(n)}$, so

\Centerline{$\lambda A=\lim_{n\to\Cal F}\mu E_{f(n)}
=\lim_{n\to\Cal F'}\mu E_n=\lambda'A$.  \Qed}

\noindent By 412K again, the identity map from $X$ to itself is
\imp\ for $\lambda$ and $\lambda'$, that is, $\lambda$ extends $\lambda'$.
}%end of proof of 538J

\leader{538K}{}\cmmnt{ Having identified the measure algebra of a
measure-centering-ultrafilter extension $\lambda$
as a probability algebra reduced
product (538Ja), we are in a position to apply the results of \S377.

\medskip

\noindent}{\bf Proposition} Let $(X,\Sigma,\mu)$ be a perfect probability
space, $\Cal F$ a measure-centering ultrafilter on $\Bbb N$ and $\lambda$
the $\Cal F$-extension of $\mu$\cmmnt{ as constructed in 538I}.

(a)(i) Let $\sequencen{f_n}$ be a sequence in $\eusm L^0(\mu)$ such that
$\{f_n^{\ssbullet}:n\in\Bbb N\}$ is bounded in the linear topological space
$L^0(\mu)$.   Then

\qquad($\alpha$) $f(x)=\lim_{n\to\Cal F}f_n(x)$ is defined in $\Bbb R$
for $\lambda$-almost every $x\in X$;

\qquad($\beta$) $f\in\eusm L^0(\lambda)$.

\quad(ii) For every $f\in\eusm L^0(\lambda)$ there is a sequence
$\sequencen{f_n}$ in $\eusm L^0(\mu)$, bounded in the sense of (i), such
that $f=\lim_{n\to\Cal F}f_n\,\,\lambda$-a.e.

(b) Suppose that $1\le p\le\infty$, and
$\sequencen{f_n}$ is a sequence in
$\eusm L^p(\mu)$ such that $\sup_{n\in\Bbb N}\|f_n\|_p$ is finite.
Set $f(x)=\lim_{n\to\Cal F}f_n(x)$ whenever this is defined in $\Bbb R$.

\quad(i)($\alpha$) $f\in\eusm L^p(\lambda)$;

\qquad($\beta$) $\|f\|_p\le\lim_{n\to\Cal F}\|f_n\|_p$.
%Fatou's Lemma

\quad(ii) Let $g$ be a conditional expectation of $f$ on $\Sigma$.

\qquad($\alpha$) If $p=1$ and $\{f_n:n\in\Bbb N\}$ is uniformly integrable, then
then $\|f\|_1=\lim_{n\to\Cal F}\|f_n\|_1$ and
$g^{\ssbullet}=\lim_{n\to\Cal F}f_n^{\ssbullet}$ for the weak topology
of $L^1(\mu)$.

\qquad($\beta$) If $1<p<\infty$, then
$g^{\ssbullet}=\lim_{n\to\Cal F}f_n^{\ssbullet}$ for the weak topology of
$L^p(\mu)$.

\woddheader{538K}{0}{0}{0}{36pt}

(c) Suppose that $1\le p\le\infty$ and $f\in\eusm L^p(\lambda)$.

\quad(i) There is a
sequence $\sequencen{f_n}$ in $\eusm L^p(\mu)$ such that
$f=\lim_{n\to\Cal F}f_n\,\,\lambda$-a.e.\
and $\|f\|_p=\sup_{n\in\Bbb N}\|f_n\|_p$.

\quad(ii) If $p=1$, we can arrange that $\sequencen{f_n}$ should
be uniformly integrable.

(d) Let $(X',\Sigma',\mu')$ be another perfect measure space, and
$\lambda'$ the $\Cal F$-extension of $\mu'$.
Let $S:L^1(\mu)\to L^1(\mu')$ be a bounded linear operator.

\allowmorestretch{468}{
\quad(i) There is a unique bounded linear operator
$\hat S:L^1(\lambda)\to L^1(\lambda')$ such that
$\hat Sf^{\ssbullet}=g^{\ssbullet}$
whenever $\sequencen{f_n}$, $\sequencen{g_n}$ are uniformly integrable
sequences in $\eusm L^1(\mu)$, $\eusm L^1(\nu)$ respectively,
$f=\lim_{n\to\Cal F}f_n\,\,\lambda$-a.e.,
$g=\lim_{n\to\Cal F}g_n\,\,\lambda'$-a.e.,
and $g_n^{\ssbullet}=Sf_n^{\ssbullet}$ for every $n\in\Bbb N$.
}

\quad(ii) The map
$S\mapsto\hat S:\eurm B(L^1(\mu);L^1(\mu'))
\to\eurm B(L^1(\lambda);L^1(\lambda'))$
is a norm-preserving Riesz homomorphism.

\proof{ We shall find that most of the work for this result has been done
in \S377.   The only new step is in (a)(i), but we shall have some checking
to do.

\medskip

{\bf (a)(i)} Let $\sequencen{\tilde f_n}$ be a sequence of
$\Sigma$-measurable functions from $X$ to $\Bbb R$ such that
$\tilde f_n=f_n\,\,\mu$-a.e.\ for every $n\in\Bbb N$.

\medskip

\qquad\grheada\ Let $\epsilon>0$.   Applying 367Rd\Latereditions\ to
$\{\tilde f_n^{\ssbullet}:n\in\Bbb N\}=\{f_n^{\ssbullet}:n\in\Bbb N\}$,
there is a $\gamma>0$ such that
$\mu E_n\le\epsilon$ for every
$n\in\Bbb N$, where $E_n=\{x:|\tilde f_n(x)|\ge\gamma\}$.
Set $E=\lim_{n\to\Cal F}E_n$,
so that $\lambda E\le\epsilon$.   For $x\in X\setminus E$,
$\{n:|\tilde f_n(x)|\le\gamma\}\in\Cal F$, so
$\lim_{n\to\Cal F}\tilde f_n(x)$ is defined in $\Bbb R$.   As $\epsilon$ is
arbitrary, $\lim_{n\to\Cal F}\tilde f_n(x)$ is defined in $\Bbb R$ for
$\lambda$-almost every $x$.   Since

\Centerline{$\{x:x\in\dom f_n$ and
$f_n(x)=\tilde f_n(x)$ for every $n\in\Bbb N\}$}

\noindent is $\mu$-conegligible, therefore $\lambda$-conegligible,
$\lim_{n\to\Cal F}f_n$ is defined in $\Bbb R\,\,\lambda$-a.e.

\medskip

\qquad\grheadb\ For any $\alpha\in\Bbb R$,

\Centerline{$\{x:\lim_{n\to\Cal F}\tilde f_n(x)>\alpha\}
=\bigcup_{k\in\Bbb N}\lim_{n\to\Cal F}\{x:f_n(x)\ge\alpha+2^{-k}\}
\in\dom\lambda$.}

\noindent So $f\eae\lim_{n\to\Cal F}\tilde f_n$ belongs to
$\eusm L^0(\lambda)$.

\medskip

\quad{\bf (ii)} At this point I seek to import the machinery of \S377.

\medskip

\qquad\grheada\
Let $(\frak A,\bar\mu)$ and $(\frak B,\bar\lambda)$ be the measure algebras
of $\mu$, $\lambda$ respectively;  recall that we can identify
$L^0(\mu)$ and $L^0(\lambda)$ with $L^0(\frak A)$ and $L^0(\frak B)$
(364Ic).   Write
$(\frak C,\bar\nu)$ for the probability algebra reduced power
$(\frak A,\bar\mu)^{\Bbb N}|\Cal F$;  let
$\phi:\frak A^{\Bbb N}\to\frak C$ be the canonical surjection, and
$\pi:\frak B\to\frak C$ the isomorphism of 538Ja;  set
$\psi=\pi^{-1}\phi:\frak A^{\Bbb N}\to\frak B$.   Then
$\bar\lambda\psi(\sequencen{a_n})=\lim_{n\to\Cal F}\bar\mu a_n$ for every
sequence $\sequencen{a_n}$ in $\frak A$, and $\psi$ is surjective.

\medskip

\qquad\grheadb\
Let $W_0\subseteq L^0(\frak A)^{\Bbb N}$ be the set of sequences bounded
for the topology of convergence in measure, and
$\eusm W_0\subseteq\eusm L^0(\mu)^{\Bbb N}$ the set of sequences
$\sequencen{f_n}$ such that $\sequencen{f_n^{\ssbullet}}\in W_0$.
Then we have a Riesz
homomorphism $T:W^0\to L^0(\frak B)$ defined by saying that
$T(\sequencen{f_n^{\ssbullet}})=(\lim_{n\to\Cal F}f_n)^{\ssbullet}$
whenever $\sequencen{f_n}\in\eusm W_0$.   \Prf\ We know from (i) that
$(\lim_{n\to\Cal F}f_n)^{\ssbullet}$ is defined in
$L^0(\lambda)\cong L^0(\frak B)$ whenever $\sequencen{f_n}\in\eusm W_0$.
(I am taking the domain of $\lim_{n\to\Cal F}f_n$ to be
$\{x:\lim_{n\to\Cal F}f_n(x)$ is defined in $\Bbb R\}$.)   Since

\Centerline{$\lim_{n\to\Cal F}f_n\eae\lim_{n\to\Cal F}g_n$}

\noindent whenever $f_n\eae g_n$ for every $n$, $T$ is
well-defined.   Since

\Centerline{$\lim_{n\to\Cal F}f_n+g_n
\eae\lim_{n\to\Cal F}f_n+\lim_{n\to\Cal F}g_n$,}

\Centerline{$\lim_{n\to\Cal F}\alpha f_n
\eae\alpha\lim_{n\to\Cal F}f_n$,
\quad$\lim_{n\to\Cal F}|f_n|
\eae|\lim_{n\to\Cal F}f_n|$}

\noindent whenever $\sequencen{f_n}$, $\sequencen{g_n}\in\eusm W_0$ and
$\alpha\in\Bbb R$, $T$ is a Riesz homomorphism.\ \Qed

\medskip

\qquad\grheadc\ If $\sequencen{a_n}$ is any sequence in $\frak A$,
$T(\sequencen{\chi a_n})=\chi\psi(\sequencen{a_n})$.   \Prf\
Express each $a_n$ as $E_n^{\ssbullet}$, where $E_n\in\Sigma$, and set
$F=\lim_{n\to\Cal F}E_n$.   In the language of 538Ja,

\Centerline{$\psi(\sequencen{a_n})
=\pi^{-1}\phi(\sequencen{a_n})
=\pi^{-1}(\sequencen{a_n}^{\ssbullet})
=F^{\ssbullet}$,}

\noindent so

\Centerline{$T(\sequencen{\chi a_n})
=(\lim_{n\to\Cal F}\chi E_n)^{\ssbullet}
=(\chi F)^{\ssbullet}
=\chi(F^{\ssbullet})
=\chi\psi(\sequencen{a_n})$.  \Qed}

\medskip

\qquad\grheadd\ Recalling that $W_0$ is just the family of sequences
$\sequencen{u_n}$ in $L^0$ such that
$\inf_{k\in\Bbb N}\sup_{n\in\Bbb N}\bar\mu\Bvalue{|u_n|>k}=0$
(367Rd again),
($\gamma$) means that we can identify $T:W_0\to L^0(\frak B)$
with the Riesz homomorphism described in 377B.   By 377D(d-i),
$T[W_0]=L^0(\frak B)$, which is what we need to prove the immediate result
here.

\medskip

{\bf (b)(i)} As in part (a) of the proof of
377C, we see that a $\|\,\|_p$-bounded
sequence in $\eusm L^p(\mu)$ will belong to $\eusm W_0$.   So we can use
377Db.

\medskip

\quad{\bf (ii)} Use 377Ec.

\medskip

{\bf (c)} Use 377Dd.

\medskip

{\bf (d)} Use 377F.
}%end of proof of 538K

\leader{538L}{Theorem} Suppose that $\zeta$ is a non-zero
countable ordinal and
$\langle\Cal F_{\xi}\rangle_{1\le\xi\le\zeta}$ is a family of Ramsey
ultrafilters on $\Bbb N$, no two isomorphic.   Let
$\langle\Cal G_{\xi}\rangle_{\xi\le\zeta}$ be the corresponding iterated
product system, as described in 538E.   Then
$\Cal G_{\zeta}$ is a measure-centering ultrafilter.

\proof{{\bf (a)}
Define $\langle(\frak A_{\xi},\bar\mu_{\xi})\rangle_{\xi\le\zeta}$
inductively, as follows.
$(\frak A_0,\bar\mu_0)=(\frak B_{\omega},\bar\nu_{\omega})$ is to be
the measure
algebra of the usual measure on $\{0,1\}^{\Bbb N}$.   Given
$\ofamily{\eta}{\xi}{(\frak A_{\eta},\bar\mu_{\eta})}$, where
$0<\xi\le\zeta$,
let $(\frak A_{\xi},\bar\mu_{\xi})$ be the probability algebra reduced
product
$\prod_{k\in\Bbb N}(\frak A_{\theta(\xi,k)},\bar\mu_{\theta(\xi,k)})
|\Cal F_{\xi}$, as described in 328A-328C.  %328A 328B 328C
At the end of the induction, write
$(\frak C,\bar\nu)$ for $(\frak A_{\zeta},\bar\mu_{\zeta})$.

\medskip

{\bf (b)} We have a family
$\langle\phi_{\xi\eta}\rangle_{\eta\le\xi\le\zeta}$
defined by induction on $\xi$, as follows.
The inductive hypothesis will be that $\phi_{\eta'\eta}$ is a
measure-preserving Boolean homomorphism from $\frak A_{\eta}$ to
$\frak A_{\eta'}$, and that
$\phi_{\eta''\eta}=\phi_{\eta''\eta'}\phi_{\eta'\eta}$ whenever
$\eta\le\eta'\le\eta''<\xi$.   For the inductive step to $\xi$,
take $\phi_{\xi\xi}$ to be the identity map on $\frak A_{\xi}$.
If $\xi>0$, set $\tilde\phi_{kj}=\phi_{\theta(\xi,k),\theta(\xi,j)}$ for
$j\le k$ in $\Bbb N$;  then 328Ea
tells us that we have measure-preserving Boolean homomorphisms
$\tilde\phi_k:\frak A_{\theta(\xi,k)}\to\frak A_{\xi}$ such that
$\tilde\phi_j=\tilde\phi_k\tilde\phi_{kj}$ for $j\le k$.   If $j\le k$ and
$\eta\le\theta(\xi,j)$, then

\Centerline{$\tilde\phi_k\phi_{\theta(\xi,k),\eta}
=\tilde\phi_k\tilde\phi_{kj}\phi_{\theta(\xi,j),\eta}
=\tilde\phi_j\phi_{\theta(\xi,j),\eta}$}

\noindent whenever $k\ge j$;  so we can take this common value for
$\phi_{\xi\eta}$.   If $\eta\le\eta'<\xi$, then take $k$ such that
$\eta'\le\theta(\xi,k)$, and see that

\Centerline{$\phi_{\xi\eta'}\phi_{\eta'\eta}
=\tilde\phi_k\phi_{\theta(\xi,k),\eta'}\phi_{\eta'\eta}
=\tilde\phi_k\phi_{\theta(\xi,k),\eta}
=\phi_{\xi\eta}$,}

\noindent so the induction proceeds.

For each $\xi\le\zeta$,
write $\pi_{\xi}$ for $\phi_{\zeta\xi}:\frak A_{\xi}\to\frak C$, and
$\frak C_{\xi}$ for the subalgebra $\pi_{\xi}[\frak A_{\xi}]$.   Of
course $\pi_{\xi}\phi_{\xi\eta}=\pi_{\eta}$,
so that $\frak C_{\eta}\subseteq\frak C_{\xi}$,
whenever $\eta\le\xi\le\zeta$.

\medskip

{\bf (c)} For each $\xi>0$, we have a canonical map
$\sequence{k}{a_k}\to\sequence{k}{a_k}^{\ssbullet}:
\prod_{k\in\Bbb N}\frak A_{\theta(\xi,k)}\to\frak A_{\xi}$.   Since every
$\pi_{\xi}:\frak A_{\xi}\to\frak C_{\xi}$ is a measure-preserving
isomorphism, we have a corresponding map
$\psi_{\xi}:\prod_{k\in\Bbb N}\frak C_{\theta(\xi,k)}\to\frak C_{\xi}$.
Reading off the basic facts of 328Ab and 328Eb, we see that

\inset{----- $\bar\nu\psi_{\xi}(\sequence{k}{c_k})
=\lim_{k\to\Cal F_{\xi}}\bar\nu c_k$ for every sequence
$\sequence{k}{c_k}\in\prod_{k\in\Bbb N}\frak C_{\theta(\xi,k)}$,

----- $\psi_{\xi}(\sequence{k}{c_k})\Bsubseteq\sup_{k\in A}c_k$ whenever
$\sequence{k}{c_k}\in\prod_{k\in\Bbb N}\frak C_{\theta(\xi,k)}$ and
$A\in\Cal F_{\xi}$}

\noindent (we can take the supremum in $\frak C$ because $\frak C_{\xi}$ is
regularly embedded in $\frak C$, as noted in 314Ga).

\medskip

{\bf (d)} Let $\family{\tau}{S}{a_{\tau}}$ be a family in
$\frak A_0=\frak B_{\omega}$ such that
$\gamma=\inf_{\tau\in S}\bar\mu a_{\tau}$ is non-zero.
By 538Fe, we can find a
disjoint family $\langle A_{\xi}\rangle_{1\le\xi\le\zeta}$ of subsets of
$\Bbb N$ such that $A_{\xi}\in\Cal F_{\xi}$ for every $\xi$.
Use these to define $T\subseteq S$ and $\alpha:T\to[0,\zeta]$ as in
538Ee.   Set $c_{\tau}=0$ for $\tau\in S\setminus T$.
For $\tau\in T$ define $c_{\tau}\in\frak C_{\alpha(\tau)}$
by induction on $\alpha(\tau)$, as
follows.   If $\alpha(\tau)=0$, set $c_{\tau}=\pi_0a_{\tau}$.   For the
inductive step to $\alpha(\tau)=\xi>0$, we know that
$\tau^{\smallfrown}\fraction{k}\in T$ and
$\alpha(\tau^{\smallfrown}\fraction{k})=\theta(\xi,k)$ whenever
$k\in A_{\xi}$ and $\tau(i)<k$ for every $i<\dom\tau$;  for other
$k$, $\tau^{\smallfrown}\fraction{k}\notin T$ so
$c_{\tau^{\smallfrown}\fraction{k}}=0\in\frak C_{\theta(\xi,k)}$.
Thus $c_{\tau^{\smallfrown}\fraction{k}}\in\frak C_{\theta(\xi,k)}$ for
every $k$, and
$\psi_{\xi}(\sequence{k}{c_{\tau^{\smallfrown}\fraction{k}}})
\in\frak C_{\xi}$;  take this for $c_{\tau}$.   Note that

\Centerline{$\bar\nu c_{\tau}
=\lim_{k\to\Cal F_{\xi}}\bar\nu c_{\tau^{\smallfrown}\fraction{k}}
\ge\inf\{\bar\nu c_{\tau^{\smallfrown}\fraction{k}}:k\in\Bbb N$,
   $\tau^{\smallfrown}\fraction{k}\in T\}$.}

\noindent
Inducing on $\alpha(\tau)$, we see that $\bar\nu c_{\tau}\ge\gamma$ for
every $\tau\in T$.   In particular, $\bar\nu c_{\emptyset}\ge\gamma$.

\medskip

{\bf (e)} For $I\subseteq\Bbb N$, set $T_I=T\cap\bigcup_{n\in\Bbb N}I^n$
and $e_I=\inf_{\tau\in T_I}c_{\tau}$;  let $\Cal S$ be the family of those
finite subsets $I$ of $\Bbb N$ such that $e_I\ne\emptyset$.   Then
$T_{\emptyset}=\{\emptyset\}$, $e_{\emptyset}=c_{\emptyset}$ and
$\emptyset\in\Cal S$.   Moreover, if $I\in\Cal S$ and $1\le\xi\le\zeta$,
then $\{k:I\cup\{k\}\in\Cal S\}\in\Cal F_{\xi}$.   \Prf\ Set
$k_0=\sup I+1$.  If $k\in A_{\xi}$ and $k\ge k_0$, set

\Centerline{$d_k
=\inf\{c_{\tau^{\smallfrown}\fraction{k}}:
  \tau\in T_I$, $\alpha(\tau)=\xi\}$.}

\noindent Set $B=\{k:k\in A_{\xi},\,k\ge k_0,\,d_k\Bcap e_I\ne 0\}$.
If $k\in B$, then

\Centerline{$T_{I\cup\{k\}}
=T_I\cup\{\tau^{\smallfrown}\fraction{k}:\tau\in T_I$,
  $\alpha(\tau)=\xi\}$,}

\noindent because every member of $T$ is strictly increasing
and $\tau^{\smallfrown}\fraction{k}$ can belong to $T$ only when
$k\in A_{\alpha(\tau)}$, that is, $\alpha(\tau)=\xi$.   So
$e_{I\cup\{k\}}=d_k\Bcap e_I\ne 0$ and $I\cup\{k\}\in\Cal S$.

\Quer\ If $B\notin\Cal F_{\xi}$, then
$B'=\{k:k\in A_{\xi}$, $k\ge k_0$, $d_k\Bcap e_I=0\}$ belongs to
$\Cal F_{\xi}$.   So

$$\eqalignno{e_I
&\Bsubseteq\inf\{c_{\tau}:\tau\in T_I,\,\alpha(\tau)=\xi\}\cr
&=\inf_{\Atop{\tau\in T_I}{\alpha(\tau)=\xi}}
   \psi_{\xi}(\sequence{k}{c_{\tau^{\smallfrown}\fraction{k}}})
=\psi_{\xi}(\sequence{k}{\inf_{\Atop{\tau\in T_I}{\alpha(\tau)=\xi}}
   c_{\tau^{\smallfrown}\fraction{k}}})\cr
\displaycause{because $\psi_{\xi}$ is a Boolean homomorphism and $T_I$ is
finite}
&\Bsubseteq\sup_{k\in B'}\inf_{\Atop{\tau\in T_I}{\alpha(\tau)=\xi}}
   c_{\tau^{\smallfrown}\fraction{k}}\cr
\displaycause{by (c)}
&=\sup_{k\in B'}d_k.\cr}$$

\noindent But $e_I\Bcap d_k=0$ for every $k\in B'$ and $e_I\ne 0$.\ \Bang

Thus $\{k:I\cup\{k\}\in\Cal S\}\supseteq B\in\Cal F_{\xi}$.\ \Qed

\medskip

{\bf (f)} For $i\in\Bbb N$ set

\Centerline{$C_i=\{k:I\cup\{k\}\in\Cal S$ whenever $I\in\Cal S$ and
  $I\subseteq i\}$,}

\noindent so that $C_i\in\Cal F_{\xi}$ for every $\xi\in[1,\zeta]$.
At this point, recall that every $\Cal F_{\xi}$ is supposed to be a Ramsey
ultrafilter.   So for each $\xi\in[1,\zeta]$ we have an
$A'_{\xi}\in\Cal F_{\xi}$ such that $A'_{\xi}\subseteq A_{\xi}\cap C_0$ and
$j\in C_i$ whenever $i$, $j\in A'_{\xi}$ and $i<j$ (538Fc).
Next, for $i\in\Bbb N$ set
$M_i=\{\alpha(\tau):\tau\in T$, $\tau(j)\le i$ whenever $j<\dom\tau\}$;
then $M_i$ is finite, so there is a
$D\in\bigcap_{1\le\xi\le\zeta}\Cal F_{\xi}$ such that whenever
$i$, $j\in D$, $i<j$ and $\xi\in M_i$, there is a $k\in A'_{\xi}$ such that
$i<k<j$ (538Ff).   Of course we can suppose that
$D\subseteq\bigcup_{1\le\xi\le\zeta}A'_{\xi}$, so that
$D\cap A_{\xi}=D\cap A'_{\xi}$ for every $\xi$.

\medskip

{\bf (g)} $J\in\Cal S$ for every finite subset $J$ of $D$.   \Prf\ Induce
on $\#(J)$.   We know that $\emptyset\in\Cal S$.   If
$i\in D$, then $\{i\}\in\Cal S$ because $D\subseteq C_0$.   For the
inductive step to $\#(J)\ge 2$, set $j=\max J$, $I=J\setminus\{j\}$ and
$i=\max I$.   Then $I\in\Cal S$, by the inductive hypothesis;  so if
$T_J=T_I$, we certainly have $J\in\Cal S$.   Otherwise, there is a
member of $T_J\setminus T_I$, and this must be of the form
$\tau^{\smallfrown}\fraction{j}$ where $\tau\in T_I$ and
$j\in A_{\alpha(\tau)}$;  as $j\in D$, $j\in A'_{\alpha(\tau)}$.
But this means that $\alpha(\tau)\in M_i$ and
there is a $k\in A'_{\alpha(\tau)}$ such that $i<k<j$.   In this case,
$j\in C_k$ and $I\subseteq k$, so $J=I\cup\{k\}$ belongs to $\Cal S$, and
the induction proceeds.\ \Qed

\medskip

{\bf (h)} Thus $\{c_{\tau}:\tau\in T_D\}$ is centered;  setting
$T^*_D=\{\tau:\tau\in T_D$, $\alpha(\tau)=0\}$,
$\{c_{\tau}:\tau\in T^*_D\}$ and therefore
$\{a_{\tau}:\tau\in T^*_D\}$ are centered.   But $T^*_D$ belongs to
$\Cal G_{\zeta}$, by 538Ee.

Since $\family{\tau}{S}{a_{\tau}}$ was chosen arbitrarily in (d) above,
$\Cal G_{\zeta}$ satisfies the condition of 538G(ii), translated to the
countably infinite set $S$, and is measure-centering.
}%end of proof of 538L

\leader{538M}{Benedikt's theorem}\cmmnt{ ({\smc Benedikt 98})}
Let $(X,\Sigma,\mu)$
be a perfect probability space.   Then there is a measure
$\lambda$ on $X$, extending $\mu$, such that
$\lambda(\lim_{n\to\Cal F}E_n)$ is defined and equal to
$\lim_{n\to\Cal F}\mu E_n$ for every sequence $\sequencen{E_n}$ in
$\Sigma$ and every Ramsey filter $\Cal F$ on $\Bbb N$.

%what is the Maharam type of $\lambda$?

\proof{{\bf (a)} If there are no Ramsey filters, we can take $\lambda=\mu$ and
stop;  so let us suppose that there is at least one Ramsey filter.
Let $\frak F$ be a family of Ramsey filters consisting of just one member
of each isomorphism class, so that every Ramsey filter is isomorphic to
some member of $\frak F$, and no two members of $\frak F$ are isomorphic.
Fix a well-ordering $\preccurlyeq$ of $\frak F$ with a greatest member
$\Cal F^*$ and a family
$\langle\theta(\xi,k)\rangle_{1\le\xi<\omega_1,k\in\Bbb N}$ such that
$\sequence{k}{\theta(\xi,k)}$ is always a non-decreasing sequence of
ordinals less than $\xi$ such that $\{\theta(\xi,k):k\in\Bbb N\}$ is
cofinal with $\xi$.

\medskip

{\bf (b)(i)} For any
non-empty countable set $D\subseteq\frak F$ containing $\Cal F^*$,
enumerate it in $\preccurlyeq$-increasing order as
$\langle\Cal F_{\xi}\rangle_{1\le\xi\le\zeta}$, and let $\Cal G_D$ be the
measure-centering ultrafilter constructed from
$\langle\Cal F_{\xi}\rangle_{1\le\xi\le\zeta}$ and
$\langle\theta(\xi,k)\rangle_{1\le\xi\le\zeta,k\in\Bbb N}$ by the method of
538E-538L;  let $\lambda_D$ be the $\Cal G_D$-extension of $\mu$ as
defined in 538I.

\medskip

\quad{\bf (ii)} For any non-empty finite set $I\subseteq\frak F$,
list it in $\preccurlyeq$-increasing order as $\Cal F_0,\ldots,\Cal F_n$,
and set $\Cal H_I=\Cal F_n\ltimes\ldots\ltimes\Cal F_0$ as defined in 538D.
By 538Ed, or otherwise, $\Cal H_I\leRK\Cal G_I$, so $\Cal H_I$ is
measure-centering (538Hb);
let $\lambda'_I$ be the $\Cal H_I$-extension of $\mu$.

\medskip

{\bf (c)} If $\emptyset\ne I\subseteq J\in[\frak F]^{<\omega}$, then
$\Cal H_I\leRK\Cal H_J$, by 538Dg, and $\lambda'_J$ extends $\lambda'_I$,
by 538Jc.   Thus
$\langle\lambda'_I\rangle_{\emptyset\ne I\in[\frak F]^{<\omega}}$ is an
upwards-directed family of probability measures on $X$.

If $\Cal I\subseteq[\frak F]^{<\omega}\setminus\{\emptyset\}$ is
countable, we have a non-empty
countable set $D\subseteq\frak F$
including $\{\Cal F^*\}\cup\bigcup\Cal I$.   Now 538Ed
tells us that $\Cal H_I\leRK\Cal G_D$ for every $I\in\Cal I$, so that
$\lambda_D$ extends $\lambda'_I$ for every $I\in\Cal I$
(538Jc again).   Thus for every countable subset of
$\{\lambda'_I:I\in[\frak F]^{<\omega}\setminus\{\emptyset\}\}$ there is a
measure on $X$ extending them all.   By 457G, there is a measure $\lambda$
on $X$ extending every $\lambda'_I$.

\medskip

{\bf (d)} If $\Cal F$ is a Ramsey ultrafilter and $\sequencen{E_n}$ is a
sequence in $\Sigma$, there is an $\Cal F'\in\frak F$ such that
$\Cal F$ and $\Cal F'$ are isomorphic.   In particular,
$\Cal F\leRK\Cal F'$, so $\tilde\lambda_{\Cal F'}$ extends
$\tilde\lambda_{\Cal F}$, where
$\tilde\lambda_{\Cal F}$, $\tilde\lambda_{\Cal F'}$ are the
$\Cal F$-extension and $\Cal F'$-extension of
$\mu$.   But $\lambda$ extends
$\lambda'_{\{\Cal F'\}}=\tilde\lambda_{\Cal F'}$ and therefore extends
$\tilde\lambda_{\Cal F}$.   Accordingly $\lambda(\lim_{n\to\Cal F}E_n)$ is
defined and equal to
$\tilde\lambda_{\Cal F}(\lim_{n\to\Cal F}E_n)=\lim_{n\to\Cal F}\mu E_n$, as
required.
}%end of proof of 538M

\leader{538N}{Measure-converging filters:  Proposition} (a) Let $\Cal F$
be a free filter on $\Bbb N$.
Let $\nu_{\omega}$ be the usual measure on $\{0,1\}^{\Bbb N}$, and
$\Tau_{\omega}$ its domain.   Then the following are equiveridical:

\inset{(i) $\Cal F$ is measure-converging;

(ii) whenever $\sequencen{F_n}$ is a sequence in $\Tau_{\omega}$
and $\lim_{n\to\infty}\nu_{\omega}F_n=1$, then
$\bigcup_{A\in\Cal F}\bigcap_{n\in A}F_n$ is conegligible;

(iii) whenever $(X,\Sigma,\mu)$ is a measure space with locally determined
negligible sets\cmmnt{ (definition:  213I)}, and $\sequencen{f_n}$ is a
sequence in $\eusm L^0\cmmnt{\mskip5mu=\eusm L^0(\mu)}$
which converges in measure to
$f\in\eusm L^0$, then $\lim_{n\to\Cal F}f_n\eae f$;

(iv) whenever $\mu$ is a Radon measure on $\Cal P\Bbb N$ such that
$\lim_{n\to\infty}\mu E_n=1$, where $E_n=\{a:n\in a\subseteq\Bbb N\}$ for
each $n$, then $\mu\Cal F=1$.}

(b)\dvAnew{2012} Every measure-converging filter is free.

(c)\dvAnew{2012} Suppose that $\Cal F$ is a measure-converging filter.

\quad(i) If $\Cal G$ is a filter on $\Bbb N$ including $\Cal F$, then
$\Cal G$ is measure-converging.

\quad(ii) If $\Cal G$ is a filter on $\Bbb N$ and
$\Cal G\leRB\Cal F$\cmmnt{ (definition:  5A6Ic)}, then $\Cal G$ is
measure-converging.

(d)\cmmnt{ (M.Foreman)} Every rapid filter is measure-converging.

(e)\cmmnt{ (M.Talagrand)} If there is a measure-converging filter,
there is a measure-converging filter which is not rapid.

(f) Let $\Cal F$ be a measure-converging filter on $\Bbb N$ and
$\Cal G$ any filter on $\Bbb N$.   Then $\Cal G\ltimes\Cal F$ is
measure-converging.

(g) If $\frakmctbl=\frak d$, there is a rapid filter.

\proof{{\bf (a)(i)$\Rightarrow$(iii)} Suppose that $\Cal F$ is
measure-converging, and that $(X,\Sigma,\mu)$, $\sequencen{f_n}$ and $f$
are as in (iii).   Let $H\in\Sigma$ be a conegligible
set such that $H\subseteq\dom f\cap\dom f_n$ and $f\restr H$ and
$f_n\restr H$ are measurable for
every $n\in\Bbb N$.   Let $k\in\Bbb N$;  set
$H_k=\{x:x\in H$, $\limsup_{n\to\Cal F}|f_n(x)-f(x)|>2^{-k}\}$.   Then
$H_k\cap F$ is negligible whenever $F\in\Sigma$ and $\mu F<\infty$.   \Prf\
If $\mu F=0$ this is trivial.   Otherwise, let $\nu=\Bover1{\mu F}\mu_F$ be
the normalized subspace measure on $F$.   For each $n\in\Bbb N$, set
$F_n=\{x:x\in F\cap H$, $|f_n(x)-f(x)|\le 2^{-k}\}$.   Then

\Centerline{$\lim_{n\to\infty}\nu(F\setminus F_n)
\le\Bover{2^k}{\mu F}\lim_{n\to\infty}\int\min(|f_n-f|,\chi F)d\mu
=0$}

\noindent because $\sequencen{f_n}\to f$ in measure.   So
$\lim_{n\to\infty}\nu F_n=1$ and
$H'=\bigcup_{A\in\Cal F}\bigcap_{n\in A}F_n$ is $\nu$-conegligible.
But $H'\cap H_k=\emptyset$, so $\mu^*(H_k\cap F)=\nu^*(H_k\cap F)=0$.\ \Qed

Since $\mu$ has locally determined negligible sets, $H_k$ is negligible.
This is true for every $k\in\Bbb N$, so $H\setminus\bigcup_{k\in\Bbb N}H_k$
is conegligible;  and $\lim_{n\to\Cal F}f_n(x)=f(x)$ for every
$x\in H\setminus\bigcup_{k\in\Bbb N}H_k$, so $\lim_{n\to\Cal F}f_n=f$ a.e.,
as required.

\medskip

\quad{\bf (iii)$\Rightarrow$(iv)} Assuming (iii), let $\mu$ and
$\sequencen{E_n}$ be as in (iv).   Set
$f_n=\chi(\Cal P\Bbb N\setminus E_n)$ for each $n$;  then
$\lim_{n\to\infty}\int f_nd\mu=0$, so $\sequencen{f_n}\to 0$ in measure,
and $H=\{a:\lim_{n\to\Cal F}f_n(a)=0\}$ is conegligible.
But for any $a\in H$,

\Centerline{$a=\{n:a\in E_n\}=\{n:f_n(x)\le\Bover12\}$}

\noindent belongs to $\Cal F$, so $H\subseteq\Cal F$ and $\mu\Cal F=1$.

\medskip

\quad{\bf (iv)$\Rightarrow$(ii)} Assume (iv), and let $\sequencen{F_n}$ be
as in (ii).   Define $\phi:\{0,1\}^{\Bbb N}\to\Cal P\Bbb N$ by setting
$\phi(x)=\{n:x\in F_n\}$ for $x\in\{0,1\}^{\Bbb N}$.   Then $\phi$ is
almost continuous (418J), so the image measure
$\mu=\nu_{\omega}\phi^{-1}$ on $\Cal P\Bbb N$ is a Radon measure (418I).
Since $F_n=\phi^{-1}[E_n]$ for each $n$, $\lim_{n\to\infty}\mu E_n=1$ and
$1=\mu\Cal F=\nu_{\omega}\phi^{-1}[\Cal F]$.   But now

\Centerline{$\bigcup_{a\in\Cal F}\bigcap_{n\in a}F_n
=\bigcup_{a\in\Cal F}\{x:a\subseteq\phi(x)\}
=\phi^{-1}[\Cal F]$}

\noindent is $\nu_{\omega}$-conegligible, as required.

\medskip

\quad{\bf (ii)$\Rightarrow$(i)} Assume (ii), and take a probability space
$(X,\Sigma,\mu)$ and a sequence $\sequencen{H_n}$ in $\Sigma$ such that
$\lim_{n\to\infty}\mu H_n\penalty-50=1$;  set
$G=\bigcup_{A\in\Cal F}\bigcap_{n\in A}H_n$.

Let $\lambda$ be the c.l.d. product
measure on $X\times\{0,1\}^{\Bbb N}$, and set

\Centerline{$W_n=H_n\times\{0,1\}^{\Bbb N}$,
\quad$V_n=\{(x,y):x\in X$, $y\in\{0,1\}^{\Bbb N}$, $y(n)=1\}$}

\noindent for $n\in\Bbb N$.   Let $\Lambda_1$ be the $\sigma$-algebra of
subsets of $X\times\{0,1\}^{\Bbb N}$ generated by
$\{W_n:n\in\Bbb N\}\cup\{V_n:n\in\Bbb N\}$, and $\lambda_1$ the completion
of the restriction $\lambda\restr\Lambda_1$.   Note that as the identity
map from $X\times\{0,1\}^{\Bbb N}$ is \imp\ for $\lambda$ and
$\lambda\restr\Lambda_1$, it is \imp\ for their completions
(234Ba);  but
$\lambda$ is complete, so this just means that $\lambda$ extends
$\lambda_1$.   Then $\lambda_1$ is a complete
atomless probability measure with countable Maharam type.   Its measure
algebra $\frak C$ is therefore isomorphic, as measure algebra, to the
measure algebra $\frak B_{\omega}$ of $\nu_{\omega}$;  let
$\pi:\frak B_{\omega}\to\frak C$ be a measure-preserving isomorphism.
By 343B, or otherwise, there is a realization
$\phi:X\times\{0,1\}^{\Bbb N}\to\{0,1\}^{\Bbb N}$,
\imp\ for $\lambda_1$ and $\nu_{\omega}$, such that
$\phi^{-1}[F]^{\ssbullet}=\pi F^{\ssbullet}$ in $\frak C$ for every
$F\in\Tau_{\omega}$.   Because $\pi$ is surjective, there is for each
$n\in\Bbb N$ an $F_n\in\Tau_{\omega}$ such that
$\phi^{-1}[F_n]\symmdiff W_n$ is $\lambda_1$-negligible.

Since

\Centerline{$\lim_{n\to\infty}\nu_{\omega}F_n
=\lim_{n\to\infty}\lambda_1W_n
=\lim_{n\to\infty}\lambda W_n
=\lim_{n\to\infty}\mu H_n=1$,}

\noindent $F=\bigcup_{A\in\Cal F}\bigcap_{n\in A}F_n$ is
$\nu_{\omega}$-conegligible, and
$\phi^{-1}[F]=\bigcup_{A\in\Cal F}\bigcap_{n\in A}\phi^{-1}[F_n]$ is
$\lambda_1$-conegligible.   We have
$G\times\{0,1\}^{\Bbb N}=\bigcup_{A\in\Cal F}\bigcap_{n\in A}W_n$,
so

\Centerline{$(G\times\{0,1\}^{\Bbb N})\symmdiff\phi^{-1}[F]
\subseteq\bigcup_{n\in\Bbb N}W_n\symmdiff\phi^{-1}[F_n]$}

\noindent is $\lambda_1$-negligible.
Thus $G\times\{0,1\}^{\Bbb N}$ is $\lambda_1$-conegligible, therefore
$\lambda$-conegligible.   But this means that $G$ is $\mu$-conegligible, by
252D applied to $G\times\{0,1\}^{\Bbb N}$;  and this is what we
needed to know.

\medskip

{\bf (b)} Let $\Cal F$ be a measure-converging filter and $m\in\Bbb N$.
Take a singleton set $X=\{x\}$ and the probability measure $\mu$ on $X$;
set $E_i=\emptyset$ for $i<n$, $X$ for $i\ge n$.   Then
$\lim_{i\to\infty}\mu E_i=1$, so there is an $A\in\Cal F$ such that
$\bigcap_{i\in A}E_i$ is non-empty.   Now $\Bbb N\setminus n\supseteq A$
belongs to $\Cal F$;  as $n$ is arbitrary, $\Cal F$ is free.

\medskip

{\bf (c)(i)} Immediate from the definition in 538Ag.

\medskip

\quad{\bf (ii)} Let $f:\Bbb N\to\Bbb N$ be a finite-to-one function such
that $\Cal G=f[[\Cal F]]$.
Let $(X,\Sigma,\mu)$ be a probability space and
$\sequencen{E_n}$ a sequence in $\Sigma$ such that
$\lim_{n\to\infty}\mu E_n=1$.   Set $F_n=E_{f(n)}$ for $n\in\Bbb N$;
because $f$ is finite-to-one, $\lim_{n\to\infty}\mu F_n=1$.   So
$H=\bigcup_{A\in\Cal F}\bigcap_{n\in A}F_n$ is conegligible.   If
$x\in H$, set $A_x=\{n:x\in E_n\}$;  then

\Centerline{$f^{-1}[A_x]=\{n:x\in E_{f(n)}\}=\{n:x\in F_n\}$}

\noindent belongs to $\Cal F$ so $A_x\in f[[\Cal F]]$ and
$x\in\bigcup_{B\in f[[\Cal F]]}\bigcap_{n\in B}E_n$.   Thus
$\bigcup_{B\in f[[\Cal F]]}\bigcap_{n\in B}E_n\supseteq H$ is conegligible.
As $(X,\Sigma,\mu)$ and $\sequencen{E_n}$ are arbitrary, $f[[\Cal F]]$ is
measure-converging.

\medskip

{\bf (d)} Let $\Cal F$ be a rapid filter on $\Bbb N$,
and $\sequencen{H_n}$ a sequence in $\Tau_{\omega}$ such that
$\lim_{n\to\infty}\nu_{\omega}H_n=1$.   Set
$G=\bigcup_{A\in\Cal F}\bigcap_{n\in A}H_n$.   Since
$\lim_{n\to\infty}(1-\nu_{\omega}H_n)=0$, there is an $A\in\Cal F$ such
that $\sum_{n\in A}1-\nu_{\omega}H_n<\infty$;   set
$H=\bigcup_{m\in\Bbb N}\bigcap_{n\in A\setminus m}H_n\subseteq G$.
Then

\Centerline{$\nu_{\omega}H
\ge\sup_{m\in\Bbb N}1-\sum_{n\in A\setminus m}(1-\nu_{\omega}H_n)
=1$,}

\noindent so $G$ is conegligible.   Thus $\Cal F$ satisfies (a-ii) and is
measure-converging.

\medskip

{\bf (e)} Let $\Cal F$ be a measure-converging filter.
Let $\sequencen{I_n}$ be a
sequence of non-empty finite subsets of $\Bbb N$ such that
$\lim_{n\to\infty}\#(I_n)=\infty$.   Let $\Cal G$ be

\Centerline{$\{A:A\subseteq\Bbb N$,
$\lim_{n\to\Cal F}\Bover1{\#(I_n)}\#(A\cap I_n)=1\}$.}

\noindent Then $\Cal G$ is a filter on $\Bbb N$.

\medskip

\quad{\bf (i)} $\Cal G$ is measure-converging.  \Prf\ Let
$\sequence{i}{H_i}$ be a sequence in $\Tau_{\omega}$ such that
$\lim_{i\to\infty}\nu_{\omega}H_i=1$, and set
$G=\bigcup_{A\in\Cal G}\bigcap_{i\in A}H_i$.   Set
$g_n=\Bover1{\#(I_n)}\sum_{i\in I_n}\chi H_i$ for each $n$;  then

\Centerline{$\lim_{n\to\infty}\int g_n
=\lim_{n\to\infty}\Bover1{\#(I_n)}\sum_{i\in I_n}\nu_{\omega}H_i
=1$}

\noindent because $\lim_{n\to\infty}\#(I_n)=\infty$ and
$\lim_{i\to\infty}\nu_{\omega}H_i=1$.   Since
$0\le g_n\le\chi\{0,1\}^{\Bbb N}$ for every $n$,
$\sequencen{g_n}\to\chi\{0,1\}^{\Bbb N}$ in measure.   By (a-iii) above,
$H=\{x:\lim_{n\to\Cal F}g_n(x)=1\}$ is conegligible.

For $x\in H$, set $A_x=\{i:x\in H_i\}$.   Then

\Centerline{$\Bover1{\#(I_n)}\#(I_n\cap A_x)
=g_n(x)\to 1$}

\noindent as $n\to\Cal F$, so $A_x\in\Cal G$ and $x\in G$.   Accordingly
$G\supseteq H$ is conegligible.   As $\sequence{i}{H_i}$ is arbitrary,
$\Cal G$ is measure-converging.\ \Qed

\medskip

\quad{\bf (ii)} $\Cal G$ is not rapid.   \Prf\
Define $\sequence{i}{t_i}$ by saying that

\Centerline{$t_i=\sup\{\Bover1{\#(I_n)}:n\in\Bbb N$, $i\in I_n\}$}

\noindent for $i\in\Bbb N$, counting $\sup\emptyset$ as $0$.
Then $\lim_{i\to\infty}t_i=0$.   If $A\in\Cal G$ and $m\in\Bbb N$, then
$B=\{n:\#(A\cap I_n)\ge\bover23\#(I_n)\}$
belongs to $\Cal F$, and must be infinite, by (b) above.
So there is an $n\in B$ such that $\#(I_n)\ge 3m$, and now

\Centerline{$\sum_{i\in A\setminus m}t_i
\ge\#(A\cap I_n\setminus m)\cdot\Bover1{\#(I_n)}\ge\Bover13$.}

\noindent As $m$ is arbitrary, $\sum_{i\in A}t_i=\infty$;  as $A$ is
arbitrary, $\Cal G$ is not rapid.\ \Qed

\medskip

{\bf (f)} Let
$\langle E_{ij}\rangle_{i,j\in\Bbb N}$ be a family in $\Tau_{\omega}$
such that $\sequence{n}{\nu_{\omega}E_{i_nj_n}}\to 1$ for some, or
any, enumeration $\sequencen{(i_n,j_n)}$ of $\Bbb N\times\Bbb N$.
Set $G=\bigcup_{C\in\Cal G\ltimes\Cal F}\bigcap_{(i,j)\in C}E_{ij}$.
For each $i\in\Bbb N$, $\lim_{j\to\infty}\nu_{\omega}E_{ij}=1$, so
$G_i=\bigcup_{A\in\Cal F}\bigcap_{j\in A}E_{ij}$ is conegligible;  set
$H=\bigcap_{i\in\Bbb N}G_i$.   For $x\in H$, set
$A_x=\{(i,j):x\in E_{ij}\}$.   As $x\in G_i$, $A_x[\{i\}]\in\Cal F$ for
every $i\in\Bbb N$;  thus $A_x\in\Cal G\ltimes\Cal F$ and $x\in G$.
So $G$ includes the conegligible set $H$, and is itself conegligible.
As $\langle E_{ij}\rangle_{i,j\in\Bbb N}$ is arbitrary, $\Cal G$ is
measure-converging.

\medskip

{\bf (g)(i)} Suppose that $\Cal E\subseteq[\Bbb N]^{\omega}$ is a
family with $\#(\Cal E)<\frakmctbl$, and that $f\in\NN$
is non-decreasing.   Then there
is an $A\subseteq\Bbb N$, meeting every member of $\Cal E$,
such that $\#(A\cap f(n))\le n$ for every $n\in\Bbb N$.   \Prf\
Consider
$X=\prod_{n\in\Bbb N}\Bbb N\setminus f(n)$.   Then $X$ is a closed subset
of $\NN$, homeomorphic to $\NN$.   For $E\in\Cal E$, set

\Centerline{$G_E=\{x:x\in X$, $E\cap x[\Bbb N]\ne\emptyset\}$;}

\noindent then $G_E$ is a dense open subset of $X$.   Writing $\Cal M(X)$
for the ideal of meager subsets of $X$,
$\#(\Cal E)<\frakmctbl=\cov\Cal M(X)$, so there is an
$x\in X\cap\bigcap_{E\in\Cal E}G_E$;  set $A=x[\Bbb N]$.\ \Qed

\medskip

\quad{\bf (ii)} Let $\ofamily{\xi}{\frak d}{f_{\xi}}$ be a cofinal family
in $\BbbN^{\Bbb N}$;  we may suppose that every $f_{\xi}$ is
strictly increasing.    Choose a non-decreasing family
$\langle\Cal E_{\xi}\rangle_{\xi\le\frak d}$
inductively, as follows.
$\Cal E_0=\{\Bbb N\setminus n:n\in\Bbb N\}$.   Given that
$\xi<\frak d=\frakmctbl$ and that
$\Cal E_{\xi}\subseteq[\Bbb N]^{<\omega}$ is a filter base with cardinal at
most $\max(\omega,\#(\xi))$, use (i) to find a set
$A_{\xi}\subseteq\Bbb N$, meeting every member of $\Cal E_{\xi}$, such that
$\#(A_{\xi}\cap f_{\xi}(n))\le n$ for every $n$;  set

\Centerline{$\Cal E_{\xi+1}
=\Cal E_{\xi}\cup\{A_{\xi}\cap E:E\in\Cal E_{\xi}\}$.}

\noindent For non-zero limit ordinals $\xi\le\frak d$ set
$\Cal E_{\xi}=\bigcup_{\eta<\xi}\Cal E_{\eta}$.

At the end of the induction, let $\Cal F$ be the filter on $\Bbb N$
generated by $\Cal E_{\frak d}$.   Then $\Cal F$ is rapid.   \Prf\ It is
free because $\Cal E_0\subseteq\Cal F$.   If $\sequencen{t_n}$ is a
sequence in $\Bbb R$ converging to $0$, let $f\in\NN$ be such that
$|t_i|\le 2^{-n}$ whenever $n\in\Bbb N$ and $i\ge f(n)$, and let
$\xi<\frak d$ be such that $f\le f_{\xi}$.   Then $A_{\xi}\in\Cal F$ and

\Centerline{$\sum_{i\in A_{\xi}}|t_i|
\le\sum_{n=0}^{\infty}
   2^{-n}\#(A_{\xi}\cap f_{\xi}(n+1)\setminus f_{\xi}(n))
\le\sum_{n=0}^{\infty}2^{-n}(n+1)$}

\noindent is finite.\ \Qed
}%end of proof of 538N

\leader{538O}{The Fatou property:  Proposition} (a)
Let $\Cal F$ be a filter on $\Bbb N$.   Let $\nu_{\omega}$ be the usual
measure on $\{0,1\}^{\Bbb N}$, and $\Tau_{\omega}$ its domain.   Then the
following are equiveridical:

\inset{(i) $\Cal F$ has the Fatou property;

(ii) whenever $\sequencen{F_n}$ is a sequence in $\Tau_{\omega}$ and
$\nu_{\omega}^*(\bigcup_{A\in\Cal F}\bigcap_{n\in A}F_n)=1$, then
$\lim_{n\to\Cal F}\nu_{\omega}F_n=1$;

(iii) whenever $(X,\Sigma,\mu)$ is a measure space and
$\sequencen{f_n}$ is a sequence of non-negative functions in
$\eusm L^0(\mu)$, then $\overline{\int}\liminf_{n\to\Cal F}f_nd\mu
\le\liminf_{n\to\Cal F}\int f_nd\mu$;

(iv) whenever $\mu$ is a Radon
probability measure on $\Cal P\Bbb N$, then
$\mu^*\Cal F\le\liminf_{n\to\Cal F}\mu E_n$, where
$E_n=\{a:n\in a\subseteq\Bbb N\}$ for each
$n\in\Bbb N$.}

(b) If $\Cal F$ and $\Cal G$ are filters on $\Bbb N$, $\Cal G\leRK\Cal F$
and $\Cal F$ has the Fatou property, then $\Cal G$ has the Fatou property.

(c) If $\Cal F$ and $\Cal G$ are filters on $\Bbb N$ with
the Fatou property, then $\Cal F\ltimes\Cal G$ has the Fatou property.

\medskip

\proof{{\bf (a) not-(iii)$\Rightarrow$not-(i)}
Suppose that $(X,\Sigma,\mu)$ is a measure space and
$\sequencen{f_n}$ a sequence of non-negative functions
in $\eusm L^0$ such that
$\overline{\int}\liminf_{n\to\Cal F}f_nd\mu
>\liminf_{n\to\Cal F}\int f_nd\mu$.   Changing the $f_n$ on negligible sets
does not change either $\overline{\int}\liminf_{n\to\Cal F}f_nd\mu$ or
$\overline{\int}\liminf_{n\to\Cal F}f_nd\mu$, so we may
assume that every $f_n$ is defined everywhere in $X$ and is
$\Sigma$-measurable.   Take $\alpha$ such that
$\overline{\int}\liminf_{n\to\Cal F}f_nd\mu
>\alpha>\liminf_{n\to\Cal F}\int f_nd\mu$;  set
$A=\{n:\int f_nd\mu\le\alpha\}$;  then $A$ meets every member of
$\Cal F$.   Since $f_n$ is integrable for every $n\in A$,
the set $G=\{x:\sup_{n\in A}f_n(x)>0\}$ is a countable
union of sets of finite measure.

Let $\lambda$ be the c.l.d.\ product measure on $G\times\Bbb R$, and
consider the ordinate sets
$W_n=\{(x,\beta):x\in G$, $0\le\beta<f_n(x)\}$ for $n\in A$.   Set
$W=\bigcup_{C\in\Cal F}\bigcap_{n\in C\cap A}W_n$;
writing $g$ for $\liminf_{n\to\Cal F}f_n$,

\Centerline{$\{(x,\beta):x\in G,\,0\le\beta<g(x)\}
\subseteq W$.}

\noindent Since $\lambda$ is a product of two $\sigma$-finite measures it
is $\sigma$-finite, and $W$ has a measurable envelope $\tilde W$ say.
Now $\lambda^*W>\alpha$.   \Prf\Quer\ Otherwise,
$\lambda\tilde W\le\alpha$.   Writing $\mu_L$ for Lebesgue measure on
$\Bbb R$,

$$\eqalignno{\alpha
\ge\lambda\tilde W
&=\int_G\mu_L\tilde W[\{x\}]\mu(dx)\cr
\displaycause{252D}
&\ge\overline{\int}_Gg\,d\mu
>\alpha.  \text{ \Bang\Qed}\cr}$$

There is therefore a set $V\subseteq\tilde W$ such that
$\alpha<\lambda V<\infty$, and now $\lambda^*(V\cap W)>\alpha$.
Let $\nu$ be the subspace measure on $V\cap W$.   Set

$$\eqalign{V_n
&=V\cap W\cap W_n\text{ if }n\in A,\cr
&=V\cap W\text{ if }n\in\Bbb N\setminus A.\cr}$$

\noindent Then

$$\eqalign{\liminf_{n\to\Cal F}\nu V_n
&=\sup_{C\in\Cal F}\inf_{n\in C}\nu V_n
\le\sup_{n\in A}\nu V_n\cr
&\le\sup_{n\in A}\lambda W_n
=\sup_{n\in A}\int f_nd\mu
\le\alpha.\cr}$$

\noindent On the other hand,

\Centerline{$\bigcup_{C\in\Cal F}\bigcap_{n\in C}V_n
=\bigcup_{C\in\Cal F}\bigcap_{n\in C\cap A}V\cap W\cap W_n
=V\cap W$}

\noindent and $\nu(V\cap W)=\lambda^*(V\cap W)>\alpha$.
Moving to a normalization of $\nu$, we see that (i) is false.

\medskip

\quad{\bf (iii)$\Rightarrow$(iv)} If $\Cal F$ satisfies (iii) and
$\mu$ is a Radon probability measure on $\Cal P\Bbb N$, set
$g=\liminf_{n\to\Cal F}\chi E_n$.   If $a\in\Cal F$, then
$\{n:\chi E_n(a)=1\}=a\in\Cal F$, so $g(a)=1$;  thus

$$\eqalignno{\mu^*\Cal F
&=\overline{\int}\chi\Cal Fd\mu\cr
\displaycause{133Je}
&\le\overline{\int}g\,d\mu
\le\liminf_{n\to\Cal F}\int\chi E_n
=\liminf_{n\to\Cal F}\mu E_n,\cr}$$

\noindent as required.

\medskip

\quad{\bf (iv)$\Rightarrow$(ii)} Given (iv),
suppose that $\sequencen{F_n}$ is a sequence in $\Tau_{\omega}$
and $\nu_{\omega}^*(\bigcup_{A\in\Cal F}\bigcap_{n\in A}F_n)=1$.
As in the corresponding part of the argument for 538Na, define
$\phi:\{0,1\}^{\Bbb N}\to\Cal P\Bbb N$ by setting
$\phi(x)=\{n:x\in F_n\}$, and let $\mu$ be the Radon measure
$\nu_{\omega}\phi^{-1}$.   Then

\Centerline{$\mu^*\Cal F=\nu_{\omega}^*\phi^{-1}[\Cal F]
=\nu_{\omega}^*(\bigcup_{A\in\Cal F}\bigcap_{n\in A}F_n)=1$}

\noindent (using 451Pc again for the first equality),
so $\lim_{n\to\Cal F}\nu_{\omega}F_n=\lim_{n\to\Cal F}\mu E_n=1$.

\medskip

\quad{\bf (ii)$\Rightarrow$(i)}
Assume (ii), and take a probability space
$(X,\Sigma,\mu)$ and a sequence $\sequencen{H_n}$ in $\Sigma$ such that
$X=\bigcup_{A\in\Cal F}\bigcap_{n\in A}H_n$.

As in the corresponding part of the argument for 538Na,
let $\lambda$ be the c.l.d. product
measure on $X\times\{0,1\}^{\Bbb N}$, and set

\Centerline{$W_n=H_n\times\{0,1\}^{\Bbb N}$,
\quad$V_n=\{(x,y):x\in X$, $y\in\{0,1\}^{\Bbb N}$, $y(n)=1\}$}

\noindent for $n\in\Bbb N$.   Let $\Lambda_1$ be the $\sigma$-algebra of
subsets of $X\times\{0,1\}^{\Bbb N}$ generated by
$\{W_n:n\in\Bbb N\}\cup\{V_n:n\in\Bbb N\}$, and $\lambda_1$ the completion
of the restriction $\lambda\restr\Lambda_1$.   As before, there is a
function
$\phi:X\times\{0,1\}^{\Bbb N}\to\{0,1\}^{\Bbb N}$, \imp\ for
$\lambda_1$ and $\nu_{\omega}$, such that there is for each
$n\in\Bbb N$ an $F_n\in\Tau_{\omega}$ such that
$\phi^{-1}[F_n]\symmdiff W_n$ is $\lambda_1$-negligible.
Set $G=\bigcup_{A\in\Cal F}\bigcap_{n\in A}F_n$.

Since $X=\bigcup_{A\in\Cal F}\bigcap_{n\in A}H_n$,
$X\times\{0,1\}^{\Bbb N}=\bigcup_{A\in\Cal F}\bigcap_{n\in A}W_n$ and

\Centerline{$(X\times\{0,1\}^{\Bbb N})\setminus\phi^{-1}[G]
\subseteq\bigcup_{n\in\Bbb N}\phi^{-1}[F_n]\symmdiff W_n$}

\noindent is $\lambda_1$-negligible.   By 413Eh,

\Centerline{$\nu^*_{\omega}G\ge\lambda_1\phi^{-1}[G]=1$.}

\noindent By (ii), $\lim_{n\to\Cal F}\nu_{\omega}F_n=1$.   But

\Centerline{$\nu_{\omega}F_n
=\lambda_1\phi^{-1}[F_n]=\lambda_1W_n=\lambda W_n=\mu H_n$}

\noindent for each $n$, so $\lim_{n\to\Cal F}\mu H_n=1$.
As $(X,\Sigma,\mu)$ and $\sequencen{H_n}$ are arbitrary, $\Cal F$ has the
Fatou property.

\medskip

{\bf (b)} Let $h:\Bbb N\to\Bbb N$ be such that $\Cal G=h[[\Cal F]]$.
Let $(X,\Sigma,\mu)$ be a probability space and $\sequencen{H_n}$ a
sequence in $\Sigma$ such that

$$\eqalign{X
&=\bigcup_{A\in\Cal G}\bigcap_{n\in A}H_n\cr
&=\bigcup_{A\subseteq\Bbb N,h^{-1}[A]\in\Cal F}\bigcap_{n\in A}H_n
=\bigcup_{A\in\Cal F}\bigcap_{n\in A}H_{h(n)}.\cr}$$

\noindent Then

$$\eqalign{1
&=\liminf_{n\to\Cal F}\mu H_{h(n)}
=\sup_{A\in\Cal F}\inf_{n\in A}\mu H_{h(n)}\cr
&\le\sup_{A\in\Cal G}\inf_{n\in A}\mu H_n
=\liminf_{n\to\Cal G}\mu H_n.\cr}$$

\noindent As $(X,\Sigma,\mu)$ and $\sequencen{H_n}$ are arbitrary,
$\Cal G$ has the Fatou property.

\medskip

{\bf (c)} Let $(X,\Sigma,\mu)$ be a probability space and
$\langle E_{ij}\rangle_{i,j\in\Bbb N}$ a family in $\Sigma$ such that
$X=\bigcup_{C\in\Cal F\ltimes\Cal G}\bigcap_{(i,j)\in C}E_{ij}$.
For each $i\in\Bbb N$, set
$F_i=\bigcup_{B\in\Cal G}\bigcap_{j\in B}E_{ij}$, and let $G_i\in\Sigma$ be
a measurable envelope of $F_i$.   Then
$\bigcup_{A\in\Cal F}\bigcap_{i\in A}G_i=X$.   \Prf\ If $x\in X$, there is
a $C\in\Cal F\ltimes\Cal G$ such that $x\in E_{ij}$ whenever
$(i,j)\in C$.   Set $A=\{i:C[\{i\}]\in\Cal G\}\in\Cal F$.
If $i\in A$, then

\Centerline{$x\in\bigcap_{j\in C[\{i\}]}E_{ij}\subseteq F_i
\subseteq G_i$,}

\noindent so $x\in\bigcap_{i\in A}G_i$.\ \Qed

Accordingly $\lim_{i\to\Cal F}\mu G_i=1$.   Take $\epsilon>0$;  then
$A=\{i:\mu G_i\ge 1-\epsilon\}$ belongs to $\Cal F$.   For each
$i\in A$,

$$\eqalignno{1-\epsilon
&\le\mu G_i
=\mu^*F_i
=\overline{\int}\chi F_i
=\overline{\int}\liminf_{j\to\Cal G}\chi E_{ij}
\le\liminf_{j\to\Cal G}\int\chi E_{ij}\cr
\displaycause{by (a-iii) above}
&=\liminf_{j\to\Cal G}\mu E_{ij},\cr}$$

\noindent so $\{j:\mu E_{ij}\ge 1-2\epsilon\}\in\Cal G$.   But this means
that $\{(i,j):\mu E_{ij}\ge 1-2\epsilon\}\in\Cal F\ltimes\Cal G$.
As $\epsilon$ is arbitrary,
$\lim_{(i,j)\to\Cal F\ltimes\Cal G}\mu E_{ij}=1$.   As $(X,\Sigma,\mu)$ and
$\langle E_{ij}\rangle_{i,j\in\Bbb N}$ are arbitrary, $\Cal F\ltimes\Cal G$
has the Fatou property.
}%end of proof of 538O

\leader{538P}{Theorem} Let $\nu:\Cal P\Bbb N\to\Bbb R$ be a bounded
finitely additive functional.   Write $\dashint\ldots d\nu$ for the
associated linear functional on $\ell^{\infty}$\cmmnt{ (see 363L)},
and set $E_n=\{a:n\in a\subseteq\Bbb N\}$ for each
$n\in\Bbb N$.   Then the following are equiveridical:

(i) whenever $\mu$ is a Radon probability measure on $\Cal P\Bbb N$,
$\int\nu(a)\mu(da)$ is defined and equal to $\dashint\mu E_n\nu(dn)$;

\allowmorestretch{468}{
(ii) whenever $\mu$ is a Radon probability measure on $[0,1]^{\Bbb N}$,
$\idashint x\,d\nu\,\mu(dx)$ is defined and equal to
\penalty-100$\dashiint x(n)\mu(dx)\nu(dn)$;
}

(iii) whenever $(X,\Sigma,\mu)$ is a probability space and
$\sequencen{f_n}$ is a uniformly bounded sequence of measurable
real-valued functions on $X$, then $\idashint f_n(x)\nu(dn)\mu(dx)$ is
defined and equal to $\dashiint f_nd\mu\,\nu(dn)$;

(iv) whenever $\sequencen{F_n}$ is a sequence of Borel subsets of
$\{0,1\}^{\Bbb N}$, $\idashint\chi F_n(x)\nu(dn)\nu_{\omega}(dx)$ is defined
and equal to $\dashint\nu_{\omega}F_n\nu(dn)$, where $\nu_{\omega}$ is the
usual measure on $\{0,1\}^{\Bbb N}$.

\proof{{\bf (i)$\Rightarrow$(ii)}\grheada\
For $t\in[0,1]$ define $h_t:[0,1]^{\Bbb N}\to\Cal P\Bbb N$ by setting
$h_t(x)=\{n:x(n)\ge t\}$ for $x\in[0,1]^{\Bbb N}$, and let
$\mu_t=\mu h_t^{-1}$ be the image measure on $\Cal P\Bbb N$.   Then
$\mu_t$ is a Radon measure for each $t$.   \Prf\
Because $h_t$ is Borel measurable and
$\Cal P\Bbb N$ is metrizable, $h_t$ is almost continuous (418J),
so $\mu_t$ is a Radon measure (418I).\ \Qed

\medskip

\quad\grheadb\
For $m\in\Bbb N$ define $v_m\in[0,1]^{\Bbb N}$ by setting

\Centerline{$v_m(n)=2^{-m}\sum_{k=1}^{2^m}\mu\{x:x(n)\ge 2^{-m}k\}$.}

\noindent Then $\|v_{m+1}-v_m\|_{\infty}\le 2^{-m-1}$.   \Prf\ For any
$n\in\Bbb N$,

$$\eqalign{v_m(n)-v_{m+1}(n)
&=2^{-m}\sum_{k=1}^{2^m}\mu\{x:x(n)\ge 2^{-m}k\}
  -2^{-m-1}\sum_{k=1}^{2^{m+1}}\mu\{x:x(n)\ge 2^{-m-1}k\}\cr
&=2^{-m-1}\sum_{k=1}^{2^m}(2\mu\{x:x(n)\ge 2^{-m}k\}
  -\mu\{x:x(n)\ge 2^{-m}k\}\cr
&\mskip150mu-\mu\{x:x(n)\ge 2^{-m-1}(2k+1)\})\cr
&=2^{-m-1}\sum_{k=1}^{2^m}\mu\{x:2^{-m}k\le x(n)<2^{-m-1}(2k+1)\}
\le 2^{-m-1}.  \text{ \Qed}\cr}$$

\noindent So $v=\lim_{m\to\infty}v_m$ is defined in $\ell^{\infty}$ and
$\dashint v\,d\nu=\lim_{m\to\infty}\dashint v_md\nu$.
Also $v(n)=\int x(n)\mu(dx)$ for every $n\in\Bbb N$, so
$\dashiint x(n)\mu(dx)\nu(dn)=\dashint v\,d\nu$.

\medskip

\quad\grheadc\ Set

\Centerline{$f(t)=\dashint\mu_tE_n\nu(dn)=\int\nu(a)\mu_t(da)$}

\noindent for each $t\in[0,1]$ (using (i)).
Then, for any $m\in\Bbb N$,

$$\eqalignno{\dashint v_md\nu
&=2^{-m}\sum_{k=1}^{2^m}\dashint\mu\{x:x(n)\ge 2^{-m}k\}\nu(dn)\cr
&=2^{-m}\sum_{k=1}^{2^m}\dashint\mu\{x:h_{2^{-m}k}(x)\in E_n\}\nu(dn)\cr
&=2^{-m}\sum_{k=1}^{2^m}\dashint\mu_{2^{-m}k}E_n\nu(dn)
=2^{-m}\sum_{k=1}^{2^m}f(2^{-m}k).\cr}$$

\medskip

\quad\grheadd\ Next, for $m\in\Bbb N$ and $x\in[0,1]^{\Bbb N}$, set
$q_m(x)=2^{-m}\sum_{k=1}^{2^m}\chi h_{2^{-m}k}(x)$, so that
$\sequence{m}{q_m(x)}$ is non-decreasing and
$\|x-q_m(x)\|_{\infty}\le 2^{-m}$ for each $m$, while
$q_m:[0,1]^{\Bbb N}\to[0,1]^{\Bbb N}$ is Borel measurable.   Now

$$\eqalign{\dashiint q_m(x)d\nu\,\mu(dx)
&=2^{-m}\sum_{k=1}^{2^m}\int\nu(h_{2^{-m}k}(x))\mu(dx)\cr
&=2^{-m}\sum_{k=1}^{2^m}\int\nu(a)\mu_{2^{-m}k}(da)
=2^{-m}\sum_{k=1}^{2^m}f(2^{-m}k).\cr}$$

\noindent Also $\sequence{m}{\dashint q_m(x)d\nu}\to\dashint x\,d\nu$
uniformly for
$x\in[0,1]^{\Bbb N}$, so $\idashint x\,d\nu\,\mu(dx)$ is defined and equal to

$$\eqalign{\lim_{m\to\infty}\idashint q_m(x)d\nu\,\mu(dx)
&=\lim_{m\to\infty}2^{-m}\sum_{k=1}^{2^m}f(2^{-m}k)
=\lim_{m\to\infty}\dashint v_md\nu\cr
&=\dashint v\,d\nu
=\dashiint x(n)\mu(dx)\nu(dn).\cr}$$

\noindent As $\mu$ is arbitrary, (ii) is true.

\medskip

{\bf (ii)$\Rightarrow$(iii)} Assume (ii), and let
$(X,\Sigma,\mu)$ be a probability space and $\sequencen{f_n}$ a
uniformly bounded sequence of measurable
real-valued functions on $X$.   As completing $\mu$ does not affect the
integral $\int\ldots d\mu$ (212Fb),
we may suppose that $\mu$ is complete.   Let
$\gamma>0$ be such that $|f_n(x)|\le\gamma$ for every $n\in\Bbb N$ and
$x\in X$,
and set $q(x)(n)=\Bover1{2\gamma}(\gamma+f_n(x))$ for all $n$ and $x$.
Then
$q:X\to[0,1]^{\Bbb N}$ is measurable, so there is a Radon probability
measure $\lambda$ on $[0,1]^{\Bbb N}$ such that $q$ is \imp\ for $\mu$
and $\lambda$.   \Prf\ Taking $\lambda_0E=\mu q^{-1}[E]$ for Borel sets
$E\subseteq[0,1]^{\Bbb N}$, $q$ is \imp\ for $\mu$ and $\lambda_0$;  taking
$\lambda$ to be the completion of $\lambda_0$, $q$ is \imp\ for $\mu$ and
$\lambda$, by 234Ba;  and $\lambda$ is a Radon measure by 433Cb.\ \QeD\
Now

$$\eqalignno{\dashiint f_nd\mu\,\nu(dn)
&=2\gamma\dashiint q(x)(n)\mu(dx)\nu(dn)-\gamma\cr
&=2\gamma\dashiint z(n)\lambda(dz)\nu(dn)-\gamma\cr
\displaycause{235Gc}
&=2\gamma\idashint z(n)\nu(dn)\lambda(dz)-\gamma\cr
\displaycause{by (ii)}
&=2\gamma\idashint q(x)(n)\nu(dn)\mu(dx)-\gamma
=\idashint f_n(x)\nu(dn)\mu(dx).\cr}$$

\noindent As $\mu$ and $\sequencen{f_n}$ are arbitrary, (iii) is true.

\medskip

{\bf (iii)$\Rightarrow$(iv)} is elementary, taking $f_n=\chi F_n$ and
$\mu=\nu_{\omega}$.

\medskip

{\bf (iv)$\Rightarrow$(i)} If (iv) is true and $\mu$ is a Radon
probability measure on $\Cal P\Bbb N$, there is an \imp\ function $\phi$
from $(\{0,1\}^{\Bbb N},\nu_{\omega})$ to $(\Cal P\Bbb N,\mu)$
(343Cd).    For each $n\in\Bbb N$, set $F_n=\phi^{-1}[E_n]$ for each $n$
and choose a Borel set $F'_n\subseteq\{0,1\}^{\Bbb N}$ such that
$\nu_{\omega}(F'_n\symmdiff F_n)=0$.   Then
$\idashint\chi F'_n(x)\nu(dn)\nu_{\omega}(dx)$ is defined and equal to

\Centerline{$\dashint\nu_{\omega}F'_n\nu(dn)
=\dashint\nu_{\omega}F_n\nu(dn)=\dashint\mu E_n\nu(dn)$.}

\noindent Now

$$\eqalignno{\dashint\mu E_n\nu(dn)
&=\idashint\chi F'_n(x)\nu(dn)\nu_{\omega}(dx)
=\idashint\chi F_n(x)\nu(dn)\nu_{\omega}(dx)\cr
\displaycause{because for almost every $x$, $\chi F_n(x)=\chi F'_n(x)$ for
every $n$}
&=\idashint\chi E_n(\phi(x))\nu(dn)\nu_{\omega}(dx)
=\idashint\chi E_n(a)\nu(dn)\mu(da)\cr
\displaycause{235Gc again}
&=\idashint\chi a(n)\nu(dn)\mu(da)
=\int\nu(a)\,\mu(da).\cr}$$

\noindent As $\mu$ is arbitrary, (i) is true.
}%end of proof of 538P

\leader{538Q}{Definition} I will say that a bounded
finitely additive functional
$\nu$ satisfying (i)-(iv) of 538P is a {\bf medial functional};
if, in addition, $\nu$ is non-negative, $\nu a=0$ for every finite set
$a\subseteq\Bbb N$ and $\nu\Bbb N=1$, I will call $\nu$
a {\bf medial limit}.
\cmmnt{I should remark that the term `medial limit' is
normally used for the associated linear
functional $\dashint\ldots d\nu$ on $\ell^{\infty}$, rather than
the additive functional $\nu$ on $\Cal P\Bbb N$;
thus $h\in(\ell^{\infty})^*$
is a medial limit if $h\ge 0$, $h(w)=\lim_{n\to\infty}w(n)$ for every
convergent sequence $w\in\ell^{\infty}$ and
$\int h(\sequencen{f_n(x)})\mu(dx)$ is defined and equal to
$h(\sequencen{\int f_nd\mu})$ whenever $(X,\Sigma,\mu)$ is a
probability space and $\sequencen{f_n}$ is a uniformly bounded sequence
of measurable real-valued functions on $X$.}

\cmmnt{Note that }538P(i) tells us that a medial
limit\cmmnt{ $\nu:\Cal P\Bbb N\to\Bbb R$} is universally
Radon-measurable\cmmnt{ (definition:  434Ec)}, therefore universally
measurable\cmmnt{ (434Fc)}.

\vleader{48pt}{538R}{Proposition}
Let $M\cmmnt{\mskip5mu\cong(\ell^{\infty})^*}$
be the $L$-space of bounded finitely additive
functionals on $\Cal P\Bbb N$, and $M_{\text{med}}\subseteq M$ the set of
medial functionals.

(a) $M_{\text{med}}$ is a band in $M$, and if
$T\in L^{\times}(\ell^{\infty};\ell^{\infty})$\cmmnt{ (definition:
355G)} and $T':M\to M$ is its
adjoint, then $T'\nu\in M_{\text{med}}$ for every $\nu\in M_{\text{med}}$.

(b) Taking $M_{\tau}$ to be the band of completely additive functionals
on $\Cal P\Bbb N$ and $M_{\text{m}}$ the band of measurable functionals,
as described in \S464,
$M_{\tau}\subseteq M_{\text{med}}\subseteq M_{\text{m}}$.

(c) Suppose that $\sequence{k}{\nu_k}$ is a norm-bounded sequence in
$M_{\text{med}}$, and that $\nu\in M_{\text{med}}$.   Set
$\tilde\nu(a)=\dashint\nu_k(a)\nu(dk)$ for $a\subseteq\Bbb N$.   Then
$\tilde\nu\in M_{\text{med}}$.

(d) Suppose that $\nu\in M$ is a medial limit, and set
$\Cal F=\{a:a\subseteq\Bbb N$, $\nu(a)=1\}$.   Then $\Cal F$ is a
measure-converging filter with the Fatou property.
%any hope of `rapid'?  did Matt know?

(e) Let $(X,\Sigma,\mu)$ and $(Y,\Tau,\lambda)$ be probability spaces, and
$T\in L^{\times}(L^{\infty}(\mu);L^{\infty}(\lambda))$.
Let $\sequencen{f_n}$, $\sequencen{g_n}$ be sequences in
$\eusm L^{\infty}(\mu)$, $\eusm L^{\infty}(\nu)$ respectively such that
$Tf_n^{\ssbullet}=g_n^{\ssbullet}$ for every $n$ and
$\sequencen{f_n^{\ssbullet}}$ is norm-bounded in $L^{\infty}(\mu)$.
Let $\nu\in M$ be a medial functional.   Then
$f(x)=\dashint f_n(x)\nu(dn)$ and
$g(y)=\dashint g_n(y)\nu(dn)$ are defined for almost every $x\in X$ and
$y\in Y$;  moreover,
$f\in\eusm L^{\infty}(\mu)$, $g\in\eusm L^{\infty}(\lambda)$ and
$Tf^{\ssbullet}=g^{\ssbullet}$.

\proof{{\bf (a)(i)}
Any of the four conditions of 538P makes it clear that
$M_{\text{med}}$ is a linear subspace of $M$.

We see also that $M_{\text{med}}$ is norm-closed in $M$.   \Prf\ Let
$\sequencen{\nu_n}$ be a sequence in $M_{\text{med}}$ which is
norm-convergent to $\nu\in M$.   If $\mu$ is a Radon probability measure on
$[0,1]^{\Bbb N}$, then
$\sequencen{\dashint x\,d\nu_n}\to\dashint x\,d\nu$ uniformly for
$x\in[0,1]^{\Bbb N}$, so

$$\eqalign{\idashint x\,d\nu\,\mu(dx)
&=\lim_{n\to\infty}\idashint x\,d\nu_n\mu(dx)\cr
&=\lim_{n\to\infty}\dashiint x(i)\mu(dx)\nu_n(di)
=\dashiint x(i)\,\mu(dx)\nu(di).\cr}$$

\noindent As $\mu$ is arbitrary, $\nu\in M_{\text{med}}$.\ \Qed

\medskip

\quad{\bf (ii)} Before completing the proof that $M_{\text{med}}$ is a
band, I deal with the second clause of (a).

\medskip

\qquad\grheada\ Recall from \S355 that
$L^{\times}(\ell^{\infty};\ell^{\infty})$ is the set of differences of
order-continuous positive linear operators from $\ell^{\infty}$ to itself.
Since $M$ can be identified with $(\ell^{\infty})^*$, any
$T\in L^{\times}(\ell^{\infty};\ell^{\infty})$ has an adjoint
$T':M\to M$ defined by saying that
$(T'\nu)(a)=\dashint T(\chi a)d\nu$ for every $a\subseteq\Bbb N$.
Since $x\mapsto\dashint Tx\,d\nu$ and $x\mapsto\dashint x\,d(T'\nu)$
both belong to $(\ell^{\infty})^*$ and agree on
$\{\chi a:a\subseteq\Bbb N\}$, they are equal, that is,
$\dashint Tx\,d\nu=\dashint x\,d(T'\nu)$ for every $x\in\ell^{\infty}$.

\medskip

\qquad\grheadb\
If $T:\ell^{\infty}\to\ell^{\infty}$ is an order-continuous positive linear
operator, it is a norm-bounded linear operator (355C), and all the functionals
$x\mapsto(Tx)(n)$ are order-continuous, therefore represented by members of
$\ell^1$;  that is, we have a family
$\langle\alpha_{ni}\rangle_{n,i\in\Bbb N}$ in $\coint{0,\infty}$ such that

\Centerline{$(Tx)(n)
=\sum_{i=0}^{\infty}\alpha_{ni}x(i)$ whenever $x\in\ell^{\infty}$
and $n\in\Bbb N$,}

\Centerline{$\sup_{n\in\Bbb N}\sum_{i=0}^{\infty}\alpha_{ni}
=\|T\|$ is finite.}

\noindent In this case, if $\nu\in M$ and $\nu'=T'\nu$ in $M$,

\Centerline{$\dashint x\,d\nu'
=\dashint(Tx)(n)\nu(dn)
=\dashint\sum_{i=0}^{\infty}\alpha_{ni}x(i)\nu(dn)$}

\noindent for every $x\in\ell^{\infty}$.

Now suppose that that $\|T\|\le 1$, so that
$\sum_{i=0}^{\infty}\alpha_{ni}\le 1$ for every $n$.   Consider the
function $\phi=T\restr[0,1]^{\Bbb N}$.   This is a function from
$[0,1]^{\Bbb N}$ to itself, and it is continuous for the product topology
on $\Bbb N$.

Take any $\nu\in M$ and Radon probability measure $\mu$ on
$[0,1]^{\Bbb N}$;  then the image measure $\mu_1=\mu\phi^{-1}$ on
$[0,1]^{\Bbb N}$ is a Radon probability measure (418I), and
$\int f(\phi(x))\mu(dx)=\int f(x)\mu_1(dx)$ for any $\mu_1$-integrable
function $f$.   In particular, setting $f(x)=\dashint x\,d\nu$,

\Centerline{$\idashint\phi(x)\,d\nu\,\mu(dx)
=\idashint x\,d\nu\,\mu_1(dx)=\dashiint x(n)\mu_1(dx)\nu(dn)$}

\noindent because $\nu\in M_{\text{med}}$.

Set $\nu'=T'\nu$.   Then we can calculate

$$\eqalignno{\dashiint x(n)\mu(dx)\nu'(dn)
&=\dashint\sum_{i=0}^{\infty}\alpha_{ni}\int x(i)\mu(dx)\nu(dn)
=\dashiint\sum_{i=0}^{\infty}\alpha_{ni}x(i)\mu(dx)\nu(dn)\cr
\displaycause{the inner integral is with respect to a genuine
$\sigma$-additive measure, so we have B.Levi's theorem}
&=\dashiint\phi(x)(n)\mu(dx)\nu(dn)
=\dashiint x(n)\mu_1(dx)\nu(dn)\cr
&=\idashint\phi(x)\,d\nu\,\mu(dx)
=\idashint Tx\,d\nu\,\mu(dx)
=\idashint x\,d\nu'\,\mu(dx).\cr}$$

\noindent As $\mu$ is arbitrary, $\nu'$ satisfies 538P(ii), and is a medial
functional.

\medskip

\qquad\grheadc\ Thus $T'\nu\in M_{\text{med}}$ whenever
$\nu\in M_{\text{med}}$ and $T:\ell^{\infty}\to\ell^{\infty}$ is positive,
order-continuous and of norm at most $1$.   As $M_{\text{med}}$ is a linear
subspace of $M$, the same is true for every positive order-continuous $T$
and for differences of such operators, that is, for every
$T\in L^{\times}(\ell^{\infty};\ell^{\infty})$, as claimed.

\medskip

\quad{\bf (iii)} I now return to the question of showing that
$M_{\text{med}}$ is a band.   The point is that if
$\nu$ is a medial functional and $b\subseteq\Bbb N$, then $\nu_b$ is a
medial functional, where $\nu_b(a)=\nu(a\cap b)$ for every
$a\subseteq\Bbb N$.   \Prf\ Define $T:\ell^{\infty}\to\ell^{\infty}$ by
setting $Tx=x\times\chi b$ for $x\in\ell^{\infty}$.   Then $T$ is a
positive order-continuous operator, and $T'\nu\in M_{\text{med}}$, by
(iii) above.   But

\Centerline{$(T'\nu)(a)=\dashint T(\chi a)d\nu=\dashint\chi(a\cap b)d\nu
=\nu(a\cap b)=\nu_b(a)$}

\noindent for every $a\subseteq\Bbb N$, so $\nu_b=T'\nu$ is a medial
functional.\ \Qed

By 436M, this is enough to ensure that $M_{\text{med}}$ is a band in $M$.

\medskip

{\bf (b)(i)} Recall that an additive functional on $\Cal P\Bbb N$
is completely additive iff it corresponds to an element of $\ell^1$,
that is,
belongs to the band generated by the elementary functionals $\delta_k$
where $\delta_k(a)=\chi a(k)$ for $k\in\Bbb N$ and $a\subseteq\Bbb N$.
To see that $\delta_k$ belongs to $M_{\text{med}}$, all we have to do is
to note that $\delta_k=\chi E_k$ where $E_k$ is defined as in 538P;
so if $\mu$ is a  Radon probability measure on $\Cal P\Bbb N$, we shall
have

\Centerline{$\int\delta_kd\mu=\mu E_k=\dashint\mu E_n\delta_k(dn)$.}

\noindent Since $M_{\text{med}}$ is a band, it must include $M_{\tau}$.

\medskip

\quad{\bf (ii)} On the other side, 538P(i) tells us that every member of
$M_{\text{med}}$ is universally measurable, and therefore belongs to
$M_{\text{m}}$, which is just the set of bounded additive
functionals which are $\Sigma$-measurable, where $\Sigma$ is the domain of
the usual measure on $\Cal P\Bbb N$.

\medskip

{\bf (c)(i)} Because $\sequence{k}{\nu_k}$ is norm-bounded, $\tilde\nu$ is
well-defined and additive;  also it is bounded.
\Prf\ If $\gamma$ is such that
$\|\nu\|\le\gamma$ and $\|\nu_k\|\le\gamma$ for every $k$, then

\Centerline{$|\tilde\nu(a)|
\le\gamma\sup_{k\in\Bbb N}|\nu_k(a)|\le\gamma^2$}

\noindent for every $a\subseteq\Bbb N$.\ \Qed

Note that

\Centerline{$\dashint\chi a\,d\tilde\nu=\tilde\nu(a)
=\dashint\nu_k(a)\nu(dk)=\dashidashint\chi a\,d\nu_k\nu(dk)$}

\noindent for every $a\subseteq\Bbb N$, so that

\Centerline{$\dashint x\,d\tilde\nu
=\dashint x(n)\,\tilde\nu(dn)
=\dashidashint x(n)\,\nu_k(dn)\nu(dk)
=\dashidashint x\,d\nu_k\nu(dk)$}

\noindent whenever $x\in\ell^{\infty}$ is a linear combination of
indicator functions, and therefore for every $x\in\ell^{\infty}$.

\medskip

\quad{\bf (ii)}
Now suppose that $(X,\Sigma,\mu)$ is a probability space and that
$\sequencen{f_n}$ is a uniformly bounded sequence of measurable real-valued
functions on $X$.   Let $(X,\hat\Sigma,\hat\mu)$ be the completion of
$(X,\Sigma,\mu)$.   For $k\in\Bbb N$ and $x\in X$ set
$g_k(x)=\dashint f_n(x)\nu_k(dn)$;  because $\nu_k$ is a medial functional,
we know that
$\int g_kd\mu=\dashiint f_n(x)\mu(dx)\nu_k(dn)$ is defined, so $g_k$ is
$\hat\Sigma$-measurable.   Consequently $\idashint g_k(x)\nu(dk)\hat\mu(dx)$
is defined and equal to $\dashiint g_k(x)\hat\mu(dx)\nu(dk)$.   It follows that

$$\eqalign{\dashiint f_n(x)\mu(dx)\tilde\nu(dn)
&=\dashidashiint f_n(x)\mu(dx)\nu_k(dn)\nu(dk)\cr
&=\dashiidashint f_n(x)\nu_k(dn)\mu(dx)\nu(dk)
=\dashiint g_k(x)\hat\mu(dx)\nu(dk)\cr
&=\idashint g_k(x)\nu(dk)\hat\mu(dx)
=\idashidashint f_n(x)\nu_k(dn)\nu(dk)\hat\mu(dx)\cr
&=\idashint f_n(x)\tilde\nu(dn)\hat\mu(dx)
=\idashint f_n(x)\tilde\nu(dn)\mu(dx).\cr}$$

\noindent (Recall that $\mu$ and $\hat\mu$ give rise to the same integrals,
by 212Fb again.)
As $(X,\Sigma,\mu)$ and $\sequencen{f_n}$ are arbitrary,
$\tilde\nu\in M_{\text{med}}$.

\medskip

{\bf (d)} Of course $\Cal F=\{\Bbb N\setminus a:\nu(a)=0\}$ is a filter.

\medskip

\quad{\bf (i)}
If $(X,\Sigma,\mu)$ is a probability space, $\sequencen{E_n}$ is a sequence
in $\Sigma$, and $\lim_{n\to\infty}\mu E_n=1$, then

\Centerline{$\idashint\chi E_n(x)\nu(dn)\mu(dx)
=\dashiint\chi E_nd\mu\,\nu(dn)=\dashint\mu E_n\nu(dn)=1$.}

\noindent So $E=\{x:\dashint\chi E_n(x)\nu(dn)=1\}$ is $\mu$-conegligible.
But if $x\in E$ and $a=\{n:x\in E_n\}$, then
$\nu a=\dashint\chi E_n(x)\nu(dn)=1$ and $a\in\Cal F$ and
$x\in\bigcap_{n\in a}E_n$.   Thus
$\bigcup_{a\in\Cal F}\bigcap_{n\in a}E_n\supseteq E$ is conegligible.
As $(X,\Sigma,\mu)$ and $\sequencen{E_n}$ are arbitrary,
$\Cal F$ is measure-converging.

\medskip

\quad{\bf (ii)} If $(X,\Sigma,\mu)$ is a probability space,
$\sequencen{E_n}$ is a sequence in $\Sigma$, and
$X=\bigcup_{A\in\Cal F}\bigcap_{n\in A}E_n$, then $\{n:x\in E_n\}\in\Cal F$
for every $x\in X$, and

\Centerline{$\dashint\mu E_n\nu(dn)
=\idashint\chi E_n(x)\nu(dn)\mu(dx)
=\int\nu\{n:x\in E_n\}\mu(dx)=1$.}

\noindent So for any $\epsilon>0$, $\nu\{n:\mu E_n\le 1-\epsilon\}=0$
and $\{n:\mu E_n\ge 1-\epsilon\}\in\Cal F$;  accordingly
$\lim_{n\to\Cal F}\mu E_n=1$.   As $(X,\Sigma,\mu)$ and $\sequencen{E_n}$
are arbitrary, $\Cal F$ has the Fatou property.

\medskip

{\bf (e)(i)}
For each $n\in\Bbb N$, we can find a $\Sigma$-measurable function
$f'_n:X\to\Bbb R$, equal almost everywhere to $f_n$, and such that
$\sup_{x\in X}|f'_n(x)|=\esssup|f_n|$.   Now $\sequencen{f'_n}$ is
uniformly bounded, so $f'(x)=\dashint f'_n(x)\nu(dn)$ is defined for every
$x\in X$;  and $f(x)$ is defined and equal to $f'(x)$ for
$\mu$-almost every $x$.   Since $f'$ is integrable, $f'$ and $f$ are
$\mu$-virtually measurable and essentially bounded, and
$f\in\eusm L^{\infty}(\mu)$.
Similarly, $g\in\eusm L^{\infty}(\lambda)$.

\medskip

\quad{\bf (ii)} If $h\in\eusm L^1(\mu)$, then
$\int f\times h\,d\mu=\dashiint f_n\times h\,d\mu\,\nu(dn)$.   \Prf\ ($\alpha$)
If $h$ is defined everywhere, measurable and bounded, then, taking $f'_n$
and $f'$ as in (i),
$(f'\times h)(x)=\dashint f'_n(x)h(x)\nu(dn)$ for every $x\in X$, so

$$\eqalign{\int f\times h\,d\mu
&=\int f'\times h\,d\mu
=\idashint(f'_n\times h)(x)\nu(dn)\mu(dx)\cr
&=\dashiint f'_n\times h\,d\mu\,\nu(dn)
=\dashiint f_n\times h\,d\mu\,\nu(dn).\cr}$$

\noindent($\beta$) In general, set $\gamma=\sup_{n\in\Bbb N}\esssup f_n$.
Given $\epsilon>0$, there is a simple
function $h'$ such that $\|h-h'\|_1\le\epsilon$, and now

$$\eqalign{|\int f\times h\,d\mu&-\dashiint f_n\times h\,d\mu\,\nu(dn)|\cr
&\le|\int f\times h\,d\mu-\int f\times h'\,d\mu|
+|\int f\times h'\,d\mu-\dashiint f_n\times h'\,d\mu\,\nu(dn)|\cr
&\mskip200mu
+|\dashiint f_n\times h'\,d\mu\,\nu(dn)-\dashiint f_n\times h\,d\mu\,\nu(dn)|\cr
&\le\|f\|_{\infty}\|h-h'\|_1
+\sup_{n\in\Bbb N}|\int f_n\times h'\,d\mu-\int f_n\times h\,d\mu|
\le 2\epsilon\gamma.\cr}$$

\noindent As $\epsilon$ is arbitrary,
$\int f\times h\,d\mu=\dashiint f_n\times h\,d\mu\,\nu(dn)$.\ \Qed

Similarly, $\int g\times h\,d\lambda=\dashiint g_n\times h\,d\lambda\,\nu(dn)$
for every $\lambda$-integrable $h$.

\medskip

\quad{\bf (iii)} If $h\in\eusm L^1(\lambda)$ there is an
$\tilde h\in\eusm L^1(\mu)$ such that
$\int\tilde h^{\ssbullet}\times v=\int h^{\ssbullet}\times Tv$ for every
$v\in L^{\infty}(\mu)$.   \Prf\ Recall that $L^1(\mu)$, $L^1(\lambda)$ can
be identified with $L^{\infty}(\mu)^{\times}$ and
$L^{\infty}(\nu)^{\times}$ (365Lb\formerly{3{}65Mb});  
perhaps I should remark that the
formulae $\int\tilde h^{\ssbullet}\times v$,
$\int h^{\ssbullet}\times Tv$ represent abstract integrals taken in
$L^1(\mu)$, $L^1(\lambda)$ respectively (242B).   Setting
$\phi(w)=\int h^{\ssbullet}\times w$ for $w\in L^{\infty}(\lambda)$,
$\phi\in L^{\infty}(\lambda)^{\times}$, so
$\phi T\in L^{\infty}(\mu)^{\times}$ (355G) and there is an
$\tilde h\in\eusm L^1(\mu)$ such that

\Centerline{$\int\tilde h^{\ssbullet}\times v=\phi(Tv)
=\int h^{\ssbullet}\times Tv$}

\noindent for every $v\in L^{\infty}(\mu)$.\ \Qed

\medskip

\quad{\bf (iv)} Take $h$ and $\tilde h$ as in (iii), and consider

$$\eqalignno{\int h^{\ssbullet}\times g^{\ssbullet}
&=\int h\times g\,d\lambda
=\dashiint h\times g_n\,d\lambda\,\nu(dn)\cr
\displaycause{by (ii)}
&=\dashint(\int h^{\ssbullet}\times g_n^{\ssbullet})\nu(dn)
=\dashint(\int h^{\ssbullet}\times Tf_n^{\ssbullet})\nu(dn)\cr
&=\dashint(\int\tilde h^{\ssbullet}\times f_n^{\ssbullet})\nu(dn)
=\dashiint\tilde h\times f_nd\mu\,\nu(dn)
=\int\tilde h\times fd\mu\cr
\displaycause{by (ii) again}
&=\int\tilde h^{\ssbullet}\times f^{\ssbullet}
=\int h^{\ssbullet}\times Tf^{\ssbullet}.\cr}$$

\noindent As $h$ is arbitrary, and the duality between $L^{\infty}(\mu)$
and $L^1(\lambda)$ is separating, $Tf^{\ssbullet}=g^{\ssbullet}$, as
required.
}%end of proof of 538R

\leader{538S}{Theorem} (a) If
$\frakmctbl=\frak c$, there is a medial limit.
%Foreman thinks enough if $\frakmctbl=\frak d$, email 5.9.07

(b)\dvAnew{2011}\cmmnt{ ({\smc Larson 09})} Suppose that the filter
dichotomy\cmmnt{ (5A6Id)} is true.   If $I$ is any set and $\nu$ is a
finitely additive real-valued functional on $\Cal PI$ which is universally
measurable for the usual topology on $\Cal PI$, then $\nu$ is completely
additive.\cmmnt{\footnote{The result developed into this form in the
course of
correspondence with J.Pachl.}} %6.11
Consequently there is no medial limit.
%universally Radon-measurable will do I suppose, but demands a little
%more from Chap 43

\proof{{\bf (a)(i)}
Let $M$ be the $L$-space of bounded additive functionals
on $\Cal P\Bbb N$.   Let us say that a subset $C$ of
$M$ is {\bf rationally convex} if $\alpha\nu+(1-\alpha)\nu'\in C$
whenever $\nu$, $\nu'\in C$ and $\alpha\in[0,1]\cap\Bbb Q$;  for
$A\subseteq M$, write $\Gamma_{\Bbb Q}(A)$ for the smallest
rationally convex set including $A$.   Set
$Q=\Gamma_{\Bbb Q}(\{\delta_n:n\in\Bbb N\})$ where
$\delta_n(a)=\chi a(n)$ for $a\subseteq\Bbb N$ and $n\in\Bbb N$.
In the language of 538Rb, $Q\subseteq M_{\tau}\subseteq M_{\text{med}}$,
so 538P(i) tells us that $\int\nu\,d\mu=\dashint\mu E_n\nu(dn)$ for every
$\nu\in Q$, where $E_n=\{a:n\in a\subseteq\Bbb N\}$ as usual.

\medskip

\quad{\bf (ii)}
Suppose that $\Cal F$ is a filter base on $Q$, consisting of rationally
convex sets, with cardinal less than $\frakmctbl$.   Let
$\mu$ be a Radon probability measure on $\Cal P\Bbb N$.
Then there is a sequence
$\sequence{k}{\nu_k}$ in $Q$ such that

\inset{$\sum_{k=0}^{\infty}\int|\nu_{k+1}(a)-\nu_k(a)|\mu(da)<\infty$,

$\{k:k\in\Bbb N$, $\nu_k\in F\}$ is infinite for every $F\in\Cal F$.}

\noindent\Prf\ Each $\nu\in Q$ is a
bounded Borel measurable real-valued function
on $\Cal P\Bbb N$;  let $u\in L^2=L^2(\mu)$ be a
$\frak T_s(L^2,L^2)$-cluster point
of $\family{\nu}{Q}{\nu^{\ssbullet}}$
along the filter generated by $\Cal F$.   For
any $F\in\Cal F$, the $\|\,\|_2$-closure of the
rationally convex set $\{\nu^{\ssbullet}:\nu\in F\}\subseteq L^2$
is convex, so
includes the weak closure of $\{\nu^{\ssbullet}:\nu\in F\}$ and
therefore contains $u$.   So $\{\nu^{\ssbullet}:\nu\in F\}$ meets
$\{v:v\in L^2$, $\|v-u\|_2\le\epsilon\}$ for every $\epsilon>0$.

Set $H_k=\{\nu:\nu\in Q$, $\|\nu^{\ssbullet}-u\|_2\le 2^{-k}\}$ for
each $k\in\Bbb N$;  then every $H_k$ meets every member of $\Cal F$.
If we give each $H_k$ its discrete topology, and take $H$ to be the product
$\prod_{k\in\Bbb N}H_k$, then $H$ is homeomorphic to $\NN$.   Writing
$\Cal M(H)$ for the ideal of meager subsets of $H$,
$\cov\Cal M(H)=\frakmctbl>\#(\Cal F)$, while

\Centerline{$\bigcup_{k\ge n}\{\alpha:\alpha\in H$, $\alpha(k)\in F\}$}

\noindent is a dense open subset of $H$ for every $F\in\Cal F$
and $n\in\Bbb N$.   There is
therefore an $\alpha\in H$ such that $\{k:\alpha(k)\in F\}$ is infinite for
every $F\in\Cal F$;  take $\nu_k=\alpha(k)$ for each $k$.   Since
$\mu$ is a probability measure,

$$\eqalignno{\int|\nu_{k+1}-\nu_k|d\mu
&\le\|\nu_{k+1}^{\ssbullet}-\nu_k^{\ssbullet}\|_2\cr
\displaycause{244E;  see 244Xd}
&\le 2^{-k-1}+2^{-k}\cr}$$

\noindent for every $k$, and
$\sum_{k=0}^{\infty}\int|\nu_{k+1}-\nu_k|d\mu$
is finite.\ \Qed

\medskip

\quad{\bf (iii)} Because a Radon
probability measure on $\Cal P\Bbb N$ is defined by its
values on the countable algebra $\frak B$ of open-and-closed sets, the
number of such measures is at most $\#(\BbbR^{\frak B})=\frak c$.
Enumerate them as $\ofamily{\xi}{\frakc}{\mu_{\xi}}$.   Choose a
non-decreasing family
$\langle\Cal F_{\xi}\rangle_{\xi\le\frakc}$ of filter bases on $Q$, as
follows.   The inductive hypothesis will be that $\Cal F_{\xi}$ has
cardinal at most $\max(\omega,\#(\xi))$ and consists of rationally convex
sets.   Start with $\Cal F_0=\{F_n:n\in\Bbb N\}$ where
$F_n=\Gamma_{\Bbb Q}(\{\delta_i:i\ge n\})$ for each $n$.
Given $\Cal F_{\xi}$ where $\xi<\frak c$, apply (ii) with
$\mu=\mu_{\xi}$ to see that there is
a sequence $\sequence{k}{\nu_{\xi k}}$ in $Q$ such that

\inset{$\sum_{k\in\Bbb N}\int|\nu_{\xi,k+1}-\nu_{\xi k}|d\mu_{\xi}<\infty$,

$\{k:\nu_{\xi k}\in F\}$ is infinite for every $F\in\Cal F_{\xi}$.}

\noindent Let $\Cal F_{\xi+1}$ be

\Centerline{$\Cal F_{\xi}
\cup\{F\cap\Gamma_{\Bbb Q}(\{\nu_{\xi k}:k\ge l\}):
F\in\Cal F_{\xi}$, $l\in\Bbb N\}$.}

\noindent For non-zero limit ordinals $\xi\le\frak c$, set
$\Cal F_{\xi}=\bigcup_{\eta<\xi}\Cal F_{\eta}$.

\medskip

\quad{\bf (iv)} At the end of the induction, let $\Cal F$ be the filter on
$M\cong(\ell^{\infty})^*$
generated by $\Cal F_{\frakc}$, and let $\theta$ be a cluster point of
$\Cal F$ for the
weak* topology of $(\ell^{\infty})^*$.   Then $\theta$ is a medial limit.
\Prf\ If $\mu$ is a Radon probability measure on $\Cal P\Bbb N$,
take $\xi<\frak c$ such that $\mu=\mu_{\xi}$.   Because
$\Gamma_{\Bbb Q}(\{\nu_{\xi k}:k\ge l\})$ belongs to $\Cal F$ for every
$l\in\Bbb N$,
$\dashint u(n)\theta(dn)=\lim_{k\to\infty}\dashint u(n)\nu_{\xi k}(dn)$ for every
$u\in\ell^{\infty}$ for which the limit is defined.   In particular,
$\theta(a)=\lim_{k\to\infty}\nu_{\xi k}(a)$ whenever $a\subseteq\Bbb N$ is
such that the limit is defined.   Because
$\sum_{k\in\Bbb N}\int|\nu_{\xi,k+1}-\nu_{\xi k}|d\mu$ is finite,
this is the case for $\mu$-almost every $a$, so

\Centerline{$\int\theta(a)\mu(da)
=\lim_{k\to\infty}\int\nu_{\xi k}(a)\mu(da)
=\lim_{k\to\infty}\dashint\mu E_n\nu_{\xi k}(dn)$;}

\noindent and because the latter limit is defined it is equal to
$\dashint\mu E_n\theta(dn)$.   As $\mu$ is arbitrary, $\theta$ satisfies
condition (i) of 538P, and is a medial functional;  because $Q\in\Cal F$,
$\theta\Bbb N=1$;  and because $\Cal F_0\subseteq\Cal F$, $\theta(a)=0$ for
every finite $a\subseteq\Bbb N$, so $\theta$ is a medial limit.\ \Qed

\medskip

{\bf (b)(i)} The key is the following.   Suppose that
$\nu:\Cal PI\to\Bbb R$ is a universally measurable additive
functional.

\medskip

\qquad\grheada\ For every set $J$ and function
$\phi:I\to J$, $\nu\phi^{-1}$ is
universally measurable, where $(\nu\phi^{-1})(b)=\nu(\phi^{-1}[b])$ for
every $b\subseteq J$.   \Prf\ We have only to observe that
$b\mapsto\phi^{-1}[b]:\Cal PJ\to\Cal PI$ is continuous, and
apply 434Df.\ \Qed

\medskip

\qquad\grheadb\ $\nu$ is bounded.   \Prf\Quer\
Otherwise, there is a disjoint sequence
$\sequence{k}{c_k}$ of subsets of $I$ such that
$\lim_{k\to\infty}|\nu c_k|=\infty$ (326D(ii)).   Enlarging $c_0$ if
necessary, we can suppose that $\bigcup_{k\in\Bbb N}c_k=I$.
Set $\phi(i)=k$ for $k\in\Bbb N$ and $i\in c_k$.   Then
$\nu\phi^{-1}[\{k\}]\to\infty$ as $k\to\infty$.   But $\nu'=\nu\phi^{-1}$
is universally measurable, therefore
$\Tau_{\Bbb N}$-measurable, where $\Tau_{\Bbb N}$ is the
domain of the usual measure $\lambda_{\Bbb N}$ on $\Cal P\Bbb N$.
Let $M$ be such
that $\lambda_{\Bbb N}E>0$ where $E=\{a:|\nu'a|\le M\}$.   Then there are
an $n\in\Bbb N$ such that for every $k\ge n$ there are $a$, $b\in E$ such
that $a\symmdiff b=\{k\}$ (345E;  recall that the natural
bijection $a\to\chi a:\Cal P\Bbb N\to\{0,1\}^{\Bbb N}$ identifies
$\lambda_{\Bbb N}$ with the usual measure on $\{0,1\}^{\Bbb N}$).
In this case, $k$ belongs to
exactly one of $a$, $b$;  suppose that $k\in a\setminus b$;  then
$|\nu'\{k\}|=|\nu a-\nu'b|\le 2M$.   This is supposed to be true for
every $k\ge n$, so $\limsup_{k\to\infty}|\nu'\{k\}|\le 2M$.\ \Bang\Qed

\medskip

\qquad\grheadc\ $|\nu|$ is universally measurable.
\Prf\ As in part (b-i) of the proof of 464K, there is a sequence
$\sequencen{c_n}$ in $\Cal PI$ such that
$\nu^+a=\lim_{n\to\infty}\nu(a\cap c_n)$ for every $a\subseteq I$.
Since all the functions $a\mapsto a\cap c_n$ are continuous,
$a\mapsto\nu(a\cap c_n)$ is universally measurable for every $n$, and
$\nu^+$ is universally measurable (use 418C).
Consequently $|\nu|=2\nu^+-\nu$ is universally measurable.\ \Qed

\medskip

\quad{\bf (ii)} If $\nu:\Cal P\Bbb N\to\coint{0,\infty}$
is a universally measurable additive functional and $\nu\{n\}=0$ for every
$n\in\Bbb N$, then $\nu=0$.   \Prf\Quer\ Otherwise, consider
$\Cal F=\{a:\nu a=\nu\Bbb N\}$.   This is a filter on $\Bbb N$
containing every cofinite set.   Let $\phi:\Bbb N\to\Bbb N$ be
finite-to-one, and write $\nu'$ for $\nu\phi^{-1}$.
Setting $\Cal I=\{a:\nu'a=0\}$, we have
a strictly positive additive functional on the quotient algebra
$\Cal P\Bbb N/\Cal I$, so $\Cal P\Bbb N/\Cal I$ is ccc and $\Cal I$ cannot
be $[\Bbb N]^{<\omega}$, that is, $\phi[[\Cal F]]$ is not
the Fr\'echet filter.   On the other hand,
$\nu'$ is universally measurable, by (i-$\alpha$), so

\Centerline{$\phi[[\Cal F]]=\{a:\phi^{-1}[a]\in\Cal F\}
=\{a:\nu'a=\nu'\Bbb N\}$}

\noindent is a universally measurable subset of $\Cal P\Bbb N$, and
cannot be an ultrafilter (464Ca).   Thus $\Cal F$ witnesses that the filter
dichotomy is false.\ \Bang\Qed

\medskip

\quad{\bf (iii)} Returning to the general case of a universally measurable
additive functional $\nu:\Cal PI\to\Bbb R$, set $\gamma_i=\nu\{i\}$ for
$i\in I$.   By (i-$\beta$),
$\sup_{J\in[I]^{<\omega}}|\sum_{j\in J}\gamma_j|
=\sup_{J\in[I]^{<\omega}}|\nu J|$ is finite, so
$\sum_{i\in I}|\gamma_i|<\infty$, and we have a functional
$\nu_1:\Cal PI\to\Bbb R$ defined by setting $\nu_1a=\sum_{i\in a}\gamma_i$
for every $a\subseteq I$.   $\nu_1$ is continuous for the topology of
$\Cal PI$, so $\nu_2=\nu-\nu_1$ is universally measurable, and
$\nu'=|\nu_2|$ is universally measurable, by (i-$\gamma$).

$\nu'J=0$ for every countable set $J\subseteq I$.   \Prf\ If $J$ is finite,
this is trivial, because

\Centerline{$|\nu_2|\{i\}
=|\nu_2\{i\}|=|\nu\{i\}-\nu_1\{i\}|=|\gamma_i-\gamma_i|=0$}

\noindent for every $i\in I$.   If $J$ is countably infinite, then
the embedding $\Cal PJ\embedsinto\Cal PI$ is continuous, so
$\nu'\restr\Cal PJ$ is universally measurable for the usual topology on
$\Cal PJ$;  also it is still zero on singletons, so (ii) tells us that it
is zero on the whole of $\Cal PJ$.\ \Qed

It follows that
$\nu'$ is zero everywhere.   \Prf\ Take $c\subseteq I$ and $\epsilon>0$.
$\nu'$ must be $\Tau_I$-measurable, where $\Tau_I$ is the domain of
the usual measure $\lambda_I$ on $\Cal PI$.   Since $\lambda_I$ is a
completion regular Radon measure (416Ub), there must be a non-negligible
zero set $K\subseteq\Cal PI$ such that $|\nu'a-\nu'b|\le\epsilon$ for all
$a$, $b\in K$;  and there is a countable set $J\subseteq I$ such that $K$
is determined by coordinates in $J$ (4A3Nc, applied to
$\{0,1\}^I\cong\Cal PI$).   Take any $a\in K$.   Then
$c_1=(c\setminus J)\cup(a\cap J)$ and $a\cap J$ both belong to $K$.
But as $\nu'(c\cap J)=0$,

\Centerline{$|\nu'c|=|\nu'c_1-\nu'(a\cap J)|\le\epsilon$.}

\noindent As $c$ and $\epsilon$ are arbitrary, $\nu'=0$.\ \Qed

Accordingly $\nu_2=0$ and $\nu=\nu_1$.   But of course $\nu_1$ is
completely additive.

\medskip

\quad{\bf (iv)} Finally, a medial limit would be a non-zero additive
functional from $\Cal P\Bbb N$ to $[0,1]$
which was universally measurable, as noted in
538Q, and zero on singletons;  and this has already been ruled out by (ii).
}%end of proof of 538S

\cmmnt{\medskip

\noindent{\bf Remark} It is possible to have medial limits when
$\frakmctbl\ll\frak c$;  see 553N.}

\exercises{\leader{538X}{Basic exercises (a)}
%\spheader 538Xa
Let $\Cal F$ be a filter on $\Bbb N$, and $I$ an infinite
subset of $\Bbb N$ such that $\Bbb N\setminus I\notin\Cal F$;  write
$\Cal F\lceil I$ for the filter $\{A\cap I:A\in\Cal F\}$.
Show that if
$\Cal F$ is free, or a $p$-point filter, or Ramsey, or rapid, or nowhere
dense, or measure-centering,
or measure-converging, or with the Fatou property, then so is
$\Cal F\lceil I$.
%538A

\spheader 538Xb
For $A\in[\Bbb N]^{\omega}$ let $f_A:\Bbb N\to A$ be the
increasing enumeration of $A$.   Let $\Cal F$ be a free filter on $\Bbb N$.
Show that $\Cal F$ is rapid iff $\{f_A:A\in\Cal F\}$ is cofinal with $\NN$.
%538A

\spheader 538Xc\dvAnew{2009}
Let $\Cal F$ be a filter which is universally
measurable (regarded as a subset
of $\Cal P(\bigcup\Cal F)$ with its usual topology), and
$\Cal G$ another filter such that $\Cal G\leRK\Cal F$.   Show that $\Cal G$
is universally measurable.
%538B Also works for universally Radon-measurable.

\spheader 538Xd
Let $\CalFr$ be the Fr\'echet filter and $\Cal F_d$ the
asymptotic density filter, the filter
of subsets of $\Bbb N$ with asymptotic density $1$.   (i) Show that
$\CalFr$ and $\Cal F_d$ are $p$-point filters.   (ii) Show that
$\CalFr\leRB\Cal F_d$ but that $\CalFr\ltimes\CalFr\not\leRK\Cal F_d$.
%538D

\spheader 538Xe(i) Let $\sequencen{\Cal F_n}$ be a sequence of
filters on $\Bbb N$, and $\Cal F$ a filter on $\Bbb N$.   Write
$\lim_{n\to\Cal F}\Cal F_n$ for the filter
$\{A:A\subseteq\Bbb N$, $\{n:n\in\Bbb N$, $A\in\Cal F_n\}\in\Cal F\}$.
Show that if every $\Cal F_n$ is rapid, then
$\lim_{n\to\Cal F}\Cal F_n$ is rapid.
(ii) Let $\Cal F$ be a rapid filter, and $\Cal G$ any filter on $\Bbb N$.
Show that $\Cal G\ltimes\Cal F$ is
rapid.   (iii) In 538E, suppose that $\Cal F_1$ is rapid.   Show that
$\Cal G_{\xi}$ is rapid for every $\xi\ge 1$.
%538E

\spheader 538Xf(i) Let $\Cal F$ be a nowhere dense filter, and
$\Cal G$ a filter on $\Bbb N$ such that $\Cal G\leRK\Cal F$.   Show that
$\Cal G$ is nowhere dense.
(ii)\dvAnew{2009} Show that a $p$-point ultrafilter is nowhere dense.
(iii)\dvAnew{2009}
In 538E, show that if every $\Cal F_{\xi}$ is a nowhere dense ultrafilter,
then $\Cal G_{\zeta}$ is a nowhere dense ultrafilter.
%538E 538Xe

\sqheader 538Xg
Let $\Cal F$ be a free filter on $\Bbb N$.   Show that the
following are equiveridical:  (i) $\Cal F$ is a Ramsey filter;  (ii)
whenever $K$ is finite, $k\in\Bbb N$ and $f:[\Bbb N]^k\to K$ is a function,
there is an $F\in\Cal F$ such that $f$ is constant on $[F]^k$;
(iii) $\Cal F$ is a $p$-point filter and whenever $\sequencen{E_n}$ is a
disjoint sequence in $[\Bbb N]^{<\omega}$, there is an $F\in\Cal F$
such that $\#(F\cap E_n)\le 1$ for every $n$;  (iv) whenever
$\sequencen{E_n}$ is a disjoint sequence in $\Cal P\Bbb N\setminus\Cal F$,
there is an $F\in\Cal F$ such that $\#(F\cap E_n)\le 1$ for every $n$.
%538F

\spheader 538Xh Let $\frak F$ be a countable family of distinct $p$-point
ultrafilters on $\Bbb N$.   Show that there is a disjoint family
$\family{\Cal F}{\frak F}{A_{\Cal F}}$ of subsets of $\Bbb N$ such that
$A_{\Cal F}\in\Cal F$ for every $\Cal F\in\frak F$.
%538F

\spheader 538Xi\dvAnew{2010}
Let $(X,\Sigma,\mu)$ be a complete
perfect probability space, $(Y,\frak S)$ a
perfectly normal compact Hausdorff space,
$\sequencen{f_n}$ a sequence of measurable functions from $X$ to $Y$,
$\Cal F$ a measure-centering ultrafilter on $\Bbb N$ and $\lambda$ the
$\Cal F$-extension of $\mu$.   (i) Setting
$f(x)=\lim_{n\to\Cal F}f_n(x)$ for $x\in X$, show that $f$ is
$\dom\lambda$-measurable.   (ii) For each $n\in\Bbb N$, show that there is
a unique Radon measure $\nu_n$ on $Y$ such that $f_n$ is \imp\ for $\mu$
and $\nu_n$.
(iii) Let $\nu$ be the limit $\lim_{n\to\Cal F}\nu_n$ for the
narrow topology on the space of Radon probability measures on $Y$ (437Jd).
Show that $f$ is \imp\ for $\lambda$ and $\nu$.   \Hint{look at the Radon
measure associated with the image measure $\lambda f^{-1}$.   You may
prefer to begin with metrizable $Y$.}
%538I

\spheader 538Xj Let $\familyiI{(\frak A_i,\bar\mu_i)}$ be a family of
probability algebras, $\Cal F$ an ultrafilter on $I$, and
$(\frak A,\bar\mu)$ the probability algebra reduced product of
$\prod_{i\in I}(\frak A_i,\bar\mu_i)|\Cal F$.   For each $i\in I$,
let $\Bsubseteqshort_i$ be the order relation on $\frak A_i$;  set
$P=\prod_{i\in I}\frak A_i$ and let $P|\Cal F$ be the
partial order reduced product of $\familyiI{(\frak A_i,\Bsubseteqshort_i)}$
modulo $\Cal F$ as defined in 5A2A.   Describe a canonical order-preserving
map from $P|\Cal F$ to $\frak A$.
%538J

\spheader 538Xk(i) Let $(\frak A,\bar\mu)$ be a homogeneous
probability algebra with Maharam type $\kappa$,
$I$ a non-empty set, $\Cal F$ an ultrafilter on $I$ and
$(\frak C,\bar\nu)$ the probability algebra reduced power
$(\frak A,\bar\mu)^I|\Cal F$.   Show that $\frak C$ is homogeneous, with
Maharam type the transversal number $\Tr_{\Cal I}(I;\kappa)$
(definition:  5A1L), where $\Cal I=\{I\setminus A:A\in\Cal F\}$.
\Hint{5A1Md, 521Eb.} %\Tr attained, M type=top density
(ii)\dvAnew{2009} % {\smc Farah Hart \& Sherman p09}
Show that if $(\frak A,\bar\mu)$ is any probability algebra
and $\Cal F$ and $\Cal G$ are non-principal
ultrafilters on $\Bbb N$, then the probability algebra reduced powers
$(\frak A,\bar\mu)^{\Bbb N}|\Cal F$ and
$(\frak A,\bar\mu)^{\Bbb N}|\Cal G$ are isomorphic.
%538J

\spheader 538Xl Let $(X,\Sigma,\mu)$ be a perfect probability space and
$\mu'$ an indefinite-integral measure over $\mu$ which is also a
probability measure.   Let $\Cal F$ be a measure-centering ultrafilter on
$\Bbb N$ and $\lambda$, $\lambda'$ the $\Cal F$-extensions of $\mu$ and
$\mu'$.   Show that $\lambda'$ is an
indefinite-integral measure over $\lambda$.
%538K  take care over  \dom(\lambda')

\sqheader 538Xm  ({\smc Benedikt 98})
(i) Let $\Cal F$ be any free filter on $\Bbb N$.   Show that
$\Cal F\ltimes\Cal F$ is not measure-centering.   \Hint{let
$\sequencen{e_n}$ be the standard generating family in $\frak B_{\omega}$,
and consider $a_{mn}=e_m\Bsetminus e_n$ if $m<n$, $1$ otherwise.}
(ii) Let $\Cal F$ be a measure-centering ultrafilter on $\Bbb N$.
Show that if $f$, $g\in\NN$ and $\{n:f(n)\ne g(n)\}\in\Cal F$, then
$f[[\Cal F]]\ne g[[\Cal F]]$.   \Hint{consider
$a_n=e_{f(n)}\Bsetminus e_{g(n)}$ if $f(n)\ne g(n)$.}
%538L "Hausdorff filter"

\spheader 538Xn\dvAnew{2009}
Let $X$ be a locally compact Hausdorff topological group,
and $\mu$ a left Haar measure on $X$.   Show that there is a
complete locally determined \lti\ measure
$\lambda$ on $X$ such that
$\lambda(\lim_{n\to\Cal F}E_n)$ is defined and equal to
$\sup_{K\subseteq X\text{ is compact}}\lim_{n\to\Cal F}\mu(E_n\cap K)$
whenever $\Cal F$ is a Ramsey ultrafilter on
$\Bbb N$ and $\sequencen{E_n}$ is a sequence of Haar measurable subsets of
$X$.
%538M

\spheader 538Xo(i) Let $\sequencen{\Cal F_n}$ be a sequence of
measure-converging filters on $\Bbb N$.
Show that $\bigcap_{n\in\Bbb N}\Cal F_n$ is measure-converging, so that
$\lim_{n\to\Cal F}\Cal F_n$ (538Xe) is measure-converging for any
filter $\Cal F$ on $\Bbb N$.
(ii) In 538E, suppose that $\Cal F_1$ is measure-converging.
Show that $\Cal G_{\xi}$ is measure-converging for every $\xi\in[1,\zeta]$.
%538N 538Xg

\spheader 538Xp\dvAnew{2012} Suppose that
$\ofamily{\xi}{\kappa}{\Cal F_{\xi}}$ is a family of measure-converging
filters, where $\kappa$ is non-zero and
less than the additivity $\add\Cal N$ of
Lebesgue measure.   Show that $\bigcap_{\xi<\kappa}\Cal F_{\xi}$ is
measure-converging.
%538N(iv) 538Xo

\spheader 538Xq(i)\dvAnew{2014}
Let $\Cal F$ be a filter on $\Bbb N$.   Show that
$\Cal F$ has the Fatou property iff $\int fd\mu$ and
$\lim_{n\to\Cal F}\int f_nd\mu$ are defined and
equal whenever $(X,\Sigma,\mu)$ is a
measure space, $g:X\to\coint{0,\infty}$ is an integrable function and
$\sequencen{f_n}$ is a sequence of measurable functions on $X$ such that
$|f_n|\leae g$ for every $n$ and $\lim_{n\to\Cal F}f_n\eae f$.
(ii) Show that a non-principal ultrafilter on $\Bbb N$ cannot
have the Fatou property.   \Hint{464Ca.}
%538O

\spheader 538Xr
Show that the asymptotic density filter (538Xd) has the Fatou property.
%538O 538Xd 538Ym

\spheader 538Xs(i)
Let $\sequencen{\Cal F_n}$ be a sequence of filters with
the Fatou property, and $\Cal F$ a filter with the Fatou property.   Show
that $\lim_{n\to\Cal F}\Cal F_n$ (538Xe) has the Fatou property.
(ii) In 538E, suppose that $\Cal F_{\xi}$ has the Fatou property for every
$\xi\in[1,\zeta]$.
Show that $\Cal G_{\xi}$ has the Fatou property for every $\xi\le\zeta$.
%538O

\spheader 538Xt Let $\nu:\Cal P\Bbb N\to\Bbb R$ be a bounded additive
functional.   (i) Show that $\nu$ is a medial functional iff
$\int\nu\{n:x\in E_n\}\mu(dx)$ is defined and equal to
$\dashint\mu E_n\nu(dn)$ whenever
$(X,\Sigma,\mu)$ is a probability space and $\sequencen{E_n}$ is a sequence
in $\Sigma$.   (ii) Show that in this case $a\mapsto\nu\phi^{-1}[a]$ is a
medial functional for any $\phi:\Bbb N\to\Bbb N$.
%538Q

\sqheader 538Xu Let $(X,\Sigma,\mu)$ be a probability space, and $\Tau$ a
$\sigma$-subalgebra of $\Sigma$.   Let $\sequencen{f_n}$ be a sequence in
$\eusm L^{\infty}(\mu)$ such that $\sup_{n\in\Bbb N}\esssup|f_n|$ is
finite, and for each $n\in\Bbb N$ let $g_n$ be a conditional expectation of
$f_n$ on $\Tau$.   Suppose that $\nu$ is a medial functional.   Show that
$f(x)=\dashint f_n(x)\nu(dn)$ and
$g(x)=\dashint g_n(x)\nu(dn)$ are defined for
almost every $x$, that $f\in\eusm L^{\infty}(\mu)$, and that $g$ is a
conditional expectation of $f$ on $\Tau$.
%538R

\spheader 538Xv (V.Bergelson) Show that there are a probability algebra
$(\frak A,\bar\mu)$ and a sequence $\sequencen{a_n}$ in $\frak A$ such that
$\inf_{n\in\Bbb N}\bar\mu a_n\penalty-100>0$
but $a_m\Bcap a_n\Bcap a_{m+n}=0$ whenever
$m$, $n>0$.   \Hint{for $n\ge 1$, set
$E_n=\{x:x\in[0,1]$, $\lfloor 3nx\rfloor\equiv 1\mod 3\}$.}
%538+

\leader{538Y}{Further exercises (a)}\dvAnew{2014}
%\spheader 538Ya
Show that if $\Cal F$ and $\Cal G$ are
filters and $\Cal F\leRK\Cal G$, then, in the language of
512A, $(\Cal F,\supseteq,\Cal F)\prGT(\Cal G,\supseteq,\Cal G)$, so that
$\ci\Cal F\le\ci\Cal G$ and $\Cal F$ is $\kappa$-complete whenever
$\kappa$ is a cardinal and $\Cal G$ is $\kappa$-complete.
%538B

\spheader 538Yb
Let $\Cal F$ be a free ultrafilter on $\Bbb N$, and suppose
that whenever $\Cal G$ is a free filter on $\Bbb N$ and
$\Cal G\leRK\Cal F$, then $\Cal F\leRK\Cal G$.   Show that $\Cal F$ is a
Ramsey ultrafilter.   \Hint{{\smc Comfort \& Negrepontis 74}.} %Th 9.6
%538D

\spheader 538Yc Show that if $\frak p=\frak c$ then there are
$2^{\frakc}$
Ramsey ultrafilters on $\Bbb N$, and therefore $2^{\frakc}$ isomorphism
classes of Ramsey ultrafilters.
%538F

\spheader 538Yd Let $\Cal F$ be an ultrafilter on $\Bbb N$.   Show that
$\Cal F$ is measure-centering iff whenever $\frak A$ is a Boolean algebra,
$D\subseteq\frak A\setminus\{0\}$ has intersection number greater than
$0$ (definition: 391H) and $\sequencen{a_n}$ is a sequence in $D$, then
there is an $A\in\Cal F$ such that $\{a_n:n\in A\}$ is centered.
%538G out of order query

\spheader 538Ye(i) Show that if $\cov\Cal N=\frak c$, there is a
measure-centering ultrafilter on $\Bbb N$
including the asymptotic density filter (538Xd).   (ii) Show that an
ultrafilter on
$\Bbb N$ including the asymptotic density filter cannot be a
$p$-point filter.
(iii) Show that a filter on $\Bbb N$ including the asymptotic density
filter cannot be a rapid filter.
%538H out of order query 53bits

\spheader 538Yf(i) Let $\Cal F$, $\Cal G$ be free filters on $\Bbb N$ such
that $\Cal F\ltimes\Cal G$ is measure-centering.   Show that there is no
free filter $\Cal H$ such that $\Cal H\leRK\Cal F$ and $\Cal H\leRK\Cal G$.
(ii) Show that if there are two non-isomorphic Ramsey
ultrafilters on $\Bbb N$, then there are  two
non-isomorphic measure-centering ultrafilters $\Cal F$, $\Cal G$ on
$\Bbb N$ such that $\Cal F\ltimes\Cal G$ is not measure-centering.
% either  \CalH_1 , \CalH_1\ltimes\CalH_2  or
%  \CalH_1\ltimes\CalH_2, \CalH_2
%538Xm 538L

\spheader 538Yg For an uncountable set $I$,
let us say that a filter $\Cal F$
on $I$ is {\bf uniform and measure-centering} if $\#(A)=\#(I)$ for every
$A\in\Cal F$ and whenever $\frak A$ is a Boolean algebra,
$\nu:\frak A\to\coint{0,\infty}$ is an additive functional, and
$\familyiI{a_i}$ is a family in $\frak A$ with
$\inf_{i\in I}\nu a_i>0$, there is an $A\in\Cal F$ such that
$\{a_i:i\in A\}$ is centered.   (i) State and prove a result corresponding
to 538G for such filters.   \Hint{in the part corresponding to 538G(iv),
use `compact' measures rather than `perfect' measures.}   (ii) State and
prove a result corresponding to 538H.   \Hint{set $\kappa=\#(I)$.
In the part corresponding to
538Hc, suppose that you have a $\kappa$-complete ultrafilter on $I$,
rather than a
Ramsey ultrafilter;  see 4A1L.   In the part corresponding to 538He,
suppose that $\kappa$ is regular and that
$\cov\Cal N_{\kappa}=2^{\kappa}$, where $\Cal N_{\kappa}$ is
the null ideal of the usual measure on $\{0,1\}^{\kappa}$.}
(iii) State and prove results corresponding to 538I-538K. %538I 538J 538K
(iv) State and prove results corresponding to 538L-538M, but with
`normal ultrafilters' in place of `Ramsey ultrafilters'.
%538M
%question:  can there be a uniform measure-centering ultrafilter on
%\frak c ?

\spheader 538Yh\dvAnew{2009} Show that if $\Cal F$ and $\Cal G$ are filters
on $\Bbb N$, $\Cal F$ is rapid and $\Cal G\leRB\Cal F$, then $\Cal G$ is
rapid.
%538N

\spheader 538Yi Give an example of filters $\Cal F$, $\Cal G$ on $\Bbb N$
such that $\Cal F$ has the Fatou property, $\Cal G\subseteq\Cal F$ and
$\Cal G$ does not have the Fatou property.
%538O  \Cal G=\Cal F_d\cap\Cal H  where  \Cal H  ultrafilter containing
%set of density  0

\spheader 538Yj(i) Let $\Cal F$ be a nowhere dense filter on $\Bbb N$,
and $\Cal I$ the
ideal $\{\Bbb N\setminus A:A\in\Cal F\}$.   Show that
$\Cal P\Bbb N/\Cal I$ is finite.
(ii) Show that a free filter with the Fatou property cannot be nowhere
dense.
%538Xq 538O

\spheader 538Yk\dvAnew{2009} Let $(X,\Sigma,\mu)$ be a probability space
and $\sequence{m}{f_m}$, $\sequencen{g_n}$ two uniformly bounded
sequences of real-valued measurable functions defined on $X$.
Let $\nu$,
$\nu':\Cal P\Bbb N\to\Bbb R$ be bounded additive functionals.   Show that
$\dashidashiint f_m\times g_nd\mu\,\nu(dm)\nu'(dn)
=\dashidashiint f_m\times g_nd\mu\,\nu'(dn)\nu(dm)$.
%538P
%both equal to $\int f\times g$.

\spheader 538Yl ({\smc Meyer 73})
Let $\nu$ be a medial limit.   Write $U$ for the set of
sequences $u\in\BbbR^{\Bbb N}$ such that
$\sup\{\dashint v\,d\nu:v\in\ell^{\infty}$, $v\le|u|\}$ is
finite;  for $u\in U$, write $\dashint u\,d\nu$ for
$\lim_{m\to\infty}\dashint\med(-m,u(n),m)\nu(dn)$
(see 364Xj).
Suppose that $(X,\Sigma,\mu)$ is a probability space and
$\sequencen{f_n}$ a sequence of $\mu$-integrable real-valued functions
on $X$ such that $\sequencen{\int|f_n|d\mu}\in U$.
(i) Show that $\sequencen{f_n(x)}\in U$ for $\mu$-almost every
$x\in X$.   Set $f(x)=\dashint f_n(x)\nu(dn)$ whenever
$\sequencen{f_n(x)}\in U$.
(ii) Show that if every $f_n$ is non-negative then
$\int fd\mu\le\dashiint f_nd\mu\,\nu(dn)$.
(iii) Show that if $\{f_n:n\in\Bbb N\}$ is uniformly integrable then
$\int fd\mu=\dashiint f_nd\mu\,\nu(dn)$.
(iv)\dvAnew{2012}
Show that if $\sequencen{f_n^{\ssbullet}}$ is weakly convergent to $0$
in $L^1(\mu)$, then $f\eae 0$.   (v)\dvAnew{2012}
Suppose that $\sequencen{f_n}$ is uniformly integrable.
Let $\Tau$ be a $\sigma$-subalgebra of $\Sigma$, and for
each $n\in\Bbb N$ let $g_n$ be a conditional expectation of $f_n$ on
$\Tau$;  set
$g(x)=\dashint g_n(x)\nu(dn)$ whenever $\sequencen{g_n(x)}\in U$.
Show that
$g$ is a conditional expectation of $f$ on $\Tau$.
%538R 53bits 538Xu

\spheader 538Ym\dvAnew{2014}
Suppose that $\Cal F$ is a filter on $\Bbb N$ with the Fatou property, and
$\sequence{n}{\nu_n}$ a sequence of medial limits.   Set
$\Cal G=\{A:A\subseteq\Bbb N$, $\lim_{n\to\Cal F}\nu_nA=1\}$.   Show that
$\Cal G$ is a filter with the Fatou property.
%538Xr 538R mt53bits

\spheader 538Yn\dvAnew{2009} Show that
$\frak u\ge\frak r(\omega,\omega)\ge\max(\cov\Cal N,\frakmctbl)$
(definitions:  5A6Ia, 529G).
%538S 529Xg

\spheader 538Yo\dvAnew{2009}(i) Show that if $\Cal F$ is a rapid filter
on $\Bbb N$, then
$\ci\Cal F\ge\frak d$.   (ii) Show that $\frak d\ge\frak g$ (definition:
5A6I(b-ii)).   (iii) Show that if $\frak u<\frak g$ there are no rapid
filters on $\Bbb N$, and if there is a measure-converging filter
there is a measure-converging ultrafilter with coinitiality $\frak u$.
%538S 5A6J 538Ya 53bits

\spheader 538Yp\dvAnew{2011} Suppose that the filter dichotomy is true.
(i) Let $\frak A$ be a Dedekind $\sigma$-complete
Boolean algebra.   Show that if $\nu:\frak A\to\Bbb R$ is an additive
functional which is universally measurable for the order-sequential
topology of $\frak A$, then
$\nu$ is countably additive.   (ii) Let $(\frak A,\bar\mu)$ be a
localizable measure algebra.   Show that if $\nu:\frak A\to\Bbb R$ is an
additive functional which is universally measurable for the measure-algebra
topology on $\frak A$, then it is continuous.
%538S

\spheader 538Yq(i) Show that there is a semigroup operation $\dot+$ on
the set $\beta\Bbb N$ of ultrafilters on $\Bbb N$ defined by saying that
$\Cal F\dot+\Cal G=+[[\Cal F\ltimes\Cal G]]$ for all $\Cal F$,
$\Cal G\in\beta\Bbb N$, where $+:\Bbb N\times\Bbb N\to\Bbb N$ is addition.
(ii) Show that if we identify $\beta\Bbb N$ with the Stone-\v{C}ech
compactification of $\Bbb N$ (4A2I(b-i)), then $\dot+$ is
continuous in the first variable.   (iii) Show that there is a
non-principal ultrafilter $\Cal F$ on $\Bbb N$ which is {\bf idempotent},
that is, $\Cal F\dot+\Cal F=\Cal F$.
\Hint{consider a minimal closed sub-semigroup of the set of non-principal
ultrafilters.}
(iv) For any function $f\in\NN$, write $\FS(f)$ for
$\{\sum_{n\in K}f(n):K\in[\Bbb N]^{<\omega}\}$;  say a {\bf finite sum set}
is a set of the form $\FS(f)$ for some strictly increasing function
$f\in\NN$.   Show that if $\Cal F$ is a non-principal idempotent
ultrafilter on $\Bbb N$ and $I\in\Cal F$, then $I$ includes a finite sum
set.   (This is a version of {\bf Hindman's theorem}.)
(v) Show that if $I\subseteq\Bbb N$ is a finite sum set
there is an idempotent ultrafilter containing $I$.
(vi) Suppose that $(\frak A,\bar\mu)$ is a probability algebra and
$\pi:\frak A\to\frak A$ is a measure-preserving Boolean homomorphism.
($\alpha$) Show that if $\Cal F$ is an idempotent ultrafilter on $\Bbb N$,
then
$\lim_{n\to\Cal F}\mu(a\Bcap\pi^na)\ge(\mu a)^2$ for every $a\in\frak A$
($\beta$) Show that there is a finite sum set $I\subseteq\Bbb N$ such that
$\{\pi^na:n\in I\}$ is centered.
(vii) Show that no idempotent ultrafilter is measure-centering.
\Hint{538Xv.}
(viii) Show that if $\Cal F$ is a $p$-point ultrafilter then
$\Cal F\dot+\Cal F$ is isomorphic to $\Cal F\ltimes\Cal F$ and is
not measure-centering.
%mt53bits
(ix) Repeat, as far as possible, for semigroups other than $(\Bbb N,+)$.
%538+

\spheader 538Yr (V.Bergelson-M.Talagrand) Show that there are a probability
algebra $(\frak A,\bar\mu)$ and a sequence $\sequencen{a_n}$ in $\frak A$
such that $\bar\mu a_n=\bover12$ for every $n\in\Bbb N$ but
$\inf_{m,n\in I}\bar\mu(a_m\Bcap a_n)=0$ whenever
$I\subseteq\Bbb N$ does not have asymptotic density $0$.
%538+ n08707.tex
}%end of exercises

\leader{538Z}{Problem}
Show that it is relatively consistent with ZFC to suppose that there are
no measure-converging filters on $\Bbb N$.
%What happens if we add Cohen reals to a model of GCH?
%What if  \frak u < \frak g ? 538Ys

\leaveitout{For a function $f:\Bbb N\to\Bbb N$, write $\FS(f)$ for
$\{\sum_{n\in I}f(n):I\in[\Bbb N]^{<\omega}\}$.   A {\bf finite sum set} is
a set of the form $\FS(f)$ for a strictly increasing sequence $f$.
Suppose that $(\frak A,\bar\mu)$ is a probability algebra and
$\sequencen{a_n}$ is a sequence in $\frak A$ such that
$\inf_{n\in\Bbb N}\mu a_n>0$.   Must there be a finite sum set
$F\subseteq\Bbb N$ such that $\{a_n:n\in F\}$ is centered?

In 538Yq(vi-$\beta$), if $\Cal F$ is idempotent,
do we actually have a set $I\in\Cal F$ such that $\{\pi^na:n\in I\}$ is
centered?
}

\endnotes{
\Notesheader{538} This is a long section, and rather a lot of ideas are
crowded into it, starting with the list in 538A.   If you have looked at
ultrafilters on $\Bbb N$ at all, you are likely to have encountered
`$p$-point', `rapid' and `Ramsey' ultrafilters, and most of
538B-538D %538B 538C 538D
and 538F will probably be familiar.   The `iterated products' of 538E will
also be a matter of adapting known concepts to my particular formulation.

Some of the slightly contorted language of 538Fe and 538Ff
(with references to
`$\#(\frak F)$') is there because we do not know how many isomorphism
classes of Ramsey filters there are.   If there are none (as in random real
models, see 553H), or one ({\smc Shelah 82}, \S VI.5),
then things are very simple.   If there are infinitely many then we could
rephrase 538Ff in terms of sequences of non-isomorphic filters.
But it is possible that there should be two, or seventeen ({\smc Shelah
98a}, p.\ 335).
%e-mail to S.S., 9.7.07;  reply 4.5.14

In 538H-538M %538I 538L 538M
I try to set out, and expand, some of the principal ideas of
{\smc Benedikt 98}.   The starting point is the observation that a Ramsey
ultrafilter gives us an extension of Lebesgue measure on $[0,1]$,
indeed of any perfect probability measure.   Observing that this property
is preserved by iterations, we are led to `measure-centering' ultrafilters.
Once we have the idea of measure-centering-ultrafilter extension
of a perfect
probability measure, we can set out to look at its properties in terms of
the (by now very extensive) general theory of this treatise.   The first
step has to be the identification of its measure algebra (538Ja, 538Xk),
followed, if
possible, by the identification of the corresponding Banach function
spaces.   It turns out that these can be reached by an alternative route
{\it not} involving special properties of the ultrafilter or the
probability space, which I have expressed in general forms in
\S\S328 and 377.   This gives a
long list of facts, which I have written out in 538Ja and 538K.
Minor variations of the measure and the filter are straightforward
(538Jb, 538Jc, 538Xl).   For iterated products of filters we have more
work to do (538L), especially if we are
to express them in a form adequate for
the objective, the universal-extension result of 538M.

You will have noticed that in the statement of 538G I speak of
`$\bigcup_{A\in\Cal F}\bigcap_{n\in A}F_n$' and
`$\liminf_{n\to\Cal F}\mu F_n$'.   Something of the sort is necessary since
in that theorem I do not insist from the outset
that $\Cal F$ should be an ultrafilter.   Of course only ultrafilters are
of interest in this context, by 538Ha, and for these we have
$\bigcup_{A\in\Cal F}\bigcap_{n\in A}F_n=\lim_{n\to\Cal F}F_n$ and
$\liminf_{n\to\Cal F}\mu F_n=\lim_{n\to\Cal F}\mu F_n$, as in 538I.

For most of this section I have kept firmly to the study of filters on
$\Bbb N$.   For measure-centering filters, at least, there are interesting
extensions to filters on uncountable sets, which I mention in 538Yg.   We
can do a good deal with the ideas of 538G-538K on cardinals less than
$\frak c$ in the presence of (for instance) Martin's axiom;  but for
anything corresponding to 538L-538M it seems that we must use a \2vm\
cardinal (541M below).

Measure-converging filters (538N) and filters with the Fatou property
(538O) form an oddly complementary pair.   I have tried to emphasize the
correspondence in the
characterizations 538Na and 538Oa
(compare 538G(v), 538Na(iv) and 538Oa(iv)),
but after this they seem to diverge.   The phrase `Fatou
property' comes from 538O(a-iii);  if you like, Fatou's Lemma says that the
Fr\'echet filter has the Fatou property.   From 538Xq(i) I see that I could
just as well have called it the `Lebesgue property'.
Note that any filter
larger than a measure-converging filter is again measure-converging, so
that if there is a measure-converging filter there is a measure-converging
ultrafilter;  but that no non-principal ultrafilter can have the Fatou
property (538Xq(ii)).   On the other hand, there are many free filters with the
Fatou property, but it is not known for sure whether there have to be
measure-converging filters.   It is possible for a
measure-converging filter to have the Fatou property (538Rd).

In the last part of the section I look at a different kind of limit.   A
`Banach limit' is an extension to $\ell^{\infty}$
of the ordinary limit regarded as a linear functional on the closed
subspace of convergent sequences;  a `medial limit' is a Banach limit which
commutes with integration in appropriate settings.   To study these
I use the formulae of repeated integration to do some
surprising things.   In 363L I tried to explain what I meant by the formula
`$\dashint\ldots d\nu$' for a {\it finitely} additive functional $\nu$.
This defines linear functionals which are positive for non-negative $\nu$.
In `repeated integrals' like $\dashiint f_n(x)\mu(dx)\nu(dn)$ (538P(iii)),
we must interpret the formula
as $\dashint\bigl(\int f_n(x)\mu(dx)\bigr)\nu(dn)$;  the `inner
integral' is an ordinary integral with respect to the countably additive
measure $\mu$, and the `outer integral' is a name for a linear functional.
In the integral $\dashint\ldots d\nu$
we have no problem with measurability,
though we must check that the integrand $n\mapsto\int f_nd\mu$ is bounded
(or, at least, satisfies the condition in 538Yl);
but when we look at the other repeated integrals, $\int\nu(a)\mu(da)$ or
$\idashint x\,d\nu\,\mu(dx)$ or $\idashint f_n(x)\nu(dn)\mu(dx)$,
the conditions of 538P must explicitly assert
that the outer integrals are defined.

Because we don't need to consider measurability, the `finitely additive
integrals' here are in some ways easy to deal with;  `disintegrations' like
$\tilde\nu=\dashint\nu_k\,\nu(dk)$
(538Rc) slide past all the usual questions.
However we must always be vigilant against the temptations of limiting
processes.   As with the Riemann integral, of course, we can integrate
the limit of a uniformly convergent sequence of functions.   But see the
manoeuvres of part (a-iii) of the proof of 538R, where
the sums $\sum_{i=0}^{\infty}\alpha_{ni}...$ demand different treatments at
different points.   And Fubini's theorem nearly disappears;  the point of
`medial functionals' is that something extraordinary has to happen before
we can expect to change the order of integration.

I have used the language of Volume 3 to express 538Re in a general form.
Of course by far the most important example is when the operator $T$ is a
conditional expectation operator (538Xu).   For more examples of operators
in $L^{\times}(L^{\infty};L^{\infty})$, see \S\S373-374.

For most of the classes of filter here, there is a question concerning
their existence.   Subject to the continuum hypothesis, there are many
Ramsey ultrafilters, and refining the argument we find that the same is
true if $\frak p=\frak c$ (538Yc).
There are many ways of forcing non-existence
of Ramsey ultrafilters, of which one of
the simplest is in 553H below.   With more difficulty, we can eliminate
$p$-point ultrafilters ({\smc Wimmers 82})
or rapid filters ({\smc Miller 80}) or nowhere dense
filters and therefore measure-centering ultrafilters
(538Hd, {\smc Shelah 98b}).   It is not known for sure that we can
eliminate measure-converging filters (538Z).

}%end of notes

\discrpage

