\frfilename{mt531.tex}
\versiondate{19.2.11/24.2.11}
\copyrightdate{2003}

\def\chaptername{Topologies and measures III}
\def\sectionname{Maharam types of Radon measures}

\newsection{531}

In the introduction to \S434 I asked

\Centerline{\it What kinds of measures can arise on what kinds of
topological space?}

\noindent In \S\S434-435, and again in \S438, I considered a variety of
topological properties and their relations with measure-theoretic
properties of Borel and Baire measures.   I passed over, however, some
natural questions concerning possible Maharam types, to which I now
return.   For a given Hausdorff space $X$, the possible measure algebras
of totally finite Radon measures on $X$ can be described in terms of the
set $\MahR(X)$ of Maharam types of \Mth\ Radon probability measures on
$X$ (531F).   For $X\ne\emptyset$, $\MahR(X)$ is of the form
$\{0\}\cup\coint{\omega,\kappa^*}$ for some infinite cardinal $\kappa^*$
(531Ef).   In 531E and 531G I give basic results from which $\MahR(X)$
can often be determined;  for obvious reasons we are primarily concerned
with compact spaces $X$.   In more abstract contexts, there are striking
relationships between precalibers of measure algebras, the sets
$\MahR(X)$ and continuous surjections onto powers of $[0,1]$, which I
examine in 531L-531N %531L 531M 531N
and 531T.   Intertwined with these, we have results relating the
character of $X$ to $\MahR(X)$ (531O-531P).   The arguments here depend
on an analysis of the structure of homogeneous measure algebras (531J,
531K, 531R).

%531A%531B easy basics
%531C M type of product
%531D defn of MahR
%531E%531F%531G%531H basic props of MahR
%531I notation
%531J%531K lemmas
%531L%531M%531N%531O%531P precalibers and Haydon's property
%531Q HS HL type omega_1
   %531R lemma
   %531S%531T with m_K

\leader{531A}{Proposition} Let $(X,\frak T,\Sigma,\mu)$ be a quasi-Radon
measure space with measure algebra $(\frak A,\bar\mu)$.

(a)\cmmnt{ The Maharam type} $\tau(\frak A)$\cmmnt{ of $\frak A$} is
at most\cmmnt{ the weight} $w(X)$\cmmnt{ of $X$}.

(b) \cmmnt{The cellularity} $c(\frak A)$\cmmnt{ of $\frak A$} is at
most\cmmnt{ the hereditary Lindel\"of number} $\hL(X)$\cmmnt{ of
$X$}.   If $\mu$ is locally finite, $c(\frak A)$ is at most\cmmnt{ the
Lindel\"of number} $L(X)$\cmmnt{ of $X$}.

(c) $\#(\{a:a\in\frak A$, $\bar\mu a<\infty\})
\le\max(1,w(X)^{\omega})$\cmmnt{, where $w(X)^{\omega}$ is the
cardinal power}.

(d) If $X$ is Hausdorff and $\mu$ is a Radon measure,
then\cmmnt{ the Maharam type} $\tau(\frak A)$\cmmnt{ of $\frak A$} is at
most\cmmnt{ the network weight} $\nw(X)$\cmmnt{ of $X$}.

\proof{{\bf (a)} Let $\Cal U$ be a base for $\frak T$ with
$\#(\Cal U)=w(X)$.   Set $B=\{U^{\ssbullet}:U\in\Cal U\}$ and let
$\frak B$ be the order-closed subalgebra of $\frak A$ generated by $B$;
set $\Tau=\{E:E\in\Sigma$, $E^{\ssbullet}\in\frak B\}$.   Then $\Tau$ is
a $\sigma$-subalgebra of $\Sigma$ containing every negligible set.

If $G\subseteq X$ is open, then $G\in\Tau$.   \Prf\ By 414Aa,
$G^{\ssbullet}=\sup\{U^{\ssbullet}:U\in\Cal U$, $U\subseteq G\}$ belongs
to $\frak B$.\ \QeD\   So every Borel set belongs to $\Tau$.   If
$E\in\Sigma$ and $\mu E<\infty$, then, because $\mu$ is inner regular
with respect to the Borel sets, there is a Borel subset $F$ of $E$ with
the same measure, so $F$, $E\setminus F$ and $E$ belong to $\Tau$.
Thus $\{a:a\in\frak A$, $\bar\mu a<\infty\}\subseteq\frak B$;  because
$\mu$ is semi-finite, $\frak B=\frak A$ and
$\tau(\frak A)\le\#(\Cal U)=w(X)$.

\medskip

{\bf (b)(i)} If $L(X)=n$ is finite, and $F_0,\ldots,F_n\subseteq X$ are
disjoint closed sets, then at least one of them is empty.   \Prf\ For
$i\le n$, set $G_i=X\setminus\bigcup_{j\le n,j\ne i}F_j$;  then
$\bigcup_{i\le n}G_i=X$, so there is some $k\le n$ such that
$\bigcup_{i\ne k}G_i=X$, and now $F_k=\emptyset$.\ \QeD\   As $\mu$ is
inner regular with respect to the closed sets,
$c(\frak A)\le n=L(X)\le\hL(X)$.

\medskip

\quad{\bf (ii)} Suppose that $\omega\le L(X)\le\hL(X)$.   Let $\Cal G$
be the family of open subsets of $X$ of finite measure.   Then there is
a set $\Cal H\subseteq\Cal G$, with cardinal at most $\hL(X)$, such that
$\bigcup\Cal H=\bigcup\Cal G$ (5A4Bf).
Now $\sup_{H\in\Cal H}H^{\ssbullet}=1$,
because $\mu$ is effectively locally finite.

If $D\subseteq\frak A\setminus\{0\}$ is disjoint, then for each
$d\in D$ take $H_d\in\Cal H$ such that $d\Bcap H_d^{\ssbullet}\ne 0$.
If $H\in\Cal H$, then $\{d:H_d=H\}$ must be countable, since
$\mu H<\infty$.   So $\#(D)\le\max(\omega,\#(\Cal H))$;  as $D$ is
arbitrary, $c(\frak A)\le\max(\omega,\hL(X))=\hL(X)$.

\medskip

\quad{\bf (iii)} Finally, if $\omega\le L(X)$ and $\mu$ is locally
finite, then in (ii) above we have $X=\bigcup\Cal G$, so we can take
$\Cal H$ to have size at most $L(X)$, and continue as before, ending
with $c(\frak A)\le\max(\omega,\#(\Cal H))=L(X)$.

\medskip

{\bf (c)} Again let $\Cal U$ be a base for the topology of $X$ with
cardinal $w(X)$.   Let $\Tau$ be the $\sigma$-subalgebra of $\Sigma$
generated by $\Cal U$.   If $E\in\Sigma$ and $\mu E<\infty$, then for
each $n\in\Bbb N$ we can find an open set $G_n$ such that
$\mu(G_n\symmdiff E)\le 2^{-n}$;  now there is an
open set $H_n$, a finite union of members of $\Cal U$, such that
$H_n\subseteq G_n$ and $\mu(G_n\setminus H_n)\le 2^{-n}$.   Setting
$F=\bigcup_{m\in\Bbb N}\bigcap_{n\in\Bbb N}H_n$, we see that $F\in\Tau$
and $E\symmdiff F$ is negligible.   Thus
$\{F^{\ssbullet}:F\in\Tau\}\supseteq\{a:\bar\mu a<\infty\}$ and

\Centerline{$\#(\{a:\bar\mu a<\infty\})\le\#(\Tau)
\le\max(1,\#(\Cal U)^{\omega})=\max(1,w(X)^{\omega})$.}

\medskip

{\bf (d)} If $a\in\frak A\setminus\{0\}$ and the principal ideal
$\frak A_a$ is \Mth, then $\tau(\frak A_a)\le\nw(X)$.   \Prf\ There is a
compact set $K\subseteq X$ such that
$0\ne K^{\ssbullet}\subseteq\frak A_a$;  let $\mu_K$ be the subspace
measure on $K$.   Then

$$\eqalignno{\tau(\frak A_a)
&=\tau(\mu_K)\le w(K)\cr
\displaycause{by (a)}
&=\nw(K)\cr
\displaycause{5A4C(a-i)}
&\le\nw(X)\cr}$$

\noindent (5A4Bb).\ \Qed

By (b), $c(\frak A)\le\#(\frak T)\le 2^{\nw(X)}$ (5A4Ba);  so 332S tells
us that $\tau(\frak A)\le\nw(X)$.
}%end of proof of 531A

\leader{531B}{}\cmmnt{ For strictly positive measures we have some
easy inequalities in the other direction.

\medskip

\noindent}{\bf Proposition} Let $(X,\Sigma,\mu)$ be a measure space,
with measure algebra $\frak A$, and $\frak T$ a topology on $X$ such
that $\Sigma$ includes a base for $\frak T$ and $\mu$ is strictly
positive.

(a) If $X$ is regular, then $w(X)\le\#(\frak A)$.

(b) If $X$ is Hausdorff, then $\#(X)\le 2^{\#(\frak A)}$.

\proof{ Set $\Cal V=\Sigma\cap\frak T$, so that $\Cal V$ is a base for
$\frak T$.  If $V$, $W\in\Cal V$ and  $V^{\ssbullet}=W^{\ssbullet}$ in
$\frak A$, then $\interior\overline{V}=\interior\overline{W}$.    \Prf\
$\mu^*(V\setminus\overline{W})\le\mu(V\setminus W)=0$, so (because $\mu$
is strictly positive) $V\subseteq\overline{W}$ and
$\overline{V}\subseteq\overline{W}$ and
$\interior\overline{V}\subseteq\interior\overline{W}$.   Similarly,
$\interior\overline{W}\subseteq\interior\overline{V}$.\ \QeD\  So if we
set $\Cal W=\{\interior\overline{V}:V\in\Cal V\}$,
$\#(\Cal W)\le\#(\frak A)$.

\medskip

{\bf (a)} If $\frak T$ is regular, $\Cal W$ is a base for $\frak T$, so
$w(X)\le\#(\Cal W)\le\#(\frak A)$.

\medskip

{\bf (b)} If $\frak T$ is Hausdorff, then for any distinct $x$,
$y\in X$, there is a $W\in\Cal W$ containing $x$ but not $y$.   \Prf\
Let $G$, $H$ be disjoint open sets containing $x$, $y$ respectively.
Take $V\in\Cal V$ such that $x\in V\subseteq G$, and set
$W=\interior\overline{V}$.\ \QeD\  So
$\#(X)\le 2^{\#(\Cal W)}\le 2^{\#(\frak A)}$.
}%end of proof of 531B

\leader{531C}{Lemma} Let $\familyiI{X_i}$ be a family of topological
spaces with product $X$, and $\mu$ a totally finite quasi-Radon measure
on $X$ with Maharam type $\kappa$.   For each $i\in I$, let $\mu_i$ be
the marginal measure on $X_i$, and $\kappa_i$ its Maharam
type.   Then $\kappa$ is at most the cardinal sum
$\sum_{i\in I}\kappa_i$.

\proof{ For each $i\in I$, let $\ofamily{\xi}{\kappa_i}{E_{i\xi}}$ be a
family in $\dom\mu_i$ such that
$\{E_{i\xi}^{\ssbullet}:\xi<\kappa_i\}\,\,\tau$-generates the measure
algebra of $\mu_i$.   Consider
$\Cal W=\{\pi_i^{-1}[E_{i\xi}]:i\in I$, $\xi<\kappa_i\}$, so that
$\Cal W\subseteq\dom\mu$ and
$\#(\Cal W)\le\sum_{i\in I}\kappa_i$.   Let $\frak B$ be the closed
subalgebra of the measure algebra $\frak A$ of $\mu$ generated by
$\{W^{\ssbullet}:W\in\Cal W\}$.

For each $i\in I$, the canonical map $\pi_i:X\to X_i$
induces a measure-preserving homomorphism
$\phi_i$ from the measure algebra $\frak A_i$ of $\mu_i$ to $\frak A$
(324M).   Now $\phi_i^{-1}[\frak B]$ is a closed subalgebra of
$\frak A_i$ containing $E_{i\xi}^{\ssbullet}$ for every $\xi<\kappa_i$,
so is the whole of $\frak A_i$, that is,
$\phi_i[\frak A_i]\subseteq\frak B$.   In particular, if
$G\subseteq X_i$ is open,
$\pi_i^{-1}[G]^{\ssbullet}=\phi_i(G^{\ssbullet})$ belongs to $\frak B$.
Now the family $\Cal V$ of open sets $V\subseteq X$ such that
$V^{\ssbullet}\in\frak B$ is closed under finite intersections and
contains $\pi_i^{-1}[G]$ whenever $i\in I$ and $G\subseteq X_i$ is open,
so $\Cal V$ is a base for the topology of $X$.   But also $\Cal V$ is
closed under arbitrary unions, because $\frak B$ is closed and $\mu$ is
$\tau$-additive (414Aa).   So $V^{\ssbullet}\in\frak B$ for every open
set $V\subseteq X$, and therefore for every Borel set $V\subseteq X$;
as $\mu$ is inner regular with respect to the Borel sets,
$\frak B=\frak A$.

Thus $\{W^{\ssbullet}:W\in\Cal W\}$ witnesses that the Maharam type
$\tau(\frak A)$ of $\mu$ is at most $\sum_{i\in I}\kappa_i$, as claimed.
}%end of proof of 531C

\leader{531D}{Definition} If $X$ is a Hausdorff space, I write
$\MahR(X)$ for the set of Maharam types of \Mth\ Radon probability
measures on $X$.   \cmmnt{Note that} $0\in\MahR(X)$ iff $X$ is
non-empty, and\cmmnt{ that} any member of $\MahR(X)$ is either $0$ or
an infinite cardinal.

\leader{531E}{Proposition} Let $X$ be any Hausdorff space.

(a) $\kappa\le w(X)$ for every $\kappa\in\MahR(X)$.

(b) $\MahR(Y)\subseteq\MahR(X)$ for every $Y\subseteq X$.

(c) $\MahR(X)=\bigcup\{\MahR(K):K\subseteq X$ is compact$\}$.

(d) If $X$ is compact and $Y$ is a continuous image of $X$,
$\MahR(Y)\subseteq\MahR(X)$.

(e) $\omega\in\MahR(X)$ iff $X$ has a compact subset which is not
scattered.

(f)\cmmnt{ ({\smc Haydon 77})} If
$\omega\le\kappa'\le\kappa\in\MahR(X)$ then $\kappa'\in\MahR(X)$.

(g) If $Y$ is another Hausdorff space, and neither $X$ nor $Y$ is empty,
then $\MahR(X\times Y)=\MahR(X)\cup\MahR(Y)$;  generally, for any
non-empty finite family $\familyiI{X_i}$ of non-empty Hausdorff spaces,
$\MahR(\prod_{i\in I}X_i)\penalty-100=\bigcup_{i\in I}\MahR(X_i)$.

\proof{{\bf (a)} This is immediate from 531Aa.

\medskip

{\bf (b)} If $\kappa\in\MahR(Y)$, there a \Mth\ Radon probability
measure $\mu$ on $Y$ with Maharam type $\kappa$.   Now $\mu$ has a
(unique) extension to a Radon probability measure $\mu'$ on $X$, setting
$\mu'(Y\setminus X)=0$.    $\mu'$ and $\mu$ have isomorphic measure
algebras, so $\mu'$ is \Mth\ and has Maharam type $\kappa$, and
$\kappa\in\MahR(X)$.

\medskip

{\bf (c)} By (b), $\MahR(K)\subseteq\MahR(X)$ for every compact set
$K\subseteq X$.   In the other direction, if $\kappa\in\MahR(X)$, there
is a \Mth\ Radon probability measure $\mu$ on $X$ with Maharam type
$\kappa$.   Let $K\subseteq X$ be a compact set with $\mu K>0$.   Then
the normalized subspace measure $\mu'=(\mu K)^{-1}\mu_K$ is a Radon
probability measure on $K$, and its measure algebra is isomorphic to a
principal ideal of the measure algebra of $\mu$, so is \Mth\ with
Maharam type $\kappa$.   Accordingly $\kappa\in\MahR(K)$.

\medskip

{\bf (d)} Take $\kappa\in\MahR(Y)$;  then there is a \Mth\ Radon
probability measure $\mu$ on $Y$ with Maharam type $\kappa$.   Let
$f:X\to Y$ be a continuous surjection.   By 418L, there is a Radon
measure $\mu'$ on $X$ such that $f$ is \imp\ for $\mu'$ and $\mu$ and
induces an isomorphism of their measure algebras.   So $\mu'$ witnesses
that $\kappa\in\MahR(X)$.

\medskip

{\bf (e)(i)} If $X$ has a compact subset $K$ which is not scattered,
then there is a continuous
surjection from $K$ onto $[0,1]$ (4A2G(j-iv)).   Of course Lebesgue
measure witnesses that $\omega\in\MahR([0,1])$, so (d) and (b)
tell us that $\omega\in\MahR(K)\subseteq\MahR(X)$.

\medskip

\quad{\bf (ii)} If every compact subset of $X$ is scattered and $\mu$ is
a \Mth\ Radon
probability measure on $X$, let $K$ be a compact set of non-zero measure
and $Z\subseteq K$ a closed self-supporting set.   Then $Z$
has an isolated point $z$ say;  in this case, $\mu\{z\}>0$ so $\{z\}$ is
an atom for $\mu$ and (because $\mu$ is \Mth) the Maharam type of $\mu$
is $0$.   As $\mu$ is arbitrary, $\omega\notin\MahR(X)$.

\medskip

{\bf (f)(i)} Suppose first that $X$ is compact.   Let $\mu$ be a \Mth\
Radon probability measure on $X$ with Maharam type $\kappa$.   Let
$\ofamily{\xi}{\kappa}{E_{\xi}}$ be a stochastically independent family
in $\dom\mu$ with
$\mu E_{\xi}=\bover12$ for every $\xi$.   For each $\xi<\kappa'$ and
$n\in\Bbb N$, let $f_{\xi n}\in C(X)$ be such that
$\int|f_{\xi n}-\chi E_{\xi}|\le 2^{-n}$ (416I).   Define
$f:X\to\BbbR^{\kappa'\times\Bbb N}$ by setting
$f(x)(\xi,n)=f_{\xi n}(x)$ for $x\in X$, $\xi<\kappa'$ and $n\in\Bbb N$.
Then $f$ is continuous, so by 418I the image measure $\nu=\mu f^{-1}$ on
the compact set $f[X]$ is a Radon measure.   For each $\xi<\kappa'$, the
set

\Centerline{$F_{\xi}=\{w:w\in f[X]$, $\lim_{n\to\infty}w(\xi,n)=1\}$}

\noindent is a Borel set, and $f^{-1}[F_{\xi}]\symmdiff E_{\xi}$ is
$\mu$-negligible;  so $\ofamily{\xi}{\kappa'}{F_{\xi}}$ is a
stochastically independent family of subsets of $f[X]$ with measure
$\bover12$.   If $\frak B$ is the measure algebra of $\nu$, and
$\frak C$ the closed subalgebra of $\frak B$ generated by
$\{F_{\xi}^{\ssbullet}:\xi<\kappa'\}$, then $\frak C$ is \Mth, with
Maharam type $\kappa'$;  at the same time,

\Centerline{$\tau(\frak B)\le w(f[X])
\le w(\BbbR^{\kappa'\times\Bbb N})=\kappa'$.}

\noindent By 332N, $\frak B$ can be embedded in $\frak C$;  by 332Q,
$\frak B$ and $\frak C$ are isomorphic, that is, $\frak B$ is \Mth\ with
Maharam type $\kappa'$, and $\nu$ witnesses that
$\kappa'\in\MahR(f[X])$.   By (d), $\kappa'\in\MahR(X)$.

\medskip

\quad{\bf (ii)} In general, (c) tells us that there is a compact set
$K\subseteq X$ such that $\kappa\in\MahR(K)$, so
$\kappa'\in\MahR(K)\subseteq\MahR(X)$.

\medskip

{\bf (g)} Because neither $Y$ nor $X$ is empty, both $X$ and $Y$ are
homeomorphic to subspaces of $X\times Y$, so (b) tells us that
$\MahR(X\times Y)\supseteq\MahR(X)\cup\MahR(Y)$.   In the other
direction, given a \Mth\ Radon probability measure $\mu$ on $X\times Y$,
let $\mu_1$, $\mu_2$ be the marginal measures on $X$ and $Y$
respectively, so that each $\mu_k$ is a Radon probability measure
(418I again).
Let $\familyiI{E_i}$, $\family{j}{J}{F_j}$ be countable partitions of
$X$, $Y$ into Borel sets such that all the subspace measures
$(\mu_1)_{E_i}$ and $(\mu_2)_{F_j}$ are \Mth.   Then
there must be $i\in J$, $j\in J$ such that $\mu(E_i\times F_j)>0$.   Let
$\mu'$ be the subspace measure $\mu_{E_i\times F_j}$;  then the Maharam
type of $\mu'$ is $\kappa$, because $\mu$ is \Mth.   Let $\mu'_1$,
$\mu'_2$ be the marginal measures of $\mu'$ on $E_i$ and $F_j$
respectively.   Then $\mu'_1$ is an indefinite-integral measure over
$(\mu_1)_{E_i}$ (415Oa), so its measure algebra is isomorphic to a
principal ideal of the measure algebra of $(\mu_1)_{E_i}$
(322K), and has the same Maharam type $\kappa_1$ say.
As in (b) above, $\kappa_1\in\MahR(X)$.   Similarly, the Maharam type
$\kappa_2$ of $\mu'_2$ belongs to $\MahR(Y)$.   Now 531C tells us that
$\kappa\le\kappa_1+\kappa_2$.   Since $\kappa$ is either zero or
infinite, it must be less than or equal to at least one of them, and
belongs to $\MahR(X)\cup\MahR(Y)$ by (f) above.

The result for general finite products now follows easily by induction
on $\#(I)$.
}%end of proof of 531E

\leader{531F}{Proposition} Let $X$ be a Hausdorff space.   Then a
totally finite measure algebra $(\frak A,\bar\mu)$ is isomorphic to the
measure algebra of a Radon measure on $X$ iff ($\alpha$) whenever
$\frak A_a$ is a non-trivial homogeneous principal ideal of $\frak A$
then $\tau(\frak A_a)\in\MahR(X)$ ($\beta$) $c(\frak A)\le\#(X)$.

\proof{{\bf (a)} If $\mu$ is a totally finite Radon measure on $X$ with
measure algebra $\frak A$ and the principal ideal $\frak A_a$ generated
by $a\in\frak A\setminus\{0\}$ is homogeneous, then there are an
$E\in\dom\mu$ such that $E^{\ssbullet}=a$ and an $F\subseteq E$ such
that $0<\mu F<\infty$.   Let $\nu$ be the Radon probability measure
$(\mu F)^{-1}\mu\LLcorner F$, that is, $\nu H=\mu(H\cap F)/\mu F$
whenever $H\subseteq X$ is such that $\mu$ measures $H\cap F$.   Then
the measure algebra of $\nu$ is isomorphic to a principal ideal of
$\frak A_a$ so is homogeneous with the same Maharam type, and
$\nu$ witnesses that $\tau(\frak A_a)\in\MahR(X)$.   Thus $\frak A$
satisfies ($\alpha$).   As for ($\beta$), if $X$ is infinite this is
trivial (because $(\frak A,\bar\mu)$ is totally finite, so $\frak A$ is
ccc), and otherwise $\frak A$ is finite, with

\Centerline{$c(\frak A)=\#(\{a:a\in\frak A$ is an atom$\})
=\#(\{x:x\in X$, $\mu\{x\}>0\})\le\#(X)$.}

\medskip

{\bf (b)} Now suppose that $(\frak A,\bar\mu)$ is a  totally finite
measure algebra satisfying the conditions.   Express it as the simple
product of a countable family $\familyiI{(\frak A_i,\bar\mu'_i)}$ of
non-zero homogeneous measure algebras (332B);  we may suppose that
$I\subseteq\Bbb N$.   For $n\in I$, set
$\kappa_n=\tau(\frak A_n)$ and $\gamma_n=\bar\mu'_n1_{\frak A_n}$.
$(\beta$) tells us that $\#(I)\le\#(X)$;  let $\family{n}I{x_n}$ be a
family of distinct elements of $X$.

Set $J=\{n:n\in I$, $\kappa_n\ge\omega\}$.   For each $n\in J$,
($\alpha$) tells us that there is a \Mth\ Radon probability measure
$\mu_n$ on $X$ with Maharam type $\kappa_n$.   Now there is a disjoint
family $\family{n}I{E_n}$ of Borel subsets of $X$ such that $\mu_nE_n>0$
for every $n\in J$.   \Prf\
Choose $\sequencen{E_n}$, $\sequencen{F_n}$ inductively, as follows.
$F_0=X\setminus\{x_n:n\in I\}$.   Given that $F_n$ is a Borel set and
$\mu_jF_n>0$ for every $j\in J\setminus n$, then if $n\notin J$ set
$E_n=\emptyset$
and $F_{n+1}=F_n$.   Otherwise, for each $j\in J$ such that $j>n$, we can
partition $F_n$ into finitely many Borel sets of $\mu_n$-measure less than
$2^{-j}\mu_nF_n$, because $\mu_n$ is atomless;   take one of these,
$G_{nj}$ say, such that $\mu_jG_{nj}>0$;  now set
$F_{n+1}=\bigcup_{j\in J,j>n}G_{nj}$ and $E_n=F_n\setminus F_{n+1}$.
Continue.\ \QeD\   Now set

\Centerline{$\mu E
=\sum_{n\in I\setminus J,x_n\in E}\gamma_n
  +\sum_{n\in J}(\mu_nE_n)^{-1}\gamma_n\mu_n(E\cap E_n)$}

\noindent whenever $E\subseteq X$ is such that $\mu_n$ measures
$E\cap E_n$ for every $n\in J$.   Of course $\mu$ is a measure.
Because every $\mu_n$ is a topological measure, so is $\mu$;  because
every $\mu_n$ is inner regular with respect to the compact sets, so is
$\mu$;  because every $\mu_n$ is complete, so is $\mu$;  thus $\mu$ is a
Radon measure.   Because every subspace measure
$(\mu_n)_{E_n}$ is \Mth\ with Maharam type $\kappa_n$, the measure
algebra of $\mu$ is isomorphic to $(\frak A,\bar\mu)$.
}%end of proof of 531F

\leader{531G}{Proposition} Let $\familyiI{X_i}$ be a family of non-empty
Hausdorff spaces with product $X$.   Then an infinite cardinal $\kappa$
belongs to $\MahR(X)$ iff {\it either}
$\kappa\le\#(\{i:i\in I$, $\#(X_i)\ge 2\})$ {\it or} $\kappa$ is
expressible as $\sup_{i\in I}\kappa_i$ where $\kappa_i\in\MahR(X_i)$ for
every $i\in I$.

\proof{{\bf (a)(i)} Suppose that $\kappa=\sup_{i\in I}\kappa_i$ where
$\kappa_i\in\MahR(X_i)$ for each $i\in I$.   For each $i$, let $\mu_i$
be a \Mth\ Radon probability measure on $X_i$ with Maharam type
$\kappa_i$ and compact support (see the proof of 531Ec).
Let $\lambda$ be the ordinary product of the measures
$\mu_i$.   By 325I, the measure algebra of $\lambda$ can be identified with
the probability
algebra free product of the measure algebras of the $\mu_i$.   It is
therefore isomorphic to the measure algebra of the usual measure on
$\{0,1\}^{\kappa'}$, where $\kappa'$ is the cardinal sum
$\sum_{i\in I}\kappa_i$;  in particular, it is homogeneous with
Maharam type $\kappa'$ (since we are supposing that $\kappa\ge\omega$).
By 417E(ii), the measure algebra of the $\tau$-additive product $\mu$ of
$\familyiI{\mu_i}$ can be identified with the measure algebra of
$\lambda$, while $\mu$ is a Radon measure (417Q).
So $\mu$ witnesses that $\kappa'\in\MahR(X)$;  by
531Ef, $\kappa\in\MahR(X)$.

\medskip

\quad{\bf (ii)} Suppose that $\omega\le\kappa\le\#(I')$ where
$I'=\{i:i\in I$, $\#(X_i)\ge 2\}$.   For $i\in I'$, let $x_i$, $y_i$ be
distinct points of $X_i$ and $\mu_i$ the
point-supported probability measure on $X_i$ such that
$\mu_i\{x_i\}=\mu_i\{y_i\}=\bover12$;  for $i\in I\setminus I'$, let
$\mu_i$ be the unique Radon probability measure on $X_i$.   As in (i)
above, the Radon measure product of $\familyiI{\mu_i}$ is \Mth, with
Maharam type $\#(I')$, so $\#(I')\in\MahR(X)$;  by
531Ef again, $\kappa\in\MahR(X)$.

\medskip

{\bf (b)} Now suppose that $\omega\le\kappa\in\MahR(X)$ and that
$\kappa>\#(I')$.   For each $i\in I$, let $\theta_i$ be the least
cardinal greater than every member of $\MahR(X_i)$.   Note that
$\kappa'\in\MahR(X_i)$ whenever $\omega\le\kappa'<\theta_i$.   Set

\Centerline{$I_1=\{i:i\in I$, $\kappa<\theta_i\}$,
\quad$Z_1=\prod_{i\in I_1}X_i$,}

\Centerline{$I_2=\{i:i\in I$, $\theta_i\le\kappa$,
$\cf\theta_i>\omega\}$,
\quad$Z_2=\prod_{i\in I_2}X_i$,}

\Centerline{$I_3=\{i:i\in I$, $\theta_i=\kappa$, $\cf\theta_i=\omega\}$,
\quad$Z_3=\prod_{i\in I_3}X_i$,}

\Centerline{$I_4=\{i:i\in I$, $\theta_i<\kappa$, $\cf\theta_i=\omega\}$,
\quad$Z_4=\prod_{i\in I_4}X_i$,}

\Centerline{$I_5=\{i:i\in I$, $\theta_i=1$, $\#(X_i)>1\}$,
\quad$Z_5=\prod_{i\in I_5}X_i$,}

\Centerline{$I_6=\{i:i\in I$, $\#(X_i)=1\}$,
\quad$Z_6=\prod_{i\in I_6}X_i$.}

\noindent Then $X$ can be identified with $\prod_{1\le k\le 6}Z_k$, so
531Eg tells us that $\kappa\in\MahR(Z_k)$ for some $k$.   As $Z_6$ is a
singleton, we actually have $\kappa\in\MahR(Z_k)$ for some $k\le 5$.

\medskip

\quad{\bf case 1} Suppose $\kappa\in\MahR(Z_1)$.   Then, in particular,
$I_1\ne\emptyset$ and there is a $j\in I$ such that $\kappa<\theta_j$.
In this case, $\kappa\in\MahR(X_j)$, and we can set $\kappa_j=\kappa$,
$\kappa_i=0$ for $i\ne j$ to find a family in $\prod_{i\in I}\MahR(X_i)$
with supremum $\kappa$.

\medskip

\quad{\bf case 2} Suppose that $\kappa\in\MahR(Z_2)$.   Let $\mu$ be a
Radon probability measure on $Z_2$ with Maharam type $\kappa$.   For
each $i\in Z_2$, let $\mu'_i$ be the marginal measure on $X_i$, and
$\kappa'_i$ its Maharam type.   By 531C,
$\kappa\le\sum_{i\in I_2}\kappa'_i$;  since $I_2\subseteq I'$,
$\kappa>\#(I_2)$;  since $\kappa$ is infinite, it must be less than or
equal to $\sup_{i\in I_2}\max(\omega,\kappa'_i)$.   On the other hand,
by 531F, each $\kappa'_i$ is either finite or the supremum of some
countable subset of $\MahR(X_i)$;  because $\cf\theta_i>\omega$,
$\kappa'_i<\theta_i$ and $\max(\omega,\kappa'_i)\in\MahR(X_i)$.
Setting

$$\eqalign{\kappa_i
&=\med(\kappa'_i,\omega,\kappa)\text{ for }i\in I_2,\cr
&=0\text{ for }i\in I\setminus I_2,\cr}$$

\noindent we have $\kappa_i\in\MahR(X_i)$ for every $i\in I$ and
$\kappa=\sup_{i\in I}\kappa_i$.

\medskip

\quad{\bf case 3} Suppose that $\kappa\in\MahR(Z_3)$.   Because
$\kappa=\theta_i\notin\MahR(X_i)$ for $i\in I_3$, 531Eg tells us that
$I_3$ must be infinite.   Let $\sequencen{i_n}$ be a sequence of
distinct elements of $I_3$.   Of course $\kappa$ itself is uncountable
and has countable cofinality, so we can find a sequence $\kappa'_n$ of
infinite cardinals less than $\kappa$ with supremum $\kappa$.   Setting
$\kappa_{i_n}=\kappa'_n$, $\kappa_i=0$ for
$i\in I\setminus\{i_n:n\in\Bbb N\}$, we have $\kappa_i\in\MahR(X_i)$ for
every $i$ and $\kappa=\sup_{i\in I}\kappa_i$.

\medskip

\quad{\bf case 4} Suppose that $\kappa\in\MahR(Z_4)$.  Following the
scheme of
case 2 above, let $\mu$ be a Radon probability measure on $Z_4$ with
Maharam type $\kappa$, and for each $i\in I_4$ let $\mu'_i$ be the
marginal measure on $X_i$ and $\kappa'_i$ its Maharam type.   Then, as
before, $\kappa\le\sup_{i\in I_4}\max(\omega,\kappa'_i)$.
At the same time,
$\kappa'_i\le\theta_i<\kappa$ for every $i$, so we must have
$\kappa=\sup_{i\in I_4}\theta_i$.   Set $\delta=\cf\kappa$.   Then we
can choose $\ofamily{\xi}{\delta}{i_{\xi}}$ inductively in $I_4$ so that
$\theta_{i_{\eta}}<\theta_{i_{\xi}}$ whenever $\eta<\xi<\delta$ and
$\sup_{\xi<\delta}\theta_{i_{\xi}}=\kappa$.   Now define
$\familyiI{\kappa_i}$ by saying

$$\eqalign{\kappa_{i_{\xi+1}}
&=\theta_{i_{\xi}}\text{ whenever }\xi<\delta,\cr
\kappa_i&=0\text{ if }i\in I\setminus\{i_{\xi+1}:\xi<\delta\}.\cr}$$

\noindent This gives $\kappa_i\in\MahR(X_i)$ for every $i$ and
$\kappa=\sup_{i\in I}\kappa_i$.

\medskip

\quad{\bf case 5} \Quer\ Suppose, if possible, that
$\kappa\in\MahR(Z_5)$.
Once again, we can find a Radon probability measure $\mu$ on $Z_5$ with
Maharam type $\kappa$, and look at its marginal measures $\mu'_i$ for
$i\in I_5$.   This time, however, every $\mu'_i$ must be purely atomic
and has Maharam type $\kappa'_i\le\omega$;  also $\#(I_5)<\kappa$.   So
our formula $\kappa\le\sum_{i\in I_5}\kappa'_i$ becomes $\kappa=\omega$.
In this case $I_5$ must be finite and
$\kappa\in\bigcup_{i\in I_5}\MahR(X_i)=\{0\}$, which is absurd.\ \Bang

Thus this case evaporates and the proof is complete.
}%end of proof of 531G

\cmmnt{
\leader{531H}{Remarks} The results above already enable us to
calculate $\MahR(X)$ for many spaces.   Of course we begin with compact
spaces (531Ec).   If $X$ is compact and Hausdorff, and $[0,1]^{\kappa}$
is a continuous image of $X$, where $\kappa$ is
an infinite cardinal, then $\kappa\in\MahR(X)$ (531Ed);  so if
$[0,1]^{w(X)}$ is a continuous image of $X$, then $\MahR(X)$ is
completely specified, being
$\{0\}\cup\{\kappa:\omega\le\kappa\le w(X)\}$ (531Ea, 531Ef).   Of
course it is not generally true that $w(X)\in\MahR(X)$ (531Xc).   But it
is quite often the case that $[0,1]^{\kappa}$ is a continuous image of
$X$ for every $\kappa\in\MahR(X)$, and I will now investigate this
phenomenon.
}%end of comment

\leader{531I}{Notation}\cmmnt{ For the rest of the section, I will use
the following notation, mostly familiar from earlier chapters of this
volume.}   For
any set $I$, let $\nu_I$ be the usual measure on
$\{0,1\}^I$, $\Tau_I$ its domain, $\Cal N_I$ its
null ideal and $(\frak B_I,\bar\nu_I)$ its measure
algebra.   \cmmnt{In this context,} I will write
$\familyiI{e_i}$ for the standard generating family in $\frak B_I$
(525A).
For $J\subseteq I$ let $\frak C_J$ be the closed subalgebra of
$\frak B_I$ generated by $\{e_i:i\in J\}$.\cmmnt{   Now for a new idea.}
For each $i\in I$, let
$\phi_i:\frak B_I\to\frak B_I$ be the
measure-preserving involution corresponding to reversal of the $i$th
coordinate in $\{0,1\}^I$, that is,
$\phi_i(e_i)=1\Bsetminus e_i$ and
$\phi_i(e_j)=e_j$ for $j\ne i$.

\leader{531J}{Lemma} Let $I$ be a set, and take
$\frak B_I$, $\frak C_J$, for $J\subseteq I$, and
$\phi_i$, for $i\in I$, as in 531I.

(a) $\bigcup\{\frak C_J:J\in[I]^{<\omega}\}$ is dense in
$\frak B_I$ for the measure-algebra topology of
$\frak B_I$.

(b) For every $a\in\frak B_I$, there is a (unique) countable
$J^*(a)\subseteq I$ such that, for $J\subseteq I$,
$a\in\frak C_J$ iff $J\supseteq J^*(a)$.

(c) $\phi_i\phi_j=\phi_j\phi_i$ for all $i$, $j\in I$.

(d) If $J\subseteq I$, $a\in\frak C_J$ and $i\in I$, then
$a\Bcap\phi_ia$, $a\Bcup\phi_ia$ belong to $\frak C_{J\setminus\{i\}}$.

(e) For $a\in\frak B_I$ and $i\in I$ we have $\phi_ia=a$
iff $i\notin J^*(a)$.

(f) $\phi_ia\in\frak C_J$ whenever $J\subseteq I$, $i\in I$
and $a\in\frak C_J$.

\proof{ Let $\familyiI{e_i}$ be the standard generating family in
$\frak B_I$.

\medskip

{\bf (a)} See 254Fe.

\medskip

{\bf (b)} See 254Rd or 325Mb.

\medskip

{\bf (c)} Because $\{e_k:k\in I\}\,\,\tau$-generates
$\frak B_I$, it is enough to check that
$\phi_i\phi_je_k=\phi_j\phi_ie_k$ for all
$i$, $j$, $k\in I$, and this is easy.

\medskip

{\bf (d)} The subalgebra
$\{(c\Bcap e_i)\Bcup(c'\Bsetminus e_i):c$,
$c'\in\frak C_{J\setminus\{i\}}\}$ generated by
$\frak C_{J\setminus\{i\}}\cup\{e_i\}$ is closed (323K), so is the
whole of $\frak C_J$ and contains $a$.   If $c$,
$c'\in\frak C_{J\setminus\{i\}}$ are such that
$a=(c\Bcap e_i)\Bcup(c'\Bsetminus e_i)$, then
$\phi_ia=(c\Bsetminus e_i)\Bcup(c'\Bcap e_i)$ and
$a\Bcap\phi_ia=c\Bcap c'$, $a\Bcup\phi_ia=c\Bcup c'$
belong to $\frak C_{J\setminus\{i\}}$.

\medskip

{\bf (e)} If $i\notin J^*(a)$ then $\phi_ia=a$ because
$\phi_i(e_j)=e_j$ for every $j\ne i$.   If
$\phi_ia=a$ then
$a=a\Bcap\phi_ia\in\frak C_{I\setminus\{i\}}$, by (d), and
$J^*(a)\subseteq I\setminus\{i\}$, that is, $i\notin J^*(a)$.

\medskip

{\bf (f)} $\frak C_J$ is the closed subalgebra of $\frak A$ generated by
$\{e_j:j\in J\}$, so $\phi_i[\frak C_J]$ is the closed
subalgebra generated by
$\{\phi_ie_j:j\in J\}\subseteq\frak C_J$ (324L).
}%end of proof of 531J

\leader{531K}{Lemma} Let $\kappa\ge\omega_2$ be a cardinal, and
$\ofamily{\xi}{\kappa}{e_{\xi}}$ the standard generating family in
$\frak B_{\kappa}$.   Suppose that we are given a family
$\ofamily{\xi}{\kappa}{a_{\xi}}$ in $\frak B_{\kappa}$.
Then there are a set $\Gamma\in[\kappa]^{\kappa}$ and a family
$\ofamily{\xi}{\kappa}{c_{\xi}}$ in $\frak B_{\kappa}$ such that

\Centerline{$c_{\xi}\subseteq a_{\xi}$,
\quad$\bar\nu_{\kappa}c_{\xi}\ge 2\bar\nu_{\kappa}a_{\xi}-1$}

\noindent for every $\xi$, and

\Centerline{$\bar\nu_{\kappa}(\inf_{\xi\in I}c_{\xi}\Bcap e_{\xi}
\Bcap\inf_{\eta\in J}c_{\eta}\Bsetminus e_{\eta})
=\Bover1{2^{\#(I\cup J)}}
  \bar\nu_{\kappa}(\inf_{\xi\in I\cup J}c_{\xi})$}

\noindent whenever $I$, $J\subseteq\Gamma$ are disjoint finite sets.
%for 531L

\proof{ Let $e_{\xi}$, $\phi_{\xi}$, for $\xi<\kappa$, $\frak C_L$, for
$L\subseteq\kappa$, and $J^*(a)$, for $a\in\frak B_{\kappa}$, be as in
531I-531J.   Set $L_{\xi}=J^*(a_{\xi})$ and
$c_{\xi}=a_{\xi}\Bcap\phi_{\xi}a_{\xi}$ for each $\xi$;  then

\Centerline{$\bar\nu_{\kappa}c_{\xi}
=\bar\nu_{\kappa}a_{\xi}+\bar\nu_{\kappa}(\phi_{\xi}a_{\xi})
  -\bar\nu_{\kappa}(a_{\xi}\Bcup\phi_{\xi}a_{\xi})
\ge 2\bar\nu_{\kappa}a_{\xi}-1$}

\noindent and $c_{\xi}\in\frak C_{L_{\xi}\setminus\{\xi\}}$ (531Jd).
By Hajnal's Free Set Theorem (5A1I(a-iii)), there is a set
$\Gamma\in[\kappa]^{\kappa}$ such that $\xi\notin L_{\eta}$ whenever $\xi$,
$\eta$ are distinct members of $\Gamma$.   (This is where we use the
hypothesis that $\kappa\ge\omega_2$.)   Now suppose that $I$,
$J\subseteq\Gamma$ are finite and disjoint.   Then
$(L_{\xi}\setminus\{\xi\})\cap(I\cup J)=\emptyset$,  so
$c_{\xi}\in\frak C_{\kappa\setminus(I\cup J)}$, for every
$\xi\in I\cup J$.   Accordingly $c=\inf_{\xi\in I\cup J}c_{\xi}$ belongs
to $\frak C_{\kappa\setminus(I\cup J)}$.   This means that $c$ and the
$e_{\xi}$, for $\xi\in I\cup J$, are stochastically independent, and

\Centerline{$\bar\nu_{\kappa}(c\Bcap\inf_{\xi\in I}e_{\xi}
  \Bcap\inf_{\eta\in J}(1\Bsetminus e_{\eta}))
=\bar\nu_{\kappa}c\cdot\prod_{\xi\in I}\bar\nu_{\kappa}e_{\xi}
  \cdot\prod_{\eta\in J}\bar\nu_{\kappa}(1\Bsetminus e_{\eta})
=\Bover1{2^{\#(I\cup J)}}\bar\nu_{\kappa}c$,}

\noindent as claimed.
}%end of proof of 531K

\leader{531L}{Theorem} Let $X$ be a normal Hausdorff space.

(a)\cmmnt{ ({\smc Haydon 77})} If $\omega\in\MahR(X)$ then
$[0,1]^{\omega}$ is a continuous image of $X$.

(b)\cmmnt{ ({\smc Haydon 77}, {\smc Plebanek 97})} If
$\kappa\ge\omega_2$ belongs to $\MahR(X)$ and
$\lambda\le\kappa$ is an infinite cardinal such that $(\kappa,\lambda)$
is a measure-precaliber pair of every probability algebra, then
$[0,1]^{\lambda}$ is a continuous image of $X$.

\proof{{\bf (a)} If $\omega\in\MahR(X)$ then $X$ has a compact subset
$K$ which is not scattered (531Ee) and
there is a continuous surjection from $K$ onto $[0,1]$ (4A2G(j-iv) again).
By Tietze's theorem (4A2F(d-ix)),
this has an extension to a continuous map from $X$ to $[0,1]$
which is still a surjection.   As
there is a continuous surjection from $[0,1]$ onto $[0,1]^{\omega}$
(5A4I(b-ii)), there is a continuous surjection from $X$ onto
$[0,1]^{\omega}$.

\medskip

{\bf (b)} Let $\mu$ be a \Mth\ Radon probability measure on $X$ with
Maharam type $\kappa$, $\Sigma$ its domain, and $(\frak A,\bar\mu)$ its
measure algebra, so that $(\frak A,\bar\mu)$ is isomorphic to the
measure algebra
$(\frak B_{\kappa},\bar\nu_{\kappa})$ as discussed in 531I-531K.   Let
$\ofamily{\xi}{\kappa}{e_{\xi}}$ be a stochastically independent
$\tau$-generating set of elements of measure $\bover12$ in $\frak A$, so
that $(\frak A,\ofamily{\xi}{\kappa}{e_{\xi}})$ is isomorphic to
$\frak B_{\kappa}$ with its standard generating family.
For each $\xi<\kappa$, let $E_{\xi}\in\Sigma$ be such that
$E_{\xi}^{\ssbullet}=e_{\xi}$ in $\frak A$.   Let
$K'_{\xi}\subseteq E_{\xi}$, $K''_{\xi}\subseteq X\setminus E_{\xi}$ be
compact sets of measure at least $\bover13$, and set
$K_{\xi}=K'_{\xi}\cup K''_{\xi}$, $a_{\xi}=K_{\xi}^{\ssbullet}$ for
$\xi<\kappa$.   By 531K, copied into $\frak A$,
there are $\ofamily{\xi}{\kappa}{c_{\xi}}$ and
$\Gamma_0\in[\kappa]^{\kappa}$ such that $c_{\xi}\Bsubseteq a_{\xi}$ and
$\bar\mu c_{\xi}\ge\bover13$ for each $\xi$, and

\Centerline{$\bar\nu_{\kappa}(\inf_{\xi\in I}c_{\xi}\Bcap e_{\xi}
  \Bcap\inf_{\eta\in J}c_{\eta}\Bsetminus e_{\eta})
=\Bover1{2^{\#(I\cup J)}}
  \bar\nu_{\kappa}(\inf_{\xi\in I\cup J}c_{\xi})$}

\noindent whenever $I$, $J\subseteq \Gamma_0$ are disjoint finite sets.

At this point, recall that $(\kappa,\lambda)$ is supposed to be a
measure-precaliber pair of every probability algebra.   So there is a
$\Gamma\in[\Gamma_0]^{\lambda}$ such that $\inf_{\xi\in I}c_{\xi}\ne 0$ for
every finite $I\subseteq\Gamma$.   It follows at once that
$\inf_{\xi\in I}a_{\xi}\Bcap e_{\xi}
\Bcap\inf_{\eta\in J}a_{\eta}\Bsetminus e_{\eta}$ is non-zero
for all disjoint finite sets $I$, $J\subseteq\Gamma$.   But this means that
$X\cap\bigcap_{\xi\in I}K'_{\xi}\cap\bigcap_{\eta\in J}K''_{\eta}$ is
non-negligible, therefore non-empty, for all disjoint finite $I$,
$J\subseteq\Gamma$.

Set $K=\bigcap_{\xi\in\Gamma}K_{\xi}$, so that $K\subseteq X$ is
compact.   Then we have a continuous function $f:K\to\{0,1\}^{\Gamma}$
defined by setting

$$\eqalign{f(x)(\xi)&=1\text{ if }x\in K\cap E_{\xi}=K\cap K'_{\xi},\cr
&=0\text{ if }x\in K\setminus E_{\xi}=K\cap K''_{\xi}.\cr}$$

\noindent Now $f$ is surjective.   \Prf\ If $w\in\{0,1\}^{\Gamma}$ and
$L\subseteq\Gamma$ is finite, then

$$\eqalign{F_L
&=\{x:x\in X,\,x\in K'_{\xi}\text{ whenever }\xi\in L
  \text{ and }w(\xi)=1,\cr
&\mskip200mu x\in K''_{\xi}\text{ whenever }\xi\in L
  \text{ and }w(\xi)=0\}\cr}$$

\noindent is a non-empty closed set.   The family
$\{F_L:L\in[\Gamma]^{<\omega}\}$ is downwards-directed, so has non-empty
intersection;  and if $x$ is any point of the intersection, $x\in K$ and
$f(x)=w$.\ \Qed

As $\#(\Gamma)=\lambda$, $\{0,1\}^{\lambda}$ is a continuous image of a
closed subset of $X$ and $[0,1]^{\lambda}$ is a continuous image of $X$
(5A4Fa).
}%end of proof of 531L

\leader{531M}{Proposition}\cmmnt{ ({\smc Plebanek 97})} If $\kappa$ is
an infinite cardinal and $[0,1]^{\kappa}$ is a continuous image
of $X$ whenever $X$ is a compact Hausdorff space such
that $\kappa\in\MahR(X)$, then $\kappa$ is a measure-precaliber of every
probability algebra.

\proof{ It will be enough to show that $\kappa$ is a measure-precaliber
of $(\frak B_{\kappa},\bar\nu_{\kappa})$ (525I(a-i)).   Let
$\ofamily{\xi}{\kappa}{a_{\xi}}$ be a family in $\frak B_{\kappa}$
such that
$\inf_{\xi<\kappa}\bar\nu_{\kappa}a_{\xi}=\alpha>0$.   Choose
$\ofamily{\xi}{\kappa}{b_{\xi}}$ in
$\frak B_{\kappa}$ inductively, as follows.   Given
$\ofamily{\eta}{\xi}{b_{\eta}}$, let $\frak D_{\xi}$ be the closed
subalgebra of $\frak B_{\kappa}$ generated by
$\{b_{\eta}:\eta<\xi\}\cup\{a_{\xi}\}$.   Because
$\frak B_{\kappa}$ is homogeneous with Maharam type
$\kappa>\tau(\frak D_{\xi})$, it is relatively atomless over
$\frak D_{\xi}$, and there is a $b\in\frak B_{\kappa}$ such that
$\bar\nu_{\kappa}(b\Bcap c)=\bover12\bar\nu_{\kappa}c$ for every
$c\in\frak D_{\xi}$ (331B).   Set $b_{\xi}=b\Bcap a_{\xi}$;  then for
any $\eta<\xi$ we have

\Centerline{$\bar\nu_{\kappa}(b_{\xi}\Bsymmdiff b_{\eta})
=\bar\nu_{\kappa}b_{\xi}+\bar\nu_{\kappa}b_{\eta}
  -2\bar\nu_{\kappa}(b_{\xi}\Bcap b_{\eta})
=\Bover12\bar\nu_{\kappa}a_{\xi}+\bar\nu_{\kappa}b_{\eta}
  -\bar\nu_{\kappa}(a_{\xi}\Bcap b_{\eta})
\ge\Bover12\bar\nu_{\kappa}a_{\xi}\ge\Bover{\alpha}2$.}

\noindent Continue.

Let $\frak C$ be the subalgebra of $\frak B_{\kappa}$ generated by
$\{b_{\xi}:\xi<\kappa\}$, and $X$ its Stone space.   Then $\frak C$ is
isomorphic to the algebra of open-and-closed subsets of $X$, so we have
a Radon measure $\mu$ on $X$ defined by saying that
$\mu\widehat{c}=\bar\nu_{\kappa}c$ for every $c\in\frak C$, writing
$\widehat{c}$ for the open-and-closed subset of $X$ corresponding to $c$
(416Qa).   Now $\mu$ is strictly positive and we can identify $\frak C$
with a topologically dense subalgebra of the measure algebra of $\mu$.
It follows that $\mu$ has a \Mth\ component of type at least $\kappa$.
\Prf\Quer\ Otherwise, there would be a set $E\subseteq X$, of measure at
least $1-\bover14\alpha$, such that the Maharam type of the subspace
measure $\mu_E$ was less than $\kappa$.   But

\Centerline{$\mu(E\cap\widehat{b}_{\xi}\symmdiff\widehat{b}_{\eta})
\ge\bar\nu_{\kappa}(b_{\xi}\Bsymmdiff b_{\eta})-\Bover{\alpha}4
\ge\Bover{\alpha}4$}

\noindent whenever $\eta<\xi<\kappa$, so the topological density of the
measure algebra of $\mu_E$ is at least $\kappa$ (5A4B(h-ii))
and the Maharam
type of $\mu_E$ is at least $\kappa$ (521Ea).\ \Bang\QeD\  Thus
$\kappa\in\MahR(X)$.

Accordingly $[0,1]^{\kappa}$ is a continuous image
of $X$.   By 5A4C(d-iii),
there is a non-empty closed subset $F$ of $X$ such
that $\chi(x,F)\ge\kappa$ for every $x\in F$.   Let $D\subseteq\kappa$
be a maximal set such that $\{F\}\cup\{\widehat{b}_{\xi}:\xi\in D\}$ has
the finite intersection property.   Set
$Z=F\cap\bigcap_{\xi\in D}\widehat{b}_{\xi}$;  then $Z$ contains a point
$z$ say.   Because $\{b_{\xi}:\xi\in D\}$ is centered, so is
$\{a_{\xi}:\xi\in D\}$.

If $x\in X\setminus\{z\}$, then there is a $c\in\frak C$ such that
$x\in\widehat{c}$ and $z\notin\widehat{c}$;  accordingly there is a
$\zeta<\kappa$ such that one of $x$, $z$ belongs to
$\widehat{b}_{\zeta}$ and the other does not.   If
$\zeta\in D$ then $z\in\widehat{b}_{\zeta}$ and
$x\notin\widehat{b}_{\zeta}$, so $x\notin Z$.   If $\zeta\notin D$ then,
by the maximality of $D$, $Z\cap\widehat{b}_{\zeta}=\emptyset$, so that
$z\notin\widehat{b}_{\zeta}$, $x\in\widehat{b}_{\zeta}$ and again
$x\notin Z$.

Accordingly $Z=\{z\}$, and $\{z\}$ can be expressed as the intersection of
$\#(D)$ relatively open sets in $F$.   By 4A2Gd, it follows that
$\#(D)\ge\chi(z,F)\ge\kappa$, and we have already seen that
$\{a_{\xi}:\xi\in D\}$ is centered.   As $\ofamily{\xi}{\kappa}{a_{\xi}}$
is arbitrary, $\kappa$ is a measure-precaliber of $\frak B_{\kappa}$, as
required.
}%end of proof of 531M

\leader{531N}{}\cmmnt{ From 531L-531M we see that if $\kappa$ is any
infinite cardinal other than $\omega_1$, then the following are
equiveridical:

\inset{(i) $\kappa$ is a measure-precaliber of every probability
algebra;

(ii) whenever $X$ is a compact Hausdorff space and $\kappa\in\MahR(X)$,
then $[0,1]^{\kappa}$ is a continuous image of $X$.}

\noindent The following variation on the construction in 531M shows that
$\omega_1$ really is a special case.

\medskip

\noindent}{\bf Proposition}\cmmnt{ ({\smc Plebanek 97})} Suppose that
there is a family $\ofamily{\xi}{\omega_1}{W_{\xi}}$ in
$\Cal N_{\omega_1}$ such that every closed subset of
$\{0,1\}^{\omega_1}\setminus\bigcup_{\xi<\omega_1}W_{\xi}$ is scattered.
Then there is a compact Hausdorff space $X$ such that
$\omega_1\in\MahR(X)$ but $[0,1]^{\omega_1}$ is not a continuous image
of $X$.

\proof{{\bf (a)} Set $E_{\xi}=\{z:z\in\{0,1\}^{\omega_1}$, $z(\xi)=1\}$
for each $\xi<\omega_1$.   Choose a family
$\langle K_{\xi n}\rangle_{\xi<\omega_1,n\in\Bbb N}$ of compact sets in
$\{0,1\}^{\omega_1}$ as follows.   Given
$\langle K_{\eta n}\rangle_{\eta<\xi,n\in\Bbb N}$, where $\xi<\omega_1$,
such that $\bigcup_{n\in\Bbb N}K_{\eta n}$ is conegligible for every
$\eta<\xi$, then for each $j\in\Bbb N$ we can find a family
$\ofamily{\eta}{\xi}{n(\xi,\eta,j)}$ in $\Bbb N$ such that
$L_{\xi j}=\bigcap_{\eta<\xi}\bigcup_{i\le n(\xi,\eta,j)}K_{\eta i}$ has
measure at least $1-2^{-j-4}$.   For $j\in\Bbb N$
choose a compact set $K'_{\xi j}\subseteq L_{\xi j}
\setminus(W_{\xi}\cup\bigcup_{i<j}K'_{\xi i})$ of measure at least
$1-2^{-j-3}-\nu_{\omega_1}(\bigcup_{i<j}K'_{\xi i})$.   Set

\Centerline{$K_{\xi,2i}=K'_{\xi i}\cap E_{\xi}$,
\quad$K_{\xi,2i+1}=K'_{\xi i}\setminus E_{\xi}$}

\noindent for each $i\in\Bbb N$, and continue.

\medskip

{\bf (b)} At the end of the induction, let $\frak C$ be the algebra of
subsets of $\{0,1\}^{\omega_1}$ generated by
$\{K_{\xi i}:\xi<\omega_1$, $i\in\Bbb N\}$, and $X$ its Stone space.
Then we have a Radon probability measure $\mu$ on $X$ defined by setting
$\mu\widehat{C}=\nu_{\omega_1}C$ for every $C\in\frak C$, where
$\widehat{C}$ is the open-and-closed subset of $X$ corresponding to $C$.   For
$\eta<\xi<\omega_1$, we have

$$\eqalign{\mu(\widehat{K}_{\eta 0}\symmdiff\widehat{K}_{\xi 0})
&=\nu_{\omega_1}(K_{\eta 0}\symmdiff K_{\xi 0})\cr
&=\nu_{\omega_1}((E_{\eta}\cap K'_{\eta 0})\symmdiff(E_{\xi}\cap K'_{\xi
0}))\cr
&\ge\nu_{\omega_1}(E_{\eta}\symmdiff E_{\xi})
  -\nu_{\omega_1}(E_{\eta}\setminus K'_{\eta 0})
  -\nu_{\omega_1}(E_{\xi}\setminus K'_{\xi 0})\cr
&\ge\Bover12-\Bover18-\Bover18
=\Bover14,\cr}$$

\noindent so the Maharam type of $\mu$ is at least $\omega_1$ and
$\omega_1\in\MahR(X)$.

\medskip

{\bf (c)} Let $F\subseteq X$ be a non-scattered closed set.   Then there
is a $\zeta<\omega_1$ such that
$F\not\subseteq\bigcup_{i\in\Bbb N}\widehat{K}_{\zeta i}$.   \Prf\Quer\
Otherwise, set

\Centerline{$R
=\bigcap_{\xi<\omega_1}\bigcup_{i\in\Bbb N}
  (F\cap\widehat{K}_{\xi i})\times K_{\xi i}
\subseteq X\times\{0,1\}^{\omega_1}$.}

\noindent Note that for each $\xi<\omega_1$ the $\widehat{K}_{\xi i}$
are disjoint open-and-compact sets covering the compact set $F$, so
$\{i:F\cap\widehat{K}_{\xi i}\ne\emptyset\}$ is finite and
$\bigcup_{i\in\Bbb N}(F\cap\widehat{K}_{\xi i})\times K_{\xi i}$ is
compact;  thus $R$ is compact.   If $(x,z)$ and $(x',z')\in R$ and
$x\ne x'$, there must be some $C\in\frak C$ such that $x\in\widehat{C}$
and $x'\notin\widehat{C}$, so there must be some $\xi<\omega_1$ and
$i\in\Bbb N$ such that
just one of $x$, $x'$ belongs to $\widehat{K}_{\xi i}$;  in this case,
only the corresponding one of $z$, $z'$ can belong to $K_{\xi i}$, and
$z\ne z'$.

Conversely, if $(x,z)$ and $(x',z')\in R$ and $z\ne z'$, there is some
$\xi$ such that $z(\xi)\ne z'(\xi)$.   In this case, if $i$,
$j\in\Bbb N$ are such that $(x,z)\in\widehat{K}_{\xi i}\times K_{\xi i}$
and $(x',z')\in\widehat{K}_{\xi j}\times K_{\xi j}$, $i\ne j$ and
$x\ne x'$.

This shows that $R$ is the graph of a bijection from $F$ to $R[F]$.
Because $R$ is a compact subset of $F\times R[F]$, it is a
homeomorphism, and $R[F]$ is not scattered.   But, for each
$\xi<\omega_1$,
$R[F]\subseteq\bigcup_{i\in\Bbb N}K_{\xi i}$ is disjoint from $W_{\xi}$;
and all compact subsets of
$\{0,1\}^{\omega_1}\setminus\bigcup_{\xi<\omega_1}W_{\xi}$ are supposed
to be scattered.\ \Bang\Qed

\medskip

{\bf (d)} Take
$x\in F\setminus\bigcup_{i\in\Bbb N}\widehat{K}_{\zeta i}$.   Then
$\chi(x,X)\le\omega$.   \Prf\ Consider the set

\Centerline{$V=\bigcap_{\eta\le\zeta,i\in\Bbb N}\{x':x'\in X$,
$x'\in\widehat{K}_{\eta i}\iff x\in\widehat{K}_{\eta i}\}$.}

\noindent This is a G$_{\delta}$ set containing $x$.   \Quer\ If there
is an $x'\in V\setminus\{x\}$, there must be some $\xi<\omega_1$ and
$j\in\Bbb N$ such that just one of $x$, $x'$ belongs to
$\widehat{K}_{\xi j}$.   In this case, $\xi>\zeta$, so
$K_{\xi j}\subseteq\bigcup_{i\le k}K_{\zeta i}$ and
$\widehat{K}_{\xi j}\subseteq\bigcup_{i\le k}\widehat{K}_{\zeta i}$ for
some $k\in\Bbb N$.
But neither $x$ nor $x'$ belongs to
$\bigcup_{i\le k}\widehat{K}_{\zeta i}$.\ \BanG\   Thus $V=\{x\}$;  by
4A2Gd again, $\chi(x,X)\le\omega$.\ \Qed

\medskip

{\bf (e)} Thus we see that whenever $F\subseteq X$ is a non-scattered
closed set, there is an $x\in F$ such that $\chi(x,X)$ is countable.
By 5A4C(d-iii), $[0,1]^{\omega_1}$ is not a continuous image of $X$.
}%end of proof of 531N

\leader{531O}{}\cmmnt{ In 531M we have a space $X$ from which there is
no surjection onto $[0,1]^{\kappa}$ because every non-empty closed set
has a point of character less than $\kappa$.   From stronger properties
of $\kappa$ we can get compact spaces with stronger topological
properties, as in the next two results.

\medskip

\noindent}{\bf Proposition} Let $\kappa$, $\kappa'$ and $\lambda$ be
infinite cardinals such that $(\kappa,\kappa')$ is not a
measure-precaliber pair of
$(\frak B_{\lambda},\bar\nu_{\lambda})$.   Then there is a compact
Hausdorff space $X$ such that $\kappa\in\MahR(X)$ and
$\chi(x,X)<\max(\kappa',\lambda^+)$ for every $x\in X$.

\proof{ Let
$\ofamily{\xi}{\kappa}{a_{\xi}}$ be a family in $\frak B_{\lambda}$,
with no centered subfamily with cardinal $\kappa'$, such that
$\inf_{\xi<\kappa}\bar\mu a_{\xi}=\alpha>0$.   Let
$\psi:\frak B_{\lambda}\to\Tau_{\lambda}$ be a lifting;  for each
$\xi<\kappa$, let $K_{\xi}\subseteq\psi a_{\xi}$ be a compact set of
measure at least $\bover12\alpha$.   If $D\subseteq\kappa$ and
$\#(D)=\kappa'$, then there is a finite set $I\subseteq D$ such that
$\inf_{\xi\in I}a_{\xi}=0$, in which case
$\bigcap_{\xi\in I}K_{\xi}
\subseteq\bigcap_{\xi\in I}\psi a_{\xi}=\emptyset$.   Thus
$\{\xi:x\in K_{\xi}\}$ has cardinal less than $\kappa'$ for every
$x\in\{0,1\}^{\lambda}$.

Set

\Centerline{$X=\bigcap_{\xi<\kappa'}\{(x,y):x\in\{0,1\}^{\lambda}$,
$y\in\{0,1\}^{\kappa}$, $x\in K_{\xi}$ or $y(\xi)=0\}$,}

\noindent so that $X$ is a compact subset of
$\{0,1\}^{\lambda}\times\{0,1\}^{\kappa}$.   Now
$\chi((x,y),X)<\max(\kappa',\lambda^+)$ for every $(x,y)\in X$.   \Prf\
Set $D=\{\xi:\xi<\kappa$, $x\in K_{\xi}\}$, so that $\#(D)<\kappa'$.
For $I\in[\lambda]^{<\omega}$ and $J\in[D]^{<\omega}$ set

\Centerline{$V_{IJ}
=\{(x',y'):(x',y')\in X$, $x'\restr I=x\restr I$,
$y'\restr J=y\restr J\}$,}

\noindent so that
$\Cal V=\{V_{IJ}:I\in[\lambda]^{<\omega}$, $J\in[D]^{<\omega}\}$ is a
downwards-directed family of closed neighbourhoods of $(x,y)$.   If
$(x',y')\in\bigcap\Cal V$, then $x'=x$, so $x'\notin K_{\xi}$ for
$\xi\in\kappa\setminus D$, and $y'(\xi)=y(\xi)=0$ for $\xi\notin D$;
also $y'\restr D=y\restr D$, so $(x',y')=(x,y)$.   Thus
$\bigcap\Cal V=\{(x,y)\}$;  by 4A2Gd once more, $\Cal V$ is a base of
neighbourhoods of $(x,y)$, and

\Centerline{$\chi((x,y),X)\le\#(\Cal V)\le\max(\#(D),\lambda)
<\max(\kappa',\lambda^+)$. \Qed}

Define $g:\{0,1\}^{\lambda}\times\{0,1\}^{\kappa}\to\{0,1\}^{\kappa}$
and $h:\{0,1\}^{\lambda}\times\{0,1\}^{\kappa}\to X$ by setting

$$\eqalign{g(x,y)(\xi)&=y(\xi)\text{ if }x\in K_{\xi},\cr
&=0\text{ otherwise},\cr
h(x,y)&=(x,g(x,y)),\cr}$$

\noindent for $\xi<\kappa$, $x\in\{0,1\}^{\lambda}$ and
$y\in\{0,1\}^{\kappa}$.   Write $\Sigma$ for the domain of the product
measure $\nu=\nu_{\lambda}\times\nu_{\kappa}$ on
$\{0,1\}^{\lambda}\times\{0,1\}^{\kappa}$.   Then the
$\sigma$-algebra
$\{F:F\subseteq X$, $h^{-1}[F]\in\Sigma\}$ contains all sets of the form
$\{(x,y):x(\eta)=1\}$ and $\{(x,y):y(\xi)=1\}$, so includes a base for
the topology of $X$ and therefore contains every open-and-closed set.
Accordingly we have an additive functional
$U\mapsto\nu h^{-1}[U]$ on the algebra of open-and-closed subsets of
$X$, which extends to a Radon probability measure $\mu$ on $X$
(416Qa again).
Set $F_{\xi}=\{(x,y):(x,y)\in X$, $y(\xi)=1\}$ for each $\xi<\kappa$;
then for any $\eta<\xi<\kappa$,

$$\eqalign{\mu(F_{\xi}\setminus F_{\eta})
&=\nu h^{-1}[F_{\xi}\setminus F_{\eta}]\cr
&\ge\nu\{(x,y):x\in K_{\xi},\,y(\xi)=1,\,y(\eta)=0\}
=\Bover14\nu_{\lambda}K_{\xi}
\ge\Bover18\alpha.\cr}$$

\noindent Once again, this shows that the measure algebra of $\mu$ must
have a homogeneous principal ideal with Maharam type at least $\kappa$,
and $\kappa\in\Mah(X)$.
}%end of proof of 531O

\leader{531P}{}\cmmnt{ Putting these ideas together with 531L, we come to
the following.

\medskip

\noindent}{\bf Proposition}\cmmnt{ ({\smc Kunen \& Mill 95},
{\smc Plebanek 95})}
Let $\kappa$ be a regular
%\query:  is this needed? OK if \cf\kappa\ge\omega_2
%the problem seems to be in (b)not(i)=>not(ii)
%what about  \kappa=\omega_{\omega} ?  using measure-precaliber, of course
infinite cardinal.   Then the following are equiveridical:

(i) $\kappa$ is a measure-precaliber of every measurable algebra;

(ii) if $X$ is a compact Hausdorff space such that $\kappa\in\MahR(X)$,
then $\chi(x,X)\ge\kappa$ for some $x\in X$.

\proof{{\bf (a)} Consider first the case $\kappa\ge\omega_2$.

\medskip

\quad{\bf (i)$\Rightarrow$(ii)} If $\kappa\in\MahR(X)$, then
$[0,1]^{\kappa}$ is a continuous image of $X$, by 531Lb.   By
5A4C(d-iii) again and 5A4Bb, it follows at once that
$\chi(x,X)\ge\kappa$ for many points $x\in X$.

\medskip

\quad{\bf not-(i)$\Rightarrow$not-(ii)} By
525Ib there is a $\lambda<\kappa$ such that $\kappa$ is not a precaliber
of $\frak B_{\lambda}$.   Now 531O tells us that there is a compact
Hausdorff space $X$ such that $\kappa\in\MahR(X)$ and
$\chi(x,X)<\max(\kappa,\lambda^+)=\kappa$ for every $x\in X$.

\medskip

{\bf (b)} Now suppose that $\kappa=\omega_1$.

\medskip

\quad{\bf (i)$\Rightarrow$(ii)}
\Quer\ Suppose, if possible, that $\omega_1$ is a precaliber of every
probability algebra, but that there is a first-countable compact
Hausdorff space $X$ with $\omega_1\in\MahR(X)$.   Let $\mu$ be a \Mth\
Radon probability measure on $X$ with Maharam type $\omega_1$, and
$(\frak A,\bar\mu)$ its measure algebra;  let
$\ofamily{\xi}{\omega_1}{c_{\xi}}$ be a $\tau$-generating stochastically
independent family of elements of measure $\bover12$ in $\frak A$.   As
in 531J, there is for each $a\in\frak A$ a countable
$J^*(a)\subseteq\omega_1$ such that $a$ belongs to the closed subalgebra
of $\frak A$ generated by $\{c_{\xi}:\xi\in J^*(a)\}$.

For each $x\in X$, let $\Cal U_x$ be a countable base of open
neighbourhoods of $x$, and set
$A_x=\{U^{\ssbullet}:U\in\Cal U_x\}$,
$J^{\dagger}(x)=\bigcup_{a\in A_x}J^*(a)$.   Then $J^{\dagger}(x)$ is
countable.   For $\xi<\omega_1$, set
$D_{\xi}=\{x:J^{\dagger}(x)\subseteq\xi\}$;  then
$\ofamily{\xi}{\omega_1}{D_{\xi}}$ is a non-decreasing family with union
$X$.   Now $\omega_1$ is supposed to be a
precaliber of $\frak A$, so there must be a $\xi<\omega_1$ such that
$D_{\xi}$ has full outer measure (525Cc).

Let $G\subseteq X$ be open.  Then $G^{\ssbullet}$ belongs to the closed
subalgebra $\frak C_{\xi}$ of $\frak A$ generated by
$\{c_{\eta}:\eta<\xi\}$.   \Prf\ For each $x\in G\cap D_{\xi}$, there is
a $U_x\in\Cal U_x$ such that $U_x\subseteq G$.   Set
$H=\bigcup\{U_x:x\in G\cap D_{\xi}\}$, so that $H\subseteq G$ is open and
$G\cap D_{\xi}=H\cap D_{\xi}$;  as $D_{\xi}$ has full outer measure,
$G\setminus H$ is negligible and $H^{\ssbullet}=G^{\ssbullet}$.   But
414Aa once more tells us that
$H^{\ssbullet}=\sup_{x\in D_{\xi}}U_x^{\ssbullet}$,
and this belongs to $\frak D_{\xi}$, because
$J^{\dagger}(x)\subseteq\xi$ for every $x\in D_{\xi}$.\ \Qed

It follows at once that $F^{\ssbullet}\in\frak C_{\xi}$ for every closed
$F\subseteq X$.   Because $\mu$ is inner regular with respect to the
closed sets, $\frak C_{\xi}$ is order-dense in $\frak A$ and
$\frak A=\frak C_{\xi}$ has Maharam type $\#(\xi)<\omega_1$.\ \Bang

Thus (i)$\Rightarrow$(ii).

\medskip

\quad{\bf not-(i)$\Rightarrow$not-(ii)} Suppose that (i) is false.

\medskip

\qquad\grheada\ By 525J,
$\cov\Cal N_{\omega_1}=\omega_1$ and there is
a family $\ofamily{\xi}{\omega_1}{A_{\xi}}$ of negligible
subsets of $\{0,1\}^{\omega_1}$ covering $\{0,1\}^{\omega_1}$.
For each $\xi<\omega_1$, let $A'_{\xi}\supseteq A_{\xi}$ be a
negligible set determined by
coordinates in a countable set $J_{\xi}\subseteq\omega_1$;  set
$\tilde A_{\xi}=\bigcup\{A'_{\eta}:\eta<\xi$, $J_{\eta}\subseteq\xi\}$;
then $\tilde A_{\xi}$ is determined by coordinates less than $\xi$.
Set $H_{\xi}=\{y\restr\xi:y\in\tilde A_{\xi}\}$, so that
$H_{\xi}$ is a $\nu_{\xi}$-negligible subset of $\{0,1\}^{\xi}$.

We see that
$\ofamily{\xi}{\omega_1}{\tilde A_{\xi}}$ is non-decreasing, and

\Centerline{$\bigcup_{\xi<\omega_1}\tilde A_{\xi}
=\bigcup_{\xi<\omega_1}A'_{\xi}=\{0,1\}^{\omega_1}$.}

\noindent Consequently $y\restr\xi\in H_{\xi}$ whenever
$\eta\le\xi<\omega_1$, $y\in\{0,1\}^{\omega_1}$ and
$y\restr\eta\in H_{\eta}$, while for every
$y\in\{0,1\}^{\omega_1}$ there is a $\xi<\omega_1$ such that
$y\restr\xi\in H_{\xi}$.

\medskip

\qquad\grheadb\ Set $Y=\{0\}\cup\{2^{-n}:n\in\Bbb N\}\subseteq[0,1]$.
For $\xi\le\omega_1$ define $\phi_{\xi}:Y^{\xi}\to\{0,1\}^{\xi}$ by setting

$$\eqalign{\phi_{\xi}(x)(\eta)
&=0\text{ if }\eta<\xi\text{ and }x(\eta)=0,\cr
&=1\text{ for other }\eta<\xi.\cr}$$

\noindent Observe that $\phi_{\xi}$ is Borel measurable for every
$\xi<\omega_1$.   Choose $\langle X_{\xi}\rangle_{\xi<\omega_1}$,
and $\langle K_{\xi n}\rangle_{\xi<\omega_1,n\in\Bbb N}$
inductively, as follows.   The inductive hypothesis will be that $X_{\xi}$
is a compact subset of $Y^{\xi}$, $\phi_{\xi}[X_{\xi}]$ is conegligible
in $\{0,1\}^{\xi}$, $\phi_{\xi}\restr X_{\xi}$ is injective
and $x\restr\eta\in X_{\eta}$ whenever $x\in X_{\xi}$
and $\eta\le\xi<\omega_1$.

Start with
$X_0=Y^0=\{\emptyset\}$ and $\phi_0:X_0\to\{0,1\}^0$ the identity map.

Given $\xi<\omega_1$ and $X_{\xi}\subseteq Y^{\xi}$,
then 433D tells us that there is a Radon measure $\mu_{\xi}$ on
$X_{\xi}$ such that $\nu_{\xi}$ is the image measure
$\mu_{\xi}\phi_{\xi}^{-1}$.   Let $\sequencen{K_{\xi n}}$ be a
disjoint sequence of compact subsets of
$X_{\xi}\setminus\phi_{\xi}^{-1}[H_{\xi}]$ with $\mu_{\xi}$-conegligible
union.   Set

$$\eqalign{X_{\xi+1}
&=\{x:x\in Y^{\xi+1},\,x\restr\xi\in X_{\xi},\,x(\xi)=0\}\cr
&\mskip150mu
\cup\bigcup_{n\in\Bbb N}\{x:x\in Y^{\xi+1},\,x\restr\xi\in K_{\xi n},\,
x(\xi)=2^{-n}\}.\cr}$$

\noindent It is easy to see that $X_{\xi+1}$ is compact and
$\phi_{\xi+1}\restr X_{\xi+1}$ is injective, while surely
$x\restr\eta\in X_{\eta}$ whenever $x\in X_{\xi+1}$ and $\eta\le\xi+1$,
just because $x\restr\xi\in X_{\xi}$.   Also

\Centerline{$\phi_{\xi+1}[X_{\xi+1}]
\supseteq\{y:y\in\{0,1\}^{\xi+1}$,
  $y\restr\xi\in\bigcup_{n\in\Bbb N}\phi_{\xi}[K_{\xi n}]\}$}

\noindent is conegligible for $\nu_{\xi+1}$ because
$\phi_{\xi}[K_{\xi n}]$ must be analytic for every $n$ and

\Centerline{$\nu_{\xi}(\bigcup_{n\in\Bbb N}\phi_{\xi}[K_{\xi n}])
=\mu_{\xi}(\bigcup_{n\in\Bbb N}K_{\xi n})=1$}

\noindent because $\phi_{\xi}\restr X_{\xi}$ is injective.

Given that $X_{\eta}$ has been defined for $\eta<\xi$,
where $\xi<\omega_1$ is a non-zero limit ordinal, set

\Centerline{$X_{\xi}=\{x:x\in Y^{\xi}$, $x\restr\eta\in X_{\eta}$ for
every $\eta<\xi\}$.}

\noindent Of course $X_{\xi}$ is compact and $\phi_{\xi}\restr X_{\xi}$ is
injective.    To see that $\phi_{\xi}[X_{\xi}]$ is conegligible, observe
that

\Centerline{$W
=\bigcap_{\eta<\xi}\{y:y\in\{0,1\}^{\xi}$,
$y\restr\eta\in\phi_{\eta}[X_{\eta}]\}$}

\noindent is conegligible.   But if $y\in W$ and we choose
$x_{\eta}\in X_{\eta}$ such that $\phi_{\eta}(x_{\eta})=y\restr\eta$ for
each $\eta<\xi$, then we must have $x_{\zeta}=x_{\eta}\restr\zeta$ whenever
$\zeta\le\eta<\xi$, because $\phi_{\zeta}\restr X_{\zeta}$ is injective;
so there is an $x\in Y^{\xi}$ such that $x_{\eta}=x\restr\eta$ for every
$\eta<\xi$, in which case $x\in X_{\xi}$ and $\phi_{\xi}(x)=y$.   Thus
$\phi_{\xi}[X_{\xi}]\supseteq W$ is conegligible.

\medskip

\qquad\grheadc\ At the end of the induction, set

\Centerline{$X=\{x:x\in Y^{\omega_1}$,
$x\restr\xi\in X_{\xi}$ for every $\xi<\omega_1\}$,
\quad$\phi=\phi_{\omega_1}\restr X$.}

\noindent As in the limit stage of the construction in ($\beta$), we see
that $X$ is a closed subset of $Y^{\omega_1}$, so with the subspace
topology is a zero-dimensional compact Hausdorff space.
This time, we do not
expect that $\phi[X]$ should be conegligible in $\{0,1\}^{\omega_1}$, but
we find that it has full outer measure.   \Prf\ If
$K\subseteq\{0,1\}^{\omega_1}$ is a
non-negligible closed G$_{\delta}$ set, there is a $\xi<\omega_1$
such that $K$ is determined by coordinates less than $\xi$.   Set
$K'=\{y\restr\xi:y\in K\}$;  then $\nu_{\xi}K'=\nu_{\omega_1}K>0$,
so there is an $x_0\in X_{\xi}$ such that $\phi_{\xi}(x_0)\in K'$.
Extending $x_0$ to $x\in Y^{\omega_1}$ by setting $x(\eta)=0$ for
$\xi\le\eta<\omega_1$, we see by induction on $\zeta$ that
$x\restr\zeta\in X_{\zeta}$ for $\xi\le\zeta<\omega_1$, so
$x\in X$;  also $\phi(x)\restr\xi=\phi_{\xi}(x_0)\in K'$, so
$\phi(x)\in K$ and $K$ meets $\phi[X]$.   As $\nu_{\omega_1}$
is completion regular, $\phi[X]$ has full outer measure.\ \Qed

\medskip

\qquad\grheadd\ $X$ is first-countable.   \Prf\ If $x\in X$, $\xi<\omega_1$
and $x(\xi)\ne 0$, then $x\restr(\xi+1)$ belongs to $X_{\xi+1}$, and there
must be some $n\in\Bbb N$ such that $x(\xi)=2^{-n}$ and
$x\restr\xi\in K_{\xi n}$;  in which case
$\phi_{\xi}(x\restr\xi)\notin H_{\xi}$.   Now take any $x\in X$.
Then there is a $\xi<\omega_1$ such that $\phi(x)\in\tilde A_{\xi}$
and $\phi_{\xi}(x)=\phi(x)\restr\xi$ belongs to $H_{\xi}$.   In this case,
$V=\{x':x'\in X$, $x'\restr\xi=x\restr\xi\}$ is a G$_{\delta}$ subset of
$X$ containing $x$.   But if $x'\in V$ then, for any $\eta\ge\xi$,
$\phi_{\eta}(x'\restr\eta)\in H_{\eta}$ and $x'(\eta)=0$.   Thus $V=\{x\}$.
By 4A2Gd, as usual, $x$ has a countable base of neighbourhoods in $X$;
as $x$ is arbitrary, $X$ is first-countable.\ \Qed

\medskip

\qquad\grheade\  By 234F, there is a measure $\lambda$ on
$X$ such that $\phi$ is \imp\ for $\lambda$ and $\nu_{\omega_1}$.
Of course $\lambda$ is a probability measure.   Now for
any $\xi<\omega_1$ and $n\in\Bbb N$,

\Centerline{$\{x:x\in X$, $x(\xi)=0\}=\{x:\phi(x)(\xi)=0\}$,}

$$\eqalign{\{x:x\in X,\,x(\xi)=2^{-n}\}
&=\{x:\phi(x)(\xi)=1,\,x\restr\xi\in K_{\xi n}\}\cr
&=\{x:\phi(x)(\xi)=1,\,\phi_{\xi}(x\restr\xi)\in\phi_{\xi}[K_{\xi n}]\}\cr
&=\{x:\phi(x)(\xi)=1,\,\phi(x)\restr\xi\in\phi_{\xi}[K_{\xi n}]\}\cr}$$

\noindent are measured by $\lambda$.   So the domain of $\lambda$ includes
a base for the topology of the zero-dimensional compact Hausdorff space
$X$.   By 416Qa,
there is a Radon measure $\mu$ on $X$ agreeing with $\lambda$ on the
open-and-closed subsets of $X$;  by the Monotone Class Theorem
(136C), $\mu$ and
$\lambda$ agree on the $\sigma$-algebra generated by the open-and-closed
sets, that is, the Baire $\sigma$-algebra of $X$ (4A3Oe).
In particular, setting
$E_{\xi}=\{x:x\in X$, $x(\xi)=0\}$ for $\xi<\omega_1$,

$$\eqalign{\mu(E_{\xi}\cap E_{\eta})
=\lambda(E_{\xi}\cap E_{\eta})
&=\nu_{\omega_1}\{y:y\in\{0,1\}^{\omega_1},\,y(\xi)=y(\eta)=0\}\cr
&=\Bover12\text{ if }\xi=\eta<\omega_1,\cr
&=\Bover14\text{ if }\xi,\,\eta<\omega_1\text{ are different.}\cr}$$

\noindent It follows that $\mu(E_{\xi}\symmdiff E_{\eta})=\bover12$ for all
distinct $\xi$, $\eta<\omega_1$, so $\mu$ has uncountable Maharam type and
$\omega_1\in\MahR(X)$.   Thus $X$ and $\mu$ witness that (ii) is false.

\medskip

{\bf (c)} Finally, if $\kappa=\omega$, both (i) and (ii) are true for
elementary reasons (525Fa).

}%end of proof of 531P

\leader{531Q}{}\cmmnt{ In 531P we see that if $\omega_1$ is not a
precaliber of every measurable algebra then there is a first-countable
compact Hausdorff space with a Radon measure with Maharam type $\omega_1$.
With a sharper hypothesis, and rather more work, we can get a
substantially stronger version, as follows.

\medskip

\noindent}{\bf Proposition} Suppose that
$\cf\Cal N_{\omega}=\omega_1$.   Then there is a hereditarily separable
perfectly normal compact Hausdorff space $X$ such that
$\omega_1\in\MahR(X)$.

\proof{ For $\eta\le\xi\le\omega_1$ and $x\in\{0,1\}^{\xi}$, set
$\pi_{\xi\eta}(x)=x\restr\eta$;  write $\pi_{\eta}$ for
$\pi_{\omega_1\eta}:\{0,1\}^{\omega_1}\to\{0,1\}^{\eta}$.

\medskip

{\bf (a)} It will be helpful to have the following fact out in the open.
Let $Y$ be a
zero-dimensional metrizable compact Hausdorff space, $\mu$ an atomless
Radon probability measure on $Y$, $A\subseteq Y$ a $\mu$-negligible set
and $\Cal Q$ a countable family of closed subsets of $Y$.   Then there
are closed sets
$K$, $L\subseteq Y$, with union $Y$, such that

\inset{$K\cup L=Y$, $K\cap L\cap A=\emptyset$, $\mu(K\cap
L)\ge\bover12$,

$K\cap Q=\overline{Q\setminus L}$
and $L\cap Q=\overline{Q\setminus K}$ for every $Q\in\Cal Q$.}

\noindent\Prf\ We can of course suppose that $\Cal Q$ is non-empty.
For each $Q\in\Cal Q$ let $D_Q$ be a countable dense subset of $Q$;  let
$S\subseteq Y\setminus(A\cup\bigcup_{Q\in\Cal Q}D_Q)$ be a closed set of
measure at least $\bover12$.
(This is where we need to know that $\mu$ is atomless, so that every
$D_Q$ is negligible.)
Let $\Cal U$ be a countable base for the topology of $Y$ consisting of
open-and-closed sets and
let $\sequencen{(U_n,Q_n)}$ run over $\Cal U\times\Cal Q$.   Choose
inductively sequences
$\sequencen{G_n}$, $\sequencen{H_n}$ of open-and-closed subsets of
$Y\setminus S$, as follows.
Start with $G_0=H_0=\emptyset$.   Given that $G_n$ and $H_n$ are
disjoint from each other and from
$S$, then

\inset{----- if $U_n\cap S=\emptyset$, take
$G_{n+1}=G_n\cup(U_n\setminus H_n)$ and
$H_{n+1}=H_n$;

----- if $U_n\cap S\cap Q_n\ne\emptyset$,
$S\cap Q_n$ is nowhere dense in $Q_n$ (because it is closed and
does not meet
$D_{Q_n}$), while $(G_n\cup H_n)\cap Q_n$ is a relatively closed set in
$Q_n$ not meeting $S$, therefore not including $Q_n\cap U_n$;  so
$Q_n\cap U_n\setminus(S\cup G_n\cup H_n)$ must have at least two points
$y$, $y'$ say,
and we can enlarge $G_n$ and $H_n$ to disjoint open-and-closed subsets
$G_{n+1}$, $H_{n+1}$ of $Y\setminus S$
containing $y$, $y'$ respectively, and therefore both meeting
$U_n\cap Q_n$;

----- otherwise, take $G_{n+1}=G_n$ and $H_{n+1}=H_n$.}

At the end of the induction, set $G=\bigcup_{n\in\Bbb N}G_n$ and
$H=\bigcup_{n\in\Bbb N}H_n$, so
that $G$ and $H$ are disjoint open subsets of $Y\setminus S$.   Now if
$y$ is any point of
$Y\setminus S$, there must be some $n$ such that $y\in U_n\subseteq
Y\setminus S$, so that
$y\in G_{n+1}\cup H_n$;  thus $Y=G\cup H\cup S$.   Set $K=G\cup
S=Y\setminus H$,
$L=H\cup S=Y\setminus G$;  then $K$ and $L$ are closed sets with union
$Y$, and $K\cap L=S$ has measure at least $\bover12$ and
is disjoint from $A$.

If $Q\in\Cal Q$, $y\in S\cap Q$ and $U$ is any neighbourhood
of $y$, there is an
$n\in\Bbb N$ such that $Q_n=Q$ and $y\in U_n\subseteq U$.   In this
case, $U_n\cap S\cap Q_n\ne\emptyset$ and $G_{n+1}$ and $H_{n+1}$ meet
$U_n\cap Q$.   As $U$ is arbitrary,

\Centerline{$y\in\overline{Q\cap G}\cap\overline{Q\cap H}
=\overline{Q\setminus L}\cap\overline{Q\setminus K}$.}

\noindent Thus $K\cap L\cap Q
\subseteq\overline{Q\setminus L}\cap\overline{Q\setminus K}$,
so $K\cap Q\subseteq\overline{Q\setminus L}$ and
$L\cap Q\subseteq\overline{Q\setminus K}$.
On the other hand, $K\supseteq Q\setminus L$ and $L\supseteq Q\setminus K$,
so $K\cap Q=\overline{Q\setminus L}$ and $L\cap Q=\overline{Q\setminus K}$.
Thus $K$ and $L$ fulfil all the specifications.\ \Qed

\wheader{531Q}{6}{2}{2}{36pt}

{\bf (b)} Choose

\Centerline{$\langle f_{\xi}\rangle_{\omega\le\xi\le\omega_1}$,
$\langle\mu_{\xi}\rangle_{\omega\le\xi<\omega_1}$,
$\langle K_{\xi}\rangle_{\omega\le\xi<\omega_1}$,
$\langle L_{\xi}\rangle_{\omega\le\xi<\omega_1}$,
$\langle X_{\xi}\rangle_{\omega\le\xi\le\omega_1}$,}

\Centerline{$\langle
Q'_{\zeta\xi}\rangle_{\omega\le\xi\le\zeta<\omega_1}$,
$\langle Q_{\xi\delta}\rangle_{\omega\le\delta\le\xi<\omega_1}$,
$\langle
Q_{\xi\delta\eta}\rangle_{\omega\le\eta\le\delta\le\xi<\omega_1}$,
$\langle A_{\zeta\xi}\rangle_{\omega\le\xi\le\zeta<\omega_1}$,
$\langle A_{\xi}\rangle_{\omega\le\xi<\omega_1}$}

\noindent as follows.
$f_{\omega}(x)=x$ for every $x\in\{0,1\}^{\omega}=X_{\omega}$.
Given that $\omega\le\xi<\omega_1$ and
$f_{\xi}$ is a Borel measurable surjection from $\{0,1\}^{\xi}$ onto a
compact subset $X_{\xi}$ of $\{0,1\}^{\xi}$, let $\mu_{\xi}$ be
the image measure $\nu_{\xi}f_{\xi}^{-1}$, so that $\mu_{\xi}$ is a
Radon measure on $\{0,1\}^{\xi}$ (433E, 418I);  of course
$\mu_{\xi}X_{\xi}=1$.  Because $\cf\Cal N_{\omega}=\omega_1$,
$\mu_{\xi}$ is inner regular with respect to a family of size at most
$\omega_1$ (524Pb),
which we may suppose to consist of closed sets;  let
$\langle Q'_{\zeta\xi}\rangle_{\xi\le\zeta<\omega_1}$ run over such a
family.   Similarly, there is a family
$\langle A_{\zeta\xi}\rangle_{\xi\le\zeta<\omega_1}$ running over a
cofinal subset of the null ideal of $\mu_{\xi}$.
Next, for $\omega\le\delta\le\xi$, let
$Q_{\xi\delta}\subseteq\pi_{\xi\delta}^{-1}[Q'_{\xi\delta}]$ be a
compact
$\mu_{\xi}$-self-supporting set of the same $\mu_{\xi}$-measure as
$\pi_{\xi\delta}^{-1}[Q'_{\xi\delta}]$.
Note that, as $\mu_{\xi}X_{\xi}=1$, every $Q_{\xi\delta}$ must be
included in $X_{\xi}$.   Set
$Q_{\xi\delta\eta}=X_{\xi}\cap\pi_{\xi\delta}^{-1}[Q_{\delta\eta}]$
for $\omega\le\eta\le\delta\le\xi$.   Set

\Centerline{$\Cal A_{\xi}
=\{\pi_{\xi\eta}^{-1}[A_{\delta\eta}]:\omega\le\eta\le\delta\le\xi\}$,
\quad$A_{\xi}=\bigcup\{A:A\in\Cal A_{\xi}$, $\mu_{\xi}A=0\}$;}

\noindent because $\Cal A_{\xi}$ is countable, $A_{\xi}$ is
$\mu_{\xi}$-negligible.   By (a) above, we can find
closed sets $K_{\xi}$, $L_{\xi}$ covering $\{0,1\}^{\xi}$ such that
$\mu_{\xi}(K_{\xi}\cap L_{\xi})\ge\bover12$, $K_{\xi}\cap L_{\xi}\cap
A_{\xi}=\emptyset$,
and $K_{\xi}\cap Q_{\xi\delta\eta}=\overline{Q_{\xi\delta\eta}\setminus
L_{\xi}}$,
$L_{\xi}\cap Q_{\xi\delta\eta}=\overline{Q_{\xi\delta\eta}\setminus
K_{\xi}}$ whenever
$\omega\le\eta\le\delta\le\xi$.

Given that $\omega<\xi\le\omega_1$ and that
$K_{\eta}$, $L_{\eta}$ are closed subsets of $\{0,1\}^{\eta}$ covering
$\{0,1\}^{\eta}$
for $\omega\le\eta<\xi$, define $f_{\xi}(x)(\eta)$ inductively, for
$x\in\{0,1\}^{\xi}$ and
$\eta<\xi$, by setting

$$\eqalign{f_{\xi}(x)(\eta)
&=x(\eta)\text{ if }\eta<\omega\text{ or }\eta\ge\omega\text{ and }
  x\restr\eta\in K_{\eta}\cap L_{\eta},\cr
&=1\text{ if }\eta\ge\omega\text{ and }x\restr\eta\notin L_{\eta},\cr
&=0\text{ if }\eta\ge\omega\text{ and }x\restr\eta\notin K_{\eta}.\cr}$$

\noindent Then $f_{\xi}:\{0,1\}^{\xi}\to\{0,1\}^{\xi}$ is Baire
measurable.   Set

\Centerline{$X_{\xi}=\bigcap_{\omega\le\eta<\xi}\{x:x\in\{0,1\}^{\xi}$,
$x(\eta)=1$ or $x\restr\eta\in L_{\eta}$, $x(\eta)=0$ or
$x\restr\eta\in K_{\eta}\}$;}

\noindent then $X_{\xi}\subseteq\{0,1\}^{\xi}$ is compact,
$f_{\xi}(x)\in X_{\xi}$ for every
$x\in\{0,1\}^{\xi}$, and $f_{\xi}(x)=x$ for every $x\in X_{\xi}$.
So $f_{\xi}[\{0,1\}^{\xi}]=X_{\xi}$ and (if $\xi<\omega_1$) the process
continues.

\medskip

{\bf (c)} At the end of the induction, write $f$ for $f_{\omega_1}$ and
$X$ for $X_{\omega_1}$.
If $z\in\{0,1\}^{\omega_1}$ and $\xi<\omega_1$, the formula for
$f_{\xi}$ in (b) shows that
$f(z)(\eta)=f_{\xi}(z\restr\xi)(\eta)$ for every $\eta<\xi$, that is,
that $f(z)\restr\xi=f_{\xi}(z\restr\xi)$.   Next, because $f$ is Baire
measurable, we have a Radon
measure $\mu$ on $\{0,1\}^{\omega_1}$ defined by saying that
$\mu V=\nu_{\omega_1}f^{-1}[V]$ for every Baire set
$V\subseteq\{0,1\}^{\omega_1}$ (432F);  of course $\mu V=0$ for every
open-and-closed set $V$ disjoint from $X$, so $\mu X=1$.

\medskip

{\bf (d)} A couple of simple facts.   I have already observed that
$\pi_{\xi}f=f_{\xi}\pi_{\xi}$ for $\omega\le\xi<\omega_1$;  consequently

\Centerline{$X_{\xi}=f_{\xi}[\{0,1\}^{\xi}]
=f_{\xi}[\pi_{\xi}[\{0,1\}^{\omega_1}]]
=\pi_{\xi}[f[\{0,1\}^{\omega_1}]]=\pi_{\xi}[X]$}

\noindent and

$$\eqalign{\mu_{\xi}V
&=\nu_{\xi}f_{\xi}^{-1}[V]
=(\nu_{\omega_1}\pi_{\xi}^{-1})f_{\xi}^{-1}[V]
=\nu_{\omega_1}(f_{\xi}\pi_{\xi})^{-1}[V]\cr
&=\nu_{\omega_1}(\pi_{\xi}f)^{-1}[V]
=(\nu_{\omega_1}f^{-1})\pi_{\xi}^{-1}[V]
=\mu\pi_{\xi}^{-1}[V]\cr}$$

\noindent for every open-and-closed set $V\subseteq\{0,1\}^{\xi}$.
Thus the Radon measures
$\mu\pi_{\xi}^{-1}$ and $\mu_{\xi}$ are identical.   Equally, if
$\omega\le\eta\le\xi<\omega_1$,

\Centerline{$X_{\eta}=\pi_{\eta}[X]=\pi_{\xi\eta}[\pi_{\xi}[X]]=\pi_{\xi
\eta}[X_{\xi}]$}

\noindent and

\Centerline{$\mu_{\eta}
=\mu\pi_{\eta}^{-1}=\mu(\pi_{\xi\eta}\pi_{\xi})^{-1}
=\mu_{\xi}\pi_{\xi\eta}^{-1}$.}

\noindent Accordingly, if $\omega\le\eta\le\delta\le\xi<\omega_1$,

\Centerline{$\mu_{\xi}\pi_{\xi\eta}^{-1}[A_{\delta\eta}]
=\mu_{\eta}A_{\delta\eta}=0$.}

\noindent Thus in fact $A_{\xi}=\bigcup\Cal A_{\xi}$ and $K_{\xi}\cap
L_{\xi}$ is disjoint from
$\pi_{\xi\eta}^{-1}[A_{\delta\eta}]$ whenever
$\omega\le\eta\le\delta\le\xi$.

\medskip

{\bf (e)} We come now to the first key idea of this construction.
If $\omega\le\eta\le\delta\le\xi<\omega_1$, then
$g_{\xi\delta\eta}=\pi_{\xi\delta}\restr Q_{\xi\delta\eta}$ is an
irreducible surjection onto $Q_{\delta\eta}$.   \Prf\ First, let us
confirm that if
$\delta\le\zeta\le\xi$ then

$$\eqalignno{\pi_{\xi\zeta}[Q_{\xi\delta\eta}]
&=\pi_{\xi\zeta}[X_{\xi}\cap\pi_{\xi\delta}^{-1}[Q_{\delta\eta}]]
=\pi_{\xi\zeta}[X_{\xi}
  \cap\pi_{\xi\zeta}^{-1}[\pi_{\zeta\delta}^{-1}[Q_{\delta\eta}]]]\cr
&=\pi_{\xi\zeta}[X_{\xi}]\cap\pi_{\zeta\delta}^{-1}[Q_{\delta\eta}]
=X_{\zeta}\cap\pi_{\zeta\delta}^{-1}[Q_{\delta\eta}]
=Q_{\zeta\delta\eta}.&(*)\cr}$$

\noindent In particular,

\Centerline{$\pi_{\xi\delta}[Q_{\xi\delta\eta}]=Q_{\delta\delta\eta}
=X_{\delta}\cap\pi_{\delta\delta}^{-1}[Q_{\delta\eta}]=Q_{\delta\eta}$.}

\noindent To see that
$g_{\xi\delta\eta}$ is irreducible, induce on $\xi$.   At the start,
$g_{\delta\delta\eta}$ is an identity function, so is certainly
irreducible.   For the inductive step to $\xi+1$, given that
$\delta\le\xi<\omega_1$ and $g_{\xi\delta\eta}$ is irreducible, consider
$h=\pi_{\xi+1,\xi}\restr Q_{\xi+1,\delta,\eta}$.   By (*),
$h[Q_{\xi+1,\delta,\eta}]=Q_{\xi\delta\eta}$.   Note that $X_{\xi+1}$
can be identified with

\Centerline{$\{(x,1):x\in X_{\xi}\cap K_{\xi}\}\cup\{(x,0):x\in
X_{\xi}\cap L_{\xi}\}$.}

\noindent If
$V\subseteq\{0,1\}^{\xi+1}$ is a cylinder set meeting
$Q_{\xi+1,\delta,\eta}$, then
$V'=\pi_{\xi+1,\xi}[V]$ is a cylinder set in $\{0,1\}^{\xi}$ meeting
$Q_{\xi\delta\eta}$.
Because

\Centerline{$K_{\xi}\cap
Q_{\xi\delta\eta}=\overline{Q_{\xi\delta\eta}\setminus L_{\xi}}$,
\quad$L_{\xi}\cap Q_{\xi\delta\eta}=\overline{Q_{\xi\delta\eta}\setminus
K_{\xi}}$,
\quad$K_{\xi}\cup L_{\xi}\supseteq Q_{\xi\delta\eta}$,}

\noindent $V'$ must meet both $Q_{\xi\delta\eta}\setminus L_{\xi}$ and
$Q_{\xi\delta\eta}\setminus K_{\xi}$.   Now $V$ must cover at least one
of

$$\eqalign{\{x:x\in\{0,1\}^{\xi+1},\,x\restr\xi\in V',\,x(\xi)=1\}
&\supseteq\{x:x\in X_{\xi+1},\,x\restr\xi\in V'\setminus L_{\xi}\}\cr
&\supseteq\{x:x\in Q_{\xi+1,\delta,\eta},\,
   x\restr\xi\in V'\cap Q_{\xi\delta\eta}\setminus L_{\xi}\},\cr}$$

$$\eqalign{\{x:x\in\{0,1\}^{\xi+1},\,x\restr\xi\in V',\,x(\xi)=0\}
&\supseteq\{x:x\in X_{\xi+1},\,x\restr\xi\in V'\setminus K_{\xi}\}\cr
&\supseteq\{x:x\in Q_{\xi+1,\delta,\eta},\,
   x\restr\xi\in V'\cap Q_{\xi\delta\eta}\setminus K_{\xi}\},\cr}$$

\noindent and $h[Q_{\xi+1,\delta,\eta}\setminus V]$ is disjoint from at
least one
of the non-empty sets $V'\cap Q_{\xi\delta\eta}\setminus L_{\xi}$,
$V'\cap Q_{\xi\delta\eta}\setminus K_{\xi}$.   As $V$ is arbitrary, $h$
is irreducible, and the composition
$g_{\xi\delta\eta}h$ is irreducible (5A4C(d-iv));  but this is
$g_{\xi+1,\delta,\eta}$.

For the inductive step to a limit ordinal $\xi$ such that
$\delta<\xi<\omega_1$, again take a
cylinder set $V\subseteq\{0,1\}^{\xi}$ meeting $Q_{\xi\delta\eta}$.
This time, because $\xi$ is a limit ordinal, there is a $\zeta$ such
that $\delta\le\zeta<\xi$ and $V$ is determined by coordinates less than
$\zeta$.   Set $V'=\pi_{\xi\zeta}[V]$;  then $V'$ is a cylinder set in
$\{0,1\}^{\zeta}$ meeting
$\pi_{\xi\zeta}[Q_{\xi\delta\eta}]=Q_{\zeta\delta\eta}$.   Now

$$\eqalign{\pi_{\xi\delta}[Q_{\xi\delta\eta}\setminus V]
&=\pi_{\zeta\delta}[\pi_{\xi\zeta}
  [Q_{\xi\delta\eta}\setminus\pi_{\xi\zeta}^{-1}[V']]]\cr
&=\pi_{\zeta\delta}[\pi_{\xi\zeta}[Q_{\xi\delta\eta}]\setminus V']
=\pi_{\zeta\delta}[Q_{\zeta\delta\eta}\setminus V']
\subset Q_{\delta\eta}\cr}$$

\noindent because
$g_{\zeta\delta\eta}:Q_{\zeta\delta\eta}\to Q_{\delta\eta}$ is irreducible.
Thus the induction continues.\ \Qed

\medskip

{\bf (f)} Repeating the last step of the argument in (e) with
$\xi=\omega_1$, we see that in fact
$\pi_{\delta}\restr X\cap\pi_{\delta}^{-1}[Q_{\delta\eta}]$ is an
irreducible surjection onto $Q_{\delta\eta}$
whenever $\omega\le\eta\le\delta<\omega_1$.   The next point to confirm
is that if $z$, $z'\in X$,
$\omega\le\eta\le\delta<\omega_1$, $z\restr\delta=z'\restr\delta$ and
$z\restr\eta\in A_{\delta\eta}$,
then $z'=z$.  \Prf\ Suppose that $\delta\le\xi<\omega_1$ and
$z\restr\xi=z'\restr\xi$.   Then
$K_{\xi}\cap L_{\xi}$ does not meet
$\pi_{\xi\eta}^{-1}[A_{\delta\eta}]$, so does not contain
$z\restr\xi$.   Accordingly

\Centerline{$z(\xi)=1
\Longrightarrow z\restr\xi\in K_{\xi}
\Longrightarrow z\restr\xi\notin L_{\xi}
\Longrightarrow z'\restr\xi\notin L_{\xi}
\Longrightarrow z'(\xi)=1$,}

\noindent and similarly $z(\xi)=0\Rightarrow z'(\xi)=0$.   So an easy
induction on $\xi$ shows that $z(\xi)=z'(\xi)$ whenever
$\delta\le\xi<\omega_1$, and $z=z'$.\ \Qed

\medskip

{\bf (g)} It follows that if $H\subseteq X$ is closed, there is a
$\xi<\omega_1$ such that
$H=X\cap\pi_{\xi}^{-1}[\pi_{\xi}[H]]$ and $\pi_{\xi}\restr H$ is
irreducible.
\Prf\ For $\omega\le\eta\le\xi<\omega_1$,

\Centerline{$\mu_{\eta}\pi_{\eta}[H]
=\mu_{\xi}\pi_{\xi\eta}^{-1}[\pi_{\eta}[H]]
=\mu_{\xi}\pi_{\xi\eta}^{-1}[\pi_{\xi\eta}[\pi_{\xi}[H]]]
\ge\mu_{\xi}\pi_{\xi}[H]$.}

\noindent So we have an $\eta<\omega_1$ such that
$\mu_{\eta}\pi_{\eta}[H]=\mu_{\xi}\pi_{\xi}[H]$ whenever
$\eta\le\xi<\omega_1$.
Now recall that $\mu_{\eta}$ is inner regular with respect to
$\{Q'_{\delta\eta}:\eta\le\delta<\omega_1\}$.   So there is a countable
set
$I\subseteq\omega_1\setminus\eta$ such that
$\family{\delta}I{Q'_{\delta\eta}}$ is disjoint,
$Q'_{\delta\eta}\subseteq\pi_{\eta}[H]$ for every $\delta\in I$ and
$\sum_{\delta\in I}\mu_{\eta}Q'_{\delta\eta}=\mu\pi_{\eta}[H]$.

For each $\delta\in I$,

$$\eqalignno{\mu_{\delta}(Q_{\delta\eta}\setminus\pi_{\delta}[H])
&\le\mu_{\delta}
  (\pi_{\delta\eta}^{-1}[Q'_{\delta\eta}]\setminus\pi_{\delta}[H])\cr
&\le\mu_{\delta}
  (\pi_{\delta\eta}^{-1}[\pi_{\eta}[H]]\setminus\pi_{\delta}[H])\cr
&=\mu_{\delta}(\pi_{\delta\eta}^{-1}[\pi_{\eta}[H]])
 -\mu_{\delta}\pi_{\delta}[H]\cr
\displaycause{because surely
$\pi_{\delta}[H]\subseteq\pi_{\delta\eta}^{-1}[\pi_{\eta}[H]]$}
&=\mu_{\eta}\pi_{\eta}[H]-\mu_{\delta}\pi_{\delta}[H]
=0\cr}$$

\noindent by the choice of $\eta$.   Because $Q_{\delta\eta}$ was
$\mu_{\delta}$-self-supporting,
and $\pi_{\delta}[H]$ is closed,
$Q_{\delta\eta}\subseteq\pi_{\delta}[H]$.   Because
$\pi_{\delta}\restr X\cap\pi_{\delta}^{-1}[Q_{\delta\eta}]$ is
irreducible,
$X\cap\pi_{\delta}^{-1}[Q_{\delta\eta}]\subseteq H$.

Set $\zeta=\sup(I\cup\{\eta\})<\omega_1$.   Since
$Q_{\delta\eta}\subseteq\pi^{-1}_{\delta\eta}[Q'_{\delta\eta}]$,
$\pi_{\zeta\delta}^{-1}[Q_{\delta\eta}]
\subseteq\pi_{\zeta\eta}^{-1}[Q'_{\delta\eta}]$ for
each $\delta\in I$;   as $\family{\delta}I{Q'_{\delta\eta}}$ is
disjoint, so is
$\family{\delta}I{\pi_{\zeta\delta}^{-1}[Q_{\delta\eta}]}$;  and

$$\eqalign{\sum_{\delta\in I}
  \mu_{\zeta}\pi_{\zeta\delta}^{-1}[Q_{\delta\eta}]
&=\sum_{\delta\in I}\mu_{\delta}Q_{\delta\eta}
=\sum_{\delta\in I}\mu_{\delta}\pi_{\delta\eta}^{-1}[Q'_{\delta\eta}]\cr
&=\sum_{\delta\in I}\mu_{\eta}Q'_{\delta\eta}
=\mu_{\eta}\pi_{\eta}[H]
=\mu_{\zeta}\pi_{\zeta}[H].\cr}$$

\noindent Because $X\cap\pi_{\delta}^{-1}[Q_{\delta\eta}]\subseteq H$,
$\pi_{\zeta}[H]\supseteq X_{\zeta}
  \cap\pi_{\zeta\delta}^{-1}[Q_{\delta\eta}]$
%$=Q_{\zeta\delta\eta}$
for every  $\delta\in I$.   So
$\pi_{\zeta}[H]\setminus\bigcup_{\delta\in I}
  \pi_{\zeta\delta}^{-1}[Q_{\delta\eta}]$
is $\mu_{\zeta}$-negligible and is included in $A_{\xi\zeta}$ for some
$\xi\ge\zeta$.
Repeating the arguments of the last two sentences at the new level, we
see that

\Centerline{$X_{\xi}\cap\bigcup_{\delta\in I}
  \pi_{\xi\delta}^{-1}[Q_{\delta\eta}]
\subseteq\pi_{\xi}[H]
\subseteq\bigcup_{\delta\in I}\pi_{\xi\delta}^{-1}[Q_{\delta\eta}]
  \cup\pi_{\xi\zeta}^{-1}A_{\xi\zeta}$.}

Now suppose that $V\subseteq\{0,1\}^{\omega_1}$ is an open set meeting
$H$.   If there is a
$z$ belonging to
$V\cap H\cap\pi_{\zeta}^{-1}[A_{\xi\zeta}]$, then $z\restr\xi\ne
z'\restr\xi$ for any other
$z'\in X$, by (f);  so $z\restr\xi\notin\pi_{\xi}[H\setminus V]$ and
$\pi_{\xi}[H\setminus V]\ne\pi_{\xi}[H]$.
Otherwise, there is a $\delta\in I$ such that
$V\cap\pi_{\delta}^{-1}[Q_{\delta\eta}]$ is not empty.   Because
$\pi_{\delta}\restr X\cap\pi_{\delta}^{-1}[Q_{\delta\eta}]$ is
irreducible, $\pi_{\delta}[H\setminus V]$ cannot cover
$Q_{\delta\eta}\subseteq\pi_{\delta}[H]$.
Thus $\pi_{\delta}[H\setminus V]\ne\pi_{\delta}[H]$;  it follows at once
that $\pi_{\xi}[H\setminus V]\ne\pi_{\xi}[H]$.   As $V$ is arbitrary,
$\pi_{\xi}\restr H$ is
irreducible.

I have still to check that $H=X\cap\pi_{\xi}^{-1}[\pi_{\xi}[H]]$.   If
$z$, $z'\in X$, $z\in H$ and $z'\restr\xi=z\restr\xi$, then if
$z\in\pi_{\zeta}^{-1}[A_{\xi\zeta}]$ we have
$z'=z\in H$.   Otherwise, there is some $\delta\in I$ such that
$z\in\pi_{\delta}^{-1}[Q_{\delta\eta}]$.   In this case,
$z'\restr\delta=z\restr\delta\in Q_{\delta\eta}$;   but
$X\cap\pi_{\delta}^{-1}[Q_{\delta\eta}]\subseteq H$, so again $z'\in H$.
So we have the
result.\ \Qed

\medskip

{\bf (h)} We are within sight of the end.   From (g) we see, first, that
if $H\subseteq X$ is closed then it is of the form
$X\cap\pi_{\eta}^{-1}[\pi_{\eta}[H]]$ for some $\eta$, so is a
zero set in $X$;  accordingly $X$ is perfectly normal, therefore
first-countable (5A4Cb).   Second, for any closed $H\subseteq X$,
there is an irreducible continuous surjection from $H$ onto a compact
metrizable space
$\pi_{\xi}[H]$;  because $\pi_{\xi}[H]$ is separable, so is $H$
(5A4C(d-i)).   It follows that
$X$ is hereditarily separable.   \Prf\ If $A\subseteq X$, then
$\overline{A}$ is separable; let
$D\subseteq\overline{A}$ be a countable dense set.   Because $X$ is
first-countable, each member of $\overline{A}$ is in the closure of a
countable subset of $A$, and there is a countable
$C\subseteq A$ such that $D\subseteq\overline{C}$.   Now $C$ is a
countable dense subset of $A$.\ \Qed

\medskip

{\bf (i)} Finally, we need to check that $\omega_1\in\MahR(X)$.   For
$\omega\le\xi<\omega_1$, set
$U_{\xi}=\{z:z\in\{0,1\}^{\omega_1}$,
$z\restr\xi\in K_{\xi}\cap L_{\xi}$, $z(\xi)=1\}$.   Then
$\mu(U_{\xi}\symmdiff E)\ge\bover14$ whenever
$E\subseteq\{0,1\}^{\omega_1}$ is a Baire set determined by
coordinates less than $\xi$.   \Prf\ Set $E'=\pi_{\xi}[E]$, so that
$E=\pi_{\xi}^{-1}[E']$ and $E'$ is a Baire set.   Then

$$\eqalign{\mu(U_{\xi}\setminus E)
&=\nu_{\omega_1}f^{-1}[U_{\xi}\setminus E]
=\nu_{\omega_1}\{z:f(z)\restr\xi\in K_{\xi}\cap L_{\xi}\setminus
E',\,f(z)(\xi)=1\}\cr
&=\nu_{\omega_1}\{z:f_{\xi}(z\restr\xi)\in K_{\xi}\cap L_{\xi}\setminus
E',\,z(\xi)=1\}\cr
&=\Bover12\nu_{\omega_1}\{z:f_{\xi}(z\restr\xi)\in K_{\xi}\cap
L_{\xi}\setminus E'\}
=\Bover12\mu_{\xi}(K_{\xi}\cap L_{\xi}\setminus E'),\cr}$$

\noindent while

$$\eqalign{\mu(E\setminus U_{\xi})
&=\nu_{\omega_1}f^{-1}[E\setminus U_{\xi}]
\ge\nu_{\omega_1}f^{-1}[E\cap\pi_{\xi}^{-1}[K_{\xi}\cap
L_{\xi}]\setminus U_{\xi}]\cr
&=\nu_{\omega_1}\{z:f(z)\restr\xi\in K_{\xi}\cap L_{\xi}\cap
E',\,f(z)(\xi)=0\}\cr
&=\nu_{\omega_1}\{z:f_{\xi}(z\restr\xi)\in K_{\xi}\cap L_{\xi}\cap
E',\,z(\xi)=0\}\cr
&=\Bover12\nu_{\omega_1}\{z:f_{\xi}(z\restr\xi)\in K_{\xi}\cap
L_{\xi}\cap E'\}
=\Bover12\mu_{\xi}(K_{\xi}\cap L_{\xi}\cap E').\cr}$$

\noindent Putting these together,

\Centerline{$\mu(E\symmdiff U_{\xi})\ge\Bover12\mu_{\xi}(K_{\xi}\cap
L_{\xi})\ge\Bover14$.
\Qed}

In particular, $\mu(U_{\eta}\symmdiff U_{\xi})\ge\bover14$ whenever
$\omega\le\eta<\xi<\omega_1$.
So $\mu$ has uncountable Maharam type.   As $\mu X=1$, so has the
subspace measure on $X$, and $\omega_1\in\MahR(X)$.   Thus $X$ has all
the properties announced.
}%end of proof of 531Q

\leader{531R}{}\cmmnt{ Returning to the ideas of 531J, we have the
following construction.

\wheader{531R}{4}{2}{2}{108pt}

\noindent}{\bf Lemma} Let $I$ be a set, and let
$\frak B_{I}$, $\familyiI{e_i}$,
$\familyiI{\phi_i}$,
$\langle\frak C_J\rangle_{J\subseteq I}$ and
$J^*:\frak B_{I}\to[I]^{\le\omega}$ be as in 531I-531J.   For
$a\in\frak B_{I}$ and $J\subseteq I$, set
$S_J(a)=\upr(a,\frak C_J)=\min\{c:a\Bsubseteq c\in\frak C_J\}$\cmmnt{,
the upper envelope of $a$ in $\frak C_J$ (313S)}.

(a) For all $a\in\frak B_{I}$, $i\in I$ and $J$,
$K\subseteq I$,

\quad(i) $S_{I}(a)=a$,

\quad(ii) $S_K(a)\Bsubseteq S_J(a)$ if $J\subseteq K$,

\quad(iii) $J^*S_J(a)\subseteq J^*(a)\cap J$,

\quad(iv) $S_{I\setminus\{i\}}(a)=a\Bcup\phi_ia$,

\quad(v) $S_JS_K(a)=S_{J\cap K}(a)$.

(b) Whenever $a\in\frak B_{I}$, $\epsilon>0$ and $m\in\Bbb N$,
there is a finite $L\subseteq I$ such that
$\bar\nu_{I}(S_J(a)\Bsetminus a)\le\epsilon$ whenever
$L\subseteq J\subseteq I$ and $\#(I\setminus J)\le m$.

\proof{{\bf (a)(i)} $\frak C_{I}=\frak B_{I}$ contains $a$.

\medskip

\quad{\bf (ii)} If $J\subseteq K$ then
$\frak C_K\supseteq\frak C_J$ contains $S_J(a)$.

\medskip

\quad{\bf (iii)} If $i\in I\setminus(J^*(a)\cap J)$ then
$S_J(a)\in\frak C_J$ so $\phi_iS_J(a)\in\frak C_J$, by 531Jf.
Also $\phi_iS_J(a)\Bsupseteq a$.  \Prf\ If $ i\notin J$ then
$\phi_iS_J(a)=S_J(a)\Bsupseteq a$, by 531Je.
If $ i\notin J^*(a)$ then
$\phi_iS_J(a)\Bsupseteq\phi_ia=a$.\ \QeD\
So $\phi_iS_J(a)\Bsupseteq S_J(a)$;  but they have the same measure,
so $\phi_iS_J(a)=S_J(a)$.
As $ i$ is arbitrary, $J^*S_J(a)\subseteq J^*(a)\cap J$,
by 531Je in the other direction.

\medskip

\quad{\bf (iv)} By 531Je again, $S_{I\setminus\{i\}}(a)
=\phi_iS_{I\setminus\{i\}}(a)\Bsupseteq\phi_ia$;  so
$S_{I\setminus\{i\}}(a)\Bsupseteq a\Bcup\phi_ia$.
On the other hand, by
531Jd, $a\Bcup\phi_i(a)$ belongs to
$\frak C_{I\setminus\{i\}}$,
so includes $S_{I\setminus\{i\}}(a)$.

\medskip

\quad{\bf (v)} By (iii),

\Centerline{$J^*S_JS_K(a)\subseteq J^*S_K(a)\cap J
\subseteq J^*(a)\cap K\cap J$,}

\noindent and $S_JS_K(a)\in\frak C_{J\cap K}$;  since
also $S_JS_K(a)\Bsupseteq S_K(a)\Bsupseteq a$,
$S_JS_K(a)\Bsupseteq S_{J\cap K}(a)$.   On the other hand,
$S_{J\cap K}(a)$ belongs to $\frak C_J$ and includes
$S_K(a)$, so includes $S_JS_K(a)$.

\medskip

{\bf (b)} Induce on $m$.   For $m=0$ the result is immediate from
(a-i).   For the
inductive step to $m+1$, take $L_0\in[I]^{<\omega}$ such that
$\bar\nu_{I}(S_K(a)\Bsetminus a)\le\bover13\epsilon$ whenever
$L_0\subseteq K$ and $\#(I\setminus K)\le m$.
By 531Ja, there are a finite
set $L_1\subseteq I$ and a $b\in\frak C_{L_1}$ such that
$\bar\nu_{I}(a\Bsymmdiff b)\le\bover13\epsilon$;  set
$L=L_0\cup L_1$.   Suppose $L\subseteq J$ and
$\#(I\setminus J)\le m+1$;  take
$ i\in I\setminus J$ and set $K=J\cup\{i\}$.   Then

\Centerline{$S_J(a)=S_{I\setminus\{i\}}S_K(a)
=S_K(a)\Bcup\phi_iS_K(a)$}

\noindent by (a-v) and (a-iv).   So

$$\eqalignno{\bar\nu_{I}(S_J(a)\Bsetminus a)
&\le\bar\nu_{I}(S_K(a)\Bsetminus a)
  +\bar\nu_{I}(\phi_iS_K(a)\Bsetminus a)\cr
&\le\Bover{\epsilon}3
  +\bar\nu_{I}\phi_i(S_K(a)\Bsetminus a)
  +\bar\nu_{I}(\phi_ia\Bsetminus a)\cr
&\le\Bover{\epsilon}3
  +\bar\nu_{I}(S_K(a)\Bsetminus a)
  +\bar\nu_{I}\phi_i(a\Bsetminus b)
  +\bar\nu_{I}(\phi_ib\Bsetminus b)
  +\bar\nu_{I}(b\Bsetminus a)\cr
&\le\Bover{2\epsilon}3
  +\bar\nu_{I}(a\Bsetminus b)
  +\bar\nu_{I}(b\Bsetminus a)
\le\epsilon\cr}$$

\noindent because $\phi_i$ is a measure-preserving Boolean
homomorphism and $\phi_ib=b$.   Thus the induction continues.
}%end of proof of 531R

\leader{531S}{}\cmmnt{ Moving on from hypotheses expressible as
statements about measure algebras, we have a further result which can be
used when Martin's axiom is true.

\medskip

\noindent}{\bf Lemma} Suppose that
$\omega_1<\frak m_{\text{K}}$\cmmnt{ (definition:  517O)}.
Let $\ofamily{\xi}{\omega_1}{e_{\xi}}$ be the standard generating family in
$\frak B_{\omega_1}$, and
$\ofamily{\xi}{\omega_1}{a_{\xi}}$ a family of elements of
$\frak B_{\omega_1}$ of measure greater than $\bover12$.
Then there is an uncountable set $\Gamma\subseteq\omega_1$
such that $\inf_{\xi\in I}(a_{\xi}\Bcap e_{\xi})
\Bcap\inf_{\eta\in J}(a_{\eta}\Bsetminus e_{\eta})$ is non-zero
whenever $I$, $J\subseteq\Gamma$ are finite and disjoint.

\proof{{\bf (a)} Define $J^*(a)$, for $a\in\frak B_{\omega_1}$, and
$S_I(a)$, for $a\in\frak B_{\omega_1}$ and $I\subseteq\omega_1$, as in
531J and 531R.
Let $P$ be the set of pairs $(c,I)$ where $I\subseteq\omega_1$ is finite,
$0\ne c\Bsubseteq\inf_{\xi\in I}a_{\xi}$ and $I\cap J^*(c)=\emptyset$.
Order $P$ by saying that $(c,I)\le(c',I')$ if $I\subseteq I'$ and
$c'\Bsubseteq c$.   Then $P$ is a partially ordered set.   For each
$\xi<\omega_1$, $a_{\xi}\Bcap\phi_{\xi}a_{\xi}$ belongs to
$\frak C_{\kappa\setminus\{\xi\}}$ (531Jd) and is non-zero, so
$p_{\xi}=(a_{\xi}\Bcap\phi_{\xi}a_{\xi},\{\xi\})$ belongs to
$P$.   The point of the proof is the following fact.

\medskip

{\bf (b)} $P$ satisfies Knaster's condition upwards.   \Prf\ Let
$\ofamily{\xi}{\omega_1}{(c_{\xi},I_{\xi})}$ be a family in $P$.   Then
there are an $\alpha>0$ and an uncountable $A_0\subseteq\omega_1$ such
that $\bar\nu_{\omega_1}(c_{\xi}\Bcap c_{\eta})\ge\alpha$ for all $\xi$,
$\eta\in A_0$ (525Tc).   Next, there is an uncountable
$A_1\subseteq A_0$ such that $\family{\xi}{A_1}{I_{\xi}}$ is a
$\Delta$-system with root $I$ say (4A1Db);  let $m\in\Bbb N$ be such
that $A_2=\{\xi:\xi\in A_1$, $\#(I_{\xi}\setminus I)=m\}$ is
uncountable.
Finally, because $J^*(c_{\eta})$ is countable for each $\eta$, and
$\family{\xi}{A_2}{I_{\xi}\setminus I}$ is disjoint, we can find an
uncountable $A_3\subseteq A_2$ such that
$J^*(c_{\eta})\cap I_{\xi}\setminus I=\emptyset$ whenever $\eta$,
$\xi\in A_3$ and $\eta<\xi$.

Take a strictly increasing sequence $\sequence{k}{\eta_k}$ in $A_3$ and
a $\zeta\in A_3$ greater than every $\eta_k$.   By 531Rb, there is a
finite set $K\subseteq\omega_1$ such that
$\bar\nu_{\omega_1}
(S_J(1\Bsetminus c_{\zeta})\Bsetminus(1\Bsetminus c_{\zeta}))<\alpha$
whenever
$K\subseteq J\subseteq\omega_1$ and $\#(\omega_1\setminus J)=m$.
Let $k\in\Bbb N$ be
such that $I_{\eta_k}\setminus I$ does not meet $K$.   Set
$c'_{\zeta}
=S_{\kappa\setminus(I_{\eta_k}\setminus I)}(1\Bsetminus c_{\zeta})$.
Then

\Centerline{$\bar\nu_{\omega_1}(c'_{\zeta}\Bcap c_{\zeta})
\le\alpha
<\bar\nu_{\omega_1}(c_{\zeta}\Bcap c_{\eta_k})$,}

\noindent so
$c=c_{\eta_k}\Bsetminus c'_{\zeta}$ is non-zero;  as
$c'_{\zeta}\Bsupseteq 1\Bsetminus c_{\zeta}$, $c\Bsubseteq c_{\zeta}$.
Set $L=I_{\eta_k}\cup I_{\zeta}$.
Then $J^*(c_{\eta_k})$ is disjoint from $I_{\eta_k}$ and from
$I_{\zeta}\setminus I$, by the choice of $A_3$, while

\Centerline{$J^*(c'_{\zeta})
\subseteq J^*(1\Bsetminus c_{\zeta})\setminus(I_{\eta_k}\setminus I)
=J^*(c_{\zeta})\setminus(I_{\eta_k}\setminus I)$}

\noindent (531R(a-iii)) is also disjoint from $L$;  so
$J^*(c)\subseteq J^*(c_{\eta_k})\cup J^*(c'_{\zeta})$ is disjoint from
$L$.   Finally,

\Centerline{$c\Bsubseteq c_{\eta_k}\Bcap c_{\zeta}
\Bsubseteq\inf_{\xi\in I_{\eta_k}}a_{\xi}
  \Bcap\inf_{\xi\in I_{\zeta}}a_{\xi}=\inf_{\xi\in L}a_{\xi}$,}

\noindent so $(c,L)\in P$.   Now $(c,L)$ dominates both
$(c_{\eta_k},I_{\eta_k})$ and $(c_{\zeta},I_{\zeta})$.

What this shows is that if we write $Q$ for

\Centerline{$\{\{\eta,\zeta\}:\eta$, $\zeta\in A_3$,
$(c_{\eta},I_{\eta})$ and $(c_{\zeta},I_{\zeta})$ are compatible upwards
in $P\}$,}

\noindent then whenever $\zeta\in A_3$ and $M\subseteq A_3\cap\zeta$ is
infinite there is an $\eta\in M$ such that $\{\eta,\zeta\}\in M$.   By
5A1Gb, there is an uncountable
$A_4\subseteq A_3$ such that $[A_4]^2\subseteq Q$, that is,
$\family{\xi}{A_4}{(c_{\xi},I_{\xi})}$ is upwards-linked.   As
$\ofamily{\xi}{\omega_1}{(c_{\xi},I_{\xi})}$ is arbitrary, $P$ satisfies
Knaster's condition upwards.\ \Qed

\medskip

{\bf (c)} By 517S, there is a sequence $\sequencen{R_n}$ of
upwards-directed subsets of $P$ covering $\{p_{\xi}:\xi<\omega_1\}$;  as $\omega_1$
is uncountable, there must be some $n$ such that
$\Gamma=\{\xi:p_{\xi}\in R_n\}$ has cardinal $\omega_1$.   In this case,
$\{p_{\xi}:\xi\in\Gamma\}$ is upwards-centered in $P$.   If $I$,
$J\subseteq\Gamma$ are finite and disjoint, then there must be a
$(c,K)\in P$ which
is an upper bound for $\{p_{\xi}:\xi\in I\cup J\}$;  now
$I\cup J\subseteq K$ does not meet $J^*(c)$, while
$c\Bsubseteq\inf_{\xi\in I\cup J}a_{\xi}$.   But this means that

\Centerline{$\bar\nu_{\omega_1}(c\Bcap\inf_{\xi\in I}e_{\xi}
  \Bcap\inf_{\eta\in J}(1\Bsetminus e_{\eta}))
=2^{-\#(I\cup J)}\bar\nu_{\omega_1}c>0$,}

\noindent and

\Centerline{$0\ne c\Bcap\inf_{\xi\in I}e_{\xi}
  \Bcap\inf_{\eta\in J}(1\Bsetminus e_{\eta})
\Bsubseteq\inf_{\xi\in I}(a_{\xi}\Bcap e_{\xi})
  \Bcap\inf_{\eta\in J}(a_{\eta}\Bsetminus e_{\eta})$.}

\noindent So we have a set $\Gamma$ of the kind required.
}%end of proof of 531S

\leader{531T}{Theorem}\cmmnt{ ({\smc Fremlin 97})} Suppose that
$\omega\le\kappa<\frak m_{\text{K}}$.   If $X$ is a normal Hausdorff
space and $\kappa\in\MahR(X)$, then $[0,1]^{\kappa}$ is a continuous
image of $X$.

\proof{ For $\kappa=0$ this is trivial.   If $\kappa\ge\omega_2$, then
$\kappa$ is a measure-precaliber of all probability algebras (525Fb), so
531Lb gives the result.

If $\kappa=\omega_1$, let $\mu$ be a
\Mth\ Radon probability measure on $X$ with Maharam type $\omega_1$, and
$(\frak A,\bar\mu)$ its measure algebra.   Let
$\ofamily{\xi}{\omega_1}{d_{\xi}}$ be a $\tau$-generating
stochastically independent
family of elements of $\frak A$ of measure $\bover12$.    For
$\xi<\omega_1$ let
$E_{\xi}\in\dom\mu$ be such that $E_{\xi}^{\ssbullet}=d_{\xi}$, and
$K'_{\xi}\subseteq E_{\xi}$,
$K''_{\xi}\subseteq X\setminus E_{\xi}$ compact sets of measure greater
than $\bover14$;  set $K_{\xi}=K'_{\xi}\cup K''_{\xi}$ and
$a_{\xi}=K_{\xi}^{\ssbullet}$ in $\frak A$.   Because
$(\frak A,\bar\mu,\ofamily{\xi}{\omega_1}{d_{\xi}})$ is isomorphic
to $(\frak B_{\omega_1},
\bar\nu_{\omega_1},\ofamily{\xi}{\omega_1}{e_{\xi}})$,
531S tells us that there is an
uncountable set $\Gamma\subseteq\omega_1$ such that

\Centerline{$0\ne\inf_{\xi\in I}(a_{\xi}\Bcap d_{\xi})
  \Bcap\inf_{\eta\in J}(a_{\eta}\Bsetminus d_{\eta})
=(X\cap\bigcap_{\xi\in I}
  K'_{\xi}\cap\bigcap_{\eta\in J}K''_{\eta})^{\ssbullet}$}

\noindent whenever $I$, $J\subseteq\Gamma$ are finite.   Just as in part
(b) of the proof of 531L, it follows that there is a continuous
surjection from $\bigcap_{\xi\in\Gamma}K_{\xi}$ onto
$\{0,1\}^{\Gamma}\cong\{0,1\}^{\omega_1}$,
and therefore a continuous surjection from $X$ onto
$[0,1]^{\omega_1}$.
}%end of proof of 531T

\exercises{\leader{531X}{Basic exercises (a)}
%\spheader 531Xa
Show that there is a Hausdorff completely regular
quasi-Radon probability space $(X,\frak T,\Sigma,\mu)$ with Maharam type
greater than $\#(X)$.   \Hint{523Ib.}
%531A

\spheader 531Xb Give an example of a separable
Radon measure space with magnitude $2^{\frakc}$.
%531A

\spheader 531Xc Let $I^{\|}$ be the split interval (343J, 419L).   Show
that $\MahR(I^{\/})=\{0,\omega\}$.
%531E

\spheader 531Xd Let $I$ be an infinite set, and $\beta I$ the
Stone-\v{C}ech compactification of the discrete space $I$.   Show that
$2^{\#(I)}$ is the greatest member of $\MahR(\beta I)$.   \Hint{5A4Ia or
515H.}
%531E

\spheader 531Xe For a topological space $X$, write $\MahqR(X)$ for the
set of Maharam types of \Mth\ quasi-Radon probability measures on $X$.
(i) Show that $\kappa\le w(X)$ for every $\kappa\in\MahqR(X)$.   (ii)
Show that $\MahqR(Y)\subseteq\MahqR(X)$ for every $Y\subseteq X$.
(iii) Show that if $Y$ is another topological space, and neither $X$ nor
$Y$ is empty, then $\MahqR(X\times Y)=\MahqR(X)\cup\MahqR(Y)$.
%531E

\sqheader 531Xf Let $X$ be a Hausdorff topological group carrying Haar
measures, and $\frak A$ its Haar measure algebra (442H, 443A).   Show that
$w(X)=\max(c(\frak A),\tau(\frak A))$.
\Hint{443Gf, 529Ba.}
Show that if $X$ is $\sigma$-compact, locally
compact, Hausdorff and not discrete then $w(X)\in\MahR(X)$.
%531E

\spheader 531Xg Let $X$ be a normal Hausdorff space and $\kappa$ an
infinite cardinal.   Suppose that whenever $Y$ is a Hausdorff continuous
image of $X$ of weight $\kappa$ then $\MahR(Y)\subseteq\kappa$.   Show
that $\MahR(X)\subseteq\kappa$.
%531F

\spheader 531Xh Let $X$ be a Hausdorff space, and $\familyiI{E_i}$ a
family of universally Radon-measurable subsets of $X$ such that
$\#(I)<\cov\Cal N_{\kappa}$ for every $\kappa$.   Show that
$\MahR(\bigcup_{i\in I}E_i)=\bigcup_{i\in I}\MahR(E_i)$.
%531F

\spheader 531Xi Let $K$ be an Eberlein compactum.   Show that
$\MahR(K)\subseteq\{0,\omega\}$.   \Hint{467Xj.}

\spheader 531Xj Let $\Cal W\subseteq\Cal N_{\omega}$ be such that every
compact subset of $\{0,1\}^{\omega}\setminus\bigcup\Cal W$ is scattered.
Show that there is a family $\Cal W'\subseteq\Cal N_{\omega_1}$
such that $\#(\Cal W')=\#(\Cal W)$ and every compact subset of
$\{0,1\}^{\omega_1}\setminus\bigcup\Cal W'$ is scattered.
%531N

\spheader 531Xk Let $(X,\frak T,\Sigma,\mu)$ be a Hausdorff quasi-Radon
probability space.   Show that the Maharam type of $\mu$ is at most
$\max(\omega,2^{\chi(X)})$.   \Hint{5A4Ba, 5A4Bg.}
%531O

\spheader 531Xl In the language of 531R, show that if $a$,
$b\in\frak B_{\kappa}$ and $I\subseteq\kappa\setminus J^*(b)$ is finite,
then $\bar\nu_{\kappa}(a\Bsymmdiff S_{\kappa\setminus I}(a))
\le 2^{\#(I)}\bar\nu_{\kappa}(a\Bsymmdiff b)$.
%531R

\spheader 531Xm Show that if $\frak m_{\text{K}}>\omega_1$ and $X$ is a
countably tight
compact Hausdorff space, then $\omega_1\notin\MahR(X)$.
%531T
%is it enough if\omega_1 a precaliber of all prob algs?
%\v{S}apirovski\v\i's theorem:  a compact countably compact Hausdorff
%space has a point-countable \pi-base (proof by Stevo).
%In 531Q we have a countably tight X such that\omega_1\in\MahR(X)

\spheader 531Xn Let $X$ be an infinite
compact Hausdorff space with a strictly
positive Radon measure $\mu$, and $P$ the set of Radon probability measures
on $X$ with its narrow topology.   Show that the
topological density of $P$ is at most
the Maharam type of $\mu$.   \Hint{the indefinite-integral measures over
$\mu$ are dense in $P$.}
%- but can d(X)  be greater than  \tau(\mu) ?

\leader{531Y}{Further exercises (a)}
%\spheader 531Ya
Let $\kappa$ be an infinite cardinal such that $\kappa=\kappa^{\frakc}$.
Show that there is a set $X\subseteq\{0,1\}^{\kappa}$, of full outer
measure for $\nu_{\kappa}$, such that every subset of $X$ with cardinal
$\frak c$ is discrete.   Show that $\MahqR(X)$ (531Xe) contains $\kappa$
but not $\omega$.
%531Xe 531E 521Dc

\spheader 531Yb
Let $X$ and $Y$ be infinite compact Hausdorff spaces, and suppose that
there is a norm-preserving linear isomorphism from the dual space
$C(X)^*$ to $C(Y)^*$.   Show that $\MahR(X)=\MahR(Y)$.
%531F 529B in an L-space, u & v disjoint iff \|\alpha u+\beta v\| is
%  what it should be
%Suppose that  C(X)^*  and  C(Y)^*  are isomorphic as linear topological
%  spaces?

\spheader 531Yc Let $\mu$ be a
$\tau$-additive Borel probability measure on a topological
space $X$, and $\kappa$ a cardinal of uncountable cofinality such
that (i) $\chi(x,X)<\cf\kappa$ for every $x\in X$ (ii) no non-negligible
measurable set can be covered by $\cf\kappa$ negligible sets.   Show
that the Maharam type of $\mu$ cannot be $\kappa$.
%531P Query:  could the Maharam type be greater than $\kappa$? mt53bits
}%end of exercises

\leader{531Z}{Problems (a)} Can there be a perfectly normal compact
Hausdorff space $X$ such that $\omega_2\in\MahR(X)$?   (See 531Q, 554Xd.)

\spheader 531Zb Can there be a hereditarily separable compact
Hausdorff space $X$ such that $\omega_2\in\MahR(X)$?
%531Q
%Recall that a countably tight compact Hausdorff space has a
%point-countable \pi_base.
%What about hereditary \pi-weight-countable?

\leaveitout{The JKR space is hereditarily separable and carries a
non-trivial Borel
measure (a copy of Lebesgue measure).}

\leaveitout{Find a ZFC example of a topological space $X$ and a Borel
probability measure on $X$ with Maharam type greater than $w(X)$.
(Note that if $\kappa$ is atomlessly-measurable then $X=\kappa$ with
its discrete topology serves.   Also there is then no non-trivial Borel
probability measure on $X$ of low Maharam type.} %531A

\leaveitout{\spheader 531Z? Let $X$ be a topological space, and suppose
that there is a quasi-Radon probability measure on $X$ with Maharam type
$\omega_1$.   Does it follow that there is an atomless quasi-Radon
measure on $X$ with Maharam type $\omega$?} %531E

\leaveitout{If $X$ is compact Hausdorff and $P(X)$ is countably tight,
can $\omega_1\in\MahR(X)$?   see {\smc Plebanek \& Sobota p14}.
}

\endnotes{
\Notesheader{531}
This section is directed to Radon measures, studying $\MahR(X)$;  of
course we can look at Maharam types of quasi-Radon measures (531Xe,
531Ya), or Borel or Baire measures for that matter.   In the next
section I shall have something to say about completion regular measures.
The function $X\mapsto\MahR(X)$ has a much more satisfying list of basic
properties (531E, 531G) than the others.

From 531L and 531T we see that there are many cardinals $\kappa$ such
that whenever $X$ is a compact Hausdorff space and $\kappa\in\MahR(X)$,
then there is a continuous function from $X$ onto $[0,1]^{\kappa}$.
Such cardinals are said to have {\bf Haydon's property}.   From 531L,
531M and 531T we see that

\inset{$\omega$ has Haydon's property (531La);

if $\kappa\ge\omega_2$ and $\kappa$ is a measure-precaliber of
$\frak B_{\kappa}$ then $\kappa$ has Haydon's property (531Lb);

if $\kappa\ge\omega$ is not a measure-precaliber of $\frak B_{\kappa}$
then $\kappa$ does not have Haydon's property (531M);

if $\omega_1<\frak m_{\text{K}}$ then $\omega_1$ has Haydon's property
(531T).}

\noindent Thus if $\frak m_{\text{K}}>\omega_1$, an infinite cardinal
$\kappa$ has Haydon's property iff it is a measure-precaliber of every
probability algebra.   $\omega_1$ really is different;  it is possible
that $\omega_1$ is a precaliber of every probability algebra but does
not have Haydon's property.   To check this, it is enough to find a
model of set theory in which $\cov\Cal N_{\omega_1}>\omega_1$ but there
is a family $\ofamily{\xi}{\omega_1}{W_{\xi}}$ as in 531N;  one is
described in 553F.

You will observe that the key arguments of this section all depend on
analysis of the measure algebras $\frak B_{\kappa}$.   We have already
seen in \S524 that many properties of a Radon measure can be determined
from its measure algebra.   Here we find that some important topological
properties of compact Hausdorff spaces can be determined by the measure
algebras of the Radon measures they carry.   The results here largely
depend for their applications on knowing enough about precalibers;  I
remind you that it seems to be still unknown whether it is possible that
every infinite cardinal is a measure-precaliber of every probability
algebra.

The remarks above have concerned the existence of continuous surjections
onto $[0,1]^{\kappa}$;  a natural place to start, because measures of
Maharam type $\kappa$ arise immediately from such surjections.   In
531O-531Q %531O 531P 531Q
I look at different measures of the richness of a compact space $X$.
Concerning characters, 531O-531P give us quite a lot of information,
slightly irregular at the edges.   I ought to offer a remark on the
context of 531Q.
In some set theories (for instance, when $\frak m>\omega_1$), we find
not only that $\omega_1$ is a precaliber of every measurable algebra,
but also that a compact Hausdorff space is hereditarily separable iff it is
hereditarily Lindel\"of ({\smc Fremlin 84a}, 44H);  so that, for
instance, a hereditarily separable compact Hausdorff space cannot carry
a Radon measure of uncountable Maharam type.   Typically, the situation
is very different if the continuum hypothesis or
Jensen's $\diamondsuit$ is true,
and 531Q is a descendant of the construction in {\smc Kunen 81} of a
non-separable hereditarily Lindel\"of compact Hausdorff space.   See
{\smc D\v{z}amonja \& Kunen 93} for further
exploration of these questions.
}%end of notes


