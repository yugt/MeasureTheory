\frfilename{mt285.tex}
\versiondate{18.9.14}
\copyrightdate{1994}

\def\chaptername{Fourier analysis}
\def\sectionname{Characteristic functions}

\newsection{285}

I come now to one of the most effective applications of Fourier
transforms, the use of `characteristic functions' to analyse
probability distributions.   It turns out not only that the Fourier
transform of a probability distribution determines the distribution
(285M) but that many of the things we want to know about a
distribution are easily calculated from its
transform\cmmnt{ (285G, 285Xi)}.    Even more strikingly, pointwise
convergence of
Fourier transforms corresponds (for sequences) to convergence for the
vague topology in
the space of distributions, so they provide a new and extremely powerful
method for proving such results as the Central Limit Theorem and
Poisson's theorem (285Q).

As the applications of the ideas here mostly belong to probability
theory, I return to probabilists' terminology, as in Chapter 27.
There will nevertheless be many points at which it is appropriate to
speak of integrals, and there will often be more than one measure in play;
so I should say directly that an integral $\int f(x)dx$ will be with
respect to Lebesgue measure (usually, but not always, one-dimensional),
as in the rest of this chapter,
while integrals with respect to other measures will be expressed
in the forms $\int fd\nu$ or $\int f(x)\nu(dx)$.

\leader{285A}{Definition (a)} Let $\nu$ be a Radon probability measure
on $\BbbR^r$\cmmnt{ (256A)}.   Then the {\bf characteristic function}
of $\nu$ is the function $\phi_{\nu}:\BbbR^r\to\Bbb C$ given by the
formula

\Centerline{$\phi_{\nu}(y)=\int e^{iy\dotproduct x}\nu(dx)$}

\noindent for every $y\in\BbbR^r$, writing
$y\dotproduct x=\eta_1\xi_1+\ldots+\eta_r\xi_r$ if
$y=(\eta_1,\ldots,\eta_r)$ and $x=(\xi_1,\ldots,\xi_r)$.

\header{285Ab}{\bf (b)} Let $X_1,\ldots,X_r$ be real-valued random
variables on the same probability space.   The {\bf characteristic
function} of $\pmb{X}=(X_1,\ldots,X_r)$ is the characteristic function
$\phi_{\pmb{X}}=\phi_{\nu_{\pmb{X}}}$ of their joint probability
distribution $\nu_{\pmb{X}}$ as defined in 271C.

\cmmnt{
\leader{285B}{Remarks (a)} By one of the ordinary accidents of history,
the definitions of `characteristic function' and `Fourier
transform' have evolved independently.   In 283Ba I remarked that the
definition of the Fourier transform remains unfixed, and that the
formulae

\Centerline{$\varhatf(y)=\int_{-\infty}^{\infty}e^{iyx}f(x)dx$,}

\Centerline{$\varcheckf(y)
=\Bover1{2\pi}\int_{-\infty}^{\infty}e^{-iyx}f(x)dx$}

\noindent are sometimes used.   On the other hand, I think that nearly
all authors agree on the definition of the characteristic function as
given above.   You may feel therefore that I should have followed their
lead, and chosen the definition of Fourier transform which best matches
the definition of characteristic function.   I did not do so largely
because I wished to emphasise the symmetry between the Fourier transform
and the inverse Fourier transform, and the correspondence between
Fourier transforms and Fourier series.   The principal advantage of
matching the definitions up would be to make the constants in such
theorems as 283F, 285Xk the same, and would be balanced by the need
to remember different constants for $\varhatf$ and $\varcheckf$ in such
results as 283M.

\header{285Bb}{\bf (b)} A secondary reason for not trying too hard to
make the formulae of this section match directly those of \S\S283-284
is that the $r$-dimensional case is at the heart of some of the most
important
applications of characteristic functions, so that it seems right to
introduce it from the beginning;  and
consequently the formulae of this section will necessarily have new
features compared with those in the body of the work so far.
}

\leader{285C}{}\cmmnt{ Of course there is a direct way to describe the
characteristic function of a family $(X_1,\ldots,X_r)$ of random
variables, as follows.

\medskip

\noindent}{\bf Proposition} Let $X_1,\ldots,X_r$ be real-valued random
variables on the same probability space, and
$\nu_{\pmb{X}}$ their joint distribution.   Then their characteristic
function $\phi_{\nu_{\pmb{X}}}$ is given by

\Centerline{$\phi_{\nu_{\pmb{X}}}(y)=\Expn(e^{iy\dotproduct \pmb{X}})
=\Expn(e^{i\eta_1X_1}e^{i\eta_2X_2}\ldots e^{i\eta_rX_r})$}

\noindent for every $y=(\eta_1,\ldots,\eta_r)\in\BbbR^r$.

\proof{ Apply 271E to the functions $h_1$, $h_2:\Bbb R^r\to\Bbb R$
defined by

\Centerline{$h_1(x)=\cos(y\dotproduct x)$,\quad
$h_2(y)=\sin(y\dotproduct x)$,}

\noindent to see that

$$\eqalign{\phi_{\nu_{\pmb{X}}}(y)
&=\int h_1(x)\nu_{\pmb{X}}(dx) + i\int h_2(x)\nu_{\pmb{X}}(dx)\cr
&=\Expn(h_1(\pmb{X})) + i\Expn(h_2(\pmb{X}))
=\Expn(e^{iy\dotproduct \pmb{X}}).\cr}$$
}%end of proof of 285C

\vleader{72pt}{285D}{}\cmmnt{ I ought to spell out the correspondence
between Fourier transforms, as defined in 283A, and characteristic
functions.

\medskip

\noindent}{\bf Proposition} Let $\nu$ be a Radon probability measure on
$\Bbb R$.   Write

\Centerline{$\varhat{\nu}(y)
=\Bover1{\sqrt{2\pi}}\int_{-\infty}^{\infty}
e^{-iyx}\nu(dx)$}

\noindent for every $y\in\Bbb R$, and $\phi_{\nu}$ for the
characteristic function of $\nu$.


(a) $\varhat{\nu}(y)=\Bover1{\sqrt{2\pi}}\phi_{\nu}(-y)$ for every
$y\in\Bbb R$.

(b) For any Lebesgue integrable complex-valued function $h$ defined
almost everywhere in $\Bbb R$,

\Centerline{$\int_{-\infty}^{\infty}\varhat{\nu}(y)h(y)dy
=\int_{-\infty}^{\infty}\varhat{h}(x)\nu(dx)$.}

(c) For any rapidly decreasing test function $h$ on $\Bbb R$\cmmnt{ (see
\S284)},

\Centerline{$\int_{-\infty}^{\infty}h(x)\nu(dx)
=\int_{-\infty}^{\infty}\varcheck{h}(y)\varhat{\nu}(y)dy$.}

(d) If $\nu$ is an indefinite-integral measure over Lebesgue measure,
with Radon-Nikod\'ym derivative $f$, then $\varhat{\nu}$ is the Fourier
transform of $f$.

\proof{{\bf (a)} This is immediate from the definitions of
$\phi_{\nu}$ and $\varhat{\nu}$.

\medskip

{\bf (b)} Because

\Centerline{$\int_{-\infty}^{\infty}\int_{-\infty}^{\infty}
|h(y)|\nu(dx)dy=\int_{-\infty}^{\infty}|h(y)|dy<\infty$,}

\noindent we may change the order of integration to see that

$$\eqalign{\int_{-\infty}^{\infty}\varhat{\nu}(y)h(y)dy
&=\Bover1{\sqrt{2\pi}}\int_{-\infty}^{\infty}\int_{-\infty}^{\infty}
e^{-iyx}h(y)\nu(dx)dy\cr
&=\Bover1{\sqrt{2\pi}}\int_{-\infty}^{\infty}\int_{-\infty}^{\infty}
e^{-iyx}h(y)dy\,\nu(dx)
=\int_{-\infty}^{\infty}\varhat{h}(x)\nu(dx).\cr}$$

\medskip

{\bf (c)} This follows immediately from (b), because $\varcheck{h}$ is
integrable and $\varcheck{h}\varhat{\phantom{h}}=h$ (284C).

\medskip

{\bf (d)} The point is just that

\Centerline{$\int h\,d\nu=\int h(x)f(x)dx$}

\noindent for every bounded Borel measurable $h:\Bbb R\to\Bbb R$ (235K),
and therefore for the functions $x\mapsto e^{-iyx}:\Bbb R\to\Bbb C$.
Now

\Centerline{$\varhat{\nu}(y)
=\Bover1{\sqrt{2\pi}}\int_{-\infty}^{\infty}e^{-iyx}\nu(dx)
=\Bover1{\sqrt{2\pi}}\int_{-\infty}^{\infty}e^{-iyx}f(x)dx
=\varhatf(y)$}

\noindent for every $y$.
}% end of proof of 285D

\leader{285E}{Lemma} Let $X$ be a normal random variable with expectation
$a$ and variance $\sigma^2$, where $\sigma>0$.   Then the characteristic
function of $X$ is given by the formula

\Centerline{$\phi(y)=e^{iya}e^{-\sigma^2y^2/2}$.}

\proof{ This is just 283N with the constants changed.   We have

$$\eqalignno{\phi(y)
&=\Expn(e^{iyX})
=\Bover1{\sigma\sqrt{2\pi}}\int_{-\infty}^{\infty}
  e^{iyx}e^{-(x-a)^2/2\sigma^2}dx\cr
\noalign{\noindent (taking the density function for $X$ given in 274Ad,
and applying 271Ic)}
&=\Bover1{\sqrt{2\pi}}\int_{-\infty}^{\infty}
  e^{iy(\sigma t+a)}e^{-t^2/2}dt\cr
\noalign{\noindent (substituting $x=\sigma t+a$)}
&=e^{iya}\sqrt{2\pi}\varhat{\psi}_1(-y\sigma)\cr
\noalign{\noindent (setting $\psi_1(x)=\bover1{\sqrt{2\pi}}e^{-x^2/2}$,
as in 283N)}
&=e^{iya}e^{-\sigma^2y^2/2}.\cr}$$
}%end of proof of 285E

\leader{285F}{}\cmmnt{ I now give results corresponding to parts of
283C, with an
extra refinement concerning independent random variables (285I).

\medskip

\noindent}{\bf Proposition} Let $\nu$ be a Radon probability measure on
$\BbbR^r$, and $\phi$ its characteristic function.

(a) $\phi(0)=1$.

(b) $\phi:\BbbR^r\to\Bbb C$ is uniformly continuous.

(c) $\phi(-y)=\overline{\phi(y)}$, $|\phi(y)|\le 1$ for every $y\in\Bbb
R^r$.

(d) If $r=1$ and $\int|x|\nu(dx)<\infty$, then $\phi'(y)$ exists and is
equal to $i\int xe^{ixy}\nu(dx)$ for every $y\in\Bbb R$.

(e) If $r=1$ and $\int x^2\nu(dx)<\infty$, then $\phi''(y)$ exists and
is equal to $-\int x^2e^{ixy}\nu(dx)$ for every $y\in\Bbb R$.

\proof{{\bf (a)} $\phi(0)=\int\chi\BbbR^r\,d\nu=\nu(\BbbR^r)=1$.

\medskip

{\bf (b)} Let $\epsilon>0$.   Let $M>0$ be such that

\Centerline{$\nu\{x:\|x\|\ge M\}\le\epsilon$,}

\noindent writing $\|x\|=\sqrt{x\dotproduct x}$ as usual.   Let
$\delta>0$
be such that $|e^{ia}-1|\le\epsilon$ whenever $|a|\le\delta$.   Now
suppose that $y$, $y'\in\BbbR^r$ are such that $\|y-y'\|\le\delta/M$.
Then whenever $\|x\|\le M$,

\Centerline{$|e^{iy\dotproduct x}-e^{iy'\dotproduct x}|
=|e^{iy'\dotproduct x}||e^{i(y-y')\dotproduct x}-1|
=|e^{i(y-y')\dotproduct x}-1|
\le\epsilon$}

\noindent because

\Centerline{$|(y-y')\dotproduct x|\le\|y-y'\|\|x\|\le\delta$.}

\noindent Consequently, writing $B$ for $\{x:\|x\|\le M\}$,

$$\eqalign{|\phi(y)-\phi(y')|
&\le\int_B|e^{iy\dotproduct x}-e^{iy'\dotproduct x}|\nu(dx)
\cr&\qquad\qquad
  +\int_{\BbbR^r\setminus B}|e^{iy\dotproduct x}|\nu(dx)
  +\int_{\BbbR^r\setminus B}|e^{iy'\dotproduct x}|\nu(dx)\cr
&\le\epsilon+\epsilon+\epsilon
=3\epsilon.\cr}$$

\noindent As $\epsilon$ is arbitrary, $\phi$ is uniformly continuous.


\medskip

{\bf (c)} This is elementary;

\Centerline{$\phi(-y)=\int e^{-iy\dotproduct x}\nu(dx)
=\overline{\intop e^{iy\dotproduct x}\nu(dx)}=\overline{\phi(y)}$,}

\Centerline{$|\phi(y)|=|\int e^{iy\dotproduct x}\nu(dx)|
\le\int|e^{iy\dotproduct x}|\nu(dx)
=1$.}

\medskip

{\bf (d)} The point is that $|\pd{}{y}e^{iyx}|=|x|$ for every $x$,
$y\in\Bbb R$.   So by 123D (applied, strictly speaking, to the real and
imaginary parts of the function)

\Centerline{$\phi'(y)
=\Bover{d}{dy}\int e^{iyx}\nu(dx)
=\int\Pd{}{y}e^{iyx}\nu(dx)
=\int ixe^{iyx}\nu(dx)$.}

\medskip

{\bf (e)} Since we now have $|\pd{}{y}xe^{iyx}|=x^2$ for every $x$, $y$,
we can repeat the argument to get

\Centerline{$\phi''(y)
=i\Bover{d}{dy}
     \int xe^{iyx}\nu(dx)
=i\int
     \Pd{}{y}xe^{iyx}\nu(dx)
=-\int
     x^2e^{iyx}\nu(dx)$.}
}%end of proof of 285F

\vleader{48pt}{285G}{Corollary} (a) Let $X$ be a real-valued random
variable
with finite expectation, and $\phi$ its characteristic function.   Then
$\phi'(0)=i\Expn(X)$.

(b) Let $X$ be a real-valued random variable with
finite variance, and $\phi$ its characteristic function.   Then
$\phi''(0)=-\Expn(X^2)$.

\proof{ We have only to match $X$ to its distribution
$\nu$, and say that

\Centerline{`$X$ has finite expectation'}

\noindent corresponds to

\Centerline{`$\int |x|\nu(dx)=\Expn(|X|)<\infty$',}

\noindent so that

\Centerline{$\phi'(0)=i\int x\,\nu(dx)=i\Expn(X)$,}

\noindent and that

\Centerline{`$X$ has finite variance'}

\noindent corresponds to

\Centerline{`$\int x^2\nu(dx)=\Expn(X^2)<\infty$',}

\noindent so that

\Centerline{$\phi''(0)=-\int x^2\,\nu(dx)=-\Expn(X^2)$,}

\noindent as in 271E.
}%end of proof of 285G

\cmmnt{
\leader{285H}{Remark} Observe that there is no result
corresponding to 283Cg (`$\lim_{|y|\to\infty}\varhatf(y)
\ifdim\pagewidth=390pt\penalty-100\fi
=0$').   If $\nu$ is the Dirac measure on $\Bbb R$ concentrated at $0$,
that is, the distribution of a random variable
which is zero almost everywhere, then $\phi(y)=1$ for every $y$.
}

\leader{285I}{Proposition} Let $X_1,\ldots,X_n$ be independent
real-valued random variables, with characteristic functions
$\phi_{1},\ldots,\phi_{n}$.   Let $\phi$ be the characteristic function
of their sum $X=X_1+\ldots+X_n$.   Then

\Centerline{$\phi(y)=\prod_{j=1}^n\phi_j(y)$}

\noindent for every $y\in\Bbb R$.

\proof{ Let $y\in\Bbb R$.   By 272E, the variables

\Centerline{$Y_j=e^{iyX_j}$}

\noindent are independent, so by 272R

\Centerline{$\phi(y)=\Expn(e^{iyX})=\Expn(e^{iy(X_1+\ldots+X_n)})
=\Expn(\prod_{j=1}^nY_j)=\prod_{j=1}^n\Expn(Y_j)
=\prod_{j=1}^n\phi_j(y)$,}

\noindent as required.
}%end of proof of 285I

\cmmnt{
\medskip

\noindent{\bf Remark} See also 285R below.
}

\leader{285J}{}\cmmnt{ There is an inversion theorem for
characteristic
functions, corresponding to 283F;  I give it in 285Xk, with an
$r$-dimensional version in 285Yb.   However, this does not seem to be as
useful as the following group of results.

\medskip

\noindent}{\bf Lemma} Let $\nu$ be a Radon probability measure on
$\Bbb R^r$, and $\phi$ its characteristic function.
Then for $1\le j\le r$ and $a>0$,

\Centerline{$\nu\{x:|\xi_j|\ge a\}
\le 7a\int_0^{1/a}(1-\Real\phi(te_j))dt$,}

\noindent where $e_j\in\BbbR^r$ is the $j$th unit vector.

\wheader{285J}{0}{0}{0}{84pt}

\proof{ We have

$$\eqalignno{7a\int_0^{1/a}(1-\Real\phi(te_j))dt
&=7a\int_0^{1/a}
   \bigl(1-\Real\int_{\Bbb R^r}e^{it\xi_j}\nu(dx)\bigr)dt\cr
&=7a\int_0^{1/a}\int_{\Bbb R^r}1-\cos(t\xi_j)\nu(dx)dt\cr
&=7a\int_{\Bbb R^r}\int_0^{1/a}1-\cos(t\xi_j)dt\,\nu(dx)\cr
\displaycause{because $(x,t)\mapsto 1-\cos(t\xi_j)$ is bounded and
$\nu\BbbR^r\cdot\Bover1a$ is finite}
&=7a\int_{\Bbb R^r}
  \bigl(\bover1a-\bover1{\xi_j}\sin\bover{\xi_j}a\bigr)\nu(dx)\cr
&\ge7a\int_{|\xi_j|\ge a}
\bigl(\bover1a-\bover1{\xi_j}\sin\bover{\xi_j}a\bigr)\nu(dx)\cr
\noalign{\noindent (because $\Bover1{\xi}\sin\Bover{\xi}{a}\le\Bover1a$
for every $\xi\ne 0$)}
&\ge\nu\{x:|\xi_j|\ge a\},\cr}$$

\noindent because

\Centerline{$\Bover{\sin\eta}{\eta}\le\Bover{\sin 1}{1}\le\Bover67$ if
$\eta\ge 1$,}

\noindent so

\Centerline{$a(\Bover1a-\Bover1{\xi_j}\sin\Bover{\xi_j}a)\ge\Bover17$}

\noindent if $|\xi_j|\ge a$.
}%end of proof of 285J

\cmmnt{
\leader{285K}{Characteristic functions and the vague topology}
The time has come to return to ideas mentioned briefly in 274L.   Fix
$r\ge 1$ and let $P$ be the set of all Radon probability measures on
$\BbbR^r$.   For any bounded continuous function $h:\BbbR^r\to\Bbb R$,
define $\rho_h:P\times P\to\Bbb R$ by setting

\Centerline{$\rho_h(\nu,\nuprime)=|\int h\,d\nu-\int h\,d\nuprime|$}

\noindent for $\nu$, $\nuprime\in P$.   Then the vague topology on $P$ is
the topology generated by the pseudometrics $\rho_h$ (274Ld).
}

\leader{285L}{Theorem} Let $\nu$, $\sequencen{\nu_n}$ be Radon
probability measures on $\BbbR^r$, with characteristic functions
$\phi$, $\sequencen{\phi_n}$.   Then the following are equiveridical:

(i) $\nu=\lim_{n\to\infty}\nu_n$ for the vague topology;

(ii) $\int h\,d\nu=\lim_{n\to\infty}\int h\,d\nu_n$ for every bounded
continuous $h:\BbbR^r\to\Bbb R$;

(iii) $\lim_{n\to\infty}\phi_n(y)=\phi(y)$ for every $y\in\BbbR^r$.

\proof{{\bf (a)} The equivalence of (i) and (ii) is virtually
the definition of the vague topology;  we have

$$\eqalignno{\lim_{n\to\infty}
&\nu_n=\nu\text{ for the vague topology}\cr
&\iff\,\lim_{n\to\infty}\rho_h(\nu_n,\nu)=0
\text{ for every bounded continuous }h\cr
\noalign{\noindent (2A3Mc)}
&\iff\,\lim_{n\to\infty}|\int h\,d\nu_n-\int h\,d\nu|=0
   \text{ for every bounded continuous }h.\cr}$$

\medskip

{\bf (b)} Next, (ii) obviously implies (iii), because

\Centerline{$\Real\phi(y)
=\int h_y\,d\nu
=\lim_{n\to\infty}h_y\,d\nu_n
=\lim_{n\to\infty}\Real\phi_n(y)$,}

\noindent setting $h_y(x)=\cos(x\dotproduct y)$ for each $x$, and
similarly

\Centerline{$\Imag\phi(y)=\lim_{n\to\infty}\Imag\phi_n(y)$}

\noindent for every $y\in\BbbR^r$.

\medskip

{\bf (c)} So we are left to prove that (iii)$\Rightarrow$(ii).   I start
by showing that, given $\epsilon>0$, there is a closed bounded set $K$
such that

\Centerline{$\nu_n(\BbbR^r\setminus K)\le\epsilon$ for every
$n\in\Bbb N$.}

\noindent\Prf\ We know that $\phi(0)=1$ and that $\phi$ is continuous at
$0$ (285Fb).   Let $a>0$ be so large that whenever $j\le r$ and
$|t|\le 1/a$ we have

\Centerline{$1-\Real\phi(te_j)\le\Bover{\epsilon}{14r}$,}

\noindent writing $e_j$ for the $j$th unit vector, as in 285J.   Then

\Centerline{$7a\int_0^{1/a}
(1-\Real\phi(te_j))dt\le\Bover{\epsilon}{2r}$}

\noindent for each $j\le r$.   By Lebesgue's Dominated Convergence
Theorem (since of course the functions $t\mapsto 1-\Real\phi_n(te_j)$
are uniformly bounded on $[0,\bover1a]$), there is an $n_0\in\Bbb N$
such that

\Centerline{$7a\int_0^{1/a}
(1-\Real\phi_n(te_j))dt\le\Bover{\epsilon}{r}$}

\noindent for every $j\le r$ and $n\ge n_0$.   But 285J tells us that now

\Centerline{$\nu_n\{x:|\xi_j|\ge a\}\le\Bover{\epsilon}{r}$}

\noindent for every $j\le r$, $n\ge n_0$.   On the other hand, there is
surely a $b\ge a$ such that

\Centerline{$\nu_n\{x:|\xi_j|\ge b\}\le\Bover{\epsilon}{r}$}

\noindent for every $j\le r$, $n<n_0$.   So, setting
$K=\{x:|\xi_j|\le b\text{ for every }j\le r\}$,

\Centerline{$\nu_n(\BbbR^r\setminus K)\le\epsilon$}

\noindent for every $n\in\Bbb N$, as required.  \Qed

\medskip

{\bf (d)} Now take any bounded continuous $h:\BbbR^r\to\Bbb R$ and
$\epsilon>0$.   Set $M=1+\sup_{x\in\BbbR^r}|h(x)|$, and let $K$ be a
bounded closed set such that

\Centerline{$\nu_n(\BbbR^r\setminus K)\le\Bover{\epsilon}M$ for every
$n\in\Bbb N$, \quad$\nu(\BbbR^r\setminus K)\le\Bover{\epsilon}M$,}

\noindent using (b) just above.   By the
Stone-Weierstrass theorem (281K) there are $y_0,\ldots,y_m\in\Bbb Q^r$
and $c_0,\ldots,c_m\in\Bbb C$ such that

\Centerline{$|h(x)-g(x)|\le\epsilon$ for every $x\in K$,}

\Centerline{$|g(x)|\le M$ for every $x\in\BbbR^r$,}

\noindent writing $g(x)=\sum_{k=0}^mc_ke^{iy_k\dotproduct x}$ for
$x\in\BbbR^r$.   Now

\Centerline{$\lim_{n\to\infty}\int g\,d\nu_n
=\lim_{n\to\infty}\sum_{k=0}^mc_k\phi_n(y_k)
=\sum_{k=0}^mc_k\phi(y_k)
=\int g\,d\nu$.}

\noindent On the other hand, for every $n\in\Bbb N$,

\Centerline{$|\int g\,d\nu_n-\int h\,d\nu_n|
\le\int_K|g-h|d\nu_n+2M\nu_n(\Bbb R\setminus K)
\le 3\epsilon$,}

\noindent and similarly $|\int g\,d\nu-\int h\,d\nu|\le 3\epsilon$.
Consequently

\Centerline{$\limsup_{n\to\infty}|\int h\,d\nu_n-\int h\,d\nu|\le
6\epsilon$.}

\noindent As $\epsilon$ is arbitrary,

\Centerline{$\lim_{n\to\infty}\int h\,d\nu_n=\int h\,d\nu$,}

\noindent and (ii) is true.
}%end of proof of 285L


\vleader{72pt}{285M}{Corollary} (a) Let $\nu$, $\nuprime$ be two Radon
probability
measures on $\BbbR^r$ with the same characteristic functions.   Then
they are equal.

(b) Let $(X_1,\ldots,X_r)$ and $(Y_1,\ldots,Y_r)$ be two families of
real-valued random variables.   If

\Centerline{$\Expn(e^{i\eta_1X_1+\ldots+i\eta_rX_r})
=\Expn(e^{i\eta_1Y_1+\ldots+i\eta_rY_r})$}

\noindent for all $\eta_1,\ldots,\eta_r\in\Bbb R$, then
$(X_1,\ldots,X_r)$ has the same joint distribution as
$(Y_1,\ldots,Y_r)$.

\proof{{\bf (a)} Applying 285L with $\nu_n=\nuprime$ for every $n$,
we see that $\int h\,d\nuprime=\int h\,d\nu$ for every bounded continuous
$h:\BbbR^r\to\Bbb R$.   By 256D(iv), $\nu=\nuprime$.

\medskip

{\bf (b)} Apply (a) with $\nu$, $\nuprime$ the two joint distributions.
}%end of proof of 285M

\cmmnt{
\leader{285N}{Remarks} Probably the most important application of this
theorem is to the standard proof of the Central Limit Theorem.   I
sketch the ideas in 285Xq and 285Yl-285Yo;  details may be found in most
serious probability texts;  two on my shelf are {\smc Shiryayev 84},
\S III.4, and {\smc Feller 66}, \S XV.6.   However, to get the full
strength of Lindeberg's version of the Central Limit Theorem we have to
work quite hard, and I therefore propose to illustrate the method with a
version of Poisson's theorem (285Q) instead.   I begin with two lemmas
which are very frequently used in results of this kind.
}

\leader{285O}{Lemma} Let
$c_0,\ldots,c_n,d_0,\ldots,d_n$ be complex numbers of
modulus at most $1$.   Then

\Centerline{$|\prod_{k=0}^nc_k-\prod_{k=0}^nd_k|
\le\sum_{k=0}^n|c_k-d_k|$.}

\proof{ Induce on $n$.   The case $n=0$ is trivial.   For
the case $n=1$ we have

$$\eqalign{|c_0c_1-d_0d_1|
&=|c_0(c_1-d_1)+(c_0-d_0)d_1|\cr
&\le|c_0||c_1-d_1|+|c_0-d_0||d_1|
\le|c_1-d_1|+|c_0-d_0|,\cr}$$

\noindent which is what we need.   For the inductive step to $n+1$, we
have

$$\eqalignno{|\prod_{k=0}^{n+1}c_k-\prod_{k=0}^{n+1}d_k|
&\le|\prod_{k=0}^nc_k-\prod_{k=0}^nd_k|
+|c_{n+1}-d_{n+1}|\cr
\noalign{\noindent (by the case just done, because
$c_{n+1}$, $d_{n+1}$, $\prod_{k=0}^nc_k$ and $\prod_{k=0}^nd_k$ all have
modulus at most $1$)}
&\le\sum_{k=0}^n|c_k-d_k|+|c_{n+1}-d_{n+1}|\cr
\noalign{\noindent (by the inductive hypothesis)}
&=\sum_{k=0}^{n+1}|c_k-d_k|,\cr}$$

\noindent so the induction continues.
}%end of proof of 285O

\leader{285P}{Lemma} Suppose that $M\ge 0$ and $\epsilon>0$.
Then there are
$\eta>0$ and $y_0,\ldots,y_n\in\Bbb R$ such that whenever $X$, $Z$ are
two real-valued random variables with $\Expn(|X|)\le M$,
$\Expn(|Z|)\le M$ and
$|\phi_X(y_j)-\phi_Z(y_j)|\le\eta$ for every $j\le n$, then
$F_X(a)\le F_Z(a+\epsilon)+\epsilon$ for every $a\in\Bbb R$, where I
write
$\phi_X$ for the characteristic function of $X$ and $F_X$ for the
distribution function of $X$.

\proof{ The case $M=0$ is trivial, as then both $X$ and $Z$ are zero a.e.,
so I will suppose henceforth that $M>0$.
Set $\delta=\Bover{\epsilon}7>0$, $b=\Bover{M}{\delta}$.

\medskip

{\bf (a)} Define $h_0:\Bbb R\to[0,1]$ by setting
$h_0(x)=\med(0,1-\Bover{x}{\delta},1)$ for $x\in\Bbb R$.
Then $h_0$ is continuous.   Let $m=\lfloor\bover{b}{\delta}\rfloor$ 
be the integer part of $\bover{b}{\delta}$, and for $-m\le k\le m+1$ set
$h_k(x)=h_0(x-k\delta)$.

By the Stone-Weierstrass theorem (281K again), there are
$y_0,\ldots,y_n\in\Bbb R$ and $c_0,\ldots,c_n\in\Bbb C$ such that,
writing $g_0(x)=\sum_{j=0}^nc_je^{iy_jx}$,

\Centerline{$|h_0(x)-g_0(x)|\le\delta$
  for every $x\in[-b-(m+1)\delta,b+m\delta]$,}

\Centerline{$|g_0(x)|\le 1$ for every $x\in\Bbb R$.}

\noindent For $-m\le k\le m+1$, set

\Centerline{$g_k(x)=g_0(x-k\delta)
=\sum_{j=0}^nc_je^{-iy_jk\delta}e^{iy_jx}$.}

\noindent Set $\eta=\delta/(1+\sum_{j=0}^n|c_j|)>0$.

\medskip

{\bf (b)} Now suppose that $X$, $Z$ are random variables such that
$\Expn(|X|)\le M$, $\Expn(|Z|)\le M$ and
$|\phi_X(y_j)-\phi_Z(y_j)|\le\eta$ for every $j\le n$.   Then for any
$k$ we have

\Centerline{$\Expn(g_k(X))
=\Expn(\sum_{j=0}^nc_je^{-iy_jk\delta}e^{iy_jX})
=\sum_{j=0}^nc_je^{-iy_jk\delta}\phi_X(y_j)$,}

\noindent and similarly

\Centerline{$\Expn(g_k(Z))
=\sum_{j=0}^nc_je^{-iy_jk\delta}\phi_Z(y_j)$,}

\noindent so

\Centerline{$|\Expn(g_k(X))-\Expn(g_k(Z))|
\le\sum_{j=0}^n|c_j||\phi_X(y_j)-\phi_Z(y_j)|
\le\sum_{j=0}^n|c_j|\eta\le\delta$.}

\noindent Next,

\Centerline{$|h_k(x)-g_k(x)|\le\delta$ for every
$x\in[-b-(m+1)\delta+k\delta,b+m\delta+k\delta]\supseteq[-b,b]$,}

\Centerline{$|h_k(x)-g_k(x)|\le 2$ for every $x$,}

\Centerline{$\Pr(|X|\ge b)\le\Bover{M}{b}=\delta$,}

\noindent so $\Expn(|h_k(X)-g_k(X)|)\le 3\delta$;  and similarly
$\Expn(|h_k(Z)-g_k(Z)|)\le 3\delta$.   Putting these together,

\Centerline{$|\Expn(h_k(X))-\Expn(h_k(Z))|\le 7\delta=\epsilon$}

\noindent whenever $-m\le k\le m+1$.

\medskip

{\bf (c)} Now suppose that $-b\le a\le b$.   Then there is a $k$ such
that $-m\le k\le m+1$ and $a\le k\delta\le a+\delta$.   Since

\Centerline{$\chi\ocint{-\infty,a}
\le\chi\ocint{-\infty,k\delta}
\le h_k
\le\chi\ocint{-\infty,(k+1)\delta}
\le\chi\ocint{-\infty,a+2\delta}$,}

\noindent we must have

\Centerline{$\Pr(X\le a)\le\Expn(h_k(X))$,}

\Centerline{$\Expn(h_{k}(Z))\le\Pr(Z\le a+2\delta)\le\Pr(Z\le
a+\epsilon)$.}

\noindent But this means that

\Centerline{$\Pr(X\le a)\le\Expn(h_k(X))\le\Expn(h_k(Z))+\epsilon
\le\Pr(Z\le a+\epsilon)+\epsilon$}

\noindent whenever $a\in[-b,b]$.

\medskip

{\bf (d)} As for the cases $a\ge b$, $a\le -b$,
we surely have

\Centerline{$b(1-F_Z(b))=b\Pr(Z>b)\le\Expn(|Z|)\le M$,}

\noindent so if $a\ge b$ then

\Centerline{$F_X(a)\le 1\le F_Z(a)+1-F_Z(b)\le F_Z(a)+\Bover{M}{b}
=F_Z(a)+\delta\le F_Z(a+\epsilon)+\epsilon$.}

\noindent Similarly,

\Centerline{$bF_X(-b)\le \Expn(|X|)\le M$,}

\noindent so

\Centerline{$F_X(a)\le\delta\le F_Z(a+\epsilon)+\epsilon$}

\noindent  for every $a\le -b$.
This completes the proof.
}%end of proof of 285P

\vleader{72pt}{285Q}{Law of Rare Events:  Theorem} For any $M\ge 0$ and
$\epsilon>0$ there is a $\delta>0$
such that whenever $X_0,\ldots,X_n$ are independent $\{0,1\}$-valued
random variables with $\Pr(X_k=1)=p_k\le\delta$ for every $k\le n$ and
$\sum_{k=0}^np_k=\lambda\le M$, and $X=X_0+\ldots+X_n$, then

\Centerline{$|\Pr(X=m)-\Bover{\lambda^m}{m!}e^{-\lambda}|\le\epsilon$}

\noindent for every $m\in\Bbb N$.

\proof{{\bf (a)} We should begin by calculating some
characteristic functions.   First, the characteristic function $\phi_k$
of $X_k$ will be given by

\Centerline{$\phi_k(y)=(1-p_k)e^{iy0}+p_ke^{iy1}=1+p_k(e^{iy}-1)$.}

\noindent Next, if $Z$ is a Poisson random variable with parameter
$\lambda$ (that is, if $\Pr(Z=m)=\lambda^me^{-\lambda}/m!$ for every
$m\in\Bbb N$;  all you need to know at this point about the Poisson
distribution is
that $\sum_{m=0}^{\infty}\lambda^me^{-\lambda}/m!=1$), then its
characteristic function $\phi_Z$ is given by

\Centerline{$\phi_Z(y)
=\sum_{m=0}^{\infty}\Bover{\lambda^m}{m!}e^{-\lambda}e^{iym}
=e^{-\lambda}\sum_{m=0}^{\infty}\Bover{(\lambda e^{iy})^m}{m!}
=e^{-\lambda}e^{\lambda e^{iy}}
=e^{\lambda(e^{iy}-1)}$.}

\medskip

{\bf (b)} Before getting down to $\delta$'s and $\eta$'s, I show how to
estimate $\phi_X(y)-\phi_Z(y)$.   We know that

\Centerline{$\phi_X(y)=\prod_{k=0}^n\phi_k(y)$}

\noindent (using 285I), while

\Centerline{$\phi_Z(y)=\prod_{k=0}^ne^{p_k(e^{iy}-1)}$.}

\noindent Because $\phi_k(y)$, $e^{p_k(e^{iy}-1)}$ all have modulus at
most $1$ (we have

\Centerline{$|e^{p_k(e^{iy}-1)}|=e^{-p_k(1-\cos y)}\le 1$,)}

\noindent 285O tells us that

\Centerline{$|\phi_X(y)-\phi_Z(y)|
\le\sum_{k=0}^n|\phi_k(y)-e^{p_k(e^{iy}-1)}|
=\sum_{k=0}^n|e^{p_k(e^{iy}-1)}-1-p_k(e^{iy}-1)|$.}

\medskip

{\bf (c)} So we have a little bit of analysis to do.   To estimate
$|e^z-1-z|$ where $\Real z\le 0$, consider the function

\Centerline{$g(t)=\Real(c(e^{tz}-1-tz))$}

\noindent where $|c|=1$.   We have $g(0)=g'(0)=0$ and

\Centerline{$|g''(t)|=|\Real(c(z^2e^{tz}))|\le|c||z^2||e^{tz}|\le|z|^2$}

\noindent for every $t\ge 0$, so that

\Centerline{$|g(1)|\le\Bover12|z|^2$}

\noindent by the (real-valued) Taylor theorem with remainder, or
otherwise.   As $c$ is arbitrary,

\Centerline{$|e^{z}-1-z|\le\Bover12|z|^2$}

\noindent whenever $\Real z\le 0$.   In particular,

\Centerline{$|e^{p_k(e^{iy}-1)}-1-p_k(e^{iy}-1)|
\le\Bover12p_k^2|e^{iy}-1|^2\le 2p_k^2$}

\noindent for each $k$, and

\Centerline{$|\phi_X(y)-\phi_Z(y)|
\le\sum_{k=0}^n|e^{p_k(e^{iy}-1)}-1-p_k(e^{iy}-1)|
\le 2\sum_{k=0}^np_k^2$}

\noindent for each $y\in\Bbb R$.

\medskip

{\bf (d)} Now for the detailed estimates.   Given $M\ge 0$ and
$\epsilon>0$, let $\eta>0$ and $y_0,\ldots,y_l\in\Bbb R$ be such that

\Centerline{$\Pr(X\le a)\le\Pr(Z\le a+\Bover12)+\Bover{\epsilon}2$}

\noindent whenever $X$, $Z$ are real-valued random variables,
$\Expn(|X|)\le M$, $\Expn(|Z|)\le M$ and
$|\phi_X(y_j)-\phi_X(y_j)|\le\eta$ for every $j\le l$ (285P).   Take
$\delta=\bover{\eta}{2M+1}$
and suppose that $X_0,\ldots,X_n$ are independent
$\{0,1\}$-valued random variables with $\Pr(X_k=1)=p_k\le\delta$ for
every $k\le n$, where $\lambda=\sum_{k=0}^np_k$ is less than or equal to
$M$.   Set $X=X_0+\ldots+X_n$
and let $Z$ be a Poisson random variable with parameter $\lambda$;  then
by the arguments of (a)-(c),

\Centerline{$|\phi_X(y)-\phi_Z(y)|\le 2\sum_{k=0}^np_k^2
\le 2\delta\sum_{k=0}^np_k=2\delta\lambda\le\eta$}

\noindent for every $y\in\Bbb R$.   Also

\Centerline{$\Expn(|X|)=\Expn(X)=\sum_{k=0}^np_k=\lambda\le M$,}

\Centerline{$\Expn(|Z|)=\Expn(Z)
=\sum_{m=0}^{\infty}m\Bover{\lambda^m}{m!}e^{-\lambda}
=e^{-\lambda}\sum_{m=1}^{\infty}\Bover{\lambda^m}{(m-1)!}
=e^{-\lambda}\sum_{m=0}^{\infty}\Bover{\lambda^{m+1}}{m!}
=\lambda\le M$.}

\noindent So

\Centerline{$\Pr(X\le a)\le\Pr(Z\le a+\Bover12)+\Bover{\epsilon}2$,}

\Centerline{$\Pr(Z\le a)\le\Pr(X\le a+\Bover12)+\Bover{\epsilon}2$}

\noindent for every $a$.   But as both $X$ and $Z$ take all their values
in $\Bbb N$,

\Centerline{$|\Pr(X\le m)-\Pr(Z\le m)|\le\Bover{\epsilon}2$}

\noindent for every $m\in\Bbb N$, and

\Centerline{$|\Pr(X=m)-\Bover{\lambda^m}{m!}e^{-\lambda}|
=|\Pr(X=m)-\Pr(Z=m)|
\le\epsilon$}

\noindent for every $m\in\Bbb N$, as required.
}%end of proof of 285Q

\leader{285R}{Convolutions}\dvro{ If}{ Recall from 257A that if}
$\nu$, $\tilde\nu$ are Radon probability measures on $\BbbR^r$ then
%\tilde\nu  not  \nuprime
%because we have  \tilde f  for its R-N deriviative
\dvro{$\phi_{\nu*\tilde\nu}(y)
=\phi_{\nu}(y)\phi_{\tilde\nu}(y)$
for every $y\in\BbbR^r$.}{they have a convolution
$\nu*\tilde\nu$ defined by writing

\Centerline{$(\nu*\tilde\nu)(E)=(\nu\times\tilde\nu)\{(x,y):x+y\in E\}$}

\noindent for every Borel set $E\subseteq\BbbR^r$, which is also a
Radon probability measure.   We can readily compute the characteristic
function $\phi_{\nu*\tilde\nu}$ from 257B:  we have

$$\eqalign{\phi_{\nu*\tilde\nu}(y)
&=\int e^{iy\dotproduct x}(\nu*\tilde\nu)(dx)
=\int e^{iy\dotproduct (x+x')}\nu(dx)\tilde\nu(dx')\cr
&=\int e^{iy\dotproduct x}e^{iy\dotproduct x'}\nu(dx)\tilde\nu(dx')
=\int e^{iy\dotproduct x}\nu(dx)
  \int e^{iy\dotproduct x'}\tilde\nu(dx')
=\phi_{\nu}(y)\phi_{\tilde\nu}(y)}$$

\noindent for every $y\in\BbbR^r$.
(Thus convolution of measures corresponds to pointwise
multiplication of characteristic functions, just as convolution of
functions corresponds to pointwise multiplication of Fourier
transforms.)   Recalling that the sum of independent random variables
corresponds to convolution of their distributions (272T), this
gives another way of looking at 285I.   Remember also that if $\nu$,
$\tilde\nu$ have Radon-Nikod\'ym derivatives $f$, $\tilde f$ with
respect to Lebesgue
measure then $f*\tilde f$ is a Radon-Nikod\'ym derivative of
$\nu*\tilde\nu$ (257F).
}%end of dvro

\leader{285S}{The vague topology and pointwise
convergence of characteristic functions}\dvro{ Write}{ In 285L we saw
that a sequence
$\sequencen{\nu_n}$ of Radon probability measures on $\BbbR^r$
converges in the vague topology to a Radon probability measure $\nu$ if
and only if

\Centerline{$\lim_{n\to\infty}\int e^{iy\dotproduct
x}\nu_n(dx)=\int
e^{iy\dotproduct x}\nu(dx)$}

\noindent for every $y\in\BbbR^r$;  that is, iff

\Centerline{$\lim_{n\to\infty}\rho'_y(\nu_n,\nu)=0$ for every $y\in\Bbb
R^r$,}

\noindent writing}

\Centerline{$\rho'_y(\nu,\nuprime)=|\int e^{iy\dotproduct x}\nu(dx)
-\int e^{iy\dotproduct x}\nuprime(dx)|$}

\noindent for Radon probability measures $\nu$, $\nuprime$ on $\BbbR^r$ and
$y\in\BbbR^r$.   \dvro{Write}{It is natural to ask whether the
pseudometrics
$\rho'_y$ actually define the vague topology.   Writing} $\frak T$ for
the vague topology and $\frak S$ for the topology defined by
$\{\rho'_y:y\in\BbbR^r\}$\cmmnt{, we surely have
$\frak S\subseteq\frak T$,
just because every $\rho'_y$ is one of the pseudometrics used in the
definition of $\frak T$.   Also we know that $\frak S$ and $\frak T$
give the same convergent sequences, and incidentally that $\frak T$ is
metrizable (see 285Xt).   But all this does not quite amount to saying
that the two topologies are the same, and indeed they are not, as the
next result shows.}%end of comment

\leader{285T}{Proposition} Suppose that $y_0,\ldots,y_n\in\Bbb R$
and $\eta>0$.   Then there are infinitely many $m\in\Bbb N$ such that
$|1-e^{iy_km}|\le\eta$ for every $k\le n$.

\proof{ Let $\eta_1,\ldots,\eta_r\in\Bbb R$ be such that
$1=\eta_0,\eta_1,\ldots,\eta_r$ are linearly independent over $\Bbb Q$
and every $y_k/2\pi$ is a linear combination of the $\eta_j$ over
$\Bbb Q$;  say $y_k=2\pi\sum_{j=0}^rq_{kj}\eta_j$ where every
$q_{kj}\in\Bbb Q$.   Express the $q_{kj}$ as $p_{kj}/p$ where each
$p_{kj}\in\Bbb Z$
and $p\in\Bbb N\setminus\{0\}$.   Set
$M=\max_{k\le n}\sum_{j=0}^r|p_{kj}|$.

Take any $m_0\in\Bbb N$ and let $\delta>0$ be such that
$|1-e^{2\pi ix}|\le\eta$ whenever $|x|\le2\pi M\delta$.   By Weyl's
Equidistribution
Theorem (281N), there are infinitely many $m$ such that
$\fraction{m\eta_j}\le\delta$ whenever $1\le j\le r$;  in particular,
there is such an $m\ge m_0$.    Set $m_j=\lfloor m\eta_j\rfloor$, 
so that $|m\eta_j-m_j|\le\delta$ for $0\le j\le r$.   Then

\Centerline{$|mpy_k-2\pi\sum_{j=0}^rp_{kj}m_j|
\le2\pi\sum_{j=0}^r|p_{kj}||m\eta_j-m_j|
\le 2\pi M\delta$,}

\noindent so that

\Centerline{$|1-e^{iy_kmp}|
=|1-\exp(i(mpy_k-2\pi\sum_{j=0}^rp_{kj}m_j))|
\le\eta$}

\noindent for every $k\le n$.
As $mp\ge m_0$ and $m_0$ is arbitrary, this proves the result.
}%end of proof of 285T

\leader{285U}{Corollary} The topologies $\frak S$ and $\frak T$ on the
space of Radon probability measures on $\Bbb R$, as described in 285S,
are different.

\proof{ Let $\delta_x$ be the Dirac measure on $\Bbb R$ concentrated at
$x$.   By 285T, every
member of $\frak S$ which contains $\delta_0$ also contains $\delta_m$
for infinitely many $m\in\Bbb N$.   On the other hand, the set

\Centerline{$G=\{\nu:\int e^{-x^2}\nu(dx)>\Bover12\}$}

\noindent is a member of $\frak T$, containing $\delta_0$, which does
not contain $\delta_m$ for any integer $m\ne 0$.
So $G\in\frak T\setminus\frak S$ and $\frak T\ne\frak S$.
}%end of proof of 285V


\exercises{
\leader{285X}{Basic exercises (a)}
%\spheader 285Xa
Let $\nu$ be a Radon probability
measure on $\BbbR^r$, where $r\ge 1$, and suppose that
$\int\|x\|\nu(dx)<\infty$.   Show that the characteristic function
$\phi$ of $\nu$ is differentiable (in the full sense of 262Fa) and
that $\pd{\phi}{\eta_j}(y)=i\int\xi_je^{iy\dotproduct x}\nu(dx)$ for
every
$j\le r$ and $y\in\BbbR^r$, using $\xi_j$, $\eta_j$ to represent the
coordinates of $x$ and $y$ as usual.
%285A

\sqheader 285Xb Let $\pmb{X}=(X_1,\ldots,X_r)$ be a family of
real-valued random variables, with characteristic function
$\phi_{\pmb{X}}$.   Show that the characteristic function $\phi_{X_j}$
of $X_j$ is given by

\Centerline{$\phi_{X_j}(y)=\phi_{\pmb{X}}(ye_j)$ for every $y\in\Bbb
R$,}

\noindent where $e_j$ is the $j$th unit vector of $\BbbR^r$.
%285A

\sqheader 285Xc Let $X$ be a real-valued random variable and
$\phi_X$ its characteristic function.   Show that

\Centerline{$\phi_{aX+b}(y)=e^{iyb}\phi_X(ay)$}

\noindent for any $a$, $b$, $y\in\Bbb R$.
%285A

\spheader 285Xd\dvAnew{2014}
Let $X$ be a real-valued random variable which is not
essentially constant, and $\phi$ its characteristic function.   Show that
$|\phi(y)|<1$ for all but countably many $y\in\Bbb R$.
\Hint{the support (256Xf) of the distribution of $X$ has distinct
points $x$, $x'$ and if $e^{iyx}\ne e^{iyx'}$ then $|\phi(y)|<1$.}
%285D

\spheader 285Xe Let $X$ be a real-valued random variable and
$\phi$ its characteristic function.

\quad{(i)} Show that for any integrable complex-valued function $h$
on $\Bbb R$,

\Centerline{$\Expn(\varhat{h}(X))=\Bover1{\sqrt{2\pi}}
\int_{-\infty}^{\infty}\phi(-y)h(y)dy$,}

\noindent writing $\varhat{h}$ for the Fourier transform of $h$.

\quad{(ii)} Show that for any rapidly decreasing test function $h$,

\Centerline{$\Expn(h(X))
=\Bover1{\sqrt{2\pi}}\int_{-\infty}^{\infty}
  \phi(y)\varhat{h}(y)dy$.}
%285D

\spheader 285Xf Let $\nu$ be a Radon probability measure on
$\Bbb R$, and suppose that its characteristic function $\phi$ is
square-integrable.   Show that $\nu$ is an indefinite-integral measure
over Lebesgue measure and that its Radon-Nikod\'ym derivatives are also
square-integrable.   \Hint{use
284O to find a square-integrable $f$ such that
$\int f\times h=\bover1{\sqrt{2\pi}}\int\phi\times\varhat{h}$ for every
rapidly
decreasing test function $h$, and ideas from the proof of 284G to show
that $\int_a^bf=\nu\ooint{a,b}$ whenever $a<b$ in $\Bbb R$.}
%285D 285Xe

\spheader 285Xg\dvAformerly{2{}85Yp} Let $\nu$ be a Radon probability
measure on $\BbbR^r$ with bounded support (definition: 256Xf).   Show that
its characteristic function is smooth.
%285D

\spheader 285Xh Let $X$ be a normal random variable with expectation $a$
and variance $\sigma^2$.   Show that $\Expn(e^X)=\exp(a+\bover12\sigma^2)$.
%285E

\sqheader 285Xi Let $\pmb{X}=(X_1,\ldots,X_r)$ be a family of
real-valued random variables with characteristic function
$\phi_{\pmb{X}}$.   Suppose that $\phi_{\pmb{X}}$ is expressible in the
form

\Centerline{$\phi_{\pmb{X}}(y)=\prod_{j=1}^r\phi_j(\eta_j)$}

\noindent for some functions $\phi_1,\ldots,\phi_r$, writing
$y=(\eta_1,\ldots,\eta_r)$ as usual.   Show that $X_1,\ldots,X_r$ are
independent.   \Hint{show that the $\phi_j$ must be multiples of the
characteristic functions of the $X_j$;  now show that the distribution
of $\pmb{X}$ has the same characteristic function as the product of the
distributions of the $X_j$.}
%285I

\spheader 285Xj Let $X_1$, $X_2$ be independent real-valued random
variables with the same distribution, and $\phi$ the characteristic
function of $X_1-X_2$.   Show that $\phi(t)=\phi(-t)\ge 0$ for every
$t\in\Bbb R$.
%285I

\spheader 285Xk Let $\nu$ be a Radon probability measure on
$\Bbb R$, with characteristic function $\phi$.   Show that

\Centerline{$\Bover12(\nu[c,d]+\nu\ooint{c,d})
=\Bover{i}{2\pi}\lim_{a\to\infty}
  \int_{-a}^a\Bover{e^{-idy}-e^{-icy}}{y}\phi(y)dy$}

\noindent whenever $c<d$ in $\Bbb R$.
\Hint{use part (a) of the proof of 283F.}
%285J

\spheader 285Xl Let $X$ be a real-valued random variable and
$\phi_X$ its characteristic function.   Show that

\Centerline{$\Pr(|X|\ge a)
\le 7a\int_0^{1/a}(1-\Real(\phi_X(y))dy$}

\noindent for every $a>0$.
%285J

\spheader 285Xm We say that a set $Q$ of Radon probability
measures on $\Bbb R$ is {\bf uniformly tight} if for every $\epsilon>0$
there is
an $M\ge 0$ such that $\nu(\Bbb R\setminus[-M,M])\le\epsilon$ for every
$\nu\in Q$.    Show that if $Q$ is any uniformly tight family of Radon
probability
measures on $\Bbb R$, and $\epsilon>0$, then there are $\eta>0$ and
$y_0,\ldots,y_n\in\Bbb R$ such that

\Centerline{$\nu\ocint{-\infty,a}
\le\nuprime\ocint{-\infty,a+\epsilon}+\epsilon$}

\noindent whenever $\nu$, $\nuprime\in Q$ and
$|\phi_{\nu}(y_j)-\phi_{\nuprime}(y_j)|\le\eta$ for every $j\le n$, writing
$\phi_{\nu}$ for the characteristic function of $\nu$.
%285M

\spheader 285Xn Let $\sequencen{\nu_n}$ be a sequence of Radon
probability measures on $\Bbb R$, and suppose that it converges for the
vague topology to a Radon probability measure $\nu$.   Show that
$\{\nu\}\cup\{\nu_n:n\in\Bbb N\}$ is uniformly tight in the sense of
285Xm.
%285M

\sqheader 285Xo Let $\nu$, $\nuprime$ be two totally finite Radon measures
on $\BbbR^r$ which agree on all closed half-spaces, that is, sets of
the form $\{x:x\dotproduct y\ge c\}$ for $c\in\Bbb R$, $y\in\BbbR^r$.
Show that $\nu=\nuprime$.   \Hint{reduce to the case $\nu\BbbR^r=\nuprime\Bbb
R^r=1$ and use 285M.}
%285M

\sqheader 285Xp\dvAformerly{2{}85Xm}
For $\gamma>0$, the {\bf Cauchy distribution} with
centre $0$ and scale parameter $\gamma$ is the Radon probability measure
$\nu_{\gamma}$ defined by the formula

\Centerline{$\nu_{\gamma}(E)
=\Bover{\gamma}{\pi}\int_E\Bover1{\gamma^2+t^2}dt$.}

\noindent (i) Show that if $X$ is a random variable with distribution
$\nu_{\gamma}$ then $\Pr(X\ge 0)=\Pr(|X|\ge\gamma)=\bover12$.
(ii) Show that the characteristic function of
$\nu_{\gamma}$ is $y\mapsto e^{-\gamma|y|}$.   \Hint{283Xq.}   (iii)
Show that if $X$ and $Y$ are independent random variables with Cauchy
distributions, both centered at $0$ and with scale parameters $\gamma$,
$\delta$ respectively, and $\alpha$, $\beta$ are not both $0$, then
$\alpha X+\beta Y$ has a Cauchy distribution centered at $0$ and with
scale parameter $|\alpha|\gamma+|\beta|\delta$.   (iv) Show that if $X$
and $Y$ are independent normally distributed random variables with expectation $0$ then $X/Y$ has a Cauchy distribution.
%285M

\sqheader 285Xq Let $X_1,X_2,\ldots$ be an independent identically
distributed sequence of random variables, all with zero expectation and
variance $1$;  let $\phi$ be their common characteristic function.   For
each $n\ge 1$, set $S_n=\bover1{\sqrt{n}}(X_1+\ldots+X_n)$.

\quad{(i)} Show that the characteristic function $\phi_n$ of $S_n$
is given by the formula $\phi_n(y)=(\phi(\Bover{y}{\sqrt{n}}))^{n}$
for each $n$.

\quad{(ii)} Show that
$|\phi_n(y)-e^{-y^2/2}|\le n|\phi(\Bover{y}{\sqrt{n}})-e^{-y^2/2n}|$
for $n\ge 1$ and $y\in\Bbb R$.

\ifdim\pagewidth>467pt\fontdimen3\tenrm=3.33pt\fi
\quad(iii) Setting $h(y)=\phi(y)-e^{-y^2/2}$, show that
$h(0)=h'(0)=h''(0)=0$ and therefore that
$\lim_{n\to\infty} nh(y/\sqrt{n})=0$, so that
$\lim_{n\to\infty}\phi_n(y)=e^{-y^2/2}$ for every $y\in\Bbb R$.
\fontdimen3\tenrm=1.67pt

\quad{(iv)} Show that $\lim_{n\to\infty}\Pr(S_n\le a)
=\Bover1{\sqrt{2\pi}}\int_{-\infty}^ae^{-x^2/2}dx$ for every
$a\in\Bbb R$.
%285N

\sqheader 285Xr\dvAformerly{2{}85Xm}
A random variable $X$ has a {\bf Poisson distribution}
with parameter $\lambda>0$ if $\Pr(X=n)=e^{-\lambda}\lambda^n/n!$ for
every $n\in\Bbb N$.    (i) Show that in this case
$\Expn(X)=\Var(X)=\lambda$.   (ii) Show that if $X$ and $Y$ are
independent random variables with Poisson distributions then $X+Y$ has a
Poisson distribution.   (iii) Find a proof of (ii) based on 285Q.
%285Q

\sqheader 285Xs For $x\in\BbbR^r$, let $\delta_x$ be the Dirac measure
on $\BbbR^r$ concentrated at $x$.
Show that $\delta_x*\delta_y=\delta_{x+y}$ for all $x$, $y\in\BbbR^r$.
%285R

\spheader 285Xt Let $P$ be the set of Radon probability measures
on $\BbbR^r$.   For $y\in\BbbR^r$, set
$\rho'_y(\nu,\nuprime)=|\phi_{\nu}(y)-\phi_{\nuprime}(y)|$ for all $\nu$,
$\nuprime\in P$, writing $\phi_{\nu}$ for the characteristic function of
$\nu$.   Set $\psi(x)=\bover1{(\sqrt{2\pi})^r}e^{-x\dotproduct x/2}$ for
$x\in\BbbR^r$.    Show that the vague topology on $P$ is
defined by the family $\{\rho_{\psi}\}\cup\{\rho'_y:y\in\Bbb Q^r\}$,
defining $\rho_{\psi}$ as in 285K, and is therefore metrizable.
\Hint{281K;  cf.\ 285Xm.}
%285S

\sqheader 285Xu\dvAformerly{2{}85Xr}
Let $\phi:\BbbR^r\to\Bbb C$ be the
characteristic function of a Radon probability measure on $\BbbR^r$.
Show that $\phi(0)=1$ and that
$\sum_{j=0}^n\sum_{k=0}^nc_j\bar c_k\phi(a_j-a_k)\ge 0$ whenever
$a_0,\ldots,a_n\in\BbbR^r$ and $c_0,\ldots,c_n\in\Bbb C$.
(`Bochner's theorem' states that these conditions are
sufficient, as well as necessary, for $\phi$ to be a characteristic
function;  see 445N in Volume 4.)
%285*

\leader{285Y}{Further exercises (a)}
%\spheader 285Ya
Let $\nu$ be a Radon probability measure on
$\BbbR^r$.   Write

\Centerline{$\varhat{\nu}(y)=\Bover1{(\sqrt{2\pi})^r}
\int e^{-iy\dotproduct x}\nu(dx)$}

\noindent for every $y\in\BbbR^r$.

\quad{(i)} Writing $\phi_{\nu}$ for the characteristic function of
$\nu$, show that $\varhat{\nu}(y)
=\Bover1{(\sqrt{2\pi})^r}\phi_{\nu}(-y)$ for every $y\in\BbbR^r$.

\quad{(ii)} Show that
$\int h(y)\varhat{\nu}(y)dy=\int\varhat h(x)\nu(dx)$
for any Lebesgue integrable complex-valued function $h$ on
$\BbbR^r$, defining the Fourier transform $\varhat{h}$ as in 283Wa.

\quad{(iii)} Show that $\int h(x)\nu(dx)
=\int\varcheck{h}(y)\varhat{\nu}(y)dy$ for any rapidly decreasing test
function $h$ on $\BbbR^r$.

\quad{(iv)} Show that if $\nu$ is an indefinite-integral measure over
Lebesgue measure, with Radon-Nikod\'ym derivative $f$, then
$\varhat{\nu}$ is the Fourier transform of $f$.
%285D

\spheader 285Yb Let $\nu$ be a Radon probability
measure on $\BbbR^r$, with characteristic function $\phi$.   Show that
whenever $c\le d$ in $\BbbR^r$ then

$$\bigl(\bover{i}{2\pi}\bigr)^r\lim_{\alpha_1,\ldots,\alpha_r\to\infty}
\int_{[-a,a]}\bigl(\prod_{j=1}^r
\Bover{e^{-i\delta_j\eta_j}-e^{-i\gamma_j\eta_j}}{\eta_j}\bigr)
\phi(y)dy$$

\noindent exists and lies between $\nu\ooint{c,d}$ and $\nu[c,d]$,
writing $a=(\alpha_1,\ldots,\alpha_r)$ and
$\ooint{c,d}=\prod_{j\le r}\ooint{\gamma_j,\delta_j}$ if
$c=(\gamma_1,\ldots,\gamma_r)$ and $d=(\delta_1,\ldots,\delta_r)$.
%285Xk, 285J, 283Wc

\spheader 285Yc Let $\sequencen{X_n}$ be an independent identically
distributed sequence of
(not-essentially-constant) random variables.   Show that
$\lim_{n\to\infty}\Pr(|\sum_{k=0}^nX_k|\le\alpha)=0$ for every
$\alpha\in\Bbb R$.
%285Xd, 285Xk, 285J

\spheader 285Yd For Radon probability measures $\nu$, $\nuprime$ on
$\BbbR^r$ set

$$\eqalign{\rho(\nu,\nuprime)=\inf\{\epsilon:\epsilon\ge 0,\,
\nu\ocint{-\infty,a}\le\nuprime\ocint{-\infty,a+\epsilon\tbf{1}}+\epsilon
&\le\nu\ocint{-\infty,a+2\epsilon\tbf{1}}+2\epsilon\cr
&\text{for every }a\in\BbbR^r\},\cr}$$

\noindent writing $\ocint{-\infty,a}
=\{(\xi_1,\ldots,\xi_r):\xi_j\le\alpha_j$ for every $j\le r\}$
when $a=(\alpha_1,\ldots,\alpha_r)$, and
$\tbf{1}=(1,\ldots,1)\in\BbbR^r$.   Show that $\rho$ is a metric on the
set of Radon probability measures on $\BbbR^r$, and that the topology it
defines is the vague topology.   (Cf.\ 274Yc.)
%285K

\spheader 285Ye Let $r\ge 1$ and let $P$ be the set of Radon
probability measures on $\BbbR^r$.   For $m\in\Bbb N$ let $\rho^*_m$ be
the pseudometric on $P$ defined by setting
$\rho^*_m(\nu,\nuprime)=\sup_{\|y\|\le m}|\phi_{\nu}(y)-\phi_{\nuprime}(y)|$
for $\nu$, $\nuprime\in P$, writing $\phi_{\nu}$ for the
characteristic function of $\nu$.   Show that $\{\rho^*_m:m\in\Bbb N\}$
defines the vague topology on $P$.
%285K

\spheader 285Yf Let $r\ge 1$.   We say that a set $Q$ of Radon
probability measures on $\BbbR^r$ is {\bf uniformly tight} if for every
$\epsilon>0$ there is a compact set $K\subseteq\BbbR^r$ such that
$\nu(\BbbR^r\setminus K)\le\epsilon$ for every $\nu\in Q$.
Show that if $Q$ is any uniformly tight family of Radon probability
measures on $\BbbR^r$, and $\epsilon>0$, then there are $\eta>0$,
$y_0,\ldots,y_n\in\BbbR^r$ such that
$\nu\ocint{-\infty,a}\le\nuprime\ocint{-\infty,a+\epsilon\tbf{1}}+\epsilon$
whenever $\nu$, $\nuprime\in Q$ and $a\in\BbbR^r$ and
$|\phi_{\nu}(y_j)-\phi_{\nuprime}(y_j)|\le\eta$ for every $j\le n$, writing
$\phi_{\nu}$ for the characteristic function of $\nu$.
%285M, 285Xm

\spheader 285Yg Show that for any $M\ge 0$ the set of Radon
probability measures $\nu$ on $\BbbR^r$ such that
$\int\|x\|\nu(dx)\le M$ is uniformly tight in the sense of 285Yf.
%285Yf, 285Xm, 285M

\spheader 285Yh Let $C_b(\BbbR^r)$ be the Banach space of
bounded continuous real-valued functions on $\BbbR^r$.

\quad{(i)} Show that any Radon probability measure $\nu$ on
$\Bbb R^r$ corresponds to a continuous linear functional
$h_{\nu}:C_b(\Bbb R^r)\to\Bbb R$, writing $h_{\nu}(f)=\int fd\nu$ for
$f\in C_b(\Bbb R^r)$.

\quad{(ii)} Show that if $h_{\nu}=h_{\nuprime}$ then $\nu=\nuprime$.

\quad{(iii)} Show that the vague topology on the set of Radon
probability measures corresponds to the weak* topology on the dual
$(C_b(\BbbR^r))^*$ of $C_b(\BbbR^r)$.
%285M mt28bits

\spheader 285Yi Let $r\ge 1$ and let $P$ be the set of Radon
probability measures on $\BbbR^r$.   For $m\in\Bbb N$ let
$\tilde\rho^*_m$ be the pseudometric on $P$ defined by setting

\Centerline{$\tilde\rho^*_m(\nu,\nuprime)
=\int_{\{y:\|y\|\le m\}}|\phi_{\nu}(y)-\phi_{\nuprime}(y)|dy$}

\noindent for $\nu$, $\nuprime\in P$, writing $\phi_{\nu}$ for the
characteristic function of $\nu$.   Show that
$\{\tilde\rho^*_m:m\in\Bbb N\}$ defines the vague topology on $P$.
%285M mt28bits

\spheader 285Yj Let $(\Omega,\Sigma,\mu)$ be a probability space.
Suppose that $\sequencen{X_n}$ is a sequence of real-valued random
variables on $\Omega$,
and $X$ another real-valued random variable on $\Omega$;  let $\phi_{X_n}$,
$\phi_X$ be the corresponding characteristic functions.   Show that the
following are equiveridical:
(i) $\lim_{n\to\infty}\Expn(h(X_n))=\Expn(h(X))$ for every bounded
continuous function $h:\Bbb R\to\Bbb R$;  (ii)
$\lim_{n\to\infty}\phi_{X_n}(y)=\phi_X(y)$ for every $y\in\Bbb R$.
%285Xm 285M

\spheader 285Yk Let $(\Omega,\Sigma,\mu)$ be a probability space, and $P$
the set of Radon probability measures on $\Bbb R$.   (i) Show that we have
a function $\psi:L^0(\mu)\to P$ defined by saying that
$\psi(X^{\ssbullet})$ is the distribution of $X$ whenever $X$ is a
real-valued random variable on $\Omega$.   (ii) Show that $\psi$ is
continuous for the topology of convergence in measure on $L^0(\mu)$ and the
vague topology on $P$.   (Compare 271Yd.)
%285Yj 285M

\spheader 285Yl Let $X$ be a real-valued random variable with
finite variance.   Show that for any $\eta\ge 0$,

\Centerline{$|\phi(y)-1-iy\Expn(X)+\Bover12y^2\Expn(X^2)|
\le\Bover16\eta|y|^3\Expn(X^2)
+y^2\Expn(\psi_{\eta}(X))$,}

\noindent writing $\phi$ for the characteristic function of $X$ and
$\psi_{\eta}(x)=0$ for $|x|\le\eta$, $x^2$ for $|x|>\eta$.
%285N

\spheader 285Ym Suppose that $\epsilon\ge\delta>0$ and that
$X_0,\ldots,X_n$ are independent
real-valued random variables such that

\Centerline{$\Expn(X_k)=0$ for every $k\le n$,
\quad$\sum_{k=0}^n\Var(X_k)=1$,
\quad$\sum_{k=0}^n\Expn(\psi_{\delta}(X_k))\le\delta$}

\noindent (writing $\psi_{\delta}(x)=0$ if $|x|\le\delta$, $x^2$ if
$|x|>\delta$).   Set $\gamma=\epsilon/\sqrt{\delta^2+\delta}$, and let
$Z$ be a standard normal random variable.   Show that

\Centerline{$|\phi(y)-e^{-y^2/2}|
\le\Bover13\epsilon|y|^3+y^2(\delta+\Expn(\psi_{\gamma}(Z)))$}

\noindent for every $y\in\Bbb R$, writing $\phi$ for the characteristic
function of $X=\sum_{k=0}^nX_k$.
% mt28bits 285Yl, 285N

\spheader 285Yn Show that for every $\epsilon>0$ there is a
$\delta>0$ such that whenever $X_0,\ldots,X_n$ are independent
real-valued random variables such that

\Centerline{$\Expn(X_k)=0$ for every $k\le n$,
\quad$\sum_{k=0}^n\Var(X_k)=1$,
\quad$\sum_{k=0}^n\Expn(\psi_{\delta}(X_k))\le\delta$}

\noindent (writing $\psi_{\delta}(x)=0$ if $|x|\le\delta$, $x^2$ if
$|x|>\delta$), then $|\phi(y)-e^{-y^2/2}|\le\epsilon(y^2+|y^3|)$
for every $y\in\Bbb R$, writing $\phi$ for the characteristic
function of $X=X_0+\ldots+X_n$.
%285Ym, 285N

\spheader 285Yo Use 285Yn to prove Lindeberg's theorem (274F).
%285Yn, 285N

\spheader 285Yp Let $r\ge 1$ and let $P$ be the set of Radon
probability measures on $\BbbR^r$.   Show that convolution, regarded as
a map from $P\times P$ to $P$, is continuous when $P$ is given the vague
topology.
%285R \Hint{281Xa and 257B will help.}

\spheader 285Yq Let $\frak S$ be the topology on $\Bbb R$
defined by $\{\rho'_y:y\in\Bbb R\}$, where
$\rho'_y(x,x')=|e^{iyx}-e^{iyx'}|$ (compare 285S).   Show that addition
and subtraction are continuous for $\frak S$ in the sense of 2A5A.
%285S

\spheader 285Yr\dvAnew{2014} Let $\nu$ be a probability measure on
$\Bbb R$.   Show that
$|\phi_{\nu}(y)-\phi_{\nu}(y')|^2\le 2(1-\Real\phi_{\nu}(y-y'))$ for any
$y$, $y'\in\Bbb R$.
%|\phi_{\nu}(y)-\phi_{\nu}(y')|^2\le\int|e^{iyx)-e^{iy'x}|^2\nu(dx)
%285J

\spheader 285Ys\dvAnew{2014} Let $\sequencen{\nu_n}$ be a sequence of
probability measures on $\Bbb R$.   Set $E=\{y:y\in\Bbb R$,
$\lim_{n\to\infty}\phi_{\nu_n}(y)=1\}$.   (i) Show that $E-E$ and
$E+E$ are included in $E$.   (ii) Show that if $E$ is not Lebesgue
negligible it is the whole of $\Bbb R$.
%285Yr 285J

\spheader 285Yt\dvAformerly{2{}85Xs}
Let $\sequencen{X_n}$ be an independent sequence of
real-valued random variables and set $S_n=\sum_{j=0}^nX_j$ for each
$n\in\Bbb N$.
Suppose that the sequence $\sequencen{\nu_{S_n}}$ of distributions is
convergent for the vague topology to a distribution.
Show that $\sequencen{S_n}$ converges in measure, therefore a.e.
%\Hint{285J, 273B, 285Ys.}
%285I  char fns of  X_n  converge
%  to  1  whenever char fns of  S_n  don't converge to  0
}%end of exercises

\endnotes{
\Notesheader{285} Just as with Fourier transforms, the
power of methods which use the characteristic functions of distributions
is based on three points:  (i) the characteristic function of a
distribution determines the distribution (285M); (ii) the properties of
interest in a distribution are reflected in accessible properties of its
characteristic function (285G, 285I, 285J) (iii) these properties of the
characteristic function are actually {\it different} from the
corresponding properties of the distribution, and are amenable to
different kinds of investigation.   Above all, the fact that (for
sequences!) convergence in the vague topology of distributions
corresponds to pointwise convergence for characteristic functions (285L)
provides us with a path to the classic limit theorems, as in 285Q and
285Xq.   In 285S-285U I show that this result for sequences does not
correspond immediately to any alternative characterization of the vague
topology, though it can be adapted in more than one way to give such a
characterization (see 285Ye-285Yi).

Concerning the Central Limit Theorem there is one conspicuous difference
between the method suggested here and that of \S274.   The previous
approach offered at least a theoretical possibility of giving an
explicit formula for $\delta$ in 274F as a function of $\epsilon$, and
hence an estimate of the rate of convergence to be expected in the
Central Limit Theorem.   The arguments in the present chapter, involving
as they do an entirely
non-constructive compactness argument in 281A, leave us with no way of
achieving such an estimate.   But in fact the method of characteristic
functions, suitably refined, is the basis of the best estimates known,
such as the Berry-Ess\'een theorem (274Hc).

In 285D I try to show how the characteristic function $\phi_{\nu}$ of a
Radon probability measure can be related to a `Fourier transform'
$\varhat{\nu}$ of $\nu$ which corresponds directly to the Fourier
transforms of functions discussed in \S\S283-284.   If $f$ is a
non-negative Lebesgue integrable function and we take $\nu$ to be the
corresponding indefinite-integral measure, then $\varhat{\nu}=\varhatf$.
Thus the concept
of `Fourier transform of a measure' is a natural extension of the
Fourier transform of an integrable function.   Looking at it from the
other side, the formula of 285Dc shows that $\nu$ can be thought of as
representing the inverse Fourier transform of $\varhat{\nu}$ in the
sense of 284H-284I.   Taking $\nu$ to be the measure which assigns a
mass $1$ to the point $0$, we get the Dirac delta function, with Fourier
transform the constant function $\chi\Bbb R$.   These ideas can be extended
without difficulty to handle convolutions of measures (285R).

It is a striking fact that while there is no satisfactory
characterization of the functions which are Fourier transforms of
integrable functions, there is a characterization of the characteristic
functions of probability distributions.   This is `Bochner's
theorem'.   I give the condition in 285Xu, asking you to prove its
necessity as an exercise;  we already have three-quarters of the
machinery to prove its sufficiency, but the last step will have to wait
for Volume 4.
}

\discrpage

