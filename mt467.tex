\frfilename{mt467.tex}
\versiondate{13.1.10}
\copyrightdate{2000}

\def\chaptername{Pointwise compact sets of measurable functions}
\def\sectionname{Locally uniformly rotund norms}

\def\normY#1{\|#1\|_{\lower1pt\hbox{\fivei Y}}}

%$\normY{y}$\quad  $q_{\normY{\,}}$ \quad  $q_{\normY{\,}}$

\newsection{*467}

In the last section I mentioned Kadec norms.   These are interesting in
themselves, but the reason for including them in this book is that in a
normed space with a Kadec norm the weak topology has the same Borel sets
as the norm topology\cmmnt{ (466Ea)}.
The same will evidently be true of any space which has an equivalent
Kadec norm.   Now Kadec norms
themselves are not uncommon, but equivalent Kadec norms appear in a
striking variety of cases.   Here I describe the principal class of spaces
(the `weakly K-countably determined' Banach spaces, 467H) which have
equivalent
Kadec norms.   In fact they have `locally uniformly rotund' norms, which
are much easier to do calculations with.

Almost everything here is pure functional analysis, mostly taken from
{\smc Deville Godefroy \& Zizler 93}, which is why I have starred the
section.   The word `measure' does not appear until 467P.   At that
point, however, we find ourselves with a striking result
(Schachermayer's theorem) which appears to need the structure theory of
weakly compactly generated Banach spaces developed in
467C-467M. %467C 467D 467E 467F 467G 467H 467I 467J 467K 467L 467M

\leader{467A}{Definition} Let $X$ be a linear space with a norm
$\|\,\|$.   $\|\,\|$ is {\bf locally uniformly rotund} or {\bf locally
uniformly convex} if whenever $\|x\|=1$ and $\epsilon>0$, there is a
$\delta>0$ such that
$\|x-y\|\le\epsilon$ whenever $\|y\|=1$ and $\|x+y\|\ge 2-\delta$.

If $X$ has a locally uniformly rotund norm, then every subspace of $X$
has a locally uniformly rotund norm.   \cmmnt{Of course any
uniformly convex norm (definition:  2A4K)
is locally uniformly rotund.}

\leader{467B}{Proposition} A locally uniformly rotund norm is a Kadec
norm.

\proof{ Let $X$ be a linear space with a locally uniformly rotund norm
$\|\,\|$.   Set $S_X=\{x:\|x\|=1\}$.   Suppose that $G$ is open for the
norm topology and that $x\in G\cap S_X$.   Then there is an $\epsilon>0$
such that $G\supseteq B(x,\epsilon)=\{y:\|y-x\|\le\epsilon\}$.   Let
$\delta>0$ be such that
$\|x-y\|\le\epsilon$ whenever $\|y\|=1$ and $\|x+y\|\ge 2-\delta$.
Now there is an
$f\in X^*$ such that $f(x)=\|f\|=1$ (3A5Ac).   So
$V=\{y:f(y)>1-\delta\}$ is open for the weak topology.   But if
$y\in V\cap S_X$, then
$\|x+y\|\ge f(x+y)\ge 2-\delta$, so $\|x-y\|\le\epsilon$ and $y\in G$.
As $x$ is arbitrary, $G\cap S_X$ is open for the weak topology on $S_X$;
as $G$ is arbitrary, the norm and weak topologies agree on $S_X$.
}%end of proof of 467B

\leader{467C}{A technical device (a)} I will use the following
notation for the rest of the section.   Let $X$ be a linear space and
$p:X\to\coint{0,\infty}$ a seminorm.   Define
$q_p:X\times X\to\coint{0,\infty}$ by setting

\Centerline{$q_p(x,y)=2p(x)^2+2p(y)^2-p(x+y)^2
=(p(x)-p(y))^2+(p(x)+p(y))^2-p(x+y)^2$}

\noindent for $x\in X$.

\spheader 467Cb A norm $\|\,\|$ on $X$ is locally uniformly rotund iff
whenever $x\in X$ and $\epsilon>0$ there is a $\delta>0$ such that
$\|x-y\|\le\epsilon$ whenever $q_{\|\,\|}(x,y)\le\delta$.

\prooflet{\medskip

\Prf{\bf (i)} Suppose that $\|\,\|$ is locally uniformly rotund,
$x\in X$ and $\epsilon>0$.   ($\alpha$) If $x=0$ then
$q_{\|\,\|}(x,y)=\|y\|^2=\|x-y\|^2$ for every $y$ so we can take
$\delta=\epsilon^2$.   ($\beta$) If $x\ne 0$ set $x'=\Bover1{\|x\|}x$.
Let $\eta>0$ be such that
$\|x'-y'\|\le\bover12\epsilon\|x\|$ whenever $\|y'\|=1$ and
$\|x'+y'\|\ge 2-\eta$.   Let $\delta>0$ be such that

\Centerline{$\delta+2\sqrt{\delta}\|x\|\le\eta\|x\|^2$,
\quad$\sqrt{\delta}<\|x\|$,
\quad$\sqrt{\delta}\le\Bover12\epsilon\|x\|^2$.}

\noindent Now if $q_{\|\,\|}(x,y)\le\delta$, we must have

\Centerline{$(\|x\|-\|y\|)^2\le\delta<\|x\|^2$,
\quad$(\|x\|+\|y\|)^2-\|x+y\|^2\le\delta$,}

\noindent so that $y\ne 0$ and

\Centerline{$\bigl|\Bover1{\|x\|}-\Bover1{\|y\|}\bigr|\|y\|
=\Bover{|\|y\|-\|x\||}{\|x\|}
\le\Bover{\sqrt{\delta}}{\|x\|}$,}

\Centerline{$\|x\|+\|y\|-\|x+y\|
\le\Bover{\delta}{\|x\|+\|y\|+\|x+y\|}\le\Bover{\delta}{\|x\|}$.}

\noindent Set $y'=\Bover1{\|y\|}y$, $y''=\Bover1{\|x\|}y$.   Then
$\|y'\|=1$, and

\Centerline{$\|y'-y''\|
=\big|\Bover1{\|y\|}-\Bover1{\|x\|}\bigr|\|y\|
\le\Bover{\sqrt{\delta}}{\|x\|}\le\Bover12\epsilon\|x\|$.}

\noindent Accordingly

$$\eqalign{\|x'+y'\|
&\ge\|x'+y''\|-\|y'-y''\|
\ge\Bover1{\|x\|}\|x+y\|-\Bover{\sqrt{\delta}}{\|x\|}\cr
&\ge\Bover1{\|x\|}(\|x\|+\|y\|-\Bover{\delta}{\|x\|})
   -\Bover{\sqrt{\delta}}{\|x\|}
=1+\Bover{\|y\|}{\|x\|}-\Bover{\delta}{\|x\|^2}
   -\Bover{\sqrt{\delta}}{\|x\|}\cr
&\ge 1+\Bover{\|x\|-\sqrt{\delta}}{\|x\|}-\Bover{\delta}{\|x\|^2}
   -\Bover{\sqrt{\delta}}{\|x\|}
=2-\Bover{\delta}{\|x\|^2}
   -\Bover{2\sqrt{\delta}}{\|x\|}
\ge 2-\eta.\cr}$$

\noindent But this means that $\|x'-y'\|\le\bover12\epsilon\|x\|$, so
that $\|x'-y''\|\le\epsilon\|x\|$ and $\|x-y\|\le\epsilon$.   As $x$ and
$\epsilon$ are arbitrary, the condition is satisfied.

\medskip

\quad{\bf (ii)} Suppose the condition is satisfied.   If $\|x\|=1$ and
$\epsilon>0$, take $\delta\in\ocint{0,2}$ such that $\|x-y\|\le\epsilon$
whenever $q(x,y)\le 4\delta$;  then if $\|y\|=1$ and
$\|x+y\|\ge 2-\delta$,
$q(x,y)=4-\|x+y\|^2\le 4\delta$ and $\|x-y\|\le\epsilon$.   As $x$ and
$\epsilon$ are arbitrary, $\|\,\|$ is locally uniformly rotund.\
\Qed}

\spheader 467Cc\cmmnt{ We have the following elementary facts.}   Let
$X$ be a linear space.

\medskip

\quad{\bf (i)} For any seminorm $p$ on $X$,
$q_p(x,y)\ge(p(x)-p(y))^2\ge 0$ for all $x$,
$y\in X$.   \prooflet{\Prf\ $(p(x)+p(y))^2-p(x+y)^2\ge 0$
because $p(x+y)\le p(x)+p(y)$.\ \Qed}

\medskip

\quad{\bf (ii)}  Suppose that $\familyiI{p_i}$ is a family of seminorms
on $X$ such that $\sum_{i\in I}p_i(x)^2$ is finite for every $x\in X$.
Set $p(x)=\sqrt{\sum_{i\in I}p_i(x)^2}$ for $x\in X$;  then $p$ is a
seminorm on $X$ and $q_p=\sum_{i\in I}q_{p_i}$.
\prooflet{\Prf\ Of course $p(\alpha x)=|\alpha|p(x)$ for
$\alpha\in\Bbb R$ and $x\in X$.   If $x\in X$, then
$p(x)=\|\phi(x)\|_2$, where $\phi(x)=\familyiI{p_i(x)}\in\ell^2(I)$.
Now for $x$, $y\in X$,

\Centerline{$0\le\phi(x+y)\le\phi(x)+\phi(y)$}

\noindent in $\ell^2(I)$, so

\Centerline{$p(x+y)=\|\phi(x+y)\|_2\le\|\phi(x)+\phi(y)\|_2
\le\|\phi(x)\|_2+\|\phi(y)\|_2=p(x)+p(y)$.}

\noindent Thus $p$ is a seminorm.   Now the calculation of
$q_p=\sum_{i\in I}q_{p_i}$ is elementary.\ \QeD}  \cmmnt{In
particular,} $q_p\ge q_{p_i}$ for every $i\in I$.

\medskip

\quad{\bf (iii)} If $\|\,\|$ is an inner product norm on $X$, then
$q_{\|\,\|}(x,y)=\|x-y\|^2$ for all $x$, $y\in X$.  \prooflet{\Prf\

$$\eqalign{2\|x\|^2+2\|y\|^2-\|x+y\|^2
&=2\innerprod{x}{x}+2\innerprod{y}{y}-\innerprod{x+y}{x+y}\cr
&=\innerprod{x}{x}+\innerprod{y}{y}-\innerprod{x}{y}-\innerprod{y}{x}
=\innerprod{x-y}{x-y}.  \text{ \Qed}\cr}$$
}%end of prooflet

\leader{467D}{Lemma} Let $(X,\|\,\|)$ be a normed space.   Suppose that
there are a space $Y$ with a locally uniformly rotund norm $\normY{\,}$
and a bounded
linear operator $T:Y\to X$ such that $T[Y]$ is dense in $X$ and, for
every $x\in X$ and $\gamma>0$, there is a $z\in Y$ such that
$\|x-Tz\|^2+\gamma\normY{z}^2=\inf_{y\in Y}\|x-Ty\|^2+\gamma\normY{y}^2$.
Then $X$ has an equivalent locally uniformly rotund norm.

\proof{{\bf (a)} For each $n\in\Bbb N$, $x\in X$ set

\Centerline{$p_n(x)
=\sqrt{\inf_{y\in Y}\|x-Ty\|^2+2^{-n}\normY{y}^2}$.}

\noindent Then $p_n:X\to\coint{0,\infty}$ is a norm on $X$, equivalent
to $\|\,\|$.   \Prf\ (i) The functionals $(x,y)\mapsto\|x-Ty\|$,
$(x,y)\mapsto 2^{-n/2}\normY{y}$ from $X\times Y$ to $\coint{0,\infty}$ are
both seminorms, so the
functional $(x,y)\mapsto\phi(x,y)=\sqrt{\|x-Ty\|^2+2^{-n}\normY{y}^2}$ also
is, by 467C(c-ii).   (ii) If $x\in X$ and $\alpha\in\Bbb R$, take
$z\in Y$ such that $p_n(x)=\phi(x,z)$;  then

\Centerline{$p_n(\alpha x)\le\phi(\alpha x,\alpha z)=|\alpha|\phi(x,z)
=|\alpha|p_n(x)$.}

\noindent If $\alpha\ne 0$, apply the same argument to see that
$p_n(x)\le|\alpha|^{-1}p_n(\alpha x)$, so that
$p_n(\alpha x)=|\alpha|p_n(x)$.
(iii) Now take any $x_1$, $x_2\in X$.   Let $z_1$, $z_2\in Y$ be such
that $p_n(x_i)=\phi(x_i,z_i)$ for both $i$.   Then

$$\eqalign{p_n(x_1+x_2)
&=2p_n(\Bover12x_1+\Bover12x_2)
\le 2\phi(\Bover12x_1+\Bover12x_2,\Bover12z_1+\Bover12z_2)\cr
&\le 2(\Bover12\phi(x_1,z_1)+\Bover12\phi(x_2,z_2))
=p_n(x_1)+p_n(x_2).\cr}$$

\noindent Thus $p_n$ is a seminorm.   (iv) $p_n(x)\le\phi(x,0)=\|x\|$
for every $x\in X$.   (vi) For any $x\in X$ and $y\in Y$, either
$\|Ty\|\ge\bover12\|x\|$ and

\Centerline{$\phi(x,y)\ge 2^{-n/2}\normY{y}
\ge\Bover1{2\cdot 2^{n/2}\|T\|}\|x\|$,}

\noindent or $\|Ty\|\le\bover12\|x\|$ and

\Centerline{$\phi(x,y)\ge\|x-Ty\|\ge\Bover12\|x\|$;}

\noindent this shows that
$p_n(x)\ge\min(\bover12,\bover122^{-n/2}\|T\|^{-1})\|x\|$.   (I am
passing over the trivial case $X=\{0\}$, $\|T\|=0$.)   In particular,
$p_n(x)=0$ only when $x=0$.   Thus $p_n$ is a norm equivalent to $p$.\
\Qed

\medskip

{\bf (b)} For any $x\in X$, $\lim_{n\to\infty}p_n(x)=0$.   \Prf\ Let
$\epsilon>0$.   Let $y\in Y$ be such that $\|x-Ty\|\le\epsilon$;  then

\Centerline{$\limsup_{n\to\infty}p_n(x)^2
\le\limsup_{n\to\infty}\|x-Ty\|^2+2^{-n}\normY{y}^2
\le\epsilon^2$.}

\noindent As $\epsilon$ is arbitrary, $\lim_{n\to\infty}p_n(x)=0$.\ \Qed

\medskip

{\bf (c)} Set $\|x\|'=\sqrt{\sum_{n=0}^{\infty}2^{-n}p_n(x)^2}$ for
$x\in X$.   The sum is always finite because $p_n(x)\le\|x\|$ for every
$n$, so $\|\,\|'\le\sqrt 2\|\,\|$ is a seminorm;  and it is a norm equivalent to
$\|\,\|$ because $p_0$ is.   Now $\|\,\|'$ is locally uniformly rotund.
\Prf\ Take $x\in X$ and $\epsilon>0$.   Let $n\in\Bbb N$ be such that
$p_n(x)\le\bover14\epsilon$.   Choose $y\in Y$ such that
$p_n(x)^2=\|x-Ty\|^2+2^{-n}\normY{y}^2$.   (This is where we really use the
hypothesis that the infimum in the definition of $p_n$ is attained.)
Let $\delta>0$ be such that $2^n\delta\le(\bover14\epsilon)^2$ and
$\|T\|\|y'-y\|\le\bover14\epsilon$
whenever $q_{\normY{\,}}(y',y)\le 2^{2n}\delta$.

If $q_{\|\,\|'}(x,x')\le\delta$, then $q_{2^{-n/2}p_n}(x,x')\le\delta$,
by 467C(c-ii), that is, $q_{p_n}(x,x')\le 2^n\delta$.   Let $y'\in Y$ be
such that $p_n(x')^2=\|x'-Ty'\|^2+2^{-n}\normY{y'}^2$.   Then

$$\eqalign{p_n(x+x')^2
&\le\|x+x'-Ty-Ty'\|^2+2^{-n}\normY{y+y'}^2\cr
&\le(\|x-Ty\|+\|x'-Ty'\|)^2+2^{-n}\normY{y+y'}^2,\cr}$$

\noindent so

$$\eqalign{q_{p_n}(x,x')
&=2p_n(x)^2+2p_n(x')^2-p_n(x+x')^2\cr
&\ge 2(\|x-Ty\|^2+2^{-n}\normY{y}^2)
  +2(\|x'-Ty'\|^2+2^{-n}\normY{y'}^2)\cr
&\qquad\qquad-(\|x-Ty\|+\|x'-Ty'\|)^2-2^{-n}\normY{y+y'}^2\cr
&=(\|x-Ty\|-\|x'-Ty'\|)^2
  +2^{-n}(2\normY{y}^2+2\normY{y'}^2-\normY{y+y'}^2).\cr}$$

\noindent This means that

\Centerline{$q_{\normY{\,}}(y,y')\le 2^nq_{p_n}(x,x')\le 2^{2n}\delta$,}

\noindent so $\|T\|\normY{y-y'}\le\bover14\epsilon$, while also

\Centerline{$\|x-Ty\|+\|x'-Ty'\|
\le 2\|x-Ty\|+\sqrt{q_{p_n}(x,x')}
\le 2p_n(x)+2^{n/2}\sqrt{\delta}
\le \Bover34\epsilon$.}

\noindent Finally

\Centerline{$\Bover1{\sqrt 2}\|x-x'\|'\le\|x-x'\|
\le\|T\|\normY{y-y'}+\|x-Ty\|+\|x'-Ty'\|
\le\Bover14\epsilon+\Bover34\epsilon
=\epsilon$.}

\noindent As $x$ and $\epsilon$ are arbitrary, this shows that $\|\,\|'$
is locally uniformly rotund.\ \Qed

This completes the proof.
}%end of proof of 467D

\leader{467E}{Theorem} Let $X$ be a separable normed space.   Then it
has an equivalent locally uniformly rotund norm.

\proof{{\bf (a)} It is enough to show that the completion of $X$ has an
equivalent locally uniformly rotund norm;  since the completion of $X$
is separable, we may suppose that $X$ itself is complete.   Let
$\sequence{i}{x_i}$ be a sequence in $X$ running
over a dense subset of $X$.   Define $T:\ell^2\to X$ by setting

\Centerline{$Ty=\sum_{i=0}^{\infty}\Bover{y(i)}{2^i(1+\|x_i\|)}x_i$}

\noindent for $y\in\ell^2=\ell^2(\Bbb N)$;  then $Ty$ is always defined
(4A4Ie);  $T$ is a linear operator and

\Centerline{$\|Ty\|
\le\sum_{i=0}^{\infty}2^{-i}|y(i)|
\le\sqrt{\sumop_{i=0}^{\infty}2^{-2i}}\,\|y\|_2$}

\noindent for every $y\in\ell^2$, by Cauchy's inequality (244Eb).  So
$T$ is a bounded linear operator.

\medskip

{\bf (b)} $T$ satisfies the conditions of 467D.   \Prf\ $T[\ell^2]$ is
dense because it contains every $x_i$.   Given $x\in X$, $\gamma>0$ and
$\alpha\ge 0$, the function
$y\mapsto\sqrt{\|x-Ty\|^2+\gamma\|y\|_2^2}$ is convex and
norm-continuous, so the set

\Centerline{$C_{\alpha}(x)
=\{y:y\in\ell^2,\,\|x-Ty\|^2+\gamma\|y\|_2^2\le\alpha^2\}$}

\noindent is convex and norm-closed.   Consequently,
$C_{\alpha}(x)$ is weakly closed (4A4Ed);  since
$\|y\|_2\le\gamma^{-1/2}\alpha$ for every $y\in C_{\alpha}(x)$,
$C_{\alpha}(x)$ is weakly compact (4A4Ka).   Set $\beta=\inf_{y\in\ell^2}\|x-Ty\|^2+\gamma\|y\|_2^2$.   Then
$\{C_{\alpha}(x):\alpha>\beta\}$ is a downwards-directed set of
non-empty weakly compact sets, so has non-empty intersection;  taking
any $z\in\bigcap_{\alpha>\beta}C_{\alpha}(x)$,
$\beta=\|x-Tz\|^2+\gamma\|z\|_2^2$.\ \Qed

\medskip

{\bf (c)} So 467D gives the result.
}%end of proof of 467E

\leader{467F}{Lemma} Let $(X,\|\,\|)$ be a Banach space, and
$\familyiI{T_i}$ a family of bounded linear operators from $X$ to itself
such that

\inset{(i) for each $i\in I$, the subspace $T_i[X]$ has an equivalent
locally uniformly rotund norm,

(ii) for each $x\in X$, $\epsilon>0$ there is a finite set
$J\subseteq I$ such that $\|x-\sum_{i\in J}T_ix\|\le\epsilon$,

(iii) for each $x\in X$, $\epsilon>0$  the set
$\{i:i\in I,\,\|T_ix\|\ge\epsilon\}$ is finite.}

\noindent  Then $X$ has an equivalent locally uniformly rotund norm.

\proof{{\bf (a)} Let  $\|\,\|_i$ be a locally uniformly rotund norm on
$X_i=T_i[X]$ equivalent to $\|\,\|$ on $X_i$.   Reducing $\|\,\|_i$ by a
scalar multiple if necessary, we may suppose that $\|T_ix\|_i\le\|x\|$
for every
$x\in X$ and $i\in I$.   By (iii), $\sup_{i\in I}\|T_ix\|$ is finite for
every $x\in X$;  by the Uniform Boundedness Theorem (3A5Ha),
$M=\sup_{i\in I}\sup_{\|x\|\le 1}\|T_ix\|$ is finite.   (This is where we
need to suppose that $X$ is complete.)   For finite sets
$J\subseteq I$ and $k\ge 1$, set

\Centerline{$p_{Jk}(x)=\sqrt{\sumop_{i\in J}\|T_ix\|_i^2
  +\bover1k\sumop_{K\subseteq J}\|x-\sumop_{i\in K}T_ix\|^2}$;}

\noindent for $n\in\Bbb N$ and $k\ge 1$ set

\Centerline{$p^{(n)}_k(x)=\sup\{p_{Jk}(x):J\subseteq I,\,\#(J)\le n\}$.}

\noindent By 467C(c-ii), as usual, all the $p_{Jk}$ are seminorms, and
it follows at once that the $p^{(n)}_k$ are seminorms.   Observe that if
$K\subseteq I$ is finite, then
$\|x-\sum_{i\in K}T_ix\|\le(1+M\#(K))\|x\|$ for every $x$, so if
$J\subseteq I$ is finite then

\Centerline{$p_{Jk}(x)\le\sqrt{\#(J)+2^{\#(J)}(1+M\#(J))}\|x\|$,}

\noindent and $p^{(n)}_k(x)\le\sqrt{n+2^n(1+Mn)}\|x\|$ whenever
$n\in\Bbb N$ and $k\ge 1$.   Setting $\beta_{nk}=2^{2n+k}$ for $n$,
$k\in\Bbb N$,

\Centerline{$\|x\|'
=\sqrt{\sumop_{n=0}^{\infty}\sumop_{k=1}^{\infty}
  \beta_{nk}^{-1}p^{(n)}_k(x)^2}$}

\noindent is finite for every $x\in X$, so that $\|\,\|'$ is a seminorm
on $X$;  moreover, $\|x\|'\le\beta\|x\|$ for every $x\in X$, where

\Centerline{$\beta
=\sqrt{\sumop_{n=0}^{\infty}\sumop_{k=1}^{\infty}
  \beta_{nk}^{-1}(n+2^n(1+Mn))}$}

\noindent is finite.   Since we also have

\Centerline{$\|x\|'\ge\Bover1{\sqrt{2}}p^{(0)}_1(x)
=\Bover1{\sqrt{2}}p_{\emptyset 1}(x)=\Bover1{\sqrt{2}}\|x\|$}

\noindent for every $x$, $\|\,\|'$ is a norm on $X$ equivalent to
$\|\,\|$.

\medskip

{\bf (b)} Now $\|\,\|'$ is locally uniformly rotund.   \Prf\ Take
$x\in X$ and $\epsilon>0$.   Let $K\subseteq I$ be a finite set such
that $\|x-\sum_{i\in K}T_ix\|\le\bover14\epsilon$;  we may suppose that
$T_ix\ne 0$ for every $i\in K$.   Set
$\alpha_1=\min_{i\in K}\|T_ix\|_i$,
$J=\{i:i\in I,\,\|T_ix\|_i\ge\alpha_1\}$ and
$\alpha_0=\sup_{i\in I\setminus J}\|T_ix\|_i$.   (For completeness, if
$K=\emptyset$, take $J=\emptyset$,  $\alpha_0=\sup_{i\in I}\|T_ix\|_i$
and $\alpha_1=\alpha_0+1$.)
Then $J$ is finite and $\alpha_0<\alpha_1$, by hypothesis (iii) of the
lemma.   Set $n=\#(J)$.   Let $k$ be so large that
$\Bover{2^n(Mn+1)}{k}\|x\|^2<\bover12(\alpha_1^2-\alpha_0^2)$.
Let $\eta>0$ be such that

\inset{$(\beta_{nk}+1)\eta
\le\min(\Bover{\epsilon^2}{16k},\alpha_1^2-\alpha_0^2)$,}

\inset{$\|T_ix-z\|\le\Bover{\epsilon}{1+4n}$ whenever $i\in K$, $z\in
X_i$ and $q_{\|\,\|_i}(T_ix,z)\le(\beta_{nk}+1)\eta$;}

\noindent this is where we use the hypothesis that every $\|\,\|_i$ is
locally uniformly rotund (and equivalent to $\|\,\|$ on $X_i$).

Now suppose that $y\in X$ and $q_{\|\,\|'}(x,y)\le\eta$.   Then
$q_{p^{(n)}_k}(x,y)\le\beta_{nk}\eta$, by 467C(c-ii).   Let $L\in[I]^n$
be such that $p^{(n)}_k(x+y)^2\le p_{Lk}(x+y)^2+\eta$.   Then

$$\eqalign{q_{p_{Lk}}(x,y)
&=2p_{Lk}(x)^2+2p_{Lk}(y)^2-p_{Lk}(x+y)^2\cr
&\le 2p^{(n)}_k(x)^2+2p^{(n)}_k(y)^2-p^{(n)}_k(x+y)^2+\eta
\le(\beta_{nk}+1)\eta.\cr}$$

\noindent We also have

$$\eqalign{2p_{Lk}(x)^2
&\ge 2p_{Lk}(x)^2+2p_{Lk}(y)^2-2p^{(n)}_k(y)^2
\ge p_{Lk}(x+y)^2-2p^{(n)}_k(y)^2\cr
&\ge p^{(n)}_k(x+y)^2-\eta-2p^{(n)}_k(y)^2
\ge 2p^{(n)}_k(x)^2-(\beta_{nk}+1)\eta,\cr}$$

\noindent so

$$\eqalign{\sum_{i\in J}\|T_ix\|_i^2
&\le p_{Jk}(x)^2
\le p_k^{(n)}(x)^2
\le p_{Lk}(x)^2+\Bover12(\beta_{nk}+1)\eta\cr
&\le\sum_{i\in L}\|T_ix\|_i^2+\Bover{2^n(Mn+1)}{k}\|x\|^2
  +\Bover12(\beta_{nk}+1)\eta\cr
&<\sum_{i\in L}\|T_ix\|_i^2+\alpha_1^2-\alpha_0^2.\cr}$$

\noindent Since $\#(L)=\#(J)$ and

\Centerline{$\|T_ix\|_i^2\le\alpha_0^2<\alpha_1^2\le\|T_jx\|_j^2$}

\noindent whenever $j\in J$ and $i\in I\setminus J$,  we must actually
have $L=J$.   In particular, $K\subseteq L$.   But this means that (by
467C(c-ii) again)

\Centerline{$q_{\|\,\|_i}(T_ix,T_iy)
\le q_{p_{Lk}}(x,y)\le(\beta_{nk}+1)\eta$}

\noindent and (by the choice of $\eta$)
$\|T_ix-T_iy\|\le\Bover{\epsilon}{1+4n}$ for every $i\in K$, so that
$\|\sum_{i\in K}T_ix-\sum_{i\in K}T_iy\|\le\bover14\epsilon$.

The last element we need is that, setting
$\tilde p(z)=\bover1{\sqrt{k}}\|z-\sum_{i\in K}T_iz\|$, $\tilde p$ is a
seminorm on $X$ and is one of the constituents of $p_{Lk}$;  so that

$$\eqalign{\Bover1k(\|x-\sum_{i\in K}T_ix\|-\|y-\sum_{i\in K}T_iy\|)^2
&\le q_{\tilde p}(x,y)
\le q_{p_{Lk}}(x,y)\cr
&\le(\beta_{nk}+1)\eta
\le\Bover{\epsilon^2}{16k},\cr}$$

\noindent and
$\bigl|\|x-\sum_{i\in K}T_ix\|-\|y-\sum_{i\in K}T_iy\|\bigr|
\le\bover14\epsilon$.   It follows that

\Centerline{$\|y-\sum_{i\in K}T_iy\|
\le\bover14\epsilon+\|x-\sum_{i\in K}T_ix\|\le\bover12\epsilon$.}

\noindent Putting these together,

\Centerline{$\|x-y\|
\le\|x-\sum_{i\in K}T_ix\|+\sum_{i\in K}\|T_ix-T_iy\|
  +\|y-\sum_{i\in K}T_iy\|
\le\epsilon$.}

\noindent And this is true whenever $q_{\|\,\|'}(x,y)\le\eta$.   As $x$
and $\epsilon$ are arbitrary, $\|\,\|'$ is locally uniformly rotund.\
\Qed
}%end of proof of 467F

\leader{467G}{Theorem} Let $X$ be a Banach space.   Suppose that there
are an ordinal $\zeta$ and a family
$\langle P_{\xi}\rangle_{\xi\le\zeta}$ of bounded linear operators from
$X$ to itself such that

\inset{(i) if $\xi\le\eta\le\zeta$ then
$P_{\xi}P_{\eta}=P_{\eta}P_{\xi}=P_{\xi}$;

(ii) $P_0(x)=0$ and $P_{\zeta}(x)=x$ for every $x\in X$;

(iii) if $\xi\le\zeta$ is a non-zero limit ordinal, then
$\lim_{\eta\uparrow\xi}P_{\eta}(x)=P_{\xi}(x)$ for every $x\in X$;

(iv) if $\xi<\zeta$ then $X_{\xi}=\{(P_{\xi+1}-P_{\xi})(x):x\in X\}$
has an equivalent locally uniformly rotund norm.}

\noindent Then $X$ has an equivalent locally uniformly rotund norm.

\medskip

\noindent{\bf Remark} A family $\langle P_{\xi}\rangle_{\xi\le\zeta}$ satisfying (i), (ii) and (iii) here is called a {\bf projectional resolution of the identity}.

\proof{ For $\xi<\zeta$ set $T_{\xi}=P_{\xi+1}-P_{\xi}$.   From
condition (i) we see easily that $T_{\xi}T_{\eta}=T_{\xi}$ if
$\xi=\eta$, $0$ otherwise;  and that $T_{\xi}P_{\eta}=T_{\xi}$ if
$\xi<\eta$, $0$ otherwise.

I seek to show that the conditions of 467F are satisfied by
$\ofamily{\xi}{\zeta}{T_{\xi}}$.   Condition (i) of 467F is just
condition (iv) here.   Let $Z$ be the set of those $x\in X$ for which
conditions (ii) and (iii) of 467F are satisfied;  that is,

\inset{for each $\epsilon>0$ there is a finite set $J\subseteq\zeta$
such that $\|x-\sum_{\xi\in J}T_{\xi}x\|\le\epsilon$, and
$\{\xi:\|T_{\xi}x\|\ge\epsilon\}$ is finite.}

\noindent Then $Z$ is a linear subspace of $X$.
For $\xi\le\zeta$, set $Y_{\xi}=P_{\xi}[X]$.   Then
$Y_{\xi}\subseteq Z$.   \Prf\ Induce on $\xi$.   Since $Y_0=\{0\}$, the
induction starts.   For the inductive step to a successor ordinal
$\xi+1\le\zeta$,
$Y_{\xi+1}=Y_{\xi}+X_{\xi}\subseteq Z$.   For the inductive step to a
non-zero limit ordinal $\xi\le\zeta$, given $x\in Y_{\xi}$ and
$\epsilon>0$, we know that there is a $\xi'<\xi$ such that
$\|P_{\eta}x-P_{\xi}x\|\le\bover13\epsilon$ whenever
$\xi'\le\eta\le\xi$.   So $\|T_{\eta}x\|\le\bover23\epsilon$ whenever
$\xi'\le\eta<\xi$, and

\Centerline{$\{\eta:\|T_{\eta}x\|\ge\epsilon\}
=\{\eta:\eta<\xi',\,\|T_{\eta}x\|\ge\epsilon\}
=\{\eta:\|T_{\eta}P_{\xi'}x\|\ge\epsilon\}$}

\noindent is finite, by the inductive hypothesis.   Moreover, there is a
finite set $J\subseteq\xi'$ such that
$\|P_{\xi'}x-\sum_{\eta\in J}T_{\eta}P_{\xi'}x\|\le\bover23\epsilon$,
and now $\|x-\sum_{\eta\in J}T_{\eta}x\|\le\epsilon$.   As $x$ and
$\epsilon$ are arbitrary, $Y_{\xi}\subseteq Z$.\ \Qed

In particular, $X=Y_{\zeta}\subseteq Z$ and conditions (ii) and (iii) of
467F are satisfied.   So 467F gives the result.
}%end of proof of 467G

\leader{467H}{Definitions (a)} A topological space $X$ is
{\bf K-countably determined} or a {\bf Lindel\"of-$\pmb{\Sigma}$} space if there are a subset $A$ of $\NN$ and an
usco-compact relation $R\subseteq A\times X$ such that $R[A]=X$.
\cmmnt{Observe that all K-analytic Hausdorff spaces (\S422) are
K-countably determined.}

\spheader 467Hb A normed space $X$ is {\bf weakly K-countably
determined} if it is K-countably determined in its weak topology.

\spheader 467Hc Let $X$ be a normed space and $Y$, $W$ closed linear
subspaces of $X$, $X^*$ respectively.   I will say that $(Y,W)$ is a
{\bf projection pair} if $X=Y\oplus W^{\smallcirc}$ and
$\|y+z\|\ge\|y\|$ for every
$y\in Y$, $z\in W^{\smallcirc}$\dvro{.}{, where

$$\eqalign{W^{\smallcirc}
&=\{z:z\in X,\,f(z)\le 1\text{ for every }f\in W\}\cr
&=\{z:z\in X,\,f(z)=0\text{ for every }f\in W\}.\cr}$$
}%end of dvro

\leader{467I}{Lemma} (a) If $X$ is a weakly K-countably determined
normed space, then any closed linear subspace of $X$ is weakly
K-countably determined.

(b) If $X$ is a weakly K-countably determined normed space, $Y$ is a
normed space, and $T:X\to Y$ is a continuous linear surjection, then $Y$
is weakly K-countably determined.

(c) If $X$ is a Banach space and $Y\subseteq X$ is a dense linear
subspace which is weakly K-countably determined, then $X$ is weakly
K-countably determined.

\proof{{\bf (a)} Let $A\subseteq\NN$, $R\subseteq A\times X$ be such
that $R$ is usco-compact (for the weak topology on $X$) and $R[A]=X$.
Let $Y$ be a (norm-\nobreak)closed linear subspace of $X$;
then $Y$ is closed
for the weak topology (3A5Ee).   Also the weak topology on $Y$ is just
the subspace topology induced by the weak topology of $X$ (4A4Ea).   Set
$R'=R\cap(A\times Y)$.   Then $R'$ is usco-compact whether regarded as a
subset of $A\times X$ or as a subset of $A\times Y$ (422Da, 422Db,
422Dg).   Since $Y=R'[A]$, $Y$ is weakly K-countably determined.

\medskip

{\bf (b)} Let $A\subseteq\NN$, $R\subseteq A\times X$ be such that $R$
is usco-compact for the weak topology on $X$ and $R[A]=X$.   Because $T$
is continuous for the weak topologies on $X$ and $Y$ (3A5Ec),

\Centerline{$R_1=\{(\phi,y)$: there is some $x\in X$ such that
$(\phi,x)\in R$ and $Tx=y\}$}

\noindent is usco-compact in $A\times Y$ (422Db, 422Df).   Also
$R_1[A]=T[R[A]]=Y$.   So $Y$ is weakly K-countably determined.

\medskip

{\bf (c)} Let $A\subseteq\NN$, $R\subseteq A\times Y$ be such that $R$
is usco-compact (for the weak topology on $Y$) and $R[A]=Y$.   Then, as
in (a), $R$ is usco-compact when regarded as a subset of $A\times X$.
By 422Dd, the set

\Centerline{$R_1
=\{(\sequencen{\phi_n},\sequencen{y_n}):(\phi_n,y_n)\in R$ for every
$n\in\Bbb N\}$}

\noindent is usco-compact in $A^{\Bbb N}\times Y^{\Bbb N}$.   Now
examine

\Centerline{$S=\{(\sequencen{y_n},x):x\in X$, $y_n\in Y$ and
$\|y_n-x\|\le 2^{-n}$ for every $n\in\Bbb N\}$.}

\noindent Then $S$ is usco-compact in $Y^{\Bbb N}\times X$.   \Prf\
(Remember that we are using weak topologies on $X$ and $Y$ throughout.)
If $\pmb{y}=\sequencen{y_n}$ is a sequence in $Y$ and
$(\pmb{y},x)\in S$, then $x=\lim_{n\to\infty}y_n$;  so $S[\{\pmb{y}\}]$
has at most one member and is certainly compact.   Let $F\subseteq X$ be
a weakly closed set and $\pmb{y}\in Y^{\Bbb N}\setminus S^{-1}[F]$.

{\bf case 1} If there are $m$, $n\in\Bbb N$ such that
$\|y_m-y_n\|>2^{-m}+2^{-n}$, let $f\in Y^*$ be such that $\|f\|\le 1$
and $f(y_m-y_n)>2^{-m}+2^{-n}$.   Then
$G=\{\pmb{z}:f(z_m-z_n)>2^{-m}+2^{-n}\}$ is an open set in $Y^{\Bbb N}$
containing $\pmb{y}$ and disjoint from $S^{-1}[F]$, because $S[G]$ is
empty.

{\bf case 2} Otherwise, $\pmb{y}$ is a Cauchy sequence and (because $X$
is a Banach space) has a limit $x\in X$, which does not belong to $F$.
Let $\delta>0$ and $f_0,\ldots,f_r\in X^*$ be such that
$\{w:|f_i(w)-f_i(x)|\le\delta$ for every $i\le r\}$ does not meet $F$.
Let $n\in\Bbb N$ be such that $2^{-n}\|f_i\|\le\bover13\delta$ for every
$i\le r$.   Then $G=\{\pmb{z}:|f_i(z_n)-f_i(y_n)|<\bover13\delta$ for
every $i\le r\}$ is an open set in $Y^{\Bbb N}$ containing $Y$.   If
$\pmb{z}\in G$ and $(\pmb{z},w)\in S$, then $\|z_n-w\|\le 2^{-n}$ so

$$\eqalign{|f_i(w)-f_i(x)|
&\le|f_i(w)-f_i(z_n)|+|f_i(z_n)-f_i(y_n)|+|f_i(y_n)-f_i(x)|\cr
&\le 2^{-n}\|f_i\|+\Bover13\delta+2^{-n}\|f_i\|
\le\delta\cr}$$

\noindent for every $i\le r$, and $w\notin F$.   Thus again
$G\cap S^{-1}[F]$ is empty.

This shows that there is always an open set containing $y$ and disjoint
from $S^{-1}[F]$.   As $y$ is arbitrary, $S^{-1}[F]$ is closed.   As $F$
is arbitrary, $S$ is usco-compact.\ \Qed

It follows that $SR_1\subseteq A^{\Bbb N}\times X$ is usco-compact
(422Df), while

\Centerline{$(SR_1)[A^{\Bbb N}]=S[R_1[A^{\Bbb N}]]=S[Y^{\Bbb N}]=X$}

\noindent because $Y$ is dense in $X$.   Finally, $A^{\Bbb N}$ is
homeomorphic to a subset of $\NN$ because it is a subspace of
$(\NN)^{\Bbb N}\cong\NN$.   So $X$ is weakly K-countably determined.
}%end of proof of 467I

\vleader{72pt}{467J}{Lemma} 
Let $X$ be a weakly K-countably determined Banach
space.   Then there is a family $\Cal M$ of subsets of $X\cup X^*$ such
that

(i) whenever $B\subseteq X\cup X^*$ there is an $M\in\Cal M$ such that
$B\subseteq M$ and $\#(M)\le\max(\omega,\#(B))$;

(ii) whenever $\Cal M'\subseteq\Cal M$ is upwards-directed, then
$\bigcup\Cal M'\in\Cal M$;

(iii) whenever $M\in\Cal M$ then
$(\overline{M\cap X},\overline{M\cap X^*})$\cmmnt{ (where the closures
are taken for the norm topologies)} is a projection pair of subspaces of
$X$ and $X^*$.

\proof{{\bf (a)} Let $A\subseteq\NN$, $R\subseteq A\times X$ be such
that $R$ is usco-compact in $A\times X$ and $R[A]=X$.   Set
$S=\bigcup_{n\in\Bbb N}\BbbN^n$ and for $\sigma\in S$ set
$F_{\sigma}=R[I_{\sigma}]$, where
$I_{\sigma}=\{\phi:\sigma\subseteq\phi\in\NN\}$;  set
$S_0=\{\sigma:\sigma\in S,\,F_{\sigma}\ne\emptyset\}$.

Let $\Cal M$ be the family of those sets $M\subseteq X\cup X^*$ such
that ($\alpha$) whenever $x$, $y\in M\cap X$ and $q\in\Bbb Q$ then $x+y$
and $qx$ belong to $M$ ($\beta$) whenever $f$, $g\in M\cap X^*$ and
$q\in\Bbb Q$ then $f+g$ and $qf$ belong to $M$ ($\gamma$)
$\|x\|=\max\{f(x):f\in M\cap X^*,\|f\|\le 1\}$ for every $x\in M\cap X$
($\delta$)
$\sup_{x\in F_{\sigma}}f(x)=\sup_{x\in F_{\sigma}\cap M}f(x)$ for every
$f\in M\cap X^*$, $\sigma\in S_0$.

\medskip

{\bf (b)} For each $x\in X$, choose $h_x\in X^*$ such that
$\|h_x\|\le 1$ and $h_x(x)=\|x\|$;  for each $f\in X^*$ and
$\sigma\in S_0$ choose a countable set $C_{f\sigma}\subseteq F_{\sigma}$
such that $\sup\{f(x):x\in C_{f\sigma}\}=\sup\{f(x):x\in F_{\sigma}\}$.
Given $B\subseteq X$, define $\sequencen{B_n}$ by setting

$$\eqalign{B_{n+1}
&=B_n\cup\{x+y:x,\,y\in B_n\cap X\}
  \cup\{qx:q\in\Bbb Q,\,x\in B_n\cap X\}\cr
&\qquad\qquad\cup\{f+g:f,\,g\in B_n\cap X\}
  \cup\{qf:q\in\Bbb Q,\,f\in B_n\cap X\}\cr
&\qquad\qquad\cup\{h_x:x\in B_n\cap X\}
  \cup\bigcup\{C_{f\sigma}:f\in B_n\cap X^*,\,\sigma\in S_0\},\cr}$$

\noindent for each $n\in\Bbb N$.   Then $M=\bigcup_{n\in\Bbb N}B_n$
belongs to $\Cal M$ and has cardinal at most $\max(\omega,\#(B))$.

\medskip

{\bf (c)} The definition of $\Cal M$ makes it plain that if
$\Cal M'\subseteq\Cal M$ is upwards-directed then $\bigcup\Cal M'$
belongs to $\Cal M$.

\medskip

{\bf (d)} Now take $M\in\Cal M$ and set $Y=\overline{M\cap X}$,
$W=\overline{M\cap X^*}$.   These are linear subspaces (2A5Ec).   If
$y\in M\cap X$ and $z\in W^{\smallcirc}$, then there is an
$f\in M\cap X^*$ such that $\|f\|\le 1$ and $f(y)=\|y\|$, so that

\Centerline{$\|y+z\|\ge f(y+z)=f(y)=\|y\|$.}

\noindent Because the function $y\mapsto\|y+z\|-\|y\|$ is continuous,
$\|y+z\|\ge\|y\|$ for every $y\in Y$ and $z\in W^{\smallcirc}$.   In
particular, if $y\in Y\cap W^{\smallcirc}$, $\|y\|\le\|y-y\|=0$ and
$y=0$, so $Y+W^{\smallcirc}=Y\oplus W^{\smallcirc}$.   If
$x\in\overline{Y+W^{\smallcirc}}$, then there are sequences
$\sequencen{y_n}$ in $Y$ and $\sequencen{z_n}$ in $W^{\smallcirc}$ such
that $x=\lim_{n\to\infty}y_n+z_n$;  now
$\|y_m-y_n\|\le\|(y_m+z_m)-(y_n+z_n)\|\to 0$ as $n\to\infty$,  so
(because $X$ is a Banach space) $\sequencen{y_n}$ is convergent to $y$
say;  in this case, $y\in Y$ and $x-y=\lim_{n\to\infty}z_n$ belongs to
$W^{\smallcirc}$, so $x\in Y+W^{\smallcirc}$.   This shows that
$Y\oplus W^{\smallcirc}$ is a closed linear subspace of $X$.

\medskip

{\bf (e)} \Quer\ Suppose, if possible, that
$Y\oplus W^{\smallcirc}\ne X$.   Then there is an
$x_0\in X\setminus(Y\oplus W^{\smallcirc})$.   By 4A4Eb, there is an
$f\in X^*$ such that $f(x_0)>0$ and $f(y)=f(z)=0$ whenever $y\in Y$ and
$z\in W^{\smallcirc}$;  multiplying $f$ by a scalar if necessary, we can
arrange that $f(x_0)=1$.   By 4A4Eg, $f$ belongs to the weak*
closure of $W$ in $X^*$.

Let $\phi\in A$ be such that $(\phi,x_0)\in R$.   Then $K=R[\{\phi\}]$
is weakly compact.   Now the weak* closure of $W$ is also its closure
for the Mackey topology of uniform convergence on weakly compact subsets
of $X$ (4A4F).   So there is a $g\in W$ such that
$|g(x)-f(x)|\le\bover17$ for every $x\in K$.   Next, because $K$ is
bounded, and $g$ belongs to the norm closure of $M\cap X^*$, there is an
$h\in M\cap X^*$ such that $|h(x)-g(x)|\le\bover17$ for every $x\in K$.
This means that $|h(x)-f(x)|\le\bover27$ for every
$x\in K$, and $K$ is included in the weakly open set
$G=\{x:|h(x)-f(x)|<\bover37\}$, that is, $\phi$ does not belong to
$R^{-1}[X\setminus G]$, which is relatively closed in $A$, because $R$
is usco-compact regarded as a relation between $A$ and $X$ with the weak
topology.   There is therefore a $\sigma\in S$ such that
$\phi\in I_{\sigma}$ and
$I_{\sigma}\cap R^{-1}[X\setminus G]=\emptyset$, that is,
$F_{\sigma}\subseteq G$.   In this case, $x_0\in F_{\sigma}$, so
$\sigma\in S_0$, while $h(x)-f(x)<\bover37$ for every
$x\in F_{\sigma}$.   But, because $M\in\Cal M$, there is a
$y\in M\cap X\cap F_{\sigma}$ such that $h(y)\ge h(x_0)-\bover17$, and as $y\in Y$ we must now have

$$\eqalign{0
&=f(y)
=h(y)-(h(y)-f(y))
>h(x_0)-\Bover17-\Bover37\cr
&=f(x_0)-(f(x_0)-h(x_0))-\Bover47
\ge 1-\Bover37-\Bover47=0,\cr}$$

\noindent which is absurd.\ \Bang

Thus $X=Y\oplus W^{\smallcirc}$ and $(Y,W)$ is a projection pair.   This
completes the proof.
}%end of proof of 467J
%elementary submodel argument

\leader{467K}{Theorem} Let $X$ be a weakly K-countably determined Banach
space.   Then it has an equivalent locally uniformly rotund norm.

\proof{ Since the completion of $X$ is weakly K-countably determined
(467Ic), we may suppose that $X$ is complete.   The proof proceeds by
induction on the weight of $X$.

\medskip

{\bf (a)} The induction starts by observing that if $w(X)\le\omega$ then
$X$ is separable (4A2Li/4A2P(a-i))
so has an equivalent locally uniformly rotund norm by 467E.

\medskip

{\bf (b)} So let us suppose that $w(X)=\kappa>\omega$ and that any
weakly K-countably determined Banach space of weight less than $\kappa$
has an equivalent locally uniformly rotund norm.

Let $\Cal M$ be a family of subsets of $X\cup X^*$ as in 467J.   Then
there is a non-decreasing family
$\langle M_{\xi}\rangle_{\xi\le\kappa}$ in $\Cal M$ such that
$\#(M_{\xi})\le\max(\omega,\#(\xi))$ for every $\xi\le\kappa$,
$M_{\kappa}$ is dense in $X$, and $M_{\xi}=\bigcup_{\eta<\xi}M_{\eta}$
for every limit ordinal $\xi\le\kappa$.   \Prf\ By 4A2Li, there is a
dense subset of $X$ of cardinal $\kappa$;  enumerate it as
$\ofamily{\xi}{\kappa}{x_{\xi}}$.   Choose $M_{\xi}$ inductively, as
follows.    $M_0=\emptyset$.   GIven $M_{\xi}$ with
$\#(M_{\xi})\le\max(\omega,\#(\xi))$, then by 467J(i) there is an
$M_{\xi+1}\in\Cal M$ such that
$M_{\xi+1}\supseteq M_{\xi}\cup\{x_{\xi}\}$ and

\Centerline{$\#(M_{\xi+1})\le\max(\omega,\#(M_{\xi}\cup\{x_{\xi}\})
\le\max(\omega,\#(\xi+1))$.}

\noindent Given that $\ofamily{\eta}{\xi}{M_{\eta}}$ is a non-decreasing
family in $\Cal M$ with $\#(M_{\eta})\le\max(\omega,\#(\eta))$ for every
$\eta<\xi$, set $M_{\xi}=\bigcup_{\eta<\xi}M_{\eta}$;  then 467J(ii)
tells us that $M_{\xi}\in\Cal M$, while
$\#(M_{\xi})\le\max(\omega,\#(\xi))$, as required by the inductive
hypothesis.\ \Qed

At the end of the induction, $M_{\kappa}\supseteq\{x_{\xi}:\xi<\kappa\}$
will be dense in $X$.

\medskip

{\bf (c)} For each $\xi<\kappa$, set $Y_{\xi}=\overline{M_{\xi}\cap X}$,
$W_{\xi}=\overline{M_{\xi}\cap X^*}$.   Then $(Y_{\xi},W_{\xi})$ is a
projection pair, by 467J(iii).    Since
$X=Y_{\xi}\oplus W_{\xi}^{\smallcirc}$, we have a projection
$P_{\xi}:X\to Y_{\xi}$ defined by saying that $P_{\xi}(y+z)=y$ whenever
$y\in Y_{\xi}$ and $z\in W_{\xi}^{\smallcirc}$.   Now
$P_{\xi}P_{\eta}=P_{\eta}P_{\xi}=P_{\xi}$ whenever $\xi\le\eta$.   \Prf\
The point is that $Y_{\xi}\subseteq Y_{\eta}$ and
$W_{\xi}\subseteq W_{\eta}$, so
$W_{\eta}^{\smallcirc}\subseteq W_{\xi}^{\smallcirc}$.   If $x\in X$,
express it as $y+z_1$ where $y\in Y_{\xi}$ and
$z_1\in W_{\xi}^{\smallcirc}$, and express $z_1$ as $y'+z$ where
$y'\in Y_{\eta}$ and $z'\in W_{\eta}^{\smallcirc}$.   Then

\Centerline{$P_{\xi}x=y\in Y_{\xi}\subseteq Y_{\eta}$}

\noindent so $P_{\eta}P_{\xi}x=P_{\xi}x$.   On the other hand,
$x=y+y'+z$, $y+y'\in Y_{\eta}$ and $z\in W_{\eta}^{\smallcirc}$, so
$P_{\eta}x=y+y'$;  and as $y'=z_1-z$ belongs to $W_{\xi}^{\smallcirc}$,
$P_{\xi}(y+y')=y$, so $P_{\xi}P_{\eta}x=P_{\xi}x$.\ \Qed

Note that the condition

\Centerline{$\|y+z\|\ge\|y\|$ whenever $y\in Y_{\xi}$,
$z\in W_{\xi}^{\smallcirc}$}

\noindent ensures that $\|P_{\xi}\|\le 1$ for every $\xi$.

Next, if $\xi\le\kappa$ is a non-zero limit ordinal,
$P_{\xi}x=\lim_{\eta\uparrow\xi}P_{\eta}x$.   \Prf\ We know that

\Centerline{$P_{\xi}x\in Y_{\xi}
=\overline{M_{\xi}\cap X}
=\overline{\bigcupop_{\eta<\xi}M_{\eta}\cap X}$.}

\noindent So, given $\epsilon>0$, there is an
$x'\in\bigcup_{\eta<\xi}M_{\eta}$ such that
$\|P_{\xi}x-x'\|\le\bover12\epsilon$.   Let $\eta<\xi$ be such that
$x'\in M_{\eta}$.   If $\eta\le\eta'\le\xi$, then

$$\eqalignno{\|P_{\xi}x-P_{\eta'}x\|
&=\|P_{\xi}(P_{\xi}x-x')-P_{\eta'}(P_{\xi}x-x')\|\cr
\displaycause{because $x'\in Y_{\eta}$, so $P_{\xi}x'=P_{\eta'}x'=x'$}
&\le 2\|P_{\xi}x-x'\|
\le\epsilon.\cr}$$

\noindent As $\epsilon$ is arbitrary,
$P_{\xi}x=\lim_{\eta\uparrow\xi}P_{\eta}x$.\ \Qed

\medskip

{\bf (d)} Now observe that every $Y_{\xi}$ is weakly K-countably
determined (467Ia), while $w(Y_{\xi})\le\max(\omega,\#(\xi))<\kappa$ for
every $\xi<\kappa$ (using 4A2Li, as usual).   So the inductive
hypothesis tells us that $Y_{\xi}=P_{\xi}[X]$ has an equivalent locally
uniformly rotund norm for every $\xi<\kappa$.   By 467G,
$X=P_{\kappa}[X]$ has an equivalent locally uniformly rotund norm.
Thus the induction proceeds.
}%end of proof of 467K

\leader{467L}{Weakly compactly generated Banach spaces}\cmmnt{ The
most important class of weakly K-{\vthsp}countably determined spaces is the following.}   A normed space $X$ is {\bf weakly compactly generated}
if there is a sequence $\sequencen{K_n}$ of weakly compact subsets of $X$ such that $\bigcup_{n\in\Bbb N}K_n$ is dense in $X$.

\allowmorestretch{468}{
\leader{467M}{Proposition}\cmmnt{ ({\smc Talagrand 75})} A weakly
compactly generated Banach space is weakly K-countably determined.
}%end of allowmorestretch

\proof{ Let $X$ be a Banach space with a sequence $\sequencen{K_n}$ of
weakly compact subsets of $X$ such that $\bigcup_{n\in\Bbb N}K_n$ is
dense in $X$.   Set

\Centerline{$L_n=\{\sum_{i=0}^n\alpha_ix_i:|\alpha_i|\le n,\,
x_i\in\bigcup_{j\le n}K_j$ for every $i\le n\}$}

\noindent for $n\in\Bbb N$.   Then every $L_n$ is weakly compact, and
$Y=\bigcup_{n\in\Bbb N}L_n$ is a linear subspace of $X$ including
$\bigcup_{n\in\Bbb N}K_n$, therefore dense.   Now $Y$ is a countable
union of weakly compact sets, therefore K-analytic for its weak topology
(422Gc, 422Hc);  in particular, it is weakly K-countably determined.
By 467Ic, $X$ is weakly K-countably determined.
}%end of proof of 467M

\leader{467N}{Theorem} Let $X$ be a Banach lattice with an
order-continuous norm\cmmnt{ (\S354)}.   Then it has an equivalent
locally uniformly rotund norm.

\proof{{\bf (a)} Consider first the case in which $X$ has a weak order
unit $e$.   Then the interval $[0,e]$ is weakly compact.   \Prf\ We have
$X^*=X^{\times}$ and $X^{**}=X^{\times\sim}$ (356D).   The canonical
identification of $X$ with its image in $X^{**}$ is an order-continuous
Riesz homomorphism from $X$ onto a solid order-dense Riesz subspace of
$X^{\times\times}$ (356I) and therefore of $X^{\times\sim}=X^{**}$.
In particular, $[0,e]\subseteq X$ is matched with an interval
$[0,\hat e]\subseteq X^{**}$.   But
$[0,\hat e]=\{\theta:\theta\in X^{**},\,0\le\theta(f)\le f(e)$ for every
$f\in(X^*)^+\}$ is weak*-closed and norm-bounded in $X^{**}$, therefore
weak*-compact;  as the weak* topology on $X^{**}$ corresponds to
the weak topology of $X$, $[0,e]$ is weakly compact.\ \Qed

Now, for each $n\in\Bbb N$, $K_n=[-ne,ne]=n[0,e]-n[0,e]$ is weakly
compact.   If $x\in X^+$, then $\sequencen{x\wedge ne}$ converges to
$X$, because the norm of $X$ is order-continuous;  so for any $x\in X$,
$\sequencen{x^+\wedge ne-x^-\wedge ne}$ converges to $x$.   Thus
$\bigcup_{n\in\Bbb N}K_n$ is dense in $X$ and $X$ is weakly compactly
generated.   By 467M and 467K, $X$ has an equivalent locally uniformly
rotund norm.

\medskip

{\bf (b)} For the general case,
let $\familyiI{x_i}$ be a maximal disjoint family in $X^+$.   For each
$i\in I$ let $X_i$ be the band in $X$ generated by $x_i$, and
$T_i:X\to X_i$ the band projection onto $X_i$ (354Ee, 353Hb).   Then
$\familyiI{T_i}$ satisfies the conditions of 467F.   \Prf\ (i) Each
$T_i$ is a continuous linear operator because the given norm $\|\,\|$ of
$X$ is a Riesz norm.   Next, each $X_i$ has a weak order unit $x_i$, and
the norm on $X_i$ is order-continuous, so (a) tells us that there is an
equivalent locally uniformly rotund norm on $X_i=T_i[X]$.   (ii) If
$x\in X$, set $x'=\sup_{i\in I}T_i|x|$;  then $(|x|-x')\wedge x_i=0$ for
every $i$, so, by the maximality of $\familyiI{x_i}$, $x'=|x|$.   If
$\epsilon>0$ then, because the norm of $X$ is order-continuous, there is
a finite $J\subseteq I$ such that

\Centerline{$\|x-\sum_{j\in J}T_ix\|
=\|x'-\sup_{j\in J}T_i|x|\|\le\epsilon$.}

\noindent Moreover, if $i\in I\setminus J$, then

\Centerline{$\|T_ix\|\le\|x-\sum_{j\in J}T_jx\|\le\epsilon$.}

\noindent Thus conditions (ii) and (iii) of 467F are satisfied.\ \Qed

Accordingly 467F tells us that $X$ has an equivalent locally uniformly
rotund norm.
}%end of proof of 467N

\leader{467O}{Eberlein compacta:  Definition} A topological space $K$ is
an {\bf Eberlein compactum} if it is homeomorphic to a weakly compact
subset of a Banach space.

\leader{467P}{Proposition} Let $K$ be a compact Hausdorff space.

(a) The following are equiveridical:

\quad(i) $K$ is an Eberlein compactum;

\quad(ii) there is a set $L\subseteq C(K)$, separating the points of
$K$, which is compact for the topology of pointwise convergence.

(b) Suppose that $K$ is an Eberlein compactum.

\quad(i) $K$ has a $\sigma$-isolated network, so is hereditarily
weakly $\theta$-refinable.

\quad(ii)\cmmnt{ ({\smc Schachermayer 77})} If $w(K)$ is measure-free,
$K$ is a Radon space.

\proof{{\bf (a)(i)$\Rightarrow$(ii)} If $K$ is an Eberlein compactum, we
may suppose that it is a weakly compact subset of a Banach space $X$.
Set $L=\{f\restr K:f\in X^*,\,\|f\|\le 1\}$;  since the map
$f\mapsto f\restr K:X^*\to C(K)\}$ is continuous for the weak* topology
of $X^*$ and the topology $\frak T_p$ of pointwise convergence on
$C(K)$, $L$ is $\frak T_p$-compact;  and $L$ separates the points of $K$
because $X^*$ separates the points of $X$.

\medskip

\quad{\bf (ii)$\Rightarrow$(i)} If $L\subseteq C(K)$ is
$\frak T_p$-compact and separates the points of $K$, set
$L_n=\{f:f\in L,\,\|f\|_{\infty}\le n\}$ for each $n\in\Bbb N$.   Then
$L_n$ is $\frak T_p$-compact for each $n$.   Set
$L'=\{0\}\cup\bigcup_{n\in\Bbb N}2^{-n}L_n$;  then $L'\subseteq C(K)$ is
norm-bounded and $\frak T_p$-compact and separates the points of $K$.
Now define $x\mapsto\hat x:K\to\BbbR^{L'}$ by setting $\hat x(f)=f(x)$
for $x\in K$ and $f\in L'$.   Then $\hat x\in C(L')$ for every $x$ and
$x\mapsto\hat x$ is continuous for the given topology of $K$ and the
topology of pointwise convergence on $C(L')$;  so the image
$\hat K=\{\hat x:x\in K\}$ is $\frak T_p$-compact.   Since it is also
bounded, it is weakly compact (462E).   But $x\mapsto\hat x$ is injective, because $L'$ separates
the points of $K$;  so $K$ is homeomorphic to $\hat K$, and is an
Eberlein compactum.

\medskip

{\bf (b)} Again suppose $K$ is actually a weakly compact subset of a
Banach space $X$.   As in 467M, set
$L_n=\{\sum_{i=0}^n\alpha_ix_i:|\alpha_i|\le n,\,x_i\in K$ for every
$i\le n\}$ for each $n\in\Bbb N$.   Then
$Y=\overline{\bigcup_{n\in\Bbb N}L_n}$ is a weakly compactly generated
Banach space.   (I am passing
over the trivial case $K=\emptyset$.)   So $Y$ has an equivalent locally
uniformly rotund norm (467M, 467K), which is a Kadec norm (467B), and
$Y$, with the weak topology, has a $\sigma$-isolated network
(466Eb).   It follows at once that $K$ has a $\sigma$-isolated
network (4A2B(a-ix)), so is hereditarily weakly $\theta$-refinable
(438Ld);  and if $w(K)$ is measure-free, $K$ is
Borel-measure-complete (438M), therefore Radon (434Jf, 434Ka).
}%end of proof of 467P

\exercises{\leader{467X}{Basic exercises (a)}%
%\spheader 467Xa
(i) Show that a continuous image of a K-countably
determined space is K-countably determined.   (ii) Show that the product
of a sequence of K-countably determined spaces is K-countably
determined.   (iii) Show that any K-countably determined topological
space is Lindel\"of.  \Hint{422De.}
(iv) Show that any Souslin-F subset of a K-countably determined
topological space is K-countably determined.   \Hint{422Hc.}
%467H

\spheader 467Xb Let $X$ be a $\sigma$-compact Hausdorff space.   Show
that a subspace $Y$ of $X$ is K-countably determined iff there is a
countable family $\Cal K$ of compact subsets of $X$ such that
$\bigcap\{K:y\in K\in\Cal K\}\subseteq Y$ for every $y\in Y$.
%467H mt46bits

\spheader 467Xc Show that if $X$ and $Y$ are weakly K-countably
determined normed spaces, then $X\times Y$, with an appropriate norm, is
weakly K-countably determined.
%467I

\spheader 467Xd Show that a normed space $X$ is weakly compactly
generated iff there is a weakly compact set $K\subseteq X$ such that the
linear subspace of $X$ generated by $K$ is dense in $X$.
%467L

\spheader 467Xe Show that any separable normed space is weakly compactly
generated.
%467Xd 467L

\spheader 467Xf Show that any reflexive Banach space is weakly compactly generated.
%467L

\sqheader 467Xg Show that if $X$ is a weakly compactly generated Banach
space, then it is K-analytic in its weak topology.   \Hint{in 467M, use
the proof of 467Ic.}
%467M

\spheader 467Xh Show that if $X$ is a Banach space and there is a set
$A\subseteq X$ such that $A$ is K-countably determined for the weak
topology and the linear subspace generated by $A$ is dense, then $X$ is
weakly K-countably determined.
%467M

\spheader 467Xi Show that the one-point compactification of any discrete
space is an Eberlein compactum.
%467P

\sqheader 467Xj Let $K$ be an Eberlein compactum,
and $\mu$ a Radon measure on $K$.
Show that $\mu$ is completion regular and inner regular with respect to
the compact metrizable subsets of $K$.   \Hint{466B.}
%467P

\leader{467Y}{Further exercises (a)}
%\spheader 467Ya
Let $\sequencen{X_n}$ be a sequence of weakly
K-countably determined normed spaces.   Investigate normed subspaces of
$\prod_{n\in\Bbb N}X_n$ which will be weakly K-countably determined.
%467I

\spheader 467Yb(i) Show that if $(\Omega,\Sigma,\mu)$ is a probability
space, then $L^1(\mu)$ has a locally uniformly rotund Riesz norm.
\Hint{apply the construction of 467D with $Y=L^2(\mu)$ and $T$ the
identity operator;  show that all the norms $p_n$ are Riesz norms.}
(ii) Show that if $X$ is any $L$-space then it has a locally uniformly
rotund Riesz norm.   \Hint{apply the construction of 467F/467N, noting
that if the $T_i$ in 467F are band projections and $\|\,\|$ and all the
$\|\,\|_i$ are Riesz norms, then all the norms in the proof of 467F are
Riesz norms.}
%467E

\spheader 467Yc\dvAnew{2011}
Let $(X,\|\,\|)$ be a Banach space, and $\frak T$ a linear
space topology on $X$ such that the unit ball of $X$ is $\frak T$-closed.
Suppose that
$\familyiI{T_i}$ is a family of bounded linear operators from $X$ to itself
such that

\inset{(i) for each $i\in I$, $T_i$ is $\frak T$-continuous as well as
norm-continuous,

(ii) for each $i\in I$, the subspace $T_i[X]$ has an equivalent
locally uniformly rotund norm for which the unit ball is closed for the
topology on $T_i[X]$ induced by $\frak T$,

(ii) for each $x\in X$, $\epsilon>0$ there is a finite set
$J\subseteq I$ such that $\|x-\sum_{i\in J}T_ix\|\le\epsilon$,

(iii) for each $x\in X$, $\epsilon>0$  the set
$\{i:i\in I,\,\|T_ix\|\ge\epsilon\}$ is finite.}

\noindent  Show that
$X$ has an equivalent locally uniformly rotund norm for which the unit ball
is $\frak T$-closed.
%467F

\spheader 467Yd\dvAnew{2011}
Let $X$ be a normed space with a locally uniformly rotund
norm, and $\frak T$ a linear space
topology on $X$ such that the unit ball of $X$ is $\frak T$-closed.
Show that every norm-Borel subset of $X$ is $\frak T$-Borel.
%466Yb

\spheader 467Ye\dvAnew{2011}
Let $\kappa$ be any cardinal, and $K$ a dyadic space.   (i) Show that
$C(K)$ has a locally uniformly rotund norm, equivalent to
the usual supremum norm $\|\,\|_{\infty}$, for which the unit ball is
closed for the topology $\frak T_p$ of pointwise convergence.
(See {\smc Deville Godefroy \& Zizler 93}, VII.1.10.)
(ii) Show that the norm topology on $C(K)$, the weak topology on $C(K)$ and
$\frak T_p$ give rise to the same Borel $\sigma$-algebras.
(iii) Show that $\frak T_p$ has a $\sigma$-isolated network.   (iv) Show
that if
$w(K)$ is measure-free, then $(C(K),\frak T_p)$ is Radon, and every
$\frak T_p$-Radon measure on $C(K)$ is norm-Radon.
%438H 438L 438M 466F 466Yb 467Yc 467Yd 467G 467P mt46bits

}%end of exercises

\endnotes{
\Notesheader{467} The purpose of this section has been to give an idea
of the scope of Proposition 466F.   `Local uniform rotundity' has an
important place in the geometrical theory of Banach spaces, but for the
many associated ideas I refer you to
{\smc Deville Godefroy \& Zizler 93}.   From our point of view, Theorem
467E is therefore purely accessory, since we know by different arguments
that on separable Banach spaces the weak and norm topologies have the
same Borel $\sigma$-algebras (4A3V).   We need it to provide the first
step in the inductive proof of 467K.

Since the concept of `K-analytic' space is one of the fundamental ideas
of Chapter 43, it is natural here to look at `K-countably determined'
spaces, especially as many of the ideas of \S422 are directly
applicable (467Xa).   But the goal of this part of the argument is
Schachermayer's theorem 467P(b-ii), which uses `weakly compactly
generated' spaces (467L).   `Eberlein compacta' are of great
interest in other ways;  they are studied at length in
{\smc Arkhangel'skii 92}.

I mention order-continuous norms here (467N) because they are prominent in the theory of Banach lattices in Volume 3.
Note that the methods here do {\it not} suffice in
general to arrange that the locally uniformly rotund norm found on $X$
is a Riesz norm;  though see 467Yb.   It is in fact the case that every Banach lattice with an order-continuous norm has an equivalent locally uniformly rotund Riesz norm,
but this requires further ideas (see
{\smc Davis Ghoussoub \& Lindenstrauss 81}).

The general question of identifying Banach spaces with equivalent Kadec
norms remains challenging.
For a recent survey see {\smc Molt\'o Orihuela Troyanski \& Valdivia 09}.
}%end of notes

\discrpage

