\frfilename{mt453.tex} 
\versiondate{22.3.10/23.3.10} 
\copyrightdate{1998} 
      
\def\chaptername{Perfect measures, disintegrations and processes} 
\def\sectionname{Strong liftings} 
      
\def\undtheta{\underline{\theta}} 
\def\undphi{\underline{\phi}} 
\def\undpsi{\underline{\psi}} 
\def\lti{left-\vthsp translation-\vthsp invariant} 
      
\newsection{453} 
      
The next step involves the concept of `strong' lifting on a 
topological measure space (453A);  I devote a few pages to describing 
the principal cases in which strong liftings are known to exist 
(453B-453J).   When we have {\it Radon} measures $\mu$ and $\nu$, with 
an {\it almost continuous} \imp\ function between them, and a {\it 
strong} lifting for $\nu$, we 
can hope for a disintegration $\family{y}{Y}{\mu_y}$ such that (almost) 
every $\mu_y$ lives on the appropriate fiber.   This is 
the content of 453K.   I end the section with a note on the relation 
between strong liftings and Stone spaces (453M) and with V.Losert's 
example of a space with no strong lifting (453N). 
      
\cmmnt{Much of the work here is based on ideas in 
{\smc Ionescu Tulcea \& Ionescu Tulcea 69}.} 
      
\leader{453A}{}\cmmnt{ The proof of the first disintegration theorem I 
presented, 452H, depended on two 
essential steps:  the use of a lifting for $(Y,\Tau,\nu)$ to define the 
finitely additive functionals $\psi_y$, and the use of a countably 
compact class to convert these into countably additive functionals.   In 
452O I observed that if our countably compact class is the family of 
compact sets in a Hausdorff space, we can get Radon measures in our 
disintegration.   Similarly, if 
we have a lifting of a special type, we can hope for special properties 
of the disintegration.   A particularly important kind of lifting, in 
this context, is the following. 
      
\medskip 
      
\noindent}{\bf Definition} Let $(X,\frak T,\Sigma,\mu)$ be a topological 
measure space.   A lifting $\phi:\Sigma\to\Sigma$ is {\bf strong} or 
{\bf of local type} if $\phi G\supseteq G$ for every open set 
$G\subseteq X$, that is, if $\phi F\subseteq F$ for every closed set 
$F\subseteq X$.   I will say that $\phi$ is {\bf almost strong} if 
$\bigcup_{G\in\frak T}\,G\setminus\phi G$ is negligible. 
      
Similarly, if $\frak A$ is the measure algebra of $\mu$, a lifting 
$\theta:\frak A\to\Sigma$ is {\bf strong} if  
$\theta G^{\ssbullet}\supseteq G$ for every open set $G\subseteq X$, and {\bf almost strong} if  
$\bigcup_{G\in\frak T}G\setminus\theta G^{\ssbullet}$ is negligible. 
      
\cmmnt{ Obviously a strong lifting is 
almost strong.} 
 
%what about the definition:  $\phi$ is strong if  
%$\phi E\subseteq\overline{E}$ for every $E\in\Sigma$?   
%works for non-topological measures? 
      
\leader{453B}{}\cmmnt{ We already have the machinery to describe a 
particularly striking class of strong liftings. 
      
\medskip 
      
\noindent}{\bf Theorem} Let $X$ be a topological group with a Haar 
measure $\mu$, and $\Sigma$ its algebra of Haar measurable sets. 
      
(a) If $\phi:\Sigma\to\Sigma$ is 
a \lti\ lifting\cmmnt{, in the sense of 447A}, then $\phi$ is strong. 
      
(b) $\mu$ has a strong lifting. 
      
\proof{{\bf (a)} Apply 447B with $Y=\{e\}$ and $\undphi=\phi$. 
      
\medskip 
      
{\bf (b)} For there is a \lti\ lifting (447J). 
}%end of proof of 453B 
      
\cmmnt{\medskip 
      
\noindent{\bf Remark} In particular, translation-invariant liftings on 
$\BbbR^r$ or $\{0,1\}^I$ (\S345) are strong. 
}%end of comment 
      
\leader{453C}{Proposition} Let $(X,\frak T,\Sigma,\mu)$ be a topological 
measure space and $\phi:\Sigma\to\Sigma$ a lifting.   Write 
$\eusm L^{\infty}$ for the space of bounded $\Sigma$-measurable 
real-valued 
functions on $X$, so that $\eusm L^{\infty}$ can be identified with 
$L^{\infty}(\Sigma)$\cmmnt{ (363H)} and the Boolean homomorphism 
$\phi:\Sigma\to\Sigma$ gives rise to a Riesz homomorphism 
$T:\eusm L^{\infty}\to\eusm L^{\infty}$\cmmnt{ (363F)}. 
      
(a) If $\phi$ is a strong lifting, then $Tf=f$ for every bounded 
continuous function $f:X\to\Bbb R$. 
      
(b) If $(X,\frak T)$ is completely regular and 
$Tf=f$ for every $f\in C_b(X)$, then 
$\phi$ is strong. 
      
\proof{{\bf (a)} Suppose first that 
$f\ge 0$.   For $\alpha\in\Bbb R$, set $G_{\alpha}=\{x:x\in 
X,\,f(x)>\alpha\}$;  then $G_{\alpha}$ is open, so $\phi 
G_{\alpha}\supseteq G_{\alpha}$.   We have $f\ge\alpha\chi G_{\alpha}$, 
so 
      
\Centerline{$Tf\ge\alpha T(\chi G_{\alpha}) 
=\alpha\chi(\phi G_{\alpha})\ge\alpha\chi G_{\alpha}$,} 
      
\noindent that is, $(Tf)(x)\ge\alpha$ whenever $f(x)>\alpha$.   As 
$\alpha$ is arbitrary, $Tf\ge f$.   At the same time, setting 
$\gamma=\|f\|_{\infty}$, we have 
      
\Centerline{$T(\gamma\chi X-f)\ge\gamma\chi X-f$, 
\quad $T(\gamma\chi X)=\gamma\chi(\phi X)=\gamma\chi X$,} 
      
\noindent so $Tf\le f$ and $Tf=f$. 
      
For general $f\in C_b(X)$, 
      
\Centerline{$Tf=T(f^+-f^-)=Tf^+-Tf^-=f^+-f^-=f$,} 
      
\noindent where $f^+$ and $f^-$ are the positive and negative parts of 
$f$. 
      
\medskip 
      
{\bf (b)} Let $G\subseteq X$ be open and $x$ any 
point of $G$.   Then there is an $f\in C_b(X)$ such 
that $f\le\chi G$ and $f(x)=1$.   In this case 
      
\Centerline{$f=Tf\le T(\chi G)=\chi(\phi G)$,} 
      
\noindent so $x\in\phi G$.   As $x$ is arbitrary, $G\subseteq\phi 
G$;  as $G$ is arbitrary, $\phi$ is strong. 
}%end of proof of 453C 
      
\leader{453D}{Proposition} Let $(X,\frak T,\Sigma,\mu)$ be a topological 
measure space. 
      
(a) If $\mu$ has a strong lifting it is strictly 
positive\cmmnt{ (definition: 411Nf)}. 
      
(b) If $\mu$ is strictly positive and complete, and has an almost strong 
lifting, it has a strong lifting. 
      
(c) If $\mu$ has an almost strong lifting it is $\tau$-additive, 
so has a support. 
      
(d) If $\mu$ is complete and $\mu X>0$ and the subspace measure $\mu_E$ 
has an almost strong lifting for some conegligible set $E\subseteq X$, 
then $\mu$ has an almost strong lifting. 
      
\proof{{\bf (a)} If $\phi:\Sigma\to\Sigma$ is a strong lifting, then 
$G\subseteq\phi G=\emptyset$ whenever $G$ is a negligible open set, so 
$\mu$ is strictly positive. 
      
\medskip 
      
{\bf (b)} If $\mu$ is strictly positive and complete and 
$\phi:\Sigma\to\Sigma$ is an almost strong lifting, set 
$A=\bigcup_{G\in\frak T}G\setminus\phi G$.   For each $x\in A$, let 
$\Cal I_x$ be the ideal of subsets of $X$ generated by 
      
\Centerline{$\{F:F\subseteq X$ is closed, $x\notin F\}\cup\{B:B\subseteq 
X$ is negligible$\}$.} 
      
\noindent Then $X\notin\Cal I_x$, because $\mu$ is strictly positive, so 
a closed set not containing $x$ cannot be conegligible.   There is 
therefore a Boolean homomorphism $\psi_x:\Cal PX\to\{0,1\}$ such that 
$\psi_xF=0$ for every $F\in\Cal I_x$ (311D).   Set 
      
\Centerline{$\tilde\phi E 
=(\phi E\setminus A)\cup\{x:x\in A,\,\psi_xE=1\}$} 
      
\noindent for $E\in\Sigma$.   It is easy to check 
that $\tilde\phi:\Sigma\to\Cal PX$ is a Boolean homomorphism.   (Compare 
the proof of 341J.)   If $E\in\Sigma$, then 
      
\Centerline{$E\symmdiff\tilde\phi E\subseteq(E\symmdiff\phi E)\cup A$} 
      
\noindent is negligible, so (because $\mu$ is complete) $\tilde\phi 
E\in\Sigma$.   If $E$ is negligible, then 
$E\in\Cal I_x$ and $\psi_xE=0$ for every $x\in A$, so $\tilde\phi E=\phi 
E=\emptyset$.   Thus $\tilde\phi$ is a lifting.   Now suppose that $x\in 
G\in\frak T$.   If $x\in A$, then $X\setminus G\in\Cal I_x$, 
so $\psi_x(X\setminus G)=0$, $\psi_xG=1$ and $x\in\tilde\phi G$.   If 
$x\notin A$, then $x\in\phi G$ and again $x\in\tilde\phi G$.   As $x$ 
and $G$ are arbitrary, $\tilde\phi$ is a strong lifting. 
      
\medskip 
      
{\bf (c)} Suppose that $\phi:\Sigma\to\Sigma$ is an almost strong 
lifting.   Let $\Cal G$ be a non-empty upwards-directed family of open 
sets with union $H$.   If $\sup_{G\in\Cal G}\mu G=\infty$, this is 
surely equal to $\mu H$.   Otherwise, there is a non-decreasing sequence 
$\sequencen{G_n}$ in $\Cal G$ such that $G\setminus G^*$ is negligible 
for every $G\in\Cal G$, where $G^*=\bigcup_{n\in\Bbb N}G_n$ (215Ab).   
Then $\phi G\subseteq\phi G^*$ 
for every $G\in\Cal G$.   This means that 
      
\Centerline{$H\setminus\phi G^* 
\subseteq\bigcup_{G\in\Cal G}G\setminus\phi G$} 
      
\noindent is negligible, because $\phi$ is almost strong, and 
      
\Centerline{$\mu H\le\mu(\phi G^*)=\mu G^* 
=\lim_{n\to\infty}\mu G_n=\sup_{G\in\Cal G}\mu G$.} 
      
\noindent As $\Cal G$ is arbitrary, $\mu$ is $\tau$-additive. 
By 411Nd, it has a support. 
      
\medskip 
      
{\bf (d)} Now suppose that $\mu$ is complete, that $\mu X>0$ and that 
there is a conegligible $E\subseteq X$ such that $\mu_E$ has an almost 
strong lifting $\phi$.   Let $\psi:\Cal PX\to\{\emptyset,X\}$ be any 
Boolean homomorphism such that $\psi A=\emptyset$ whenever $A$ is 
negligible.   (This is where I use the hypothesis that $X$ is not 
negligible.)   Define $\tilde\phi:\Sigma\to\Cal PX$ by setting 
      
\Centerline{$\tilde\phi F=\phi(E\cap F)\cup(\psi F\setminus E)$.} 
      
\noindent Then $\tilde\phi$ is a Boolean homomorphism because $\phi$ and 
$\psi$ are; 
      
\Centerline{$F\symmdiff\tilde\phi F 
  \subseteq((E\cap F)\symmdiff\phi(E\cap F))\cup(X\setminus E)$} 
      
\noindent is negligible, so $\tilde\phi F\in\Sigma$, for every 
$F\in\Sigma$, because $\mu$ is complete;  and if $F$ is negligible, then 
$\phi(E\cap F)=\psi F=\emptyset$ so $\tilde\phi F=\emptyset$.   Thus 
$\tilde\phi$ is a lifting.   Finally, 
      
\Centerline{$\bigcup_{G\in\frak T}G\setminus\tilde\phi G 
\subseteq(X\setminus E)\cup\bigcup_{G\in\frak T}((G\cap 
E)\setminus\phi(G\cap E))$} 
      
\noindent is negligible because $\phi$ is almost strong and $E$ is 
conegligible. 
}%end of proof of 453D 
      
\leader{453E}{Proposition} Let $(X,\frak T,\Sigma,\mu)$ be a complete 
strictly localizable topological measure space with an almost strong 
lifting, and $A\subseteq X$ a non-negligible set.   Then the subspace 
measure $\mu_A$ has an almost strong lifting. 
      
\proof{ Let $\phi:\Sigma\to\Sigma$ be an almost strong lifting. 
Because $\mu$ is strictly localizable, $A$ has a measurable envelope $W$ 
say (put 213J and 213L together).   Write $\Sigma_A$ for the 
subspace $\sigma$-algebra on $A$.   Let 
$\psi:\Sigma_A\to\{\emptyset,A\}$ be any Boolean homomorphism such that 
$\psi H=\emptyset$ for every negligible set $H\subseteq A$. 
      
If $E$, $F\in\Sigma$ and $E\cap A=F\cap A$, then $\phi E\cap\phi W=\phi 
F\cap\phi W$.   \Prf\ 
      
\Centerline{$\mu((E\symmdiff F)\cap W)=\mu^*((E\symmdiff F)\cap A)=0$,} 
      
\noindent so 
      
\Centerline{$(\phi E\cap\phi W)\symmdiff(\phi F\cap\phi W) 
=\phi((E\symmdiff F)\cap W)=\emptyset$.   \Qed} 
      
We can therefore define a function $\tilde\phi:\Sigma_A\to\Cal PA$ by 
setting 
      
\Centerline{$\tilde\phi H=(\phi E\cap\phi W\cap A)\cup(\psi 
H\setminus\phi W)$} 
      
\noindent whenever $E\in\Sigma$ and $H=E\cap A$.   It is easy to check 
that $\tilde\phi$ is a Boolean homomorphism.   If $E\in\Sigma$ then 
      
\Centerline{$(E\cap A)\symmdiff\tilde\phi(E\cap A) 
\subseteq(E\symmdiff\phi E)\cup(A\setminus\phi W) 
\subseteq(E\symmdiff\phi E)\cup(W\setminus\phi W)$} 
      
\noindent is negligible, so $\tilde\phi(E\cap A)\in\Sigma_A$ (because 
$\mu$ and $\mu_A$ are complete).   If $H\in\Sigma_A$ is negligible, then 
      
\Centerline{$\tilde\phi H\subseteq\phi H\cup\psi H=\emptyset$,} 
      
\noindent so $\tilde\phi$ is a lifting for $\mu_A$. 
      
Now set $B=(A\setminus\phi W)\cup\bigcup_{G\in\frak T}G\setminus\phi G$. 
Because $\phi$ is almost strong, $B$ is negligible.   If $H\subseteq 
A$ is relatively open, then $H\setminus\tilde\phi H\subseteq B$.   \Prf\ 
Take $x\in H\setminus\tilde\phi H$.   Express $H$ as $G\cap A$ where 
$G\subseteq X$ is open.   If $x\in\phi W$, then $x\notin\phi G$ so $x\in 
B$;  if $x\notin\phi W$, then of course $x\in B$.\ \QeD\  Thus 
      
\Centerline{$\bigcup\{H\setminus\tilde\phi H:H\subseteq A$ is relatively 
open$\}\subseteq B$} 
      
\noindent is negligible and $\tilde\phi$ is almost strong. 
}%end of proof of 453E 
      
\leader{453F}{Proposition} Let $(X,\frak T,\Sigma,\mu)$ be a complete 
strictly localizable topological measure space. 
      
(a) If $\frak T$ has a countable network, 
any lifting for $\mu$ is almost strong. 
      
(b) Suppose that $\mu X>0$ and $\mu$ is inner regular with respect to 
      
\Centerline{$\Cal K=\{K:K\in\Sigma,\,\mu_K$ has an almost strong 
lifting$\}$,} 
      
\noindent where $\mu_K$ is the subspace measure on $K$.   Then 
$\mu$ has an almost strong lifting. 
      
\proof{{\bf (a)} Let $\Cal E$ be a countable network for 
$\frak T$, and $\phi:\Sigma\to\Sigma$ a lifting.   For each  
$E\in\Cal E$, let $\hat E$ be a measurable envelope of $E$ (213J/213L
again).   Then 
      
$$\eqalignno{\bigcup_{G\in\frak T}G\setminus\phi G 
&=\bigcup_{G\in\frak T,E\in\Cal E,E\subseteq G}E\setminus\phi G
\subseteq\bigcup_{G\in\frak T,E\in\Cal E,E\subseteq G} 
  \hat E\setminus\phi\hat E\cr 
\displaycause{because if $E\subseteq G\in\frak T$, then $G\in\Sigma$, so $\mu(\hat E\setminus G)=0$ and $\phi\hat E\subseteq\phi G$} 
&\subseteq\bigcup_{E\in\Cal E}\hat E\setminus\phi\hat E\cr}$$ 
      
\noindent is negligible, so $\phi$ is almost strong. 
      
\medskip 
      
{\bf (b)} Let $\Cal L\subseteq\Cal K$ be a disjoint family such that 
$\mu^*A=\sum_{L\in\Cal L}\mu^*(A\cap L)$ for every $A\subseteq X$ 
(412Ib).   For each $L\in\Cal L$, let $\Sigma_L$ be the corresponding 
subspace $\sigma$-algebra and $\phi_L:\Sigma_L\to\Sigma_L$ an almost 
strong lifting.   Set $E=\bigcup\Cal L$;  then 
      
\Centerline{$\mu^*(X\setminus E)=\sum_{L\in\Cal L}\mu(L\setminus E)=0$,} 
      
\noindent so $E$ is conegligible.   For $F\in\Sigma_E$ set $\phi 
F=\bigcup_{L\in\Cal L}\phi_L(F\cap L)$;  then $\phi$ is a Boolean 
homomorphism from $\Sigma_E$ to $\Cal PE$.   If $F\in\Sigma_E$, then 
      
\Centerline{$\mu^*(F\symmdiff\phi F) 
=\sum_{L\in\Cal L}\mu^*(L\cap(F\symmdiff\phi F)) 
=\sum_{L\in\Cal L}\mu^*((F\cap L)\symmdiff\phi_L(F\cap L)) 
=0$,} 
      
\noindent while if $\mu F=0$ then $\phi_L(F\cap L)=\emptyset$ for every 
$L$, so $\phi F=\emptyset$.   Thus $\phi$ is a lifting.   Now set 
      
\Centerline{$A=\bigcup\{H\setminus\phi H:H\subseteq E$ is relatively 
open$\}$.} 
      
\noindent If $L\in\Cal L$, then 
      
\Centerline{$A\cap L 
=\bigcup\{(H\cap L)\setminus\phi_L(H\cap L):H\subseteq E$ is relatively 
open$\}$} 
      
\noindent is negligible, because $\phi_L$ is almost strong;  thus $\phi$ 
is an almost strong lifting for $\mu_E$.   By 453Dd, $\mu$ also has an 
almost strong lifting. 
}%end of proof of 453F 
      
\leader{453G}{Corollary} (a) A non-zero quasi-Radon measure on a 
separable metrizable space has an almost strong lifting. 
      
(b) A non-zero Radon measure $\mu$ on an analytic Hausdorff space $X$ 
has an almost strong lifting. 
      
\proof{{\bf (a)} A quasi-Radon measure is complete and strictly 
localizable (415A), so, if non-zero, has a lifting (341K).   A separable 
metrizable space has a countable network (4A2P(a-iii)),  
so this lifting must be almost strong. 
      
\medskip 
      
{\bf (b)} If $K\subseteq X$ is compact and non-negligible, it is 
metrizable (423Dc), so that the subspace measure $\mu_K$ 
has an almost strong lifting, by (a);  as $\mu$ is tight (that is, inner 
regular with respect to the closed 
compact sets), it has an almost strong lifting, by 453Fb. 
}%end of proof of 453G 
      
\cmmnt{\medskip 
      
\noindent{\bf Remark} In particular, Lebesgue measure on $\BbbR^r$ has 
an almost strong lifting and therefore, by 453Db, a strong lifting, as 
already noted in 453B. 
}%end of comment 
      
\leader{453H}{Lemma} Let $(X,\Sigma,\mu)$ be a complete locally 
determined measure space and $\frak T$ a topology on $X$ generated by a 
family $\Cal U\subseteq\Sigma$.   Suppose that $\phi:\Sigma\to\Sigma$ is 
a lifting such that $\phi U\supseteq U$ for every $U\in\Cal U$.   Then 
$\mu$ is a $\tau$-additive topological measure, and $\phi$ is a strong 
lifting. 
      
\proof{ Of course $\phi$ is a lower density, and $\phi X=X$, so by 414P 
we have a density topology 
      
\Centerline{$\frak T_d=\{E:E\in\Sigma,\,E\subseteq\phi E\}$} 
      
\noindent with respect to which $\mu$ is a $\tau$-additive topological 
measure.   But our hypothesis is that $\Cal U\subseteq\frak T_d$, so 
$\frak T\subseteq\frak T_d$ and $\mu$ is a $\tau$-additive topological 
measure with respect to $\frak T$.   Also, of course, 
$\phi G\supseteq G$ for every $G\in\frak T$, so $\phi$ is a strong 
lifting. 
}%end of proof of 453H 
      
\leader{453I}{Proposition} Let 
$\familyiI{(X_i,\frak T_i,\Sigma_i,\mu_i)}$ be a family of topological 
probability spaces such that every $\frak T_i$ has a countable network  
and every $\mu_i$ is strictly positive.   Let $\lambda$ be the 
(ordinary) complete product measure on $X=\prod_{i\in I}X_i$.   Then 
$\lambda$ is a $\tau$-additive topological measure and has a strong 
lifting. 
      
\proof{{\bf (a)} The strategy of the proof is as follows.   We may 
suppose that $I=\kappa$ is a cardinal.   Write $\Lambda$ for the domain 
of $\lambda$, and for each $\xi\le\kappa$ let $\Lambda_{\xi}$ be the 
$\sigma$-algebra of members of $\Lambda$ determined by coordinates less than $\xi$;  write $\pi_{\xi}:X\to X_{\xi}$ for the canonical map.   I seek to define a lifting $\phi:\Lambda\to\Lambda$ such that 
$\phi W\supseteq W$ for every open set $W\in\Lambda$.   This will be the 
last in a family $\langle\phi_{\xi}\rangle_{\xi\le\kappa}$ of partial 
liftings, constructed inductively as in the proof of 341H, with 
$\dom\phi_{\xi}=\Lambda_{\xi}$ for each $\xi$.   The inductive 
hypothesis will be that $\phi_{\xi}$ extends $\phi_{\eta}$ whenever 
$\eta\le\xi$, and 
$\phi_{\xi}\pi_{\eta}^{-1}[G]\supseteq\pi_{\eta}^{-1}[G]$ for every 
$\eta<\xi$ and every open $G\subseteq X_{\eta}$. 
      
The induction starts with $\Lambda_0=\{\emptyset,X\}$, 
$\phi_0\emptyset=\emptyset$, $\phi_0X=X$.   For $\xi\le\kappa$, set 
$\frak B_{\xi}=\{W^{\ssbullet}:W\in\Lambda_{\xi}\}$. 
      
\medskip 
      
{\bf (b)} {\it Inductive step to a successor ordinal $\xi+1$} Suppose 
that $\phi_{\xi}$ has been defined, where $\xi<\kappa$. 
      
\medskip 
      
\quad{\bf (i)} By 341Nb, there is a lifting 
$\phi'_{\xi}:\Lambda\to\Lambda$ extending $\phi_{\xi}$.   Let 
$\Cal E_{\xi}$ be a countable network for 
$\frak T_{\xi}$.   
For each $E\in\Cal E_{\xi}$ let $\hat E$ be a measurable envelope of $E$.   Set 
      
\Centerline{$Q 
=\bigcup\{\pi_{\xi}^{-1}[\hat E] 
  \setminus\phi'_{\xi}(\pi_{\xi}^{-1}[\hat E]): 
E\in\Cal E_{\xi}\}$;} 
      
\noindent then $Q$ is negligible. 
      
\medskip 
      
\quad{\bf (ii)} For $x\in Q$, let $\Cal I_x\subseteq\Lambda$ be the 
ideal generated by 
      
\Centerline{$\{W:W\in\Lambda_{\xi},\,x\notin\phi_{\xi}W\} 
\cup\{\pi_{\xi}^{-1}[F]:F\subseteq X_{\xi}$ is closed, 
$\pi_{\xi}(x)\notin F\} 
\cup\{W:\lambda W=0\}$.} 
      
\noindent Then $X\notin\Cal I_x$.   \Prf\Quer\ Otherwise, there are a 
$W\in\Lambda_{\xi}$, a closed $F\subseteq X_{\xi}$ and a negligible $W'\in\Lambda$ such that $W\cup W'\cup\pi_{\xi}^{-1}[F]=X$ while 
$x\notin\phi_{\xi}W\cup\pi_{\xi}^{-1}[F]$.   But in this case 
      
$$\eqalign{0 
&=\lambda W' 
\ge\lambda((X\setminus W)\cap(X\setminus\pi^{-1}_{\xi}[F]))\cr 
&=\lambda(X\setminus W)\cdot\lambda(X\setminus\pi_{\xi}^{-1}[F]) 
=\lambda(X\setminus W)\cdot\mu_{\xi}(X_{\xi}\setminus F) 
>0\cr}$$ 
      
\noindent because $\mu_{\xi}$ is strictly positive and 
$\phi_{\xi}W\ne X$.\ \Bang\Qed 
      
There is therefore a Boolean homomorphism $\psi_x:\Lambda\to\{0,1\}$ 
which is zero on $\Cal I_x$. 
      
\medskip 
      
\quad{\bf (iii)} Set 
      
\Centerline{$\phi_{\xi+1}W=(\phi'_{\xi}W\setminus Q)\cup\{x:x\in 
Q,\,\psi_xW=1\}$} 
      
\noindent for every $W\in\Lambda_{\xi+1}$.   Then $\phi_{\xi+1}$ is a 
Boolean homomorphism from $\Lambda_{\xi+1}$ to $\Cal PX$.   Because 
$\phi_{\xi+1}W\symmdiff\phi'_{\xi}W\subseteq Q$ is negligible, 
$\phi_{\xi+1}W\in\Lambda$ and $W\symmdiff\phi_{\xi+1}W$ is negligible for every $W\in\Lambda_{\xi+1}$.   If $\lambda 
W=0$ then $\phi'_{\xi}W=\emptyset$ and $\psi_xW=0$ for every $x\in Q$, 
so $\phi_{\xi+1}W=\emptyset$;  thus 
$\phi_{\xi+1}:\Lambda_{\xi+1}\to\Lambda$ is a partial lifting. 
If $W\in\Lambda_{\xi}$, then, for $x\in Q$, 
      
$$\eqalign{x\in\phi_{\xi}W 
&\Longrightarrow x\notin\phi_{\xi}(X\setminus W) 
\Longrightarrow X\setminus W\in\Cal I_x\cr 
&\Longrightarrow \psi_x(X\setminus W)=0 
\Longrightarrow \psi_xW=1\iff x\in\phi_{\xi+1}W\cr 
&\Longrightarrow W\notin\Cal I_x 
\Longrightarrow x\in\phi_{\xi}W,\cr}$$ 
      
\noindent so $\phi_{\xi+1}W=\phi_{\xi}W$.   Thus $\phi_{\xi+1}$ extends 
$\phi_{\xi}$. 
      
\medskip 
      
\quad{\bf (iv)} Suppose that $\eta\le\xi$ and $G\subseteq X_{\eta}$ is 
open. 
If $\eta<\xi$ then 
      
\Centerline{$\phi_{\xi+1}(\pi_{\eta}^{-1}[G]) 
=\phi_{\xi}(\pi_{\eta}^{-1}[G]) 
\supseteq \pi_{\eta}^{-1}[G]$} 
      
\noindent by the inductive hypothesis.   If $\eta=\xi$, take any 
$x\in\pi_{\xi}^{-1}[G]$.   If $x\in Q$, then 
$X\setminus\pi^{-1}_{\xi}[G]\in\Cal I_x$, so 
$\psi_x(\pi_{\xi}^{-1}[G])=1$ and $x\in\phi_{\xi+1}(\pi_{\xi}^{-1}[G])$. 
If $x\notin Q$, there is an $E\in\Cal E_{\xi}$ such that  
$x(\xi)\in E\subseteq G$.   In this case, $x(\xi)\in\hat E$, so 
      
\Centerline{$x\in\pi_{\xi}^{-1}[\hat E]\setminus Q 
\subseteq\phi'_{\xi}(\pi_{\xi}^{-1}[\hat E])\setminus Q 
\subseteq\phi_{\xi+1}(\pi_{\xi}^{-1}[\hat E]) 
\subseteq\phi_{\xi+1}(\pi_{\xi}^{-1}[G])$} 
      
\noindent because $\hat E\setminus G$ and  
$\pi_{\xi}^{-1}[\hat E]\setminus\pi_{\xi}^{-1}[G]$ are negligible.   As $x$ is arbitary,  
$\pi^{-1}_{\eta}[G]\subseteq\phi_{\xi+1}(\pi^{-1}_{\eta}[G])$ in this case also. 
      
Thus the induction continues. 
      
\medskip 
      
{\bf (c)} {\it Inductive step to a non-zero limit ordinal $\xi$ of 
countable cofinality} Suppose that $0<\xi\le\kappa$, that 
$\cf\xi=\omega$ and that $\phi_{\eta}$ has been defined for every 
$\eta<\xi$.   Let $\sequencen{\zeta_n}$ be a non-decreasing sequence in 
$\xi$ with limit $\xi$.   Then $\frak B_{\xi}$ is the closed subalgebra 
of $\frak A$ generated by $\bigcup_{n\in\Bbb N}\frak B_{\zeta_n}$ (using 
254N and 254Fe, or otherwise).   By 341G, there is a partial lower 
density $\undphi:\Lambda_{\xi}\to\Lambda$ extending every 
$\phi_{\zeta_n}$, and 
therefore extending $\phi_{\eta}$ for every $\eta<\xi$.   By 341Jb 
(applied to $\lambda\restrp\widehat{\Lambda}_{\xi}$, where 
$\widehat{\Lambda}_{\xi}$ is the $\sigma$-subalgebra of $\Lambda$ 
generated by $\Lambda_{\xi}\cup\{W:\lambda W=0\}$), there is a partial 
lifting $\phi_{\xi}:\Lambda_{\xi}\to\Lambda$ such that 
$\undphi W\subseteq\phi_{\xi}W$ for every $W\in\Lambda_{\xi}$. 
      
If $\eta<\xi$ and $W\in\Lambda_{\eta}$, then 
      
\Centerline{$\phi_{\eta}W=\undphi W\subseteq\phi_{\xi}W$, 
\quad$X\setminus\phi_{\eta}W=\phi_{\eta}(X\setminus W) 
\subseteq\phi_{\xi}(X\setminus W)=X\setminus\phi_{\xi}W$,} 
      
\noindent so $\phi_{\xi}$ extends $\phi_{\eta}$.   If $\eta<\xi$ and $G\subseteq X_{\eta}$ is open, 
      
\Centerline{$\phi_{\xi}(\pi_{\eta}^{-1}[G]) 
=\phi_{\eta+1}(\pi_{\eta}^{-1}[G]) 
\supseteq\pi_{\eta}^{-1}[G]$.} 
      
\noindent So again the induction continues. 
      
\medskip 
      
{\bf (d)} {\it Inductive step to a limit ordinal $\xi$ of uncountable 
cofinality} In this case, 
$\frak B_{\xi}=\bigcup_{\eta<\xi}\frak B_{\eta}$, as in the proof of 
341H;  so there will be a unique 
partial lifting $\phi_{\xi}:\Lambda_{\xi}\to\Lambda$ extending 
$\phi_{\eta}$ for every $\eta<\xi$ (set $\phi_{\xi}W=\phi_{\eta}W'$ 
whenever $W\in\Lambda_{\xi}$, $\eta<\xi$, $W'\in\Lambda_{\eta}$ and 
$W\symmdiff W'$ is negligible).   As in (c), we again have 
      
\Centerline{$\phi_{\xi}(\pi_{\eta}^{-1}[G]) 
=\phi_{\eta+1}(\pi_{\eta}^{-1}[G]) 
\supseteq \pi_{\eta}^{-1}[G]$} 
      
\noindent whenever $\eta<\xi$ and $G\subseteq X_{\eta}$ is open. 
      
\medskip 
      
{\bf (e)} At the end of the induction, we have a lifting 
$\phi=\phi_{\kappa}$ of $\Lambda$ such that $\phi U\supseteq U$ for 
every $U\in\Cal U$, where 
$\Cal U=\{\pi_{\xi}^{-1}[G]:\xi<\kappa,\,G\in\frak T_{\xi}\}$. 
By 453H, 
$\lambda$ is a $\tau$-additive topological measure and $\phi$ is a 
strong lifting. 
}%end of proof of 453I 
      
\leader{453J}{Corollary} Let $\familyiI{(X_i,\frak T_i,\Sigma_i,\mu_i)}$ 
be a family of quasi-Radon probability spaces such that every 
$\frak T_i$ has a countable network consisting of measurable sets and 
every $\mu_i$ is strictly positive.   Then the ordinary product measure 
$\lambda$ on $X=\prod_{i\in I}X_i$ is quasi-Radon and has a strong 
lifting. 
If every $X_i$ is compact and Hausdorff, then $\lambda$ is a Radon 
measure. 
      
\proof{ We have just seen that $\lambda$ is a $\tau$-additive 
topological measure with a strong lifting; 
but also it is inner regular with respect to the closed sets, by 412Ua, 
so it is a quasi-Radon measure. 
If all the $X_i$ are compact and Hausdorff, so is $X$, so $\lambda$ is a 
Radon measure (416G). 
}%end of proof of 453J 
%for compact Hausdorff X_i, would it be enough here if every 
%(X_i,\Sigma_i) were countably separated?  yes, see 453Xf 
      
\leader{453K}{}\cmmnt{ We come now to the construction of 
disintegrations from strong liftings. 
      
\medskip 
      
\noindent}{\bf Theorem} Let $(X,\frak T,\Sigma,\mu)$ and 
$(Y,\frak S,\Tau,\nu)$ be Radon measure spaces and $f:X\to Y$ an almost 
continuous 
\imp\ function.   Suppose that $\nu$ has an almost strong lifting. 
Then there is a disintegration $\family{y}{Y}{\mu_y}$ of $\mu$ over 
$\nu$ such that every $\mu_y$ is a Radon measure and 
$\mu_yX=\mu_yf^{-1}[\{y\}]=1$ for almost every $y\in Y$. 
      
\proof{{\bf (a)(i)}  
Suppose first that $X$ is compact, $\mu$ is a probability 
measure and that $f$ is continuous. 
      
Turn back to the proofs of 452H-452I.   In part (a) of the proof 
of 452H, suppose that the lifting $\theta:\frak B\to\Tau$ 
corresponds to an almost strong 
lifting $\phi:\Tau\to\Tau$ (see 341Ba).   Set 
$B=\bigcup_{H\in\frak S}H\setminus\phi H$, so that $B$ is negligible. 
In part (c) of the proof of 452H, take $\Cal K$ to be 
the family of compact subsets of $X$.   Then all the $\mu_y$, as 
constructed in 452H, will be Radon probability measures. 
For every $y$, $f^{-1}[\{y\}]$ is a closed set, so is 
necessarily measured by $\mu_y$.   But also it is $\mu_y$-conegligible 
for every $y\in Y\setminus B$. 
\Prf\ Let $K\subseteq X\setminus f^{-1}[\{y\}]$ be a compact set.   Then 
$f[K]$ is a compact set not containing $y$.   Because $Y$ is 
Hausdorff, there is an open set $H$ 
containing $y$ such that $\overline{H}\cap f[K]=\emptyset$ (4A2F(h-i)). 
Now 
      
\Centerline{$y\in H\setminus B\subseteq\phi H 
\subseteq\phi\overline{H}$.} 
      
\noindent Let $E$ be the compact set $f^{-1}[\overline{H}]$.   Taking  
$T:L^{\infty}(\mu)\to L^{\infty}(\nu)$ as in part (a) of the proof of 452I, 
$T(\chi E^{\ssbullet})=\chi\overline{H}^{\ssbullet}$, so 
      
\Centerline{$\psi_yE=(ST(\chi E^{\ssbullet}))(y) 
=(S(\chi\overline{H}^{\ssbullet}))(y)=(\chi(\phi\overline{H}))(y)=1$.} 
      
\noindent Because $E\in\Cal K$, $\mu_yE\ge\psi_yE$;  since we always 
have $\mu_yX=1$, $E$ is $\mu_y$-conegligible.   But  
$K\cap E=\emptyset$, so $\mu_yK=0$.   As $K$ is arbitrary,  
$\mu_y(X\setminus f^{-1}[\{y\}])=0$. \Qed 
      
Thus $\mu_yf^{-1}[\{y\}]=1$ for almost every $y\in Y$, while
$\mu_yX\le 1$ for every $y$.
 
\medskip 
 
\quad{\bf (ii)} The result for totally finite $\mu$ and $\nu$ and 
continuous $f$ follows at once. 
      
\medskip 
      
{\bf (b)} Now suppose that $\mu$ and $\nu$ are totally finite, and that $f$ 
is almost continuous. 
      
\medskip 
      
\quad{\bf (i)} Let $\Cal K$ be the family of subsets $K\subseteq X$ such 
that 
      
\inset{$K$ is compact and $f\restr K$ is continuous,} 
      
\inset{whenever $F\in\Tau$ and $\nu(F\cap f[K])>0$ then 
$\mu(K\cap f^{-1}[F])>0$,} 
      
\inset{either $K=\emptyset$ or $\mu K>0$.} 
      
\noindent Take any $E\in\Sigma$ such that $\mu E>0$.   Then there is a 
$K\in\Cal K$ such that $K\subseteq E$ and $\mu K>0$.   \Prf\ Let 
$K_0\subseteq E$ 
be a compact set such that $f\restr K_0$ is continuous and $\mu K_0>0$. 
Let $\delta>0$ be such that $\mu K_0-\delta\nu Y>0$.   For compact sets 
$K\subseteq K_0$ set $q(K)=\mu K-\delta\nu f[K]$.   Choose 
$\sequencen{\alpha_n}$, $\langle K_n\rangle_{n\ge 1}$ as follows. 
Given that $K_n$ is a compact subset of $K_0$, where $n\in\Bbb N$, set 
      
\Centerline{$\alpha_n=\sup\{q(K):K\subseteq K_n$ is compact$\}$,} 
      
\noindent and choose a compact subset $K_{n+1}$ of $K_n$ such that 
$q(K_{n+1})\ge\max(q(K_n),\alpha_n-2^{-n})$.   Continue. 
      
Set $K=\bigcap_{n\in\Bbb N}K_n$.   We have 
      
$$\eqalign{q(K) 
&=\mu K-\delta\nu f[K]\cr 
&\ge\lim_{n\to\infty}\mu K_n-\delta\inf_{n\in\Bbb N}\nu f[K_n] 
=\lim_{n\to\infty}q(K_n) 
=\sup_{n\in\Bbb N}q(K_n)\cr}$$ 
      
\noindent because $\sequencen{q(K_n)}$ is non-decreasing.   Of course 
$K\subseteq E$, 
      
\Centerline{$\mu K\ge q(K)\ge q(K_0)>0$,} 
      
\noindent and $f\restr K$ is continuous because $K\subseteq K_0$. 
      
\Quer\ If there is an $F\in\Tau$ such that $\nu(F\cap f[K])>0$ but 
$\mu(K\cap f^{-1}[F])=0$, take a compact set $K'\subseteq K\setminus 
f^{-1}[F]$ such that $\mu K'>\mu K-\delta\nu(F\cap f[K])$.   Then 
$f[K']\subseteq f[K]\setminus F$, so 
      
\Centerline{$q(K') 
=\mu K'-\delta\nu f[K'] 
\ge\mu K'-\delta(\nu f[K]-\nu(F\cap f[K])) 
>\mu K-\delta\nu f[K] 
=q(K)$.} 
      
\noindent Let $n\in\Bbb N$ be such that $q(K')>q(K)+2^{-n}$;  then $K'$ 
is a compact subset of $K_n$, so 
      
\Centerline{$\alpha_n\ge q(K')>q(K)+2^{-n} 
\ge q(K_{n+1})+2^{-n}\ge\alpha_n$,} 
      
\noindent which is impossible.\ \BanG\  
Thus $K$ belongs to $\Cal K$ and will serve.\ \Qed 
      
\medskip 
      
\quad{\bf (ii)} By 342B, there is a countable disjoint set 
$\Cal K_0\subseteq\Cal K$ 
such that $\mu(X\setminus\bigcup\Cal K_0)=0$.   Enumerate $\Cal K_0$ as 
$\langle K_n\rangle_{n<\#(\Cal K_0)}$;  for convenience of notation, if 
$\Cal K_0$ is finite, set $K_n=\emptyset$ for $n\ge\#(\Cal K_0)$, so 
that every $K_n$ belongs to $\Cal K$ and $\mu 
E=\sum_{n=0}^{\infty}\mu(E\cap K_n)$ for every $E\in\Sigma$. 
      
\medskip 
      
\quad{\bf (iii)} For each $n\in\Bbb N$, define $\lambda_n:\Tau\to\Bbb R$ 
by setting $\lambda_nF=\mu(K_n\cap f^{-1}[F])$ for every $F\in\Tau$. 
Then $\lambda_n$ is a measure dominated by $\nu$, so 
there is a $\Tau$-measurable $g_n:Y\to[0,1]$ such that 
$\lambda_nF=\int_Fg_n$ for every 
$F\in\Tau$, by the Radon-Nikod\'ym theorem.   Because 
$\lambda_n(Y\setminus f[K_n])=0$, we may suppose that $g_n(y)=0$ for 
$y\notin f[K_n]$.   We have 
      
\Centerline{$\int_F\sum_{n=0}^{\infty}g_n 
=\sum_{n=0}^{\infty}\int_Fg_n 
=\sum_{n=0}^{\infty}\mu(K_n\cap f^{-1}[F]) 
=\mu f^{-1}[F] 
=\nu F$} 
      
\noindent for every $F\in\Tau$, so $\sum_{n=0}^{\infty}g_n(y)=1$ for 
$\nu$-almost every $y$.   Reducing the $g_n$ further on a set of measure 
zero, if need be, we may suppose that $\sum_{n=0}^{\infty}g_n(y)\le 1$ 
for 
every $y$. 
      
\medskip 
      
\quad{\bf (iv)} For each $n\in\Bbb N$, let $\tilde\lambda_n$ be the 
subspace measure on $f[K_n]$ induced by $\lambda_n$, and $\tilde\mu_n$ 
the subspace measure on $K_n$ induced by $\mu$.   Then $f\restr K_n$ is 
\imp\ for $\tilde\mu_n$ and $\tilde\lambda_n$.   Also, $\tilde\lambda_n$ 
has an almost strong lifting.   \Prf\ If $K_n=\emptyset$, this is 
trivial.   Otherwise, $\nu f[K_n]\ge\mu K_n>0$, so the subspace measure 
$\tilde\nu_n$ induced by $\nu$ on $f[K_n]$ has an almost strong lifting, 
by 453E.   But $\tilde\nu_n$ and $\tilde\lambda_n$ have the same domain 
$\Tau\cap\Cal P(f[K_n])$ and the same null ideal, because 
$K_n\in\Cal K$;  so an almost strong lifting for $\tilde\nu_n$ is an 
almost strong lifting for $\tilde\lambda_n$.\ \Qed 
      
By (a) above, we can find a disintegration 
$\family{y}{f[K_n]}{\mu_{ny}}$ of $\tilde\mu_n$ over $\tilde\lambda_n$ 
such that every $\mu_{ny}$ is a Radon measure on $K_n$, 
$\mu_{ny}K_n\le 1$ for every $y$ and 
      
\Centerline{$\mu_{ny}\{x:x\in K_n,\,f(x)=y\}=1$} 
      
\noindent for $\tilde\lambda_n$-almost every $y\in f[K_n]$, that is, for 
$\nu$-almost every $y\in f[K_n]$.   For $y\in Y\setminus f[K_n]$, let 
$\mu_{ny}$ be the zero measure on $K_n$. 
      
\wheader{453K}{6}{2}{2}{48pt} 
      
\quad{\bf (v)} Now, for $y\in Y$, set 
      
\Centerline{$\mu_yE=\sum_{n=0}^{\infty}g_n(y)\mu_{ny}(E\cap K_n)$} 
      
\noindent for all those $E\subseteq X$ such that the sum is defined. 
Then $\mu_y$ is a Radon measure and $\mu_yX\le 1$.   \Prf\ Because every 
$\mu_{ny}$ is a complete measure, so is $\mu_y$.   We have 
      
\Centerline{$\mu_yX=\sum_{n=0}^{\infty}g_n(y)\mu_{ny}K_n 
\le\sum_{n=0}^{\infty}g_n(y)\le 1$} 
      
\noindent by the choice of the $g_n$.   If $G\subseteq X$ is open then 
$\mu_{ny}$ measures $G\cap K_n$ for every $n$, so $\mu_y$ measures $G$; 
accordingly $\mu_y$ measures every compact set.   If $\mu_yE>0$, there 
is some $n\in\Bbb N$ such that $g_n(y)>0$ and $\mu_{ny}(E\cap K_n)>0$; 
now there is a compact set $K\subseteq E\cap K_n$ such that 
$\mu_{ny}K>0$, in which case $\mu_yK>0$.   By 412B, $\mu_y$ is tight, 
and is a Radon measure.\ \Qed 
      
\medskip 
      
\quad{\bf (vi)} $\family{y}{Y}{\mu_y}$ is a disintegration of $\mu$ over 
$\nu$.   \Prf\ If $E\in\Sigma$ then 
      
$$\eqalignno{\mu E 
&=\sum_{n=0}^{\infty}\mu(E\cap K_n)\cr 
\displaycause{by the choice of the $K_n$ in (ii) above} 
&=\sum_{n=0}^{\infty}\tilde\mu_n(E\cap K_n) 
=\sum_{n=0}^{\infty} 
  \int_{f[K_n]}\mu_{ny}(E\cap K_n)\tilde\lambda_n(dy)\cr 
\noalign{\noindent (because $\family{y}{f[K_n]}{\mu_{ny}}$ is a 
disintegration of $\tilde\mu_n$ over $\tilde\lambda_n$)} 
&=\sum_{n=0}^{\infty} 
  \int_{f[K_n]}\mu_{ny}(E\cap K_n)\lambda_n(dy) 
=\sum_{n=0}^{\infty}\int\mu_{ny}(E\cap K_n)\lambda_n(dy)\cr 
\noalign{\noindent (because $\lambda_n(Y\setminus f[K_n])=0$)} 
&=\sum_{n=0}^{\infty}\int g_n(y)\mu_{ny}(E\cap K_n)\nu(dy)\cr 
\noalign{\noindent (235A)} 
&=\int\sum_{n=0}^{\infty}g_n(y)\mu_{ny}(E\cap K_n)\nu(dy) 
=\int\mu_yE\,\nu(dy).\text{ \Qed}\cr}$$ 
      
\woddheader{453K}{4}{2}{2}{40pt}
      
\quad{\bf (vii)} It follows that
$\mu_yf^{-1}[\{y\}]=1$ for almost every $y$.   \Prf\ 
      
$$\eqalign{\{y:\mu_yf^{-1}[\{y\}]\ne 1\} 
&\subseteq\{y:\mu_yX\ne 1\} 
  \cup\{y:\mu^*_y(X\setminus f^{-1}[\{y\}])>0\}\cr 
&\subseteq\{y:\mu_yX\ne 1\} 
  \cup\bigcup_{n\in\Bbb N} 
  \{y:y\in f[K_n],\,\mu_{ny}(K_n\setminus f^{-1}[\{y\}])>0\}\cr}$$ 
      
\noindent is negligible.\ \Qed 
      
\medskip 
      
{\bf (c)} Now let us turn to the general case.   This proceeds just as 
in 452O.   Let $\familyiI{Y_i}$ be a decomposition of $Y$.   For each 
$i\in I$, take $X_i$, $\lambda_i$ and $\nu_i$ as in the proof of 452O. 
Note that $\lambda_i$ and $\nu_i$ are Radon measures, so that we can 
apply (b) above to find a disintegration $\family{y}{Y_i}{\tilde\mu_y}$ 
of $\lambda_i$ over $\nu_i$ such that every $\tilde\mu_y$ is a Radon 
measure and $\tilde\mu_yX_i=\tilde\mu_yf^{-1}[\{y\}]=1$ for 
$\nu_i$-almost every $y\in Y_i$.   Just as in 452O, we can set 
      
\Centerline{$\mu_yE=\tilde\mu_y(E\cap X_i)$} 
      
\noindent whenever $y\in Y_i$ and $\mu_y$ measures $E\cap X_i$, to 
obtain a disintegration $\family{y}{Y}{\mu_y}$ of $\mu$ over $\nu$ in 
which every $\mu_y$ is a Radon measure and $\mu_yX=1$ for almost every 
$y$;  and this time 
      
\Centerline{$\{y:y\in Y,\,\mu_yf^{-1}[\{y\}]\ne 1\} 
=\bigcup_{i\in I}\{y:y\in Y_i,\,\tilde\mu_yf^{-1}[\{y\}]\ne 1\}$} 
      
\noindent is negligible.   So we have a disintegration of the required 
type. 
}%end of proof of 453K 
      
\cmmnt{ 
\leader{453L}{Remark} If $f$ is surjective, we can arrange that 
every $\mu_y$ is a Radon probability measure for which 
$X_y=f^{-1}[\{y\}]$ is $\mu_y$-conegligible, just by changing some of the 
$\mu_y$ to Dirac measures.   If $f$ is not surjective, then we can still 
(if $X$ itself is not empty) arrange that every $\mu_y$ is a Radon 
probability measure;  but it might be more appropriate to make some of 
the $\mu_y$ the zero measure, so that $X_y$ is always 
$\mu_y$-conegligible. 
      
I have continued to express this theorem in terms of measures $\mu_y$ on 
the whole space $X$.   Of course, if we take it that $X_y$ 
is to be $\mu_y$-conegligible for every $y$, it will sometimes be easier to 
think of $\mu_y$ as a measure on $X_y$;  this is very much what we do in 
the case of Fubini's theorem, where all the $X_y$ are, in effect, the 
same. 
}%end of comment 
      
\leader{453M}{Strong liftings and Stone spaces} Let  
$(X,\frak T,\Sigma,\mu)$ be a quasi-Radon measure space, and  
$(Z,\frak S,\Tau,\nu)$ the Stone space of the measure algebra  
$(\frak A,\bar\mu)$ 
of $\mu$.   For $E\in\Sigma$ let $E^*\subseteq Z$ be the open-and-closed 
set corresponding to the equivalence class $E^{\ssbullet}\in\frak A$. 
Let $R$ be the relation 
      
\Centerline{$\bigcap_{F\subseteq X\text{ is closed}}
\{(z,x):z\in Z\setminus F^*
  \text{ or }x\in F\}\subseteq Z\times X$\dvro{.}{}} 
      
\noindent\cmmnt{(415Q).  }For every lifting $\phi:\Sigma\to\Sigma$ we 
have a unique function $g_{\phi}:X\to Z$ such that 
$\phi E=g_{\phi}^{-1}[E^*]$ for every $E\in\Sigma$\cmmnt{ (see 341P)}. 
\cmmnt{Now we have the following easy facts.} 
      
\spheader 453Ma $\phi$ is strong iff $(g_{\phi}(x),x)\in R$ for every 
$x\in X$.   \prooflet{\Prf\ 
      
$$\eqalign{(g_{\phi}(x),x)\in R&\text{ for every }x\in X\cr 
&\iff x\in F\text{ whenever }F\text{ is closed and }g_{\phi}(x) 
  \in F^*\cr 
&\iff F\subseteq g_{\phi}^{-1}[F^*] 
  \text{ for every closed set }F\subseteq X\cr 
&\iff F\subseteq \phi F\text{ for every closed set }F\subseteq X\cr 
&\iff \phi\text{ is strong.\ \Qed}\cr}$$ 
}%end of prooflet 
      
\spheader 453Mb If $\frak T$ is Hausdorff, so that $R$ is the graph of a 
function $f$\cmmnt{ (415Ra)}, then $\phi$ is strong iff 
$fg_{\phi}(x)=x$ for every $x\in X$.   \prooflet{(For 
$(g_{\phi}(x),x)\in R$ iff $fg_{\phi}(x)=x$.)} 
      
\leader{453N}{Losert's example}\cmmnt{ ({\smc Losert 79})} There is a 
compact Hausdorff space with a strictly positive 
completion regular Radon probability measure which has no strong 
lifting. 
      
\proof{{\bf (a)} Let $\nu$ be the usual measure on $\{0,1\}^{\Bbb N}=Y$. 
Let $M\subseteq Y$ be a closed nowhere dense set such that $\nu M>0$ 
(cf.\ 419B), and $\nu_1$ a Radon probability measure on $Y$ such that 
$\nu_1M=1$ (e.g., a Dirac measure concentrated at some point of $M$). 
      
Let $I$ be any set of cardinal at least $\omega_2$ such that 
$I\cap(I\times I)=\emptyset$.   Let $\lambda$ be the product measure on 
$Y^I$, giving each factor the measure $\nu$;  of course $\lambda$ can be 
identified with the usual measure on 
$\{0,1\}^{\Bbb N\times I}$ (254N).   Note that $\lambda$ and $\nu$ are 
both strictly positive.   For $i\in I$ write 
$M_i=\{z:z\in Y^I,\,z(i)\in M\}$;  then $M_i$ is closed in $Y^I$. 
      
Set $A=\{(i,j):i,\,j\in I,\,i\ne j\}$.   For $z\in Y^I$ and $(i,j)\in A$ 
let $\nu^{(z)}_{ij}$ be the Radon probability measure on $Y$ given by 
setting 
      
$$\eqalign{\nu^{(z)}_{ij} 
&=\nu_1\text{ if }z\in M_i\cap M_j,\cr 
&=\nu\text{ otherwise}.\cr}$$ 
      
\noindent Now, for $z\in Y^I$, let $\lambda_z$ be the Radon product 
measure of $\family{(i,j)}{A}{\nu^{(z)}_{ij}}$ on $Y^A$. 
      
\medskip 
      
{\bf (b)} Let $\Cal U$ be the family of sets $U\subseteq Y^A$ of the 
form $\{u:u(i,j)\in U_{ij}$ for $(i,j)\in B\}$, where $B\subseteq A$ is 
finite and $U_{ij}\subseteq Y$ is open-and-closed for every 
$(i,j)\in B$.   Then the function $z\mapsto\lambda_zU:Y^I\to[0,1]$ is 
Borel measurable for every $U\in\Cal U$.   \Prf\ Express $U$ in the 
given form.   For $C\subseteq B$ set 
      
\Centerline{$E_C=\{z:z\in Y^I$, 
$C=\{(i,j):(i,j)\in B,\,z\in M_i\cap M_j\}\}$,} 
      
\noindent so that $\langle E_C\rangle_{C\subseteq B}$ is a partition of 
$Y^I$ into Borel sets.   For any $C\subseteq B$, 
      
$$\lambda_zU 
=\prod_{(i,j)\in B}\nu^{(z)}_{ij}(U_{ij}) 
=\prod_{(i,j)\in C}\nu_1U_{ij} 
  \cdot\prod_{(i,j)\in B\setminus C}\nu U_{ij}$$ 
      
\noindent is constant for $z\in E_C$.\ \Qed 
      
\medskip 
      
{\bf (c)} There is a Radon measure $\mu$ on $X=Y^I\times Y^A$ 
specified by the formula 
      
\Centerline{$\mu E=\int\lambda_zE[\{z\}]\lambda(dz)$} 
      
\noindent for every Baire set $E\subseteq X$.   \Prf\ Let $\Cal E$ be 
the class of those sets $E\subseteq X$ such that 
$\int\lambda_zE[\{z\}]\lambda(dz)$ is defined.   Then $\Cal E$ is closed 
under monotone limits of sequences, and $E\setminus E'\in\Cal E$ 
whenever $E$, $E'\in\Cal E$ and $E'\subseteq E$;  also $\Cal E$ contains 
all the basic open-and-closed sets in $X$ of the form 
$V\times U$, where $V\subseteq Y^I$ is open-and-closed and $U\in\Cal U$. 
By the Monotone Class Theorem (136B), $\Cal E$ includes the 
$\sigma$-algebra generated by such sets, which is the Baire 
$\sigma$-algebra $\CalBa$ of $X$ (4A3Of).   Of course 
$E\mapsto\int\lambda_zE[\{z\}]\lambda(dz)$ is countably additive on 
$\CalBa$, so is a Baire measure on $X$, and has a unique extension to 
a Radon measure, by 432F.\ \Qed 
      
$\mu$ is strictly positive.  \Prf\ Let $W\subseteq X$ be any non-empty 
open set.   Then it includes an open set of the form $V\times U$ where 
$V=\{z:z\in Y^I,\,z(i)\in V_i$ for every $i\in J\}$, 
$U=\{u:u\in Y^A,\,u(j,k)\in U_{jk}$ for every $(j,k)\in B\}$, 
$J\subseteq I$ and 
$B\subseteq A$ are finite sets, and $V_i$, $U_{jk}\subseteq Y$ are 
non-empty open sets for every $i\in J$ and $(j,k)\in B$.   Now $\nu$ is 
strictly positive, so $\lambda V'>0$, where 
      
\Centerline{$V'=\{z:z\in V,\,z\notin M_j$ whenever $(j,k)\in B\}$.} 
      
\noindent (This is where we need to know that the $M_j$ are nowhere 
dense.)   But if $z\in V'$ then $\nu_{jk}^{(z)}=\nu$ for every 
$(j,k)\in B$, so 
      
\Centerline{$\lambda_zU=\prod_{(j,k)\in B}\nu U_{jk}>0$.} 
      
\noindent Accordingly 
      
\Centerline{$\mu W\ge\int_{V'}\lambda_zU\lambda(dz)>0$.} 
      
\noindent As $W$ is arbitrary, $\mu$ is strictly positive.\ \Qed 
      
Write $\Sigma$ for the domain of $\mu$. 
      
\medskip 
      
{\bf (d)} Fix on a self-supporting compact set $K\subseteq X$.   I seek 
to show that, regarded as a subset of $Y^{I\cup A}$, $K$ is determined 
by coordinates in some countable set. 
      
\medskip 
      
\quad{\bf (i)} There is a zero set $L\supseteq K$ such that 
$\mu L=\mu K$.   \Prf\ Let $\sequencen{K_n}$ be a sequence of compact 
subsets of 
$X\setminus K$ such that $\lim_{n\to\infty}\mu K_n=\mu(X\setminus K)$. 
For each $n\in\Bbb N$ there is a continuous function $f_n:X\to[0,1]$ 
which is zero on $K$ and $1$ on $K_n$;  now $L=\{x:f_n(x)=0$ for every 
$n\in\Bbb N\}$ is a zero set including $K$ and of the same measure as 
$K$.\ \Qed 
      
\medskip 
      
\quad{\bf (ii)} By 4A3Nc, $L$ is determined by coordinates in a 
countable subset of $I\cup A$, that is, there are countable sets 
$J_0\subseteq I$, 
$B_0\subseteq A$ such that whenever $(z,u)\in L$, $(z',u')\in X$, 
$z\restr J_0=z'\restr J_0$ and $u\restr B_0=u'\restr B_0$ we shall have 
$(z',u')\in L$.   Set 
      
\Centerline{$J=J_0\cup\{i:(i,j)\in B_0\}\cup\{j:(i,j)\in B_0\}$, 
\quad$B=A\cap(J\times J)$;} 
      
\noindent  then $J\supseteq J_0$ and $B\supseteq B_0$ are still 
countable, and $L$ is determined by coordinates in $J\cup B$. 
      
\medskip 
      
\quad{\bf (iii)} Take any $(z_0,u_0)\in X\setminus K$.   Because $K$ is 
closed, we can find finite sets $J_1\subseteq I$ and $B_1\subseteq A$, 
open-and-closed sets $V_i\subseteq Y$ for $i\in J_1$, and 
open-and-closed sets $U_{ij}\subseteq Y$ for $(i,j)\in B_1$, such that 
      
\Centerline{$W=\{(z,u):z(i)\in V_i$ for every $i\in J_1$, 
$u(i,j)\in U_{ij}$ for every $(i,j)\in B_1\}$} 
      
\noindent contains $(z_0,u_0)$ and is disjoint from $K$.   Set 
      
$$\eqalign{W_1=\{(z,u):(z,u)\in X,\,&z(i)\in V_i\text{ for every }i\in 
J_1\cap J,\cr 
&u(i,j)\in U_{ij}\text{ for every }(i,j)\in B_1\cap B\},\cr}$$ 
      
\Centerline{$Q=\{z:z\in Y^I,\,\lambda_z((L\cap W_1)[\{z\}])>0\}$,} 
      
\noindent so that $W_1$ is an open-and-closed set in $X$ and $Q$ is a 
Borel set in $Y^I$ ((b) above).   Now $Q$ is determined by coordinates 
in $J$. 
\Prf\ Suppose that $z\in Q$, $z'\in Y^I$ and $z\restr J=z'\restr J$. 
Because both $L$ and $W_1$ are determined by coordinates in $J\cup B$, 
$(L\cap W_1)[\{z\}]=(L\cap W_1)[\{z'\}]=H$ say, and $H$ is determined by 
coordinates in $B$.   At the same time, for any $(i,j)\in B$, 
$M_i\cap M_j$ is determined by coordinates in $J$, so contains $z$ iff 
it contains $z'$, and $\nu^{(z)}_{ij}=\nu^{(z')}_{ij}$.   This means 
that, writing $\lambda'_z$ and $\lambda'_{z'}$ for the products of 
$\family{(i,j)}{B}{\nu^{(z)}_{ij}}$ and 
$\family{(i,j)}{B}{\nu^{(z')}_{ij}}$ on $Y^B$, 
$\lambda'_z=\lambda'_{z'}$.   So 
      
\Centerline{$\lambda_{z'}((L\cap W_1)[\{z'\}]) 
=\lambda_{z'}H=\lambda'_{z'}H'=\lambda'_zH'=\lambda_zH 
=\lambda_z((L\cap W_1)[\{z\}])>0$,} 
      
\noindent where $H'=\{u\restr B:u\in H\}$ (254Ob), and $z'\in Q$.\ \Qed 
      
\medskip 
      
\quad{\bf (iv)} Set 
      
\Centerline{$J_2 
=(\{i:(i,j)\in B_1\setminus B\}\cup\{j:(i,j)\in B_1\setminus B\}) 
  \setminus J$.} 
      
\noindent Then $J_2$ is a finite subset of $I\setminus J$, and 
$B_1\subseteq(J\cup J_2)\times(J\cup J_2)$.   Set 
      
\Centerline{$G=\{z:z\in Y^I,\,z(i)\notin M$ for every $i\in J_2\}$,} 
      
\noindent so that $G$ is a dense open subset of $Y^I$. 
%3{}A3Ea, 3{}A3Ie 
Set 
      
\Centerline{$G_1=\{z:z\in Y^I,\,z(i)\in V_i$ for every $i\in 
J_1\setminus J\}$.} 
      
\noindent Then $G_1$ is a non-empty open set, so $G\cap G_1\ne\emptyset$ 
and $\lambda(G\cap G_1)>0$. 
      
\medskip 
      
\quad{\bf (v)} Set 
      
\Centerline{$U=\{u:u\in Y^A,\,u(i,j)\in U_{ij}$ for every $(i,j)\in 
B_1\setminus B\}$.} 
      
\noindent If $z\in G$, then $z\notin M_i\cap M_j$ whenever $(i,j)\in 
B_1\setminus B$, so $\nu_{ij}^{(z)}=\nu$ for every $(i,j)\in 
B_1\setminus B$, and 
      
\Centerline{$\lambda_zU=\prod_{(i,j)\in B_1\setminus B}\nu U_{ij}>0$.} 
      
\medskip 
      
\quad{\bf (vi)} \Quer\ Suppose, if possible, that $\lambda Q>0$. 
Because $Q$ is determined by coordinates in $J$ and $G\cap G_1$ is 
determined by coordinates in $J_2\cup(J_1\setminus J)$, 
      
\Centerline{$\lambda(Q\cap G\cap G_1) 
=\lambda Q\cdot\lambda(G\cap G_1)>0$.} 
      
\noindent If $z\in Q\cap G\cap G_1$, 
      
$$\eqalignno{\lambda_z((L\cap W)[\{z\}]) 
&=\lambda_z(U\cap(L\cap W_1)[\{z\}])\cr 
\noalign{\noindent (because $W=W_1\cap(Y^I\times U)\cap(G_1\times Y^A)$, 
and $z\in G_1$)} 
&=\lambda_zU\cdot\lambda_z((L\cap W_1)[\{z\}])\cr 
\noalign{\noindent (because $(L\cap W_1)[\{z\}]$ is determined by 
coordinates in $B$, while $U$ is determined by coordinates in 
$B_1\setminus B$, and $\lambda_z$ is a product measure)} 
&>0\cr}$$ 
      
\noindent because $z\in G\cap Q$.   But this means that 
      
\Centerline{$0<\int\lambda_z((L\cap W)[\{z\}])\lambda(dz) 
=\mu(L\cap W)=\mu(K\cap W)=\mu\emptyset$,} 
      
\noindent which is absurd.\ \Bang 
      
Thus $\lambda Q$ must be zero. 
      
\medskip 
      
\quad{\bf (vii)} Consequently 
      
\Centerline{$\mu(K\cap W_1) 
=\mu(L\cap W_1)=\int\lambda_z((L\cap W_1)[\{z\}])\lambda(dz) 
=0$;} 
      
\noindent because $K$ is self-supporting, $K\cap W_1=\emptyset$.   And 
$W_1$ contains $(z_0,u_0)$ and is determined by coordinates in $J\cup 
B$. 
      
\medskip 
      
\quad{\bf (viii)} What this means is that there can be no $(z,u)\in K$ 
such that $z\restr J=z_0\restr J$ and $u\restr B=u_0\restr B$.   At this 
point, recall that $(z_0,u_0)$ was an arbitrary point of $X\setminus K$. 
So what must be happening is that $K$ is determined by coordinates in 
the countable set $J\cup B$.   By 4A3Nc again, in the other 
direction, $K$ is a zero set. 
      
\medskip 
      
{\bf (e)} Part (d) shows that every self-supporting compact subset of 
$X$ is a zero set.   Since $\mu$ is certainly inner regular with respect 
to the self-supporting compact sets, it is inner regular with respect to 
the zero sets, that is, is completion regular. 
      
It follows that whenever $E\in\Sigma$ there is an $E'\subseteq E$, 
determined by coordinates in a countable subset of $I\cup A$, such that 
$E\setminus E'$ is negligible.   (Take $E'$ to be a countable union of 
self-supporting compact sets.) 
      
\medskip 
      
{\bf (f)} \Quer\ Now suppose, if possible, that we could find a strong 
lifting $\phi$ for $\mu$.   For each $i\in I$, take a set 
$E_i\subseteq\phi(M_i\times Y^A)$ such that 
$\mu E_i=\mu\phi(M_i\times Y^A)$ and $E_i$ is determined by coordinates 
in $J_i\cup B_i$, where $J_i\subseteq I$ and $B_i\subseteq A$ are 
countable.   Set 
      
\Centerline{$J^*_i=\{j:(j,k)\in B_i\}\cup\{k:(j,k)\in B_i\}$,} 
      
\noindent so that $J^*_i$ also is countable.   Because 
$\#(I)\ge\omega_2$, there are distinct $i$, $j\in I$ such that 
$i\notin J^*_j$ and $j\notin J^*_i$ (4A1Ea).   So 
$(i,j)\notin B_i\cup B_j$. 
      
Set 
      
\Centerline{$F=\{u:u\in Y^A,\,u(i,j)\in M\}$.} 
      
\noindent Then $\mu((M_i\cap M_j)\times(Y^A\setminus F))=0$.   \Prf\ If 
$z\in M_i\cap M_j$, then 
      
\Centerline{$\lambda_z(Y^A\setminus F)=\nu^{(z)}_{ij}(Y\setminus M)=0$.} 
      
\noindent But $(M_i\cap M_j)\times(Y^A\setminus F)$ is a Baire set, so 
      
$$\eqalign{\mu((M_i\cap M_j)\times(Y^A\setminus F)) 
&=\int\lambda_z((M_i\cap M_j)\times(Y^A\setminus F))[\{z\}]\lambda(dz)\cr 
&=\int_{M_i\cap M_j}\lambda_z(Y^A\setminus F)\lambda(dz) 
=0.\text{ \Qed}\cr}$$ 
      
\noindent Accordingly 
      
$$\eqalignno{E_i\cap E_j 
&\subseteq\phi(M_i\times Y^A)\cap\phi(M_j\times Y^A) 
=\phi((M_i\cap M_j)\times Y^A) 
\subseteq\phi(Y^I\times F)\cr 
\noalign{\noindent (because 
$((M_i\cap M_j)\times Y^A)\setminus(Y^I\times F)$ is negligible)} 
&\subseteq Y^I\times F\cr}$$ 
      
\noindent because $Y^I\times F$ is closed and $\phi$ is supposed to be 
strong.   However, $E_i\cap E_j$ is determined by coordinates in 
$J_i\cup J_j\cup B_i\cup B_j$, while $Y^I\times F$ is determined by 
coordinates in $\{(i,j)\}$, which does not meet $B_i\cup B_j$. 
So either $E_i\cap E_j$ is empty or $F=Y^A$.   But $F\ne Y^A$ because 
$M\ne Y$, while 
      
$$\eqalign{\mu(E_i\cap E_j) 
&=\mu(\phi(M_i\times Y^A)\cap\phi(M_j\times Y^A)) 
=\mu((M_i\cap M_j)\times Y^A)\cr 
&=\lambda(M_i\cap M_j) 
=(\nu M)^2 
>0,\cr}$$ 
      
\noindent so $E_i\cap E_j\ne\emptyset$.\ \Bang 
      
\medskip 
      
{\bf (g)} Thus $\mu$ has no strong lifting, as claimed. 
}%end of proof of 453N (Losert) 
      
\exercises{ 
\leader{453X}{Basic exercises $\pmb{>}$(a)} 
%\spheader 453Xa 
Let $(\frak A,\bar\mu)$ be a measure algebra and 
$(Z,\frak T,\Sigma,\mu)$ its Stone space.   Show that the canonical 
lifting for $\mu$ (341O) is strong. 
%453A 
      
\spheader 453Xb Let $\frak S$ be the right-facing Sorgenfrey topology on 
$\Bbb R$ (415Xc).   Show that there is a lifting for Lebesgue measure on 
$\Bbb R$ which is strong with respect to $\frak S$. 
\Hint{set $\undphi E=\{x:\lim_{\delta\downarrow 0} 
\Bover1{\delta}\mu(E\cap[x,x+\delta])=1\}$, and use 341Jb.} 
%453A 
      
\sqheader 453Xc Let $\mu$ be the usual measure on the split interval 
(343J, 419L).   Show that $\mu$ has a strong lifting. 
%453Xb, 453A 
      
\spheader 453Xd Let $(X,\frak T,\Sigma,\mu)$ be a complete locally 
determined topological measure space such that $\mu$ is inner regular 
with respect to the closed sets, and $\phi:\Sigma\to\Sigma$ a strong 
lifting.   Show that $\mu$ is a quasi-Radon measure with respect to the 
lifting topology $\frak T_l$ (414Q).   Show that if $\frak T$ is regular 
then $\frak T_l\supseteq\frak T$. 
%453A 
      
\spheader 453Xe Let $(X,\frak T,\Sigma,\mu)$ be a topological measure 
space which has an almost strong lifting.   Show that any non-zero
indefinite-integral measure over $\mu$ (234J\formerly{2{}34B})  
has an almost strong lifting. 
%453E 
      
\spheader 453Xf Let $(X,\frak T,\Sigma,\mu)$ be a Radon measure space 
such that $(X,\Sigma,\mu)$ is countably separated and $\mu X>0$;  for 
example, 
$(X,\frak T)$ could be an analytic space (433B).   Show that $\mu$ is 
inner regular with respect to the compact metrizable subsets of $X$, so 
has an almost strong lifting.   \Hint{there is an injective measurable 
$f:X\to\Bbb R$, which must be almost continuous.} 
%453F 
      
\spheader 453Xg Let $(X,\frak T,\Sigma,\mu)$ be a complete locally 
determined topological measure space such that $\mu$ is effectively 
locally finite and inner regular with respect to the closed sets, and 
$\undphi:\Sigma\to\Sigma$ a lower density such that 
$\undphi G\supseteq G$ for every open $G\subseteq X$. 
Show that $\mu$ is a quasi-Radon measure with respect to both $\frak T$ 
and the density topology associated with $\undphi$. 
%453H 
      
\spheader 453Xh Let $(X,\frak T,\Sigma,\mu)$ be a quasi-Radon 
measure space and $\undphi:\Sigma\to\Sigma$ a lower density such that 
$\undphi G\supseteq G$ for every open $G\subseteq X$.   Let 
$\family{x}{X}{G_x}$ be a family of open sets in $X$ such that 
$x\notin\undphi(X\setminus G_x)$ for every $x\in X$.   (i) Show that 
$A\setminus\bigcup_{x\in A}(G_x\cap U_x)$ is negligible whenever 
$A\subseteq X$ and $U_x$ is a neighbourhood of $x$ for every $x\in A$. 
(ii) Let $\frak S$ be the topology on $X$ generated by 
$\frak T\cup\{\{x\}\cup G_x:x\in X\}$.   Show that $\mu$ is quasi-Radon 
with respect to $\frak S$. 
%453Xg 
      
\spheader 453Xi Let $X$ and $Y$ be Hausdorff spaces, and $\mu$ a Radon 
probability measure on $X\times Y$;  set $\pi(x,y)=y$ for $x\in X$, 
$y\in Y$, and let $\nu$ be the image measure $\mu\pi^{-1}$.   Suppose 
that $\nu$ has an almost strong lifting.   Show 
that there is a family $\family{y}{Y}{\mu_y}$ of Radon probability 
measures on $X$ such that $\mu E=\int\mu_y(E^{-1}[\{y\}])\nu(dy)$ for 
every $E\in\dom\mu$. 
%453K 
      
\spheader 453Xj Use 453Xe to simplify part (b) of the proof of 453K. 
%453K 
      
\spheader 453Xk In 453N, show that $\int\lambda_zE[\{z\}]\lambda(dz)$ is 
defined and equal to $\mu E$ whenever $\mu$ measures $E$. 
%453N 
      
      
\leader{453Y}{Further exercises (a)} 
%\spheader 453Ya 
Let $(Y,\frak S,\Tau,\nu)$ be a Radon measure space such 
that $\nu Y>0$ and whenever $(X,\frak T,\Sigma,\mu)$ is a Radon measure 
space and $f:X\to Y$ is an almost continuous \imp\ function, then there 
is a disintegration $\family{y}{Y}{\mu_y}$ of $\mu$ over $\nu$ such that 
$\mu_yf^{-1}[\{y\}]=1$ for almost every $y$.   Show that $\nu$ has an 
almost strong lifting.   \Hint{Start with the case in which $Y$ is 
compact.   Take $f:X\to Y$ to be the function described in 415R, 416V 
and 453Mb.   Set $\undphi E=\{y:\mu_yE^*=\mu_yX=1\}$.} 
%453M 
      
\leader{453Z}{Problems (a)} 
If $(X,\frak T,\Sigma,\mu)$ and $(Y,\frak S,\Tau,\nu)$ are compact Radon 
measure spaces with strong liftings, does their product necessarily have 
a strong lifting?   What if they are both Stone spaces? 
      
\spheader 453Zb If $(X,\frak T,\Sigma,\mu)$ is a Radon probability space 
with countable Maharam type, must it have an almost strong lifting? 
}%end of exercises 
      
\endnotes{\Notesheader{453} 
As I noted in \S452, early theorems on disintegrations concentrated on 
cases in which all the measure spaces involved were `standard' in 
that the measures were defined on standard Borel spaces (\S424), or were 
the completions of such 
measures.   Under these conditions the distinction between 452I and 453K 
becomes blurred;  measures (when completed) have to be Radon measures 
(433Cb), liftings have to be almost strong (453F) and disintegrations 
have to be concentrated on fibers (452Gc). 
Theorem 453K provides disintegrations concentrated on fibers without any 
limitation on the size of the spaces involved, though making strong 
topological assumptions. 
      
The strength of 453K derives from the remarkable variety of the (Radon) 
measure spaces which have strong liftings, as in 453F, 453G, 453I and 
453J.   For some ten years there were hopes that every strictly positive
Radon measure had a strong lifting, which were finally dashed by 
{\smc Losert 79};  I give a version of the example in 453N.   This 
is a special construction, and it remains unclear whether some much more 
direct approach might yield another example (453Za).   I should perhaps 
remark straight away that if the continuum hypothesis is true, then any 
strictly positive Radon measure with Maharam type at most $\omega_1$ 
has a strong lifting (see 535I in Volume 5).   In 
particular, subject to the continuum hypothesis, $Z\times Z$ has a 
strong lifting, where $Z$ is the Stone space of the Lebesgue measure 
algebra, and we have a positive answer to 453Zb. 
}%end of notes 
      
\discrversionA{ 
\bigskip 
      
{\bf question} If $\mu$ is a Radon probability measure, does it have a 
lifting $\phi$ such that $\mu$ is inner regular with respect to 
$\{\phi E:E\in\Sigma\}$? 
      
If there is such a lifting, then there are lots of closed $F$ such that  
$\phi F\subseteq F$.   But it is not clear that enough of them are 
self-supporting.   What about Losert's example? 
}{} 
      
\discrpage 
      
