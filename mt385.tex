\frfilename{mt385.tex}
\versiondate{21.10.03}
\copyrightdate{1997}
     
\def\AmuA{\Aut_{\bar\mu}\frak A}

\def\chaptername{Automorphisms}
\def\sectionname{Entropy}
     
\newsection{385}
     
Perhaps the most glaring problem associated with the theory of
measure-preserving homomorphisms and automorphisms is the fact that we
have no generally effective method of determining when two homomorphisms
are the same, in the sense that two structures $(\frak A,\bar\mu,\pi)$
and $(\frak B,\bar\nu,\phi)$ are isomorphic, where $(\frak A,\bar\mu)$
and $(\frak B,\bar\nu)$ are measure algebras and $\pi:\frak A\to\frak
A$, $\phi:\frak B\to\frak B$ are Boolean homomorphisms.   Of course the
first part of the problem is to decide whether $(\frak A,\bar\mu)$ and
$(\frak B,\bar\nu)$ are isomorphic;  but this is solved (at least for
localizable algebras) by Maharam's theorem\cmmnt{ (see 332J)}.   The difficulty
lies in the homomorphisms.   Even when we know that $(\frak A,\bar\mu)$
and $(\frak B,\bar\nu)$ are both isomorphic to the Lebesgue measure
algebra, the extraordinary variety of constructions of
homomorphisms -- corresponding in part to the variety of
measure spaces with such measure
algebras, each with its own natural \imp\ functions -- means that the
question of which are isomorphic to each other is continually being
raised.   In this section I give the most elementary ideas associated
with the concept of `entropy', up to the Kolmogorov-Sina\v\i\ theorem
(385P).   This is an invariant which can be attached to any
measure-preserving homomorphism on a probability algebra, and therefore
provides a useful method for distinguishing non-isomorphic
homomorphisms.
     
The main work of the section deals with homomorphisms on measure
algebras, but as many of the most important ones arise from \imp\
functions on measure spaces.   I comment on the extra problems arising in the isomorphism problem for such functions (385T-385V).   I should
remark that some of the lemmas will be repeated in stronger forms in the
next section.
     
\leader{385A}{Notation} %(2002) 3{}84A
Throughout this section, I will use the letter
$q$ to denote the function from $\coint{0,\infty}$ to $\Bbb R$
defined by saying that $q(t)=-t\ln t=t\ln\bover1t$ if $t>0$, $q(0)=0$.
     
\def\Caption{The function $q$}
\picture{mt385a}{100pt}
     
\noindent We shall need the following straightforward facts concerning
$q$.
     
\spheader 385Aa $q$ is continuous on $\coint{0,\infty}$ and
differentiable on $\ooint{0,\infty}$;  $q'(t)=-1-\ln t$ and
$q''(t)=-\bover1t$ for $t>0$.    \prooflet{Because $q''$ is negative,}
$q$ is concave\cmmnt{, that is, $-q$ is convex}.   $q$ has a unique
maximum at $(\bover1e,\bover1e)$.
     
\spheader 385Ab \prooflet{If $s\ge 0$ and $t>0$ then $q'(s+t)\le q'(t)$;
consequently}
     
\Centerline{$q(s+t)\cmmnt{\mskip5mu
=q(s)+\biggerint_0^tq'(s+\tau)d\tau}\le q(s)+q(t)$}
     
\noindent for $s$, $t\ge 0$.    \cmmnt{It follows that
$q(\sum_{i=0}^ns_i)\le\sum_{i=0}^nq(s_i)$ for all $s_0,\ldots,s_n\ge 0$
and (because $q$ is continuous)}
$q(\sum_{i=0}^{\infty}s_i)\le\sum_{i=0}^{\infty}q(s_i)$ for every
non-negative summable series $\sequence{i}{s_i}$.
     
\spheader 385Ac If $s$, $t\ge 0$ then $q(st)=sq(t)+tq(s)$;  more
generally, if $n\ge 1$ and $s_i\ge 0$ for $i\le n$ then
     
\Centerline{$q(\prod_{i=0}^ns_i)=\sum_{j=0}^nq(s_j)\prod_{i\ne j}s_i$.}
     
\spheader 385Ad The function $t\mapsto q(t)+q(1-t)$ has a unique maximum
at $(\bover12,\ln 2)$.
\prooflet{($\bover{d}{dt}(q(t)+q(1-t))=\ln\bover{1-t}{t}$.)}
\cmmnt{It
follows that for every $\epsilon>0$ there is a $\delta>0$ such that
$|t-\bover12|\le\epsilon$ whenever $q(t)+q(1-t)\ge\ln 2-\delta$.}
     
\spheader 385Ae If $0\le t\le\bover12$, then $q(1-t)\le q(t)$.
\prooflet{\Prf\ Set $f(t)=q(t)-q(1-t)$.   Then
     
\Centerline{$f''(t)=-\Bover1t+\Bover1{1-t}=\Bover{2t-1}{t(1-t)}\le 0$}
     
\noindent for $0<t\le\bover12$, while $f(0)=f(\bover12)=0$, so
$f(t)\ge 0$ for $0\le t\le\bover12$.\ \Qed}
     
\spheader 385Af(i) If $\frak A$ is a Dedekind $\sigma$-complete Boolean
algebra, I will write $\bar q$ for the function from $L^0(\frak A)^+$
to $L^0(\frak A)$ defined from $q$\cmmnt{ (364H)}.   Note that
\cmmnt{because $0\le q(t)\le 1$ for $t\in [0,1]$,}
$0\le\bar q(u)\le\chi 1$ if $0\le u\le\chi 1$.
     
\quad(ii)\cmmnt{ By (b),}
$\bar q(u+v)\le\bar q(u)+\bar q(v)$ for all $u$, $v\ge 0$ in
$L^0(\frak A)$.   \prooflet{(Represent $\frak A$ as the measure algebra
of a measure space, so that $\bar q(f^{\ssbullet})=(qf)^{\ssbullet}$, as
in 364Ib.)}
     
\quad(iii) Similarly, if $u$, $v\in L^0(\frak A)^+$, then
$\bar q(u\times v)=u\times\bar q(v)+v\times\bar q(u)$.
     
\leader{385B}{Lemma} %(2002) 3{}84B
Let $(\frak A,\bar\mu)$ be a probability algebra,
$\frak B$ a closed subalgebra of $\frak A$, and
$P:L^1(\frak A,\bar\mu)\to L^1(\frak A,\bar\mu)$ the corresponding
conditional expectation operator\cmmnt{ (365R)}.   Then
$\int\bar q(u)\le q(\int u)$ and $P(\bar q(u))\le\bar q(Pu)$ for every
$u\in L^{\infty}(\frak A)^+$.
     
\proof{ Apply the remarks in 365Rb to $-q$.
($\bar q(u)\in L^{\infty}\subseteq L^1$ for every $u\in(L^{\infty})^+$
because $q$ is
bounded on every bounded interval in $\coint{0,\infty}$.)
}%end of proof of 385B
     
\leader{385C}{Definition} %(2002) 3{}84C
Let $(\frak A,\bar\mu)$ be a probability algebra.   If $A$ is a
partition of unity in $\frak A$, its {\bf entropy} is
$H(A)=\sum_{a\in A}q(\bar\mu a)$\cmmnt{, where $q$ is the function
defined in 385A}.
     
\cmmnt{\medskip
     
\noindent{\bf Remarks (a)} In the definition of `partition of unity'
(311Gc) I allowed $0$ to belong to the family.   In the present context
this is a mild irritant, and when convenient I shall remove $0$ from the
partitions of unity considered here (as in 385F below).   But
because $q(0)=0$, it makes no
difference;   $H(A)=H(A\setminus\{0\})$ whenever $A$ is a partition of
unity.   So if you wish you can read `partition of unity' in this
section to mean `partition of unity not containing $0$', if you are
willing to make an occasional amendment in a formula.   In important
cases, in fact, $A$ is of the form $\{a_i:i\in I\}$ or $\{a_i:i\in
I\}\setminus\{0\}$, where $\langle
a_i\rangle_{i\in I}$ is an indexed partition of unity, with $a_i\Bcap
a_j=0$ for $i\ne j$, but no restriction in the number of $i$ with
$a_i=0$;  in this case, we still have $H(A)=\sum_{i\in I}q(\bar\mu
a_i)$.
     
\medskip
     
{\bf (b)} Many authors prefer to use $\log_2$ in place of $\ln$.   This
makes sense in terms of one of the intuitive approaches to entropy as
the `information' associated with a partition.   See {\smc Petersen
83}, \S5.1.
}%end of comment
     
\leader{385D}{Definition} %(2002) 3{}84D
Let $(\frak A,\bar\mu)$ be a probability
algebra, $\frak B$ a closed subalgebra of $\frak A$ and $A$ a partition
of unity in $\frak A$.   Let
$P:L^1(\frak A,\bar\mu)\to L^1(\frak A,\bar\mu)$ be the conditional 
expectation operator associated with
$\frak B$.   Then the {\bf conditional entropy of $A$ on $\frak B$} is
     
\Centerline{$H(A|\frak B)=\sum_{a\in A}$$\textfont3=\twelveex\int\bar
q(P\chi a)$\dvro{.}{,}}
     
\noindent\cmmnt{where $\bar q$ is defined as in 385Af.}
     
\leader{385E}{Elementary remarks (a)} %(2002) 3{}84E
In the formula
     
\Centerline{$\sum_{a\in A}$$\textfont3=\twelveex \int \bar q(P\chi a)$,}
     
\noindent\cmmnt{we have $0\le P(\chi a)\le\chi 1$ for every $a$, so
$\bar q(P\chi a)\ge 0$ and }every term in the sum is non-negative;
accordingly $H(A|\frak B)$ is well-defined in $[0,\infty]$.
     
\spheader 385Eb $H(A)=H(A|\{0,1\})$\prooflet{, since if
$\frak B=\{0,1\}$ then $P(\chi a)=\bar\mu a\chi 1$, so that
$\int\bar q(P\chi a)= q(\bar\mu a)$}.   If
$A\subseteq\frak B$, $H(A|\frak B)=0$\prooflet{, since
$P(\chi a)=\chi a$, $\bar q(P\chi a)=0$ for every $a$}.
     
\leader{385F}{Definition} %(2002) 3{}84F
If $\frak A$ is a Boolean algebra and $A$,
$B\subseteq\frak A$ are partitions of unity, I write $A\vee B$ for the
partition of unity $\{a\Bcap b:a\in A,\,b\in B\}\setminus\{0\}$.
\cmmnt{(See 385Xq.)}
     
\leader{385G}{Lemma} %(2002) 3{}84G
Let $(\frak A,\bar\mu)$ be a probability algebra
and $\frak B$ a closed subalgebra.   Let $A\subseteq\frak A$ be a
partition of unity.
     
(a) If $B$ is another partition of unity in $\frak A$, then
     
\Centerline{$H(A|\frak B)\le H(A\vee B|\frak B)
\le H(A|\frak B)+H(B|\frak B)$.}
     
(b) If $\frak B$ is purely atomic and $D$ is the set of its atoms, then
$H(A\vee D)=H(D)+H(A|\frak B)$.
     
(c) If $\frak C\subseteq\frak B$ is a smaller closed subalgebra of
$\frak A$, then $H(A|\frak C)\ge H(A|\frak B)$.   In particular,
$H(A)\ge H(A|\frak B)$.
     
(d) Suppose that $\sequencen{\frak B_n}$ is a non-decreasing sequence of
closed subalgebras of $\frak A$ such that
$\frak B=\overline{\bigcup_{n\in\Bbb N}\frak B_n}$.   If $H(A)<\infty$
then
     
\Centerline{$H(A|\frak B)=\lim_{n\to\infty}H(A|\frak B_n)$.}
     
\noindent In particular, if $A\subseteq\frak B$ then
$\lim_{n\to\infty}H(A|\frak B_n)=0$.
     
\proof{ Write $P$ for the conditional expectation operator on 
$L^1(\frak A,\bar\mu)$ associated with $\frak B$.
     
\medskip
     
{\bf (a)(i)} If $B$ is infinite, enumerate it as $\sequence{j}{b_j}$;
if it is finite, enumerate it as $\langle b_j\rangle_{j\le n}$ and set
$b_j=0$ for $j>n$.   For any $a\in A$,
     
\Centerline{$\chi a=\sum_{j=0}^{\infty}\chi(a\Bcap b_j)$,
\quad$P(\chi a)=\sum_{j=0}^{\infty}P\chi(a\Bcap b_j)$,}
     
$$\eqalign{\bar q(P\chi a)
&=\lim_{n\to\infty}\bar q(\sum_{j=0}^nP\chi(a\Bcap b_j))\cr
&\le\lim_{n\to\infty}\sum_{j=0}^n\bar q(P\chi(a\Bcap b_j))
=\sum_{j=0}^{\infty}\bar q(P\chi(a\Bcap b_j))\cr}$$
     
\noindent where all the infinite sums are to be regarded as
order*-limits of the corresponding finite sums (see \S367), and the
middle inequality is a consequence of 385A(f-ii).   Accordingly
     
$$\eqalign{H(A\vee B|\frak B)
&=\sum_{a\in A,b\in B,a\Bcap b\ne 0}\int\bar q(P\chi(a\Bcap b))\cr
&=\sum_{a\in A}\sum_{j=0}^{\infty}\int\bar q(P\chi(a\Bcap b_j))
\ge\sum_{a\in A}\int\bar q(P\chi a)
=H(A|\frak B).\cr}$$
     
\medskip
     
\quad{\bf (ii)} Suppose for the moment that $A$ and $B$ are both finite.
For $a\in\frak A$ set
$u_a=P(\chi a)$.   If $a$, $b\in\frak A$ we have
$0\le u_{a\Bcap b}\le u_b$ in $L^0(\frak B)$, so we may choose
$v_{ab}\in L^0(\frak B)$ such
that $0\le v_{ab}\le\chi 1$ and $u_{a\Bcap b}=v_{ab}\times u_b$.
     
For any $b\in B$, $\sum_{a\in A}u_{a\Bcap b}=u_b$ (because
$\sum_{a\in A}\chi(a\Bcap b)=\chi b$), so
$u_b\times\sum_{a\in A}v_{ab}=u_b$.
Since $\Bvalue{|\bar q(u_b)|>0}\Bsubseteq\Bvalue{u_b>0}$,
$\bar q(u_b)\times\sum_{a\in A}v_{ab}=\bar q(u_b)$.
     
For any $a\in A$,
     
$$\eqalignno{\bar q(u_a)
&=\bar q(\sum_{b\in B}u_{a\Bcap b})
=\bar q(\sum_{b\in B}u_b\times v_{ab})
=\bar q(P(\sum_{b\in B}\chi b\times v_{ab}))\cr
\noalign{\noindent (because $v_{ab}\in L^0(\frak B)$ for every $b$, so
$P(\chi b\times v_{ab})=P(\chi b)\times v_{ab}$)}
&\ge P(\bar q(\sum_{b\in B}\chi b\times v_{ab}))\cr
\displaycause{385B}
&=P(\sum_{b\in B}\chi b\times\bar q(v_{ab}))\cr
\noalign{\noindent (because $B$ is disjoint)}
&=\sum_{b\in B}u_b\times\bar q(v_{ab})\cr}$$
     
\noindent (because $\bar q(v_{ab})\in L^0(\frak B)$ for every $b$).
     
Putting these together,
     
$$\eqalignno{H(A\vee B|\frak B)
&=\sum_{a\in A,b\in B}\int\bar q(u_{a\Bcap b})
=\sum_{a\in A,b\in B}\int\bar q(u_b\times v_{ab})\cr
&=\sum_{a\in A,b\in B}\int u_b\times\bar q(v_{ab})
   +\sum_{a\in A,b\in B}\int v_{ab}\times\bar q(u_b)\cr
\displaycause{385A(f-iii)}
&\le\sum_{a\in A}\int\bar q(u_a)
   +\sum_{b\in B}\int\bar q(u_b)
=H(A|\frak B)+H(B|\frak B).\cr}$$
     
\medskip
     
\quad{\bf (iii)} For general partitions of unity $A$ and $B$, take any
finite set $C\subseteq A\vee B$.   Then
$C\subseteq\{a\Bcap b:a\in A_0,\,b\in B_0\}$ where $A_0\subseteq A$ and
$B_0\subseteq B$ are finite.   Set
     
\Centerline{$A'=A_0\cup\{1\Bsetminus\sup A_0\}$,
\quad$B'=B_0\cup\{1\Bsetminus\sup B_0\}$,}
     
\noindent so that $A'$ and $B'$ are finite partitions of unity and
$C\subseteq A'\vee B'$.   Now
     
$$\eqalignno{\sum_{c\in C}\int\bar q(P\chi c)
&\le\sum_{c\in A'\vee B'}\int\bar q(P\chi c)
=H(A'\vee B'|\frak B)
\le H(A'|\frak B)+H(B'|\frak B)\cr
\displaycause{by (ii)}
&\le H(A'\vee A|\frak B)+H(B'\vee B|\frak B)\cr
\displaycause{by (i)}
&=H(A|\frak B)+H(B|\frak B).\cr}$$
     
\noindent As $C$ is arbitrary,
     
\Centerline{$H(A\vee B|\frak B)
=\sum_{c\in A\vee B}$$\textfont3=\twelveex \int\bar q(P\chi c)
\le H(A|\frak B)+H(B|\frak B)$.}
     
\medskip
     
{\bf (b)} It follows from 385Ab that
$\sum_{d\in D}q(\bar\mu(a\Bcap d))\ge q(\bar\mu a)$ for any $a\in A$.
     
Now, because $\frak B$ is purely atomic and $D$ is its set of atoms,
     
\Centerline{$P(\chi a)
=\sum_{d\in D}\Bover{\bar\mu(a\Bcap d)}{\bar\mu d}\chi d$,
\quad$\bar q(P(\chi a))
=\sum_{d\in D}q(\Bover{\bar\mu(a\Bcap d)}{\bar\mu d})\chi d$}
     
\noindent for every $a\in A$,
     
\Centerline{$H(A|\frak B)
=\sum_{a\in A,d\in D}q(\Bover{\bar\mu(a\Bcap d)}{\bar\mu d})\bar\mu d$.}
     
\noindent Putting these together,
     
$$\eqalignno{H(A\vee D)
&=\sum_{a\in A,d\in D} q(\bar\mu(a\Bcap d))
=\sum_{a\in A,d\in D}q(\Bover{\bar\mu(a\Bcap d)}{\bar\mu d})\bar\mu d
  +\Bover{\bar\mu(a\Bcap d)}{\bar\mu d} q(\bar\mu d)\cr
\displaycause{385Ac}
&=H(A|\frak B)+\sum_{d\in D} q(\bar\mu d)
=H(A|\frak B)+H(D).\cr}$$
     
\medskip
     
{\bf (c)} Write $P_{\frak C}$ for the conditional expectation operator
corresponding to $\frak C$.   If $a\in\frak A$,
     
\Centerline{$\bar q(P_{\frak C}\chi a)
=\bar q(P_{\frak C}P\chi a)
\ge P_{\frak C}\bar q(P\chi a)$}
     
\noindent by 385B.   So
     
\Centerline{$H(A|\frak C)=\sum_{a\in A}$$\textfont3=\twelveex \int\bar
q(P_{\frak C}\chi a)
\ge$$\sum_{a\in A}$$\textfont3=\twelveex \int P_{\frak C}\bar q(P\chi a)
=$$\sum_{a\in A}$$\textfont3=\twelveex \int \bar q(P\chi a)
=H(A|\frak B)$.}
     
Taking $\frak C=\{0,1\}$, we get $H(A)\ge H(A|\frak B)$.
     
\medskip
     
{\bf (d)} Let $P_n$ be the conditional expectation operator
corresponding to $\frak B_n$, for each $n$.   Fix $a\in A$.   Then
$P(\chi a)$ is the order*-limit of $\sequencen{P_n(\chi a)}$, by
L\'evy's martingale theorem (367Jb).
Consequently (because $q$ is continuous)
$\sequencen{\bar q(P_n\chi a)}$ is order*-convergent to
$\bar q(P\chi a)$ for every $a\in A$ (367H).   Also,
because $0\le P_n\chi a\le\chi 1$ for every $n$,
$0\le\bar q(P_n\chi a)\le\bover1e\chi 1$ for every $n$.   By the
Dominated Convergence Theorem (367I),
$\lim_{n\to\infty}\int\bar q(P_n\chi a)=\int\bar q(P\chi a)$.
     
By 385B, we also have
     
\Centerline{$ 0\le\int\bar q(P_n\chi a)\le q(\int
P_n(\chi a))
=q(\int\chi a)=q(\bar\mu a)$}
     
\noindent for every $a\in A$ and $n\in\Bbb N$;  since also
     
\Centerline{$ 0\le\int\bar q(P\chi a)=q(\bar\mu
a)$,}
     
\noindent we have
$|\int\bar q(P_n\chi a)-\int\bar q(P\chi a)|\le q(\bar\mu a)$ for every
$a\in A$, $n\in\Bbb N$.
     
Now we are supposing that $H(A)$ is finite.   Given
$\epsilon>0$, we can find a finite set $I\subseteq A$ such that
$\sum_{a\in A\setminus I}q(\bar\mu a)\le\epsilon$, and an $n_0\in\Bbb N$
such that
     
\Centerline{$\sum_{a\in I}
 $$\textfont3=\twelveex |\int\bar q(P_n\chi a)-\int\bar q(P\chi
a)|\le\epsilon$}
     
\noindent for every $n\ge n_0$;  in which case
     
\Centerline{$\sum_{a\in A\setminus I}
 $$\textfont3=\twelveex |\int\bar q(P_n\chi a)-\int\bar q(P\chi a)|
\le$$\sum_{a\in A\setminus I}q(\bar\mu a)\le\epsilon$}
     
\noindent and $|H(A|\frak B_n)-H(A|\frak B)|\le 2\epsilon$ for every
$n\ge n_0$.   As $\epsilon$ is arbitrary,
\discrcenter{390pt}{$H(A|\frak B)=\lim_{n\to\infty}H(A|\frak B_n)$.}
}%end of proof of 385G
     
\leader{385H}{Corollary} %(2002) 3{}84H
Let $(\frak A,\bar\mu)$ be a probability
algebra and $A$, $B$ two partitions of unity in $\frak A$.   Then
$H(A)\le H(A\vee B)\le H(A)+H(B)$.
     
\proof{ Take $\frak B=\{0,1\}$ in 385Ga.
}%end of proof of 385H
     
\leader{385I}{Lemma} %(2002) 3{}84I
Let $(\frak A,\bar\mu)$ be a probability algebra,
and $\pi:\frak A\to\frak A$ a measure-preserving Boolean homomorphism.
If $A\subseteq \frak A$ is a partition of unity, then $H(\pi[A])=H(A)$.
     
\proof{ $\sum_{a\in A} q(\bar\mu\pi a)=\sum_{a\in A} q(\bar\mu a)$.}
     
\leader{385J}{Lemma} %(2002) 3{}84J
Let $(\frak A,\bar\mu)$ be a measure algebra.   Let
$A$ be the set of its atoms.   Then the following are equiveridical:
     
(i) {\it either} $\frak A$ is not purely atomic {\it or}  $\frak A$ is
purely atomic and $H(A)=\infty$;
     
(ii) there is a partition of unity $B\subseteq\frak A$ such that
$H(B)=\infty$;
     
(iii) for every $\gamma\in\Bbb R$ there is a finite partition of unity
$C\subseteq\frak A$ such that $H(C)\ge\gamma$.
     
\proof{{\bf (i)$\Rightarrow$(ii)} We need examine only the case in which
$\frak A$ is not purely atomic.   Let $a\in\frak A$ be a non-zero
element such that the principal ideal $\frak A_a$ is atomless.   By 331C
we can choose inductively a disjoint sequence $\sequencen{a_n}$ such
that $a_n\Bsubseteq a$ and $\bar\mu a_n=2^{-n-1}\bar\mu a$.   Now, for
each $n\in\Bbb N$, choose a disjoint set $B_n$ such that
     
\Centerline{$\#(B_n)=2^{2^n}$,
\quad$b\Bsubseteq a_n$ and $\bar\mu b=2^{-2^n}\bar\mu a_n$ for each
$b\in B_n$.}
     
\noindent Set
     
\Centerline{$B=\bigcup_{n\in\Bbb N}B_n\cup\{1\Bsetminus a\}$.}
     
\noindent Then $B$ is a partition of unity in $\frak A$ and
     
$$\eqalign{H(B)
&\ge\sum_{n=0}^{\infty}\sum_{b\in B_n}q(\bar\mu B_n)
=\sum_{n=0}^{\infty}2^{2^n}
  q\bigl(\bover{\bar\mu a}{2^{n+1+2^n}}\bigr)\cr
&=\sum_{n=0}^{\infty}\bover{\bar\mu a}{2^{n+1}}
  \ln\bigl(\bover{2^{n+1+2^n}}{\bar\mu a}\bigr)
\ge\sum_{n=0}^{\infty}\bover{\bar\mu a}{2^{n+1}}2^n\ln 2
=\infty.\cr}$$
     
\medskip
     
{\bf (ii)$\Rightarrow$(iii)} Enumerate $B$ as $\sequence{i}{b_i}$.   For
each $n\in\Bbb N$, $C_n=\{b_i:i\le n\}\cup\{1\Bsetminus\sup_{i\le
n}b_i\}$ is a finite partition of unity, and
     
\Centerline{$\lim_{n\to\infty}H(C_n)
\ge\lim_{n\to\infty}\sum_{i=0}^nq(\bar\mu b_i)
=H(B)=\infty$.}
     
\medskip
     
{\bf (iii)$\Rightarrow$(i)} We need only consider the case in which
$\frak A$ is purely atomic.   In this case,
$A\vee C=A$ for every partition of unity $C\subseteq\frak A$, so
$H(C)\le
H(A)$ for every $C$ (385H), and $H(A)$ must be infinite.
}%end of proof of 385J
     
\leader{385K}{Definition} %(2002) 3{}84K
Let $\frak A$ be a Boolean algebra.
If $\pi:\frak A\to\frak A$ is an order-continuous Boolean homomorphism,
$A\subseteq\frak A$ is a partition of unity and $n\ge 1$, write
$D_n(A,\pi)$ for\cmmnt{ the
partition of unity generated by $\{\pi^ia:a\in A,\,0\le i<n\}$, that
is,} $\{\inf_{i<n}\pi^ia_i:a_i\in A$ for every $i<n\}\setminus\{0\}$.
\cmmnt{It will occasionally be convenient to take $D_0(A,\pi)=\{1\}$ (or $\emptyset$ in the trivial case $\frak A=\{0\}$).   Observe that $D_1(A,\pi)=A\setminus\{0\}$ and
     
\Centerline{$D_{n+1}(A,\pi)=D_n(A,\pi)\vee\pi^n[A]=A\vee\pi[D_n(A,\pi)]$
}
     
\noindent for every $n\in\Bbb N$.}
     
\leader{385L}{Lemma} %(2002) 3{}84L
Let $(\frak A,\bar\mu)$ be a probability algebra
and $\pi:\frak A\to\frak A$ a measure-preserving Boolean homomorphism.
Let $A\subseteq\frak A$ be a partition of unity.   Then
\discrcenter{390pt}{$\lim_{n\to\infty}\bover1{n}H(D_n(A,\pi))
=\inf_{n\ge 1}\bover1nH(D_n(A,\pi))$ }is defined in $[0,\infty]$.
     
\proof{{\bf (a)} Set $\alpha_0=0$, $\alpha_n=H(D_{n}(A,\pi))$ for
$n\ge 1$.   Then $\alpha_{m+n}\le \alpha_m+\alpha_n$ for all $m$,
$n\ge 0$.   \Prf\ If $m$, $n\ge 1$,
$D_{m+n}(A,\pi)=D_m(A,\pi)\vee\pi^m[D_n(A,\pi)]$.   So 385Ga tells us
that
     
\Centerline{$H(D_{m+n}(A,\pi))\le H(D_m(A,\pi))+H(\pi^m[D_n(A,\pi)])
= H(D_m(A,\pi))+H(D_n(A,\pi))$}
     
\noindent because $\pi$ is measure-preserving.\ \Qed
     
\medskip
     
{\bf (b)}  If $\alpha_1=\infty$ then of course
$H(D_n(A,\pi))\ge H(A)=\infty$ for every $n\ge 1$, by 385H, so
$\inf_{n\ge 1}\bover1nH(D_n(A,\pi))\penalty-100=\infty
=\lim_{n\to\infty}\bover1nH(D_n(A,\pi))$.   Otherwise,
$\alpha_n\le n\alpha_1$ is finite for every $n$.   Set $\alpha=\inf_{n\ge 1}\bover1n\alpha_n$.   If $\epsilon>0$
there is an $m\ge 1$ such that $\bover1m\alpha_m\le\alpha+\epsilon$.
Set $M=\max_{j<m}\alpha_j$.    Now, for any $n\ge m$, there are
$k\ge 1$, $j<m$ such that $n=km+j$, so that
     
\Centerline{$\alpha_n\le k\alpha_m+\alpha_j$,
\quad$\Bover1n\alpha_n\le\Bover{k}{n}\alpha_m+\Bover{M}n
\le\Bover{1}{m}\alpha_m+\Bover{M}n$.}
     
\noindent Accordingly
$\limsup_{n\to\infty}\bover1n\alpha_n\le\alpha+\epsilon$.   As
$\epsilon$ is arbitrary,
     
\Centerline{$\alpha\le\liminf_{n\to\infty}\Bover1n\alpha_n
\le\limsup_{n\to\infty}\Bover1n\alpha_n\le\alpha$}
     
\noindent and $\lim_{n\to\infty}\bover1n\alpha_n=\alpha$ is defined in
$[0,\infty]$.
}%end of proof of 385L
     
\cmmnt{\medskip
     
\noindent{\bf Remark} See also 385Yb and 386Lc below.}
     
\leader{385M}{Definition} %(2002) 3{}84M
Let $(\frak A,\bar\mu)$ be a probability
algebra, and $\pi:\frak A\to\frak A$ a measure-preserving Boolean
homomorphism.   For any partition of unity $A\subseteq\frak A$, set
     
\Centerline{$h(\pi,A)=\inf_{n\ge 1}\bover1n H(D_n(A,\pi))
=\lim_{n\to\infty}\bover1nH(D_n(A,\pi))$\dvro{.}{}}
     
\noindent\cmmnt{(385L).   }Now the {\bf entropy} of $\pi$ is
     
\Centerline{$h(\pi)=\sup\{h(\pi,A):A\subseteq\frak A$ is a finite
partition of unity$\}$.}
     
\medskip
     
\noindent{\bf Remarks (a)}\cmmnt{ We always have}
     
\Centerline{$h(\pi,A)\le\cmmnt{H(D_1(A,\pi))=}H(A)$.}
     
\spheader 385Mb Observe that if $\pi$ is the identity automorphism
then\cmmnt{ $D_n(A,\pi)=A\setminus\{0\}$ for every $A$ and $n$, so
that} $h(\pi)=0$.
     
\leader{385N}{Lemma} %(2002) 3{}84N
Let $(\frak A,\bar\mu)$ be a probability algebra
and $A$, $B$ two partitions of unity in $\frak A$.   Let
$\pi:\frak A\to\frak A$ be a measure-preserving Boolean homomorphism.   Then $h(\pi,A)\le h(\pi,B)+H(A|\frak B)$, where $\frak B$ is the closed
subalgebra of $\frak A$ generated by $B$.
     
\proof{ We may suppose that $0\notin B$, since removing $0$ from $B$
changes neither $D_n(B,\pi)$ nor $\frak B$.   For each $n\in\Bbb N$,
set $A_n=\pi^n[A]$ and $B_n=\pi^n[B]$.   Let $\frak B_n=\pi^n[\frak B]$ be the closed subalgebra of $\frak A$ generated by $B_n$, and
$\frak B^*_n$
the closed subalgebra of $\frak A$ generated by $D_n(B,\pi)$.   Then
$H(A_n|\frak B_n)=H(A|\frak B)$ for each $n$.   \Prf\ The point is that,
because $\frak B$ is purely atomic and $B$ is its set of atoms,
     
\Centerline{$H(A|\frak B)=\sum_{a\in A,b\in B} q(\Bover{\bar\mu(a\Bcap
b)}{\bar\mu b})\bar\mu b$}
     
\noindent as in the proof of 385Gb.   Similarly,
     
\Centerline{$H(A_n|\frak B_n)=\sum_{a\in A,b\in
B} q(\Bover{\bar\mu(\pi^n a\Bcap \pi^n b)}{\bar\mu(\pi^n b)})\bar\mu
(\pi^n b)=H(A|\frak B)$.  \Qed}
     
Accordingly, for any $n\ge 1$,
     
$$\eqalignno{H(D_n(A,\pi)|\frak B^*_n)
&\le\sum_{i=0}^{n-1}H(A_i|\frak B^*_n)\cr
\noalign{\noindent (by 385Ga)}
&\le\sum_{i=0}^{n-1}H(A_i|\frak B_i)\cr
\noalign{\noindent (by 385Gc)}
&=n H(A|\frak B).\cr}$$
     
\noindent Now
     
$$\eqalignno{h(\pi,A)
&=\lim_{n\to\infty}\Bover1n H(D_n(A,\pi))
\le\limsup_{n\to\infty}\Bover1n H(D_n(A,\pi)\vee D_n(B,\pi))\cr
\displaycause{385Ga}
&\le\limsup_{n\to\infty}\Bover1n H(D_n(B,\pi))
  +\Bover1n H(D_n(A,\pi)|\frak B^*_n)\cr
\noalign{\noindent (385Gb)}
&\le h(\pi,B)+H(A|\frak B).\cr}$$
}%end of proof of 385N
     
\leader{385O}{Lemma} %(2002) 3{}84O
Let $(\frak A,\bar\mu)$ be a probability algebra,
$\pi:\frak A\to\frak A$ a measure-preserving Boolean homomorphism, and
$A\subseteq \frak A$ a partition of unity such that $H(A)<\infty$.
Then $h(\pi,A)\le h(\pi)$.
     
\proof{ If $A$ is finite, this is immediate from the definition of
$h(\pi)$;  so suppose that $A$ is
infinite.   Enumerate $A$ as $\sequence{i}{a_i}$.   For each
$n\in\Bbb N$ let $\frak B_n$ be the subalgebra of $\frak A$ generated by
$a_0,\ldots,a_n$;  set
$\frak B=\overline{\bigcup_{n\in\Bbb N}\frak B_n}$.   Then
$A\subseteq\frak B$, so
     
\Centerline{$\lim_{n\to\infty}H(A|\frak B_n)=H(A\frak B)=0$}
     
\noindent by 385Eb and 385Gd.   Accordingly, using 385N,
     
\Centerline{$h(\pi,A)\le h(\pi,B_n)+H(A|\frak B_n)
\le h(\pi)+H(A|\frak B_n)\to h(\pi)$}
     
\noindent as $n\to\infty$, and $h(\pi,A)\le h(\pi)$.
}%end of proof of 385O
     
\leader{385P}{Theorem} %(2002) 3{}84P
({\smc Kolmogorov 58}, {\smc Sina\v\i\ 59}) Let
$(\frak A,\bar\mu)$ be a probability algebra, and $\pi:\frak A\to\frak
A$ a measure-preserving Boolean homomorphism.
     
(i) Suppose that $A\subseteq\frak A$ is a partition of unity such that
$H(A)<\infty$ and the closed subalgebra of $\frak A$ generated by
$\bigcup_{n\in\Bbb N}\pi^n[A]$ is $\frak A$ itself.   Then
$h(\pi)=h(\pi,A)$.
     
(ii) Suppose that $\pi$ is an automorphism, and that $A\subseteq\frak A$
is a partition of unity such that $H(A)<\infty$ and the closed
subalgebra of $\frak A$ generated by $\bigcup_{n\in\Bbb Z}\pi^n[A]$ is
$\frak A$ itself.   Then $h(\pi)=h(\pi,A)$.
     
\proof{ I take the two arguments together.   In both cases, by 385O, we
have $h(\pi,A)\le h(\pi)$, so I have to show that if $B\subseteq\frak A$
is any finite partition of unity, then $h(\pi,B)\le h(\pi,A)$.   For
(i), let $A_n$ be the partition of unity generated by $\bigcup_{0\le
j<n}\pi^j[A]$;  for (ii), let $A_n$ be the partition of unity generated
by $\bigcup_{-n\le j<n}\pi^j[A]$.   Then $h(\pi,A_n)=h(\pi,A)$ for every
$n$.   \Prf\ In case (i), we have
$D_m(A_n,\pi)=D_{m+n}(A,\pi)$ for every $m$, so that
     
$$\eqalign{\lim_{m\to\infty}\Bover1m H(D_m(A_n,\pi))
&=\lim_{m\to\infty}\Bover1m H(D_{m+n}(A,\pi))\cr
&=\lim_{m\to\infty}\Bover1m H(D_{m}(A,\pi)).\cr}$$
     
\noindent In case (ii), we have
$D_m(A_n,\pi)=\pi^{-n}[D_{m+2n}(A,\pi)]$ for every $m$, so that
     
$$\eqalign{\lim_{m\to\infty}\Bover1m H(D_m(A_n,\pi))
&=\lim_{m\to\infty}\Bover1m H(D_{m+2n}(A,\pi))\cr
&=\lim_{m\to\infty}\Bover1m H(D_{m}(A,\pi)). \text{ \Qed}\cr}$$
     
Let $\frak A_n$ be the purely atomic closed subalgebra of $\frak A$
generated by $A_n$;  our hypothesis is that the closed subalgebra
generated by $\bigcup_{n\in\Bbb N}A_n$ is $\frak A$ itself, that is,
that $\bigcup_{n\in\Bbb N}\frak A_n$ is dense.   But this means that
$\lim_{n\to\infty}H(B|\frak A_n)=0$ (385Gd).   Since
     
\Centerline{$h(\pi,B)\le h(\pi,A_n)+H(B|\frak A_n)=h(\pi,A)
+H(B|\frak A_n)$}
     
\noindent for every $n$ (385N), we have the result.
}%end of proof of 385P
     
\vleader{60pt}{385Q}{Bernoulli shifts} %(2002) 3{}84Q
Let $(\frak A,\bar\mu)$ be a probability
algebra, and $\pi:\frak A\to\frak A$ a measure-preserving Boolean
homomorphism.
     
\spheader 385Qa $\pi$ is a {\bf one-sided Bernoulli shift} if there is a 
closed subalgebra $\frak A_0$ in $\frak A$ such that (i)
$\sequence{k}{\pi^k[\frak A_0]}$ is stochastically
independent\cmmnt{ (that is,
$\bar\mu(\inf_{j\le k}\pi^ja_j)=\prod_{j=0}^k\bar\mu a_j$ for all
$a_0,\ldots,a_k\in\frak A_0$;  see 325L)} (ii) the closed subalgebra of
$\frak A$ generated by $\bigcup_{k\in\Bbb N}\pi^k[\frak A_0]$ is $\frak
A$ itself.   In this case $\frak A_0$ is a {\bf root algebra} for $\pi$.
     
\spheader 385Qb $\pi$ is a {\bf two-sided Bernoulli shift} if it is an
automorphism and there is a closed subalgebra $\frak A_0$ in $\frak A$
such that (i) $\family{k}{\Bbb Z}{\pi^k[\frak A_0]}$ is independent (ii)
the closed subalgebra of $\frak A$ generated by
$\bigcup_{k\in\Bbb Z}\pi^k[\frak A_0]$ is $\frak A$ itself.   In this
case $\frak A_0$ is a {\bf root algebra} for $\pi$.
     
\cmmnt{It is important to be aware that a Bernoulli shift can have
many, and (in the case of a two-sided shift) very different, root
algebras;  this is the subject of \S387 below.}
     
\leader{385R}{Theorem} %(2002) 3{}84R
Let $(\frak A,\bar\mu)$ be a probability algebra
and $\pi:\frak A\to\frak B$ a Bernoulli shift, either one- or two-sided,
with root algebra $\frak A_0$.
     
(i) If $\frak A_0$ is purely atomic, then $h(\pi)=H(A)$, where $A$ is
the set of atoms of $\frak A_0$.
     
(ii) If $\frak A_0$ is not purely atomic, then $h(\pi)=\infty$.
     
\proof{{\bf (a)} The point is that for any partition of unity
$C\subseteq\frak A_0\setminus\{0\}$, $h(\pi,C)=H(C)$.  \Prf\ For any $n\ge 1$,
$D_n(C,\pi)$ is the partition of unity consisting of elements of the
form $\inf_{j<n}\pi^jc_j$, where $c_0,\ldots,c_{n-1}\in C$.   So
     
$$\eqalignno{H(D_n(C,\pi))
&=\sum_{c_0,\ldots,c_{n-1}\in C}
 q(\bar\mu(\inf_{j<n}\pi^jc_j))
=\sum_{c_0,\ldots,c_{n-1}\in C}
 q(\prod_{j=0}^{n-1}\bar\mu c_j))\cr
&=\sum_{c_0,\ldots,c_{n-1}\in C}
 \sum_{j=0}^{n-1}q(\bar\mu c_j)\prod_{i\ne j}\bar\mu c_i\cr
\displaycause{385Ac}
&=\sum_{j=0}^{n-1}\sum_{c\in C} q(\bar\mu c)
=n H(C).\cr}$$
     
\noindent So
\Centerline{$h(\pi,C)=\lim_{n\to\infty}\Bover1nH(D_n(C,\pi))=H(C)$.
\Qed}
     
\medskip
     
{\bf (b)} If $\frak A_0$ is purely atomic and $H(A)<\infty$, the result
can now be read off from 385P, because the closed subalgebra of $\frak
A$
generated by $A$ is $\frak A_0$ and the closed subalgebra of $\frak A$
generated by $\bigcup_{k\in\Bbb N}\pi^k[A]$ or $\bigcup_{k\in\Bbb
Z}\pi^k[A]$ is $\frak A$;  so $h(\pi)=h(\pi,A)=H(A)$.
     
\medskip
     
{\bf (c)} Otherwise, 385J tells us that there are finite partitions of
unity $C\subseteq\frak A_0$ such that $H(C)$ is arbitrarily large.
Since $h(\pi)\ge h(\pi,C)=H(C)$ for any such $C$, by (a) and the
definition of $h(\pi)$, $h(\pi)$ must be infinite, as claimed.
}%end of proof of 385R
     
\leader{385S}{Remarks (a)}\cmmnt{ The %(2002) 3{}84S
standard construction of a
Bernoulli shift is from a product space, as follows.}   If
$(X,\Sigma,\mu_0)$ is any probability space, write $\mu$ for the product
measure on $X^{\Bbb N}$;  let $(\frak A,\bar\mu)$ be the measure algebra
of $\mu$, and $\frak A_0\subseteq\frak A$ the set of equivalence classes
of sets of the form $\{x:x(0)\in E\}$ where $E\in\Sigma$, so that
$(\frak A_0,\bar\mu\restrp\frak A_0)$ can be identified with the measure
algebra of
$\mu_0$.   We have an \imp\ function $f:X^{\Bbb N}\to X^{\Bbb N}$
defined by setting
     
\Centerline{$f(x)(n)=x(n+1)$ for every $x\in X^{\Bbb N}$, $n\in\Bbb N$,}
     
\noindent and $f$ induces\cmmnt{, as usual,} a measure-preserving
homomorphism $\pi:\frak A\to\frak A$.   Now $\pi$ is a one-sided
Bernoulli shift with root algebra $\frak A_0$.   \prooflet{\Prf\ (i)
If $a_0,\ldots,a_k\in\frak A_0$, express each $a_j$ as $\{x:x(0)\in
E_j\}^{\ssbullet}$, where $E_j\in\Sigma$.   Now
     
\Centerline{$\pi^ja_j=\{x:(f^j(x))(0)\in E_j\}^{\ssbullet}
=\{x:x(j)\in E_j\}^{\ssbullet}$}
     
\noindent for each $j$, so
     
\Centerline{$\bar\mu(\inf_{j\le k}\pi^ja_j)
=\mu(\bigcap_{j\le k}\{x:x(j)\in E_j\})
=\prod_{j=0}^k\mu_0 E_j
=\prod_{j=0}^k\bar\mu a_j$.}
     
\noindent Thus $\sequence{k}{\pi^k[\frak A_0]}$ is independent.   (ii)
The closed subalgebra $\frak A'$ of $\frak A$ generated by
$\bigcup_{k\in\Bbb N}\pi^k[\frak A_0]$ must contain $\{x:x(k)\in
E\}^{\ssbullet}$ for every $k\in\Bbb N$, $E\in\Sigma$, so must contain
$W^{\ssbullet}$ for every $W$ in the $\sigma$-algebra generated by sets
of the form $\{x:x(k)\in E\}$;  but every set measured by $\mu$ is
equivalent to such a set $W$.   So $\frak A'=\frak A$.\ \Qed
}%end of prooflet
     
\spheader 385Sb\cmmnt{ The same method gives us two-sided Bernoulli
shifts.}
Again let $(X,\Sigma,\mu_0)$ be a probability space, and\cmmnt{ this
time} write $\mu$ for the product measure on $X^{\Bbb Z}$;
\cmmnt{again} let $(\frak A,\bar\mu)$ be the measure algebra of $\mu$,
and $\frak A_0\subseteq\frak A$ the set of equivalence classes of sets
of the form $\{x:x(0)\in E\}$ where $E\in\Sigma$\cmmnt{, so that
$(\frak A_0,\bar\mu\restrp\frak A_0)$ can once more be identified with
the
measure algebra of $\mu_0$}.   This time, we have a measure space
automorphism $f:X^{\Bbb Z}\to X^{\Bbb Z}$ defined by setting
     
\Centerline{$f(x)(n)=x(n+1)$ for every $x\in X^{\Bbb Z}$, $n\in\Bbb Z$,}
     
\noindent and $f$ induces a measure-preserving automorphism
$\pi:\frak A\to\frak A$.   \cmmnt{The arguments used above show that}
$\pi$ is a two-sided Bernoulli shift with root algebra $\frak A_0$.

It follows that if $(\frak A,\bar\mu)$ is an atomless
homogeneous probability algebra it has a two-sided Bernouilli
shift.   \prooflet{\Prf\ We can identify $(\frak A,\bar\mu)$ 
with the measure algebra of the usual measure
on $\{0,1\}^{\kappa\times\Bbb Z}\cong(\{0,1\}^{\kappa})^{\Bbb Z}$, where
$\kappa$ is the Maharam type of $\frak A$.\ \Qed}
     
\spheader 385Sc I remarked above that a Bernoulli shift will normally
have many root algebras.   But\cmmnt{ it is important to know that},
up to isomorphism, 
any probability algebra is the root algebra of just one Bernoulli shift 
of each type.
     
\medskip
     
\prooflet{
\Prf{\bf (i)} Given a probability algebra $(\frak A_0,\bar\mu_0)$ then
we can identify it with the measure algebra of a probability space
$(X,\Sigma,\mu_0)$ (321J), and now the constructions of (a) and (b)
provide Bernoulli shifts with root algebras isomorphic to
$(\frak A_0,\bar\mu_0)$.
     
\medskip
     
{\bf (ii)} Let $(\frak A,\bar\mu)$ and $(\frak B,\bar\nu)$ be
probability algebras with one-sided Bernoulli shifts $\pi$, $\phi$ with
root algebras $\frak A_0$, $\frak B_0$, and suppose that
$\theta_0:\frak A_0\to\frak B_0$ is a measure-preserving isomorphism.
Then $(\frak A,\bar\mu)$ can be identified with the probability algebra
free product
of $\sequence{k}{\pi^k[\frak A_0]}$ (325L), while $(\frak B,\bar\nu)$
can be identified with the probability algebra free product of
$\sequence{k}{\pi^k[\frak B_0]}$.   For each $k\in\Bbb N$,
$\phi^k\theta_0(\pi^k)^{-1}$ is a measure-preserving isomorphism between
$\pi^k[\frak A_0]$ and $\phi^k[\frak B_0]$.   Assembling these, we have
a measure-preserving isomorphism $\theta:\frak A\to\frak B$ such that
$\theta a=\phi^k\theta_0(\pi^k)^{-1}a$ whenever $k\in\Bbb N$ and
$a\in\pi^k[\frak A_0]$, that is, $\theta\pi^ka=\phi^k\theta_0a$ for
every
$a\in\frak A_0$, $k\in\Bbb N$.   Of course $\theta$ extends $\theta_0$.
     
If we set
     
\Centerline{$\frak C=\{a:a\in\frak A,\,\theta\pi a=\phi\theta a\}$,}
     
\noindent then $\frak C$ is a closed subalgebra of $\frak A$.   If
$a\in\frak A_0$ and $k\in\Bbb N$, then
     
\Centerline{$\theta\pi(\pi^ka)=\theta\pi^{k+1}a=\phi^{k+1}\theta_0 a
=\phi(\phi^k\theta_0a)=\phi\theta(\pi^ka)$,}
     
\noindent so $\pi^ka\in\frak C$.   Thus $\phi^k[\frak A_0]\subseteq\frak
C$ for every $k\in\Bbb N$, and $\frak C=\frak A$.
     
This means that $\theta:\frak A\to\frak B$ is such that
$\phi=\theta\pi\theta^{-1}$;  $\theta$ is an isomorphism between the
structures $(\frak A,\bar\mu,\pi)$ and $(\frak B,\bar\nu,\phi)$
extending
the isomorphism $\theta_0$ from $\frak A_0$ to $\frak B_0$.
     
\medskip
     
\quad{\bf (iii)} Now suppose that $(\frak A,\bar\mu)$ and 
$(\frak B,\bar\nu)$ are probability algebras with two-sided Bernoulli 
shifts
$\pi$, $\phi$ with root algebras $\frak A_0$, $\frak B_0$, and suppose
that $\theta_0:\frak A_0\to\frak B_0$ is a measure-preserving
isomorphism.   Repeating (ii) word for word, but changing each $\Bbb N$
into $\Bbb Z$, we find that $\theta_0$ has an extension to a
measure-preserving isomorphism $\theta:\frak A\to\frak B$ such that
$\theta\pi=\phi\theta$, so that once more the structures 
$(\frak A,\bar\mu,\pi)$ and $(\frak B,\bar\nu,\phi)$ are isomorphic.\ 
\Qed
}%end of prooflet
     
\spheader 385Sd The classic problem to which the theory of this section
was directed was the following:  suppose we have two two-sided Bernoulli
shifts $\pi$ and $\phi$, one based on a root algebra with two atoms of
measure $\bover12$ and the other on a root algebra with three atoms of
measure $\bover13$;  are they isomorphic?   The Kolmogorov-Sina\v\i\
theorem tells us that they are not, because $h(\pi)=\ln 2$ and
$h(\phi)=\ln 3$ are different.   \cmmnt{The question of which
Bernoulli shifts
{\it are} isomorphic is addressed, and (for countably-generated
algebras) solved, in \S387 below.}
     
\spheader 385Se We shall need to know that any Bernoulli shift (either
one- or two-sided) is ergodic.   In fact, it is mixing.
\prooflet{\Prf\ Let $(\frak A,\bar\mu)$
be a probability algebra and $\pi:\frak A\to\frak A$ a Bernoulli shift
with root algebra $\frak A_0$.   Let $\frak B$ be the subalgebra of
$\frak A$ generated by $\bigcup_{k\in\Bbb N}\pi^k[\frak A_0]$ (if $\pi$
is one-sided) or by $\bigcup_{k\in\Bbb Z}\pi^k[\frak A_0]$ (if $\pi$ is
two-sided).   If $b$, $c\in\frak B$, there is some $n\in\Bbb N$ such
that both belong to the algebra $\frak B_n$ generated by
$\bigcup_{j\le n}\pi^j[\frak A_0]$ (if $\pi$ is one-sided) or by
$\bigcup_{|j|\le
n}\pi^j[\frak A_0]$ (if $\pi$ is two-sided).   If now $k>2n$, $\pi^kb$
belongs to the algebra generated by $\bigcup_{j>n}\pi^j[\frak A_0]$.
But this is independent of $\frak B_n$ (cf.\ 325Xg, 272K), so
     
\Centerline{$\bar\mu(c\Bcap\pi^kb)=\bar\mu c\cdot\bar\mu(\pi^k b)
=\bar\mu c\cdot\bar\mu b$.}
     
\noindent And this is true for every $k\ge n$.   Generally, if $b$,
$c\in\frak A$ and $\epsilon>0$, there are $b'$, $c'\in\frak B$ such that
$\bar\mu(b\Bsymmdiff b')\le\epsilon$ and
$\bar\mu(c\Bsymmdiff c')\le\epsilon$, so that
     
$$\eqalign{\limsup_{k\to\infty}
  |\bar\mu(c\Bcap\pi^kb)-\bar\mu c\cdot\bar\mu b|
&\le\limsup_{k\to\infty}
  |\bar\mu(c'\Bcap\pi^kb')-\bar\mu c'\cdot\bar\mu b'|\cr
&\qquad\qquad+\bar\mu(c\Bsymmdiff c')
  +\bar\mu(\pi^kb\Bsymmdiff\pi^kb')
  +|\bar\mu c\cdot\bar\mu b-\bar\mu c'\cdot\bar\mu b'|\cr
&\le 0+\epsilon+\epsilon+|\bar\mu c-\bar\mu c'|
  +|\bar\mu b-\bar\mu b'|
\le 4\epsilon.\cr}$$
     
\noindent As $\epsilon$, $b$ and $c$ are arbitrary, $\pi$ is mixing.
By 372Qa, it is ergodic.\ \Qed
}%end of prooflet
     
\spheader 385Sf\cmmnt{ The following elementary remark will be
useful.}  If $(\frak A,\bar\mu)$ is a probability algebra,
$\pi:\frak A\to\frak A$ is a measure-preserving automorphism, and
$\frak A_0\subseteq\frak A$ is a closed subalgebra such that
$\sequence{k}{\pi^k[\frak A_0]}$ is independent, then
$\family{k}{\Bbb Z}{\pi^k[\frak A_0]}$ is independent.
\prooflet{\Prf\ If $J\subseteq\Bbb Z$ is finite and $\family{j}{J}{a_j}$
is a family in $\frak A_0$, take $n\in\Bbb N$ such that $-n\le j$ for
every $j\in J$; then
     
\Centerline{$\bar\mu(\inf_{j\in J}\pi^ja_j)
=\bar\mu(\inf_{j\in J}\pi^{n+j}a_j)
=\prod_{j\in J}\bar\mu a_j$.  \Qed}
}%end of prooflet
     
\spheader 385Sg\dvAnew{2011}
It is I hope obvious, but perhaps I should 
\cmmnt{explicitly }say:  if $(\frak A,\bar\mu)$ is a probability algebra, 
$\phi:\frak A\to\frak A$ is a measure-preserving automorphism, and
$\pi:\frak A\to\frak A$ is a (one- or two-sided) Bernouilli shift with a
root algebra $\frak A_0$, then
$\phi\pi\phi^{-1}$ is a Bernouilli shift and $\phi[\frak A_0]$ is a root
algebra for $\phi\pi\phi^{-1}$.

\cmmnt{
\leader{385T}{Isomorphic homomorphisms (a)} %(2002) 3{}84T
In this
section I have spoken of `isomorphic homomorphisms' without offering a
formal definition.   I hope that my intention was indeed obvious, and
that the next sentence will merely confirm what you have already
assumed.   If $(\frak A_1,\bar\mu_1)$ and $(\frak A_2,\bar\mu_2)$ are
measure algebras, and $\pi_1:\frak A_1\to\frak A_2$,
$\pi_2:\frak A_2\to\frak A_2$ are functions, then I say that
$(\frak A_1,\bar\mu_1,\pi_1)$ and $(\frak A_2,\bar\mu_2,\pi_2)$ are
isomorphic if there is a measure-preserving isomorphism
$\phi:\frak A_1\to\frak A_2$ such that $\pi_2=\phi\pi_1\phi^{-1}$.   In
this context, using
Maharam's theorem or otherwise, we can expect to be able to decide
whether $(\frak A_1,\bar\mu_1)$ and $(\frak A_2,\bar\mu_2)$ are or are
not isomorphic;  and if they are, we have a good hope of being able to
describe a measure-preserving isomorphism
$\theta:\frak A_1\to\frak A_2$.   In this case, of course,
$(\frak A_2,\bar\mu_2,\pi_2)$ will be
isomorphic to $(\frak A_1,\bar\mu_1,\pi'_2)$ where
$\pi'_2=\theta^{-1}\pi_2\theta$.   So now we have to decide whether
$(\frak A_1,\bar\mu_1,\pi_1)$ is isomorphic to
$(\frak A_1,\bar\mu_1,\pi'_2)$;  and when $\pi_1$, $\pi_2$ are
measure-preserving Boolean automorphisms, this is just the question of
whether $\pi_1$, $\pi'_2$ are conjugate in the group
$\Aut_{\bar\mu_1}(\frak A_1)$ of measure-preserving automorphisms of
$\frak A_1$.   Thus the isomorphism problem, as stated here, is very
close to the classical group-theoretic problem of identifying the
conjugacy classes in $\Aut_{\bar\mu}(\frak A)$ for a measure algebra
$(\frak A,\bar\mu)$.   But we also want to look at
measure-preserving homomorphisms which are not automorphisms, so there
would be something left even if the conjugacy problem were solved.   (In
effect, we are studying conjugacy in the semigroup of all
measure-preserving Boolean homomorphisms, not just in its group of
invertible elements.)
     
The point of the calculation of the entropy of a homomorphism is that it
is an invariant under this kind of isomorphism;  so that if $\pi_1$,
$\pi_2$ have different entropies then $(\frak A_1,\bar\mu_1,\pi_1)$ and
$(\frak A_2,\bar\mu_2,\pi_2)$ cannot be isomorphic.   Of course the
properties of being `ergodic' or `mixing' (see 372O) are also
invariant.
     
\spheader 385Tb All the main work of this section has been done in terms
of measure algebras;  part of my purpose in this volume has been to
insist that this is often the right way to proceed, and to establish a
language which makes the arguments smooth and natural.   But of course a
large proportion of the most important homomorphisms arise in the
context of measure spaces, and I take a moment to discuss such
applications.   Suppose that we have two quadruples
$(X_1,\Sigma_1,\mu_1,f_1)$ and $(X_2,\Sigma_2,\mu_2,f_2)$ where, for each $i$, $(X_i,\Sigma_i,\mu_i)$ is a measure space and $f_i:X_i\to X_i$ is an \imp\ function.   Then we have associated structures
$(\frak A_1,\bar\mu_1,\pi_1)$ and $(\frak A_2,\bar\mu_2,\pi_2)$ where $(\frak A_i,\bar\mu_i)$ is the measure algebra of $(X_i,\Sigma_i,\mu_i)$ and $\pi_i:\frak A_i\to\frak A_i$ is the measure-preserving homomorphism
defined by the usual formula
$\pi_iE^{\ssbullet}=f_i^{-1}[E]^{\ssbullet}$.
Now we can call $(X_1,\Sigma_1,\mu_1,f_1)$ and
$(X_2,\Sigma_2,\mu_2,f_2)$ isomorphic if there is a measure space
isomorphism $g:X_1\to X_2$ such that $f_2=gf_1g^{-1}$.   In this case
$(\frak A_1,\bar\mu_1,\pi_1)$ and $(\frak A_2,\bar\mu_2,\pi_2)$ are
isomorphic under the obvious isomorphism
$\phi(E^{\ssbullet})=g[E]^{\ssbullet}$ for every $E\in\Sigma_1$.
     
It is not the case that if the $(\frak A_i,\bar\mu_i,\pi_i)$ are
isomorphic, then the $(X_i,\Sigma_i,\mu_i,f_i)$ are;  in fact we do not
even need to have an isomorphism of the measure spaces (for instance,
one could be Lebesgue measure, and the other the Stone space of the
Lebesgue measure algebra).   Even when $(\frak A_1,\bar\mu_1,\pi_1)$ and
$(\frak A_2,\bar\mu_2,\pi_2)$ are actually identical, $f_1$ and $f_2$
need not be isomorphic.   There are two examples in \S343 of a
probability space $(X,\Sigma,\mu)$ with a measure space automorphism
$f:X\to X$ such that $f(x)\ne x$ for every $x\in X$ but the
corresponding automorphism on the measure algebra is the identity
(343I, 343J);  writing $\iota$ for the identity map from $X$ to itself,
$(X,\Sigma,\mu,\iota)$ and $(X,\Sigma,\mu,f)$ are non-isomorphic but give 
rise to the same $(\frak A,\bar\mu,\pi)$.
     
\spheader 385Tc Even with Lebesgue measure, we can have a problem in a
formal sense.   Take $(X,\Sigma,\mu)$ to be $[0,1]$ with Lebesgue
measure, and set $f(0)=1$, $f(1)=0$, $f(x)=x$ for $x\in\ooint{0,1}$;
then $f$ is not isomorphic to the identity function on $X$, but induces
the identity automorphism on the measure algebra.   But in this case we
can sort things out just by discarding the negligible set $\{0,1\}$, and
for Lebesgue measure such a procedure is effective in a wide variety of
situations.   To formalize it I offer the following definition.
}%end of comment
     
\leader{385U}{Definition} %(2002) 3{}84U
Let $(X_1,\Sigma_1,\mu_1)$ and
$(X_2,\Sigma_2,\mu_2)$ be measure spaces, and $f_1:X_1\to X_1$,
$f_2:X_2\to X_2$ two \imp\ functions.   I will say that the structures
$(X_1,\Sigma_1,\mu_1,f_1)$ and $(X_2,\Sigma_2,\mu_2,f_2)$ are {\bf
almost isomorphic} if there are conegligible sets $X_i'\subseteq X_i$
such that $f_i[X'_i]\subseteq X'_i$ for both $i$ and the structures
$(X'_i,\Sigma'_i,\mu'_i,f'_i)$ are isomorphic\cmmnt{ in the sense of
385Tb}, where $\Sigma'_i$ is the algebra of relatively measurable
subsets of $X'_i$, $\mu'_i$ is the subspace measure on $X'_i$ and
$f'_i=f_i\restr X'_i$.
     
\leader{385V}{}\cmmnt{ I %(2002) 3{}84V
leave the elementary properties of this
notion to the exercises (385Xn-385Xp), but I spell out the result for
which the definition is devised.   I phrase it in the language of
\S\S342-343;  if the terms are not immediately familiar, start by
imagining that both
$(X_i,\Sigma_i,\mu_i)$ are measurable subspaces of $\Bbb R$ endowed with
some Radon measure (342J, 343H), or indeed that both are $[0,1]$
with Lebesgue measure.
     
\medskip
     
\noindent}{\bf Proposition} Let $(X_1,\Sigma_1,\mu_1)$ and
$(X_2,\Sigma_2,\mu_2)$ be perfect, complete, strictly localizable and
countably separated measure spaces, and $(\frak A_1,\bar\mu_1)$,
$(\frak A_2,\bar\mu_2)$ their measure algebras.   Suppose that
$f_1:X_1\to X_1$, $f_2:X_2\to X_2$ are \imp\ functions and that
$\pi_1:\frak A_1\to\frak A_1$, $\pi_2:\frak A_2\to\frak A_2$ are the
measure-preserving Boolean
homomorphisms they induce.   If $(\frak A_1,\bar\mu_1,\pi_1)$ and
$(\frak A_2,\bar\mu_2,\pi_2)$ are isomorphic, then
$(X_1,\Sigma_1,\mu_1,f_1)$ and $(X_2,\Sigma_2,\mu_2,f_2)$ are almost
isomorphic.
     
\proof{ Because $(\frak A_1,\bar\mu_1)$ and $(\frak A_2,\bar\mu_2)$ are
isomorphic, we surely have $\mu_1X_1=\mu_2X_2$.   If both are zero, we
can take $X'_1=X'_2=\emptyset$ and stop;  so let us suppose that
$\mu_1X_1>0$.   Let $\phi:\frak A_1\to\frak A_2$ be a measure-preserving
automorphism such that $\pi_2=\phi\pi_1\phi^{-1}$.   Because both
$\mu_1$ and $\mu_2$ are complete and strictly localizable and compact
(343K), there are \imp\ functions $g_1:X_1\to X_2$ and $g_2:X_2\to X_1$
representing
$\phi^{-1}$, $\phi$ respectively (343B).   Now $g_1g_2:X_2\to X_2$,
$g_2g_1:X_1\to X_1$, $f_2g_1:X_1\to X_2$ and $g_1f_1:X_1\to X_2$
represent, respectively, the identity automorphism on $\frak A_2$, the
identity automorphism on $\frak A_1$, the homomorphism
$\phi^{-1}\pi_2=\pi_1\phi^{-1}:\frak A_2\to\frak A_1$ and the
homomorphism $\pi_1\phi^{-1}$ again.   Next, because
both $\mu_1$ and $\mu_2$ are countably separated, the sets
$E_1=\{x:g_2g_1(x)=x\}$, $H=\{x:f_2g_1(x)=g_1f_1(x)\}$ and
$E_2=\{y:g_1g_2(y)=y\}$ are all conegligible (343F).   As in part (b) of
the proof of 344I, $g_1\restr E_1$ and $g_2\restr E_2$ are the two
halves of a bijection, a measure space isomorphism if $E_1$ and $E_2$
are given their subspace measures.   Set $G_0=E_1\cap H$, and for
$n\in\Bbb N$ set
$G_{n+1}=G_n\cap f_1^{-1}[G_n]$.   Then every $G_n$ is conegligible, so
$X'_1=\bigcap_{n\in\Bbb N}G_n$ is conegligible.   Because $X'_1$ is a
conegligible subset of $E_1$, $h=g_1\restr X'_1$ is a measure space
isomorphism between $X'_1$ and $X'_2=g_1[X'_1]$, which is conegligible
in $X_2$.    Because $f_1[G_{n+1}]\subseteq G_n$ for each $n$,
$f_1[X'_1]\subseteq X'_1$.   Because $X'_1\subseteq H$,
$g_1f_1(x)=f_2g_1(x)$ for every $x\in X'_1$.   Next, if $y\in X'_2$,
$g_2(y)\in X'_1$, so
     
\Centerline{$f_2(y)=f_2g_1g_2(y)=g_1f_1g_2(y)\in g_1[f_1[X'_1]]
\subseteq g_1[X'_1]=X'_2$.}
     
\noindent Accordingly we have $f'_2=hf'_1h^{-1}$, where $f'_i=f_i\restr
X'_i$ for both $i$.
     
Thus $h$ is an isomorphism between $(X'_1,f'_1)$ and $(X'_2,f'_2)$, and
$(X_1,\Sigma_1,\mu_1,f_1)$ and $(X_2,\Sigma_2,\mu_2,f_2)$ are almost
isomorphic.
}%end of proof of 385V
     
\exercises{\leader{385X}{Basic exercises (a)} 
%\spheader 385Xa %(2002) 3{}84Xa
Let $(\frak A,\bar\mu)$ be a probability algebra and $A\subseteq\frak A$
a partition of unity.   Show that if $\#(A)=n$ then $H(A)\le\ln n$.
%385C
     
\sqheader 385Xb %(2002) 3{}84Xb
Let $(\frak A,\bar\mu)$ be a probability algebra,
$\frak B$ a closed subalgebra of $\frak A$ and $A$ a
partition of unity in $\frak A$, enumerated as $\sequencen{a_n}$.   Set
$a_n^*=\sup_{i>n}a_i$, $A_n=\{a_0,\ldots,a_n,a_n^*\}$ for each
$n$.   Show that $H(A_n|\frak B)\le H(A_{n+1}|\frak B)$ for every $n$,
and that $H(A|\frak B)=\lim_{n\to\infty}H(A_n|\frak B)$.
%385G
     
\spheader 385Xc %(2002) 3{}84Xc
Let $(\frak A,\bar\mu)$ be a probability algebra,
$\frak B$ a closed subalgebra of $\frak A$ and $A$ a partition of unity
in $\frak A$.   Show that $H(A|\frak B)=0$ iff $A\subseteq\frak B$.
%385G
     
\spheader 385Xd %(2002) 3{}84Xd
Let $(\frak A,\bar\mu)$ be a probability algebra,
$\frak B$ a closed subalgebra of $\frak A$ and $A$ a partition of unity
in $\frak A$.   Show that $H(A|\frak B)=H(A)$ iff
$\bar\mu(a\Bcap b)=\bar\mu a\cdot\bar\mu b$ for every $a\in A$,
$b\in\frak B$.   \Hint{for
`only if', start with the case $\frak B=\{0,b,1\Bsetminus b,1\}$ and
use 385Gc.}
%385G
     
\spheader 385Xe %(2002) 3{}84Xe
Let $(\frak A,\bar\mu)$ be a probability algebra and
$A$, $B$ two partitions of unity in $\frak A$.   Show that
$H(A\vee B)=H(A)+H(B)$ iff $\bar\mu(a\Bcap b)=\bar\mu a\cdot\bar\mu b$
for all $a\in A$, $b\in B$.   Show that $H(A\vee B)=H(A)$ iff every
member of $A$ is included in some member of $B$, that is, iff
$A=A\vee B$.
%385G
     
\spheader 385Xf %(2002) 3{}84Xf
Let $\langle(\frak A_i,\bar\mu_i)\rangle_{i\in I}$ be a
family of probability algebras, with probability algebra free product
$(\frak C,\bar\lambda)$ (325K).   Suppose that
$\pi_i:\frak A_i\to\frak A_i$ is a measure-preserving Boolean
homomorphism for each $i\in I$, and
that $\pi:\frak C\to\frak C$ is the measure-preserving Boolean
homomorphism they induce.   Show that $h(\pi)=\sum_{i\in I}h(\pi_i)$.
\Hint{use 385Gb and 385Gd to show that $h(\pi)$ is the supremum of
$h(\pi,A)$ as $A$ runs over the finite partitions of unity in
$\bigotimes_{i\in I}\frak A_i$.   Use this to reduce to the case
$I=\{0,1\}$.   Now show that if $A_i\subseteq\frak A_i$ is a finite
partition of unity for each $i$, and $A=\{a_0\otimes a_1:a_0\in
A_0,\,a_1\in A_1\}$, then $H(A)=H(A_0)+H(A_1)$, so that
$h(\pi,A)=h(\pi_0,A_0)+h(\pi_1,A_1)$.}
%385M
     
\sqheader 385Xg %(2002) 3{}84Xg
Let $(\frak A,\bar\mu)$ be a probability algebra and
$\pi:\frak A\to\frak A$ a measure-preserving automorphism.   Show that
$h(\pi^{-1})=h(\pi)$.
%385M
     
\spheader 385Xh %(2002) 3{}84Xh
Let $(\frak A,\bar\mu)$ be a probability algebra and
$\pi:\frak A\to\frak A$ a measure-preserving Boolean homomorphism.
Show that $h(\pi^k)=k h(\pi)$ for any $k\in\Bbb N$.   \Hint{if
$A\subseteq\frak A$ is a partition of unity,  $h(\pi^k,A)\le
h(\pi^k,D_k(A,\pi))=kh(\pi,A)$.}
%385M
     
\sqheader 385Xi %(2002) 3{}84Xi
Let $(\frak A,\bar\mu)$ be a probability algebra and
$\pi:\frak A\to\frak A$ a
measure-preserving Boolean homomorphism.   (i) Suppose there is a
partition of unity $A\subseteq\frak A$ such that ($\alpha$)
$\bar\mu(a\Bcap\pi b)=\bar\mu a\cdot\bar\mu b$ for every $a\in A$,
$b\in\frak A$ ($\beta$) $\frak A$ is the closed subalgebra of itself
generated by $\bigcup_{n\in\Bbb N}\pi^n[A]$.   Show that $\pi$ is a
one-sided Bernoulli shift, and that $h(\pi)=H(A)$.   (ii) Suppose that
$\pi$ is a one-sided Bernoulli shift of finite entropy.   Show that
there is a partition of unity satisfying ($\alpha$) and ($\beta$).
%385Q
     
\sqheader 385Xj %(2002) 3{}84Xj
Let $(\frak A,\bar\mu)$ be the measure algebra of
Lebesgue measure on $\coint{0,1}$.   Fix an integer $k\ge 2$, and define
$f:\coint{0,1}\to\coint{0,1}$ by setting $f(x)=\fraction{kx}$, the
fractional part of $kx$, for every $x\in\coint{0,1}$;  let
$\pi:\frak A\to\frak A$ be the corresponding homomorphism.
(Cf.\ 372Xt.)   Show
that $\pi$ is a one-sided Bernoulli shift and that $h(\pi)=\ln k$.
\Hint{in 385Xi, set $A=\{a_0,\ldots,a_{k-1}\}$ where
$a_i=\coint{\bover{i}{k},\bover{i+1}k}^{\ssbullet}$ for $i<k$.}
%385Q, 385Xi
     
\sqheader 385Xk %(2002) 3{}84Xk
Let $(\frak A,\bar\mu)$ be the measure algebra of
Lebesgue measure on $[0,1]$.   Set $f(x)=2\min(x,1-x)$ for $x\in[0,1]$
(see 372Xp).   Show that the corresponding homomorphism
$\pi:\frak A\to\frak A$ is a one-sided Bernoulli shift and that
$h(\pi)=\ln 2$.
\Hint{in 385Xi, set $A=\{a,1\Bsetminus a\}$ where
$a=[0,\bover12]^{\ssbullet}$.}
%385Q, 385Xi
     
\spheader 385Xl %(2002) 3{}84Xl
Let $(\frak A,\bar\mu)$ be a probability algebra and
$\pi:\frak A\to\frak A$ a two-sided Bernoulli shift.   Show that
$\pi^{-1}$ is a two-sided Bernoulli shift and $\pi$ and $\pi^{-1}$ are conjugate in $\AmuA$, so that $\pi$ is a product of two
involutions in $\Aut_{\bar\mu}(\frak A)$.
%385Q
     
\spheader 385Xm %(2002) 3{}84Xm
Let $\langle(\frak A_i,\bar\mu_i)\rangle_{i\in I}$ be a
family of probability algebras, and $(\frak C,\bar\lambda)$ their
probability algebra free product.   Suppose that for each
$i\in I$ we have a measure-preserving Boolean homomorphism
$\pi_i:\frak A_i\to\frak A_i$, and that $\pi:\frak C\to\frak C$ is the
measure-preserving homomorphism induced by
$\langle\pi_i\rangle_{i\in I}$ (325Xe).   (i) Show that if every $\pi_i$
is a one-sided
Bernoulli shift so is $\pi$.   (ii) Show that if every $\pi_i$ is a
two-sided Bernoulli shift so is $\pi$.
%385Q
     
\spheader 385Xn %(2002) 3{}84Xn
Show that the relation `almost isomorphic to' (385U)
is an equivalence relation.
%385U
     
\spheader 385Xo %(2002) 3{}84Xo
Show that the concept of `almost isomorphism'
described in 385U is not changed if we amend the definition to require
that the subspaces $X'_1$, $X'_2$ should be measurable.
%385V
     
\spheader 385Xp %(2002) 3{}84Xp
Show that if $(X_1,\Sigma_1,\mu_1,f_1)$ and
$(X_2,\Sigma_2,\mu_2,f_2)$ are almost isomorphic quadruples as described
in 385U, then $(\frak A_1,\bar\mu_1,\pi_1)$ and $(\frak
A_2,\bar\mu_2,\pi_2)$ are isomorphic, where for each
$i\,\,(\frak A_i,\bar\mu_i)$ is the measure algebra of
$(X_i,\Sigma_i,\mu_i)$ and
$\pi_i:\frak A_i\to\frak A_i$ is the measure-preserving Boolean
homomorphism derived from $f_i:X_i\to X_i$.
%385V
     
\spheader 385Xq %(2002) 3{}84Xq
Let $(\frak A,\bar\mu)$ be a probability algebra, and
write $\Cal A$ for the set of partitions of unity in
$\frak A$ not containing $0$, ordered by saying that $A\le B$ if every
member
of $B$ is included in some member of $A$.   (i) Show that $\Cal A$ is a
Dedekind complete lattice, and can be identified with the lattice of
purely atomic closed subalgebras of $\frak A$.   Show that for $A$,
$B\in\Cal A$, $A\vee B$, as defined in 385F, is $\sup\{A,B\}$ in $\Cal
A$.   (ii) Show that $H(A\vee B)+H(A\wedge B)\le
H(A)+H(B)$ for all $A$, $B\in\Cal A$, where $\vee$, $\wedge$ are the
lattice operations on $\Cal A$.   (iii) Set $\Cal A_1=\{A:A\in\Cal
A,\,H(A)<\infty\}$.   For $A$, $B\in\Cal A_1$ set
$\rho(A,B)=2 H(A\vee B)-H(A)-H(B)$.   Show that $\rho$ is a metric on
$\Cal A_1$ (the {\bf entropy metric}).   (iv) Show that if $\pi:\frak
A\to\frak A$ is a measure-preserving Boolean homomorphism, then
$|h(\pi,A)-h(\pi,B)|\le\rho(A,B)$ for all $A$, $B\in\Cal A_1$.   (iv)
Show that the lattice operations $\vee$ and $\wedge$ are
$\rho$-continuous
on $\Cal A_1$.   (v) Show that $H:\Cal A_1\to \coint{0,\infty}$ is
order-continuous.   (vi) Show that if $\frak B$ is any closed subalgebra
of $\frak A$, then $A\mapsto H(A|\frak B)$ is order-continuous on
$\Cal A_1$.
%385+
     
\spheader 385Xr %(2002) 3{}84Xr
Let $(\frak A,\bar\mu)$ be a probability algebra and $\frak B$ a topologically dense subalgebra of $\frak A$.   (i) Show that if
$\langle a_i\rangle_{i\le n}$ is a partition of unity in $\frak A$ and $\epsilon>0$, there is a partition $\langle b_i\rangle_{i\le n}$ of unity
in $\frak B$ such that $\bar\mu(a_i\Bsymmdiff b_i)\le\epsilon$ for every $i\le n$.   (ii) Show that if $A$ is a finite partition of unity in
$\frak A$ and $\epsilon>0$ then there is a finite partition of unity $D\subseteq\frak B$ such that $H(A\vee D)\le H(A)+\epsilon$.   (iii) Show that if
$\pi:\frak A\to\frak A$ is a measure-preserving Boolean homomorphism, then $h(\pi)=\sup\{h(\pi,D):D\subseteq\frak B$ is a finite partition of unity$\}$.
\Hint{385N, 385Gb.}
%385+

\spheader 385Xs Let $(\frak A,\bar\mu)$ be a probability algebra 
and $\pi:\frak A\to\frak A$ an ergodic measure-preserving Boolean 
homomorphism.   Show that if $h(\pi)>0$ then $\frak A$ is atomless.
%385S
     
\leader{385Y}{Further exercises (a)}
%\spheader 385Ya %(2002) 3{}84Ya
Let $(\frak A,\bar\mu)$ be a probability algebra, and
write $\frak P$ for the lattice of closed subalgebras of $\frak A$.
Show that if $A$ is any partition of unity in $\frak A$ of finite
entropy, then the
order-preserving function $\frak B\mapsto -H(A|\frak B):
\frak P\to\ocint{-\infty,0}$ is order-continuous.
%385G
     
\spheader 385Yb %(2002) 3{}84Yb
Let $(\frak A,\bar\mu)$ be a probability algebra, $A$ a
partition of unity in $\frak A$ of finite entropy, and
$\pi:\frak A\to\frak A$ a
measure-preserving Boolean homomorphism.   Show that
$h(\pi,A)=\lim_{n\to\infty}H(A|\frak B_n)$, where $\frak B_n$ is the
closed subalgebra of $\frak A$ generated by
$\bigcup_{1\le i\le n}\pi^i[A]$.   \Hint{use 385Gb to show that
$H(A|\frak B_n)=H(D_{n+1}(A,\pi))-H(D_n(A,\pi))$ and observe that
$\lim_{n\to\infty}H(A|\frak B_n)$ is defined.}
%385Ya (385G), 385M
     
\spheader 385Yc %(2002) 3{}84Yc
Let $(\frak A,\bar\mu)$ be a probability algebra and
$\pi:\frak A\to\frak A$ a measure-preserving Boolean homomorphism.
Suppose that there is a partition of unity $A$ of finite entropy such
that the closed subalgebra of $\frak A$ generated by $\bigcup_{i\ge
1}\pi^i[A]$ is $\frak A$.   Show that $h(\pi)=0$.   \Hint{use 385Yb and
385Pa.}
%385P, 385Yb (385G)
     
\spheader 385Yd %(2002) 3{}84Yd
Let $\mu$ be Lebesgue measure on $\coint{0,1}$, and take
any $\alpha\in\ooint{0,1}$.   Let
$f:\coint{0,1}\to\coint{0,1}$ be the measure space automorphism defined
by saying that $f(x)$ is to be one of $x+\alpha$, $x+\alpha-1$.   Let
$(\frak A,\bar\mu)$ be the measure algebra of $(\coint{0,1},\mu)$ and
$\pi:\frak A\to\frak A$ the measure-preserving automorphism
corresponding to $f$.   Show that $h(\pi)=0$.   \Hint{if
$\alpha\in\Bbb Q$, use 385Xh;  otherwise use 385Yc with
$A=\{a,1\Bsetminus a\}$ where $a=\coint{0,\bover12}^{\ssbullet}$.}
%385Yc (385G, 385P, 385Yb), 385Xh (385M)
     
\spheader 385Ye %(2002) 3{}84Ye
Set $X=[0,1]\setminus\Bbb Q$, let $\nu$ be the measure
on $X$ defined by setting $\nu E=\bover1{\ln 2}\int_E\bover{1}{1+x}dx$
for every Lebesgue measurable set $E\subseteq X$, and for $x\in X$ let
$f(x)$ be the fractional part $\fraction{\bover1x}$ of $\bover1x$.
Recall that $f$ is \imp\ for $\nu$ (see 372M).  Let $(\frak A,\bar\nu)$ be the measure algebra of $(X,\nu)$ and $\pi:\frak A\to\frak A$ the
homomorphism corresponding to $f$.   Show that $h(\pi)=\pi^2/6\ln 2$.
\Hint{use the Kolmogorov-Sina\v\i\ theorem and 372Yh(v).}
%385P
     
\spheader 385Yf Let $(\frak A,\bar\mu)$ be a probability algebra, and
$\phi:\frak A\to\frak A$ a one-sided Bernoulli shift.   
Show that there are a probability algebra $(\frak C,\bar\lambda)$, a
two-sided Bernouilli shift $\tilde\phi:\frak C\to\frak C$, and a
measure-preserving Boolean homomorphism $\pi:\frak A\to\frak C$ such that
$\tilde\phi\pi=\pi\phi$.   \Hint{328J.}
%385S
     
\spheader 385Yg %(2002) 3{}84Yf
Consider the triplets $(\coint{0,1},\mu_1,f_1)$ and
$([0,1],\mu_2,f_2)$ where $\mu_1$, $\mu_2$ are Lebesgue measure on
$\coint{0,1}$, $[0,1]$ respectively, $f_1(x)=\fraction{2x}$ for each
$x\in\coint{0,1}$, and $f_2(x)=2\min(x,1-x)$ for each $x\in [0,1]$.
Show that these structures are almost isomorphic in the sense of 385U,
and give a formula for an almost-isomorphism.
%385U, 385Xj, 385Xk
     
\spheader 385Yh %(2002) 3{}84Yg
Let $(\frak A,\bar\mu)$ be a probability algebra, and
$\Cal A_1$ the set of partitions of unity of finite entropy not
containing $0$, as in 385Xq.   Show that $\Cal A_1$ is complete under
the entropy metric.   \Hint{show that if $\sequencen{A_n}$ is a
non-decreasing sequence in $\Cal A_1$ and $\sup_{n\in\Bbb
N}H(A_n)<\infty$, then the closed subalgebra of $\frak A$ generated by
$\bigcup_{n\in\Bbb N}A_n$ is purely atomic.}
%385Xq
}%end of exercises
     
\cmmnt{
\Notesheader{385} In preparing this section I have been heavily
influenced by {\smc Petersen 83}.   I have taken almost the shortest
possible route to Theorem 385P, the original application of the theory,
ignoring both the many extensions of these ideas and their intuitive
underpinning in the concept of the quantity of `information' carried
by a partition.   For both of these I refer you to {\smc Petersen 83}.
The techniques described there are I think sufficiently powerful to make
possible the calculation of the entropy of any of the measure-preserving
homomorphisms which have yet appeared in this treatise.
     
Of course the idea of entropy of a partition, or of a homomorphism, can
be translated into the language of probability spaces and \imp\
functions;  if $(X,\Sigma,\mu)$ is a probability space, with measure
algebra $(\frak A,\bar\mu)$, then partitions of unity in $\frak A$
correspond (subject to decisions on how to treat negligible sets) to
countable partitions of $X$ into measurable sets, and an \imp\ function
$f:X\to X$ gives rise to a measure-preserving homomorphism
$\pi_f:\frak A\to\frak A$;  so we can define the entropy of $f$ to be
$h(\pi_f)$.   The whole point of the language I have sought to develop
in this volume is that we can do this when and if we choose;  in
particular, we are not limited to those homomorphisms which are
representable by \imp\ functions.   But of course a large proportion of
the most important examples do arise in this way (see 385Xj, 385Xk).
The same two examples are instructive from another point of view:  the
case $k=2$ of 385Xj is (almost) isomorphic to the tent map of 385Xk.
The similarity is obvious, but exhibiting an actual isomorphism is I
think another matter (385Yg).
     
I must say `almost' isomorphic here because the doubling map on
$\coint{0,1}$ is everywhere two-to-one, while the tent map is not, so
they cannot be isomorphic in any exact sense.   This is the problem
grappled with in 385T-385V.   In some moods I would say that a dislike
of such contortions is a sign of civilized taste.   Certainly it is part
of my motivation for working with measure algebras whenever possible.
But I have to say also that new ideas in this topic arise more often
than not from actual measure spaces, and that it is absolutely necessary
to be able to operate in the more concrete context.
}%end of comment
     
\discrpage
     
