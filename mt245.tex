\frfilename{mt245.tex}
\versiondate{25.3.06}
\copyrightdate{1995}
\def\halfarrow{\rightharpoonup}

\loadeusm

\def\chaptername{Function spaces}
\def\sectionname{Convergence in measure}

\newsection{245}

I come now to an
important and interesting topology on the spaces $\eusm L^0$ and $L^0$.
I start with the definition (245A) and with properties which echo those
of the $L^p$ spaces for $p\ge 1$ (245D-245E).   In 245G-245J I describe
the most useful relationships between this topology and the norm
topologies of the $L^p$ spaces.   For $\sigma$-finite spaces, it is
metrizable (245Eb) and sequential convergence can be described in terms
of pointwise convergence of sequences of functions (245K-245L).

\leader{245A}{Definitions}  Let $(X,\Sigma,\mu)$ be a measure space.

\header{245Aa}{\bf (a)}   For any measurable set $F\subseteq X$ of
finite measure, we have a functional $\tau_F$ on 
$\eusm L^0=\eusm L^0(\mu)$ defined by setting

\Centerline{$\tau_F(f)=\int|f|\wedge\chi F$}

\noindent for every $f\in\eusm L^0$.
\cmmnt{(The integral exists in $\Bbb R$ because $|f|\wedge\chi F$
belongs to
$\eusm L^0$ and is dominated by the integrable function $\chi F$).}
Now $\tau_F(f+g)\le\tau_F(f)+\tau_F(g)$ whenever $f$, $g\in\eusm L^0$.
\prooflet{\Prf\ We need only observe that

\Centerline{$\min(|(f+g)(x)|,\chi F(x))
\le\min(|f(x)|,\chi F(x))+\min(|g(x)|,\chi F(x))$}

\noindent for every $x\in\dom f\cap\dom g$, which is almost every
$x\in X$.\ \QeD}
Consequently, setting $\rho_F(f,g)=\tau_F(f-g)$,\cmmnt{ we have

\Centerline{$\rho_F(f,h)=\tau_F((f-g)+(g-h))
\le\tau_F(f-g)+\tau_F(g-h)=\rho_F(f,g)+\rho_F(g,h)$,}

\Centerline{$\rho_F(f,g)=\tau_F(f-g)\ge 0$,}

\Centerline{$\rho_F(f,g)=\tau_F(f-g)=\tau_F(g-f)=\rho_F(g,f)$}

\noindent for all $f$, $g$, $h\in\eusm L^0$;
that is,} $\rho_F$ is a pseudometric on $\eusm L^0$.

\header{245Ab}{\bf (b)} The family

\Centerline{$\{\rho_F:F\in\Sigma,\,\mu F<\infty\}$}

\noindent now defines a topology on $\eusm L^0$\cmmnt{ (2A3F)};   I
will call it the topology of
{\bf convergence in measure} on $\eusm L^0$.

\header{245Ac}{\bf (c)} If $f$, $g\in\eusm L^0$ and $f\eae g$,
then\cmmnt{ $|f|\wedge\chi F\eae|g|\wedge\chi F$ and}
$\tau_F(f)=\tau_F(g)$, for
every set $F$ of finite measure.   Consequently we have functionals
$\bar\tau_F$ on $L^0=L^0(\mu)$ defined by writing

\Centerline{$\bar\tau_F(f^{\ssbullet})=\tau_F(f)$}

\noindent whenever $f\in\eusm L^0$, $F\in\Sigma$ and $\mu F<\infty$.
Corresponding to these we have pseudometrics $\bar\rho_F$ defined by
either of the formulae

\Centerline{$\bar\rho_F(u,v)=\bar\tau_F(u-v)$,
\quad$\bar\rho_F(f^{\ssbullet},g^{\ssbullet})=\rho_F(f,g)$}

\noindent for $u$, $v\in L^0$, $f$, $g\in\eusm L^0$ and $F$ of finite
measure.   The family of these pseudometrics defines the {\bf topology
of convergence in measure} on $L^0$.

\spheader 245Ad I shall allow myself to say that a sequence
(in $\eusm L^0$ or $L^0$) {\bf converges in measure} if it converges
for the
topology of convergence in measure\cmmnt{ (in the sense of 2A3M)}.

\cmmnt{
\leader{245B}{Remarks (a)} Of course the topologies of $\eusm L^0$,
$L^0$ are about as closely related as it is possible for them to be.
Not only is the topology of $L^0$ the quotient of the topology on
$\eusm L^0$ (that is, a set $G\subseteq L^0$ is open iff
$\{f:f^{\ssbullet}\in G\}$ is open in $\eusm L^0$), but every open set in
$\eusm L^0$ is the
inverse image under the quotient map of an open set in $L^0$.


\header{245Bb}{\bf (b)}
It is convenient to note that if $F_0,\ldots,F_n$ are measurable sets of
finite measure with union $F$, then, in the notation of 245A,
$\tau_{F_i}\le\tau_{F}$ for every $i$;   this
means that a set $G\subseteq \eusm L^0$ is open for the topology of
convergence in measure iff for every $f\in G$ we can find a single set
$F$ of finite measure and a $\delta>0$ such that

\Centerline{$\rho_F(g,f)\le\delta\Longrightarrow g\in G$.}

\noindent Similarly, a set $G\subseteq L^0$ is open for the topology of
convergence in measure iff for every $u\in G$ we can find a set
$F$ of finite measure and a $\delta>0$ such that

\Centerline{$\bar\rho_F(v,u)\le\delta\Longrightarrow v\in G$.}

\header{245Bc}{\bf (c)} The phrase `topology of convergence in
measure' agrees well enough with standard usage when
$(X,\Sigma,\mu)$ is totally finite.   But a {\bf warning!} the phrase
`topology of convergence in measure' is also used for the topology
defined by the metric of 245Ye below, even when $\mu X=\infty$.   I have
seen the phrase {\bf local convergence in measure} used for the topology
of 245A.   Most authors ignore
non-$\sigma$-finite spaces in this context.   However I hold that
245D-245E below are of sufficient interest to make the extension
worth while.
}%end of comment

\leader{245C}{Pointwise convergence}\cmmnt{ The topology of
convergence in measure is almost definable in terms of `pointwise convergence', which is one of the roots of measure theory.   The
correspondence is closest in $\sigma$-finite measure spaces (see 245K),
but there is still a very important relationship in the general case, as
follows.}   Let $(X,\Sigma,\mu)$ be a measure space, and write $\eusm
L^0=\eusm L^0(\mu)$, $L^0=L^0(\mu)$.

\spheader 245Ca If $\sequencen{f_n}$ is a sequence in $\eusm L^0$
converging almost everywhere to $f\in\eusm L^0$, then
$\sequencen{f_n}\to f$ in measure.   \prooflet{\Prf\ By 2A3Mc, I have
only to show that $\lim_{n\to\infty}\rho_F(f_n,f)=0$ whenever
$\mu F<\infty$.   But $\sequencen{|f_n-f|\wedge\chi F}$ converges to $0$
a.e.\ and is dominated by the integrable function $\chi F$, so by
Lebesgue's Dominated Convergence Theorem

\Centerline{$\lim_{n\to\infty}\rho_F(f_n,f)
=\lim_{n\to\infty}\int|f_n-f|\wedge\chi F=0$.   \Qed}}

\spheader 245Cb \cmmnt{ To formulate a corresponding result applicable
to $L^0$, we need the following concept.   If $\sequencen{f_n}$,
$\sequencen{g_n}$ are sequences in $\eusm L^0$ such that
$f_n^{\ssbullet}=g_n^{\ssbullet}$ for every $n$, and $f$,
$g\in\eusm L^0$ are such that $f^{\ssbullet}=g^{\ssbullet}$, and
$\sequencen{f_n}\to f$ a.e., then $\sequencen{g_n}\to g$
a.e.\prooflet{, because

$$\eqalign{\{x:x\in\dom f&\cap\dom g\cap\bigcap_{n\in\Bbb N}\dom
f_n\cap\bigcap_{n\in\Bbb N}g_n,\cr
&g(x)=f(x)=\lim_{n\to\infty}f_n(x),\,f_n(x)
=g_n(x)\Forall n\in \Bbb N\}\cr}$$

\noindent is conegligible}.   Consequently we have a definition
applicable to sequences in $L^0$;  we can say that, for}
\dvro{For}{} $f$, $f_n\in\eusm L^0$,
$\sequencen{f_n^{\ssbullet}}$ is {\bf order*-convergent}, or {\bf
order*-converges}, to $f^{\ssbullet}$ iff $f\eae\lim_{n\to\infty}f_n$.   In\cmmnt{ this case, of course, $\sequencen{f_n}\to f$ in
measure.   Thus, in} $L^0$, a sequence $\sequencen{u_n}$ which
order*-converges to $u\in L^0$ also converges to $u$ in measure.

\cmmnt{\medskip

\noindent{\bf Remark} I suggest alternative descriptions of
order-convergence in 245Xc;  the conditions (iii)-(vi) there are in
forms adapted to more general structures.
}%end of comment

\spheader 245Cc\cmmnt{ For a typical example of a sequence which is
convergent in measure without being order-convergent, consider the
following.}   Take $\mu$ to be Lebesgue measure on $[0,1]$, and set
$f_n(x)=2^m$ if $x\in[2^{-m}k,2^{-m}(k+1)]$, $0$ otherwise, where
$k=k(n)\in\Bbb N$, $m=m(n)\in\Bbb N$ are defined by saying that
$n+1=2^{m}+k$ and $0\le k<2^m$.   Then $\sequencen{f_n}\to 0$ for the
topology of convergence in measure\cmmnt{ (since $\rho_F(f_n,0)\le
2^{-m}$ if $F\subseteq[0,1]$ is measurable and
$2^m-1\le n$)}, though $\sequencen{f_n}$ is not convergent
to $0$ almost everywhere\cmmnt{ (indeed,
$\limsup_{n\to\infty}f_n=\infty$ everywhere)}.

\leader{245D}{Proposition} Let $(X,\Sigma,\mu)$ be any measure space.

(a) The topology of convergence in measure is a linear space topology
on $L^0=L^0(\mu)$.

(b) The maps $\vee$, $\wedge:L^0\times L^0\to L^0$, and  $u\mapsto|u|$,
$u\mapsto u^+$, $u\mapsto u^-:L^0\to L^0$ are all continuous.

(c) The map $\times:L^0\times L^0\to L^0$ is continuous.

(d) For any continuous function $h:\Bbb R\to\Bbb R$, the corresponding
function $\bar h:L^0\to L^0$\cmmnt{ (241I)} is continuous.

\proof{{\bf (a)} The point is that the functionals $\bar\tau_F$,
as defined in 245Ac, are F-seminorms in the sense of 2A5B.   \Prf\
Fix a set $F\in\Sigma$ of finite measure.   I noted in 245Aa that

\Centerline{$\tau_F(f+g)\le\tau_F(f)+\tau_F(g)$ for
all $f$, $g\in\eusm L^0$,}

\noindent so

\Centerline{$\bar\tau_F(u+v)\le\bar\tau_F(u)+\bar\tau_F(v)$ for
all $u$, $v\in L^0$.}

\noindent Next,

$$\eqalignno{\bar\tau_F(cu)&\le\bar\tau_F(u)\text{ whenever }u\in L^0
  \text{ and }|c|\le 1&\text{(*)}\cr}$$

\noindent because $|cf|\wedge\chi F\leae|f|\wedge\chi F$ whenever
$f\in\eusm L^0$ and $|c|\le 1$.   Finally, given $u\in L^0$ and
$\epsilon>0$, let $f\in\eusm L^0$ be such that $f^{\ssbullet}=u$.   Then

\Centerline{$\lim_{n\to\infty}|2^{-n}f|\wedge\chi F=0$ a.e.,}

\noindent so by Lebesgue's Dominated Convergence Theorem

\Centerline{$\lim_{n\to\infty}\bar\tau_F(2^{-n}u)
=\lim_{n\to\infty}\int|2^{-n}f|\wedge\chi F=0$,}

\noindent and there is an $n$ such that
$\bar\tau_F(2^{-n}u)\le\epsilon$.   It follows (by (*) just above)
that $\bar\tau_F(cu)\le\epsilon$ whenever $|c|\le 2^{-n}$.   As
$\epsilon$ is arbitrary, $\lim_{c\to 0}\bar\tau_F(u)=0$ for every
$u\in L^0$;  which is the third condition in 2A5B.\ \Qed

Now 2A5B tells us that the topology defined by the $\bar\tau_F$ is
a linear space topology.

\medskip

{\bf (b)} For any $u$, $v\in L^0$, $||u|-|v||\le|u-v|$, so
$\bar\rho_F(|u|,|v|)\le\bar\rho_F(u,v)$ for every set $F$ of finite
measure.   By 2A3H, $|\,\,|:L^0\to L^0$ is continuous.   Now

\Centerline{$u\vee v=\Bover12(u+v+|u-v|)$,
\quad$u\wedge v=\Bover12(u+v-|u-v|)$,}

\Centerline{$u^+=u\wedge 0$,
\quad$u^-=(-u)\wedge 0$.}

\noindent As addition and subtraction are continuous, so are $\vee$,
$\wedge$, $^+$ and $^-$.

\medskip

{\bf (c)} Take $u_0$, $v_0\in L^0$, $F\in\Sigma$ a set of finite
measure and $\epsilon>0$.   Represent $u_0$ and $v_0$ as
$f_0^{\ssbullet}$, $g_0^{\ssbullet}$ respectively, where $f_0$,
$g_0:X\to\Bbb R$ are $\Sigma$-measurable (241Bk).   If we set

\Centerline{$F_m=\{x:x\in F,\,|f_0(x)|+|g_0(x)|\le m\}$,}

\noindent then $\sequence{m}{F_m}$ is a non-decreasing sequence of sets
with union $F$, so there is an $m\in\Bbb N$ such that $\mu(F\setminus
F_m)\le\bover12\epsilon$.   Let $\delta>0$ be such that
$(2m+\mu F)\delta^2+2\delta\le\bover12\epsilon$.

Now suppose that $u$, $v\in L^0$ are such that
$\bar\rho_F(u,u_0)\le\delta^2$ and $\bar\rho_F(v,v_0)\le\delta^2$.   Let
$f$, $g:X\to\Bbb R$ be measurable functions such that $f^{\ssbullet}=u$
and $v^{\ssbullet}=v$.   Then

\Centerline{$\mu\{x:x\in F,\,|f(x)-f_0(x)|\ge\delta\}\le\delta$,
\quad$\mu\{x:x\in F,\,|g(x)-g_0(x)|\ge\delta\}\le\delta$,}

\noindent so that

\Centerline{$\mu\{x:x\in F,\,|f(x)-f_0(x)||g(x)-g_0(x)|\ge\delta^2\}
\le 2\delta$}

\noindent and

\Centerline{$\int_F\min(1,|f-f_0|\times|g-g_0|)
\le\delta^2\mu F+2\delta$.}

\noindent Also

\Centerline{$|f\times g-f_0\times g_0|
\le|f-f_0|\times|g-g_0| + |f_0|\times|g-g_0| + |f-f_0|\times|g_0|$,}

\noindent so that

$$\eqalignno{\bar\rho_F(u\times v,u_0\times v_0)
&=\int_F\min(1,|f\times g-f_0\times g_0|)\cr
&\le\Bover12\epsilon+\int_{F_m}\min(1,|f\times g-f_0\times g_0|)\cr
&\le\Bover12\epsilon+\int_{F_m}\min(1,|f-f_0|\times|g-g_0|
        +m|g-g_0|+m|f-f_0|)\cr
&\le\Bover12\epsilon+\int_F\min(1,|f-f_0|\times|g-g_0|)\cr
&\qquad\qquad\qquad\qquad+m\int_F\min(1,|g-g_0|)
        +m\int_F\min(1,|f-f_0|)\cr
&\le\Bover12\epsilon+\delta^2\mu F+2\delta+2m\delta^2
\le\epsilon.\cr}$$

\noindent As $F$ and $\epsilon$ are arbitrary, $\times$ is continuous at
$(u_0,v_0)$;  as $u_0$ and $v_0$ are arbitrary, $\times$ is continuous.

\medskip

{\bf (d)} Take $u\in L^0$, $F\in\Sigma$ of finite measure and
$\epsilon>0$.   Then there is a $\delta>0$ such that
$\rho_F(\bar h(v),\bar h(u))\le\epsilon$ whenever $\rho_F(v,u)\le\delta$.
\Prf\Quer\ Otherwise, we can find, for each $n\in\Bbb N$, a $v_n$ such
that $\bar\rho_F(v_n,u)\le 4^{-n}$ but
$\bar\rho_F(\bar h(v_n),\bar h(u))>\epsilon$.
Express $u$ as $f^{\ssbullet}$ and $v_n$ as
$g_n^{\ssbullet}$ where $f$, $g_n:X\to\Bbb R$ are measurable.   Set

\Centerline{$E_n=\{x:x\in F,\,|g_n(x)-f(x)|\ge 2^{-n}\}$}

\noindent for each $n$.   Then $\bar\rho_F(v_n,u)\ge 2^{-n}\mu E_n$, so
$\mu E_n\le 2^{-n}$ for each $n$, and $E=\bigcap_{n\in\Bbb
N}\bigcup_{m\ge n}E_m$ is negligible.   But
$\lim_{n\to\infty}g_n(x)=f(x)$ for every $x\in F\setminus E$, so
(because $h$ is continuous) $\lim_{n\to\infty}h(g_n(x))=h(f(x))$ for
every $x\in F\setminus E$.   Consequently (by Lebesgue's Dominated
Convergence Theorem, as always)

\Centerline{$\lim_{n\to\infty}\bar\rho_F(\bar h(v_n),\bar h(u))
=\lim_{n\to\infty}\int_F\min(1,|h(g_n(x))-h(f(x))|\mu(dx)=0$,}

\noindent which is impossible.\ \Bang\Qed

By 2A3H, $\bar h$ is continuous.
}%end of proof of 245D

\cmmnt{\medskip

\noindent{\bf Remark} I cannot say that the topology of convergence
in measure on $\eusm L^0$ is a linear space topology solely because (on
the definitions I have chosen) $\eusm L^0$ is not in general a linear
space.}



\leader{245E}{}\cmmnt{ I turn now to the principal theorem relating
the properties of the topological linear space $L^0(\mu)$ to the
classification of measure spaces in Chapter 21.

\medskip

\noindent}{\bf Theorem} Let $(X,\Sigma,\mu)$ be a measure space.   Let
$\frak T$ be the topology of convergence in measure  on
$L^0=L^0(\mu)$\cmmnt{, as described in 245A}.

(a) $(X,\Sigma,\mu)$ is semi-finite iff $\frak T$ is Hausdorff.

(b) $(X,\Sigma,\mu)$ is $\sigma$-finite iff $\frak T$ is metrizable.

(c) $(X,\Sigma,\mu)$ is  localizable iff $\frak T$ is Hausdorff and
$L^0$ is complete under $\frak T$.

\wheader{245E}{0}{0}{0}{48pt}

\proof{ I use the pseudometrics $\rho_F$ on $\eusm
L^0=\eusm L^0(\mu)$, $\bar\rho_F$ on $L^0$ described in 245A.

\medskip

{\bf (a)(i)} Suppose that $(X,\Sigma,\mu)$ is semi-finite
and that $u$, $v$ are distinct members of $L^0$.   Express them as
$f^{\ssbullet}$ and $g^{\ssbullet}$ where $f$ and $g$ are measurable
functions from $X$ to $\Bbb R$.   Then $\mu\{x:f(x)\ne g(x)\}>0$ so,
because $(X,\Sigma,\mu)$ is semi-finite, there is a set $F\in\Sigma$ of
finite measure such that $\mu\{x:x\in F,\,f(x)\ne g(x)\}>0$.   Now

\Centerline{$\bar\rho_F(u,v)=\int_F\min(|f(x)-g(x)|,1)dx>0$}

\noindent (see 122Rc).   As $u$ and $v$ are arbitrary, $\frak T$ is
Hausdorff (2A3L).

\medskip

\quad{\bf (ii)} Suppose that $\frak T$ is Hausdorff and that
$E\in\Sigma$, $\mu E>0$.    Then $u=\chi E^{\ssbullet}\ne 0$ so there is
an $F\in\Sigma$ such that $\mu F<\infty$ and $\bar\rho_F(u,0)\ne 0$,
that is, $\mu(E\cap F)>0$.
Now $E\cap F$ is a non-negligible set of finite measure included in $E$.
As $E$ is arbitrary, $(X,\Sigma,\mu)$ is semi-finite.

\medskip

{\bf (b)(i)} Suppose that $(X,\Sigma,\mu)$ is $\sigma$-finite.   Let
$\sequencen{E_n}$ be a non-decreasing sequence of sets of finite measure
covering $X$.   Set

$$\bar\rho(u,v)
=\sum_{n=0}^{\infty}\bover{\bar\rho_{E_n}(u,v)}{1+2^n\mu
E_n}$$

\noindent for $u$, $v\in L^0$.   Then $\bar\rho$ is a metric on $L^0$.
\Prf\ Because every $\bar\rho_{E_n}$ is a pseudometric, so is
$\bar\rho$.
If
$\bar\rho(u,v)=0$, express $u$ as $f^{\ssbullet}$, $v$ as
$g^{\ssbullet}$ where $f$, $g\in\eusm L^0(\mu)$;   then

\Centerline{$\int|f-g|\wedge\chi E_n=\bar\rho_{E_n}(u,v)=0$,}

\noindent so $f=g$ almost everywhere in $E_n$, for every $n$.
Because
$X=\bigcup_{n\in\Bbb N}E_n$, $f\eae g$ and $u=v$.\ \Qed\

If $F\in\Sigma$ and $\mu F<\infty$ and $\epsilon>0$, take $n$ such that
$\mu(F\setminus E_n)\le\bover12\epsilon$.   If $u$, $v\in L^0$ and
$\bar\rho(u,v)\le\epsilon/2(1+2^n\mu E_n)$, then
$\bar\rho_F(u,v)\le\epsilon$.   \Prf\ Express $u$ as $f^{\ssbullet}$,
$v=g^{\ssbullet}$ where $f$, $g\in\eusm L^0$.   Then

\Centerline{$\int|u-v|\wedge \chi E_n
=\bar\rho_{E_n}(u,v)\le(1+2^n\mu E_n)\bar\rho(u,v)
\le\Bover{\epsilon}2$,}

\noindent while

\Centerline{$\int|f-g|\wedge \chi(F\setminus E_n)
\le\mu(F\setminus E_n)\le\Bover{\epsilon}2$,}


\noindent so

\Centerline{$\bar\rho_F(u,v)=\int|f-g|\wedge\chi F
\le\int|f-g|\wedge\chi E_n+\int|f-g|\wedge\chi(F\setminus E_n)
\le\Bover{\epsilon}2+\Bover{\epsilon}2
=\epsilon$.  \Qed}

In the other direction, given $\epsilon>0$, take $n\in\Bbb N$ such that
$2^{-n}\le\bover12\epsilon$;  then $\bar\rho(u,v)\le\epsilon$ whenever
$\bar\rho_{E_n}(u,v)\le\epsilon/2(n+1)$.

These show that $\bar\rho$ defines the same topology as the
$\bar\rho_F$ (2A3Ib), so that $\frak T$, the topology defined by the
$\bar\rho_F$, is
metrizable.

\medskip

\quad{\bf (ii)} Suppose that $\frak T$ is metrizable.    Let
$\bar\rho$ be a
metric defining $\frak T$.   For each $n\in\Bbb N$ there must be a
measurable set $F_n$ of finite measure and a $\delta_n>0$ such that

\Centerline{$\bar\rho_{F_n}(u,0)\le \delta_n\Longrightarrow\bar\rho(u,0)
\le 2^{-n}$.}

\noindent Set $E=X\setminus\bigcup_{n\in\Bbb N}F_n$.   \Quer\ If $E$ is
not negligible, then $u=\chi E^{\ssbullet}\ne 0$;  because $\bar\rho$
is a metric, there is an $n\in\Bbb N$ such that $\bar\rho(u,0)>2^{-n}$;
now


\Centerline{$\mu(E\cap F_n)=\bar\rho_{F_n}(u,0)>\delta_n$.}

\noindent But $E\cap F_n=\emptyset$.\ \Bang

So $\mu E=0<\infty$.
Now $X=E\cup\bigcup_{n\in\Bbb N}F_n$ is a countable union of sets of
finite measure, so $\mu$ is $\sigma$-finite.

\medskip

{\bf (c)} By (a), either hypothesis ensures that $\mu$ is semi-finite
and that $\frak T$ is Hausdorff.

\medskip

\quad{\bf (i)} Suppose that $(X,\Sigma,\mu)$ is localizable.   Let
$\Cal F$ be a Cauchy filter on $L^0$ (2A5F).   For each measurable set
$F$ of finite measure, choose a sequence $\sequencen{A_n(F)}$ of members
of $\Cal F$ such that $\bar\rho_F(u,v)\le 4^{-n}$ for every $u$,
$v\in A_n(F)$ and
every $n$ (2A5G).   Choose $u_{Fn}\in\bigcap_{k\le n}A_n(F)$ for each
$n$;  then $\bar\rho_F(u_{F,n+1},u_{Fn})\le 2^{-n}$ for each $n$.
Express each
$u_{Fn}$ as $f_{Fn}^{\ssbullet}$ where $f_{Fn}$ is a measurable function
from $X$ to $\Bbb R$.   Then

\Centerline{$\mu\{x:x\in F,\,|f_{F,n+1}(x)-f_{Fn}(x)|\ge 2^{-n}\}
\le 2^n\bar\rho_F(u_{F,n+1},u_{Fn})\le 2^{-n}$}

\noindent for each $n$.   It follows that $\sequencen{f_{Fn}}$ must
converge almost everywhere in $F$.  \Prf\ Set

\Centerline{$H_n=\{x:x\in F,\,|f_{F,n+1}(x)-f_{Fn}(x)|\ge 2^{-n}\}$.}

\noindent Then $\mu H_n\le 2^{-n}$ for each $n$, so

\Centerline{$\mu(\bigcap_{n\in\Bbb N}\bigcup_{m\ge n}H_m)
\le\inf_{n\in\Bbb N}\sum_{m=n}^{\infty}2^{-m}
=0$.}

\noindent If $x\in F\setminus\bigcap_{n\in\Bbb N}\bigcup_{m\ge n}H_m$,
then there is some $k$ such that $x\in F\setminus\bigcup_{m\ge k}H_m$,
so that $|f_{F,m+1}(x)-f_{Fm}(x)|\le 2^{-m}$ for every $m\ge k$, and
$\sequencen{f_{Fn}(x)}$ is Cauchy, therefore convergent.\ \Qed

Set $f_F(x)=\lim_{n\to\infty}f_{Fn}(x)$ for every $x\in F$ such that the
limit is defined in $\Bbb R$, so
that $f_F$ is measurable and defined almost everywhere in $F$.

If $E$, $F$ are two sets of finite measure and $E\subseteq F$, then
$\bar\rho_E(u_{En},u_{Fn})\le 2\cdot 4^{-n}$ for each $n$.   \Prf\
$A_n(E)$ and
$A_n(F)$ both belong to $\Cal F$, so must have a point $w$ in common;
now

$$\eqalign{\bar\rho_E(u_{En},u_{Fn})
&\le\bar\rho_E(u_{En},w)+\bar\rho_E(w,u_{Fn})\cr
&\le\bar\rho_E(u_{En},w)+\bar\rho_F(w,u_{Fn})
\le 4^{-n}+4^{-n}.  \text{  \Qed}\cr}$$



\noindent Consequently

\Centerline{$\mu\{x:x\in E,\,|f_{Fn}(x)-f_{En}(x)|\ge 2^{-n}\}
\le 2^n\bar\rho_E(u_{Fn},u_{En})\le 2^{-n+1}$}

\noindent for each $n$, and $\lim_{n\to\infty}f_{Fn}-f_{En}=0$ almost
everywhere in $E$;  so that $f_E=f_F$ a.e.\ on $E$.

Consequently, if $E$ and $F$ are any two sets of finite measure,
$f_E=f_F$ a.e.\ on $E\cap F$, because both are equal almost everywhere
on $E\cap F$ to $f_{E\cup F}$.

Because $\mu$ is localizable, it follows that there is an
$f\in\eusm L^0$ such that $f=f_E$ a.e.\ on $E$
for every measurable set $E$ of finite measure (213N).   Consider
$u=f^{\ssbullet}\in L^0$.   For any set $E$ of finite measure,

\Centerline{$\bar\rho_E(u,u_{En})
=\int_E\min(1,|f(x)-f_{En}(x)|)dx
=\int_E\min(1,|f_E(x)-f_{En}(x)|)dx\to 0$}

\noindent as $n\to\infty$, using Lebesgue's Dominated Convergence
Theorem.   Now

$$\eqalign{\inf_{A\in\Cal F}\sup_{v\in A}\bar\rho_E(v,u)
&\le\inf_{n\in\Bbb N}\sup_{v\in A_{En}}\bar\rho_E(v,u)\cr
&\le\inf_{n\in\Bbb N}\sup_{v\in A_{En}}
   (\bar\rho_E(v,u_{En})+\bar\rho_E(u,u_{En}))\cr
&\le\inf_{n\in\Bbb N}(4^{-n}+\bar\rho_E(u,u_{En}))
=0.\cr}$$

\noindent As $E$ is arbitrary, $\Cal F\to u$.   As $\Cal F$ is
arbitrary, $L^0$ is complete under $\frak T$.

\medskip

\quad{\bf (ii)} Now suppose that $L^0$ is complete under $\frak T$ and
let $\Cal E$ be any family of sets in $\Sigma$.   Let $\Cal E'$ be

\Centerline{$\{\bigcup\Cal E_0:\Cal E_0$ is a finite subset of
$\Cal E\}$.}

\noindent Then the union of any two members of $\Cal E'$ belongs to
$\Cal E'$.   Let $\Cal F$ be the set

\Centerline{$\{A:A\subseteq L^0$, $A\supseteq A_E$ for
some $E\in\Cal E'\}$,}

\noindent where for $E\in\Cal E'$ I write

\Centerline{$A_E=\{\chi F^{\ssbullet}:F\in\Cal E'$, $F\supseteq E\}$.}

\noindent Then $\Cal F$ is a filter on $L^0$, because $A_E\cap
A_F=A_{E\cup F}$ for all $E$, $F\in\Cal E'$.

In fact $\Cal F$ is a Cauchy filter.   \Prf\ Let $H$ be any set of
finite measure and $\epsilon>0$.   Set $\gamma=\sup_{E\in\Cal
E'}\mu(H\cap E)$ and take $E\in\Cal E'$ such that $\mu(H\cap E)\ge
\gamma-\epsilon$.   Consider $A_E\in \Cal F$.   If $F$, $G\in\Cal E'$
and $F\supseteq E$, $G\supseteq E$ then

$$\eqalign{\bar\rho_H(\chi F^{\ssbullet},\chi G^{\ssbullet})
&=\mu(H\cap(F\symmdiff G))
=\mu(H\cap(F\cup G))-\mu(H\cap F\cap G)\cr
&\le\gamma-\mu(H\cap E)
\le\epsilon.\cr}$$

\noindent Thus $\bar\rho_H(u,v)\le\epsilon$ for all $u$, $v\in A_E$.
As $H$ and $\epsilon$ are arbitrary, $\Cal F$ is Cauchy.\ \Qed

Because $L^0$ is complete under $\frak T$, $\Cal F$ has a limit $w$ say.
Express $w$ as $h^{\ssbullet}$, where $h:X\to\Bbb R$ is measurable, and
consider $G=\{x:h(x)>\bover12\}$.

\vthsp\Quer\ If $E\in \Cal E$ and $\mu(E\setminus G)>0$, let
$F\subseteq E\setminus G$ be a set of non-zero finite measure.   Then
$\{u:\bar\rho_F(u,w)<\bover12\mu F\}$ belongs to $\Cal F$, so meets
$A_E$;   let $H\in\Cal E'$ be such that $E\subseteq H$ and
$\bar\rho_F(\chi H^{\ssbullet},w)<\bover12\mu F$.   Then

\Centerline{$\int_F\min(1,|1-h(x)|)
=\bar\rho_F(\chi H^{\ssbullet},w)<\Bover12\mu F$;}

\noindent but because $F\cap G=\emptyset$, $1-h(x)\ge\bover12$ for every
$x\in F$, so this is impossible.\ \Bang

Thus $E\setminus G$ is negligible for every $E\in\Cal E$.

Now suppose that $H\in\Sigma$ and $\mu(G\setminus H)>0$.   Then there is
an $E\in\Cal E$ such that $\mu(E\setminus H)>0$.   \Prf\ Let
$F\subseteq G\setminus H$ be a set of non-zero finite measure.   Let
$u\in A_{\emptyset}$ be such that $\bar\rho_F(u,w)<\bover12\mu F$.
Then $u$ is of the form $\chi C^{\ssbullet}$ for some $C\in\Cal E'$, and

\Centerline{$\int_F\min(1,|\chi C(x) - h(x)|)<\Bover12\mu F$.}

\noindent As $h(x)\ge\bover12$ for every $x\in F$, $\mu(C\cap F)>0$.
But $C$ is a finite union of members of $\Cal E$, so there is an
$E\in\Cal E$ such that $\mu(E\cap F)>0$, and now $\mu(E\setminus
H)>0$.\ \Qed

As $H$ is arbitrary, $G$ is an essential supremum of $\Cal E$ in
$\Sigma$.   As $\Cal E$ is arbitrary, $(X,\Sigma,\mu)$ is localizable.
}%end of proof of 245E

\leader{245F}{Alternative description of the topology of convergence in
measure} Let us return to arbitrary measure spaces $(X,\Sigma,\mu)$.

\header{245Fa}{\bf (a)} For any $F\in\Sigma$ of finite measure and
$\epsilon>0$ define $\tau_{F\epsilon}:\eusm L^0\to\coint{0,\infty}$ by

\Centerline{$\tau_{F\epsilon}(f)
=\mu^*\{x:x\in F\cap\dom f,\,|f(x)|>\epsilon\}$}

\noindent for $f\in\eusm L^0$\cmmnt{, taking $\mu^*$ to be the outer
measure defined from $\mu$ (132B)}.   If $f$, $g\in\eusm L^0$ and
$f\eae g$, then\cmmnt{

\Centerline{$\{x:x\in F\cap\dom f,\,|f(x)|>\epsilon\}
\symmdiff\{x:x\in F\cap\dom g,\,|g(x)|>\epsilon\}$}

\noindent is negligible, so} $\tau_{F\epsilon}(f)=\tau_{F\epsilon}(g)$;
accordingly we have a functional from $L^0$ to $\coint{0,\infty}$, given
by

\Centerline{$\bar\tau_{F\epsilon}(u)=\tau_{F\epsilon}(f)$}

\noindent whenever $f\in\eusm L^0$ and $u=f^{\ssbullet}\in L^0$.

\header{245Fb}{\bf (b)} \cmmnt{Now $\tau_{F\epsilon}$ is not (except
in trivial
cases) subadditive, so does not define a pseudometric on $\eusm L^0$ or
$L^0$.   But we can say that, for $f\in\eusm L^0$,

\Centerline{$\tau_F(f)\le\epsilon\min(1,\epsilon)
\Longrightarrow\tau_{F\epsilon}(f)\le\epsilon
\Longrightarrow\tau_F(f)\le\epsilon(1+\mu F)$.}

\noindent\prooflet{(The point is that if $E\subseteq\dom f$ is a
measurable
conegligible set such that $f\restr E$ is measurable, then

\Centerline{$\tau_F(f)=\int_{E\cap F}\min(f(x),1)dx$,
\quad$\tau_{F\epsilon}(f)=\mu\{x:x\in E\cap F,\,f(x)>\epsilon\}$.)}
}

This means that a set $G\subseteq\eusm L^0$ is open for the
topology of convergence in measure iff for every $f\in G$ we can find a
set $F$ of finite measure and $\epsilon$, $\delta>0$ such that

\Centerline{$\tau_{F\epsilon}(g-f)\le\delta\Longrightarrow g\in G$.}

\noindent Of course $\tau_{F\delta}(f)\ge\tau_{F\epsilon}(f)$
whenever $\delta\le\epsilon$, so we can equally say:  }%end of comment
$G\subseteq\eusm L^0$ is open for the
topology of convergence in measure iff for every $f\in G$ we can find a
set $F$ of finite measure and $\epsilon>0$ such that

\Centerline{$\tau_{F\epsilon}(g-f)\le\epsilon\Longrightarrow g\in G$.}

\noindent\cmmnt{Similarly, }$G\subseteq L^0$ is open for the
topology of convergence in measure on $L^0$ iff for every $u\in G$ we
can find a set $F$ of finite measure and $\epsilon>0$ such that

\Centerline{$\bar\tau_{F\epsilon}(v-u)\le\epsilon
\Longrightarrow v\in G$.}

\header{245Fc}{\bf (c)} It follows at once that
a sequence $\sequencen{f_n}$ in $\eusm L^0=\eusm L^0(\mu)$
converges in measure to $f\in\eusm L^0$ iff

\Centerline{$\lim_{n\to\infty}
  \mu^*\{x:x\in F\cap\dom f\cap\dom f_n,\,|f_n(x)-f(x)|>\epsilon\}
=0$}

\noindent whenever $F\in\Sigma$, $\mu F<\infty$ and $\epsilon>0$.
Similarly, a sequence $\sequencen{u_n}$ in $L^0$ converges in measure to
$u$ iff
$\lim_{n\to\infty}\bar\tau_{F\epsilon}(u-u_n)=0$ whenever $\mu
F<\infty$ and $\epsilon>0$.

\header{245Fd}{\bf (d)} In particular, if $(X,\Sigma,\mu)$ is totally
finite, $\sequencen{f_n}\to f$ in $\eusm L^0$ iff

\Centerline{$\lim_{n\to\infty}
  \mu^*\{x:x\in \dom f\cap\dom f_n,\,|f(x)-f_n(x)|>\epsilon\}
=0$}

\noindent for every $\epsilon>0$, and $\sequencen{u_n}\to u$ in $L^0$
iff

\Centerline{$\lim_{n\to\infty}\bar\tau_{X\epsilon}(u-u_n)=0$}

\noindent for every $\epsilon>0$.

\leader{245G}{Embedding $L^p$ in $L^0$:  Proposition} Let
$(X,\Sigma,\mu)$ be any measure space.   Then for any $p\in[1,\infty]$,
the embedding of $L^p=L^p(\mu)$ in $L^0=L^0(\mu)$ is continuous for the
norm topology of $L^p$ and the topology of convergence in measure on
$L^0$.

\proof{ Suppose that $u$, $v\in L^p$ and that $\mu F<\infty$.   Then
$(\chi F)^{\ssbullet}$ belongs to $L^q$, where $q=p/(p-1)$ (taking $q=1$
if $p=\infty$, $q=\infty$ if $p=1$ as usual), and

\Centerline{$\bar\rho_F(u,v)\le\int|u-v|\times(\chi F)^{\ssbullet}
\le\|u-v\|_p\|\chi F^{\ssbullet}\|_q$}

\noindent (244Eb). By 2A3H, this is enough to ensure that the
embedding $L^p\embedsinto L^0$ is continuous.
}%end of proof of 245G

\leader{245H}{}\cmmnt{ The case of $L^1$ is so important that I go
farther with it.

\medskip

\noindent}{\bf Proposition} Let $(X,\Sigma,\mu)$ be a measure space.

(a)(i) If $f\in\eusm L^1=\eusm L^1(\mu)$ and $\epsilon>0$, there are a
$\delta>0$ and a set $F\in\Sigma$ of finite measure such that
$\int|f-g|\le\epsilon$ whenever $g\in\eusm L^1$,
$\int|g|\le\int|f|+\delta$ and $\rho_F(f,g)\le\delta$.

\quad(ii) For any sequence $\sequencen{f_n}$ in $\eusm L^1$ and any
$f\in\eusm L^1$, $\lim_{n\to\infty}\int|f-f_n|=0$ iff
$\sequencen{f_n}\to f$ in measure and
$\limsup_{n\to\infty}\int|f_n|\le\int|f|$.

(b)(i) If $u\in L^1=L^1(\mu)$ and $\epsilon>0$, there are a $\delta>0$
and a set $F\in\Sigma$ of finite measure such that
$\|u-v\|_1\le\epsilon$ whenever $v\in L^1$, $\|v\|_1\le\|u\|_1+\delta$
and $\bar\rho_F(u,v)\le\delta$.

\quad(ii) For any sequence $\sequencen{u_n}$ in $L^1$ and any $u\in
L^1$, $\sequencen{u_n}\to u$ for $\|\,\|_1$ iff $\sequencen{u_n}\to u$
in measure and $\limsup_{n\to\infty}\|u_n\|_1\le\|u\|_1$.

\proof{{\bf (a)(i)}
We know that there are a set $F$ of finite measure and an $\eta>0$ such
that $\int_E|f|\le\bover15\epsilon$ whenever $\mu(E\cap F)\le\eta$
(225A).   Take $\delta>0$ such that $\delta(\epsilon+5\mu
F)\le\epsilon\eta$ and $\delta\le\bover15\epsilon$.
Then if $\int|g|\le\int|f|+\delta$ and $\rho_F(f,g)\le\delta$, let
$G\subseteq\dom f\cap\dom g$ be a conegligible measurable set such that
$f\restr G$ and $g\restr G$ are both measurable.   Set

\Centerline{$E=\{x:x\in F\cap G,\,
|f(x)-g(x)|\ge\Bover{\epsilon}{\epsilon+5\mu F}\}$;}

\noindent then

\Centerline{$\delta\ge\rho_F(f,g)
\ge\Bover{\epsilon}{\epsilon+5\mu F}\mu E$,}

\noindent so $\mu E\le\eta$.   Set $H=F\setminus E$, so that
$\mu(F\setminus H)\le\eta$ and $\int_{X\setminus
H}|f|\le\bover15{\epsilon}$.   On the other hand, for almost every $x\in
H$, $|f(x)-g(x)|\le\bover{\epsilon}{\epsilon+5\mu F}$, so
$\int_H|f-g|\le\bover15\epsilon$ and

\Centerline{$\int_{H}|g|\ge\int_{H}|f|-\Bover15\epsilon
\ge\int|f|-\int_{X\setminus H}|f|-\Bover15\epsilon
\ge\int|f|-\Bover25\epsilon$.}

\noindent Since $\int|g|\le\int|f|+\bover15\epsilon$, $\int_{X\setminus
H}|g|\le\bover35{\epsilon}$.   Now this means that

\Centerline{$\int|g-f|
\le\int_{X\setminus H}|g|+\int_{X\setminus H}|f|+\int_H|g-f|
\le\Bover35\epsilon+\Bover15\epsilon+\Bover15\epsilon
=\epsilon$,}

\noindent as required.

\medskip

\quad{\bf (ii)} If
$\lim_{n\to\infty}\int|f-f_n|=0$, that is,
$\lim_{n\to\infty}f_n^{\ssbullet}=f^{\ssbullet}$ in $L^1$, then by 245G
we must have $\sequencen{f_n^{\ssbullet}}\to f^{\ssbullet}$ in $L^0$,
that is, $\sequencen{f_n}\to f$ for the topology of convergence in
measure.   Also, of course, $\lim_{n\to\infty}\int|f_n|=\int|f|$.

In the other direction, if $\limsup_{n\to\infty}\int|f_n|\le\int|f|$ and
$\sequencen{f_n}\to f$ for the topology of convergence in measure, then
whenever $\delta>0$ and $\mu F<\infty$ there must be an $m\in\Bbb N$
such that $\int|f_n|\le\int|f|+\delta$, $\rho_F(f,f_n)\le\delta$ for
every $n\ge m$;  so (i) tells us that $\lim_{n\to\infty}\int|f_n-f|=0$.

\medskip

{\bf (b)} This now follows immediately if we express $u$ as
$f^{\ssbullet}$, $v$ as $g^{\ssbullet}$ and $u_n$ as $f_n^{\ssbullet}$.
}%end of proof of 245H

\cmmnt{
\vleader{48pt}{245I}{Remarks (a)} I think the phenomenon here is so important
that it is worth looking at some elementary examples.

(i) If $\mu$ is
counting measure on $\Bbb N$, and we set $f_n(n)=1$, $f_n(i)=0$ for
$i\ne n$, then $\sequencen{f_n}\to 0$ in measure, while $\int|f_n|=1$
for every $n$.

(ii) If $\mu$ is Lebesgue measure on $[0,1]$, and we
set $f_n(x)=2^n$ for $0<x\le 2^{-n}$, $0$ for other $x$, then again
$\sequencen{f_n}\to 0$ in measure, while $\int|f_n|=1$ for every $n$.

(iii) In 245Cc we have another sequence $\sequencen{f_n}$ converging to
$0$ in measure, while $\int|f_n|=1$ for every $n$.   In all these cases
(as required by Fatou's Lemma, at least in (i) and (ii)) we have
$\int|f|\le\liminf_{n\to\infty}\int|f_n|$.   (The next proposition shows
that this applies to any sequence which is convergent in measure.)

The common feature of these examples is the way in which somehow the
$f_n$ escape to infinity, either laterally (in (i)) or vertically (in
(iii)) or both (in (ii)).   Note that in all three examples we can set
$f'_n=2^nf_n$ to obtain a sequence still converging to $0$ in measure,
but with $\lim_{n\to\infty}\int|f'_n|=\infty$.

\spheader 245Ib In 245H, I have used the explicit formulations
`$\lim_{n\to\infty}\int|f_n-f|=0$' (for sequences of functions),
`$\sequencen{u_n}\to u$ for $\|\,\|_1$' (for sequences in $L^1$).
These are often expressed by saying that $\sequencen{f_n}$,
$\sequencen{u_n}$ are {\bf convergent in mean} to $f$, $u$ respectively.
}%end of comment

\leader{245J}{}\cmmnt{ For semi-finite spaces we have a further
relationship.

\medskip

\noindent}{\bf Proposition} Let $(X,\Sigma,\mu)$ be a semi-finite
measure space.   Write $\eusm L^0=\eusm L^0(\mu)$, etc.

(a)(i) For any $a\ge 0$, the set $\{f:f\in\eusm L^1,\,\int|f|\le a\}$ is
closed in $\eusm L^0$ for the topology of convergence in measure.

\quad(ii) If $\sequencen{f_n}$ is a sequence in $\eusm L^1$ which is
convergent in measure to $f\in\eusm L^0$, and
$\liminf_{n\to\infty}\int|f_n|<\infty$, then $f$ is integrable and
$\int|f|\le\liminf_{n\to\infty}\int|f_n|$.

(b)(i) For any $a\ge 0$, the set  $\{u:u\in L^1,\,\|u\|_1\le a\}$ is
closed in $L^0$ for the topology of convergence in measure.

\quad(ii) If $\sequencen{u_n}$ is a sequence in $L^1$ which is
convergent in measure to $u\in L^0$, and
$\liminf_{n\to\infty}\|u_n\|_1<\infty$, then $u\in L^1$ and
$\|u\|_1\le\liminf_{n\to\infty}\|u_n\|_1$.

\proof{{\bf (a)(i)} Set $A=\{f:f\in\eusm L^1,\,\int|f|\le a\}$, and let
$g$ be any member of the closure of $A$ in $\eusm L^0$.   Let $h$ be any
simple function such that $0\le h\leae|g|$, and $\epsilon>0$.   If
$h=0$ then of course $\int h\le a$.   Otherwise, setting
$F=\{x:h(x)>0\}$ and $M=\sup_{x\in X}h(x)$, there is an $f\in A$ such that
$\mu^*\{x:x\in F\cap\dom f\cap\dom g,\,
|f(x)-g(x)|\ge\epsilon\}\le\epsilon$ (245F);  let
$E\supseteq\{x:x\in F\cap\dom f\cap\dom g,\,|f(x)-g(x)|\ge\epsilon\}$
be a measurable set of measure
at most $\epsilon$.   Then $h\leae|f|+\epsilon\chi F+M\chi E$, so
$\int h\le a+\epsilon(M+\mu F)$.   As $\epsilon$ is arbitrary,
$\int h\le a$.   But we are supposing that $\mu$ is semi-finite, so
this is enough to ensure that $g$ is integrable and that $\int|g|\le a$
(213B), that is, that $g\in A$.   As $g$ is arbitrary, $A$ is closed.

\medskip

\quad{\bf (ii)}
Now if $\sequencen{f_n}$ is convergent in measure to $f$, and
$\liminf_{n\to\infty}\int|f_n|=a$, then for any $\epsilon>0$ there is a
subsequence $\sequence{k}{f_{n(k)}}$ such that $\int|f_{n(k)}|\le
a+\epsilon$ for every $k$;  since $\sequence{k}{f_{n(k)}}$ still
converges in measure to $f$, $\int|f|\le a+\epsilon$.   As $\epsilon$ is
arbitrary, $\int|f|\le a$.

\medskip

{\bf (b)} As in 245H, this is just a translation of part (a) into the
language of $L^1$ and $L^0$.
}%end of proof of 245J


\leader{245K}{}\cmmnt{ For $\sigma$-finite measure spaces, the
topology of
convergence in measure on $L^0$ is metrizable, so can be described
effectively in terms of convergent sequences;  it is therefore important
that we have, in this case, a sharp characterisation of sequential
convergence in measure.

\medskip

\noindent}{\bf Proposition}
Let $(X,\Sigma,\mu)$ be a $\sigma$-finite measure space.   Then

(a) a sequence $\sequencen{f_n}$ in $\eusm L^0$ converges in measure to
$f\in\eusm L^0$ iff every subsequence
of $\sequencen{f_n}$ has a sub-subsequence converging to $f$ almost
everywhere;

(b) a sequence $\sequencen{u_n}$ in $L^0$ converges in measure to $u\in
L^0$ iff every subsequence
of $\sequencen{u_n}$ has a sub-subsequence which order*-converges to $u$.

\proof{{\bf (a)(i)} Suppose that
$\sequencen{f_n}\to f$, that is, that
$\lim_{n\to\infty}\int|f-f_n|\wedge \chi F=0$ for every set $F$ of
finite measure.   Let $\sequencen{E_k}$ be a non-decreasing sequence of
sets
of finite measure covering $X$, and let
$\sequence{k}{n(k)}$ be a strictly increasing sequence in $\Bbb N$ such
that $\int|f-f_{n(k)}|\wedge\chi E_k\le 4^{-k}$ for every
$k\in\Bbb N$.   Then
$\sum_{k=0}^{\infty}|f-f_{n(k)}|\wedge\chi E_k$ is finite almost everywhere
(242E);   but $\lim_{k\to\infty}f_{n(k)}(x)=f(x)$ whenever
$\sum_{k=0}^{\infty}\min(1,|f(x)-f_{n(k)}(x)|)<\infty$, so
$\sequence{k}{f_{n(k)}}\to f$ a.e.

\medskip

\quad{\bf (ii)} The same applies to every
subsequence of $\sequencen{f_n}$, so that every subsequence of
$\sequencen{f_n}$ has a sub-subsequence converging to $f$ almost
everywhere.

\medskip

\quad{\bf (iii)} Now suppose that $\sequencen{f_n}$ does
not converge to $f$.   Then there is an open set $U$ containing $f$ such
that $\{n:f_n\notin U\}$ is infinite, that is, $\sequencen{f_n}$ has a
subsequence $\sequencen{f'_n}$ with $f'_n\notin U$ for every $n$.   But
now no sub-subsequence of $\sequencen{f'_n}$ converges to $f$ in
measure, so no such sub-subsequence can converge almost everywhere, by
245Ca.

\medskip

{\bf (b)} This follows immediately from (a) if we express $u$ as
$f^{\ssbullet}$, $u_n$ as $f_n^{\ssbullet}$.
}%end of proof of 245K


\leader{245L}{Corollary} Let $(X,\Sigma,\mu)$ be a $\sigma$-finite
measure space.

(a) A subset $A$ of $\eusm L^0=\eusm L^0(\mu)$ is closed for the
topology of convergence in measure iff $f\in A$ whenever $f\in\eusm L^0$
and there is a sequence $\sequencen{f_n}$ in $A$ such that
$f\eae\lim_{n\to\infty}f_n$.

(b) A subset $A$ of $L^0=L^0(\mu)$ is closed for the topology of
convergence in measure iff $u\in A$ whenever $u\in L^0$ and there is a
sequence $\sequencen{u_n}$ in $A$ order*-converging to $u$.

\proof{{\bf (a)(i)} If $A$ is closed for the topology of convergence in
measure, and $\sequencen{f_n}$ is a sequence in $A$ converging to $f$
almost everywhere, then $\sequencen{f_n}$ converges to $f$ in measure,
so surely $f\in A$ (since otherwise all but finitely many of the $f_n$
would have to belong to the open set $\eusm L^0\setminus A$).

\medskip

\quad{\bf (ii)} If $A$ is not closed, there is an
$f\in\overline{A}\setminus A$.   The topology can be defined by a metric
$\rho$ (245Eb), and we can choose a sequence $\sequencen{f_n}$ in $A$
such that $\rho(f_n,f)\le 2^{-n}$ for every $n$, so that
$\sequencen{f_n}\to f$ in measure.   By 245K, $\sequencen{f_n}$ has a
subsequence $\sequencen{f'_n}$ converging a.e.\ to $f$, and this
witnesses that $A$ fails to satisfy the condition.

\medskip

{\bf (b)} This follows immediately, because $A\subseteq L^0$ is closed
iff $\{f:f^{\ssbullet}\in A\}$ is closed in $\eusm L^0$.
}%end of proof of 245L

\leader{245M}{Complex $L^0$} In 241J I briefly discussed the
adaptations needed to construct the complex linear space $L^0_{\Bbb C}$.
The formulae of 245A may be used unchanged to
define topologies of convergence in measure on $\eusm L^0_{\Bbb C}$ and
$L^0_{\Bbb C}$.   I think that every word of 245B-245L still applies
if we replace each $L^0$ or $\eusm L^0$ with $L^0_{\Bbb C}$ or $\eusm
L^0_{\Bbb C}$.   \cmmnt{Alternatively, to relate the `real' and
`complex' forms of 245E, for instance, we can observe that because

$$\eqalign{\max(\rho_F(\Real(u),\Real(v)),\rho_F(\Imag(u),\Imag(v)))
&\le\rho_F(u,v)\cr
&\le\rho_F(\Real(u),\Real(v))+\rho_F(\Imag(u),\Imag(v))\cr}$$

\noindent for all $u$, $v\in L^0$ and all sets $F$ of finite measure,
$L^0_{\Bbb C}$ can be identified, as uniform space, with $L^0\times
L^0$, so is Hausdorff, or metrizable, or complete iff $L^0$ is.}

\exercises{
\leader{245X}{Basic exercises $\pmb{>}$(a)}
%\spheader 245Xa
Let $X$ be any set, and $\mu$ counting measure on $X$.   Show that the
topology of convergence in measure on $\eusm L^0(\mu)=\Bbb R^X$ is just
the product topology on $\Bbb R^X$ regarded as a product of copies of
$\Bbb R$.
%245A

\sqheader 245Xb
Let $(X,\Sigma,\mu)$ be any measure space, and $(X,\hat\Sigma,\hat\mu)$
its completion.   Show that the topologies of convergence in measure on
$\eusm L^0(\mu)=\eusm L^0(\hat\mu)$ (241Xb), corresponding to the
families $\{\rho_F:F\in\Sigma,\,\mu F<\infty\}$,
$\{\rho_F:F\in\hat\Sigma,\,\hat\mu F<\infty\}$ are the same.
%245A

\sqheader 245Xc Let $(X,\Sigma,\mu)$ be any measure space;  set
$L^0=L^0(\mu)$.   Let $u$, $u_n\in L^0$ for
$n\in\Bbb N$.   Show that the following are equiveridical:

\quad(i) $\sequencen{u_n}$ order*-converges to $u$ in the sense of
245C;

\quad (ii) there are measurable functions $f$, $f_n:X\to\Bbb R$
such that $f^{\ssbullet}=u$, $f_n^{\ssbullet}=u_n$ for every
$n\in\Bbb N$, and $f(x)=\lim_{n\to\infty}f_n(x)$ for every $x\in X$;

\quad (iii)
$u=\inf_{n\in\Bbb N}\sup_{m\ge n}u_m=\sup_{n\in\Bbb N}\inf_{m\ge n}u_m$,
the infima and suprema being taken in $L^0$;

\quad (iv) $\inf_{n\in\Bbb N}\sup_{m\ge n}|u-u_m|=0$ in $L^0$;

\quad (v) there is a non-increasing sequence $\sequencen{v_n}$ in $L^0$
such that $\inf_{n\in\Bbb N}v_n=0$ in $L^0$ and $u-v_n\le u_n\le u+v_n$
for every $n\in\Bbb N$;

\quad (vi) there are sequences $\sequencen{v_n}$, $\sequencen{w_n}$ in
$L^0$ such that $\sequencen{v_n}$ is non-decreasing, $\sequencen{w_n}$
is non-increasing, $\sup_{n\in\Bbb N}v_n=u=\inf_{n\in\Bbb N}w_n$ and
$v_n\le u_n\le w_n$ for every $n\in\Bbb N$.
%245C

\spheader 245Xd  Let $(X,\Sigma,\mu)$ be a semi-finite measure space.
Show that a sequence $\sequencen{u_n}$ in $L^0=L^0(\mu)$ is
order*-convergent to $u\in L^0$ iff $\{|u_n|:n\in\Bbb N\}$ is bounded
above in
$L^0$ and $\sequencen{\sup_{m\ge n}|u_m-u|}\to 0$ for the topology of
convergence in measure.
%245C

\spheader 245Xe
Write out proofs that $L^0(\mu)$ is complete (as linear topological
space) adapted to the special
cases (i) $\mu X=1$ (ii) $\mu$ is $\sigma$-finite, taking advantage of
any simplifications you can find.
%245E

\spheader 245Xf Let $(X,\Sigma,\mu)$ be a measure space and $r\ge 1$;
let $h:\Bbb R^r\to\Bbb R$ be a continuous function.   (i) Suppose that
for $1\le k\le r$ we are given a sequence $\sequencen{f_{kn}}$ in
$\eusm L^0={\eusm L}^0(\mu)$ converging in measure to
$f_k\in{\eusm L}^0$.      Show that
$\sequencen{h(f_{1n},\ldots,f_{kn})}$ converges in measure to
$h(f_1,\ldots,f_k)$.   (ii) Generally, show that
$(f_1,\ldots,f_r)\mapsto h(f_1,\ldots,f_r):(\eusm L^0)^r\to\eusm L^0$ is
continuous for the topology of convergence in measure.   (iii) Show that
the corresponding function $\bar h:(L^0)^r\to L^0$ (241Xh) is
continuous for the topology of convergence in measure.
%245F

\spheader 245Xg Let $(X,\Sigma,\mu)$ be a measure space and
$u\in L^1(\mu)$.   Show that
$v\mapsto\int u\times v:L^{\infty}\to\Bbb R$ is continuous for the
topology of convergence in measure on the unit ball of $L^{\infty}$, but
not, as a rule, on the whole of $L^{\infty}$.
%245G

\spheader 245Xh Let $(X,\Sigma,\mu)$ be a measure space and $v$ a
non-negative member of $L^1=L^1(\mu)$.   Show that on the set
$A=\{u:u\in L^1,\,|u|\le v\}$ the subspace topologies (2A3C) induced
by the norm topology of $L^1$ and the topology of convergence in measure
are the same.   \Hint{given $\epsilon>0$, take $F\in\Sigma$ of finite
measure and $M\ge 0$ such that
$\int(|v|-M\chi F^{\ssbullet})^+\le\epsilon$.   Show that
$\|u-u'\|_1\le\epsilon+M\bar\rho_F(u,u')$ for all $u$, $u'\in A$.}
%245H

\spheader 245Xi Let $(X,\Sigma,\mu)$ be a measure space and $\Cal F$ a
filter on $L^1=L^1(\mu)$ which is convergent, for the topology of
convergence in measure, to $u\in L^1$.   Show that $\Cal F\to u$ for the
norm topology of $L^1$ iff
$\inf_{A\in\Cal F}\sup_{v\in A}\|v\|_1\le\|u\|_1$.
%245H

\spheader 245Xj Let $(X,\Sigma,\mu)$ be a measure space and
$p\in\coint{1,\infty}$.   Suppose that $\sequencen{u_n}$ is a sequence in
$L^p(\mu)$ which converges for $\|\,\|_p$ to $u\in L^p(\mu)$.
Show that $\sequencen{|u_n|^p}\to|u|^p$ for $\|\,\|_1$.
\Hint{245G, 245Dd, 245H.}
%245H added 2002

\sqheader 245Xk Let $(X,\Sigma,\mu)$ be a semi-finite measure space and
$p\in[1,\infty]$, $a\ge 0$.   Show that
$\{u:u\in L^p(\mu),\,\|u\|_p\le a\}$ is closed in $L^0(\mu)$ for the
topology of convergence in measure.
%245J

\spheader 245Xl Let $(X,\Sigma,\mu)$ be a measure space, and
$\sequencen{u_n}$ a sequence in $L^p=L^p(\mu)$, where $1\le p<\infty$.
Let $u\in L^p$.   Show that the following are equiveridical:  (i)
$u=\lim_{n\to\infty}u_n$ for the norm
topology of $L^p$ (ii) $\sequencen{u_n}\to u$ for the topology of
convergence in measure and $\lim_{n\to\infty}\|u_n\|_p=\|u\|_p$
(iii) $\sequencen{u_n}\to u$ for the topology of
convergence in measure and $\limsup_{n\to\infty}\|u_n\|_p\le\|u\|_p$.
%245Xk, 245J

\spheader 245Xm Let $X$ be a set and $\mu$, $\nu$ two measures on $X$
with the same measurable sets and the same negligible sets.   (i) Show
that $\eusm L^0(\mu)=\eusm L^0(\nu)$ and $L^0(\mu)=L^0(\nu)$.   (ii)
Show that if both $\mu$ and $\nu$ are semi-finite,
then they define the same topology of convergence in measure on
$\eusm L^0$ and $L^0$.
\Hint{use 215A to show that if $\mu E<\infty$ then
$\mu E=\sup\{\mu F:F\subseteq E,\,\nu F<\infty\}$.}
%245L

\leader{245Y}{Further exercises (a)}
%\spheader 245Ya
Let $(X,\Sigma,\mu)$ be a measure space and give $\Sigma$ the topology
described in 232Ya.   Show that $\chi:\Sigma\to\eusm L^0(\mu)$ is a
homeomorphism between $\Sigma$ and its image $\chi[\Sigma]$ in
$\eusm L^0$, if $\eusm L^0$ is given the topology of convergence in
measure and $\chi[\Sigma]$ the subspace topology.
%245A

\spheader 245Yb
Let $(X,\Sigma,\mu)$ be a measure space and $Y$
any subset of $X$;  let $\mu_Y$ be the subspace measure on $Y$.   Let
$T:L^0(\mu)\to L^0(\mu_Y)$ be the canonical map defined by
setting $T(f^{\ssbullet})=(f\restrp Y)^{\ssbullet}$ for every
$f\in\eusm L^0(\mu)$ (241Yg).
Show that $T$ is continuous for the topologies of
convergence in measure on $L^0(\mu)$ and $L^0(\mu_Y)$.
%245A

\spheader 245Yc Let $(X,\Sigma,\mu)$ be a measure space, and
$\tilde\mu$ the c.l.d.\ version of $\mu$.   Show that the map
$T:L^0(\mu)\to L^0(\tilde\mu)$ induced by the inclusion
$\eusm L^0(\mu)\subseteq\eusm L^0(\tilde\mu)$ (241Yf) is continuous for
the topologies of convergence in measure.
%245A

\spheader 245Yd Let $(X,\Sigma,\mu)$ be a measure space, and
give $L^0=L^0(\mu)$ the topology of convergence in measure.   Let
$A\subseteq L^0$ be a non-empty downwards-directed set, and suppose that
$\inf A=0$ in $L^0$.   (i) Let
$F\in\Sigma$ be any set of finite measure, and define $\bar\tau_F$ as
in 245A;  show that $\inf_{u\in A}\bar\tau_F(u)=0$.   \Hint{set
$\gamma=\inf_{u\in A}\bar\tau_F(u)$;  find a non-increasing sequence
$\sequencen{u_n}$ in $A$ such that
$\lim_{n\to\infty}\bar\tau_F(u_n)=\gamma$;  set $v=(\chi
F)^{\ssbullet}\wedge\inf_{n\in\Bbb N}u_n$ and show that $u\wedge v=v$
for every $u\in A$, so that $v=0$.}   (ii) Show that if $U$ is any open
set containing $0$, there is a $u\in A$ such that $v\in U$ whenever
$0\le v\le u$.
%245A

\spheader 245Ye Let $(X,\Sigma,\mu)$ be a measure space.   (i) Show that
for $u\in L^0=L^0(\mu)$ we may define $\psi_a(u)$, for $a\ge 0$, by
setting $\psi_a(u)=\mu\{x:|f(x)|\ge a\}$ whenever $f:X\to\Bbb R$ is a
measurable function and $f^{\ssbullet}=u$.   (ii) Define $\rho:L^0\times
L^0\to[0,1]$ by setting
$\rho(u,v)=\min(\{1\}\cup\{a:a\ge 0,\,\psi_a(u-v)\le a\}$.   Show that
$\rho$ is a metric on $L^0$, that $L^0$ is complete under $\rho$, and that $+$, $-$, $\wedge$, $\vee:L^0\times L^0\to L^0$ are continuous for
$\rho$.   (iii) Show that $c\mapsto cu:\Bbb R\to L^0$ is continuous for
every $u\in L^0$ iff $(X,\Sigma,\mu)$ is totally finite, and that in
this case $\rho$ defines the topology of convergence in measure on
$L^0$.
%245D

\spheader 245Yf Let $(X,\Sigma,\mu)$ be a localizable measure space and
$A\subseteq L^0=L^0(\mu)$ a non-empty upwards-directed set which is
bounded in the linear topological space sense (i.e., such that for every
neighbourhood $U$ of $0$ in $L^0$ there is a $k\in\Bbb N$ such that
$A\subseteq kU$).   Show that $A$ is bounded above in $L^0$, and that
its supremum belongs to its closure.
%245E

\spheader 245Yg Let $(X,\Sigma,\mu)$ be a measure space,
$p\in\coint{1,\infty}$ and $v$ a
non-negative member of $L^p=L^p(\mu)$.   Show that on the set
$A=\{u:u\in L^p,\,|u|\le v\}$ the subspace topologies  induced by the
norm topology of $L^p$ and the topology of convergence in measure are
the same.
%245F

\spheader 245Yh Let $S$ be the set of all sequences $s:\Bbb N\to\Bbb N$
such that $\lim_{n\to\infty}s(n)=\infty$.   For every $s\in S$, let
$(X_s,\Sigma_s,\mu_s)$ be $[0,1]$ with Lebesgue measure, and let
$(X,\Sigma,\mu)$ be the direct sum of
$\family{s}{S}{(X_s,\Sigma_s,\mu_s)}$ (214L).   For $s\in S$,
$t\in[0,1]$, $n\in\Bbb N$ set $h_n(s,t)=f_{s(n)}(t)$, where
$\sequencen{f_n}$ is the sequence of 245Cc.
Show that $\sequencen{h_n}\to 0$ for the topology of
convergence in measure on $\eusm L^0(\mu)$, but that
$\sequencen{h_n}$ has no subsequence which is convergent to $0$ almost
everywhere.
%245K

\spheader 245Yi Let $X$ be a set, and suppose we are given a relation
$\rightharpoonup$ between sequences in $X$ and members of $X$ such that
($\alpha$) if $x_n=x$ for every $n$ then $\sequencen{x_n}\rightharpoonup
x$ ($\beta$) $\sequencen{x'_n}\rightharpoonup x$ whenever
$\sequencen{x_n}\rightharpoonup x$ and $\sequencen{x'_n}$ is a
subsequence of $\sequencen{x_n}$.
Show that we have a topology $\frak T$ on $X$ defined by saying that a
subset $G$ of $X$ belongs to $\frak T$ iff whenever $\sequencen{x_n}$ is
a sequence in $X$ and $\sequencen{x_n}\rightharpoonup x\in G$ then some
$x_n$ belongs to $G$.   Show that a sequence $\sequencen{x_n}$ in $X$ is
$\frak T$-convergent to $x$ iff every subsequence of $\sequencen{x_n}$
has a sub-subsequence $\sequencen{x''_n}$ such that
$\sequencen{x''_n}\halfarrow x$.

\spheader 245Yj Let $\mu$ be Lebesgue measure on $\BbbR^r$.   Show that
$L^0(\mu)$ is separable for the topology of convergence in measure.
\Hint{244I.}
}%end of exercises

\endnotes{
\Notesheader{245} In this section I am inviting you to regard the
topology of (local) convergence in measure as the standard topology on
$L^0$, just as the norms define the standard topologies on $L^p$ spaces
for $p\ge 1$.   The definition I have chosen is designed to make
addition and scalar multiplication and the operations $\vee$, $\wedge$
and $\times$ continuous (245D);  see also 245Xf.   From the point of
view of functional analysis these properties are more important than
metrizability or even completeness.

Just as the algebraic and order structure of $L^0$ can be described in
terms of the general theory of Riesz spaces, the more advanced results
241G and 245E also have
interpretations in the general theory.   It is not an
accident that (for semi-finite measure spaces) $L^0$ is Dedekind
complete iff it is complete as uniform space;  you may find the relevant
generalizations in 23K and 24E of {\smc Fremlin 74}.   Of course it is
exactly because the two kinds of completeness are interrelated that I
feel it necessary to use the phrase `Dedekind completeness' to
distinguish this particular kind of order-completeness from the more
familiar uniformity-completeness described in 2A5F.

The usefulness of the topology of convergence in measure derives in
large part from 245G-245J and the $L^p$ versions 245Xk and 245Xl.
Some of the ideas here can be related to a question arising out of the
basic convergence theorems.   If $\sequencen{f_n}$ is a sequence of
integrable functions converging (pointwise) to a function $f$, in what
ways can $\int f$ fail to be $\lim_{n\to\infty}\int f_n$?   In the
language of this section, this translates into:  if we have a sequence
(or filter) in $L^1$ converging for the topology of convergence in
measure, in what ways can it fail to converge for the norm topology of
$L^1$?   The first answer is Lebesgue's Dominated Convergence Theorem:
this cannot happen if the sequence is dominated, that is, lies within
some set of the form $\{u:|u|\le v\}$ where $v\in L^1$.   (See 245Xh and
245Yg.)   I will return to this in the next section.   For the moment,
though, 245H tells us that if $\sequencen{u_n}$ converges in measure to
$u\in L^1$, but not for the topology of $L^1$, it is because
$\limsup_{n\to\infty}\|u_n\|_1$ is too big;  some of its weight is being
lost at infinity, as in the examples of 245I.   If $\sequencen{u_n}$
actually order*-converges to $u$, then Fatou's Lemma tells us that
$\liminf_{n\to\infty}\|u_n\|_1\ge\|u\|_1$, that is, that the limit
cannot have greater weight (as measured by $\|\,\|_1$) than the sequence
provides.   245J and 245Xk are generalizations of this to convergence in
measure.   If you want a generalization of B.Levi's theorem, then
242Yf remains the best expression in the language of this chapter;  but
245Yf is a version in terms of the concepts of the present section.

In the case of $\sigma$-finite spaces, we have an alternative
description of the topology of convergence in measure (245L) which makes
no use of any of the functionals or pseudo-metrics in 245A.   This can
be expressed, at least in the context of $L^0$, in terms of a standard
result from general topology (245Yi).   You will see that that result
gives a recipe for a topology on $L^0$ which could be applied in any
measure space.   What is remarkable is that for $\sigma$-finite spaces
we get a linear space topology.
}%end of notes


\discrpage
