\frfilename{mt362.tex}
\versiondate{31.12.10}
\copyrightdate{1996}

\def\chaptername{Function spaces}
\def\sectionname{{$S^{\sim}$}}

\newsection{362}

The next stage in our journey is the systematic investigation of linear
functionals on spaces $S=S(\frak A)$.   We already know that these
correspond to additive real-valued functionals on the algebra $\frak A$
(361F).   My purpose here is to show how the structure of the Riesz
space dual $S^{\sim}$ and its bands is related to the classes of
additive functionals introduced in \S\S326-327.   The first step is just
to check the identification of the linear and order structures of
$S^{\sim}$ and the space $M$ of bounded finitely additive functionals
(362A);   all the ideas needed for this have already been set out, and
the basic properties of $S^{\sim}$ are covered by the general results in
\S356.   Next, we need to be able to describe the operations on $M$
corresponding to the Riesz space operations $|\,\,|$, $\vee$, $\wedge$
on $S^{\sim}$, and the band projections from $S^{\sim}$
onto $S^{\sim}_c$ and $S^{\times}$;  these are dealt with in 362B, with
a supplementary remark in 362D.   In
the case of measure algebras, we have some further important bands which
present themselves in $M$, rather than in $S^{\sim}$, and which are
treated in 362C.   Since all these spaces are $L$-spaces, it is worth
taking a moment to identify their uniformly integrable subsets;  I do
this in 362E.

While some of the ideas here have interesting extensions to the case in
which $\frak A$ is a Boolean ring without identity, these can I think be
left to one side;  the work of this section will be done on the
assumption that every $\frak A$ is a Boolean algebra.

\leader{362A}{Theorem} Let $\frak A$ be a Boolean algebra.   Write $S$
for $S(\frak A)$.

(a) The partially ordered linear space of all finitely additive
real-valued functionals on $\frak A$ may be identified with the
partially ordered linear space of all real-valued linear functionals on
$S$.

(b) The linear space of bounded finitely additive real-valued
functionals on $\frak A$ may be identified with the $L$-space
$S^{\sim}$ of order-bounded linear functionals on $S$.   If
$f\in S^{\sim}$ corresponds to $\nu:\frak A\to\Bbb R$, then
$f^+\in S^{\sim}$ corresponds to $\nu^+$, where

\Centerline{$\nu^+a=\sup_{b\Bsubseteqshort a}\nu b$}

\noindent for every $a\in\frak A$, and

\Centerline{$\|f\|=\sup_{a\in\frak A}\nu a-\nu(1\Bsetminus a)$.}

(c) The linear space of bounded countably additive real-valued
functionals on $\frak A$ may be identified with the $L$-space
$S^{\sim}_c$.

(d) The linear space of completely additive real-valued functionals on
$\frak A$ may be identified with the $L$-space $S^{\times}$.

\proof{ By 361F, we have a canonical one-to-one correspondence between
linear functionals $f:S\to\Bbb R$ and additive functionals
$\nu_f:\frak A\to\Bbb R$, given by setting $\nu_f=f\chi$.

\medskip

{\bf (a)} Now
it is clear that $\nu_{f+g}=\nu_f+\nu_g$, $\nu_{\alpha f}=\alpha\nu_f$
for all $f$, $g$ and $\alpha$, so this one-to-one correspondence is a
linear space isomorphism.   To see that it is also an order-isomorphism,
we need note only that $\nu_f$ is non-negative iff $f$ is, by 361Ga.

\medskip

{\bf (b)} Recall from 356N that, because $S$ is a Riesz space with
order unit (361Ee), $S^{\sim}$ has a corresponding norm under which it
is an $L$-space.

\medskip

\quad {\bf (i)} If $f\in S^{\sim}$, then

\Centerline{$\sup_{b\in\frak A}|\nu_fb|
=\sup_{b\in\frak A}|f(\chi b)|\le\sup\{|f(u)|:u\in S,\,|u|\le\chi 1\}$}

\noindent is finite, and $\nu_f$ is bounded.

\medskip

\quad{\bf (ii)} Now suppose that $\nu_f$ is bounded and that $v\in S^+$.
Then there is an $\alpha\ge 0$ such that $v\le\alpha\chi 1$ (361Ee).
If $u\in S$ and $|u|\le v$, then we can express $u$ as
$\sum_{i=0}^n\alpha_i\chi a_i$ where $a_0,\ldots,a_n$ are disjoint
(361Eb);  now $|\alpha_i|\le\alpha$ whenever $a_i\ne 0$, so

\Centerline{$|f(u)|=|\sum_{i=0}^n\alpha_i\nu_fa_i|
\le\alpha\sum_{i=0}^n|\nu_fa_i|
=\alpha(\nu_fc_1-\nu_fc_2)
\le 2\alpha\sup_{b\in\frak A}|\nu_fb|$,}

\noindent setting $c_1=\sup\{a_i:i\le n,\,\nu_fa_i\ge 0\}$,
$c_2=\sup\{a_i:i\le n,\,\nu_fa_i<0\}$.   This shows that
$\{f(u):|u|\le v\}$ is bounded.   As $v$ is arbitrary, $f\in S^{\sim}$
(356Aa).

\medskip

\quad{\bf (iii)} To check the correspondence between $f^+$ and
$\nu_f^+$, refine the arguments of (i) and (ii)  as follows.   Take any
$f\in S^{\sim}$.    If $a\in\frak A$,

\Centerline{$\nu_f^+a=\sup_{b\Bsubseteqshort a}\nu_fb
=\sup_{b\Bsubseteqshort a}f(\chi b)\le\sup\{f(u):u\in S,\,0\le u\le\chi a\}
=f^+(\chi a)$.}

\noindent On the other hand, if $u\in S$ and $0\le u\le\chi a$, then we
can express $u$ as $\sum_{i=0}^n\alpha_i\chi a_i$ where $a_0,\ldots,a_n$
are disjoint;  now $0\le\alpha_i\le 1$ whenever $a_i\ne 0$, so

\Centerline{$f(u)=\sum_{i=0}^n\alpha_i\nu_fa_i\le\nu_fc
\le\nu_f^+a$,}

\noindent where $c=\sup\{a_i:i\le n,\,\nu_fa_i\ge 0\}$.
As $u$ is arbitrary, $f^+(\chi a)\le\nu_f^+a$.   This shows that
$\nu_f^+=f^+\chi$ is finitely additive, and that $\nu_f^+=\nu_{f^+}$, as
claimed.

\medskip

\quad{\bf (iv)} Now, for any $f\in S^{\sim}$,

$$\eqalignno{\|f\|
&=|f|(\chi 1)\cr
\noalign{\noindent (356N)}
&=(2f^+-f)(\chi 1)
=2\nu_f^+1-\nu_f1\cr
\noalign{\noindent (by (iii) just above)}
&=\sup_{a\in\frak A}2\nu_fa-\nu_f1
=\sup_{a\in\frak A}\nu_fa-\nu_f(1\Bsetminus a).\cr}$$

\medskip

{\bf (c)} If $f\ge 0$ in $S^{\sim}$, then $f$ is sequentially
order-continuous iff $\nu_f$ is sequentially order-continuous (361Gb),
that is, iff $\nu_f$ is countably additive (326Kc).   Generally, an
order-bounded linear functional belongs to $S^{\sim}_c$ iff it is
expressible as the difference of two sequentially
order-continuous positive linear functionals (356Ab), while a
bounded finitely additive functional is countably additive iff it is
expressible as the difference of two non-negative countably additive
functionals (326L);  so in the present context $f\in S^{\sim}_c$ iff
$\nu_f$ is bounded and countably additive.

\medskip

{\bf (d)} If $f\ge 0$ in $S^{\sim}$, then $f$ is
order-continuous iff $\nu_f$ is order-continuous (361Gb), that is, iff
$\nu_f$ is completely additive (326Oc).   Generally, an
order-bounded linear functional belongs to $S^{\times}$ iff it is
expressible as the difference of two
order-continuous positive linear functionals (356Ac), while a
finitely additive functional is completely additive iff it is
expressible as the difference of two non-negative completely additive
functionals (326Q);  so in the present context $f\in S^{\times}$ iff
$\nu_f$ is completely additive.
}%end of proof of 362A

\leader{362B}{Spaces of finitely additive
\dvrocolon{functionals}}\cmmnt{ The
identifications in the last theorem mean that we can relate the Riesz
space structure of $S(\frak A)^{\sim}$ to constructions involving
finitely additive functionals.   I have already set out the most useful
facts as exercises (326Yd, 326Ym, 326Yn, 326Yp, 326Yq);  it is now time
to repeat them more formally.

\medskip

\noindent}{\bf Theorem} Let $\frak A$ be a Boolean algebra.
Let $M$ be the Riesz space of bounded finitely
additive real-valued functionals on $\frak A$, $M_{\sigma}\subseteq M$
the space of bounded countably additive functionals, and
$M_{\tau}\subseteq M_{\sigma}$ the space of completely additive
functionals.

(a) For any $\mu$, $\nu\in M$, $\mu\vee\nu$, $\mu\wedge\nu$ and $|\nu|$
are defined by the formulae

\Centerline{$(\mu\vee\nu)(a)
=\sup_{b\Bsubseteqshort a}\mu b+\nu(a\Bsetminus b)$,}

\Centerline{$(\mu\wedge\nu)(a)=\inf_{b\Bsubseteqshort a}\mu b+\nu(a\Bsetminus
b)$,}

\Centerline{$|\nu|(a)=\sup_{b\Bsubseteqshort a}\nu b-\nu(a\Bsetminus b)
=\sup_{b,c\Bsubseteqshort a}\nu b-\nu c$}

\noindent for every $a\in\frak A$.   Setting

\Centerline{$\|\nu\|=|\nu|(1)
=\sup_{a\in\frak A}\nu a-\nu(1\Bsetminus a)$,}

\noindent $M$ becomes an $L$-space.

(b) $M_{\sigma}$ and $M_{\tau}$ are projection bands in $M$, therefore
$L$-spaces in their own right.   In particular, $|\nu|\in M_{\sigma}$
for every $\nu\in M_{\sigma}$, and $|\nu|\in M_{\tau}$ for every
$\nu\in M_{\tau}$.

(c) The band projection $P_{\sigma}:M\to M_{\sigma}$ is defined by the
formula

\Centerline{$(P_{\sigma}\nu)(c)=\inf\{\sup_{n\in\Bbb N}\nu a_n:
 \sequencen{a_n}$ is a non-decreasing sequence with supremum $c\}$}

\noindent whenever $c\in\frak A$ and $\nu\ge 0$ in $M$.

(d) The band projection $P_{\tau}:M\to M_{\tau}$ is defined by the
formula

\Centerline{$(P_{\tau}\nu)(c)=\inf\{\sup_{a\in A}\nu a:
 A$ is a non-empty upwards-directed set with supremum $c\}$}

\noindent whenever $c\in\frak A$ and $\nu\ge 0$ in $M$.

(e) If $A\subseteq M$ is upwards-directed, then $A$ is bounded above in
$M$ iff $\{\nu 1:\nu\in A\}$ is bounded above in $\Bbb R$, and in this
case (if $A\ne\emptyset$) $\sup A$ is defined by the formula

\Centerline{$(\sup A)(a)=\sup_{\nu\in A}\nu a$ for every $a\in\frak A$.}

\wheader{362B}{0}{0}{0}{60pt}

(f) Suppose that $\mu$, $\nu\in M$.

\quad(i) The following are equiveridical:

\qquad($\alpha$) $\nu$ belongs to the band in $M$ generated by $\mu$;

\qquad ($\beta$) for every $\epsilon>0$ there is a $\delta>0$ such that
$|\nu a|\le\epsilon$ whenever $|\mu|a\le\delta$;

\qquad ($\gamma$) $\lim_{n\to\infty}\nu a_n=0$ whenever
$\sequencen{a_n}$ is a non-increasing sequence in $\frak A$ such that
$\lim_{n\to\infty}|\mu|(a_n)=0$.

\quad(ii) Now suppose that $\mu$, $\nu\ge 0$, and let $\nu_1$, $\nu_2$
be the components of $\nu$ in the band generated by $\mu$ and its
complement.   Then

\Centerline{$\nu_1c
=\sup_{\delta>0}\inf_{\mu a\le\delta}\nu(c\Bsetminus a)$,
\quad$\nu_2c
=\inf_{\delta>0}\sup_{a\Bsubseteqshort c,\mu a\le\delta}\nu a$}

\noindent for every $c\in\frak A$.

\proof{{\bf (a)} Of course $\mu\vee\nu=\nu+(\mu-\nu)^+$,
$\mu\wedge\nu=\nu-(\nu-\mu)^+$, $|\nu|=\nu\vee(-\nu)$ (352D), so the
formula of 362Ab gives

$$\eqalign{(\mu\vee\nu)(a)
&=\nu a+\sup_{b\Bsubseteqshort a}\mu b-\nu b
=\sup_{b\Bsubseteqshort a}\mu b+\nu(a\Bsetminus b),\cr
(\mu\wedge\nu)(a)
&=\nu a-\sup_{b\Bsubseteqshort a}\nu b-\mu b
=\inf_{b\Bsubseteqshort a}\mu b+\nu(a\Bsetminus b),\cr
|\nu|(a)
&=\sup_{b\Bsubseteqshort a}\nu b-\nu(a\Bsetminus b)
\le\sup_{b,c\Bsubseteqshort a}\nu b-\nu c
=\sup_{b,c\Bsubseteqshort a}\nu(b\Bsetminus c)-\nu(c\Bsetminus b)\cr
&\le\sup_{b,c\Bsubseteqshort a}|\nu|(b\Bsetminus c)+|\nu|(c\Bsetminus b)
=\sup_{b,c\Bsubseteqshort a}|\nu|(b\Bsymmdiff c)
\le|\nu|(a).\cr}$$

\noindent The formula offered for $\|\nu\|$ corresponds exactly to the
formula in 362Ab for the norm of the associated member of
$S(\frak A)^{\sim}$;  because $S(\frak A)^{\sim}$ is an $L$-space under
its norm, so is $M$.

\medskip

{\bf (b)} By 362Ac-362Ad, $M_{\sigma}$ and $M_{\tau}$ may be identified
with $S(\frak A)^{\sim}_c$ and $S(\frak A)^{\times}$, which are
bands in $S(\frak A)^{\sim}$ (356B), therefore projection bands (353I);
so that $M_{\sigma}$
and $M_{\tau}$ are projection bands in $M$, and are $L$-spaces in their
own right (354O).

\medskip

{\bf (c)} Take any $\nu\ge 0$ in $M$.   Set

\Centerline{$\nu_{\sigma}c=\inf\{\sup_{n\in\Bbb N}\nu
a_n:\sequencen{a_n}$ is a non-decreasing sequence with supremum $c\}$}

\noindent for every $c\in\frak A$.   Then of course
$0\le\nu_{\sigma}c\le\nu c$ for every $c$.
The point is that $\nu_{\sigma}$ is countably additive.
\Prf\
Let $\sequence{i}{c_i}$ be a disjoint sequence in $\frak A$, with
supremum $c$.   Then for any $\epsilon>0$ we have non-decreasing
sequences $\sequencen{a_n}$, $\sequencen{a_{in}}$, for $i\in\Bbb N$,
such that

\Centerline{$\sup_{n\in\Bbb N}a_n=c$,
\quad$\sup_{n\in\Bbb N}a_{in}=c_i$ for $i\in\Bbb N$,}

\Centerline{$\sup_{n\in\Bbb N}\nu a_n\le \nu_{\sigma}c+\epsilon$,}

\Centerline{$\sup_{n\in\Bbb N}\nu a_{in}\le\nu_{\sigma}c_i
+2^{-i}\epsilon$ for every $i\in\Bbb N$.}

\noindent Set $b_n=\sup_{i\le n}a_{in}$ for each $n$;  then
$\sequencen{b_n}$ is non-decreasing, and

\Centerline{$\sup_{n\in\Bbb N}b_n=\sup_{i,n\in\Bbb N}a_{in}
=\sup_{i\in\Bbb N}c_i=c$,}

\noindent so

$$\eqalign{\nu_{\sigma}c
&\le\sup_{n\in\Bbb N}\nu b_n
=\sup_{n\in\Bbb N}\sum_{i=0}^n\nu a_{in}\cr
&=\sum_{i=0}^{\infty}\sup_{n\in\Bbb N}\nu a_{in}
\le\sum_{i=0}^{\infty}\nu_{\sigma}c_i+2^{-i}\epsilon
=\sum_{i=0}^{\infty}\nu_{\sigma}c_i+2\epsilon.\cr}$$

\noindent On the other hand, $\sequencen{a_n\Bcap c_i}$ is a
non-decreasing sequence with supremum $c\Bcap c_i=c_i$ for each $i$, so
$\nu_{\sigma}c_i\le\sup_{n\in\Bbb N}\nu(a_n\Bcap c_i)$, and

$$\eqalignno{\sum_{i=0}^{\infty}\nu_{\sigma}c_i
&\le\sum_{i=0}^{\infty}\sup_{n\in\Bbb N}\nu(a_n\Bcap c_i)
=\sup_{n\in\Bbb N}\sum_{i=0}^{\infty}\nu(a_n\Bcap c_i)\cr
\noalign{\noindent(because $\sequencen{a_n}$ is non-decreasing)}
&\le\sup_{n\in\Bbb N}\nu a_n\cr
\noalign{\noindent(because $\sequence{i}{c_i}$ is disjoint)}
&\le\nu_{\sigma}c+\epsilon.\cr}$$

\noindent As $\epsilon$ is arbitrary,
$\nu_{\sigma}c=\sum_{i=0}^{\infty}\nu_{\sigma}c_i$;  as
$\sequence{i}{c_i}$ is arbitrary, $\nu_{\sigma}$ is countably
additive.\ \Qed

Thus $\nu_{\sigma}\in M_{\sigma}$.   On the other hand, if
$\nuprime\in M_{\sigma}$ and $0\le\nuprime\le \nu$, then whenever $c\in\frak A$
and $\sequencen{a_n}$ is a non-decreasing sequence with supremum $c$,

\Centerline{$\nuprime c=\sup_{n\in\Bbb N}\nuprime a_n\le\sup_{n\in\Bbb N}\nu
a_n$.}

\noindent So we must have $\nuprime c\le\nu_{\sigma}c$.   This means that

\Centerline{$\nu_{\sigma}=\sup\{\nuprime:\nuprime\in
M_{\sigma},\,\nuprime\le\nu\}=P_{\sigma}\nu$,}

\noindent as claimed.

\medskip

{\bf (d)} The same ideas, with essentially elementary modifications,
deal with the completely additive part.   Take any $\nu\ge 0$ in $M$.
Set

\Centerline{$\nu_{\tau}c=\inf\{\sup_{a\in A}\nu
a:A$ is a non-empty upwards-directed set with supremum $c\}$}

\noindent for every $c\in\frak A$.   Then of course
$0\le\nu_{\tau}c\le\nu c$ for every $c$.
The point is that $\nu_{\tau}$ is completely additive.
\Prf\ Note first that if $c\in\frak A$, $\epsilon>0$ there is a
non-empty upwards-directed $A$, with supremum $c$, such that $\sup_{a\in
A}\nu a\le\nu_{\tau}c+\epsilon\nu c$;  for if $\nu c=0$ we can take
$A=\{c\}$.   Now let $\langle c_i\rangle_{i\in I}$ be a partition of
unity in $\frak A$.
Then for any $\epsilon>0$ we have non-empty
upwards-directed sets $A$, $A_i$, for $i\in I$, such that

\Centerline{$\sup A=1$,
\quad$\sup A_i=c_i$ for $i\in I$,
\quad$\sup_{a\in A}\nu a\le \nu_{\tau}1+\epsilon\nu 1$,}

\Centerline{$\sup_{a\in A_i}\nu a\le\nu_{\tau}c_i+\epsilon\nu c_i$ for
every
$i\in I$.}

\noindent Set

\Centerline{$B=\{\sup_{i\in J}a_i:J\subseteq I$ is finite, $a_i\in A_i$
for every $i\in J\}$;}

\noindent then $B$ is non-empty and upwards-directed,
and

\Centerline{$\sup B=\sup(\bigcup_{i\in I}A_i)=1$,}

\noindent so

$$\eqalign{\nu_{\tau}1
&\le\sup_{b\in B}\nu b
=\sup\{\sum_{i\in J}\nu a_i:J\subseteq I\text{ is finite},\,
a_i\in A_i\Forall i\in J\}\cr
&\le\sum_{i\in I}\nu_{\tau}c_i+\epsilon\nu c_i
\le\epsilon\nu 1+\sum_{i\in I}\nu_{\tau}c_i.\cr}$$

\noindent On the other hand, $A'_i=\{a\Bcap c_i:a\in A\}$ is a
non-empty upwards-directed set with supremum $c_i$ for each $i$, so
$\nu_{\tau}c_i\le\sup_{a\in A'_i}\nu a$, and

$$\eqalignno{\sum_{i\in I}\nu_{\tau}c_i
&\le\sum_{i\in I}\sup_{a\in A}\nu(a\Bcap c_i)
=\sup_{a\in A}\sum_{i\in I}\nu(a\Bcap c_i)\cr
&\le\sup_{a\in A}\nu a
\le\nu_{\tau}1+\epsilon\nu 1.\cr}$$

\noindent As $\epsilon$ is arbitrary,
$\nu_{\tau}c=\sum_{i\in I}\nu_{\tau}c_i$;  as
$\langle c_i\rangle_{i\in I}$ is arbitrary, $\nu_{\tau}$ is completely
additive, by 326R.\ \Qed

Thus $\nu_{\tau}\in M_{\tau}$.   On the other hand, if
$\nuprime\in M_{\tau}$ and $0\le\nuprime\le \nu$, then whenever
$c\in\frak A$
and $A$ is a non-empty upwards-directed set with supremum $c$,

\Centerline{$\nuprime c=\sup_{a\in A}\nuprime a\le\sup_{a\in A}\nu a$}

\noindent (using 326Oc).   So we must have $\nuprime c\le\nu_{\tau}c$.   This
means that

\Centerline{$\nu_{\tau}=\sup\{\nuprime:\nuprime\in
M_{\tau},\,\nuprime\le\nu\}=P_{\tau}\nu$,}

\noindent as claimed.

\medskip

{\bf (e)} If $A$ is empty, of course it is bounded above in $M$, and
$\{\nu 1:\nu\in A\}=\emptyset$ is bounded above in $\Bbb R$;  so let us
suppose that $A$ is not empty.   In this case, if $\lambda_0\in M$ is an
upper bound for $A$, then $\lambda_01$ is an upper bound for $\{\nu
1:\nu\in A\}$.   On the other hand, if $\sup_{\nu\in A}\nu 1=\gamma$ is
finite, $\gamma^*=\sup\{\nu a:\nu\in A,\,a\in\frak A\}$ is finite.
\Prf\ Fix $\nu_0\in A$.   Set $\gamma_1=\sup_{a\in\frak
A}|\nu_0a|<\infty$.   Then for any $\nu\in A$ and $a\in\frak A$ there is a
$\nuprime\in A$ such that $\nu_0\vee\nu\le\nuprime$, so that

\Centerline{$\nu a\le\nuprime a=\nuprime1-\nuprime(1\Bsetminus a)
\le\gamma-\nu_0(1\Bsetminus a)\le\gamma+\gamma_1$.}

\noindent So

\Centerline{$\gamma^*\le\gamma+\gamma_1<\infty$.   \Qed}

Set $\lambda a=\sup_{\nu\in A}\nu a$ for every $a\in\frak A$.    Then
$\lambda:\frak A\to\Bbb R$ is additive.   \Prf\ If $a$, $b\in\frak A$
are disjoint, then

$$\eqalignno{\lambda(a\Bcup b)
&=\sup_{\nu\in A}\nu(a\Bcup b)
=\sup_{\nu\in A}\nu a+\nu b
=\sup_{\nu\in A}\nu a+\sup_{\nu\in A}\nu b\cr
\noalign{\noindent (because $A$ is upwards-directed)}
&=\lambda a+\lambda b.\text{ \Qed}\cr}$$

\noindent Also $\lambda a\le\gamma^*$ for every $a$, so

\Centerline{$|\lambda a|=\max(\lambda a,-\lambda a)
=\max(\lambda a,\lambda(1\Bsetminus a)-\lambda 1)
\le\gamma^*+|\lambda 1|$}

\noindent for every $a\in\frak A$, and $\lambda$ is bounded.

This shows that $\lambda\in M$, so that $A$ is bounded above in $M$.
Of course $\lambda$ must be actually the least upper bound of $A$ in
$M$.

\medskip

{\bf (f)(i)}\grheada$\Rightarrow$\grheadb\ Suppose that $\nu$ belongs to
the band in $M$ generated by $\mu$, that is,
$|\nu|=\sup_{n\in\Bbb N}|\nu|\wedge n|\mu|$ (352Vb).   Let $\epsilon>0$.
Then there is an
$n\in\Bbb N$ such that $|\nu|(1)\le\bover12\epsilon+(|\nu|\wedge
n|\mu|)(1)$ ((e) above).   Set $\delta=\bover1{2n+1}\epsilon>0$.   If
$|\mu|(a)\le\delta$, then

$$\eqalign{|\nu a|
&\le|\nu|(a)
=(|\nu|\wedge n|\mu|)(a)+(|\nu|-|\nu|\wedge n|\mu|)(a)\cr
&\le n|\mu|(a)+(|\nu|-|\nu|\wedge n|\mu|)(1)
\le n\delta+\Bover12\epsilon
\le\epsilon.\cr}$$

\noindent So ($\beta$) is satisfied.

\medskip

\quad{\bf not-($\pmb{\alpha}$)$\Rightarrow$not-($\pmb{\beta}$)}
Suppose that $\nu$ does not belong
to the band in $M$ generated by $|\mu|$.   Then there is a $\nu_1>0$
such that $\nu_1\le|\nu|$ and $\nu_1\wedge|\mu|=0$ (353C).   For any
$\delta>0$, there is an $a\in\frak A$ such that $\nu_1(1\Bsetminus
a)+|\mu|(a)\le\min(\delta,\bover12\nu_11)$ ((a) above);  now
$|\mu|(a)\le\delta$ but

\Centerline{$|\nu|(a)\ge\nu_1a=\nu_11-\nu_1(1\Bsetminus a)
\ge\nu_11-\bover12\nu_11=\bover12\nu_11$.}

\noindent Thus $\mu$, $\nu$ do not satisfy ($\beta$) (with
$\epsilon=\bover12\nu_11$).

\medskip

\quad\grheadb$\Rightarrow$\grheadc\  is trivial.

\medskip

\quad\grheadc$\Rightarrow$\grheada\  Observe first that if
$\sequence{k}{c_k}$ is a non-increasing sequence in $\frak A$ such that
$\lim_{k\to\infty}|\mu|c_k=0$, then $\lim_{k\to\infty}\nu^+c_k=0$.
\Prf\ Let $\epsilon>0$.   Because $\nu^+\wedge\nu^-=0$, there is a
$b\in\frak A$ such that $\nu^+b+\nu^-(1\Bsetminus b)\le\epsilon$, by
part (a).   Now $\sequence{k}{c_k\Bsetminus b}$ is non-increasing and
$\lim_{k\to\infty}|\mu|(c_k\Bsetminus b)=0$, so
$\lim_{k\to\infty}\nu(c_k\Bsetminus b)=0$ and

$$\eqalign{\limsup_{k\to\infty}\nu^+c_k
&=\limsup_{k\to\infty}\nu^+(c_k\Bcap b)+\nu(c_k\Bsetminus b)
   +\nu^-(c_k\Bsetminus b)\cr
&\le\nu^+b+\nu^-(1\Bsetminus b)\le\epsilon.\cr}$$

\noindent As $\epsilon$ is arbitrary,
$\lim_{k\to\infty}\nu^+c_k=0$.\ \Qed

\Quer\ Now suppose, if possible, that $\nu^+$ does not belong to the
band generated by $\mu$.   Then there is a $\nu_1>0$ such that
$\nu_1\le\nu^+$ and $\nu_1\wedge|\mu|=0$.   Set
$\epsilon=\bover14\nu_11>0$.   For each $n\in\Bbb N$, we can choose
$a_n\in\frak A$ such that
$|\mu|a_n+\nu_1(1\Bsetminus a_n)\le 2^{-n}\epsilon$, by part (a) again.
For $n\ge k$, set $b_{kn}=\sup_{k\le i\le n}a_i$;  then

\Centerline{$|\mu|b_{kn}\le\sum_{i=k}^n|\mu|a_i\le 2^{-k+1}\epsilon$,}

\noindent and $\langle b_{kn}\rangle_{n\ge k}$ is non-decreasing.   Set
$\gamma_k=\sup_{n\ge k}\nu_1b_{kn}$ and choose $m(k)\ge k$
such that $\nu_1b_{k,m(k)}\ge\gamma_k-2^{-k}\epsilon$.   Setting
$b_k=b_{k,m(k)}$, we see that $b_k\Bcup b_{k+1}=b_{kn}$ where
$n=\max(m(k),m(k+1))$, so that

\Centerline{$\nu_1(b_k\Bcup b_{k+1})\le\gamma_k
\le\nu_1b_k+2^{-k}\epsilon$}

\noindent and $\nu_1(b_{k+1}\Bsetminus b_k)\le 2^{-k}\epsilon$.   Set
$c_k=\inf_{i\le k}b_i$ for each $k$;  then

\Centerline{$\nu_1(b_{k+1}\Bsetminus c_{k+1})
=\nu_1(b_{k+1}\Bsetminus c_k)
\le\nu_1(b_{k+1}\Bsetminus b_k)+\nu_1(b_k\Bsetminus c_k)
\le 2^{-k}\epsilon+\nu_1(b_k\Bsetminus c_k)$}

\noindent for each $k$;  inducing on $k$, we see that

\Centerline{$\nu_1(b_k\Bsetminus c_k)\le\sum_{i=0}^{k-1}2^{-i}\epsilon
\le 2\epsilon$}

\noindent for every $k$.   This means that

\Centerline{$\nu^+c_k\ge\nu_1c_k
\ge\nu_1b_k-2\epsilon\ge\nu_1a_k-2\epsilon
=\nu_11-\nu_1(1\Bsetminus a_k)-2\epsilon
\ge4\epsilon -\epsilon-2\epsilon=\epsilon$}

\noindent for every $k\in\Bbb N$.   On the other hand,
$\sequence{k}{c_k}$ is a non-increasing sequence and

\Centerline{$|\mu|c_k\le|\mu|b_k\le 2^{-k+1}\epsilon$}

\noindent for every $k$, which contradicts the paragraph just
above.\ \Bang

This means that $\nu^+$ must belong to the band generated by $\mu$.
Similarly $\nu^-=(-\nu)^+$ belongs to the band generated by $\mu$ and
$\nu=\nu^++\nu^-$ also does.

\medskip

\quad{\bf (ii)} Take $c\in\frak A$.   Set

\Centerline{$\beta_1=\sup_{\delta>0}\inf_{\mu a\le\delta}\nu(c\Bsetminus
a)$,
\quad$\beta_2=\inf_{\delta>0}\sup_{a\Bsubseteqshort c,\mu a\le\delta}\nu a$.}

\noindent Then

\Centerline{$\beta_1=\sup_{\delta>0}\inf_{a\Bsubseteqshort c,\mu
a\le\delta}\nu(c\Bsetminus a)=\nu c-\beta_2$.}

\noindent Take any $\epsilon>0$.   Because $\nu_1$ belongs to the band
generated by $\mu$, part (i) tells us that there is a $\delta>0$ such
that $\nu_1 a\le\epsilon$ whenever $\mu a\le\delta$.   In this case, if
$\mu a\le\delta$,

\Centerline{$\nu(c\Bsetminus a)=\nu c-\nu(c\Bcap a)\ge\nu c-\epsilon
\ge\nu_1c-\epsilon$;}

\noindent thus

\Centerline{$\beta_1\ge\inf_{\mu a\le\delta}\nu(c\Bsetminus a)
\ge\nu_1c-\epsilon$.}

\noindent As $\epsilon$ is arbitrary, $\beta_1\ge\nu_1c$.   On the other
hand, given $\epsilon$, $\delta>0$, there is an $a\Bsubseteq c$ such
that $\mu a+\nu_2(c\Bsetminus a)\le\min(\delta,\epsilon)$, because
$\mu\wedge\nu_2=0$ (using (a) again).   In this case, of course, $\mu
a\le\delta$, while

\Centerline{$\nu a\ge\nu_2a=\nu_2c-\nu_2(c\Bsetminus a)
\ge\nu_2c-\epsilon$.}

\noindent Thus $\sup_{a\Bsubseteqshort c,\mu a\le\delta}\nu a
\ge\nu_2c-\epsilon$.   As $\delta$ is arbitrary,
$\beta_2\ge\nu_2c-\epsilon$.   As $\epsilon$ is arbitrary,
$\beta_2\ge\nu_2c$;  but as

\Centerline{$\beta_1+\beta_2=\nu c=\nu_1c+\nu_2c$,}

\noindent $\beta_i=\nu_ic$ for both $i$, as claimed.
}%end of proof of 362B

\medskip

\noindent{\bf Remark} The $L$-space norm $\|\,\|$ on $M$, described in (a)
above, is the {\bf total variation norm}.

\leader{362C}{}\cmmnt{ The formula in 362B(f-i) has, I hope, already
reminded you of the concept of `absolutely continuous' additive
functional from the Radon-Nikod\'ym theorem (Chapter 23, \S327).   The
expressions in 362Bf are limited by the assumption that $\mu$, like
$\nu$, is finite-valued.   If we relax this we get an alternative
version of some of the same ideas.

\medskip

\noindent}{\bf Theorem} Let $(\frak A,\bar\mu)$ be a measure algebra
and $M$ be the Riesz space of bounded finitely
additive real-valued functionals on $\frak A$.   Write

\Centerline{$M_{ac}=\{\nu:\nu\in M$ is absolutely continuous with
respect to $\bar\mu\}$\dvro{,}{}}

\cmmnt{\noindent (see 327A),}

\Centerline{$M_{tc}=\{\nu:\nu\in M$ is continuous with respect to the
measure-algebra topology on $\frak A\}$,}

\Centerline{$M_t=\{\nu:\nu\in M$,
$|\nu|1=\sup_{\bar\mu a<\infty}|\nu|a\}$.}

\noindent Then $M_{ac}$, $M_{tc}$ and $M_t$ are bands in $M$.

\proof{{\bf (a)(i)} It is easy to check that $M_{ac}$ is a linear
subspace of $M$.

\medskip

\quad{\bf (ii)} If $\nu\in M_{ac}$, $\nuprime\in M$ and $|\nuprime|\le|\nu|$
then $\nuprime\in M_{ac}$.   \Prf\ Given $\epsilon>0$ there is a $\delta>0$
such that $|\nu a|\le\bover12\epsilon$ whenever $\bar\mu a\le\delta$.
Now

\Centerline{$|\nuprime a|\le|\nuprime|(a)\le|\nu|(a)
\le 2\sup_{c\Bsubseteqshort a}|\nu c|\le\epsilon$}

\noindent (using the formula for $|\nu|$ in 362Ba) whenever $\bar\mu
a\le\delta$.   As $\epsilon$ is arbitrary, $\nuprime$ is absolutely
continuous.\ \Qed

\medskip

\quad{\bf (iii)} If $A\subseteq M_{ac}$ is non-empty and
upwards-directed
and $\nu=\sup A$ in $M$, then $\nu\in M_{ac}$.    \Prf\ Let
$\epsilon>0$.   Then there is a $\nuprime\in A$ such that $\nu
1\le\nuprime1+\bover12\epsilon$ (362Be).   Now there is a $\delta>0$ such
that $|\nu
a|\le\bover12\epsilon$ whenever $\bar\mu a\le\delta$.   If now $\bar\mu
a\le\delta$,

\Centerline{$|\nu a|\le|\nuprime a|+(\nu-\nuprime)(a)
\le\bover12\epsilon+(\nu-\nuprime)(1)
\le\epsilon$.}

\noindent As $\epsilon$ is arbitrary, $\nu$ is absolutely continuous
with respect to $\bar\mu$.\ \Qed

Putting these together, we see that $M_{ac}$ is a band.

\medskip

{\bf (b)(i)} We know that $M_{tc}$ consists just of those $\nu\in M$
which are continuous at $0$ (327Bc).   Of course this is a linear
subspace of $M$.

\medskip

\quad{\bf (ii)} If $\nu\in M_{tc}$, $\nuprime\in M$ and $|\nuprime|\le|\nu|$
then $|\nu|\in M_{tc}$.   \Prf\ Write
$\frak A^f=\{d:d\in\frak A,\,\bar\mu d<\infty\}$.
Given $\epsilon>0$ there are $d\in\frak A^f$,
$\delta>0$ such that $|\nu a|\le\bover12\epsilon$
whenever $\bar\mu(a\Bcap d)\le\delta$.   Now

\Centerline{$|\nuprime a|\le|\nuprime|(a)\le|\nu|(a)
\le 2\sup_{c\Bsubseteqshort a}|\nu c|\le\epsilon$}

\noindent whenever $\bar\mu(a\Bcap d)\le\delta$.   As $\epsilon$ is
arbitrary, $\nuprime$ is continuous at $0$ and belongs to $M_{tc}$.\ \Qed

\medskip

\quad{\bf (iii)} If $A\subseteq M_{tc}$ is non-empty and
upwards-directed and
$\nu=\sup A$ in $M$, then $\nu\in M_{tc}$.    \Prf\ Let $\epsilon>0$.
Then there is a $\nuprime\in A$ such that
$\nu 1\le\nuprime1+\bover12\epsilon$.
There are $d\in\frak A^f$, $\delta>0$ such that
$|\nu a|\le\bover12\epsilon$ whenever $\bar\mu(a\Bcap d)\le\delta$.
If now $\bar\mu(a\Bcap d)\le\delta$,

\Centerline{$|\nu a|\le|\nuprime a|+(\nu-\nuprime)(a)
\le\bover12\epsilon+(\nu-\nuprime)(1)
\le\epsilon$.}

\noindent As $\epsilon$ is arbitrary, $\nu$ is continuous at $0$,
therefore belongs to $M_{tc}$.\ \Qed

Putting these together, we see that $M_{tc}$ is a band.

\medskip

{\bf (c)(i)} $M_t$ is a linear subspace of $M$.   \Prf\ Suppose that
$\nu_1$, $\nu_2\in M_t$ and $\alpha\in\Bbb R$.   Given $\epsilon>0$,
there are $a_1$, $a_2\in\frak A^f$ such that
$|\nu_1|(1\Bsetminus a_1)\le\bover{\epsilon}{1+|\alpha|}$
and $|\nu_2|(1\Bsetminus a_2)\le\epsilon$.   Set $a=a_1\Bcup a_2$;
then $\bar\mu a<\infty$ and

\Centerline{$|\nu_1+\nu_2|(1\Bsetminus a)\le 2\epsilon$,
\quad$|\alpha\nu_1|(1\Bsetminus a)\le\epsilon$.}

\noindent As $\epsilon$ is arbitrary, $\nu_1+\nu_2$ and $\alpha\nu_1$
belong to $M_t$;  as $\nu_1$, $\nu_2$ and $\alpha$ are arbitrary, $M_t$
is a linear subspace of $M$.\ \Qed

\medskip

\quad{\bf (ii)} If $\nu\in M_t$, $\nuprime\in M$ and $|\nuprime|\le|\nu|$ then

\Centerline{$\inf_{\bar\mu a<\infty}|\nuprime|(1\Bsetminus
a)\le\inf_{\bar\mu a<\infty}|\nu|(1\Bsetminus a)=0$,}

\noindent so $\nuprime\in M_t$.    Thus $M_t$ is a solid linear subspace of
$M$.


\medskip

\quad{\bf (iii)} If $A\subseteq M_t^+$ is non-empty and upwards-directed
and $\nu=\sup A$ is defined in $M$, then $\nu\in M_t$.   \Prf\

\Centerline{$|\nu|1
=\nu1
=\sup_{\nuprime\in A}\nuprime1
=\sup_{\nuprime\in A,\bar\mu a<\infty}\nuprime a
=\sup_{\bar\mu a<\infty}\nu a$.}

\noindent As $A$ is arbitrary, $\nu\in M_t$.\ \QeD\  Thus $M_t$ is a
band in $M$.
}%end of proof of 362C

\vleader{108pt}{362D}{}\cmmnt{ For semi-finite measure algebras, among
others, the formula of 362Bd takes a special form.

\medskip

\noindent}{\bf Proposition} Let $\frak A$ be a \wsid\ Boolean algebra.
Let $M$ be the space of bounded finitely additive functionals on
$\frak A$, $M_{\tau}\subseteq M$ the space of completely additive
functionals,
and $P_{\tau}:M\to M_{\tau}$ the band projection\cmmnt{, as in 362B}.
Then for any $\nu\in M^+$ and $c\in\frak A$ there is a non-empty
upwards-directed set $A\subseteq\frak A$ with supremum $c$ such that
$(P_{\tau}\nu)(c)=\sup_{a\in A}\nu a$\cmmnt{;  that is, the `inf'
in 362Bd can be read as `min'}.

\proof{ By 362Bd, we can find for each $n$ a non-empty upwards-directed
$A_n$, with supremum $c$, such that
$\sup_{a\in A_n}\nu a\le(P_{\tau}\nu)(c)+2^{-n}$.   Set
$B_n=\{c\Bsetminus a:a\in A_n\}$ for
each $n$, so that $B_n$ is downwards-directed and has infimum $0$.
Because $\frak A$ is \wsid,

\Centerline{$B=\{b:$ for every $n\in\Bbb N$ there is a $b'\in B_n$ such
that $b\Bsupseteq b'\}$}

\noindent is also a downwards-directed set with infimum $0$.
Consequently $A=\{c\Bsetminus b:b\in B\}$ is upwards-directed and has
supremum $c$.   Moreover, for any $n\in\Bbb N$ and $a\in A$, there is an
$a'\in A_n$ such that $a\Bsubseteq a'$;  so, using 362Bd again and
referring to the choice of $A_n$,

\Centerline{$(P_{\tau}\nu)(c)
\le\sup_{a\in A}\nu a
\le\sup_{a'\in A_n}\nu a'
\le(P_{\tau}\nu)(c)+2^{-n}$.}

\noindent As $n$ is arbitrary, $A$ has the required property.
}%end of proof of 362D

\leader{362E}{Uniformly integrable\dvrocolon{ sets}}\cmmnt{ The spaces
$S^{\sim}$, $S^{\sim}_c$ and $S^{\times}$ of 362A, or, if you prefer,
the spaces $M$, $M_{\sigma}$, $M_{\tau}$, $M_{ac}$, $M_{tc}$, $M_t$ of
362B-362C, are all $L$-spaces, and any serious study of them must
involve a discussion of their uniformly integrable ( = relatively weakly
compact) subsets.   The basic work has been done in 356O;  I spell
out its application in this context.

\medskip

\noindent}{\bf Theorem} Let $\frak A$ be a Boolean algebra and $M$ the
$L$-space of bounded finitely additive functionals on $\frak A$.   Then
a norm-bounded set $C\subseteq M$ is uniformly integrable iff
$\lim_{n\to\infty}\sup_{\nu\in C}|\nu a_n|=0$ for every disjoint
sequence $\sequencen{a_n}$ in $\frak A$.

\proof{ Write $S$ for $S(\frak A)$ and $\tilde C$ for the set
$\{f:f\in S^{\sim},\,f\chi\in C\}$.   Because the map
$f\mapsto f\chi$ is a
normed Riesz space isomorphism between $S^{\sim}$ and $M$, $\tilde C$ is
uniformly integrable in $M$ iff $C$ is uniformly integrable in
$S^{\sim}$.

\medskip

{\bf (a)} Suppose that $C$ is uniformly integrable and that
$\sequencen{a_n}$ is a disjoint sequence in $\frak A$.   Then
$\sequencen{\chi a_n}$ is a disjoint order-bounded sequence in
$S^{\sim}$, while $\tilde C$ is uniformly integrable, so
$\lim_{n\to\infty}\sup_{f\in\tilde C}|f(\chi a_n)|=0$, by 356O;
but this means that $\lim_{n\to\infty}\sup_{\nu\in C}|\nu a_n|=0$.  Thus
the condition is satisfied.

\medskip

{\bf (b)} Now suppose that $C$ is not uniformly integrable.   By
356O, in the other direction, there is a disjoint sequence
$\sequencen{u_n}$ in $S$ such
that $0\le u_n\le\chi 1$ for each $n$ and
$\limsup_{n\to\infty}\sup_{f\in\tilde C}|f(u_n)|>0$.   For each $n$,
take $c_n=\Bvalue{u_n>0}$ (361Eg);  then $0\le u_n\le\chi c_n$ and
$\sequencen{c_n}$ is disjoint.   Now

$$\eqalign{\limsup_{n\to\infty}\sup_{\nu\in C}|\nu|(c_n)
&=\limsup_{n\to\infty}\sup_{f\in\tilde C}|f|(\chi c_n)\cr
&\ge\limsup_{n\to\infty}\sup_{f\in\tilde C}|f(u_n)|
>0.\cr}$$

\noindent So if we choose $\nu_n\in C$ such that
$|\nu_n|(c_n)\ge\bover12\sup_{\nu\in C}|\nu|(c_n)$, we shall have
$\limsup_{n\to\infty}|\nu_n|(c_n)>0$.   Next, for each $n$, we can find
$a_n\Bsubseteq c_n$ such that $|\nu_na_n|\ge\bover12|\nu_n|(c_n)$, so
that

\Centerline{$\limsup_{n\in\Bbb N}\sup_{\nu\in C}|\nu a_n|
\ge\limsup_{n\to\infty}|\nu_na_n|>0$.}

\noindent Since $\sequencen{a_n}$, like $\sequencen{c_n}$, is disjoint,
the condition is not satisfied.   This completes the proof.
}%end of proof of 362E

\exercises{\leader{362X}{Basic exercises (a)} Let $\frak A$ be a
Dedekind $\sigma$-complete Boolean algebra and $\nu_1$, $\nu_2$ two
countably additive functionals on $\frak A$.   Show that
$|\nu_1|\wedge|\nu_2|=0$ in the Riesz space of bounded finitely additive
functionals on $\frak A$ iff there is a $c\in\frak A$ such that
$\nu_1a=\nu_1(a\Bcap c)$ and $\nu_2a=\nu_2(a\Bsetminus c)$ for every
$a\in\frak A$.
%362B

\spheader 362Xb Let $(\frak A,\bar\mu)$ be a measure algebra, and take
$M$, $M_{ac}$ as in 362C.
Show that for any non-negative $\nu\in M$, the component $\nu_{ac}$ of
$\nu$ in $M_{ac}$ is given by the formula

\Centerline{$\nu_{ac}c
=\sup_{\delta>0}\inf_{\bar\mu a\le\delta}\nu(c\Bsetminus a)$.}
%362C

\spheader 362Xc Let $(\frak A,\bar\mu)$ be a measure algebra, and take
$M$, $M_t$ as in 362C.   (i) Show that $M_t$ is just the set of those
$\nu\in M$ such that $\nu a=\lim_{b\to\Cal F}\nu(a\Bcap b)$ for every
$a\in\frak A$, where $\Cal F$ is the filter on $\frak A$ generated by
the sets $\{b:b\in\frak A^f,\,b\Bsupseteq b_0\}$ as
$b_0$ runs over the set $\frak A^f$ of elements of $\frak A$ of
finite measure.   (ii)
Show that the complementary band $M_t^{\perp}$ of $M_t$ in $M$ is just
the set of those $\nu\in M$ such that $\nu a=0$ for every $a\in\frak A^f$.
(iii) Show that for any $\nu\in M$, its component $\nu_t$ in $M_t$ is
given by the formula $\nu_ta=\lim_{b\to\Cal F}\nu(a\Bcap b)$ for every
$a\in \frak A$.
%362C

\spheader 362Xd Let $(\frak A,\bar\mu)$ be a measure algebra.   Write
$M$, $M_{\sigma}$, $M_{\tau}$, $M_{ac}$, $M_{tc}$ and $M_t$ as in
362B-362C.   Show that (i) $M_{\sigma}\subseteq M_{ac}$ (ii)
$M_{ac}\cap M_t=M_{tc}\subseteq M_{\tau}$ (iii) if $(\frak A,\bar\mu)$
is $\sigma$-finite, then $M_{\sigma}=M_{tc}$.
%362C

\spheader 362Xe Let $\frak A$ be a Boolean algebra, and $M$ the space of
bounded additive functionals on $\frak A$.   Let us say that a
non-zero finitely additive functional $\nu:\frak A\to\Bbb R$ is
{\bf atomic} if whenever $a$, $b\in\frak A$ and $a\Bcap b=0$ then at least
one of $\nu a$, $\nu b$ is zero.   (i) Show that for a non-zero finitely
additive functional $\nu$ on $\frak A$ the following are equiveridical:
($\alpha$) $\nu$ is
atomic; ($\beta$) $\nu\in M$ and $|\nu|$ is atomic;
($\gamma$) $\nu\in M$ and the corresponding linear functional
$f_{|\nu|}=|f_{\nu}|\in S(\frak A)^{\sim}$ is a Riesz homomorphism;
($\delta$) there are a
multiplicative linear functional $f:S(\frak A)\to\Bbb R$ and an
$\alpha\in\Bbb R$ such that $\nu a=\alpha f(\chi a)$ for every
$a\in\frak A$;  ($\epsilon$) $\nu\in M$
and the band in $M$ generated by $\nu$
is the set of multiples of $\nu$.
(ii) Show that a completely additive functional
$\nu:\frak A\to\Bbb R$ is atomic iff there are $a\in\frak A$ and
$\alpha\in\Bbb R\setminus\{0\}$ such that $a$ is an atom in $\frak A$
and $\nu b=\alpha$ when $a\Bsubseteq b$, $0$ when $a\Bcap b=0$.

\spheader 362Xf Let $\frak A$ be a Boolean algebra.   (i) Show that the
properly atomless functionals (definition:  326F) form a band $M_c$
in the Riesz space $M$ of all bounded finitely additive functionals on
$\frak A$.  (ii) Show that the complementary band $M_c^{\perp}$ consists
of just those $\nu\in M$ expressible as a sum $\sum_{i\in I}\nu_i$ of
countably many atomic functionals $\nu_i\in M$.
(iii) Show that if $\frak A$ is purely atomic then a
properly atomless completely additive functional on
$\frak A$ must be $0$.
%362Xe

\spheader 362Xg Let $X$ be a set and $\Sigma$ an algebra of subsets of
$X$.   Let $M$ be the Riesz space of bounded finitely additive
functionals on $\Sigma$, $M_{\tau}$ the space of completely additive
functionals and $M_p$ the space of functionals expressible in the form
$\nu E=\sum_{x\in E}\alpha_x$ for some absolutely summable family
$\langle\alpha_x\rangle_{x\in X}$ of real numbers.   (i) Show that $M_p$
is a band in $M$.    (ii) Show that if all singleton subsets of $X$
belong to $\Sigma$ then $M_p=M_{\tau}$.   (iii) Show that if $\Sigma$ is
a $\sigma$-algebra then every member of $M_p$ is countably additive.
(iv) Show that if $X$ is a compact
zero-dimensional Hausdorff space and $\Sigma$ is the algebra of
open-and-closed subsets of $X$ then the complementary band $M_p^{\perp}$
of $M_p$ in $M$ is the band $M_c$ of
properly atomless functionals described in 362Xf.
%362Xf

\spheader 362Xh Let $(X,\Sigma,\mu)$ be a measure space.   Let $M$ be
the Riesz space of bounded finitely additive functionals on $\Sigma$ and
$M_{\sigma}$ the space of bounded countably additive functionals.
Let $M_{tc}$, $M_{ac}$ be
the spaces of truly continuous and bounded absolutely continuous
additive functionals as defined in 232A.   Show that
$M_{tc}$ and $M_{ac}$ are bands in $M$ and that
$M_{tc}\subseteq M_{\sigma}\cap M_{ac}$.    Show that if $\mu$ is
$\sigma$-finite then $M_{tc}=M_{\sigma}\cap M_{ac}$.

\spheader 362Xi Let $\frak A$ be a Boolean algebra and
$M$ the Riesz space
of bounded finitely additive functionals on $\frak A$.  (i) For any
non-empty downwards-directed set $A\subseteq\frak A$ set
$N_A=\{\nu:\nu\in M,\,\inf_{a\in A}|\nu|a=0\}$.   Show that $N_A$ is a
band in $M$.   (ii) For any non-empty set $\Cal A$ of non-empty
downwards-directed sets in $\frak A$ set
$M_{\Cal A}=\{\nu:\nu\in M,\,\inf_{a\in A}|\nu|a=0\Forall A\in\Cal A\}$.
Show that $M_{\Cal A}$ is a band in $M$.
(iii)  Explain how to represent as such $M_{\Cal A}$ the bands
$M_{\sigma}$, $M_{\tau}$, $M_t$, $M_{ac}$, $M_{tc}$
described in 362B-362C, and also
any band generated by a single element of $M$.
(iv) Suppose, in (ii), that $\Cal A$ has the property
that for any $A$, $A'\in\Cal A$ there is a $B\in\Cal A$ such that for
every $b\in B$ there are $a\in A$, $a'\in A'$ such that $a\Bcup
a'\Bsubseteq b$.   Show that for any non-negative $\nu\in M$, the
component $\nu_1$ of $\nu$ in $M_{\Cal A}$ is given by the formula
$\nu_1c=\inf_{A\in\Cal A}\sup_{a\in A}\nu(c\Bsetminus a)$, so that the
component $\nu_2$ of $\nu$ in $M_{\Cal A}^{\perp}$ is given by the
formula $\nu_2c=\sup_{A\in\Cal A}\inf_{a\in A}\nu(c\Bcap a)$.
(Cf.\ 356Yb.)

\leader{362Y}{Further exercises (a)}
%\spheader 362Ya
Let $\frak A$ be a Boolean algebra.
Let $\frak C$ be the band algebra of the Riesz space $M$ of bounded
finitely additive functionals on $\frak A$ (353B).   Show that the
bands $M_{\sigma}$, $M_{\tau}$, $M_c$ (362B, 362Xf)  generate a
subalgebra $\frak C_0$ of $\frak C$ with at most six atoms.   Give an
example in which $\frak C_0$ has six atoms.   How many atoms can it have
if (i) $\frak A$ is atomless (ii) $\frak A$ is purely atomic (iii)
$\frak A$ is Dedekind $\sigma$-complete?
%362B

\spheader 362Yb Let $(\frak A,\bar\mu)$ be a measure algebra.   Let
$\frak C$ be the band algebra of the Riesz space $M$ of bounded
finitely additive functionals on $\frak A$.   Show that the bands
$M_{\sigma}$, $M_{\tau}$, $M_c$, $M_{ac}$, $M_{tc}$, $M_t$ (362B, 362C,
362Xf)  generate a subalgebra $\frak C_0$ of $\frak C$ with at
most twelve atoms.   Give an example in which $\frak C_0$ has twelve
atoms.   How many atoms can it have if (i) $\frak A$ is atomless (ii)
$\frak A$ is purely atomic (iii) $(\frak A,\bar\mu)$ is semi-finite (iv)
$(\frak A,\bar\mu)$ is localizable (v) $(\frak A,\bar\mu)$ is
$\sigma$-finite (vi) $(\frak A,\bar\mu)$ is totally finite?
%362C

\spheader 362Yc Give an example of a set $X$, a $\sigma$-algebra
$\Sigma$ of subsets of $X$, and a functional in $M_p$ (as defined in
362Xg) which is not completely additive.

\spheader 362Yd Let $U$ be a Riesz space and $f$, $g\in U^{\sim}$.
Show that the following are equiveridical:  ($\alpha$) $g$ is in the
band in $U^{\sim}$ generated by $f$;  ($\beta$) for every $u\in U^+$,
$\epsilon>0$ there is a $\delta>0$ such that $|g(v)|\le\epsilon$
whenever $0\le v\le u$ and $|f|(v)\le\delta$;   ($\gamma$)
$\lim_{n\to\infty}g(u_n)=0$ whenever $\sequencen{u_n}$ is a
non-increasing sequence in $U^+$ and $\lim_{n\to\infty}|f|(u_n)=0$.
%362B

\spheader 362Ye Let $\frak A$ be a weakly $\sigma$-distributive Boolean
algebra (316Ye).   Show that the `inf' in the formula for
$P_{\sigma}\nu$ in 362Bc can be replaced by `min'.
%362D

\spheader 362Yf Let $\frak A$ be any Boolean algebra and $M$ the space
of bounded finitely additive functionals on $\frak A$.   Let
$C\subseteq M$ be such that $\sup_{\nu\in C}|\nu a|<\infty$ for every
$a\in\frak A$.
(i) Suppose that $\sup_{n\in\Bbb N}\sup_{\nu\in C}|\nu a_n|$ is finite
for every disjoint sequence $\sequencen{a_n}$ in $\frak A$.   Show that
$C$ is norm-bounded.   (ii) Suppose that
$\lim_{n\to\infty}\sup_{\nu\in C}|\nu a_n|=0$ for every disjoint
sequence $\sequencen{a_n}$ in $\frak A$.   Show that $C$ is uniformly
integrable.
%362E

\spheader 362Yg Let $\frak A$ be a Boolean algebra and $M_{\tau}$ the
space of completely additive functionals on $\frak A$.   Let
$C\subseteq M_{\tau}$ be such that $\sup_{\nu\in C}|\nu a|<\infty$ for
every atom $a\in\frak A$.  (i) Suppose that
$\sup_{n\in\Bbb N}\sup_{\nu\in C}|\nu a_n|$ is finite for every disjoint
sequence $\sequencen{a_n}$ in $\frak A$.   Show that $C$ is
norm-bounded.   (ii) Suppose that
$\lim_{n\to\infty}\sup_{\nu\in C}|\nu a_n|=0$ for every disjoint
sequence $\sequencen{a_n}$ in $\frak A$.   Show that $C$ is uniformly
integrable.
%362E

\spheader 362Yh Let $\frak A$ be a Dedekind $\sigma$-complete Boolean
algebra and $\sequencen{\nu_n}$ a sequence of countably additive
real-valued functionals on $\frak A$.   Suppose that
$\nu a=\lim_{n\to\infty}\nu_na$ is defined in $\Bbb R$ for every
$a\in\frak A$.   Show that $\nu$ is countably additive and that
$\{\nu_n:n\in\Bbb N\}$ is uniformly integrable.   \Hint{246Yg.}   Show
that if every $\nu_n$ is completely additive, so is $\nu$.
%362E

\spheader 362Yi Let $\frak A$ be a Boolean algebra, $M$ the Riesz space
of bounded finitely additive functionals on $\frak A$, and
$M_c\subseteq M$ the band of properly atomless functionals (362Xf).
Show that for a
non-negative $\nu\in M$ the component $\nu_c$ of $\nu$ in $M_c$ is given
by the formula

\Centerline{$\nu_ca=\inf_{\delta>0}\sup\{\sum_{i=0}^n\nu a_i:
a_0,\ldots,a_n\Bsubseteq a$ are disjoint, $\nu a_i\le\delta$ for every
$i\}$}

\noindent for each $a\in\frak A$.

\spheader 362Yj Let $\frak A$ be a Boolean algebra and $M$ the
$L$-space of bounded additive real-valued functionals on $\frak A$.
Show that the complexification of $M$, as defined in 354Yl, can be
identified with the Banach space of bounded additive functionals
$\nu:\frak A\to\Bbb C$, writing

\Centerline{$\|\nu\|=\sup\{\sum_{i=0}^n|\nu a_i|:a_0,\ldots,a_n$ are
disjoint elements of $\frak A\}$}

\noindent for such $\nu$.

\spheader 362Yk Let $\frak A$ be a Boolean algebra and $M$ the $L$-space of
bounded additive real-valued functionals on $\frak A$.   Suppose that $M_0$
is a norm-closed linear subspace of $M$ and that
$a\mapsto\nu(a\Bcap c):\frak A\to\Bbb R$ belongs to $M_0$
whenever $\nu\in M_0$ and $c\in\frak A$.   Show that $M_0$ is a band in
$M$.   \Hint{436L.}
}%end of exercises

\cmmnt{
\Notesheader{362} The Boolean algebras most immediately important in
measure theory are of course $\sigma$-algebras of measurable sets and
their quotient measure algebras.   It is therefore natural to begin any
investigation by concentrating on Dedekind $\sigma$-complete algebras.
Nevertheless, in this section and the last (and in \S326), I have gone
to some trouble not to specialize to $\sigma$-complete algebras except
when necessary.   Partly this is just force of habit, but partly it is
because I wish to lay a foundation for a further step forward:  the
investigation of the ways in which additive functionals on general
Boolean algebras reflect the concepts of measure theory, and indeed can
generate them.   Some of the results in this direction can be
surprising.   I do not think it obvious that the condition ($\gamma$) in
362B(f-i), for instance, is sufficient in the absence of any hypothesis
of Dedekind $\sigma$-completeness or countable additivity.

Given a Boolean algebra $\frak A$ with the associated Riesz space
$M\cong S(\frak A)^{\sim}$ of bounded additive functionals on
$\frak A$, we now have a substantial list of bands in $M$:  $M_{\sigma}$,
$M_{\tau}$, $M_c$ (362Xf), and for a measure algebra the further bands
$M_{ac}$,
$M_{tc}$ and $M_t$;  for an algebra of sets we also have $M_p$ (362Xg).
These bands can be used to generate finite subalgebras of the band
algebra of $M$ (362Ya-362Yb), and for any such finite subalgebra we have
a corresponding decomposition of $M$ as a direct sum of the bands which
are the atoms of the subalgebra (352Tb).   This decomposition of $M$
can be regarded as a recipe for decomposing its members into finite sums
of functionals with special properties.   What I called the `Lebesgue
decomposition' in 232I is just such a recipe.   In that context I had
a measure space $(X,\Sigma,\mu)$ and was looking at the countably
additive functionals from $\Sigma$ to $\Bbb R$, that is, at $M_{\sigma}$
in the language of this section, and the bands involved in the
decomposition were $M_p$, $M_{ac}$ and $M_{tc}$.   But I hope that it
will be plain that these ideas can be refined indefinitely, as we refine
the classification of additive functionals.   At each stage, of course,
the exact enumeration of the subalgebra of bands generated by the
classification (as in 362Ya-362Yb) is a necessary check that we have
understood the relationships between the classes we have described.

These decompositions are of such importance that it is worth examining
the corresponding band projections.   I give formulae for the action of
band projections on (non-negative) functionals in 362Bc, 362Bd,
362B(f-ii),
362Xb, 362Xc(iii), 362Xi(iv) and 362Yi.   Of course these are readily
adapted to give formulae for the projections onto the complementary
bands, as in 362Bf and 362Xi.

If we have an algebra of sets, the completely additive functionals are
(usually) of relatively minor importance;  in the standard examples,
they correspond to functionals defined as weighted sums of point masses
(362Xg(ii)).   The point is that measure algebras $\frak A$ appear as
quotients of $\sigma$-algebras $\Sigma$ of sets by $\sigma$-ideals $\Cal
I$;  consequently the countably additive functionals on $\frak A$
correspond exactly to the countably additive functionals on $\Sigma$
which are zero on $\Cal I$;  but the canonical homomorphism from
$\Sigma$ to $\frak A$ is hardly ever order-continuous, so
completely additive functionals on $\frak A$ rarely correspond to
completely additive
functionals on $\Sigma$.
On the other hand, when we are looking at countably additive functionals
on $\Sigma$, we have to consider the possibility that they are singular
in the sense that they are carried on some member of $\Cal I$;  in the
measure algebra context this possibility disappears, and we can often be
sure that every countably additive functional is absolutely continuous,
as in 327Bb.

For any Boolean algebra $\frak A$, we can regard it as the algebra of
open-and-closed subsets of its Stone space $Z$;  the points of $Z$
correspond to Boolean homomorphisms from $\frak A$ to $\{0,1\}$, which
are the normalised `atomic elements'
in the space of additive functionals on
$\frak A$ (362Xe, 362Xg(iv)).   It is the case that all non-negative
additive functionals on a Boolean algebra $\frak A$ can be represented
by appropriate measures on its Stone space (see 416Q in Volume 4), but I
prefer to
hold this result back until it can take its place among other theorems
on representing functionals by measures and integrals.

It is one of the leitmotivs of this chapter, that Boolean algebras and
Riesz spaces are Siamese twins;  again and again, matching results are
proved by the application of identical ideas.   A typical example is the
pair 362B(f-i) and 362Yd.   Many of us have been tempted to try to
describe something which would provide a common generalization of
Boolean algebras and Riesz spaces (and lattice-ordered groups).   I have
not yet seen any such structure which was worth the trouble.   Most of
the time, in this chapter, I shall be using ideas from the general
theory of Riesz spaces to suggest and illuminate questions in measure
theory;  but if you pursue this subject you will surely find that
intuitions often come to you first in the context of Boolean algebras,
and the applications to Riesz spaces are secondary.

In 362E I give a condition for uniform integrability in terms of
disjoint sequences, following the pattern established in 246G and
repeated in 354R and 356O.   The condition of 362E assumes that
the set is norm-bounded;  but if you have 246G to hand, you will see
that it can be done with weaker assumptions involving atoms, as in
362Yf-362Yg.

I mention once again the Banach-Ulam problem:  if $\frak A$ is Dedekind
complete, can $S(\frak A)^{\sim}_c$ be different from
$S(\frak A)^{\times}$?   This is obviously equivalent to the form given
in the notes to \S326 above.   See 363S below.
}%end of comment

\discrpage

