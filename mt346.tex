\frfilename{mt346.tex}
\versiondate{17.12.10}
\copyrightdate{1995}

\def\undtheta{\underline{\theta}}
\def\undphi{{\underline{\phi}}}
\def\undpsi{\underline{\psi}}
\def\upr{\mathop{\text{upr}}}
\def\tmputp{\underline{\tilde\phi}}

\def\chaptername{Liftings}
\def\sectionname{Consistent liftings}

\newsection{346}

I turn now to a different type of condition
which we should naturally prefer our liftings to satisfy.   If
we have a product measure $\mu$ on a product $X=\prod_{i\in I}X_i$ of
probability spaces, then we can look for liftings $\phi$ which `respect
coordinates', that is, are
compatible with the product structure in the sense that they factor
through subproducts (346A).   There seem to be obstacles in the way of
the natural conjecture (346Za), and I give the partial results which are
known.   For \Mth\ spaces $X_i$, there is always a
lifting which respects coordinates (346E), and indeed the
translation-invariant liftings of \S345
on $\{0,1\}^I$ already have this property (346C).   There is always a
lower density for the product measure
which respects coordinates, and we can ask
for a little more (346G);  using the full strength of 346G, we can
enlarge this lower density to a lifting which respects single
coordinates and initial segments of a well-ordered product (346H).   In
the case in which all the factors are copies of each other, we can
arrange for the induced liftings on the factors to be copies also (346I,
346J\cmmnt{, 346Ye}).   I end the section with an important fact about
Stone spaces which is relevant here (346K-346L).

\leader{346A}{Definition} Let
$\langle(X_i,\Sigma_i,\mu_i)\rangle_{i\in I}$ be a family of
probability spaces, with product $(X,\Sigma,\mu)$.   I will say that
a lifting $\phi:\Sigma\to\Sigma$ {\bf respects coordinates} if $\phi E$
is determined by coordinates in $J$ whenever $E\in\Sigma$ is determined
by coordinates in $J\subseteq I$.

\cmmnt{
\medskip

\noindent{\bf Remark} Recall that a set $E\subseteq X$ is
`determined by coordinates in $J$' if $x'\in E$ whenever $x\in E$,
$x'\in X$ and $x'\restr J=x\restr J$;  that is, if $E$ is expressible as
$\pi_J^{-1}[F]$ for some $F\subseteq\prod_{i\in J}X_i$, where
$\pi_J(x)=x\restr J$ for every $x\in X$;  that is, if
$E=\pi_J^{-1}[\pi_J[E]]$.   See 254M.   Recall also that in this case,
if $E$ is measured by the product measure on $X$, then $\pi_J[E]$ is
measured by the product measure on $\prod_{i\in J}X_i$ (254Ob).
}%end of comment

\leader{346B}{Lemma} (a)\dvAnew{2010} Let $(X,\Sigma,\mu)$ be a measure
space with a lifting $\phi:\Sigma\to\Sigma$.   Suppose that $Y$ is a set
and $f:X\to Y$ a 
surjective function such that whenever $E\in\Sigma$ is such that
$f^{-1}[f[E]]=E$, then $f^{-1}[f[\phi E]]=\phi E$.
Then we have a lifting 
$\psi$ for the image measure $\mu f^{-1}$ defined by the formula

\Centerline{$f^{-1}[\psi F]=\phi(f^{-1}[F])$ whenever $F\subseteq Y$ and
$f^{-1}[F]\in\Sigma$.}

(b) Let
$\langle(X_i,\Sigma_i,\mu_i)\rangle_{i\in I}$ be a family of
probability spaces, with product $(Z,\Lambda,\lambda)$.   For
$J\subseteq I$ let $(Z_J,\Lambda_J,\lambda_J)$ be the product of
$\langle(X_i,\Sigma_i,\mu_i)\rangle_{i\in J}$, and $\pi_J:Z\to Z_J$
the canonical map.   Let $\phi:\Lambda\to\Lambda$ be a lifting.   If
$J\subseteq I$ is such that $\phi W$ is determined by coordinates in $J$
whenever $W\in\Lambda$ is determined by coordinates in $J$, then $\phi$
induces a lifting $\phi_J:\Lambda_J\to\Lambda_J$ defined by the formula

\Centerline{$\pi_J^{-1}[\phi_J E]
=\phi(\pi_J^{-1}[E])$ for every $E\in\Lambda_J$.}

\proof{{\bf (a)} Set $\psi F=f[\phi(f^{-1}[F])]$ for
$F\in\dom(\mu f^{-1})$.   Because $f$ is surjective, $\psi Y=Y$,
and it is now elementary to check that
$\psi$ is a lifting for $\mu f^{-1}$.

\medskip

{\bf (b)} By 254Oa, $\lambda_J$ is the image measure $\lambda\pi_J^{-1}$,
so we can use (a).
}%end of proof of 346B

\cmmnt{\medskip

\noindent{\bf Remark} Of course we frequently wish to use part (b) here
with a singleton set $J=\{j\}$.   In this case we must remember that
$(Z_J,\Sigma_J,\lambda_J)$ corresponds to the {\it completion} of
the probability space $(X_j,\Sigma_j,\mu_j)$.
}%end of comment

\leader{346C}{Theorem} Let $I$ be any set, and $\nu_I$ the usual
measure on $\cmmnt{X=\mskip5mu}\{0,1\}^I$.   
Then any translation-invariant lifting for $\nu_I$ respects coordinates.

\proof{ Suppose that $E\subseteq X$ is a measurable set determined by
coordinates in $J\subseteq I$;  take $x\in\phi E$ and $x'\in X$ such
that $x'\restr J=x\restr J$.   Set $y=x'-x$;  then $y(i)=0$ for
$i\in J$, so that $E+y=y$.   Now

\Centerline{$x'=x+y\in\phi E + y=\phi(E+y)=\phi E$}

\noindent because $\phi$ is translation-invariant.   As $x$, $x'$ are
arbitrary, $\phi E$ is determined by coordinates in $J$.   As $E$ and
$J$ are arbitrary, $\phi$ respects coordinates.
}%end of proof of 346C

\leader{346D}{}\cmmnt{ I describe a standard method of constructing
liftings from other liftings.

\medskip

\noindent}{\bf Lemma} Let $(X,\Sigma,\mu)$ and $(Y,\Tau,\nu)$ be
measure spaces, with measure algebras $\frak A$, $\frak B$;  suppose
that $f:X\to Y$ represents an isomorphism
$F^{\ssbullet}\mapsto f^{-1}[F]^{\ssbullet}:\frak B\to\frak A$.   Then
if $\phi:\Tau\to\Tau$ is a lifting for $\nu$, there is a
corresponding lifting $\phi':\Sigma\to\Sigma$ given by the formula

\Centerline{$\phi'E=f^{-1}[\phi F]$ whenever
$\mu(E\symmdiff f^{-1}[F])=0$.}

\proof{ If we say that $\pi:\frak B\to\frak A$ is the isomorphism
induced by $f$, then

\Centerline{$\phi'E=f^{-1}[\theta(\pi^{-1} E^{\ssbullet})]$,}

\noindent where $\theta:\frak B\to\Tau$ is the lifting corresponding to
$\phi:\Tau\to\Tau$.   Since $\theta$, $\pi^{-1}$ and
$F\mapsto f^{-1}[F]$ are all Boolean homomorphisms, so is $\phi'$, and
it is easy to check that $(\phi'E)^{\ssbullet}=E^{\ssbullet}$ for every
$E\in\Sigma$ and that $\phi'E=\emptyset$ if $\mu E=0$.
}%end of proof of 346D

\cmmnt{\medskip

\noindent{\bf Remark} Compare the construction in 341P.
}

\leader{346E}{Theorem} Let
$\langle(X_i,\Sigma_i,\mu_i)\rangle_{i\in I}$ be a family of \Mth\
probability spaces, with product
$(X,\Sigma,\mu)$.   Then there is a lifting for $\mu$ which
respects coordinates.

\proof{{\bf (a)} Replacing each $\mu_i$ by its completion does not
change $\mu$ (254I), so we may suppose that all the $\mu_i$ are
complete.   In this case there is for each $i$ an isomorphism between
the measure algebra $(\frak A_i,\bar\mu_i)$ of $\mu_i$ and the measure
algebra $(\frak B_{J_i},\bar\nu_{J_i})$ of some $\{0,1\}^{J_i}$ with its
usual
measure $\nu_{J_i}$ (331L).   We may suppose that the sets $J_i$ are
disjoint.   Each $\nu_{J_i}$ is compact (342Jd), so the isomorphisms are
represented by \imp\ functions $f_i:X_i\to\{0,1\}^{J_i}$ (343Ca).

Set $K=\bigcup_{i\in I}J_i$, and let $\nu_K$ be the usual measure on
$Y=\{0,1\}^K$, $\Tau_K$ its domain.  We have a natural bijection between
$\prod_{i\in I}\{0,1\}^{J_i}$ and $Y$, so we obtain a function
$f:X\to Y$;  literally speaking,

\Centerline{$f(x)(j)=f_i(x(i))(j)$}

\noindent for $i\in I$, $j\in J_i$ and $x\in X$.

\medskip

{\bf (b)} Now $f$ is
\imp\ and induces an isomorphism between the measure algebras $\frak A$,
$\frak B_K$ of $\mu$, $\nu_K$.

\medskip

\Prf{\bf (i)} If $L\subseteq K$ is finite and $z\in\{0,1\}^L$, then,
setting $L_i=L\cap J_i$ for $i\in I$,

$$\eqalignno{\mu\{x:x\in X,\,f(x)\restr L=z\}
&=\mu(\prod_{i\in I}\{w:w\in X_i,\,
  f_i(w)\restr L_i=z\restr L_i\})\cr
&=\prod_{i\in I}\mu_i\{w:w\in X_i,\,
  f_i(w)\restr L_i=z\restr L_i\}\cr
&=\prod_{i\in I}\nu_{J_i}\{v:v\in\{0,1\}^{J_i},\,
  v\restr L_i=z\restr L_i\}\cr
\noalign{\noindent (because every $f_i$ is \imp)}
&=\prod_{i\in I}2^{-\#(L_i)}
=2^{-\#(L)}
=\nu_K\{y:y\in Y,\,y\restr L=z\}.\cr}$$

\noindent So $\mu f^{-1}[C]=\nu_KC$ for every basic cylinder set
$C\subseteq Y$.   By 254G, $f$ is \imp.

\medskip

\quad{\bf (ii)} Accordingly $f$ induces a measure-preserving
homomorphism
$\pi:\frak B_K\to\frak A$.   To see that $\pi$ is surjective, consider

\Centerline{$\Lambda'=\{E:E$ is $\Sigma$-measurable,
$E^{\ssbullet}\in\pi[\frak B_K]\}$.}

\noindent Because $\pi[\frak B_K]$ is a closed subalgebra of $\frak A$
(324Kb), $\Lambda'$ is a $\sigma$-subalgebra of the domain $\Lambda$ of
$\mu$, and of course it contains all $\mu$-negligible sets.   If
$i\in J$ and $G\in\Sigma_i$, then there is an $H\subseteq\{0,1\}^{J_i}$
such that $G\symmdiff f_i^{-1}[H]$ is $\mu_i$-negligible.   Now if
$E=\{x:x\in X,\,x(i)\in G\}$ and $F=\{y:y\in Y,\,y\restr J_i\in H\}$,

\Centerline{$E\symmdiff f^{-1}[F]
=\{x:x(i)\in G\symmdiff f_i^{-1}[H]\}$}

\noindent is $\mu$-negligible, and $E\in\Lambda'$.   But this means
that $\Lambda'\supseteq\Tensorhat_{i\in I}\Sigma_i$, and must therefore
be the whole of $\Lambda$ (254Ff).
\Qed

\medskip

{\bf (c)} By 345C, there is a translation-invariant lifting $\phi$ for
$\nu_K$;  by 346C, this respects coordinates.   By 346D, we have a
corresponding lifting $\phi'$ for $\mu$ such that

\Centerline{$\phi'f^{-1}[F]=f^{-1}[\phi F]$}

\noindent for every $F\in\Tau_K$.   Now suppose that
$E\in\Lambda$ is determined by coordinates in $L\subseteq I$.   Then
there is an $E'$ belonging to the $\sigma$-algebra $\Lambda'_L$
generated by

\Centerline{$\{\{x:x(i)\in G\}:i\in L,\,G\in\Sigma_i\}$}

\noindent such that $\mu(E\symmdiff E')=0$ (254Ob).   Write
$\Tau_L$ for the family of sets in $\Tau_K$ determined by coordinates in
$\bigcup_{i\in L}J_i$.   Then, just as in (b-ii), every member of
$\Lambda'_L$ differs by a negligible set from some set of the form
$f^{-1}[F]$ with $F\in\Tau_L$.   So there is an $F\in\Tau_L$ such that
$E\symmdiff f^{-1}[F]$ is $\mu$-negligible.   Consequently

\Centerline{$\phi'E=\phi'f^{-1}[F]=f^{-1}[\phi F]$.}

\noindent But $\phi$ respects coordinates, so $\phi F$ is determined by
coordinates in $\bigcup_{i\in L}J_i$.   It follows at once that
$f^{-1}[\phi F]$ is determined by coordinates in $L$;  that is, that
$\phi'E$ is determined by coordinates in $L$.   As $E$ and $L$ are
arbitrary, $\phi'$ respects coordinates, and witnesses the truth of the
theorem.
}%end of proof of 346E

\leader{346F}{}\cmmnt{ It seems to be unknown whether 346E is true of
arbitrary probability spaces (346Za);  I give some partial results in
this direction.   The following general method of constructing lower
densities will be useful.

\medskip

\noindent}{\bf Lemma} Let $(X,\Sigma,\mu)$ and $(Y,\Tau,\nu)$ be
complete probability spaces, with product $(X\times
Y,\Lambda,\lambda)$.   If $\undphi:\Lambda\to\Lambda$ is a lower
density, then we have a lower density $\undphi_1:\Sigma\to\Sigma$
defined by saying that

\Centerline{$\undphi_1E=\{x:x\in X,\,\{y:(x,y)\in\undphi(E\times Y)\}$
is conegligible in $Y\}$}

\noindent for every $E\in\Sigma$.

\proof{ For $E\in\Sigma$, $(E\times Y)\symmdiff\undphi(E\times Y)$ is
negligible, so that

\Centerline{$H_x
=\{y:(x,y)\in (E\times Y)\symmdiff\undphi(E\times Y)\}$}

\noindent is $\nu$-negligible for almost
every $x\in X$ (252D).   Now $E\symmdiff\undphi_1E=\{x:H_x$ is not
negligible$\}$ is negligible, so $\undphi_1E\in\Sigma$.
If $E$, $F\in\Sigma$, then

\Centerline{$\undphi((E\cap F)\times Y)=\undphi((E\times Y)\cap(F\times
Y))=\undphi(E\times Y)\cap\undphi(F\times Y)$,}

\noindent so that

\Centerline{$\{y:(x,y)\in\undphi((E\cap F)\times Y)\}
=\{y:(x,y)\in\undphi(E\times Y)\}\cap\{y:(x,y)\in\undphi(F\times Y)\}$}

\noindent is conegligible iff both
$\{y:(x,y)\in\undphi(E\times Y)\}$ and $\{y:(x,y)\in\undphi(F\times
Y)\}$ are conegligible, and $\undphi_1(E\cap
F)=\undphi_1E\cap\undphi_1F$.

The rest is easy.   Of course $\undphi(\emptyset\times Y)=\emptyset$ so
$\undphi_1\emptyset=\emptyset$.   If $E$, $F\in\Sigma$ and $E\symmdiff
F$ is negligible, then $(E\times Y)\symmdiff(F\times Y)$ is negligible,
$\undphi(E\times Y)=\undphi(F\times Y)$ and $\undphi_1E=\undphi_1F$.
So $\undphi_1$ is a lower density, as claimed.
}%end of proof of 346F

\leader{346G}{Theorem} Let
$\langle(X_i,\Sigma_i,\mu_i)\rangle_{i\in I}$ be a family of probability
spaces with product
$(X,\Sigma,\mu)$.   For $J\subseteq I$ let $\Sigma_J$ be the
set of members of $\Sigma$ which are determined by coordinates in $J$.
Then there is a lower density $\undphi:\Sigma\to\Sigma$ such that

(i) whenever $J\subseteq I$ and $E\in\Sigma_J$ then
$\undphi E\in\Sigma_J$,

(ii) whenever $J$, $K\subseteq I$ are disjoint, $E\in\Sigma_J$ and
$F\in\Sigma_K$ then $\undphi(E\cup F)=\undphi E\cup\undphi F$.

\proof{ For each $i\in I$, set $Y_i=X_i^{\Bbb N}$, with the product
measure $\nu_i$;  set $Y=\prod_{i\in I}Y_i$, with its product
measure $\nu$;  set $Z_i=X_i\times Y_i$, with its product measure
$\lambda_i$, and $Z=\prod_{i\in I}Z_i$, with its product measure
$\lambda$.   Then the natural identification of
$Z=\prod_{i\in I}X_i\times Y_i$ with
$\prod_{i\in I}X_i\times\prod_{i\in I}Y_i=X\times Y$ makes $\lambda$
correspond to the product of $\mu$ and $\nu$ (254N).

Each $(Z_i,\lambda_i)$ can be identified with an infinite power of
$(X_i,\mu_i)$, and is therefore \Mth\ (334E).
Consequently there is a lifting $\phi:\Lambda\to\Lambda$ which
respects coordinates (346E).   Regarding $(Z,\lambda)$ as the
product of $(X,\mu)$ and $(Y,\nu)$, we see that $\phi$ induces
a lower density $\undphi:\Sigma\to\Sigma$ by the formula of 346F.

If $J\subseteq I$ and $E\in\Sigma$ is determined by coordinates in $J$,
then $E\times Y$ (regarded as a subset of $\prod_{i\in I}Z_i$) is
determined by coordinates in $J$, so $\phi(E\times Y)$ also is.   Now
suppose that $x\in\undphi E$, $x'\in X$ and $x\restr J=x'\restr J$.
Then for any $y\in Y$, $(x\restr J,y\restr J)=(x'\restr J,y\restr J)$,
so $(x,y)\in\phi(E\times Y)$ iff $(x',y)\in\phi(E\times Y)$.   Thus

\Centerline{$\{y:(x',y)\in\phi(E\times
Y)\}=\{y:(x,y)\in\phi(E\times Y)\}$}

\noindent is conegligible in $Y$, and $x'\in\undphi E$.   This shows
that $\undphi E$ is determined by coordinates in $J$.

Now suppose that $J$ and $K$ are disjoint subsets of $I$, that $E$,
$F\in\Sigma$ are determined by coordinates in $J$, $K$ respectively, and
that $x\notin\undphi E\cup\undphi F$.   Then
$A=\{y:(x,y)\notin\phi(E\times Y)\}$ and
$B=\{y:(x,y)\notin\phi(F\times Y)\}$ are non-negligible.   As noted
just above, $\phi(E\times Y)$ is determined by coordinates in $J$, so
$A$ is determined by coordinates in $J$, and can be expressed as
$\{y:y\restr J\in A'\}$, where $A'\subseteq Y_J=\prod_{i\in J}Y_i$.
Because $y\mapsto y\restr J:Y\to Y_J$ is \imp, $A'$ cannot be negligible
in $Y_J$.    Similarly, $B$ can be expressed as $\{y:y\restr K\in B'\}$
for some non-negligible $B'\subseteq Y_K$.

By 251S/251Wm, $A'\times B'\times Y_{I\setminus(J\cup K)}$, regarded as
a subset of $Y$, is non-negligible, that is,

\Centerline{$C=\{y:y\in Y,\,y\restr J\in A',\,y\restr K\in B'\}$}

\noindent is non-negligible.   But

\Centerline{$C=A\cap B
=\{y:(x,y)\notin\phi(E\times Y)\cup\phi(F\times Y)\}
=\{y:(x,y)\notin\phi((E\cup F)\times Y\}$.}

\noindent So $x\notin\undphi(E\cup F)$.   As $x$ is arbitrary,
$\undphi(E\cup F)\subseteq\undphi E\cup\undphi F$;  but of course
$\undphi E\cup\undphi F\subseteq\undphi(E\cup F)$, because $\undphi$ is
a lower density, so that $\undphi(E\cup F)=\undphi E\cup\undphi F$, as
required.
}%end of proof of 346G

\cmmnt{\medskip

\noindent{\bf Remark} See {\smc Macheras Musia\l\ \& Strauss 99} for an
alternative proof.
}%end of comment

\leader{346H}{Theorem} Let $\zeta$ be an ordinal, and
$\langle(X_{\xi},\Sigma_{\xi},\mu_{\xi})\rangle_{\xi<\zeta}$ a family
of probability spaces, with product $(Z,\Lambda,\lambda)$.
For $J\subseteq\zeta$ let $\Lambda_J$ be the set of those $W\in\Lambda$
which are determined by coordinates in $J$.   Then there is a lifting
$\phi:\Lambda\to\Lambda$ such that $\phi W\in\Lambda_J$ whenever
$W\in\Lambda_J$ and $J$ is {\it either} a singleton subset of $\zeta$
{\it or} an initial segment of $\zeta$.

\proof{{\bf (a)} Let $P$ be the set of all lower densities
$\undphi:\Lambda\to\Lambda$ such that, for every $\xi<\zeta$, (i)
whenever $E\in\Lambda_{\xi}$ then $\undphi
E\in\Lambda_{\xi}$ (ii) whenever $E\in\Lambda_{\{\xi\}}$ then $\undphi
E\in\Lambda_{\{\xi\}}$  (iii) whenever $E\in\Lambda_{\xi}$ and
$F\in\Lambda_{\zeta\setminus\xi}$ then $\undphi(E\cup F)=\undphi
E\cup\undphi F$.   By 346G, $P$ is not empty.   Order $P$ by saying that
$\undphi\le\undphi'$ if $\undphi E\subseteq\undphi'E$ for every
$E\in\Lambda$;  then $P$ is a partially ordered set.   Note that if
$\undphi\in P$ then $\undphi Z=Z$ (because $\Lambda_0=\{\emptyset,Z\}$).

\medskip

{\bf (b)} Any non-empty totally ordered subset $Q$ of $P$ has an upper
bound in $P$.   \Prf\ Define $\undphi^*:\Lambda\to\Cal PX$ by setting
$\undphi^*E=\bigcup_{\undphi\in Q}\undphi E$ for every $E\in\Lambda$.
(i)

\Centerline{$\undphi^*\emptyset=\bigcup_{\undphi\in
Q}\emptyset=\emptyset$.}

\noindent (ii) If $E$, $F\in\Lambda$ and $\lambda(E\symmdiff F)=0$
then $\undphi E=\undphi F$ for every $\undphi\in Q$ so
$\undphi^*E=\undphi^*F$.   (iii) If $E$, $F\in\Lambda$ and $E\subseteq
F$ then $\undphi E\subseteq\undphi F$ for every $\undphi\in Q$ so
$\undphi^*E\subseteq\undphi^*F$.   (iv) If $E$, $F\in\Lambda$ and
$x\in\undphi^*E\cap\undphi^*F$, then there are $\undphi_1$,
$\undphi_2\in Q$ such that $x\in\undphi_1E\cap\undphi_2F$;   now either
$\undphi_1\le\undphi_2$ or $\undphi_2\le\undphi_1$, so that

\Centerline{$x\in(\undphi_1E\cap\undphi_1F)
\cup(\undphi_2E\cap\undphi_2F)=\undphi_1(E\cap F)\cup\undphi_2(E\cap F)
\subseteq\undphi^*(E\cap F)$.}

\noindent Accordingly
$\undphi^*E\cap\undphi^*F\subseteq\undphi^*(E\cap F)$ and
$\undphi^*E\cap\undphi^*F=\undphi^*(E\cap F)$.   (v) Taking any
$\undphi_0\in Q$, we have $\undphi_0E\subseteq\undphi^*E$ for every
$E\in\Lambda$, so (because $\lambda$ is complete) $\undphi^*$ is a
lower density, by 341Ib.   (vi) Now suppose that $J\subseteq I$ is
either a singleton $\{\xi\}$ or an initial segment $\xi$, and that
$E\in\Lambda_J$.   Then $\undphi E$ is determined by coordinates in $J$
for every $\undphi\in Q$, so $\undphi^*E$ is determined by coordinates
in $J$.   (vii) Finally, suppose that $\xi<\zeta$ and that
$E\in\Lambda_{\xi}$, $F\in\Lambda_{\zeta\setminus\xi}$.   If
$x\in\undphi^*(E\cup F)$ then there is a $\undphi\in Q$ such that

\Centerline{$x\in\undphi(E\cup F)=\undphi E\cup\undphi F
\subseteq\undphi^* E\cup\undphi^* F$.}

\noindent So $\undphi^*(E\cup F)\subseteq\undphi^*E\cup\undphi^*F$ and
(using (iii) again) $\undphi^*(E\cup F)=\undphi^*E\cup\undphi^*F$.
Thus
$\undphi^*$ belongs to $P$ and is an upper bound for $Q$ in $P$.\ \Qed

By Zorn's Lemma, $P$ has a maximal element $\tmputp$.

\medskip


{\bf (c)} For any $H\in\Lambda$ we may define a function $\undphi_H$ as
follows.   Set $A_H=Z\setminus(\tmputp H\cup\tmputp(Z\setminus H))$,

\Centerline{$\undphi_H E=\tmputp E\cup(A_H\cap \tmputp(H\cup E))$}

\noindent for $E\in\Lambda$.   Then $\undphi_H$ is a lower density.
\Prf\ (i) Because $H\symmdiff\tmputp H$ and $(Z\setminus
H)\symmdiff\tmputp(Z\setminus H)$ are both negligible, $A_H$ is
negligible and $\undphi_HE$ is measurable and
$(\undphi_H E)^{\ssbullet}=(\tmputp E)^{\ssbullet}=E^{\ssbullet}$ for
every $E\in\Lambda$.
(ii) Because $A_H\cap\tmputp H=\emptyset$,
$\undphi_H\emptyset=\emptyset$.
(iii) If $E$, $F\in\Lambda$ and $\lambda(E\symmdiff F)=0$ then
$\tmputp E=\tmputp F$ and $\tmputp(E\cup H)=\tmputp(F\cup H)$, so
$\undphi_H E=\undphi_H F$.
(iv) If $E$, $F\in\Lambda$ and $E\subseteq F$ then $\tmputp
E\subseteq\tmputp F$ and $\tmputp(E\cup H)\subseteq\tmputp(F\cup H)$, so
$\undphi_H E\subseteq\undphi_H F$.
(v) If $E$, $F\in\Lambda$ and $x\in\undphi_H E\cap\undphi_H F$, then

\qquad($\alpha$) if $x\notin A_H$,

\Centerline{$x\in\tmputp E\cap\tmputp
F=\tmputp(E\cap F)\subseteq\undphi_H(E\cap F)$,}

\qquad($\beta$) if $x\in A_H$,

\Centerline{$x\in\tmputp(E\cup H)\cap\tmputp(F\cup
H)=\tmputp((E\cap F)\cup H)\subseteq\undphi_H(E\cap F)$.}

\noindent Thus $\undphi_H E\cap\undphi_H F\subseteq\undphi_H(E\cap F)$
and $\undphi_H E\cap\undphi_H F=\undphi_H(E\cap F)$.\ \Qed

\medskip

{\bf (d)}
It is worth noting the following.

\medskip

\quad{\bf (i)} If $E$, $H\in\Lambda$ and $\tmputp(E\cup H)=\tmputp
E\cup\tmputp H$ then $\undphi_HE=\tmputp E$.   \Prf\ We have

\Centerline{$\undphi_H E
=\tmputp E\cup(A_H\cap\tmputp(E\cup H))
=\tmputp E\cup(A_H\cap\tmputp E)\cup(A_H\cap\tmputp H)
=\tmputp E$}

\noindent because $A_H\cap\tmputp H=\emptyset$.\ \Qed

\wheader{346H}{6}{2}{2}{36pt}

\quad{\bf (ii)} If $H\in\Lambda$ and $\undphi_H\in P$ then $\tmputp
H\cup\tmputp(Z\setminus H)=Z$.   \Prf\ By the maximality of $\tmputp$,
we must have $\undphi_H=\tmputp$.   But

\Centerline{$A_H=\undphi_H(Z\setminus H)\setminus\tmputp(Z\setminus
H)$,}

\noindent so $A_H=\emptyset$, that is, $\tmputp H\cup\tmputp(Z\setminus
H)=Z$.\ \Qed

\medskip

\quad{\bf (iii)} If $E$, $F\in\Lambda$ and
$\tmputp E\cup\tmputp(Z\setminus E)=Z$, then
$\tmputp(E\cup F)=\tmputp E\cup\tmputp F$.   \Prf\

\Centerline{$\tmputp(E\cup F)\setminus\tmputp E
=\tmputp(E\cup F)\cap\tmputp(Z\setminus E)
=\tmputp((E\cup F)\cap(Z\setminus E))
=\tmputp(F\setminus E)
\subseteq\tmputp F$,}

\noindent so  $\tmputp(E\cup F)\subseteq\tmputp E\cup\tmputp F$;  as the
reverse inclusion is true for all $E$ and $F$, we have the result.
\Qed

\medskip

{\bf (e)} If $\xi<\zeta$ and $H\in\Lambda_{\{\xi\}}$, then
$\undphi_H\in P$.

\medskip

\Prf{\bf (i)} If $J\subseteq I$ is either a singleton or an inital
segment, and $E\in\Lambda_J$, then

\qquad($\alpha$) if $\xi\in J$, $E\cup H$ and $\tmputp E$ and
$\tmputp(E\cup H)$ and $A_H$ all belong to $\Lambda_J$, so
$\undphi_HE\in\Lambda_J$.

\qquad($\beta$) If $\xi\notin J$,
$\tmputp(E\cup H)=\tmputp E\cup\tmputp H$, because there is some $\eta$
such that $J\subseteq\eta$ and $\{\xi\}\subseteq\zeta\setminus\eta$;
so $\undphi_HE=\tmputp E\in\Lambda_J$ by (d-i).

\medskip

\quad{\bf (ii)} If $\eta<\zeta$, $E\in\Lambda_{\eta}$ and
$F\in\Lambda_{\zeta\setminus\eta}$, then

\inset{if $\xi<\eta$, $E\cup H\in\Lambda_{\eta}$ so $\tmputp(E\cup F\cup
H)=\tmputp(E\cup H)\cup\tmputp F$, and

$$\eqalign{\undphi_H(E\cup F)
&=\tmputp(E\cup F)\cup(A_H\cap\tmputp(E\cup F\cup H))\cr
&=\tmputp E\cup\tmputp F\cup(A_H\cap\tmputp(E\cup H))
   \cup(A_H\cap\tmputp F)
\subseteq\undphi_H E\cup\undphi_H F;\cr}$$
}

\inset{if $\eta\le\xi$, $F\cup H\in\Lambda_{\zeta\setminus\eta}$ so
$\tmputp(E\cup F\cup H)=\tmputp(E)\cup\tmputp(F\cup H)$, and

$$\eqalign{\undphi_H(E\cup F)
&=\tmputp(E\cup F)\cup(A_H\cap\tmputp(E\cup F\cup H))\cr
&=\tmputp E\cup\tmputp F\cup(A_H\cap\tmputp E))
   \cup(A_H\cap\tmputp(F\cup H))
\subseteq\undphi_H E\cup\undphi_H F;\cr}$$
}

\noindent accordingly $\undphi_H(E\cup F)=\undphi_H E\cup\undphi_H F$.
\Qed


By (d-ii) we have

\Centerline{$\tmputp H\cup\tmputp(Z\setminus H)=Z$}

\noindent whenever $\xi<\zeta$ and $H\in\Lambda_{\{\xi\}}$.

\medskip

{\bf (f)} If $\xi\le\zeta$ and $H\in\Lambda_{\xi}$, then
$\undphi_H\in P$.   \Prf\ Induce on $\xi$.   For $\xi=0$,
$H\in\Lambda_0=\{\emptyset,Z\}$ so $\tmputp H$ is either $\emptyset$ or
$Z$, $A_H=\emptyset$ and $\undphi_H=\tmputp$ belongs to $P$.
For the inductive step to
$\xi\le\zeta$, we have the following.

\medskip

\quad{\bf (i)} If $\eta<\zeta$ and $E\in\Lambda_{\eta}$, then

\qquad($\alpha$) if $\xi\le\eta$, $E\cup H$ and $\tmputp E$ and
$\tmputp(E\cup H)$ and $A_H$ all belong to $\Lambda_{\eta}$, so
$\undphi_HE\in\Lambda_{\eta}$.

\qquad($\beta$) if $\eta<\xi$, then, by the inductive hypothesis,
$\undphi_E\in P$, $\tmputp E=Z\setminus\tmputp(Z\setminus E)$ and
$\tmputp(E\cup H)=\tmputp E\cup\tmputp H$, by (d-ii) and (d-iii) above;
so $\undphi_HE=\tmputp E\in\Lambda_{\eta}$ by (d-i).

\medskip

\quad{\bf (ii)} If $\eta<\zeta$ and $E\in\Lambda_{\{\eta\}}$, then,
by (e), $\tmputp E\cup\tmputp(Z\setminus E)=Z$, so that $\tmputp(E\cup
H)=\tmputp E\cup\tmputp H$, by (d-iii), and $\undphi_HE=\tmputp
E\in\Lambda_{\{\eta\}}$, by (d-i).

\medskip

\quad{\bf (iii)} If $\eta<\zeta$, $E\in\Lambda_{\eta}$ and
$F\in\Lambda_{\zeta\setminus\eta}$, then

\qquad($\alpha$) if $\xi\le\eta$, then $E\cup H\in\Lambda_{\eta}$ and
$F\in\Lambda_{\zeta\setminus\eta}$, so that
$\tmputp(E\cup F\cup H)=\tmputp(E\cup H)\cup\tmputp F$, and

$$\eqalign{\undphi_H(E\cup F)
&=\tmputp(E\cup F)\cup(A_H\cap\tmputp(E\cup F\cup H))\cr
&=\tmputp E\cup\tmputp F\cup(A_H\cap\tmputp(E\cup H))
   \cup(A_H\cap\tmputp F)
\subseteq\undphi_H E\cup\undphi_H F,\cr}$$


\noindent as in (e-ii) above, and accordingly
$\undphi_H(E\cup F)=\undphi_H E\cup\undphi_H F$.

\qquad($\beta$) If $\eta<\xi$ then, as in (ii), using the inductive
hypothesis, we have
$\tmputp(E\cup F\cup H)=\tmputp E\cup\tmputp(F\cup H)$, and (just as in
($\alpha$)) we get $\undphi_H(E\cup F)=\undphi_H E\cup\undphi_H F$.

Thus $\undphi_H\in P$ and the induction continues.
\Qed

\medskip

{\bf (g)} But the case $\xi=\zeta$ of (f) just tells us that

\Centerline{$\tmputp H\cup\tmputp(Z\setminus H)=Z$}

\noindent for every $H\in\Lambda$.   This means that $\tmputp$ is
actually a lifting (since it preserves intersections and complements).
And the definition of $P$ is just what is needed to ensure that it is a
lifting of the right type.
}%end of proof of 346H

\cmmnt{\medskip

\noindent{\bf Remark} This result is due to {\smc Macheras \& Strauss
96b}.
}%end of comment

\leader{346I}{Theorem} Let $(X,\Sigma,\mu)$ be a complete
probability space.   For any set $I$, write $\lambda_I$ for the
product measure on $X^I$, $\Lambda_I$ for its domain and
$\pi_{Ii}(x)=x(i)$ for $x\in X^I$, $i\in I$.   Then there is a lifting
$\psi:\Sigma\to\Sigma$ such that for every set $I$ there
is a lifting $\phi:\Lambda_I\to\Lambda_I$ such that
$\phi(\pi_{Ii}^{-1}[E])=\pi_{Ii}^{-1}[\psi E]$ whenever $E\in\Sigma$ and
$i\in I$.

\proof{ \Quer\ Suppose, if possible, otherwise.

Let $\Psi$ be the set of all liftings for $\mu$.   We are supposing
that for every $\psi\in\Psi$ there is a set $I_{\psi}$ for which there
is no lifting for $\lambda_{I_{\psi}}$ consistent with $\psi$ in the
sense above.   Let $\kappa$ be a cardinal greater than
$\max(\omega,\#(\Psi),\sup_{\psi\in\Psi}\#(I_{\psi}))$.   Let
$\phi_0:\Lambda_{\kappa}\to\Lambda_{\kappa}$ be a lifting satisfying the
conditions of 346H.   346Bb tells us that for every $\xi<\kappa$ we have
a lifting $\psi$ for $\mu$ defined by the formula
$\pi_{\kappa\xi}^{-1}[\psi E]=\phi_0(\pi_{\kappa\xi}^{-1}[E])$.
For $\psi\in\Psi$ set

\Centerline{$K_{\psi}
=\{\xi:\xi<\kappa,\,\phi_0(\pi_{\kappa\xi}^{-1}[E])
  =\pi_{\kappa\xi}^{-1}[\psi E]$ for every $E\in\Sigma\}$.}

\noindent Then $\bigcup_{\psi\in\Psi}K_{\psi}=\kappa$, so
$\kappa\le\max(\omega,\#(\Psi),\sup_{\psi\in\Psi}\#(K_{\psi}))$ and
there is some $\psi\in\Psi$ such that $\#(K_{\psi})>\#(I_{\psi})$.
Take $I\subseteq K_{\psi}$ such that $\#(I)=\#(I_{\psi})$.

We may regard $X^{\kappa}$ as $X^I\times X^{\kappa\setminus I}$, and in
this form we can use the method of 346F to obtain a lower density
$\undphi:\Lambda_I\to\Lambda_I$ from
$\phi_0:\Lambda_{\kappa}\to\Lambda_{\kappa}$.   Now

\Centerline{$\undphi(\pi_{I\xi}^{-1}[E])=\pi_{I\xi}^{-1}[\psi E]$ for
every $E\in\Sigma$, $\xi\in I$.}

\noindent\Prf\ The point is that
$\pi_{I\xi}^{-1}[E]\times X^{\kappa\setminus I}$ corresponds to
$\pi_{\kappa\xi}^{-1}[E]\subseteq X^{\kappa}$, while
$\phi_0(\pi_{\kappa\xi}^{-1}[E])=\pi_{\kappa\xi}^{-1}[\psi E]$ can be
identified with
$\pi_{I\xi}^{-1}[\psi E]\times X^{\kappa\setminus I}$.   Now the
construction
of 346F obviously makes $\undphi(\pi_{I\xi}^{-1}[E])$ equal to
$\pi_{I\xi}^{-1}[\psi E]$.\ \Qed

By 341Jb, there is a lifting $\phi:\Lambda_I\to\Lambda_I$ such that
$\phi W\supseteq\undphi W$ for every $W\in\Lambda_I$.   But now we must
have

$$\eqalign{\pi_{I\xi}^{-1}[\psi E]
&=\undphi(\pi_{I\xi}^{-1}[E])
\subseteq\phi(\pi_{I\xi}^{-1}[E])\cr
&=X^I\setminus\phi(\pi_{I\xi}^{-1}[X\setminus E])
\subseteq X^I\setminus\undphi(\pi_{I\xi}^{-1}[X\setminus E])\cr
&=X^I\setminus\pi_{I\xi}^{-1}[\psi(X\setminus E)]
=X^I\setminus\pi_{I\xi}^{-1}[X\setminus\psi E]
=\pi_{I\xi}^{-1}[\psi E]\cr}$$

\noindent and $\phi(\pi_{I\xi}^{-1}[E])=\pi_{I\xi}^{-1}[\psi E]$ for
every $E\in\Sigma$ and $\xi\in I$.   But since $\#(I)=\#(I_{\psi})$, this
must be impossible, by the choice of $I_{\psi}$.\ \Bang

This contradiction proves the theorem.
}%end of proof of 346I

\leader{346J}{Consistent liftings} Let $(X,\Sigma,\mu)$ be a measure
space.   A lifting $\psi:\Sigma\to\Sigma$ is {\bf consistent} if for
every $n\ge 1$ there is a lifting $\phi_n$ of the product measure on
$X^n$ such that $\phi_n(E_1\times\ldots\times E_n)
=\psi E_1\times\ldots\times\psi E_n$ for all $E_1,\ldots,E_n\in\Sigma$.
Thus\cmmnt{ 346I tells us, in part, that every complete probability
space has a consistent lifting;  it follows that} every non-trivial
complete totally finite measure space has a consistent lifting.

\cmmnt{I do not suppose you will be surprised to be told that not all
liftings on probability spaces are consistent.   What may be surprising
is the fact that one of the standard liftings already introduced is not
consistent.   This depends on a general fact about Stone spaces of
measure algebras which has further important applications, so I present
it as a lemma.}

\leader{346K}{Lemma} Let $(Z,\Tau,\nu)$ be the Stone space of the
measure algebra of
Lebesgue measure on $[0,1]$, and let $\lambda$ be the product
measure on $Z\times Z$, with $\Lambda$ its domain.   Then there is a set
$W\in\Lambda$, with $\lambda W<1$, such that $\lambda^*\tilde W=1$,
where

\Centerline{$\tilde W
=\bigcup\{G\times H:G,\,H\subseteq Z$ are open-and-closed,
  $(G\times H)\setminus W$ is negligible$\}$.}

\cmmnt{\medskip

\noindent{\bf Remark} For the sake of anybody who has already become
acquainted with the alternative measures which can be put on the product
of topological measure spaces, I ought to insist that the `product
measure' $\lambda$ here is, as always in this volume, the ordinary
completed product measure as defined in Chapter 25.
}

\proof{{\bf (a)} Let $\sequencen{E_n}$ be a sequence of measurable
subsets of $[0,1]$, stochastically independent for Lebesgue measure
$\mu$ on $[0,1]$, such that $\mu E_n=\bover1{n+2}$ for each $n$.
Set $a_n=E_n^{\ssbullet}$ in the measure of algebra of $\mu$, and
$E_n^*=\widehat{a}_n$ the corresponding compact open subset of $Z$.
Set $W=\bigcup_{n\in\Bbb N}E_n^*\times E_n^*$.   Then

\Centerline{$\lambda W\le\sum_{n=0}^{\infty}(\nu
E_n)^2=\sum_{n=2}^{\infty}\Bover1{n^2}<1$.}

\Quer\ Suppose, if possible, that $\lambda^*\tilde W<1$.   Then there
are sequences $\sequencen{G_n}$, $\sequencen{H_n}$ in $\Tau$ such that
$\tilde W\subseteq\bigcup_{n\in\Bbb N}G_n\times H_n$ and
$\lambda(\bigcup_{n\in\Bbb N}G_n\times H_n)<1$.   Recall from
322Rc that

\Centerline{$\nu F
=\inf\{\nu G:G$ is compact and open, $F\subseteq G\}$}

\noindent for every $F\in\Tau$.   Accordingly we can find compact open
sets $\tilde G_n$, $\tilde H_n$ such that $G_n\subseteq \tilde G_n$,
$H_n\subseteq
\tilde H_n$ for every $n\in\Bbb N$ and

\Centerline{$\sum_{n=0}^{\infty}\nu(\tilde G_n\setminus
G_n)+\sum_{n=0}^{\infty}\nu(\tilde H_n\setminus H_n)
<1-\lambda(\bigcup_{n\in\Bbb N}G_n\times H_n)$,}

\noindent so that
$\lambda(\bigcup_{n\in\Bbb N}\tilde G_n\times \tilde H_n)<1$.

Let $\Cal U_0$ be the family

\Centerline{$\{Z\}\cup\{E_n^*:n\in\Bbb N\}\cup\{Z\setminus
\tilde G_n:n\in\Bbb N\}\cup\{Z\setminus \tilde H_n:n\in\Bbb N\}$,}

\noindent so that $\Cal U_0$ is a countable subset of $\Tau$.   Let
$\Cal U$ be the set of finite intersections $U_0\cap U_1\cap\ldots\cap U_n$
where $U_0,\ldots,U_n\in\Cal U_0$, so that $\Cal U$ also is a
countable subset of $\Tau$, and $\Cal U$ is closed under $\cap$.

\medskip

{\bf (b)} For $U\in\Cal U$, define $Q(U)$ as follows.   If
$\nu U=0$, then $Q(U)=U$.   Otherwise,

\Centerline{$Q(U)
=Z\setminus\bigcup\{E_n^*:n\in\Bbb N,\,\nu(E_n^*\cap U)>0\}$.}

\noindent Then $\nu Q(U)$ is always $0$.   \Prf\ Of course this is
true if $\nu U=0$, so suppose that $\nu U>0$.   Set
$I=\{n:\nu(E_n^*\cap U)=0\}$.   Then we have
$\nu U'>0$, where $U'=U\setminus\bigcup_{n\in I}E_n^*$, and
$Z\setminus E_n^*\supseteq U'$ for every $n\in I$.   Because
$\sequencen{E_n}$ is stochastically independent for $\mu$,
$\sequencen{E_n^*}$ is stochastically independent for $\nu$, while

\Centerline{$\nu(\bigcup_{n\in I}E_n^*)\le 1-\nu U'<1$.}

\noindent By the
Borel-Cantelli lemma (273K), $\sum_{n\in I}\nu E_n^*<\infty$.
Consequently
$\sum_{n\in\Bbb N\setminus I}\nu E_n^*=\infty$, because $\sum_{n=0}^{\infty}\bover1{n+2}$ is infinite, so

\Centerline{$\nu(Z\setminus Q(U))
=\nu(\bigcup_{n\in\Bbb N\setminus I}E_n^*)=1$,}

\noindent and $\nu Q(U)=0$.\ \Qed

\medskip

{\bf (c)} Set $Q_0=\bigcup_{U\in\Cal U}Q(U)$;  because $\Cal U$ is
countable, $Q_0$ is negligible.   Accordingly $(Z\setminus Q_0)^2$ has
measure $1$ and cannot be included in $\bigcup_{n\in\Bbb N}\tilde
G_n\times
\tilde H_n$;  take $(w,z)\in(Z\setminus Q_0)^2\setminus\bigcup_{n\in\Bbb
N}\tilde G_n\times \tilde H_n$.

\medskip

{\bf (d)} We can find sequences $\sequencen{C_n}$, $\sequencen{D_n}$,
$\sequencen{U_n}$ and $\sequencen{V_n}$ in $\Cal U$ such that

\inset{$w\in U_{n+1}\subseteq U_n$, $z\in V_{n+1}\subseteq V_n$,
$(U_{n+1}\times V_{n+1})\cap(\tilde G_n\times \tilde H_n)=\emptyset$,}

\inset{$\nu C_n>0$, $\nu D_n>0$,}

\inset{$C_n\subseteq U_{n}$, $D_n\subseteq V_{n+1}$,}

\inset{$C_n\times V_{n+1}\subseteq W$, $U_{n+1}\times D_n\subseteq W$}

\noindent for every $n\in\Bbb N$.   \Prf\ Build the sequences
inductively, as follows.   Start with $U_0=V_0=Z$.   Given that
$w\in U_n\in\Cal U$ and $z\in V_n\in\Cal U$,
then we know that $(w,z)\notin
\tilde G_n\times \tilde H_n$.   If $w\notin \tilde G_n$, set
$U'_n=U_n\setminus \tilde G_n$,
$V'_n=V_n$;  otherwise set $U'_n=U_n$, $V'_n=V_n\setminus \tilde H_n$.
In
either case, we have $w\in U'_n\in\Cal U$, $z\in V'_n\in\Cal U$ and
$(U'_n\times V'_n)\cap(\tilde G_n\times \tilde H_n)=\emptyset$.

Because $U'_n\in\Cal U$, $w\notin Q(U'_n)$.   But $w\in U'_n$, so this
must be because $\nu U'_n>0$.    Now $z\notin Q(U'_n)$, so
$z\in\bigcup\{E^*_k:k\in\Bbb N,\,\nu(E^*_k\cap U'_n)>0\}$.   Take
some $k\in\Bbb N$ such that $z\in E^*_k$ and $\nu(E^*_k\cap
U'_n)>0$, and set

\Centerline{$V_{n+1}=V'_n\cap E^*_k$,\quad $C_n=E^*_k\cap U'_n$,}

\noindent so that

\Centerline{$z\in V_{n+1}\in\Cal U$,\quad $C_n\subseteq U_n$, \quad
$C_n\times V_{n+1}\subseteq E^*_k\times E^*_k\subseteq W$,\quad $\nu
C_n>0$.}

\noindent Next, $z\notin Q(V_{n+1})$ and $\nu V_{n+1}>0$;  also
$w\notin Q(V_{n+1})$, so there is an $l$ such that $w\in E^*_l$ and
$\nu(E^*_l\cap V_{n+1})>0$.   Set

\Centerline{$U_{n+1}=U'_n\cap E^*_l$,\quad $D_n=E^*_l\cap V_{n+1}$,}

\noindent so that

\Centerline{$w\in U_{n+1}\in\Cal U$,\quad $D_n\subseteq V_{n+1}$, \quad
$U_{n+1}\times D_{n}\subseteq E^*_l\times E^*_l\subseteq W$,\quad
$\nu D_n>0$,}

\Centerline{$(U_{n+1}\times V_{n+1})\cap(\tilde G_n\times \tilde
H_n)\subseteq
(U'_n\times V'_n)\cap(\tilde G_n\times \tilde H_n)=\emptyset$,}

\noindent and continue the process.
\Qed

\medskip

{\bf (e)} Setting $C=\bigcup_{n\in\Bbb N}C_n$
and $D=\bigcup_{n\in\Bbb N}D_n$ we see that $C\times D\subseteq W$.
\Prf\ If $m\le n$,
$D_n\subseteq V_{n+1}\subseteq V_{m+1}$, so $C_m\times D_n\subseteq W$.
If $m>n$, $C_m\subseteq U_m\subseteq U_{n+1}$, so $C_m\times
D_n\subseteq W$.\ \Qed

Recall from 321K that the measurable sets of $Z$ are precisely those of
the form $G\symmdiff M$ where $M$ is nowhere dense and negligible and
$G$ is compact and open.   There must therefore be compact open sets
$G$, $H\subseteq Z$ such that $G\symmdiff C$ and $H\symmdiff D$ are
negligible.   Consequently

\Centerline{$(G\times H)\setminus W\subseteq ((G\setminus C)\times
Z)\cup(Z\times(H\setminus D))$}

\noindent is negligible, and

\Centerline{$G\times H\subseteq\tilde W\subseteq\bigcup_{n\in\Bbb
N}\tilde G_n\times \tilde H_n$.}

\noindent But because $G\times H$ is compact (3A3J), and all the
$\tilde G_n\times \tilde H_n$ are open, there must be some $n$ such that
$G\times H\subseteq\bigcup_{k\le n}\tilde G_k\times\tilde H_k=S$ say.
Now $(U_{k+1}\times V_{k+1})\cap(\tilde G_k\times \tilde H_k)=\emptyset$
for every $k$, so

\Centerline{$(C_{n+2}\times D_{n+2})\cap(G\times H)
\subseteq(U_{n+1}\times V_{n+1})\cap S
=\emptyset$,}

\noindent and either $C_{n+2}\cap G=\emptyset$ or
$D_{n+2}\cap H=\emptyset$.   Since

\Centerline{$C_{n+2}\setminus G\subseteq C\setminus G$,
\quad$D_{n+2}\setminus H\subseteq D\setminus H$}

\noindent are both negligible, one of $C_{n+2}$, $D_{n+2}$ is
negligible.   But the construction took care to ensure that all the
$C_k$, $D_k$ were non-negligible.\ \Bang

\medskip

{\bf (f)} Thus $\lambda^*\tilde W=1$, as required.
}%end of proof of 346K

\leader{346L}{Proposition} Let $(Z,\Tau,\nu)$ be the Stone space of
the measure algebra of Lebesgue measure on $[0,1]$.   Let
$\psi:\Tau\to\Tau$ be the canonical lifting, defined by setting
$\psi E=G$ whenever $E\in\Tau$, $G$ is open-and-closed and
$E\symmdiff G$ is negligible\cmmnt{ (341O)}.   Then $\psi$ is not
consistent.

\proof{ \Quer\ Suppose, if possible, that $\phi$ is a lifting on
$Z\times Z$ such that $\phi(E\times F)=\psi E\times\psi F$ for every
$E$, $F\in\Tau$.   Let $W\subseteq Z\times Z$ be a set as in 346K, and
consider $\phi W$.   If $G$, $H\subseteq Z$ are open-and-closed and
$(G\times H)\setminus W$ is negligible, then

\Centerline{$G\times H=\psi G\times\psi H=\phi(G\times H)\subseteq\phi
W$;}

\noindent that is, in the language of 346K, we must have    $\tilde
W\subseteq\phi W$.   But this means that

\Centerline{$\lambda(\phi W)\ge\lambda^*\tilde W=1>\lambda
W$,}

\noindent which is impossible.\ \Bang

Thus $\psi$ fails the first test and cannot be consistent.
}%end of proof of 346L

\exercises{\leader{346X}{Basic exercises (a)}
%\spheader 346Xa
Let $(X,\Sigma,\mu)$ be a measure space and
$\sequencen{\undphi_n}$ a sequence of lower densities for $\mu$.
(i) Show that $E\mapsto\bigcap_{n\in\Bbb N}\undphi_nE$ and
$E\mapsto\bigcup_{n\in\Bbb N}\bigcap_{m\ge n}\undphi_mE$ are also lower
densities for $\mu$.   (ii) Show that if $\mu$ is complete and $\Cal F$
is any filter on $\Bbb N$, then
$E\mapsto\bigcup_{F\in\Cal F}\bigcap_{n\in F}\undphi_nE$ is a lower
density for $\mu$.

\spheader 346Xb Let $(X,\Sigma,\mu)$ be a strictly localizable
measure space, and $G$ a countable group of measure space automorphisms
from $X$ to itself.   Show that there is a lower density
$\undphi:\Sigma\to\Sigma$ which is $G$-invariant in the sense that
$\undphi(g^{-1}[E])=g^{-1}[\undphi E]$ for every $E\in\Sigma$
and $g\in G$.
\Hint{set $\undphi E=\bigcap_{g\in G}g[\undphi_0(g^{-1}[E])]$.}
%346Xa

\sqheader 346Xc   Show that there is no lifting $\phi$ of Lebesgue
measure on $[0,1]^2$ which is `symmetric' in the sense that
$\phi(E^{-1})=(\phi E)^{-1}$ for every measurable set $E$, writing
$E^{-1}=\{(y,x):(x,y)\in E\}$.   \Hint{345Xc.}
%346E

\spheader 346Xd Let $(X,\Sigma,\mu)$
be a measure space and $\undphi$ a lower density for $\mu$.  Take
$H\in\Sigma$ and set $A=X\setminus(\undphi H\cup\undphi(Z\setminus H))$,
$\undphi' E=\undphi E\cup(A\cap \undphi(H\cup E))$ for $E\in\Sigma$.
Show that $\undphi'$ is a lower density.
%346H

\spheader 346Xe\dvAnew{2010}
Describe the connections between 346B, 346D and 346F.
%346F

\sqheader 346Xf Suppose, in 341H, that $(X,\Sigma,\mu)$ is a product of
probability spaces, and that in the proof, instead of taking
$\langle a_{\xi}\rangle_{\xi<\kappa}$ to run over the whole measure
algebra $\frak A$, we take it to run over the elements of $\frak A$
expressible as $E^{\ssbullet}$ where $E\in\Sigma$ is determined by a
single coordinate.   Show that the resulting lower density $\undtheta$
respects coordinates in the sense that $\undtheta E^{\ssbullet}$ is
determined by
coordinates in $J$ whenever $E\in\Sigma$ is determined by coordinates in
$J$.   (Compare {\smc Macheras \& Strauss 95}, Theorem 2.)
%346G

\sqheader 346Xg\dvAformerly{3{}46Xe;  revised 2010}
Let $\undphi$ be lower Lebesgue density on $\Bbb R$, and
$\phi$ a translation-invariant lifting for Lebesgue measure on
$\Bbb R$ such that $\phi E\supseteq\undphi E$ for every measurable set
$E$.   Show that $\phi$ is consistent.  ({\it Hint\/}:  given $n\ge 1$,
let $\undphi_n$ be lower Lebesgue density on $\BbbR^n$.
Let $\Cal I$ be the ideal generated by

\Centerline{$\{W:\tbf{0}\in\undphi_n(\BbbR^n\setminus W)\}
  \cup\bigcup_{i<n}\{\pi_i^{-1}[E]:0\in\phi(\Bbb R\setminus E)\}$;}

\noindent show that $\BbbR^n\notin\Cal I$, so that we can use the
method of 345B to construct a lifting for Lebesgue measure on $\BbbR^n$.)
%346J   do we need \phi to be t-i?

\spheader 346Xh Show that Lemma 346K is valid for any $(Z,\Tau,\nu)$
which is the Stone space of an atomless probability space.
%346K

\leader{346Y}{Further exercises (a)}
%\spheader 346Ya 
Let $(X_1,\Sigma_1,\mu_1),\ldots,(X_n,\Sigma_n,\mu_n)$
be probability spaces with product $(X,\Sigma,\mu)$.   Show that there
is a lifting for $\mu$ which respects coordinates.   ({\smc Burke n95}.)
%346E decompose each factor into M-homog pieces

\spheader 346Yb Let $(X,\Sigma,\mu)$ be a
probability space, $I$ any set, and $\lambda$ the product measure on
$X^I$.   Show that there is a lower density for $\lambda$ which is
invariant under transpositions of pairs of coordinates.
%mt34bits 346Zb 346Xc 346E

\spheader 346Yc Suppose that $(X,\Sigma,\mu)$
and $(Y,\Tau,\nu)$ are complete probability spaces with product
$(X\times Y,\Lambda,\lambda)$.   Show that for any lifting
$\psi_1:\Sigma\to\Sigma$ there are liftings $\psi_2:\Tau\to\Tau$ and
$\phi:\Lambda\to\Lambda$ such that
$\phi(E\times F)=\psi_1E\times\psi_2F$ for all $E\in\Sigma$, $F\in\Tau$.
\Hint{use the methods of \S341.   In the inductive construction of
341H, start with $\undphi_0(E\times Y)=(\psi_1E)\times Y$ for every
$E\in\Sigma$.   Extend each lower density $\undphi_{\xi}$ to the algebra
generated by $\dom(\undphi_{\xi})\cup\{X\times F_{\xi}\}$ for some
$F_{\xi}\in\Tau$.   Make sure that $\undphi_{\xi}(X\times F)$ is always
of the form $X\times F'$, and that
$\undphi_{\xi}((E\times Y)\cup(X\times F))
=\undphi_{\xi}(E\times Y)\cup\undphi_{\xi}(X\times F)$;  adapt the
construction of
341G to maintain this.   Use the method of 346H to generate a lifting
from the final lower density $\undphi$.   See {\smc Macheras \& Strauss
96a}, Theorem 4.}
%346H

\spheader 346Yd Use 346Yc and induction on $\zeta$ to prove 346H.
({\smc Macheras \& Strauss 96b}.)
%346Yc 346H

\spheader 346Ye Let $(X,\Sigma,\mu)$ be a complete probability space.
Show that there is a lifting $\psi:\Sigma\to\Sigma$ such that whenever
$\langle(X_i,\Sigma_i,\mu_i)\rangle_{i\in I}$ is a family of probability
spaces, with product measure $\lambda$, there is a lifting $\phi$ for
$\lambda$ such that
$\phi(\pi_i^{-1}[E])=\pi_i^{-1}[\psi E]$ whenever $E\in\Sigma$ and
$i\in I$ is such that $(X_i,\Sigma_i,\mu_i)=(X,\Sigma,\mu)$, writing
$\pi_i(x)=x(i)$ for $x\in\prod_{i\in I}X_i$.
%346I
}%end of exercises

\leader{346Z}{Problems (a)} Let
$\langle(X_i,\Sigma_i,\mu_i)\rangle_{i\in I}$ be a family of
probability spaces, with product $(X,\Sigma,\mu)$.   Is there always
a lifting for $\mu$ which respects coordinates in the sense of 346A?

\spheader 346Zb Is there a lower density $\undphi$ for the usual measure
on $\{0,1\}^{\Bbb N}$ which is invariant under all permutations of
coordinates?

\endnotes{
\Notesheader{346} I ought to say at once that in writing this section I
have been greatly assisted by M.R.Burke.

The theorem that every complete probability space has a consistent
lifting (346J) is due to {\smc Talagrand 82a};  it is the inspiration
for the whole of the section.   `Consistent' liftings were devised in
response to some very interesting questions (see {\smc Talagrand 84},
\S6) which I do not discuss here;
one will be mentioned in Theorem 465P in Volume 4.   My aim
here is rather to suggest further ways in which a lifting on a product
space can
be consistent with the product structure.   The labour is substantial
and the results achieved are curiously partial.   I offer 346Za as the
easiest natural question which does not appear amenable to the methods I
describe.

The arguments I use are based on the fact that the translation-invariant
measures of 345C already respect coordinates
(346C).   Maharam's theorem now
makes it easy to show that any product of \Mth\
probability spaces has a lifting which respects coordinates
(346E).   A kind of
projection argument (346F) makes it possible to obtain a lower density
which respects coordinates on any product of probability spaces (346G).
In fact the methods of \S341, very slightly refined, automatically
produce such lower densities (346Xf).   But the extra power of 346G lies
in the condition (ii):  if $E$ and $F$ are `fully
independent' in the sense of being determined by coordinates in disjoint sets, then $\undphi(E\cup F)=\undphi E\cup\undphi F$, that is,
$\undphi$ is making a tentative step towards being a lifting.
(Remember that the difference between a lifting and a lower density is
mostly that a lifting preserves finite unions as well as finite
intersections;  see 341Xa.)   This can also be achieved by a
modification of the previous method, but we have to work harder at one
point in the proof.

The next step is to move to liftings which continue, as far as possible,
to respect coordinates.   Here there seem to be quite new obstacles, and
346H is the best result I know;  the lifting respects {\it individual}
coordinates, and also, for a given well-ordering of the index set,
initial segments of the coordinates.   The treatment
of initial segments makes essential use of the well-ordering, which is
what leaves 346Za open.

Finally, if all the factors are identical, we can seek lower densities
and liftings which are invariant under permutation of coordinates.
I give 345Xc and 346Xc as examples to show that we must not just assume
that a symmetry in the underlying measure space can be reflected in a
symmetry of a lifting.   The problems there concern liftings themselves,
not lower densities, since we can frequently find lower densities which
share symmetries (346Xb, 346Yb).   (Even for lower densities
there seem to be difficulties if we are more ambitious (346Zb).)
However a very simple argument (346I) shows that at least we can make
each individual coordinate look more or less the same, as long as we do
not investigate its relations with others.

Still on the question of whether, and when, liftings can be `good', note
346L/346Xh and 346Xg.   The most natural liftings for Lebesgue
measure are necessarily consistent;  but the only example we have of a
truly canonical lifting is not consistent in any non-trivial context.

I have deliberately used a variety of techniques here, even though 346H
(for instance) has an alternative proof based on the ideas of \S341
(346Yc-346Yd).   In particular, I give some of the standard methods of
constructing liftings and lower densities (346B, 346D, 346F, 346Xd,
346Xa).   In fact 346D was one of the elements of Maharam's original
proof of the lifting theorem ({\smc Maharam 58}).
}%end of comment

\frnewpage


