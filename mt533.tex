\frfilename{mt533.tex}
\versiondate{4.1.14}
\copyrightdate{2007}

\def\chaptername{Topologies and measures III}
\def\sectionname{Special topics}

\def\ureg{\mathop{\text{ureg}}}

\newsection{533}

I present notes on certain questions which can be answered if we make
particular assumptions concerning values of the cardinals considered in
\S\S523-524.   The first cluster (533A-533E) %533A 533B 533C 533D 533E
looks at Radon and quasi-Radon measures in contexts in which the additivity
of Lebesgue measure is large compared with other cardinals of the
structures considered.   Developing ideas which arose in the course of
\S531, I discuss `uniform regularity' in perfectly normal and
first-countable spaces (533H).   We also have a complete description of
the cardinals $\kappa$ for which $\BbbR^{\kappa}$ is measure-compact
(533J).

As previously, I write $\Cal N(\mu)$ for the null ideal of a measure
$\mu$;   $\nu_{\kappa}$ will be the usual measure on $\{0,1\}^{\kappa}$
and $\Cal N_{\kappa}=\Cal N(\nu_{\kappa})$ its null ideal.

\leader{533A}{Lemma} Let $(X,\Sigma,\mu)$ be a semi-finite measure space
with measure algebra $(\frak A,\bar\mu)$.   If
$\ofamily{\xi}{\kappa}{\Cal K_{\xi}}$ is a family of ideals in $\Sigma$
such that $\mu$ is inner regular with respect to every
$\Cal K_{\xi}$ and
$\kappa<\min(\add\Cal N(\mu),\wdistr(\frak A))$, then $\mu$ is inner
regular with respect to $\bigcap_{\xi<\kappa}\Cal K_{\xi}$.

\proof{ Take $E\in\Sigma$ and $\gamma<\mu E$.   Then there is an
$E_1\in\Sigma$ such that $E_1\subseteq E$ and $\gamma<\mu E_1<\infty$.
For $\xi<\kappa$, $D_{\xi}=\{K^{\ssbullet}:K\in\Cal K_{\xi}\}$ is closed
under finite unions and is order-dense in $\frak A$, so includes a
partition of unity $A_{\xi}$.   Now there is a partition $B$ of unity in
$\frak A$ such that $\{a:a\in A_{\xi}$, $a\Bcap b\ne 0\}$ is finite for
every $b\in B$ and $\xi<\kappa$.   Let $B'\subseteq B$ be a finite set
such that $\bar\mu(E_1^{\ssbullet}\Bcap\sup B')\ge\gamma$, and let
$E_2\subseteq E_1$ be such that
$E_2^{\ssbullet}=E_1^{\ssbullet}\Bcap\sup B'$.   For any $\xi<\kappa$,

\Centerline{$A'_{\xi}
=\{a:a\in A_{\xi}$, $a\Bcap E_2^{\ssbullet}\ne 0\}
\subseteq\bigcup_{b\in B'}\{a:a\in A_{\xi}$, $a\Bcap b\ne 0\}$}

\noindent is finite, so $\sup A'_{\xi}$ belongs to $D_{\xi}$
and can be expressed
as $K_{\xi}^{\ssbullet}$ for some $K_{\xi}\in\Cal K_{\xi}$.   Now
$E_2^{\ssbullet}\Bsubseteq\sup A'_{\xi}$ so $E_2\setminus K_{\xi}$ is
negligible.   As $\kappa<\add\Cal N(\mu)$, we have a negligible
$H\in\Sigma$ including $\bigcup_{\xi<\kappa}E_2\setminus K_{\xi}$;  now
$E'=E_2\setminus H\subseteq E$, $\mu E'\ge\gamma$ and
$E'\in\bigcap_{\xi<\kappa}\Cal K_{\xi}$.   As $E$ and $\gamma$ are
arbitrary, $\mu$ is inner regular with respect to
$\bigcap_{\xi<\kappa}\Cal K_{\xi}$.
}%end of proof of 533A

\cmmnt{\medskip

\noindent{\bf Remark} Of course this result is covered by 412Ac unless
$\wdistr(\frak A)>\omega_1$, which nearly forces $\frak A$ to have
countable Maharam type (524Mb).
}

\leader{533B}{Corollary} Let $(X,\Sigma,\mu)$ be a totally finite
measure space with countable Maharam type.   If $\Cal E\subseteq\Sigma$,
$\#(\Cal E)<\min(\add\Cal N_{\omega},\add\Cal N(\mu))$ and $\epsilon>0$,
there is a set $F\in\Sigma$ such that $\mu(X\setminus F)\le\epsilon$ and
$\{E\cap F:E\in\Cal E\}$ is countable.

\proof{ Let $(\frak A,\bar\mu)$ be the measure algebra of $\mu$.   Then
$\frak A$ is separable in its measure-algebra topology (521Ea).   Let
$\Cal H\subseteq\Sigma$ be a countable set
such that $\{H^{\ssbullet}:H\in\Cal H\}$ is
dense in $\frak A$.   For $E\in\Cal E$ and $n\in\Bbb N$ choose
$H_{En}\in\Cal H$ such that
$\mu(E\symmdiff H_{En})\le 2^{-n}$;  let $\Cal K_E$ be the family of
measurable sets $K$ such that $K$ is disjoint from
$\bigcup_{i\ge n}E\symmdiff H_{Ei}$ for some $n$.   Then $\mu$ is inner
regular with respect to $\Cal K_E$.   Because
$\#(\Cal E)<\min(\wdistr(\frak A),\add\Cal N(\mu))$ (524Mb), $\mu$ is
inner regular with respect to $\bigcap_{E\in\Cal E}\Cal K_E$ (533A) and
there is an $F\in\bigcap\Cal K_E$ such that $\mu F\ge\mu X-\epsilon$.
If $E\in\Cal E$, there is an $n\in\Bbb N$ such that
$F\cap(E\symmdiff H_{En})=\emptyset$, that is, $F\cap E=F\cap H_{En}$;
so $\{F\cap E:E\in\Cal E\}\subseteq\{F\cap H:H\in\Cal H\}$ is countable.
}%end of proof of 533B

\leader{533C}{Proposition} Let $(X,\frak T,\Sigma,\mu)$ be a quasi-Radon
measure space with countable Maharam type.

(a) If $w(X)<\add\Cal N_{\omega}$, then $\mu$ is inner regular with
respect to the second-countable subsets of $X$;  if moreover $\frak T$
is regular and Hausdorff, then $\mu$ is inner regular with respect to
the metrizable subsets of $X$.

(b) If $Y$ is a topological space of weight less than
$\add\Cal N_{\omega}$, then any measurable function $f:X\to Y$ is almost
continuous.

(c) If $\familyiI{Y_i}$ is a family of topological spaces, with
$\#(I)<\add\Cal N_{\omega}$, and
$f_i:X\to Y_i$ is almost continuous for every $i$, then
$x\mapsto\cmmnt{ f(x)=}\familyiI{f_i(x)}:X\to\prod_{i\in I}Y_i$ is
almost continuous.

\proof{ Note first that $\add\Cal N(\mu)\ge\add\Cal N_{\omega}$, by
524Ta.

\medskip

{\bf (a)} Let
$\Cal U$ be a base for $\frak T$ with $\#(\Cal U)<\add\Cal N_{\omega}$.
Set

\Centerline{$\Cal F=\{F:F\subseteq X$, $\{F\cap U:U\in\Cal U\}$ is
countable$\}$.}

\noindent Then $\mu$ is inner regular with respect to $\Cal F$.   \Prf\
If $E\in\Sigma$ and $\gamma<\mu E$, let $H\in\Sigma$ be such that
$H\subseteq E$ and $\gamma<\mu H<\infty$.   Then the subspace measure
$\mu_H$ still has countable Maharam type (use 322I and 514Ed) and

\Centerline{$\add\Cal N(\mu_H)\ge\add\Cal N(\mu)\ge\add\Cal N_{\omega}
>\#(\{H\cap U:U\in\Cal U\})$.}

\noindent By 533B, there is an $F\in\dom\mu_H$ such that
$\mu_HF\ge\gamma$ and $\{F\cap H\cap U:U\in\Cal U\}$ is countable;  now
$F\in\Cal F$, $F\subseteq E$ and $\mu F\ge\gamma$.\ \QeD\   But every
member of $\Cal F$ is second-countable (use 4A2B(a-vi)).
If $\frak T$ is regular and Hausdorff, then every member of $\Cal F$ is
separable and metrizable (4A2Pb).

\medskip

{\bf (b)} If $f:X\to Y$ is measurable, let $\Cal V$ be a base for the
topology of $Y$ with $\#(\Cal V)<\add\Cal N_{\omega}$.   Suppose that
$E\in\Sigma$ and $\gamma<\mu E$.   By 533B, there is an $F\in\Sigma$
such that $F\subseteq E$, $\gamma<\mu F<\infty$ and
$\{F\cap f^{-1}[V]:V\in\Cal V\}$ is countable.
It follows that $\{f[F]\cap V:V\in\Cal V\}$ is countable, so that the
subspace topology on $f[F]$ is second-countable (4A2B(a-vi) again).
Giving $F$ its subspace topology $\frak T_F$ and measure $\mu_F$, $\mu_F$
is inner regular with respect to the closed sets (412Pc).   If
$H\subseteq f[F]$ is relatively open in $f[F]$, it is of the form
$G\cap f[F]$ where $G$ is an open subset of $Y$, so that
$(f\restr F)^{-1}[H]=F\cap f^{-1}[G]$ is measured by $\mu_F$;  thus
$f\restr F:F\to f[F]$ is measurable.   By 418J, $f\restr F$ is
almost continuous, and there is a $K\in\Sigma$ such that
$K\subseteq F$, $\mu K\ge\gamma$ and $f\restr K$ is continuous.

As $E$ and $\gamma$ are arbitrary, $f$ is almost continuous.

\medskip

{\bf (c)} For each $i\in I$, set
$\Cal K_i=\{K:K\in\Sigma$, $f_i\restr K$ is continuous$\}$.
Then $\Cal K_i$ is an ideal in $\Sigma$ and $\mu$ is inner regular with
respect to $\Cal K_i$.    Also, as in 533B, $\#(I)<\wdistr(\frak A)$,
where $\frak A$ is the measure algebra of $\mu$.
So $\mu$ is inner regular with respect to
$\Cal K=\bigcap_{i\in I}\Cal K_i$, by 533A.   But $f\restr K$ is
continuous for every $K\in\Cal K$, so $f$ is almost continuous.
}%end of proof of 533C

\leader{533D}{Proposition} Let $(X,\frak T)$ be a
first-countable compact Hausdorff space such that
$\cff[w(X)]^{\le\omega}<\add\Cal N_{\omega}$, and $\mu$ a Radon measure
on $X$ with countable Maharam type.   Then $\mu$ is inner regular with
respect to the metrizable zero sets.

\proof{ Set $\kappa=w(X)$.   Then there is an injective continuous
function $f:X\to[0,1]^{\kappa}$ (5A4Cc).   Let $\Cal I$ be a cofinal
subset of $[\kappa]^{\le\omega}$ with $\#(\Cal I)<\add\Cal N_{\omega}$.
By 524Pa, $\add\mu\ge\add\Cal N_{\omega}$.

For $I\in\Cal I$ and $x\in X$ set $f_I(x)=f(x)\restr I$.   We need to know
that for every $x\in X$ there is an $I\in\Cal I$ such that
$\{x\}=f_I^{-1}[f_I[\{x\}]]$.   \Prf\ Set $F_I=f_I^{-1}[f_I[\{x\}]]$ for
each $I$.   Because $\Cal I$ is upwards-directed,
$\family{I}{\Cal I}{F_I}$ is downwards-directed.   Because $f$ is
injective and $\bigcup\Cal I=\kappa$, $\bigcap_{I\in\Cal I}F_I=\{x\}$.
Let $\Cal V$ be a countable base of open neighbourhoods of $x$.   For
each $V\in\Cal V$ there is an $I_V\in\Cal I$ such that
$F_{I_V}\cap(X\setminus V)=\emptyset$.   Let $I\in\Cal I$ be such that
$\bigcup_{V\in\Cal V}I_V\subseteq I$;  then $F_I=\{x\}$.\ \Qed

For $I\in\Cal I$, let $\lambda_I$ be the image measure $\mu f_I^{-1}$ on
$[0,1]^I$;  note that $\lambda_I$ is a Radon measure (418I).   Of course
$\add\lambda_I$ is also at least
$\add\Cal N_{\omega}$, and in particular is greater than $\kappa$.   If
$G\subseteq X$ is open, then $G$ and $f_I[G]$ are expressible as unions
of at most $\kappa$ compact sets, so $\lambda_I$ measures $f_I[G]$.

There is an $I\in\Cal I$ such that $\mu f_I^{-1}[f_I[G]]=\mu G$ for
every open set $G\subseteq X$.   \Prf\Quer\ Suppose, if possible,
otherwise.   For each $I\in\Cal I$ choose an open set $G_I\subseteq X$
such that $E_I=f_I^{-1}[f_I[G_I]]\setminus G_I$ is non-negligible;
because $\lambda_I$ measures $f_I[G_I]$, $\mu$ measures $E_I$.   Set
$E'_I=\bigcup_{J\in\Cal I,J\supseteq I}E_J$ for each $I\in\Cal I$;
because $\#(\Cal I)<\add\mu$, $\mu$ measures $E'_I$.   Note that
$E'_I\subseteq E'_J$ whenever $J\subseteq I$ in $\Cal I$;  moreover, any
sequence in $\Cal I$ has an upper bound in $\Cal I$.   There is
therefore an $M\in\Cal I$ such that $E'_M\setminus E'_I$ is negligible
for every $I\in\Cal I$.   Again because $\#(\Cal I)<\add\mu$,
$E'_M\setminus\bigcap_{I\in\Cal I}E'_I$ is negligible;  as $E'_M$ is not
negligible, there is an $x\in\bigcap_{I\in\Cal I}E'_I$.   But there is
an $I\in\Cal I$ such that $\{x\}=f_I^{-1}[f_I[\{x\}]]$, so
$x\notin E_J$ for any $J\supseteq I$.\ \Bang\Qed

Let $\Cal U$ be a base for the topology of $X$ with $\#(\Cal U)=\kappa$.
Then $\bigcup_{U\in\Cal U}f_I^{-1}[f_I[U]]\setminus U$ is
$\mu$-negligible;  let $Y$ be its complement.   If $x\in X$ and $y\in Y$
and $x\ne y$, there is a $U\in\Cal U$ containing $x$ but not $y$, so
$f_I^{-1}[f_I[U]]$ contains $x$ and not $y$ and $f(x)\ne f(y)$.   If
$F\subseteq Y$ is compact, then $F$ is homeomorphic to the metrizable
$f_I[F]$, so is metrizable, and $F=f_I^{-1}[f_I[F]]$ is a zero set.   As
$\mu$ is surely inner regular with respect to the compact subsets of the
conegligible set $Y$, it is inner regular with respect to the metrizable
zero sets.
}%end of proof of 533D

\leader{533E}{Corollary} Suppose that
$\cov\Cal N_{\omega_1}>\omega_1$.   Let $(X,\frak T)$ be a
first-countable K-analytic Hausdorff space such that
$\cff[w(X)]^{\le\omega}<\add\Cal N_{\omega}$.   Then $X$ is a Radon
space.

\proof{ Let $\mu$ be a totally finite Borel measure on $X$,
$E\subseteq X$ a Borel set and $\gamma<\mu E$.   Because $X$ is
K-analytic, there is a compact set $K\subseteq X$ such that
$\mu(E\cap K)>\gamma$ (apply 432B to the measure $\mu\LLcorner E$).
Let $\lambda$ be the Radon measure on
$K$ defined by saying that $\int fd\lambda=\int_Kfd\mu$ for every
$f\in C(K)$ (using the Riesz Representation Theorem, 436J/436K).
Because $\cov\Cal N_{\omega_1}>\omega_1$, $\omega_1$ is a precaliber of
every measurable algebra (525J);  as $K$ is first-countable,
$\omega_1\notin\MahR(K)$ (531P) and $\lambda$ must have countable
Maharam type (531Ef).   By 533D, $\lambda$ is completion regular.   But if
$F\subseteq K$ is a zero set (for the subspace topology of $K$), there
is a non-increasing sequence $\sequencen{f_n}$ in $C(K)$ with infimum
$\chi F$, so

\Centerline{$\lambda F=\lim_{n\to\infty}\int f_nd\lambda
=\lim_{n\to\infty}\int_Kf_nd\mu=\mu F$.}

\noindent Accordingly

\Centerline{$\lambda H
=\sup\{\lambda F:F\subseteq H$ is a zero set$\}
=\sup\{\mu F:F\subseteq H$ is a zero set$\}\le\mu H$}

\noindent for every Borel set $H\subseteq K$.   As $\lambda K=\mu K$,
$\lambda$ agrees with $\mu$ on the Borel subsets of $K$.   In
particular, $\lambda(E\cap K)>\gamma$;  now there is a compact set
$L\subseteq E\cap K$ such that $\gamma\le\lambda L=\mu L$.

As $E$ and $\gamma$ are arbitrary, $\mu$ is tight;  as $\mu$ is
arbitrary, $X$ is a Radon space.
}%end of proof of 533E

\leader{533F}{Definition} Let $X$ be a topological space and $\mu$ a
topological measure on $X$.   I will say that $\mu$ is
{\bf uniformly regular} if there is a countable family $\Cal V$ of open
sets in $X$ such that $G\setminus\bigcup\{V:V\in\Cal V$, $V\subseteq G\}$
is negligible for every open set $G\subseteq X$.

\vleader{72pt}{533G}{Lemma} Let $(X,\frak T,\Sigma,\mu)$ be a compact Radon
measure space.

(a) The following are equiveridical:

\quad(i) $\mu$ is uniformly regular;

\quad(ii) there are a metrizable space $Z$ and a continuous function
$f:X\to Z$ such that $\mu f^{-1}[f[F]]=\mu F$ for every closed
$F\subseteq X$;

\quad(iii) there is a countable family $\Cal H$ of cozero sets in $X$
such that $\mu G=\sup\{\mu H:H\in\Cal H$, $H\subseteq G\}$ for every
open set $G\subseteq X$;

\quad(iv) there is a countable family $\Cal E$ of zero sets in $X$ such
that $\mu G=\sup\{\mu E:E\in\Cal E$, $E\subseteq G\}$ for every open set
$G\subseteq X$.

(b) If $\frak T$ is perfectly normal, the following are equiveridical:

\quad(i) $\mu$ is uniformly regular;

\quad(ii) there are a metrizable space $Z$ and a continuous function
$f:X\to Z$ such that $\mu f^{-1}[f[E]]=\mu E$ for every $E\in\Sigma$;

\quad(iii) there are a metrizable space $Z$ and a continuous function
$f:X\to Z$ such that $f[G]\ne f[X]$ whenever $G\subseteq X$ is open and
$\mu G<\mu X$;

\quad(iv) there is a countable family $\Cal E$ of closed sets in $X$
such that $\mu G=\sup\{\mu E:E\in\Cal E$, $E\subseteq G\}$ for every
open set $G\subseteq X$.

\proof{{\bf (a)(i)$\Rightarrow$(iii)} Given $\Cal V$ as in 533F, then for
each $V\in\Cal V$ there is a cozero set $H_V\subseteq V$ of the same
measure.   \Prf\ $\frak T$ is completely regular, so
$\Cal H_V=\{H:H\subseteq V$ is a cozero set$\}$ has union $V$;  $\mu$ is
$\tau$-additive, so there is a sequence $\sequencen{H_n}$ in $\Cal H_V$
such that $\mu V=\mu(\bigcup_{n\in\Bbb N}H_n)$;  set
$H_V=\bigcup_{n\in\Bbb N}H_n$;  by 4A2C(b-iii), $H_V$ is a cozero
set.\ \QeD\  Now $\Cal H=\{H_V:V\in\Cal V\}$ witnesses that (iii) is
true.

\medskip

\quad{\bf (iii)$\Rightarrow$(iv)} Given $\Cal H$ as in (iii), then for
each $H\in\Cal H$ let $\sequencen{F_n(H)}$ be a non-decreasing sequence
of zero sets with union $H$ (4A2C(b-vi)).   Set
$\Cal E=\{F_n(H):H\in\Cal H$, $n\in\Bbb N\}$, so that $\Cal E$ is a
countable family of zero sets.   If $G\subseteq X$ is open,

\Centerline{$\mu G=\sup_{H\in\Cal H,H\subseteq G}\mu H
=\sup_{H\in\Cal H,H\subseteq G,n\in\Bbb N}\mu F_n(H)
\le\sup_{E\in\Cal E,E\subseteq G}\mu E\le\mu G$,}

\noindent so $\Cal E$ witnesses that (iv) is true.

\medskip

\quad{\bf (iv)$\Rightarrow$(ii)} Given $\Cal E$ as in (iv),
then for each $E\in\Cal E$ choose a continuous $f_E:X\to\Bbb R$ such
that $E=f_E^{-1}[\{0\}]$, and set $f(x)=\family{E}{\Cal E}{f_E(x)}$ for
$x\in X$.   Then
$f:X\to Z=\BbbR^{\Cal E}$ is continuous and $Z$ is metrizable and
$f^{-1}[f[E]]=E$ for every $E\in\Cal E$.   If $F\subseteq X$ is closed,
set $\Cal E_0=\{E:E\in\Cal E$, $E\cap F=\emptyset\}$.   Then
$\bigcup\Cal E_0$ has the same measure as $X\setminus F$ and does not
meet $f^{-1}[f[F]]$, so $\mu f^{-1}[f[F]]=\mu F$.   As $F$ is arbitrary,
$f$ and $Z$ witness that $\mu$ satisfies (ii).

\medskip

\quad{\bf (ii)$\Rightarrow$(i)} Take $Z$ and
$f:X\to Z$ as in (ii).   Replacing $Z$ by $f[X]$ if
necessary, we may suppose that $f$ is surjective, so that $Z$ is
compact, therefore second-countable (4A2P(a-ii)).   Let $\Cal U$ be a
countable base for the topology of $Z$ closed under finite unions, and
set $\Cal V=\{f^{-1}[U]:U\in\Cal U\}$, so that $\Cal V$ is a countable
family of open sets in $X$.   If $G\subseteq X$ is
open, set $F=X\setminus G$,
$\Cal U_0=\{U:U\in\Cal U$, $U\cap f[F]=\emptyset\}$,
$\Cal V_0=\{f^{-1}[U]:U\in\Cal U_0\}$.   Then
$Z\setminus f[F]=\bigcup\Cal U_0$ so
$X\setminus f^{-1}[f[F]]=\bigcup\Cal V_0$ and (because $\Cal U_0$ and
$\Cal V_0$ are closed under finite unions)

$$\eqalign{\sup\{\mu V:V\in\Cal V,\,V\subseteq G\}
&\ge\sup_{V\in\Cal V_0}\mu V
=\mu(X\setminus f^{-1}[f[F]])\cr
&=\mu X-\mu f^{-1}[f[F]]
=\mu X-\mu F
=\mu G.\cr}$$

\noindent Thus $\Cal V$ witnesses that $\mu$ is uniformly regular.

\medskip

{\bf (b)(i)$\Rightarrow$(iii)} If $\mu$ is uniformly regular, then by
(a-ii) there are
a metrizable space $Z$ and a continuous function $f:X\to Z$ such that
$\mu f^{-1}[f[F]]=\mu F$ for every closed $F\subseteq X$.   If now
$G\subseteq X$ is open and $\mu G<\mu X$, there is a sequence
$\sequencen{F_n}$ of closed sets with union $G$, because $\frak T$ is
perfectly normal.   In this case
$f^{-1}[f[G]]=\bigcup_{n\in\Bbb N}f^{-1}[f[F_n]]$ has the same measure
as $G$, so is not the whole of $X$, and $f[G]\ne f[X]$.   Thus $f$ and
$Z$ witness that (iii) is true.

\medskip

\quad{\bf (iii)$\Rightarrow$(ii)} Take $Z$ and $f$ from (iii).
Let $\nu$ be the image measure $\mu f^{-1}$ on $Z$;  then $\mu$ is a
Radon measure (418I again).   \Quer\ If $E\in\Sigma$ and
$\mu^*f^{-1}[f[E]]>\mu E$, let $E'\supseteq E$ be a Borel set such that
$\mu E'=\mu E$.   Because $X$ is perfectly normal, $E'$ belongs to the
Baire $\sigma$-algebra of $X$ (4A3Kb), so is Souslin-F (421L), therefore
K-analytic (422Hb);  consequently $f[E']$ is K-analytic (422Gd)
therefore measured by $\nu$ (432A).   This means that
$f^{-1}[f[E']]\in\Sigma$, and of course

\Centerline{$\mu f^{-1}[f[E']]\ge\mu^*f^{-1}[f[E]]>\mu E=\mu E'$.}

\noindent We can therefore find open sets $G\supseteq E'$ and
$G'\supseteq X\setminus f^{-1}[f[E']]$ such that $\mu G+\mu G'<\mu X$.
But now $G\cup G'$ is an open set of measure less than $\mu X$ and
$f[G\cup G']=f[X]$, which is supposed to be impossible.\ \Bang

Thus, for any $E\in\Sigma$, we have $\mu^*f^{-1}[f[E]]=\mu E$;  of
course it follows at once that $f^{-1}[f[E]]$ is measurable, with the
same measure as $E$, as required by (ii).

\medskip

\quad{\bf (ii)$\Rightarrow$(i)$\Leftrightarrow$(iv)}
These follow immediately from (a), because all closed sets in $X$ are zero
sets.
}%end of proof of 533G

\leader{533H}{Theorem} (a) Suppose that
$\cov\Cal N_{\omega_1}>\omega_1$.   Let
$X$ be a perfectly normal compact Hausdorff space.   Then every Radon
measure on $X$ is uniformly regular.

(b)\cmmnt{ ({\smc Plebanek 00})} Suppose that
$\cov\Cal N_{\omega_1}>\omega_1=\non\Cal N_{\omega}$.   Let
$X$ be a first-countable compact Hausdorff space.   Then every Radon
measure on $X$ is uniformly regular.

\proof{{\bf (a)} Let $\mu$ be a Radon measure on $X$.   \Quer\ If $\mu$
is not uniformly regular, then we can choose
$\ofamily{\xi}{\omega_1}{g_{\xi}}$ and
$\ofamily{\xi}{\omega_1}{G_{\xi}}$
inductively, as follows.   Given that $g_{\eta}:X\to\Bbb R$ is
continuous for every $\eta<\xi$, set
$f_{\xi}(x)=\ofamily{\eta}{\xi}{g_{\eta}(x)}$ for $x\in X$, so that
$f_{\xi}:X\to\BbbR^{\xi}$ is continuous.   By 533G(b-iii), there is an
open set $G_{\xi}$ such that $\mu G_{\xi}<\mu X$ and
$f_{\xi}[G_{\xi}]=f_{\xi}[X]$;  now $G_{\xi}$ is a cozero set and there
is a continuous function $g_{\xi}:X\to\Bbb R$ such that
$G_{\xi}=\{x:g_{\xi}(x)\ne 0\}$.   Continue.

At the end of the induction, we have a continuous function
$f_{\omega_1}:X\to\BbbR^{\omega_1}$, setting
$f_{\omega_1}(x)=\ofamily{\xi}{\omega_1}{g_{\xi}(x)}$ for each $x$.
Now $\omega_1$ is a precaliber of every measurable algebra (525J again),
and
$\mu(X\setminus G_{\xi})>0$ for each $\xi$, so there is an $x\in X$ such
that $A=\{\xi:x\notin G_{\xi}\}$ is uncountable (525Ca).   Set
$H=\{y:f_{\omega_1}(y)\ne f_{\omega_1}(x)\}$;  then $H$ is an open set,
so expressible as $\bigcup_{n\in\Bbb N}K_n$ where each $K_n$ is compact.
For each $\xi\in A$ there is an
$x_{\xi}\in G_{\xi}$ such that $f_{\xi}(x_{\xi})=f_{\xi}(x)$.   As
$g_{\xi}(x_{\xi})\ne 0=g_{\xi}(x)$, $x_{\xi}\in H$.   Let $n\in\Bbb N$
be such that $A'=\{\xi:\xi\in A$, $x_{\xi}\in K_n\}$ is uncountable.
Then

\Centerline{$f_{\omega_1}(x)
\in\overline{\{f_{\omega_1}(x_{\xi}):\xi\in A'\}}
\subseteq f_{\omega_1}[K_n]$;}

\noindent but this is impossible, because $K_n\subseteq H$.\ \Bang

So $\mu$ must be uniformly regular, as required.

\medskip

{\bf (b)} Let $\mu$ be a Radon measure on $X$.   If $\mu X=0$ then of
course $\mu$ is uniformly regular;  suppose $\mu X>0$.   As in (a)
and the proof of 533E, the Maharam type of $\mu$ is
countable.   Let $\frak A$ be the measure algebra of $\mu$;  then
$d(\frak A)\le\non\Cal N_{\omega}$ (524Me), so there is a set
$A\subseteq X$, of full outer measure, with $\#(A)\le\omega_1$ (521Lc).
For each
$x\in X$, let $\sequencen{V_{xn}}$ run over a base of neighbourhoods of
$x$.   Let $\Cal H$ be the family of sets
expressible as finite unions of $V_{xn}$ for $x\in A$ and $n\in\Bbb N$,
so that $\Cal H$ is a family of open sets in $X$ and
$\#(\Cal H)\le\omega_1$.

For any open $G\subseteq X$,
$\mu G=\sup\{\mu H:H\in\Cal H$, $H\subseteq G\}$.   \Prf\ Set
$H^*=\bigcup\{H:H\in\Cal H$, $H\subseteq G\}$.   For any $x\in A\cap G$,
there is an $n\in\Bbb N$ such that $V_{xn}\subseteq G$, and now
$V_{xn}\in\Cal H$, so $x\in H^*$.   Thus $G\setminus H^*$ does not meet
$A$;  as $A$ has full outer measure,

\Centerline{$\mu G=\mu H^*=\sup\{\mu H:H\in\Cal H$, $H\subseteq G\}$}

\noindent because $\{H:H\in\Cal H$, $H\subseteq G\}$ is closed under
finite unions.\ \QeD\  So there is a countable
$\Cal H'\subseteq\{H:H\in\Cal H$, $H\subseteq G\}$ such that
$\mu G=\sup_{H\in\Cal H'}\mu H$.

Let $\ofamily{\xi}{\omega_1}{H_{\xi}}$ run over $\Cal H$.   For
$\xi<\omega_1$, set

\Centerline{$\Cal G_{\xi}
=\{G:G\subseteq X$ is open, $\mu G=\sup\{\mu H_{\eta}:\eta\le\xi$,
$H_{\eta}\subseteq G\}\}$.}

\noindent Then $\bigcup_{\xi<\omega_1}\Cal G_{\xi}=\frak T$.   For each
$\xi<\omega_1$, set

\Centerline{$Y_{\xi}
=\{y:y\in X$, $V_{yn}\in\Cal G_{\xi}$ for every $n\in\Bbb N\}$;}

\noindent then $X=\bigcup_{\xi<\omega_1}Y_{\xi}$.   Now there is a
$\xi<\omega_1$ such that $Y_{\xi}$ has full outer measure.   \Prf\ Let
$\xi$ be such that $\mu^*Y_{\xi}=\mu^*Y_{\eta}$ for every $\eta\ge\xi$.
\Quer\ If $\mu^*Y_{\xi}<\mu X$, let $K\subseteq X\setminus Y_{\xi}$ be a
non-negligible measurable set.   Then the subspace measure $\mu_K$ is a
Radon measure with countable Maharam type, so

\Centerline{$\cov\Cal N(\mu_K)\ge\cov\Cal N_{\omega}
\ge\cov\Cal N_{\omega_1}>\omega_1$.}

\noindent Since $K\subseteq\bigcup_{\eta<\omega_1}Y_{\eta}$, there must
be some $\eta<\omega_1$ such that $\mu^*_K(K\cap Y_{\eta})>0$;   but now
$\mu^*(K\cap Y_{\eta})>0$ and $\eta>\xi$ and

\Centerline{$\mu^*Y_{\eta}
=\mu^*(Y_{\eta}\setminus K)+\mu^*(Y_{\eta}\cap K)>\mu^*Y_{\xi}$.  \Bang}

\noindent So $Y_{\xi}$ has full outer measure.\ \Qed

Set $\Cal H_{\xi}=\{H_{\eta}:\eta\le\xi\}$.   If $G\subseteq X$ is open,
and $H^*=\bigcup\{H:H\in\Cal H_{\xi}$, $H\subseteq G\}$, then
$G\setminus H^*$ is negligible.   \Prf\ Set
$\Cal V=\{V_{yn}:y\in Y_{\xi}$, $n\in\Bbb N$, $V_{yn}\subseteq G\}$,
$H_1^*=\bigcup\Cal V$.   Then $Y_{\xi}$ does not meet
$G\setminus H_1^*$, so $\mu H_1^*=\mu G$.   Let
$\Cal V_0\subseteq\Cal V$ be a countable set such that
$\mu(\bigcup\Cal V_0)=\mu G$.   If $V\in\Cal V_0$, then
$V\in\Cal G_{\xi}$ and $V\subseteq G$ so $V\setminus H^*$ is negligible.
Accordingly

\Centerline{$G\setminus H^*
\subseteq(G\setminus\bigcup\Cal V_0)
  \cup\bigcup_{V\in\Cal V_0}(V\setminus H^*)$}

\noindent is negligible.\ \QeD\   So if we take $\Cal H'$ to be the set
of finite unions of members of $\Cal H_{\xi}$, $\Cal H'$ will be a
countable family of open sets and $\mu G=\sup\{\mu H:H\in\Cal H'$,
$H\subseteq G\}$ for every open $G\subseteq X$.   Thus $\mu$ is
uniformly regular.
}%end of proof of 533H

%{\bf Problem} For (a), is it enough if $\add\Cal N_{\omega}>\omega_1$?

\leader{533I}{}\cmmnt{ We know from 435Fb/435H and 439P
that $\BbbR^{\Bbb N}$ is measure-compact and $\BbbR^{\frak c}$ is not.
It turns out that we already have a language in which to express a
necessary and sufficient condition for $\BbbR^{\kappa}$ to be
measure-compact.   To give the result in its full strength I repeat a
definition from 435Xk.

\medskip

\noindent}{\bf Definition} A completely regular space $X$ is {\bf
strongly measure-compact} if
$\mu X=\sup\{\mu^*K:K\subseteq X$ is compact$\}$ for every totally
finite Baire measure $\mu$ on $X$.

\cmmnt{\medskip

\noindent{\bf Remark} For the elementary properties of these spaces, see
435Xk.   I repeat one here:  a completely regular space $X$ is strongly
measure-compact iff it is measure-compact and pre-Radon.
\prooflet{\Prf{\bf (i)}
Suppose that $X$ is measure-compact and pre-Radon and
that $\mu$ is a totally finite Baire measure on $X$.   Because $X$ is
measure-compact, $\mu$ has an extension to a quasi-Radon measure
$\tilde\mu$ (435D);  because $X$ is pre-Radon, $\tilde\mu$ is Radon
(434Jb) and

$$\eqalign{\mu X
&=\tilde\mu X
=\sup_{K\subseteq X\text{ is compact}}\tilde\mu K\cr
&=\sup_{K\subseteq X\text{ is compact}}\tilde\mu^*K
\le\sup_{K\subseteq X\text{ is compact}}\mu^*K
\le\mu X.\cr}$$

\noindent As $\mu$ is arbitrary, $X$ is strongly measure-compact.
{\bf (ii)}
Suppose that $X$ is strongly measure-compact.   ($\alpha$) Let $\mu$ be
a Baire probability measure on $X$.   Then there is a
non-negligible compact set, so $X$ cannot be covered by the negligible
open sets;  by 435Fa, this is enough to ensure that $X$ is
measure-compact.   ($\beta$) Now let $\mu$ be a totally finite
$\tau$-additive Borel measure on $X$.   Write $\nu$ for the restriction
of $\mu$ to the Baire $\sigma$-algebra of $X$.   Then there is a compact
set $K\subseteq X$ which is not $\nu$-negligible.   \Quer\ If
$\mu(X\setminus K)=\mu X$, then, because $\mu$ is $\tau$-additive and
$X$ is regular, there is a closed set
$F\subseteq X\setminus K$ such that $\mu F+\nu^*K>\mu X$.   Because $X$
is completely regular, there is a zero set $G$ including $K$ and
disjoint from $F$, in which case $\nu^*K>\mu G=\nu G$, which is
impossible.\ \BanG\   So $\mu K>0$;  by 434J(a-iii), this tells us that
$X$ is pre-Radon.\ \Qed
}}

\vleader{72pt}{533J}{Theorem}\cmmnt{ (see {\smc Fremlin 77})} Let $\kappa$
be a cardinal.   Then the following are equiveridical:

(i) $\BbbR^{\kappa}$ is measure-compact;

(ii) if $\ofamily{\xi}{\kappa}{X_{\xi}}$ is a family of strongly
measure-compact completely regular Hausdorff spaces then
$\prod_{\xi<\kappa}X_{\xi}$ is measure-compact;

(iii) whenever $X$ is a compact Hausdorff space and
$\ofamily{\xi}{\kappa}{G_{\xi}}$ is a family of cozero sets in $X$, then
$X\cap\bigcap_{\xi<\kappa}G_{\xi}$ is measure-compact;

(iv) for any Radon measure, the union of $\kappa$ or fewer closed
negligible sets has inner measure zero;

(v) for any Radon measure, the union of $\kappa$ or fewer
negligible sets has inner measure zero;

(vi) $\kappa<\cov\Cal N(\mu)$ for any Radon measure $\mu$;

(vii) $\kappa<\cov\Cal N_{\kappa}$;

(viii) $\kappa<\frak m(\frak A)$ for every measurable algebra $\frak A$.

\proof{{\bf not-(iv)$\Rightarrow$not-(i)} Suppose that $X$ is a
Hausdorff space, $\mu$ is a Radon measure on $X$ and
$\ofamily{\xi}{\kappa}{F_{\xi}}$ is a family of closed $\mu$-negligible
subsets of $X$ such that $\mu_*(\bigcup_{\xi<\kappa}F_{\xi})>0$.   Then
there is a compact set $K\subseteq\bigcup_{\xi<\kappa}F_{\xi}$ such that
$\mu K>0$.

For each $\xi<\kappa$, there is a continuous $g_{\xi}:K\to\coint{0,1}$
such that $g_{\xi}(z)=0$ for $z\in K\cap F_{\xi}$ and
$g_{\xi}^{-1}[\{0\}]$ is negligible.   \Prf\ For each $n\in\Bbb N$,
there is a compact set $L_n\subseteq K\setminus F_{\xi}$ such that
$\mu L_n\ge\mu K-2^{-n}$;  there is a continuous $f_n:K\to[0,1]$ such that
$f_n(z)=0$ for $z\in K\cap F_{\xi}$, $1$ for $z\in L_n$;  set
$g_{\xi}=\sum_{n=0}^{\infty}2^{-n-2}f_n$.\ \QeD\   Set
$g(z)=\ofamily{\xi}{\kappa}{g_{\xi}(z)}$ for $z\in K$, so that
$g:K\to\coint{0,1}^{\kappa}$ is continuous.

Let $\nu$ be the Baire measure on $[0,1]^{\kappa}$ defined by
setting $\nu H=\mu g^{-1}[H]$ for every Baire set
$H\subseteq[0,1]^{\kappa}$.   Then $\ooint{0,1}^{\kappa}$ has full outer
measure for $\nu$.   \Prf\ If $H\subseteq[0,1]^{\kappa}$ is a Baire
set including $\ooint{0,1}^{\kappa}$, then $H$ is determined by
coordinates in some countable subset $I$ of $\kappa$ (4A3Mb).   If
$z\in K$ and $g_{\xi}(z)>0$ for every $\xi\in I$, then
$g(z)\restr I\in\ooint{0,1}^I$ is equal to $w\restr I$ for some
$w\in H$, so $g(z)\in H$.   Thus $g^{-1}[H]$ includes
$\{z:z\in K$, $g_{\xi}(z)>0$ for every $\xi\in I\}$ and

\Centerline{$\nu H=\mu g^{-1}[H]
\ge\mu\{z:g_{\xi}(z)>0$ for every $\xi\in I\}
=\mu K=\nu[0,1]^{\kappa}$. \Qed}

On the other hand, every point $y$ of $\ooint{0,1}^{\kappa}$ belongs to
a $\nu$-negligible cozero set.   \Prf\ $g[K]$ is a compact set not
containing $y$, so there is a cozero set $W$ containing $y$ and disjoint
from $g[K]$, and now $\nu W=0$.\ \Qed

Let $\nu_0$ be the subspace measure on $\ooint{0,1}^{\kappa}$.   By
4A3Nd, $\nu_0$ is a Baire measure on $\ooint{0,1}^{\kappa}$.   If
$y\in\ooint{0,1}^{\kappa}$ it belongs to a $\nu$-negligible cozero
set $W\subseteq[0,1]^{\kappa}$, and now $W\cap\ooint{0,1}^{\kappa}$ is a
$\nu_0$-negligible cozero set in $\ooint{0,1}^{\kappa}$ containing
$y$.   At the same time,

\Centerline{$\nu_0\ooint{0,1}^{\kappa}=\nu[0,1]^{\kappa}
=\mu K>0$.}

\noindent So $\nu_0$ witnesses that $\ooint{0,1}^{\kappa}$ is
not measure-compact;  as $\BbbR^{\kappa}$ is homeomorphic to
$\ooint{0,1}^{\kappa}$, it also is not measure-compact.

\medskip

{\bf (iv)$\Rightarrow$(iii)} Suppose that (iv) is true and that we have
$X$ and $\ofamily{\xi}{\kappa}{G_{\xi}}$, as in (iii), with a Baire
probability measure $\mu$ on $Y=X\cap\bigcap_{\xi<\kappa}G_{\xi}$.   Let
$\nu$ be the Radon probability measure on $X$ defined by saying that
$\int fd\nu=\int(f\restr Y)d\mu$ for every $f\in C(X)$ (436J/436K again).
Then $\nu G_{\xi}=1$ for each $\xi<\kappa$.   \Prf\ Let $f:X\to\Bbb R$
be a continuous function such that $G_{\xi}=\{x:x\in X$, $f(x)\ne 0\}$.
Set $f_n=n|f|\wedge\chi X$ for each $n$.   Then
$\lim_{n\to\infty}f_n=\chi G_{\xi}$, so

\Centerline{$\nu G_{\xi}
=\lim_{n\to\infty}\int f_nd\nu
=\lim_{n\to\infty}\int(f_n\restr Y)d\mu=\mu Y=1$.\ \Qed}

\noindent By (iv), $\nu_*(\bigcup_{\xi<\kappa}(X\setminus G_{\xi}))=0$,
that is, $Y$ has full outer measure.   In particular, $Y$ must meet the
support of $\nu$;  take any $z$ in the intersection.   If $U$ is a
cozero set in $Y$ containing $z$, there is an open set $G\subseteq X$
such that $U=G\cap Y$;  now there is a continuous $f:X\to[0,1]$ such
that $f(z)=1$ and $f(x)=0$ for $x\in X\setminus G$;  in this case

\Centerline{$\mu U\ge\int(f\restr Y)d\mu=\int fd\nu>0$}

\noindent because $\{x:f(x)>0\}$ is an open set meeting the support of
$\nu$.   This shows that $Y$ is not covered by the $\mu$-negligible
relatively cozero sets;  as $\mu$ is arbitrary, $Y$ is measure-compact
(435Fa).

\medskip

{\bf (iii)$\Rightarrow$(i)} We can express $\BbbR^{\kappa}$ in the form of
(iii) by taking $X=[-\infty,\infty]^{\kappa}$ and
$G_{\xi}=\{x:x(\xi)$ is finite$\}$ for each $\xi$.

\medskip

{\bf (iv)$\Rightarrow$(vii)} Let $Z$ be the Stone space of the measure
algebra of $\nu_{\kappa}$, and $\lambda$ its usual measure.   If
$\ofamily{\xi}{\kappa}{E_{\xi}}$ is a family of $\lambda$-negligible
sets, then, because $\lambda$ is inner regular with respect to the
open-and-closed sets, we can find negligible zero sets
$F_{\xi}\supseteq E_{\xi}$ for each $\xi$.   By (iv),
$\{F_{\xi}:\xi<\kappa\}$ cannot cover $Z$, so the same is true of
$\{E_{\xi}:\xi<\kappa\}$.   Thus $\cov\Cal N(\lambda)>\kappa$.   By
524Jb, $\cov\Cal N_{\kappa}>\kappa$.

\medskip

{\bf (vii)$\Rightarrow$(vi)} Let $\theta$ be
$\min\{\cov\Cal N(\nu):\nu$ is a non-zero Radon measure$\}$.   By 524Pc,
there is an infinite cardinal $\kappa'$ such that
$\theta=\cov\Cal N_{\kappa'}$;  by 523F, $\theta=\cov\Cal N_{\theta}$.
\Quer\ If $\theta\le\kappa$, then 523B tells us that

\Centerline{$\kappa<\cov\Cal N_{\kappa}
\le\cov\Cal N_{\theta}=\theta$.  \Bang}

\noindent So $\theta>\kappa$, as required.

\medskip

{\bf (vi)$\Rightarrow$(v)} If (vi) is true, $(X,\mu)$
is a Radon measure space, $\ofamily{\xi}{\kappa}{F_{\xi}}$ is a family
of negligible sets, and $E\subseteq\bigcup_{\xi<\kappa}F_{\xi}$
is a measurable set, then the subspace measure $\mu_E$ is a Radon
measure (416Rb), while $E$ can be covered by $\kappa$ negligible sets;
by (vi), $\mu E=0$;  as $E$ is arbitrary,
$\mu_*(\bigcup_{\xi<\kappa}F_{\xi})=0$.

\medskip

{\bf (v)$\Rightarrow$(ii)} Suppose that (v) is true, that
$\ofamily{\xi}{\kappa}{X_{\xi}}$ is a family of strongly
measure-compact completely regular Hausdorff spaces with product $X$,
and that $\mu$ is a Baire probability measure on $X$.   For each
$\xi<\kappa$ let $Z_{\xi}$ be the Stone-\v{C}ech compactification of
$X_{\xi}$;  set $Z=\prod_{\xi<\kappa}Z_{\xi}$, and $\pi_{\xi}(z)=z(\xi)$
for $z\in Z$, $\xi<\kappa$.   Then we have a Radon probability measure
$\lambda$ on $Z$ defined by saying that
$\int g\,d\lambda=\int_X(g\restr X)d\mu$ for every $g\in C(Z)$.   Note
that if $W\subseteq Z$ is a zero set, there is a non-increasing sequence
$\sequencen{g_n}$ in $C(Z)$ with infimum $\chi W$, so that

\Centerline{$\lambda W=\inf_{n\in\Bbb N}\int g_nd\lambda
=\inf_{n\in\Bbb N}\int_X(g_n\restr X)d\mu=\mu(W\cap X)$.}

Now $\lambda\pi_{\xi}^{-1}[X_{\xi}]=1$ for each $\xi$.   \Prf\ Let
$\epsilon>0$.   We have a Baire probability measure $\mu_{\xi}$ on
$X_{\xi}$ defined by setting $\mu_{\xi}E=\mu(X\cap\pi_{\xi}^{-1}[E])$
for every Baire set $E\subseteq X_{\xi}$, and a Radon measure
$\lambda_{\xi}=\lambda\pi_{\xi}^{-1}$ on $Z_{\xi}$.   Because $X_{\xi}$
is strongly measure-compact, there is a compact set $K\subseteq X_{\xi}$
such that
$\mu_{\xi}^*K\ge 1-\epsilon$.   Now $K$ is still compact when regarded
as a subset of $Z_{\xi}$, so there is a zero set $F\subseteq Z_{\xi}$,
including $K$, such that $\lambda_{\xi}F=\lambda_{\xi}K$.   In this
case, $F\cap X_{\xi}$ is a zero set in $X_{\xi}$ including $K$, so

$$\eqalign{\lambda_*\pi_{\xi}^{-1}[X_{\xi}]
&\ge\lambda\pi_{\xi}^{-1}[K]
=\lambda_{\xi}K
=\lambda_{\xi}F
=\lambda\pi_{\xi}^{-1}[F]\cr
&=\mu(X\cap\pi_{\xi}^{-1}[F])
=\mu_{\xi}(F\cap X_{\xi})
\ge\mu_{\xi}^*K
\ge 1-\epsilon.\cr}$$

\noindent As $\epsilon$ is arbitrary, we have the result.\ \Qed

By (v), $X=Z\cap\bigcap_{\xi<\kappa}\pi_{\xi}^{-1}[X_{\xi}]$ has full
outer measure for $\lambda$.   Let $\Cal G$ be the family of
$\mu$-negligible cozero sets in $X$ and $\Cal H$ the family of
$\lambda$-negligible open sets in $Z$.   If $x\in G\in\Cal G$, then
there is a continuous function $g:Z\to[0,1]$ such that $g(x)=1$ and
$H=\{y:y\in X$, $g(y)>0\}$ is included in $G$;  now
$\int g\,d\lambda=\int(g\restr X)d\mu=0$, so $\lambda H=0$.   This shows
that $\bigcup\Cal G\subseteq\bigcup\Cal H$ is $\lambda$-negligible, and,
in particular, is not the whole of $X$.   By 435Fa as usual,
this is enough to show that $X$ is measure-compact, as required.

\medskip

{\bf (ii)$\Rightarrow$(i)} is elementary, because $\Bbb R$ is certainly
strongly measure-compact.

\medskip

{\bf (vi)$\Rightarrow$(viii)$\Rightarrow$(vii)} are immediate from 524Md.
}%end of proof of 533J

\exercises{\leader{533X}{Basic exercises (a)}
%\spheader 533Xa
Describe a family $\family{t}{\Bbb R}{\Cal K_t}$ such that every $\Cal K_t$
consists of compact sets,
Lebesgue measure on $\Bbb R$ is inner regular with respect to every
$\Cal K_t$, but $\bigcap_{t\in\Bbb R}\Cal K_t=\emptyset$.
%533A

\spheader 533Xb Let $\mu$ be a uniformly regular topological
measure on a topological space $X$.   (i) Show that if $A\subseteq X$
then the subspace measure on $A$ is uniformly regular.   (ii) Show that any
indefinite-integral measure over $\mu$ is uniformly regular.   (iii) Show
that if $Y$ is another topological space and $f:X\to Y$ is a continuous
open map, then the image measure $\mu f^{-1}$ is uniformly regular.
%533F

\spheader 533Xc Show that any Radon measure on the split interval is
uniformly regular.   \Hint{419L.}
%533F

\spheader 533Xd ({\smc Babiker 76}) Let $X$ and $Y$ be compact Hausdorff
spaces, $\mu$ a Radon measure on $X$, $f:X\to Y$ a continuous surjection
and $\nu=\mu f^{-1}$ the image measure on $Y$.   Show that the following
are equiveridical:  (i) $\nu f[F]=\mu F$ for every closed
$F\subseteq X$;  (ii)
$\int g\,d\mu=\inf\{\int h\,d\nu:h\in C(Y)$, $hf\ge g\}$ for every
$g\in C(X)$;  (iii) for every $g\in C(X)$,
$\{y:g$ is constant on $f^{-1}[\{y\}]\}$ is $\nu$-conegligible.
%533F

\spheader 533Xe Show that any uniformly regular Borel measure has
countable Maharam type.
%533G

\spheader 533Xf Let $\familyiI{X_i}$ be a countable family of topological
spaces with product $X$, and $\mu$ a $\tau$-additive topological
measure on $X$.   Suppose that the marginal measure of $\mu$ on $X_i$ is
uniformly regular for every $i\in I$.
Show that $\mu$ is uniformly regular.
%533G mt53bits

\spheader 533Xg Let $X$ be $[0,1]\times\{0,1\}$ with the topology
generated by

$$\eqalign{\{G\times\{0,1\}:G\subseteq[0,1]
&\text{ is relatively open for the usual topology}\}\cr
&\cup\{\{(t,1)\}:t\in[0,1]\}
\cup\{X\setminus\{(t,1)\}:t\in[0,1]\}.\cr}$$

\noindent Show that $X$ is compact and Hausdorff.   Let $\mu$ be the
Radon measure on $X$ which is the image of Lebesgue measure on $[0,1]$
under the map $t\mapsto(t,0)$.   Show that $\mu$ is uniformly regular
but not completion regular.
%533G

\spheader 533Xh Let $X$ be a topological space and $\mu$ a
uniformly regular topological probability measure on $X$.
Show that there is an equidistributed sequence in $X$.
%533G

\spheader 533Xi Show that there is a first-countable compact Hausdorff
space with a uniformly regular topological probability measure,
inner regular with respect to the closed sets, which is not
$\tau$-additive.   \Hint{439K.}
%533G

\leader{533Y}{Further exercises (a)}
%\spheader 533Ya
({\smc Pol 82}) Let $X$ be a compact Hausdorff space and
$\mu$ a uniformly regular Radon measure on $X$.   Show that if we give
the space $M^+_{\text{R}}$ of Radon measures on $X$ its narrow topology
(437Jd) then $\chi(\mu,M^+_{\text{R}})\le\omega$.
%533G

\spheader 533Yb For a topological measure $\mu$ on a space
$X$, write $\ureg(\mu)$ for the smallest size of any family $\Cal V$ of
open subsets of $X$ such that
$G\setminus\bigcup\{V:V\in\Cal V$, $V\subseteq G\}$ is negligible for every
open $G\subseteq X$.   (i) Show that if $\mu$ is inner regular with
respect to the Borel sets then the Maharam type $\tau(\mu)$ of $\mu$
is at most $\ureg(\mu)$.   (ii) Show that if $X$ is compact and Hausdorff
and $\mu$ is a Radon measure, then
$\ureg(\mu)\le\max(\non\Cal N_{\tau(\mu)},\chi(X))$.
(iii) Show that if $X$ is compact and Hausdorff,
$\mu$ is a Radon probability measure and
$\cov\Cal N_{\tau(\mu)}>\ureg(\mu)$, then $\mu$ has an equidistributed
sequence.
%533H

\spheader 533Yc ({\smc Plebanek 00})  Suppose that $\kappa$ is a regular
infinite cardinal such that
$\non\Cal N_{\kappa}<\cov\Cal N_{\kappa}=\kappa$.   Let $(X,\mu)$ be a
Radon probability space such that $\chi(X)<\kappa$.   Show that $\mu$
has an equidistributed sequence.
%533Yb
%By 531P, $\kappa>\tau(\mu)$ so $\ureg(\mu)<\kappa$ if $X$ compact

\spheader 533Yd Let $(X,\frak T,\Sigma,\mu)$ be a Radon measure space with
countable Maharam type, $\Cal A\subseteq\Sigma$ a set with cardinal less
than $\add\Cal N_{\omega}$, and $\frak S$ the topology on $X$ generated by
$\frak T\cup\Cal A$.   Show that $\mu$ is $\frak S$-Radon.
%533B out of order query
}%end of exercises

\leader{533Z}{Problem} For which cardinals $\kappa$ is $\BbbR^{\kappa}$
Borel-measure-compact?

\discrversionA{If $\kappa<\cov\Cal N_{\kappa}$, yes, by 435Fd.   What if
$\kappa=\omega_1=\cov\Cal N_{\omega}<\frak p$?
$\BbbR^{\omega_1}$ is never Borel-measure-complete, by 434Id, because it
includes $\omega_1$.}{}

\endnotes{
\Notesheader{533} I suppose that from the standpoint of measure theory
the most fundamental of all the properties of $\omega$ is the fact that
the union of countably many Lebesgue negligible sets is again Lebesgue
negligible;  this is of course shared by every
$\kappa<\add\Cal N_{\omega}$ (which is in effect the definition of
$\add\Cal N_{\omega}$).   In 533A-533E %533A 533B 533C 533D 533E
and 533J we have results showing that uncountable cardinals can be
`almost countable' in other ways.   In each case the fact that $\omega$
has the property examined is either trivial (as in 533B) or a basic
result from Volume 4 (as in 533Cb, 533Cc and 533E).   Similarly, the
fact that $\frak c$ does {\it not} have any of these properties is
attested by classical examples.
If you are familiar with Martin's axiom you will not be surprised to
observe that everything here is sorted out if we assume that
$\frak m=\frak c$.

533H does not quite fit this pattern, and the hypothesis in 533Hb
definitely contradicts Martin's axiom.   `Uniformly regular' measures
got squeezed out of \S434 by shortage of space;  in the exercises
533Xb-533Xi % 533Xb 533Xc 533Xd 533Xe 533Xf 533Xg 533Xh 533Xi
I sketch some of what was missed.   Here I mention them
just to show that
there is more to say on the subject of first-countable and perfectly
normal spaces than I put into 531P and 531Q.   Another phenomenon of
interest is the occurrence of measures which are inner regular with respect
to a family of compact metrizable sets (462J, 533Ca, 533D).
}%end of notes

\discrpage

