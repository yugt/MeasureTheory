\frfilename{mt5a6.tex}\dvAnew{2009}
\versiondate{4.8.10/27.11.12}
\copyrightdate{2009}

\def\chaptername{Appendix}
\def\sectionname{Special axioms}

\long\def\doubleinset#1{\inset{\inset{\parindent=-20pt #1}}}
\def\Glsquare{{Gl}$\Square$}
\def\Sing{\mathop{\text{Sing}}}
\def\VVdash{\mskip5mu\vrule height 7.5pt depth 2.5pt width 0.5pt
  \mskip2.5mu\vrule height 7.5pt depth 2.5pt width 0.5pt
  \vrule height 2.75pt depth -2.25pt width 4pt\mskip2mu}
\def\VVdP{\VVdash_{\Bbb P}}


\def\JechM{{\smc Jech 03}} %........
\def\Jech{{\smc Jech 78}} %.....
%{\smc Devlin 84}} %..... serious problems according to ARDM
%{\smc Kanamori 03} %....
\def\JWI{{\smc Just \& Weese 96}} %.
\def\JWII{{\smc Just \& Weese 97}} %.
%\def\DS{{\smc Drake \& Singh 96}} 

\newsection{5A6}

This section contains very brief accounts of
some of the undecidable propositions
and special axioms which are used in this volume, with a few of their most
basic consequences:  the generalized
continuum hypothesis, the axiom of constructibility,
Jensen's Covering Lemma, square principles, Chang's transfer
principle, Todor\v{c}evi\'c's $p$-ideal dichotomy and the filter dichotomy.

%The \dvAformerly here, and later, gives very odd reports in the log file,
%but output seems to be sensible
\leader{5A6A}{The generalized continuum 
hypothesis}\dvAformerly{5{}A1P} {\bf (a)} The generalized
continuum hypothesis is the assertion

\Centerline{(GCH)\quad $2^{\kappa}=\kappa^+$ for every infinite cardinal
$\kappa$.}

\spheader 5A6Ab If GCH is true, then for infinite cardinals $\kappa$,
$\lambda$

$$\eqalign{\cff[\kappa]^{\le\lambda}
&=1\text{ if }\kappa\le\lambda,\cr
&=\kappa\text{ if }\lambda<\cf\kappa,\cr
&=\kappa^+\text{ otherwise}.\cr}$$

\noindent\prooflet{\Prf\ If $\kappa\le\lambda$, use 5A1E(e-i).   If
$\lambda<\kappa$ then

\Centerline{$\cff[\kappa]^{\le\lambda}\le\#([\kappa]^{\le\lambda})
\le\#(\Cal P\kappa)=2^{\kappa}=\kappa^+$.}

\noindent If $\lambda<\theta=\cf\kappa$, then
$[\kappa]^{\le\lambda}=\bigcup_{\xi<\kappa}[\xi]^{\le\lambda}$ so

\Centerline{$\cff[\kappa]^{\le\lambda}
\le\max(\kappa,\sup_{\xi<\kappa}\cff[\xi]^{\le\lambda})
\le\max(\kappa,\sup_{\xi<\kappa}\#(\xi)^+)
=\kappa$}

\noindent and we have equality (using the other part of 5A1E(e-i)).
If $\lambda=\theta$ then 5A1E(e-v) tells us that
$\cff[\kappa]^{\lambda}$ is greater than $\kappa$, so must be $\kappa^+$.
If $\theta<\lambda<\kappa$ then, by 5A1E(e-ii),

\Centerline{$\kappa<\cff[\kappa]^{\le\theta}
\le\max(\cff[\kappa]^{\le\lambda},\cff[\lambda]^{\le\theta})
\le\max(\cff[\kappa]^{\le\lambda},\lambda^+)
\le\max(\cff[\kappa]^{\le\lambda},\kappa)$,}

\noindent so again $\cff[\kappa]^{\le\lambda}=\kappa^+$.\ \Qed}

\spheader 5A6Ac If GCH is true, then for infinite cardinals $\kappa$ and
$\lambda$, the cardinal power $\kappa^{\lambda}$ is
$2^{\lambda}$ if $\kappa\le\lambda$,
$\kappa$ if $\lambda<\cf\kappa$,
$\kappa^+$ otherwise.
\prooflet{(Put (b) and 5A1E(e-iii) together.)}

\leader{5A6B}{$L$, $0^{\sharp}$ and Jensen's Covering
Lemma}\dvAformerly{5{}A1Q} {\bf (a)(i)}
Let $L$ be the class of {\bf
constructible} sets\prooflet{ (\JechM, \S13;
\Jech, \S12;
{\smc Kanamori 03}, \S3;  {\smc Kunen 80}, chap.\ VI)}. %;  \DS, chap.\ 8
The {\bf axiom of constructibility} is

\Centerline{(\VeqL)\quad Every set is constructible.}

\noindent \VeqL\ implies GCH
\prooflet{(\JechM, 13.20;
\Jech, Theorem 34;  {\smc Kunen 80}, \S VI.4)}. %;  \DS, \S8.6

\medskip

\quad{\bf (ii)}\cmmnt{ I will call on the following
three properties of $L$ in the
remarks below.   To make sense of them you will of course need to look at
the proper definition.   Only the third has any real content.}
Every ordinal belongs to $L$;
if $A$, $B\in L$ then $A\cap B\in L$;
if $\kappa$ is a cardinal, then $\#(L\cap\Cal P\kappa)\le\kappa^+$.

\spheader 5A6Bb $0^{\sharp}$, if it exists, is a set of sentences in a
countable formal language\prooflet{ (\JechM, \S18;
{\smc Kanamori 03}, \S9)}.   \cmmnt{I will not
attempt to explain further;  I mention $0^{\sharp}$ only so that you will
be able to explore the literature for proofs of the assertions below.}
I will write `$\exists 0^{\sharp}$' for the assertion
`$0^{\sharp}$ exists'.

{\bf Jensen's Covering Lemma} is the assertion

\doubleinset{(CL)\quad for every uncountable set $A$ of ordinals,
there is a
constructible set of the same cardinality including $A$.}

\noindent\cmmnt{Now }Jensen's Covering Theorem is

\Centerline{CL iff not-$\exists 0^{\sharp}$\dvro{.}{}}

\noindent\prooflet{({\smc Jech 03}, Theorem 18.30.)}

\spheader 5A6Bc\cmmnt{ The importance to us of $0^{\sharp}$ is that
there are
relatively direct proofs that \VeqL\ implies not-$\exists 0^{\sharp}$
(\JechM, \S18), and
that not-$\exists 0^{\sharp}$ is true
in any set forcing extension of a model of not-$\exists 0^{\sharp}$;  see
\JechM, Exercise 18.2 or \Jech, Exercise 30.2.   So} CL
implies that $\VVdP\,$CL for every forcing notion $\Bbb P$ of the kind
considered in \S5A3.

\leader{5A6C}{Theorem}\dvAformerly{5{}A1R} Assume that CL is true.

(a) For infinite cardinals $\kappa$ and $\lambda$,

$$\eqalign{\cff[\kappa]^{\le\lambda}
&=1\text{ if }\kappa\le\lambda,\cr
&=\kappa\text{ if }\lambda<\cf\kappa,\cr
&=\kappa^+\text{ otherwise}.\cr}$$

(b) If $\kappa$ and $\lambda$ are infinite cardinals, then the cardinal
power $\kappa^{\lambda}$ is
$2^{\lambda}$ if $\kappa\le 2^{\lambda}$,
$\kappa$ if $\lambda<\cf\kappa$ and $2^{\lambda}\le\kappa$,
and $\kappa^+$ otherwise.

%(c) If $X$ is a set and $\Cal I$ is a proper ideal of subsets of $X$
%containing singletons, then $\Cal I$ is not $(\add\Cal I)^+$-saturated.

\proof{{\bf (a)(i)} If $\omega_1\le\lambda<\kappa$ then
$\cff[\kappa]^{\le\lambda}\le\kappa^+$.   \Prf\ By CL, every
$A\in[\kappa]^{\lambda}$
is included in a $B\in[L]^{\lambda}$;  now $\kappa\in L$ so
$B\cap\kappa\in L$ and $A\subseteq B\cap\kappa\in[\kappa]^{\lambda}$.
Thus $L\cap[\kappa]^{\lambda}$ is cofinal with $[\kappa]^{\lambda}$ and
$[\kappa]^{\le\lambda}$.   But $\#(L\cap\Cal P\kappa)\le\kappa^+$
(5A6B(a-ii)), so $\cf[\kappa]^{\lambda}\le\kappa^+$.\ \Qed

\medskip

\quad{\bf (ii)} It follows that $\cff[\kappa]^{\le\lambda}\le\kappa^+$ for all
infinite cardinals $\kappa$ and $\lambda$.   \Prf\ The case
$\lambda\ge\kappa$ is trivial, so only the case
$\lambda=\omega<\kappa$ remains.   But

\Centerline{$\cff[\kappa]^{\le\omega}
\le\max(\cff[\kappa]^{\omega_1},\cff[\omega_1]^{\le\omega})
\le\max(\kappa^+,\omega_1)=\kappa^+$}

\noindent by 5A1E(e-ii) and (i).\ \Qed

\medskip

\quad{\bf (iii)} If $\lambda<\cf\kappa$ then
$\cff[\kappa]^{\le\lambda}\le\kappa$.   \Prf\
$[\kappa]^{\le\lambda}=\bigcup_{\xi<\kappa}[\xi]^{\le\lambda}$, so

\Centerline{$\cff[\kappa]^{\le\lambda}
\le\max(\kappa,\sup_{\xi<\kappa}\cff[\xi]^{\le\lambda})
\le\max(\kappa,\sup_{\xi<\kappa}\#(\xi)^+)
=\kappa$.  \Qed}

\medskip

\quad{\bf (iv)} If $\cf\kappa\le\lambda<\kappa$ then
$\cff[\kappa]^{\le\lambda}>\kappa$.   \Prf\ Set $\theta=\cf\kappa$.
Then

$$\eqalign{\kappa
&<\cff[\kappa]^{\le\theta}
\le\max(\cff[\kappa]^{\le\lambda},\cff[\lambda]^{\le\theta})\cr
&\le\max(\cff[\kappa]^{\le\lambda},\lambda^+)
\le\max(\cff[\kappa]^{\le\lambda},\kappa)\cr}$$

\noindent(5A1E(e-v), 5A1E(e-ii), (ii) above), so
$\cff[\kappa]^{\le\lambda}>\kappa$.\ \Qed

Putting this together with 5A1E(e-i), (ii) and (iii) we have the result.

\medskip

{\bf (b)} As 5A6Ac.

%\medskip

%{\bf (c)} \JechM, 22.22
%\Cal I would have to be precipitous
}%end of proof of 5A6C

\leader{5A6D}{Square principles}\dvAformerly{5{}A1S} {\bf (a)(i)}
Let $\Sing$ be the class of non-zero limit
ordinals which are not regular cardinals.   Global Square
%$\Glsquare$,
is the statement

\inset{there is a family
$\family{\xi}{\Sing}{C_{\xi}}$ such that

\quad for every $\xi\in\Sing$, $C_{\xi}$ is a closed cofinal set
in $\xi$;

\quad$\otp C_{\xi}<\xi$ for every $\xi\in\Sing$;

\quad if $\xi\in\Sing$ and $\zeta>0$ is such that
$\zeta=\sup(\zeta\cap C_{\xi})$, then $\zeta\in\Sing$ and
$C_{\zeta}=\zeta\cap C_{\xi}$.}

\medskip

\quad{\bf (ii)} For an infinite cardinal $\kappa$,
let $\square_{\kappa}$ be the statement

\inset{there is a family $\ofamily{\xi}{\kappa^+}{C_{\xi}}$ of
sets such that

\quad for every $\xi<\kappa^+$, $C_{\xi}\subseteq\xi$ is a
closed cofinal set in $\xi$;

\quad if $\cf\xi<\kappa$ then $\#(C_{\xi})<\kappa$;

\quad whenever $\xi<\kappa^+$ and $\zeta<\xi$ is such that
$\zeta=\sup(\zeta\cap C_{\xi})$, then
$C_{\zeta}=\zeta\cap C_{\xi}$.
}%end of inset
%5{}18I

\spheader 5A6Db The axiom of constructibility implies Global
Square\prooflet{ ({\smc Friedman \& Koepke 97})}.
%forthcoming article by P Welch, Sigma^* Fine Structure, Handbook of
%Set Theory, 2009?
Global Square implies that
$\square_{\kappa}$ is true for every infinite cardinal $\kappa$
\prooflet{({\smc Devlin 84}, VI.6.2)}.   Jensen's Covering Lemma
implies that $\square_{\kappa}$ is true for every singular
infinite cardinal $\kappa$\prooflet{ ({\smc Devlin 84}, V.5.6)}.

\spheader 5A6Dc If $\kappa$ is an uncountable cardinal and
$\ofamily{\xi}{\kappa^+}{C_{\xi}}$ is a family as in
(a-ii), then $\otp C_{\xi}\le\kappa$ for every $\xi<\kappa^+$.
\prooflet{\Prf\ If $\cf\xi<\kappa$ this is immediate from the second clause
of $\square_{\kappa}$.
Otherwise, $\kappa$ is regular, and $\cf C_{\xi}=\kappa$.
\Quer\ If $\otp C_{\xi}>\kappa$ then $\otp C_{\xi}>\kappa+\omega$ so
there is a $\zeta\in C_{\xi}$ such that
$\otp(\zeta\cap C_{\xi})=\kappa+\omega$;  but now $\cf\zeta=\omega<\kappa$
and $C_{\zeta}=\zeta\cap C_{\xi}$ has cardinal $\kappa$, which is not
allowed.\ \Bang\Qed}

%Jech 78 speaks of `box principle', \square_{\omega_2}
%Jech 03, 23.4 has a slightly stronger version, requiring also
%  \#(C_{\alpha})<\kappa  if  \alpha<\kappa^+  a limit and
%  \cf\alpha<\kappa
%5{}18 notes

\vleader{48pt}{5A6E}{Lemma}\dvAformerly{5{}A1T} Suppose that $\kappa$ is 
an uncountable cardinal
with countable cofinality such that $\square_{\kappa}$ is true.   Then
there is a family $\ofamily{\xi}{\kappa^+}{I_{\xi}}$ of countably
infinite subsets of $\kappa$ such that

\inset{$I_{\xi}\cap I_{\eta}$ is finite whenever $\eta<\xi<\kappa^+$,

$\{\xi:\xi<\kappa^+$, $I\cap I_{\xi}$ is infinite$\}$ is countable for
every countable $I\subseteq\kappa$.}

\proof{ Let $\ofamily{\xi}{\kappa^+}{C_{\xi}}$ be a family as in
5A6D(a-ii).   Let $\sequencen{\kappa_n}$ be a non-decreasing
sequence of infinite
cardinals less than $\kappa$ with supremum $\kappa$.   Define
$\ofamily{\xi}{\kappa^+}{f_{\xi}}$ in $\prod_{n\in\Bbb N}\kappa_n^+$
as follows.   $f_0(n)=0$ for every $n$.   Given $f_{\xi}$, set
$f_{\xi+1}(n)=f_{\xi}(n)+1$ for every $n$.   Given
$\ofamily{\eta}{\xi}{f_{\eta}}$, where $\xi<\kappa^+$ is a non-zero
limit ordinal, set

\Centerline{$f_{\xi}(n)
=\sup\{f_{\eta}(n):\eta\in C_{\xi}$, $\#(\eta\cap C_{\xi})<\kappa_n\}$}

\noindent for each $n$;  because
$\{\eta:\eta\in C_{\xi}$, $\#(\eta\cap C_{\xi})<\kappa_n\}$ has
cardinal at most $\kappa_n$, $f_{\xi}(n)<\kappa_n^+$.   Continue.

We find that if $\eta<\xi<\kappa^+$ then
$\{n:f_{\xi}(n)\le f_{\eta}(n)\}$ is finite.   \Prf\ Induce on $\xi$.
If $\xi=0$ there is nothing to prove.   If $\xi=\zeta+1$ then
$\{n:f_{\xi}(n)\le f_{\eta}(n)\}=\{n:f_{\zeta}(n)<f_{\eta}(n)\}$
is finite.   If $\xi$ is a non-zero limit ordinal, let
$\zeta\in C_{\xi}$ be such that $\eta<\zeta$.   Because
$\otp C_{\xi}\le\kappa$ (5A6Dc),
$\#(\zeta\cap C_{\xi})<\kappa_m$ for some
$m$.   Now $f_{\xi}(n)\ge f_{\zeta}(n)$ for every $n\ge m$, so
$\{n:f_{\xi}(n)\le f_{\eta}(n)\}
\subseteq m\cup\{n:f_{\zeta}(n)\le f_{\eta}(n)\}$ is finite.\ \Qed

If $I\subseteq\Bbb N\times\kappa$ is countable, then
$B=\{\xi:\xi<\kappa^+$, $I\cap f_{\xi}$ is infinite$\}$ is countable,
where in this formula I am identifying $f_{\xi}$ with its graph, as usual.
\Prf\Quer\ Otherwise, let $B'\subseteq B$ be a set with order type
$\omega_1$, and set $\xi=\sup B'<\kappa^+$.   Set

\Centerline{$I'=\{(n,\alpha):(n,\alpha)\in I$,
$\alpha\le f_{\eta}(n)$ for some $\eta\in C_{\xi}\}$.}

\noindent Because $I$ and $I'$ are countable, while
$\cf C_{\xi}=\cf\xi=\omega_1$, there is a $\zeta\in C_{\xi}$ such that
$\zeta=\sup(\zeta\cap C_{\xi})$ and

\Centerline{$I'=\{(n,\alpha):(n,\alpha)\in I$,
$\alpha\le f_{\eta}(n)$ for some $\eta\in\zeta\cap C_{\xi}\}$.}

\noindent Take $\eta\in B'$ such that $\eta>\zeta$, and
$\zeta'\in C_{\xi}$ such that $\zeta'>\eta$.   Then there is an
$m\in\Bbb N$ such that $\#(\zeta\cap C_{\xi})<\kappa_m$ and
$f_{\zeta}(n)<f_{\eta}(n)<f_{\zeta'}(n)$ for every $n\ge m$.   As
$\eta\in B$, there is an $n\ge m$ such that $(n,f_{\eta}(n))\in I$;  as
$f_{\eta}(n)<f_{\zeta'}(n)$, $(n,f_{\eta}(n))\in I'$ and there is an
$\eta'\in\zeta\cap C_{\xi}$ such that $f_{\eta}(n)\le f_{\eta'}(n)$.
But now we have $\eta'\in C_{\zeta}$ and
$\#(\eta'\cap C_{\zeta})\le\#(C_{\zeta})<\kappa_n$ and
$f_{\zeta}(n)<f_{\eta'}(n)$, contrary to the choice of $f_{\zeta}$.\
\Bang\Qed

Thus if we set $I_{\xi}=f_{\xi}$ for $\xi<\kappa^+$ we have an
appropriate family of sets in $\Bbb N\times\kappa^+$ which can be
transferred to $\kappa^+$ by any bijection.
}%end of proof of 5A6E

\leader{5A6F}{Chang's transfer principle}\dvAformerly{5{}A1U} {\bf (a)} 
If $\lambda_0$, $\lambda_1$, $\kappa_0$ and $\kappa_1$ are cardinals, then
$(\kappa_1,\lambda_1)\doubleheadrightarrow(\kappa_0,\lambda_0)$
means

\inset{whenever $f:[\kappa_1]^{<\omega}\to\lambda_1$ is a function, there
is an $A\in[\kappa_1]^{\kappa_0}$ such that
$\#(f[\,[A]^{<\omega}])\le\lambda_0$.}

\cmmnt{\noindent
For the original model-theoretic version of this principle, and
the proof that it comes to the same thing, see
{\smc Kanamori 03}, 8.1.   For
various combinatorial consequences, see `Chang's conjecture' in
{\smc Erd\H{o}s Hajnal M\'at\'e \& Rado 84}.}

\cmmnt{In this book, }I write CTP$(\kappa,\lambda)$ for the statement

\Centerline{$(\kappa,\lambda)\doubleheadrightarrow(\omega_1,\omega)$.}
\cmmnt{

\noindent What is commonly called `Chang's conjecture' is
CTP$(\omega_2,\omega_1)$.   For a model of
GCH + CTP$(\omega_{\omega+1},\omega_{\omega})$, see
{\smc Levinski Magidor \& Shelah 90}.}

\spheader 5A6Fb Suppose that CTP$(\kappa,\lambda)$ is true.

\medskip

\quad{\bf (i)} If $f:[\kappa]^{<\omega}\to[\lambda]^{\le\omega}$ is a
function, then there is an uncountable $A\subseteq\kappa$ such that
$\bigcup\{f(I):I\in[A]^{<\omega}\}$ is countable.   \prooflet{\Prf\
Enumerate $\Bbb N\times\Bbb N$ as $\sequencen{(k_n,m_n)}$ in
such a way that $m_n\le n$ for every $n\in\Bbb N$.   For
$I\in[\kappa]^{<\omega}$ let $\sequence{k}{f_k(I)}$ be a sequence running
over $f(I)\cup\{0\}$.   (I am passing over the trivial case $\lambda=0$.)
Now, for $n\in\Bbb N$ and
$I\in[\kappa]^n$, enumerate $I$ in ascending order as
$\ofamily{i}{n}{\xi_i}$ and set $g(I)=f_{k_n}(\{\xi_i:i<m_n\})$.
There is an uncountable $A\subseteq\kappa$ such that
$B=\{g(I):I\in[A]^{<\omega}\}$ is countable;  we may suppose that $A$ has
order type $\omega_1$.   If $J\in[A]^{<\omega}$ and $k\in\Bbb N$, let
$n\in\Bbb N$ be such that $k_n=k$ and $m_n=\#(J)$;  let $I\in[A]^n$ be such
that $J$ consists of the first $m_n$ elements of $I$;  then $f_k(J)=g(I)$
belongs to $B$.   As $J$ and $k$ are arbitrary,
$\bigcup\{f(I):I\in[A]^{<\omega}\}\subseteq B$ is countable.\ \Qed}

\medskip

\quad{\bf (ii)} If
$\ofamily{\xi}{\kappa}{A_{\xi}}$ is any family of countable subsets of
$\lambda$, then there is a countable $A\subseteq\lambda$ such that
$\{\xi:A_{\xi}\subseteq A\}$ is uncountable.   \prooflet{\Prf\ In (i), take
$f(I)=\bigcup_{\xi\in I}A_{\xi}$ for $I\in[\kappa]^{<\omega}$.\ \Qed}

\spheader 5A6Fc CL implies that
CTP$(\kappa,\lambda)$ is false except when 
$\lambda\le\omega$\prooflet{ ({\smc Kanamori 03}, 8.3, 21.1 and 21.4)}.

\leader{5A6G}{Todor\v{c}evi\'c's $p$-ideal
dichotomy}\dvAformerly{5{}A1V} {\bf (a)} Let $X$ be a set
and $\Cal I$ an ideal of subsets of $X$.   Then $\Cal I$ is a
{\bf $p$-ideal} if for every sequence $\sequencen{I_n}$ in $\Cal I$ there
is an $I\in\Cal I$ such that $I_n\setminus I$ is finite for every
$n\in\Bbb N$.   \cmmnt{(Compare 538Ab.)}

\spheader 5A6Gb Now Todor\v{c}evi\'c's $p$-ideal dichotomy is the statement

\doubleinset{(TPID)\quad
whenever $X$ is a set and $\Cal I\subseteq[X]^{\le\omega}$
is a $p$-ideal of countable subsets of $X$, then
{\it either} there is a $B\in[X]^{\omega_1}$ such that
$[B]^{\le\omega}\subseteq\Cal I$
{\it or} $X$ is expressible as $\bigcup_{n\in\Bbb N}X_n$ where
$\Cal I\cap\Cal PX_n\subseteq[X_n]^{<\omega}$ for every $n\in\Bbb N$.}

\noindent This is a consequence of the Proper Forcing Axiom, and
implies that $\square_{\kappa}$ is false for every
$\kappa\ge\omega_1$\cmmnt{ ({\smc Todor\v{c}evi\'c 00})}.
%mt5abits so also contradicts CL

\leader{*5A6H}{Analytic $P$-ideals:  Theorem}\dvAformerly{5{}A1W}
Suppose that the Proper Forcing Axiom is true.
Take a non-empty set $D\subseteq\coint{0,\infty}^{\Bbb N}$ and set

\Centerline{$\Cal I=\{I:I\subseteq\Bbb N$,
  $\lim_{n\to\infty}\sup_{z\in D}\sum_{i\in I\setminus n}z(i)=0\}$,}

\noindent so that $\Cal I$ is an ideal of subsets of $\Bbb N$.
Let $\frak A$ be the quotient Boolean algebra $\Cal P\Bbb N/\Cal I$.
Then for every $\pi\in\Aut\frak A$ there are sets $I$, $J\in\Cal I$ and a
bijection
$h:\Bbb N\setminus I\to\Bbb N\setminus J$ representing $\pi$ in the sense
that $\pi(A^{\ssbullet})=(h^{-1}[A])^{\ssbullet}$ for every
$A\subseteq\Bbb N$.
\prooflet{({\smc Farah 00}, 3.4.6.)}
%5{}56

\leader{5A6I}{}{\bf $\frak u$, $\frak g$ and the filter
dichotomy:}\dvAnew{2009} {\bf Definitions
(a)} The {\bf ultrafilter number} $\frak u$ is 
the least cardinal of any filter base
generating a free ultrafilter on $\Bbb N$\cmmnt{, that is,
$\min\{\ci\Cal F:\Cal F$ is a free ultrafilter on $\Bbb N\}$}.

\spheader 5A6Ib{\bf (i)} A family $A$ of infinite subsets of $\Bbb N$
is {\bf groupwise dense} if

\inset{($\alpha$) whenever $a\in A$, $a'\in[\Bbb N]^{\omega}$ and
$a'\setminus a$ is finite, then $a'\in A$,

($\beta$) whenever $\phi:\Bbb N\to\Bbb N$ is finite-to-one, 
there is an infinite $c\subseteq\Bbb N$ such that
$\phi^{-1}[c]\in A$.}

\noindent (A function $f:X\to Y$ is `finite-to-one' if $f^{-1}[\{y\}]$ is
finite for every $y\in Y$.)

\medskip

\quad{\bf (ii)} The {\bf groupwise density number} $\frak g$ is
the least cardinal of any collection $\Bbb A$
of groupwise dense subsets of $[\Bbb N]^{\omega}$ such that
$\bigcap\Bbb A=\emptyset$.

%\frak g\le\cf(\frak d) ??
%\frak g is regular and uncountable (Mildenberger)

\cmmnt{\medskip

\quad{\bf (iii)} For a model in which $\omega_1=\frak u<\frak g$ see
{\smc Blass \& Laflamme 89}.}

\spheader 5A6Ic 
For filters $\Cal F$ on $X$ and $\Cal G$ on $Y$, 
say that $\Cal F\leRB\Cal G$ if there is a finite-to-one $\phi:Y\to X$ such
that $\Cal F=\phi[[\Cal G]]$.    (This is the {\bf Rudin-Blass ordering} of
filters.)   \cmmnt{Note that }$\Cal F\leRB\Cal F$ for every 
filter $\Cal F$, 
and if $\Cal F\leRB\Cal G$ and $\Cal G\leRB\Cal H$ then
$\Cal F\leRB\Cal H$ (and $\Cal F\leRK\Cal G$\cmmnt{, of course}).

\spheader 5A6Id The {\bf filter dichotomy} is the statement

\doubleinset{(FD)\quad
For every free filter $\Cal F$ on $\Bbb N$ either $\CalFr\leRB\Cal F$,
where $\CalFr$ is the Fr\'echet filter, or there is an ultrafilter $\Cal G$
on $\Bbb N$ such that $\Cal G\leRB\Cal F$.}
%5{}38S

\noindent

\leader{*5A6J}{Proposition}\dvAnew{2009}\cmmnt{ ({\smc Blass \& 
Laflamme 89})}
If $\frak u<\frak g$ then the filter dichotomy is true.

\proof{ Let $\Cal F$ be a free filter on $\Bbb N$ such that
$\CalFr\not\leRB\Cal F$, where $\CalFr$ is the Fr\'echet filter.

\medskip

{\bf (a)}   For subsets $a$, $b$, $c$ of $\Bbb N$ I will say that 
{\bf $b$ interpolates between $a$ and $c$}, and write $(a(b)c)$,
if whenever $i\in a$, $k\in c$ and $i\le k$ then there is a $j\in b$ such
that $i\le j\le k$.   Now if $b\subseteq\Bbb N$ is infinite,

\Centerline{$A_b=\{a:a\in[\Bbb N]^{\omega}$, $(a(b)c)$ for some 
$c\in\Cal F\}$}

\noindent is groupwise dense.   \Prf\ ($\alpha)$ If $a\in A_b$, 
$a'\in[\Bbb N]^{\omega}$ and
$a'\setminus a$ is finite, let $c\in\Cal F$ be such that $(a(b)c)$.   
Let $j_0\in b$ be such that 
$a'\setminus a\subseteq j_0$, and set $c'=c\setminus j_0\in\Cal F$.
If $i\in a'$, $k\in c'$ and $i\le k$, either $i\le j_0$ and 
$i\le j_0\le k$, or $i\in a$ and there is a $j\in b$ such that 
$i\le j\le k$;  thus $(a'(b)c')$ and 
$a'\in A_b$.   ($\beta$) If $\phi:\Bbb N\to\Bbb N$ is finite-to-one,
we can choose strictly increasing sequences
$\sequence{r}{l_r}$, $\sequence{r}{m_r}$ in $\Bbb N$ such that

\inset{$l_0\in\phi[\Bbb N]$,

given $l_r$, $m_r\in b$ and $i<m_r$ whenever $\phi(i)=l_r$,

given $m_r$, $l_{r+1}\in\phi[\Bbb N]$ and $i\ge m_r$ whenever
$\phi(i)=l_{r+1}$.}

\noindent Set $\psi(i)=\#(\{r:m_r\le i\})$ for $i\in\Bbb N$;  then
$\psi:\Bbb N\to\Bbb N$ is finite-to-one.   As $\Cal F$ is free,
$\CalFr\subseteq\psi[[\Cal F]]$;  as $\CalFr\not\leRB\Cal F$,
$\psi[[\Cal F]]\ne\CalFr$ and there is an infinite set 
$d\subseteq\Bbb N$ such that
$d'=\Bbb N\setminus d\in\psi[[\Cal F]]$.   Of course we can suppose that
$0\notin d$.

If $i\in\psi^{-1}[d]$, $k\in\psi^{-1}[d']$ and $i\le k$, then
$\psi(i)\in d$, $\psi(k)\in d'$ and $\psi(i)\le\psi(k)$, so
$\psi(i)<\psi(k)$;  setting $r=\psi(i)$, we have $i<m_r\le k$, while
$m_r\in b$.   This shows that $(\psi^{-1}[d](b)\psi^{-1}[d'])$, so that
$\psi^{-1}[d]\in A_b$.
Next, setting $c=\{l_r:r\in d\}$, $c$ is infinite;  and if
$i\in\phi^{-1}[c]$, we have an $r\in d$ such that $\phi(i)=l_r$, in which
case (as $r>0$) $m_{r-1}\le i<m_r$ and $\psi(i)=r$.   So
$\phi^{-1}[c]\subseteq\psi^{-1}[d]$;  also $\phi^{-1}[c]$ is infinite
(because $l_r\in\phi[\Bbb N]$ for every $r$), so $\phi^{-1}[c]\in A_b$.
As $\phi$ is arbitrary, $A_b$ is groupwise dense.\ \Qed

\medskip

{\bf (b)(i)} 
Now let $\Cal G$ be a non-principal ultrafilter on $\Bbb N$ which
has a filter base $B$ with cardinal $\frak u$.   Because
$\frak u<\frak g$, there is an $a\in\bigcap_{b\in B}A_b$.
This time, set $\phi(i)=\#(a\cap(i+1))$ for $i\in\Bbb N$;  
as $a$ is infinite, $\phi$ is finite-to-one.   Set $m=\min a$.
If $b\in B$, $c\subseteq\Bbb N$ and $(a(b)c)$, then 
$\phi[c\setminus m]\subseteq\phi[b]$.
\Prf\ If $k\in c\setminus m$, set $i=\max(a\cap(k+1))$;  then
$i\le k$ so there is a $j\in b$ such that $i\le j\le k$, and now
$a\cap(j+1)=a\cap(k+1)=a\cap(i+1)$ so $\phi(k)=\phi(j)\in\phi[b]$.\ \Qed\

\medskip

\quad{\bf (ii)}
It follows that $\phi[[\Cal G]]\subseteq\phi[[\Cal F]]$.   \Prf\ If
$G\in\phi[[\Cal G]]$ there is a $b\in B$ such that
$b\subseteq\phi^{-1}[G]$.   Now $a\in A_b$ so there is a $c\in\Cal F$ such
that $(a(b)c)$;  in this case, setting $m=\min a$, 
$c\setminus m\in\Cal F$ and $\phi[c\setminus m]\in\phi[[\Cal F]]$.
By (i) just above, $\phi[c\setminus m]\subseteq\phi[b]$, while
$\phi[b]\subseteq G$.   So $G\in\phi[[\Cal F]]$.   As $G$ is arbitrary,
$\phi[[\Cal G]]\subseteq\phi[[\Cal F]]$.\ \Qed

\medskip

\quad{\bf (iii)} We supposed that
$\Cal G$ was an ultrafilter, so $\phi[[\Cal G]]$ is an ultrafilter (2A1N)
and must be equal to $\phi[[\Cal F]]\leRB\Cal F$.   Thus
we have the second alternative in the statement of FD.   As $\Cal F$ is
arbitrary, FD is true.
}%end of proof of 5A6J

