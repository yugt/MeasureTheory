\frfilename{mt386.tex}
\versiondate{17.11.03}
\copyrightdate{2003}
     
\def\barln{\mathop{\bar{\text{ln}}}}
     
\def\chaptername{Automorphisms}
\def\sectionname{More about entropy}
     
\newsection{386}
     
In preparation for the next two sections, I present a number of basic
facts concerning measure-preserving homomorphisms and entropy.
Compared with the work to follow, they are mostly fairly elementary, but
the Halmos-Rokhlin-Kakutani lemma (386C) and the
Shannon-McMillan-Breiman
theorem (386E), in their full strengths, go farther than one might
expect.
     
\leader{386A}{}\cmmnt{ %3{8}5D
I start by returning to the notion of `recurrence' from
381L-381P, %381L 381M 381N 381P
in its original home.
     
\medskip
     
\noindent}{\bf Theorem} Let $(\frak A,\bar\mu)$ be a totally finite
measure algebra and $\pi:\frak A\to\frak A$ a measure-preserving
Boolean homomorphism.   Then $\pi$ is recurrent on every $a\in\frak A$.
     
\proof{ If $a\in\frak A$ is non-zero, then
$\sum_{k=0}^{\infty}\bar\mu(\pi^ka)=\infty>\mu 1$, so there are $i<j$ such that $0\ne\pi^ia\Bcap\pi^ja=\pi^i(a\Bcap\pi^{j-i}a)$ and
$a\Bcap\pi^{j-i}a\ne 0$.   Thus (ii) of 381O is satisfied;  by 381O,
$\pi$ is recurrent on every $a\in\frak A$.
}%end of proof of 386A
     
\leader{386B}{Corollary} %3{8}5D
Let $(\frak A,\bar\mu)$ be a totally finite measure algebra and
$\pi:\frak A\to\frak A$ a measure-preserving Boolean homomorphism.   Let
$\frak C$ be its fixed-point subalgebra $\{c:c\in\frak A$, $\pi c=c\}$.
Then
     
\Centerline{$\sup_{k\ge n}\pi^ka=\upr(a,\frak C)
\cmmnt{\mskip5mu=\inf\{c:a\Bsubseteq c\in\frak C\}\in\frak C}$}
     
\noindent for any $a\in\frak A$ and $n\in\Bbb N$.
     
\proof{ By 386A and 381O, $a\Bsubseteq\sup_{k\ge 1}\pi^ka$.   Set
$a^*=\sup_{k\in\Bbb N}\pi^ka$;  by 381Kb, $a^*=\sup_{k\ge n}\pi^ka$ for
every $n$;  by 381Ka, $a^*\in\frak C$.   Also, of course,
$a^*\Bsubseteq c$ whenever $a\Bsubseteq c\in\frak C$, so
$a^*=\upr(a,\frak C)$.
}%end of proof of 386B
     
\leader{386C}{The Halmos-Rokhlin-Kakutani lemma} Let %3{8}5E
$(\frak A,\bar\mu)$
be a totally finite measure algebra and $\pi:\frak A\to\frak A$ a
measure-preserving Boolean homomorphism, with fixed-point subalgebra
$\frak C$.   Then the following are equiveridical:
     
(i) $\pi$ is aperiodic;
     
(ii) $\frak A$ is relatively atomless over
$\frak C$\cmmnt{ (definition: 331A)};
     
(iii) whenever $n\ge 1$ and $0\le\gamma<\bover1n$
there is an $a\in\frak A$ such
that $a$, $\pi a$, $\pi^2a,\ldots,\pi^{n-1}a$ are disjoint and
$\bar\mu(a\Bcap c)=\gamma\bar\mu c$ for every $c\in\frak C$;
     
(iv) whenever $n\ge 1$, $0\le\gamma<\bover1n$ and $B\subseteq\frak A$ is
finite, there is an $a\in\frak A$ such
that $a$, $\pi a$, $\pi^2a,\ldots,\pi^{n-1}a$ are disjoint and
$\bar\mu(a\Bcap b)=\gamma\bar\mu b$ for every $b\in B$.
     
\proof{ Note that $\frak C$ is (order-\nobreak)closed because $\pi$
is (order-\nobreak)continuous (324Kb).
     
\medskip
     
{\bf (i)$\Rightarrow$(ii)} Put 386A and 381P together.
     
\medskip
     
{\bf (ii)$\Rightarrow$(iii)} Set $\delta=\bover1n(\bover1n-\gamma)>0$.
By 331B, there is a $d\in\frak A$ such that
$\bar\mu(c\Bcap d)=\delta\bar\mu c$ for
every $c\in\frak C$.   Set $d_k=\pi^kd\Bsetminus\sup_{i<k}\pi^id$ for
$k\in\Bbb N$.   Note that
     
\Centerline{$d_{j+k}=\pi^{j+k}d\Bsetminus\sup_{i<j+k}\pi^id
\Bsubseteq\pi^{j+k}d\Bsetminus\sup_{i<k}\pi^{j+i}d
=\pi^jd_k$}
     
\noindent whenever $j$, $k\in\Bbb N$.   Next,
$\pi^id_j\Bcap d_k\Bsubseteq\sup_{m\le i}d_m$ for any $i$, $j$,
$k\in\Bbb N$ such that
$i+j\ne k$.   \Prf\ ($\alpha$)
If $k\le i$ this is obvious.   ($\beta$) If $i<k<i+j$ then
     
\Centerline{$\pi^id_j\Bcap d_k
\Bsubseteq\pi^id_j\Bcap\pi^id_{k-i}=\pi^i(d_j\Bcap d_{k-i})=0$.}
     
\noindent ($\gamma$) If $i+j<k$, then
     
\Centerline{$\pi^id_j\Bcap d_k\Bsubseteq\pi^{i+j}d\Bcap d_k=0$. \Qed}
     
Setting $c^*=\sup_{i\in\Bbb N}d_i=\sup_{i\in\Bbb N}\pi^id$, we have
$c^*\in\frak C$, by 386B, so that
$\bar\mu(d\Bsetminus c^*)=\delta\bar\mu(1\Bsetminus c^*)$;  but as
$d\Bsubseteq c^*$, $c^*=1$.
     
Set $a^*=\sup_{m\in\Bbb N}d_{mn}$ (the $mn$ here is a product, not a
double subscript!), $d^*=\sup_{i<n}d_i=\sup_{i<n}\pi^id$.   Then
     
\Centerline{$\bar\mu(c\Bcap d^*)
\le\sum_{i=0}^{n-1}\bar\mu(c\Bcap\pi^id)
=\sum_{i=0}^{n-1}\bar\mu\pi^i(c\Bcap d)
=n\bar\mu(c\Bcap d)
=n\delta\bar\mu c$}
     
\noindent for every $c\in\frak C$.
Next, $\pi^id_{mn}\Bsupseteq d_{mn+i}$ for all $m$ and $i$, so
     
\Centerline{$\sup_{i<n}\pi^ia^*=\sup_{i\in\Bbb N}d_i=1$.}
     
\noindent Consequently
     
\Centerline{$\bar\mu c
\le\sum_{i=0}^{n-1}\bar\mu(c\Bcap\pi^ia^*)
=n\bar\mu(c\Bcap a^*)$,}
     
\Centerline{$\bar\mu(c\Bcap a^*\Bsetminus d^*)
\ge\bar\mu(c\Bcap a^*)-\bar\mu(c\Bcap d^*)
\ge(\bover1n-n\delta)\bar\mu c=\gamma\bar\mu c$}
     
\noindent for every $c\in\frak C$.
     
By 331B again (applied to the principal ideal of $\frak A$ generated by
$a^*\Bsetminus d^*)$ there is an $a\Bsubseteq a^*\Bsetminus d^*$ such
that $\bar\mu(a\Bcap c)=\gamma\bar\mu c$ for every $c\in\frak C$.   For
$0<i<n$,
     
\Centerline{$\pi^ia^*\Bcap a^*
=\sup_{k,l\in\Bbb N}\pi^id_{kn}\Bcap d_{ln}\Bsubseteq\sup_{m\le i}d_m
\Bsubseteq d^*$,}
     
\noindent so $\pi^ia\Bcap a=0$;  accordingly $a$,
$\pi a,\ldots,\pi^{n-1}a$ are all disjoint and (iii) is satisfied.
     
\medskip
     
{\bf (iii)$\Rightarrow$(iv)} Note that $\frak A$ is certainly atomless,
since for every $k\ge 1$ we can find a $c\in\frak A$ such that
$c,\pi c,\ldots,\pi^{k-1}c$ are disjoint and
$\bar\mu c=\bover{\bar\mu 1}{k+1}$,
so that we have a partition of unity consisting of sets of measure
$\bover{\bar\mu 1}{k+1}$.   Let
$B'$ be the set of atoms of the (finite) subalgebra of
$\frak A$ generated by $B$, and $m=\#(B')$.   Let
$\delta>0$ and $r$, $k\in\Bbb N$  be such that
     
\Centerline{$3\delta\le(1-n\gamma)\bar\mu b$ for every $b\in B'$,
\quad$m(\bar\mu 1)^2<r\delta^2$,
\quad$k\delta\ge\bar\mu 1$.}
     
\noindent By (iii), there is a $c\in\frak A$ such that
$c,\pi c,\ldots,\pi^{nr(k+1)-1}c$ are disjoint and
$\bar\mu(\sup_{i<nr(k+1)}\pi^ic)
\ifdim\pagewidth=390pt\penalty-100\fi
=1-\delta$.   For $j<r$, set
$e_j=\sup_{l\le k,i<n}\pi^{n(k+1)j+nl+i}c$,
$d_j=\sup_{l<k}\pi^{n(k+1)j+nl}c$.   Observe that
$d_j$, $\pi d_j,\ldots,\pi^{n-1}d_j$ are disjoint, and that $\pi^id_j\Bsubseteq e_j$
for $i<2n$.   Set $e=\sup_{j<r}e_j=\sup_{i<nr(k+1)}\pi^ic$, so that
$\bar\mu e=1-\delta$.
     
Suppose we choose $d\in\frak A$ by the following random process.   Take
$s(0),\ldots,s(r-1)$ independently in $\{0,\ldots,n-1\}$, so that
$\Pr(s(j)=l)=\bover1n$ for each $l<n$, and set
$d=\sup_{j<r}\pi^{s(j)}d_j$.   Because we certainly have
$\pi^i\pi^{s(j)}d_j\Bsubseteq e_j$ whenever $i<n$,
$d$, $\pi d,\ldots,\pi^{n-1}d$ will be disjoint.   Now for any $b\in\frak A$,
     
\Centerline{$\Pr\bigl(\bar\mu(d\Bcap b)
  \le\Bover1n(\bar\mu b-3\delta)\bigr)
<\Bover1m$.}
     
\noindent\Prf\ We can express the random variable $\bar\mu(d\Bcap b)$ as
$X=\sum_{j=0}^{r-1}X_j$, where $X_j=\bar\mu(\pi^{s(j)}d_j\Bcap b)$.
Then the $X_j$ are independent random variables.   For each $j$, $X_j$
takes values between $0$ and
$\bar\mu d_j=k\bar\mu c\le\Bover{\bar\mu 1}{nr}$, and
has expectation $\bover1n\bar\mu(e'_j\Bcap b)$, where
     
\Centerline{$e'_j=\sup_{i<n}\pi^id_j
=\sup_{l<k,i<n}\pi^{n(k+1)j+nl+i}c$.}
     
\noindent So $X$ has expectation $\bover1n\bar\mu(e'\Bcap b)$ where
$e'=\sup_{j<r}e'_j$.   Now
     
\Centerline{$e_j\Bsetminus e'_j
=\sup_{i<n}\pi^{n(k+1)j+nk+i}c$}
     
\noindent has measure $n\bar\mu c\le\Bover{n\bar\mu 1}{nr(k+1)}$ for
each $j$, so $\bar\mu(e\Bsetminus e')\le\Bover{\bar\mu 1}{k+1}$ and
$\bar\mu(1\Bsetminus e')\le 2\delta$;  thus
$\Expn(X)\ge\bover1n(\bar\mu b-2\delta)$, while
     
\Centerline{$\Var(X)=\sum_{j=0}^{r-1}\Var(X_j)
\le r\bigl(\Bover{\bar\mu 1}{nr}\bigr)^2
=\Bover{(\bar\mu 1)^2}{n^2r}$.}
     
\noindent But this means that
     
\Centerline{$\Bover{(\bar\mu 1)^2}{n^2r}
\ge\bigl(\Bover{\delta}n\bigr)^2
  \Pr\bigl(X\le\Bover1n(\bar\mu b-3\delta)\bigr)$,}
     
\noindent and
     
\Centerline{$\Pr\bigl(X\le\Bover1n(\bar\mu b-3\delta)\bigr)
\le\Bover{(\bar\mu 1)^2}{r\delta^2}
<\Bover1m$}
     
\noindent by the choice of $r$.\ \Qed
     
This is true for every $b\in B'$, while $\#(B')=m$.   There must
therefore be some choice of $s(0),\ldots,s(r-1)$ such that,
taking $d^*=\sup_{j<r}\pi^{s(j)}d_j$,
     
\Centerline{$\bar\mu(d^*\Bcap b)\ge\Bover1n(\bar\mu b-3\delta)
\ge\gamma\bar\mu b$}
     
\noindent for every $b\in B'$, while $d^*,\pi
d^*,\ldots,\pi^{n-1}d^*$ are
disjoint.   Because $\frak A$ is atomless, there is a $d\Bsubseteq d^*$
such that $\bar\mu(d\Bcap b)=\gamma\bar\mu b$ for every $b\in B'$.
Since every member of $B$ is a disjoint union of members of $B'$,
$\bar\mu(d\Bcap b)=\gamma\bar\mu b$ for every $b\in B$.
     
\medskip
     
{\bf(iv)$\Rightarrow$(i)} If $a\in\frak A\setminus\{0,1\}$ and $n\ge 1$
then (iv) tells us that there is a $b\in\frak A$ such that $b$,
$\pi b,\ldots,\pi^nb$ are all disjoint and
$\bar\mu(1\Bsetminus\sup_{i\le n}\pi^ib)<\bar\mu a$.   Now there must
be some $i<n$ such that $d=\pi^ib\Bcap a\ne 0$, in which case
     
\Centerline{$d\Bcap\pi^nd\Bsubseteq\pi^ib\Bcap\pi^{i+n}b
=\pi^i(b\Bcap\pi^nb)=0$,}
     
\noindent and $\pi^nd\ne d$.   As $n$ and $a$ are arbitrary, $\pi$ is
aperiodic.
}%end of proof of 386C
     
\leader{386D}{Corollary} %3{8}5F
An ergodic measure-preserving Boolean
homomorphism on an atomless totally finite measure algebra is aperiodic.
     
\proof{ By 372P, this is (ii)$\Rightarrow$(i) of 386C in the case
$\frak C=\{0,1\}$ (compare 381P).
}%end of proof of 386D
     
\leader{386E}{}\cmmnt{ I %3{8}5G
turn now to a celebrated result which is a
kind of strong law of large numbers.
     
\medskip
     
\noindent}{\bf The Shannon-McMillan-Breiman theorem} Let
$(\frak A,\bar\mu)$ be a probability algebra, $\pi:\frak A\to\frak A$ a
measure-preserving Boolean homomorphism and $A\subseteq\frak A$ a
partition of unity of finite entropy.   For each $n\ge 1$, set
     
\Centerline{$w_n
=\Bover1n\sum_{d\in D_n(A,\pi)}\ln(\Bover1{\bar\mu d})\chi d$,}
     
\noindent where $D_n(A,\pi)$ is the partition of unity generated by
$\{\pi^ia:a\in A,\,i<n\}$\cmmnt{, as in 385K}.   Then
$\sequencen{w_n}$ is norm-convergent in 
$\cmmnt{L^1=}L^1(\frak A,\bar\mu)$ to
$w$ say;  moreover, $\sequencen{w_n}$ is order*-convergent to
$w$\cmmnt{ (definition: 367A)}.   If $T:L^0(\frak A)\to L^0(\frak A)$
is the Riesz homomorphism defined by $\pi$\cmmnt{, so that
$T(\chi a)=\chi(\pi a)$ for every $a\in\frak A$ (364P)}, then $Tw=w$.
     
\proof{ ({\smc Petersen 83}) We may suppose that $0\notin A$.
     
\medskip
     
{\bf (a)} For each $n\in\Bbb N$, let $\frak B_n$ be the subalgebra of
$\frak A$ generated by
$\{\pi^ia:a\in A,\,1\le i\le n\}$, $B_n$ the set of its atoms, and $P_n$
the corresponding conditional
expectation operator on $L^1$ (365R).   Let $\frak B$ be
the closed subalgebra of $\frak A$ generated by
$\bigcup_{n\in\Bbb N}\frak B_n$, and $P$ the corresponding conditional
expectation operator.   Observe that
$B_n=\pi[D_n(A,\pi)]$ and that, in the
language of 385F, $D_{n+1}(A,\pi)=A\vee B_n$.   Let $\frak C$ be the
fixed-point subalgebra of $\pi$ and $Q$ the associated
conditional expectation.
Set $L^0=L^0(\frak A)$, and let $\barln$ be the function from
$\{v:\Bvalue{v>0}=1\}$ to $L^0$ corresponding to
$\ln:\ooint{0,\infty}\to\Bbb R$ (364H).
     
\medskip
     
{\bf (b)} It will save a moment later if I note a simple fact here:
if $v\in L^1$, then $\langle\bover1nT^nv\rangle_{n\ge 1}$ is
order*-convergent and $\|\,\|_1$-convergent to $0$.   \Prf\ We know
from the ergodic theorem (372G) that $\sequencen{\tilde v_n}$ is
order*-convergent and $\|\,\,\|_1$-convergent
to $Qv$, where $\tilde v_n=\bover1{n+1}\sum_{i=0}^nT^iv$.   Now
$\bover1nT^nv=\bover{n+1}n\tilde v_n-\tilde v_{n-1}$ is
order*-convergent and $\|\,\|_1$-convergent to
$Qv-Qv=0$ (using 367C for `order*-convergent').\ \Qed
     
\medskip
     
{\bf (c)} Set
     
\Centerline{$v_n=\sum_{a\in A}P_n(\chi a)\times\chi a
=\sum_{a\in A,b\in B_n}\Bover{\bar\mu(a\Bcap b)}{\bar\mu b}
  \chi(a\Bcap
b)$.}
     
\noindent By L\'evy's martingale theorem (275I, 367Jb),
     
\Centerline{$\sequencen{v_n\times\chi a}
=\sequencen{P_n(\chi a)\times\chi a}$}
     
\noindent is order*-convergent to $P(\chi a)\times\chi a$
for every $a\in A$;  consequently $\sequencen{v_n}$
order*-converges to $v=\sum_{a\in A}P(\chi a)\times\chi a$.   It
follows that $\sequencen{\barln v_n}$ order*-converges to $\barln v$.
\Prf\ The point is that, for any $a\in A$ and $n\in\Bbb N$,
$a\Bsubseteq\Bvalue{P_n(\chi a)>0}$, so that $\Bvalue{v_n>0}=1$ for
every $n$, and $\barln v_n$ is defined.   Similarly, $\barln v$ is
defined, and $\sequencen{\barln v_n}$ order*-converges to $\barln v$
by 367H.\ \QeD\  As $0\le v_n\le\chi 1$ for every $n$,
$\sequencen{v_n}\to v$ for $\|\,\|_1$, by the Dominated Convergence
Theorem (367I).
     
Next, $\sequencen{\barln v_n}$ is order-bounded in $L^1$.   \Prf\ Of
course $\barln v_n\le 0$ for every $n$, because
$P_n(\chi a)\le P_n(\chi 1)\le\chi 1$ for each $a$, so $v_n\le\chi 1$.
To see that
$\{\barln v_n:n\in\Bbb N\}$ is bounded below in $L^1$, we use an idea
from the fundamental martingale inequality 275D.   Set
$v_*=\inf_{n\in\Bbb N}v_n$.   For $\alpha>0$, $a\in A$ and $n\in\Bbb N$
set
     
\Centerline{$b_{an}(\alpha)
=\Bvalue{P_n(\chi a)<\alpha}
\Bcap\inf_{i<n}\Bvalue{P_i(\chi a)\ge\alpha}$,}
     
\noindent so that
     
\Centerline{$\Bvalue{v_*<\alpha}=\sup_{a\in A,n\in\Bbb N}a\Bcap
b_{an}(\alpha)$.}
     
\noindent Now $b_{an}(\alpha)\in\frak B_n$, so
     
\Centerline{$\bar\mu(a\Bcap b_{an}(\alpha))
=\int_{b_{an}(\alpha)}\chi a
=\int_{b_{an}(\alpha)}P_n(\chi a)
\le\alpha\bar\mu(b_{an}(\alpha))$,}
     
\noindent and
     
$$\eqalign{\bar\mu(a\Bcap\Bvalue{v_*<\alpha})
&\le\min(\bar\mu a,\sum_{n=0}^{\infty}\bar\mu(a\Bcap b_{an}(\alpha)))\cr
&\le\min(\bar\mu a,\alpha\sum_{n=0}^{\infty}\bar\mu b_{an}(\alpha))
\le\min(\bar\mu a,\alpha).\cr}$$
     
\noindent Letting $\alpha\downarrow 0$,
$\bar\mu(a\Bcap\Bvalue{v_*=0})=0$ for every $a\in A$, so
$\Bvalue{v_*>0}=1$, and $\barln v_*$ is defined.   Moreover,
     
\Centerline{$\bar\mu(a\Bcap\Bvalue{-\barln v_*>-\ln\alpha})
=\bar\mu(a\Bcap\Bvalue{v_*<\alpha})\le\min(\bar\mu a,\alpha)$}
     
\noindent for every $a\in A$, $\alpha>0$;  that is,
     
\Centerline{$\bar\mu(a\Bcap\Bvalue{-\barln v_*>\beta})
\le\min(\bar\mu a,e^{-\beta})$}
     
\noindent for every $a\in A$ and $\beta\in\Bbb R$.   Accordingly
     
$$\eqalign{\int(-\barln v_*)
&=\int_0^{\infty}\bar\mu\Bvalue{-\barln v_*>\beta}d\beta
=\sum_{a\in A}\int_0^{\infty}
  \bar\mu(a\Bcap\Bvalue{-\barln v_*>\beta})d\beta\cr
&\le\sum_{a\in A}\int_0^{\infty}\min(\bar\mu a,e^{-\beta})d\beta\cr
&=\sum_{a\in A}\bigl(\int_0^{\ln(1/\bar\mu a)}\bar\mu a\,d\beta
  +\int_{\ln(1/\bar\mu a)}^{\infty}e^{-\beta}d\beta\bigr)\cr
&=\sum_{a\in A}\bigl(\ln(\Bover1{\bar\mu a})\bar\mu a
  +e^{\ln\bar\mu a}\bigr)\cr
&=\sum_{a\in A}\ln(\Bover1{\bar\mu a})\bar\mu a
  +\sum_{a\in A}\bar\mu a
=H(A)+1<\infty\cr}$$
     
\noindent because $A$ has finite entropy.   But this means that
$\barln v_*$ belongs to $L^1$, and of course it is a lower bound for
$\{\barln v_n:n\in\Bbb N\}$.\ \QeD\
     
By 367I again, $\barln v\in L^1$ and $\sequencen{\barln v_n}\to\barln v$
for $\|\,\|_1$.
     
\wheader{386E}{4}{2}{2}{60pt}
     
{\bf (d)} Fix $n\in\Bbb N$ for the moment.   For each $d\in
D_{n+1}(A,\pi)$ let $d'$ be the unique element of $B_n$ such that
$d\Bsubseteq d'$.   Then
     
$$\eqalign{(n+1)w_{n+1}
&=\sum_{d\in D_{n+1}(A,\pi)}\ln(\Bover1{\bar\mu d'})\chi d
  -\sum_{d\in D_{n+1}(A,\pi)}
     \ln(\Bover{\bar\mu d}{\bar\mu d'})\chi d\cr
&=\sum_{b\in B_n}\ln(\Bover1{\bar\mu b})\chi b
  -\sum_{\Atop{\Atop{a\in A}{b\in B_n}}{a\Bcap b\ne 0}}
     \ln(\Bover{\bar\mu(a\Bcap b)}{\bar\mu b})\chi(a\Bcap b)\cr
&=\sum_{d\in D_n(A,\pi)}\ln(\Bover1{\bar\mu(\pi d)})\chi(\pi d)
  -\barln v_n
=T(nw_n)-\barln v_n.\cr}$$
     
\noindent Inducing on $n$, starting from
     
\Centerline{$w_1=\sum_{a\in A}\ln(\Bover1{\bar\mu a})\chi a
=-\barln v_0$,}
     
\noindent we get
     
\Centerline{$nw_n=\sum_{i=0}^{n-1}T^i(-\barln v_{n-i-1})$,
\quad$w_n=\Bover1n\sum_{i=0}^{n-1}T^i(-\barln v_{n-i-1})$}
     
\noindent for every $n\ge 1$.
     
\medskip
     
{\bf (e)} Set $w'_n=\bover1n\sum_{i=0}^{n-1}T^i(-\barln v)$ for
$n\ge 1$.   By the Ergodic Theorem,
$\langle w'_n\rangle_{n\ge 1}$ is order*-convergent and
$\|\,\|_1$-convergent to $w=Q(-\barln v)$, and $Tw=w$.   To estimate
$w_n-w'_n$, set $u_n^*=\sup_{k\ge n}|\barln v_k-\barln v|$ for each
$n\in\Bbb N$.   Then $\sequencen{u^*_n}$ is a non-increasing sequence,
$u_0^*\in L^1$ (by (c) above), and $\inf_{n\in\Bbb N}u_n^*=0$ because
$\sequencen{\barln v_n}$ order*-converges to $\barln v$.   Now,
whenever $n>m\in\Bbb N$,
     
$$\eqalign{|w_n-w'_n|
&\le\Bover1n\sum_{i=0}^{n-1}T^i|\barln v-\barln v_{n-i-1}|\cr
&=\Bover1n\bigl(\sum_{i=0}^{n-m-1}T^i|\barln v-\barln v_{n-i-1}|
  +\sum_{i=n-m}^{n-1}T^i|\barln v-\barln v_{n-i-1}|\bigr)\cr
&\le\Bover1n\bigl(\sum_{i=0}^{n-m-1}T^iu_m^*
  +\sum_{j=0}^{m-1}T^{n-1-j}|\barln v-\barln v_j|\bigr)\cr
&\le\Bover1{n-m}\bigl(\sum_{i=0}^{n-m-1}T^iu_m^*
  +\sum_{j=0}^{m-1}T^{n-1-j}u_0^*\bigr)\cr
&=\Bover1{n-m}\sum_{i=0}^{n-m-1}T^iu_m^*
  +\Bover1{n-m}T^{n-m}\sum_{j=0}^{m-1}T^{m-1-j}u_0^*\cr
&=\Bover1{n-m}\sum_{i=0}^{n-m-1}T^iu_m^*
  +\Bover1{n-m}T^{n-m}\tilde u_m,\cr}$$
     
\noindent setting $\tilde u_m=\sum_{j=0}^{m-1}T^{m-1-j}u_0^*$.
     
Holding $m$ fixed and letting $n\to\infty$, we know that
     
\Centerline{$\Bover1{n-m}\sum_{i=0}^{n-m-1}T^iu_m^*$}
     
\noindent is order*-convergent and $\|\,\,\|_1$-convergent to
$Qu_m^*$.   As for the other term,
$\bover1{n-m}T^{n-m}\tilde u_m$ is order*-convergent and
$\|\,\|_1$-convergent to $0$, by
(b).   What this means is that
     
\Centerline{$\limsup_{n\to\infty}|w_n-w'_n|
\le Qu_m^*$,}
     
\Centerline{$\limsup_{n\to\infty}\|w_n-w'_n\|_1
\le\|Qu_m^*\|_1$}
     
\noindent for every $m\in\Bbb N$.   Since $\sequence{m}{Qu_m^*}$ is
surely a non-decreasing sequence with infimum $0$,
     
\Centerline{$\limsup_{n\to\infty}|w_n-w'_n|=0$,
\quad$\limsup_{n\to\infty}\|w_n-w'_n\|_1=0$.}
     
\noindent Since $w'_n$ is order*-convergent and
$\|\,\,\|_1$-convergent to $w$, so is $w_n$.
}%end of proof of 386E
     
\leader{386F}{Corollary} %3{8}5H
If, in 386E, $\pi$ is ergodic, then
$\sequencen{w_n}$ is order*-convergent and
$\|\,\,\|_1$-convergent to $h(\pi,A)\chi 1$.
     
\proof{ Because the limit $w$ in 386E has $Tw=w$, it must be of the form
$\gamma\chi 1$, because $\pi$ is ergodic (372Q(a-ii)).   
Now $\gamma=\int w$ must be
     
$$\eqalignno{\lim_{n\to\infty}\int w_n
&=\lim_{n\to\infty}\Bover1n\sum_{d\in
  D_n(A,\pi)}\ln(\Bover1{\bar\mu d})\bar\mu d
=\lim_{n\to\infty}\Bover1n\sum_{d\in
  D_n(A,\pi)}q(\bar\mu d)\cr
\displaycause{where $q$ is the function of 385A}
&=\lim_{n\to\infty}\Bover1nH(D_n(A,\pi))
=h(\pi,A).\cr}$$
}%end of proof of 386F
     
\leader{386G}{Definition} %3{8}5I
Set $p(t)=t\ln t$ for $t>0$, $p(0)=0$;  for
any Dedekind $\sigma$-complete Boolean algebra $\frak A$, let
$\bar p:L^0(\frak A)^+\to L^0(\frak A)$ be the corresponding
function\cmmnt{, as in 364H}.   \cmmnt{(Thus $p=-q$ where $q$ is the
function of 385A.)}
     
\leader{386H}{Lemma}\cmmnt{ ({\smc Csisz\'ar 67}, %3{8}5J
{\smc Kullback 67})} Let $(\frak A,\bar\mu)$ be a probability algebra,
and $u$ a member of $L^1(\frak A,\bar\mu)^+$ such that $\int u=1$.  Then
     
\Centerline{$(\int|u-\chi 1|)^2\le 2\int\bar p(u)$.}
     
\proof{ Set $a=\Bvalue{u<1}$, $\alpha=\bar\mu a$, $\beta=\int_au$,
$b=1\Bsetminus a$.   Then $\bar\mu b=1-\alpha$ and $\int_bu=1-\beta$.
Surely $\beta\le\alpha<1$.   If $\alpha=0$ then $u=\chi 1$ and the
result is trivial;  so let us suppose that $0<\alpha<1$.   Because the
function $p$ is convex,
     
\Centerline{$\int_a\bar p(u)
\ge\bar\mu a\cdot p(\Bover1{\bar\mu a}\int_au)
=\alpha p(\Bover{\beta}{\alpha})=p(\beta)-\beta\ln\alpha$,}
     
\noindent (using 233Ib/365Rb for the inequality), and similarly
     
\Centerline{$\int_b\bar p(u)
\ge p(1-\beta)-(1-\beta)\ln(1-\alpha)$.}
     
\noindent Also
     
\Centerline{$\int|u-\chi 1|
=\int_a(\chi 1-u)+\int_b(u-\chi 1)
=\alpha-\beta+(1-\beta)-(1-\alpha)
=2(\alpha-\beta)$,}
     
\noindent so
     
$$\eqalign{\int\bar p(u)-\Bover12(\int|u-\chi 1|)^2
&\ge p(\beta)-\beta\ln\alpha
+p(1-\beta)-(1-\beta)\ln(1-\alpha)-2(\alpha-\beta)^2\cr
&=\phi(\beta)\cr}$$
     
\noindent say.   Now $\phi$
is continuous on $[0,1]$ and arbitrarily often differentiable on
$\ooint{0,1}$,
     
\Centerline{$\phi(\alpha)=0$,}
     
\Centerline{$\phi'(t)=\ln t-\ln\alpha-\ln(1-t)+\ln(1-\alpha)
+4(\alpha-t)$ for $t\in\ooint{0,1}$,}
     
\Centerline{$\phi'(\alpha)=0$,}
     
\Centerline{$\phi''(t)=\Bover1t+\Bover1{1-t}-4\ge 0$ for
$t\in\ooint{0,1}$.}
     
\noindent So $\phi(t)\ge 0$ for $t\in[0,1]$ and, in particular,
$\phi(\beta)\ge 0$;  but this means that
     
\Centerline{$\int\bar p(u)-\Bover12(\int|u-\chi 1|)^2
\ge 0$,}
     
\noindent that is, $(\int|u-\chi 1|)^2\le 2\int\bar p(u)$, as claimed.
}%end of proof of 386H
     
\leader{386I}{Corollary} %3{8}5K
Whenever $(\frak A,\bar\mu)$ is a probability
algebra and $A$, $B$ are partitions of unity of finite entropy,
     
$$\sum_{a\in A,b\in B}|\bar\mu(a\Bcap b)-\bar\mu a\cdot\bar\mu b|
\le\sqrt{2(H(A)+H(B)-H(A\vee B))}.$$
     
\proof{ Replacing $A$, $B$ by $A\setminus\{0\}$ and $B\setminus\{0\}$ if
necessary, we may suppose that neither $A$ nor $B$ contains $\{0\}$.
Let $(\frak C,\bar\lambda)$ be the probability algebra free product of
$(\frak A,\bar\mu)$ with itself (325E, 325K).   Set
     
\Centerline{$u=\sum_{a\in A,b\in B}
\Bover{\bar\mu(a\Bcap b)}{\bar\mu a\cdot\bar\mu b}\chi(a\otimes b)
\in L^0(\frak C)$;}
     
\noindent then $u$ is non-negative and integrable and
$\int u=\sum_{a\in A,b\in B}\bar\mu(a\Bcap b)=1$.   Now
     
$$\eqalign{\int\bar p(u)
&=\sum_{a\in A,b\in B}
  \bar\mu(a\Bcap b)
  \ln\Bover{\bar\mu(a\Bcap b)}{\bar\mu a\cdot\bar\mu b}\cr
&=-H(A\vee B)
  -\sum_{a\in A,b\in B}\bar\mu(a\Bcap b)\ln\bar\mu a
  -\sum_{a\in A,b\in B}\bar\mu(a\Bcap b)\ln\bar\mu b\cr
&=-H(A\vee B)
  -\sum_{a\in A}\bar\mu a\ln\bar\mu a
  -\sum_{b\in B}\bar\mu b\ln\bar\mu b\cr
&=H(A)+H(B)
  -H(A\vee B).\cr}$$
     
\noindent On the other hand,
     
\Centerline{$\int|u-\chi 1|
=\sum_{a\in A,b\in B}\bar\mu a\cdot\bar\mu b
   |\Bover{\bar\mu(a\Bcap b)}{\bar\mu a\cdot\bar\mu b}-1|
=\sum_{a\in A,b\in B}|\bar\mu(a\Bcap b)-\bar\mu a\cdot\bar\mu b|$.}
     
\noindent So what we are seeking to prove is that
     
\Centerline{$\int|u-\chi 1|
\le\sqrt{2\intop\bar p(u)}$,}
     
\noindent which is 386H.
}%end of proof of 386I
     
\leader{386J}{}\cmmnt{ The %3{8}5L
next six lemmas are notes on more or
less elementary facts which will be used at various points in the next
section.   The first two are nearly trivial.
     
\medskip
     
\noindent}{\bf Lemma} Let $(\frak A,\bar\mu)$ be a probability algebra
and $\familyiI{a_i}$, $\familyiI{b_i}$ two partitions of unity in
$\frak A$.   Then
     
\Centerline{$\bar\mu(\sup_{i\in I}a_i\Bcap b_i)
=1-\Bover12\sum_{i\in I}\bar\mu(a_i\Bsymmdiff b_i)$.}
     
\proof{
     
$$\eqalign{\bar\mu(\sup_{i\in I}a_i\Bcap b_i)
=\sum_{i\in I}\bar\mu(a_i\Bcap b_i)
&=\sum_{i\in I}\Bover12(\bar\mu a_i+\bar\mu b_i
   -\bar\mu(a_i\Bsymmdiff b_i))\cr
&=1-\Bover12\sum_{i\in I}\bar\mu(a_i\Bsymmdiff b_i).\cr}$$
     
}%end of proof of 386J
     
\leader{386K}{Lemma} %3{8}5M
Let $(\frak A,\bar\mu)$ be a totally finite measure
algebra, $\sequence{k}{B_k}$ a non-decreasing sequence of subsets of
$\frak A$ such that $0\in B_0$, and $\familyiI{c_i}$ a partition of
unity in $\frak A$ such that $c_i\in\overline{\bigcup_{k\in\Bbb N}B_k}$
for every $i\in I$.   Then
     
\Centerline{$\lim_{k\to\infty}\sup_{i\in I}\rho(c_i,B_k)=0$\dvro{.}{,}}
     
\cmmnt{\noindent writing
$\rho(c,B)=\inf_{b\in B}\bar\mu(c\Bsymmdiff b)$ for $c\in\frak A$ and
non-empty $B\subseteq\frak A$, as in 3A4I.}
     
\proof{ Let $\epsilon>0$.   Then
$J=\{j:j\in I,\,\bar\mu c_j\ge\epsilon\}$ is finite.   For each
$j\in J$, $\lim_{k\to\infty}\rho(c_i,B_k)=0$, by 3A4I, while
     
\Centerline{$\rho(c_i,B_k)\le\bar\mu(c_i\Bsymmdiff 0)
=\bar\mu c_i\le\epsilon$}
     
\noindent for every $i\in I\setminus J$.   So
     
\Centerline{$\limsup_{k\to\infty}\sup_{i\in I}\rho(c_i,B_k)
\le\max(\epsilon,\limsup_{k\to\infty}\sup_{i\in J}\rho(c_i,B_k))
=\epsilon$.}
     
\noindent As $\epsilon$ is arbitrary, we have the result.
}%end of proof of 386K
     
\leader{386L}{Lemma} %3{8}5N
Let $(\frak A,\bar\mu)$ be a probability algebra,
and $\pi:\frak A\to\frak A$ a measure-preserving Boolean homomorphism.
Let $A$, $B$ and $C$ be partitions of unity in $\frak A$.
     
(a) $H(A\vee B\vee C)+H(C)\le H(B\vee C)+H(A\vee C)$.
     
(b) $h(\pi,A)\le h(\pi,A\vee B)\le h(\pi,A)+h(\pi,B)\le h(\pi,A)+H(B)$.
     
(c) If $H(A)<\infty$,
     
$$\eqalign{h(\pi, A)
&=\inf_{n\in\Bbb N}H(D_{n+1}(A,\pi))-H(D_n(A,\pi))\cr
&=\lim_{n\to\infty}H(D_{n+1}(A,\pi))-H(D_n(A,\pi)).\cr}$$
     
(d) If $H(A)<\infty$ and $\frak B$ is any closed subalgebra of $\frak A$
such that $\pi[\frak B]\subseteq\frak B$, then
$h(\pi,A)\le h(\pi\restrp\frak B)+H(A|\frak B)$.
     
\proof{{\bf (a)} Let $\frak C$ be the closed subalgebra of $\frak A$
generated by $C$, so that $\frak C$ is purely atomic and $C$ is the set
of its atoms.   Then
     
$$\eqalign{H(A\vee B\vee C)+H(C)
&=H(A\vee B|\frak C)+2H(C)\cr
&\le H(A|\frak C)+H(B|\frak C)+2H(C)
=H(A\vee C)+H(B\vee C)\cr}$$
     
\noindent by 385Gb and 385Ga.
     
\medskip
     
{\bf (b)} We need only observe that
$D_n(A\vee B,\pi)=D_n(A,\pi)\vee D_n(B,\pi)$ for every $n\in\Bbb N$,
being the partition of unity
generated by $\{\pi^ia:i<n,\,a\in A\}\cup\{\pi^ib:i<n,\,b\in B\}$.
Consequently
     
$$\eqalign{h(\pi,A)
&=\lim_{n\to\infty}\Bover1nH(D_n(A,\pi))
\le\lim_{n\to\infty}\Bover1nH(D_n(A,\pi)\vee D_n(B,\pi))\cr
&=\lim_{n\to\infty}\Bover1nH(D_n(A\vee B,\pi))
=h(\pi,A\vee B)\cr
&\le\lim_{n\to\infty}\Bover1n(H(D_n(A,\pi)+H(D_n(B,\pi)))
=h(\pi,A)+h(\pi,B)\cr
&\le h(\pi,A)+H(B)\cr}$$
     
\noindent as remarked in 385M.
     
\medskip
     
{\bf (c)} Set $\gamma_n=H(D_{n+1}(A,\pi))-H(D_n(A,\pi))$ for each
$n\in\Bbb N$.   By 385H, $\gamma_n\ge 0$.   From (a) we see that
     
$$\eqalign{\gamma_{n+1}
&=H(A\vee\pi[D_{n+1}(A,\pi)])-H(A\vee\pi[D_{n}(A,\pi)])\cr
&\le H(\pi[D_{n+1}(A,\pi)])-H(\pi[D_n(A,\pi])
=\gamma_n\cr}$$
     
\noindent for every $n\in\Bbb N$.   So
$\lim_{n\to\infty}\gamma_n=\inf_{n\in\Bbb N}\gamma_n$;  write $\gamma$
for the common value.   Now
     
\Centerline{$h(\pi,A)=\lim_{n\to\infty}\Bover1nH(D_n(A,\pi))
=\lim_{n\to\infty}\Bover1n\sum_{i=0}^{n-1}\gamma_i=\gamma$}
     
\noindent (273Ca).
     
\medskip
     
{\bf (d)} Let $P:L^1_{\bar\mu}\to L^1_{\bar\mu}$ be the conditional
expectation operator corresponding to $\frak B$.   Let
$\sequence{k}{b_k}$ be a sequence running over
$\{\Bvalue{P(\chi a)>q}:a\in A,\,q\in\Bbb Q\}$, so that $b_k\in\frak B$
for every $k$,
and for each $k\in\Bbb N$ let $\frak B_k\subseteq\frak B$ be the
subalgebra generated by $\{b_i:i\le k\}$;  let $P_k$ be the conditional
expectation operator corresponding to $\frak B_k$.   Writing
$\frak B_{\infty}\subseteq\frak B$ for
$\overline{\bigcup_{k\in\Bbb N}\frak B_k}$, and $P_{\infty}$ for the
corresponding conditional expectation
operator, then $P(\chi a)\in L^0(\frak B_{\infty})$, so
$P_{\infty}(\chi a)=P(\chi a)$, for every $a\in A$.   So
     
\Centerline{$H(A|\frak B)
=\sum_{a\in A}\int\bar q(P\chi a)
=H(A|\frak B_{\infty})=\lim_{k\to\infty}H(A|\frak B_k)$,}
     
\noindent by 385Gd.
     
For each $k$, let $B_k$ be the set of atoms of $\frak B_k$.   Then
     
\Centerline{$h(\pi,A)\le h(\pi,B_k)+H(A|\frak B_k)
\le h(\pi\restrp\frak B)+H(A|\frak B_k)$}
     
\noindent by 385N and the definition of $h(\pi\restrp\frak B)$.   So
     
\Centerline{$h(\pi,A)
\le h(\pi\restrp\frak B)+\lim_{k\to\infty}H(A|\frak B_k)
=h(\pi\restrp\frak B)+H(A|\frak B)$.}
}%end of proof of 386L
     
\vleader{72pt}{386M}{Lemma} %3{8}5O
Let $(\frak A,\bar\mu)$ be a probability algebra
and $\frak B$ a closed subalgebra.
     
(a) There is a function $h:\frak A\to\frak B$ such that
$\bar\mu(a\Bsymmdiff h(a))=\rho(a,\frak B)$ for every $a\in\frak A$ and
$h(a)\Bcap h(a')=0$ whenever $a\Bcap a'=0$.
     
(b) If $A$ is a partition of unity in $\frak A$, then
$H(A|\frak B)\le\sum_{a\in A}q(\rho(a,\frak B))$\cmmnt{, where $q$ is
the function of 385A}.
     
(c) If $\frak B$ is atomless and $\sequence{i}{a_i}$ is a partition of
unity in $\frak A$,  then there is a partition of unity
$\sequence{i}{b_i}$ in $\frak B$ such that $\bar\mu b_i=\bar\mu a_i$ and
$\bar\mu(b_i\Bsymmdiff a_i)\le 2\rho(a_i,\frak B)$ for every
$i\in\Bbb N$.
     
\proof{{\bf (a)} Let $P:L^1_{\bar\mu}\to L^1_{\bar\mu}$ be the
conditional expectation operator associated with $\frak B$.   For any
$b\in\frak B$,
     
$$\eqalign{\int|P(\chi a)-\chi b|
&=\int_{1\Bsetminus b}P(\chi a)+\bar\mu b-\int_bP(\chi a)
=\int_{1\Bsetminus b}\chi a+\bar\mu b-\int_b\chi a\cr
&=\bar\mu(a\Bsetminus b)+\bar\mu b-\bar\mu(a\Bcap b)
=\bar\mu(a\Bsymmdiff b).\cr}$$
     
If $a\in\frak A$ set $h(a)=\Bvalue{P(\chi a)>\bover12}$.   Then
$|P(\chi a)-\chi h(a)|\le|P(\chi a)-\chi b|$ for any $b\in\frak B$, so
     
$$\eqalign{\rho(a,\frak B)
&=\inf_{b\in\frak B}\bar\mu(a\Bsymmdiff b)
=\inf_{b\in\frak B}\int|P(\chi a)-\chi b|\cr
&=\int|P(\chi a)-\chi h(a)|
=\bar\mu(a\Bsymmdiff h(a)).\cr}$$
     
If $a\Bcap a'=0$, then
     
\Centerline{$P(\chi a)+P(\chi a')=P\chi(a\Bcup a')\le\chi 1$,}
     
\noindent so
     
\Centerline{$h(a)\Bcap h(a')
=\Bvalue{P(\chi a)>\bover12}\Bcap\Bvalue{P(\chi a')>\bover12}
\Bsubseteq\Bvalue{P(\chi a)+P(\chi a')>1}=0$,}
     
\noindent by 364Ea.
     
\medskip
     
{\bf (b)} By 385Ae, $q(t)\le q(1-t)$
whenever $\bover12\le t\le 1$.  Consequently
$q(t)\le q(\min(t,1-t))$ for every $t\in[0,1]$, and
$\bar q(u)\le\bar q(u\wedge(\chi 1-u))$ whenever $u\in L^0(\frak A)$ and
$0\le u\le\chi 1$.   Fix $a\in A$ for the moment.   We have
     
\Centerline{$\bar q(P(\chi a))
\le\bar q(P(\chi a)\wedge(\chi 1-P(\chi a))
=\bar q(|P(\chi a)-\chi h(a)|)$.}
     
\noindent Consequently
     
$$\eqalignno{\int\bar q(P\chi a)
&\le\int\bar q(|P(\chi a)-\chi h(a)|)
\le q\bigl(\int|P(\chi a)-\chi h(a)|\bigr)\cr
\noalign{\noindent (because $q$ is concave)}
&=q(\rho(a,\frak B)).\cr}$$
     
\noindent Summing over $a$,
     
\Centerline{$H(A|\frak B)
=\sum_{a\in A}\int\bar q(P\chi a)
\le\sum_{a\in A}q(\rho(a,\frak B))$.}
     
\medskip
     
{\bf (c)} Set $b'_i=h(a_i)$ for each $i\in\Bbb N$.
Then $\sequence{i}{b'_i}$ is disjoint.   Next,
for each $i\in\Bbb N$, take $b''_i\in\frak B$ such that
$b''_i\Bsubseteq b'_i$ and
$\bar\mu b''_i=\min(\bar\mu a_i,\bar\mu b'_i)$;  then
$\sequence{i}{b''_i}$ is disjoint and $\bar\mu b''_i\le\bar\mu a_i$ for
every $i$.   We can therefore find a partition of unity
$\sequence{i}{b_i}$ such that $b_i\Bsupseteq b''_i$ and
$\bar\mu b_i=\bar\mu a_i$ for every $i$.   (Use 331C to choose
$\sequence{i}{d_i}$ inductively so that
$d_i\Bsubseteq 1\Bsetminus(\sup_{j<i}d_j\Bcup\sup_{j\in\Bbb N}b''_j)$
and $\bar\mu d_i=\bar\mu a_i-\bar\mu b''_i$ for each $i$, and set
$b_i=b''_i\Bcup d_i$.)
     
Take any $i\in\Bbb N$.   If $\bar\mu b'_i>\bar\mu a_i$, then
     
$$\eqalign{\bar\mu(a_i\Bsymmdiff b_i)
&=\bar\mu(a_i\Bsymmdiff b''_i)
\le\bar\mu(a_i\Bsymmdiff b'_i)+\bar\mu(b'_i\Bsymmdiff b''_i)\cr
&=\bar\mu(a_i\Bsymmdiff b'_i)+\bar\mu b'_i-\bar\mu a_i
\le 2\bar\mu(a_i\Bsymmdiff b'_i)
=2\rho(a_i,\frak B).\cr}$$
     
\noindent If $\bar\mu b'_i\le\bar\mu a_i$, then
     
$$\eqalign{\bar\mu(a_i\Bsymmdiff b_i)
&\le\bar\mu(a_i\Bsymmdiff b'_i)+\bar\mu(b'_i\Bsymmdiff b_i)\cr
&=\bar\mu(a_i\Bsymmdiff b'_i)+\bar\mu a_i-\bar\mu b'_i
\le 2\bar\mu(a_i\Bsymmdiff b'_i)
=2\rho(a_i,\frak B).\cr}$$
}%end of proof of 386M
     
\leader{386N}{Lemma} %3{8}5P
Let $(\frak A,\bar\mu)$ be a probability algebra
and $\pi:\frak A\to\frak A$ a measure-preserving automorphism.   Suppose
that $B\subseteq\frak A$.   For $k\in\Bbb N$, let $\frak B_k$ be the
closed subalgebra of $\frak A$ generated by
$\{\pi^jb:b\in B,\,|j|\le k\}$, and
let $\frak B$ be the closed subalgebra of $\frak A$ generated by
$\{\pi^jb:b\in B,\,j\in\Bbb Z\}$.   
     
(a) $\frak B$ is the topological closure
$\overline{\bigcup_{k\in\Bbb N}\frak B_k}$.
     
(b) $\pi[\frak B]=\frak B$.
     
(c) If $\frak C$ is any closed subalgebra of $\frak A$ such that
$\pi[\frak C]=\frak C$, and $a\in\frak B_k$, then
     
\Centerline{$\rho(a,\frak C)\le(2k+1)\sum_{b\in B}\rho(b,\frak C)$.}
     
\proof{{\bf (a)} Because $\sequence{k}{\frak B_k}$ is non-decreasing,
$\bigcup_{k\in\Bbb N}\frak B_k$ is a subalgebra of $\frak A$, so its
closure also is (323J), and must be $\frak B$.
     
\medskip
     
{\bf (b)} Of course $\pi^{-1}[\frak B_{k+1}]$ is a closed subalgebra of
$\frak A$ containing $\pi^jb$ whenever $|j|\le k$ and $b\in B$, so
includes $\frak B_k$;  thus
$\pi[\frak B_k]\subseteq\frak B_{k+1}\subseteq\frak B$ for every $k$,
and
     
\Centerline{$\pi[\frak B]
=\pi[\overline{\bigcupop_{k\in\Bbb N}\frak B_k}]
\subseteq\overline{\bigcupop_{k\in\Bbb N}\pi[\frak B_k]}
\subseteq\overline{\frak B}
\subseteq\frak B$}
     
\noindent because $\pi$ is continuous (324Kb).
Similarly, $\pi^{-1}[\frak B]\subseteq\frak B$ and
$\pi[\frak B]=\frak B$.
     
\medskip
     
{\bf (c)} For each $b\in B$, choose $c_b\in\frak C$ such that
$\bar\mu(b\Bsymmdiff c_b)=\rho(c_b,\frak C)$ (386Ma).   Set
     
\Centerline{$e=\sup_{|j|\le k}\sup_{b\in B}\pi^j(b\Bsymmdiff c_b)$;}
     
\noindent then
     
\Centerline{$\bar\mu e\le(2k+1)\sum_{b\in B}\bar\mu(b\Bsymmdiff c_b)
=(2k+1)\sum_{b\in B}\rho(b,\frak C)$.}
     
\noindent Now
     
\Centerline{$\frak B'=\{d:d\in\frak A,\Exists c\in\frak C$ such that
$d\Bsetminus e=c\Bsetminus e\}$}
     
\noindent is a subalgebra of $\frak A$.   By 314F(a-i), applied to the
order-continuous homomorphism
$c\mapsto c\Bsetminus e:\frak C\to\frak A_{1\Bsetminus e}$,
$\{c\Bsetminus e:c\in\frak C\}$ is an order-closed subalgebra of the
principal ideal $\frak A_{1\Bsetminus e}$;  by 313Id, applied to the
order-continuous function
$d\mapsto d\Bsetminus e:\frak A\to\frak A_{1\Bsetminus e}$, $\frak B'$
is order-closed.   If $b\in B$ and $|j|\le k$,
then $\pi^jb\Bsymmdiff\pi^jc_b\Bsubseteq e$, so $\pi^jb\in\frak B'$;
accordingly $\frak B'\supseteq\frak B_k$.   Now $a\in\frak B_k$, so
there is a $c\in\frak C$ such that $a\Bsymmdiff c\Bsubseteq e$, and
     
\Centerline{$\rho(a,\frak C)\le\bar\mu(a\Bsymmdiff c)\le\bar\mu e
\le(2k+1)\sum_{b\in B}\rho(b,\frak C)$,}
     
\noindent as claimed.
}%end of proof of 386N
     
\leader{386O}{Lemma} %3{8}5Q
Let $(\frak A,\bar\mu)$ be a probability algebra
and suppose {\it either} that $\frak A$ is not purely atomic {\it or}
that it is purely atomic and $H(D_0)=\infty$, where $D_0$ is the set of
atoms of
$\frak A$.   Then whenever $A\subseteq\frak A$ is a partition of unity
and $H(A)\le\gamma\le\infty$, there is a partition of unity $B$,
refining $A$, such that $H(B)=\gamma$.
     
\proof{{\bf (a)} By 385J, there is a partition of unity $D_1$ such
that $H(D_1)=\infty$.    Set $D=D_1\vee A$;  then we still have
$H(D)=\infty$.
Enumerate $D$ as $\sequence{i}{d_i}$.   Choose $\sequence{k}{B_k}$
inductively, as follows.  $B_0=A$.   Given that $B_k$ is a partition of
unity, then if $H(B_k\vee\{d_k,1\Bsetminus d_k\})\le\gamma$, set
$B_{k+1}=B_k\vee\{d_k,1\Bsetminus d_k\}$;  otherwise set $B_{k+1}=B_k$.
     
Let $\frak B$ be the closed subalgebra of $\frak A$ generated by
$\bigcup_{k\in\Bbb N}B_k$.   Note that, for each $d\in D$,
     
\Centerline{$\{c:c\in\frak A,\,d\Bsubseteq c$ or $d\Bcap c=0\}$}
     
\noindent is a closed subalgebra of $\frak A$ including every $B_k$, so
includes $\frak B$.   If $b\in\frak B\setminus\{0\}$, there is surely
some $d\in D$ such that $b\Bcap d\ne 0$, so $b\Bsupseteq d$;  thus
$\frak B$ must be purely atomic.   Let $B$ be the set of atoms of
$\frak B$.   Because $A=B_0\subseteq\frak B$, $B$ refines $A$.
     
\medskip
     
{\bf (b)} $H(B)\le\gamma$.   \Prf\ For each $k\in\Bbb N$, let
$\frak B_k$ be the closed subalgebra of $\frak A$ generated by $B_k$, so
that $\frak B=\overline{\bigcup_{k\in\Bbb N}\frak B_k}$.   Suppose that
$b_0,\ldots,b_n$ are distinct members of $B$.   Then for each
$k\in\Bbb N$ we can find disjoint $b_{0k},\ldots,b_{nk}\in\frak B_k$
such that $\bar\mu(b_{ik}\Bsymmdiff b_i)\le\rho(b_i,\frak B_k)$ for
every $i\le n$ (386Ma).   Accordingly
$\bar\mu b_i=\lim_{k\to\infty}\bar\mu b_{ik}$ for each $i$, and
     
\Centerline{$\sum_{i=0}^nq(\bar\mu b_i)
=\lim_{k\to\infty}\sum_{i=0}^nq(\bar\mu b_{ik})
\le\sup_{k\in\Bbb N}H(B_k)\le\gamma$.}
     
\noindent As $b_0,\ldots,b_n$ are arbitrary, $H(B)\le\gamma$.\ \Qed
     
\medskip
     
{\bf (c)} $H(B)\ge\gamma$.   \Prf\Quer\ Suppose otherwise.   We know
that
     
\Centerline{$\lim_{k\to\infty}H(\{d_k,1\Bsetminus d_k\})
=\lim_{k\to\infty}q(\bar\mu d_k)+q(1-\bar\mu d_k)=0$.}
     
\noindent Let $m\in\Bbb N$ be such that $H(B)+H(\{d_k,1\Bsetminus
d_k\})\le\gamma$ for every $k\ge m$.   Because $B$ refines $B_k$, we
must
have
     
\Centerline{$H(B_k\vee\{d_k,1\Bsetminus d_k\})
\le H(B_k)+H(\{d_k,1\Bsetminus d_k\})\le\gamma$,}
     
\noindent so that $B_{k+1}=B_k\vee\{d_k,1\Bsetminus d_k\}$ for every
$k\ge m$.   But this means that $d_k\in B$ for every $k\ge m$, so that
     
\Centerline{$\gamma>H(B)\ge\sum_{k=m}^{\infty}q(\bar\mu d_k)=\infty$,}
     
\noindent which is impossible.\ \Bang\Qed
     
Thus $B$ has the required properties.
}%end of proof of 386O
     
\exercises{
\leader{386X}{Basic exercises (a)}
%\spheader 386Xa
Let $(\frak A,\bar\mu)$ be a totally finite
measure algebra and $\pi:\frak A\to\frak A$ a measure-preserving Boolean
homomorphism with fixed-point subalgebra $\frak C$.
Take any $a\in\frak A$ and
set $a_n=\pi^na\Bsetminus\penalty-100\sup_{1\le i<n}\pi^ia$ for $n\ge 1$.
Show that $\sum_{n=1}^{\infty}n\bar\mu(a\Bcap a_n)=\bar\mu(\upr(a,\frak C))$.
\Hint{for $0\le j<k$ set
$a_{jk}=\pi^j(a\Bcap a_{k-j})$.   Show that if $r\in\Bbb N$, then $\langle a_{jk}\rangle_{j\le r<k}$ is disjoint.}
     
\spheader 386Xb
Let $(X,\Sigma,\mu)$ be a totally finite measure space
and $f:X\to X$ an inverse-measure-preserving function.   Take
$E\in\Sigma$ and set $F=\{x:\,\exists\,n\ge 1,\,f^n(x)\in E\}$.   (i)
Show that $E\setminus F$ is negligible.   (ii) For $x\in E\cap F$ set
$k_x=\min\{n:n\ge 1,\,f^n(x)\in E\}$.   Show that
$\int_Ek_x\mu(dx)=\mu F$.   (This is a simple form of the {\bf
Recurrence Theorem}.)
%386Xa
     
\spheader 386Xc %3{8}5Xd
Let $(\frak A,\bar\mu)$ be a totally finite measure
algebra, $\sequence{k}{B_k}$ a non-decreasing sequence of subsets of
$\frak A$ such that $0\in B_0$, and $\familyiI{c_i}$ a partition of
unity in $\frak A$.   Show that
     
\Centerline{$\lim_{k\to\infty}\sum_{i\in I}\rho(c_i,B_k)
=\sum_{i\in I}\rho(c_i,B)$}
     
\noindent where $B=\overline{\bigcup_{k\in\Bbb N}B_k}$.
%386K
     
\sqheader 386Xd %3{8}5Xe
Let $(\frak A,\bar\mu)$ be a probability algebra,
$\pi:\frak A\to\frak A$ a measure-preserving Boolean homomorphism and
$A$ a partition of unity in $\frak A$.
Show that $h(\pi,D_n(A,\pi))=h(\pi,A)=h(\pi,\pi[A])$ for any $n\ge 1$.
%386L
     
\spheader 386Xe %3{8}5Xf
Let $(\frak A,\bar\mu)$ be a totally finite measure
algebra and
$\pi:\frak A\to\frak A$ a measure-preserving Boolean homomorphism.
Suppose that
$B\subseteq\frak A$.   For $k\in\Bbb N$, let $\frak B_k$ be the closed
subalgebra of $\frak A$ generated by $\{\pi^jb:b\in B,\,j\le k\}$, and
let $\frak B$ be the closed subalgebra of $\frak A$ generated by
$\{\pi^jb:b\in B,\,j\in\Bbb N\}$.   Show that
     
\Centerline{$\frak B=\overline{\bigcupop_{k\in\Bbb N}\frak B_k}$,
\quad$\pi[\frak B]\subseteq\frak B$,}
     
\noindent and that if $\frak C$ is any subalgebra of $\frak A$
such that $\pi[\frak C]\Bsubseteq\frak C$, and $a\in\frak B_k$, then
$\rho(a,\frak C)\le(k+1)\sum_{b\in B}\rho(b,\frak C)$.
%386N

\spheader 386Xf Prove 332L without using Maharam's theorem.
%386A query out of order
     
\leader{386Y}{Further exercises (a)}
%\spheader 386Ya %3{8}5Yb
Let $(\frak A,\bar\mu)$ be a totally finite measure algebra and
$\pi:\frak A\to\frak A$ an aperiodic
measure-preserving Boolean homomorphism.   Set $\frak C=\{c:\pi c=c\}$.
Show that whenever $n\ge 1$, $0\le\gamma<\bover1n$ and
$B\subseteq\frak A$ is finite, there is an $a\in\frak A$ such
that $a$, $\pi a$, $\pi^2a,\ldots,\pi^{n-1}a$ are disjoint and
$\bar\mu(a\Bcap b\Bcap c)=\gamma\bar\mu(b\Bcap c)$ for every $b\in B$,
$c\in\frak C$.
%386C
     
\spheader 386Yb %3{8}5Yc
Let $(\frak A,\bar\mu)$ be a probability algebra, and
$\pi:\frak A\to\frak A$ a measure-preserving Boolean homomorphism.   Let
$\frak P$ be the set of all closed subalgebras of $\frak A$ which are
invariant under $\pi$, ordered by inclusion.   Show that
$\frak B\mapsto h(\pi\restrp\frak{B}):\frak P\to[0,\infty]$ is
order-preserving and
{\bf order-continuous on the left}, in the sense that if 
$\frak Q\subseteq\frak P$ is non-empty and upwards-directed then
$h(\pi\restr\sup\frak Q)=\sup_{\frak B\in\frak Q}h(\pi\restrp\frak{B})$.
%386L
}%end of exercises
     
\endnotes{\Notesheader{386} I have taken the trouble to give sharp forms
of the Halmos-Rokhlin-Kakutani lemma (386C) and the Czisz\'ar-Kullback
inequality (386H);  while it is possible to get through the principal
results of the next two sections with rather less, the formulae become
better focused if we have the exact expressions available.   Of course
one can always go farther still (386Ya).   Ornstein's theorem in \S387
(though not Sina\v\i's, as stated there) can be deduced from the special
case of the Shannon-McMillan-Breiman theorem (386E) in which the
homomorphism $\pi$ is a Bernoulli shift.
}%end of notes
     
\discrpage
     
