\frfilename{mt4a6.tex}
\versiondate{8.12.10}
\copyrightdate{2000}

\def\chaptername{Appendix}
\def\sectionname{Banach algebras}
\def\Folland{{\smc Folland 95}}
\def\HR{{\smc Hewitt \& Ross 63}}
\def\Rudin{{\smc Rudin 91}}

\newsection{4A6}

I give results which are needed for Chapter 44.   Those down to 4A6K
should be in any introductory text on normed algebras;  
4A6L-4A6O, %4A6L 4A6M 4A6N 4A6O
as expressed here, are a little more specialized.
As with normed spaces or linear topological spaces, Banach
algebras may be defined over either $\Bbb R$ or $\Bbb C$.   In \S445 we
need complex Banach algebras, but in \S446 I think the ideas are clearer
in the context of real Banach algebras.   Accordingly, as in \S2A4, I
express as much as possible of the theory in terms applicable equally to
either, speaking of `normed algebras' or `Banach algebras' without
qualification, and using the symbol $\RoverC$ to represent the field of
scalars.   Since (at least, if you keep to the path indicated here) the
ideas are independent of which field we work with, you will have no
difficulty in applying the arguments given in \Folland\ or \HR\ for the
complex case to the real case.   In 4A6B and 4A6I-4A6K, %4A6I 4A6J 4A6K
however, we have results which apply only to `complex' Banach algebras,
in which the underlying field is taken to be $\Bbb C$.

\leader{4A6A}{Definition (a)}\cmmnt{ I repeat a definition from
\S2A4.}   A {\bf normed algebra} is a normed
space $U$ together with a multiplication, a binary operator $\times$ on
$U$, such that

\Centerline{$u\times(v\times w)=(u\times v)\times w$,}

\Centerline{$u\times(v+w)=(u\times v)+(u\times w)$,
\quad$(u+v)\times w=(u\times w)+(v\times w)$,}

\Centerline{$(\alpha u)\times v=u\times(\alpha v)=\alpha(u\times v)$,}

\Centerline{$\|u\times v\|\le\|u\|\|v\|$}

\noindent for all $u$, $v$, $w\in U$ and $\alpha\in\RoverC$.
A normed algebra is {\bf commutative} if its
multiplication is commutative.

\spheader 4A6Ab A {\bf Banach algebra} is a normed algebra which is a
Banach space.   A {\bf unital Banach algebra} is a Banach
algebra with a multiplicative identity $e$ such that $\|e\|=1$.
\cmmnt{({\bf Warning:}  some authors %Rudin 91
reserve the term `Banach algebra' for what I call a `unital Banach
algebra'.)}

In a unital Banach algebra I will always use the letter $e$ for the
identity.
%Folland 1.1

\leader{4A6B}{Stone-Weierstrass Theorem:  fourth form} Let $X$ be a
locally compact Hausdorff space, and $C_0=C_0(X;\Bbb C)$ the complex
Banach algebra of continuous functions $f:X\to\Bbb C$ such that
$\{x:|f(x)|\ge\epsilon\}$ is compact for every $\epsilon>0$.   Let
$A\subseteq C_0$ be such that

\inset{$A$ is a linear subspace of $C_0$,}

\inset{$f\times g\in A$ for every $f$, $g\in A$,}

\inset{the complex conjugate $\bar f$ of $f$ belongs to $A$ for every
$f\in A$,}

\inset{for every $x\in X$ there is an $f\in A$ such that $f(x)\ne 0$,}

\inset{whenever $x$, $y$ are distinct points of $X$ there is an $f\in A$
such that $f(x)\ne f(y)$.}

\noindent Then $A$ is $\|\,\,\|_{\infty}$-dense in $C_0$.
%4{}45K

\proof{ Let $X_{\infty}=X\cup\{\infty\}$ be the one-point
compactification of $X$
(3A3O).   For $f\in C_0$ write $f^{\#}$ for the extension of $f$ to
$X\cup\{\infty\}$ with $f^{\#}(\infty)=0$, so that
$f^{\#}\in C_b(X_{\infty};\Bbb C)$.   Let
$B\subseteq C_b(X_{\infty};\Bbb C)$ be the set of all
functions of the form $f^{\#}+\alpha\chi X_{\infty}$ where
$f\in A$ and $\alpha\in\Bbb C$.
Then $B$ is a subalgebra of $C_b(X\cup\{\infty\})$ which contains
complex conjugates of its members and constant functions and separates
the points of $X_{\infty}$.

Take any $h\in C_0$ and $\epsilon>0$.   By the `third form' of the
Stone-Weierstrass theorem (281G), there is a $g\in B$ such that
$\|g-h^{\#}\|_{\infty}\le\bover12\epsilon$.   Express $g$ as
$f^{\#}+\alpha\chi X_{\infty}$ where $f\in A$ and $\alpha\in\Bbb C$.
Then

\Centerline{$|\alpha|=|g(\infty)|
=|g(\infty)-h^{\#}(\infty)|\le\bover12\epsilon$,}

\noindent so

\Centerline{$\|h-f\|_{\infty}=\|h^{\#}-f^{\#}\|_{\infty}
\le\|h^{\#}-g\|_{\infty}+\|g-f^{\#}\|_{\infty}
\le\bover12\epsilon+|\alpha|
\le\epsilon$.}

\noindent As $h$ and $\epsilon$ are arbitrary, $A$ is dense in $C_0$.
}%end of proof of 4A6B

\leader{4A6C}{Proposition} If $U$ is any Banach space other than
$\{0\}$, then the
space $\eurm B(U;U)$ of bounded linear operators from $U$ to itself is a
unital Banach algebra.   \prooflet{({\smc K\"othe 69}, 14.6.)}

\leader{4A6D}{Proposition} Any normed algebra $U$ can be embedded as
a subalgebra of a unital Banach algebra $V$, in such a way that if $U$
is commutative so is $V$.   \prooflet{(\Folland, \S1.3;  \HR, C.3.)}
%4A6K  4A6F

\leader{4A6E}{Proposition} Let $U$ be a unital Banach algebra and
$W\subseteq U$ a closed proper ideal.   Then $U/W$, with the quotient
linear structure, ring structure and norm, is a unital Banach algebra.
\prooflet{(\HR, C.2.)}
%4A6J

\leader{4A6F}{Proposition} If $U$ is a Banach algebra and
$\phi:U\to\RoverC$ is a multiplicative linear functional, then
$|\phi(u)|\le\|u\|$ for every $u\in U$.

\proof{\Quer\ Otherwise, there is a $u$ such that $|\phi(u)|>\|u\|$;
set $v=\Bover1{\phi(u)}u$, so that $\phi(v)=1$ and $\|v\|<1$.   Since
$\|v^n\|\le\|v\|^n$ for every $n\ge 1$, 
$w=\sum_{n\in\Bbb N\setminus\{0\}}v^n$ is
defined in $U$ (4A4Ie), and $w=vw+v$ (because $u\mapsto vu$ is a continuous
linear operator, so we can use 4A4Bh to see that
$vw=\sum_{n\in\Bbb N\setminus\{0\}}v^{n+1}$).   But this means that
$\phi(w)=\phi(v)\phi(w)+\phi(v)=\phi(w)+1$, which is impossible.\ \Bang
}%end of proof of 4A6F

\leader{4A6G}{Definition} Let $U$ be a normed algebra and $u\in U$.

\spheader 4A6Ga For any $u\in U$, $\lim_{n\to\infty}\|u^n\|^{1/n}$
is defined and equal to $\inf_{n\ge 1}\|u^n\|^{1/n}$.
\prooflet{(\HR, C.4.)}

\spheader 4A6Gb This common value is\cmmnt{ called} the {\bf spectral
radius} of $u$.

\leader{4A6H}{Theorem} If $U$ is a unital Banach algebra,
then the set $R$ of invertible elements is open, and
$u\mapsto u^{-1}$ is a continuous function from $R$ to itself.   
If $v\in U$ and $\|v-e\|<1$, then $v\in R$ and
$\|v^{-1}-e\|\le\Bover{\|v-e\|}{1-\|v-e\|}$.
\prooflet{(\Folland, 1.4;  \HR, C.8 \& C.10;  \Rudin, 10.7 \& 10.12.)}

\vleader{48pt}{4A6I}{Theorem} Let $U$ be a complex unital Banach algebra and
$u\in U$.   Write $r$ for the spectral radius of $u$.

(a) If $\zeta\in\Bbb C$ and $|\zeta|>r$ then $\zeta e-u$ is invertible.

(b) There is a $\zeta$ such that $|\zeta|=r$ and $\zeta e-u$ is not
invertible.
%4A6K 4A6J

\proof{ \Folland, 1.8;  \HR, C.24;  \Rudin, 1.13.}
%mt4abits

\leader{4A6J}{Theorem} Let $U$ be a commutative complex unital Banach
algebra, and $u\in U$.   Then for any $\zeta\in\Bbb C$ the following are
equiveridical:

(i) there is a non-zero multiplicative linear functional
$\phi:U\to\Bbb C$ such that $\phi(u)=\zeta$;

(ii) $\zeta e-u$ is not invertible.
%4A6K

\proof{ \Folland, 1.13;  \HR, C.20;  \Rudin, 11.5.}

\leader{4A6K}{Corollary} Let $U$ be a commutative complex Banach
algebra and $u\in U$.   Then its spectral radius $r(u)$ is
$\max\{|\phi(u)|:\phi$ is a multiplicative linear functional on $U\}$.
\prooflet{(\Folland, 1.13;  \HR, C.20;  \Rudin, 11.9.)}

\leader{4A6L}{Exponentiation} Let $U$ be a unital Banach algebra.
For any $u\in U$,\cmmnt{ $\sum_{k=0}^{\infty}\|\Bover1{k!}u^k\|
\penalty-100\le\sum_{k=0}^{\infty}\Bover1{k!}\|u\|^k$ is finite, so}

\Centerline{$\exp(u)=\sum_{k\in\Bbb N}\Bover1{k!}u^k$}

\noindent is defined in $U$\cmmnt{ (4A4Ie)}.   \cmmnt{(In this
formula, interpret $u^0$ as $e$ for every $u$.)}
%4A6M, 4A6N

\leader{4A6M}{Lemma} Let $U$ be a unital Banach algebra.

(a) If $u$, $v\in U$ and $\max(\|u\|,\|v\|)\le\gamma$ then
$\|\exp(u)-\exp(v)-u+v\|\le\|u-v\|(\exp\gamma-1)$.   So if
$\max(\|u\|,\|v\|)\le\bover23$ and $\exp(u)=\exp(v)$ then $u=v$.
%4A6K

(b) If $\|u-e\|\le\bover16$ then there is a $v$ such that $\exp(v)=u$
and $\|v\|\le 2\|u-e\|$.
%4A6N

(c) If $u$, $v\in U$ and $uv=vu$ then $\exp(u+v)=\exp(u)\exp(v)$.
%4A6N

\proof{{\bf (a)} Note first that if $k\ge 1$ then

$$\eqalignno{\|u^k-v^k\|
&=\|\sum_{i=0}^{k-1}u^{k-i}v^i-u^{k-i-1}v^{i+1}\|
=\|\sum_{i=0}^{k-1}u^{k-i-1}(u-v)v^i\|\cr
&\le\sum_{i=0}^{k-1}\|u\|^{k-i-1}\|u-v\|\|v\|^i
\le\sum_{i=0}^{k-1}\gamma^{k-1}\|u-v\|
=k\gamma^{k-1}\|u-v\|.\cr}$$

\noindent So

$$\eqalign{\|\exp(u)-\exp(v)-u+v\|
&=\|\sum_{k\in\Bbb N\setminus\{0,1\}}\Bover1{k!}(u^k-v^k)\|
\le\sum_{k=2}^{\infty}\Bover1{k!}\|u^k-v^k\|\cr
&\le\sum_{k=2}^{\infty}\Bover{k}{k!}\gamma^{k-1}\|u-v\|
=\|u-v\|(\exp\gamma-1).\cr}$$

\noindent Now if $\exp(u)=\exp(v)$ and $\gamma\le\bover23$,
$0\le\exp\gamma-1<1$ so $\|u-v\|=0$ and $u=v$.

\medskip

{\bf (b)} Set $\gamma=\|u-e\|$.   Define $\sequencen{v_n}$ in $U$ by
setting $v_0=0$, $v_{n+1}=v_n+u-\exp(v_n)$ for $n\in\Bbb N$.   Then

\Centerline{$\|v_{n+1}-v_n\|=\|u-\exp(v_n)\|\le 2^{-n}\gamma$,
\quad$\|v_n\|\le 2(1-2^{-n})\gamma$}

\noindent for every $n\in\Bbb N$.   \Prf\ Induce on $n$.   The induction
starts with $\|v_0\|=0$ and $\|u-\exp(v_0)\|=\|u-e\|=\gamma$.   Given
that $\|v_n\|\le 2(1-2^{-n})\gamma$ and
$\|u-\exp(v_n)\|\le 2^{-n}\gamma$, then

\Centerline{$\|v_{n+1}\|
\le\|v_n\|+\|u-\exp(v_n)\|
\le 2(1-2^{-n})\gamma+2^{-n}\gamma
=2(1-2^{-n-1})\gamma$.}

\noindent Now $\max(\|v_{n+1}\|,\|v_n\|)\le 2\gamma\le\bover13$, so

$$\eqalign{\|u-\exp(v_{n+1})\|
&=\|v_{n+1}-v_n+\exp(v_n)-\exp(v_{n+1})\|
\le\|v_{n+1}-v_n\|(\exp\Bover13-1)\cr
&\le\Bover12\|v_{n+1}-v_n\|
=\Bover12\|u-\exp(v_n)\|
\le 2^{-n-1}\gamma,\cr}$$

\noindent and the induction continues.\ \Qed

Since $\sum_{n=0}^{\infty}\|v_{n+1}-v_n\|$ is finite,
$v=\lim_{n\to\infty}v_n$ is defined in $U$, and
$\|v\|=\lim_{n\to\infty}\|v_n\|\le 2\gamma$.   Accordingly

$$\eqalign{\|\exp(v)-\exp(v_n)\|
&\le\|v-v_n\|+\|\exp(v)-\exp(v_n)-v+v_n\|\cr
&\le\|v-v_n\|(1+\exp\Bover13-1)
\to 0\cr}$$

\noindent as $n\to\infty$, and $\exp(v)=\lim_{n\to\infty}\exp(v_n)=u$.

\medskip

{\bf (c)} Because $uv=vu$,
$(u+v)^m=\sum_{j=0}^m\Bover{m!}{j!(m-j)!}u^jv^{m-j}$ for every
$m\in\Bbb N$ (induce on $m$;  the point is that $uv^j=v^ju$ for every
$j\in\Bbb N$).   Next, $\sum_{j,k\in\Bbb N}\Bover1{j!k!}\|u\|^j\|v\|^k$
is finite.   So

$$\eqalignno{\exp(u+v)
&=\sum_{m\in\Bbb N}\Bover1{m!}(u+v)^m\cr
&=\sum_{m\in\Bbb N}\bigl(\sum_{j+k=m}
  \Bover1{j!k!}u^jv^k\bigr)
=\sum_{(j,k)\in\Bbb N\times\Bbb N}\Bover1{j!k!}u^jv^k\cr
\displaycause{using 4A4I(e-ii)}
&=\sum_{j\in\Bbb N}\bigl(\sum_{k\in\Bbb N}\Bover1{j!k!}u^jv^k)\cr
\displaycause{4A4I(e-ii) again}
&=\sum_{j\in\Bbb N}
  \bigl(\Bover1{j!}u^j\sum_{k\in\Bbb N}\Bover1{k!}v^k\bigr)\cr
\displaycause{by 4A4Bh, because $w\mapsto\Bover1{j!}u^jw$ is a
continuous linear operator for each $j$}
&=\sum_{j\in\Bbb N}\Bover1{j!}u^j\exp(v)
=\bigl(\sum_{j\in\Bbb N}\Bover1{j!}u^j\bigr)\exp(v)\cr
\displaycause{4A4Bh again}
&=\exp(u)\exp(v),\cr}$$

\noindent as claimed.
}%end of proof of 4A6M

\leader{4A6N}{Lemma} If $U$ is a unital Banach algebra, $u\in U$ and
$\|u^n-e\|\le\bover16$ for every $n\in\Bbb N$, then $u=e$.

\proof{ For every $n\in\Bbb N$ there is a $v_n\in U$ such
that $\exp(v_n)=u^{2^n}$ and $\|v_n\|\le\bover13$ (4A6Mb).   Then
$\exp(v_{n+1})=\exp(v_n)^2=\exp(2v_n)$ (4A6Mc),
$\|v_{n+1}\|\le\bover13$ and $\|2v_n\|\le\bover13$ so $v_{n+1}=2v_n$ for
every $n$ (4A6Ma).   Inducing on $n$, $v_n=2^nv_0$ for every $n$, so
that $\|v_0\|\le 2^{-n}\|v_n\|\to 0$ as $n\to\infty$, and $u=\exp(v_0)=e$.
}%end of proof of 4A6N

\leader{4A6O}{Proposition} Let $U$ be a normed algebra, and $U^*$, 
$U^{**}$ its dual and bidual as a normed space.   For a bounded linear
operator $T:U\to U$ let $T':U^*\to U^*$ be the adjoint of $T$ and
$T'':U^{**}\to U^{**}$ the adjoint of $T'$.

(a) We have bilinear maps, all of norm at most $1$,

$$\eqalign{(f,x)\mapsto f\frsmallcirc x&:U^*\times U\to U^*,\cr
(\phi,f)\mapsto \phi\frsmallcirc f&:U^{**}\times U^*\to U^*,\cr
(\phi,\psi)\mapsto \phi\frsmallcirc\psi&:U^{**}\times U^{**}\to U^{**}
\cr}$$

\noindent defined by the formulae

$$\eqalign{(f\frsmallcirc x)(y)&=f(xy),\cr
(\phi\frsmallcirc f)(x)&=\phi(f\frsmallcirc x),\cr
(\phi\frsmallcirc\psi)(f)&=\phi(\psi\frsmallcirc f)\cr}$$

\noindent for all $x$, $y\in U$, $f\in U^*$ and $\phi$, $\psi\in U^{**}$.

(b)(i) Suppose that $S:U\to U$ is a bounded linear operator such that
$S(xy)=(Sx)y$ for all $x$, $y\in U$.   Then
$S''(\phi\frsmallcirc\psi)=(S''\phi)\frsmallcirc\psi$ for all
$\phi$, $\psi\in U^{**}$.

\quad(ii) Suppose that $T:U\to U$ is a bounded linear operator such that
$T(xy)=x(Ty)$ for all $x$, $y\in U$.   Then 
$T''(\phi\frsmallcirc\psi)=\phi\frsmallcirc(T''\psi)$ for
all $\phi$, $\psi\in U^{**}$.

\proof{{\bf (a)} The calculations are elementary if we take them one at a
time.

\medskip

{\bf (b)(i)}\grheada
 $(S'f)\frsmallcirc x=f\frsmallcirc(Sx)$ for every $f\in U^*$
and $x\in U$.   \Prf\ 

\Centerline{$((S'f)\frsmallcirc x)(y)=(S'f)(xy)=f(S(xy))=f((Sx)y)
=(f\frsmallcirc(Sx))(y)$}

\noindent for every $y\in U$.\ \Qed

\medskip

\qquad\grheadb\ 
$\psi\frsmallcirc(S'f)=S'(\psi\frsmallcirc f)$ for every $f\in U^*$.   \Prf\

\Centerline{$(\psi\frsmallcirc(S'f))(x)
=\psi((S'f)\frsmallcirc x)=\psi(f\frsmallcirc(Sx))
=(\psi\frsmallcirc f)(Sx)=(S'(\psi\frsmallcirc f))(x)$}

\noindent for every $x\in U$.\ \Qed

\medskip

\qquad\grheadc\ So

$$\eqalign{(S''(\phi\frsmallcirc\psi))(f)
&=(\phi\frsmallcirc\psi)(S'f)
=\phi(\psi\frsmallcirc(S'f))\cr
&=\phi(S'(\psi\frsmallcirc f))
=(S''\phi)(\psi\frsmallcirc f)
=((S''\phi)\frsmallcirc\psi)(f)\cr}$$

\noindent for every $f\in U^*$, and 
$S''(\phi\frsmallcirc\psi)=(S''\phi)\frsmallcirc\psi$.

\medskip

\quad{\bf (ii)}\grheada\ $(T'f)\frsmallcirc x=T'(f\frsmallcirc x)$ for
every $f\in U^*$ and $x\in U$.   \Prf\

\Centerline{$((T'f)\frsmallcirc x)(y)
=(T'f)(xy)
=f(T(xy))
=f(x(Ty))
=(f\smallcirc x)(Ty)
=(T'(f\frsmallcirc x))(y)$}

\noindent for every $y\in U$.\ \Qed

\medskip

\qquad\grheadb\ $\psi\frsmallcirc(T'f)=(T''\psi)\frsmallcirc f$ for every 
$f\in U^*$.   \Prf\ 

$$\eqalign{(\psi\frsmallcirc(T'f))(x)
&=\psi((T'f)\frsmallcirc x)
=\psi(T'(f\frsmallcirc x))\cr
&=(T''\psi)(f\frsmallcirc x)
=((T''\psi)\frsmallcirc f)(x)\cr}$$

\noindent for every $x\in U$.\ \Qed

\medskip

\qquad\grheadc\ So

\Centerline{$(T''(\phi\frsmallcirc\psi))(f) 
=(\phi\frsmallcirc\psi)(T'f)
=\phi(\psi\frsmallcirc(T'f))
=\phi((T''\psi)\frsmallcirc f)
=(\phi\frsmallcirc(T''\psi))(f)$}

\noindent for every $f\in U'$, and
$T''(\phi\frsmallcirc\psi)=\phi\frsmallcirc(T''\psi)$.
}%end of proof of 4A6O

\cmmnt{\medskip

\noindent{\bf Remark} I must not abandon you at this point without telling
you that $\frsmallcirc:U^{**}\times U^{**}\to U^{**}$ is an 
{\bf Arens multiplication}, and that it is associative, so that
that $U^{**}$ is a Banach algebra.
}

%\exercises{\leader{4A6X}{Basic exercises (a)}

%\leader{4A6Y}{Further exercises (a)}
%}%end of exercises

%\endnotes{
%\Notesheader{4A6}

%}%end of notes

\discrpage

