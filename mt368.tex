\frfilename{mt368.tex}
\versiondate{16.9.09}
\copyrightdate{1997}

\def\chaptername{Function spaces}
\def\sectionname{Embedding Riesz spaces in $L^0$}

\newsection{368}

In this section I turn to the representation of Archimedean Riesz spaces
as function spaces.   Any Archimedean Riesz space $U$ can be represented
as an order-dense subspace of $L^0(\frak A)$, where $\frak A$ is its
band algebra (368E).   Consequently we get representations of
Archimedean Riesz
spaces as quotients of subspaces of $\Bbb R^X$ (368F) and as subspaces
of $C^{\infty}(X)$ (368G), and a notion of `Dedekind completion'
(368I-368J).   Closely associated with these is the fact that we have a
very general extension theorem for order-continuous Riesz homomorphisms
into $L^0$ spaces (368B).   I give a characterization of $L^0$
spaces in terms of lateral completeness (368M\cmmnt{, 368Yd}), and I
discuss \wsid\ Riesz spaces (368N-368S).


\leader{368A}{Lemma} Let $\frak A$ be a Dedekind $\sigma$-complete
Boolean algebra, and $A\subseteq(L^0)^+$ a set with no upper bound in
$L^0$, where
$L^0=L^0(\frak A)$.   If {\it either} $A$ is countable {\it or}
$\frak A$ is Dedekind complete, there is a $v>0$ in $L^0$ such that
$nv=\sup_{u\in A}u\wedge nv$ for every $n\in\Bbb N$.

\proof{ The hypothesis `$A$ is countable or $\frak A$ is Dedekind
complete' ensures that $c_{\alpha}=\sup_{u\in A}\Bvalue{u>\alpha}$ is
defined for each $\alpha$.   By 364L(a-i),
$c=\inf_{n\in\Bbb N}c_n=\inf_{\alpha\in\Bbb R}c_{\alpha}$ is non-zero.
Now for any $n\ge 1$, $\alpha\in\Bbb R$

\Centerline{$\Bvalue{\sup_{u\in A}(u\wedge n\chi c)>\alpha}
=\sup_{u\in A}\Bvalue{u>\alpha}\Bcap\Bvalue{\chi c>\bover{\alpha}n}
=\Bvalue{\chi c>\bover{\alpha}n},$}

\noindent because if $\alpha\ge 0$ then

\Centerline{$\sup_{u\in A}\Bvalue{u>\alpha}
=c_{\alpha}\Bsupseteq c\Bsupseteq\Bvalue{\chi c>\bover{\alpha}n}$,}

\noindent while if $\alpha<0$ then (because $A$ is a non-empty subset of
$(L^0)^+$)

\Centerline{$\sup_{u\in A}\Bvalue{u>\alpha}=1
=\Bvalue{\chi c>\bover{\alpha}n}$.}

\noindent So $\sup_{u\in A}u\wedge n\chi c=n\chi c$ for every $n\ge 1$,
and
we can take $v=\chi c$.   (The case $n=0$ is of course trivial.)
}%end of proof of 368A

\leader{368B}{Theorem} Let $\frak A$ be a Dedekind complete Boolean
algebra, $U$ an Archimedean Riesz space, $V$ an order-dense Riesz
subspace of $U$ and $T:V\to L^0=L^0(\frak A)$ an order-continuous Riesz
homomorphism.   Then $T$ has a unique extension to an
order-continuous Riesz homomorphism $\tilde T:U\to L^0$.
\proof{{\bf (a)}  The key to the proof is the following:  if $u\ge 0$ in
$U$, then $\{Tv:v\in V,\,0\le v\le u\}$ is bounded above in $L^0$.
\Prf\Quer\ Suppose, if possible, otherwise.   Then by 368A there is a
$w>0$ in $L^0$ such that $nw=\sup_{v\in A}nw\wedge Tv$ for every
$n\in\Bbb N$,
where $A=\{v:v\in V,\,0\le v\le u\}$.   In particular, there is a
$v_0\in A$ such that $w_0=w\wedge Tv_0>0$.   Because $U$ is Archimedean,
$\inf_{k\ge 1}\bover1ku=0$, so
$v_0=\sup_{k\ge 1}(v_0-\bover1ku)^+$.
Because $V$ is order-dense in $U$, $v_0=\sup B$ where

\Centerline{$B
=\{v:v\in V,\,0\le v\le (v_0-\Bover1ku)^+$ for some $k\ge 1\}$.}

\noindent Because $T$ is order-continuous, $Tv_0=\sup T[B]$ in $L^0$,
and there is a $v_1\in B$ such that $w_1=w_0\wedge Tv_1>0$.   Let
$k\ge 1$
be such that $v_1\le (v_0-\bover1ku)^+$.   Then for any $m\in\Bbb N$,

$$\eqalignno{mv_1\wedge u
&\le(mv_1\wedge kv_0)+(mv_1\wedge(u-kv_0)^+)\cr
\displaycause{352Fa}
&\le kv_0+(m+k)(v_1\wedge(\Bover1ku-v_0)^+)
=kv_0.\cr}$$

\noindent So for any $v\in A$, $m\in\Bbb N$,

\Centerline{$mw_1\wedge Tv=mw_1\wedge mTv_1\wedge Tv
\le T(mv_1\wedge v)\le T(mv_1\wedge u)\le T(kv_0)=kTv_0$.}

\noindent But this means that, for $m\in\Bbb N$,

\Centerline{$mw_1=mw_1\wedge mw=\sup_{v\in A}mw_1\wedge(mw\wedge Tv)
=\sup_{v\in A}mw_1\wedge Tv\le kTv_0$,}

\noindent which is impossible because $L^0$ is Archimedean
and $w_1>0$.\ \Bang\Qed

\medskip

{\bf (b)}  Because $L^0$ is Dedekind complete,
$\sup\{Tv:v\in V,\,0\le v\le u\}$ is defined in $L^0$ for every
$u\in U$.   By 355F, $T$ has a
unique extension to an order-continuous Riesz homomorphism from $U$ to
$L^0$.
}%end of proof of 368B

\leader{368C}{Corollary} Let $\frak A$ and $\frak B$ be Dedekind
complete Boolean algebras and $U$, $V$ order-dense Riesz subspaces of
$L^0(\frak A)$, $L^0(\frak B)$ respectively.   Then any Riesz space
isomorphism
between $U$ and $V$ extends uniquely to a Riesz space isomorphism
between $L^0(\frak A)$ and $L^0(\frak B)$;  and in this case $\frak A$
and $\frak B$ must be isomorphic as Boolean algebras.

\proof{ If $T:U\to V$ is a Riesz space isomorphism, then 368B tells us
that we have (unique)
order-continuous Riesz homomorphisms $\tilde T:L^0(\frak A)\to L^0(\frak B)$
and $\tilde T':L^0(\frak B)\to L^0(\frak A)$ extending $T$, $T^{-1}$
respectively.   Now $\tilde T'\tilde T:L^0(\frak A)\to L^0(\frak A)$ is an
order-continuous Riesz homomorphism agreeing with the identity on $U$,
so must be the identity on $L^0(\frak A)$;  similarly $\tilde T\tilde T'$ is the
identity on $L^0(\frak B)$, and $\tilde T$ is a Riesz space isomorphism.
To see that $\frak A$ and $\frak B$ are isomorphic, recall that by 364O
they can be identified with the algebras of projection bands of
$L^0(\frak A)$ and $L^0(\frak B)$, which must be isomorphic.
}%end of proof of 368C

\leader{368D}{Corollary} Suppose that $\frak A$ is a Dedekind
$\sigma$-complete Boolean algebra, and that $U$ is an order-dense Riesz
subspace of $L^0(\frak A)$ which is isomorphic, as Riesz space, to
$L^0(\frak B)$ for some Dedekind complete Boolean algebra $\frak B$.
Then $U=L^0(\frak A)$ and $\frak A$ is isomorphic to $\frak B$ (so, in
particular, is Dedekind complete).

\proof{ The identity mapping $U\to U$ is surely an order-continuous
Riesz homomorphism, so by 368B extends to an order-continuous Riesz
homomorphism
$\tilde T:L^0(\frak A)\to U$.   Now $\tilde T$ must be injective,
because if $u\ne 0$ in $L^0(\frak A)$ there is a $u'\in U$ such that
$0<u'\le|u|$,
so that $0<u'\le|\tilde Tu|$.   So we must have $U=L^0(\frak A)$ and
$\tilde T$ the identity map.   By 364O again, $\frak A\cong\frak B$.
}%end of proof of 368D

\leader{368E}{Theorem} Let $U$ be any Archimedean Riesz space, and
$\frak A$ its band algebra\cmmnt{ (353B)}.   Then $U$ can be
embedded as an order-dense Riesz subspace of $L^0(\frak A)$.

\proof{{\bf (a)} If $U=\{0\}$ then $\frak A=\{0\}$,
$L^0=L^0(\frak A)=\{0\}$ and the result is trivial;  I shall therefore
suppose henceforth
that $U$ is non-trivial.   Note that by 352Q $\frak A$ is Dedekind
complete.

Let $C\subseteq U^+\setminus\{0\}$ be a maximal disjoint set (in the
sense of 352C);  to obtain such a set apply Zorn's lemma to the family
of all disjoint subsets of $U^+\setminus\{0\}$.   Now I can write down
the formula for the
embedding $T:U\to L^0$ immediately, though there will be a good deal of
work to do in justification:  for $u\in U$ and $\alpha\in\Bbb R$,
$\Bvalue{Tu>\alpha}$ will be the band in $U$ generated by

\Centerline{$\{e\wedge(u-\alpha e)^+: e\in C\}$.}

\noindent (For once, I allow myself to use the formula $\Bvalue{\ldots}$
without checking immediately that it represents a member of $L^0$;  all
I
claim for the moment is that $\Bvalue{Tu>\alpha}$ is a member of $\frak
A$
determined by $u$ and $\alpha$.)

\medskip

{\bf (b)} Before getting down to the main argument, I make some remarks
which will be useful later.

\medskip

\quad{\bf (i)} If $u>0$ in $U$, then there is some $e\in C$ such that
$u\wedge e>0$, since otherwise we ought to have added $u$ to $C$.   Thus
$C^{\perp}=\{0\}$.
\medskip

\quad{\bf (ii)} If $u\in U$ and $e\in C$ and $\alpha\in\Bbb R$, then
$v=e\wedge(\alpha e-u)^+$ belongs to $\Bvalue{Tu>\alpha}^{\perp}$.
\Prf\
If $e'\in C$, then either $e'\ne e$ so

\Centerline{$v\wedge e'\wedge(u-\alpha e')^+\le e\wedge e'=0$,}

\noindent or $e'=e$ and

\Centerline{$v\wedge e'\wedge(u-\alpha e')^+
\le (\alpha e-u)^+\wedge(u-\alpha e)^+=0$.}

\noindent   Accordingly $\Bvalue{Tu>\alpha}$ is included in the band
$\{v\}^{\perp}$ and $v\in\Bvalue{Tu>\alpha}^{\perp}$.\ \Qed

\medskip

{\bf (c)} Now I must confirm that the formula given for
$\Bvalue{Tu>\alpha}$ is consistent with the conditions laid down in
364Aa.   \Prf\ Take $u\in U$.

\medskip

\quad{\bf (i)} If $\alpha\le\beta$ then

\Centerline{$0\le e\wedge(u-\beta e)^+
\le e\wedge(u-\alpha e)^+\in\Bvalue{Tu>\alpha}$}

\noindent so $e\wedge(u-\beta e)^+\in\Bvalue{Tu>\alpha}$, for every
$e\in C$, and $\Bvalue{Tu>\beta}\subseteq\Bvalue{Tu>\alpha}$.

\medskip

\quad{\bf (ii)} Given
$\alpha\in\Bbb R$, set $W=\sup_{\beta>\alpha}\Bvalue{Tu>\beta}$ in
$\frak A$, that is, the band in $U$ generated by
$\{e\wedge(u-\beta e)^+:e\in C,\,\beta>\alpha\}$.   Then for each
$e\in C$,

\Centerline{$\sup_{\beta>\alpha}e\wedge(u-\beta e)^+
=e\wedge(u-\inf_{\beta>\alpha}\beta e)^+
=e\wedge(u-\alpha e)^+$}

\noindent using the general distributive laws in $U$ (352E), the
translation-invariance of the order (351D) and the fact that $U$ is
Archimedean (to see that $\alpha e=\inf_{\beta>\alpha}\beta e$).   So
$e\wedge(u-\alpha e)^+\in W$;  as $e$ is arbitrary,
$\Bvalue{Tu>\alpha}\subseteq W$ and $\Bvalue{Tu>\alpha}=W$.

\medskip

\quad{\bf (iii)} Now set $W=\inf_{n\in\Bbb N}\Bvalue{Tu>n}$.   For any
$e\in C$, $n\in\Bbb N$ we have

\Centerline{$e\wedge(ne-u)^+\in\Bvalue{Tu>n}^{\perp}\subseteq
W^{\perp}$,}

\noindent so that

\Centerline{$e\wedge(e-\bover1nu^+)^+\le e\wedge(e-\bover1nu)^+\in
W^{\perp}$}

\noindent for every $n\ge 1$ and

\Centerline{$e=\sup_{n\ge 1}e\wedge(e-\bover1nu^+)^+\in W^{\perp}$. }

\noindent Thus $C\subseteq W^{\perp}$ and $W\subseteq C^{\perp}=\{0\}$.
So we have $\inf_{n\in\Bbb N}\Bvalue{Tu>n}=0$.

\medskip

\quad{\bf (iv)} Finally, set
$W=\sup_{n\in\Bbb N}\Bvalue{Tu>-n}$.   Then

\Centerline{$e\wedge(e-\bover1nu^-)^+\le e\wedge(e+\bover1nu)^+\le
e\wedge(u+ne)^+\in W$}

\noindent for every $n\ge 1$ and $e\in C$, so

\Centerline{$e=\sup_{n\ge 1}e\wedge(e-\bover1nu^-)^+\in W$}

\noindent for every $e\in C$ and $W^{\perp}=\{0\}$, $W=U$.   Thus all
three conditions of 364Aa are satisfied.\ \Qed

\medskip

{\bf (d)} Thus we have a well-defined map $T:U\to L^0$.
I show next that $T(u+v)=Tu+Tv$ for all $u$, $v\in U$.   \Prf\ I rely on
the formulae in 364D and 364Ea, and on partitions of unity in $\frak A$,
constructed as follows.   Fix $n\ge 1$ for the moment.   Then we know
that

\Centerline{$\sup_{i\in\Bbb Z}\Bvalue{Tu>\bover{i}{n}}=1$,
\quad$\inf_{i\in\Bbb Z}\Bvalue{Tu>\bover{i}{n}}=0$.}

\noindent So setting

\Centerline{$V_i
=\Bvalue{Tu>\bover{i}n}\Bsetminus\Bvalue{Tu>\bover{i+1}n}
=\Bvalue{Tu>\bover{i}n}\cap\Bvalue{Tu>\bover{i+1}n}^{\perp}$,}

\noindent $\langle V_i\rangle_{i\in\Bbb Z}$ is a partition of unity in
$\frak A$.   Similarly, $\langle W_i\rangle_{i\in\Bbb Z}$ is a partition
of unity, where

\Centerline{$W_i
=\Bvalue{Tv>\bover{i}n}\cap\Bvalue{Tv>\bover{i+1}n}^{\perp}$.}

\noindent Now, for any $i$, $j$, $k\in\Bbb Z$ such that $i+j\ge k$,

\Centerline{$V_i\cap W_j
\subseteq\Bvalue{Tu>\bover{i}n}\cap\Bvalue{Tv>\bover{j}n}
\subseteq\Bvalue{Tu+Tv>\bover{i+j}n}
\subseteq\Bvalue{Tu+Tv>\bover{k}n}$;}

\noindent thus

\Centerline{$\Bvalue{Tu+Tv>\bover{k}n}
\supseteq\sup_{i+j\ge k}V_i\cap W_j$.}

\noindent On the other hand, if $q\in\Bbb Q$ and $k\in\Bbb Z$, there is
an $i\in\Bbb Z$ such that $\bover{i}n\le q<\bover{i+1}n$, so that

\Centerline{$\Bvalue{Tu>q}\cap\Bvalue{Tv>\bover{k+1}n-q}
\subseteq\Bvalue{Tu>\bover{i}n}\cap\Bvalue{Tv>\bover{k-i}n}
\subseteq\sup_{i+j\ge k}V_i\cap W_j$;}

\noindent thus for any $k\in\Bbb Z$
\Centerline{$\Bvalue{Tu+Tv>\bover{k+1}n}\subseteq\sup_{i+j\ge k}V_i\cap
W_j\subseteq\Bvalue{Tu+Tv>\bover{k}n}$.}


Also, if $0<w\in V_i\cap W_j$ and $e\in C$ then

\Centerline{$w\wedge e\wedge(u-\Bover{i+1}ne)^+
=w\wedge e\wedge(v-\Bover{j+1}ne)^+=0$,}

\noindent so that

\Centerline{$w\wedge e\wedge(u+v-\Bover{i+j+2}ne)^+=0$}

\noindent because

\Centerline{$(u+v-\Bover{i+j+2}ne)^+
\le(u-\Bover{i+1}ne)^++(v-\Bover{j+1}ne)^+$}

\noindent by 352Fc.   But this means that
$V_i\cap W_j\cap\Bvalue{T(u+v)>\bover{i+j+2}n}=\{0\}$.   Turning this
round,

\Centerline{$\Bvalue{T(u+v)>\bover{k+1}n}\cap\sup_{i+j\le k-1}V_i\cap
W_j=0$,}

\noindent and because $\sup_{i,j\in\Bbb Z}V_i\cap W_j=U$ in $\frak A$,

\Centerline{$\Bvalue{T(u+v)>\bover{k+1}n}\subseteq\sup_{i+j\ge k}V_i\cap
W_j$.}

\noindent Finally, if $i+j\ge k$ and $0<w\in V_i\cap V_j$, then there is
an $e\in C$ such that $w_1=w\wedge e\wedge(u-\bover{i}ne)^+>0$;  there
is an $e'\in C$ such that $w_2=w_1\wedge e'\wedge(v-\bover{j}ne')^+>0$;
of course $e=e'$, and

$$\eqalign{0
<w_2
\le e\wedge(u-\Bover{i}ne)^+\wedge(v-\Bover{j}n)^+
&\le e\wedge(u+v-\Bover{i+j}ne)^+\cr
&\in\Bvalue{T(u+v)>\Bover{i+j}n}
\subseteq\Bvalue{T(u+v)>\Bover{k}n}\cr}$$

\noindent using 352Fc.   This shows that
$w\not\in\Bvalue{T(u+v)>\bover{k}n}^{\perp}$;  as $w$ is arbitrary,
$V_i\cap W_j\subseteq\Bvalue{T(u+v)>\bover{k}n}$;  so we get

\Centerline{$\sup_{i+j\ge k}V_i\cap W_j\subseteq
\Bvalue{T(u+v)>\bover{k}{n}}$.}

Putting these four facts together, we see that

\Centerline{$\Bvalue{T(u+v)>\bover{k+1}n}\subseteq\sup_{i+j\ge k}V_i\cap
W_j\subseteq\Bvalue{Tu+Tv>\bover{k}n}$,}
\Centerline{$\Bvalue{Tu+Tv>\bover{k+1}n}\subseteq\sup_{i+j\ge k}V_i\cap
W_j\subseteq\Bvalue{T(u+v)>\bover{k}n}$}

\noindent for all $n\ge 1$ and $k\in\Bbb Z$.   But this means that we must
have

\Centerline{$\Bvalue{T(u+v)>\beta}\subseteq\Bvalue{Tu+Tv>\alpha}$,
\quad$\Bvalue{Tu+Tv>\beta}\subseteq\Bvalue{T(u+v)>\alpha}$}

\noindent whenever $\alpha<\beta$.   Consequently

$$\eqalign{\Bvalue{Tu+Tv>\alpha}
&=\sup_{\beta>\alpha}\Bvalue{Tu+Tv>\beta}
\subseteq\Bvalue{T(u+v)>\alpha}\cr
&=\sup_{\beta>\alpha}\Bvalue{T(u+v)>\beta}
\subseteq\Bvalue{Tu+Tv>\alpha}\cr}$$

\noindent and $\Bvalue{Tu+Tv>\alpha}=\Bvalue{T(u+v)>\alpha}$ for every
$\alpha$, that is, $T(u+v)=Tu+Tv$.\ \Qed

\medskip

{\bf (e)} The hardest part is over.   If $u\in U$, $\gamma>0$ and
$\alpha\in\Bbb R$, then for any $e\in C$

\Centerline{$\min(1,\Bover1{\gamma})(e\wedge(\gamma u-\alpha e)^+)
\le e\wedge(u-\Bover{\alpha}{\gamma}e)^+
\le\max(1,\Bover1{\gamma})(e\wedge(\gamma u-\alpha e)^+)$,}

\noindent so

\Centerline{$\Bvalue{T(\gamma
u)>\alpha}=\Bvalue{Tu>\bover{\alpha}{\gamma}}
=\Bvalue{\gamma Tu>\alpha}$;}

\noindent as $\alpha$ is arbitrary, $\gamma Tu=T(\gamma u)$;  as
$\gamma$ and $u$ are arbitrary, $T$ is linear.   (We need only check
linearity for
$\gamma>0$ because we know from the additivity of $T$ that $T(-u)=-Tu$
for every $u$.)

\medskip

{\bf (f)} To see that $T$ is a Riesz homomorphism, take any $u\in U$ and
$\alpha\in\Bbb R$ and consider the band
$\Bvalue{Tu>\alpha}\Bcup\Bvalue{-Tu>\alpha}\penalty-100=\Bvalue{|Tu|>\alpha}$ (by
364L(a-ii)).    This is the band generated by
$\{e\wedge(u-\alpha e)^+:e\in C\}\cup\{e\wedge(-u-\alpha e)^+:e\in C\}$.
But this must also be the band generated by

\Centerline{$\{(e\wedge(u-\alpha e)^+)\vee(e\wedge(-u-\alpha e)^+):
e\in C\}=\{e\wedge(|u|-\alpha e)^+:e\in C\}$,}

\noindent which is $\Bvalue{T|u|>\alpha}$.   Thus
$\Bvalue{|Tu|>\alpha}=\Bvalue{T|u|>\alpha}$ for every $\alpha$ and
$|Tu|=T|u|$.   As $u$ is arbitrary, $T$ is a Riesz homomorphism.

\medskip

{\bf (g)} To see that $T$ is injective, take any non-zero $u\in U$.
Then
there must be some $e\in C$ such that $|u|\wedge e\ne 0$, and some
$\alpha>0$ such that $|u|\wedge e\not\le\alpha e$, so that
$e\wedge(|u|-\alpha e)^+\ne 0$ and $\Bvalue{T|u|>\alpha}\ne\{0\}$ and
$T|u|\ne 0$ and $Tu\ne 0$.

Thus $T$ embeds $U$ as a Riesz subspace of $L^0$.

\medskip

{\bf (h)} Finally, I must check that $T[U]$ is order-dense in $L^0$.
\Prf\ Let $p>0$ in $L^0$.   Then there is some $\alpha>0$ such that
$V=\Bvalue{p>\alpha}\ne 0$.   Take $u>0$ in $V$.   Let $e\in C$ be such
that $u\wedge e>0$.   Then $v=u\wedge\alpha e>0$.   Now
$e\wedge(v-\alpha e)^+=0$;  but also $e'\wedge v=0$ for every $e'\in C$
distinct from $e$, so that $\Bvalue{Tv>\alpha}=\{0\}$.   Next, $v\in V$,
so $e'\wedge(v-\beta e')^+\in V$ whenever $e'\in C$ and $\beta\ge 0$,
and $\Bvalue{Tv>\beta}\subseteq V$ for every $\beta\ge 0$.   Accordingly
we have

$$\eqalign{\Bvalue{Tv>\beta}&=\{0\}\subseteq\Bvalue{p>\beta}
  \text{ if }\beta\ge\alpha,\cr
&\subseteq V\subseteq\Bvalue{p>\beta}
\text{ if }0\le\beta<\alpha,\cr
&=U=\Bvalue{p>\beta}
\text{ if }\beta<0,\cr}$$

\noindent and $Tv\le p$.   Also $Tv>0$, by (g).   As $p$ is arbitrary,
$T[U]$ is order-dense in $L^0$.\ \Qed
}%end of proof of 368E

\leader{368F}{Corollary} A Riesz space $U$ is Archimedean iff it is
isomorphic to a Riesz subspace of some reduced power $\Bbb R^X|\Cal F$,
where $X$ is a set and $\Cal F$ is a filter on $X$ such that
$\bigcap_{n\in\Bbb N}F_n\in\Cal F$ whenever $\sequencen{F_n}$ is a
sequence in $\Cal F$.

\proof{{\bf (a)} If $U$ is an Archimedean Riesz space, then by 368E
there is a space of the form $L^0=L^0(\frak A)$ such that $U$ can be
embedded into $L^0$.   As in the proof of 364D, $L^0$ is isomorphic to
some space of the form $\eusm L^0(\Sigma)/\eusm W$, where $\Sigma$ is a
$\sigma$-algebra of subsets of some set $X$ and
$\eusm W=\{f:f\in\eusm L^0,\,\{x:f(x)\ne 0\}\in\Cal I\}$, $\Cal I$ being
a $\sigma$-ideal of
$\Sigma$.   But now $\Cal F=\{A:A\cup E=X$ for some $E\in\Cal I\}$ is a
filter on $X$ such that $\bigcap_{n\in\Bbb N}F_n\in\Cal F$ for every
sequence $\sequencen{F_n}$ in $\Cal F$.   (I am passing over the trivial
case $X\in\Cal I$, since then $U$ must be $\{0\}$.)   And
$\eusm L^0/\eusm W$ is
(isomorphic to) the image of $\eusm L^0$ in $\Bbb R^X|\Cal F$, since
$\eusm W=\{f:f\in\eusm L^0,\,\{x:f(x)=0\}\in\Cal F\}$.   Thus $U$ is
isomorphic to a Riesz subspace of $\Bbb R^X|\Cal F$.

\medskip

{\bf (b)} On the other hand, if $\Cal F$ is a filter on $X$ closed under
countable intersections, then
$\eusm W=\{f:f\in\Bbb R^X,\,\{x:f(x)=0\}\in\Cal F\}$ is a sequentially
order-closed solid linear
subspace of the Dedekind $\sigma$-complete Riesz space $\Bbb R^X$, so
that $\Bbb R^X|\Cal F=\Bbb R^X/\eusm W$ is Dedekind
$\sigma$-complete (353J(a-iii)) and all its Riesz subspaces must be
Archimedean (353Ha, 351Rc).
}%end of proof of 368F

\leader{368G}{Corollary} Every Archimedean Riesz space $U$ is isomorphic
to an order-dense Riesz subspace of some space
$C^{\infty}(X)$\cmmnt{ (definition:  364V)}, where
$X$ is an extremally disconnected compact Hausdorff space.

\proof{ Let $Z$ be the Stone space of the band algebra $\frak A$ of $U$.
Because $\frak A$ is Dedekind complete (352Q), $Z$ is extremally
disconnected and $\frak A$ can be identified
with the regular open algebra $\RO(Z)$ of $Z$ (314S).   By 364V,
$L^0(\RO(Z))$ can be identified with $C^{\infty}(Z)$.    So an
embedding
of $U$ as an order-dense Riesz subspace of $L^0(\frak A)$ (368E) can be
regarded as an embedding of $U$ as an order-dense Riesz subspace of
$C^{\infty}(Z)$.
}%end of proof of 368G

\leader{368H}{Corollary} Any Dedekind complete Riesz space $U$ is
isomorphic to an order-dense solid linear subspace of $L^0(\frak A)$ for
some Dedekind complete Boolean algebra $\frak A$.

\proof{ Embed $U$ in $L^0=L^0(\frak A)$ as in 368E;  because $U$ is
order-dense in $L^0$ and (in itself) Dedekind complete, it is solid
(353K).
}%end of proof of 368H

\leader{368I}{Corollary} Let $U$ be an Archimedean Riesz space.   Then
$U$ can be embedded as an order-dense Riesz subspace of a Dedekind
complete
Riesz space $V$ in such a way that the solid linear subspace of $V$
generated by $U$ is $V$ itself,
and this can be done in essentially only one way.
If $W$ is any other Dedekind complete Riesz space and $T:U\to W$ is an
order-continuous positive linear operator, there is a unique positive
linear operator $\tilde T:V\to W$ extending $T$.

\proof{ By 368E, we may suppose that $U$ is actually an
order-dense Riesz subspace of $L^0(\frak A)$, where $\frak A$ is a
Dedekind complete Boolean algebra.   In this case, we can take $V$ to be
the solid linear subspace generated by $U$, that is, $\{v:|v|\le u$ for
some $u\in U\}$;  being a solid linear subspace of the Dedekind complete
Riesz space $L^0(\frak A)$, $V$ is Dedekind complete, and of course $U$
is order-dense in $V$.

If $W$ is any other Dedekind complete Riesz space and $T:U\to W$ is an
order-continuous positive linear operator, then for any $v\in V^+$ there
is a $u_0\in U$ such that $v\le u_0$, so that $Tu_0$ is an upper bound
for $\{Tu:u\in U,\,0\le u\le v\}$;  as $W$ is Dedekind complete,
$\sup_{u\in U,0\le u\le v}Tu$ is defined in $W$.   By 355Fa, $T$ has a
unique extension to an order-continuous positive linear operator from
$V$ to $W$.

In particular, if $V_1$ is another Dedekind complete Riesz space in
which $U$ can be embedded as an order-dense Riesz subspace, this
embedding of $U$
extends to an embedding of $V$;  since $V$ is Dedekind complete, its
copy in $V_1$ must be a solid linear subspace, so if $V_1$ is the solid
linear subspace of itself generated by $U$, we get an identification
between $V$
and $V_1$, uniquely determined by the embeddings of $U$ in $V$ and
$V_1$.
}%end of proof of 368I

\leader{368J}{Definition} If $U$ is an Archimedean Riesz space, a {\bf
Dedekind completion} of $U$ is a Dedekind complete Riesz space $V$
together
with an embedding of $U$ in $V$ as an order-dense Riesz subspace of $V$
such that the solid linear subspace of $V$ generated by $U$ is $V$
itself.
\cmmnt{368I tells us that every Archimedean Riesz space $U$ has an
essentially unique Dedekind completion, so that we may speak of `the'
Dedekind completion of $U$.}

\leader{368K}{}\cmmnt{ This is a convenient point at which to give a
characterization of the Riesz spaces $L^0(\frak A)$.

\medskip

\noindent}{\bf Lemma} Let $\frak A$ be a Dedekind $\sigma$-complete
Boolean algebra.
Suppose that $A\subseteq L^0(\frak A)^+$ is disjoint.   If {\it either}
$A$ is countable {\it or} $\frak A$ is Dedekind complete, 
$A$ is bounded above in $L^0(\frak A)$.

\proof{ If $A=\emptyset$, this is trivial;  suppose that $A$ is not
empty.   For $n\in\Bbb N$, set $a_n=\sup_{u\in A}\Bvalue{u>n}$;  this is
always defined;  set $a=\inf_{n\in\Bbb N}a_n$.   Now $a=0$.   \Prf\Quer\
Otherwise, there must be a $u\in A$ such that $a'=a\Bcap\Bvalue{u>0}\ne
0$, since $a\Bsubseteq a_0$.   But now, for any $n$, and any $v\in
A\setminus\{u\}$,

\Centerline{$a'\Bcap\Bvalue{v>n}\Bsubseteq\Bvalue{u>0}\Bcap\Bvalue{v>0}
=0$,}

\noindent so that $a'\Bsubseteq\Bvalue{u>n}$.   As $n$ is arbitrary,
$\inf_{n\in\Bbb N}\Bvalue{u>n}\ne 0$, which is impossible.\ \Bang\Qed\

By 364L(a-i), $A$ is bounded above.
}%end of proof of 368K

\leader{368L}{Definition} A Riesz space $U$ is called {\bf laterally
complete} or {\bf universally complete} if $A$ is bounded above whenever
$A\subseteq U^+$ is disjoint.

\vleader{108pt}{368M}{Theorem} Let $U$ be an Archimedean Riesz space.   Then the
following are equiveridical:

(i) there is a Dedekind complete Boolean algebra $\frak A$ such that $U$
is isomorphic to $L^0(\frak A)$;

(ii) $U$ is Dedekind $\sigma$-complete and laterally complete;

(iii) whenever $V$ is an Archimedean Riesz space, $V_0$ is an
order-dense Riesz subspace of $V$ and $T:V_0\to U$ is an
order-continuous Riesz homomorphism, there is a positive linear operator
$\tilde T:V\to U$ extending $T$.

\proof{{\bf (a)(i)$\Rightarrow$(ii)} and {\bf (i)$\Rightarrow$(iii)} are
covered by 368K and 368B.

\medskip

{\bf (b)(ii)$\Rightarrow$(i)} Assume (ii).   By 368E, we may suppose
that $U$ is actually an order-dense Riesz subspace of $L^0=L^0(\frak A)$
for a Dedekind complete Boolean algebra $\frak A$.

\medskip

\quad\grheada\ If $u\in U^+$ and $a\in\frak A$ then
$u\times\chi a\in U$.
\Prf\ Set $A=\{v:v\in U,\,0\le v\le\chi a\}$, and let $C\subseteq A$ be
a maximal disjoint set;  then $w=\sup C$ is defined in $U$, and is also
the supremum in $L^0$.   Set $b=\Bvalue{w>0}$.   As $w\le\chi a$,
$b\Bsubseteq a$.   \Quer\ If $b\ne a$, then $\chi(a\Bsetminus b)>0$, and
there is a
$v'\in U$ such that $0<v'\le\chi(a\Bsetminus b)$;  but now $v'\in A$ and
$v'\wedge w=0$, so $v'\wedge v=0$ for every $v\in C$, and we ought to
have added $v'$ to $C$.\ \BanG\  Thus $\Bvalue{w>0}=a$.

Now consider $u'=\sup_{n\in\Bbb N}u\wedge nw$;  as $U$ is Dedekind
$\sigma$-complete, $u'\in U$.   Since $\Bvalue{u'>0}\Bsubseteq a$,
$u'\le
u\times\chi a$.   On the other hand,

\Centerline{$u\times\chi\Bvalue{w>\bover1n}\times\chi\Bvalue{u\le n}
\le u\wedge n^2w\le u'$}

\noindent for every $n\ge 1$, so, taking the supremum over $n$,
$u\times\chi a\le u'$.   Accordingly

\Centerline{$u\times\chi a=u'\in U$,}

\noindent as required.\ \Qed

\medskip

\quad\grheadb\ If $w\ge 0$ in $L^0$, there is a $u\in U$ such that
$\bover12w\le u\le w$.   \Prf\ Set

\Centerline{$A=\{u:u\in U,\,0\le u\le w\}$,}

\Centerline{$C=\{a:a\in\frak A,\,a\Bsubseteq\Bvalue{u-\bover12w\ge 0}$
for
some $u\in A\}$.}

\noindent Then $\sup A=w$, so $C$ is order-dense in $\frak A$.   (If
$a\in\frak A\setminus\{0\}$, either $a\Bcap\Bvalue{w>0}=0$ and
$a\subseteq\Bvalue{0-\bover12w\ge 0}$, so $a\in C$, or there is a $u\in
U$
such that $0<u\le w\times\chi a$.   In the latter case there is some $n$
such that $2^nu\le w$ and $2^{n+1}u\not\le w$, and now
$c=a\Bcap\Bvalue{2^nu-\bover12w\ge 0}$ is a non-zero member of $C$
included in $a$.)   Let $D\subseteq C$ be a partition of unity and for
each $d\in D$
choose $u_d\in A$ such that $d\Bsubseteq\Bvalue{u_d-\bover12w\ge 0}$.
By ($\alpha$), $u_d\times\chi d\in U$ for every $d\in D$, so
$u=\sup_{d\in D}u_d\times\chi d\in U$.   Now $u\le w$, but also
$\Bvalue{u-\bover12w\ge 0}\Bsupseteq d$ for every $d\in D$, so is equal
to $1$, and $u\ge\bover12w$, as required.\ \Qed

\medskip

\quad\grheadc\ Given $w\ge 0$ in $L^0$, we can therefore choose
$\sequencen{u_n}$, $\sequencen{v_n}$ inductively such that $v_0=0$ and

\Centerline{$u_n\in U$,
\quad$\Bover12(w-v_n)\le u_n\le w-v_n$,
\quad$v_{n+1}=v_n+u_n$}

\noindent for every $n\in\Bbb N$.   Now $\sequencen{v_n}$ is a
non-decreasing sequence in $U$ and $w-v_n\le 2^{-n}w$ for every $n$, so
$w=\sup_{n\in\Bbb N}v_n\in U$.

As $w$ is arbitrary, $(L^0)^+\subseteq U$ and $U=L^0$ is of the right
form.

\medskip

{\bf (c)(iii)$\Rightarrow$(i)} As in (b), we may suppose that $U$ is an
order-dense Riesz subspace of $L^0$.   But now apply condition (iii)
with $V=L^0$, $V_0=U$ and $T$ the identity operator.   There is an
extension $\tilde T:L^0\to U$.   If $v\ge 0$ in $L^0$, $\tilde Tv\ge Tu=u$
whenever
$u\in U$ and $u\le v$, so $\tilde Tv\ge v$,
since $v=\sup\{u:u\in U,\,0\le u\le v\}$ in $L^0$.   Similarly,
$\tilde T(\tilde Tv-v)\ge\tilde Tv-v$.   But as $\tilde Tv\in U$,
$\tilde T(\tilde Tv)=T(\tilde Tv)=\tilde Tv$ and
$\tilde T(\tilde Tv-v)=0$, so
$v=\tilde Tv\in U$.   As $v$ is arbitrary, $U=L^0$.
}%end of proof of 368M

\leader{368N}{Weakly $(\sigma,\infty)$-distributive Riesz
\dvrocolon{spaces}}\cmmnt{ We are now ready to look at the class of
Riesz spaces corresponding to the \wsid\ Boolean algebras of \S316.

\medskip

\noindent}{\bf Definition} Let $U$ be a Riesz space.   Then $U$ is {\bf
\wsid} if whenever $\sequencen{A_n}$ is a sequence of non-empty
downwards-directed subsets of $U^+$, each with infimum $0$, and
$\bigcup_{n\in\Bbb N}A_n$ has an upper bound in $U$, then

\Centerline{$\{u:u\in U$, for every $n\in\Bbb N$ there is a $v\in A_n$
such that $v\le u\}$}

\noindent has infimum $0$ in $U$.

\cmmnt{\medskip

\noindent{\bf Remark} Because the definition looks only at sequences
$\sequencen{A_n}$ such that $\bigcup_{n\in\Bbb N}A_n$ is order-bounded,
we can invert it, as follows:  a Riesz space $U$ is \wsid\ iff whenever
$\sequencen{A_n}$ is a sequence of
non-empty upwards-directed subsets of $U^+$, all with supremum $u_0$,
then

\Centerline{$\{u:u\in U^+$, for every $n\in\Bbb N$ there is a $v\in A_n$
such that $u\le v\}$}

\noindent also has supremum $u_0$.
}%end of comment

\leader{368O}{Lemma} Let $U$ be an Archimedean Riesz space.   Then the
following are equiveridical:

(i) $U$ is not \wsid;

(ii) there are a $u>0$ in $U$ and a sequence $\sequencen{A_n}$ of
non-empty downwards-directed sets, all with infimum $0$, such that
$\sup_{n\in\Bbb N}u_n=u$ whenever $u_n\in A_n$ for every $n\in\Bbb N$.

\proof{(ii)$\Rightarrow$(i) is immediate from the definition of `\wsid'.
For (i)$\Rightarrow$(ii), suppose that $U$ is not \wsid.   Then there is
a sequence $\sequencen{A_n}$ of non-empty downwards-directed sets, all
with infimum $0$, such that $\bigcup_{n\in\Bbb N}A_n$ is bounded above,
but

\Centerline{$A=\{w:w\in U$, for every $n\in\Bbb N$ there is a $v\in A_n$
such that $v\le w\}$}

\noindent does not have infimum $0$.   Let $u>0$ be a lower bound for
$A$, and set $A'_n=\{u\wedge v:v\in A_n\}$ for each $n\in\Bbb N$.   Then
each $A'_n$ is a non-empty downwards-directed set with infimum $0$.
Let $\sequencen{u_n}$ be a sequence such that $u_n\in A'_n$ for every
$n$.   Express each $u_n$ as $u\wedge v_n$ where $v_n\in A_n$.   Let $B$
be the set of upper bounds of $\{v_n:n\in\Bbb N\}$.   Then
$\inf_{w\in B,n\in\Bbb N}w-v_n=0$, because $U$ is Archimedean (353F),
while $B\subseteq A$, so
$u\le w$ for every $w\in B$.   If $u'$ is any upper bound for
$\{u_n:n\in\Bbb N\}$, then

\Centerline{$u-u'\le u-u\wedge v_n=(u-v_n)^+\le(w-v_n)^+=w-v_n$}

\noindent whenever $n\in\Bbb N$ and $w\in B$.   So $u'\ge u$.   Thus
$u=\sup_{n\in\Bbb N}u_n$.   As $\sequencen{u_n}$ is arbitrary, $u$ and
$\sequencen{A'_n}$ witness that (ii) is true.
}%end of proof of 368O

\leader{368P}{Proposition} (a) A regularly embedded Riesz subspace of an
Archimedean \wsid\ Riesz space is \wsid.

%query:  do we need "Archimedean" here?  for order-dense ssps, no pb
%(368Ye)

(b) An Archimedean Riesz space with a \wsid\ order-dense Riesz subspace
is \wsid.

%query:  do we need "Archimedean" here?

(c) If $U$ is a Riesz space such that $U^{\times}$ separates the points
of $U$, then $U$ is \wsid;  in particular, $U^{\sim}$ and $U^{\times}$
are \wsid\ for every Riesz space $U$.

\proof{{\bf (a)} Suppose that $U$ is an Archimedean Riesz space and that
$V\subseteq U$ is a regularly embedded Riesz subspace which is not
\wsid.   Then 368O tells us that there are a $v>0$ in $V$ and a sequence
$\sequencen{A_n}$ of non-empty downwards-directed subsets of $V$, all
with infimum $0$ in $V$, such that $\sup_{n\in\Bbb N}v_n=v$ in $V$ whenever $v_n\in A_n$ for every $n\in\Bbb N$.   Because $V$ is regularly embedded in
$U$, $\inf A_n=0$ in $U$ for every $n$ and $\sup_{n\in\Bbb N}v_n=v$ in
$U$ for every sequence $\sequencen{v_n}\in\prod_{n\in\Bbb N}A_n$, so $U$ is not \wsid.   Turning this round, we have (a).

\medskip

{\bf (b)} Let $U$ be an Archimedean Riesz space which is not \wsid, and
$V$ an order-dense Riesz subspace of $U$.   By 368O again, there are a
$u^*>0$ in $U$ and a sequence $\sequencen{A_n}$ of non-empty
downwards-directed sets in $U$, all with infimum $0$, such that
$\sup_{n\in\Bbb N}u_n=u^*$ whenever
$u_n\in A_n$ for every $n$.   Let $v\in V$ be such that $0<v\le u^*$.
Set

\Centerline{$B_n=\{w:w\in V$, there is some $u\in A_n$ such that
$v\wedge u\le w\le v\}$}

\noindent for each $n\in\Bbb N$.   Because $A_n$ is downwards-directed,
$w\wedge w'\in B_n$ for all $w$, $w'\in B_n$;  $v\in B_n$, so
$B_n\ne\emptyset$;  and $\inf B_n=0$ in $V$.   \Prf\ Setting

\Centerline{$C=\{w:w\in V^+$, there is some $u\in A_n$ such that
$w\le(v-u)^+\}$,}

\noindent then (because $V$ is order-dense) any upper bound for $C$ in
$U$ is also an upper bound of $\{(v-u)^+:u\in A_n\}$.   But

\Centerline{$\sup_{u\in A_n}(v-u)^+=(v-\inf A_n)^+=v$,}

\noindent so $v=\sup C$ in $U$ and $\inf B_n=\inf\{v-w:w\in C\}=0$ in
$U$ and in $V$.\ \Qed

Now if $v_n\in B_n$ for every $n\in\Bbb N$, we can choose $u_n\in A_n$
such that $v\wedge u_n\le v_n\le v$ for every $n$, so that

\Centerline{$v=v\wedge u^*=v\wedge\sup_{n\in\Bbb N}u_n
=\sup_{n\in\Bbb N}v\wedge u_n\le\sup_{n\in\Bbb N}v_n\le v$,}

\noindent and $v=\sup_{n\in\Bbb N}v_n$.   Thus $\sequencen{B_n}$
witnesses that $V$ is not \wsid.

\medskip

{\bf (c)} Now suppose that $U^{\times}$ separates the points of $U$.
In this case $U$ is surely Archimedean (356G).   \Quer\ If $U$ is not
\wsid, there are a $u>0$ in $U$ and a sequence $\sequencen{A_n}$ of
non-empty downwards-directed sets, all with infimum $0$, such that
$\sup_{n\in\Bbb N}u_n=u$ whenever $u_n\in A_n$ for each $n$.   Take
$f\in U^{\times}$ such
that $f(u)\ne 0$;  replacing $f$ by $|f|$ if necessary, we may suppose
that $f>0$.   Set $\delta=f(u)>0$.   For each $n\in\Bbb N$, there is a
$u_n\in A_n$ such that $f(u_n)\le 2^{-n-2}\delta$.   But in this case
$\sequencen{\sup_{i<n}u_i}$ is a
non-decreasing sequence with supremum $u$, so

\Centerline{$f(u)=\lim_{n\to\infty}f(\sup_{i\le
n}u_i)\le\sum_{i=0}^{\infty}f(u_i)\le\bover12\delta<f(u)$,}

\noindent which is absurd.\ \BanG\  Thus $U$ is \wsid.

For any Riesz space $U$, $U$ acts on $U^{\sim}$ as a subspace of
$U^{\sim\times}$ (356F);  as $U$ surely separates the points of
$U^{\sim}$, so does $U^{\sim\times}$.   So $U^{\sim}$ is \wsid.   Now
$U^{\times}$ is a
band in $U^{\sim}$ (356B), so is regularly embedded, and must also be
\wsid, by (a) above.
}%end of proof of 368P

\leader{368Q}{Theorem} (a) For any Boolean algebra $\frak A$, $\frak A$
is \wsid\ iff $S(\frak A)$ is \wsid\ iff $L^{\infty}(\frak A)$ is \wsid.

(b) For a Dedekind $\sigma$-complete Boolean algebra $\frak A$,
$L^0(\frak A)$ is \wsid\ iff $\frak A$ is \wsid.

\proof{{\bf (a)(i)} \Quer\ Suppose, if possible, that $\frak A$ is
\wsid\ but $S=S(\frak A)$ is not.   By 368O, as usual, we have a $u>0$
in $S$ and
a sequence $\sequencen{A_n}$ of non-empty downwards-directed sets in
$S$, all with
infimum $0$, such that $u=\sup_{n\in\Bbb N}u_n$ whenever $u_n\in A_n$
for every $n$.   Let $\alpha>0$ be such that $c=\Bvalue{u>\alpha}\ne 0$
(361Eg), and consider

\Centerline{$B_n=\{\Bvalue{v>\alpha}:v\in A_n\}\subseteq\frak A$}

\noindent for each $n\in\Bbb N$.   Then each $B_n$ is
downwards-directed (because $A_n$ is), and $\inf B_n=0$ in $\frak A$
(because if $b$ is a lower bound of $B_n$, $\alpha\chi b\le v$ for every
$v\in A_n$).   Because $\frak A$ is \wsid, there must be some $a\in\frak
A$
such that $a\notBsupseteq c$ but there is, for every $n\in\Bbb N$, a
$b_n\in B_n$ such that $a\Bsupseteq b_n$.   Take $v_n\in A_n$ such that
$b_n=\Bvalue{v_n>\alpha}$, so that

\Centerline{$v_n\le\alpha\chi 1\vee\|v_n\|_{\infty}\chi b_n
\le\alpha\chi 1\vee\|u\|_{\infty}\chi a$.}

\noindent Since $u=\sup_{n\in\Bbb N}v_n$, $u\le\alpha\chi
1\vee\|u\|_{\infty}\chi a$.   But in this case

\Centerline{$c=\Bvalue{u>\alpha}\Bsubseteq a$,}

\noindent contradicting the choice of $a$.\ \Bang

Thus $S$ must be \wsid\ if $\frak A$ is.

\medskip

\quad{\bf (ii)} Now suppose that $S$ is \wsid, and let $\sequencen{B_n}$
be
a sequence of non-empty downwards-directed subsets of $\frak A$, all
with
infimum $0$.   Set $A_n=\{\chi b:b\in B_n\}$ for each $n$;  then
$A_n\subseteq S$ is non-empty, downwards-directed and has infimum $0$ in
$S$, because $\chi:\frak A\to S$ is order-continuous (361Ef).   Set

\Centerline{$A=\{v:v\in S$, for every $n\in\Bbb N$ there is a $u\in A_n$
such that $u\le v\}$,}

\Centerline{$B=\{b:b\in\frak A$, for every $n\in\Bbb N$ there is an
$a\in B_n$ such that $a\Bsubseteq b\}$.}

\noindent \Quer\ If $0$ is not the greatest lower bound of $B$, take a
non-zero lower bound $c$.   Because $S$ is \wsid, $\inf A=0$, and there
is a $v\in A$ such that $\chi c\not\le v$.   Express $v$ as
$\sum_{i=0}^n\alpha_i\chi a_i$, where $\langle a_i\rangle_{i\le n}$ is
disjoint, and set $a=\sup\{a_i:i\le n,\,\alpha_i\ge 1\}$;  then
$\chi a\le v$, so $c\notBsubseteq a$.   For each $n$ there is a
$b_n\in B_n$ such
that $\chi b_n\le v$.   But in this case $b_n\Bsubseteq a$ for each
$n\in\Bbb N$, so that $a\in B$;  which means that $c$ is not a lower
bound for $B$.\ \Bang

Thus $\inf B=0$ in $\frak A$.   As $\sequencen{B_n}$ is arbitrary,
$\frak A$ is \wsid.

\medskip

\quad{\bf (iii)} Thus $S$ is \wsid\ iff $\frak A$ is.   But $S$ is
order-dense in $L^{\infty}=L^{\infty}(\frak A)$ (363C), therefore
regularly embedded (352Ne), so 368Pa-b tell us that $S$ is \wsid\ iff
$L^{\infty}$ is.

\medskip

{\bf (b)} In the same way, because $S$ can be regarded as an order-dense
Riesz subspace of $L^0=L^0(\frak A)$ (364Ja), $L^0$ is \wsid\ iff $S$ is,
that is, iff $\frak A$ is.
}%end of proof of 368Q

\leader{368R}{Corollary} An Archimedean Riesz space is \wsid\ iff its
band algebra is \wsid.

\proof{ Let $U$ be an Archimedean Riesz space and $\frak A$ its band
algebra.   By 368E, $U$ is isomorphic to an order-dense Riesz subspace
of $L^0=L^0(\frak A)$.   By 368P, $U$ is \wsid\ iff $L^0$ is;  and by
368Qb $L^0$ is \wsid\ iff $\frak A$ is.
}%end of proof of 368R

\leader{368S}{Corollary} If $(\frak A,\bar\mu)$ is a semi-finite measure
algebra, any regularly embedded Riesz subspace (in particular, any solid
linear subspace and any order-dense Riesz subspace) of $L^0(\frak A)$ is
\wsid.

\proof{ By 322F, $\frak A$ is \wsid;  by 368Qb, $L^0(\frak A)$ is \wsid;
by 368Pa, any regularly embedded Riesz subspace is \wsid.
}%end of proof of 368S

\exercises{\leader{368X}{Basic exercises (a)}
%\spheader 368Xa
Let $X$ be an uncountable set and $\Sigma$ the countable-cocountable
$\sigma$-algebra of subsets of $X$.   Show that there is a family
$A\subseteq L^0=L^0(\Sigma)$ such that $u\wedge v=0$ for all distinct
$u$, $v\in A$ but $A$ has no upper bound in $L^0$.
Show moreover that if $w>0$ in $L^0$ then there is an $n\in\Bbb N$ such
that $nw\ne\sup_{u\in A}u\wedge nw$.
%368A

\spheader 368Xb\dvAnew{2009}
Let $U$ be a linear space, $\frak A$ a Dedekind complete
Boolean algebra, and $p:U\to L^0=L^0(\frak A)$ a function such that
$p(u+v)\le p(u)+p(v)$ and $p(\alpha u)=\alpha p(u)$ whenever $u$, $v\in U$
and $\alpha\ge 0$.   Suppose that $V\subseteq U$ is a linear subspace and
$T:V\to L^0$ is a linear operator such that $Tv\le p(v)$ for every
$v\in V$.   Show that there is a linear operator
$\tilde T:U\to L^0$, extending
$T$, such that $\tilde Tu\le p(u)$ for every $u\in U$.
\Hint{part A of the proof of 363R.}
%368B

\spheader 368Xc\dvAformerly{3{}68Xb}
Let $\frak A$ be any Boolean algebra, and
$\widehat{\frak A}$ its Dedekind completion (314U).   Show that
$L^{\infty}(\widehat{\frak A})$ can be identified with the Dedekind
completions of $S(\frak A)$ and $L^{\infty}(\frak A)$.
%368J

\spheader 368Xd\dvAformerly{3{}68Xc} Explain how to prove 368K from 368A.
%368K

\spheader 368Xe\dvAformerly{3{}68Xd} Show that any product of \wsid\ Riesz spaces is \wsid.
%368N

\spheader 368Xf\dvAformerly{3{}68Xe}
Let $\frak A$ be a Dedekind complete \wsid\ Boolean
algebra.   Show that a
set $A\subseteq L^0=L^0(\frak A)$ is order-bounded iff
$\sequencen{2^{-n}u_n}$ order*-converges to $0$ in $L^0$ whenever
$\sequencen{u_n}$ is a sequence in $A$.   \Hint{use 368A.   If $v>0$ and
$v=\sup_{u\in A}v\wedge 2^{-n}u$ for every $n$, we can find a $w>0$ and
a sequence $\sequencen{u_n}$ in $A$ such that $w\le 2^{-n}u_n$ for every
$n$.}
%368R

\spheader 368Xg\dvAformerly{3{}68Xf}
Give a direct proof of 368S, using the ideas of 322F,
but not relying on it or on 368Q.
%368S

\leader{368Y}{Further exercises (a)}
%\spheader 368Ya
(i) Use 364T-364U to show that if $X$ is any compact Hausdorff space
then $C(X)$ can be regarded as an order-dense Riesz subspace of
$L^0(\RO(X))$, where $\RO(X)$ is the regular open algebra of $X$.
(ii) Use 353M
to show that any Archimedean Riesz space with order unit can be embedded
as an order-dense Riesz subspace of some $L^0(\RO(X))$.   (iii) Let $U$
be an Archimedean Riesz space and $C\subseteq U^+$ a maximal disjoint
set, as in part (a) of the proof of 368E.   For $e\in C$ let $U_e$ be
the solid linear
subspace of $U$ generated by $e$, and let $V$ be the solid linear
subspace of $U$ generated by $C$.   Show that $V$ can be embedded as an
order-dense
Riesz subspace of $\prod_{e\in C}U_e$ and therefore in $\prod_{e\in
C}L^0(\RO(X_e))\cong L^0(\prod_{e\in C}\RO(X_e))$ for a suitable
family of regular open algebras $\RO(X_e)$.   (iv) Now use 368B to
complete a proof of 368E.
%368E

\spheader 368Yb Let $U$ be any Archimedean Riesz space.   Let $\Cal V$
be the family of pairs $(A,B)$ of non-empty subsets of $U$ such that $B$
is the set of upper bounds of $A$ and $A$ is the set of lower bounds of
$B$.
Show that $\Cal V$ can be given the structure of a Dedekind complete
Riesz
space defined by the formulae

\Centerline{$(A_1,B_1)+(A_2,B_2)=(A,B)$ iff $A_1+A_2\subseteq A$,
$B_1+B_2\subseteq B$,}

\Centerline{$\alpha(A,B)=(\alpha A,\alpha B)$ if $\alpha>0$,}

\Centerline{$(A_1,B_1)\le(A_2,B_2)$ iff $A_1\subseteq A_2$.}

\noindent Show that $u\mapsto(\ocint{-\infty,u},\coint{u,\infty})$
defines
an embedding of $U$ as an order-dense Riesz subspace of $\Cal V$, so
that
$\Cal V$ may be identified with the Dedekind completion of $U$.
%368J

\spheader 368Yc Work through the proof of 364T when $X$ is compact,
Hausdorff and extremally disconnected, and show that it is easier than
the general case.   Hence show that 368Yb can be used to shorten the
proof of 368E sketched in 368Ya.
%368J, 368Yb

\spheader 368Yd Let $U$ be a Riesz space.   Show that the following are
equiveridical:  (i) $U$ is isomorphic, as Riesz space, to $L^0(\frak A)$
for some Dedekind $\sigma$-complete Boolean algebra $\frak A$ (ii) $U$
is Dedekind $\sigma$-complete and has a weak order unit and whenever
$A\subseteq U^+$ is countable and disjoint then $A$ is
bounded above in $U$.
%368M

\spheader 368Ye Let $U$ be a \wsid\ Riesz space and $V$ a Riesz subspace
of $U$ which is {\it either} solid {\it or} order-dense.   Show that $V$
is \wsid.
%368P

\spheader 368Yf Show that $C([0,1])$ is not \wsid.   (Compare 316J.)
%368R

\spheader 368Yg Let $\frak A$ be a ccc \wsid\ Boolean algebra.
Suppose we have a double sequence
$\family{(i,j)}{\Bbb N\times\Bbb N}{a_{ij}}$ in
$\frak A$ such that $\sequence{j}{a_{ij}}$ order*-converges to $a_i$
in $\frak A$
for each $i$, while $\sequence{i}{a_i}$ order*-converges to $a$.
Show that there
is a strictly increasing sequence $\sequence{i}{n(i)}$ such that
$\sequence{i}{a_{i,n(i)}}$ order*-converges to $a$.
%368R

\spheader 368Yh Let $U$ be a \wsid\ Riesz space with the countable sup
property.   Suppose we have an order-bounded double sequence
$\family{(i,j)}{\Bbb N\times\Bbb N}{u_{ij}}$ in
$U$ such that $\sequence{j}{u_{ij}}$ order*-converges to $u_i$ in $U$
for each $i$, while $\sequence{i}{u_i}$ order*-converges to $u$.
Show that there
is a strictly increasing sequence $\sequence{i}{n(i)}$ such that
$\sequence{i}{u_{i,n(i)}}$ order*-converges to $u$.
%368R

\spheader 368Yi Let $\frak A$ be a ccc \wsid\ Dedekind complete Boolean
algebra.   Show that there is a topology on $L^0=L^0(\frak A)$ such that
the closure of any $A\subseteq L^0$ is precisely the set of
order*-limits of sequences in $A$.   (Cf.\ 367Yk.)
%368Yh 368R

\spheader 368Yj Let $U$ be a \wsid\ Riesz space and $f:U\to\Bbb R$ a
positive linear functional;  write $f_{\tau}$ for the component of $f$
in $U^{\times}$.   (i) Show that for any $u\in U^+$ there is an
upwards-directed $A\subseteq[0,u]$, with supremum $u$, such that
$f_{\tau}(u)=\sup_{v\in A}f(v)$.   (See 356Xe, 362D.)  (ii) Show that if
$f$ is strictly positive, so is $f_{\tau}$.   (Compare 391D.)
%368R
}%end of exercises

\cmmnt{
\Notesheader{368} 368A-368B are manifestations of a principle which will
reappear in \S375:   Dedekind complete $L^0$ spaces are in some
sense `maximal'.   If we have an order-dense subspace $U$ of such an
$L^0$, then any Archimedean Riesz space including $U$ as an order-dense
subspace can itself be embedded in $L^0$ (368B).   In fact this property
characterizes Dedekind complete $L^0$ spaces (368M).   Moreover, any
Archimedean Riesz space $U$ can be embedded in this way (368E);
by 368C, the $L^0$ space (though not the embedding) is unique up to
isomorphism.   If $U$ and $V$ are Archimedean Riesz spaces, each
embedded as an order-dense Riesz subspace of a Dedekind complete $L^0$
space, then
any order-continuous Riesz homomorphism from $U$ to $V$ extends uniquely
to the $L^0$ spaces (368B).   If one Dedekind complete $L^0$ space is
embedded as an
order-dense Riesz subspace of another, they must in fact be the same
(368D).   Thus we can say that every Archimedean Riesz space $U$ can be
extended to a Dedekind complete $L^0$ space, in a way which respects
order-continuous Riesz homomorphisms, and that this extension is
maximal, in that $U$ cannot be order-dense in any larger space.

The proof of 368E which I give is long because I am using a bare-hands
approach.   Alternative methods shift the burdens.   For instance, if we
take the trouble to develop a direct construction of the `Dedekind
completion' of a Riesz space (368Yb), then we need prove the theorem
only for Dedekind complete Riesz spaces.   A more substantial aid is the
representation theorem for Archimedean Riesz spaces with order unit
(353M);  I sketch an argument in 368Ya.   The drawback to this
approach is the proof of Theorem 364T, which seems to be quite as long
as the direct proof of 368E which I give here.   Of course we need 364T
only for compact Hausdorff spaces, which are usefully easier than the
general case (364U, 368Yc).

368G is a version of Ogasawara's representation theorem for Archimedean
Riesz spaces.   Both this and 368F can be regarded as expressions of the
principle that an Archimedean Riesz space is `nearly' a space of
functions.

I have remarked before on the parallels between the theories of Boolean
algebras and Archimedean Riesz spaces.   The notion of
`weak $(\sigma,\infty)$-distributivity' is one of the more striking
correspondences.   (Compare, for instance, 316Xi(i) with 368Pa.)   What is
really important to us, of course, is the fact that the function spaces
of measure theory are mostly \wsid, by 368S.   Of course this is easy to
prove directly (368Xg), but I think that the argument through 368Q gives
a better idea of what is really happening here.   Some of the features
of `order*-convergence', as defined in \S367, are related to weak
$(\sigma,\infty)$-distributivity (compare 367Yi, 367Yp);
in 368Yi I describe a topology which
can be thought of as an abstract version of the topology of convergence
in measure on the $L^0$ space of a $\sigma$-finite measure algebra
(367M).
}%end of comment

\discrpage

