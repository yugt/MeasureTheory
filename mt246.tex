\frfilename{mt246.tex}
\versiondate{17.11.06}
\loadeusm

\def\chaptername{Function spaces}
\def\sectionname{Uniform integrability}

\newsection{246}

The next topic is a fairly specialized one, but it is of great
importance, for different reasons, in both probability theory and
functional analysis, and it therefore seems worth while giving a proper
treatment straight away.

\vleader{72pt}{246A}{Definition} Let $(X,\Sigma,\mu)$ be a measure space.

\header{246Aa}(a) A set $A\subseteq{\eusm L}^1(\mu)$ is {\bf uniformly
integrable} if
for every $\epsilon>0$ we can find a set $E\in\Sigma$, of finite
measure, and an $M\ge 0$ such that

\Centerline{$\int(|f|-M\chi E)^+\le\epsilon$ for every $f\in A$.}

\header{246Ab}(b) A set $A\subseteq L^1(\mu)$ is {\bf uniformly
integrable} if for
every $\epsilon>0$ we can find a set $E\in\Sigma$, of finite measure,
and an $M\ge 0$ such that

\Centerline{$\int(|u|-M\chi E^{\ssbullet})^+\le\epsilon$ for every
$u\in A$.}

\leader{246B}{Remarks} \cmmnt{{\bf (a)} Recall the formulae from 241Ef:
$u^+=u\vee 0$, so $(u-v)^+=u-u\wedge v$.

\header{246Bb}{\bf (b)} The phrase `uniformly integrable' is not
particularly helpful.   But of course we can observe that for any
particular integrable function $f$, there are simple functions
approximating $f$ for $\|\,\|_1$ (242M), and such functions will be
bounded (in
modulus) by functions of the form $M\chi E$, with $\mu E<\infty$;  thus
singleton subsets of $\eusm L^1$ and $L^1$ are uniformly integrable.
A general uniformly integrable set of functions is one in which $M$ and
$E$ can be chosen uniformly over the set.

\medskip

}{\bf (c)}\cmmnt{ It will I hope be clear from the
definitions that} $A\subseteq\eusm L^1$
is uniformly integrable iff $\{f^{\ssbullet}:f\in A\}\subseteq L^1$ is
uniformly integrable.

\header{246Bd}{\bf (d)}\dvro{ If}{ There is a useful simplification in
the definition if} $\mu X<\infty$\dvro{}{ (in particular, if
$(X,\Sigma,\mu)$ is a
probability space).   In this case} a set $A\subseteq L^1(\mu)$ is
uniformly integrable iff

\Centerline{$\inf_{M\ge 0}\sup_{u\in A}\int(|u|-Me)^+=0$}

\noindent iff

\Centerline{$\lim_{M\to\infty}\sup_{u\in A}\int(|u|-Me)^+=0$,}

\noindent writing $e=\chi X^{\ssbullet}\in L^1(\mu)$.   \prooflet{(For
if $\sup_{u\in A}\int(|u|-M\chi E^{\ssbullet})^+\le\epsilon$, then
$\int(|u|-M'e)^+\le\epsilon$ for every $M'\ge M$.)}
\cmmnt{Similarly, }$A\subseteq\eusm L^1(\mu)$ is uniformly integrable
iff

\Centerline{$\lim_{M\to\infty}\sup_{f\in A}\int(|f|-M\chi X)^+
=0$}

\noindent iff

\Centerline{$\inf_{M\ge 0}\sup_{f\in A}\int(|f|-M\chi X)^+=0$.}

\cmmnt{\medskip

\noindent{\bf Warning!} Some authors use the phrase `uniformly
integrable' for sets satisfying the conditions in (d) even when $\mu$ is
not totally finite.}

\leader{246C}{}\cmmnt{ We have the following wide-ranging stability
properties of the class of uniformly integrable sets in $L^1$ or $\eusm
L^1$.

\medskip

\noindent}{\bf Proposition} Let $(X,\Sigma,\mu)$ be a measure space and
$A$ a uniformly integrable subset of $L^1(\mu)$.

(a) $A$ is bounded for the norm $\|\,\|_1$.

(b) Any subset of $A$ is uniformly integrable.

(c) For any $a\in\Bbb R$, $aA=\{au:u\in A\}$ is uniformly integrable.

(d) There is a uniformly integrable $C\supseteq A$ such that $C$ is
convex and $\|\,\|_1$-closed and $v\in C$ whenever $u\in C$ and
$|v|\le|u|$.

(e) If $B$ is another uniformly integrable subset of $L^1$, then
$A\cup B$ and $A+B=\{u+v:u\in A,\,v\in B\}$ are uniformly integrable.

\proof{ Write $\Sigma^f$ for $\{E:E\in\Sigma,\,\mu E<\infty\}$.

\medskip

{\bf (a)} There must be $E\in\Sigma^f$, $M\ge 0$ such that
$\int(|u|-M\chi E^{\ssbullet})^+\le 1$ for every $u\in A$;  now

\Centerline{$\|u\|_1\le\int(|u|-M\chi E^{\ssbullet})^+
+\int M\chi E^{\ssbullet}\le 1+M\mu E$}

\noindent for every $u\in A$, so $A$ is bounded.

\medskip

{\bf (b)} This is immediate from the definition 246Ab.

\medskip

{\bf (c)} Given $\epsilon>0$, we can find $E\in\Sigma^f$, $M\ge 0$ such
that $|a|\int_E(|u|-M\chi E^{\ssbullet})^+\le\epsilon$ for every $u\in
A$;  now $\int_E(|v|-|a|M\chi E^{\ssbullet})^+\le\epsilon$ for every
$v\in aA$.

\medskip

{\bf (d)} If $A$ is empty, take $C=A$.   Otherwise, try

\Centerline{$C=\{v:v\in L^1,\,\int(|v|-w)^+
\le\sup_{u\in A}\int(|u|-w)^+$
for every $w\in L^1(\mu)\}$.}

\noindent Evidently $A\subseteq C$, and $C$ satisfies the definition
246Ab because $A$ does, considering $w$ of the form $M\chi
E^{\ssbullet}$ where $E\in\Sigma^f$ and $M\ge 0$.   The
functionals

\Centerline{$v\mapsto\int(|v|-w)^+:L^1(\mu)\to\Bbb R$}

\noindent are all continuous for $\|\,\|_1$ (because the operators
$v\mapsto|v|$, $v\mapsto v-w$, $v\mapsto v^+$, $v\mapsto\int v$ are
continuous), so $C$ is closed.   If $|v'|\le|v|$ and $v\in C$, then

\Centerline{$\int(|v'|-w)^+\le\int(|v|-w)^+
\le\sup_{u\in A}\int(|u|-w)^+$}

\noindent for every $w$, and $v'\in C$.   If $v=av_1+bv_2$ where  $v_1$,
$v_2\in C$, $a\in[0,1]$ and $b=1-a$, then
$|v|\le a|v_1|+b|v_2|$, so

\Centerline{$|v|-w\le(a|v_1|-aw)+(b|v_2|-bw)
\le(a|v_1|-aw)^++(b|v_2|-bw)^+$}

\noindent and

\Centerline{$(|v|-w)^+\le a(|v_1|-w)^++b(|v_2|-w)^+$}

\noindent for every $w$;  accordingly

$$\eqalign{\int(|v|-w)^+
&\le a\int(|v_1|-w)^++b\int(|v_2|-w)^+\cr
&\le(a+b)\sup_{u\in A}\int(|u|-w)^+
=\sup_{u\in A}\int(|u|-w)^+\cr}$$

\noindent for every $w$, and $v\in C$.

Thus $C$ has all the required properties.

\medskip

{\bf (e)} I show first that $A\cup B$ is uniformly integrable.
\Prf\ Given $\epsilon>0$, let $M_1$, $M_2\ge 0$ and $E_1$,
$E_2\in\Sigma^f$ be such that

\Centerline{$\int(|u|-M_1\chi E_1^{\ssbullet})^+\le\epsilon$
for every $u\in A$,}

\Centerline{$\int(|u|-M_2\chi E_2^{\ssbullet})^+\le\epsilon$
for every $u\in B$.}

\noindent Set $M=\max(M_1,M_2)$, $E=E_1\cup E_2$;  then $\mu E<\infty$ and

\Centerline{$\int(|u|-M\chi E^{\ssbullet})^+\le\epsilon$
for every $u\in A\cup B$.}

\noindent As $\epsilon$ is arbitrary, $A\cup B$ is uniformly integrable.
\Qed

Now (d) tells us that there is a convex uniformly integrable set $C$
including $A\cup B$, and in this case $A+B\subseteq 2C$, so $A+B$ is
also uniformly integrable, using (b) and (c).
}%end of proof of 246C

\leader{246D}{Proposition} Let $(X,\Sigma,\mu)$ be a probability space
and $A\subseteq L^1(\mu)$ a uniformly integrable set.   Then there is a
convex, $\|\,\|_1$-closed uniformly integrable set $C\subseteq L^1$ such
that $A\subseteq C$, $w\in C$ whenever $v\in C$ and $|w|\le|v|$, and
$Pv\in C$ whenever $v\in C$ and $P$ is the conditional expectation
operator associated with a $\sigma$-subalgebra of $\Sigma$.

\proof{ Set

\Centerline{$C=\{v:v\in L^1(\mu),\,
  \int(|v|-Me)^+\le\sup_{u\in A}\int(|u|-Me)^+$ for every
$M\ge 0\}$,}

\noindent writing $e=\chi X^{\ssbullet}$ as usual.
The arguments in the proof of 246Cd make it plain that $C\supseteq A$ is
uniformly integrable, convex and closed, and that $w\in C$ whenever
$v\in C$ and $|w|\le|v|$.   As for the conditional expectation operators,
if $v\in C$, $\Tau$ is a $\sigma$-subalgebra of $\Sigma$, $P$ is the
associated conditional expectation operator, and $M\ge 0$, then

\Centerline{$|Pv|\le P|v|=P((|v|\wedge Me)+(|v|-Me)^+)
\le Me+P((|v|-Me)^+)$,}

\noindent so

\Centerline{$(|Pv|-Me)^+\le P((|v|-Me)^+)$}

\noindent and

\Centerline{$\int(|Pv|-Me)^+\le\int P(|v|-Me)^+=\int(|v|-Me)^+
\le\sup_{u\in A}\int(|u|-Me)^+$;}

\noindent as $M$ is arbitrary, $Pv\in C$.
}

\cmmnt{
\leader{246E}{Remarks (a)} Of course 246D has an expression in terms of
$\eusm L^1$ rather than $L^1$:  if $(X,\Sigma,\mu)$ is a probability
space and $A\subseteq\eusm L^1(\mu)$ is uniformly integrable, then there
is a uniformly integrable set $C\supseteq A$ such that  (i)
$af+(1-a)g\in C$ whenever $f$, $g\in C$
and $a\in[0,1]$ (ii) $g\in C$ whenever $f\in C$, $g\in\eusm L^0(\mu)$
and $|g|\leae|f|$ (iii) $f\in C$ whenever there is a sequence
$\sequencen{f_n}$ in $C$ such that $\lim_{n\to\infty}\int|f-f_n|=0$ (iv)
$g\in C$ whenever there is an $f\in C$ such that $g$ is a conditional
expectation of $f$ with respect to some $\sigma$-subalgebra of $\Sigma$.

\header{246Eb}{\bf (b)} In fact, there are obvious extensions of
246D;  the proof there already shows that $T[C]\subseteq C$ whenever
$T:L^1(\mu)\to L^1(\mu)$ is an order-preserving linear operator such
that $\|Tu\|_1\le\|u\|_1$ for every $u\in L^1(\mu)$ and
$\|Tu\|_{\infty}\le\|u\|_{\infty}$ for every $u\in L^1(\mu)\cap
L^{\infty}(\mu)$ (246Yc).
If we had done a bit more of the theory of operators on Riesz spaces I
should be able to take you a good deal farther along this road;  for
instance, it is not in fact necessary to assume that the operators $T$
of the last sentence are
order-preserving.   I will return to this in Chapter 37 in the next
volume.

\header{246Ec}{\bf (c)} Moreover, the main theorem of the next section
will show that for any measure spaces $(X,\Sigma,\mu)$, $(Y,\Tau,\nu)$,
$T[A]$ will be uniformly integrable in $L^1(\nu)$ whenever
$A\subseteq L^1(\mu)$ is uniformly integrable and
$T:L^1(\mu)\to L^1(\nu)$ is a continuous linear operator (247D).
}%end of comment

\leader{246F}{}\cmmnt{ We shall need an elementary lemma which I have
not so far spelt out.

\medskip

\noindent}{\bf Lemma} Let $(X,\Sigma,\mu)$ be a measure space.   Then
for
any $u\in L^1(\mu)$,

\Centerline{$\|u\|_1\le 2\sup_{E\in\Sigma}|\int_Eu|$.}

\proof{ Express $u$ as $f^{\ssbullet}$ where $f:X\to\Bbb R$ is
measurable.   Set $F=\{x:f(x)\ge 0\}$.   Then

\Centerline{$\|u\|_1=\int|f|=|\int_Ff|+|\int_{X\setminus F}f|
\le 2\sup_{E\in\Sigma}|\int_Ef|=2\sup_{E\in\Sigma}|\int_Eu|$.}
}

\leader{246G}{}\cmmnt{ Now we come to some of the remarkable
alternative descriptions of uniform integrability.

\medskip

\noindent}{\bf Theorem} Let $(X,\Sigma,\mu)$ be any measure space and
$A$ a non-empty subset of $L^1(\mu)$.   Then the following are
equiveridical:

\quad(i) $A$ is uniformly integrable;

\quad(ii) $\sup_{u\in A}|\int_Fu|<\infty$ for every $\mu$-atom
$F\in\Sigma$, and for every $\epsilon>0$ there
are $E\in\Sigma$, $\delta>0$ such that $\mu E<\infty$ and
$|\int_Fu|\le\epsilon$ whenever $u\in A$, $F\in\Sigma$ and
$\mu(F\cap E)\le\delta$;

\quad(iii) $\sup_{u\in A}|\int_Fu|<\infty$ for every $\mu$-atom
$F\in\Sigma$, and $\lim_{n\to\infty}\sup_{u\in A}|\int_{F_n}u|=0$
whenever $\sequencen{F_n}$ is a disjoint sequence in $\Sigma$;

\quad(iv) $\sup_{u\in A}|\int_Fu|<\infty$ for every $\mu$-atom
$F\in\Sigma$, and $\lim_{n\to\infty}\sup_{u\in A}|\int_{F_n}u|=0$
whenever $\sequencen{F_n}$ is a non-increasing sequence in $\Sigma$ with
empty intersection.

\cmmnt{\medskip

\noindent{\bf Remark} I use the phrase `$\mu$-atom' to emphasize that
I mean an atom in the measure space sense (211I).}

\proof{{\bf (a)(i)$\Rightarrow$(iv)} Suppose that $A$ is uniformly
integrable.  Then surely if $F\in\Sigma$ is a $\mu$-atom,

\Centerline{$\sup_{u\in A}|\int_Fu|
\le\sup_{u\in A}\|u\|_1<\infty$,}

\noindent by 246Ca.   Now suppose that $\sequencen{F_n}$ is a
non-increasing sequence in $\Sigma$ with empty intersection, and that
$\epsilon>0$.   Take $E\in\Sigma$, $M\ge 0$ such that $\mu E<\infty$ and
$\int(|u|-M\chi E^{\ssbullet})^+\le\bover12\epsilon$ whenever $u\in A$.
Then for all $n$ large enough, $M\mu(F_n\cap E)\le\bover12\epsilon$, so
that

\Centerline{$|\int_{F_n}u|\le\int_{F_n}|u|
\le\int(|u|-M\chi E^{\ssbullet})^++\int_{F_n}M\chi E^{\ssbullet}
\le\Bover{\epsilon}2+M\mu(F_n\cap E)\le\epsilon$}

\noindent for every $u\in A$.   As $\epsilon$ is arbitrary,
$\lim_{n\to\infty}\sup_{u\in A}|\int_{F_n}u|=0$, and (iv) is true.

\medskip

{\bf (b)(iv)$\Rightarrow$(iii)} Suppose that (iv) is true.   Then of
course $\sup_{u\in A}|\int_Fu|<\infty$ for every $\mu$-atom
$F\in\Sigma$.   \Quer\ Suppose, if possible, that $\sequencen{F_n}$ is a
disjoint sequence in $\Sigma$ such that
\discrcenter{468pt}{$\epsilon
=\limsup_{n\to\infty}\sup_{u\in A}\min(1,\bover13|\int_{F_n}u|)
>0$. }Set $H_n=\bigcup_{i\ge n}F_i$ for
each $n$, so that $\sequencen{H_n}$ is non-increasing and has empty
intersection, and $\int_{H_n}u\to 0$ as $n\to\infty$
for every $u\in L^1(\mu)$.
Choose $\sequence{i}{n_i}$, $\sequence{i}{m_i}$, $\sequence{i}{u_i}$
inductively, as follows.   $n_0=0$.   Given $n_i\in\Bbb N$, take $m_i\ge
n_i$, $u_i\in A$ such that $|\int_{F_{m_i}}u_i|\ge 2\epsilon$.   Take
$n_{i+1}>m_i$ such that $\int_{H_{n_{i+1}}}|u_i|\le\epsilon$.
Continue.

Set $G_k=\bigcup_{i\ge k}F_{m_i}$ for each $k$.   Then
$\sequence{k}{G_k}$ is a non-increasing sequence in $\Sigma$ with empty
intersection.   But $F_{m_i}\subseteq G_i\subseteq F_{m_i}\cup
H_{n_{i+1}}$, so

\Centerline{$|\int_{G_i}u_i|
\ge|\int_{F_{m_i}}u_i|-|\int_{G_i\setminus F_{m_i}}u_i|
\ge 2\epsilon-\int_{H_{n_{i+1}}}|u_i|
\ge\epsilon$}

\noindent for every $i$, contradicting the hypothesis (iv).  \Bang

This means that $\lim_{n\to\infty}\sup_{u\in A}|\int_{F_n}u|$ must be
zero, and (iii) is true.

\medskip

{\bf (c)(iii)$\Rightarrow$(ii)} We still have $\sup_{u\in
A}|\int_Fu|<\infty$ for every $\mu$-atom $F$.    \Quer\ Suppose, if
possible, that there is an $\epsilon>0$ such that for
every measurable set $E$ of finite measure and every $\delta>0$ there
are $u\in A$, $F\in\Sigma$ such that $\mu(F\cap E)\le\delta$ and
$|\int_Fu|\ge\epsilon$.   Choose a sequence $\sequencen{E_n}$ of sets of
finite measure, a sequence $\sequencen{G_n}$ in $\Sigma$, a sequence
$\sequencen{\delta_n}$ of strictly positive real numbers and a sequence
$\sequencen{u_n}$ in $A$ as follows.      Given $u_k$, $E_k$, $\delta_k$
for $k<n$, choose $u_n\in A$ and $G_n\in\Sigma$  such that
$\mu(G_n\cap\bigcup_{k<n}E_k)\le 2^{-n}\min(\{1\}\cup\{\delta_k:k<n\})$
and $|\int_{G_n}u_n|\ge\epsilon$;  then choose a set $E_n$ of finite
measure and a $\delta_n>0$ such that $\int_F|u_n|\le\bover12\epsilon$
whenever $F\in\Sigma$ and $\mu(F\cap E_n)\le\delta_n$ (see 225A).
Continue.

On completing the induction, set $F_n=E_n\cap
G_n\setminus\bigcup_{k>n}G_k$ for each $n$;  then $\sequencen{F_n}$ is a
disjoint sequence in $\Sigma$.   By the choice of $G_k$,

\Centerline{$\mu(E_n\cap\bigcup_{k>n}G_k)
\le\sum_{k=n+1}^{\infty}2^{-k}\delta_n\le\delta_n$,}

\noindent so $\mu(E_n\cap(G_n\setminus F_n))\le\delta_n$ and
$\int_{G_n\setminus F_n}|u_n|\le \bover12\epsilon$.   This means that
$|\int_{F_n}u_n|\ge|\int_{G_n}u_n|-\bover12\epsilon\ge\bover12\epsilon$.
But this is contrary to the hypothesis (iii).   \Bang

\medskip

{\bf (d)(ii)$\Rightarrow$(i)}\grheada\  Assume (ii).    Let
$\epsilon>0$.   Then there are $E\in\Sigma$, $\delta>0$ such that
$\mu E<\infty$ and
$|\int_Fu|\le\epsilon$ whenever $u\in A$, $F\in\Sigma$ and
$\mu(F\cap E)\le\delta$.   Now $\sup_{u\in A}\int_E|u|<\infty$.   \Prf\
Write $\Cal I$ for the family of those $F\in\Sigma$ such that
$F\subseteq E$ and $\sup_{u\in A}\int_F|u|$ is finite.   If
$F\subseteq E$ is an atom for $\mu$, then
$\sup_{u\in A}\int_F|u|=\sup_{u\in A}|\int_Fu|<\infty$, so $F\in\Cal I$.
(The point is that if $f:X\to\Bbb R$ is a measurable function such that
$f^{\ssbullet}=u$, then one of $F'=\{x:x\in F,\,f(x)\ge 0\}$,
$F''=\{x:x\in F,\,f(x)<0\}$ must be negligible, so that $\int_F|u|$ is
either $\int_{F'}u=\int_Fu$ or $-\int_{F''}u=-\int_Fu$.)   If
$F\in\Sigma$, $F\subseteq E$ and $\mu F\le\delta$ then

\Centerline{$\sup_{u\in A}\int_F|u|
\le 2\sup_{u\in A,G\in\Sigma,G\subseteq F}|\int_Gu|
\le 2\epsilon$}

\noindent (by 246F), so $F\in\Cal I$.   Next, if $F$, $G\in\Cal I$ then
$\sup_{u\in A}\int_{F\cup G}|u|
\le\sup_{u\in A}\int_F|u|+\sup_{u\in A}\int_G|u|$ is finite, so
$F\cup G\in \Cal I$.   Finally, if
$\sequencen{F_n}$ is any sequence in $\Cal I$, and
$F=\bigcup_{n\in\Bbb N}F_n$, there is some $n\in\Bbb N$ such that
$\mu(F\setminus\bigcup_{i\le n}F_i)\le\delta$;  now
$\bigcup_{i\le n}F_i$ and
$F\setminus\bigcup_{i\le n}F_i$ both belong to $\Cal I$, so
$F\in\Cal I$.

By 215Ab, there is an $F\in\Cal I$ such that $H\setminus F$ is
negligible for every $H\in\Cal I$.   Now observe that $E\setminus F$
cannot include any non-negligible member of $\Cal I$;  in particular,
cannot include either an atom or a non-negligible set of measure less
than $\delta$.   But this means that the subspace measure on
$E\setminus F$ is atomless, totally finite and has no non-negligible
measurable sets
of measure less than $\delta$;  by 215D, $\mu(E\setminus F)=0$ and
$E\setminus F$ and $E$ belong to $\Cal I$, as required.\ \Qed

Since $\int_{X\setminus E}|u|\le\delta$ for every $u\in A$,
$\gamma=\sup_{u\in A}\int|u|$ is finite.

\medskip

\quad\grheadb\
Set $M=\gamma/\delta$.   If $u\in A$, express $u$ as $f^{\ssbullet}$,
where $f:X\to\Bbb R$ is measurable, and consider

\Centerline{$F=\{x:f(x)\ge M\chi E(x)\}$.}

\noindent Then

\Centerline{$M\mu(F\cap E)\le\int_Ff=\int_Fu\le\gamma$,}

\noindent so $\mu(F\cap E)\le\gamma/M=\delta$.   Accordingly
$\int_Fu\le\epsilon$.   Similarly, $\int_{F'}(-u)\le\epsilon$, writing
$F'=\{x:-f(x)\ge M\chi E(x)\}$.   But this means that

\Centerline{$\int(|u|-M\chi E^{\ssbullet})^+
=\int(|f|-M\chi E)^+\le\int_{F\cup F'}|f|
=\int_{F\cup F'}|u|\le 2\epsilon$}

\noindent for every $u\in A$.  As $\epsilon$ is arbitrary, $A$ is
uniformly integrable.
}%end of proof of 246G

\cmmnt{
\leader{246H}{Remarks (a)} Of course conditions (ii)-(iv) of this
theorem, like (i), have direct translations in terms of members of
$\eusm L^1$.   Thus a non-empty set $A\subseteq\eusm L^1$ is uniformly
integrable iff $\sup_{f\in A}|\int_Ff|$ is finite for every atom
$F\in\Sigma$ and

\inset{{\it either} for every $\epsilon>0$ we can
find $E\in\Sigma$, $\delta>0$ such that $\mu E<\infty$ and
$|\int_Ff|\le\epsilon$ whenever $f\in A$, $F\in\Sigma$ and $\mu(F\cap
E)\le\delta$
}

\inset{{\it or} $\lim_{n\to\infty}\sup_{f\in A}|\int_{F_n}f|=0$ for
every disjoint sequence $\sequencen{F_n}$ in $\Sigma$}

\inset{{\it or} $\lim_{n\to\infty}\sup_{f\in A}|\int_{F_n}f|=0$ for
every non-increasing sequence $\sequencen{F_n}$ in $\Sigma$ with empty
intersection.}

\header{246Hb}{\bf (b)} There are innumerable further equivalent
expressions
characterizing uniform integrability;  every author has his own
favourite.   Many of them are variants on (i)-(iv) of this theorem, as
in 246I and 246Yd-246Yf.    For a condition of a quite different kind,
see Theorem 247C.
}%end of comment

\leader{246I}{Corollary} Let $(X,\Sigma,\mu)$ be a probability space.
For $f\in\eusm L^0(\mu)$, $M\ge 0$ set
$F(f,M)=\{x:x\in\dom f,\,|f(x)|\ge M\}$.
Then a non-empty set $A\subseteq\eusm L^1(\mu)$ is
uniformly integrable iff
\discrcenter{468pt}{$\lim_{M\to\infty}\sup_{f\in A}\int_{F(f,M)}|f|=0$.}

\proof{{\bf (a)} If $A$ satisfies the condition, then

\Centerline{$\inf_{M\ge 0}\sup_{f\in A}\int(|f|-M\chi X)^+
\le\inf_{M\ge 0}\sup_{f\in A}\int_{F(f,M)}|f|=0$,}

\noindent so $A$ is uniformly integrable.

\medskip

{\bf (b)} If $A$ is uniformly integrable, and $\epsilon>0$, there is an
$M_0\ge 0$ such that $\int(|f|-M_0\chi X)^+\le\epsilon$ for every $f\in
A$;  also, $\gamma=\sup_{f\in A}\int|f|$ is finite (246Ca).   Take any
$M\ge M_0\max(1,(1+\gamma)/\epsilon)$.   If $f\in A$, then

\Centerline{$|f|\times\chi F(f,M)
\le (|f|-M_0\chi X)^++M_0\chi F(f,M)
\le (|f|-M_0\chi X)^++\Bover{\epsilon}{\gamma+1}|f|$}

\noindent everywhere on $\dom f$, so

\Centerline{$\int_{F(f,M)}|f|
\le\int(|f|-M_0\chi X)^++\Bover{\epsilon}{\gamma+1}\int|f|
\le 2\epsilon$.}

\noindent As $\epsilon$ is arbitrary, $\lim_{M\to\infty}\sup_{f\in
A}\int_{F(f,M)}|f|=0$.
}%end of proof of 246I

\leader{246J}{}\cmmnt{ The next step is to set out some remarkable
connexions
between uniform integrability and the topology of convergence in measure
discussed in the last section.

\medskip

\noindent}{\bf Theorem}
Let $(X,\Sigma,\mu)$ be a measure space.

\ifdim\pagewidth>467pt\fontdimen4\tenrm=1.5pt\fi
(a) If $\sequencen{f_n}$ is a uniformly integrable sequence of
real-valued functions on $X$, and $f(x)=\lim_{n\to\infty}f_n(x)$ for
almost every $x\in X$, then $f$ is integrable and
$\lim_{n\to\infty}\int|f_n-f|=0$;  consequently
$\int f=\lim_{n\to\infty}\int f_n$.
\fontdimen4\tenrm=1.11pt

(b) If $A\subseteq L^1=L^1(\mu)$ is uniformly integrable, then the norm
topology of $L^1$ and the topology of convergence in measure of
$L^0=L^0(\mu)$ agree on $A$.

(c) For any $u\in L^1$ and any sequence $\sequencen{u_n}$ in $L^1$, the
following are equiveridical:

\qquad(i) $u=\lim_{n\to\infty}u_n$ for $\|\,\|_1$;

\qquad(ii) $\{u_n:n\in\Bbb N\}$ is uniformly integrable and
$\sequencen{u_n}$ converges to $u$ in measure.

(d) If $(X,\Sigma,\mu)$ is semi-finite, and $A\subseteq L^1$ is
uniformly integrable, then the closure $\overline{A}$ of $A$ in $L^0$
for the topology of convergence in measure is still a uniformly
integrable subset of $L^1$.

\proof{{\bf (a)} Note first that because $\sup_{n\in\Bbb
N}\int|f_n|<\infty$ (246Ca) and $|f|=\liminf_{n\to\infty}|f_n|$,
Fatou's Lemma assures us that $|f|$ is integrable, with
$\int|f|\le\limsup_{n\to\infty}\int|f_n|$.
It follows immediately that
$\{f_n-f:n\in\Bbb N\}$ is
uniformly integrable, being the sum of two uniformly integrable
sets (246Cc, 246Ce).

Given $\epsilon>0$, there are $M\ge 0$, $E\in\Sigma$ such that $\mu
E<\infty$ and $\int(|f_n-f|-M\chi E)^+\le\epsilon$ for every $n\in\Bbb
N$.   Also $|f_n-f|\wedge M\chi E\to 0$ a.e., so

$$\eqalign{\limsup_{n\to\infty}\int|f_n-f|
&\le\limsup_{n\to\infty}\int(|f_n-f|-M\chi E)^+\cr
&\qquad\qquad+\limsup_{n\to\infty}\int|f_n-f|\wedge M\chi E\cr
&\le\epsilon,\cr}$$

\noindent by Lebesgue's Dominated Convergence Theorem.   As $\epsilon$
is arbitrary, $\lim_{n\to\infty}\int|f_n-f|=0$ and
$\lim_{n\to\infty}\int f_n-f=0$.

\medskip

{\bf (b)} Let $\frak T_A$, $\frak S_A$ be the topologies on $A$ induced
by the norm topology of $L^1$ and the topology of convergence in measure
on $L^0$ respectively.

\medskip

\quad{\bf (i)} Given $\epsilon>0$,   let $F\in\Sigma$, $M\ge 0$ be such
that $\mu F<\infty$ and
$\int(|v|-M\chi F^{\ssbullet})^+\le\epsilon$ for every $v\in A$,
and consider $\bar\rho_F$, defined as in 245A.   Then for any $f$,
$g\in\eusm L^0$,

\Centerline{$|f-g|\le(|f|-M\chi F)^+
+(|g|-M\chi F)^++M(|f-g|\wedge \chi F)$}

\noindent everywhere on $\dom f\cap \dom g$, so

\Centerline{$|u-v|\le(|u|-M\chi F^{\ssbullet})^+
+(|v|-M\chi F^{\ssbullet})^++M(|u-v|\wedge\chi F^{\ssbullet})$}

\noindent for all $u$, $v\in L^0$.   Consequently

\Centerline{$\|u-v\|_1\le 2\epsilon+M\bar\rho_F(u,v)$}

\noindent for all $u$, $v\in A$.

This means that, given $\epsilon>0$, we can find $F$, $M$ such that, for
$u$, $v\in A$,

\Centerline{$\bar\rho_F(u,v)
\le\Bover{\epsilon}{1+M}\Longrightarrow\|u-v\|_1\le 3\epsilon$.}

\noindent It follows that every subset of $A$ which is open for
$\frak T_A$ is open for $\frak S_A$ (2A3Ib).

\medskip

\quad{\bf (ii)} In the other direction, we have
$\bar\rho_F(u,v)\le\|u-v\|_1$
for every $u\in L^1$ and every set $F$ of finite measure, so every
subset of $A$ which is open for $\frak S_A$ is open for $\frak T_A$.

\medskip

{\bf (c)} If $\sequencen{u_n}\to u$ for $\|\,\|_1$,
$A=\{u_n:n\in\Bbb N\}$ is uniformly integrable.   \Prf\ Given
$\epsilon>0$, let $m$ be such that $\|u_n-u\|_1\le\epsilon$ whenever
$n\ge m$.   Set $v=|u|+\sum_{i\le m}|u_i|\in L^1$, and let $M\ge 0$,
$E\in\Sigma$ be such that $\mu E$ is finite and
$\int_E(v-M\chi E^{\ssbullet})^+\le\epsilon$.   Then, for $w\in A$,

\Centerline{$(|w|-M\chi E^{\ssbullet})^+
\le(|w|-v)^++(v-M\chi E^{\ssbullet})^+$,}

\noindent so

\Centerline{$\int_E(|w|-M\chi E^{\ssbullet})^+
\le\|(|w|-v)^+\|_1+\int_E(v-M\chi E^{\ssbullet})^+
\le 2\epsilon$. \Qed}

Thus on either hypothesis we can be sure that $\{u_n:n\in\Bbb N\}$ and
$A=\{u\}\cup\{u_n:n\in\Bbb N\}$ are uniformly integrable, so that the
two topologies agree on $A$ (by (b)) and $\sequencen{u_n}$ converges to
$u$ in one topology iff it converges to $u$ in the other.

\medskip

{\bf (d)} Because $A$ is $\|\,\|_1$-bounded (246Ca) and $\mu$ is
semi-finite, $\overline{A}\subseteq L^1$ (245J(b-i)).   Given
$\epsilon>0$, let $M\ge 0$, $E\in\Sigma$ be such that $\mu E<\infty$ and
$\int(|u|-M\chi E^{\ssbullet})^+\le\epsilon$ for every $u\in A$.   Now
the maps $u\mapsto|u|$, $u\mapsto u-M\chi E^{\ssbullet}$,
$u\mapsto u^+:L^0\to L^0$
are all continuous for the topology of convergence in
measure (245D), while $\{u:\|u\|_1\le\epsilon\}$ is closed for the same
topology (245J again), so $\{u:u\in L^0,\,\int(|u|-M\chi
E^{\ssbullet})^+\le\epsilon\}$ is closed and must include
$\overline{A}$.   Thus $\int(|u|-M\chi E^{\ssbullet})^+\le\epsilon$ for
every $u\in\overline{A}$.   As $\epsilon$ is arbitrary, $\overline{A}$
is uniformly integrable.
}%end of proof of 246J

\leader{246K}{Complex $\eusm L^1$ and $L^1$}\dvro{ For}{ The definitions
and theorems above can be repeated without difficulty for spaces of
(equivalence classes of) complex-valued functions, with just one
variation:  in the complex equivalent of 246F, the constant must be
changed.   It is easy to see that, for} $u\in L^1_{\Bbb C}(\mu)$,
\dvro{$\|u\|_1\le 4\sup_{F\in\Sigma}|\int_Fu|$.}{

$$\eqalign{\|u\|_1
&\le \|\Real(u)\|_1+\|\Imag(u)\|_1\cr
&\le 2\sup_{F\in\Sigma}|\int_F\Real(u)|
   +2\sup_{F\in\Sigma}|\int_F\Imag(u)|
\le 4\sup_{F\in\Sigma}|\int_Fu|.\cr}$$
}

\cmmnt{\noindent (In fact, $\|u\|_1\le\pi\sup_{F\in\Sigma}|\int_Fu|$;
see 246Yl and 252Yt.)
Consequently some of the arguments of 246G need to be written out with
different constants, but the results, as stated, are unaffected.
}%end of comment

\exercises{
\leader{246X}{Basic exercises (a)}
%\spheader 246Xa
Let $(X,\Sigma,\mu)$ be a measure space and $A$ a subset of
$L^1=L^1(\mu)$.   Show that the following are equiveridical:  (i) $A$ is
uniformly integrable;  (ii) for every $\epsilon>0$ there is a $w\ge 0$
in $L^1$ such that $\int(|u|-w)^+\le\epsilon$ for every $u\in A$;
(iii) $\sequencen{(|u_{n+1}|-\sup_{i\le n}|u_i|)^+}\to 0$ in $L^1$ for
every sequence $\sequencen{u_n}$ in $A$.   \Hint{for
(ii)$\Rightarrow$(iii), set $v_n=\sup_{i\le n}|u_i|$ and note that
$\sequencen{v_n\wedge w}$ is convergent in $L^1$ for every $w\ge 0$.}

\sqheader 246Xb
Let $(X,\Sigma,\mu)$ be a totally finite measure space.   Show
that for any $p>1$ and $M\ge 0$ the set
$\{f:f\in{\eusm L}^p(\mu),\,\|f\|_p\le M\}$ is uniformly integrable.
\Hint{$\int(|f|-M\chi X)^+\le M^{1-p}\int|f|^p$.}
%246B used in \zz5R query

\sqheader 246Xc Let $\mu$ be counting
measure on $\Bbb N$.   Show that a set $A\subseteq\eusm L^1(\mu)=\ell^1$
is uniformly integrable iff (i) $\sup_{f\in A}|f(n)|<\infty$ for every
$n\in \Bbb N$ (ii) for every $\epsilon>0$ there is an $m\in\Bbb N$ such
that $\sum_{n=m}^{\infty}|f(n)|\le\epsilon$ for every $f\in A$.
%246G

\spheader 246Xd Let $X$ be a set, and let $\mu$ be counting
measure on $X$.   Show that a set $A\subseteq\eusm L^1(\mu)=\ell^1(X)$
is uniformly integrable iff (i) $\sup_{f\in A}|f(x)|<\infty$ for every
$x\in X$ (ii) for every $\epsilon>0$ there is a finite set $I\subseteq
X$ such that $\sum_{x\in X\setminus I}|f(x)|\le\epsilon$ for every $f\in
A$.   Show that in this case $A$ is relatively compact for the norm
topology of $\ell^1(X)$.
%246G

\spheader 246Xe Let $(X,\Sigma,\mu)$ be a measure space, $\delta>0$, and
$\Cal I\subseteq\Sigma$ a family such that (i) every atom belongs to
$\Cal I$ (ii) $E\in\Cal I$ whenever $E\in\Sigma$ and $\mu E\le\delta$
(iii) $E\cup F\in\Cal I$ whenever $E$, $F\in\Cal I$ and $E\cap
F=\emptyset$.   Show that every set of finite measure belongs to
$\Cal I$.
%246G

\spheader 246Xf Let $(X,\Sigma,\mu)$ and $(Y,\Tau,\nu)$ be measure
spaces and $\phi:X\to Y$ an \imp\ function.   Show that a set
$A\subseteq\eusm L^1(\nu)$ is uniformly integrable iff $\{g\phi:g\in
A\}$ is uniformly integrable in $\eusm L^1(\mu)$.   \Hint{use 246G for
`if', 246A for `only if'.}
%246G

\sqheader 246Xg Let $(X,\Sigma,\mu)$ be a measure space and
$p\in\coint{1,\infty}$.   Let $\sequencen{f_n}$ be a sequence in
$\eusm L^p=\eusm L^p(\mu)$ such that $\{|f_n|^p:n\in\Bbb N\}$ is
uniformly
integrable and $f_n\to f$ a.e.   Show that $f\in\eusm L^p$ and
$\lim_{n\to\infty}\int|f_n-f|^p=0$.
%246J

\spheader 246Xh Let $(X,\Sigma,\mu)$ be a semi-finite measure space and
$p\in\coint{1,\infty}$.   Let $\sequencen{u_n}$ be a sequence in
$L^p=L^p(\mu)$ and $u\in L^0(\mu)$.   Show that the following are
equiveridical:  (i) $u\in L^p$ and $\sequencen{u_n}$ converges to $u$ for
$\|\,\|_p$ (ii) $\sequencen{u_n}$ converges in measure to $u$ and
$\{|u_n|^p:n\in\Bbb N\}$ is uniformly integrable.   \Hint{245Xl.}
%246J

\spheader 246Xi Let $(X,\Sigma,\mu)$ be a totally finite measure space,
and $1\le p<r\le\infty$.   Let $\sequencen{u_n}$ be a
$\|\,\|_r$-bounded sequence in $L^r(\mu)$ which converges in measure to
$u\in L^0(\mu)$.   Show that $\sequencen{u_n}$ converges to $u$ for
$\|\,\|_p$.   \Hint{show that $\{|u_n|^p:n\in\Bbb N\}$ is uniformly
integrable.}
%246J, 246Xh

\leader{246Y}{Further exercises (a)}
%\spheader 246Ya
\ifdim\pagewidth>467pt\fontdimen4\tenrm=1.5pt\fi
Let $(X,\Sigma,\mu)$ be a totally finite measure space.   Show that
$A\subseteq\eusm L^1(\mu)$ is uniformly integrable iff there is a convex
function $\phi:\coint{0,\infty}\to\Bbb R$ such that
$\lim_{a\to\infty}\phi(a)/a=\infty$ and
$\sup_{f\in A}\int\phi(|f|)<\infty$.
\fontdimen4\tenrm=1.11pt
%246B

\spheader 246Yb For any metric space $(Z,\rho)$, let $\Cal C_Z$ be the
family of closed subsets of $Z$, and for $F$,
$F'\in\Cal C_Z\setminus\{\emptyset\}$ set
$\tilde\rho(F,F')=\min(1,\max(\sup_{z\in F}\inf_{z'\in F'}\rho(z,z'),
\sup_{z'\in F'}\inf_{z\in F}\rho(z,z')))$.
Show that $\tilde\rho$ is a metric on $\Cal C_Z\setminus\{\emptyset\}$
(it is the {\bf
Hausdorff metric}).    Show that if $(Z,\rho)$ is complete then the family $\Cal K_Z\setminus\{\emptyset\}$ of non-empty compact
subsets of $Z$ is closed for $\tilde\rho$.   Now let $(X,\Sigma,\mu)$ be
any measure space and take $Z=L^1=L^1(\mu)$, $\rho(z,z')=\|z-z'\|_1$ for
$z$, $z'\in Z$.   Show that the family of non-empty closed uniformly
integrable subsets of $L^1$ is a closed subset of
$\Cal C_Z\setminus\{\emptyset\}$ including
$\Cal K_Z\setminus\{\emptyset\}$.
%246B

\spheader 246Yc Let $(X,\Sigma,\mu)$ be a totally finite measure space
and $A\subseteq L^1(\mu)$ a uniformly integrable set.   Show that there
is a uniformly integrable set $C\supseteq A$ such that (i) $C$ is convex
and closed in $L^0(\mu)$ for the topology of convergence in measure (ii)
if $u\in C$ and $|v|\le|u|$ then $v\in C$ (iii) if $T$ belongs to the
set $\Cal T^+$ of operators from $L^1(\mu)=M^{1,\infty}(\mu)$ to itself,
as described in 244Xm, then $T[C]\subseteq C$.
%246C

%\ifdim\pagewidth>467pt\fontdimen4\tenrm=1.5pt\fi
\spheader 246Yd Let $\mu$ be Lebesgue measure on
$\Bbb R$.   Show that a set $A\subseteq\eusm L^1(\mu)$ is uniformly
integrable iff $\lim_{n\to\infty}\int_{F_n}f_n
\ifdim\pagewidth>467pt\penalty-100\fi
=0$ for every disjoint
sequence $\sequencen{F_n}$ of compact sets in $\Bbb R$ and every
sequence $\sequencen{f_n}$ in $A$.
%246G
%\fontdimen4\tenrm=1.11pt

\spheader 246Ye Let $\mu$ be Lebesgue measure on
$\Bbb R$.   Show that a set $A\subseteq\eusm L^1(\mu)$ is uniformly
integrable iff $\lim_{n\to\infty}\int_{G_n}f_n
\ifdim\pagewidth>467pt\penalty-100\fi
=0$ for every disjoint
sequence $\sequencen{G_n}$ of open sets in $\Bbb R$ and every sequence
$\sequencen{f_n}$ in $A$.
%246G, 246Yd

\spheader 246Yf Repeat 246Yd and 246Ye for Lebesgue measure on
arbitrary subsets of $\BbbR^r$.
%246G, 246Yd, 246Ye

\spheader 246Yg Let $X$ be a set and $\Sigma$ a $\sigma$-algebra of
subsets of $X$.   Let $\sequencen{\nu_n}$ be a sequence of countably
additive functionals on $\Sigma$ such that
$\nu E=\lim_{n\to\infty}\nu_nE$ is defined for every $E\in\Sigma$.
Show that $\lim_{n\to\infty}\nu_nF_n=0$ whenever $\sequencen{F_n}$ is a
disjoint sequence in $\Sigma$.   \Hint{suppose otherwise.   By taking
suitable subsequences reduce to the case in which
$|\nu_nF_i-\nu F_i|\le 2^{-n}\epsilon$ for $i<n$,
$|\nu_nF_n|\ge 3\epsilon$,
$|\nu_nF_i|\le 2^{-i}\epsilon$ for $i>n$.   Set
$F=\bigcup_{i\in\Bbb N}F_{2i+1}$ and show that
$|\nu_{2n+1}F-\nu_{2n}F|\ge\epsilon$ for every $n$.}
%246G %231Xf

\spheader 246Yh Let $(X,\Sigma,\mu)$ be a measure space and
$\sequencen{u_n}$ a sequence in $L^1=L^1(\mu)$ such that
$\lim_{n\to\infty}\int_Fu_n$ is defined for every $F\in\Sigma$.   Show
that $\{u_n:n\in\Bbb N\}$ is uniformly integrable.   \Hint{suppose not.
Then there are a disjoint sequence $\sequencen{F_n}$ in $\Sigma$ and a
subsequence $\sequencen{u'_n}$ of $\sequencen{u_n}$ such that
$\inf_{n\in\Bbb N}|\int_{F_n}u'_n|=\epsilon>0$.   But this contradicts
246Yg.}
%246G

\spheader 246Yi In 246Yg, show that $\nu$ is countably additive.
\Hint{Set $\mu=\sum_{n=0}^{\infty}a_n\nu_n$ for a suitable sequence
$\sequencen{a_n}$ of strictly positive numbers.   For each $n$ choose a
Radon-Nikod\'ym derivative $f_n$ of $\nu_n$ with respect to $\mu$.
Show that $\{f_n:n\in\Bbb N\}$ is uniformly integrable, so that $\nu$ is
truly continuous.}(This is the {\bf Vitali-Hahn-Saks theorem}.)
%246G

\spheader 246Yj Let $(X,\Sigma,\mu)$ be any measure space, and
$A\subseteq L^1(\mu)$.   Show that the following are equiveridical:  (i)
$A$ is $\|\,\|_1$-bounded;   (ii) $\sup_{u\in A}|\int_Fu|<\infty$
for every $\mu$-atom $F\in\Sigma$ and $\limsup_{n\to\infty}\sup_{u\in
A}|\int_{F_n}u|<\infty$ for every disjoint sequence
$\sequencen{F_n}$ of measurable sets of finite measure;  (iii)
$\sup_{u\in A}|\int_Eu|<\infty$ for every $E\in\Sigma$.   \Hint{show
that $\sequencen{a_nu_n}$ is uniformly integrable whenever
$\lim_{n\to\infty}a_n=0$ in $\Bbb R$ and $\sequencen{u_n}$ is a sequence
in $A$.}

%246G

\spheader 246Yk Let $(X,\Sigma,\mu)$ be a measure space and $A\subseteq
L^1(\mu)$ a non-empty set.   Show that the following are equiveridical:
(i) $A$ is uniformly integrable;  (ii) whenever $B\subseteq
L^{\infty}(\mu)$ is non-empty and downwards-directed and has infimum $0$
in $L^{\infty}(\mu)$ then $\inf_{v\in B}\sup_{u\in A}|\int u\times
v|=0$.   \Hint{for (i)$\Rightarrow$(ii), note that $\inf_{v\in B}w\times
v=0$ for every $w\ge 0$ in $L^0$.   For (ii)$\Rightarrow$(i), use
246G(iv).}
%246G

\spheader 246Yl Set $f(x)=e^{ix}$ for $x\in[-\pi,\pi]$.   Show that
$|\int_Ef|\le 2$ for every $E\subseteq[-\pi,\pi]$.
%246K
}%end of exercises

\endnotes{\Notesheader{246} I am holding over to the next section the
most striking property of uniformly integrable sets (they are the
relatively weakly compact sets in $L^1$) because this demands some
non-trivial ideas from functional analysis and general topology.   In
this section I give the results which can be regarded as essentially
measure-theoretic in inspiration.   The most important new concept, or
technique, is that of `disjoint-sequence theorem'.   A typical
example is in condition (iii) of 246G, relating uniform integrability to
the behaviour of functionals on disjoint sequences of sets.   I give
variants of this in 246Yd-246Yf, and 246Yg-246Yj are further results in
which similar methods can be used.   The central result of the next
section (247C) will also use disjoint sequences in the proof, and they
will appear more than once in Chapter 35 in the next volume.

The phrase `uniformly integrable' ought to mean something like
`uniformly approximable by simple functions', and the definition 246A
can be forced into such a form, but I do not think it very useful to do
so.   However condition (ii) of 246G amounts to something like
`uniformly truly continuous', if we think of members of $L^1$ as truly
continuous functionals on $\Sigma$, as in 242I.   (See 246Yi.)   Note
that in each
of the statements (ii)-(iv) of 246G we need to take special note of any
atoms for the measure, since they are not controlled by the main
condition imposed.
In an atomless measure space, of course, we have a simplification here,
as in 246Yd-246Yf.

Another way of justifying the `uniformly' in `uniformly
integrable' is by considering functionals $\theta_w$ where $w\ge 0$ in
$L^1$, setting $\theta_w(u)=\int(|u|-w)^+$ for $u\in L^1$;  then
$A\subseteq L^1$ is uniformly integrable iff $\theta_w\to 0$ uniformly
on $A$ as $w$ rises in $L^1$ (246Xa).   It is sometimes useful to know
that if this is true at all then it is necessarily witnessed by elements
$w$ which can be built directly from materials at hand (see (iii) of
246Xa).   Furthermore, the sets
$A_{w\epsilon}=\{u:\theta_w(u)\le\epsilon\}$ are always convex,
$\|\,\|_1$-closed and `solid' (if $u\in A_{w\epsilon}$ and $|v|\le
|u|$ then $v\in A_{w\epsilon}$)(246Cd);  they are closed under pointwise
convergence of sequences (246Ja) and in semi-finite measure spaces they
are closed
for the topology of convergence in measure (246Jd);  in probability
spaces, for level $w$, they are closed under conditional expectations
(246D) and similar operators (246Yc).   Consequently we can expect that
any uniformly integrable set will be included in a uniformly integrable
set which is closed under operations of several different types.

Yet another `uniform' property of uniformly integrable sets is in 246Yk.
The norm $\|\,\|_{\infty}$ is never (in interesting cases)
order-continuous in the way that other $\|\,\|_p$ are (244Ye);  but the
uniformly integrable subsets of $L^1$ provide interesting
order-continuous seminorms on $L^{\infty}$.

246J supplements results from \S245.   In the notes to that section I
mentioned the question:  if $\sequencen{f_n}\to f$ a.e., in what ways
can $\sequencen{\int f_n}$ fail to converge to $\int f$?   Here we find
that $\sequencen{\int|f_n-f|}\to 0$ iff $\{f_n:n\in\Bbb N\}$ is
uniformly integrable;  this is a way of making precise the expression
`none of the weight of the sequence is lost at infinity'.   Generally,
for sequences, convergence in $\|\,\|_p$, for $p\in\coint{1,\infty}$, is
convergence in measure for
$p$th-power-uniformly-integrable sequences (246Xh).




}%end of notes

\discrpage

