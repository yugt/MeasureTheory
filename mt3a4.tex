\frfilename{mt3a4.tex}
\versiondate{30.1.08}
\copyrightdate{1996}

\def\chaptername{Appendix}
\def\sectionname{Uniformities}

%\def\ref{\medskip\noindent < \medskip}
\def\Bourbaki{{\smc Bourbaki 66}}
\def\Dugundji{{\smc Dugundji 66}}
\def\Engelking{{\smc Engelking 89}}
\def\Gaal{{\smc Gaal 64}}
\def\James{{\smc James 87}}
%\def\Kelley{{\smc Kelley 55}}
\def\Schaefer{{\smc Schaefer 66}}
\def\Schubert{{\smc Schubert 68}}

\newsection{3A4}

I continue the work of \S3A3 with some notes on uniformities, so as to
be able to discuss completeness and the extension of uniformly
continuous functions in non-metrizable contexts (3A4F-3A4H).
\cmmnt{As in \S3A3, most of the individual steps are elementary;  I
mark exceptions with a $*$.}%end of comment

\leader{3A4A}{Uniformities (a)} Let $X$ be a set.   A {\bf uniformity}
on $X$ is a filter $\Cal W$ on $X\times X$ such that

\quad(i) $(x,x)\in W$ for every $x\in X$, $W\in\Cal W$;

\quad(ii) for every $W\in\Cal W$,
$W^{-1}=\{(y,x):(x,y)\in W\}\in\Cal W$;

\quad(iii) for every $W\in\Cal W$, there is a $V\in\Cal W$ such that

\Centerline{$V\frsmallcirc V
=\{(x,z):\,\Exists y,\,(x,y)\in V\,\&\,(y,z)\in V\}\subseteq W$.}

\noindent It is convenient to allow the special case $X=\emptyset$,
$\Cal W=\{\emptyset\}$\cmmnt{, even though this is not properly
speaking a filter}.

The pair $(X,\Cal W)$ is now a {\bf uniform space}.

\spheader 3A4Ab If $\Cal W$ is a uniformity on a set $X$, the associated
topology $\frak T$ is the set of sets $G\subseteq X$ such that
for every $x\in G$ there is a $W\in\Cal W$ such that
$W[\{x\}]\cmmnt{\mskip5mu=\{y:(x,y)\in W\}}$ is included in $G$.
\prooflet{(\Engelking, 8.1.1;  \Bourbaki, II.1.2;  \Gaal, p.\
48;  \Schubert, p.\ 115;  \James, p.\ 101.)}

\spheader 3A4Ac We say that a uniformity is {\bf Hausdorff} if the
associated topology is Hausdorff.

\spheader 3A4Ad If $U$ is a linear topological space, then it has an
associated uniformity

$$\eqalign{\Cal W=\{W:W\subseteq U\times U,\text{ there is an}
&\text{ open set } G\text{ containing }0\cr
&\text{ such that }(u,v)\in W\text{ whenever }u-v\in G\},\cr}$$

\noindent and $\Cal W$ defines the topology of $U$ in the sense of (b) 
above\prooflet{ (\Schaefer, I.1.4)}.

\leader{3A4B}{Uniformities and pseudometrics (a)} If $\Rho$ is a family
of pseudometrics on a set $X$, then the associated uniformity is the
smallest uniformity on $X$ containing all the sets
$W(\rho;\epsilon)=\{(x,y):\rho(x,y)<\epsilon\}$
as $\rho$ runs over $\Rho$, $\epsilon$ over $\ooint{0,\infty}$.
\prooflet{(\Engelking, 8.1.18;  \Bourbaki, IX.1.2.)}

\spheader 3A4Bb If $\Cal W$ is the uniformity defined by a family $\Rho$
of pseudometrics, then the topology associated with $\Cal W$ is the
topology defined from $\Rho$\cmmnt{ (2A3F)}.
\prooflet{(\Dugundji, p.\ 203.)}

\spheader 3A4Bc A uniformity $\Cal W$ is {\bf metrizable} if it can be
defined by a single metric.

\spheader 3A4Bd If $U$ is a linear space with a topology defined from a
family of functionals $\tau:U\to\coint{0,\infty}$ such that
$\tau(u+v)\le\tau(u)+\tau(v)$, $\tau(\alpha u)\le\tau(u)$ when
$|\alpha|\le 1$, and
$\lim_{\alpha\to 0}\tau(\alpha u)=0$\cmmnt{ (2A5B)}, the
uniformity defined from the topology\cmmnt{ (3A4Ad)} coincides with
the uniformity defined from the pseudometrics
$\rho_{\tau}(u,v)=\tau(u-v)$.   \prooflet{(Immediate from the
definitions.)}

\leader{3A4C}{Uniform continuity (a)} If $(X,\Cal W)$ and $(Y,\Cal V)$
are uniform spaces, a function $\phi:X\to Y$ is {\bf uniformly
continuous} if $\{(x,y):(\phi(x),\phi(y))\in V\}$ belongs to
$\Cal W$ for every $V\in\Cal V$.

\header{3A4Cb}{\bf (b)} The composition of uniformly continuous
functions is uniformly continuous.
\prooflet{(\Bourbaki, II.2.1;  \Schubert, p.\ 118.)}

\spheader 3A4Cc If uniformities $\Cal W$, $\Cal V$ on sets $X$, $Y$ are
defined by
non-empty families $\Rho$, $\Theta$ of pseudometrics, then a function
$\phi:X\to Y$ is uniformly continuous iff for every $\theta\in\Theta$,
$\epsilon>0$ there are $\rho_0,\ldots,\rho_n\in\Rho$ and $\delta>0$ such
that $\theta(\phi(x),\phi(y))\le\epsilon$ whenever $x$, $y\in X$ and
$\max_{i\le n}\rho_i(x,y)\le\delta$.
%3{}25D
\prooflet{(Elementary verification.)}

\leaveitout{If $\Rho$ is upwards-directed then we can say:
$\phi:X\to Y$ is uniformly continuous iff for every $\theta\in\Theta$,
$\epsilon>0$ there are $\rho\in\Rho$ and $\delta>0$ such
that $\theta(\phi(x),\phi(y))\le\epsilon$ whenever $x$, $y\in X$ and
$\rho(x,y)\le\delta$.
\ref
}%end of leaveitout

\spheader 3A4Cd A uniformly continuous function is continuous for the
associated topologies.
\prooflet{(\Bourbaki, II.2.1;  \Schubert, p.\ 118;  \James, p.\ 102.)}

\spheader 3A4Ce Two metrics $\rho$, $\sigma$ on a set $X$ are {\bf
uniformly equivalent} if they give rise to the same uniformity\cmmnt{;  
that is, if for every $\epsilon>0$ there is a $\delta>0$ such that

\Centerline{$\rho(x,y)\le\delta\Rightarrow\sigma(x,y)\le\epsilon$,
\quad$\sigma(x,y)\le\delta\Rightarrow\rho(x,y)\le\epsilon$.}}

\leaveitout{If $\Rho$ and $\Theta$ are two non-empty families of
pseudometrics on
$X$, then they define the same uniformity iff

\inset{for every $\rho\in\Rho$, $\epsilon>0$ there are $\delta>0$,
$\theta_0,\ldots,\theta_n\in\Theta$ such that $\rho(x,y)\le\epsilon$
whenever $\max_{i\le n}\theta_i(x,y)\le\delta$}

\noindent and

\inset{for every $\theta\in\Theta$, $\epsilon>0$ there are $\delta>0$,
$\rho_0,\ldots,\rho_n\in\Rho$ such that $\theta(x,y)\le\epsilon$
whenever $\max_{i\le n}\rho_i(x,y)\le\delta$.}
\ref

If $\Rho$ and $\Theta$ are upwards-directed (2A3Fe), this simplifies to

\inset{for every $\rho\in\Rho$, $\epsilon>0$ there are $\delta>0$,
$\theta\in\Theta$ such that $\rho(x,y)\le\epsilon$ whenever
$\theta(x,y)\le\delta$}

\noindent and

\inset{for every $\theta\in\Theta$, $\epsilon>0$ there are $\delta>0$,
$\rho\in\Rho$ such that $\theta(x,y)\le\epsilon$ whenever
$\rho(x,y)\le\delta$.}
}%end of leaveitout

\leader{3A4D}{Subspaces (a)} If $(X,\Cal W)$ is a uniform space and $Y$
is any subset of $X$, then $\Cal W_Y=\{W\cap(Y\times Y):W\in\Cal W\}$ is
a uniformity on $Y$;  it is the {\bf subspace uniformity}.
\prooflet{(\Bourbaki, II.2.4;  \Schubert, p.\ 122.)}

\header{3A4Db}{\bf (b)} If $\Cal W$ defines a topology $\frak T$ on $X$,
then the topology defined by $\Cal W_Y$ is the subspace topology on
$Y$\cmmnt{, as defined in 2A3C}.
\prooflet{(\Schubert, p.\ 122;  \James, p.\ 103.)}

\header{3A4Dc}{\bf (c)} If $\Cal W$ is defined by a family $\Rho$ of
pseudometrics on $X$, then $\Cal W_Y$ is defined by
$\{\rho\,\restr\,Y\times Y:\rho\in\Rho\}$.
\prooflet{(Elementary verification.)}

\leader{3A4E}{Product uniformities (a)} If $(X,\Cal U)$ and $(Y,\Cal V)$
are uniform spaces, the {\bf product uniformity} is the
smallest uniformity $\Cal W$ on $X\times Y$ containing all sets of the
form

\Centerline{$\{((x,y),(x',y')):(x,x')\in U,\,(y,y')\in V\}$}

\noindent as $U$ runs over $\Cal U$ and $V$ over $\Cal V$.
\prooflet{(\Engelking, \S8.2;  \Bourbaki, II.2.6;  \Schubert, p.\ 124;
\James, p.\ 93.)}

\header{3A4Eb}{\bf (b)} If $\Cal U$, $\Cal V$ are defined from families
$\Rho$, $\Theta$ of pseudometrics, then $\Cal W$ will be defined by the
family $\{\tilde\rho:\rho\in\Rho\}\cup\{\bar\theta:\theta\in\Theta\}$,
writing

\Centerline{$\tilde\rho((x,y),(x',y'))=\rho(x,x'),
\quad\bar\theta((x,y),(x',y'))=\theta(y,y')$}

\noindent as in 2A3Tb.
\prooflet{(Elementary verification.)}

\header{3A4Ec}{\bf (c)} If $(X,\Cal U)$, $(Y,\Cal V)$ and $(Z,\Cal W)$
are uniform spaces, a map $\phi:Z\to X\times Y$ is uniformly continuous
iff the coordinate maps $\phi_1:Z\to X$ and $\phi_2:Z\to Y$ are
uniformly continuous.
\prooflet{(\Engelking, 8.2.1;  \Bourbaki, II.2.6;  \Schubert, p.\ 125;
\James, p.\ 93.)}

\leader{3A4F}{Completeness (a)} If $\Cal W$ is a uniformity on a set
$X$, a filter $\Cal F$ on $X$ is {\bf Cauchy} if for every $W\in\Cal W$
there is an $F\in\Cal F$ such that $F\times F\subseteq W$.

Any convergent filter in a uniform space is Cauchy.
\prooflet{(\Bourbaki, II.3.1;  \Gaal, p.\ 276;  \Schubert, p.\ 134;
\James, p.\ 109.)}

\header{3A4Fb}{\bf (b)} A uniform space is {\bf complete} if every
Cauchy filter is convergent.

\header{3A4Fc}{\bf (c)} If $\Cal W$ is defined from a family $\Rho$ of
pseudometrics, then a
filter $\Cal F$ on $X$ is Cauchy iff for every $\rho\in\Rho$ and
$\epsilon>0$ there is an $F\in\Cal F$ such that $\rho(x,y)\le\epsilon$
for all $x$, $y\in F$;  equivalently, for every $\rho\in\Rho$,
$\epsilon>0$ there is an $x\in X$ such that
$U(x;\rho;\epsilon)\in\Cal F$.   \prooflet{(Elementary verification.)}

\spheader 3A4Fd A complete subspace of a Hausdorff uniform space is
closed.
\prooflet{(\Engelking, 8.3.6;  \Bourbaki, II.3.4;  \Schubert, p.\ 135;
\James, p.\ 148.)}

\spheader 3A4Fe A metric space is complete iff every Cauchy sequence
converges\cmmnt{ (cf.\ 2A4Db)}.
\prooflet{(\Schubert, p.\ 141;  \Gaal, p.\ 276;  \James, p.\ 150.)}

\spheader 3A4Ff If $(X,\rho)$ is a complete metric space, $D\subseteq X$
a dense subset, $(Y,\sigma)$ a metric space and $f:X\to Y$ is an
{\bf isometry} (that is, $\sigma(f(x),f(x'))=\rho(x,x')$ for all $x$,
$x'\in X$), then $f[X]$ is precisely the closure of $f[D]$ in $Y$.
\prooflet{(For $f[X]$ must be complete, and we can use (d).)}

\spheader 3A4Fg If $U$ is a linear space with a linear space topology
and the associated uniformity\cmmnt{ (3A4Ad)}, then a filter $\Cal F$
on $U$ is Cauchy iff for every open set $G$ containing $0$ there is an
$F\in\Cal F$ such that $F-F\subseteq G$\cmmnt{ (cf.\ 2A5F)}.
\prooflet{(Immediate from the definitions.)}

\leader{3A4G}{Extension of uniformly continuous functions:  Theorem} If
$(X,\Cal W)$ is a uniform space, $(Y,\Cal V)$ is a complete uniform
space, $D\subseteq X$ is a dense subset of $X$, and $\phi:D\to Y$ is
uniformly continuous\cmmnt{ (for the subspace uniformity of $D$)},
then there is a uniformly continuous $\hat\phi:X\to Y$ extending $\phi$.
If $Y$ is Hausdorff, the extension is unique.
\cmmnt{$*$} \prooflet{(\Engelking, 8.3.10;  \Bourbaki, II.3.6;  \Gaal,
p.\ 300;  \Schubert, p.\ 137;  \James, p.\ 152.)}

In particular, if $(X,\rho)$ is a metric space, $(Y,\sigma)$ is a
complete metric space, $D\subseteq X$ is a dense subset, and
$\phi:D\to Y$ is an isometry, then there is a unique isometry
$\hat\phi:X\to Y$ extending $\phi$.

\leader{3A4H}{Completions (a) Theorem} If $(X,\Cal W)$ is any Hausdorff
uniform space, then we can find a complete Hausdorff uniform space
$(\hat X,\hat{\Cal W})$ in which $X$ is embedded as a dense
subspace;   moreover, any two such spaces are essentially unique.
\cmmnt{$*$} \prooflet{(\Engelking, 8.3.12;  \Bourbaki, II.3.7;  \Gaal,
p.\ 297 \& p.\ 300;  \Schubert, p.\ 139;  \James, p.\ 156.)}

\spheader 3A4Hb Such a space $(\hat X,\hat{\Cal W})$ is called a
{\bf completion} of $(X,\Cal W)$.   Because it is unique up to
isomorphism as a uniform space, we may call it `the' completion.

\spheader 3A4Hc If $\Cal W$ is the
uniformity defined by a metric $\rho$ on a set $X$, then there is a
unique extension of $\rho$ to a metric $\hat\rho$ on $\hat X$ defining
the uniformity $\hat{\Cal W}$.
\prooflet{(\Bourbaki, IX.1.3.)}

\leader{3A4I}{A note on metric spaces}\cmmnt{ I mention some
elementary facts.}   Let $(X,\rho)$ be a metric
space.   If $x\in X$ and $A\subseteq X$ is non-empty, set

\Centerline{$\rho(x,A)=\inf_{y\in A}\rho(x,y)$.}

\noindent Then $\rho(x,A)=0$ iff $x\in\overline{A}$\cmmnt{ (2A3Kb)}.
If $B\subseteq X$ is another non-empty set, then

\Centerline{$\rho(x,B)\le\rho(x,A)+\sup_{y\in A}\rho(y,B)$.}

\noindent In particular, $\rho(x,\overline{A})=\rho(x,A)$.   If
$\sequencen{A_n}$ is a non-decreasing sequence of non-empty sets with
union $A$, then

\Centerline{$\rho(x,A)=\lim_{n\to\infty}\rho(x,A_n)$.}

\discrpage

