\frfilename{mt316.tex}
\versiondate{26.1.09}
\copyrightdate{2000}

\def\chaptername{Boolean algebras}
\def\sectionname{Further topics}

\newsection{316}

I introduce
three special properties of Boolean algebras which will be of great
importance in the rest of this volume:
the countable chain condition (316A-316F), weak
$(\sigma,\infty)$-distributivity (316G-316J) and homogeneity
(316N-316Q). %316N 316O 316P 316Q
I add some brief notes on atoms in Boolean algebras (316K-316L),
with a characterization of the algebra of open-and-closed subsets of
$\{0,1\}^{\Bbb N}$ (316M).

\leader{316A}{Definitions (a)} A Boolean algebra $\frak A$ is {\bf ccc},
or satisfies the {\bf countable chain condition}, if every disjoint
subset of $\frak A$ is countable.

\spheader 316Ab A topological space $X$ is {\bf ccc}, or
satisfies the {\bf countable chain condition}, or has {\bf Souslin's
property}, if every disjoint
collection of open sets in $X$ is countable.

\leader{316B}{Theorem} A Boolean algebra $\frak A$ is ccc iff its Stone
space $Z$ is ccc.

\proof{{\bf (a)} If $\frak A$ is ccc and $\Cal G$ is a disjoint family
of open sets in $Z$, then for each
$G\in\Cal G'=\Cal G\setminus\{\emptyset\}$
we can find a non-zero $a_G\in\frak A$ such that the corresponding
open-and-closed set $\widehat{a}_G$ is included in $G$.
Now $\{a_G:G\in\Cal G'\}$ is a disjoint family in $\frak A$, so is
countable;  since $a_G\ne a_H$ for distinct $G$, $H\in\Cal G'$,
$\Cal G'$ and $\Cal G$ must be countable.   As $\Cal G$ is arbitrary,
$Z$ is ccc.

\medskip

{\bf (b)} If $Z$ is ccc and $A\subseteq\frak A$ is disjoint, then
$\{\widehat{a}:a\in A\}$ is a disjoint family of open subsets of $Z$, so
must be countable, and $A$ is countable.   As $A$ is arbitrary,
$\frak A$ is ccc.
}%end of proof of 316B

\leader{316C}{Proposition} Let $\frak A$ be a Dedekind
$\sigma$-complete Boolean algebra and $\Cal I$ a $\sigma$-ideal of
$\frak A$.   Then the quotient algebra $\frak B=\frak A/\Cal I$ is ccc
iff every disjoint family in $\frak A\setminus\Cal I$ is countable.

\proof{{\bf (a)} Suppose that $\frak B$ is ccc and that $A$ is a
disjoint family in $\frak A\setminus\Cal I$.   Then
$\{a^{\ssbullet}:a\in A\}$ is a
disjoint family in $\frak B$, therefore countable, and
$a^{\ssbullet}\ne b^{\ssbullet}$ when $a$, $b$ are distinct members of
$A$;  so $A$ is countable.

\medskip

{\bf (b)} Now suppose that $\frak B$ is not ccc.   Then there is an
uncountable disjoint set $B\subseteq\frak B$.   Of course
$B\setminus\{0\}$ is still uncountable, so may be enumerated as
$\langle b_{\xi}\rangle_{\xi<\kappa}$, where $\kappa$ is an uncountable
cardinal (2A1K), so that $\omega_1\le\kappa$.   For each $\xi<\omega_1$,
choose $a_{\xi}\in \frak A$ such that $a_{\xi}^{\ssbullet}=b_{\xi}$.
Of course $a_{\xi}\notin\Cal I$.   If $\eta<\xi<\omega_1$, then
$b_{\eta}\Bcap b_{\xi}=0$, so $a_{\xi}\Bcap a_{\eta}\in\Cal I$.
Because $\xi<\omega_1$, it is countable;  because $\Cal I$ is
a $\sigma$-ideal, and $\frak A$ is Dedekind $\sigma$-complete,

\Centerline{$d_{\xi}=\sup_{\eta<\xi}a_{\xi}\Bcap a_{\eta}$,}

\noindent belongs to $\Cal I$, and

\Centerline{$c_{\xi}=a_{\xi}\setminus d_{\xi}
\in\frak A\setminus\Cal I$.}

\noindent But now of course

\Centerline{$c_{\xi}\Bcap c_{\eta}\subseteq c_{\xi}\Bcap
a_{\eta}\subseteq
c_{\xi}\Bcap d_{\xi}=0$}

\noindent whenever $\eta<\xi<\omega_1$, so $\{c_{\xi}:\xi<\omega_1\}$ is
an uncountable disjoint family in $\frak A\setminus\Cal I$.
}%end of proof of 316C

\cmmnt{
\medskip

\noindent{\bf Remark} An ideal $\Cal I$ satisfying the conditions of
this proposition is said to be {\bf $\omega_1$-saturated} in $\frak A$.
%Kechris:  "$\Cal I$ has ccc in $\frak A$"
}%end of comment

\leader{316D}{Corollary} Let $X$ be a set, $\Sigma$ a $\sigma$-algebra
of subsets of $X$, and $\Cal I$ a $\sigma$-ideal of $\Sigma$.   Then
the quotient algebra $\Sigma/\Cal I$ is ccc iff every disjoint family in
$\Sigma\setminus\Cal I$ is countable.

\leader{316E}{Proposition} Let $\frak A$ be a ccc Boolean algebra.
Then for any $A\subseteq\frak A$ there is a countable $B\subseteq A$
such that $B$ has the same upper and lower bounds as $A$.

\proof{{\bf (a)} Set

\Centerline{$D=\bigcup_{a\in A}\{d:d\Bsubseteq a\}$.}

\noindent Applying Zorn's lemma to the family $\Cal C$ of disjoint
subsets
of $D$, we have a maximal $C_0\in\Cal C$.   For each $c\in C_0$ choose a
$b_c\in A$ such that $c\Bsubseteq b_c$, and set $B_0=\{b_c:c\in C_0\}$.
Because $\frak A$ is ccc, $C_0$ is countable, so $B_0$ also is
countable.
\Quer\ If there is an upper bound $e$ for $B_0$ which is not an upper
bound for $A$, take $a\in A$ such that $c'=a\Bsetminus e\ne 0$;  then
$c'\in D$ and $c'\Bcap c=c'\Bcap b_c=0$ for every $c\in C_0$, so
$C_0\cup\{c'\}\in\Cal C$;  but $C_0$ was supposed to be maximal in
$\Cal C$.\ \BanG\   Thus every
upper bound for $B_0$ is also an upper bound for $A$.

\medskip

{\bf (b)} Similarly, there is a countable set $B_1'\subseteq
A'=\{1\Bsetminus a:a\in A\}$ such that every upper bound for $B_1'$ is
an upper bound for $A'$.   Set $B_1=\{1\Bsetminus b:b\in B_1'\}$;  then
$B_1$ is a countable subset of $A$ and every lower bound for $B_1$ is a
lower bound for $A$.   Try $B=B_0\cup B_1$.   Then $B$ is a countable
subset of
$A$ and every upper (resp. lower) bound for $B$ is an upper (resp.
lower) bound for $A$;  so that $B$ must have exactly the same upper and
lower bounds as $A$ has.
}%end of proof of 316E

\leader{316F}{Corollary} Let $\frak A$ be a ccc Boolean algebra.

(a) If $\frak A$ is Dedekind $\sigma$-complete it is Dedekind complete.

(b) If $A\subseteq\frak A$ is sequentially order-closed it is
order-closed.

(c) If $Q$ is any partially ordered set and $\phi:\frak A\to Q$ is a
sequentially order-continuous order-preserving function, it is
order-continuous.

(d) If $\frak B$ is another Boolean algebra and $\pi:\frak A\to\frak B$
is a sequentially order-continuous Boolean homomorphism, it is
order-continuous.

\proof{{\bf (a)} If $A$ is any
subset of $\frak A$, let $B\subseteq A$ be a countable set with the same
upper bounds as $A$;  then $\sup B$ is defined in $\frak A$ and must be
$\sup A$.

\medskip

{\bf (b)} Suppose that $B\subseteq A$ is non-empty and upwards-directed
and has a supremum $a$ in $\frak A$.   Then there is a non-empty
countable set
$C\subseteq B$ with the same upper bounds as $B$.   Let
$\sequencen{c_n}$ be
a sequence running over $C$.   Because $B$ is upwards-directed, we can
choose $\sequencen{b_n}$ inductively such that

\Centerline{$b_0=c_0$,\quad $b_{n+1}\in B$, $b_{n+1}\supseteq b_n\Bcup
c_{n+1}$ for every $n\in\Bbb N$.}

\noindent Now any upper bound for $\{b_n:n\in\Bbb N\}$ must also be an
upper bound for $\{c_n:n\in\Bbb N\}=C$, so is an upper bound for the
whole set $B$.   But this means that $a=\sup_{n\in\Bbb N}b_n$.   As
$\sequencen{b_n}$
is a non-decreasing sequence in $A$, and $A$ is sequentially
order-closed, $a\in A$.

In the same way, if $B\subseteq A$ is downwards-directed and has an
infimum in $\frak A$, this is also the infimum of some non-increasing
sequence in $B$, so must belong to $A$.   Thus $A$ is order-closed.

\medskip

{\bf (c)(i)} Suppose that $A\subseteq\frak A$ is a non-empty
upwards-directed set with supremum $a_0\in\frak A$.   As in (b), there
is a non-decreasing sequence $\sequencen{c_n}$ in $\frak A$ with
supremum $a_0$.   Because $\phi$ is sequentially order-continuous,
$\phi a_0=\sup_{n\in\Bbb N}\phi c_n$ in $Q$.   But this means that
$\phi a_0$ must be the least upper bound of $\phi[A]$.

\medskip

{\bf (ii)} Similarly, if $A\subseteq\frak A$ is a non-empty
downwards-directed set with infimum $a_0$, there is a non-increasing
sequence $\sequencen{c_n}$ in $A$ with infimum $a_0$, so that

\Centerline{$\inf\phi[A]=\inf_{n\in\Bbb N}\phi c_n=\phi a_0$.}

\noindent Putting this together with (i), we see that $\phi$ is
order-continuous, as claimed.

\medskip

{\bf (d)} This is a special case of (c).
}%end of proof of 316F

\leader{316G}{Definition} Let $\frak A$ be a Boolean algebra.   I will say
that $\frak A$ is {\bf \wsid} if whenever
$\sequencen{A_n}$ is a sequence of downwards-directed
subsets of $\frak A$ and $\inf A_n=0$ for every $n$, then $\inf B=0$,
where

\Centerline{$B=\{b:b\in\frak A$, for every $n\in\Bbb N$ there is an
$a\in A_n$ such that $b\Bsupseteq a\}$.}

\vleader{60pt}{316H}{Proposition} Let $\frak A$ be a Boolean algebra.
Then the following are equiveridical:

(i) $\frak A$ is \wsid;

(ii) whenever
$\sequencen{A_n}$ is a sequence of partitions of unity in $\frak A$,
there is a partition of unity $B$ in $\frak A$ such that
$\{a:a\in A_n$, $a\Bcap b\ne 0\}$ is finite for every $n\in\Bbb N$ and
$b\in B$;

(iii) whenever $\sequencen{A_n}$ is a sequence of upwards-directed
subsets of $\frak A$, each with a supremum $c_n=\sup A_n$, and

\Centerline{$B=\{b:b\in\frak A$, for every $n\in\Bbb N$ there is an
$a\in A_n$ such that $b\Bsubseteq a\}$,}

\noindent then $\inf\{c_n\Bsetminus b:n\in\Bbb N$, $b\in B\}=0$;

(iv) whenever $\sequencen{A_n}$ is a sequence of upwards-directed
subsets of $\frak A$, each with a supremum $c_n=\sup A_n$, and
$\inf_{n\in\Bbb N}c_n=c$ is defined, then $c=\sup B$, where

\Centerline{$B=\{b:b\in\frak A$, for every $n\in\Bbb N$ there is an
$a\in A_n$ such that $b\Bsubseteq a\}$.}

\proof{{\bf (i)$\Rightarrow$(ii)} Suppose that $\frak A$ is \wsid\ and
that $\sequencen{A_n}$ is a sequence of partitions of unity in $\frak A$.
For each $n\in\Bbb N$, set

\Centerline{$C_n=\{1\Bsetminus\sup D:D\in A_n$ is finite$\}$,}

\noindent so that $C_n$ is downwards-directed and has infimum $0$.   Set
$E=\{e:$ for every $n\in\Bbb N$ there is a $c\in C_n$ such that
$c\Bsubseteq e\}$;  then $\inf E=0$.   So $B_0=\{b:b\Bcap e=0$ for some
$e\in E\}$ is order-dense in $\frak A$ and includes a partition $B$ of
unity.   If $n\in\Bbb N$ and $b\in B$, take $e\in E$ such that
$b\Bcap e=0$, $c\in C_n$ such that $c\Bsubseteq e$, and a finite set
$D\subseteq A_n$ such that $c=1\Bsetminus\sup D$;  then

\Centerline{$b\Bsubseteq 1\Bsetminus e
\Bsubseteq 1\Bsetminus c\Bsubseteq\sup D$}

\noindent and $\{a:a\in A_n$, $a\Bcap b\ne 0\}\subseteq D$ is finite.
As $\sequencen{A_n}$ is arbitrary, (ii) is true.

\medskip

{\bf (ii)$\Rightarrow$(iii)} Suppose that (ii) is true, and
that $\sequencen{A_n}$ is a sequence of upwards-directed subsets of
$\frak A$, each with a supremum $c_n=\sup A_n$.   For each $n\in\Bbb N$,

\Centerline{$D_n
=\{d:d\Bsubseteq 1\Bsetminus c_n\}
  \cup\bigcup_{a\in A_n}\{d:d\Bsubseteq a\}$}

\noindent is order-dense in $\frak A$, so there is a partition of unity
$D'_n\subseteq D_n$ (313K).   By (ii), there is a partition
of unity $E$ such that $\{d:d\in D'_n$, $d\Bcap e\ne 0\}$ is finite for
every $n\in\Bbb N$ and $e\in E$.   \Quer\ Suppose, if possible, that
$\{c_n\Bsetminus b:n\in\Bbb N$, $b\in B\}$ has a non-zero lower bound
$c$.   Let $e\in E$ be such that $c\Bcap e\ne 0$.   For each
$n\in\Bbb N$, set $D''_n=\{d:d\in D'_n$, $c\Bcap e\Bcap d\ne 0\}$.
Then $D''_n$ is finite so $d_n=\sup D''_n$ is defined and
$c\Bcap e\Bsubseteq d_n$.   Also, because $c\Bsubseteq c_n$, each element of
$D''_n$ is included in a member of $A_n$;  as $A_n$ is upwards-directed,
so are $d_n$ and $c\Bcap e$.   As $n$ is arbitrary, $c\Bcap e\in B$;  and
$c$ was supposed to be disjoint from every member of $B$.\ \Bang

Thus $\inf\{c_n\Bsetminus b:n\in\Bbb N$, $b\in B\}=0$;  as
$\sequencen{A_n}$ is arbitrary, (iii) is true.

\medskip

{\bf (iii)$\Rightarrow$(iv)} Suppose that (iii) is true and
that $A_n$, $c_n$, $c$ and $B$ are as in the statement of (iv).   Then

\Centerline{$\inf\{c\Bsetminus b:b\in B\}
=\inf\{c_n\Bsetminus b:n\in\Bbb N$, $b\in B\}=0$;}

\noindent as $b\Bsubseteq c_n$ whenever $b\in B$ and $n\in\Bbb N$,
we have $b\Bsubseteq c$ for every $b\in B$, and $\sup B=c$, by
313Ab.   Thus (iv) is true.

\medskip

{\bf (iv)$\Rightarrow$(i)} Suppose that (iv) is true and
that $A_n$ and $B$ are as in 316G.    Set
$A'_n=\{1\Bsetminus a:a\in A_n\}$, so that $A'_n$ is an upwards-directed
set with supremum $1$ for each $n$, and

\Centerline{$B'
=\{b:$ for every $n\in\Bbb N$ there is an
$a\in A'_n$ such that $b\Bsubseteq a\}
=\{1\Bsetminus b:b\in B\}$;}

\noindent then

\Centerline{$\inf B=1\Bsetminus\sup B'
=1\Bsetminus\inf_{n\in\Bbb N}\sup A'_n=0$,}

\noindent as required.
}%end of proof of 316H

\leader{316I}{}\cmmnt{ As usual, a characterization of the property in
terms of the Stone spaces is extremely valuable.

\medskip

\noindent}{\bf Theorem} Let $\frak A$ be a Boolean algebra, and $Z$ its
Stone space.   Then $\frak A$ is \wsid\ iff
every meager set in $Z$ is nowhere dense.

\proof{{\bf (a)} The point is that if $M\subseteq Z$ then $M$ is nowhere
dense iff there is a partition of unity $A$ in $\frak A$ such that
$M\cap\widehat{a}=\emptyset$ for every $a\in A$.   \Prf\ If $M$ is
nowhere dense, then $\{a:M\cap\widehat{a}=\emptyset\}$ is order-dense in
$\frak A$, so includes a partition of unity.   In the other direction,
if $A$ is a partition of unity such that $M$ is disjoint from
$\widehat{a}$ for every $a\in A$, then $\sup A=1$ so
$G=\bigcup_{a\in A}\widehat{a}$ is dense (313Ca);  now $G$ is a dense
open set disjoint from $M$, so $M$ is nowhere dense.\ \Qed

\medskip

{\bf (b)} Suppose that $\frak A$ is \wsid\ and that $M$ is a
meager subset of $Z$.
Then $M$ can be expressed as $\bigcup_{n\in\Bbb N}M_n$ where each $M_n$
is nowhere dense.   For each $n\in\Bbb N$, let $A_n$ be a partition of
unity such that $M_n\cap\widehat{a}=\emptyset$ for every $a\in A_n$.
By 316H(i)$\Rightarrow$(ii), there is a partition of unity $B$ such that
$\{a:a\in A_n$, $a\Bcap b\ne 0\}$ is finite for every $n\in\Bbb N$ and
$b\in B$.   Now $M_n\cap\widehat{b}=\emptyset$ for every $n\in\Bbb N$
and $b\in B$.   \Prf\ $C=\{a:a\in A_n$, $b\cap a\ne 0\}$ is finite.   So
$F=\bigcup_{a\in C}\widehat{a}$ is closed and $G=\widehat{b}\setminus F$
is open.   But $G\cap\widehat{a}=\emptyset$ for every $a\in A$, so $G$
is empty and $\widehat{b}\subseteq F\subseteq Z\setminus M_n$.\ \QeD\
Accordingly $M\cap\widehat{b}=\emptyset$ for every $b\in B$ and $M$ is
nowhere dense.

\medskip

{\bf (c)} Suppose that every meager set in $Z$ is nowhere dense, and
that $\sequencen{A_n}$ is a sequence of partitions of unity in
$\frak A$.   Then
$M_n=Z\setminus\bigcup_{a\in A_n}\widehat{a}$ is nowhere dense for each
$n$ (313Cc), so
$M=\bigcup_{n\in\Bbb N}M_n$ is meager, therefore nowhere dense.   Let
$B$ be a partition of unity in $\frak A$ such that
$M\cap\widehat{b}=\emptyset$ for every $b\in B$.   If $n\in\Bbb N$ and
$b\in B$, then

\Centerline{$\widehat{b}\subseteq Z\setminus M
\subseteq Z\setminus M_n=\bigcup_{a\in A_n}\widehat{a}$.}

\noindent As $\widehat{b}$ is compact, there is some finite
$C\subseteq A$ such that
$\widehat{b}\subseteq\bigcup_{a\in C}\widehat{a}$ and
$b\Bsubseteq\sup C$;   but this means that
$\{a:a\in A_n$, $a\Bcap b\ne 0\}\subseteq C$ is finite.   As
$\sequencen{A_n}$ is arbitrary, $\frak A$ is \wsid, by
316H(ii)$\Rightarrow$(i).
}%end of proof of 316I

\leader{316J}{The regular open algebra of \dvrocolon{$\Bbb R$}}\cmmnt{
For examples of \wsid\ algebras, I think we can wait for
Chapter 32 (see also 393C).   But the standard example of an algebra
which is {\it not} \wsid\ is of such importance that (even
though it has nothing to do with measure theory, narrowly defined) I
think it right to describe it here.

\medskip

\noindent}{\bf Proposition} The algebra $\RO(\Bbb R)$ of regular open
subsets of $\Bbb R$\cmmnt{ (314O)} is not \wsid.

\proof{ Enumerate $\Bbb Q$ as
$\sequencen{q_n}$.   For each $n\in\Bbb N$, set

\Centerline{$A_n=\{G:G\in\RO(\Bbb R),\,q_i\in G$ for every $i\le n\}$.}

\noindent Then $A_n$ is downwards-directed, and

\Centerline{$\inf A_n=\interior\bigcap A_n
=\interior\{q_i:i\le n\}=\emptyset$.}

\noindent But if $A\subseteq\RO(\Bbb R)$ is such that

\inset{for every $n\in\Bbb N$, $G\in A$ there is an $H\in A_n$ such that
$H\subseteq G$,}

\noindent then we must have $\Bbb Q\subseteq G$ for every $G\in A$, so
that

\Centerline{$\Bbb R
=\interior\overline{\Bbb Q}\subseteq\interior\overline{G}=G$}

\noindent for every $G\in A$, and $A\subseteq\{\Bbb R\}$;  which means
that $\inf A\ne\emptyset$ in $\RO(\Bbb R)$, and 316G cannot be
satisfied.
}%end of proof of 316J

\leader{316K}{Atoms in Boolean algebras (a)} If $\frak A$ is a Boolean
algebra, an {\bf atom} in $\frak A$ is a non-zero $a\in\frak A$ such
that
the only elements included in $a$ are $0$ and $a$.

\header{316Kb}{\bf (b)} A Boolean algebra is {\bf atomless} if it has no
atoms.

\header{316Kc}{\bf (c)} A Boolean algebra is {\bf purely atomic} if
every non-zero element includes an atom.

\leader{316L}{Proposition} Let $\frak A$ be a Boolean algebra, with
Stone space $Z$.

(a) There is a one-to-one correspondence between atoms $a$ of $\frak A$
and isolated points $z\in Z$, given by the formula $\widehat{a}=\{z\}$.

(b) $\frak A$ is atomless iff $Z$ has no isolated points.

(c) $\frak A$ is purely atomic iff the isolated points of $Z$ form a
dense subset of $Z$.

\proof{{\bf (a)(i)} If $z$ is an isolated point in $Z$, then $\{z\}$ is
an open-and-closed subset of $Z$, so is of the form $\widehat{a}$ for
some $a\in\frak A$;  now if $b\Bsubseteq a$, $\widehat{b}$ must be
either $\emptyset$ or $\{z\}$, so $b$ must be either $a$ or $0$, and $a$
is an atom.

\medskip

\quad{\bf (ii)} If $a\in\frak A$ and $\widehat{a}$ has two points $z$
and $w$, then (because $Z$ is Hausdorff, 311I) there is an open set $G$
containing $z$ but not $w$.   Now there is a $c\in\frak A$ such that
$z\in\widehat{c}\subseteq G$, so that $a\Bcap c$ must be different from
both $0$ and $a$, and $a$ is not an atom.

\medskip

{\bf (b)} This follows immediately from (a).

\medskip

{\bf (c)} From (a) we see that $\frak A$ is purely atomic iff
$\widehat{a}$ contains an isolated point for every non-zero
$a\in\frak A$;  since every
non-empty open set in $Z$ includes a non-empty set of the form
$\widehat{a}$, this happens iff every non-empty open set in $Z$ contains
an isolated point, that is, iff the set of isolated points is dense.
}%end of proof of 316L

\leader{316M}{Proposition}\dvAformerly{3{}93F}
Let $\frak B$ be the algebra of open-and-closed
subsets of $\{0,1\}^{\Bbb N}$.   Then a Boolean algebra $\frak A$ is
isomorphic to $\frak B$ iff it is atomless, countable and not $\{0\}$.

\proof{{\bf (a)} I must check that $\frak B$ has the declared
properties.   The point is that it is the subalgebra $\frak B'$ of $\Cal PX$
generated by $\{b_i:i\in\Bbb N\}$, where I write
$X=\{0,1\}^{\Bbb N}$, $b_i=\{x:x\in X,\,x(i)=1\}$.   \Prf\ Of course $b_i$ and its
complement $\{x:x(i)=0\}$ are open, so $b_i\in\frak B$ for each $i$, and
$\frak B'\subseteq\frak B$.   In the other direction, the open cylinder
sets of
$X$ are all of the form $c_z=\{x:x(i)=z(i)$ for every $i\in J\}$, where
$J\subseteq I$ is finite and $z\in\{0,1\}^J$;  now

\Centerline{$c_z
=X\cap\bigcap_{z(i)=1}b_i\setminus\bigcup_{z(i)=0}b_i\in\frak B'$.}

\noindent If $b\in\frak B$ then $b$ is expressible as a union of such
cylinder sets, because it is open;  but also it is compact, so is the
union of finitely many of them, and must belong to $\frak B'$.   Thus
$\frak B=\frak B'$, as claimed.\ \Qed

For each $n\in\Bbb N$ let
$\frak B_n$ be the finite subalgebra of $\frak B$ generated by
$\{b_i:i<n\}$ (so that $\frak B_0=\{0,1\}$).   Then
$\sequencen{\frak B_n}$ is an increasing sequence of subalgebras of
$\frak B$ with union $\frak B$;  so $\frak B$ is countable.
Also $b\Bcap b_n$, $b\Bsetminus b_n$ are non-zero for every
$n\in\Bbb N$ and non-zero
$b\in\frak B_n$, so no member of any $\frak B_n$ can be an
atom in $\frak B$, and $\frak B$ is atomless.

\medskip

{\bf (b)} Now suppose that $\frak A$ is another algebra with the same
properties.
Enumerate $\frak A$ as $\sequencen{a_n}$.
Choose finite subalgebras
$\frak A_n\subseteq\frak A$ and isomorphisms
$\pi_n:\frak A_n\to\frak B_n$ as follows.
$\frak A_0=\{0,1\}$, $\pi_00=0$, $\pi_01=1$.   Given
$\frak A_n$ and $\pi_n$, let $A_n$ be the set of atoms of $\frak A_n$.
For $a\in A_n$, choose $a'\in\frak A$ such that

\inset{if $a_n\Bcap a$, $a_n\Bsetminus a$ are both non-zero, then
$a'=a_n\Bcap a$;}

\inset{otherwise, $a'\Bsubseteq a$ is any element such that $a'$,
$a\Bsetminus a'$ are both non-zero.}

\noindent (This is where I use the hypothesis that $\frak A$ is
atomless.)   Set $a'_n=\sup_{a\in A_n}a'$.   Then we see that
$a\Bcap a'_n$, $a\Bsetminus a'_n$ are non-zero for every $a\in A_n$ and
therefore for every non-zero $a\in\frak A_n$, that is, that

\Centerline{$\sup\{a:a\in\frak A_n,\,a\Bsubseteq a'_n\}=0$,
\quad$\inf\{a:a\in\frak A_n,\,a\Bsupseteq a'_n\}=1$.}

\noindent Let $\frak A_{n+1}$ be the subalgebra of $\frak A$ generated
by $\frak A_n\cup\{a'_n\}$.   Since we have

\Centerline{$\sup\{b:b\in\frak B_n,\,b\Bsubseteq b_n\}=0$,
\quad$\inf\{b:b\in\frak B_n,\,b\Bsupseteq b_n\}=1$,}

\noindent there is a (unique) extension of $\pi_n:\frak A_n\to\frak B_n$
to a homomorphism $\pi_{n+1}:\frak A_{n+1}\to\frak B_{n+1}$ such that
$\pi_{n+1}a'_n=b_n$ (312O).   Since we similarly  have an extension
$\phi$ of
$\pi_n^{-1}$ to a homomorphism from $\frak B_{n+1}$ to $\frak A_{n+1}$
with $\phi b_n=a'_n$, and since $\phi\pi_{n+1}$, $\pi_{n+1}\phi$ must be
the respective identity homomorphisms, $\pi_{n+1}$ is an isomorphism,
and the induction continues.

Since $\pi_{n+1}$ extends $\pi_n$ for each $n$, these isomorphisms join
together to give us an isomorphism

\Centerline{$\pi:\bigcup_{n\in\Bbb N}\frak A_n\to\bigcup_{n\in\Bbb
N}\frak B_n=\frak B$.}

\noindent Observe next that the construction ensures that
$a_n\in\frak A_{n+1}$ for each $n$, since $a_n\Bcap a$ is either $0$ or
$a$ or $a'_n\Bcap a$ for every $a\in A_n$, and in all cases belongs to
$\frak A_{n+1}$.   So $\bigcup_{n\in\Bbb N}\frak A_n$ contains every
$a_n$ and
(by the choice of $\sequencen{a_n}$) must be the whole of $\frak A$.
Thus $\pi:\frak A\to\frak B$ witnesses that $\frak A\cong\frak B$.
}%end of proof of 316M

\leader{316N}{Definition}\dvAformerly{3{}31M}
A Boolean algebra $\frak A$ is {\bf
homogeneous} if every non-trivial principal ideal of $\frak A$ is
isomorphic to $\frak A$.

\leader{*316O}{Lemma} Let $\frak A$ be a Dedekind complete
Boolean algebra such that

\Centerline{$D
=\{d:d\in\frak A$, $\frak A$ is isomorphic to the principal ideal
$\frak A_d\}$}

\noindent is order-dense in $\frak A$.   Then $\frak A$ is homogeneous.

\proof{{\bf (a)} If $\frak A=\{0\}$ then it has no non-trivial principal
ideals, so is homogeneous.
If $\frak A$ is not atomless, let $a\in\frak A$ be an
atom;  then there is a non-zero $d\in D$ such that $d\Bsubseteq a$ and
$d=a$;  so $\frak A\cong\frak A_d=\{0,d\}$ and $\frak A=\{0,1\}$ is
homogeneous because its only non-trivial principal ideal is itself.
So suppose henceforth that $\frak A$ is atomless and not $\{0\}$.

\medskip

{\bf (b)} Take any $a\in\frak A\setminus\{0\}$.
Choose $\sequencen{a_n}$ inductively in $\frak A$ in such a way that
$a_0=a$ and that $a_{n+1}\Bsubseteq a_n$ is neither $0$ nor $a_n$ for any
$n$.   Let $D'$ be

\Centerline{$\{d:d\in D$, either $d\Bsubseteq\inf_{n\in\Bbb N}a_n$
or $d\Bsubseteq a_n\Bsetminus a_{n+1}$ for some $n$ or
$d\Bsubseteq 1\Bsetminus a_0\}$.}

\noindent Then $D'\subseteq D$ is still order-dense.
Let $C\subseteq D'$ be a partition of unity (313K);  then $C$ is infinite.
We have

\Centerline{$\frak A\cong\prod_{c\in C}\frak A_c\cong\frak A^C$}

\noindent (315F).
Moreover, every member of $C$ is either included in $a$ or disjoint from
it, so setting $C'=\{c:c\in C$, $c\Bsubseteq a\}$ we see that $C'$ is a
partition of unity in $\frak A_a$;  as $\frak A_a$ is Dedekind complete
(314Ea),

\Centerline{$\frak A_a\cong\prod_{c\in C'}\frak A_c\cong\frak A^{C'}
\cong(\frak A^C)^{C'}\cong\frak A^{C\times C'}\cong\frak A^C\cong\frak A$}

\noindent (because $C$ is infinite and $C'$ is not empty,
so $\#(C\times C')=\#(C)$).   As $a$ is arbitrary, $\frak A$ is
homogeneous.
}%end of proof of 316O

\leader{*316P}{Proposition}\dvAnew{2005}
Let $\frak A$ be a homogeneous Boolean algebra.
Then its Dedekind completion\cmmnt{ $\widehat{\frak A}$} is homogeneous.

\proof{ Regarding $\frak A$ as a subset of $\widehat{\frak A}$, it is
order-dense.   Next, if $a\in\frak A$, then the principal ideal
$\widehat{\frak A}_a$ which it generates in $\widehat{\frak A}$ can be
identified with the Dedekind completion of the principal ideal $\frak A_a$
which it generates in $\frak A$.   \Prf\ $\frak A_a$ is order-dense in
$\widehat{\frak A}_a$ and $\widehat{\frak A}_a$ is Dedekind complete, so
we can use 314Ub.\ \QeD\  But this means that

\Centerline{$\widehat{\frak A}_a\cong\widehat{\frak A_a}
\cong\widehat{\frak A}$}

\noindent for every $a\in\frak A\setminus\{0\}$.   As
$\frak A\setminus\{0\}$ is order-dense in $\widehat{\frak A}$,
316O tells us that $\widehat{\frak A}$ is homogeneous.
}%end of proof of 316P

\leader{*316Q}{Proposition}\dvAnew{2008}
The free product of any family of homogeneous
Boolean algebras is homogeneous.

\proof{{\bf (a)}
Let $\familyiI{\frak A_i}$ be a family of homogeneous Boolean
algebras and $\frak A=\bigotimes_{i\in I}\frak A_i$
their free product;  let
$\varepsilon_i:\frak A_i\to\frak A$ be the canonical homomorphisms.
If any of the $\frak A_i$ is $\{0\}$, so is $\frak A$ (315Kd), and the
result is trivial;  so let us suppose that every $\frak A_i$ has at least
two elements.   If $\frak A_i=\{0,1\}$ for every $i\in I$, then
$\frak A=\{0,1\}$ is homogeneous;  so we may suppose that at least one
$\frak A_i$ is infinite.

\medskip

{\bf (b)} If we have a family $\familyiI{a_i}$ such that
$a_i\in\frak A_i$ for every
$i$ and $J=\{i:a_i\ne 1\}$ is finite, consider
$a=\inf_{i\in J}\varepsilon_i(a_i)$ in $\frak A$.   Then
$\frak A_a\cong\bigotimes_{i\in I}(\frak A_i)_{a_i}$.   \Prf\
For $j\in I$, let $\varepsilon_j'$ be the canonical homomorphism from
$(\frak A_j)_{a_j}$ to $\bigotimes_{i\in I}(\frak A_i)_{a_i}$.
Set $\phi_i(c)=a\Bcap\varepsilon_i(c)$ for $i\in I$ and
$c\in(\frak A_i)_{a_i}$.   Then $\phi_i:(\frak A_i)_{a_i}\to\frak A_a$ is
always a Boolean homomorphism, so we have a Boolean homomorphism
$\phi:\bigotimes_{i\in I}(\frak A_i)_{a_i}\to\frak A_a$ such that
$\phi_i=\phi\varepsilon'_i$ for each $i$ (315J).

If $K\subseteq I$ is finite, $J\subseteq K$,
$b_k\in(\frak A_k)_{a_k}$ for each $k\in K$ and
$b$ is the infimum $\inf_{k\in K}\varepsilon'_k(b_k)$ taken in
$\bigotimes_{i\in I}(\frak A_i)_{a_i}$, then

$$\eqalignno{\phi(b)
&=\inf_{k\in K}\phi\varepsilon'_k(b_k)\cr
\displaycause{here taking the infimum in $\frak A_a$}
&=\inf_{k\in K}\phi_k(b_k)
=\inf_{k\in K}a\Bcap\varepsilon_k(b_k)
=a\Bcap\inf_{k\in K}\varepsilon_k(b_k)\cr
\displaycause{here taking the infimum in $\frak A$}
&=\inf_{k\in K}\varepsilon_k(b_k)\cr}$$

\noindent because $K\supseteq J$.

Now suppose that
$b\in\bigotimes_{i\in I}(\frak A_i)_{a_i}$ is non-zero.   Then there are a
finite $K\subseteq I$ and a family $\family{k}{K}{b_k}$ such that
$b_k\in(\frak A_k)_{a_k}\setminus\{0\}$ for each $k$ and
$b$ includes $\inf_{k\in K}\varepsilon'_kb_k$ in
$\bigotimes_{i\in I}(\frak A_i)_{a_i}$ (315Kb).   Set $K'=J\cup K$ and
$b_k=a_k$ for $k\in J\setminus K$.   Then

\Centerline{$\phi(b)
\Bsupseteq\phi(\inf_{k\in K}\varepsilon'_k(b_k))
\Bsupseteq\inf_{k\in K'}\phi\varepsilon'_k(b_k)
=\inf_{k\in K'}\varepsilon_k(b_k)
\ne 0$}

\noindent (315K(e-ii)).   As $b$ is arbitrary, $\phi$ is injective.

To see that $\phi$ is surjective, use 315Kb;  every element of
$\frak A_a$ is expressible as a finite union of elements of the form
$c=\inf_{k\in K}\varepsilon_k(c_k)$ where $K\subseteq I$ is finite and
$c_k\in\frak A_k$ for each $k\in K$.   Again set $K'=J\cup K$;
this time, take $c_k=1$ for any $k\in J\setminus K$.  Then

$$\eqalign{c
&=c\Bcap a
=\inf_{k\in K'}\varepsilon_k(c_k)\Bcap\inf_{k\in K'}\varepsilon_k(a_k)\cr
&=\inf_{k\in K'}(\varepsilon_k(c_k)\Bcap\varepsilon_k(a_k))
=\inf_{k\in K'}(\varepsilon_k(c_k\Bcap a_k))\cr
&=\phi(\inf_{k\in K'}\varepsilon'_k(c_k\Bcap a_k)
\in\phi[\bigotimes_{i\in I}(\frak A_i)_{a_i}].\cr}$$

\noindent So
$\frak A_a\subseteq\phi[\bigotimes_{i\in I}(\frak A_i)_{a_i}]$.\ \Qed

\medskip

{\bf (c)} Let $A$ be the set of those $a\in\frak A$ expressible in the form
considered in (b), with every $a_i$ non-zero.   If $a\in A$, then

\Centerline{$\frak A_a
\cong\bigotimes_{i\in I}(\frak A_i)_{a_i}
\cong\bigotimes_{i\in I}\frak A_i
=\frak A$}

\noindent because every $\frak A_i$ is homogeneous.

\medskip

{\bf (d)} We need to know that $\frak A$ is isomorphic to the simple power
$\frak A^n$ for every $n\ge 1$.   \Prf\ We are supposing that there is a
$k\in I$ such that $\frak A_k$ is infinite.   In this case there must be
a partition of unity $(d_1,\ldots,d_n)$ in $\frak A_k\setminus\{0\}$.
(Induce on $n$, noting at the inductive step that if $\{d_1,\ldots,d_n\}$
is a partition of unity then not all the $d_j$ can be atoms, because
$\#(\frak A_k)>2^n$.)   Now, setting $a^{(j)}=\varepsilon_k(d_j)$ for each
$j$, $(a^{(1)},\ldots,a^{(n)})$ is a partition of unity in $\frak A$ and
(by (c)) all the principal ideals $\frak A_{a^{(j)}}$ are isomorphic to
$\frak A$, so

\Centerline{$\frak A\cong\prod_{j\le n}\frak A_{a^{(j)}}\cong\frak A^n$}

\noindent by 315F(i).\ \Qed

\medskip

{\bf (e)} Now take any $a\in\frak A\setminus\{0\}$.   Then $a$ is
expressible as $\sup_{1\le j\le n}a^{(j)}$ where $a^{(1)},\ldots,a^{(n)}$
are disjoint members of $A$ (315Kb).   So, putting (c) and (d) together,

\Centerline{$\frak A_a\cong\prod_{1\le j\le n}\frak A_{a^{(j)}}
\cong\frak A^n\cong\frak A$.}

\noindent As $a$ is arbitrary, $\frak A$ is homogeneous.
}%end of proof of 316Q

%Question.   We can certainly have  \frak A\otimes\frak B  homogeneous
% without  \frak A  being homogeneous.   But what if neither factor is
% homogeneous?

%What about free products of \wsid\ algebras?

\exercises{
\leader{316X}{Basic exercises (a)}\dvAformerly{3{}16Xg}
%\spheader 316Xa
Let $\frak A$ be a Dedekind $\sigma$-complete
Boolean algebra.   Show that it is ccc iff there is no family
$\langle a_{\xi}\rangle_{\xi<\omega_1}$ in $\frak A$ such that
$a_{\xi}\Bsubset a_{\eta}$ whenever $\xi<\eta<\omega_1$.
%316C

\spheader 316Xb\dvAformerly{3{}16Xi}
Let $\frak A$ be a ccc Boolean algebra.   Show that if
$\Cal I$ is a $\sigma$-ideal of $\frak A$, then it is order-closed, and
$\frak A/\Cal I$ is ccc.
%316C

\spheader 316Xc\dvAformerly{3{}16Xh}
Let $\frak A$ be a
Boolean algebra and $\Cal I$ an order-closed ideal of $\frak A$.
Show that $\frak A/\Cal I$ is ccc iff there is no uncountable disjoint
family in $\frak A\setminus\Cal I$.
%316C

\spheader 316Xd\dvAformerly{3{}16Xj}
Let $\frak A$ be a Boolean algebra.   Show that the
following are equiveridical:  (i) $\frak A$ is ccc;  (ii) every
$\sigma$-ideal of $\frak A$ is order-closed;  (iii) every
$\sigma$-subalgebra of $\frak A$ is order-closed;  (iv) every
sequentially order-continuous Boolean homomorphism from $\frak A$ to
another Boolean algebra is order-continuous.
\Hint{313Q.}
%316Xc 316Xb 316

\spheader 316Xe\dvAformerly{3{}16Xw, previously 3{}16Xx}
Show that any purely atomic Boolean algebra is \wsid.
%316K

\sqheader 316Xf\dvAformerly{3{}16Xs, previously 3{}16Xt}
Let $\frak A$ be a Dedekind complete purely atomic
Boolean algebra.   Show that it is isomorphic to $\Cal PA$, where $A$ is
the set of atoms of $\frak A$.
%316K used in 539O%

\spheader 316Xg\dvAnew{2009}
Show that a homogeneous Boolean algebra is either
atomless or $\{0,1\}$.
%316N

\spheader 316Xh\dvAformerly{3{}16Xa}
Let $\frak A$ be a Boolean algebra, and $\frak B$ a subalgebra of
$\frak A$.   Show that if $\frak A$ is ccc, then $\frak B$ is ccc.
%316A

\spheader 316Xi\dvAformerly{3{}16Xm and 3{}16Xr, previously 3{}16Xs}
Let $\frak A$ be a Boolean algebra, and $\frak B$
a subalgebra of $\frak A$ which is regularly embedded in
$\frak A$.   (i)
Show that if $\frak A$ is \wsid, then $\frak B$ is \wsid.
(ii) Show that every atom
of $\frak A$ is included in an atom of $\frak B$.
(iii) Show that if $\frak A$ is purely atomic, so is $\frak B$.
$\pmb{>}$(iv) Show that if $\frak B$ is atomless, so is
$\frak A$. %used in 528H
%316H 316K

\spheader 316Xj\dvAformerly{3{}16Xf, 3{}16Xo and 3{}16Xq (previously
3{}16Xr)} Let $\frak A$ be a Boolean algebra, and $\frak B$ an
order-dense subalgebra of $\frak A$.
(i) Show that $\frak A$ is ccc iff $\frak B$ is ccc.
(ii) Show that $\frak A$ is \wsid\ iff $\frak B$ is \wsid.
(iii) Show that $\frak A$ and
$\frak B$ have the same atoms, so that $\frak A$ is atomless, or purely
atomic, iff $\frak B$ is.
%316K

\spheader 316Xk\dvAformerly{3{}16Xi, 3{}16Xn and 3{}16Xt (previously
3{}16Xu)} Let $\frak A$ and $\frak B$ be Boolean algebras, and
$\pi:\frak A\to\frak B$ a surjective order-continuous Boolean homomorphism.
(i) Show that if $\frak A$ is ccc, then $\frak B$ is ccc.
(ii) Show that if $\frak A$ is \wsid, then $\frak B$ is \wsid.
(iii) Show that if $\frak A$ is atomless, then $\frak B$ is atomless.
(iv) Show that if $\frak A$ is purely atomic, then $\frak B$ is
purely atomic.
%316K

\spheader 316Xl\dvAnew{2009} Let $\frak A$ and $\frak B$ be Boolean algebras,
neither $\{0\}$, and
$\frak A\otimes\frak B$ their free product.   (i) Show that if
$\frak A\otimes\frak B$ is ccc, then $\frak A$ and $\frak B$ are both ccc.
(ii) Show that if $\frak A\otimes\frak B$ is \wsid,
then $\frak A$ and $\frak B$ are both \wsid.
(iii) Show that $\frak A\otimes\frak B$ is atomless iff either
$\frak A$ or $\frak B$ is atomless.
(iv) Show that $\frak A\otimes\frak B$ is purely atomic iff $\frak A$ and
$\frak B$ are both purely atomic.
%316K

\spheader 316Xm\dvAformerly{3{}16Xb, 3{}16Xk and 3{}16Xu (previously
3{}16Xv)} Let $\frak A$ be a Boolean algebra and $\frak A_a$ a
principal ideal of $\frak A$.
(i) Show that of $\frak A$ is ccc, then $\frak A_a$ is ccc.
(ii) Show that if $\frak A$ is \wsid, then $\frak A_a$ is \wsid.
(iii) Show that if $\frak A$ is atomless, then $\frak A_a$ is atomless.
(iv) Show that if $\frak A$ is purely atomic, then $\frak A_a$ is purely
atomic.
(v)\dvAnew{2009} Show that if $\frak A$ is homogeneous, then $\frak A_a$ is homogeneous.
%316Xg 316N

\spheader 316Xn\dvAformerly{3{}16Xc, 3{}16Xl and 3{}16Xv (previously
3{}16Xw)}
Let $\langle\frak A_i\rangle_{i\in I}$ be a
family of Boolean algebras, with simple product $\frak A$.
(i) Show that $\frak A$ is
ccc iff every $\frak A_i$ is ccc and $\{i:\frak A_i\ne\{0\}\}$ is
countable.
(ii) Show that $\frak A$ is \wsid\ iff every $\frak A_i$ is \wsid.
(iii) Show that $\frak A$ is atomless iff every $\frak A_i$ is atomless.
(iv) Show that $\frak A$
is purely atomic iff every $\frak A_i$ is purely atomic.
%316Xm 316N

\sqheader 316Xo\dvAformerly{3{}16Xd}
Let $X$ be a separable topological space.   Show that $X$ is ccc.
%316A%

\spheader 316Xp\dvAformerly{3{}16Xe}
Let $X$ be a topological space, and $\RO(X)$ its regular
open algebra.   Show that $X$ is ccc iff $\RO(X)$ is ccc.
%316A%

\spheader 316Xq\dvAformerly{3{}16Xo}
Let $X$ be a zero-dimensional compact
Hausdorff space.   Show that the regular open algebra of $X$ is \wsid\
iff the algebra of open-and-closed subsets of $X$ is \wsid.
%316Xj 316H%

\spheader 316Xr\dvAformerly{3{}16Xp}
Show that the algebra of open-and-closed subsets of
$\{0,1\}^{\Bbb N}$, with its usual topology, is ccc,
homogeneous
%new 2009
and not \wsid.
%316N

\spheader 316Xs\dvAformerly{3{}16Xe}
Show that the regular open algebra $\RO(\Bbb R)$ is ccc
and homogeneous.
% new 2009
%316O

\leader{316Y}{Further exercises (a)}
%\spheader 316Ya
Let $I$ be any set.   Show that $\{0,1\}^I$, with its usual topology, is
ccc.   \Hint{show that if $E\subseteq\{0,1\}^I$ is a
non-empty
open-and-closed set, then $\nu_IE>0$, where $\nu_I$ is the usual measure on
$\{0,1\}^I$.}
%316A

\spheader 316Yb\dvAformerly{3{}16Yc}
Let $\frak A$ be a Boolean algebra and $Z$ its Stone
space.   Show that $\frak A$ is ccc iff every nowhere dense subset of
$Z$ is included in a nowhere dense zero set.
%316B%

\spheader 316Yc\dvAformerly{3{}16Yd}
Let $X$ be a zero-dimensional topological space.
Show that $X$ is ccc iff the regular open algebra of $X$ is ccc iff the
algebra of open-and-closed subsets of $X$ is ccc.
%316B%

\spheader 316Yd\dvAformerly{3{}16Ye}
Set $X=\{0,1\}^{\omega_1}$, and for $\xi<\omega_1$
set $E_{\xi}=\{x:x\in X,\,x(\xi)=1\}$.   Let $\Sigma$ be the algebra of
subsets of $X$ generated by
$\{E_{\xi}:\xi<\omega_1\}\cup\{\{x\}:x\in X\}$,
and $\Cal I$ the $\sigma$-ideal of $\Sigma$ generated by
$\{E_{\xi}\cap E_{\eta}:\xi<\eta<\omega_1\}\cup\{\{x\}:x\in X\}$.  Show
that $\Sigma/\Cal I$ is not ccc, but that there is no uncountable
disjoint family in $\Sigma\setminus\Cal I$.
%316D%

\spheader 316Ye\dvAformerly{3{}16Yg}
Let $\frak A$ be a Boolean algebra.   $\frak A$ is
{\bf weakly $\sigma$-distributive} if whenever
$\sequencen{A_n}$ is a sequence of countable partitions of unity in
$\frak A$ then there is a partition $B$ of unity such that
$\{a:a\in A_n$, $a\Bcap b\ne 0\}$ is finite for every $b\in B$ and
$n\in\Bbb N$.   (Dedekind complete weakly $\sigma$-distributive algebras
are also called {\bf $\omega^{\omega}$-bounding}.)   $\frak A$ has the
{\bf Egorov property} if whenever
$\sequencen{A_n}$ is a sequence of countable partitions of unity in
$\frak A$ then there is a {\it countable} partition $B$ of unity such that
$\{a:a\in A_n$, $a\Bcap b\ne 0\}$ is finite for every $b\in B$ and
$n\in\Bbb N$.   (i) Show that if $\frak A$ is ccc
then it is \wsid\ iff it has the Egorov property iff it is weakly
$\sigma$-distributive.   (ii) Show that $\Cal P(\BbbN^{\Bbb N})$ does
not have the Egorov property, even though it is \wsid.   \Hint{try
$a_{nm}=\{f:f(n)=m\}$.}
%316H%

\spheader 316Yf\dvAnew{2009} Let $\frak A$ be a Dedekind $\sigma$-complete
Boolean algebra with the Egorov property
and $I$ a $\sigma$-ideal of $\frak A$.   Show that $\frak A/I$ has the
Egorov property.
%316Ye 316H

\spheader 316Yg\dvAformerly{3{}16Yf}
Let $X$ be a regular topological space and $\RO(X)$ its
regular open algebra.   Show that $\RO(X)$ is \wsid\ iff every meager
set in $X$ is nowhere dense.
%316I%

\spheader 316Yh\dvAformerly{3{}16Yh}
Let $\frak A$ be a Boolean algebra and $Z$ its Stone
space.   (i) Show that $\frak A$ is weakly $\sigma$-distributive iff the
union of any sequence of nowhere dense zero sets in $Z$ is nowhere
dense.
(ii) Show that $\frak A$ has the Egorov property iff the union of any
sequence of nowhere dense zero sets in $Z$ is included in a nowhere
dense zero set.
%316I%

\spheader 316Yi Let $\frak A$ be a Dedekind $\sigma$-complete \wsid\
Boolean algebra, $Z$ its Stone space, $\Cal E$ the algebra of
open-and-closed subsets of $Z$, $\Cal M$ the $\sigma$-ideal of meager
subsets of $Z$, and $\Sigma$ the algebra
$\{E\symmdiff M:E\in\Cal E,\,M\in\Cal M\}$, as in 314M.   (i) Let
$f:Z\to\Bbb R$ be a function.   Show that $f$ is
$\Sigma$-measurable iff there is a dense open set
$G\subseteq Z$ such that $f\restr G$ is continuous.   (ii) Now suppose
that $\frak A$ is Dedekind complete and that $f:Z\to\Bbb R$ is a
bounded function.   Show that $f$ is $\Sigma$-measurable iff there is
a continuous function $g:Z\to\Bbb R$ such that $\{z:f(z)\ne g(z)\}$
is meager.   \Hint{if $G$ is a dense open set and $f\restr G$ is
continuous, the closure in $Z\times\Bbb R$ of the graph of
$f\restr G$ is a function, because $Z$ is extremally disconnected.}
%316I%

\spheader 316Yj\dvAformerly{3{}16Yb}
Show that the Stone space of $\RO(\Bbb R)$ is separable.
More generally, show
that if a topological space $X$ is separable so is the Stone space of
its regular open algebra.
%316J%

\spheader 316Yk\dvAformerly{3{}16Yj}(i) Let $X$ be a non-empty separable
Hausdorff space
without isolated points.   Show that its regular open algebra is not
\wsid.   (ii)
Let $(X,\rho)$ be a non-empty metric space without isolated points.
Show that its regular open algebra is not \wsid.   (iii) Let $I$ be any
infinite
set.   Show that the algebra of open-and-closed subsets of $\{0,1\}^I$
is not \wsid.   Show that the regular open algebra of $\{0,1\}^I$ is not
\wsid.
%316J%

\wheader{316Yl}{0}{0}{0}{36pt}

\spheader 316Yl\dvAformerly{3{}16Yk} For any set $X$, write

\Centerline{$\Cal C_X=\{I:I\subseteq X$ is finite$\}
  \cup\{X\setminus I:I\subseteq X$ is finite$\}$.}

\noindent (i) Show that $\Cal C_X$ is an algebra of subsets of $X$ (the
{\bf finite-cofinite algebra}).   (ii) Show that a Boolean algebra is
purely atomic iff it has an order-dense subalgebra isomorphic  to the
finite-cofinite algebra of some set.   (iii) Show that a Dedekind
$\sigma$-complete Boolean algebra is purely atomic iff it has an
order-dense subalgebra isomorphic to the countable-cocountable algebra
of some set.
%316K%

\spheader 316Ym\dvAformerly{3{}16Yl} Let $\langle\frak A_i\rangle_{i\in I}$ be a family
of Boolean algebras, none of them $\{0\}$, with free product $\frak A$.
(i) Show that $\frak A$ is purely atomic iff every $\frak A_i$ is purely
atomic and $\{i:\frak A_i\ne\{0,1\}\}$ is finite.   (ii) Show that
$\frak A$ is atomless iff either some $\frak A_i$ is atomless or
$\{i:\frak A_i\ne\{0,1\}\}$ is infinite.
%316K

\spheader 316Yn\dvAformerly{3{}16Yp}
Let $X$ be a Hausdorff space and $\RO(X)$ its
regular open algebra.   (i) Show that the atoms of $\RO(X)$ are
precisely
the sets $\{x\}$ where $x$ is an isolated point in $X$.  (ii) Show that
$\RO(X)$ is atomless iff $X$ has no isolated points.   (iii) Show that
$\RO(X)$ is purely atomic iff the set of isolated points in $X$ is
dense in $X$.
%316L

\spheader 316Yo\dvAformerly{3{}16Yn} Show that a Boolean algebra is isomorphic to
$\RO(\Bbb R)$ iff it is atomless, Dedekind complete,
has a countable order-dense subalgebra and is not $\{0\}$.
%316M 316K

\spheader 316Yp\dvAnew{2009}
Let $\familyiI{\frak A_i}$ be a family of Boolean algebras,
none of them $\{0\}$, and $\frak A$ their simple product.   Show that
$\frak A$ is homogeneous iff (i) $\frak A_i$ is isomorphic to $\frak A_j$
for all $i$, $j\in I$ (ii) for every $i\in I$ there is a partition of
unity $A\subseteq\frak A_i\setminus\{0\}$ with $\#(A)=\#(I)$.
%316N

\spheader 316Yq\dvAnew{2009} Let $\frak A$ be a Boolean algebra such that
$\{d:d\in\frak A$, $\frak A_d\cong\frak A\}$ is order-dense in $\frak A$.
Show that the Dedekind completion $\widehat{\frak A}$ is homogeneous.
%316O

\spheader 316Yr\dvAformerly{3{}16Yo} Write $[\Bbb N]^{<\omega}$ for the ideal of
$\Cal P\Bbb N$ consisting of the finite subsets of $\Bbb N$.   Show that
$\Cal P\Bbb N/[\Bbb N]^{<\omega}$ is atomless, \wsid\ and not ccc,
and that its Dedekind completion is homogeneous.
%new 2009
%316K 316O 316Yq

\spheader 316Ys\dvAnew{2009}
Show that the regular open algebra of $\{0,1\}^I$ is
homogeneous for any infinite set $I$.
%316P %new 2009

}%end of exercises

\cmmnt{
\Notesheader{316} The phrase `countable chain condition' is perhaps
unfortunate, since the disjoint sets to which the definition 316A refers
could more naturally be called `antichains';  but there is in fact a
connexion between countable chains and countable antichains (316Xa).
While some authors speak of the `countable antichain condition' or
`cac', the term
`ccc' has become solidly established.   In the Boolean algebra context,
it could equally well be called the `countable sup property' (316E).

The countable chain condition can be thought of as a restriction on the
`width' of a Boolean algebra;  it means that the algebra cannot spread
too far laterally (see 316Xn(i)), though it may be indefinitely complex in
other ways.   Generally it means that in a wide variety of contexts we
need look
only at countable families and monotonic sequences, rather than
arbitrary
families and directed sets (316E, 316F, 316Ye).   Many of the ideas of
316C-316E %316C 316D 316E
have already appeared in 215B;  see 322G below.

I remarked in the notes to \S313 that the distributive laws described in
313B have important generalizations, of which
`weak $(\sigma,\infty)$-distributivity' and its cousin
`weak $\sigma$-distributivity' (316Ye) are two.
They are characteristic of the measure algebras which are the chief
subject of this volume.   The `Egorov property' (316Ye again) is an
alternative formulation applicable to ccc spaces.

Of course every property of Boolean algebras has a reflection in a
topological property of their Stone spaces;  happily,
most of the concepts of
this section correspond to reasonably natural topological expressions
(316B, 316I, 316L, 316Yh).   `Homogeneity' is the odd one out.   In fact
only the definition of `homogeneous' Boolean algebra is particularly worth
noting at this stage.   The homogeneous algebras we are primarily
interested in will appear in \S331, and they are too special for any
general theory to be very helpful.

With five new properties (ccc, \wsid, atomless, purely atomic,
homogeneous) to
incorporate into the constructions of the last few sections, a very
large number of questions can be asked;   most are elementary.
In 316Xh-316Xn    % 316Xh 316Xi 316Xj 316Xk 316Xl 316Xm 316Xn
I list the properties which are inherited by subalgebras, order-continuous
homomorphic images, free products, principal ideals and simple products.
The countable chain
condition is so important that it is worth noting that a sequentially
order-continuous image of a ccc algebra is ccc (316Xb), and that there
is a useful necessary and sufficient condition for a sequentially
order-continuous image of a $\sigma$-complete algebra to be ccc (316C,
316D, 316Xc;  but see also 316Yd).   To see that sequentially
order-continuous images do not inherit weak
$(\sigma,\infty)$-distributivity, recall that the regular open algebra
of $\Bbb R$ is isomorphic to the quotient of the
Baire-property algebra $\widehat{\Cal B}$ of $\Bbb R$ by the meager
ideal $\Cal M$
(314Yd);  but that $\widehat{\Cal B}$ is purely atomic (since it
contains all singletons), therefore \wsid\ (316Xe).   Similarly,
$\Cal P\Bbb N/[\Bbb N]^{<\omega}$ is a non-ccc image of a ccc algebra
(316Yr).   Free products of \wsid\ algebras need not be \wsid\ (325Ye).
There are important cases in which the simple product of homogeneous
algebras is homogeneous (316Yp).

The definitions here provide a language in which a remarkably
interesting question can be asked:  is the free product of ccc Boolean
algebras always
ccc?   equivalently, is the product of ccc topological spaces always
ccc?   What is special about this question is that it cannot be answered
within the ordinary rules of mathematics (even including the axiom of
choice);  it is
undecidable, in the same way that the continuum hypothesis is.   I will
deal with a variety of undecidable questions in Volume 5;  this particular
one is mentioned in 516U, 517Xe and 553J.

I have taken the opportunity to mention three of the most important of
all Boolean algebras:  the algebra of open-and-closed subsets of
$\{0,1\}^{\Bbb N}$ (316M, 316Xr),
the regular open algebra of $\Bbb R$ (316J, 316Xs, 316Yo) and the quotient
$\Cal P\Bbb N/[\Bbb N]^{<\omega}$ (316Yr).   A fourth algebra which
belongs in
this company is the Lebesgue measure algebra, which is atomless, ccc,
\wsid\ and homogeneous (so that
every countable subset of its Stone space $Z$ is nowhere dense, and $Z$
is a non-separable ccc space);  but for this I wait for the next
chapter.
}%end of notes

\frnewpage
