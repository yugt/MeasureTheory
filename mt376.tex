\frfilename{mt376.tex}
\versiondate{8.4.10}
\copyrightdate{1996}
\def\esssup{\mathop{\text{ess sup}}}

\def\chaptername{Linear operators between function spaces}
\def\sectionname{Kernel operators}

\newsection{376}

The theory of linear integral equations is in large part the theory of
operators $T$ defined from formulae of the type

\Centerline{$(Tf)(y)=\int k(x,y)f(x)dx$}

\noindent for some function $k$ of two variables.   I make no attempt to
study the general theory here.   However, the concepts developed in this
book make it easy to discuss certain aspects of such operators defined
between the `function spaces' of measure theory, meaning spaces of
equivalence classes of functions, and indeed allow us to do some of the
work in the abstract theory of Riesz spaces, omitting all formal mention
of measures (376D, 376H, 376P).   I give a very brief account of two
theorems characterizing kernel operators in the abstract (376E, 376H),
with corollaries to show the form these theorems can take in the
ordinary language of integral kernels (376J, 376N).   To give an idea of
the kind of results we can hope for in this area, I go a bit farther
with operators with domain $L^1$ (376Mb, 376P, 376S).

I take the opportunity to spell out versions of results from \S253 in
the language of this volume (376B-376C).

\cmmnt{
\leader{376A}{Kernel operators} To give an idea of where this
section is going, I will try to describe the central idea in a
relatively concrete special case.   Let $(X,\Sigma,\mu)$ and
$(Y,\Tau,\nu)$ be $\sigma$-finite measure spaces;  you can take them
both to be $[0,1]$ with Lebesgue measure if you like.   Let $\lambda$ be
the product measure on $X\times Y$.   If $k\in\eusm L^1(\lambda)$, then
$\int k(x,y)dx$ is defined for almost every $y$, by Fubini's theorem;
so if $f\in\eusm L^{\infty}(\mu)$ then $g(y)=\int k(x,y)f(x)dx$
is defined for almost every $y$.   Also

\Centerline{$\int g(y)dy=\int k(x,y)f(x)dxdy$}

\noindent is defined, because $(x,y)\mapsto k(x,y)f(x)$ is
$\lambda$-virtually measurable, defined $\lambda$-a.e.\ and is
dominated by a multiple of the integrable function $k$.
Thus $k$ defines a function from $\eusm L^{\infty}(\mu)$ to $\eusm
L^1(\nu)$.   Changing $f$ on a set of measure $0$ will not change $g$,
so we can think of this as an operator from $L^{\infty}(\mu)$ to $\eusm
L^1(\nu)$;  and of course we can move immediately to the equivalence
class of $g$ in $L^1(\nu)$, so getting an operator $T_k$ from
$L^{\infty}(\mu)$ to $L^1(\nu)$.   This operator is plainly linear;
also it is easy to check that $\pm T_k\le T_{|k|}$, so that $T_k\in\eurm
L^{\sim}(L^{\infty}(\mu);L^1(\nu))$, and that $\|T_k\|\le\int|k|$.
Moreover, changing $k$ on a $\lambda$-negligible set does not change
$T_k$, so that in fact we can speak of $T_w$ for any $w\in
L^1(\lambda)$.

I think it is obvious, even before investigating them, that operators
representable in this way will be important.   We can immediately ask
what their properties will be and whether there is any straightforward
way of recognising them.   We can look at the properties of the map
$w\mapsto T_w:L^1(\lambda)\to\eurm L^{\sim}(L^{\infty}(\mu);L^1(\nu))$.
And we can ask what happens when $L^{\infty}(\mu)$ and $L^1(\nu)$ are
replaced by other function spaces, defined by extended Fatou norms or
otherwise.   Theorems 376E and 376H are answers to questions of this kind.

It turns out that the formula $g(y)=\int k(x,y)f(x)dx$ gives rise to a
variety of technical problems, and it is much easier to characterize
$Tu$ in terms of its action on the dual.   In the language of the
special case above, if $h\in\eusm L^{\infty}(\nu)$, then we shall have

\Centerline{$\int k(x,y)f(x)h(y)d(x,y)=\int g(y)h(y)dy$;}

\noindent since $g^{\ssbullet}\in L^1(\nu)$ is entirely determined by
the integrals $\int g(y)h(y)dy$ as $h$ runs over $\eusm
L^{\infty}(\nu)$, we can define the operator $T$ in terms of the
functional $(f,h)\mapsto\int k(x,y)f(x)h(y)d(x,y)$.   This enables us to
extend the results from the case of $\sigma$-finite spaces to general
strictly localizable spaces;  perhaps more to the point in the present
context, it gives them natural expressions in terms of function spaces
defined from measure algebras rather than measure spaces, as in 376E.

Before going farther along this road, however, I give a couple of
results relating the theorems of \S253 to the methods of this volume.
}%end of comment

\leader{376B}{The canonical map $L^0\times L^0\to L^0$:  Proposition}
Let $(\frak A,\bar\mu)$ and $(\frak B,\bar\nu)$ be semi-finite measure
algebras, and $(\frak C,\bar\lambda)$ their localizable measure algebra
free product\cmmnt{ (325E)}.   Then we have a bilinear operator
$(u,v)\mapsto u\otimes v:
L^0(\frak A)\times L^0(\frak B)\to L^0(\frak C)$ with the following
properties.

(a) For any $u\in L^0(\frak A)$, $v\in L^0(\frak B)$ and $\alpha\in\Bbb R$,

\Centerline{$\Bvalue{u\otimes\chi 1_{\frak B}>\alpha}
=\Bvalue{u>\alpha}\otimes 1_{\frak B}$,
\quad $\Bvalue{\chi 1_{\frak A}\otimes v>\alpha}
=1_{\frak A}\otimes\Bvalue{v>\alpha}$}

\noindent where for $a\in\frak A$, $b\in\frak B$ I write $a\otimes b$
for the corresponding member of
$\frak A\otimes\frak B$\cmmnt{ (315N)}, identified with a subalgebra
of $\frak C$\cmmnt{ (325Dc)}.

(b)(i) For any $u\in L^0(\frak A)^+$, the map
$v\mapsto u\otimes v:L^0(\frak B)\to L^0(\frak C)$ is an
order-continuous multiplicative Riesz homomorphism.

\quad (ii) For any $v\in L^0(\frak B)^+$, the map
$u\mapsto u\otimes v:L^0(\frak A)\to L^0(\frak C)$ is an
order-continuous multiplicative Riesz homomorphism.

(c) In particular, $|u\otimes v|=|u|\otimes|v|$ for all
$u\in L^0(\frak A)$ and $v\in L^0(\frak B)$.

(d) For any $u\in L^0(\frak A)^+$ and $v\in L^0(\frak B)^+$,
$\Bvalue{u\otimes v>0}=\Bvalue{u>0}\otimes\Bvalue{v>0}$.

\proof{ The canonical maps $a\mapsto a\otimes 1_{\frak B}$,
$b\mapsto 1_{\frak A}\otimes b$ from $\frak A$, $\frak B$ to $\frak C$
are order-continuous Boolean homomorphisms (325Da), so induce
order-continuous multiplicative Riesz homomorphisms from $L^0(\frak A)$
and $L^0(\frak B)$ to $L^0(\frak C)$ (364P);  write $\tilde u$,
$\tilde v$ for the images of
$u\in L^0(\frak A)$, $v\in L^0(\frak B)$.   Observe that
$|\tilde u|=|u|^{\sim}$, $|\tilde v|=|v|^{\sim}$ and
$(\chi 1_{\frak A})^{\sim}=(\chi 1_{\frak B})^{\sim}=\chi 1_{\frak C}$.
Now set $u\otimes v=\tilde u\times\tilde v$.   The properties
listed in (a)-(c) are just a matter of putting the definition in
364Pa together with the fact that $L^0(\frak C)$ is an $f$-algebra
(364D).   As for
$\Bvalue{u\otimes v>0}=\Bvalue{\tilde u\times\tilde v>0}$, this is (for
non-negative $u$, $v$) just

\Centerline{$\Bvalue{\tilde u>0}\Bcap\Bvalue{\tilde v>0}
=(\Bvalue{u>0}\otimes 1_{\frak B})
  \Bcap(1_{\frak A}\otimes\Bvalue{v>0})
=\Bvalue{u>0}\otimes\Bvalue{v>0}$.}
}%end of proof of 376B

\leader{376C}{}\cmmnt{ For $L^1$ spaces we have a similar result, with
additions corresponding to the Banach lattice structures of the three
spaces.

\medskip

\noindent}{\bf Theorem} Let $(\frak A,\bar\mu)$ and $(\frak B,\bar\nu)$
be semi-finite measure algebras with localizable measure algebra free
product $(\frak C,\bar\lambda)$.

(a) If $u\in L^1_{\bar\mu}=L^1(\frak A,\bar\mu)$ and
$v\in L^1_{\bar\nu}=L^1(\frak B,\bar\nu)$ then
$u\otimes v\in L^1_{\bar\lambda}=L^1(\frak C,\bar\lambda)$ and

\Centerline{$\int u\otimes v=\int u\int v$,
\quad $\|u\otimes v\|_1=\|u\|_1\|v\|_1$.}

(b) Let $W$ be a Banach space and
$\phi:L^1_{\bar\mu}\times L^1_{\bar\nu}\to W$ a bounded bilinear operator.
Then there is a unique
bounded linear operator $T:L^1_{\bar\lambda}\to W$ such that
$T(u\otimes v)=\phi(u,v)$ for all $u\in L^1_{\bar\mu}$ and
$v\in L^1_{\bar\nu}$, and $\|T\|=\|\phi\|$.

(c) Suppose, in (b), that $W$ is a Banach lattice.   Then

\quad(i) $T$ is positive iff $\phi(u,v)\ge 0$ for all $u$, $v\ge 0$;

\quad(ii) $T$ is a Riesz homomorphism iff
$u\mapsto\phi(u,v_0):L^1_{\bar\mu}\to W$ and
$v\mapsto\phi(u_0,v):L^1_{\bar\nu}\to W$ are Riesz homomorphisms for all
$v_0\ge 0$ in $L^1_{\bar\nu}$ and $u_0\ge 0$ in $L^1_{\bar\mu}$.

\proof{{\bf (a)} I refer to the proof of 325D.   Let $(X,\Sigma,\mu)$
and $(Y,\Tau,\nu)$ be the Stone spaces of $(\frak A,\bar\mu)$ and
$(\frak B,\bar\nu)$ (321K), so that $(\frak C,\bar\lambda)$ can be
identified with the measure algebra of the c.l.d.\ product measure
$\lambda$ on $X\times Y$ (part (a) of the proof of 325D), and
$L^1_{\bar\mu}$, $L^1_{\bar\nu}$, $L^1_{\bar\lambda}$ can be identified
with $L^1(\mu)$, $L^1(\nu)$ and $L^1(\lambda)$ (365B).    Now if
$f\in\eusm L^0(\mu)$ and $g\in\eusm L^0(\nu)$ then
$f\otimes g\in\eusm L^0(\lambda)$ (253Cb), and it is easy to check that
$(f\otimes g)^{\ssbullet}\in L^0(\bar\lambda)$ corresponds to
$f^{\ssbullet}\otimes g^{\ssbullet}$ as defined in 376B.   (Look first
at the cases in which
one of $f$, $g$ is a constant function with value $1$.)     By 253E, we
have a canonical map
$(f^{\ssbullet},g^{\ssbullet})\mapsto(f\otimes g)^{\ssbullet}$ from
$L^1(\mu)\times L^1(\nu)$ to $L^1(\lambda)$, with
$\int f\otimes g=\int f\int g$ (253D);   so that if
$u\in L^1_{\bar\mu}$ and $v\in L^1_{\bar\nu}$ we must have
$u\otimes v\in L^1_{\bar\lambda}$, with $\int u\otimes v=\int u\int v$.
As in 253E, it follows that $\|u\otimes v\|_1=\|u\|_1\|v\|_1$.

\medskip

{\bf (b)} In view of the situation described in (a) above, this is now
just a translation of the same result about $L^1(\mu)$, $L^1(\nu)$ and
$L^1(\lambda)$, which is Theorem 253F.

\medskip

{\bf (c)} Identifying the algebraic free product $\frak A\otimes\frak B$
with its canonical image in $\frak C$ (325Dc), I write
$(\frak A\otimes\frak B)^f$ for
$\{c:c\in\frak A\otimes\frak B,\,\bar\lambda c<\infty\}$,
so that $(\frak A\otimes\frak B)^f$ is a subring of $\frak C$.
Recall that any member of $\frak A\otimes\frak B$ is expressible
as $\sup_{i\le n}a_i\otimes b_i$ where $a_0,\ldots,a_n$ are disjoint
(315Oa);  evidently this will belong to $(\frak A\otimes\frak B)^f$ iff
$\bar\mu a_i\cdot\bar\nu b_i$ is finite for every $i$.

The next fact to lift from previous theorems is in part (e) of the proof
of 253F:  the linear span $M$ of
$\{\chi(a\otimes b):a\in\frak A^f,\,b\in\frak B^f\}$ is norm-dense in
$L^1_{\bar\lambda}$.    Of
course $M$ can also be regarded as the linear span of
$\{\chi c:c\in(\frak A\otimes\frak B)^f\}$, or
$S(\frak A\otimes\frak B)^f$.
(Strictly speaking, this last remark relies on 361J;  the identity map
from $(\frak A\otimes\frak B)^f$ to $\frak C$ induces an injective Riesz
homomorphism from $S(\frak A\otimes\frak B)^f$ into
$S(\frak C)\subseteq L^0(\frak C)$.   To see that $\chi c\in M$ for
every $c\in(\frak A\otimes\frak B)^f$, we need to know that $c$ can be
expressed as a
disjoint union of members of $\frak A\otimes\frak B$, as noted above.)

\medskip

\quad{\bf (i)} If $T$ is positive then of course
$\phi(u,v)=T(u\otimes v)\ge 0$ whenever $u$, $v\ge 0$, since
$u\otimes v\ge 0$.   On the other
hand, if $\phi$ is non-negative on $U^+\times V^+$, then, in particular,
$T\chi(a\otimes b)=\phi(\chi a,\chi b)\ge 0$ whenever
$\bar\mu a\cdot\bar\nu b<\infty$.   Consequently $T(\chi c)\ge 0$ for
every $c\in(\frak A\otimes\frak B)^f$ and $Tw\ge 0$ whenever $w\ge 0$ in
$M\cong S(\frak A\otimes\frak B)^f$, as in 361Ga.

Now this means that $T|w|\ge 0$ whenever $w\in M$.   But as $M$ is
norm-dense in $L^1_{\bar\lambda}$, $w\mapsto T|w|$ is continuous and
$W^+$ is closed, it follows that $T|w|\ge 0$ for every
$w\in L^1_{\bar\lambda}$, that is, that $T$ is positive.

\medskip

\quad{\bf (ii)} If $T$ is a Riesz homomorphism then of course
$u\mapsto\phi(u,v_0)=T(u\otimes v_0)$ and
$v\mapsto\phi(u_0,v)=T(u_0\otimes
v)$ are Riesz homomorphisms for $v_0$, $u_0\ge 0$.   On the other hand,
if all these maps are Riesz homomorphisms, then, in particular,

$$\eqalign{T\chi(a\otimes b)\wedge T\chi(a'\otimes b')
&=\phi(\chi a,\chi b)\wedge\phi(\chi a',\chi b')\cr
&\le \phi(\chi a,\chi b+\chi b')\wedge\phi(\chi a',\chi b+\chi b')\cr
&=\phi(\chi a\wedge\chi a',\chi b+\chi b')
=0\cr}$$

\noindent whenever $a$, $a'\in\frak A^f$, $b$, $b'\in\frak B^f$ and
$a\Bcap a'=0$.   Similarly, $T\chi(a\otimes b)\wedge T\chi(a'\otimes
b')=0$ if $b\Bcap b'=0$.   But this means that $T\chi c\wedge T\chi
c'=0$ whenever $c$, $c'\in(\frak A\otimes\frak B)^f$ and $c\Bcap c'=0$.
\Prf\ Express $c$, $c'$ as $\sup_{i\le m}a_i\otimes b_i$, $\sup_{j\le
n}a'_j\otimes b'_j$ where $a_i$, $a'_j$, $b_i$, $b'_j$ all have finite
measure.   Now if $i\le m$, $j\le n$, $(a_i\Bcap a'_j)\otimes(b_i\Bcap
b'_j)=(a_i\otimes b_i)\Bcap(a'_j\otimes b'_j)=0$, so one of $a_i\Bcap
a'_j$, $b_i\Bcap b'_j$ must be zero, and in either case
$T\chi(a_i\otimes b_i)\wedge T\chi(a'_j\otimes b'_j)=0$.   Accordingly

$$\eqalign{T\chi c\wedge T\chi c'
&\le(\sum_{i=0}^mT\chi(a_i\otimes b_i))
   \wedge(\sum_{j=0}^nT\chi(a'_j\times b'_j))\cr
&\le\sum_{i=0}^m\sum_{j=0}^nT\chi(a_i\otimes b_i)
   \wedge T\chi(a'_j\otimes b'_j)
=0,\cr}$$

\noindent using 352Fa for the second inequality.\ \Qed

This implies that $T\restr M$ must be a Riesz homomorphism (361Gc), that
is, $T|w|=|Tw|$ for all $w\in M$.   Again because $M$ is dense in
$L^1_{\bar\lambda}$, $T|w|=|Tw|$ for every $w\in L^1_{\bar\lambda}$, and
$T$ is a Riesz homomorphism.
}%end of proof of 376C

\leader{376D}{Abstract integral operators:  Definition}\cmmnt{ The
following concept will be used repeatedly in the theorems below;  it is
perhaps worth giving it a name.}   Let $U$ be a Riesz space and $V$ a
Dedekind complete Riesz space\cmmnt{, so that $\eurm L^{\times}(U;V)$
is a Dedekind complete Riesz space (355H)}.   If $f\in U^{\times}$ and
$v\in V$ write $P_{fv}u=f(u)v$ for each $u\in U$;  then
$P_{fv}\in\eurm L^{\times}(U;V)$.   \prooflet{\Prf\ If $f\ge 0$ in
$U^{\times}$ and $v\ge 0$ in $V^{\times}$ then $P_{fv}$ is a positive
linear operator from $U$ to $V$ which is
order-continuous because if $A\subseteq U$ is non-empty,
downwards-directed and has infimum $0$, then (as $V$ is Archimedean)

\Centerline{$\inf_{u\in A}P_{fv}(u)=\inf_{u\in A}f(u)v=0$.}

\noindent Of course $(f,g)\mapsto P_{fg}$ is bilinear, so
$P_{fv}\in \eurm L^{\times}(U;V)$ for every $f\in U^{\times}$,
$v\in V$.\ \QeD\ }%end of prooflet
\cmmnt{Now} I call a linear operator from $U$ to $V$ an {\bf abstract integral
operator} if it is in the band in $\eurm L^{\times}(U;V)$ generated by
$\{P_{fv}:f\in U^{\times},\,v\in V\}$.

\cmmnt{The first result describes these operators when $U$, $V$ are
expressed as subspaces of $L^0(\frak A)$, $L^0(\frak B)$ for measure
algebras $\frak A$, $\frak B$ and $V$ is perfect.}

\leader{376E}{Theorem} Let $(\frak
A,\bar\mu)$ and $(\frak B,\bar\nu)$ be semi-finite measure algebras,
with localizable measure algebra free product $(\frak C,\bar\lambda)$,
and $U\subseteq L^0(\frak A)$, $V\subseteq L^0(\frak B)$ order-dense
Riesz subspaces.   Write $W$ for the set of those $w\in L^0(\frak C)$
such that $w\times(u\otimes v)$ is integrable for every $u\in U$ and
$v\in V$.   Then we have an operator $w\mapsto T_w:W\to \eurm
L^{\times}(U;V^{\times})$ defined by setting

\Centerline{$T_w(u)(v)=\int w\times(u\otimes v)$}

\noindent for every $w\in W$, $u\in U$ and $v\in V$.   The map
$w\mapsto T_w$ is a Riesz space isomorphism between $W$ and the band of
abstract integral operators in $\eurm L^{\times}(U;V^{\times})$.

\proof{{\bf (a)} The first thing to check is that the formula offered
does define a member $T_w(u)$ of $V^{\times}$ for any $w\in W$ and
$u\in U$.   \Prf\    Of course $T_w(u)$ is a linear operator because
$\int$ is linear and $\otimes$ and $\times$ are bilinear.   It belongs to
$V^{\sim}$ because, writing $g(v)=\int |w|\times(|u|\otimes v)$, $g$ is
a positive linear operator and $|T_w(u)(v)|\le g(|v|)$ for every $v$.
(I am here using 376Bc to see that $|w\times(u\otimes
Fv)|=|w|\times(|u|\otimes|v|)$.)   Also $g\in V^{\times}$ because
$v\mapsto|u|\otimes v$, $w'\mapsto |w|\times w'$ and $\int$ are all
order-continuous;  so $T_w(u)$ also belongs to $V^{\times}$.\ \Qed

\medskip

{\bf (b)} Next, for any given $w\in W$, the map $T_w:U\to V^{\times}$ is
linear (again because $\otimes$ and $\times$ are bilinear).   It is
helpful to note that $W$ is a solid linear subspace of $L^0(\frak C)$.
Now if $w\ge 0$ in $W$, then $T_w\in\eurm L^{\times}(U;V^{\times})$.
\Prf\ If $u$, $v\ge 0$ then $u\otimes v\ge 0$,
$w\times(u\otimes v)\ge 0$ and $T_w(u)(v)\ge 0$;   as $v$ is arbitrary,
$T_w(u)\ge 0$ whenever $u\ge 0$;  as
$u$ is arbitrary, $T_w$ is positive.   If $A\subseteq U$ is non-empty,
downwards-directed and has infimum $0$, then $T_w[A]$ is
downwards-directed, and for any $v\in V^+$

\Centerline{$(\inf T_w[A])(v)=\inf_{u\in A}T_w(u)(v)
=\inf_{u\in A}\int w\times(u\otimes v)=0$}

\noindent because $u\mapsto u\otimes v$ is order-continuous.   So $\inf
T_w[A]=0$;  as $A$ is arbitrary, $T_w$ is order-continuous.\ \Qed\

For general $w\in W$, we now have $T_w=T_{w^+}-T_{w^-}\in
L^{\times}(U;V^{\times})$.

\medskip

{\bf (c)} Ths shows that $w\mapsto T_w$ is a map from $W$ to
$L^{\times}(U;V^{\times})$.   Running through the formulae once again,
it is linear, positive and order-continuous;  this last because, given a
non-empty downwards-directed $C\subseteq W$ with infimum $0$, then for
any $u\in U^+$, $v\in V^+$

\Centerline{$(\inf_{w\in C}T_w)(u)(v)
\le\inf_{w\in C}\int w\times(u\otimes v)=0$}

\noindent (because $\int$ and $\times$ are order-continuous);  as $v$ is
arbitrary, $(\inf_{w\in C}T_w)(u)=0$;  as $u$ is arbitrary, $\inf_{w\in
C}T_w=0$.

\medskip

{\bf (d)} All this is easy, being nothing but a string of applications
of the elementary properties of $\otimes$, $\times$ and $\int$.   But I
think a new idea is needed for the next fact:  the map $w\mapsto
T_w:W\to\eurm L^{\times}(U;V^{\times})$ is a Riesz homomorphism.   \Prf\
Write $\frak D$ for the set of those $d\in\frak C$ such that $T_w\wedge
T_{w'}=0$ whenever $w$, $w'\in W^+$, $\Bvalue{w>0}\Bsubseteq d$ and
$\Bvalue{w'>0}\Bsubseteq 1_{\frak C}\Bsetminus d$.
(i) If $d_1$, $d_2\in \frak D$, $w$, $w'\in W^+$,
$\Bvalue{w>0}\Bsubseteq d_1\Bcup d_2$ and $\Bvalue{w'>0}\Bcap(d_1\Bcup
d_2)=0$, then set $w_1=w\times\chi d_1$, $w_2=w-w_1$.   In this case

\Centerline{$\Bvalue{w_1>0}\Bsubseteq d_1$,
\quad$\Bvalue{w_2>0}\Bsubseteq d_2$,}

\noindent so

\Centerline{$T_{w_1}\wedge T_{w'}=T_{w_2}\wedge T_{w'}=0$,
\quad$T_w\wedge T_{w'}
\le(T_{w_1}\wedge T_{w'})+(T_{w_2}\wedge T_{w'})=0$.}

\noindent As $w$, $w'$ are arbitrary, $d_1\Bcup d_2\in \frak D$.   Thus
$\frak D$ is closed under $\Bcup$.   (ii) The symmetry of the definition
of $\frak D$ means that $1_{\frak C}\Bsetminus d\in \frak D$ whenever
$d\in \frak D$.   (iii) Of course $0\in \frak D$, just because $T_w=0$
if $w\in W^+$ and $\Bvalue{w>0}=0$;  so $\frak D$ is a subalgebra of
$\frak C$.   (iv) If $D\subseteq \frak D$ is non-empty and
upwards-directed, with supremum $c$ in $\frak C$, and if $w$, $w'\in
W^+$ are such that $\Bvalue{w>0}\Bsubseteq c$, $\Bvalue{w'>0}\Bcap c=0$,
then consider $\{w\times\chi d:d\in D\}$.   This is upwards-directed,
with supremum $w$;  so $T_w=\sup_{d\in D}T_{w\times\chi d}$, because the
map $q\mapsto T_q$ is order-continuous.   Also $T_{w\times\chi d}\wedge
T_{w'}=0$ for every $d\in D$, so $T_w\wedge T_{w'}=0$.   As $w$, $w'$
are arbitrary, $c\in\frak D$;  as $D$ is arbitrary, $\frak D$ is an
order-closed subalgebra of $\frak C$.   (v) If $a\in\frak A$ and $w$,
$w'\in W^+$ are such that $\Bvalue{w>0}\Bsubseteq a\otimes 1_{\frak B}$ and
$\Bvalue{w'>0}\Bcap(a\otimes 1_{\frak B})=0$, then any $u\in U^+$ is
expressible as $u_1+u_2$ where $u_1=u\times\chi a$,
$u_2=u\times\chi(1_{\frak A}\Bsetminus a)$.   Now

\Centerline{$T_w(u_2)(v)=\int w\times(u_2\otimes v)
=\int w\times\chi(a\otimes 1_{\frak B})\times(u\otimes v)
  \times\chi((1_{\frak A}\Bsetminus a)\otimes 1_{\frak B})
=0$}

\noindent for every $v\in V$, so $T_w(u_2)=0$.   Similarly,
$T_{w'}(u_1)=0$.   But this means that

\Centerline{$(T_w\wedge T_{w'})(u)\le T_w(u_2)+T_{w'}(u_1)=0$.}

\noindent As $u$ is arbitrary, $T_w\wedge T_{w'}=0$;  as $w$ and $w'$
are arbitrary, $a\otimes 1_{\frak B}\in \frak D$.   (vi)  Now suppose
that $b\in\frak B$ and that $w$, $w'\in W^+$ are such that
$\Bvalue{w>0}\Bsubseteq 1_{\frak A}\otimes b$ and
$\Bvalue{w'>0}\Bcap(1_{\frak A}\otimes b)=0$.   If $u\in U^+$ and
$v\in V^+$ then

\Centerline{$(T_w\wedge T_{w'})(u)(v)
\le\int w\times(u\otimes(v\times\chi(1_{\frak B}\Bsetminus b)))
  +\int w'\times(u\otimes(v\times\chi b))
=0$.}

\noindent As $u$, $v$ are arbitrary, $T_w\wedge T_{w'}=0$;  as $w$ and
$w'$ are arbitrary, $1_{\frak A}\otimes b\in\frak D$.   (vii) This means
that $\frak D$ is an order-closed subalgebra of $\frak C$ including
$\frak A\otimes\frak B$, and is therefore the whole of $\frak C$
(325D(c-ii)).   (viii) Now take any $w$, $w'\in W$ such that
$w\wedge w'=0$,
and consider $c=\Bvalue{w>0}$.   Then $\Bvalue{w'>0}\Bsubseteq 1_{\frak
C}\Bsetminus c$ and $c\in\frak D$, so $T_w\wedge T_{w'}=0$.   This is
what we need to be sure that $w\mapsto T_w$ is a Riesz homomorphism
(352G).\ \Qed

\medskip

{\bf (e)} The map $w\mapsto T_w$ is injective.   \Prf\ (i) If $w>0$ in
$W$, then consider

\Centerline{$A=\{a:a\in\frak A,\,\exists\,u\in U,\,\chi a\le u\}$,
\quad$B=\{b:b\in\frak B,\,\exists\,v\in V,\,\chi b\le v\}$.}

\noindent Because $U$ and $V$ are order-dense in $L^0(\frak A)$ and
$L^0(\frak B)$ respectively, $A$ and $B$ are order-dense in $\frak A$
and $\frak B$.   Also both are upwards-directed.   So
$\sup_{a\in A,b\in B}a\otimes b=1_{\frak C}$ and
$0<\int w=\sup_{a\in A,b\in B}\int_{a\otimes b}w$.   Take $a\in A$,
$b\in B$ such that
$\int_{a\otimes b}w>0$;  then there are $u\in U$, $v\in V$ such that
$\chi a\le u$ and $\chi b\le v$, so that

\Centerline{$T_w(u)(v)\ge\int_{a\otimes b}w>0$}

\noindent and $T_w>0$.   (ii) For general non-zero $w\in W$, we now have
$|T_w|=T_{|w|}>0$ so $T_w\ne 0$.\ \Qed

Thus $w\mapsto T_w$ is an order-continuous injective Riesz homomorphism.

\medskip

{\bf (f)} Write $\tilde W$ for $\{T_w:w\in W\}$, so that $\tilde W$ is a
Riesz subspace of $\eurm L^{\times}(U;V^{\times})$ isomorphic to $W$,
and $\widehat{W}$ for the band it generates in
$\eurm L^{\times}(U;V^{\times})$.   Then $\tilde W$ is order-dense in
$\widehat{W}$.   \Prf\ Suppose that $S>0$ in
$\widehat{W}=\tilde W^{\perp\perp}$ (353Ba).   Then
$S\notin\tilde W^{\perp}$, so
there is a $w\in W$ such that $S\wedge T_w>0$.
Set $w_1=w\wedge\chi 1_{\frak C}$.   Then
$w=\sup_{n\in\Bbb N}w\wedge nw_1$, so
$T_w=\sup_{n\in\Bbb N}T_w\wedge nT_{w_1}$ and $R=S\wedge T_{w_1}>0$.

Set $U_1=U\cap L^1(\frak A,\bar\mu)$.   Because $U$ is an order-dense
Riesz subspace of $L^0(\frak A)$, $U_1$ is an order-dense Riesz subspace
of $L^1_{\bar\mu}=L^1(\frak A,\bar\mu)$, therefore also norm-dense.
Similarly $V_1=V\cap L^1(\frak B,\bar\nu)$ is a norm-dense Riesz
subspace of $L^1_{\bar\nu}=L^1(\frak B,\bar\nu)$.   Define
$\phi_0:U_1\times V_1\to\Bbb R$ by setting $\phi_0(u,v)=R(u)(v)$ for
$u\in U_1$ and $v\in V_1$.   Then $\phi_0$ is bilinear, and

$$\eqalign{|\phi_0(u,v)|
&=|R(u)(v)|\le|R(u)|(|v|)
\le R(|u|)(|v|)
\le T_{w_1}(|u|)(|v|)\cr
&=\biggerint w_1\times(|u|\otimes|v|)
\le\biggerint|u|\otimes|v|
=\|u\|_1\|v\|_1\cr}$$

\noindent for all $u\in U_1$, $v\in V_1$, because $0\le R\le T_{w_1}$ in
$\eurm L^{\times}(U;V^{\times})$.   Because $U_1$, $V_1$ are
norm-dense in $L^1_{\bar\mu}$, $L^1_{\bar\nu}$ respectively, $\phi_0$
has a unique extension to a continuous bilinear operator
$\phi:L^1_{\bar\mu}\times L^1_{\bar\nu}\to\Bbb R$.    (To reduce this to
standard results on linear operators, think of $R$ as a function from
$U_1$ to $V_1^*$;  since every member of $V_1^*$ has a unique extension
to a member of $(L^1_{\bar\nu})^*$, we get a corresponding function
$R_1:U_1\to(L^1_{\bar\nu})^*$ which is continuous and linear, so has a
unique extension to a continuous linear operator
$R_2:L^1_{\bar\mu}\to(L^1_{\bar\nu})^*$, and we set
$\phi(u,v)=R_2(u)(v)$.)

By 376C, there is a unique
$h\in(L^1_{\bar\lambda})^*=L^1(\frak C,\bar\lambda)^*$
such that $h(u\otimes v)=\phi(u,v)$ for every $u\in L^1_{\bar\mu}$ and
$v\in L^1_{\bar\nu}$.   Because $(\frak C,\bar\lambda)$
is localizable, this $h$ corresponds to a $w'\in L^{\infty}(\frak C)$
(365Mc), and

\Centerline{$\int w'\times(u\otimes v)=h(u\otimes v)=\phi_0(u,v)
=R(u)(v)$}

\noindent for every $u\in U_1$, $v\in V_1$.

Because $U_1$ is norm-dense in $L^1_{\bar\mu}$, $U_1^+$ is dense in
$(L^1_{\bar\mu})^+$, and similarly $V_1^+$ is dense in
$(L^1_{\bar\nu})^+$, so $U_1^+\times V_1^+$ is dense in
$(L^1_{\bar\mu})^+\times(L^1_{\bar\nu})^+$;  now $\phi_0$ is
non-negative on $U_1^+\times V_1^+$, so $\phi$ (being continuous) is
non-negative on $(L^1_{\bar\mu})^+\times(L^1_{\bar\nu})^+$.   By 376Cc,
$h\ge 0$ in $(L^1_{\bar\lambda})^*$ and $w'\ge 0$ in
$L^{\infty}(\frak C)$.   In the same way, because
$\phi_0(u,v)\le T_w(u)(v)$ for
$u\in U_1^+$ and $v\in V_1^+$, $w'\le w_1\le w$ in $L^0(\frak C)$, so
$w'\in W$.   We have

\Centerline{$T_{w'}(u)(v)=\int w'\times(u\otimes v)=R(u)(v)$}

\noindent for all $u\in U_1$, $v\in V_1$.   If $u\in U_1^+$, then
$T_{w'}(u)$ and $R(u)$ are both order-continuous, so must be identical,
since $V_1$ is order-dense in $V$.   This means that $T_{w'}$ and $R$
agree on $U_1$.   But as both are themselves
order-continuous linear operators, and $U_1$ is order-dense in $U$, they
must be equal.

Thus $0<T_{w'}\le S$ in $\eurm L^{\times}(U;V^{\times})$.   As $S$ is
arbitrary, $\tilde W$ is quasi-order-dense in $\widehat{W}$, therefore
order-dense (353A).\ \Qed

\medskip

{\bf (g)} Because $w\mapsto T_w:W\mapsto\tilde W$ is an injective Riesz
homomorphism, we have an inverse map $Q:\tilde W\to L^0(\frak C)$,
setting $Q(T_w)=w$;  this is a Riesz homomorphism, and it is
order-continuous because $W$ is solid in $L^0(\frak C)$, so that the
embedding $W\embedsinto L^0(\frak C)$ is order-continuous.   By 368B, $Q$
has an extension to an order-continuous Riesz homomorphism
$\tilde Q:\widehat{W}\to L^0(\frak C)$.
Because $Q(S)>0$ whenever $S>0$ in
$\tilde W$, $\tilde Q(S)>0$ whenever $S>0$ in $\widehat{W}$, so
$\tilde Q$ is injective.
Now $\tilde Q(S)\in W$ for every $S\in\widehat{W}$.
\Prf\ It is enough to look at non-negative $S$.   In this case,
$\tilde Q(S)$ must be $\sup\{\tilde Q(T_w):w\in W,\,T_w\le S\}=\sup C$,
where $C=\{w:T_w\le S\}\subseteq W$.   Take $u\in U^+$ and $v\in V^+$.
Then $\{w\times(u\otimes v):w\in C\}$ is upwards-directed, because $C$ is,
and

\Centerline{$\sup_{w\in C}\int w\times(u\otimes v)
=\sup_{w\in C}T_w(u)(v)
\le S(u)(v)<\infty$.}

\noindent So
$\tilde Q(S)\times(u\otimes v)=\sup_{w\in C}w\times(u\otimes v)$ belongs
to $L^1_{\bar\lambda}$ (365Df).   As $u$ and $v$
are arbitrary, $\tilde Q(S)\in W$.\ \Qed

\medskip

{\bf (h)} Of course this means that $\tilde W=\widehat{W}$ and
$\tilde Q=Q$,
that is, that $w\mapsto T_w:W\mapsto\widehat{W}$ is a Riesz space
isomorphism.

\medskip

{\bf (i)} I have still to check on the identification of $\widehat{W}$
as the band $Z$ of abstract integral operators in
$\eurm L^{\times}(U;V^{\times})$.   Write $P_{fg}(u)=f(u)g$ for
$f\in U^{\times}$, $g\in V^{\times}$ and $u\in U$.

Set

\Centerline{$U^{\#}=\{u:u\in L^0(\frak A),\,u\times u'\in L^1_{\bar\mu}$
for every $u'\in U\}$,}

\Centerline{$V^{\#}=\{v:v\in L^0(\frak B),\,v\times v'\in L^1_{\bar\nu}$
for every $v'\in V\}$.}

\noindent From 369C we know that if we set $f_u(u')=\int u\times u'$ for
$u\in U^{\#}$ and $u'\in U$, then $f_u\in U^{\times}$ for every
$u\in U^{\#}$, and $u\mapsto f_u$ is an isomorphism between $U^{\#}$ and
an order-dense Riesz subspace of $U^{\times}$.   Similarly, setting
$g_v(v')=\int v\times v'$ for $v\in V^{\#}$ and $v'\in V$,
$v\mapsto g_v$
is an isomorphism between $V^{\#}$ and an order-dense Riesz subspace of
$V^{\times}$.

If $u\in U^{\#}$ and $v\in V^{\#}$ then

\Centerline{$\int(u\otimes v)\times(u'\otimes v')
=\int(u\times u')\otimes(v\times v')
=(\int u\times u')(\int v\times v')
=f_u(u')g_v(v')$}

\noindent for every $u'\in U$, $v'\in V$, so $u\otimes v\in W$ and
$T_{u\otimes v}=P_{f_ug_v}$.

Now take $f\in(U^{\times})^+$ and $g\in (V^{\times})^+$.
Set $A=\{u:u\in U^{\#},\,u\ge 0,\,f_u\le f\}$ and
$B=\{v:v\in V^{\#},\,v\ge 0,\,g_v\le g\}$.   These are upwards-directed,
so $C=\{u\otimes v:u\in A,\,v\in B\}$ is upwards-directed in
$L^0(\frak C)$.   Because $\{f_u:u\in U^{\#}\}$ is order-dense in
$U^{\times}$, $f=\sup_{u\in A}f_u$;  by 355Ed,
$f(u')=\sup_{u\in A}f_u(u')$ for every $u'\in U^+$.  Similarly,
$g(v')=\sup_{v\in B}f_v(v')$ for every $v'\in V^+$.

\Quer\ Suppose, if possible, that $C$ is not bounded above in $L^0(\frak
C)$.   Because $\frak C$ and $L^0(\frak C)$ are Dedekind complete,

\Centerline{$c
=\inf_{n\in\Bbb N}\sup_{u\in A,v\in B}\Bvalue{u\otimes v\ge n}$}

\noindent must be non-zero (364L(a-i)).   Because $U$ and $V$ are
order-dense
in $L^0(\frak A)$, $L^0(\frak B)$ respectively,

\Centerline{$1_{\frak A}=\sup\{\Bvalue{u'>0}:u'\in U\}$,
\quad$1_{\frak B}=\sup\{\Bvalue{v'>0}:v'\in V\}$,}

\noindent and there are $u'\in U^+$, $v'\in V^+$ such that
$c\Bcap\Bvalue{u'>0}\otimes\Bvalue{v'>0}\ne 0$, so that
$\int_cu'\otimes v'>0$.   But now, for any $n\in\Bbb N$,

$$\eqalignno{f(u')g(v')
&\ge\sup_{u\in A,v\in B}f_u(u')g_v(v')\cr
&=\sup_{u\in A,v\in B}\int(u\otimes v)\times(u'\otimes v')\cr
&\ge\sup_{u\in A,v\in B}\int((u\otimes v)\wedge n\chi c)\times(u'\otimes
v')\cr
&=\int\sup_{u\in A,v\in B}((u\otimes v)\wedge n\chi c)\times(u'\otimes
v')\cr
\noalign{\noindent (because $w\mapsto\int w\times(u'\otimes v')$ is
order-continuous)}
&=\int(n\chi c)\times(u'\otimes v')
=n\int_cu'\otimes v',\cr}$$

\noindent which is impossible.\ \Bang

Thus $C$ is bounded above in $L^0(\frak C)$, and has a supremum $w\in
L^0(\frak C)$.   If $u'\in U^+$, $v'\in V^+$ then

$$\eqalign{\int w\times(u'\otimes v')
&=\sup_{u\in A,v\in B}\int(u\otimes v)\times(u'\otimes v')\cr
&=\sup_{u\in A,v\in B}f_u(u')g_v(v')
=f(u')g(v')
=P_{fg}(u')(v').\cr}$$

\noindent Thus $w\in W$ and

\Centerline{$P_{fg}=T_w\in\tilde W\subseteq\widehat{W}$.}

\noindent And this is true for any non-negative $f\in U^{\times}$ and
$g\in V^{\times}$.   Of course it follows that $P_{fg}\in\widehat{W}$ for
every $f\in U^{\times}$, $g\in V^{\times}$;  as $\widehat{W}$ is a band,
it must include $Z$.

\medskip

{\bf (j)} Finally, $\widehat{W}\subseteq Z$.   \Prf\ Since
$Z=Z^{\perp\perp}$, it is enough to show that $\widehat{W}\cap
Z^{\perp}=\{0\}$.   Take any $T>0$ in $\widehat{W}$.   There are
$u'_0\in U^+$, $v'_0\in V^+$ such that $T(u'_0)(v'_0)>0$.   So there is
a $v\in V^{\#}$ such that $0\le g_v\le T(u'_0)$ and $g_v(v'_0)>0$, that
is, $\int v\times v'_0>0$.   Because $V$ is order-dense in
$L^0(\frak B)$, there is a $v'_1\in V$ such that
$0<v'_1\le v'_0\times\chi\Bvalue{v>0}$, so that

\Centerline{$0<\int v\times v'_1=g_v(v'_1)\le T(u'_0)(v'_1)$}

\noindent and $\Bvalue{v'_1>0}\Bsubseteq\Bvalue{v>0}$.

Now consider the functional $u'\mapsto h(u')=T(u')(v'_1):U\to\Bbb R$.
This belongs to $(U^{\times})^+$ and $h(u'_0)>0$, so there is a
$u\in U^{\#}$ such that $0\le f_u\le h$ and $f_u(u'_0)>0$.   This time,
$\int u\times u'_0>0$ so (because $U$ is order-dense in $L^0(\frak A)$)
there is a $u'_1\in U$ such that $h(u'_1)>0$ and
$\Bvalue{u'_1>0}\Bsubseteq\Bvalue{u>0}$.

We can express $T$ as $T_w$ where $w\in W^+$.   In this case, we have

\Centerline{$\int w\times(u'_1\otimes v'_1)=T(u'_1)(v'_1)=h(u'_1)>0$,}

\noindent so

$$\eqalign{0
&\ne\Bvalue{w>0}\cap\Bvalue{u'_1\otimes v'_1>0}
=\Bvalue{w>0}\cap(\Bvalue{u'_1>0}\otimes\Bvalue{v'_1>0})\cr
&\Bsubseteq\Bvalue{w>0}\cap(\Bvalue{u>0}\otimes\Bvalue{v>0})
=\Bvalue{w>0}\cap\Bvalue{u\otimes v>0},\cr}$$

\noindent and $w\wedge(u\otimes v)>0$, so

\Centerline{$T_w\wedge P_{f_ug_v}
=T_w\wedge T_{u\otimes v}
=T_{w\wedge(u\otimes v)}>0$.}

\noindent Thus $T\notin Z^{\perp}$.   Accordingly
$\widehat{W}\cap Z^{\perp}=\{0\}$ and
$\widehat{W}\subseteq Z^{\perp\perp}=Z$.\ \Qed

Since we already know that $Z\subseteq\widehat{W}$, this completes the
proof.
}%end of proof of 376E

\vleader{72pt}{376F}{Corollary} Let $(\frak A,\bar\mu)$ and
$(\frak B,\bar\nu)$
be localizable measure algebras, with localizable measure algebra free
product $(\frak C,\bar\lambda)$.   Let $U\subseteq L^0(\frak A)$,
$V\subseteq L^0(\frak B)$ be perfect order-dense solid linear subspaces,
and $T:U\to V$ a linear operator.   Then the following are
equiveridical:

(i) $T$ is an abstract integral operator;

(ii) there is a $w\in L^0(\frak C)$ such that $\int w\times(u\otimes
v')$ is defined and equal to $\int Tu\times v'$ whenever $u\in U$ and
$v'\in L^0(\frak B)$ is such that $v'\times v$ is integrable for every
$v\in V$.

\proof{ Setting $V^{\#}=\{v':v'\in L^0(\frak B),\,v\times v'\in L^1$ for
every $v\in V\}$, we know that we can identify $V^{\#}$ with
$V^{\times}$ and $V$ with $(V^{\#})^{\times}$ (369C).   So the
equivalence of (i) and (ii) is just 376E applied to $V^{\#}$ in place of
$V$.
}%end of proof of 376F

\leader{376G}{Lemma} Let $U$ be a Riesz space, $V$ an Archimedean Riesz
space, $T:U\to V$ a linear operator, $f\in (U^{\sim})^+$ and $e\in V^+$.
Suppose that $0\le Tu\le f(u)e$ for every $u\in U^+$.   Then if
$\sequencen{u_n}$ is a sequence in $U$ such that
$\lim_{n\to\infty}g(u_n)=0$ whenever $g\in U^{\sim}$ and $|g|\le f$,
$\sequencen{Tu_n}$ order*-converges to $0$ in
$V$\cmmnt{ (definition:  367A)}.

\proof{ Let $V_e$ be the solid linear subspace of $V$ generated by $e$;
then $Tu\in V_e$ for every $u\in U$.   We can identify $V_e$ with an
order-dense and norm-dense Riesz subspace of $C(X)$, where $X$ is a
compact Hausdorff space, with $e$ corresponding to $\chi X$ (353M).
For $x\in X$, set $g_x(u)=(Tu)(x)$ for every $u\in U$;  then
$0\le g_x(u)\le f(u)$ for $u\ge 0$, so $|g_x|\le f$ and
$\lim_{n\to\infty}(Tu_n)(x)=0$.   As $x$ is arbitrary,
$\sequencen{Tu_n}$ order*-converges to $0$ in $C(X)$, by 367K, and
therefore in $V_e$, because $V_e$ is
order-dense in $C(X)$ (367E).   But $V_e$, regarded as a subspace of
$V$, is solid, so 367E tells us also that $\sequencen{Tu_n}$
order*-converges to $0$ in $V$.
}%end of proof of 376G

\leader{376H}{Theorem} Let $U$ be a Riesz space and $V$ a
\wsid\ Dedekind complete Riesz space\cmmnt{ (definition: 368N)}.
Suppose that $T\in\eurm L^{\times}(U;V)$.
Then the following are equiveridical:

(i) $T$ is an abstract integral operator;

(ii) whenever $\sequencen{u_n}$ is an order-bounded sequence in $U^+$
and $\lim_{n\to\infty}f(u_n)=0$ for every $f\in U^{\times}$, then
$\sequencen{Tu_n}$ order*-converges to $0$ in $V$;

(iii) whenever $\sequencen{u_n}$ is an order-bounded sequence in $U$
and $\lim_{n\to\infty}f(u_n)=0$ for every $f\in U^{\times}$, then
$\sequencen{Tu_n}$ order*-converges to $0$ in $V$.

\proof{ For $f\in U^{\times}$, $v\in V$ and $u\in U$ set
$P_{fv}(u)=f(u)v$.   Write $Z\subseteq\eurm L^{\times}(U;V)$ for the
band of abstract integral operators.

\medskip

{\bf (a)(i)$\Rightarrow$(iii)} Suppose that $T\in Z^+$, and that
$\sequencen{u_n}$ is an order-bounded sequence in $U$ such that
$\lim_{n\to\infty}f(u_n)=0$ for every $f\in U^{\times}$.   Note that
$\{P_{fv}:f\in U^{\times+},\,v\in V^+\}$ is upwards-directed, so that
$T=\sup\{T\wedge P_{fv}:f\in U^{\times+},\,v\in V^+\}$ (352Va).

Take $u^*\in U^+$ such that $|u_n|\le u^*$ for every $n$, and set
$w=\inf_{n\in\Bbb N}\sup_{m\ge n}Tu_m$, which is defined because
$|Tu_n|\le Tu^*$ for every $n$.   Now $w\le(T-P_{fv})^+(u^*)$ for
every $f\in U^{\times+}$ and $v\in V^+$.   \Prf\ Setting
$T_1=T\wedge P_{fv}$, $w_0=(T-P_{fv})^+(u^*)$ we have

\Centerline{$Tu_n-T_1u_n
\le|T-T_1|(u^*)
=(T-P_{fv})^+(u^*)=w_0$}

\noindent for every $n\in\Bbb N$, so $Tu_n\le w_0+T_1u_n$.   On the
other hand, $0\le T_1u\le f(u)v$ for every $u\in U^+$, so by 376G we
must have $\inf_{n\in\Bbb N}\sup_{m\ge n}T_1u_m=0$.   Accordingly

\Centerline{$w\le w_0+\inf_{n\in\Bbb N}\sup_{m\ge n}T_1u_m=w_0$.   \Qed}

\noindent But as $\inf\{(T-P_{fv})^+:f\in U^{\times+},\,v\in V^+\}=0$,
$w\le 0$.    Similarly (or applying the same argument to
$\sequencen{-u_n}$), $\sup_{n\in\Bbb N}\inf_{n\in\Bbb N}Tu_n\ge 0$ and
$\sequencen{Tu_n}$ order*-converges to zero.

For general $T\in Z$, this shows that $\sequencen{T^+u_n}$ and
$\sequencen{T^-u_n}$ both order*-converge to $0$, so
$\sequencen{Tu_n}$ order*-converges to $0$, by 367Cd.
As $\sequencen{u_n}$ is arbitrary, (iii) is satisfied.

\medskip

{\bf (b)(iii)$\Rightarrow$(ii)} is trivial.

\medskip

{\bf (c)(ii)$\Rightarrow$(i)} \Quer\ Now suppose, if possible, that (ii)
is satisfied, but that $T\notin Z$.   Because $\eurm L^{\times}(U;V)$ is
Dedekind complete (355H), $Z$ is a projection band (353I), so $T$ is
expressible as
$T_1+T_2$ where $T_1\in Z$, $T_2\in Z^{\perp}$ and $T_2\ne 0$.   At
least one of $T_2^+$, $T_2^-$ is non-zero;  replacing $T$ by $-T$ if
need be, we may suppose that $T_2^+>0$.

Because $T_2^+$, like $T$, belongs to $\eurm L^{\times}(U;V)$, its
kernel $U_0$ is a band in $U$, which cannot be the whole of $U$, and
there is a $u_0>0$ in $U_0^{\perp}$.   In this case $T_2^+u_0>0$;
because $T_2^+\wedge(T_2^-+|T_1|)=0$, there is a $u_1\in[0,u_0]$ such
that $T_2^+(u_0-u_1)+(T_2^-+|T_1|)(u_1)\not\ge T_2^+u_0$, so that

\Centerline{$Tu_1\ge T_2u_1-|T_1|(u_1)\not\le 0$}

\noindent and $Tu_1\ne 0$.   Now this means that the sequence
$(Tu_1,Tu_1,\ldots)$ is not order*-convergent to zero, so there must
be some $f\in U^{\times}$ such that $(f(u_1),f(u_1),\ldots)$ does not
converge to $0$, that is, $f(u_1)\ne 0$;  replacing $f$ by $|f|$ if
necessary, we may suppose that $f\ge 0$ and that $f(u_1)>0$.

By 356H, there is a $u_2$ such that $0<u_2\le u_1$ and $g(u_2)=0$
whenever $g\in U^{\times}$ and $g\wedge f=0$.
Because $0<u_2\le u_0$, $u_2\in U_0^{\perp}$ and $v_0=T_2^+u_2>0$.
Consider $P_{fv_0}\in Z$.   Because $T_2\in Z^{\perp}$, $T_2^+\wedge
P_{fv_0}=0$;  set $S=P_{fv_0}+T_2^-$, so that $T_2^+\wedge S=0$.   Then

\Centerline{$\inf_{u\in[0,u_2]}T_2^+(u_2-u)+Su=0$,
\quad$\sup_{u\in[0,u_2]}T_2^+u-Su=v_0$}

\noindent (use 355Ec for the first equality, and then subtract both
sides from $v_0$).   Now $Su\ge f(u)v_0$ for every $u\ge 0$, so that for
any $\epsilon>0$

\Centerline{$\sup_{u\in[0,u_2],f(u)\ge\epsilon}T_2^+u-Su
\le(1-\epsilon)v_0$}

\noindent and accordingly

\Centerline{$\sup_{u\in[0,u_2],f(u)\le\epsilon}T_2^+u=v_0$,}

\noindent since the join of these two suprema is surely at least $v_0$,
while the second is at most $v_0$.
Note also that

\Centerline{$v_0=\sup_{u\in[0,u_2],f(u)\le\epsilon}T_2^+u
=\sup_{0\le u'\le u\le u_2,f(u)\le\epsilon}T_2u'
=\sup_{0\le u'\le u_2,f(u')\le\epsilon}T_2u'$.}

For $k\in\Bbb N$ set $A_k=\{u:0\le u\le u_2,\,f(u)\le 2^{-k}\}$.   We
know that

\Centerline{$B_k=\{\sup_{u\in I}T_2u:I\subseteq A_k$ is finite$\}$}

\noindent is an upwards-directed set with supremum $v_0$ for each $k$.
Because $V$ is \wsid, we can find a sequence $\sequence{k}{v'_k}$ such
that $v'_k\in B_k$ for every $k$ and $v_1=\inf_{k\in\Bbb N}v'_k>0$.
For each $k$ let $I_k\subseteq A_k$ be a finite set such that
$v'_k=\sup_{u\in I_k}T_2u$.

Because each $I_k$ is finite, we can build a sequence $\sequencen{u'_n}$
in $[0,u_2]$ enumerating each in turn, so that
$\lim_{n\to\infty}f(u'_n)=0$ (since $f(u)\le 2^{-k}$ if $u\in I_k$)
while $\sup_{m\ge n}T_2u'_m\ge v_1$ for every $n$ (since $\{u'_m:m\ge
n\}$ always includes some $I_k$).   Now $\sequencen{T_2u'_n}$ does not
order*-converge to $0$.

However, $\lim_{n\to\infty}g(u'_n)=0$ for every $g\in U^{\times}$.
\Prf\ Express $|g|$ as $g_1+g_2$ where $g_1$ belongs to the band of
$U^{\times}$ generated by $f$ and $g_2\wedge f=0$ (353Hc).   Then
$g_2(u'_n)=g_2(u_2)=0$ for every $n$, by the choice of $u_2$.   Also
$g_1=\sup_{n\in\Bbb N}g_1\wedge nf$ (352Vb);  so, given
$\epsilon>0$, there is an $m\in\Bbb N$ such that
$(g_1-mf)^+(u_2)\le\epsilon$ and $(g_1-mf)^+(u'_n)\le\epsilon$ for every
$n\in\Bbb N$.   But this means that

\Centerline{$|g(u'_n)|\le|g|(u'_n)\le\epsilon+mf(u'_n)$}

\noindent for every $n$, and $\limsup_{n\to\infty}|g(u'_n)|\le\epsilon$;
as $\epsilon$ is arbitrary, $\lim_{n\to\infty}g(u'_n)=0$.\ \Qed

Now, however, part (a) of this proof tells us that $\sequencen{T_1u'_n}$
is order*-convergent to $0$, because $T_1\in Z$, while
$\sequencen{Tu'_n}$ is order*-convergent to $0$, by hypothesis;  so
$\sequencen{T_2u'_n}=\sequencen{Tu'_n-T_1u'_n}$ order*-converges to
$0$.\ \Bang

This contradiction shows that every operator satisfying the condition
(ii) must be in $Z$.
}%end of proof of 376H

\leader{376I}{}\cmmnt{ The following elementary remark will be useful
for the next corollary and also for Theorem 376S.

\medskip

\noindent}{\bf Lemma} Let $(X,\Sigma,\mu)$ be a $\sigma$-finite measure
space and $U$ an order-dense solid linear subspace of $L^0(\mu)$.   Then
there is a non-decreasing sequence $\sequencen{X_n}$ of measurable
subsets of $X$, with union $X$, such that $\chi X_n^{\ssbullet}\in U$
for every $n\in\Bbb N$.

\proof{ Write $\frak A$ for the measure
algebra of $\mu$, so that $L^0(\mu)$ can be identified with
$L^0(\frak A)$ (364Ic).   $A=\{a:a\in\frak A\setminus\{0\},\,\chi a\in U\}$
is order-dense in $\frak A$, so includes a partition of unity
$\langle a_i\rangle_{i\in I}$.
Because $\mu$ is $\sigma$-finite, $\frak A$ is
ccc (322G) and $I$ is countable, so we can take $I$ to be a subset of
$\Bbb N$.   Choose $E_i\in\Sigma$ such that $E_i^{\ssbullet}=a_i$ for
$i\in I$;  set $E=X\setminus\bigcup_{i\in I}E_i$,
$X_n=E\cup\bigcup_{i\in I,i\le n}E_i$ for $n\in\Bbb N$.
}%end of proof of 376I

\leader{376J}{Corollary} Let $(X,\Sigma,\mu)$ and $(Y,\Tau,\nu)$ be
$\sigma$-finite measure spaces, with product measure $\lambda$ on
$X\times Y$.   Let $U\subseteq L^0(\mu)$,
$V\subseteq L^0(\nu)$ be perfect order-dense solid linear subspaces, and
$T:U\to V$ a linear operator.   Write
$\eusm U=\{f:f\in\eusm L^0(\mu),\,f^{\ssbullet}\in U\}$,
$\eusm V^{\#}=\{h:h\in\eusm L^0(\nu),\,h^{\ssbullet}\times v\in L^1$ for
every $v\in V\}$.   Then the following are equiveridical:

(i) $T$ is an abstract integral operator;

(ii) there is a $k\in\eusm L^0(\lambda)$ such that

\quad ($\alpha$) $\int|k(x,y)f(x)h(y)|d(x,y)<\infty$ for every
$f\in\eusm U$, $h\in\eusm V^{\#}$,

\quad ($\beta$) if $f\in\eusm U$ and we set $g(y)=\int k(x,y)f(x)dx$
wherever this is defined, then $g\in\eusm L^0(\nu)$ and
$Tf^{\ssbullet}=g^{\ssbullet}$;

(iii) $T\in\eurm L^{\sim}(U;V)$ and whenever $\sequencen{u_n}$ is an
order-bounded sequence in $U^+$ and $\lim_{n\to\infty}h(u_n)=0$ for
every $h\in U^{\times}$, then $\sequencen{Tu_n}$ order*-converges to
$0$ in $V$.

\cmmnt{\medskip

\noindent{\bf Remark} I write `$d(x,y)$' above to indicate integration
with respect to the product measure $\lambda$.   Recall that in the
terminology of \S251, $\lambda$ can be taken to be either the
`primitive' or `c.l.d.' product measure (251K).}

\proof{ The idea is of course to identify $L^0(\mu)$ and $L^0(\nu)$ with
$L^0(\frak A)$ and $L^0(\frak B)$, where $(\frak A,\bar\mu)$ and $(\frak
B,\bar\nu)$ are the measure algebras of $\mu$ and $\nu$, so that their
localizable measure algebra free product can be identified with the
measure algebra of $\lambda$ (325E), while
$V^{\#}=\{h^{\ssbullet}:h\in\eusm V^{\#}\}$ can be identified with
$V^{\times}$, because $(T,\Tau,\nu)$ is localizable (see the last sentence
in 369C).

\medskip

{\bf (a)(i)$\Rightarrow$(ii)}
By 376F, there is a $w\in L^0(\lambda)$ such
that $\int w\times(u\otimes v')$ is defined and equal to
$\int Tu\times v'$ whenever $u\in U$ and $v'\in V^{\#}$.   Express $w$ as
$k^{\ssbullet}$ where $k\in\eusm L^0(\lambda)$.   If $f\in\eusm U$ and
$h\in\eusm V^{\#}$ then $\int|k(x,y)f(x)h(y)|d(x,y)
=\int|w\times(f^{\ssbullet}\otimes h^{\ssbullet}|$ is finite, so
(ii-$\alpha$) is satisfied.

Now take any $f\in\eusm U$, and set $g(y)=\int k(x,y)f(x)dx$ whenever
this is defined in $\Bbb R$.
Write $\Cal F$ for the set of those $F\in\Tau$ such that
$\chi F\in \eusm V^{\#}$.   Then for any $F\in\Cal F$, $g$ is defined
almost everywhere in $F$ and $g\restr F$ is $\nu$-virtually measurable.
\Prf\ $\int k(x,y)f(x)\chi F(y)d(x,y)$ is defined in $\Bbb R$, so by
Fubini's theorem (252B, 252C) $g_F(y)=\int k(x,y)f(x)\chi F(y)dx$ is
defined for almost every $y$, and is
$\nu$-virtually measurable;  now $g\restr F=g_F\restr F$.\ \Qed\
Next, there is a sequence $\sequencen{F_n}$ in $\Cal F$ with union $Y$,
by 376I, because $V$ is perfect and order-dense, so $V^{\#}$ must also
be order-dense in $L^0(\nu)$.

For each $n\in\Bbb N$, there is a measurable set $F'_n\subseteq
F_n\cap\dom g$ such that $g\restr F_n$ is measurable and $F_n\setminus
F'_n$ is negligible.   Setting $G=\bigcup_{n\in\Bbb N}F'_n$, $G$ is
conegligible and $g\restr G$ is measurable, so $g\in\eusm L^0(\nu)$.

If $\tilde g\in L^0(\nu)$ represents $Tu\in L^0(\nu)$, then for any
$F\in\Cal F$

\Centerline{$\int_F\tilde g=\int Tu\times(\chi F)^{\ssbullet}
=\int_Fg$.}

\noindent In particular, this is true whenever $F\in\Tau$ and
$F\subseteq F_n$.   So $g$ and $\tilde g$ agree almost everywhere in
$F_n$, for each $n$, and $g\eae\tilde g$.   Thus $g$ also represents
$Tu$, as required in (ii-$\beta$).

\medskip

{\bf (b)(ii)$\Rightarrow$(i)} Set $w=k^{\ssbullet}$ in $L^0(\lambda)$.
If $f\in\eusm U$ and $h\in\eusm V^{\#}$ the hypothesis ($\alpha$)
tells us that $(x,y)\mapsto k(x,y)f(x)h(y)$ is integrable (because it
surely belongs to $\eusm L^0(\lambda)$).   By Fubini's theorem,

\Centerline{$\int k(x,y)f(x)h(y)d(x,y)
=\int g(y)h(y)dy$}

\noindent where $g(y)=\int k(x,y)f(x)dx$ for almost every $y$, so that
$Tf^{\ssbullet}=g^{\ssbullet}$, by ($\beta$).   But this means that,
setting $u=f^{\ssbullet}$ and $v'=h^{\ssbullet}$,

\Centerline{$\int w\times(u\otimes v')=\int Tu\times v'$;}

\noindent and this is true for every $u\in U$, $v'\in V^{\#}$.

Thus $T$ satisfies the condition 376F(ii), and is an abstract integral
operator.

\medskip

{\bf (b)(i)$\Rightarrow$(iii)} Because $V$ is \wsid\ (368S), this is
covered by 376H(i)$\Rightarrow$(iii).

\medskip

{\bf (c)(iii)$\Rightarrow$(i)} Suppose that $T$ satisfies (iii).   The
point is that $T^+$ is order-continuous.   \Prf\Quer\
Otherwise, let $A\subseteq U$ be a
non-empty downwards-directed set, with infimum $0$, such that
$v_0=\inf_{u\in A}T^+(u)>0$.   Let $\sequencen{X_n}$ be a non-decreasing
sequence of sets of finite measure covering $X$, and set
$a_n=X_n^{\ssbullet}$ for each $n$.   For each $n$,
$\inf_{u\in A}\Bvalue{u>2^{-n}}=0$, so we can find $\tilde u_n\in A$
such that $\bar\mu(a_n\Bcap\Bvalue{\tilde u_n>2^{-n}})\le 2^{-n}$.   Set
$u_n=\inf_{i\le n}\tilde u_i$ for each $n$;  then $\sequencen{u_n}$ is
non-increasing and has infimum $0$;  also, $[0,u_n]$ meets $A$ for each
$n$, so that $v_0\le\sup\{Tu:0\le u\le u_n\}$ for each $n$.   Because
$V$ is \wsid, we can find a sequence $\sequencen{I_n}$
of finite sets such that $I_n\subseteq[0,u_n]$ for each $n$ and
$v_1=\inf_{n\in\Bbb N}\sup_{u\in I_n}(Tu)^+>0$.   Enumerating
$\bigcup_{n\in\Bbb N}I_n$ as $\sequencen{u'_n}$, as in part (c) of the
proof of 376H, we see that $\sequencen{u'_n}$ is order-bounded and
$\lim_{n\to\infty}f(u'_n)=0$ for every $f\in U^{\times}$ (indeed,
$\sequencen{u'_n}$ order*-converges to $0$ in $U$), while
$\sequencen{Tu'_n}\not\to^* 0$ in $V$.\ \Bang\Qed

Similarly, $T^-$ is order-continuous, so $T\in\eurm L^{\times}(U;V)$.
Accordingly $T$ is an abstract integral operator by condition (ii) of
376H.
}%end of proof of 376J

\leader{376K}{}\cmmnt{ As an application of the ideas above, I give
a result due to N.Dunford (376N) which was one of the inspirations
underlying the theory.   Following the method of {\smc Zaanen 83},  I
begin with a couple of elementary lemmas.

\wheader{376K}{4}{2}{2}{60pt}

\noindent}{\bf Lemma} Let $U$ and $V$ be Riesz spaces.   Then there is a
Riesz space isomorphism $T\mapsto T':\eurm
L^{\times}(U;V^{\times})\to\eurm L^{\times}(V;U^{\times})$ defined by
the formula

\Centerline{$(T'v)(u)=(Tu)(v)$ for every $u\in U$, $v\in V$.}

\noindent If we write
$P_{fg}(u)=f(u)g$ for $f\in U^{\times}$, $g\in V^{\times}$ and $u\in U$,
then $P_{fg}\in\eurm L^{\times}(U;V^{\times})$ and $P'_{fg}=P_{gf}$ in
$\eurm L^{\times}(V;U^{\times})$.
Consequently $T$ is an abstract integral operator iff $T'$ is.

\proof{ All the ideas involved have already appeared.   For positive
$T\in \eurm L^{\times}(U;V^{\times})$ the functional $(u,v)\mapsto
(Tu)(v)$ is bilinear and order-continuous in each variable separately;
so (just as in the first part of the proof of 376E) corresponds to a
$T'\in\eurm L^{\times}(V;U^{\times})$.   The map $T\mapsto T':
\eurm L^{\times}(U;V^{\times})^+\to\eurm L^{\times}(V;U^{\times})^+$ is
evidently an additive, order-preserving bijection, so extends to an
isomorphism between $\eurm L^{\times}(U;V^{\times})$ and $\eurm
L^{\times}(V;U^{\times})$ given by the same formula.   I remarked in
part (i) of the proof of 376E that every $P_{fg}$ belongs to $\eurm
L^{\times}(U;V^{\times})$, and the identification $P'_{fg}=P_{gf}$ is
just a matter of checking the formulae.   Of course it follows at once
that the bands of abstract integral operators must also be matched by
the map $T\mapsto T'$.
}%end of proof of 376K

\leader{376L}{Lemma} Let $U$ be a Banach lattice with an order-continuous
norm.   If $w\in U^+$ there is a $g\in(U^{\times})^+$ such that for every
$\epsilon>0$ there is a $\delta>0$ such that $\|u\|\le\epsilon$ whenever
$0\le u\le w$ and $g(u)\le\delta$.

\woddheader{376L}{0}{0}{0}{24pt} %p243

\proof{{\bf (a)}  As remarked in 356D, $U^*=U^{\sim}=U^{\times}$.
Set

\Centerline{$A=\{v:v\in U$ and there is an $f\in(U^{\times})^+$ such
that $f(u)>0$ whenever $0<u\le|v|\}$.}

\noindent Then $v'\in A$ whenever $|v'|\le|v|\in A$ and $v+v'\in A$ for
all $v$, $v'\in A$ (if $f(u)>0$ whenever $0<u\le|v|$ and $f'(u)>0$
whenever $0<u\le|v'|$, then $(f+f')(u)>0$ whenever $0<u\le|v+v'|$);
moreover, if $v_0>0$ in $U$, there is a $v\in A$ such that $0<v\le v_0$.
\Prf\ Because $U^{\times}=U^*$ separates the points of $U$, there is a
$g>0$ in $U^{\times}$ such that $g(v_0)>0$;  now by 356H there is a
$v\in\ocint{0,v_0}$ such that $g$ is strictly positive on $\ocint{0,v}$,
so that $v\in A$.\ \QeD\   But this means that $A$ is an order-dense
solid linear subspace of $U$.

\medskip

{\bf (b)} In fact $w\in A$.   \Prf\ $w=\sup B$, where $B=A\cap[0,w]$.
Because $B$ is upwards-directed, $w\in\overline{B}$ (354Ea), and there
is a sequence $\sequencen{u'_n}$
in $B$ converging to $w$ for the norm.   For each $n$, choose
$f_n\in(U^{\times})^+$
such that $f_n(u)>0$ whenever $0<u\le u'_n$.   Set

\Centerline{$f=\sum_{n=0}^{\infty}\Bover1{2^{n}(1+\|f_n\|)}f_n$}

\noindent in $U^*=U^{\times}$.   Then
whenever $0<u\le w$ there is some $n\in\Bbb N$ such that $u\wedge
u'_n>0$, so that $f_n(u)>0$ and $f(u)>0$.   So $f$ witnesses that $w\in
A$.\ \Qed

\medskip

{\bf (c)} Take $g\in(U^{\times})^+$ such that $g(u)>0$ whenever
$0<u\le w$.   This $g$ serves.   \Prf\Quer\  Otherwise, there is some
$\epsilon>0$ such that for every $n\in\Bbb N$ we can find a
$u_n\in[0,w]$ with $g(u_n)\le 2^{-n}$ and $\|u_n\|\ge\epsilon$.   Set
$v_n=\sup_{i\ge n}u_i$;  then $0\le v_n\le w$, $g(v_n)\le 2^{-n+1}$ and
$\|v_n\|\ge\epsilon$ for every $n\in\Bbb N$.   But $\sequencen{v_n}$ is
non-decreasing, so $v=\inf_{n\in\Bbb N}v_n$ must be non-zero, while
$0\le v\le w$ and $g(v)=0$;  which is impossible.\ \Bang\Qed

Thus we have found an appropriate $g$.
}%end of proof of 376L

\leader{376M}{Theorem} (a) Let $U$ be a Banach lattice with an
order-continuous norm and $V$ a Dedekind complete $M$-space.
Then every bounded linear operator from $U$ to $V$ is an abstract
integral operator.

(b) Let $U$ be an $L$-space and $V$ a Banach lattice with
order-continuous
norm.   Then every bounded linear operator from $U$ to $V^{\times}$ is
an abstract integral operator.

\proof{{\bf (a)} By 355Kb and 355C,
$\eurm L^{\times}(U;V)=\eurm L^{\sim}(U;V)\subseteq\eurm B(U;V)$;  but
since norm-bounded sets in $V$ are also order-bounded,
$\{Tu:|u|\le u_0\}$ is bounded above in $V$ for every
$T\in\eurm B(U;V)$ and $u_0\in U^+$, and
$\eurm B(U;V)=\eurm L^{\times}(U;V)$.

I repeat ideas from the proof of 376H.   (I cannot quote 376H directly
as I am not assuming that $V$ is \wsid.)   \Quer\ Suppose, if possible,
that $\eurm B(U;V)$ is not the band $Z$ of abstract integral operators.
In this case there is a $T>0$ in
$Z^{\perp}$.   Take $u_1\ge 0$ such that $v_0=Tu_1$ is non-zero.
Let $f\ge 0$ in
$U^{\times}$ be such that for every $\epsilon>0$ there is a $\delta>0$
such that $\|u\|\le\epsilon$ whenever $0\le u\le u_1$ and
$f(u)\le\delta$ (376L).   Then, just as in part (c) of
the proof of 376H,

\Centerline{$\sup_{u\in[0,u_1],f(u)\le\delta}Tu=v_0$}

\noindent for every $\delta>0$.   But there is a $\delta>0$ such that
$\|T\|\|u\|\le\bover12\|v_0\|$ whenever $0\le u\le u_1$ and
$f(u)\le\delta$;  in which case
$\|\sup_{u\in[0,u_1],f(u)\le\delta}Tu\|\le\bover12\|v_0\|$, which is
impossible.\ \Bang

Thus $Z=\eurm B(U;V)$, as required.

\medskip

{\bf (b)} Because $V$ has an order-continuous norm,
$V^*=V^{\times}=V^{\sim}$;  and the norm of $V^*$ is a Fatou norm with
the Levi property (356Da).   So
$\eurm B(U;V^*)=\eurm L^{\times}(U;V^{\times})$, by 371C.
By 376K, this is canonically
isomorphic to $\eurm L^{\times}(V;U^{\times})$.   Now $U^{\times}=U^*$
is an $M$-space (356Pb).   By (a), every member of
$\eurm L^{\times}(V;U^{\times})$ is an abstract integral operator;
but the isomorphism between $\eurm L^{\times}(V;U^{\times})$ and
$\eurm L^{\times}(U;V^{\times})$ matches the abstract integral operators in
each space (376K), so
every member of $\eurm B(U;V^*)$ is also an abstract integral operator,
as claimed.
}%end of proof of 376M

\leader{376N}{Corollary:  Dunford's theorem} Let $(X,\Sigma,\mu)$ and
$(Y,\Tau,\nu)$ be $\sigma$-finite measure spaces and
$T:L^1(\mu)\to L^p(\nu)$ a bounded linear operator, where $1<p\le\infty$.
Then there is a measurable function $k:X\times Y\to\Bbb R$ such that
$Tf^{\ssbullet}=g_f^{\ssbullet}$, where
$g_f(y)=\int k(x,y)f(x)dx$ almost everywhere, for every
$f\in\eusm L^1(\mu)$.

\proof{ Set $q=\bover{p}{p-1}$ if $p$ is finite, $1$ if $p=\infty$.   We
can identify $L^p(\nu)$ with $V^{\times}$, where
$V=L^q(\nu)\cong L^p(\nu)^{\times}$ (366Dc, 365Mc) has an
order-continuous norm because $1\le q<\infty$.   By 376Mb, $T$ is an
abstract integral operator.   By 376F/376J, $T$ is
represented by a kernel, as claimed.
}%end of proof of 376N

\leader{376O}{}\cmmnt{ Under the right conditions, weakly compact
operators are abstract integral operators.

\medskip

\noindent}{\bf Lemma} Let $U$ be a Riesz space, and $W$ a solid linear
subspace of $U^{\sim}$.   If $C\subseteq U$ is relatively compact for
the weak topology $\frak T_s(U,W)$\cmmnt{ (3A5E)}, then for
every $g\in W^+$ and $\epsilon>0$ there is a $u^*\in U^+$ such that
$g(|u|-u^*)^+\le\epsilon$ for every $u\in C$.

\proof{ Let $W_g$ be the solid linear subspace of $W$ generated by $g$.
Then $W_g$ is an Archimedean Riesz space with order unit, so
$W_g^{\times}$ is a band in the $L$-space $W_g^*=W_g^{\sim}$ (356Na),
and is therefore an $L$-space in its own right (354O).   For $u\in U$,
$h\in W^{\times}_g$ set $(Tu)(h)=h(u)$;  then $T$ is an order-continuous
Riesz homomomorphism from $U$ to $W^{\times}_g$ (356F).

Now $W_g$ is perfect.
\Prf\ I use 356K.   $W_g$ is Dedekind complete because it is a solid
linear subspace of the Dedekind complete space $U^{\sim}$.
$W_g^{\times}$ separates the points of $W$ because $T[U]$ does.
If $A\subseteq W_g$ is
upwards-directed and $\sup_{h\in A}\phi(h)$ is finite for every
$\phi\in W_g^{\times}$, then $A$ acts on $W_g^{\times}$ as a set of
bounded linear functionals which, by the Uniform Boundedness Theorem
(3A5Ha), is uniformly bounded;  that is, there is some $M\ge 0$ such
that
$\sup_{h\in A}|\phi(h)|\le M\|\phi\|$ for every $\phi\in W_g^{\times}$.
Because $g$ is the standard order unit of $W_g$, we have
$\|\phi\|=|\phi|(g)$ and $|\phi(h)|\le M|\phi|(g)$ for every
$\phi\in W_g^{\times}$ and $h\in A$.   In particular,

\Centerline{$h(u)\le|h(u)|=|(Tu)(h)|\le M|Tu|(g)=M(Tu)(g)=Mg(u)$}

\noindent for every $h\in A$ and $u\in U^+$.   But this means that
$h\le Mg$ for every $h\in A$ and $A$ is bounded above in $W_g$.   Thus
all the conditions of 356K are satisfied and $W_g$ is perfect.\ \Qed

Accordingly $T$ is continuous for the topologies $\frak T_s(U,W)$ and
$\frak T_s(W_g^{\times},W_g^{\times\times})$, because every element
$\phi$ of $W_g^{\times\times}$ corresponds to a member of
$W_g\subseteq W$, so 3A5Ec applies.

Now we are supposing that $C$ is relatively compact for
$\frak T_s(U,W)$, that is, is included in some compact set $C'$;
accordingly $T[C']$ is compact and $T[C]$ is relatively compact for
$\frak T_s(W_g^{\times},W_g^{\times\times})$.   Since $W_g^{\times}$ is
an $L$-space, $T[C]$ is uniformly integrable (356Q);  consequently
(ignoring the trivial case $C=\emptyset$) there are
$\phi_0,\ldots,\phi_n\in T[C]$ such that
$\|(|\phi|-\sup_{i\le n}|\phi_i|)^+\|\le\epsilon$ for every
$\phi\in T[C]$ (354Rb), so that
$(|\phi|-\sup_{i\le n}|\phi_i|)^+(g)\le\epsilon$ for every
$\phi\in T[C]$.

Translating this back into terms of $C$ itself, and recalling that $T$
is a Riesz homomorphism, we see that there are $u_0,\ldots,u_n\in C$
such that $g(|u|-\sup_{i\le n}|u_i|)^+\le\epsilon$ for every $u\in C$.
Setting $u^*=\sup_{i\le n}|u_i|$ we have the result.
}%end of proof of 376O

\leader{376P}{Theorem} Let $U$ be an $L$-space and $V$ a perfect Riesz
space.   If $T:U\to V$ is a linear operator such that
$\{Tu:u\in U,\,\|u\|\le 1\}$ is relatively compact for the weak topology
$\frak T_s(V,V^{\times})$, then $T$ is an abstract integral operator.

\proof{{\bf (a)} For any $g\ge 0$ in $V^{\times}$,
$M_g=\sup_{\|u\|\le 1}g(|Tu|)$ is finite.   \Prf\ By 376O, there is a
$v^*\in V^+$ such that
$g(|Tu|-v^*)^+\le 1$ whenever $\|u\|\le 1$;  now $M_g\le g(v^*)+1$.\
\Qed\   Considering $\|u\|^{-1}u$, we see that $g(|Tu|)\le M_g\|u\|$ for
every $u\in U$.

Next, we find that $T\in\eurm L^{\sim}(U;V)$.   \Prf\ Take $u\in U^+$.
Set

\Centerline{$B=\{\sum_{i=0}^n|Tu_i|:u_0,\ldots,u_n\in U^+,\,
\sum_{i=0}^nu_i=u\}\subseteq V^+$.}

\noindent Then $B$ is upwards-directed.   (Cf.\ 371A.)   If $g\ge 0$ in
$V^{\times}$,

$$\eqalign{\sup_{v\in B}g(v)
&=\sup\{\sum_{i=0}^ng(|Tu_i|):\sum_{i=0}^nu_i=u\}\cr
&\le\sup\{\sum_{i=0}^nM_g\|u_i\|:\sum_{i=0}^nu_i=u\}
=M_g\|u\|\cr}$$

\noindent is finite.   By 356K, $B$ is bounded above in $V$;  and of
course any upper bound for $B$ is also an upper bound for
$\{Tu':0\le u'\le u\}$.   As $u$ is arbitrary, $T$ is order-bounded.\
\Qed

Because $U$ is a Banach lattice with an order-continuous norm,
$T\in\eurm L^{\times}(U;V)$ (355Kb).

\medskip

{\bf (b)} Since we can identify $\eurm L^{\times}(U;V)$ with
$\eurm L^{\times}(U;V^{\times\times})$, we have an adjoint operator
$T'\in\eurm L^{\times}(V^{\times};U^{\times})$, as in 376K.   Now if
$g\ge 0$ in
$V^{\times}$ and $\sequencen{g_n}$ is a sequence in $[0,g]$ such that
$\lim_{n\to\infty}g_n(v)=0$ for every $v\in V$, $\sequencen{T'g_n}$
order*-converges to $0$ in $U^{\times}$.   \Prf\ For any
$\epsilon>0$, there is a
$v^*\in V^+$ such that $g(|Tu|-v^*)^+\le\epsilon$ whenever $\|u\|\le 1$;
consequently

$$\eqalign{\|T'g_n\|
&=\sup_{\|u\|\le 1}(T'g_n)(u)
=\sup_{\|u\|\le 1}g_n(Tu)\cr
&\le g_n(v^*)+\sup_{\|u\|\le 1}g_n(|Tu|-v^*)^+\cr
&\le g_n(v^*)+\sup_{\|u\|\le 1}g(|Tu|-v^*)^+
\le g_n(v^*)+\epsilon\cr}$$

\noindent for every $n\in\Bbb N$.   As $\lim_{n\to\infty}g_n(v^*)=0$,
$\limsup_{n\to\infty}\|T'g_n\|\le\epsilon$;  as $\epsilon$ is arbitrary,
$\sequencen{\|T'g_n\|}\to 0$.   But as $U^{\times}$ is an $M$-space
(356Pb), it follows that $\sequencen{T'g_n}$ order*-converges to
$0$.\ \Qed

By 368Pc, $U^{\times}$ is \wsid.   By 376H, $T'$ is an abstract integral
operator, so $T$ also is, by 376K.
}%end of proof of 376P

\leader{376Q}{Corollary} Let $(X,\Sigma,\mu)$ and
$(Y,\Tau,\nu)$ be $\sigma$-finite measure spaces and
$T:L^1(\mu)\to L^1(\nu)$ a weakly compact linear operator.
Then there is a function $k:X\times Y\to\Bbb R$ such that
$Tf^{\ssbullet}=g_f^{\ssbullet}$, where
$g_f(y)=\int k(x,y)f(x)dx$ almost everywhere, for every
$f\in\eusm L^1(\mu)$.

\proof{ This follows from 376P and 376J, just as in 376N.
}%end of proof of 376Q

\leader{376R}{}\cmmnt{ So far I have mentioned actual kernel functions
$k(x,y)$ only as a way of giving slightly more concrete form to the
abstract kernels of 376E.   But of course they can provide new
structures and insights.   I give one result as an example.   The
following lemma is useful.

\medskip

\noindent}{\bf Lemma} Let $(X,\Sigma,\mu)$ be a measure space,
$(Y,\Tau,\nu)$ a $\sigma$-finite measure space, and $\lambda$ the
c.l.d.\ product measure on $X\times Y$.   Suppose that $k$ is a
$\lambda$-integrable real-valued function.   Then for any $\epsilon>0$
there is a finite partition $E_0,\ldots,E_n$ of $X$ into measurable sets
such that
$\|k-k_1\|_1\le\epsilon$, where

$$\eqalign{k_1(x,y)&=\Bover1{\mu E_i}\int_{E_i}k(t,y)dt
\text{ whenever }x\in E_i,\,0<\mu E_i<\infty\cr
&\hskip12em\text{ and the integral is defined in }\Bbb R,\cr
&=0\text{ in all other cases}.\cr}$$

\proof{ Once again I refer to the proof of 253F:  there are sets
$H_0,\ldots,H_r$ of finite measure in $X$, sets $F_0,\ldots,F_r$ of
finite measure in $Y$, and $\alpha_0,\ldots,\alpha_r$ such that
$\|k-k_2\|_1\le\bover12\epsilon$, where
$k_2=\sum_{j=0}^r\alpha_i\chi(H_j\times F_j)$.   Let $E_0,\ldots,E_n$ be
the partition of $X$ generated by $\{H_i:i\le r\}$.    Then for any
$i\le n$, $\int_{E_i\times Y}|k-k_1|$ is defined and is at most
$2\int_{E_i\times Y}|k-k_2|$.   \Prf\ If $\mu E_i=0$, this is trivial,
as both are zero.   If $\mu E_i=\infty$, then again the result is
elementary, since both $k_1$ and $k_2$ are zero on $E_i\times Y$.   So
let us suppose that $0<\mu E_i<\infty$.   In this case
$\int_{E_i}k(t,y)dt$ must be defined for almost every $y$, by Fubini's
theorem.   So $k_1$ is defined
almost everywhere in $E_i\times Y$, and

\Centerline{$\int_{E_i\times Y}|k-k_1|
=\int_Y\int_{E_i}|k(x,y)-k_1(x,y)|dxdy$.}

\noindent Now take some fixed $y\in Y$ such that

\Centerline{$\beta=\Bover1{\mu E_i}\int_{E_i}k(t,y)dt$}

\noindent is defined.   Then $\beta=k_1(x,y)$ for every $x\in E_i$.
For every $x\in E_i$, we must have
$k_2(x,y)=\alpha$ where $\alpha=\sum\{\alpha_j:E_i\subseteq H_j,\,y\in
F_j\}$.   But in this case, because
$\int_{E_i}k(x,y)-\beta\,dx=0$, we have

\Centerline{$\int_{E_i}\max(0,k(x,y)-\beta)dx
=\int_{E_i}\max(0,\beta-k(x,y))dx
=\Bover12\int_{E_i}|k(x,y)-k_1(x,y)|dx$.}

\noindent If $\beta\ge\alpha$,

\Centerline{$\int_{E_i}\max(0,k(x,y)-\beta)dx
\le\int_{E_i}\max(0,k(x,y)-\alpha)dx\le\int_{E_i}|k(x,y)-k_2(x,y)|dx$;}

\noindent if $\beta\le\alpha$,

\Centerline{$\int_{E_i}\max(0,\beta-k(x,y))dx
\le\int_{E_i}\max(0,\alpha-k(x,y))dx\le\int_{E_i}|k(x,y)-k_2(x,y)|dx$;}

\noindent in either case,

\Centerline{$\Bover12\int_{E_i}|k(x,y)-k_1(x,y)|dx
\le\int_{E_i}|k(x,y)-k_2(x,y)|dx$.}

\noindent This is true for almost every $y$, so integrating with respect
to $y$ we get the result.\ \Qed

Now, summing over $i$, we get

\Centerline{$\int|k-k_1|\le 2\int|k-k_2|\le\epsilon$,}

\noindent as required.
}%end of proof of 376R

\leader{376S}{Theorem} Let $(X,\Sigma,\mu)$ be a complete locally
determined measure space, $(Y,\Tau,\nu)$ a $\sigma$-finite measure
space, and $\lambda$ the c.l.d.\ product measure on $X\times Y$.   Let
$\tau$ be an extended Fatou norm on $L^0(\nu)$ and write
$\eusm L^{\tau'}$ for
$\{g:g\in\eusm L^0(\nu),\,\tau'(g^{\ssbullet})<\infty\}$,
where $\tau'$ is the associate extended Fatou norm of
$\tau$\cmmnt{ (369H-369I)}.   Suppose that $k\in\eusm L^0(\lambda)$ is
such that $k\times(f\otimes g)$ is integrable whenever
$f\in\eusm L^1(\mu)$ and $g\in\eusm L^{\tau'}$.
Then we have a corresponding linear operator $T:L^1(\mu)\to L^{\tau}$
defined by saying that
$\int(Tf^{\ssbullet})\times g^{\ssbullet}=\int k\times(f\otimes g)$
whenever $f\in\eusm L^1(\mu)$ and $g\in\eusm L^{\tau'}$.

For $x\in X$ set $k_x(y)=k(x,y)$ whenever this is defined.    Then
$k_x\in\eusm L^0(\nu)$ for almost every $x$;  set
$v_x=k_x^{\ssbullet}\in L^0(\nu)$ for such $x$.   In this case
$x\mapsto\tau(v_x)$ is measurable and defined and finite almost
everywhere, and $\|T\|=\esssup_x\tau(v_x)$.

\cmmnt{\medskip

\noindent{\bf Remarks} The discussion of extended Fatou norms in \S369
regarded them as functionals on spaces of the form $L^0(\frak A)$.   I
trust that no-one will be offended if I now speak of an extended Fatou
norm on $L^0(\nu)$, with the associated function spaces $L^{\tau}$,
$L^{\tau'}\subseteq L^0$, taking for granted the identification in
364Ic.

Recall that $(f\otimes g)(x,y)=f(x)g(y)$ for $x\in\dom f$ and $y\in\dom g$
(253B).

\wheader{376S}{0}{0}{0}{24pt}

By `$\esssup_x\tau(v_x)$' I mean

\Centerline{$\inf\{M:M\ge 0,\,\{x:v_x$ is defined and $\tau(v_x)\le M\}$
is conegligible$\}$}

\noindent (see 243D).
}%end of comment

\proof{{\bf (a)} To see that the formula
$(f,g)\mapsto\int k\times(f\otimes g)$ gives rise to an operator in
$\eurm L^{\times}(U;(L^{\tau'})^{\times})$, it is perhaps quickest to
repeat the
argument of parts (a) and (b) of the proof of 376E.   (We are not quite
in a position to quote 376E, as stated, because the localizable measure
algebra free product there might be strictly larger than the measure
algebra of $\lambda$;  see 325B.)   The first step, of course, is to
note that changing $f$ or $g$ on a negligible set does not affect the
integral $\int k\times(f\otimes g)$, so that we have a bilinear
functional on $L^1\times L^{\tau'}$;  and the other essential element is
the fact that the maps
$f^{\ssbullet}\mapsto(f\otimes\chi Y)^{\ssbullet}$,
$g^{\ssbullet}\mapsto (\chi X\otimes g)^{\ssbullet}$
are order-continuous (put 325A and 364Pc together).

By 369K, we can identify $(L^{\tau'})^{\times}$ with $L^{\tau}$, so that
$T$ becomes an operator in $\eurm L^{\times}(U;L^{\tau})$.   Note that
it must be norm-bounded (355C).

\medskip

{\bf (b)} By 376I, there is a non-decreasing sequence $\sequencen{Y_n}$
of measurable sets in $Y$, covering $Y$, such that
$\chi Y_n\in\eusm L^{\tau'}$ for every $n$.
Set $X_0=\{x:x\in X,\,k_x\in\eusm L^0(\nu)\}$.   Then $X_0$ is
conegligible in $X$.   \Prf\ Let $E\in\Sigma$ be any set of finite
measure.   Then for any $n\in\Bbb N$, $k\times(\chi E\otimes\chi Y_n)$
is integrable, that is, $\int_{E\times Y_n}k$ is defined and finite;  so
by Fubini's theorem $\int_{Y_n}k_x$ is defined and finite for almost
every $x\in E$.   Consequently, for almost every $x\in E$,
$k_x\times\chi Y_n\in\eusm L^0(\nu)$ for every $n\in\Bbb N$, that is,
$k_x\in\eusm L^0(\nu)$, that is, $x\in X_0$.

Thus $E\setminus X_0$ is negligible for every set $E$ of finite measure.
Because $\mu$ is complete and locally determined, $X_0$ is
conegligible.\ \Qed

This means that $v_x$ and $\tau(v_x)$ are defined for almost every $x$.

\medskip

{\bf (c)} $\tau(v_x)\le\|T\|$ for almost every $x$.   \Prf\ Take any
$E\in\Sigma$ of finite measure, and $n\in\Bbb N$.   Then
$k\times\chi(E\times Y_n)$ is integrable.   For each $r\in\Bbb N$, there
is a finite partition $E_{r0},\ldots,E_{r,m(r)}$ of $E$ into measurable
sets such that $\int_{E\times Y_n}|k-k^{(r)}|\le 2^{-r}$, where

$$\eqalign{k^{(r)}(x,y)&=\Bover1{\mu E_{ri}}\int_{E_{ri}}k(t,y)dt
\text{ whenever }y\in Y_n,\,x\in E_{ri},\,\mu E_{ri}>0\cr
&\hskip12em\text{ and the integral is defined in }\Bbb R\cr
&=0\text{ otherwise}\cr}$$

\noindent (376R).   Now $k^{(r)}$ also is integrable over $E\times Y_n$,
so $k^{(r)}_x\in\eusm L^0(\nu)$ for almost every $x\in E$, writing
$k^{(r)}_x(y)=k^{(r)}(x,y)$, and we can speak of
$v^{(r)}_x=(k^{(r)}_x)^{\ssbullet}$ for almost every $x$.   Note that
$k^{(r)}_x=k^{(r)}_{x'}$ whenever $x$, $x'$ belong to the same $E_{ri}$.

If $\mu E_{ri}>0$, then $v^{(r)}_x$ must be defined for every $x\in
E_{ri}$.   If $v'\in L^{\tau'}$ is represented by $g\in\eusm L^{\tau'}$
then

$$\eqalign{\int k\times(\chi E_{ri}\otimes(g\times\chi Y_n))
&=\int_{E_{ri}\times Y_n}k(t,y)g(y)d(t,y)\cr
&=\mu E_{ri}\int k^{(r)}(x,y)g(y)dy
=\mu E_{ri}\int v^{(r)}_x\times v'\cr}$$

\noindent for any $x\in E_{ri}$.   But this means that

\Centerline{$\mu E_{ri}\int v^{(r)}_x\times v'=\int T(\chi
E_{ri}^{\ssbullet})\times v'\times\chi Y_n^{\ssbullet}$}

\noindent for every $v'\in L^{\tau'}$, so

\Centerline{$v^{(r)}_x=\Bover1{\mu E_{ri}}T(\chi
E_{ri}^{\ssbullet})\times\chi Y_n^{\ssbullet}$,
\quad$\tau(v^{(r)}_x)\le\Bover1{\mu E_{ri}}\|T\|\|\chi
E_{ri}^{\ssbullet}\|_1=\|T\|$}

\noindent for every $x\in E_{ri}$.   This is true whenever
$\mu E_{ri}>0$, so in fact $\tau(v^{(r)}_x)\le\|T\|$ for almost every
$x\in E$.

Because $\sum_{r\in\Bbb N}\int_{E\times Y_n}|k-k^{(r)}|<\infty$, we must
have $k(x,y)=\lim_{r\to\infty}k^{(r)}(x,y)$ for almost every
$(x,y)\in E\times Y_n$.   Consequently, for almost every $x\in E$,
$k(x,y)=\lim_{r\to\infty}k^{(r)}(x,y)$ for almost every $y\in Y_n$, that
is, $\sequence{r}{v^{(r)}_x}$ order*-converges to
$v_x\times\chi Y_n^{\ssbullet}$
(in $L^0(\nu)$) for almost every $x\in E$.   But this means that, for
almost every $x\in E$,

\Centerline{$\tau(v_x\times\chi Y_n^{\ssbullet})
\le\liminf_{r\to\infty}\tau(v^{(r)}_x)\le\|T\|$}

\noindent (369Mc).   Now

\Centerline{$\tau(v_x)=\lim_{n\to\infty}\tau(v_x\times\chi
Y_n^{\ssbullet})\le\|T\|$}

\noindent for almost every $x\in E$.

As in (b), this implies (since $E$ is arbitrary) that
$\tau(v_x)\le\|T\|$ for almost every $x\in X$.\ \Qed

\medskip

{\bf (d)} I now show that $x\mapsto\tau(v_x)$ is measurable.   \Prf\
Take $\gamma\in\coint{0,\infty}$ and set
$A=\{x:x\in X_0,\,\tau(v_x)\le\gamma\}$.   Suppose that $\mu E<\infty$.
Let $G$ be a measurable envelope of $A\cap E$ (132Ee).   Set
$\tilde k(x,y)=k(x,y)$ when $x\in G$ and $(x,y)\in\dom k$, $0$
otherwise.   If $f\in\eusm L^1(\mu)$ and $g\in\eusm L^{\tau'}$, then

\Centerline{$\int\tilde k(x,y)f(x)g(y)d(x,y)
=\int_{G\times Y}k(x,y)f(x)g(y)d(x,y)=\int_Gf(x)\int_Yk(x,y)g(y)dydx$}

\noindent is defined.

Take any $g\in\eusm L^{\tau'}$.
For $x\in X_0$, set $h(x)=\int|\tilde k(x,y)g(y)|dy$.   Then $h$ is
finite almost everywhere and measurable.   For $x\in A\cap E$,

\Centerline{$\int|\tilde k(x,y)g(y)|dy
=\int|v_x\times g^{\ssbullet}|\le\gamma\tau'(g^{\ssbullet})$.}

\noindent So the measurable set
$G'=\{x:h(x)\le\gamma\tau'(g^{\ssbullet})\}$ includes $A\cap E$, and
$\mu(G\setminus G')=0$.   Consequently

\Centerline{$|\int\tilde k(x,y)f(x)g(y)d(x,y)|\le\int_G|f(x)|h(x)dx
\le\gamma\|f\|_1\tau'(g^{\ssbullet})$,}

\noindent and this is true whenever $f\in\eusm L^1(\mu)$.

Now we have an operator $\tilde T:L^1(\mu)\to L^{\tau}$ defined by the
formula

\Centerline{$\int(\tilde Tf^{\ssbullet})\times g^{\ssbullet}
=\int\tilde k\times(f\otimes g)$ when $f\in\eusm L^1(\nu)$ and
$g\in\eusm L^{\tau'}$,}

\noindent and the formula just above tells us that
$|\int\tilde Tu\times v'|\le\gamma\|u\|_1\tau'(v')$ for every
$u\in L^1(\nu)$ and $v'\in L^{\tau'}$;  that is,
$\tau(\tilde Tu)\le\gamma\|u\|_1$ for every $u\in L^1(\mu)$;  that is,
$\|\tilde T\|\le\gamma$.   But now (c) tells us
that $\tau(\tilde v_x)\le\gamma$ for almost every $x\in X$, where
$\tilde v_x$ is the equivalence class of $y\mapsto \tilde k(x,y)$, that
is, $\tilde v_x=v_x$ for $x\in G\cap X_0$, $0$ for $x\in X\setminus G$.
So $\tau(v_x)\le\gamma$ for almost every $x\in G$, and $G\setminus A$ is
negligible.   But this means that $A\cap E$ is measurable.   As $E$ is
arbitrary, $A$ is measurable;  as $\gamma$ is arbitrary,
$x\mapsto\tau(v_x)$ is measurable.\ \Qed

\medskip

{\bf (e)} Finally, the ideas in (d) show that
$\|T\|\le\esssup_x\tau(v_x)$.   \Prf\ Set $M=\esssup_x\tau(v_x)$.   If
$f\in\eusm L^1(\mu)$ and $g\in\eusm L^{\tau'}$, then

\Centerline{$\int|k(x,y)f(x)g(y)|d(x,y)
\le\int|f(x)|\tau(v_x)\tau'(g^{\ssbullet})dx
\le M\|f\|_1\tau'(g^{\ssbullet})$;}

\noindent as $g$ is arbitrary, $\tau(Tf^{\ssbullet})\le M\|f\|_1$;  as
$f$ is arbitrary, $\|T\|\le M$.\ \Qed
}%end of proof of 376S

\exercises{\leader{376X}{Basic exercises $\pmb{>}$(a)}
%\spheader 376Xa
Let $\mu$ be Lebesgue measure on $\Bbb R$.   Let $h$ be a
$\mu$-integrable
real-valued function with $\|h\|_1\le 1$, and set $k(x,y)=h(y-x)$
whenever this is defined.   Show that if $f$ is in either
$\eusm L^1(\mu)$ or $\eusm L^{\infty}(\mu)$ then
$g(y)=\int k(x,y)f(x)dx$ is
defined for almost every $y\in\Bbb R$, and that this formula gives rise
to an operator $T\in\Cal T^{\times}_{\bar\mu,\bar\mu}$ as defined in
373Ab.   \Hint{255H.}
%376A

\spheader 376Xb Let $(\frak A,\bar\mu)$ and $(\frak B,\bar\nu)$ be
semi-finite measure algebras with localizable measure algebra free
product $(\frak C,\bar\lambda)$, and take $p\in[1,\infty]$.   Show that
if $u\in L^p(\frak A,\bar\mu)$ and $v\in L^p(\frak B,\bar\nu)$ then
$u\otimes v\in L^p(\frak C,\bar\lambda)$ and
$\|u\otimes v\|_p=\|u\|_p\|v\|_p$.
%376C

\sqheader 376Xc Let $U$, $V$, $W$ be Riesz spaces, of which $V$ and $W$
are Dedekind complete, and suppose that $T\in\eurm L^{\times}(U;V)$ and
$S\in\eurm L^{\times}(V;W)$.   Show that if either $S$ or $T$ is an
abstract integral operator, so is $ST$.
%376D

\spheader 376Xd Let $h$ be a Lebesgue integrable function on $\Bbb R$,
and $f$ a square-integrable function.   Suppose that $\sequencen{f_n}$
is a sequence of measurable functions such that ($\alpha$) $|f_n|\le f$
for every $n$ ($\beta$) $\lim_{n\to\infty}\int_Ef_n=0$ for every
measurable set $E$ of finite measure.   Show that
$\lim_{n\to\infty}(h*f_n)(y)=0$ for almost every $y\in\Bbb R$, where
$h*f_n$ is the convolution of $h$ and $f_n$.   \Hint{376Xa, 376H.}
%376H, 376Xa

\spheader 376Xe Let $U$ and $V$ be Riesz spaces, of which $V$ is
Dedekind complete.   Suppose that $W\subseteq U^{\sim}$ is a solid
linear subspace, and that $T$ belongs to the band in $\eurm L^{\sim}(U;V)$
generated by operators of the form $u\mapsto f(u)v$,
where $f\in W$ and $v\in V$.   Show that whenever $\sequencen{u_n}$ is
an order-bounded sequence in $U$ such that $\lim_{n\to\infty}f(u_n)=0$
for every $f\in W$, then $\sequencen{Tu_n}$ order*-converges to $0$
in $V$.
%376H

\spheader 376Xf Let $(\frak A,\bar\mu)$ be a semi-finite measure algebra
and $U\subseteq L^0=L^0(\frak A)$ an order-dense Riesz subspace such
that $U^{\times}$ separates the points of $U$.   Let $\sequencen{u_n}$
be an order-bounded sequence in $U$.   Show that the following are
equiveridical:  (i) $\lim_{n\to\infty}f(|u_n|)=0$ for every $f\in
U^{\times}$;  (ii) $\sequencen{u_n}\to 0$ for the topology of
convergence in measure on $L^0$.   \Hint{by 367T, condition (ii) is
intrinsic to $U$, so we can replace $(\frak A,\bar\mu)$ by a localizable
algebra and use the representation in 369D.}
%376H

\spheader 376Xg Let $U$ be a Banach lattice with an order-continuous
norm, and $V$ a \wsid\ Riesz space.  Show that for $T\in\eurm
L^{\sim}(U;V)$ the following are equiveridical:  (i) $T$ belongs to the
band in $\eurm L^{\sim}(U;V)$ generated by operators of the form
$u\mapsto f(u)v$ where $f\in U^{\sim}$, $v\in V$;  (ii)
$\sequencen{Tu_n}$ order*-converges to $0$ in $V$ whenever
$\sequencen{u_n}$ is
an order-bounded sequence in $U^+$ which is norm-convergent to $0$;
(iii) $\sequencen{Tu_n}$ order*-converges to $0$ in $V$ whenever
$\sequencen{u_n}$ is an order-bounded sequence in $U$ which is weakly
convergent to $0$.
%376H

\spheader 376Xh Let $(X,\Sigma,\mu)$ and $(Y,\Tau,\nu)$ be
$\sigma$-finite measure spaces, with product measure $\lambda$ on
$X\times Y$, and measure algebras $(\frak A,\bar\mu)$,
$(\frak B,\bar\nu)$.   Suppose that $k\in\eusm L^0(\lambda)$.   Show
that the
following are equiveridical:  (i)($\alpha$) if $f\in\eusm L^1(\mu)$ then
$g_f(y)=\int k(x,y)f(x)dx$ is defined for almost every $y$ and
$g_f\in\eusm L^1(\nu)$ ($\beta$) there is an operator
$T\in\Cal T^{\times}_{\bar\mu,\bar\nu}$ defined by setting
$Tf^{\ssbullet}=g_f^{\ssbullet}$
for every $f\in \eusm L^1(\mu)$;  (ii) $\int|k(x,y)|dy\le 1$ for almost
every $x\in X$, $\int|k(x,y)|dx\le 1$ for almost every $y\in Y$.
%376J

\sqheader 376Xi(i) Show that there is a compact linear operator from
$\ell^2$ to itself which is not in $\eurm L^{\sim}(\ell^2;\ell^2)$.
\Hint{start from the operator $S$ of 371Ye.}   (ii) Show that the
identity operator on $\ell^2$ is an abstract integral operator.
%376J

\sqheader 376Xj Let $\mu$ be Lebesgue measure on $[0,1]$.   (i) Give an
example of a measurable function $k:[0,1]^2\to\Bbb R$ such that, for any
$f\in\eusm L^2(\mu)$, $g_f(y)=\int k(x,y)f(x)dx$ is defined
for every $y$ and $\|g_f\|_2=\|f\|_2$, but $k$ is not integrable, so the
linear isometry on $L^2=L^2(\mu)$ defined by $k$ does not belong to
$\eurm L^{\sim}(L^2;L^2)$.   (ii) Show that the identity operator on $L^2$
is not an abstract integral operator.
%376Xi %376J

\spheader 376Xk Let $(X,\Sigma,\mu)$ be a $\sigma$-finite measure space
and $(Y,\Tau,\nu)$ a complete locally determined measure space.   Let
$U\subseteq L^0(\mu)$, $V\subseteq L^0(\nu)$ be solid linear subspaces,
of which $V$ is order-dense;  write
$V^{\#}=\{v:v\in L^0(\nu),\,v\times v'$ is integrable for every
$v'\in V\}$,
$\eusm U=\{f:f\in\eusm L^0(\nu),\,f^{\ssbullet}\in U\}$,
$\eusm V=\{g:g\in\eusm L^0(\nu),\,g^{\ssbullet}\in V\}$,
$\eusm V^{\#}=\{h:h\in\eusm L^0(\nu),\,h^{\ssbullet}\in V^{\#}\}$.
Let $\lambda$ be the c.l.d.\ product measure on $X\times Y$, and
$k\in\eusm L^0(\lambda)$ a function such that $k\times(f\otimes g)$ is
integrable for whenever $f\in\eusm U$ and $g\in\eusm V$.
(i) Show that for
any $f\in\eusm U$, $h_f(y)=\int k(x,y)f(x)dx$ is defined for almost
every $y\in Y$, and that $h_f\in\eusm V^{\#}$.   (ii) Show that we have
a map $T\in\eurm L^{\times}(U;V^{\#})$ defined {\it either} by writing
$Tf^{\ssbullet}=h_f^{\ssbullet}$ for every $f\in\eusm U$ {\it or} by
writing
$\int(Tf^{\ssbullet})\times g^{\ssbullet}=\int k\times(f\otimes g)$ for
every $f\in\eusm U$ and $g\in\eusm V$.
%376J

\spheader 376Xl Let $(X,\Sigma,\mu)$, $(Y,\Tau,\nu)$ and
$(Z,\Lambda,\lambda)$ be $\sigma$-finite measure spaces, and $U$, $V$,
$W$ perfect order-dense solid linear subspaces of $L^0(\mu)$, $L^0(\nu)$
and $L^0(\lambda)$ respectively.   Suppose that $T:U\to V$ and $S:V\to
W$ are abstract integral operators corresponding to kernels
$k_1\in\eusm L^0(\mu\times\nu)$, $k_2\in\eusm L^0(\nu\times\lambda)$,
writing
$\mu\times\nu$ for the (c.l.d.\ or primitive) product measure on
$X\times Y$.   Show that $ST:U\to W$ is represented by the kernel
$k\in\eusm L^0(\mu\times\lambda)$ defined by setting
$k(x,z)=\int k_1(x,y)k_2(y,z)dy$ whenever this integral is defined.
%376J

\spheader 376Xm Let $U$ be a perfect Riesz space.   Show that a set
$C\subseteq U$ is relatively compact for $\frak T_s(U,U^{\times})$ iff
for every $g\in(U^{\times})^+$, $\epsilon>0$ there is a $u^*\in U$ such
that $g(|u|-u^*)^+\le\epsilon$ for every $u\in C$.   \Hint{376O and the
proof of 356Q.}

\sqheader 376Xn Let $\mu$ be Lebesgue measure on $[0,1]$, and $\nu$
counting measure on $[0,1]$.   Set $k(x,y)=1$ if $x=y$, $0$ otherwise.
Show that 376S fails in this context (with, e.g.,
$\tau=\|\,\|_{\infty}$).
%376S

\spheader 376Xo Suppose, in 376Xk, that $U=L^{\tau}$ for some extended
Fatou norm on $L^0(\mu)$ and that $V=L^1(\nu)$, so that
$V^{\#}=L^{\infty}(\nu)$.   Set $k_y(x)=k(x,y)$ whenever this is
defined, $w_y=k_y^{\ssbullet}$ whenever $k_y\in\eusm L^0(\mu)$.   Show
that $w_y\in L^{\tau'}$ for almost every $y\in Y$, and that the norm of
$T$ in $\eurm B(L^{\tau};L^{\infty})$ is $\esssup_y\tau'(w_y)$.
\Hint{do the case of totally finite $Y$ first.}
%376S, 376Xk

\leader{376Y}{Further exercises (a)}
%\spheader 376Ya
Let $U$, $V$ and $W$ be linear spaces (over any field $F$) and
$\phi:U\times V\to W$ a bilinear operator.   Let $W_0$ be the linear subspace
of $W$ generated by $\phi[U\times V]$.   Show that the following are
equiveridical:  (i) for every linear space $Z$ over $F$ and every
bilinear
$\psi:U\times V\to Z$, there is a (unique) linear operator $T:W_0\to Z$
such
that $T\phi=\psi$ (ii) whenever $u_0,\ldots,u_n\in U$ are linearly
independent and $v_0,\ldots,v_n\in V$ are non-zero,
$\sum_{i=0}^n\phi(u_i,v_i)\ne 0$
(iii) whenever $u_0,\ldots,u_n\in U$ are non-zero and
$v_0,\ldots,v_n\in V$ are linearly independent,
$\sum_{i=0}^n\phi(u_i,v_i)\ne 0$ (iv) for
any Hamel bases $\langle u_i\rangle_{i\in I}$,
$\langle v_j\rangle_{j\in J}$ of $U$ and $V$,
$\langle\phi(u_i,v_j)\rangle_{i\in I,j\in J}$ is a
Hamel basis of $W_0$
(v) for some pair $\langle u_i\rangle_{i\in I}$,
$\langle v_j\rangle_{j\in J}$ of Hamel bases of $U$ and $V$,
$\langle\phi(u_i,v_j)\rangle_{i\in I,j\in J}$ is a Hamel basis of $W_0$.
%`Hamel basis' not defined until v4

\spheader 376Yb
Let $(\frak A,\bar\mu)$, $(\frak B,\bar\nu)$ be semi-finite measure
algebras, and $(\frak C,\bar\lambda)$ their localizable measure algebra
free product.   Show that $\otimes:L^0(\frak A)\times L^0(\frak B)\to
L^0(\frak C)$ satisfies the equivalent conditions of 376Ya.
%376B, 376Ya

\spheader 376Yc Let $(X,\Sigma,\mu)$ and $(Y,\Tau,\nu)$ be semi-finite
measure spaces and $\lambda$ the c.l.d.\ product measure on $X\times Y$.
Show that the map $(f,g)\mapsto f\otimes g:
\eusm L^0(\mu)\times\eusm L^0(\nu)\to\eusm L^0(\lambda)$ induces a map
$(u,v)\mapsto u\otimes v:L^0(\mu)\times L^0(\nu)\to L^0(\lambda)$
possessing all the properties described in 376B and 376Ya, subject to
a suitable interpretation of the formula
$\otimes:\frak A\times\frak B\to\frak C$.
%376B, 376Ya

\spheader 376Yd Let $(\frak B_{\omega_1},\bar\nu_{\omega_1})$
be the measure algebra of
$\{0,1\}^{\omega_1}$ with its usual measure, and
$\langle a_{\xi}\rangle_{\xi<\omega_1}$ a stochastically independent
(definition:  325Xf) family
of elements of measure $\bover12$ in $\frak B_{\omega_1}$.   Set
$U=L^2(\frak B_{\omega_1},\bar\nu_{\omega_1})$ and
$V=\{v:v\in\BbbR^{\omega_1},\,\{\xi:v(\xi)\ne 0\}$ is
countable$\}$.   Define $T:U\to\BbbR^{\omega_1}$ by setting
$Tu(\xi)=2\int_{a_{\xi}}u-\int u$ for $\xi<\omega_1$, $u\in U$.     Show
that (i) $Tu\in V$ for every $u\in U$ (ii)
$\sequencen{Tu_n}$ order*-converges to $0$ in $V$ whenever
$\sequencen{u_n}$ is a sequence in $U$ such that
$\lim_{n\to\infty}f(u_n)=0$ for every $f\in U^{\times}$ (iii)
$T\notin\eurm L^{\sim}(U;V)$.
%376H

\spheader 376Ye Let $U$ be a Riesz space with the countable sup property
(definition:  241Ye)
such that $U^{\times}$ separates the points of $U$, and
$\sequencen{u_n}$ a sequence in $U$.   Show that the
following are equiveridical:  (i) $\lim_{n\to\infty}f(v\wedge|u_n|)=0$
for every $f\in U^{\times}$, $v\in U^+$;  (ii) every subsequence of
$\sequencen{u_n}$ has a
sub-subsequence which is order*-convergent to $0$.
%376H

\spheader 376Yf Let $U$ be an Archimedean Riesz space and $\frak A$ a
\wsid\ Dedekind complete Boolean algebra.   Suppose that $T:U\to
L^0=L^0(\frak A)$ is a linear operator such that
$\sequencen{|Tu_n|}$ order*-converges to $0$ in $L^0$ whenever
$\sequencen{u_n}$ is order-bounded and order*-convergent to $0$ in
$U$.   Show that $T\in\eurm L^{\sim}_c(U;L^0)$ (definition:  355G),
so that if $U$ has the
countable sup property then $T\in\eurm L^{\times}(U;L^0)$.
%376H

\spheader 376Yg Suppose that $(Y,\Tau,\nu)$ is a probability space in
which $\Tau=\Cal PY$ and $\nu\{y\}=0$ for every $y\in Y$.    (See
363S.)   Take $X=Y$ and let $\mu$ be counting measure on $X$;  let
$\lambda$ be the c.l.d.\ product measure on $X\times Y$, and set
$k(x,y)=1$ if $x=y$, $0$ otherwise.   Show that we have an operator
$T:L^{\infty}(\mu)\to L^{\infty}(\nu)$ defined by setting
$Tf=g^{\ssbullet}$ whenever $f\in L^{\infty}(\mu)\cong\ell^{\infty}(X)$
and $g(y)=\int k(x,y)f(x)dx=f(y)$ for every $y\in Y$.   Show that $T$
satisfies the conditions (ii) and (iii) of 376J but does not belong to
$\eurm L^{\times}(L^{\infty}(\mu);L^{\infty}(\nu))$.
%376J

\spheader 376Yh Give an example of an abstract integral operator
$T:\ell^2\to L^1(\mu)$, where $\mu$ is Lebesgue measure on $[0,1]$, such
that $\sequencen{Te_n}$ is not order*-convergent in $L^1(\mu)$, where
$\sequencen{e_n}$ is the standard orthonormal sequence in $\ell^2$.
%376J

\spheader 376Yi Set $k(m,n)=1/\pi(n-m+\bover12)$ for $m$, $n\in\Bbb Z$.
(i) Show that
$\sum_{n=-\infty}^{\infty}k(m,n)^2=1$ and
$\sum_{n=-\infty}^{\infty}k(m,n)k(m',n)=0$ for all distinct $m$,
$m'\in\Bbb Z$.   \Hint{find the Fourier series of
$x\mapsto e^{i(m+\bover12)x}$ and use 282K.}
(ii) Show that there is a norm-preserving linear operator $T$ from
$\ell^2=\ell^2(\Bbb Z)$ to itself given by the formula
$(Tu)(n)=\sum_{m=-\infty}^{\infty}k(m,n)u(m)$.   (iii) Show
that $T^2$ is the identity operator on $\ell^2$.   (iv) Show that
$T\notin\eurm L^{\sim}(\ell^2;\ell^2)$.   \Hint{consider
$\sum_{m,n=-\infty}^{\infty}|k(m,n)|x(m)x(n)$ where
$x(n)=1/\sqrt{|n|}\ln|n|$ for $|n|\ge 2$.}   ($T$ is a form of the
{\bf Hilbert transform.})
%376J

\spheader 376Yj Let $U$ be an $L$-space and $V$ a Banach lattice with an
order-continuous norm.   Let $T\in\eurm L^{\sim}(U;V)$.   Show that the
following are equiveridical:  (i) $T$ is an abstract integral operator;
(ii) $T[C]$ is norm-compact in $V$ whenever $C$ is weakly compact in
$U$.   \Hint{start with the case in which $C$ is order-bounded, and
remember that it is weakly sequentially compact.}
%376M

\spheader 376Yk Let $(X,\Sigma,\mu)$ be a complete locally determined
measure space and $(Y,\Tau,\nu)$, $(Z,\Lambda,\lambda)$ two
$\sigma$-finite measure spaces.   Suppose that $\tau$, $\theta$ are
extended Fatou norms on $L^0(\nu)$, $L^0(\lambda)$ respectively, and
that $T:L^1(\mu)\to L^{\tau}$ is an abstract integral operator, with
corresponding kernel $k\in\eusm L^0(\mu\times\nu)$, while $S\in\eurm
L^{\times}(L^{\tau};L^{\theta})$, so that $ST:L^1(\mu)\to L^{\theta}$ is
an abstract integral operator (376Xc);  let
$\tilde k\in\eusm L^0(\mu\times\lambda)$ be the corresponding kernel.
For $x\in X$ set
$v_x=k_x^{\ssbullet}$ when this is defined in $L^{\tau}$, as in 376S,
and similarly take $w_x=\tilde k_x^{\ssbullet}\in L^{\theta}$.   Show
that $Sv_x=w_x$ for almost every $x\in X$.
%376S
}%end of exercises

\endnotes{
\Notesheader{376}
I leave 376Yb to the exercises because I do not rely on it for any of
the work here, but of course it is an essential aspect of the map
$\otimes:L^0(\frak A)\times L^0(\frak B)\to L^0(\frak C)$ I discuss in
this section.   The conditions in 376Ya are characterizations of the
`tensor product' of two linear spaces, a construction of great
importance
in abstract linear algebra (and, indeed, in modern applied linear
algebra;  it is by no means trivial even in the finite-dimensional
case).   In particular, note that conditions (ii), (iii) of 376Ya apply
to arbitrary subspaces of $U$ and $V$ if they apply to $U$ and $V$
themselves.

The principal ideas used in 376B-376C have already been set out in
\S\S253 and 325.   Here I do little more than list the references.   I
remark however that it is quite striking that $L^1(\frak C,\bar\lambda)$
should have no fewer than three universal mapping theorems attached to
it (376Cb, 376C(c-i) and 376C(c-ii)).

The real work of this section begins in 376E.
As usual, much of the proof is taken up with relatively straightforward
verifications, as in parts (a) and (b), while part (i) is just a
manoeuvre to show that it doesn't matter if $\frak A$ and $\frak B$
aren't Dedekind complete, because $\frak C$ is.   But I think that parts
(d), (f) and (j) have ideas in them.   In particular, part (f) is a kind
of application of the Radon-Nikod\'ym theorem (through the
identification of $L^1(\frak C,\bar\lambda)^*$ with
$L^{\infty}(\frak C)$).

I have split 376E from 376H because the former demands the language of
measure algebras, while the latter can be put into the language of pure
Riesz space theory.   Asking for a \wsid\ space $V$ in 376H is a way of
applying the ideas to $V=L^0$ as well as to Banach function spaces.
(When $V=L^0$, indeed, variations on the hypotheses are possible, using
376Yf.)   But it is a reminder of one of the directions in which it is
often possible to find generalizations of ideas beginning in measure
theory.

The condition `$\lim_{n\to\infty}f(u_n)=0$ for every
$f\in U^{\times}$' (376H(ii)) seems natural in this context, and
gives marginally greater generality than some alternatives (because it
does the right thing when $U^{\times}$ does not separate the points of
$U$), but it is not the only way of expressing the idea;  see 376Xf and
376Ye.   Note that the conditions (ii) and (iii) of 376H are
significantly
different.   In 376H(iii) we could easily have $|u_n|=u^*$ for every
$n$;  for instance, if $u_n=2\chi a_n-\chi 1$ for some stochastically
independent sequence $\sequencen{a_n}$ of elements of measure $\bover12$
in a probability algebra (272Ye).

If you have studied compact linear operators between Banach spaces
(definition:  3A5La), you will have encountered the condition
`$Tu_n\to 0$ strongly whenever $u_n\to 0$ weakly'.   The conditions in
376H and
376J are of this type.   If a sequence $\sequencen{u_n}$ in a Riesz
space $U$ is order-bounded and order*-convergent to $0$, then
$\lim_{n\to\infty}f(u_n)=0$ for every $f\in U^{\times}$ (367Xf).
Visibly this latter condition is associated
with weak convergence, and order*-convergence is (in Banach lattices)
closely related to norm convergence (367D).   In the context of
376H, an abstract integral operator is one which transforms convergent
sequences of a weak type into convergent sequences of a stronger type.
The relationship between the classes of (weakly) compact operators and
abstract integral operators is interesting, but outside the scope of
this book;  I leave you with 376P-376Q and 376Y, and a pair of
elementary examples to guard against extravagant conjecture (376Xi).

376O belongs to an extensive general theory of weak compactness in
perfect Riesz spaces, based on adaptations of the concept of `uniform
integrability'.  I give the next step in 376Xm.   For more information
see {\smc Fremlin 74a}, chap.\ 8.

Note that 376Mb and 376P overlap when $V^{\times}$ in 376Mb is reflexive
-- for instance, when $V$ is an $L^p$ space for some
$p\in\ooint{1,\infty}$ -- since then every bounded linear operator from
$L^1$ to $V^{\times}$ must be weakly compact.   For more information on the
representation of operators see {\smc Dunford \& Schwartz 57},
particularly Table VI in the notes to Chapter VI.

As soon as we leave formulations in terms of the spaces $L^0(\frak A)$
and their subspaces, and return to the original conception of a
kernel operator in terms of integrating functions against sections of
a kernel, we are necessarily involved in the pathology of Fubini's
theorem for general measure spaces.   In general, the repeated integrals
$\iint k(x,y)dxdy$, $\iint k(x,y)dydx$ need not be equal, and something
has to give (376Xn).   Of course this particular worry disappears if the
spaces are $\sigma$-finite, as in 376J.  In 376S I take the trouble to
offer a more general condition,
mostly as a reminder that the techniques developed in Volume 2
do enable us sometimes to go beyond the $\sigma$-finite case.   Note
that this is one of the many contexts in which anything we can prove
about probability spaces will be true of all $\sigma$-finite spaces;
but that we cannot make the next step, to all strictly localizable
spaces.

376S verges on the theory of integration of vector-valued functions,
which I don't wish to enter here;  but it also seems to have a natural
place in the context of this chapter.   It is of course a special
property of $L^1$ spaces.   The formula
$\|T_k\|=\esssup_x\tau(k_x^{\ssbullet})$ shows that
$\|T_{|k|}\|=\|T_k\|$;
now we know fron 376E that $T_{|k|}=|T_k|$, so we get a special case of
the Chacon-Krengel theorem (371D).   Reversing the roles of $X$ and
$Y$, we find ourselves with an operator from $L^{\tau}$ to $L^{\infty}$
(376Xo), which is the other standard context in which $\|T\|=\||T|\|$
(371Xd).   I include two exercises on $L^2$ spaces (376Xj, 376Yi)
designed to emphasize the fact that $\eurm B(U;V)$ is included in
$\eurm L^{\sim}(U;V)$ only in very special cases.

The history of the theory here is even more confusing than that of
mathematics in general, because so many of the ideas were developed in
national schools in very imperfect contact with each other.   My own
account gives no hint of how this material arose;  I ought in
particular to note that 376N is one of the oldest results, coming
(essentially) from {\smc Dunford 36}.   For further references,
see {\smc Zaanen 83}, chap.\ 13.
}%end of comment

\discrpage

