\frfilename{mt422.tex}
\versiondate{4.12.04}
\copyrightdate{2001}

\def\NN{\BbbN^{\Bbb N}}

\def\chaptername{Descriptive set theory}
\def\sectionname{K-analytic spaces}

\newsection{422}

I introduce K-analytic spaces, defined in terms of usco-compact
relations.   The first step is to define the latter (422A) and give
their fundamental properties (422B-422E).   I reach K-analytic spaces
themselves in 422F, with an outline of the most important facts about
them in 422G-422K.

\leader{422A}{Definition} Let $X$ and $Y$ be Hausdorff spaces.   A
relation $R\subseteq X\times Y$ is {\bf usco-compact} if

\inset{($\alpha$) $R[\{x\}]$ is a compact subset of $Y$ for every
$x\in X$,}

\inset{($\beta$) $R^{-1}[F]$ is a closed subset of $X$ for every closed
set $F\subseteq Y$.}

\cmmnt{\noindent (Relations satisfying condition ($\beta$) are
sometimes called `upper semi-continuous'.)}

\leader{422B}{}\cmmnt{ The following elementary remark will be useful.

\medskip

\noindent}{\bf Lemma} Let $X$ and $Y$ be Hausdorff spaces and
$R\subseteq X\times Y$ an usco-compact relation.   If $x\in X$ and
$H$ is an open subset of $Y$ including $R[\{x\}]$, there is an open set
$G\subseteq X$, containing $x$, such that $R[G]\subseteq H$.

\proof{ Set $G=X\setminus R^{-1}[Y\setminus H]$.   Because $Y\setminus
H$ is closed, so is $R^{-1}[Y\setminus H]$, and $G$ is open.   Of course
$R[G]\subseteq H$, and $x\in G$ because $R[\{x\}]\subseteq H$.
}%end of proof of 422B

\leader{422C}{Proposition} Let $X$ and $Y$ be Hausdorff spaces.   Then a
subset $R$ of $X\times Y$ is an usco-compact relation iff whenever
$\Cal F$ is an ultrafilter on $X\times Y$, containing $R$, such that the
first-coordinate image
$\pi_1[[\Cal F]]$ of $\Cal F$ has a limit in $X$, then $\Cal F$ has a
limit in $R$.

\proof{ Recall that, writing $\pi_1(x,y)=x$ and $\pi_2(x,y)=y$ for
$(x,y)\in X\times Y$,

\Centerline{$\pi_1[[\Cal F]]
=\{A:A\subseteq X,\,\pi_1^{-1}[A]\in\Cal F\}
=\{A:A\subseteq X,\,A\times Y\in\Cal F\}$}

\noindent (2A1Ib), and that $\Cal F\to(x,y)$ iff $\pi_1[[\Cal F]]\to x$
and $\pi_2[[\Cal F]]\to y$ (3A3Ic).

\medskip

{\bf (a)} Suppose that $R$ is usco-compact and that $\Cal F$ is an
ultrafilter on $X\times Y$, containing $R$, such that $\pi_1[[\Cal F]]$
has a limit $x\in X$.   \Quer\ If $\Cal F$ has no limit in $R$, then, in
particular, it does not converge to $(x,y)$ for any $y\in R[\{x\}]$;
that is, $\pi_2[[\Cal F]]$ does not converge to any point of $R[\{x\}]$,
that is, every point of $R[\{x\}]$ belongs to an open set not belonging
to $\pi_2[[\Cal F]]$.   Because $R[\{x\}]$ is compact, it is covered by
a finite union of open sets not belonging to $\pi_2[[\Cal F]]$;  but as
$\pi_2[[\Cal F]]$ is an ultrafilter (2A1N), there is an open set
$H\supseteq R[\{x\}]$ such that $Y\setminus H\in\pi_2[[\Cal F]]$.

Now 422B tells us that there is an open set $G$ containing $x$ such that
$R[G]\subseteq H$.   In this case, $G\in\pi_1[[\Cal F]]$ so $G\times
Y\in\Cal F$;  at the same time, $X\times(Y\setminus H)\in\Cal F$.   So

\Centerline{$R\cap(G\times Y)\cap(X\times(Y\setminus H))\in\Cal F$.}
\noindent But this is empty, by the choice of $G$;  which is
intolerable.\ \Bang

Thus $\Cal F$ has a limit in $R$, as required.

\wheader{422C}{4}{2}{2}{48pt}

{\bf (b)} Now suppose that $R$ has the property described.

\medskip

\quad{\bf (i)} Let $x\in X$, and suppose that $\Cal G$ is an ultrafilter
on $Y$ containing $R[\{x\}]$.   Set $h(y)=(x,y)$ for $y\in Y$;  then
$\Cal F=h[[\Cal G]]$ is an ultrafilter on $X\times Y$ containing $R$.
The image $\pi_1[[\Cal F]]$ is just the principal filter generated by
$\{x\}$, so certainly converges to $x$;  accordingly $\Cal F$ must
converge to some point $(x,y)\in R$, and $\Cal G=\pi_2[[\Cal F]]$
converges to $y\in R[\{x\}]$.   As $\Cal G$ is arbitrary, $R[\{x\}]$ is
compact (2A3R).

\medskip

\quad{\bf (ii)} Let $F\subseteq Y$ be closed, and
$x\in\overline{R^{-1}[F]}\subseteq X$.   Consider

\Centerline{$\Cal E=\{R,X\times F\}\cup\{G\times Y:G\subseteq X$
is open, $x\in G\}$.}

\noindent Then $\Cal E$ has the finite intersection property.   \Prf\ If
$G_0,\ldots,G_n$ are open sets containing $x$, then $R^{-1}[F]$ meets
$G_0\cap\ldots\cap G_n$ in $z$ say, and now
$(z,y)\in R\cap(X\times F)\cap\bigcap_{i\le n}(G_i\times Y)$ for some
$y\in F$.\ \QeD\   Let
$\Cal F$ be an ultrafilter on $X\times Y$ including $\Cal E$
(4A1Ia).   Because $G\times Y\in\Cal E\subseteq\Cal F$ for every
open set $G$ containing $x$, $\pi_1[[\Cal F]]\to x$, so $\Cal F$
converges to some point
$(x,y)$ of $R$.   Because $X\times F$ is a
closed set belonging to $\Cal E\subseteq\Cal F$, $y\in F$ and
$x\in R^{-1}[F]$.   As $x$ is arbitrary, $R^{-1}[F]$ is closed;  as $F$
is arbitrary, $R$ satisfies condition ($\beta$) of 422A, and is
usco-compact.
}%end of proof of 422C

\leader{422D}{Lemma} (a) Let $X$ and $Y$ be Hausdorff spaces.
If $R\subseteq X\times Y$ is an usco-compact relation, then $R$ is
closed in $X\times Y$.

(b) Let $X$ and $Y$ be Hausdorff spaces.
If $R\subseteq X\times Y$ is an usco-compact relation and
$R'\subseteq R$ is a closed set, then $R'$ is usco-compact.

(c) Let $X$ and $Y$ be Hausdorff spaces.   If $f:X\to Y$ is a continuous
function, then its graph is an usco-compact relation.

(d) Let $\familyiI{X_i}$ and $\familyiI{Y_i}$ be families of Hausdorff
spaces, and $R_i\subseteq X_i\times Y_i$ an usco-compact relation for
each $i$.   Set $X=\prod_{i\in I}X_i$, $Y=\prod_{i\in I}Y_i$ and

\Centerline{$R=\{(x,y):x\in X,\,y\in Y,\,(x(i),y(i))\in R_i$ for every
$i\in I\}$.}

\noindent Then $R$ is usco-compact in $X\times Y$.

(e) Let $X$ and $Y$ be Hausdorff spaces, and $R\subseteq X\times Y$ an
usco-compact relation.   Then (i) $R[K]$ is a compact subset of $Y$ for
any compact subset $K$ of $X$ (ii) $R[L]$ is a Lindel\"of subset of $Y$
for any Lindel\"of subset $L$ of $X$.

(f) Let $X$, $Y$ and $Z$ be Hausdorff spaces, and
$R\subseteq X\times Y$, $S\subseteq Y\times Z$ usco-compact relations.
Then the composition

\Centerline{$S\frsmallcirc R=\{(x,z):$ there is some $y\in Y$ such that
$(x,y)\in R$ and $(y,z)\in S\}$}

\noindent is usco-compact in $X\times Z$.

(g) Let $X$ and $Y$ be Hausdorff spaces and $Y_0$ any subset of $Y$.
Then a relation $R\subseteq X\times Y_0$ is usco-compact when regarded
as a relation between $X$ and $Y_0$ iff it is usco-compact when regarded
as a relation between $X$ and $Y$.

(h)\dvAnew{2011} 
Let $Y$ be a Hausdorff space and $R\subseteq\NN\times Y$ an
usco-compact relation.   Set

\Centerline{$R'=\{(\alpha,y):\alpha\in\NN$, $y\in Y$ and there is a 
$\beta\le\alpha$ such that $(\beta,y)\in R\}$.}

\noindent Then $R'$ is usco-compact.

\proof{{\bf (a)} If $(x,y)\in\overline{R}$, there is an ultrafilter
$\Cal F$ containing $R$ and converging to $(x,y)$ (4A2Bc).   By 422C,
$\Cal F$ must have a limit in $R$;  but as $X\times Y$ is Hausdorff,
this limit
must be $(x,y)$, and $(x,y)\in R$.   As $(x,y)$ is arbitrary, $R$ is
closed.

\medskip

{\bf (b)} It is obvious that $R'$ will satisfy the condition of 422C if
$R$ does.

\medskip

{\bf (c)} $f[\{x\}]=\{f(x)\}$ is surely compact for every $x\in X$, and
$f^{-1}[F]$ is closed for every closed set $F\subseteq Y$ because $f$ is
continuous.

\medskip

{\bf (d)} For $i\in I$, $x\in X$, $y\in Y$ set
$\phi_i(x,y)=(x(i),y(i))$.
If $\Cal F$ is an ultrafilter on $X\times Y$ containing $R$ such that
$\pi_1[[\Cal F]]$ has a limit in $X$, then

\Centerline{$\pi_1\phi_i[[\Cal F]]=\psi_i\pi_1[[\Cal F]]$}

\noindent has a limit in $X_i$ for every $i\in I$, writing
$\psi_i(x)=x(i)$
for $i\in I$ and $x\in X$.   But $\phi_i[[\Cal F]]$ contains $R_i$, so has
a limit $(x_0(i),y_0(i))$ in $X_i\times Y_i$, for each $i$.
Accordingly $(x_0,y_0)$ is a limit of $\Cal F$ in $X\times Y$ (3A3Ic).
As $\Cal F$ is arbitrary, $R$ is usco-compact.

\medskip

{\bf (e)} For the moment, let $L$ be any subset of $X$.    Let $\Cal H$
be a family of open sets in $Y$ covering $R[L]$.   Let $\Cal G$ be the
family of those open sets $G\subseteq X$ such that $R[G]$ can be covered
by finitely many members of $\Cal H$.   Then $\Cal G$ covers $L$.
\Prf\ If $x\in L$, then $R[\{x\}]$ is a compact subset of
$R[L]\subseteq\bigcup\Cal H$, so there is a finite set
$\Cal H'\subseteq\Cal H$ covering
$R[\{x\}]$.   Now there is an open set $G$ containing $x$ such that
$R[G]\subseteq\bigcup\Cal H'$, by 422B.\ \Qed

\medskip

\quad{\bf (i)} If $L$ is compact, then there must be a finite subfamily
$\Cal G'$ of $\Cal G$ covering $L$;  now $R[L]\subseteq R[\bigcup\Cal
G']$
is covered by finitely many members of $\Cal H$.   As $\Cal H$ is
arbitrary, $R[L]$ is compact.

\medskip

\quad{\bf (ii)} If $L$ is Lindel\"of, then there must be a countable
subfamily $\Cal G'$ of $\Cal G$ covering $L$;  now
$R[L]\subseteq R[\bigcup\Cal G']$ is covered by countably many members
of $\Cal H$.   As $\Cal H$ is arbitrary, $R[L]$ is Lindel\"of.

\medskip

{\bf (f)} If $x\in X$ then $R[\{x\}]\subseteq Y$ is compact, so
$(SR)[\{x\}]=S[R[\{x\}]]$ is compact, by (e-i).   If $F\subseteq Z$ is
closed then $S^{-1}[F]\subseteq Y$ is closed so
$(SR)^{-1}[F]=R^{-1}[S^{-1}[F]]$ is closed.

\medskip

{\bf (g)(i)} Suppose that $R$ is usco-compact when regarded as a subset
of $X\times Y_0$.   Set $S=\{(y,y):y\in Y_0\}$;  by (c), $S$ is
usco-compact when regarded as a subset of $Y_0\times Y$, so by (f)
$R=SR$ is usco-compact when regarded as a subset of $X\times Y$.

\medskip

\quad{\bf (ii)} If $R$ is usco-compact when regarded as a subset of
$X\times Y$, and $x\in X$, then $R[\{x\}]$ is a subset of $Y_0$ which is
compact for the topology of $Y$, therefore for the subspace topology of
$Y_0$.   If $F\subseteq Y_0$ is closed for the subspace topology, it is
of the form $F'\cap Y_0$ for some closed $F'\subseteq Y$,
so $R^{-1}[F]=R^{-1}[F']$ is closed in $X$.   As $x$ and $F$ are
arbitrary, $R$ is usco-compact in $X\times Y_0$.

\medskip

{\bf (h)} Set
$S=\{(\alpha,\beta):\beta\le\alpha\in\NN\}$.   Then $S$ is usco-compact in
$\NN\times\NN$.   \Prf\ 
$S[\{\alpha\}]=\{\beta:\beta\le\alpha\}$ is a product of finite sets, so is
compact, for every $\alpha\in\NN$.
If $F\subseteq\NN$ is closed and $\sequencen{\alpha_n}$ is a convergent
sequence in $S^{-1}[F]$ with limit $\alpha\in\NN$, 
then for every $n\in\Bbb N$ there is a
$\beta_n\in F$ such that $\beta_n\le\alpha_n$.   For any $i\in\Bbb N$,
$\sup_{n\in\Bbb N}\beta_n(i)\le\sup_{n\in\Bbb N}\alpha_n(i)$ is finite, so
$\{\beta_n:n\in\Bbb N\}$ is relatively compact and $\sequencen{\beta_n}$
has a cluster point $\beta$ say;  now

\Centerline{$\beta(i)\le\limsup_{n\to\infty}\beta_n(i)
\le\limsup_{n\to\infty}\alpha_n(i)=\alpha(i)$}

\noindent for every $i\in\Bbb N$, so $\beta\le\alpha$.   Also, of course,
$\beta\in F$, so $\alpha\in S^{-1}[F]$.   As $\sequencen{\alpha_n}$ is
arbitrary, $S^{-1}[F]$ is closed;  as $F$ is arbitrary, $S$ is
usco-compact.\ \Qed

Now $R'=S\frsmallcirc R$ is usco-compact, by (f) above.
}%end of proof of 422D

\leader{422E}{}\cmmnt{ The following lemma is actually very important
in the structure theory of K-analytic spaces (see 422Yb).   It will be
useful to us in 423C below.

\medskip

\noindent}{\bf Lemma} Let $X$ and $Y$ be Hausdorff spaces, and
$R\subseteq X\times Y$ an usco-compact relation.   If $X$ is regular, so
is $R$\cmmnt{ (in its subspace topology)}.

\proof{ \Quer\ Suppose, if possible, otherwise;  that there are a closed
set $F\subseteq R$ and an $(x,y)\in R\setminus F$ which cannot be
separated from $F$ by open sets (in $R$).   If $G$, $H$ are open sets
containing $x$, $y$ respectively, then $R\cap(G\times H)$,
$R\setminus(\overline{G}\times\overline{H})$ are disjoint relatively
open sets in $R$, so the latter cannot include $F$;  that is,
$F\cap(\overline{G}\times\overline{H})\ne\emptyset$ whenever $G$, $H$
are open, $x\in G$ and $y\in H$.   Accordingly there is an ultrafilter
$\Cal F$ on $X\times Y$ such that
$F\cap(\overline{G}\times\overline{H})\in\Cal F$
whenever $G\subseteq X$ and $H\subseteq Y$ are open sets containing $x$,
$y$ respectively.   In this case $R\in\Cal F$, and
$\overline{G}\in\pi_1[[\Cal F]]$ for every open set $G$ containing $x$.
Because the topology of
$X$ is regular, every open set containing $x$ includes $\overline{G}$
for some smaller open set $G$ containing $x$, and belongs to
$\pi_1[[\Cal F]]$;  thus $\pi_1[[\Cal F]]\to x$ in $X$.   Because $R$ is
usco-compact, $\Cal F$
has a limit in $R$, which must be of the form $(x,y')$.   Because
$F\in\Cal F$ is closed (in $R$), $(x,y')\in F$.   But also
$y'\in\overline{H}$ for
every open set $H$ containing $y$, since $X\times\overline{H}$ is a
closed set belonging to $\Cal F$;  because the topology of $Y$ is
Hausdorff, $y'$ must be equal to $y$, and $(x,y)\in F$, which is
absurd.\ \Bang
}%end of proof of 422E


\leader{422F}{Definition}\cmmnt{ ({\smc Frol\'\i k 61})} Let $X$ be a
Hausdorff space.   Then $X$ is
{\bf K-analytic} if there is an usco-compact relation
$R\subseteq\NN\times X$ such that $R[\NN]=X$.

If $X$ is a Hausdorff space, we call a subset of $X$ {\bf K-analytic} if
it is a K-analytic space in its subspace topology.


\leader{422G}{Theorem} (a) Let $X$ be a Hausdorff space.   Then a subset
$A$ of $X$ is K-analytic iff there is an usco-compact relation
$R\subseteq\NN\times X$ such that $R[\NN]=A$.

(b) $\NN$ is K-analytic.

(c) Compact Hausdorff spaces are K-analytic.

(d) If $X$ and $Y$ are Hausdorff spaces and $R\subseteq X\times Y$ is an
usco-compact relation, then $R[A]$ is K-analytic whenever $A\subseteq X$
is K-analytic.  In particular, a Hausdorff continuous image of a
K-analytic Hausdorff space is K-analytic.

(e) A product of countably many K-analytic Hausdorff spaces is
K-analytic.

(f) A closed subset of a K-analytic Hausdorff space is K-analytic.

(g) A K-analytic Hausdorff space is Lindel\"of, so a regular K-analytic
Hausdorff space is completely regular.

\proof{{\bf (a)} $A$ is K-analytic iff there is an usco-compact relation
$R\subseteq\NN\times A$ with projection $A$.   But a subset of
$\NN\times X$ with projection $A$ is usco-compact in $\NN\times A$ iff
it is usco-compact in $\NN\times X$, by 422Dg.

\medskip

{\bf (b)} The identity function from $\NN$ to itself is an usco-compact
relation, by 422Dc.

\medskip

{\bf (c)} If $X$ is compact, then $R=\NN\times X$ is an
usco-compact relation (because $R[\{\phi\}]=X$ is compact for every
$\phi\in\NN$, while $R^{-1}[F]$ is either $\NN$ or $\emptyset$ for every
closed $F\subseteq X$), so $X=R[\NN]$ is K-analytic.

\medskip

{\bf (d)} By (a), there is an usco-compact relation
$S\subseteq\NN\times X$ such that $S[\NN]=A$.   Now
$R\frsmallcirc S\subseteq\NN\times Y$ is usco-compact, by 422Df, and
$RS[\NN]=R[A]$.

In particular, if $X$ itself is K-analytic and $f:X\to Y$ is a
continuous surjection, $f$ is an
usco-compact relation (422Dc), so $Y=f[X]$ is K-analytic.

\medskip

{\bf (e)} Let $\familyiI{X_i}$ be a countable family of K-analytic
Hausdorff spaces with product $X$.   If $I=\emptyset$ then
$X=\{\emptyset\}$ is compact,
therefore K-analytic.   Otherwise, choose for each $i\in I$ an
usco-compact relation $R_i\subseteq\NN\times X_i$ such that
$R_i[\NN]=X_i$.   Set

\Centerline{$R=\{(\phi,x):\phi\in(\NN)^I,\,x\in X,\,
(\phi(i),x(i))\in R_i$ for every $i\in I\}$.}

\noindent By 422Dd, $R$ is an usco-compact relation in
$(\NN)^I\times X$, and it is easy to see that $R[(\NN)^I]=X$.   But
$(\NN)^I\cong\BbbN^{\Bbb N\times I}$ is homeomorphic to $\NN$, because
$I$ is countable,
so we can identify $R$ with a relation in $\NN\times X$ which is still
usco-compact, and $X$ is K-analytic.

\medskip

{\bf (f)} Let $X$ be a K-analytic Hausdorff space and $F$ a closed
subset.   Let $R\subseteq\NN\times X$ be an usco-compact relation such
that $R[\NN]=X$.   Set $R'=R\cap(\NN\times F)$.   Then $R'$ is a closed
subset of $R$, so is
usco-compact (422Da).   By (a), $F=R'[\NN]$ is K-analytic.

\medskip

{\bf (g)} Let $X$ be a K-analytic Hausdorff space.   $\NN$ is
Lindel\"of (4A2Ub), and there is an usco-compact relation $R$ such
that $R[\NN]=X$, so that $X$ is Lindel\"of, by 422D(e-ii).   4A2H(b-i)
now tells us that if $X$ is regular it is completely regular.
}%end of proof of 422G

\leader{422H}{Theorem} (a) If $X$ is a Hausdorff space, then any
K-analytic subset of $X$ is Souslin-F in $X$.

(b) If $X$ is a K-analytic Hausdorff space, then a subset of $X$ is
K-analytic iff it is Souslin-F in $X$.

(c) For any Hausdorff space $X$, the family of K-analytic subsets of $X$
is closed under Souslin's operation.

\proof{{\bf (a)} If $A\subseteq X$ is K-analytic, there is an
usco-compact relation $R\subseteq\NN\times X$ such that $R[\NN]=A$, by
422Ga.   By 422Da, $R$ is a closed set;  so $A$ is Souslin-F by 421J.

\medskip

{\bf (b)} Now suppose that $X$ itself is K-analytic, and that
$A\subseteq X$ is Souslin-F in $X$.   Then there is a closed set
$R\subseteq\NN\times X$ such that $R[\NN]=A$ (421J, in the other
direction).   $\NN\times X$ is
K-analytic (422Gb, 422Ge), and $R$ is closed, therefore itself
K-analytic (422Gf);  so
its continuous image $A$ is
K-analytic, by 422Gd.

\medskip

{\bf (c)(i)} The first step is to show that the union of a sequence of
K-analytic subsets of $X$ is K-analytic.   \Prf\ Let $\sequencen{A_n}$
be a sequence of K-analytic sets, with union $A$.   For each 
$n\in\Bbb N$, let $R_n\subseteq\NN\times X$ be an usco-compact relation 
such that
$R_n[\NN]=A_n$.   In $(\Bbb N\times\NN)\times X$ let $R$ be the set

\Centerline{$\{((n,\phi),x):n\in\Bbb N,\,(\phi,x)\in R_n\}$.}

\noindent If $(n,\phi)\in\Bbb N\times\NN$, then
$R[\{(n,\phi)\}]=R_n[\{\phi\}]$ is compact;  if $F\subseteq X$ is
closed, then

\Centerline{$R^{-1}[F]=\{(n,\phi):n\in\Bbb N,\,\phi\in R_n^{-1}[F]\}$}

\noindent is closed in $\Bbb N\times\NN$.   So $R$ is
usco-compact, and of course

\Centerline{$R[\Bbb N\times\NN]=\bigcup_{n\in\Bbb N}R_n[\NN]=A.$}

\noindent As $\Bbb N\times\NN$ is K-analytic (in fact, homeomorphic to
$\NN$), $A$ is K-analytic.\ \Qed

\medskip

\quad{\bf (ii)} Now suppose that $\family{\sigma}{S^*}{A_{\sigma}}$ is a
Souslin scheme consisting of K-analytic sets with kernel $A$.   Then
$X'=\bigcup_{\sigma\in S^*}A_{\sigma}$ is K-analytic, by (i).   By (a),
every $A_{\sigma}$ is Souslin-F when regarded as a subset of $X'$.   But
since the family of Souslin-F subsets of $X'$ is closed under Souslin's
operation, by 421D, $A$ also is Souslin-F in $X'$.   By (b) of this
theorem, $A$ is K-analytic.   As $\family{\sigma}{S^*}{A_{\sigma}}$ is
arbitrary, we have the result.
}%end of proof of 422H


\leader{422I}{}\cmmnt{ It seems that for the measure-theoretic
results of \S432, at least, the following result (the `First
Separation Theorem') is not essential.   However I do not think it
possible to get a firm grasp on K-analytic and analytic spaces without
knowing some version of it, so I present it here.   It is most often
used through the forms in 422J and 422Xd below.

\wheader{422I}{4}{2}{2}{48pt}

\noindent}{\bf Lemma} Let $X$ be a Hausdorff space.   Let $\Cal E$ be a
family of subsets of $X$ such that (i) $\bigcup_{n\in\Bbb N}E_n$ and
$\bigcap_{n\in\Bbb N}E_n$ belong to
$\Cal E$ whenever $\sequencen{E_n}$ is a sequence in $\Cal E$
(ii) whenever $x$, $y$ are distinct points of $X$, there are disjoint
$E$, $F\in\Cal E$ such that $x\in\interior E$ and $y\in\interior F$.
Then whenever $A$, $B$ are disjoint non-empty K-analytic subsets of $X$,
there are disjoint $E$, $F\in\Cal E$ such that $A\subseteq E$ and
$B\subseteq F$.

\proof{{\bf (a)} We need to know that if $K$, $L$ are disjoint non-empty
compact subsets of $X$, there are disjoint $E$, $F\in\Cal E$ such that
$K\subseteq\interior E$ and $L\subseteq\interior F$.   \Prf\ For any
point $(x,y)\in K\times L$, we can find disjoint $E_{xy}$,
$F_{xy}\in\Cal E$ such that $x\in\interior E$ and $y\in\interior F$.
Because $L$ is compact and non-empty, there is for each $x\in K$ a
non-empty finite set
$I_x\subseteq L$ such that $L\subseteq\bigcup_{y\in I_x}\interior
F_{xy}$.   Set $E_x=\bigcap_{y\in I_x}E_{xy}$, $F_x=\bigcup_{y\in
I_x}F_{xy}$;  then $E_x$, $F_x$ are disjoint members of $\Cal E$,
$x\in\interior E_x$ and $L\subseteq\interior F_x$.   Because $K$ is
compact and not empty, there is a non-empty finite set $J\subseteq K$
such that
$K\subseteq\bigcup_{x\in J}\interior E_x$.   Set $E=\bigcup_{x\in
J}E_x$, $F=\bigcap_{x\in J}F_x$;  then $E$, $F\in\Cal E$, $E\cap
F=\emptyset$, $K\subseteq\interior E$ and $L\subseteq\interior F$, as
required.\ \Qed

\medskip

{\bf (b)} Let $Q$, $R\subseteq\NN\times X$ be usco-compact relations
such that $Q[\NN]=A$ and $R[\NN]=B$.    For each
$\sigma\in S=\bigcup_{n\in\Bbb N}\BbbN^n$, set

\Centerline{$I_{\sigma}=\{\phi:\sigma\subseteq\phi\in\NN\}$,
\quad$A_{\sigma}=Q[I_{\sigma}]$,
\quad$B_{\sigma}=R[I_{\sigma}]$,}

\noindent so that $A=A_{\emptyset}$ and $A_{\sigma}=\bigcup_{i\in\Bbb
N}A_{\sigma^{\smallfrown}\fraction{i}}$ for every $\sigma$.

\medskip

{\bf (c)} Write $T$ for the set of pairs

\Centerline{$\{(\sigma,\tau):\sigma,\,\tau\in S$ and there are
disjoint $E$, $F\in\Cal E$ such that $A_{\sigma}\subseteq E$ and
$B_{\tau}\subseteq F\}$.}

\noindent If $\sigma$, $\tau\in S$ are such that
$(\sigma^{\smallfrown}\fraction{i},\tau^{\smallfrown}\fraction{j})\in T$ for every $i$,
$j\in\Bbb N$, then $(\sigma,\tau)\in T$.   \Prf\ For each $i$, $j\in\Bbb
N$ take disjoint $E_{ij}$, $F_{ij}\in\Cal E$ such that

\Centerline{$A_{\sigma^{\smallfrown}\fraction{i}}\subseteq E_{ij}$,
\quad$B_{\tau^{\smallfrown}\fraction{j}}\subseteq F_{ij}$.}

\noindent Then $E=\bigcup_{i\in\Bbb N}\bigcap_{j\in\Bbb N}E_{ij}$,
$F=\bigcap_{i\in\Bbb N}\bigcup_{j\in\Bbb N}F_{ij}$ are disjoint and
belong to $\Cal E$, and $A_{\sigma}\subseteq E$, $B_{\tau}\subseteq
F$.   So $(\sigma,\tau)\in T$.\ \Qed

\medskip

{\bf (d)} \Quer\ Now suppose, if possible, that there are no disjoint
$E$, $F\in\Cal E$ such that $A\subseteq E$ and $B\subseteq F$;  that is,
that $(\emptyset,\emptyset)\notin T$.   By (c),
used repeatedly, we can find sequences
$\sequence{i}{\phi(i)}$, $\sequence{i}{\psi(i)}$ such that
$(\phi\restr n,\psi\restr n)\notin T$ for every $n\in\Bbb N$.   Set
$K=Q[\{\phi\}]$, $L=R[\{\psi\}]$.   These are compact (because $R$ is
usco-compact) and disjoint (because $K\subseteq A$ and $L\subseteq B$).
By (a), there are disjoint $E$, $F\in\Cal E$ such that
$K\subseteq\interior E$ and $L\subseteq\interior F$.

By 422B, there are open sets $U$, $V\subseteq\NN$ such that

\Centerline{$\phi\in U$,
\quad$Q[U]\subseteq\interior E$,
\quad$\psi\in V$,
\quad$R[V]\subseteq\interior F$.}

\noindent But now there is some $n\in\Bbb N$ such that
$I_{\phi\restr n}\subseteq U$ and $I_{\psi\restr n}\subseteq V$, in
which case

\Centerline{$A_{\phi\restr n}\subseteq E$,
\quad$B_{\psi\restr n}\subseteq F$,}

\noindent and $(\phi\restr n,\psi\restr n)\in T$, contrary to the choice
of $\phi$ and $\psi$.\ \Bang

This contradiction shows that the lemma is true.
}%end of proof of 422I

\leader{422J}{Corollary} Let $X$ be a Hausdorff space and $A$, $B$
disjoint K-analytic subsets of $X$.   Then there is a Borel set which
includes $A$ and is disjoint from $B$.

\proof{ Apply 422I with $\Cal E$ the Borel $\sigma$-algebra of $X$.
}%end of proof of 422J

\leader{*422K}{}\cmmnt{ I give the next step in the theory of
`constituents' begun in 421N-421Q.
%421N 421O 421P 421Q

\medskip

\noindent}{\bf Theorem} Let $X$ be a Hausdorff space.

(i) Suppose that $X$ is regular.   Let $A\subseteq X$ be a K-analytic
set.   Then there is a
non-decreasing family $\ofamily{\xi}{\omega_1}{E_{\xi}}$ of Borel sets
in $X$, with union $X\setminus A$, such that every Souslin-F subset of
$X$ disjoint from $A$ is included in some $E_{\xi}$.

(ii) Suppose that $X$ is regular.   Let $A\subseteq X$ be a Souslin-F
set.   Then there is a
non-decreasing family $\ofamily{\xi}{\omega_1}{E_{\xi}}$ of Borel sets
in $X$, with union $X\setminus A$, such that every K-analytic subset of
$X\setminus A$ is included in some $E_{\xi}$.

(iii) Let $A\subseteq X$ be a K-analytic set.   Then there is a
non-decreasing family $\ofamily{\xi}{\omega_1}{E_{\xi}}$ of Borel sets
in $X$, with union $X\setminus A$, such that every K-analytic subset of
$X\setminus A$ is included in some $E_{\xi}$.

%what if X is not regular?  Can we still separate disjoint K-analytic
%sets in this way?

\proof{{\bf (a)} The first two parts depend on the following fact:  if
$X$ is regular, $R\subseteq\NN\times X$ is usco-compact,
$\family{\sigma}{S^*}{F_{\sigma}}$
is a Souslin scheme consisting of closed sets with kernel $B$, and
$R[\NN]\cap B=\emptyset$, then for any $\phi$, $\psi\in\NN$ there is
an $n\ge 1$ such that $\overline{R[I_{\phi\restr n}]}
\cap\bigcap_{1\le i\le n}F_{\psi\restr i}$ is empty, where I write
$I_{\sigma}=\{\theta:\sigma\subseteq\theta\in\NN\}$ for
$\sigma\in S^*=\bigcup_{n\ge 1}\Bbb N^n$.   \Prf\ We know that
$K=R[\{\phi\}]$ is a compact set disjoint from the closed set
$\bigcap_{n\ge 1}F_{\psi\restr n}$.   So there is some $m\ge 1$ such
that $K\cap F=\emptyset$ where
$F=\bigcap_{1\le i\le m}F_{\psi\restr i}$.   Because $X$ is regular,
there are disjoint open sets $G$, $H\subseteq X$ such that
$K\subseteq G$ and $F\subseteq H$ (4A2F(h-ii)).   Now
$R^{-1}[X\setminus G]$ is a closed set not containing $\phi$, so there
is some $n$ such that $R[I_{\phi\restr n}]\subseteq G$.   Of course we
can take $n\ge m$, and in this case

\Centerline{$\overline{R[I_{\phi\restr n}]}
  \cap\bigcap_{1\le i\le n}F_{\psi\restr i}
\subseteq\overline{G}\cap F=\emptyset$,}

\noindent as required.\ \Qed

\medskip

{\bf (b)(i)} Suppose that $A\subseteq X$ is K-analytic.   Then there is
an usco-compact set $R\subseteq\NN\times X$ such that $R[\NN]=A$, and
$R$ is closed (422Da), so that $A$ is the kernel of the Souslin scheme
$\family{\sigma}{S^*}{\overline{R[I_{\sigma}]}}$ (421I).   For $x\in X$
set
$T_x=\{\sigma:\sigma\in S^*,\,x\in\overline{R[I_{\sigma}]}\}$,
as in 421Ng, and let
$r(T_x)<\omega_1$ be the rank of the tree $T_x$.   Then
$E_{\xi}=\{x:x\in X\setminus A,\,r(T_x)\le\xi\}$ is a Borel set for
every $\xi<\omega_1$, by 421Ob.   Now suppose that
$B\subseteq X\setminus A$ is a Souslin-F set.   Then it is the kernel of
a Souslin scheme $\family{\sigma}{S^*}{F_{\sigma}}$ consisting of closed
sets.   If $\phi$, $\psi\in\NN$ then by (a) above there is an $n\ge 1$
such that
$\overline{R[I_{\phi\restr n}]}\cap\bigcap_{1\le i\le n}F_{\psi\restr i}$
is empty.   By 421Q, there
must be some $\xi<\omega_1$ such that $B\subseteq E_{\xi}$.   So
$\ofamily{\xi}{\omega_1}{E_{\xi}}$ is a suitable family.

\medskip

\quad{\bf (ii)} The other part is almost the same.
Suppose that $A\subseteq X$ is Souslin-F.   Then
it is the kernel of a Souslin scheme $\family{\sigma}{S^*}{F_{\sigma}}$
consisting of closed sets.   For $x\in X$ set

\Centerline{$T_x=\bigcup_{n\ge 1}\{\sigma:\sigma\in\Bbb N^n,\,
  x\in\bigcap_{1\le i\le n}F_{\sigma\restr i}\}$,}

\noindent and let $r(T_x)<\omega_1$ be the rank of the tree $T_x$.
Then $E_{\xi}=\{x:x\in X\setminus A,\,r(T_x)\le\xi\}$ is a Borel set for
every $\xi<\omega_1$.   Now let $B\subseteq X\setminus A$ be a
K-analytic set.   There is an usco-compact set $R\subseteq\NN\times X$
such that $R[\NN]=B$, and $B$ is the kernel of the Souslin scheme
$\family{\sigma}{S^*}{\overline{R[I_{\sigma}]}}$.   If $\phi$,
$\psi\in\NN$ then by (a) above there is an $n\ge 1$ such that
$\bigcap_{1\le i\le n}F_{\psi\restr i}
\cap\overline{R[I_{\phi\restr n}]}$ is empty.   So there must be some
$\xi<\omega_1$ such that $B\subseteq E_{\xi}$.   Thus here again
$\ofamily{\xi}{\omega_1}{E_{\xi}}$ is a suitable family.

\medskip

{\bf (c)} If $X$ is not regular, we still have a version of the result
in (a), as follows:  if $R$, $S\subseteq\NN\times X$ and
$R[\NN]\cap S[\NN]=\emptyset$, then for any $\phi$, $\psi\in\NN$ there
is an $n\ge 1$ such that
$\overline{R[I{\phi\restr n}]}\cap\overline{S[I_{\psi\restr n}]}$ is
empty.   \Prf\ This time, $R[\{\phi\}]$ and $S[\{\psi\}]$ are disjoint
compact sets, so there are disjoint open sets $G$, $H$ with
$R[\{\phi\}]\subseteq G$ and $S[\{\psi\}]\subseteq H$ (4A2F(h-i)).  Now
$R[I_{\phi\restr n}]\subseteq G$ and $S[I_{\psi\restr n}]\subseteq H$
for all $n$ large enough.\ \Qed

Now the argument of (b-i), with $F_{\sigma}=S[I_{\sigma}]$, gives part
(iii).
}%end of proof of 422K

\exercises{\leader{422X}{Basic exercises (a)}
%\spheader 422Xa
Let $X$ and $Y$ be Hausdorff spaces and $X_0$ a closed subset of $X$.
Show that a relation $R\subseteq X_0\times Y$ is
usco-compact when regarded as a relation between $X_0$ and $Y$ iff it is
usco-compact when regarded as a relation between $X$ and $Y$.
%422D

\spheader 422Xb Show that a locally compact Hausdorff space is
K-analytic iff it is Lindel\"of iff it is $\sigma$-compact.
%422H

\sqheader 422Xc Prove 422Hc from first principles, without using 421D.
({\it Hint\/}:  if $\family{\sigma}{S^*}{R_{\sigma}}$ is a Souslin scheme
of usco-compact relations in $\NN\times X$,

\Centerline{$\{((\phi,\family{\sigma}{S^*}{\psi_{\sigma}}),x):
(\psi_{\phi\restr n},x)\in R_{\phi\restr n}$ for every $n\ge 1\}$}

\noindent is usco-compact in $(\NN\times(\NN)^{S^*})\times X$.)
%422H

\spheader 422Xd Let $X$ be a completely regular Hausdorff space and $A$,
$B$ disjoint K-analytic subsets of $X$.   Show that there is
a Baire set including $A$ and disjoint from $B$.
%422J

\spheader 422Xe Let $X$ be a K-analytic Hausdorff space.   (i) Show that
every Baire subset of $X$ is K-analytic.   \Hint{apply
136Xi to the family of K-analytic subsets of $X$.}
(ii) Show that if $X$ is regular, it is
perfectly normal iff it is hereditarily Lindel\"of iff every open subset
of $X$ is K-analytic.

\spheader 422Xf Let $X$ be a Hausdorff space in which every open set is
K-analytic.   Show that every Borel set is K-analytic.

\leader{422Y}{Further exercises (a)}
%\spheader 422Ya
Let $X$ be a completely regular Hausdorff space, and $\beta X$ its
Stone-\v Cech compactification.   Show that $X$ is K-analytic iff it is
a Souslin-F set in $\beta X$.
%422H

\spheader 422Yb Show that a Hausdorff space is K-analytic iff it is a
continuous image of a K$_{\sigma\delta}$ set in a compact Hausdorff
space, that is, a set expressible as
$\bigcap_{m\in\Bbb N}\bigcup_{n\in\Bbb N}K_{mn}$ where every $K_{mn}$ is
compact.   \Hint{Write $\Cal K^*$ for
the class of Hausdorff continuous images of K$_{\sigma\delta}$
subsets of compact Hausdorff spaces.   (i) Show that $\NN$ is a
K$_{\sigma\delta}$ set in $Y^{\Bbb N}$, where $Y$ is the one-point
compactification of $\Bbb N$.   (ii) Show that if $X$ is a compact
Hausdorff space and $R\subseteq\NN\times X$ is closed, then $R\in\Cal
K^*$.  (iii) Show that if $X$ is a compact Hausdorff space, then every
Souslin-F subset of $X$ belongs to $\Cal K^*$.   (iv) Show that if $X$
is a regular K-analytic Hausdorff space, then $X\in\Cal K^*$.   (v) Show
that if $X$ is any Hausdorff space and $R\subseteq\NN\times X$ is an
usco-compact relation, then $R\in\Cal K^*$.   See {\smc Jayne 76}.}
%422H

\spheader 422Yc Let $X$ be a normal space and $\Cal C$ the family of
countably compact closed subsets of $X$.   Let $A$, $B$ be disjoint sets
obtainable from $\Cal C$ by Souslin's operation.   (For instance, if $X$
itself is countably compact, $A$ and $B$ could be disjoint Souslin-F
sets.)   Show that there is a Borel set including $A$ and disjoint from
$B$.   \Hint{adapt the proof of 422I.}
%422I

\spheader 422Yd Let $X$ be a Hausdorff space and $\sequencen{A_n}$ a
sequence of K-analytic subsets of $X$ such that
$\bigcap_{n\in\Bbb N}A_n=\emptyset$.   Show that there is a sequence
$\sequencen{E_n}$ of
Borel sets such that $A_n\subseteq E_n$ for every $n\in\Bbb N$ and
$\bigcap_{n\in\Bbb N}E_n=\emptyset$.   \Hint{for each $n\in\Bbb N$
choose an usco-compact $R_n\subseteq\NN\times X$ with projection $A_n$.
Consider the set $T=\{\sequencen{\sigma_n}:\exists$ Borel
$E_n,\,R_n[I_{\sigma_n}]\subseteq E_n\Forall n$,
$\bigcap_{n\in\Bbb N}E_n=\emptyset\}$.}
%422I

\spheader 422Ye Explain how to prove 422J from 421Q, without using 422I.
%422J

\spheader 422Yf Let $X$ be a set and $\frak S$, $\frak T$ two Hausdorff
topologies on $X$ such that $\frak S\subseteq\frak T$ and $(X,\frak T)$
is K-analytic.   Show that $\frak S$ and $\frak T$ yield the same
K-analytic subspaces of $X$.

\spheader 422Yg\dvAnew{2013} Let $X$ be a Hausdorff space, 
$\Cal K$ the family of K-analytic subsets of $X$, $Y$ a set and $\Cal H$ a
family of subsets of $Y$ containing $\emptyset$.   
Show that $R[X]\in\Cal S(\Cal H)$
for every $R\in\Cal S(\{K\times H:K\in\Cal K$, $H\in\Cal H\})$.
}%end of exercises

\endnotes{
\Notesheader{422}
In a sense, this section starts at the deep end of its topic.
`Descriptive set theory' originally developed in the context of the real
line and associated spaces, and this remains the centre of the subject.
But it turns out that some of the same arguments can be used in much
more general contexts, and in particular greatly illuminate the theory
of Radon measures on Hausdorff spaces.   I find that a helpful way to
look at K-analytic spaces is to regard them as a common generalization
of compact Hausdorff spaces and Souslin-F subsets of $\Bbb R$;  if you
like, any theorem which is true of both these classes has a fair chance
of being true of all K-analytic spaces.   In the next section we shall
come to the special properties of the more restricted class of
`analytic' spaces, which are much closer to the separable metric spaces
of the original theory.

The phrase `usco-compact' is neither elegant nor transparent, but is
adequately established and (in view of the frequency with which it is
needed) seems preferable to less concise alternatives.   If we think of
a relation $R\subseteq X\times Y$ as a function $x\mapsto R[\{x\}]$ from
$x$ to $\Cal PY$, then an usco-compact relation is one which takes
compact values and is `upper semi-continuous' in the sense that
$\{x:R[\{x\}]\subseteq H\}$ is open for every open set $H\subseteq Y$;
just as a real-valued function is upper semi-continuous if
$\{x:f(x)<\alpha\}$ is open for every $\alpha$.

This is not supposed to be a book on general topology, and in my account
of the topological properties of K-analytic spaces I have concentrated
on facts which are useful when proving that spaces are K-analytic, on
the assumption that these will be valuable when we seek to apply
the results of \S432 below.   Other properties are mentioned only when
they are relevant to the measure-theoretic results which are my real
concern, and readers already acquainted with this area may be startled
by some of my omissions.   For a proper treatment of the subject, I
refer you to {\smc Rogers 80}.   As usual, however, I take technical
details seriously in
the material I do cover.   I hope you will not find that such results as
422Dg and 422Ga try your patience too far.   I think a moment's thought
will persuade you that it
is of the highest importance that K-analyticity (like compactness) is an
intrinsic property.   In contrast, the property of being `Souslin-F',
like the property of being closed, depends on the surrounding space.   A
completely regular Hausdorff space is compact iff it must be a closed
set in any surrounding Hausdorff space iff it is closed in its
Stone-\v Cech compactification;  and it is K-analytic iff it must be a
Souslin-F set in any surrounding Hausdorff space iff it is a Souslin-F
set in its Stone-\v Cech compactification (422Ya).

For regular spaces, 422K gives us another version of the First
Separation Theorem.   But this one is simultaneously more restricted in
its scope (it does not seem to have applications to Baire
$\sigma$-algebras, for instance) and very much more powerful in its
application.   When all Borel sets are Souslin-F, as in the next
section, it tells us something very important about the cofinal
structure of the Souslin-F subsets of the complement of a K-analytic
set.
}%end of notes

\discrpage


