\frfilename{mt511.tex}
\versiondate{10.10.13}
\copyrightdate{2003}

\def\nw{\mathop{\text{nw}}}
\def\triplepc#1#2#3{(#1,#2,\hbox{$<$}#3)}
\def\bu{\mathop{\text{bu}}}

\def\chaptername{Cardinal functions}
\def\sectionname{Definitions}

\newsection{511}

A large proportion of the ideas of this volume will be expressed in
terms of cardinal numbers associated with the structures of measure
theory.   For any measure space $(X,\Sigma,\mu)$ we have, at least, the
structures $(X,\Sigma)$, $(X,\Sigma,\Cal N(\mu))$ (where $\Cal N(\mu)$
is the null ideal of $\mu$) and the measure algebra
$\frak A=\Sigma/\Sigma\cap\Cal N(\mu)$;  each of these types of
structure has
a family of cardinal functions associated with it, starting from the
obvious ones $\#(X)$, $\#(\Sigma)$ and $\#(\frak A)$.   For the measure
algebra $\frak A$, we quickly find that we have cardinals naturally
associated with its Boolean structure and others naturally associated
with the topological structure of its Stone space;  of course the most
important ones are those which can be described in both languages.   The
actual measure $\mu:\Sigma\to[0,\infty]$, and its daughter
$\bar\mu:\frak A\to[0,\infty]$, will be less conspicuous here;   for most
of the questions addressed in this volume, replacing a measure by
another with the same measurable sets and the same negligible sets will
make no difference.

In this section I list the definitions on which the rest of the chapter
depends, with a handful of elementary results to give you practice with
the definitions.

\leader{511A}{Pre-ordered sets}\cmmnt{ When we come to the theory of
forcing in Chapter 55, there will be technical advantages in using a
generalization of the concept of `partial order'.}   A {\bf pre-ordered
set} is a set $P$ together with a relation $\le$ on $P$ such that

\inset{if $p\le q$ and $q\le r$ then $p\le r$,

$p\le p$ for every $p\in P$\dvro{.}{;}}

\noindent\cmmnt{that is, $\le$ is transitive and reflexive but need not
be antisymmetric.  }As with partial orders,
I will write $p\ge q$ to mean $q\le p$;
$[p,q]=\{r:p\le r$ and $r\le q\}$;  $\coint{p,\infty}=\{q:p\le q\}$,
$\ocint{-\infty,p}=\{q:q\le p\}$.   An {\bf upper} (resp.\ {\bf lower})
{\bf bound} for a set $A\subseteq P$ will be a $p\in P$ such that $q\le p$
(resp.\ $p\le q$) for every $q\in A$.   If $(Q,\le)$ is another pre-ordered
set, I will say that $f:P\to Q$ is {\bf order-preserving} if
$f(p)\le f(p')$ whenever $p\le p'$ in $P$.   If $\familyiI{(P_i,\le_i)}$ is
a family of pre-ordered sets, their {\bf product} is the pre-ordered set
$(P,\le)$ where $P=\prod_{i\in I}P_i$ and, for $p$, $q\in P$,
$p\le q$ iff $p(i)\le_iq(i)$ for every $i\in I$\cmmnt{ (cf.\ 315C)}.

If $(P,\le)$ is a pre-ordered set, we have an equivalence relation $\sim$
on $P$ defined by saying that $p\sim q$ if $p\le q$ and $q\le p$.   Now we
have a canonical partial order on the set $\tilde P$ of equivalence
classes defined by saying that $p^{\ssbullet}\le q^{\ssbullet}$ iff
$p\le q$.   \cmmnt{For all ordinary purposes, $(P,\le)$ and
$(\tilde P,\le)$ carry the same structural information, and the move to the
true partial order is natural and convenient.   It occasionally happens
(see 512Ee below, for instance, and also the theory of iterated forcing
in {\smc Kunen 80}, chap.\ VIII) that it is
helpful to have a language which enables us to
dispense with this step, thereby simplifying some basic definitions.
However the extra generality leads to no new ideas, and I expect that
most readers will prefer to do nearly all their thinking in
the context of partially ordered sets.}

\leader{511B}{Definitions} Let $(P,\le)$ be any pre-ordered set.

\spheader 511Ba A subset $Q$ of $P$ is {\bf cofinal} with $P$ if for
every $p\in P$ there is a $q\in Q$ such that $p\le q$.   The
{\bf cofinality} of $P$, $\cf P$, is the least cardinal of any cofinal
subset of $P$.

\spheader 511Bb The {\bf additivity} of $P$, $\add P$, is the least
cardinal of any subset of $P$ with no upper bound in $P$.   If there is
no such set, write $\add P=\infty$.

\spheader 511Bc A subset $Q$ of $P$ is {\bf coinitial} with $P$ if for
every $p\in P$ there is a $q\in Q$ such that $q\le p$.   The
{\bf coinitiality} of $P$, $\ci P$, is the least cardinal of any
coinitial subset of $P$.

\spheader 511Bd Two elements $p$, $p'$ of $P$ are
{\bf compatible upwards} if
$\coint{p,\infty}\cap\coint{p',\infty}\ne\emptyset$\cmmnt{, that is, if
$\{p,p'\}$ has an upper bound in $P$};  otherwise they are
{\bf incompatible upwards}.   A subset $A$ of $P$ is an
{\bf up-antichain} if no two
distinct elements of $A$ are compatible upwards.   The {\bf
upwards cellularity} of $P$ is
$c^{\uparrow}(P)=\sup\{\#(A):A\subseteq P$ is an up-antichain in $P\}$;
the {\bf upwards saturation} of $P$, $\sat^{\uparrow}(P)$, is the least
cardinal $\kappa$ such that there is no up-antichain in $P$ of size
$\kappa$.   $P$ is called {\bf upwards-ccc} if it has no uncountable
up-antichain\cmmnt{, that is, $c^{\uparrow}(P)\le\omega$, that is,
$\sat^{\uparrow}(P)\le\omega_1$}.

%Note:  Juhasz (2001) uses  $\hat c$  for $\sat$

\spheader 511Be Two elements $p$, $p'$ of $P$ are
{\bf compatible downwards} if
$\ocint{-\infty,p}\cap\ocint{-\infty,p'}\ne\emptyset$\cmmnt{, that is, if
$\{p,p'\}$ has a lower bound in $P$};  otherwise they are
{\bf incompatible downwards}.   A subset $A$ of $P$ is a
{\bf down-antichain} if no two
distinct elements of $A$ are compatible downwards.
The {\bf downwards cellularity} of $P$ is
$\cdownarrow(P)=\sup\{\#(A):A\subseteq P$ is a down-antichain in
$P\}$;  the {\bf downwards saturation} of $P$, $\sat^{\downarrow}(P)$,
is the least $\kappa$ such that there is no down-antichain in
$P$ with cardinal $\kappa$.   $P$ is called {\bf downwards-ccc} if it has
no uncountable down-antichain\cmmnt{, that is,
$\cdownarrow(P)\le\omega$, that is,
$\sat^{\downarrow}(P)\le\omega_1$}.

\spheader 511Bf If $\kappa$ is a cardinal, a subset $A$ of $P$ is
{\bf upwards-$\hbox{$<$}\kappa$-linked} in $P$ if every subset of $A$ of
cardinal less than $\kappa$ is bounded above in $P$.   The {\bf
upwards $\hbox{$<$}\kappa$-linking number} of $P$,
$\link_{<\kappa}^{\uparrow}(P)$, is the smallest
cardinal of any cover of $P$ by upwards-$\hbox{$<$}\kappa$-linked sets.

A subset $A$ of $P$ is {\bf upwards-$\kappa$-linked} in $P$ if it is
upwards-$\hbox{$<$}\kappa^+$-linked\cmmnt{, that is, every member of
$[A]^{\le\kappa}$ is bounded above in $P$}.   The {\bf
upwards $\kappa$-linking number} of $P$,
$\link_{\kappa}^{\uparrow}(P)=\link_{<\kappa^+}^{\uparrow}(P)$, is the
smallest cardinal of any cover of $P$ by upwards-$\kappa$-linked sets.

Similarly, a subset $A$ of $P$ is
{\bf downwards-$\hbox{$<$}\kappa$-linked} if every member of
$[A]^{<\kappa}$ has a lower bound in $P$, and
{\bf downwards-$\kappa$-linked} if it is
downwards-$\hbox{$<$}\kappa^+$-linked;  the
{\bf downwards $\hbox{$<$}\kappa$-linking number} of $P$,
$\link_{<\kappa}^{\downarrow}(P)$, is the smallest
cardinal of any cover of $P$ by
downwards-$\hbox{$<$}\kappa$-linked sets,
and $\link_{\kappa}^{\downarrow}(P)=\link_{<\kappa^+}^{\downarrow}(P)$.

\spheader 511Bg The most important cases of (f) above are $\kappa=2$ and
$\kappa=\omega$.   A subset $A$ of $P$ is {\bf upwards-linked} if any
two members of $A$ are compatible upwards in $P$, and
{\bf upwards-centered} if it is upwards-$\hbox{$<$}\omega$-linked\cmmnt{,
that is, any finite subset of $A$ has
an upper bound in $P$}.   The {\bf upwards linking number} of $P$,
$\link^{\uparrow}(P)=\link_2^{\uparrow}(P)$, is the least cardinal of any
cover of $P$ by upwards-linked sets, and the {\bf upwards centering
number} of $P$, $\duparrow(P)=\link_{<\omega}^{\uparrow}(P)$,
is the least cardinal of any cover of $P$ by upwards-centered sets.

\allowmorestretch{500}{\cmmnt{Similarly, }$A\subseteq P$ is
{\bf downwards-linked} if any two members
of $A$ are compatible downwards in $P$, and {\bf down\-wards-centered} if
any finite subset of $A$ has a lower bound in $P$;  the
{\bf downwards linking number} of $P$ is
$\link^{\downarrow}(P)=\link_2^{\downarrow}(P)$, and the
{\bf downwards centering number} of
$P$ is $\ddownarrow(P)=\link_{<\omega}^{\downarrow}(P)$.
}

If $\link^{\uparrow}(P)\le\omega$ (resp.\
$\link^{\downarrow}(P)\le\omega$) we say that $P$ is
{\bf $\sigma$-linked upwards} (resp.\ {\bf downwards}).
If $\duparrow(P)\le\omega$ (resp.\
$\ddownarrow(P)\le\omega$) we say that $P$ is
{\bf $\sigma$-centered upwards} (resp.\ {\bf downwards}).

\spheader 511Bh The {\bf upwards Martin number} $\frak m^{\uparrow}(P)$
of $P$ is the smallest cardinal of any family $\Cal Q$ of cofinal
subsets of $P$ such that there is some $p\in P$ such that no
upwards-linked subset of $P$ containing $p$ meets
every member of $\Cal Q$;  if there is no such family $\Cal Q$, write
$\frak m^{\uparrow}(P)=\infty$.

Similarly, the {\bf downwards Martin number} $\frak m^{\downarrow}(P)$
of $P$ is the smallest cardinal of any family $\Cal Q$ of coinitial
subsets of $P$ such that there is some $p\in P$ such that no
downwards-linked subset of $P$ containing $p$ meets every member
of $\Cal Q$, or $\infty$ if there is no such $\Cal Q$.

\spheader 511Bi A {\bf Freese-Nation function} on $P$ is a function
$f:P\to\Cal PP$ such that whenever $p\le q$ in $P$ then
$[p,q]\cap f(p)\cap f(q)$ is non-empty.   The {\bf Freese-Nation number}
of $P$, $\FN(P)$, is the least $\kappa$ such that there is a
Freese-Nation function $f:P\to[P]^{<\kappa}$.
The {\bf regular Freese-Nation number}
of $P$, $\FN^*(P)$, is the least regular infinite $\kappa$
such that there is a
Freese-Nation function $f:P\to[P]^{<\kappa}$.
%$P$ has the
%{\bf Freese-Nation property} if $\FN(P)\le\omega$, and the
%{\bf weak Freese-Nation property} if $\FN(P)\le\omega_1$.
If $Q$ is a subset of $P$, the {\bf Freese-Nation index} of $Q$ in $P$
is the least cardinal $\kappa$ such that
$\cf(Q\cap\ocint{-\infty,p})<\kappa$
and $\ci(Q\cap\coint{p,\infty})<\kappa$ for every $p\in P$.

\spheader 511Bj The {\bf (principal) bursting number} $\bu P$ of $P$ is
the least cardinal $\kappa$ such that there is a cofinal subset $Q$ of
$P$ such that

\Centerline{$\#(\{q:q\in Q$, $q\le p$, $p\not\le q\})<\kappa$}

\noindent for every $p\in P$.
%we want  bu \le cf  when  P  has top element

\spheader 511Bk\cmmnt{ It will be convenient to have a phrase for the
following phenomenon.}   I will say that $P$ is {\bf separative upwards} if
whenever $p$, $q\in P$ and $p\not\le q$ there is a $q'\ge q$ which is
incompatible upwards with $p$.   \cmmnt{Similarly, of course,} $P$ is
{\bf separative downwards} if whenever $p$,
$q\in P$ and $p\not\ge q$ there is a $q'\le q$ which is incompatible
downwards with $p$.

\cmmnt{
\leader{511C}{On the symbol $\infty$} I note that in the
definitions above I have introduced expressions of the form
`$\add P=\infty$'.   The `$\infty$' here must be rigorously
distinguished from the `$\infty$' of ordinary measure theory, which can
be regarded as a top point added to the set of real numbers.   The
`$\infty$' of 511B is rather a top point added to the class of ordinals.
But it is convenient, and fairly safe, to use formulae like
`$\add P\le\add Q$' on the understanding that
$\add P\le\infty$ for every pre-ordered set $P$, while
$\infty\le\add Q$ only when $\add Q=\infty$.   Of course we have to be
careful to distinguish between `$\add P<\infty$' (meaning that
there is a subset of $P$ with no upper bound in $P$) and `$\add P$ is
finite' (meaning that $\add P<\omega$).
}%end of comment

\leader{511D}{Definitions} Let $\frak A$ be a Boolean algebra.   I write
$\frak A^+$ for the set $\frak A\setminus\{0\}$ of non-zero elements of
$\frak A$ and
$\frak A^-$ for $\frak A\setminus\{1\}$, so that the partially ordered
sets $(\frak A^-,\Bsubseteqshort)$ and $(\frak A^+,\Bsupseteqshort)$ are
isomorphic.

\spheader 511Da The {\bf Maharam type} $\tau(\frak A)$ of $\frak A$ is
the smallest cardinal of any subset\cmmnt{ $B$} of $\frak A$ which
$\tau$-generates $\frak A$\cmmnt{ in the sense that the order-closed
subalgebra of $\frak A$ including $B$ is $\frak A$ itself}.
\cmmnt{(See Chapter 33.)}

\spheader 511Db The {\bf cellularity} of $\frak A$ is

\Centerline{$c(\frak A)
=c^{\uparrow}(\frak A^-)
=\cdownarrow(\frak A^+)
=\sup\{\#(C):C\subseteq\frak A^+$ is disjoint$\}$.}

\noindent The {\bf saturation} of $\frak A$ is

\Centerline{$\sat(\frak A)
=\sat^{\uparrow}(\frak A^-)
=\sat^{\downarrow}(\frak A^+)
=\sup\{\#(C)^+:C\subseteq\frak A^+$ is disjoint$\}$\dvro{.}{,}}

\cmmnt{\noindent that is, the smallest cardinal $\kappa$ such that
there is no disjoint family of size $\kappa$ in $\frak A^+$.}

%following Koppelberg

\spheader 511Dc The {\bf $\pi$-weight} or {\bf density}
%Koppelberg 2.4.8
$\pi(\frak A)$ of $\frak A$ is
$\cf\frak A^-=\ci\frak A^+$\cmmnt{, that is, the smallest cardinal of
any order-dense subset of $\frak A$}.

\spheader 511Dd Let $\kappa$ be a cardinal.   A subset $A$ of
$\frak A^+$ is {\bf $\hbox{$<$}\kappa$-linked} if it is
downwards-$\hbox{$<$}\kappa$-linked in $\frak A^+$\cmmnt{, that is, no
$B\in[A]^{<\kappa}$ has infimum $0$}, and {\bf $\kappa$-linked} if it is
$\hbox{$<$}\kappa^+$-linked\cmmnt{, that is, every
$B\in[A]^{\le\kappa}$ has a non-zero lower bound}.
The {\bf $\hbox{$<$}\kappa$-linking number}
$\link_{<\kappa}(\frak A)$ of $\frak A$ is
$\link_{<\kappa}^{\downarrow}(\frak A^+)$,
the least cardinal of any family of
$\hbox{$<$}\kappa$-linked sets covering $\frak A^+$;  and the
{\bf $\kappa$-linking number} $\link_{\kappa}(\frak A)$ of $\frak A$ is
$\link_{<\kappa^+}(\frak A)$\cmmnt{, that is, the least cardinal of
any cover of $\frak A^+$ by $\kappa$-linked sets}.

\spheader 511De\cmmnt{ As in 511Bg, I say that} $A\subseteq\frak A^+$ is
{\bf linked} if no two members of $A$ are disjoint;  the
{\bf linking number} of $\frak A$ is $\link(\frak A)=\link_2(\frak A)$,
the least cardinal of any cover of $\frak A^+$ by linked sets.
\cmmnt{Similarly, }$A\subseteq\frak A^+$ is {\bf centered} if
$\inf I\ne 0$ for any finite $I\subseteq A$\cmmnt{;  that
is, if $A$ is downwards-centered in $\frak A^+$}.   The {\bf centering
number}
%my invention
$d(\frak A)$ of $\frak A$ is
$\duparrow(\frak A^-)=\ddownarrow(\frak A^+)$\cmmnt{, that is, the
smallest cardinal of any cover of $\frak A^+$ by centered sets}.
$\frak A$ is {\bf $\sigma$-$m$-linked} if
$\link_m(\frak A)\le\omega$;  \cmmnt{in particular,} it is
{\bf $\sigma$-linked} iff $\link(\frak A)\le\omega$.
$\frak A$ is {\bf $\sigma$-centered} if $d(\frak A)\le\omega$.

\spheader 511Df If $\kappa$ is any cardinal, $\frak A$ is {\bf weakly
$(\kappa,\infty)$-distributive} if whenever
$\ofamily{\xi}{\kappa}{A_{\xi}}$ is a family of partitions of unity in
$\frak A$, there is a partition $B$ of unity such that
$\{a:a\in A_{\xi}$, $a\Bcap b\ne 0\}$ is finite for every $b\in B$ and
$\xi<\kappa$.    Now the {\bf weak distributivity} $\wdistr(\frak A)$ of
$\frak A$ is the least cardinal $\kappa$ such that $\frak A$ is not
weakly $(\kappa,\infty)$-distributive.   (If there is no such cardinal,
write $\wdistr(\frak A)=\infty$.)

\spheader 511Dg The {\bf Martin number} %my invention
  %maybe `Baire number' or `Nov\'ak number'
$\frak m(\frak A)$ of $\frak A$ is the downwards Martin number of
$\frak A^+$\cmmnt{, that is, the smallest cardinal of any
family $\Cal B$ of coinitial subsets of $\frak A^+$ for
which there is some $a\in\frak A^+$ such that no
linked subset of $\frak A$ containing $a$ meets every member of
$\Cal B$;   or $\infty$ if there is no such $\Cal B$}.

\spheader 511Dh The {\bf Freese-Nation number} of $\frak A$,
$\FN(\frak A)$, is the Freese-Nation number of the partially ordered set
$(\frak A,\Bsubseteqshort)$.
%\cmmnt{;  we say that $\frak A$
%has the
%(weak) Freese-Nation property if it has the property for this partial
%ordering}.
The {\bf regular Freese-Nation number} $\FN^*(\frak A)$
of $\frak A$ is the regular Freese-Nation number of
$(\frak A,\Bsubseteqshort)$\cmmnt{, that is, the smallest regular infinite
cardinal greater than or equal to $\FN(\frak A)$}.

\spheader 511Di\dvArevised{2014} If
$\kappa$ is a cardinal, a {\bf tight $\kappa$-filtration} of $\frak A$
is a family $\ofamily{\xi}{\zeta}{a_{\xi}}$ in $\frak A$, where $\zeta$
is an ordinal, such that, writing
$\frak A_{\alpha}$ for the subalgebra of $\frak A$ generated by
$\{a_{\xi}:\xi<\alpha\}$,
($\alpha$) $\frak A_{\zeta}=\frak A$ ($\beta$) for every $\alpha<\zeta$,
the Freese-Nation index of $\frak A_{\alpha}$ in $\frak A$ is
at most $\kappa$.
If $\frak A$ has a tight $\kappa$-filtration, I will say that it is {\bf
tightly $\kappa$-filtered}.

\leader{511E}{Precalibers (a)} Let $(P,\le)$ be a pre-ordered set.

\medskip

\quad{\bf (i)} I will say that $\triplepc{\kappa}{\lambda}{\theta}$ is
an {\bf upwards precaliber triple} of $P$ if $\kappa$, $\lambda$ and
$\theta$
are cardinals, and whenever $\ofamily{\xi}{\kappa}{p_{\xi}}$ is a family
in $P$ then there is a set
$\Gamma\in[\kappa]^{\lambda}$ such that $\{p_{\xi}:\xi\in I\}$ has an
upper bound in $P$ for every $I\in[\Gamma]^{<\theta}$.

\cmmnt{Similarly, }$\triplepc{\kappa}{\lambda}{\theta}$ is a
{\bf downwards precaliber
triple} of $P$ if $\kappa$, $\lambda$ and $\theta$ are cardinals and
whenever $\ofamily{\xi}{\kappa}{p_{\xi}}$ is a family in $P$ then there
is a set
$\Gamma\in[\kappa]^{\lambda}$ such that $\{p_{\xi}:\xi\in I\}$ has a
lower bound in $P$ for every $I\in[\Gamma]^{<\theta}$.

\medskip

\quad{\bf (ii)} An {\bf upwards precaliber pair} of $P$ is a pair
$(\kappa,\lambda)$ of cardinals such that
$\triplepc{\kappa}{\lambda}{\omega}$ is
an upwards precaliber triple of $P$\cmmnt{, that is, whenever
$\ofamily{\xi}{\kappa}{p_{\xi}}$ is a family in $P$ there is a
$\Gamma\in[\kappa]^{\lambda}$ such that $\{p_{\xi}:\xi\in\Gamma\}$ is
upwards-centered in $P$}.

A {\bf downwards precaliber pair} of $P$ is a pair $(\kappa,\lambda)$ of
cardinals such that $\triplepc{\kappa}{\lambda}{\omega}$ is a downwards
precaliber triple of $P$.

\medskip

\quad{\bf (iii)} An {\bf up-} (resp.\ {\bf down-}) {\bf precaliber} of
$P$ is a cardinal $\kappa$ such that $(\kappa,\kappa)$ is an upwards
(resp.\ downwards) precaliber pair of $P$.

\spheader 511Eb Let $(X,\frak T)$ be a topological space.   Then
$\triplepc{\kappa}{\lambda}{\theta}$ is a
{\bf precaliber triple} of $X$ if it is
a downwards precaliber triple of $\frak T\setminus\{\emptyset\}$;
$(\kappa,\lambda)$ is a {\bf precaliber pair} of $X$ if it is a
downwards precaliber pair of
$\frak T\setminus\{\emptyset\}$;  and $\kappa$ is a {\bf precaliber} of
$X$ if it is a down-precaliber of $\frak T\setminus\{\emptyset\}$.

\spheader 511Ec Let $\frak A$ be a Boolean algebra.   Then
$\triplepc{\kappa}{\lambda}{\theta}$ is a
{\bf precaliber triple} of $\frak A$ if
it is a downwards precaliber triple of $\frak A^+$;
$(\kappa,\lambda)$ is a {\bf precaliber pair} of $\frak A$ if it is a
downwards precaliber pair of $\frak A^+$;  and $\kappa$ is a
{\bf precaliber} of $\frak A$ if it is a down-precaliber of
$\frak A^+$.

\spheader 511Ed If $(\frak A,\bar\mu)$ is a measure algebra, then
$\triplepc{\kappa}{\lambda}{\theta}$ is a {\bf measure-precaliber
triple} of $(\frak A,\bar\mu)$ if whenever
$\ofamily{\xi}{\kappa}{a_{\xi}}$ is a family in
$\frak A$ such that $\inf_{\xi<\kappa}\bar\mu a_{\xi}>0$, then there is
a $\Gamma\in[\kappa]^{\lambda}$ such that $\{a_{\xi}:\xi\in I\}$ has a
non-zero lower bound for every $I\in[\Gamma]^{<\theta}$.
Now $(\kappa,\lambda)$ is a {\bf measure-precaliber pair} of
$(\frak A,\bar\mu)$ if $\triplepc{\kappa}{\lambda}{\omega}$ is a
measure-precaliber triple, and $\kappa$ is a {\bf measure-precaliber} of
$(\frak A,\bar\mu)$ if $(\kappa,\kappa)$ is a measure-precaliber pair.

\spheader 511Ee In this context, I will say that
$(\kappa,\lambda,\theta)$ is a precaliber triple (in any sense) if
$\triplepc{\kappa}{\lambda}{\theta^+}$ is a precaliber triple as defined
above;  and similarly for measure-precaliber triples.

\spheader 511Ef I will say that one of the structures here satisfies
{\bf Knaster's condition} if it has $(\omega_1,\omega_1,2)$ as a
precaliber triple\cmmnt{, that is, if every uncountable set has an
uncountable linked subset}.  (For pre-ordered sets I will speak of
`Knaster's condition upwards' or `Knaster's condition downwards'.)

\leader{511F}{Definitions} Let $X$ be a set and $\Cal I$ an ideal of
subsets of $X$.

\spheader 511Fa Taking $\Cal I$ to be partially ordered by
$\subseteq$, we can speak of $\add\Cal I$ and $\cf\Cal I$\cmmnt{ in
the sense of 511B}.   $\Cal I$ is called {\bf $\kappa$-additive} or
{\bf $\kappa$-complete} if $\kappa\le\add\Cal I$\cmmnt{, that is, if
$\bigcup\Cal E\in\Cal I$ for every $\Cal E\in[\Cal I]^{<\kappa}$}.

\cmmnt{In addition we have three other cardinals which will be
important to us.}

\spheader 511Fb The {\bf uniformity} of $\Cal I$ is

\Centerline{$\non\Cal I=\min\{\#(A):A\subseteq X$, $A\notin\Cal I\}$,}

\noindent or $\infty$ if there is no such set $A$.   \cmmnt{(Note the
hidden variable $X$ in this notation;  if any confusion seems possible,
I will write $\non(X,\Cal I)$.   Many authors prefer
unif $\Cal I$.)}

\spheader 511Fc The {\bf shrinking number} of
$\Cal I$, $\shr\Cal I$, is the smallest cardinal $\kappa$ such that
whenever $A\in\Cal PX\setminus\Cal I$ there is a
$B\in[A]^{\le\kappa}\setminus\Cal I$.
\cmmnt{(Again, we need to know $X$ as well as $\Cal I$ to determine
$\shr\Cal I$, and if necessary I will write $\shr(X,\Cal I)$.)}
The {\bf augmented shrinking number} $\shr^+(\Cal I)$ is the smallest
$\kappa$ such that
whenever $A\in\Cal PX\setminus\Cal I$ there is a
$B\in[A]^{<\kappa}\setminus\Cal I$.

\spheader 511Fd The {\bf covering number} of $\Cal I$ is

\Centerline{$\cov\Cal I=\min\{\#(\Cal E):\Cal E\subseteq\Cal I$,
$\bigcup\Cal E=X\}$,}

\noindent or $\infty$ if there is no such set $\Cal E$.
\cmmnt{(Once more, $X$ is a hidden variable here, and I may write
$\cov(X,\Cal I)$.)}

\leader{511G}{Definition} Let $(X,\Sigma,\mu)$ be a measure space.

\spheader 511Ga If $\kappa$ is a cardinal,
$\mu$ is {\bf $\kappa$-additive} if
$\bigcup\Cal E\in\Sigma$ and $\mu(\bigcup\Cal E)=\sum_{E\in\Cal E}\mu E$
for every disjoint family $\Cal E\in[\Sigma]^{<\kappa}$.   The {\bf
additivity} $\add\mu$ of $\mu$ is the largest cardinal $\kappa$ such
that $\mu$ is
$\kappa$-additive, or $\infty$ if $\mu$ is $\kappa$-additive for every
$\kappa$.

\spheader 511Gb\dvAnew{2013} The {\bf $\pi$-weight} $\pi(\mu)$ of $\mu$
is the coinitiality of $\Sigma\setminus\Cal N(\mu)$, where
$\Cal N(\mu)$ is the null ideal of $\mu$.

\cmmnt{
\spheader 511Gc Recall that the Maharam type $\tau(\mu)$ of $\mu$ is the
Maharam type of the measure algebra of $\mu$ (331Fc).
}

\leader{511H}{Elementary facts:  pre-ordered sets} Let $P$ be a
pre-ordered set.

\spheader 511Ha If $\tilde P$ is the partially ordered set of equivalence
classes in $P$, as described in 511A, all the cardinal functions defined in
511B have the same values for $P$ and
$\tilde P$.    \cmmnt{(The point is that $p$ is an upper bound for
$A\subseteq P$ iff $p^{\ssbullet}$ is an upper bound for
$\{q^{\ssbullet}:q\in A\}\subseteq\tilde P$.)}
\cmmnt{Similarly, }$P$ and $\tilde P$ will have the same triple
precalibers, precaliber pairs and precalibers.

\spheader 511Hb \cmmnt{Obviously,
}$c^{\uparrow}(P)\le\sat^{\uparrow}(P)$.   \cmmnt{(In fact
$c^{\uparrow}(P)$ is determined by $\sat^{\uparrow}(P)$;  see 513Bc
below.)}   If $\kappa\le\lambda$ are cardinals then

\Centerline{$\link_{<\kappa}^{\uparrow}(P)
\le\link_{<\lambda}^{\uparrow}(P)\le\cf P$\dvro{.}{,}}

\noindent\cmmnt{because every upwards-$\hbox{$<$}\lambda$-linked set is
upwards-$\hbox{$<$}\kappa$-linked and every set $\ocint{-\infty,p}$ is
upwards-$\hbox{$<$}\lambda$-linked.
}$c^{\uparrow}(P)\le\link^{\uparrow}(P)$\prooflet{, because if
$A\subseteq P$ is an up-antichain then no upwards-linked set can contain
more than one point of $A$}.   \cmmnt{It follows that}

\Centerline{$\link^{\uparrow}(P)
=\link_{<3}^{\uparrow}(P)\le\link_{<\omega}^{\uparrow}(P)
=\duparrow(P)\le\cf P$.}

\noindent\cmmnt{Of course }$\cf P\le\#(P)$.   \cmmnt{Similarly,}

\Centerline{$\link^{\downarrow}_{<\kappa}(P)
\le\link^{\downarrow}_{<\lambda}(P)\le\ci P$}

\noindent whenever $\kappa\le\lambda$, and

\Centerline{$\cdownarrow(P)\le\link^{\downarrow}(P)
\le\ddownarrow(P)\le\ci P\le\#(P)$,
\quad$\cdownarrow(P)\le\sat^{\downarrow}P$.}

\spheader 511Hc $P$ is empty iff $\cf P=0$ iff
$\ci P=0$ iff $\add P=0$ iff $\duparrow(P)=0$ iff
$\ddownarrow(P)=0$ iff $\link^{\uparrow}(P)=0$ iff
$\link^{\downarrow}(P)=0$ iff $c^{\uparrow}(P)=0$ iff
$\cdownarrow(P)=0$ iff
$\sat^{\uparrow}(P)=1$ iff $\sat^{\downarrow}(P)=1$ iff $\FN(P)=0$.

\spheader 511Hd $P$ is upwards-directed iff
$c^{\uparrow}(P)\le 1$ iff $\sat^{\uparrow}(P)\le 2$ iff
$\link^{\uparrow}(P)\le 1$ iff $\duparrow(P)\le 1$.   \cmmnt{Similarly,}
$P$ is downwards-directed iff $\cdownarrow(P)\le 1$ iff
$\sat^{\downarrow}(P)\le 2$ iff $\link^{\downarrow}(P)\le 1$ iff
$\ddownarrow(P)\le 1$.

If $P$ is not empty, it is upwards-directed iff $\add P>2$ iff
$\add P\ge\omega$.

\spheader 511He If $P$ is partially ordered, it has a greatest element
iff $\cf P=1$ iff
$\add P=\infty$.   Otherwise, $\add P\le\cf P$\prooflet{, since no
cofinal subset of $P$ can have an upper bound in $P$}.

\spheader 511Hf If $P$ is totally ordered, then $\cf P\le\add P$.
\prooflet{\Prf\ If $A\subseteq P$ has no upper bound in $P$ it must be
cofinal with $P$.\ \Qed}

\spheader 511Hg If $\familyiI{P_i}$ is a non-empty family of non-empty
pre-ordered sets with product $P$,
then $\add P=\min_{i\in I}\add P_i$.
\prooflet{\Prf\ A set $A\subseteq P$ lacks an upper bound in $P$ iff there
is an $i\in I$ such that $\{p(i):p\in A\}$ is unbounded above in $P_i$.\
\Qed}

\leader{511I}{Elementary facts:  Boolean algebras}\dvAformerly{5{}11J} Let
$\frak A$ be a Boolean algebra.

\spheader 511Ia

\Centerline{$\link_{<\kappa}(\frak A)\le\link_{<\lambda}(\frak A)
\le\pi(\frak A)$}

\noindent whenever $\kappa\le\lambda$,

\Centerline{$c(\frak A)\le\link(\frak A)\le d(\frak A)
\le\pi(\frak A)\le\#(\frak A)$,
\quad$c(\frak A)\le\sat(\frak A)$.}

\noindent\cmmnt{In addition, }$\tau(\frak A)
\le\pi(\frak A)$\prooflet{ because any order-dense
subset of $\frak A\,\,\tau$-generates $\frak A$}.

\spheader 511Ib $\frak A=\{0\}$ iff $\pi(\frak A)=0$ iff
$\link(\frak A)=0$ iff $d(\frak A)=0$ iff $c(\frak A)=0$ iff
$\sat(\frak A)=1$.

\spheader 511Ic If $\frak A$ is finite, then
$c(\frak A)=\link(\frak A)=d(\frak A)=\pi(\frak A)$ is the number of
atoms of $\frak A$, $\sat(\frak A)=c(\frak A)+1$ and
$\#(\frak A)=2^{c(\frak A)}$, while
$\tau(\frak A)=\lceil\log_2c(\frak A)\rceil$, unless $\frak A=\{0\}$, in
which case $\tau(\frak A)=0$.   If
$\frak A$ is infinite then $c(\frak A)$, $\link(\frak A)$, $d(\frak A)$,
$\pi(\frak A)$, $\sat(\frak A)$
and $\tau(\frak A)$ are all infinite.

\spheader 511Id\cmmnt{ Note that} $\frak A$ is `ccc' just when
$c(\frak A)\le\omega$, that is, $\sat(\frak A)\le\omega_1$.   $\frak A$
is \wsid\cmmnt{, in the sense of
316G,} iff $\wdistr(\frak A)\ge\omega_1$.

\spheader 511Ie{\bf (i)} If $\frak A$ is purely atomic,
$\wdistr(\frak A)=\infty$.    \prooflet{\Prf\ Suppose that
$\ofamily{\xi}{\kappa}{A_{\xi}}$ is any family of partitions of unity in
$\frak A$.   Then the set $B$ of atoms of $\frak A$ is a partition of
unity, and $\{a:a\in A_{\xi}$, $a\Bcap b\ne 0\}$ has just one member for
every $b\in B$ and $\xi<\kappa$.   As
$\ofamily{\xi}{\kappa}{A_{\xi}}$ is arbitrary,
$\wdistr(\frak A)=\infty$.\ \Qed}

\medskip

\quad{\bf (ii)} If $\frak A$ is not purely atomic,
$\wdistr(\frak A)\le\pi(\frak A)$.
\prooflet{\Prf\ Let $c\in\frak A^+$ be disjoint
from every atom of $\frak A$, and $D\subseteq\frak A$ an order-dense set of
size $\pi(\frak A)$;  let $D'$ be $\{d:d\in D$, $d\Bsubseteq c\}$.
For $d\in D'$, there is a disjoint sequence of non-zero elements included
in $d$;  let $A_d$ be a partition of unity in $\frak A$ including
such a sequence.   If $B$ is any partition of unity in $\frak A$, there are
a $b\in B$ such that $b\Bcap c\ne 0$, and a $d\in D'$ such that
$d\Bsubseteq b\Bcap c$;  now $\{a:a\in A_d$, $b\Bcap a\ne 0\}$ is infinite.
So $\family{d}{D'}{A_d}$ witnesses that
$\wdistr(\frak A)\le\#(D')\le\pi(\frak A)$.\ \Qed}

\spheader 511If $\frak m(\frak A)=\infty$ iff $\frak A$ is purely
atomic.   \prooflet{\Prf\ Write $\Cal B$ for the family of all coinitial
subsets of $\frak A^+$.   (i) If $\frak A$ is purely atomic and
$a\in\frak A^+$, then there is an atom $d\Bsubseteq a$;  now $d\in B$
for every $B\in\Cal B$, so $\{d,a\}$ is a linked subset of $\frak A$
meeting every member of $\Cal B$.   Accordingly
$\frak m(\frak A)=\infty$.   (ii) If $\frak A$ is not purely atomic, let
$a\in\frak A^+$ be such that no atom of $\frak A$ is included in $a$.
\Quer\ If $A$ is a linked subset of $\frak A$ containing $a$ and meeting
every member of $\Cal B$, set
$B=\frak A^+\setminus A$.   If $b\in\frak A^+$, then either $b\Bcap a=0$
and $b\in B$, or there are non-zero disjoint $b'$,
$b''\Bsubseteq b\Bcap a$ and one of $b'$, $b''$ must belong to $B$.   So
$B\in\Cal B$, which is impossible.\ \BanG\
So $\frak m(\frak A)\le\#(\Cal B)<\infty$.\ \Qed}

\leader{511J}{Elementary facts:  ideals of sets}\dvAformerly{5{}11K}  Let
$X$ be a set and $\Cal I$ an ideal of subsets of $X$.

\spheader 511Ja $\add\Cal I\ge\omega$\prooflet{, by the definition of
`ideal of sets'}.

\spheader 511Jb
$\shr\Cal I=\sup\{\non(A,\Cal I\cap\Cal PA):A\in\Cal PX\setminus\Cal I\}$,
counting $\sup\emptyset$ as $0$;
$\shr\Cal I\le\#(X)$;   $\shr\Cal I\le\shr^+\Cal I\le(\shr\Cal I)^+$;
if $\shr\Cal I$ is a successor cardinal, $\shr^+\Cal I=(\shr\Cal I)^+$.

\spheader 511Jc Suppose that $\Cal I$ covers $X$ but does not contain
$X$.   Then $\add\Cal I\le\cov\Cal I\le\cf\Cal I$ and
$\add\Cal I\le\non\Cal I\le\shr\Cal I\le\cf\Cal I$.   \prooflet{\Prf\
Let $\Cal J$ be a subset of $\Cal I$ with cardinal $\cov\Cal I$
covering $X$;  let $\Cal K$ be a cofinal
subset of $\Cal I$ with cardinal $\cf\Cal I$;  let
$A\in\Cal PX\setminus\Cal I$ be such that $\#(A)=\non\Cal I$.
(i) $\Cal J$ cannot have an upper bound in $\Cal I$, so
$\add\Cal I\le\#(\Cal J)=\cov\Cal I$.   (ii)
$\bigcup\Cal K=\bigcup\Cal I=X$, so $\cov\Cal I\le\#(\Cal K)=\cf\Cal I$.
(iii) For each $x\in A$ we can
find an $I_x\in\Cal I$ containing $x$;  now $\{I_x:x\in A\}$ cannot have
an upper bound in $\Cal I$, so $\add\Cal I\le\#(A)=\non\Cal I$.   (iv)
By (b), $\shr\Cal I\ge\non\Cal I$.   (v) Take any $B\subseteq X$ such that
$B\notin\Cal I$.   Then for each $K\in\Cal K$ we can find an
$x_K\in B\setminus K$;  now $B'=\{x_K:K\in\Cal K\}$ is not included in
any member of $\Cal K$, so cannot belong to $\Cal I$, while
$B'\subseteq B$ and $\#(B')\le\#(\Cal K)=\cf\Cal I$.   As $B$ is
arbitrary, $\shr\Cal I\le\cf\Cal I$.\ \Qed}

\spheader 511Jd Suppose that $X\in\Cal I$.   Then
$\add\Cal I=\non\Cal I=\infty$,
$\cov\Cal I\le 1$ (with $\cov\Cal I=0$ iff $X=\emptyset$) and
$\shr\Cal I=0$.

\spheader 511Je Suppose that $\Cal I$ has a greatest member which is not
$X$.   Then $\add\Cal I=\cov\Cal I=\infty$ and
$\non\Cal I=\shr\Cal I=\cf\Cal I=1$.

\spheader 511Jf Suppose that $\Cal I$ has no greatest member and does
not cover $X$.   Then $\add\Cal I\le\cf\Cal I$\prooflet{ (511He)},
$\non\Cal I=\shr\Cal I=1$ and
$\cov\Cal I=\infty$.

\spheader 511Jg Suppose that $Y\subseteq X$, and set
$\Cal I_Y=\Cal I\cap\Cal PY$, regarded as an ideal of subsets of $Y$.
Then $\add\Cal I_Y\ge\add\Cal I$, $\non\Cal I_Y\ge\non\Cal I$,
$\shr\Cal I_Y\le\shr\Cal I$, $\shr^+\Cal I_Y\le\shr^+\Cal I$,
$\cov\Cal I_Y\le\cov\Cal I$ and $\cf\Cal I_Y\le\cf\Cal I$.


\exercises{\leader{511X}{Basic exercises (a)}
%\spheader 511Xa
Let $P$ be a partially ordered set and $\kappa\ge 3$ a  cardinal.   Show
that $\add P\ge\kappa$ iff $(\lambda,\lambda,\lambda)$ is an
upwards precaliber triple of $P$ for every $\lambda<\kappa$.
%511E

\sqheader 511Xb Let $X$ be a compact Hausdorff space.   Show that a pair
$(\kappa,\lambda)$ of cardinals is a precaliber pair of $X$ iff whenever
$\ofamily{\xi}{\kappa}{G_{\xi}}$ is a family of non-empty open subsets
of $X$ there is an $x\in X$ such that $\{\xi:x\in G_{\xi}\}$ has
cardinal at least $\lambda$.
%511E out of order query

\spheader 511Xc Let $(X,\Sigma,\mu)$ be a measure space.   For
$A\subseteq X$ write $\mu_A$ for the subspace measure on $A$, and
$\Cal N(\mu)$, $\Cal N(\mu_A)$ for the corresponding null ideals.   Show
that $\shr(X,\Cal N(\mu))
=\sup\{\non(A,\Cal N(\mu_A)):A\in\Cal PX\setminus\Cal N(\mu)\}$.
%511F

%\query gap

\spheader 511Xd Let $(X,\Sigma,\mu)$ be a measure space, and let
$\hat\mu$, $\tilde\mu$ be the completion and c.l.d.\ version of $\mu$.
(i) Let $\Cal N(\mu)=\Cal N(\hat\mu)$ and
$\Cal N(\tilde\mu)$ be the corresponding null ideals.   Show that
$\add\Cal N(\mu)\le\add\Cal N(\tilde\mu)$,
$\cov\Cal N(\mu)\ge\cov\Cal N(\tilde\mu)$,
$\non\Cal N(\mu)\le\non\Cal N(\tilde\mu)$,
$\shr\Cal N(\mu)\ge\shr\Cal N(\tilde\mu)$
and $\shr^+\Cal N(\mu)\ge\shr^+\Cal N(\tilde\mu)$.
(ii) Show that $\add\mu\le\add\hat\mu\le\add\tilde\mu$,
$\pi(\mu)=\pi(\hat\mu)\le\pi(\tilde\mu)$\dvAnew{2013} and
$\tau(\mu)=\tau(\hat\mu)\ge\tau(\tilde\mu)$\dvAnew{2014}.
%511G

%\query gap

\spheader 511Xe Show that if $P$ is a partially ordered set and
$c^{\uparrow}(P)<\omega$ then
$c^{\uparrow}(P)=\link^{\uparrow}(P)=\duparrow(P)$ and
$\frak m^{\uparrow}(P)=\infty$.
%511H

\spheader 511Xf Let $P$ be a partially ordered set.   Show that $\omega$
is an up-precaliber of $P$ iff $c^{\uparrow}(P)<\omega$.
%511H

\sqheader 511Xg(i) Show that if $P$ is a partially ordered set and
$\kappa$ is an up-precaliber of $P$, then $\cf\kappa$ is
also an up-precaliber of $P$.   (ii) Show that if $\kappa$ is a cardinal
and $\cf\kappa>\cf P$ then $\kappa$ is an up-precaliber of $P$.
%511H

\sqheader 511Xh Give $\Bbb R$, $\Bbb N$ and $\Bbb Q$ their usual total
orders.   Show that $\FN(\Bbb N)=\FN(\Bbb Q)=\omega$ and that
$\FN(\Bbb R)=\omega_1$.
%511H

\spheader 511Xi Show that if $P$ is a partially ordered set and
$\#(P)\ge 3$ then $\FN(P)\le\#(P)$.   \Hint{consider separately the
cases $P$ infinite, $P$ finite with no greatest member, and $P$ finite
with greatest and least members.}
%511H

\spheader 511Xj Let $P$ be a partially ordered set and $\Cal Q$ a
family of subsets of $P$ with $\#(\Cal Q)<\add P$.   Show that if
$\bigcup\Cal Q$ is cofinal with $P$ then one of the members of $\Cal Q$
is cofinal with $P$.
%511H

\spheader 511Xk Let $U$ be a Riesz space and $\kappa$ a cardinal.   Then
$U$ is {\bf weakly $(\kappa,\infty)$-distributive} if
whenever $\ofamily{\xi}{\kappa}{A_{\xi}}$ is a family of non-empty
downwards-directed subsets of $U^+$, each with infimum $0$, and
$\bigcup_{\xi<\kappa}A_{\xi}$ has an upper bound in $U$, then

\Centerline{$\{u:u\in U$, for every $\xi<\kappa$ there is a
$v\in A_{\xi}$ such that $v\le u\}$}

\noindent has infimum $0$ in $U$.   Show that an Archimedean Riesz space
is weakly $(\kappa,\infty)$-distributive iff its band algebra is.
\Hint{368R.}
%511I

\spheader 511Xl\dvAnew{2014} Let $X$ be a set and $\Cal I$ an ideal of
subsets of $X$.   Show that the coinitiality $\ci(\Cal PX\setminus\Cal I)$
is at most $\#(X)^{\shr\Cal I}$.
%511J

\leader{511Y}{Further exercises (a)}(i)
%\spheader 511Ya (i)
Show that $\duparrow(P)\le 2^{\link^{\uparrow}(P)}$ for every
partially ordered set $P$.
% if \ofamily}\xi}{\kappa}{A_{\xi}} maximal up-linked subsets covering  P
% for  p\in P  say  C_p=\{\x:p\in A_{\xi}\}
% if  C_p=C_q  and  p\le p'  then  p'\in A_{\xi} , \xi\in C_p=C_q  and
% q , p'  are linked;  hence  \{p:C_p=C\}  up-centered
(ii) Show that there is a partially ordered set $P$ such that
$\duparrow(P)=\omega$ but $P$ cannot be covered by countably many
upwards-directed sets.
%511H mt51bits n06328

\spheader 511Yb Let $\kappa$ be an infinite cardinal, with its usual
well-ordering.  Show that $\FN(\kappa)=\kappa$.
%511H

\spheader 511Yc(i) Find a semi-finite measure space $(X,\Sigma,\mu)$
such that $\cf\Cal N(\mu)<\cf\Cal N(\tilde\mu)$, where $\Cal N(\mu)$ and
$\Cal N(\tilde\mu)$
are the null ideals of $\mu$ and its c.l.d.\ version.
(ii) Find a semi-finite measure space $(X,\Sigma,\mu)$ such that
$\add\Cal N(\tilde\mu)>\add\Cal N(\mu)$ and
$\cf\Cal N(\tilde\mu)<\cf\Cal N(\mu)$.
%511J mt51bits
}%end of exercises

\endnotes{
\Notesheader{511} Because $(P,\ge)$ is a pre-ordered set whenever
$(P,\le)$ is, any cardinal function on pre-ordered sets is bound
to appear in two mirror-image forms.   It does not quite follow that we
have to set up a language with a complete set of mirror pairs of
definitions, and indeed I have omitted the reflections of `additivity'
and `bursting number';   but the naturally arising pre-ordered
sets to which we shall want to apply these ideas may appear in either
orientation.   The most natural conversions to topological spaces and
Boolean algebras use the families of non-empty open sets and non-zero
elements, which are `active downwards', so that we have such formulae as
$\pi(\frak A)=\ci\frak A^+$ and
$c(X)=\cdownarrow(\frak T\setminus\{\emptyset\})$;  but we could
equally well
say that $\pi(\frak A)=\cf\frak A^-$ or that $c(X)$ is the
upwards-cellularity of the partially ordered set of proper closed
subsets of $X$.

Most readers, especially those acquainted with Volumes 3 and 4 of this
treatise, will be more familiar with topological spaces and Boolean
algebras than with general pre-ordered sets, and will prefer to
approach the concepts here through the formulations in 5A4A and 511D.
But even in the present chapter we shall be looking at questions which
demand substantial fragments of the theory of general partially ordered
sets, and I think it is useful to grapple with these immediately.
The list of definitions above
is a long one, and the functions here vary widely in importance;  but I
hope you will come to agree that all are associated with interesting
questions.

I apologise for introducing two cardinal functions to represent the
`breadth' of a pre-ordered set (or topological space or Boolean
algebra), its `cellularity'
and `saturation'.   It turns out that the saturation of a space
determines its cellularity (513Bc), which seems to render the concept of
`cellularity' unnecessary;  but it is well-established and makes some
formulae simpler.
%332E 332F 332R
This is an example of
a standard problem:  whenever we give a name to
a supremum, we find ourselves
asking whether the supremum is attained.   The question of whether
cellularity is attained turns out to be rather interesting (513B again).
In the case of shrinking numbers, the ordinary shrinking number
$\shr\Cal I$ is the one which has been most studied, but I shall have some
results which are more elegantly expressed in terms of the augmented
shrinking number $\shr^+\Cal I$.

I give very little space here to the functions $\frak m()$ and
$\wdistr()$ and to precalibers;   these are bound to be a bit
mysterious.   Later in the chapter I will explore their relations
with each other and with other cardinal functions.   You may recognise
them as belonging to the general area associated with Martin's axiom
({\smc Fremlin 84a}, or \S517 below).   `Precaliber pairs' have a
slightly more direct description in the context of compact Hausdorff
spaces (511Xb).   `Freese-Nation numbers' relate to quite different
aspects of the structure of ordered sets.   As will be made
clear in the next two sections, all the other cardinal functions defined
in 511B refer to the cofinal (or coinitial) structure of a partially
ordered set;  the Freese-Nation number, by contrast, tells us something
about the nature of intervals inside it.   We see a difference already
in the formula for the
Freese-Nation number of a Boolean algebra, which refers to the whole
algebra $\frak A$ rather than to $\frak A^+$.   Another
signal is the fact that it is not a trivial matter to calculate the
Freese-Nation number of a finite partially ordered set.

The only cardinal functions I have explicitly defined for measure spaces
are the additivity and $\pi$-weight of a measure (511G), and even these
are, in the most
important cases, reducible to the additivity of the null ideal (521A) and
the $\pi$-weight of the measure algebra (521Da).
I give a pair of warming-up exercises (511Xc-511Xd),
but we shall hardly see `measure' again until Chapter 52.
For the questions studied in this volume, the important
cardinals associated with a measure $\mu$ are those defined from its
measure algebra together with the four cardinals $\add\Cal N(\mu)$,
$\cov\Cal N(\mu)$, $\non\Cal N(\mu)$ and $\cf\Cal N(\mu)$.   In
particular, the
additivity of Lebesgue measure will have a special position.   In the
case of a topological measure space, of course, we can investigate
relationships between the cardinal functions of the topology and the
cardinal functions of the measure.   I will come to such questions in
Chapter 53.
}%end of notes

\discrpage

