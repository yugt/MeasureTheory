\frfilename{mt458.tex}
\versiondate{16.1.07}
\copyrightdate{2007}

\def\chaptername{Perfect measures, disintegrations and processes}
\def\sectionname{Relative independence and relative products}

\newsection{458}

Stochastic independence is one of the central concepts of probability
theory, and pervades measure theory.   We come now to a generalization
of great importance.   If $X_1$, $X_2$ and $Y$ are random
variables, we may find that $X_1$ and $X_2$ are `relatively independent
over $Y$', or `independent when conditioned on $Y$', in the sense that
if we know the value of $Y$, then we learn nothing further about one of
the $X_i$ if we are told the value of the other.   For any stochastic
process, where information comes to us piecemeal, this idea is likely to
be fundamental.   In this section I set out a general framework for
discussion of relative independence (458A), introducing relative
distributions (458I) and relative independence in measure algebras
(458L-458M).   In the second half of the section I look at
`relative product measures'
(458N, 458Q), giving the basic existence theorems (458O, 458S, 458T).

\leader{458A}{Relative independence} Let
$(X,\Sigma,\mu)$ be a probability space and $\Tau$ a $\sigma$-subalgebra
of $\Sigma$.

\spheader 458Aa I say that a family $\familyiI{E_i}$ in $\Sigma$ is
{\bf relatively (stochastically)
independent} over $\Tau$ if whenever $J\subseteq I$ is finite and not
empty, and $g_i$ is a conditional expectation of
$\chi E_i$ on $\Tau$ for each $i\in J$,
then $\mu(F\cap\bigcap_{i\in J}E_i)=\int_F\prod_{i\in J}g_id\mu$ for every
$F\in\Tau$;  that is, $\prod_{i\in J}g_i$ is a conditional expectation of
$\chi(\bigcap_{i\in J}E_i)$ on $\Tau$.
\cmmnt{(Note that this does not depend on which
conditional expectations $g_i$ we take, since any two conditional
expectations of $\chi E_i$ must be equal almost everywhere.)}
A family $\familyiI{\Sigma_i}$ of
subalgebras of $\Sigma$ is {\bf relatively independent} over
$\Tau$ if $\familyiI{E_i}$ is relatively independent over $\Tau$
whenever $E_i\in\Sigma_i$ for every $i\in I$.

\spheader 458Ab I say that a family $\familyiI{f_i}$ in
$\eusm L^0(\mu)$\cmmnt{ (the space of almost-everywhere-defined
virtually measurable real-valued functions, or `random variables')} is
{\bf relatively independent} over $\Tau$ if
$\familyiI{\Sigma_i}$ is relatively independent over $\Tau$ (with
respect to the completion of $\mu$), where $\Sigma_i$ is the
$\sigma$-algebra defined by $f_i$\cmmnt{ in the sense of 272C, that
is, the $\sigma$-algebra generated by
$\{f_i^{-1}[F]:F\subseteq\Bbb R$ is a Borel set$\}$}.

\spheader 458Ac I remark at once that a family of $\sigma$-algebras or
random variables is relatively independent iff every finite subfamily
is\cmmnt{ (cf.\ 272Bb)}.

\spheader 458Ad\cmmnt{ It will be convenient to have a shorthand referring
to  lattices of $\sigma$-algebras of sets.}   If $\Sigma$, $\Tau$ are
algebras of subsets of a set $X$, I will write $\Sigma\vee\Tau$
for the $\sigma$-algebra
of subsets of $X$ generated by $\Sigma\cup\Tau$;  \cmmnt{similarly,}
if $\familyiI{\Sigma_i}$ is a family of algebras of
subsets of $X$, then $\bigvee_{i\in I}\Sigma_i$ will be the
$\sigma$-algebra generated by $\bigcup_{i\in I}\Sigma_i$.
\cmmnt{Note that the functions
$\vee$, $\bigvee$ here are always supposed to yield
$\sigma$-algebras, even if we start with algebras which are not closed
under countable unions, so that
$\Sigma\vee\Sigma$ could in principle be strictly larger than $\Sigma$.}

\leader{458B}{}\cmmnt{ There are some surprising results at the very
beginning of the theory of relative independence;  see 458Xa, for instance.
On the positive side, we have the following facts.

\medskip

\noindent}{\bf Lemma} Let
$(X,\Sigma,\mu)$ be a probability space, $\Tau$ a $\sigma$-subalgebra
of $\Sigma$, and $\familyiI{\Sigma_i}$ a family of subalgebras of
$\Sigma$ such that $\Tau\subseteq\bigcup_{i\in I}\Sigma_i$.
Suppose that whenever $J\subseteq I$ is finite and not
empty, $E_i\in\Sigma_i$ and $g_i$ is a conditional expectation of
$\chi E_i$ on $\Tau$ for each $i\in J$,
then $\mu(\bigcap_{i\in J}E_i)=\int\prod_{i\in J}g_id\mu$.   Then
$\familyiI{\Sigma_i}$ is relatively independent over $\Tau$.

\proof{ Take $F\in\Tau$, a finite non-empty $J\subseteq I$ and
$E_i\in\Sigma_i$ for $i\in J$.   Let $j\in I$ be such that $F\in\Sigma_j$.
Set $K=J\cup\{j\}$;  if $j\notin J$, set $E_j=X$.   Now set
$E_j'=E_j\cap F$ and $E_i'=E_i$ for $i\in K\setminus\{j\}$.

For $i\in K$, let $g_i$ be a conditional expectation of $\chi E_i$ on
$\Tau$.   Set $g'_j=g_j\times\chi F$ and $g'_i=g_i$ for
$i\in K\setminus\{j\}$;  then $g'_i$ is a conditional expectation of
$\chi E'_i$ for each $i\in K$.   So we have

\Centerline{$\mu(F\cap\bigcap_{i\in J}E_i)
=\mu(\bigcap_{i\in K}E'_i)
=\int\prod_{i\in K}g'_id\mu
=\int_F\prod_{i\in J}g_id\mu$.}

\noindent As $F$ and $\family{i}{J}{E_i}$ are arbitrary,
$\familyiI{\Sigma_i}$ is relatively independent over $\Tau$.
}%end of proof of 458B

\leader{458C}{Proposition}
Let $(X,\Sigma,\mu)$ be a probability space, $\Bbb T$ a non-empty
upwards-directed family of $\sigma$-subalgebras of $\Sigma$,
and $\familyiI{\Sigma_i}$ a family of
$\sigma$-subalgebras of $\Sigma$ which is relatively independent over
$\Tau$ for every $\Tau\in\Bbb T$.   Then $\familyiI{\Sigma_i}$ is
relatively independent over $\bigvee\Bbb T$.

%no such general result for downwards-directed families, see
%458Ya

\proof{{\bf (a)} Suppose first that $\Bbb T$ is countable;
because it is upwards-directed, there is a
non-decreasing sequence $\sequencen{\Tau_n}$ in $\Bbb T$ such
that $\bigcup\Bbb T=\bigcup_{n\in\Bbb N}\Tau_n$ and
$\bigvee\Bbb T=\bigvee_{n\in\Bbb N}\Tau_n$.
Take a non-empty finite set $J\subseteq I$
and $E_i\in\Sigma_i$ for $i\in J$;  set $E=\bigcap_{i\in J}E_i$.
For $i\in J$, let $g_{ni}$ be a conditional
expectation of $\chi E_i$ on $\Tau_n$ for each $n$;  then
$g_i=\lim_{n\to\infty}g_{ni}$ is a conditional expectation of $\chi E_i$
on $\bigvee\Bbb T$ (275I).   Similarly, if $h_n$ is a
conditional expectation of $\chi E$ on $\Tau_n$ for each $n$,
$h=\lim_{n\to\infty}h_n$ is a
conditional expectation of $\chi E$ on $\bigvee\Bbb T$.
Since $\family{i}{J}{E_i}$ is relatively independent over $\Tau_n$,
$h_n\eae\prod_{i\in J}g_{ni}$ for each $n$;  accordingly
$h\eae\prod_{i\in J}h_i$, and $\prod_{i\in J}h_i$ is a conditional
expectation of $\chi E$ on $\bigvee\Bbb T$.
As $\family{i}{J}{E_i}$ is arbitrary,
$\familyiI{\Sigma_i}$ is relatively independent over $\bigvee\Bbb T$.

\medskip

{\bf (b)} For the general case, take a non-empty finite
$J\subseteq I$ and $E_i\in\Sigma_i$ for $i\in J$;  set
$E=\bigcap_{i\in I}E_i$.   For each $i\in J$, let
$g_i:X\to[0,1]$ be a $\bigvee\Bbb T$-measurable conditional expectation of
$\chi E_i$ on $\bigvee\Bbb T$, and $g:X\to[0,1]$ a
$\bigvee\Bbb T$-measurable
conditional expectation of $\chi E$ on $\bigvee\Bbb T$.
Then for every $i\in J$ and $q\in\Bbb Q$ there is
a countable set $\Bbb T_{iq}\subseteq\Bbb T$ such that
$\{x:g_i(x)\ge q\}\in\bigvee\Bbb T_{iq}$;
similarly, there is for each $q\in\Bbb Q$ a
countable set $\Bbb T'_q\subseteq\Bbb T$ such that
$\{x:g(x)\ge q\}\in\bigvee\Bbb T'_q$.
Let $\tilde{\Bbb T}$ be a countable upwards-directed subset of $\Bbb T$
including
$\bigcup_{i\in J,q\in\Bbb Q}\Bbb T_{iq}\cup\bigcup_{q\in\Bbb Q}\Bbb T'_q$.
Then every $g_i$ is $\bigvee{\tilde\Bbb T}$-measurable,
so is a conditional expectation of $\chi E_i$ on $\bigvee{\tilde\Bbb T}$;
similarly, $g$ is a conditional expectation of $\chi E$ on
$\bigvee{\tilde\Bbb T}$.   By (i), $g\eae\prod_{i\in J}g_i$, so
$\prod_{i\in J}g_i$ is a conditional expectation of $\chi E$ on
$\bigvee\Bbb T$.   As $\family{i}{J}{E_i}$ is arbitrary,
$\familyiI{\Sigma_i}$ is relatively independent over
$\bigvee\Bbb T$, as claimed.
}%end of proof of 458C

\leader{458D}{Proposition} Let $(X,\Sigma,\mu)$ be a probability space,
$\Tau$ a $\sigma$-subalgebra of $\Sigma$ and $\familyiI{\Sigma_i}$ a
family of subalgebras of $\Sigma$ which is relatively independent over
$\Tau$.

(a) If $J\subseteq I$ and $\Sigma'_i$ is a subalgebra of $\Sigma_i$ for
$i\in J$, then $\family{i}{J}{\Sigma'_j}$ is relatively independent over
$\Tau$.

(b) Set $\Sigma^*_i=\Sigma_i\vee\Tau$ for $i\in I$.
Then $\familyiI{\Sigma^*_i}$ is relatively independent over $\Tau$.

(c)\dvAnew{2007} If $\Cal E\subseteq\bigcup_{i\in I}\Sigma_i$, then
$\familyiI{\Sigma_i}$ is relatively independent over
the $\sigma$-algebra generated by $\Tau\cup\Cal E$.

\proof{{\bf (a)} Immediate from the definition in 458Aa.

\medskip

{\bf (b)(i)} Suppose that $F_0\in\Tau$ and that $\Sigma'_i$ is the algebra
generated by $\Sigma_i\cup\{F_0\}$ for each $i\in I$.   Then
$\familyiI{\Sigma'_i}$ is relatively independent over $\Tau$.
\Prf\ Suppose that $J\subseteq I$ is finite and not empty, and that
$E_i\in\Sigma'_i$ for each $i\in J$.   For $i\in I$, we can express $E_i$
as $(G_i\cap F_0)\cup(H_i\setminus F_0)$, where $G_i$, $H_i\in\Sigma_i$.   Let
$g_i$, $h_i$ be conditional expectations of $\chi G_i$, $\chi H_i$ on
$\Tau$;  then $f_i=g_i\times\chi F_0+h_i\times\chi(X\setminus F_0)$ is a
conditional expectation of $\chi E$ on $\Tau$.   Now, for any
$F\in\Tau$, we have

$$\eqalignno{\int_F\prod_{i\in J}f_i
&=\int_F\prod_{i\in J}g_i\times\chi F_0
  +\prod_{i\in J}h_i\times\chi(X\setminus F_0)\cr
&=\int_{F\cap F_0}\prod_{i\in J}g_i
  +\int_{F\setminus F_0}\prod_{i\in J}h_i
=\mu(F\cap\bigcap_{i\in J}G_i\cap F_0)
  +\mu(F\cap\bigcap_{i\in J}H_i\setminus F_0)\cr
\displaycause{because the families
$\family{i}{J}{G_i}$ and $\family{i}{J}{H_i}$ are
both relatively independent over $\Tau$}
&=\mu(F\cap\bigcap_{i\in J}E_i).\cr}$$

\noindent As $\family{i}{J}{E_i}$ is arbitrary, $\familyiI{\Sigma'_i}$ is
relatively independent over $\Tau$.\ \Qed

\medskip

\quad{\bf (ii)} Suppose that $\Cal E\subseteq\Tau$ is finite, and that
$\Sigma'_i$ is the algebra generated by $\Sigma_i\cup\Cal E$ for each $i$.
Then $\familyiI{\Sigma'_i}$ is relatively independent over $\Tau$.   \Prf\
Induce on $\#(\Cal E)$, using (i) for the inductive step.\ \Qed

\medskip

\quad{\bf (iii)} Suppose that $\Sigma'_i$ is the algebra generated by
$\Sigma_i\cup\Tau$ for
each $i\in I$.   Then $\familyiI{\Sigma'_i}$ is
relatively independent over $\Tau$.   \Prf\ If $J\subseteq I$ is finite and
not empty, and
$E_i\in\Sigma'_i$ for each $i\in J$, then there is a finite set
$\Cal E\subseteq\Tau$ such that $E_i$ belongs to the algebra $\Sigma''_i$
generated by $E_i\cup\Cal E$ for every $i\in J$.   By (ii),
$\familyiI{\Sigma''_i}$ is relatively independent over $\Tau$, so
$\family{i}{J}{E_i}$ is relatively independent over $\Tau$;  as
$\family{i}{J}{E_i}$ is arbitrary, $\familyiI{\Sigma'_i}$ is
relatively independent over $\Tau$.\ \Qed

\medskip

\quad{\bf (iv)}
Finally, suppose that $J\subseteq I$ is finite and not
empty, that $E_i\in\Sigma^*_i$ for each $i\in J$, that $F\in\Tau$
and that $\epsilon>0$.
For $i\in J$, let $\Sigma'_i$ be the algebra generated by
$\Sigma_i\cup\Tau$;  then there is an $E'_i\in\Sigma'_i$ such that
$\mu(E'_i\symmdiff E_i)\le\epsilon$ (136H).
%from 2007
Let $g_i$, $g'_i$ be conditional expectations of $\chi E_i$, $\chi E'_i$ on
$\Tau$;  we can arrange that they are all defined on the whole of $X$ and
take values in $[0,1]$.   Then

$$\eqalignno{|\mu(F\cap\bigcap_{i\in J}E_i)-\int_F\prod_{i\in J}g_i|
&\le\sum_{i\in J}\mu(E_i\symmdiff E'_i)
  +|\mu(F\cap\bigcap_{i\in J}E'_i)-\int_F\prod_{i\in J}g'_i|\cr
&\mskip250mu  +\int_F|\prod_{i\in J}g'_i-\prod_{i\in J}g_i|\cr
&\le\epsilon\#(J)+0
  +\int_F\sum_{i\in J}|g'_i-g_i|\cr
\displaycause{(iii) above and 285O}
&\le\epsilon\#(J)+\sum_{i\in J}\int|g'_i-g_i|\cr
&\le\epsilon\#(J)+\sum_{i\in J}\int|\chi E'_i-\chi E_i|\cr
\displaycause{233J or 242Je}
&=\epsilon\#(J)+\sum_{i\in J}\mu(E'_i\symmdiff E_i)
\le 2\epsilon\#(J).\cr}$$

\noindent As $\epsilon$ is arbitrary,

\Centerline{$\mu(F\cap\bigcap_{i\in J}E_i)=\int_F\prod_{i\in J}g_i$.}

\noindent As $\family{i}{J}{E_i}$ and $F$ are arbitrary,
$\familyiI{\Sigma^*_i}$ is relatively independent.

\medskip

{\bf (c)} For any $\Cal E\subseteq\bigcup_{i\in I}\Sigma_i$,
write $\Tau_{\Cal E}$ for the
$\sigma$-algebra generated by $\Tau\cup\Cal E$.

\medskip

\quad{\bf (i)}
Suppose that $i$, $j\in I$ are distinct, $E\in\Sigma_i$, $g$ is a
conditional expectation of $\chi E$ on $\Tau$, and $H\in\Sigma_j$.   Then
$g$ is a conditional expectation of $\chi E$ on $\Tau_{\{H\}}$.
\Prf\ Let $h$ be a conditional expectation of $\chi H$ on
$\Tau$.   If $F\in\Tau$, then

$$\eqalignno{\mu(F\cap H\cap E_i)
&=\int_Fg\times h\cr
\displaycause{because $\Sigma_j$ and $\Sigma_i$ are relatively independent
over $\Tau$}
&=\int_Fg\times\chi H\cr
\displaycause{because $g\times h$ is a conditional expectation of
$g\times\chi H$ on $\Tau$, see 233Eg}
&=\int_{F\cap H}g.\cr}$$

\noindent Similarly, $\mu(F\cap E_i\setminus H)=\int_{F\setminus H}g$.
Now any $G\in\Tau_{\{H\}}$ is expressible as
$(F_1\cap H)\cup(F_2\setminus H)$ where $F_1$, $F_2\in\Tau$, so that

\Centerline{$\mu(G\cap E)=\mu(F_1\cap E\cap H)+\mu(F_2\cap E\setminus H)
=\int_{F_1\cap H}g+\int_{F_2\setminus H}g=\int_Gg$,}

\noindent as required.\ \Qed

\medskip

\quad{\bf (ii)} If $j\in I$ and $H\in\Sigma_j$, $\familyiI{\Sigma_i}$ is
relatively independent over $\Tau_{\{H\}}$.

\medskip

\Prf\ \grheada\ Let $J\subseteq I$ be a non-empty finite set
containing $j$, and
$\family{i}{J}{E_i}$ a family such that $E_i\in\Sigma_i$ for $i\in J$.
Set $K=J\setminus\{j\}$.
For $i\in K$, let $g_i:X\to[0,1]$ be a $\Tau$-measurable conditional
expectation of $\chi E_i$ on $\Tau$.
Then $g_i$ is a conditional expectation of $\chi E_i$ on $\Tau_{\{H\}}$, by
(i).   Let $g_j$ be a conditional expectation of $\chi E_j$ on
$\Tau_{\{H\}}$, and $g'_j$ a conditional
expectation of $\chi(E_j\cap H)$ on $\Tau$.   Then, for any $F\in\Tau$,

$$\eqalignno{\mu(F\cap H\cap\bigcap_{i\in J}E_i)
&=\mu(F\cap(E_j\cap H)\cap\bigcap_{i\in K}E_i)
=\int_Fg'_j\times\prod_{i\in K}g_i\cr
\displaycause{because $\family{i}{J}{\Sigma_i}$ is relatively independent
over $\Tau$}
&=\int_F\chi(E_j\cap H)\times\prod_{i\in K}g_i\cr
\displaycause{233Eg again, because $\prod_{i\in K}g_i$ is bounded and
$\Tau$-measurable}
&=\int_{F\cap H}\chi E_j\times\prod_{i\in K}g_i
=\int_{F\cap H}g_j\times\prod_{i\in K}g_i\cr}$$

\noindent (because $\prod_{i\in K}g_i$ is bounded and
$\Tau_{\{H\}}$-measurable).   Similarly,

\Centerline{$\mu(F\cap\bigcap_{i\in J}E_i\setminus H)
=\int_{F\setminus H}g_j\times\prod_{i\in K}g_i
=\int_{F\setminus H}\prod_{i\in J}g_i$}

\noindent for every $F\in\Tau$;   putting these together, as in (i),

\Centerline{$\mu(G\cap\bigcap_{i\in J{}}E_i)
=\int_G\prod_{i\in J}g_i$}

\noindent for every $G\in\Tau_{\{H\}}$, and $\prod_{i\in J}g_i$ is a
conditional expectation of $\chi(\bigcap_{i\in J}E_i)$ on $\Tau_{\{H\}}$.

\medskip

\qquad\grheadb\
This is not exactly the formula demanded by the definition
in 458Aa, because I supposed that $j\in J$;  but if we have a non-empty
finite $J\subseteq I\setminus\{j\}$ and
$\family{i}{J}{E_j}\in\prod_{i\in J}\Sigma_j$, set $J'=J\cup\{j\}$ and
$E_j=X$ to see that there is a family $\family{i}{J'}{g_i}$ such that $g_i$
is a conditional expectation of $\chi E_i$ on $\Tau_{\{H\}}$ for every $i$,
and

\Centerline{$\mu(G\cap\bigcap_{i\in J}E_i)
=\mu(G\cap\bigcap_{i\in J'}E_i)
=\int_G\prod_{i\in J'}g_i
=\int_G\prod_{i\in J}g_i$}

\noindent for every $G\in\Tau_{\{H\}}$.
So $\familyiI{\Sigma_i}$ really is relatively
independent over $\Tau_{\{H\}}$.\ \Qed

\medskip

\quad{\bf (iii)} Inducing on $\#(\Cal E)$, we see that
$\familyiI{\Sigma_i}$
is relatively independent over $\Tau_{\Cal E}$ whenever
$\Cal E\subseteq\bigcup_{i\in I}\Sigma_i$ is finite.
By 458C,
$\familyiI{\Sigma_i}$
is relatively independent over $\Tau_{\Cal E}$ for every
$\Cal E\subseteq\bigcup_{i\in I}\Sigma_i$.
}%end of proof of 458D

\leader{458E}{Example}\cmmnt{ The simplest examples of relatively
independent $\sigma$-algebras arise as follows.}  Let $(X,\Sigma,\mu)$
be a probability space, $\familyiI{\Tau_i}$ an independent family of
$\sigma$-subalgebras of $\Sigma$\cmmnt{, as in 272Ab}, and $\Tau$ a
$\sigma$-subalgebra of $\Sigma$ which is independent of
$\bigvee_{i\in I}\Tau_i$.   For each
$i\in I$, let $\Sigma_i$ be $\Tau\vee\Tau_i$.
Then $\familyiI{\Sigma_i}$ is relatively independent over $\Tau$.

\proof{ In view of 458Db, it is enough to show that $\familyiI{\Tau_i}$ is
relatively independent over $\Tau$.   But if we have a non-empty finite
$J\subseteq I$ and $E_i\in\Tau_i$ for $i\in I$, then
$\mu(E_i\cap F)=\mu E_i\cdot\mu F$ for $F\in\Tau$, so
$f_i=\mu E_i\cdot\chi X$ is a
conditional expectation of $\chi E_i$ on $\Tau$, for each $i$.
Similarly, setting $E=\bigcap_{i\in J}E_i$, $\mu E\cdot\chi X$
is a conditional
expectation of $\chi E$ on $\Tau$.   Since $\mu E=\prod_{i\in J}\mu E_j$,
$\prod_{i\in J}f_i$ is a conditional expectation of $\chi E$ on $\Tau$,
which is what we need to know.
}%end of proof of 458E

\leader{458F}{}\cmmnt{ The following facts are elementary but occasionally
useful.

\wheader{458F}{4}{2}{2}{60pt}

\noindent}{\bf Proposition} Let $(X,\Sigma,\mu)$ be a probability space and
$\Tau$ a $\sigma$-subalgebra of $\Sigma$.

(a)\dvAnew{2007} Let $\familyiI{f_i}$ be a family of
non-negative $\mu$-integrable functions on $X$ which is relatively
independent over $\Tau$.   For each $i\in I$ let
$g_i$ be a conditional expectation of $f_i$ on $\Tau$.   Then for any
$F\in\Tau$ and $i_0,\ldots,i_n\in I$,

\Centerline{$\int_F\prod_{j=0}^ng_{i_j}\le\int_F\prod_{j=0}^nf_{i_j}$}

\noindent with equality if all the $i_j$ are distinct.

(b)\dvAnew{2009}
Suppose that $\Sigma_1$, $\Sigma_2$ are $\sigma$-subalgebras of
$\Sigma$ which are relatively independent over $\Tau$, and that
$f\in\eusm L^1(\mu\restr\Sigma_1)$.   If $g$ is a conditional expectation
of $f$ on $\Tau$, then it is a conditional expectation of $f$ on
$\Tau\vee\Sigma_2$.

\proof{{\bf (a)}
Let $\Sigma_i$ be the $\sigma$-algebra generated by $f_i$ for
each $i$, so that $\familyiI{\Sigma_i}$ is relatively independent over
$\Tau$.

\medskip

\quad{\bf (i)} To begin with, suppose that $i_0,\ldots,i_n$ are all
different.

\medskip

\qquad\grheada\
If $f_i=\chi E_i$ for each $i\in I$, where $E_i\in\Sigma_i$,
the result is just the definition of `relative independence' in 458Aa.

\medskip

\qquad\grheadb\ Because both sides of the desired equality are
multilinear expressions of the inputs, and conditional expectation is an
essentially linear operation, the same is true if all the $f_i$ are simple
functions.

\medskip

\qquad\grheadc\ For general non-negative
integrable random variables $f_i$, let
$\sequence{k}{f_{ik}}$ be a non-decreasing sequence
of non-negative $\Sigma_i$-simple
functions converging almost everywhere to $f_i$ for each $i$, and $g_{ik}$
a conditional expectation of $f_{ik}$ for all $i$ and $k$.   Then
$\sequence{k}{g_{ik}}$ is non-decreasing almost everywhere and converges
a.e.\ to the given conditional expectation $g_i$ of $f_i$.   So

\Centerline{$\int_F\prod_{j=0}^ng_{i_j}
=\lim_{k\to\infty}\int_F\prod_{j=0}^ng_{i_jk}
=\lim_{k\to\infty}\int_F\prod_{j=0}^nf_{i_jk}
=\int_F\prod_{j=0}^nf_{i_j}$,}

\noindent as required.

\medskip

\quad{\bf (ii)}\grheada\
Now suppose that the $i_0,\ldots,i_n$ are not all distinct,
but that all the $f_{i_j}$ are bounded.   Let $l_0,\ldots,l_m$
enumerate
$\{i_0,\ldots,i_n\}$ and for $j\le m$ set $k_j=\#(\{r:i_r=l_j\})$.
For each $j\le m$, let $h_j$ be a conditional expectation of
$f_{l_j}^{k_j}=|f_{l_j}|^{k_j}$.   Because $t\mapsto|t|^{k_j}$ is convex,
$g_{l_j}^{k_j}\leae h_j$ (233J).   So

$$\eqalignno{\int_F\prod_{j=0}^ng_{i_j}
&=\int_F\prod_{j=0}^mg_{l_j}^{k_j}
\le\int_F\prod_{j=0}^mh_j
=\int_F\prod_{j=0}^mf_{l_j}^{k_j}\cr
\displaycause{by part (i), because each $f_{l_j}^{k_j}$ is
$\Sigma_{l_j}$-measurable}
&=\int_F\prod_{j=0}^nf_{i_j},\cr}$$

\noindent as required.

\medskip

\qquad\grheadb\ Finally, for the general case, take simple functions
$f_{ik}$ and conditional expectations $g_{ik}$ as in
(a-iii) above.   Then

\Centerline{$\int_F\prod_{j=0}^ng_{i_j}
=\lim_{k\to\infty}\int_F\prod_{j=0}^ng_{i_jk}
\le\lim_{k\to\infty}\prod_{j=0}^n\int_Ff_{i_jk}
=\int_F\prod_{j=0}^nf_{i_j}$,}

\noindent and the proof is complete.

\medskip

{\bf (b)} Adjusting $f$ on a negligible set if necessary, we may
suppose that $f$ is $\Sigma_1$-measurable.
Take any $F\in\Sigma_2\vee\Tau$, and let $h\ge 0$ be
a conditional expectation of $\chi F$ on $\Tau$.
By 458Db and 458Da, $\Sigma_1$ and
$\Sigma_2\vee\Tau$ are relatively independent over $\Tau$, so $f$ and
$\chi F$ are relatively independent over $\Tau$.   Accordingly

$$\eqalignno{\int_Ff
&=\int f\times\chi F
=\int g\times h\cr
\displaycause{applying (a) to the positive and negative parts of $f$}
&=\int g\times\chi F\cr
\displaycause{233K}
&=\int_Fg.\cr}$$

\noindent As $F$ is arbitrary and

\Centerline{$g\in\eusm L^1(\mu\restr\Tau)
\subseteq\eusm L^1(\mu\restr\Sigma_2\vee\Tau)$,}

\noindent $g$ is a conditional expectation of $f$ on
$\Sigma_2\vee\Tau$.
}%end of proof of 458F

\medskip

\noindent{\bf Remark} In (a),
I have avoided speaking of conditional expectations
of products $\prod_{j=0}^nf_{i_j}$ because these need not be integrable
functions.   But when $\prod_{j=0}^nf_{i_j}$ is integrable and has a
conditional expectation $g$, then we must have
$\prod_{j=0}^ng_{i_j}\leae g$, with equality almost everywhere
when the $i_j$ are distinct.

%can probably do without the next bit
\leader{*458G}{}\cmmnt{ It is sometimes useful to know that
`relative independence' can be defined without using the apparatus of
conditional expectations;  indeed, we have a formulation which can be used
with finitely additive functionals rather than measures.

\medskip

\noindent}{\bf Lemma} Let $(X,\Sigma,\mu)$ be a probability space,
$\Tau$ a $\sigma$-subalgebra of $\Sigma$, and $\familyiI{\Sigma_i}$ a
family of $\sigma$-subalgebras of $\Sigma$.
Let $\Bbb T$ be the family of finite
subalgebras of $\Tau$.   For $\Lambda\in\Bbb T$ write $\Cal A_{\Lambda}$
for the set of non-negligible atoms in $\Lambda$.   For non-empty finite
$J\subseteq I$, $\family{i}{J}{E_i}\in\prod_{i\in J}\Sigma_i$ and
$F\in\Tau$, set

\Centerline{$\phi_{\Lambda}(F,\family{i}{J}{E_i})
=\sum_{H\in\Cal A_{\Lambda}}\mu(H\cap F)
  \cdot\prod_{i\in J}\Bover{\mu(E_i\cap H)}{\mu H}$.}

\noindent Then $\familyiI{\Sigma_i}$ is relatively independent over
$\Tau$ iff $\lim_{\Lambda\in\Bbb T,\Lambda\uparrow}
\phi_{\Lambda}(F,\family{i}{J}{E_i})=\mu(F\cap\bigcap_{i\in J}E_i)$
whenever
$J\subseteq I$ is finite and not empty, $E_i\in\Sigma_i$ for every
$i\in J$ and $F\in\Tau$.

\proof{{\bf (a)} The point is just that if $J\subseteq I$ is finite and
not empty, $E_i\in\Sigma_i$ for $i\in J$, $g_i$ is a conditional
expectation of $\chi E_i$ on $\Tau$ for each $i$, and $F\in\Tau$, then
$\int_F\prod_{i\in J}g_id\mu
=\lim_{\Lambda\uparrow}\phi_{\Lambda}(F,\family{i}{J}{E_i})$.   \Prf\
Adjusting each $g_i$ on a negligible set if necessary, we may suppose
that it is $\Tau$-measurable, defined
everywhere on $X$ and takes values between $0$ and $1$.

Fix $n\in\Bbb N$ for the moment.   Let $\Lambda_n$ be the finite
subalgebra
of $\Tau$ generated by sets of the form $\{x:g_i(x)\le 2^{-n}k\}$ for
$i\in J$ and $k\le 2^n$, and $\Lambda$ any finite
subalgebra of $\Sigma_0$ including $\Lambda_n$.   If $H$ is an atom of
$\Lambda$ and $\mu H>0$, then there are integers $k_i$, for $i\in J$,
such that
$2^{-n}k_i\le g_i(x)<2^{-n}(k_i+1)$ for every $i\in J$ and $x\in H$.
So

\Centerline{$2^{-n}k_i\le\Bover{\mu_i(E\cap H)}{\mu H}<2^{-n}(k_i+1)$}

\noindent for each $i$.   Accordingly

$$\eqalign{\sum_{H\in\Cal A_{\Lambda}}\mu(H\cap F)
   \cdot\prod_{i\in J}2^{-n}k_i
&\le\phi_{\Lambda}(F,\family{i}{J}{E_i})\cr
&\le\sum_{H\in\Cal A_{\Lambda}}\mu(H\cap F)\cdot\prod_{i\in J}
   \min(1,2^{-n}(k_i+1)),\cr}$$

\noindent that is,

\Centerline{$\int_F\prod_{i\in J}g'_{in}d\mu
\le\phi_{\Lambda}(F,\family{i}{J}{E_i})
\le\int_F\prod_{i\in J}g''_{in}d\mu$,}

\noindent where $g'_{in}(x)=2^{-n}k$, $g''_{in}(x)=\min(1,2^{-n}(k+1))$
when $2^{-n}k\le g_i(x)<2^{-n}(k+1)$.   But this means that

$$\eqalignno{|\phi_{\Lambda}(F,\family{i}{J}{E_i})
  -\int_F\prod_{i\in J}g_id\mu|
&\le\max(\int|\prod_{i\in J}g'_{in}-\prod_{i\in J}g_i|d\mu,
  \int|\prod_{i\in J}g''_{in}-\prod_{i\in J}g_i|d\mu)\cr
&\le\max(\int\sum_{i\in J}|g'_{in}-g_i|d\mu,
  \int\sum_{i\in J}|g''_{in}-g_i|d\mu)\cr
\displaycause{because all the $g_i$, $g'_{in}$, $g''_{in}$ take values
in $[0,1]$}
&\le 2^{-n}\#(J).\cr}$$

\noindent Since this is true for every $\Lambda\supseteq\Lambda_n$ and
every $n\in\Bbb N$,
$\lim_{\Lambda\uparrow}\phi_{\Lambda}(F,\family{i}{J}{E_i})
=\int_F\prod_{i\in J}g_id\mu$.\ \Qed

\medskip

{\bf (b)} Accordingly the condition given exactly matches the
definition in 458A.
}%end of proof of 458G

\leader{458H}{}\cmmnt{ All the fundamental theorems concerning
stochastic independence have relativized forms.   A simple one is the
following.

\medskip

\noindent}{\bf Proposition}\cmmnt{ (Compare 272K.)}  Let
$(X,\Sigma,\mu)$ be a probability space and $\Tau$ a $\sigma$-subalgebra of
$\Sigma$.
Let $\familyiI{\Sigma_i}$ be a family of $\sigma$-subalgebras of
$\Sigma$ which is relatively independent over $\Tau$.   Let
$\family{j}{J}{I_j}$ be a partition of $I$, and for each $j\in J$ let
$\tilde\Sigma_j$ be $\bigvee_{i\in I_j}\Sigma_i$.

(a) If $\familyiI{\Sigma_i}$ is relatively independent over $\Tau$,
then $\family{j}{J}{\tilde\Sigma_j}$ is
relatively independent over $\Tau$.

(b) Suppose that
$\family{j}{J}{\tilde\Sigma_j}$ is relatively independent over $\Tau$ and
that $\family{i}{I_j}{\Sigma_i}$ is relatively independent over $\Tau$ for
every $j\in J$.   Then $\familyiI{\Sigma_i}$ is relatively independent
over $\Tau$.

\proof{ For each $E\in\Sigma$ let $f_E$ be a conditional expectation of
$\chi E$ on $\Tau$.

\medskip

{\bf (a)} Take any finite $K\subseteq J$, and let $\pmb{W}$
be the set of families $\family{j}{K}{W_j}$ such that
$W_j\in\tilde\Sigma_j$ for each $j\in K$ and
$\mu(F\cap\bigcap_{j\in K}W_j)=\int_F\prod_{j\in K}f_{W_j}d\mu$ for
every $F\in\Tau$.   For each $j\in K$, let $\Cal C_j$ be the family of
measurable cylinders
expressible as $W=X\cap\bigcap_{i\in L}E_i$ where $L\subseteq I_j$
is finite and $E_i\in\Sigma_i$ for $i\in L$.   Note that in this case

\Centerline{$\mu(F\cap W)
=\mu(F\cap\bigcap_{i\in L}E_i)=\int_F\prod_{i\in L}f_{E_i}d\mu$}

\noindent for every $F\in\Tau$, so $f_W\eae\prod_{i\in L}f_{E_i}$,
taking the product to be $\chi X$ if $L$ is empty.

If $W_j\in\Cal C_j$ for each $j\in K$, then
$\family{j}{K}{W_j}\in\pmb{W}$.   \Prf\ Express $W_j$ as
$X\cap\bigcap_{i\in L_j}E_i$ where $L_j\subseteq I_j$ is finite and
$E_i\in\Sigma_i$ whenever $j\in K$ and $i\in L_j$.   Then,
setting $L=\bigcup_{j\in K}L_j$,

$$\eqalignno{\mu(F\cap\bigcap_{j\in K}W_j)
&=\mu(F\cap\bigcap_{i\in L}E_i)
=\int_F\prod_{i\in L}f_{E_i}d\mu\cr
\displaycause{because $\familyiI{\Sigma_i}$ is relatively independent}
&=\int_F\prod_{j\in K}\prod_{i\in L_j}f_{E_i}d\mu
=\int_F\prod_{j\in K}f_{W_i}d\mu\cr}$$

\noindent for every $F\in\Tau$.\ \Qed

Observe next that if we fix $k\in K$, and a family
$\family{j}{K\setminus\{k\}}{W_j}$, then the set of those
$W_k\in\tilde\Sigma_k$ such that $\family{j}{K}{W_j}\in\pmb{W}$ is a
Dynkin class, so if it includes $\Cal C_k$ it must include the
$\sigma$-algebra generated by $\Cal C_k$, viz., $\tilde\Sigma_k$.   Now
an easy induction on $n$ shows that if
$\family{j}{K}{W_j}\in\prod_{j\in K}\tilde\Sigma_j$ and
$\#(\{j:W_j\notin\Cal C_j\})=n$, then $\family{j}{K}{W_j}\in\pmb{W}$.
Taking $n=\#(K)$ we see that
$\prod_{j\in K}\tilde\Sigma_j\subseteq\pmb{W}$.

As this is true for every finite $K\subseteq J$,
$\family{j}{J}{\tilde\Sigma_j}$ is relatively independent over $\Tau$,
as claimed.

\medskip

{\bf (b)} This time, let $K\subseteq I$ be a
non-empty finite set, and $E_i\in\Sigma_i$ for $i\in K$.   Set
$L=\{j:j\in J$, $K\cap I_j\ne\emptyset\}$, and for $j\in L$ set
$G_j=\bigcap_{i\in K\cap I_j}E_i$;  set
$E=\bigcap_{i\in J}E_i=\bigcap_{j\in L}G_j$.
Because $\family{i}{I_j}{\Sigma_i}$ is
relatively independent over $\Tau$,
$f_{G_j}\eae\prod_{i\in K\cap I_j}f_{E_i}$.   Because
$\family{j}{J}{\tilde\Sigma_j}$ is relatively independent over
$\Tau$,

\Centerline{$f_E\eae\prod_{j\in L}f_{G_j}\eae\prod_{i\in K}f_{E_i}$.}

\noindent As $\family{i}{K}{E_i}$ is arbitrary, we have the result.
}%end of proof of 458H

\leader{458I}{}\cmmnt{ For the next, we need
a concept of `relative probability distribution', as follows.

\medskip

\noindent}{\bf Definition} Let $(X,\Sigma,\mu)$ be a probability space,
$\Tau$ a $\sigma$-subalgebra of $\Sigma$, and
$f\in\eusm L^0(\mu)$.   Then a {\bf relative distribution} of $f$ over
$\Tau$ will be a family $\family{x}{X}{\nu_x}$ of Radon probability
measures on $\Bbb R$ such that $x\mapsto\nu_x(H):X\to[0,1]$ is
$\Tau$-measurable and $\int_F\nu_x(H)\mu(dx)=\mu(F\cap f^{-1}[H])$
for every Borel set $H\subseteq\Bbb R$ and every $F\in\Tau$\cmmnt{,
that is,
$x\mapsto\nu_xH$ is a conditional expectation of $\chi f^{-1}[H]$ on
$\Tau$}.

\leader{458J}{Theorem} Let $(X,\Sigma,\mu)$ be a probability space,
$\Tau$ a $\sigma$-subalgebra of $\Sigma$, and $f\in\eusm L^0(\mu)$.
Then there is a relative distribution of $f$ over $\Tau$, which is
essentially unique in the sense that if
$\family{x}{X}{\nu_x}$ and $\family{x}{X}{\nuprime_x}$ are two such
families, then $\nu_x=\nuprime_x$ for $\mu\restrp\Tau$-almost every $x$.

\proof{{\bf (a)} Write $\mu_0$ for the restriction of $\mu$ to $\Tau$,
$\hat\mu$ for the completion of $\mu$, $\hat\Sigma$ for the domain of
$\hat\mu$, and $\Cal B$ for the Borel $\sigma$-algebra of $\Bbb R$.
Then the function $x\mapsto (x,f(x)):\dom f\to X\times\Bbb R$ is
$(\hat\Sigma,\Tau\tensorhat\Cal B)$-measurable, just because
$F\cap f^{-1}[H]\in\hat\Sigma$ for every $F\in\Tau$ and $H\in\Cal B$.
So we have a probability measure $\lambda$ on $X\times\Bbb R$ defined by
setting
$\lambda W=\mu\{x:(x,f(x))\in W\}$ for every $W\in\Tau\tensorhat\Cal B$.
The marginal measure on $\Bbb R$ is tight
just because it is a Borel probability measure (433Ca).
By 452M, we have a family $\family{x}{X}{\lambda_x}$ of Radon
probability measures on $\Bbb R$ such that
$\lambda W=\int\lambda_xW[\{x\}]\mu_0(dx)$ for every
$W\in\Tau\tensorhat\Cal B$.

\medskip

{\bf (b)} The functions $x\mapsto\lambda_xH$ need not be
$\Tau$-measurable.   However, if we set

\Centerline{$g_q(x)=\lambda_x(\ocint{-\infty,q})$}

\noindent for $q\in\Bbb Q$, then every $g_q$ is $\mu_0$-virtually
measurable, so there is a $\mu_0$-conegligible set $G$ such that every
$g_q\restr G$ is $\Tau$-measurable.   By the Monotone Class Theorem
(136B), the family

\Centerline{$\{H:H\in\Cal B,\,x\mapsto\lambda_xH:G\to[0,1]$ is
$\Tau$-measurable$\}$}

\noindent is the whole of $\Cal B$.    So if we set $\nu_x=\lambda_x$
for $x\in G$, and take $\lambda_x$ to be the Dirac measure concentrated at
$0$ (for
instance) for $x\in X\setminus G$, we shall have a relative distribution
of $f$ over $\Tau$ as defined in 458I.

\medskip

{\bf (c)} Now suppose that $\family{x}{X}{\nuprime_x}$ is another
relative
distribution of $f$ over $\Tau$.   Then for each $H\in\Cal B$ we have
$\int_F\nu_xH\mu(dx)=\int_F\nuprime_xH\mu(dx)$ for every $F\in\Tau$, so
that
$\nu_xH=\nuprime_xH$ for $\mu_0$-almost every $x$.   But this means that
for $\mu_0$-almost every $x$, we have $\nu_xH=\nuprime_xH$ for every
interval $H$ with rational endpoints;  and for such $x$ we must have
$\nu_x=\nuprime_x$ (415H(v)).
}%end of proof of 458J

\leader{458K}{}\cmmnt{ Now we can state and prove a result
corresponding to 272G.

\medskip

\noindent}{\bf Theorem} Let $(X,\Sigma,\mu)$ be a probability space,
$\Tau$ a $\sigma$-subalgebra of $\Sigma$, and
$\familyiI{f_i}$ a family in $\eusm L^0(\mu)$.   For each $i\in I$, let
$\family{x}{X}{\nu_{ix}}$ be a relative distribution of $f_i$ over
$\Tau$, and $\tilde f_i:X\to\Bbb R$ an arbitrary extension of $f_i$ to
the whole of $X$.   Then the following are equiveridical:

(i) $\familyiI{f_i}$ is relatively independent over $\Tau$;

(ii) for any Baire set $W\subseteq\BbbR^I$ and any $F\in\Tau$,

\Centerline{$\hat\mu(F\cap\pmb{f}^{-1}[W])
=\int_F\lambda_xW\mu(dx)$,}

\noindent where $\hat\mu$ is the completion of $\mu$,
$\pmb{f}(x)=\familyiI{\tilde f_i(x)}$ for $x\in X$, and $\lambda_x$ is
the product of $\familyiI{\nu_{ix}}$ for each $x$;

(iii) for any non-negative Baire measurable function $h:\BbbR^I\to\Bbb R$
and any $F\in\Tau$,

\Centerline{$\int_Fh\pmb{f}d\mu
=\int_F\int h\,d\lambda_x\mu(dx)$.}


\proof{{\bf (a)} Note first that if $i\in I$ and $H\subseteq\Bbb R$ is a
Borel set, then $\int_F\nu_{ix}H\mu(dx)=\hat\mu(F\cap f_i^{-1}[H])$ for
every $F\in\Tau$, so $x\mapsto\nu_{ix}H$ is a conditional expectation of
$\chi f_i^{-1}[H]$ on $\Tau$.

Suppose that $\familyiI{f_i}$ is relatively independent, and $F\in\Tau$.
Let $\Cal C$ be the family of Baire measurable cylinders
of $\BbbR^I$ expressible in the
form $C=\{z:z\in\BbbR^I,\,z(i)\in H_i$ for every $i\in J\}$ where
$J\subseteq I$ is finite and $H_i\subseteq\BbbR$ is a Borel set for each
$i\in J$.   For such a set $C$,

$$\eqalignno{\hat\mu(F\cap\pmb{f}^{-1}[C])
&=\hat\mu(F\cap\bigcap_{i\in J}\tilde f_i^{-1}[H_i])
=\int_F\prod_{i=0}^n\nu_{ix}H_i\mu(dx)\cr
\displaycause{interpreting an empty product as $\chi X$}
&=\int_F\lambda_xC\mu(dx).\cr}$$

\noindent So

\Centerline{$\Cal W=\{W:W\subseteq\BbbR^I$,
  $\mu(F\cap\pmb{f}^{-1}[W])=\biggerint\lambda_xW\mu(dx)\}$}

\noindent includes $\Cal C$;  since it
is a Dynkin class, it contains every Baire subset of $\BbbR^I$
(by the Monotone Class Theorem, 136B), and (ii) is true.

\medskip

{\bf (b)} Now suppose that (ii) is true.   Let $\Sigma_i$ be the
$\sigma$-algebra defined by $f_i$ for each $i$.   If $J\subseteq I$ is
finite and $E_i\in\Sigma_i$
for each $i\in J$, then there are Borel sets $H_i\subseteq\Bbb R$ such
that $E_i\symmdiff f_i^{-1}[H_i]$ is negligible for each $i$, so that
$x\mapsto\nu_{ix}H_i$ is a conditional expectation of $\chi E_i$ on
$\Tau$.   Now by the same equations as before, in the opposite
direction,

$$\eqalignno{\int_F\prod_{i\in J}\nu_{ix}H_i\mu(dx)
&=\int_F\lambda_xC\mu(dx)\cr
\displaycause{where $C=\{z:z(i)\in H_i$ for $i\in J\}$}
&=\hat\mu(F\cap\pmb{f}^{-1}[C])
=\hat\mu(F\cap\bigcap_{i\in J}f_i^{-1}[H_i])
=\hat\mu(F\cap\bigcap_{i\in J}E_i)\cr}$$

\noindent for every $F\in\Tau$.   As $\family{i}{J}{E_i}$ is arbitrary,
$\familyiI{\Sigma_i}$ and $\familyiI{f_i}$ are
relatively independent.

\medskip

{\bf (c)} Thus (i)$\Leftrightarrow$(ii).   For (ii)$\Rightarrow$(iii), observe that
(ii) covers the case in which $h$ is an indicator function $\chi W$;
for the general case, express $h$ as the supremum of a non-decreasing
sequence of linear combinations of indicator functions, as usual.
And (iii)$\Rightarrow$(ii) is trivial.
}%end of proof of 458K

\cmmnt{\medskip

\noindent{\bf Remarks} Of course the ungainly shift to $\tilde f_i$ is
unnecessary if $I$ is countable;  but for uncountable $I$ the
intersection $\bigcap_{i\in I}\dom f_i$, which is the only suitable
domain for $\pmb{f}$, may not be conegligible.

I said that $\lambda_x$ should be `the product of
$\familyiI{\nu_{ix}}$'.   Since the $\nu_{ix}$ are Radon probability
measures, we have two possible interpretations of this:  either the
`ordinary' product measure of \S254 or the `quasi-Radon' product measure
of \S417.   But as we are interested only in the values of $\lambda_xW$
for Baire sets $W$, it makes no difference which we use.
}%end of comment

\leader{458L}{Measure algebras}\dvAnew{2008,
with fragments from the former 4{}58H}\cmmnt{ We can look at
the same ideas
in the context of measure algebras.    Let $(\frak A,\bar\mu)$ be a
probability algebra and $\frak C$ a closed subalgebra of $\frak A$.

\medskip

}{\bf (a)} If $a\in\frak A$, then we can say that
$u\in L^{\infty}(\frak C)$ is the conditional expectation of $\chi a$ on
$\frak C$ if $\int_cu=\bar\mu(c\Bcap a)$ for every
$c\in\frak C$\cmmnt{ (365R)}.   Now we can say that a family
$\familyiI{b_i}$ in $\frak A$ is {\bf relatively (stochastically)
independent} over $\frak C$ if
$\bar\mu(c\Bcap\inf_{i\in J}b_i)=\int_c\prod_{i\in J}u_i$
whenever $J\subseteq I$ is a non-empty finite set
and $u_i$ is the conditional expectation of $\chi b_i$ on $\frak C$ for
every $i\in J$;  while a family
$\familyiI{\frak B_i}$ of subalgebras of $\frak A$ is {\bf relatively
(stochastically) independent over
$\frak C$} if $\familyiI{b_i}$ is relatively independent over $\frak C$
whenever $b_i\in\frak B_i$ for every $i\in I$.

Corresponding to 458Ab, we can say that a family $\familyiI{w_i}$ in
$L^0(\frak A)$ is {\bf relatively (stochastically) independent over
$\frak C$} if $\familyiI{\frak B_i}$ is relatively stochastically
independent, where $\frak B_i$ is the closed subalgebra of $\frak A$
generated by $\{\Bvalue{w_i>\alpha}:\alpha\in\Bbb R\}$ for each $i$.

Returning to the original form of these ideas, we say that a family
$\familyiI{b_i}$ in $\frak A$ is {\bf (stochastically) independent} if
it is relatively independent over $\{0,1\}$, that is, if
$\bar\mu(\inf_{i\in J}b_i)=\prod_{i\in J}\bar\mu b_i$ whenever
$J\subseteq I$ is finite.   Similarly, a family $\familyiI{\frak B_i}$ of
subalgebras of $\frak A$ is (stochastically) independent if
$\bar\mu(\inf_{i\in J}b_i)=\prod_{i\in J}\bar\mu b_i$ whenever
$J\subseteq I$ is finite and $b_i\in\frak B_i$ for every $i$.

\spheader 458Lb
Let $(X,\Sigma,\mu)$ be a probability space and
$(\frak A,\bar\mu)$ its measure algebra.   Let $\familyiI{E_i}$,
$\familyiI{\Sigma_i}$ and $\familyiI{f_i}$ be, respectively, a family in
$\Sigma$, a family of subalgebras of $\Sigma$, and a family of
$\mu$-virtually measurable real-valued functions defined almost everywhere
on $X$;  let $\Tau$ be a $\sigma$-subalgebra of $\Sigma$.   For $i\in I$,
set $a_i=E_i^{\ssbullet}\in\frak A$,
$\frak B_i=\{E^{\ssbullet}:E\in\Sigma_i\}$, and
$w_i=f_i^{\ssbullet}\in L^0(\frak A)$, identified with
$L^0(\mu)$\cmmnt{ (364Ic\formerly{3{}64Jc})}.
Let $\Tau$ be a $\sigma$-subalgebra of
$\frak A$ and $\frak C=\{F^{\ssbullet}:F\in\Tau\}$.   Then

\inset{$\familyiI{a_i}$ is relatively independent over $\frak C$ iff
$\familyiI{E_i}$ is relatively independent over $\Tau$,

$\familyiI{\frak B_i}$ is relatively independent over $\frak C$ iff
$\familyiI{\Sigma_i}$ is relatively independent over $\Tau$,

$\familyiI{w_i}$ is relatively independent over $\frak C$ iff
$\familyiI{f_i}$ is relatively independent over $\Tau$.}

\prooflet{\noindent\Prf\ The point is that if $f\in\eusm L^1(\mu)$ (in
particular, if $f=\chi E$ for some $E\in\Sigma$), and
$g\in\eusm L^1(\mu\restrp\Tau)\subseteq\eusm L^1(\mu)$ is a conditional
expectation of $f$ on $\Tau$, then $g^{\ssbullet}$ is a conditional
expectation of $f^{\ssbullet}$ on $\frak C$;  see 242J and 365R.\ \Qed}

\spheader 458Lc
Corresponding to 458B, we see that if
$\familyiI{\frak A_i}$ is a family of subalgebras of $\frak A$ such that
$\frak C\subseteq\bigcup_{i\in I}\frak A_i$, and
$\int\prod_{i\in J}u_id\bar\mu=\bar\mu(\inf_{i\in J}a_i)$ whenever
$J\subseteq I$ is finite and not empty and $a_i\in\frak A_i$,
$u_i\in L^{\infty}(\frak C)$ is a conditional expectation of $\chi a_i$ on
$\frak C$ for each $i$, then $\familyiI{\frak A_i}$ is relatively
independent over $\frak C$.

\spheader 458Ld
Corresponding to 458Db, we see that if $\familyiI{\frak B_i}$ is a
family of subalgebras of $\frak A$  which is relatively independent over
$\frak C$, and $\frak B_i^*$ is the closed subalgebra of $\frak A$
generated by $\frak B_i\cup\frak C$ for each $i$, then
$\familyiI{\frak B_i^*}$ is relatively independent over $\frak C$.
\prooflet{The most natural proof, from where we are now
standing, is to express $(\frak A,\bar\mu)$ as the measure algebra of a
probability space $(X,\Sigma,\mu)$, set
$\Tau=\{F:F^{\ssbullet}\in\frak C\}$ and
$\Sigma_i=\{E:E^{\ssbullet}\in\frak B_i\}$ for each $i\in I$, and use
458D.}

Corresponding to 458Dc, we see that if $\familyiI{\frak B_i}$ is a family
of subalgebras of $\frak A$ which is relatively independent over $\frak C$,
$D_i\subseteq\frak B_i$ for every $i\in I$, and $\frak D$ is the closed
subalgebra of $\frak A$ generated by $\frak C\cup\bigcup_{i\in I}D_i$, then
$\familyiI{\frak B_i}$ is relatively independent over $\frak D$.

\spheader 458Le Following 458H, we have the result that if
$\familyiI{\frak B_i}$ is relatively independent over $\frak C$, and
$\family{j}{J}{I_j}$ is a partition of $I$, and $\tilde{\frak B}_j$ is
the closed subalgebra of $\frak A$ generated by
$\bigcup_{i\in I_j}\frak B_i$ for every $j\in J$, then
$\family{j}{J}{\tilde{\frak B}_j}$ is relatively independent over
$\frak C$.

\spheader 458Lf Note that\cmmnt{ if $a\in\frak A$ and $u$ is the
conditional expectation of $\chi a$ on $\frak C$, then
$\Bvalue{u>0}=\upr(a,\frak C)$, by 365Rc.   So} if
$\familyiI{\frak B_i}$ is a family of subalgebras of $\frak A$ which is
relatively independent over $\frak C$, and $J\subseteq I$ is finite, and
$b_i\in\frak B_i$ for each $i\in J$, then $\inf_{i\in J}b_i=0$ iff
$\inf_{i\in J}\upr(b_i,\frak C)=0$.   \prooflet{(If $u_i$ is a
conditional expectation of $\chi b_i$ on $\frak C$ for each $i$, then

\Centerline{$\inf_{i\in J}\upr(b_i,\frak C)
=\inf_{i\in J}\Bvalue{u_i>0}
=\Bvalue{\prod_{i\in J}u_i>0}$}

\noindent is zero iff
$\bar\mu(\inf_{i\in J}b_i)=\int\prod_{i\in J}u_i=0$.)}

\spheader 458Lg\dvAformerly{4{}58Hc}\cmmnt{ We have a straightforward 
version of 458E, as follows.}   If $\familyiI{\frak C_i}$
is a stochastically independent family of closed subalgebras of
$\frak A$, $\frak C$ is independent of
the algebra generated by $\bigcup_{i\in I}\frak C_i$, and $\frak B_i$ is
the closed subalgebra of $\frak A$ generated by $\frak C\cup\frak C_i$
for each $i$, then $\familyiI{\frak B_i}$ is relatively independent over
$\frak C$.   \prooflet{(Either repeat the proof of 458E, looking at
$\frak B_{i0}=\frak C_i$ and $\frak B_{i1}=\frak C$ for each $i$, or
move to a measure space representing $\frak A$ and quote 458E.)}

\spheader 458Lh\cmmnt{ Similarly, we can translate 458F into this
language.}  Let
$P:L^1(\frak A,\bar\mu)\to L^1(\frak C,\bar\mu\restrp\frak C)
\subseteq L^1(\frak A,\bar\mu)$
be the conditional expectation operator associated with
$\frak C$\cmmnt{ (365R)}.   Suppose that
$\familyiI{\frak B_i}$ is a family of closed subalgebras of $\frak A$ which
is relatively independent over $\frak C$.   Then

\Centerline{$\int_c\prod_{j=0}^nPu_j\le\int_c\prod_{j=0}^nu_j$}

\noindent whenever $c\in\frak C$, $i_0,\ldots,i_n\in I$ and
$u_j\in L^1(\frak B_{i_j},\bar\mu\restrp\frak B_{i_j})^+$ for each
$j\le n$, with equality if $i_0,\ldots,i_n$ are distinct.

\leader{458M}{Proposition}\dvAnew{2012}
Let $(\frak A,\bar\mu)$ be a probability algebra and
$\frak B$, $\frak C$ closed subalgebras of $\frak A$.   Write
$P_{\frak B}$, $P_{\frak C}$ and $P_{\frak B\cap\frak C}$
for the conditional expectation operators associated with
$\frak B$, $\frak C$ and $\frak B\cap\frak C$.   Then the following are
equiveridical:

(i) $\frak B$ and $\frak C$ are relatively independent over
$\frak B\cap\frak C$;

(ii) $P_{\frak B\cap\frak C}(v\times w)
=P_{\frak B\cap\frak C}v\times P_{\frak B\cap\frak C}w$ whenever
$v\in L^{\infty}(\frak B)$ and $w\in L^{\infty}(\frak C)$;

(iii) $P_{\frak B}P_{\frak C}=P_{\frak B\cap\frak C}$;

(iv) $P_{\frak B}P_{\frak C}=P_{\frak C}P_{\frak B}$;

(v) $P_{\frak B}u\in L^0(\frak C)$
for every $u\in L^1(\frak C,\bar\mu\restrp\frak C)$.

\proof{ Write $P$ for $P_{\frak B\cap\frak C}$.

\medskip

{\bf (i)$\Rightarrow$(ii)} If (i) is true,
$v\in L^{\infty}(\frak B)$ and $w\in L^{\infty}(\frak C)$, then
$Pv\times Pw$ certainly belongs to
$L^{\infty}(\frak B\cap\frak C)$, and if $d\in\frak B\cap\frak C$,
$\int_dPv\times Pw=\int_dv\times w$ by 458Lh.   So
$Pv\times Pw=P(v\times w)$.

\medskip

{\bf (ii)$\Rightarrow$(i)} If (ii) is true, $b\in\frak B$, $c\in\frak C$
and $d\in\frak B\cap\frak C$, then

\Centerline{$\bar\mu(d\Bcap b\Bcap c)=\int_d\chi b\times\chi c
=\int_dP(\chi b\times\chi c)=\int_dP(\chi b)\times P(\chi c)$}

\noindent as required by the definition in 458La.

\medskip

{\bf (ii)$\Rightarrow$(iii)} Suppose that (ii) is true.
First note that if $w\in L^{\infty}(\frak C)$ then
$Pw=P_{\frak B}w$.   \Prf\ Of course
$Pw\in L^{\infty}(\frak B\cap\frak C)\subseteq L^{\infty}(\frak B)$.
If $b\in\frak B$, then

$$\eqalignno{\int_bw
&=\int\chi b\times w
=\int P(\chi b\times w)
=\int P\chi b\times Pw
=\int\chi b\times Pw\cr
\displaycause{365Ra}
&=\int_bPw,\cr}$$

\noindent so that $Pw$ possesses the defining properties of
$P_{\frak B}w$.\ \Qed

But this means that if $u\in L^{\infty}(\frak A)$,
$P_{\frak B}P_{\frak C}u=PP_{\frak C}u$, which in turn is equal to
$Pu$ just because $\frak B\cap\frak C\subseteq\frak C$ (see 233Eh).
As $u$ is arbitrary, $P_{\frak B}P_{\frak C}$ agrees with $P$ on
$L^{\infty}(\frak A)$;  but $L^{\infty}(\frak A)$ is $\|\,\|_1$-dense in
$L^1(\frak A,\bar\mu)$, and $P_{\frak B}P_{\frak C}$ and $P$ are both
$\|\,\|_1$-continuous, so they agree everywhere on
$L^1(\frak A,\bar\mu)$ and are equal, as required by (iii).

\medskip

{\bf (iii)$\Rightarrow$(ii)} Suppose that (iii) is true, and that
$v\in L^{\infty}(\frak B)$, $w\in L^{\infty}(\frak C)$ and
$d\in\frak B\cap\frak C$.   Then

$$\eqalignno{\int_dPv\times Pw
&=\int\chi d\times Pv\times Pw
=\int\chi d\times v\times Pw\cr
\displaycause{because $\chi d\times Pw\in L^{\infty}(\frak B\cap\frak C)$}
&=\int\chi d\times v\times P_{\frak B}w\cr
\displaycause{because $P_{\frak C}w=w$}
&=\int\chi d\times v\times w\cr
\displaycause{because $\chi d\times v\in L^{\infty}(\frak B)$}
&=\int_dv\times w.\cr}$$

\noindent As $d$ is arbitrary and
$Pv\times Pw\in L^{\infty}(\frak B\cap\frak C)$,
$Pv\times Pw=P(v\times w)$.

\medskip

{\bf (i)$\Rightarrow$(iv)} follows immediately from (i)$\Rightarrow$(iii)
and the symmetry of the relation `$\frak B$ and $\frak C$ are
relatively independent over $\frak B\cap\frak C$'.

\medskip

{\bf (iv)$\Rightarrow$(v)} If (iv) is true and
$u\in L^1(\frak C,\bar\mu\restrp\frak C)$, then

\Centerline{$P_{\frak B}u=P_{\frak B}P_{\frak C}u
=P_{\frak C}P_{\frak B}u\in L^0(\frak C)$,}

\noindent so (v) is true.

\medskip

{\bf (v)$\Rightarrow$(iii)} If (v) is true, and
$u\in L^1_{\bar\mu}$, then
$P_{\frak C}u\in L^1(\frak C,\bar\mu\restrp\frak C)$, so
$P_{\frak B}P_{\frak C}u$ belongs to
$L^0(\frak C)\cap L^0(\frak B)=L^0(\frak B\cap\frak C)$, and of course

\Centerline{$\int_dP_{\frak B}P_{\frak C}u
=\int_dP_{\frak C}u=\int_du$}

\noindent for every
$d\in\frak B\cap\frak C$.   So $P_{\frak B}P_{\frak C}u=Pu$.
As $u$ is arbitrary, (iii) is true.
}%end of proof of 458M

\leader{458N}{Relative free products of probability algebras:
Definition}\dvAformerly{4{}58I} Let
$\familyiI{(\frak A_i,\bar\mu_i)}$ be a family of probability algebras
and $(\frak C,\bar\nu)$ a probability algebra, and suppose that we are
given a measure-preserving Boolean homomomorphism
$\pi_i:\frak C\to\frak A_i$
for each $i\in I$.   A {\bf relative free product} of
$\familyiI{(\frak A_i,\bar\mu_i,\pi_i)}$ over
$(\frak C,\bar\nu)$ is a probability algebra
$(\frak A,\bar\mu)$, together with a measure-preserving Boolean
homomorphism $\phi_i:\frak A_i\to\frak A$ for each $i\in I$, such that

\inset{$\frak A$ is the closed subalgebra of itself generated by
$\bigcup_{i\in I}\phi_i[\frak A_i]$,

$\phi_i\pi_i=\phi_j\pi_j:\frak C\to\frak A$ for all $i$,
$j\in I$,

writing $\frak D$ for the common value of the $\phi_i[\pi_i[\frak C]]$,
$\familyiI{\phi_i[\frak A_i]}$ is relatively independent over
$\frak D$.}

\cmmnt{\medskip

\noindent{\bf Remark} The homomorphisms $\pi_i$ and $\phi_i$ are essential
for the formal content of this definition, and will necessarily appear in
the basic result 458O;  but conceptually they are a nuisance;  we should
much prefer to think of every $\frak A_i$ as a subalgebra of $\frak A$, and
of $\frak C$ as actually equal to $\frak D$.   It may help if I spell out
the key condition `$\familyiI{\phi_i[\frak A_i]}$ is relatively independent
over $\frak D$' in terms of $\frak C$ and the $\pi_i$.

The common value $\pi$ of the
$\phi_i\pi_i$ is a measure-preserving isomorphism between $\frak C$ and
$\frak D$, so gives rise to an $f$-algebra isomorphism
$S:L^0(\frak C)\to L^0(\frak D)$ such that
$S(\chi c)=\chi(\pi c)$ for every $c\in\frak C$ (364P\formerly{3{}64R});
note that $S[L^{\infty}(\frak C)]=L^{\infty}(\frak D)$ and
$\int Su\,d\bar\mu=\int u\,d\bar\nu$ for every $u\in L^1(\frak C)$
(365O).   If $u\in L^{\infty}(\frak C)$ and $d\in\frak D$, then

$$\eqalign{\int_dSu\,d\bar\mu
&=\int Su\times\chi d\,d\bar\mu
=\int Su\times S(\chi(\pi^{-1}d))d\bar\mu\cr
&=\int S(u\times\chi(\pi^{-1}d))d\bar\mu
=\int u\times\chi(\pi^{-1}d)\,d\bar\nu
=\int_{\pi^{-1}d}u\,d\bar\nu.\cr}$$

Next, for $i\in I$ and
$a\in\frak A_i$, we have a completely additive functional
$c\mapsto\bar\mu_i(a\Bcap\pi_ic):\frak C\to[0,1]$;  let
$u_{ia}\in L^{\infty}(\frak C)$ be a corresponding Radon-Nikod\'ym
derivative, so that $\int_cu_{ia}d\bar\nu=\bar\mu_i(a\Bcap\pi_ic)$ for
every $c\in\frak C$ (365E).   (Thus $u_{ia}\in L^{\infty}(\frak C)$
corresponds to the conditional expectation of $\chi a$ on the algebra
$\pi_i[\frak C]\subseteq\frak A_i$.)    The image $Su_{ia}$ in
$L^{\infty}(\frak D)$ is defined by the property

\Centerline{$\int_dSu_{ia}d\bar\mu
=\int_{\pi^{-1}d}u_{ia}d\bar\nu
=\bar\mu_i(a\Bcap\pi_i(\phi_i\pi_i)^{-1}d)
=\bar\mu(\phi_ia\Bcap d)$}

\noindent for every $d\in\frak D$;  that is, $Su_{ia}$ is the conditional
expectation of $\chi(\phi_ia)$ on $\frak D$.

Note also that $\frak D\subseteq\phi_i[\frak A_i]$ for every $i\in I$.
So we can use the criterion of 458B/458Lc to see that

$$\eqalign{\familyiI{\phi_i[\frak A_i]}&\text{ is relatively independent
  over }\frak D\cr
&\text{iff }\bar\mu(\inf_{i\in J}\phi_ia_i)
  =\int\prod_{i\in J}Su_{i,a_i}d\bar\mu\cr
&\mskip50mu\text{ whenever }J\subseteq I
  \text{ is finite and not empty and }a_i\in\frak A_i\text{ for }i\in J\cr
&\text{iff }\bar\mu(\inf_{i\in J}\phi_ia_i)
  =\int\prod_{i\in J}u_{i,a_i}d\bar\nu\cr
&\mskip50mu\text{ whenever }J\subseteq I
  \text{ is finite and not empty and } a_i\in\frak A_i
  \text{ for }i\in J\cr}$$

\noindent because $S$ is multiplicative, so we always have

\Centerline{$\int\prod_{i\in J}Su_{i,a_i}d\bar\mu
=\int S(\prod_{i\in J}u_{i,a_i})d\bar\mu
=\int\prod_{i\in J}u_{i,a_i}d\bar\nu$.}
}%end of comment

\leader{458O}{Theorem}\dvAformerly{4{}58J} Let
$\familyiI{(\frak A_i,\bar\mu_i)}$ be a family of probability algebras,
$(\frak C,\bar\nu)$ a probability algebra and
$\pi_i:\frak C\to\frak A_i$ a
measure-preserving Boolean homomomorphism for each $i\in I$.  Then
$\familyiI{(\frak A_i,\bar\mu_i,\pi_i)}$ has an essentially unique
relative free product over $(\frak C,\bar\nu)$.

\proof{{\bf (a)(i)} Let $\frak B$ be the free product of
$\familyiI{\frak A_i}$ (315I\formerly{3{}15H});  write
$\varepsilon_i:\frak A_i\to\frak B$ for the canonical embedding of
$\frak A_i$ in $\frak B$.   For each $i\in I$, $a\in\frak A_i$ let
$u_{ia}\in L^{\infty}(\frak C)$ be such that
$\int_cu_{ia}d\bar\nu=\bar\mu_i(a\Bcap\pi_ic)$ for
every $c\in\frak C$ (458N).   Of course $u_{i1}=\chi 1$ in
$L^{\infty}(\frak C)$ for every $i\in I$, interpreting the `$1$' in
$u_{i1}$ in the Boolean algebra $\frak A_i$, and the `$1$' in $\chi 1$
in the Boolean algebra $\frak C$;  so (this time interpreting `$1$' in
$\frak B$) $\lambda 1=1$ (the final `$1$' being a real number, of
course).

Because the map
$a\mapsto u_{ia}:\frak A_i\to L^{\infty}(\frak C)$ is additive for each
$i$, 326E\formerly{3{}26Q} tells us that there is a unique additive
functional $\lambda:\frak B\to[0,1]$ such that

\Centerline{$\lambda(\inf_{i\in J}\varepsilon_ia_i)
=\biggerint\prod_{i\in J}u_{i,a_i}d\bar\nu$}

\noindent whenever $J\subseteq I$ is a non-empty finite set and
$a_i\in\frak A_i$ for every $i\in J$.

\medskip

\quad{\bf (ii)} By 392I, there are a probability algebra
$(\frak A,\bar\mu)$ and a
Boolean homomorphism $\phi:\frak B\to\frak A$ such that
$\lambda=\bar\mu\phi$.   We can of course suppose that $\frak A$ is the
order-closed subalgebra of itself generated by $\phi[\frak B]$ (which is
in fact automatically the case if we use the construction in the proof
of 392I).

For each $i\in I$, set $\phi_i=\phi\varepsilon_i:\frak A_i\to\frak A$.
It is a Boolean homomorphism because $\phi$ and $\varepsilon_i$ are.
If $a\in\frak A_i$, then

\Centerline{$\bar\mu\phi_ia
=\bar\mu(\phi\varepsilon_ia)=\lambda\varepsilon_ia
=\int u_{ia}d\bar\nu
=\bar\mu_i(a\Bcap\pi_i1)=\bar\mu_ia$,}

\noindent so $\phi_i$ is measure-preserving.

If $i$, $j\in I$ and $c\in\frak C$ then

$$\eqalign{\bar\mu(\phi_i\pi_ic\Bsymmdiff\phi_j\pi_jc)
&=\lambda(\varepsilon_i\pi_ic\Bsymmdiff\varepsilon_j\pi_jc)\cr
&=\lambda(\varepsilon_i\pi_ic)+\lambda(\varepsilon_j\pi_jc)
  -2\lambda(\varepsilon_i\pi_ic\Bcap\varepsilon_j\pi_jc)\cr
&=\int u_{i,\pi_ic}d\bar\nu+\int u_{j,\pi_jc}d\bar\nu
  -2\int u_{i,\pi_ic}\times u_{j,\pi_jc}d\bar\nu\cr
&=\int\chi c\,d\bar\nu+\int\chi c\,d\bar\nu
  -2\int\chi c\times\chi c\,d\bar\nu
=0.\cr}$$

\noindent So $\phi_i\pi_i=\phi_j\pi_j$.  Let $\frak D$ be the common value
of $\phi_i[\pi_i[\frak C]]$.   (In the trivial case $I=\emptyset$, take
$\frak D=\frak A=\{0,1\}$.)

\medskip

\quad{\bf (iii)} Suppose that
$J\subseteq I$ is finite and not empty and
that $a_i\in\frak A_i$ for each $i\in J$.   Then

$$\eqalign{\bar\mu(\inf_{i\in J}\phi_ia_i)
&=\lambda(\inf_{i\in J}\varepsilon_ia_i)
=\int\prod_{i\in J}u_{i,a_i}d\bar\nu\cr
&=\int\prod_{i\in J}u_{i,a_i}d\bar\nu
=\int\prod_{i\in J}u_{i,a_i}d\bar\nu.\cr}$$

\noindent But this is precisely the condition described in 458N, so
$\familyiI{\phi_i[\frak A_i]}$ is relatively independent over $\frak D$,
and $(\frak A,\bar\mu,\familyiI{\phi_i})$ is a relative free product of
$\familyiI{(\frak A_i,\bar\mu_i,\pi_i)}$ over $(\frak C,\bar\nu)$.

\medskip

{\bf (b)} Now suppose that $(\frak A',\bar\mu',\familyiI{\phi'_i})$ is
another relative free product of $\familyiI{(\frak A_i,\bar\mu_i,\pi_i)}$
over $(\frak C,\bar\nu)$.   Then we have a Boolean
homomorphism $\psi:\frak B\to\frak A'$ such that
$\phi'_i=\psi\varepsilon_i$ for every $i\in I$ (315J\formerly{3{}15I}).   In this case,
$\bar\mu'\psi=\lambda$.   \Prf\ Let $\pi'$ be the common value of
$\phi'_i\pi_i$ for $i\in I$, set $\frak D'=\pi'[\frak C]$, and let
$S':L^0(\frak C)\to L^0(\frak D')$ be the isomorphism corresponding to
$\pi':\frak C\to\frak D$.   If $a\in\frak A$ and $c\in\frak C$, then

\Centerline{$\bar\mu'(\phi'_ia\Bcap\pi'c)
=\bar\mu_i(a\Bcap\pi_ic)
=\bar\mu(\phi_ia\Bcap\pi c)
=\int_cu_{ia}d\bar\nu
=\int_{\pi'c}S'u_{ia}d\bar\mu'$}

\noindent by 365H (compare 458N above);  it follows that $S'u_{ia}$
is the conditional expectation of $\chi(\phi'_ia)$ on $\frak D'$.

If $J\subseteq I$ is finite and not empty, and $a_i\in\frak A_i$ for
$i\in J$, then

$$\eqalign{\bar\mu'(\psi(\inf_{i\in J}\varepsilon_ia_i))
&=\bar\mu'(\inf_{i\in J}\phi'_ia_i)
=\int\prod_{i\in J}S'u_{i,a_i}d\bar\mu'\cr
&=\int S'(\prod_{i\in J}u_{i,a_i})d\bar\mu'
=\int\prod_{i\in J}u_{i,a_i}d\bar\nu
=\lambda(\inf_{i\in J}\varepsilon_ia_i).\cr}$$

\noindent Because $\lambda$ is the only additive functional on $\frak B$
taking the right values on elements of this form,
$\bar\mu'\psi=\lambda$.\ \Qed

In particular, $\psi b=0$ whenever $b\in\frak B$ and $\lambda b=0$.  It
follows that $\psi b=\psi b'$ whenever $b$, $b'\in\frak B$ and
$\phi b=\phi b'$, since in this case
$\lambda(b\Bsymmdiff b')=\bar\mu(\phi b\Bsymmdiff\phi b')$ is zero.
So we have a function $\theta:\phi[\frak B]\to\frak A'$ defined by
setting $\theta(\phi b)=\psi b$ for every $b\in\frak B$, and of course
$\theta$ is a Boolean homomorphism;  moreover,

\Centerline{$\bar\mu'\theta(\phi b)=\bar\mu'\psi b=\lambda b
=\bar\mu\phi b$}

\noindent for every $b$, so $\theta$ is measure-preserving and an
isometry for the
measure metrics of $\frak A$ and $\frak A'$.   If $i\in I$ and
$a\in\frak A$, then

\Centerline{$\theta\phi_ia=\theta\phi\varepsilon_ia=\psi\varepsilon_ia
=\phi_i'a$,}

\noindent so $\theta\phi_i=\phi'_i$ for every $i$.   Because $\frak A$
and $\frak A'$ are the closed subalgebras generated by
$\bigcup_{i\in I}\phi_i[\frak A_i]$ and
$\bigcup_{i\in I}\phi'_i[\frak A_i]$ respectively, $\phi[\frak B]$ and
$\psi[\frak B]$ are dense (323J).   The isometry $\theta$ therefore
extends uniquely to a measure algebra isomorphism
$\hat\theta:\frak A\to\frak A'$ which must be the unique isomorphism
such that $\hat\theta\phi_i=\phi'_i$ for every $i$.   Thus
$(\frak A,\bar\mu,\familyiI{\phi_i})$ and
$(\frak A',\bar\mu',\familyiI{\phi'_i})$ are isomorphic, and the
relative free product is essentially unique.
}%end of proof of 458O

\leader{458P}{}\cmmnt{ Developing the argument of the last part of
the proof of 458O, we have the following.

\medskip

\noindent}{\bf Theorem} Let
$\familyiI{(\frak A_i,\bar\mu_i)}$, $\familyiI{(\frak A'_i,\bar\mu'_i)}$
be two families of probability algebras, and
$\psi_i:\frak A_i\to\frak A'_i$ a measure-preserving Boolean
homomorphism for each $i$.   Let $(\frak C,\bar\nu)$,
$(\frak C',\bar\nu')$ be probability algebras and
$\pi_i:\frak C\to\frak A_i$, $\pi'_i:\frak C'\to\frak A'_i$
measure-preserving Boolean homomomorphisms for each $i\in I$;  suppose
that we have a measure-preserving isomorphism $\psi:\frak C\to\frak C'$
such that $\pi'_i\psi=\psi_i\pi_i:\frak C\to\frak A'_i$ for each $i$.
Let $(\frak A,\bar\mu,\familyiI{\phi_i})$ and
$(\frak A',\bar\mu',\familyiI{\phi'_i})$ be relative free products of
$\familyiI{(\frak A_i,\bar\mu_i,\pi_i)}$,
$\familyiI{(\frak A'_i,\bar\mu'_i,\pi'_i)}$ over $(\frak C,\bar\nu)$,
$(\frak C',\bar\nu')$ respectively.   Then there is a unique
measure-preserving Boolean homomorphism $\hat\psi:\frak A\to\frak A'$
such that $\hat\psi\phi_i=\phi'_i\pi_i:\frak A_i\to\frak A'$ for every
$i\in I$.

\proof{ By the uniqueness assertion of 458O, we may suppose that
$(\frak A,\bar\mu,\familyiI{\phi_i})$ has been constructed by the method
in the proof of 458O.

\medskip

{\bf (a)} For $i\in A$, $a\in\frak A$, $a'\in\frak A'$ let
$u_{ia}\in L^{\infty}(\frak C)$, $u'_{ia'}\in L^{\infty}(\frak C')$ be
defined as in the proof of 458O, so that

\Centerline{$\int_cu_{ia}d\bar\nu=\bar\mu_i(a\Bcap\pi_ic)$,
\quad$\int_{c'}u'_{ia'}d\bar\nu'=\bar\mu'_i(a'\Bcap\pi'_ic')$}

\noindent whenever $c\in\frak C$ and $c'\in\frak C'$.   Let
$T:L^0(\frak C)\to L^0(\frak C')$ be the $f$-algebra isomorphism
such that $T(\chi c)=\chi(\psi c)$ for every
$c\in\frak C$.   Now $u'_{i,\psi_ia}=Tu_{ia}$ whenever
$i\in I$ and $a\in\frak A_i$.   \Prf\ If $c\in\frak C$, then

$$\eqalignno{\int_{\psi c}Tu_{ia}d\bar\nu'
&=\int_cu_{ia}d\bar\nu
=\bar\mu_i(a\Bcap\pi_ic)\cr
&=\bar\mu_i'\psi_i(a\Bcap\pi_ic)
=\bar\mu_i'(\psi_ia\Bcap\pi'_i\psi c)
=\int_{\psi c}u'_{i,\psi_ia}d\bar\nu'.\cr}$$

\noindent Because $\psi$ is surjective, it follows that
$Tu_{ia}=u'_{i,\psi_ia}$.\ \Qed

\medskip

{\bf (b)} Let $\frak B$ be the free product of $\familyiI{\frak A_i}$
and $\varepsilon_i:\frak A_i\to\frak B$ the canonical embedding for each
$i$;  let $\lambda$ be the functional on $\frak B$ defined by the
process of (a-i) in the proof of 458O.   By 315J, there is a Boolean
homomorphism $\theta:\frak B\to\frak A'$ such that
$\theta\varepsilon_i=\phi'_i\psi_i:\frak A_i\to\frak A'$ for every $i$.
Now $\bar\mu'\theta=\lambda$.   \Prf\ If $J\subseteq I$ is finite and
$a_i\in\frak A_i$ for every $i\in J$, then

$$\eqalignno{\bar\mu'\theta(\inf_{i\in J}\varepsilon_ia_i)
&=\bar\mu'(\inf_{i\in J}\phi'_i\psi_ia_i)
=\int\prod_{i\in J}u'_{i,\psi_ia_i}d\bar\nu'\cr
&=\int\prod_{i\in J}Tu_{i,a_i}d\bar\nu'
=\int T(\prod_{i\in J}u_{i,a_i})d\bar\nu'\cr
\displaycause{because $T\restr L^{\infty}(\frak C)$ is multiplicative}
&=\int\prod_{i\in J}u_{i,a_i}d\bar\nu
=\lambda(\inf_{i\in J}\varepsilon_ia_i).\cr}$$

\noindent As $\lambda$, $\theta$ and $\bar\nu$ are all additive,
$\lambda=\bar\nu\theta'$ (using 315Kb\formerly{3{}15J}).\ \Qed

\medskip

{\bf (c)} Let $\phi:\frak B\to\frak A$ be the map described in (a-ii) of
the proof of 458O.   Then $\bar\mu(\phi b)=\lambda b=\bar\mu'(\theta b)$
for every $b\in\frak B$;  in particular,

\Centerline{$\phi b=0\mskip8mu\Longrightarrow\mskip8mu\bar\mu(\phi b)=0
\mskip8mu\Longrightarrow\mskip8mu\bar\mu'(\theta b)=0
\mskip8mu\Longrightarrow\mskip8mu\theta b=0$.}

\noindent There is therefore a Boolean homomorphism
$\tilde\theta:\phi[\frak B]\to\frak A'$ such that
$\tilde\theta\phi=\theta$, and $\tilde\theta$ is measure-preserving on
$\phi[\frak B]$.   Since $\phi[\frak B]$ is topologically dense in
$\frak A$ (use 323J), $\tilde\theta$ has an extension to a
measure-preserving Boolean homomorphism $\hat\psi:\frak A\to\frak A'$
(324O).   Now, for $i\in I$ and $a\in\frak A_i$,

\Centerline{$\hat\psi\phi_ia
=\hat\psi\phi\varepsilon_ia
=\tilde\theta\phi\varepsilon_ia
=\theta\varepsilon_ia
=\phi'_i\psi_ia$,}

\noindent as required.

\medskip

{\bf (d)} To see that $\hat\psi$ is unique, we need observe only that
the given formula defines it on the subalgebra $\phi[\frak B]$ and that
this is topologically dense in $\frak A$, while $\hat\psi$, being
measure-preserving, must be continuous.
}%end of proof of 458P

\leader{458Q}{Relative product measures:  Definitions (a)}
Let $\familyiI{X_i}$ be a family of sets, $Y$ a set, and
$\pi_i:X_i\to Y$ a function for each $i\in I$.   The {\bf fiber product}
of $\familyiI{(X_i,\pi_i)}$ is the set
$\Delta=\{x:x\in\prod_{i\in I}X_i$, $\pi_ix(i)=\pi_jx(j)$ for all $i$,
$j\in I\}$.

\spheader 458Qb  Let
$\familyiI{(X_i,\Sigma_i,\mu_i)}$ be a family of probability spaces
and $(Y,\Tau,\nu)$ a probability space, and suppose that we are given an
\imp\ function $\pi_i:X_i\to Y$ for each $i\in I$;  let $\Delta$ be the
fiber product of $\familyiI{(X_i,\pi_i)}$.   A {\bf relative
product measure} on $\Delta$ is a probability measure $\mu$ on $\Delta$
such that

\inset{($\dagger$) whenever $J\subseteq I$ is finite and not empty and
$E_i\in\Sigma_i$ for $i\in J$, and $g_i$ is a Radon-Nikod\'ym derivative
with respect to $\nu$ of the
functional $F\mapsto\mu_i(E\cap\pi_i^{-1}[F]):\Tau\to[0,1]$ for each
$i\in J$, then $\mu\{x:x\in\Delta$, $x(i)\in E_i$ for every $i\in J\}$
is defined and equal to $\int\prod_{i\in J}g_id\nu$;

($\ddagger$) for every $W\in\Sigma$ there is a $W'$ in the
$\sigma$-algebra generated by
$\{\{x:x\in\Delta$, $x(i)\in E\}:i\in I$, $E\in\Sigma_i\}$ such that
$\mu(W\symmdiff W')=0$.}

\medskip

\noindent{\bf Remark} If $\mu$ is a
relative product measure of $\familyiI{(\mu_i,\pi_i)}$ over
$\nu$, then all the functions $x\mapsto x(i):\Delta\to X_i$ are \imp.
\prooflet{\Prf\ The condition ($\dagger$) tells us that if
$E\in\Sigma_i$ and $g$ is any Radon-Nikod\'ym derivative of
$F\mapsto\mu_i(E\cap\pi_i^{-1}[F])$, then

\Centerline{$\mu\{x:x(i)\in E\}=\int g\,d\nu=\mu_iE$.\ \Qed}

\noindent}It follows that if $I$ is not empty then we have an \imp\
function $\pi:\Delta\to Y$ defined by setting $\pi x=\pi_ix(i)$ whenever
$x\in\Delta$ and $i\in I$.

\cmmnt{Note that when verifying ($\dagger$) we need check the equality
$\mu\{x:x\in\Delta$, $x(i)\in E_i$ for every $i\in J\}
=\int\prod_{i\in J}g_id\nu$ for
only one representative family $\family{i}{J}{g_i}$ of Radon-Nikod\'ym
derivatives for any given $\family{i}{J}{E_i}$.}

\vleader{36pt}{458R}{Proposition} Suppose that
$\familyiI{(X_i,\Sigma_i,\mu_i)}$ is a family of probability spaces,
$(Y,\Tau,\nu)$ a probability space, $\pi_i:X_i\to Y$ an \imp\
function for each $i\in I$, $\Delta$ the fiber product of
$\familyiI{(X_i,\pi_i)}$ and $\mu$ a relative product measure of
$\familyiI{(\mu_i,\pi_i)}$.   Let $(\frak A_i,\bar\mu_i)$,
$(\frak C,\bar\nu)$ and $(\frak A,\bar\mu)$
be the measure algebras of $\mu_i$, $\nu$ and
$\mu$ respectively, and for $i\in I$ define
$\bar\pi_i:\frak C\to\frak A_i$ and $\bar\phi_i:\frak A_i\to\frak A$ by
setting $\bar\pi_iF^{\ssbullet}=\pi_i^{-1}[F]^{\ssbullet}$,
$\bar\phi_iE^{\ssbullet}=\{x:x\in\Delta$, $x(i)\in E\}^{\ssbullet}$
whenever $F\in\Tau$ and
$E\in\Sigma_i$.   Then $(\frak A,\bar\mu,\familyiI{\bar\phi_i})$ is a
relative free product of $\familyiI{(\frak A_i,\bar\mu_i,\bar\pi_i)}$
over $(\frak C,\bar\nu)$.

\proof{ The case $I=\emptyset$ is trivial (if you care to follow through
the definitions to the letter, $\Delta=\prod_{i\in I}X_i=\{\emptyset\}$
and $\frak A$ is the two-point algebra).   So I will take it that $I$ is
not empty.   For $i\in I$ define $\phi_i:\Delta\to X_i$ by setting
$\phi_i(x)=x(i)$ for $x\in\Delta$.

\medskip

{\bf (a)} Of course we have to check that all the $\bar\pi_i$ and
$\bar\phi_i$ are measure-preserving Boolean homomorphisms between the
appropriate algebras, but in view of the remark following the definition
458Q, this is elementary.   The condition that $\frak A$ should be the
closed subalgebra generated by
$\bigcup_{i\in I}\bar\phi_i[\frak A_i]$ is just a translation of the
condition ($\ddagger$).

\medskip

{\bf (b)} As $I$ is not empty, we have a well-defined \imp\ map
$\pi:\Delta\to Y$ given by the formula $\pi(x)=\pi_ix(i)$ whenever
$x\in\Delta$ and $i\in I$.   Let
$\bar\pi:\frak C\to\frak A$ be the corresponding measure-preserving
homomorphism, so that $\bar\pi=\bar\phi_i\bar\pi_i$ for every $i$.   Set
$\frak D=\bar\pi[\frak C]\subseteq\frak A$, and let
$T:L^{\infty}(\frak C)\to L^{\infty}(\frak D)$ be the $f$-algebra
isomorphism corresponding to $\bar\pi$ (363F).   For $i\in I$ and
$E\in\Sigma_i$, let $g_{iE}$ be a Radon-Nikod\'ym derivative with respect
to $\nu$ of the functional $F\mapsto\mu_i(E\cap\pi_i^{-1}[F])$,
and set $u_{iE}=Tg_{iE}^{\ssbullet}\in L^{\infty}(\frak D)$.
Then $u_{iE}$ is the conditional expectation of
$\chi\{x:x(i)\in E\}^{\ssbullet}$ on $\frak D$.   \Prf\ If
$d\in\frak D$, it is of the form
$\bar\pi F^{\ssbullet}=\bar\phi_i\bar\pi_iF^{\ssbullet}$
where $F\in\Tau$, so that $\chi d=T(\chi F)^{\ssbullet}$ and

$$\eqalign{\int_du_{iE}d\bar\mu
&=\int u_{iE}\times\chi d\,d\bar\mu
=\int T(g_{iE}^{\ssbullet}\times\chi F^{\ssbullet})d\bar\mu\cr
&=\int g_{iE}^{\ssbullet}\times\chi F^{\ssbullet}d\bar\nu
=\int_Fg_{iE}d\nu\cr
&=\mu_i(E\cap\pi_i^{-1}[F])
=\mu(\phi_i^{-1}[E]\cap\phi_i^{-1}[\pi_i^{-1}[F]])
=\bar\mu(d\Bcap\phi_i^{-1}[E]^{\ssbullet}).\cr}$$

\noindent As $d$ is arbitrary, we have the result.\ \Qed

\medskip

{\bf (c)} It follows that if $J\subseteq I$ is finite and not empty, and
$a_i\in\bar\phi_i[\frak A_i]$ and $v_i$ is the conditional expectation
of $\chi a_i$ on $\frak D$ for each $i\in J$, then
$\bar\mu(\inf_{i\in J}a_i)=\int\prod_{i\in J}v_id\bar\mu$.   \Prf\
Express $a_i$ as $\phi_i^{-1}[E_i]^{\ssbullet}$, where $E_i\in\Sigma_i$,
so that $v_i=u_{i,E_i}$ for each $i$.   Then

$$\eqalign{\int\prod_{i\in J}v_id\bar\mu
&=\int\prod_{i\in J}Tg_{i,E_i}^{\ssbullet}d\bar\mu
=\int T(\prod_{i\in J}g_{i,E_i}^{\ssbullet})d\bar\mu
=\int\prod_{i\in J}g_{i,E_i}^{\ssbullet}d\bar\nu\cr
&=\int\prod_{i\in J}g_{i,E_i}d\nu
=\mu(\bigcap_{i\in J}\phi_i^{-1}[E_i])
=\bar\mu(\inf_{i\in J}a_i).  \text{ \Qed}\cr}$$

\noindent But this is exactly what we need to know to see that
$\familyiI{\bar\phi_i[\frak A_i]}$ is relatively independent over
$\frak D$, completing the proof that
$(\frak A,\bar\mu,\familyiI{\bar\phi_i})$ is a relative free product of
$\familyiI{(\frak A_i,\bar\mu_i,\bar\pi_i)}$ over $(\frak C,\bar\nu)$.
}%end of proof of 458R

\leader{458S}{}\cmmnt{ There is no general result on relative product
measures to match 458O (see 458Xj-458Xm). %458Xj 458Xk 458Xl 458Xm
The general question
of when we can expect relative product measures to exist seems
interesting (458Ye, 458Yf).   Here I give a couple of sample results
dealing with important special cases.

\medskip

\noindent}{\bf Proposition} Let $\familyiI{(X_i,\Sigma_i,\mu_i)}$ be a
family of probability spaces, $(Y,\Tau,\nu)$ a probability space, and
$\pi_i:X_i\to Y$ an \imp\ function for each $i$.   Suppose that for each
$i$ we have a disintegration $\family{y}{Y}{\mu_{iy}}$ of $\mu_i$ such
that $\mu_{iy}^*\pi_i^{-1}[\{y\}]=\mu_{iy}X_i=1$ for every $y\in Y$.
Let $\Delta\subseteq\prod_{i\in I}X_i$ be the fiber product of
$\familyiI{(X_i,\pi_i)}$, and $\Upsilon$ the subspace $\sigma$-algebra
on $\Delta$ induced by
$\Tensorhat_{i\in I}\Sigma_i$.   For $y\in Y$, let $\lambda_y$ be the
product of
$\familyiI{\mu_{iy}}$, $(\lambda_y)_{\Delta}$ the subspace measure on
$\Delta$ and $\lambda'_y$ its restriction to $\Upsilon$.   Then
$\mu W=\int\lambda'_yW\nu(dy)$ is defined for every
$W\in\Upsilon$, and $\mu$ is a
relative product measure of $\familyiI{(\mu_i,\pi_i)}$ over $\nu$.

\proof{ If $y\in Y$, then

\Centerline{$\lambda'_y\Delta=\lambda^*_y\Delta
=(\lambda_y)^*(\prod_{i\in I}\pi_i^{-1}[\{y\}])=1$}

\noindent (254Lb).   For $i\in I$ and $E\in\Sigma_i$ set
$g_{iE}(y)=\mu_{iy}E$ when this is defined;  then
$g_{iE}$ is a Radon-Nikod\'ym derivative of
$F\mapsto\mu_i(E\cap\pi_i^{-1}[F]):\Tau\to[0,1]$ (452Qa).   Write $X$
for $\prod_{i\in I}X_i$;  for
$i\in I$ and $x\in X$ set $\phi_i(x)=x(i)$.   If
$J\subseteq I$ is finite and $E_i\in\Sigma_i$ for each $i\in J$, then

$$\eqalignno{\int\lambda'_y(\Delta
  \cap\bigcap_{i\in J}\phi_i^{-1}[E_i])\nu(dy)
&=\int(\lambda_y)_{\Delta}(\Delta
  \cap\bigcap_{i\in J}\phi_i^{-1}[E_i])\nu(dy)\cr
&=\int\lambda_y(X\cap\bigcap_{i\in J}\phi_i^{-1}[E_i])\nu(dy)\cr
\displaycause{because $\lambda_y^*\Delta=1$ and $\lambda_y$ measures
every $\phi_i^{-1}[E_i]$ for almost every $y$}
&=\int\prod_{i\in J}\mu_{iy}E_i\nu(dy)
=\int\prod_{i\in J}g_{i,E_i}d\nu.\cr}$$

\noindent In particular,
$\int\lambda'_y(\Delta\cap\bigcap_{i\in J}\phi_i^{-1}[E_i])\nu(dy)$ is
defined.

The set
$\{W:W\subseteq X$, $\int\lambda'_y(W\cap\Delta)\nu(dy)$ is defined$\}$
is a Dynkin class of subsets of $X$ containing
$\bigcap_{i\in J}\phi_i^{-1}[E_i]$ whenever $J\subseteq I$ is
finite and
not empty and $E_i\in\Sigma_i$ for each $i\in J$;  by the Monotone Class
Theorem, it includes $\Tensorhat_{i\in I}\Sigma_i$.   So
$\mu W=\int\lambda'_yW\nu(dy)$ is defined for every $W\in\Upsilon$.
Moreover, the formula displayed above tells us that
$\mu(\Delta\cap\bigcap_{i\in I}\phi_i^{-1}[E_i])
=\int\prod_{i\in J}g_{i,E_i}d\nu$
whenever $J\subseteq I$ is finite and $E_i\in\Sigma_i$ for each
$i\in I$.   Thus ($\dagger$) of 458Q is satisfied.   And ($\ddagger$) is
true by the choice of $\Upsilon$.
}%end of proof of 458S

\leader{458T}{}\cmmnt{ The latitude I have permitted in the definition
of `relative product' makes it possible to look for relative product
measures with further properties, as in the following.

\medskip

\noindent}{\bf Proposition} Let
$\familyiI{(X_i,\frak T_i,\Sigma_i,\mu_i)}$ be a family of compact Radon
probability spaces, $(Y,\frak S,\Tau,\nu)$ a Radon probability space,
and $\pi_i:X_i\to Y$ a continuous \imp\ function for each $i$.   Then
$\familyiI{(\mu_i,\pi_i)}$ has a relative product measure $\mu$
over $\nu$ which is a Radon measure for the topology on the fiber
product of $\familyiI{(X_i,\pi_i)}$ induced by the product topology on
$\prod_{i\in I}X_i$.

\proof{{\bf (a)} For each $i\in I$, $E\in\Sigma_i$ let $g_{iE}$ be a
Radon-Nikod\'ym derivative of
$F\mapsto\mu_i(E\cap\pi_i^{-1}[F]):\Tau\to[0,1]$.   Let $\Cal C$ be the
family of measurable cylinders in $X=\prod_{i\in I}X_i$;  write
$\phi_i(x)=x(i)$ for $x\in X$ and $i\in I$.   We have a
functional $\lambda_0:\Cal C\to[0,1]$ defined by setting

\Centerline{$\lambda_0(\bigcap_{i\in J}\phi_i^{-1}[E_i])
=\int\prod_{i\in J}g_{i,E_i}d\nu$}

\noindent whenever $J\subseteq I$ is finite and not empty and
$E_i\in\Sigma_i$ for every $i\in I$.   It is easy to check that
$\lambda_0$ is additive in the sense of 454E so (because every $\mu_i$
is perfect, by 416Wa and 343K) it has an extension to a measure
$\lambda$ on $X$ with domain $\Tensorhat_{i\in I}\Sigma_i$.   By 454Aa,
with $\Cal K$ the family of compact subsets of $X$, $\lambda$ is inner
regular with respect to the compact sets.   By 416O, there is a Radon
measure $\tilde\lambda$ on $X$ extending $\lambda$.

\medskip

{\bf (b)} Let $\Delta$ be the fiber product of $\familyiI{(X_i,\pi_i)}$.
Now the point is that
$\Delta$ is $\tilde\lambda$-conegligible.   \Prf\ Because every $\pi_i$
is continuous, $\Delta$ is closed.   \Quer\ If it is not conegligible,
then, because $\tilde\lambda$ is $\tau$-additive, there must be a basic
open set of non-zero measure disjoint from $\Delta$;  express such a set
as $W=\prod_{i\in J}\phi_i^{-1}[G_i]$ where $J\subseteq I$ is finite and
$G_i\subseteq X_i$ is
open for each $i\in J$.   Because $\tilde\lambda$ is inner regular with
respect to the compact sets, there is a compact set $K\subseteq W$ such
that $\tilde\lambda K>0$;  setting $K_i=\phi_i[K]$,
$K_i\subseteq G_i$ is compact for each $i$ and
$W'=\bigcap_{i\in J}\phi_i^{-1}[K_i]$ is non-negligible.   Now we have

\Centerline{$0<\tilde\lambda(\prod_{i\in J}\phi_i^{-1}[K_i])
=\lambda_0(\prod_{i\in J}\phi_i^{-1}[K_i])
=\int\prod_{i\in J}g_{i,K_i}d\nu$,}

\noindent so $F=\{y:y\in Y$, $g_{i,K_i}(y)>0$ for every $i\in J\}$ is
non-negligible.   On the other hand, for each $i\in J$ we have

\Centerline{$\int_{Y\setminus\pi_i[K_i]}g_{i,K_i}d\nu
=\mu_i(K_i\Bcap\pi_i^{-1}[Y\setminus K_i])=0$}

\noindent so that $F\setminus\pi_i[K_i]$ is negligible.   Accordingly
$\bigcap_{i\in J}\pi_i[K_i]$ is non-negligible, and must meet the
support $Y_0$ of $Y$;  let $y$ be any point of the intersection.   For
$i\in J$, choose $x(i)\in K_i$ such that $\pi_ix(i)=y$.   For
$i\in I\setminus J$, $\pi_i[X_i]$ is a compact subset of $Y$, and
$\nu\pi_i[X_i]=\mu_i\pi_i^{-1}[\pi_i[X_i]]=1$, so
$Y_0\subseteq\pi_i[X_i]$ and we can therefore choose $x(i)\in X_i$ with
$\pi_ix(i)=y$.   This defines $x\in\Delta$.   But as $x(i)\in K_i$ for
$i\in J$, we also have

\Centerline{$x\in\bigcap_{i\in J}\phi_i^{-1}[K_i]
\subseteq\bigcap_{i\in J}\phi_i^{-1}[G_i]\subseteq X\setminus\Delta$,}

\noindent which is impossible.\ \Bang

Thus $\Delta$ is $\tilde\lambda$-conegligible, as claimed.\ \Qed

\medskip

{\bf (c)} Let $\mu$ be the subspace measure on $\Delta$ induced by
$\tilde\lambda$, and $\Sigma$ its domain, so that $\mu$ is a Radon
probability measure on $\Delta$ with its subspace topology
(416Rb).   Concerning ($\dagger$) of 458Q, if $J\subseteq I$ is finite
and not empty and $E_i\in\Sigma_i$ for $i\in J$, then

\Centerline{$\mu(\Delta\cap\bigcap_{i\in J}\phi_i^{-1}[E_i])
=\tilde\lambda(\bigcap_{i\in J}\phi_i^{-1}[E_i])
=\lambda_0(\bigcap_{i\in J}\phi_i^{-1}[E_i])
=\int\prod_{i\in J}g_{i,E_i}d\nu$,}

\noindent as required.   Finally, for ($\ddagger$), the $\sigma$-algebra
$\Upsilon$ of subsets of $\Delta$ generated by
$\{\Delta\cap\phi_i^{-1}[E]:i\in I$, $E\in\Sigma_i\}$ is just the subspace
$\sigma$-algebra induced by $\Tensorhat_{i\in I}\Sigma_i$.   Let
$\frak A$ be the measure algebra of $\tilde\lambda$ and
$\frak B\subseteq\frak A$ the set
$\{W^{\ssbullet}:W\in\Tensorhat_{i\in I}\Sigma_i\}$.   Then $\frak B$ is
a closed subalgebra of $\frak A$.   If $W\subseteq X$ is open, then for
every $\epsilon>0$ there is a $W_0\in\Tensorhat_{i\in I}\Sigma_i$ such
that $W_0\subseteq W$ and $\tilde\lambda(W\setminus W_0)\le\epsilon$, so
$W^{\ssbullet}\in\frak B$;  accordingly $\{W:W^{\ssbullet}\in\frak B\}$
contains every open set and every Borel set and must be the whole of
$\dom\lambda$.   Returning to the measure $\mu$, we see that if
$W\in\Sigma$ there must be a $W_0\in\Tensorhat_{i\in I}\Sigma_i$ such
that $\tilde\lambda(W\symmdiff W_0)=0$;  now $W_0\cap\Delta\in\Upsilon$
and
$\mu(W\symmdiff(W_0\cap\Delta))=0$.    So ($\ddagger$) also is true, and
we have a relative product measure of the declared type.
}%end of proof of 458T

\leader{458U}{}\cmmnt{ We can of course make a general search through
theorems about product measures, looking for ways of re-presenting them
as theorems about relative product measures.   There is an associative
law, for instance (458Xr).   To give an idea of what is to be expected,
I offer a result corresponding to 253D.

\medskip

\noindent}{\bf Proposition} Let $(X_1,\Sigma_1,\mu_1)$,
$(X_2,\Sigma_2,\mu_2)$ and $(Y,\Tau,\nu)$ be probability spaces, and
$\pi_1:X_1\to Y$, $\pi_2:X_2\to Y$ \imp\ functions.   Let $\Delta$ be
the fiber product of $(X_1,\pi_1)$ and $(X_2,\pi_2)$, and suppose that
$\mu$ is a relative product measure of $(\mu_1,\pi_1)$ and
$(\mu_2,\pi_2)$ over $\nu$;  set $\pi x=\pi_1x(1)=\pi_2x(2)$ for
$x\in\Delta$.   Take $f_1\in\eusm L^1(\mu_1)$ and
$f_2\in\eusm L^2(\mu_2)$, and set
$(f_1\otimes f_2)(x)=f_1(x(1))f_2(x(2))$ when
$x\in\Delta\cap(\dom f_1\times\dom f_2)$.   For $i=1$, $2$ let
$g_i\in\eusm L^1(\nu)$ be a Radon-Nikod\'ym derivative of
$F\mapsto\int_{\pi_i^{-1}[F]}f_id\mu_i:\Tau\to\Bbb R$.
Then $\int_Fg_1\times g_2d\nu=\int_{\pi^{-1}[F]}f_1\otimes f_2d\mu$ for
every $F\in\Tau$.

\proof{ When $f_1$ and $f_2$ are indicator functions of measurable
sets, this is just the definition of `relative product measure'.
The formula for $g_i$ corresponds to a linear operator from $L^1(\mu_i)$
to $L^1(\nu)$, so the result is true for simple functions $f_1$ and
$f_2$.   If $f_1$ and $f_2$ are almost everywhere limits of
non-decreasing sequences $\sequencen{f_{1n}}$, $\sequencen{f_{2n}}$ of
non-negative simple functions, then the corresponding sequences
$\sequencen{g_{1n}}$, $\sequencen{g_{2n}}$ will also be non-decreasing
and non-negative and convergent to $g_1$, $g_2\,\,\nu$-a.e.;  moreover,
because $x\mapsto x(1)$ and $x\mapsto x(2)$ are \imp,
$f_1\otimes f_2=\lim_{n\to\infty}f_{1n}\otimes f_{2n}\,\,\mu$-a.e.   So
in this case we shall have

$$\eqalign{\int_{\pi^{-1}[F]}f_1\otimes f_2d\mu
&=\lim_{n\to\infty}\int_{\pi^{-1}[F]}f_{1n}\otimes f_{2n}d\mu\cr
&=\lim_{n\to\infty}\int_Fg_{1n}\times g_{2n}d\nu
=\int_Fg_1\times g_2d\nu\cr}$$

\noindent for every $F\in\Tau$.   Finally, considering positive and
negative parts, we can extend the result to general integrable $f_1$ and
$f_2$.
}%end of proof of 458U

\exercises{\leader{458X}{Basic exercises $\pmb{>}$(a)}
%\sqheader 458Xa
Find an example of a probability space $(X,\Sigma,\mu)$
with $\sigma$-subalgebras $\Sigma_1$, $\Sigma_2$ and $\Tau$ of $\Sigma$
such that $\Sigma_1$ and $\Sigma_2$ are independent but are not
relatively independent over $\Tau$.
%458A

\spheader 458Xb Let $(X,\Sigma,\mu)$ be a probability space and $\Tau$,
$\Sigma_1$ and $\Sigma_2\,\,\sigma$-subalgebras of $\Sigma$.   Show that if
$\Sigma_1\subseteq\Tau$ then $\Sigma_1$ and $\Sigma_2$ are relatively
independent over $\Tau$.
%458A

\sqheader 458Xc Let $(X,\Sigma,\mu)$ be a probability space and $\Tau$ a
$\sigma$-subalgebra of $\Sigma$.   Let $\familyiI{\Cal E_i}$ be a family of
subsets of $\Sigma$ such that (i) each $\Cal E_i$ is closed under finite
intersections (ii) $\familyiI{E_i}$ is relatively independent over $\Tau$
whenever $E_i\in\Cal E_i$ for every $i$.   For each $i\in I$,
let $\Sigma_i$ be the
$\sigma$-subalgebra of $\Sigma$ generated by $\Cal E_i$.   Show that
$\familyiI{\Sigma_i}$ is relatively independent over $\Tau$.
%458D

\sqheader 458Xd Let $(X,\Sigma,\mu)$ be a probability space and $\Tau$ a
$\sigma$-subalgebra of $\Sigma$.   Let $f_1$, $f_2$ be $\mu$-integrable
real-valued functions which are relatively independent over $\Tau$, and
suppose that $f_1\times f_2$ is integrable.
Let $g_1$, $g_2$ be conditional expectations of $f_1$, $f_2$ on $\Tau$.
Show that $g_1\times g_2$ is a conditional expectation of
$f_1\times f_2$ on $\Tau$.
%458F  out of order query

\spheader 458Xe In 458I, show that (writing $\hat\mu$ for the completion
of $\mu$) $\hat\mu(F\cap f^{-1}[H])=\int_F\nu_xH\mu(dx)$ for every
$F\in\Tau$ and every universally measurable $H\subseteq\Bbb R$.
%458I

\spheader 458Xf Let $(X,\Sigma,\mu)$ be a probability space and $\Sigma_1$,
$\Sigma_2$ and $\Tau\,\,\sigma$-subalgebras.   Show that the following are
equiveridical:  (i) $\Sigma_1$ and $\Sigma_2$ are relatively independent
over $\Tau$;  (ii) whenever $f\in\eusm L^1(\mu\restr\Sigma_1)$ and $g$ is a
conditional expectation of $f$ on $\Tau$, then $g$ is a conditional
expectation of $f$ on $\Sigma_2\vee\Tau$.
%458H

\spheader 458Xg Prove 458Ld directly from 313G, without appealing to 458H.
%458L

\spheader 458Xh Let $\familyiI{(\frak A_i,\bar\mu_i)}$ be a family of
probability algebras.   Show that their probability algebra free product
(325K) can be identified with their relative free product over
$(\frak C,\bar\nu)$ if $\frak C$ is the two-element
Boolean algebra, $\bar\nu$ its unique probability measure, and
$\pi_i:\frak C\to\frak A_i$ the trivial Boolean homomorphism for every
$i$.
%458O

\sqheader 458Xi Let $Y$ be a set, $\familyiI{Z_i}$ a family of sets, and
$\pi_i:Y\times Z_i\to Y$ the canonical map for each $i$.   Show that the
fiber product of $\familyiI{(Y\times Z_i,\pi_i)}$ can be identified with
$Y\times\prod_{i\in I}Z_i$.
%458Q

\spheader 458Xj Let $\nu$ be Lebesgue
measure on $[0,1]$, and $X_1$,
$X_2\subseteq[0,1]$ disjoint sets with outer measure $1$.   For each
$i\in\{1,2\}$ let $\mu_i$ be the
subspace measure on $X_i$ and
$\pi_i:X_i\to[0,1]$ the identity map.   Show that $(\mu_1,\pi_1)$ and
$(\mu_2,\pi_2)$ have no relative product measure over $\nu$.
%458Q

\spheader 458Xk Let $\nu$ be the usual measure on the split interval
$I^{\|}$ (343J), and $\mu$ Lebesgue measure on
$[0,1]$.   Set $\pi_1(t)=t^+$, $\pi_2(t)=t^-$ for
$t\in[0,1]$.   Show that $(\mu,\pi_1)$ and
$(\mu,\pi_2)$ have no relative product measure over $\nu$.
%458Q

\spheader 458Xl Let $\nu$ be Lebesgue measure on $[0,1]$.   For each
$t\in[0,1]$, set $X_t=[0,1]\setminus\{t\}$;  let
$\mu_t$ be the subspace measure on $X_t$ and
$\pi_t:X_t\to[0,1]$ the identity map.   Show that
$\family{t}{[0,1]}{(\mu_t,\pi_t)}$ has no relative product measure
over $\nu$.
%458Q

\spheader 458Xm(i) Show that there is a set $X\subseteq[0,1]^2$ with outer
planar Lebesgue measure $1$ and just one point in each vertical section.
\Hint{419H-419I.}  (ii) Set $X_1=X_2=X$ and
$\mu_1=\mu_2$ the subspace measure on $X$;
let $(Y,\Tau,\nu)$ be $[0,1]$ with Lebesgue measure, and $\pi_1=\pi_2$ the
first-coordinate projection from $X$ to $Y$.   Show that
$(\mu_1,\pi_1)$ and $(\mu_2,\pi_2)$ have no
relative product measure over $\nu$.
%458Q

\spheader 458Xn Let $(X,\Sigma,\mu)$ be a probability space, $\Tau$ a
$\sigma$-subalgebra of $\Sigma$, and $\familyiI{\Sigma_i}$ a family of
$\sigma$-subalgebras of $\Sigma$, all including $\Tau$.   Set
$\pi_i(x)=x$ for every
$x\in X$.   Show that $\familyiI{\Sigma_i}$ is relatively independent
over $\Tau$ iff  $\mu\restr\Sigma^*$ is a
relative product measure of $\familyiI{(\mu\restr\Sigma_i,\pi_i)}$
over $\mu\restrp\Tau$, where $\Sigma^*=\bigvee_{i\in I}\Sigma_i$.
%458Q

\spheader 458Xo(i) Let $\familyiI{(X_i,\Sigma_i,\mu_i)}$ be a family of
probability spaces and $(X,\Sigma,\mu)$ their ordinary probability space
product.   Show that $\mu$ is a relative
product measure of $\familyiI{(\mu_i,\pi_i)}$ over
$\nu$ where $Y$ is a singleton set, $\nu$ its
unique probability measure, and $\pi_i:X_i\to Y$ the unique function for
each $i$.   (ii) Let $\familyiI{(X_i,\frak T_i,\Sigma_i,\mu_i)}$ be a
family of quasi-Radon probability spaces and $(X,\frak T,\Sigma,\mu)$
their quasi-Radon probability space product (417R).   Show that
$\mu$ is a relative product measure of
$\familyiI{\mu_i}$ in the same sense as in (i).
%458Q

\spheader 458Xp Suppose that $\familyiI{(X_i,\Sigma_i,\mu_i)}$ is a
family of probability spaces and $(Y,\Tau,\nu)$ is a probability space,
and that for each $i\in I$ we are given an \imp\ function
$\pi_i:X_i\to Y$.   Write $\hat\mu_i$ and
$\hat\nu$ for the completions of $\mu_i$, $\nu$ respectively.   Show
that $\familyiI{(\mu_i,\pi_i)}$ has a relative product measure over
$\nu$ iff $\familyiI{(\hat\mu_i,\pi_i)}$ has a relative product measure
over $\hat\nu$.
%458Q

\spheader 458Xq Let $\familyiI{(X_i,\Sigma_i,\mu_i)}$ be a
family of probability spaces, $(Y,\Tau,\nu)$ a probability space, and
$\pi_i:X_i\to Y$ an \imp\ function for each $i\in I$.   Show that if
$\familyiI{(\mu_i,\pi_i)}$ has a relative product measure over $\nu$, so
does $\family{i}{J}{(\mu_i,\pi_i)}$ for any $J\subseteq I$.
%458Q

\spheader 458Xr Let $\familyiI{(X_i,\Sigma_i,\mu_i)}$ be a
family of probability spaces, $(Y,\Tau,\nu)$ a probability space, and
$\pi_i:X_i\to Y$ an \imp\ function for each $i\in I$.   Let
$\family{k}{K}{J_k}$ be a partition of $I$ into non-empty sets.   For
each $k\in K$, let $\Delta_k$ be the fiber product of
$\family{i}{J_k}{(X_i,\pi_i)}$;  suppose that $\tilde\mu_k$ is a
relative product measure of $\family{i}{J_k}{(\mu_i,\pi_i)}$.
Define $\tilde\pi_k:\Delta_k\to Y$ by
setting $\tilde\pi_k(x)=\pi_ix(i)$ whenever $x\in\Delta_k$ and
$i\in J_k$, so that $\tilde\pi_k$ is \imp.  Suppose that $\mu$ is a
relative product measure of $\family{k}{K}{(\tilde\mu_k,\tilde\pi_k)}$
over $\nu$.   Show that $\mu$ can be regarded as a relative product
measure of $\familyiI{(\mu_i,\pi_i)}$ over $\nu$.
%458Q

\spheader 458Xs Let $\familyiI{(X_i,\Sigma_i,\mu_i)}$ and
$\familyiI{(X'_i,\Sigma'_i,\mu'_i)}$ be two families of
probability spaces, $(Y,\Tau,\nu)$ and $(Y',\Tau',\nu')$ probability
spaces, and $\pi_i:X_i\to Y$, $\pi'_i:X'_i\to Y'$ \imp\ functions for
each $i$.   Suppose that we have a measure space isomorphism $g:Y\to Y'$
and \imp\ functions $f_i:X_i\to X'_i$, for
$i\in I$, such that $g\pi_i=\pi'_if_i$ for every $i$.   Show that if
there is a relative product measure of $\familyiI{(\mu_i,\pi_i)}$
over $\nu$, then there is a relative product measure of
$\familyiI{(\mu'_i,\pi'_i)}$ over $\nu'$.
%458R

\spheader 458Xt
Let $\familyiI{(X_i,\Sigma_i,\mu_i)}$ be a countable
family of probability spaces, $(Y,\Tau,\nu)$ a probability space, and
$\pi_i:X_i\to Y$ an \imp\ function for each $i$.   Suppose that for each
$i$ we have a disintegration of $\mu_i$ over $\nu$ which is strongly
consistent with $\pi_i$.   Show that
$\familyiI{(\mu_i,\pi_i)}$ has a relative product measure over
$\nu$.
%458S

\spheader 458Xu Let $(X,\Sigma,\mu)$, $(X',\Sigma',\mu')$ and
$(Y,\Tau,\nu)$ be probability spaces.   Suppose that $\pi:X\to Y$
and $\pi':X'\to Y$ are \imp\ functions, and that $\mu'$ has a
disintegration $\family{y}{Y}{\mu'_y}$ over $(Y,\Tau,\nu)$ which is
strongly consistent with $\pi'$.   Show that $(\mu,\pi)$ and
$(\mu',\pi')$
have a relative product measure over $\nu$.   \Hint{set
$\lambda W=\int\mu'_{\pi(x)}W[\{x\}]\mu(dx)$ for every
$W\in\Sigma\tensorhat\Sigma'$.}
%458S

\sqheader 458Xv Let $Y$ be a Hausdorff space, $\familyiI{Z_i}$ a family
of Hausdorff spaces, $\mu_i$ a Radon probability measure on
$Z_i\times Y$ and $\pi_i:Y\times Z_i\to Y$ the canonical map for each
$i$.   Suppose that all the image measures $\mu_i\pi_i^{-1}$ on $Y$ are
the same, and that all but countably many of the $Z_i$ are compact.
Show that there is a Radon probability measure $\mu$ on
$Y\times\prod_{i\in I}Z_i$ such that $\mu_i=\mu\phi_i^{-1}$ for each
$i$, where $\phi_i(y,z)=(y,z(i))$ for $y\in Y$, $z\in\prod_{j\in I}Z_j$.
%458T

\spheader 458Xw Let $\familyiI{(X_i,\frak T_i,\Sigma_i,\mu_i)}$ be a
countable family of Radon
probability spaces, $(Y,\frak S,\Tau,\nu)$ a Radon probability space,
and $\pi_i:X_i\to Y$ an almost continuous \imp\ function for each $i$.
Show
that $\familyiI{(\mu_i,\pi_i)}$ has a relative product measure
over $\nu$ which is a Radon measure for the topology on the fiber
product of $\familyiI{(X_i,\pi_i)}$ induced by the product topology on
$\prod_{i\in I}X_i$.   Discuss the relation of this result to 418Q.
%458T

\leader{458Y}{Further exercises (a)}%
%\spheader 458Ya
(i) Let $(X,\Sigma,\mu)$ be a probability space,
$\sequencen{\Tau_n}$ a non-increasing sequence of $\sigma$-subalgebras of
$\Sigma$ with intersection $\Tau$, and $\familyiI{\Sigma_i}$ a family of
subalgebras of $\Sigma$.   Suppose that $\familyiI{\Sigma_i}$ is relatively
independent over $\Tau_n$ for every $n$.   Show that it is relatively
independent over $\Tau$.   \Hint{275K.}
(ii)\dvAnew{2010} Give an example of a probability space
$(X,\Sigma,\mu)$, a downwards-directed
family $\Bbb T$ of $\sigma$-subalgebras of $\Sigma$,
and a family $\familyiI{E_i}$ in $\Sigma$ which is
relatively independent over $\Tau$ for every $\Tau\in\Bbb T$, but not over
$\bigcap\Bbb T$.
%458C

\spheader 458Yb Let $(X,\Sigma,\mu)$ be a probability space, $\Tau$ a
$\sigma$-subalgebra of $\Sigma$, and $\sequencen{\Sigma_n}$ a sequence
of $\sigma$-subalgebras of $\Sigma$ which is relatively independent over
$\Tau$.   For each $n\in\Bbb N$ let $\Sigma^*_n$ be
$\bigvee_{m\ge n}\Sigma_m$, and set
$\Sigma_{\infty}=\bigcap_{n\in\Bbb N}\Sigma^*_n$.   Show that for every
$E\in\Sigma_{\infty}$ there is an $F\in\Tau$ such that $E\symmdiff F$ is
negligible.   (Compare 272O.)
%relative zero-one law
%458H

\spheader 458Yc Let $(X,\Sigma,\mu)$ be a probability space, $\Tau$ a
$\sigma$-subalgebra of $\Sigma$, and $f$, $g\in\eusm L^0(\mu)$
relatively independent over $\Tau$;  suppose that $\family{x}{X}{\nu_x}$
and $\family{x}{X}{\nuprime_x}$ are relative distributions of $f$ and
$g$ over $\Tau$.   Show that $\family{x}{X}{\nu_x*\nuprime_x}$ is a
relative distribution of $f+g$ over $\Tau$.   (Compare 272T.)
%relative convolutions
%458H

\spheader 458Yd Let $(X,\Sigma,\mu)$ be a probability space, $\Tau$ a
$\sigma$-subalgebra of $\Sigma$, and $\sequencen{f_n}$ a sequence in
$\eusm L^2(\mu)$ such that $\sequencen{f_n}$ is relatively independent
over $\Tau$ and $\int_Ff_nd\mu=0$ for every $n\in\Bbb N$ and every
$F\in\Tau$.   (i) Suppose that $\sequencen{\beta_n}$ is a non-decreasing
sequence in $\ooint{0,\infty}$, diverging to $\infty$, such that
$\sum_{n=0}^{\infty}\Bover1{\beta_n^2}\|f_n\|_2^2<\infty$.   Show that
$\lim_{n\to\infty}\Bover1{\beta_n}\sum_{i=0}^nf_i=0$ a.e.   (ii)
Suppose that $\sup_{n\in\Bbb N}\|f_n\|_2<\infty$.   Show that
$\lim_{n\to\infty}\bover1{n+1}\sum_{i=0}^nf_i=0$ a.e.   (Compare 273D.)
%relative strong law
%458H

\spheader 458Ye Let $\familyiI{(X_i,\Sigma_i,\mu_i)}$ be a family of
measure spaces, $(Y,\Tau,\nu)$ a measure space, and $\pi_i:X_i\to Y$ an
\imp\ function for each $i\in I$.   For $i\in I$ and $E\in\Sigma_i$ let
$g_{iE}$ be a Radon-Nikod\'ym derivative of the functional
$F\mapsto\mu_i(E\cap\pi_i^{-1}[F])$.   Let $\Cal C$ be the family of
measurable cylinders in $X=\prod_{i\in I}X_i$.   If
$C=\{x:x\in X$, $x(i)\in E_i$ for every $i\in J\}$ where
$J\subseteq I$
is finite and not empty and $E_i\in\Sigma_i$ for $i\in J$, set
$\lambda_0C=\int\prod_{i\in J}g_{i,E_i}d\nu$.   Let $\Delta\subseteq X$
be the fiber product of $\familyiI{(X_i,\pi_i)}$.   Show that the
following are equiveridical:  (i) $\familyiI{(\mu_i,\pi_i)}$ has a
relative product measure over $\nu$ (ii) whenever
$\sequencen{C_n}$ is a sequence in $\Cal C$ covering $\Delta$,
$\sum_{n=0}^{\infty}\lambda_0C_n\ge 1$.
%458Q

\spheader 458Yf Let $\familyiI{(X_i,\Sigma_i,\mu_i)}$ be a family of
probability spaces, $(Y,\Tau,\nu)$ a probability space, and
$\pi_i:X_i\to Y$ a
surjective \imp\ function for each $i\in I$.   Suppose that
$\family{i}{J}{(\mu_i,\pi_i)}$ has a relative product measure over $\nu$
for every countable $J\subseteq I$.   Show that
$\family{i}{I}{(\mu_i,\pi_i)}$ has a relative product measure over $\nu$.
%458Ye 458Q

\spheader 458Yg Let $\familyiI{(X_i,\Sigma_i,\mu_i)}$ be a countable
family of perfect
probability spaces, $(Y,\Tau,\nu)$ a countably separated
probability space, and
$\pi_i:X_i\to Y$ an \imp\ function for each $i\in I$.   Show that
$\familyiI{(\mu_i,\pi_i)}$ has a relative product measure over $\nu$.
%458Ye 458S

\spheader 458Yh Let $(\frak A,\bar\mu)$ be a probability algebra and
$\frak C$ a closed subalgebra of $\frak A$.   Let
$\frak C_0\subseteq\frak C$ be the core subalgebra of countable Maharam
type described in the
canonical form of such structures given in 333N.   Show that there is a
closed subalgebra $\frak B$ of $\frak A$, including $\frak C_0$, such that
$\frak B$ and $\frak C$ are relatively independent over $\frak C_0$, and
$\frak A$ is the closed subalgebra of itself generated by
$\frak B\cup\frak C$.
%+

\spheader 458Yi\dvAnew{2009} Let $X$ be a set, $\Sigma$ a $\sigma$-algebra
of subsets of $X$, and $\Tau$ a $\sigma$-subalgebra of $\Sigma$.   Let
$\familyiI{\Cal E_i}$ be a family of subsets of $\Sigma$ such that (i)
$E\cap F\in\Cal E_i$ whenever $i\in I$ and $E$, $F\in\Cal E_i$ (ii)
$\familyiI{E_i}$ is relatively independent over $\Tau$ whenever
$E_i\in\Cal E_i$ for every $i\in I$.   For each $i\in I$,
let $\Sigma_i$ be the $\sigma$-algebra generated by $\Cal E_i$.   Show that
$\familyiI{\Sigma_i}$ is relatively independent over $\Tau$.
%458D out of order query

 }%end of exercises

\endnotes{
\Notesheader{458}
The elementary theory of relative independence has two aspects.   First,
there is the matter of systematically formulating and verifying appropriate
variations on standard results on stochastic independence;  458F, 458H,
458J, 458K, 458Xd, 458Yb-458Yd %458Yb 458Yc 458Yd
come under this heading.   More interestingly, we study the new phenomena
associated with changes in the core $\sigma$-algebras, as in 458C, 458D
and 458Xa.

At a couple of points in Volume 3 (Dye's theorem, in \S388, and
Kawada's theorem, in \S395) I took the trouble to generalize standard
theorems to `non-ergodic' forms.   In both 388L and 395P the results are
complicated
by potentially non-trivial closed subalgebras of the probability algebra
we are studying.   I remarked on both occasions that the generalization
is only a matter of technique, but I do not suppose that it was obvious
just why this must be so.   It is however a fundamental theorem of the
topic of `random reals' in the theory of forcing that {\it any} theorem
about probability algebras must have a relativized form as a theorem
about probability algebras with arbitrary closed subalgebras.   The
concept of `relative Maharam type' from \S333, for instance, is what
matches `Maharam type' for simple algebras;  the concept of
`exchangeable' sequence (definition:  459C)
is what matches `independent identically
distributed' sequence.   (In probability theory, the keyword
is `mixture'.)   In this section I present another example in the idea
of `relatively independent' closed subalgebras (458L-458M).   I should
emphasize that the forcing method, when we eventually come to it
in \S556 in Volume 5,
will not as a rule apply directly to measure spaces;  it
deals with measure algebras.   But of course the ideas generated by this
theory can often be profitably applied to constructions in measure
spaces, and this is what I am seeking to do with relatively independent
$\sigma$-algebras and relative product measures.

Just as independent $\sigma$-algebras are associated with product spaces
(272J), relatively independent algebras are associated with relative
products (458Xn).   The archetype of a relative product measure is
458S;  it is a kind of disintegrated product.   It is frequently
profitable to express the `relative' concepts of measure theory in terms
of disintegrations.

I introduce `relative free products' of probability algebras before
proceeding
to measure spaces because the uniqueness property proved in 458O shows
that we have an unambiguous definition.   For measure spaces it seems
for the moment better to leave ourselves a bit of freedom, not (for
instance) favouring one product construction over another (458Xo).
The requirement that a relative product measure be carried by the fiber
product is seriously limiting
(458Xj-458Xl, %458Xj 458Xk 458Xl
458Ye), and forces us to seek strongly consistent disintegrations
(458S), at least for uncountable products (see 458Xt).   However, as we
might hope, the special case of compact spaces with Radon measures and
continuous functions is amenable to a different approach (458T);  and we
have a one-sided method for the product of two spaces (458Xu) which is
reminiscent of 454C and 457F.

There are corresponding complications when we
come to look at maps between different relative products.   For
measure algebras, we have a natural theorem (458P), based on the
same algebraic considerations as the corresponding theorems in \S\S315
and 325;  the only possibly surprising feature is the need to assume
that $\psi:\frak C\to\frak C'$ is actually an isomorphism.   For measure
spaces there is a similar result (458Xs).

}%end of notes

\discrpage

%}}}}}

