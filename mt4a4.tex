\frfilename{mt4a4.tex}
\versiondate{19.6.13}
\copyrightdate{2002}

\def\Bourbaki{{\smc Bourbaki 87}}
\def\Schaefer{{\smc Schaefer 71}}
\def\Kothe{{\smc K\"othe 69}}
%\def\NB{{> Narici \& Beckenstein 85}}
\def\DS{{\smc Dunford \& Schwartz 57}}
\def\Rudin{{\smc Rudin 91}}
\def\Taylor{{\smc Taylor 64}}
\def\Jameson{{\smc Jameson 74}}

\def\chaptername{Appendix}
\def\sectionname{Locally convex spaces}

\newsection{4A4}

As in \S3A5, all the ideas, and nearly all the results as stated below,
are applicable to complex linear spaces;  but for the purposes of this
volume the real case will almost always
be sufficient, and for definiteness you may
take it that the scalar field is $\Bbb R$, except in 4A4J-4A4K.
(Complex Hilbert spaces arise naturally in \S445.)

\leader{4A4A}{Linear spaces (a)} If $U$ is a linear space, a
{\bf Hamel basis} for $U$ is a maximal linearly independent family
$\familyiI{u_i}$ in $U$, so that every member of $U$ is uniquely
expressible as $\sum_{i\in J}\alpha_iu_i$ for some finite
$J\subseteq I$ and $\family{i}{J}{\alpha_i}\in(\Bbb R\setminus\{0\})^J$.
%4A4

\spheader 4A4Ab Every linear space has a Hamel basis.
\prooflet{(\Schaefer, p.\ 10;  \Kothe, \S7.3.)}
%Bourbaki, Algebra, p 25

\spheader 4A4Ac If $U$ is a linear space, I write $U'$ for the
{\bf algebraic dual} of $U$, the linear
space of all linear functionals from $U$ to $\Bbb R$.
%4A4

\leader{4A4B}{Linear topological spaces}\cmmnt{ (see \S2A5)}
{\bf (a)} If $U$ is a linear topological space, and $V$ is a linear
subspace of $U$, then $V$, with the linear structure and topology
induced by those of $U$, is again a linear topological space.
\prooflet{(\Bourbaki, I.1.3; \Schaefer, \S I.2; \Kothe, \S15.2.)}

\spheader 4A4Bb If $\familyiI{U_i}$ is any family of linear topological
spaces, then $U=\prod_{i\in I}U_i$, with the product linear space
structure and topology, is again a linear topological space.
\prooflet{(\Bourbaki, I.1.3;  \Schaefer, \S I.2;  \Kothe, \S15.4.)}
% \NB, \S5.6; \KN, II.5.9
\cmmnt{In particular, }$\Bbb R^X$, with its usual linear and topological
structures, is a linear topological space, for any set $X$.
%4A4

\spheader 4A4Bc If $U$ and $V$ are linear topological spaces, the set of
continuous linear operators from $U$ to $V$ is a linear subspace of the
space $\eurm L(U;V)$ of all linear operators from $U$ to $V$.
\prooflet{}
If $U$, $V$ and $W$ are linear topological spaces, and $T:U\to V$ and
$S:V\to W$ are continuous linear operators, then $ST:U\to W$ is a
continuous linear operator.
%4A4Bd etc

\spheader 4A4Bd If $U$ is a linear topological space, I will write $U^*$
for the {\bf dual} of $U$, the space of all continuous linear
functionals from $U$ to
$\Bbb R$\cmmnt{ (compare 2A4H)}.   $U^*$ is a linear subspace of
$U'$\cmmnt{ as defined in 4A4Ac}.   The {\bf weak topology} on $U$,
$\frak T_s(U,U^*)$, is that defined\cmmnt{ by the method of 2A5B} from
the seminorms $u\mapsto|f(u)|$ as $f$ runs over
$U^*$\cmmnt{ (compare
2A5Ia)}. The {\bf weak* topology} on $U^*$, $\frak T_s(U^*,U)$, is
that defined from the seminorms
$f\mapsto|f(u)|$ as $u$ runs over $U$\cmmnt{ (compare 2A5Ig)}.
\cmmnt{By 2A5B, both are linear space topologies.}
If $U$ and $V$ are linear topological spaces, $T:U\to V$ is a continuous
linear operator, and $g\in V^*$, then $gT\in U^*$\prooflet{ ((c)
above)};  \cmmnt{consequently} $T$ is
$(\frak T_s(U,U^*),\frak T_s(V,V^*))$-continuous.

\spheader 4A4Be If $U=\prod_{i\in I}U_i$ is a product of linear
topological spaces, then every element of $U^*$ is of the form
$u\mapsto\sum_{i\in J}f_i(u(i))$ where $J\subseteq I$ is finite and
$f_i\in U_i^*$ for every $i\in J$.
\prooflet{(\Bourbaki, II.6.6; \Schaefer, IV.4.3;  \Kothe, \S22.5.)}
Consequently the weak topology on $U$ is the
product of the weak topologies on the $U_i$.

\spheader 4A4Bf Let $U$ be a linear topological space.   For $A\subseteq U$
write $A^{\smallcirc}$ for its {\bf polar} set
$\{f:f\in U^*$, $f(x)\le 1$ for every $x\in A\}$ in $U^*$.
If $G$ is a neighbourhood of $0$ in $U$, then $G^{\smallcirc}$ is a
$\frak T_s(U^*,U)$-compact subset of $U^*$\cmmnt{ (compare 3A5F)}.
\prooflet{(\Schaefer, III.4.3;  \Kothe, \S20.9;  \Rudin, 3.15.)}

\spheader 4A4Bg Let $U$ be a linear topological
space.   If $D\subseteq U$ is non-empty and closed under addition and
multiplication by rationals, $\overline{D}$ is a linear subspace of $U$.
\prooflet{\Prf\ The linear span

\Centerline{$V=\{\sum_{i=0}^n\alpha_iu_i:u_0,\ldots,u_n\in D,\,
\alpha_0,\ldots,\alpha_n\in\Bbb R\}$}

\noindent of

\Centerline{$D=\{\sum_{i=0}^n\alpha_iu_i:u_0,\ldots,u_n\in D,\,
\alpha_0,\ldots,\alpha_n\in\Bbb Q\}$}

\noindent is included in $\overline{D}$, because addition and scalar
multiplication are continuous;  so $\overline{D}=\overline{V}$ is a
linear subspace.\ \QeD}
If $A\subseteq U$ is separable, then the closed linear subspace
generated by $A$ is separable.   \prooflet{\Prf\ Let $D_0\subseteq A$ be
a countable dense subset;  then

\Centerline{$D=\{\sum_{i=0}^n\alpha_iu_i:u_0,\ldots,u_n\in D_0,\,
\alpha_0,\ldots,\alpha_n\in\Bbb Q\}$}

\noindent is countable, and $\overline{D}$ is separable;  but
$\overline{D}$ is the closed linear subspace generated by $A$.\ \Qed}
%5{}27

\spheader 4A4Bh If $\familyiI{u_i}$ is an indexed family in a Hausdorff
linear topological space $U$ and $u\in U$, we say that
$u=\sum_{i\in I}u_i$ if for every neighbourhood $G$ of $u$ there is a
finite set $J\subseteq I$ such that $\sum_{i\in K}u_i\in G$ whenever
$K\subseteq I$ is finite and $J\subseteq K$\cmmnt{ (compare 226Ad)}.

If $\familyiI{v_i}$ is another family with the same index set, and
$v=\sum_{i\in I}v_i$ is defined, then $\sum_{i\in I}(u_i+v_i)$ is defined
and equal to $u+v$.   \prooflet{\Prf\
If $G$ is a neighbourhood of $u+v$, there
are neighbourhoods $H$, $H'$ of $u$, $v$ respectively such that
$H+H'\subseteq G$;  there are finite sets $J$, $J'\subseteq I$ such that
$\sum_{i\in K}u_i\in H$ whenever $J\subseteq K\in[I]^{<\omega}$ and
$\sum_{i\in K}v_i\in H'$ whenever $J'\subseteq K\in[I]^{\omega}$;  now
$\sum_{i\in K}u_i+v_i\in G$ whenever
$J\cup J'\subseteq K\in[I]^{<\omega}$.\ \Qed}

If now $V$ is another Hausdorff linear topological space and $T:U\to V$
is a continuous linear operator,
$\sum_{i\in I}Tu_i=T(\sum_{i\in I}u_i)$ if the right-hand-side is
defined.  \prooflet{\Prf\ Set $u=\sum_{i\in I}u_i$.   If $H$ is an open
set containing $Tu$, then $T^{-1}[H]$ is an open set containing $u$, so
there is a $J\in[I]^{<\omega}$ (notation: 3A1J) such that
$\sum_{i\in K}u_i\in T^{-1}[H]$ and $\sum_{i\in K}Tu_i\in H$ whenever
$J\subseteq K\in[I]^{<\omega}$.\ \Qed}
%4A4Ie

\spheader 4A4Bi If $U$ is a Hausdorff linear topological space, then any
finite-dimensional linear subspace of $U$ is closed.
\prooflet{(\Schaefer I.3.3;  \Taylor, 3.12-C;  \Rudin, 1.2.1.)}
%\Bourbaki\query

\spheader 4A4Bj\dvAnew{2011}
If $U$ is a first-countable Hausdorff linear topological space which
(regarded as a linear topological space) is complete, then there is a
metric $\rho$ on $U$, defining its topology, under which $U$ is complete.
\prooflet{\Prf\ Let $\Cal W$ be the uniformity of $U$ (3A4Ad).   We know
there is a sequence $\sequencen{G_n}$ running over a base of neighbourhoods
of $0$ in $U$;  setting $W_n=\{(u,v):u-v\in G_n\}$ for each $n$,
$\{W_n:n\in\Bbb N\}$ generates $\Cal W$.   So there is a metric $\rho$ on
$U$ defining $\Cal W$ (4A2Jb).   Because $X$ is $\Cal W$-complete, it is
$\rho$-complete ({\smc Engelking 89}, 8.3.5).\ \Qed}

\leaveitout{
\spheader  If $U$ is a Hausdorff linear topological space, then its
completion $\hat U$\cmmnt{ (3A4H)} has a unique linear structure
extending that of $U$ and rendering it a linear topological space.
\prooflet{(\Bourbaki, I.1.4;  \Schaefer, I.1.5;  \Kothe, \S15.3.)}
%\NB, 5.7.3;  \KN, II.7.11
Every member of $U^*$ has a unique extension to a member of $\hat U^*$,
so $U^*$ and $\hat U^*$ can be identified as linear spaces.
\prooflet{(\Bourbaki, III.3.3;  \Kothe, \S15.9.)}
}%end of leaveitout

\leader{4A4C}{Locally convex spaces (a)} A linear topological space
is {\bf locally convex} if the convex open sets
form a base for the topology.

\spheader 4A4Cb A linear topological space is locally convex iff its
topology can be defined by a family of seminorms\cmmnt{ (2A5B, 2A5D)}.
\prooflet{(\Bourbaki, II.4.1;  \Schaefer, \S II.4;  \Kothe, \S18.1.)}
%\KN, II.6.6;  \NB, 6.5.1
%??

\spheader 4A4Cc Let $U$ be a linear space and $\tau$ a seminorm on $U$.
\cmmnt{Then} $N_{\tau}=\{u:\tau(u)=0\}$ is a linear subspace of $X$.
On the quotient space $U/N_{\tau}$ we have a norm defined by setting
$\|u^{\ssbullet}\|=\tau(u)$ for every $u\in U$.
\prooflet{(\Bourbaki, II.1.3;  \Schaefer, II.5.4;  \Rudin, 1.43.)}

\spheader 4A4Cd Let $U$ be a locally convex linear topological space,
and $\Tau$ the family of continuous seminorms on $U$.   For each
$\tau\in\Tau$, write $N_{\tau}=\{u:\tau(u)=0\}$\cmmnt{, as in (c)
above,} and $\pi_{\tau}$ for the canonical map from
$U$ to $U_{\tau}=U/N_{\tau}$.   Give each $U_{\tau}$ its norm, and set
$\Cal G_{\tau}=\{\pi_{\tau}^{-1}[H]:H\subseteq U_{\tau}$ is open$\}$.
Then $\bigcup_{\tau\in\Tau}\Cal G_{\tau}$ is a base for the topology of
$X$ closed under finite
%actually, countable
unions.
%and finite intersections.
\prooflet{(\Schaefer, II.5.4.)}

\spheader 4A4Ce A linear subspace of a locally convex linear topological
space is locally convex.  \prooflet{(\Bourbaki, II.4.3;  \Kothe,
\S18.3.)}   The product of any family of
locally convex linear topological
spaces\cmmnt{ (4A4Bb)} is locally convex.
\prooflet{(\Bourbaki, II.4.3;  \Kothe, \S18.3.)}
%\Grothendieck, \S1.6

\spheader 4A4Cf If $U$ is a metrizable locally convex linear topological
space, its topology can be defined by a sequence of seminorms.
\prooflet{(\Bourbaki, II.4.1;  \Kothe, \S18.2.)}

\spheader 4A4Cg Let $U$ be a linear space and $V$ a linear subspace
of\cmmnt{ the space} $U'$\cmmnt{ of all linear functionals on $U$}.
Let $\frak T_s(V,U)$ be the topology on $V$ generated by the
seminorms $f\mapsto|f(u)|$ as $u$ runs over $U$\cmmnt{ (compare
4A4Bd)}, and let $\phi:V\to\Bbb R$ be a $\frak T_s(V,U)$-continuous
linear functional.   Then there is a $u\in U$ such that $\phi(f)=f(u)$
for every $f\in V$.
\prooflet{(\Bourbaki, IV.1.1;  \Schaefer, IV.1.2;  \Kothe, \S20.2;
\Rudin, 3.10;  \DS, II.3.9;  \Taylor, 3.81-A.)}

\spheader 4A4Ch {\bf Grothendieck's theorem} If $U$ is a complete
locally convex Hausdorff linear topological space, and $\phi$ is a
linear functional on the dual $U^*$ such that
$\phi\restr G^{\smallcirc}$ is
$\frak T_s(U^*,U)$-continuous for every neighbourhood $G$ of $0$ in
$U$, then $\phi$ is of
the form $f\mapsto f(u)$ for some $u\in U$.
\prooflet{(\Bourbaki, III.3.6;  \Schaefer, IV.6.2;  \Kothe, \S21.9.)}

\leaveitout{
\spheader  The completion of a Hausdorff locally convex linear
topological space\cmmnt{ ()} is locally convex.
\prooflet{(\Bourbaki, II.4.1;  \Schaefer, II.4.1;  \Kothe, \S18.4.)}
%\Grothendieck, \S1.6
If $U$ is a complete Hausdorff locally convex linear topological space
it is isomorphic to a closed subspace of a product of Banach spaces.
\prooflet{(\Bourbaki, II.4.3;  \Schaefer, II.5.4;  \Kothe, \S18.3.)}
}%end of leaveitout

\leader{4A4D}{Hahn-Banach theorem (a)} Let $U$ be a linear space and
$\theta:U\to\coint{0,\infty}$ a seminorm.

\quad(i) If $V\subseteq U$ is a
linear subspace and $g:V\to\Bbb R$ is a linear functional such that
$|g(v)|\le\theta(v)$ for every $v\in V$, then there is a linear
functional $f:U\to\Bbb R$, extending $g$, such that $|f(u)|\le\theta(u)$
for every $u\in U$.

\quad(ii) If $u_0\in U$ then there is
a linear functional $f:U\to\Bbb R$ such that $f(u_0)=\theta(u_0)$ and
$|f(u)|\le\theta(u)$ for every $u\in U$.
\prooflet{(\Bourbaki, II.3.2;  \Rudin, 3.3;
\DS, II.3.11;  \Taylor, 3.7-C;  or use 3A5Aa.)}  
%\NB, 8.4.5;  \KN, I.3.4
%\query check exact refs

\spheader 4A4Db Let $U$ be a linear topological space and $G$, $H$ two
disjoint convex sets in $U$, of which one has non-empty interior.   Then
there are a non-zero $f\in U^*$ and an
$\alpha\in\Bbb R$ such that $f(u)\le\alpha\le f(v)$ for every $u\in G$,
$v\in H$, so that $f(u)<\alpha$ for every $u\in\interior G$ and
$\alpha<f(v)$ for every $u\in\interior H$.
\prooflet{(\Bourbaki, II.5.2;  \Schaefer, II.9.1;  \Kothe,
\S17.1.)}

\leader{4A4E}{The Hahn-Banach theorem in locally convex spaces} Let $U$
be a locally convex linear topological space.

\spheader 4A4Ea If $V\subseteq U$ is a linear subspace, then every
member of $V^*$ extends to a member of $U^*$\cmmnt{ (compare 3A5Ab)}.
\prooflet{(\Bourbaki, II.4.1;  \Schaefer, II.4.2;  \Kothe, \S20.1;
\Rudin, 3.6;  \Taylor, 3.8-D.)}
%\NB, 8.4.6
\cmmnt{

Consequently }$\frak T_s(V,V^*)$ is\cmmnt{ just} the subspace
topology on $V$ induced by $\frak T_s(U,U^*)$.  \prooflet{}

\spheader 4A4Eb Let $C\subseteq U$ be a non-empty closed convex set.
If $u\in U$ then $u\in C$ iff $f(u)\le\sup_{v\in C}f(v)$ for every
$f\in U^*$ iff $f(u)\ge\inf_{v\in C}f(v)$ for every $f\in U^*$.
\prooflet{(\Bourbaki, II.5.3;  \Schaefer, II.9.2;  \Kothe, \S20.7;  \DS,
V.2.12.)}

If $V\subseteq U$ is a closed linear subspace and $u\in U\setminus V$
there
is an $f\in U^*$ such that $f(u)\ne 0$ and $f(v)=0$ for every $v\in V$.
\prooflet{(\Bourbaki, II.5.3;  \Kothe, \S20.1;  \Rudin, 3.5;  \Taylor,
3.8-E.)}
%\NB, 8.5.5

\spheader 4A4Ec If $U$ is Hausdorff, $U^*$ separates its
points\cmmnt{ (compare 3A5Ae)}.
\prooflet{(\Bourbaki, II.4.1;  \Rudin, 3.4.)}
%\NB, 8.5.5

\spheader 4A4Ed If $u\in U$ belongs to the $\frak T_s(U,U^*)$-closure of
a convex set $C\subseteq U$, it belongs to the closure of
$C$\cmmnt{ (compare 3A5Ee)}.
\prooflet{(\Schaefer, II.9.2;  \Kothe, \S20.7;  \Rudin, 3.12.)}
%\NB, 9.7.1
In particular, if $C$ is closed, it is $\frak T_s(U,U^*)$-closed.
\prooflet{(\DS, V.2.13.)}

\spheader 4A4Ee If $C$, $C'\subseteq U$ are disjoint non-empty closed
convex sets, of
which one is compact, there is an $f\in U^*$ such that
$\sup_{u\in C}f(u)<\inf_{u\in C'}f(u)$.
\prooflet{(Apply (b) to $C-C'$.  See \Bourbaki, II.5.3;  \Schaefer,
II.9.2;  \Kothe, \S20.7; \Rudin, 3.4;  \DS, V.3.13.)}
%\NB, 8.6.5

\spheader 4A4Ef Let $V$ be a linear subspace of
$U'$.   Let $K\subseteq U$ be a non-empty $\frak T_s(U,V)$-compact
convex set, and $\phi_0:V\to\Bbb R$ a linear functional such that
$\phi_0(f)\le\sup_{u\in K}f(u)$ for every $f\in V$.   Then there is a
$u_0\in K$ such that $\phi_0(f)=f(u_0)$ for every $f\in V$.
\prooflet{\Prf\ Give $U$ the topology $\frak T_s(U,V)$ and $V'$ the
topology $\frak T_s(V',V)$
(4A4Cg).   For $u\in U$, $f\in V$ set $\hat u(f)=f(u)$;  then
$u\mapsto\hat u$ is a continuous linear operator from $U$ to $V'$
(use 2A3H), so
$\hat K=\{\hat u:u\in K\}$ is a compact convex subset of $V'$.

\Quer\ Suppose, if possible, that $\phi_0\notin\hat K$.   By (b),
there is a continuous linear functional $\pmb{\theta}:V'\to\Bbb R$
such that $\pmb{\theta}(\phi_0)>\sup_{u\in K}\pmb{\theta}(\hat u)$.   But
there is an $f\in V$ such that $\pmb{\theta}(\phi)=\phi(f)$ for every
$\phi\in V'$ (4A4Cg), so that
$\phi_0(f)>\sup_{u\in K}f(u)$, contrary to hypothesis.\ \BanG\  So there
is a $u_0\in K$ such that $\phi_0=\hat u_0$, as claimed.\ \Qed}

\spheader 4A4Eg {\bf The Bipolar Theorem} Let $A\subseteq U'$ be a
non-empty set.   Set
$A^{\smallcirc}=\{u:u\in U$, $f(u)\le 1$ for every
$f\in A\}$\cmmnt{ (compare 4A4Bf)}.   If $g\in U'$ is such that
$g(u)\le 1$ for every $u\in A^{\smallcirc}$, then $g$ belongs to the
$\frak T_s(U',U)$-closed convex hull of $A\cup\{0\}$.
\prooflet{\Prf\ Put 4A4Cg and (b) above together, as in (f).   See
\Bourbaki,
II.6.3;  \Schaefer, IV.1.5;  \Kothe, 20.8.\ \Qed}

\spheader 4A4Eh Let $W$ be a linear subspace of $U'$
separating the points of $U$.   Then $W$ is $\frak T_s(U',U)$-dense in
$U'$.   \prooflet{(For $W^0=\{0\}$.)}

\leader{4A4F}{The Mackey topology} Let $U$ be a linear space and $V$ a
linear subspace of $U'$.   The {\bf Mackey topology} $\frak T_k(V,U)$ on
$V$ is the topology of uniform convergence on convex
$\frak T_s(U,V)$-compact subsets of $U$.   Every
$\frak T_k(V,U)$-continuous linear functional on $V$ is of the form
$f\mapsto f(u)$ for some $u\in U$\prooflet{ (use 4A4Ef)}.   So every
$\frak T_k(V,U)$-closed convex set is
$\frak T_s(V,U)$-closed\prooflet{, by 4A4Ed}.   \cmmnt{(See \Bourbaki,
IV.1.1;  \Schaefer, IV.3.2;  \Kothe, 21.4.)}

\leader{4A4G}{Extreme points (a)} Let $X$ be a real linear space, and
$C\subseteq X$ a convex set.   An element of $C$ is an {\bf extreme}
point of $C$ if it is not expressible as a convex combination of two
other members of $C$\cmmnt{;  equivalently, if it is not expressible
as $\bover12(x+y)$ where $x$, $y\in C$ are distinct}.

\spheader 4A4Gb {\bf The Kre\v\i n-Mil'man theorem} Let $U$ be a
Hausdorff locally convex linear topological space and $K\subseteq U$ a
compact convex set.   Then $K$ is the closed convex hull of the set of
its extreme points.
\prooflet{(\Bourbaki, II.7.1;  \Schaefer, II.10.4;  \Kothe, \S25.1;
\Rudin, 3.22.)}
%\NB, 10.3.4

\spheader 4A4Gc Let $U$ and $V$ be Hausdorff locally convex linear
topological spaces, $T:U\to V$ a continuous linear operator,
$K\subseteq X$ a compact convex set and $v$ any extreme point of
$T[K]\subseteq V$.   Then there is an extreme point $u$ of $K$ such
that $Tu=v$.  \prooflet{\Prf\ Set $K_1=\{u':u'\in K,\,Tu'=v\}$.   Then
$K_1$ is a compact convex set so has an extreme point $u$.   \Quer\
If $u$ is not an extreme point of $K$, it is expressible as
$\alpha u_1+(1-\alpha)u_2$ where $u_1$, $u_2$ are distinct points of $K$
and $\alpha\in\ooint{0,1}$.   So $v=\alpha Tu_1+(1-\alpha)Tu_2$, and we
must
have $Tu_1=Tv=Tu_2$, because $v$ is an extreme point of $T[K]$;  but
this means that $u_1$, $u_2\in K_1$ and $u$ is not an extreme point of
$K_1$.\ \BanG\  So $u$ has the required properties.\ \Qed}

\leader{4A4H}{Proposition} Let $I$ be a set, $W$ a closed linear
subspace of $\BbbR^I$, $U$ a linear topological
space and $V$ a Hausdorff linear topological space.   Let $K\subseteq U$
be a compact set and $T:U\times\BbbR^I\to V$ a continuous linear
operator.   Then $T[K\times W]$ is closed.

\proof{ Take $v_0\in\overline{T[K\times W]}$.   Let $J\subseteq I$ be a
maximal set such that

\inset{\noindent whenever $L\subseteq J$ is finite and $H\subseteq V$ is
an open set containing $v_0$, there are a $u\in K$ and an $x\in W$
such that $x(i)=0$ for every $i\in L$ and $T(u,x)\in H$.}

\noindent Let $\Cal F$ be the filter on $U\times\BbbR^I$ generated by
the closed set $K\times W$, the sets $\{(u,x):x(i)=0\}$ for $i\in J$,
and the sets
$T^{-1}[H]$ for open sets $H$ containing $v_0$.   Then for any $j\in I$
there is an $F\in\Cal F$ such that $\{x(j):(u,x)\in F\}$ is bounded.
\Prf\Quer\ Suppose, if possible, otherwise.   Then, in particular,
$j\notin J$.   So there must be a finite set $L\subseteq J$ and an open
set $H$ containing $v_0$ such that $x(j)\ne 0$ whenever $u\in K$,
$x\in W$, $x(i)=0$ for every $i\in L$ and $T(u,x)\in H$.   By 2A5C, or
otherwise, there is a
neighbourhood $G_0$ of $0$ in $V$ such that $v_0+G_0-G_0\subseteq H$ and
$\alpha v\in G_0$ whenever $v\in G_0$ and $|\alpha|\le 1$.
Fix $u^*\in K$, $x^*\in W$ such that $x^*(i)=0$ for every $i\in L$ and
$T(u^*,x^*)\in v_0+G_0$.   If $x\in W$ and $x(i)=0$ for every
$i\in L$ and $x(j)=x^*(j)$, then $T(u^*,x^*-x)\notin v_0+G_0-G_0$ so
$T(0,x)\notin G_0$.   It follows that $T(0,x)\notin G_0$ whenever
$x\in W$ and $x(i)=0$ for every $i\in L$ and $|x(j)|\ge|x^*(j)|$.

Let $G$ be a neighbourhood of $0$ in $V$ such that $G+G-G-G\subseteq
G_0$.   We are supposing that $\{x(j):(u,x)\in F\}$ is unbounded for
every
$F\in\Cal F$.   So for every $n\in\Bbb N$ there are $u_n\in K$ and
$x_n\in W$ such that $x_n(i)=0$ for $i\in L$,
$T(u_n,x_n)\in v_0+G$ and $|x_n(j)|\ge n$.   Let $u\in K$ be a cluster
point of $\sequencen{u_n}$.  Then $T(u,0)$ is a cluster point of
$\sequencen{T(u_n,0)}$, so $M=\{n:T(u_n,0)\in T(u,0)+G\}$ is infinite.
For $n\in M$, $T(0,x_n)=T(u_n,x_n)-T(u_n,0)\in v_0-T(u,0)+G-G$;  so if
$m$,
$n\in M$, $T(0,x_m-x_n)\in G-G-(G-G)\subseteq G_0$.   But note now that
$(x_m-x_n)(i)=0$ for all $m$, $n\in\Bbb N$ and $i\in L$, and that
because $M$ is infinite there are certainly $m$, $n\in M$ such that
$|x_m(j)-x_n(j)|\ge|x^*(j)|$;  which contradicts the last paragraph.\
\Bang\Qed

Now let $\Cal G$ be any ultrafilter on $U\times\BbbR^I$ including
$\Cal F$.   Then for every $i\in I$ there is a $\gamma_i<\infty$ such
that $\Cal G$ contains $\{(u,x):|x(i)|\le\gamma_i\}$.   It follows that
$\Cal G$ has a limit $(\hat u,\hat x)$ in $K\times W$.   Now the
image filter $T[[\Cal G]]$ (2A1Ib) converges to $T(\hat u,\hat x)$;
since $T[[\Cal F]]\to v_0$, and the topology of $V$ is Hausdorff,
$v_0=T(\hat u,\hat x)\in T[K\times W]$.   As $v_0$ is arbitrary,
$T[K\times W]$ is closed.
}%end of proof of 4A4H

\leader{4A4I}{Normed spaces (a)}
Two norms $\|\,\,\|$, $\|\,\,\|'$ on a linear space $U$
give rise to the same topology iff they are {\bf equivalent} in the
sense that, for some $M\ge 0$,

\Centerline{$\|x\|\le M\|x\|'$,\quad$\|x\|'\le M\|x\|$}

\noindent for every $x\in U$.  \prooflet{(\Kothe, \S14.2;
\Taylor, 3.1-D;  \Jameson, 2.8.)}
%\Grothendieck, \S1.5

\spheader 4A4Ib If $U$ and $V$ are normed spaces, $T:U\to V$ is a linear
operator and $gT:U\to\Bbb R$ is continuous for every $g\in V^*$, then
$T$ is a bounded operator.  \prooflet{(\Jameson, 27.6.)}

\spheader 4A4Ic If $U$ is any normed space, its dual $U^*$, under its
usual norm\cmmnt{ (2A4H)}, is a Banach space.  \prooflet{(\Rudin, 4.1;
\DS, II.3.9;  \Kothe, \S14.5.)}
%4A4Id

\spheader 4A4Id If $U$ is a separable normed space, its dual $U^*$
\cmmnt{(regarded as a normed space)} is isometrically
isomorphic to a closed linear subspace of $\ell^{\infty}$.
\prooflet{\Prf\ Let $\sequencen{x_n}$ run over a dense subset of
the unit ball of $U$ (4A2P(a-iv));  define $T:U^*\to\ell^{\infty}$ by
setting $(Tf)(n)=f(x_n)$ for every $n$.   $T$ is a linear isometry
between $U^*$ and $T[U^*]$, which is closed because $U^*$ is complete
(4A4Ic, 3A4Fd).\ \Qed}

\spheader 4A4Ie Let $U$ be a Banach space.   Suppose that
$\familyiI{u_i}$ is a
family in $U$ such that $\gamma=\sum_{i\in I}\|u_i\|<\infty$.

\quad(i) $\sum_{i\in I}u_i$ is defined\cmmnt{ in the sense of 4A4Bh}.
\prooflet{\Prf\ For $J\in[I]^{<\omega}$, set
$v_J=\sum_{i\in J}u_i$.   For each $n\in\Bbb N$, there is a
$J_n\in[I]^{<\omega}$ such that $\gamma-\sum_{i\in K}\|u_i\|\le 2^{-n}$
whenever $J_n\subseteq K\in[I]^{<\omega}$.   Now

\Centerline{$\|v_{J_m}-v_{J_n}\|\le\sum_{i\in J_m\symmdiff J_n}\|u_i\|
\le 2^{-m}+2^{-n}$}

\noindent for all $m$, $n\in\Bbb N$, so $\sequencen{v_{J_n}}$ is a
Cauchy sequence and has a limit $v$ say.   If now $n\in\Bbb N$ and
$J_n\subseteq K\in[I]^{<\omega}$,

$$\eqalign{\|v-\sum_{i\in K}u_i\|
&=\lim_{m\to\infty}\|v_{J_m}-\sum_{i\in K}u_i\|
\le\limsup_{m\to\infty}\sum_{i\in J_m\symmdiff K}\|u_i\|\cr
&\le\lim_{m\to\infty}2^{-m}+2^{-n}
=2^{-n},\cr}$$

\noindent so $v=\sum_{i\in I}u_i$.\ \Qed}

\quad(ii) Now if $\family{j}{J}{I_j}$ is any partition of $I$,
$w_j=\sum_{i\in I_j}u_i$ is defined for every $j$, and
$\sum_{j\in J}w_j$ is defined and equal to $\sum_{i\in I}u_i$.

\spheader 4A4If Let $U$ be a normed space.   For $u\in U$, define
$\hat u\in U^{**}=(U^*)^*$ by setting $\hat u(f)=f(u)$ for every
$f\in U^*$.   Then $\{\hat u:u\in U,\,\|u\|\le 1\}$ is weak*-dense in
$\{\phi:\phi\in U^{**},\,\|\phi\|=1\}$.   \prooflet{(Apply 4A4Eg with
$A=\{\hat u:\|u\|\le 1\}$.)}

\leader{4A4J}{Inner product spaces (a)}
Let $U$ be an inner product space over
$\RoverC$\cmmnt{ (3A5M)}.
An {\bf orthonormal family} in $U$ is a family $\familyiI{e_i}$ in $U$
such that $\innerprod{e_i}{e_j}=0$ if $i\ne j$, $1$ if $i=j$.
An {\bf orthonormal basis} in $U$ is an orthonormal family
$\familyiI{e_i}$ in $U$ such that the closed
linear subspace of $U$ generated by $\{e_i:i\in I\}$ is $U$ itself.

\spheader 4A4Jb If $U$, $V$ are inner product spaces over $\Bbb R$
and $T:U\to V$ is an isometry such that $T(0)=0$, then
$\innerprod{Tu}{Tv}=\innerprod{u}{v}$ for all $u$, $v\in U$ and $T$ is
linear.   \prooflet{\Prf\ ($\alpha$)

\Centerline{$\innerprod{Tu}{Tv}
=\Bover12(\|Tu\|^2+\|Tv\|^2-\|Tu-Tv\|^2)
=\Bover12(\|u\|^2+\|v\|^2-\|u-v\|^2)
=\innerprod{u}{v}$.}

\noindent($\beta$) For any $u$, $v\in U$,

$$\eqalign{\|T(u+v)-Tu-Tv\|^2
&=\|T(u+v)\|^2+\|Tu\|^2+\|Tv\|^2\cr
&\hskip5em   -2\innerprod{T(u+v)}{Tu}-2\innerprod{T(u+v)}{Tv}
   +2\innerprod{Tu}{Tv}\cr
&=\|u+v\|^2+\|u\|^2+\|v\|^2\cr
&\hskip5em -2\innerprod{u+v}{u}-2\innerprod{u+v}{v}
   +2\innerprod{u}{v}\cr
&=0.\cr}$$

\noindent So $T$ is additive.   ($\gamma$) Consequently $T(qu)=qTu$ for
every $u\in U$ and $q\in\Bbb Q$;  as $T$ is continuous, it is linear.\
\Qed}

\spheader 4A4Jc If $U$, $V$ are inner product spaces over $\Bbb C$ and
$T:U\to V$ is a linear operator such that $\|Tu\|=\|u\|$ for every
$u\in U$, then $\innerprod{Tu}{Tv}=\innerprod{u}{v}$ for all $u$, $v\in U$.
\prooflet{\Prf\

\Centerline{$\Real\innerprod{Tu}{Tv}
=\Bover12(\|Tu\|^2+\|Tv\|^2-\|Tu-Tv\|^2)
=\Real\innerprod{u}{v}$,}

\Centerline{$\Imag\innerprod{Tu}{Tv}
=-\Real(i\innerprod{Tu}{Tv})
=-\Real\innerprod{T(iu)}{Tv}
=-\Real\innerprod{iu}{v}
=\Imag\innerprod{u}{v}$.  \Qed}}

\spheader 4A4Jd If $U$ is an inner product space over $\RoverC$, a
linear operator $T:U\to U$ is {\bf self-adjoint} if
$\innerprod{Tu}{v}=\innerprod{u}{Tv}$ for all $u$, $v\in U$.
%4A4M

\spheader 4A4Je If $U$ is a finite-dimensional inner product space over
$\Bbb R$, it
is isomorphic to Euclidean space $\BbbR^r$, where $r=\dim U$.
\prooflet{(\Taylor, 3.21-A.)}   In particular, any
finite-dimensional inner product space is a Hilbert space.

\spheader 4A4Jf If $U$ is an inner product space over $\RoverC$ and
$V\subseteq U$ is a linear subspace of $U$, then
$V^{\perp}=\{x:x\in U,\,\innerprod{x}{y}=0$ for every $y\in V\}$ is a
linear subspace of $U$, and $\|x+y\|^2=\|x\|^2+\|y\|^2$ for $x\in V$,
$y\in V^{\perp}$.   If $V$ is complete (in particular, if $V$ is
finite-dimensional), then $U=V\oplus V^{\perp}$.   \prooflet{(\Rudin,
12.4;  \Bourbaki, V.1.6;  \Taylor, 4.82-A.)}

\spheader 4A4Jg If $U$ is an inner product space over $\Bbb R$ and
$v_1$, $v_2\in U$ are such
that $\|v_1\|=\|v_2\|=1$, there is a linear operator $T:U\to U$ such
that $Tv_1=v_2$ and $\|Tu\|=\|u\|$ and $\|Tu-u\|\le\|v_1-v_2\|\|u\|$ for
every $u\in U$.   \prooflet{\Prf\ If $v_2$ is a multiple of $v_1$, say
$v_2=\alpha v_1$, take $Tu=\alpha u$ for every $u$.   Otherwise, set
$w=v_2-\innerprod{v_2}{v_1}v_1$ and $w_1=\Bover1{\|w\|}w$, so that
$v_2=\cos\theta v_1+\sin\theta w_1$, where
$\theta=\arccos\innerprod{v_2}{v_1}$.   Let $V$ be the two-dimensional
linear subspace of $U$ generated by $v_1$ and $w_1$, so that
$U=V\oplus V^{\perp}$.   Define a linear operator $T:U\to U$ by saying
that $Tv_1=v_2$, $Tw_1=\cos\theta w_1-\sin\theta v_1$ and $Tu=u$ for
$u\in V^{\perp}$.   Then $T$ acts on $V$ as a simple rotation through an
angle $\theta$, so $\|Tv\|=1$ and $\|Tv-v\|=\|v_2-v_1\|$ whenever
$v\in V$ and $\|v\|=1$;  generally, if $u\in U$, then $\|Tu\|=\|u\|$
and

\Centerline{$\|Tu-u\|=\|T(Pu)-Pu\|=\|v_2-v_1\|\|Pu\|
\le\|v_2-v_1\|\|u\|$,}

\noindent where $P$ is the orthogonal projection of $U$ onto $V$.\
\Qed}

\spheader 4A4Jh Let $U$ be an inner product space over $\RoverC$, and
$\familyiI{u_i}$
a countable family in $U$.   Then there is a countable orthonormal
family $\family{j}{J}{v_j}$ in $U$ such that $\{v_j:j\in J\}$ and
$\{u_i:i\in I\}$ span the same linear subspace of $U$.
\prooflet{\Prf\ We can suppose that $I\subseteq\Bbb N$;  set $u_i=0$ for
$i\in\Bbb N\setminus I$.   Define $\sequencen{v_n}$ inductively by
setting $v_n'=u_n-\sum_{i<n}\innerprod{u_n}{v_i}v_i$, $v_n=0$ if
$v'_n=0$, $\Bover1{\|v'_n\|}v'_n$ otherwise.   Set $J=\{n:v_n\ne 0\}$.\
\Qed}

\spheader 4A4Ji\dvAnew{2008}
Let $U$ be an inner product space over $\RoverC$, and
$\family{i}{I}{e_i}$ an orthonormal family in $U$.   Then
$\sum_{i\in I}|\innerprod{u}{e_i}|^2\le\|u\|^2$ for every $u\in U$.
\prooflet{(\DS, p.\ 252;  \Taylor, 3.2-D.)}
\leaveitout{Take any finite $J\subseteq I$ and set
$v=\sum_{i\in J}\innerprod{u}{e_i}e_i$.   Then
$\innerprod{v}{e_i}=\innerprod{u}{e_i}$ for every $i\in J$, so
$\innerprod{u}{v}=\|v\|^2$ and 

\Centerline{$\sum_{i\in J}|\innerprod{u}{e_i}|^2
=\|v\|^2
=2\Real\innerprod{u}{v}-\|v\|^2
=\|u\|^2-\|u-v\|^2
\le\|u\|^2$.}

\noindent As $J$ is arbitrary, 
$\sum_{i\in I}|\innerprod{u}{e_i}|^2\le\|u\|^2$.}

\spheader 4A4Jj Let $U$ be an inner product space over $\RoverC$, and
$C\subseteq U$ a convex set.   Then there is at most one point $u\in C$
such that $\|u\|\le\|v\|$ for every $v\in C$.   \prooflet{\Prf\ If $u$,
$u'$ both have this property, then $v=\bover12(u+u')\in C$, and
$\|u\|=\|u'\|\le\|v\|$;  but $4\|v\|^2+\|u-u'\|^2=2(\|u\|^2+\|u'\|^2)$,
so $\|u-u'\|=0$ and $u=u'$.\ \Qed}

For such a $u$, $\|u\|^2\le\Real\innerprod{u}{v}$ for every $v\in C$.
\prooflet{\Prf\ For $\alpha\in\ocint{0,1}$,

\Centerline{$\|u\|^2\le\|\alpha v+(1-\alpha)u\|^2
=\|u\|^2+2\alpha(\Real\innerprod{u}{v}-\|u\|^2)
  +\alpha^2\|v-u\|^2$,}

\noindent so $\Real\innerprod{u}{v}-\|u\|^2
\ge-\lim_{\alpha\downarrow 0}\bover12\alpha\|v-u\|^2$.\ \Qed}

\leader{4A4K}{Hilbert spaces (a)}
If $U$ is a real or complex Hilbert space, its unit ball is compact in
the weak topology
$\frak T_s(U,U^*)$;  any bounded set is relatively compact for
$\frak T_s(U,U^*)$.  \prooflet{(\Bourbaki, V.1.7;  \DS, IV.4.6.)}

\spheader 4A4Kb If $U$ is a real or complex Hilbert space, any
norm-bounded sequence in $U$ has a weakly convergent subsequence.
\prooflet{(462D;  \DS, IV.4.7.)}

\spheader 4A4Kc If $U$ is a real or complex Hilbert space and
$\familyiI{u_i}$ is any
orthonormal family in $U$, then it can be extended to an orthonormal
basis.   \prooflet{(\DS, IV.4.10;  \Taylor, 3.2-I.)}   In particular,
$U$ has an orthonormal basis.
% \Bourbaki\query

\leader{4A4L}{Compact operators}\cmmnt{ (see 3A5La)}
{\bf (a)} Let $U$, $V$ and $W$ be Banach spaces.   If
$T\in\eurm B(U;V)$ and $S\in\eurm B(V;W)$ and either $S$ or $T$ is a
compact operator, then $ST$ is compact.
\prooflet{(\DS, VI.5.4;  \Jameson, 34.2.)}

\spheader 4A4Lb If $U$ is a Banach space, $T\in\eurm B(U;U)$ is a
compact linear operator and $\gamma\ne 0$ then
$\{u:Tu=\gamma u\}$ is finite-dimensional.
\prooflet{(\Rudin, 4.18;  \Taylor, 5.5-C;  \DS, VII.4.5;  \Jameson,
34.8.)}

\leader{4A4M}{Self-adjoint compact operators} If $U$ is a
Hilbert space and $T:U\to U$ is a
self-adjoint compact linear operator, then $T[U]$ is included in
the closed linear span of $\{Tv:v$ is an eigenvector of $T\}$.
\prooflet{(\Taylor, 6.4-B.)}

\leader{4A4N}{Max-flow Min-cut
Theorem}\cmmnt{ ({\smc Ford \& Fulkerson 56})} Let $(V,E,\gamma)$ be a (finite) transportation network, that is,

\inset{$V$ is a finite set of `vertices',

$E\subseteq\{(v,v'):v,\,v'\in V,\,v\ne v'\}$ is a set of (directed)
`edges',

$\gamma:E\to\coint{0,\infty}$ is a function;}

\noindent we regard a member $e=(v,v')$ of $E$ as `starting' at $v$ and
`ending' at $v'$, and $\gamma(e)$ is the `capacity' of the edge $e$.
Suppose that $v_0$, $v_1\in V$ are distinct vertices
such that no edge ends at $v_0$ and no edge starts at $v_1$.
%is this needed?  yes, for calculation in (iv)
Then we have a `flow' $\phi:E\to\coint{0,\infty}$ and a `cut'
$X\subseteq E$ such that

(i) for every $v\in V\setminus\{v_0,v_1\}$,

\Centerline{$\sum_{e\in E,e\text{ starts at }v}\phi(e)
=\sum_{e\in E,e\text{ ends at }v}\phi(e)$,}

(ii) $\phi(e)\le\gamma(e)$ for every $e\in E$,

(iii) there is no path from $v_0$ to $v_1$ using only edges in
$E\setminus X$,

(iv) $\sum_{e\in E,e\text{ starts at }v_0}\phi(e)
=\sum_{e\in E,e\text{ ends at }v_1}\phi(e)
=\sum_{e\in X}\gamma(e)$.

\proof{ {\smc Bollob\'as 79}, \S III.1;  {\smc Anderson 87},
12.3.1.
}%end of proof of 4A4N

\discrpage

