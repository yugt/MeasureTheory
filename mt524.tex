\frfilename{mt524.tex}
\versiondate{29.9.10/10.9.13}
\copyrightdate{2003}

\def\chaptername{Cardinal functions of measure theory}
\def\sectionname{Radon measures}

\newsection{524}

It is a remarkable fact that for a Radon measure the principal cardinal
functions are determined by its measure algebra (524J), so can in most
cases be calculated in terms of the cardinals of
the last section (524P-524Q).   The proof of this seems to require a
substantial excursion involving not only measure algebras but also the
Banach lattices $\ell^1(\kappa)$ and/or the $\kappa$-localization
relation (524D, 524E).   The same machinery gives us formulae for the
cardinal functions of measurable algebras (524M).   The results of \S518
can be translated directly to give partial information on the
Freese-Nation numbers of measurable algebras (524O).   For covering
number and uniformity, we can see from 521L that strictly localizable
compact measures follow Radon measures.
I know of no such general results for any other class of measure, but
there are some bounds for cardinal functions of countably compact and
quasi-Radon measures, which I give in 524R-524T.  %524R 524S 524T

\leader{524A}{Notation} If $(X,\Sigma,\mu)$ is a measure space,
$\Cal N(\mu)$ will be the null ideal of $\mu$.
For any cardinal $\kappa$, $\nu_{\kappa}$ will be the usual measure on
$\{0,1\}^{\kappa}$, $\Tau_{\kappa}$ its domain and
$(\frak B_{\kappa},\bar\nu_{\kappa})$ its measure algebra.   As in
\S\S522-523, I will
write $\Cal N_{\kappa}$ for $\Cal N(\nu_{\kappa})$ and $\Cal N$ for the
null ideal of Lebesgue measure on $\Bbb R$, so that
$(\Bbb R,\Cal N)$ and $(\{0,1\}^{\omega},\Cal N_{\omega})$ are
isomorphic\cmmnt{ (522Wa)}.
If $\frak A$ is any Boolean algebra, I write $\frak A^+$ for
$\frak A\setminus\{0\}$ and $\frak A^-$ for $\frak A\setminus\{1\}$.
If $(A,R,B)$ is a supported relation, $R^{\strprime}$ is the
relation $\{(a,I):a\in R^{-1}[I]\}$\cmmnt{ (see 512F)}.
For any cardinal $\kappa$, $(\kappa^{\Bbb N},\subseteq^*,\Cal S_{\kappa})$
will be the $\kappa$-localization relation\cmmnt{ (522K)}.

\leader{524B}{Proposition} Let $(X,\frak T,\Sigma,\mu)$ be a
$\sigma$-finite
Radon measure space with Maharam type $\kappa$.   Then
$\Cal N(\mu)\prT\frak B_{\kappa}^-$.

\proof{{\bf (a)} Suppose, to begin with, that $\mu X=1$ and
$\kappa\ge\omega$.   Let $\frak A$ be the measure algebra of the Radon
product measure $\tilde\lambda$ on $Y=X^{\Bbb N}$.   Then
$\frak A\cong\frak B_{\kappa}$.   \Prf\ By 417E(ii), $\frak A$ is
isomorphic
to the measure algebra of the usual product measure $\lambda$ on $Y$,
which by 334E is isomorphic to $\frak B_{\kappa}$.\ \Qed

For $E\in\Cal N(\mu)$, let
$\sequence{i}{F_{Ei}}$ be a sequence of closed subsets of
$X$ such that $E\cap F_{Ei}=\emptyset$ and $\mu F_{Ei}\ge 1-2^{-i-1}$
for every $n\in\Bbb N$.   Then

\Centerline{$\tilde\lambda(\prod_{i\in\Bbb N}F_{Ei})
=\lambda(\prod_{i\in\Bbb N}F_{Ei})
\ge\prod_{i\in\Bbb N}(1-2^{-i-1})>0$;}

\noindent set

\Centerline{$\phi(E)=(Y\setminus\prod_{i\in\Bbb N}F_{Ei})^{\ssbullet}
\in\frak A^-$.}

For $b\in\frak A^-$ let $K_b\subseteq Y$ be a non-empty compact
self-supporting set such that $K_b^{\ssbullet}\Bcap b=0$.   Set
$\pi_i(y)=y(i)$ for $i\in\Bbb N$ and $y\in Y$.   Then each
$\pi_i[K_b]\subseteq X$ is compact and
$K_b\subseteq\prod_{i\in\Bbb N}\pi_i^{-1}[\pi_i[K_b]]$, so
$\prod_{i\in\Bbb N}\mu\pi_i[K_b]>0$ and
$\sup_{i\in\Bbb N}\mu\pi_i[K_b]=1$;  set

\Centerline{$\psi(b)
=X\setminus\bigcup_{i\in\Bbb N}\pi_i[K_b]
\in\Cal N(\mu)$.}

If $E\in\Cal N(\mu)$ and $b\in\frak A^-$ and $\phi(E)\Bsubseteq b$
and $j\in\Bbb N$, then

\Centerline{$K_b\setminus\pi_j^{-1}[F_{Ej}]
\subseteq K_b\setminus\prod_{i\in\Bbb N}F_{Ei}$}

\noindent is negligible.   As $K_b$ is self-supporting,
$K_b\setminus\pi_j^{-1}[F_{Ej}]$ is empty and
$\pi_j[K_b]\subseteq F_{Ej}$.   But this means that
$\pi_j[K_b]\cap E=\emptyset$ for every $j\in\Bbb N$, so that
$E\subseteq\psi(b)$.

This shows that $\phi$ is a Tukey function, so that
$\Cal N(\mu)\prT\frak A^-\cong\frak B_{\kappa}^-$.

\medskip

{\bf (b)} If $\kappa$ is finite, $\Cal N(\mu)$ has a greatest member and
the constant function with value $0$ is a Tukey function from
$\Cal N(\mu)$ to $\frak B_{\kappa}^-$ and the result is trivial.   If
$\kappa$ is infinite and $\mu X\ne 1$, then, because $\mu$ is
$\sigma$-finite and not trivial, there is a function
$f:X\to\ooint{0,\infty}$ such that $\int fd\mu=1$ (215B(ix)).   Let
$\nu$ be the corresponding indefinite-integral measure;  then $\nu$ is a
Radon probability measure (416S) with the same measurable sets and the
same negligible sets as $\mu$ (234L),
so $\Sigma/\Cal N(\nu)=\Sigma/\Cal N(\mu)$ has Maharam type $\kappa$.
In this case, (a) tells us that
$\Cal N(\mu)=\Cal N(\nu)\prT\frak B_{\kappa}^-$.
}%end of proof of 524B

\leader{524C}{Lemma} Let $P$ be a partially ordered set such that
$p\vee q=\sup\{p,q\}$ is defined for all $p$, $q\in P$.   Suppose that
$\rho$ is a metric on $P$ such that $P$ is complete\cmmnt{ (as a
metric space)} and $\vee:P\times P\to P$ is
uniformly continuous with respect to $\rho$.   Let $Q\subseteq P$ be an
open set, and $\kappa\ge d(Q)$ a cardinal.   Then
$(Q,\le^{\strprime},[Q]^{<\omega})
\prGT(\ell^1(\kappa),\le,\ell^1(\kappa))$.   If $Q$ is upwards-directed,
then $Q\prT\ell^1(\kappa)$.

\proof{{\bf (a)} If $Q$ is finite, then we can set $\phi(q)=0$ for every
$q\in Q$, $\psi(x)=Q$ for every $x\in\ell^1(\kappa)$ and $(\phi,\psi)$
will be a Galois-Tukey connection from
$(Q,\le^{\strprime},[Q]^{<\omega})$ to
$(\ell^1(\kappa),\le,\ell^1(\kappa))$.
So let us suppose that $Q$ and $\kappa$ are infinite.

\medskip

{\bf (b)} Let $\ofamily{\xi}{\kappa}{q_{\xi}}$ run over a dense subset
of $Q$.   For each $q\in Q$ let $m(q)\in\Bbb N$ be such that
$\{p:p\in P$, $\rho(p,q)\le 2^{-m(q)}\}\subseteq Q$.   For each
$n\in\Bbb N$, let $\delta_n>0$ be such that
$\rho(\sup I,\sup J)\le 2^{-n}$ whenever
$\emptyset\ne I\subseteq J\subseteq P$ and $\#(J)\le 2^n$ and
$\max_{q\in J}\min_{p\in I}\rho(p,q)\le 2\delta_n$;   such exists
because $\ofamily{i}{k}{p_i}\mapsto\sup_{i<k}p_i:P^k\to P$ is uniformly
continuous whenever $k>0$, and in particular when $k=2^n$.   Reducing
the $\delta_n$ if necessary, we may suppose that
$\delta_{n+1}\le\delta_n\le 2^{-n}$ for every $n$.

\medskip

{\bf (c)} Define $\phi:Q\to\ell^1(\kappa)$ as follows.   Given $p\in Q$,
choose a sequence $\sequencen{\xi(p,n)}$ in $\kappa$ such that
$\rho(p,q_{\xi(p,n)})\le\delta_{n+1}$ for every $n$.   Take
$\phi(p)\in\ell^1(\kappa)$ such that

\Centerline{$\phi(p)(m(p))\ge 1$,
\quad$\phi(p)(\xi(p,n))\ge 2^{-n}$ for every $n\in\Bbb N$}

\noindent (regarding $m(p)$ as a finite ordinal).

\medskip

{\bf (d)} Define $\psi:\ell^1(\kappa)\to[Q]^{<\omega}$ as follows.
Given $x\in\ell^1(\kappa)$, set
$K_n(x)=\{q_{\xi}:\xi<\kappa,\,x(\xi)\ge 2^{-n}\}$ for $n\in\Bbb N$.
Then

\Centerline{$\sum_{n=0}^{\infty}2^{-n}\#(K_n(x))
\le\sum_{\xi<\kappa}\sum\{2^{-n}:x(\xi)\ge 2^{-n}\}\le 2\|x\|_1<\infty$,}

\noindent so there
is a $k(x)\in\Bbb N$ such that $x(n)<1$ for $n\in\omega\setminus k(x)$
and also $\#(K_n(x))\le 2^n$ for $n\ge k(x)$.   Set
$\tilde K(x)=K_{k(x)}(x)$.   For $s\in\tilde K(x)$ set

\Centerline{$I(x,s,k(x))=\{s\}$,
\quad$I(x,s,n+1)
=\{q:q\in K_{n+1}(x),\,\rho(q,I(x,s,n))\le 2\delta_{n+1}\}$}

\noindent for $n\ge k(x)$, writing $\rho(q,I)$ for
$\inf_{q'\in I}\rho(q,q')$.   Because $\sequencen{K_n(x)}$ is
non-decreasing, so is $\langle I(x,s,n)\rangle_{n\ge k(x)}$.   Set
$r_{xsn}=\sup I(x,s,n)$ in $P$ for $n\ge k(x)$;  then
$\rho(r_{x,s,n+1},r_{xsn})\le 2^{-n-1}$ for every $n\ge k(x)$, by the
choice of $\delta_{n+1}$, so $r_{xs}=\lim_{n\to\infty}r_{xsn}$ is
defined in $P$.   Set $\psi(x)=Q\cap\{r_{xs}:s\in\tilde K(x)\}$.

\medskip

{\bf (e)} Now $(\phi,\psi)$ is a Galois-Tukey connection from
$(Q,\le^{\strprime},[Q]^{<\omega})$ to
$(\ell^1(\kappa),\le,\ell^1(\kappa))$.
\Prf\ Suppose that $p\in Q$ and $x\in\ell^1(\kappa)$ are such that
$\phi(p)\le x$.   Then $q_{\xi(p,n)}\in K_n(x)$ for every $n$, so
$s=q_{\xi(p,k(x))}\in\tilde K(x)$.   Also $q_{\xi(p,n)}\in I(x,s,n)$ for
every $n\ge k(x)$, because

\Centerline{$\rho(q_{\xi(p,n+1)},q_{\xi(p,n)})
\le\delta_{n+2}+\delta_{n+1}\le 2\delta_{n+1}$}

\noindent for every $n$.   So $q_{\xi(p,n)}\le r_{xsn}$ for every
$n\ge k(x)$.   It follows that

\Centerline{$p\vee r_{xs}=\lim_{n\to\infty}q_{\xi(p,n)}\vee r_{xsn}
=\lim_{n\to\infty}r_{xsn}=r_{xs}$}

\noindent and $p\le r_{xs}$.

By the choice of $k(x)$, we also have $\phi(p)(n)<1$ for $n\ge k(x)$, so
that $m(p)<k(x)$.   We therefore have

$$\eqalignno{\rho(r_{xs},p)
&\le\rho(q_{\xi(p,k(x))},p)
  +\sum_{n=k(x)}^{\infty}\rho(r_{x,s,n+1},r_{xsn})\cr
\displaycause{because $s=q_{\xi(p,k(x))}$ is the unique member of
$I(x,s,k(x))$, so is equal to $r_{x,s,k(x)}$}
&\le\delta_{k(x)+1}+\sum_{n=k(x)}^{\infty}2^{-n-1}
\le 2^{-k(x)-1}+2^{-k(x)}
\le 2^{-m(p)}\cr}$$

\noindent and $r_{xs}\in Q$.   So $p\le r_{xs}\in\psi(x)$ and
$p\le^{\strprime}\psi(x)$.   As $p$ and $x$ are arbitrary,
$(\phi,\psi)$ is a Galois-Tukey connection.\ \Qed

\medskip

{\bf (f)} So $(Q,\le^{\strprime},[Q]^{<\omega})
\prGT(\ell^1(\kappa),\le,\ell^1(\kappa))$, as claimed.

Finally, if $Q$ is upwards-directed, then $\add Q\ge\omega$, so
$(Q,\le,Q)\equivGT(Q,\le^{\strprime},[Q]^{<\omega})$ (513Id) and
$(Q,\penalty-50\le\nobreak,\penalty-50Q)\penalty-100
   \prGT(\ell^1(\kappa),\le,\ell^1(\kappa))$, that is,
$Q\prT\ell^1(\kappa)$.
}%end of proof of 524C

\leader{524D}{Proposition} If $\kappa$ is any cardinal,

\Centerline{$(\frak B_{\kappa}^-,\Bsubseteqshort^{\strprime},
  [\frak B_{\kappa}^-]^{<\omega})
\prGT(\ell^1(\kappa),\le,\ell^1(\kappa))$.}

\proof{ If $\kappa$ is finite then $\frak B_{\kappa}^-$ is finite and
the result is trivial.   Otherwise, if we give $\frak B_{\kappa}$ its
measure metric $\rho$ (323Ad), then it is a complete metric space in
which $\Bcup$ is uniformly continuous (323Gc, 323B) and
$\frak B_{\kappa}^-=\frak B_{\kappa}\setminus\{1\}$ is an open set.
Now the topological density of $\frak B_{\kappa}^-$ and
$\frak B_{\kappa}$ is $\kappa$, by 521E;  so
524C gives the result.
}%end of proof of 524D

\leader{524E}{Proposition} Let $\kappa$ be an infinite cardinal.   Then

\Centerline{$(\ell^1(\kappa),\le^{\strprime},
  [\ell^1(\kappa)]^{\le\omega})
\prGT(\kappa^{\Bbb N},\subseteq^*,\Cal S_{\kappa})$.}

\proof{{\bf (a)} For each $i\in\Bbb N$, let
$\ofamily{\xi}{\kappa}{z_{i\xi}}$ run over a norm-dense subset of
$\{x:x\in\ell^1(\kappa)^+,\,\|x\|_1\le 4^{-i}\}$.   Now there is a
function $\phi:\ell^1(\kappa)\to\kappa^{\Bbb N}$ such that

\Centerline{for every $x\in\ell^1(\kappa)$, $n\in\Bbb N$ there is a
$k\in\Bbb N$ such that $x\le k\sum_{i=n}^{\infty}z_{i,\phi(x)(i)}$.}

\noindent\Prf\ Given $x\in\ell^1(\kappa)$, choose $\sequencen{x_n}$,
$\sequencen{\xi_n}$, $\sequencen{k_n}$ inductively, as follows.   Take
$k_0\ge 1$ such that $\|x^+\|_1\le k_0$;  set $x_0=k_0^{-1}x^+$ and take
$\xi_0<\kappa$ such that $\|x_0-z_{0\xi_0}\|_1<\bover14$.   Given that
$x_n\in\ell^1(\kappa)^+$, $\xi_n<\kappa$ are such that
$\|x_n-z_{n\xi_n}\|_1<4^{-n-1}$, let $k_{n+1}\ge 1$ be such that
$\|x_{n+1}\|_1\le 4^{-n-1}$ where
$x_{n+1}=(x_n-z_{n\xi_n})^++k_{n+1}^{-1}x^+$, and take
$\xi_{n+1}<\kappa$ such that
$\|x_{n+1}-z_{n+1,\xi_{n+1}}\|_1<4^{-n-2}$;  continue.
At the end of the process, set $\phi(x)=\sequencen{\xi_n}$.

Now, for any $n\in\Bbb N$, we have $x\le x^+\le k_nx_n$.   But we also
have, for any $m\ge n$, $x_{m+1}\ge x_m-z_{m\xi_m}$, so that
$x_n\le x_m+\sum_{i=n}^{m-1}z_{i\xi_i}$ for every $m\ge n$.
Since $\|x_m\|_1\le 4^{-m}$ for every $m$,
$\lim_{m\to\infty}x_m=0$ and

\Centerline{$x\le k_nx_n\le k_n\sum_{i=n}^{\infty}z_{i,\phi(x)(i)}$.}

\noindent As $x$ and $n$ are arbitrary, $\phi$ is a suitable function.\
\Qed

\medskip

{\bf (b)} Define $\psi_0:\Cal S_{\kappa}\to\ell^1(\kappa)$ by setting
$\psi_0(S)=\sum_{(i,\xi)\in S}z_{i\xi}$;  because

\Centerline{$\sum_{(i,\xi)\in S}\|z_{i\xi}\|_1
\le\sum_{i=0}^{\infty}4^{-i}\#(S[\{i\}])\le\sum_{i=0}^{\infty}2^{-i}$}

\noindent is finite, $\psi_0(S)$ is well defined in $\ell^1(\kappa)$ for
every $S\in\Cal S_{\kappa}$ (4A4Ie).   Now define
$\psi:\Cal S_{\kappa}\to[\ell^1(\kappa)]^{\le\omega}$ by setting
$\psi(S)=\{k\psi_0(S):k\in\Bbb N\}$ for $S\in\Cal S_{\kappa}$.

\medskip

{\bf (c)} If $x\in\ell^1$ and $S\in\Cal S_{\kappa}$ are such that
$\phi(x)\subseteq^*S$, then $x\le^{\strprime}\psi(S)$.   \Prf\ Let
$n\in\Bbb N$ be such that $(i,\phi(x)(i))\in S$ for every $i\ge n$.
Then there is a $k\in\Bbb N$ such that

\Centerline{$x\le k\sum_{i=n}^{\infty}z_{i,\phi(x)(i)}
\le k\sum_{(i,\xi)\in S}z_{i\xi}=k\psi_0(S)\in\psi(S)$.  \Qed}

\medskip

{\bf (d)} Thus $(\phi,\psi)$ is a Galois-Tukey connection and

\Centerline{$(\ell^1(\kappa),\le^{\strprime},
[\ell^1(\kappa)]^{\le\omega})
\prGT(\kappa^{\Bbb N},\subseteq^*,\Cal S_{\kappa})$.}
}%end of proof of 524E

\leader{524F}{Lemma} Let $(X,\Sigma,\mu)$ be a
countably compact measure space with
Maharam type $\kappa$.

(a) If $\mu$ is a \Mth\ probability measure, there is a family
$\ofamily{\xi}{\kappa}{E_{\xi}}$ in $\Cal N(\mu)$ such
that $\bigcup_{\xi\in A}E_{\xi}$ has full outer
measure for every uncountable $A\subseteq\kappa$.

(b) If $\mu$ is $\sigma$-finite, there is a family $\ofamily{\xi}{\kappa}{E_{\xi}}$ in
$\Cal N(\mu)$ such that $\bigcup_{\xi\in A}E_{\xi}$ is non-negligible
for every uncountable $A\subseteq\kappa$.

\proof{ Let $\frak A$ be the measure algebra of $(X,\Sigma,\mu)$.

\medskip

{\bf (a)} If $\kappa$ is countable, we can take $E_{\xi}=\emptyset$ for
every $\xi$.   Otherwise, $\frak A$ is
$\tau$-generated by a stochastically independent family
$\ofamily{\xi}{\kappa}{a_{\xi}}$ of elements of measure $\bover12$, and
for every $G\in\Sigma$ there is a smallest countable set
$I_G\subseteq\kappa$ such that $G^{\ssbullet}$ is in the closed
subalgebra of $\frak A$ generated by $\{a_{\xi}:\xi\in I_G\}$ (254Rd or
325Mb).
For each $\xi<\kappa$ choose $F_{\xi}\in\Sigma$ such that
$F_{\xi}^{\ssbullet}=a_{\xi}$.   Let $\Cal K$ be a countably compact
class such that $\mu$ is inner regular with respect to $\Cal K$.

Let $\ofamily{\xi}{\kappa}{J_{\xi}}$ be a disjoint family of subsets of
$\kappa$ all with cardinal $\omega_1$.
For each $\xi<\kappa$ choose $\sequencen{K_{\xi n}}$,
$\sequencen{\alpha_{\xi n}}$ inductively, as follows.
$\alpha_{\xi 0}=\min J_{\xi}$.   Given $\alpha_{\xi n}$ and
$\ofamily{i}{n}{K_{\xi i}}$, let $K_{\xi n}\in\Cal K$ be such that
$K_{\xi n}\cap F_{\alpha_{\xi n}}=\emptyset$ and
$\mu(K_{\xi n})\ge\bover12(1-3^{-n-2})$;  now let $\alpha_{\xi,n+1}$ be
a member of $J_{\xi}$ not belonging to
$I_{K_{\xi i}}\cup\{\alpha_{\xi i}\}$ for any $i\le n$.   Continue.
Set

\Centerline{$E_{\xi}=\bigcup_{n\in\Bbb N}\bigcap_{m\ge n}K_{\xi m}
\subseteq\bigcup_{n\in\Bbb N}\bigcap_{m\ge n}(X\setminus F_{\alpha_{\xi m}})$,}

\noindent so that $E_{\xi}$ is negligible, because all the $\alpha_{\xi m}$
are different, so that $\sequence{m}{F_{\alpha_{\xi m}}}$ is stochastically
independent.

Now suppose that
$A\subseteq\kappa$ is uncountable, and that $F\subseteq X$ is measurable and not negligible.   Let
$K\in\Cal K$ be such that $K\subseteq F$ and $\mu K>0$;  let $\xi\in A$ be such that
$I_K\cap J_{\xi}=\emptyset$;  let $n\in\Bbb N$ be such that $\mu K\ge 3^{-n-1}$.
Set
$G_m=K\cap\bigcap_{n\le i<m}K_{\xi i}$ for $m\ge n$.   Then
$I_{G_m}\subseteq I_K\cup\bigcup_{i<m}I_{K_{\xi i}}$ does not contain
$\alpha_{\xi m}$, for any $m$.   This means that

\Centerline{$\mu G_{m+1}
=\mu(G_m\cap K_{\xi m})
\ge\mu(G_m\setminus F_{\alpha_{\xi m}})-\Bover{3^{-m-2}}{2}
=\Bover12(\mu G_m-3^{-m-2})$}

\noindent for every $m\ge n$, and an easy induction shows that
$\mu G_m\ge 3^{-m-1}$ for every $m$.   But this tells us that every
$G_m$ is non-empty;  because $\Cal K$ is a countably compact class,
$K\cap E_{\xi}\supseteq\bigcap_{m\ge n}G_m$ is non-empty, and $F$ meets
$E_{\xi}$.

As $F$ is arbitrary, $\bigcup_{\xi\in A}E_{\xi}$ has full outer measure.

\medskip

{\bf (b)} For the general case, because $\mu$ is $\sigma$-finite, there
is a countable partition of unity $\familyiI{a_i}$ in
$\frak A$ such that all the principal ideals
$\frak A_{a_i}$ are totally finite and \Mth\ (use 332A),
and we can find a partition $\familyiI{X_i}$ of $X$ into measurable sets
such that $X_i^{\ssbullet}=a_i$ for each $i$.   Moreover, the subspace
measure $\mu_{X_i}$ on $X_i$ is countably compact (451Db).
Writing $\kappa_i$ for the Maharam type of $\frak A_{a_i}$, there is a
family $\ofamily{\xi}{\kappa_i}{E_{i\xi}}$ of negligible subsets of
$X_i$ such that $\{\xi:\xi<\kappa_i,\,E_{i\xi}\subseteq E\}$ is
countable for every negligible set $E$.
(Apply (a) to a scalar multiple of
$\mu_{X_i}$.)   Now we know from 332S that
$\kappa=\sup_{i\in I}\kappa_i=\#(\{(i,\xi):i\in I,\,\xi<\kappa_i\})$.
On the other hand, for any negligible set $E\subseteq X$,
$\{(i,\xi):i\in I,\,\xi<\kappa_i,\,E_{i\xi}\subseteq E\}$ is countable.
So if we re-enumerate $\langle E_{i\xi}\rangle_{i\in I,\xi<\kappa_i}$ as
$\ofamily{\xi}{\kappa}{E_{\xi}}$ we shall have an appropriate family.
}%end of proof of 524F

\leader{524G}{Proposition} Let $(X,\frak T,\Sigma,\mu)$ be a
\Mth\ Radon probability space with Maharam type $\kappa\ge\omega$.   Then
$(\kappa^{\Bbb N},\subseteq^*,\Cal S_{\kappa})
\prGT(\Cal N(\mu),\subseteq\penalty+100,\Cal N(\mu))$.

\proof{ (Compare 522M.)

\medskip

{\bf (a)} By 524F, there is a family $\ofamily{\xi}{\kappa}{E_{\xi}}$ in
$\Cal N(\mu)$ such that $\{\xi:E_{\xi}\subseteq E\}$ is countable for
every $E\in\Cal N(\mu)$.
Next, because the measure algebra of $\mu$ is isomorphic to the measure
algebra of the usual measure on $[0,1]^{\Bbb N\times\kappa}$, there is a
stochastically independent family
$\langle G_{i\xi}\rangle_{i\in\Bbb N,\xi<\kappa}$ in $\Sigma$ such that
$\mu G_{i\xi}=2^{-i}$ for every $i\in\Bbb N$ and $\xi<\kappa$.
For $f\in\kappa^{\Bbb N}$ set

\Centerline{$\phi(f)=\bigcup_{n\in\Bbb N}E_{f(n)}
\cup\bigcap_{n\in\Bbb N}\bigcup_{m\ge n}G_{m,f(m)}\in\Cal N(\mu)$.}

\medskip

{\bf (b)} Take $E\in\Cal N(\mu)$ and set
$I_E=\{\xi:E_{\xi}\subseteq E\}$, so that $I_E$ is countable.
Define $\pi_E:X\to\{0,1\}^{\Bbb N\times I_E}$ by setting
$\pi_E(x)(i,\xi)=1$ if $x\in G_{i\xi}$, $0$ otherwise.
Then there is a non-empty compact self-supporting set $K_E$ such that
$\pi_E\restr K_E$ is continuous.   \Prf\
Then $\pi_E$ is measurable, therefore almost continuous (418J), and
there is a non-negligible measurable set $H\subseteq X\setminus E$ such
that $\pi_E\restr H$ is continuous.   Because $\mu$ is inner regular
with respect to the compact self-supporting sets, there is a
non-negligible compact self-supporting $K_E\subseteq H$, and this has
the required property.\ \Qed

$\pi_E[K_E]$ is compact.   Let $\sequencen{W_n(E)}$ run
over the family of open-and-closed subsets $W$ of
$\{0,1\}^{\Bbb N\times I_E}$ meeting $\pi_E[K_E]$.   Then
$\pi_E^{-1}[W_n(E)]$ is a non-empty relatively open subset of $K_E$ for
every $n$;  because $K_E$ is self-supporting, $\pi_E^{-1}[W_n(E)]$ is
never negligible.   Set

\Centerline{$J(E,n,i)
=\{\xi:\xi\in I_E,\,\pi_E^{-1}[W_n(E)]\cap G_{i\xi}=\emptyset\}$}

\noindent for $n$, $i\in\Bbb N$.   Because
$\langle G_{i\xi}\rangle_{i\in\Bbb N,\xi\in I_E}$ is stochastically
independent,

\Centerline{$\sum_{i=0}^{\infty}2^{-i}\#(J(E,n,i))
=\sum\{\mu G_{i\xi}:
  i\in\Bbb N,\,\xi\in I_E,\,G_{i\xi}\cap\pi_E^{-1}[W_n(E)]=\emptyset\}$}

\noindent is finite, by the Borel-Cantelli lemma (273K).   For each $n$,
let $k(E,n)\in\Bbb N$ be such that $2^{-i}\#(J(E,n,i))\le 2^{-n-1}$ for
$i\ge k(E,n)$, and set

\Centerline{$\psi(E)
=\bigcup_{n\in\Bbb N}\{(i,\xi):i\ge k(E,n),\,\xi\in J(E,n,i)\}
\subseteq\Bbb N\times\kappa$.}

\noindent Then

$$\eqalign{\#(\{\xi:(i,\xi)\in\psi(E)\}
&\le\sum_{n\in\Bbb N,k(E,n)\le i}\#(J(E,n,i))\cr
&\le\sum_{n\in\Bbb N,k(E,n)\le i}2^{-n-1}2^i
\le 2^i\cr}$$

\noindent for every $i\in\Bbb N$, and $\psi(E)\in\Cal S_{\kappa}$.

\medskip

{\bf (c)} Now $(\phi,\psi)$ is a Galois-Tukey connection from
$(\kappa^{\Bbb N},\subseteq^*,\Cal S_{\kappa})$ to
$(\Cal N(\mu),\subseteq,\Cal N(\mu))$.   \Prf\ Suppose that
$f\in\kappa^{\Bbb N}$ and $E\in\Cal N(\mu)$ are such that
$\phi(f)\subseteq E$.   Because $E_{f(n)}\subseteq\phi(f)$,
$f(n)\in I_E$ for every $n\in\Bbb N$.   Next, $K_E$ does not meet
$\phi(f)$, so $K_E\cap\bigcap_{n\in\Bbb N}\bigcup_{m\ge n}G_{m,f(m)}$ is
empty, that is,

\Centerline{$\pi_E[K_E]\cap
\bigcap_{n\in\Bbb N}\bigcup_{m\ge n}
\{w:w\in\{0,1\}^{\Bbb N\times I_E},\,w(m,f(m))=1\}=\emptyset$.}

\noindent By Baire's theorem, there is some $m\in\Bbb N$ such that

\Centerline{$\pi_E[K_E]\cap\bigcup_{i\ge m}
\{w:w\in\{0,1\}^{\Bbb N\times I_E},\,w(i,f(i))=1\}$}

\noindent is not dense in $\pi_E[K_E]$, and there is an $n\in\Bbb N$
such that

\Centerline{$W_n(E)\cap\bigcup_{i\ge m}
\{w:w\in\{0,1\}^{\Bbb N\times I_E},\,w(i,f(i))=1\}=\emptyset$.}

\noindent In this case, $f(i)\in J(E,n,i)$ for every $i\ge m$.   But
this means that $(i,f(i))\in\psi(E)$ for every $i\ge\max(m,k(E,n))$, so
that $f\subseteq^*\psi(E)$.   As $f$ and $E$ are arbitrary,
$(\phi,\psi)$ is a Galois-Tukey connection.\ \Qed

\medskip

{\bf (d)} Thus $\phi$ and $\psi$ witness that
$(\kappa^{\Bbb N},\subseteq^*,\Cal S_{\kappa})
\prGT(\Cal N(\mu),\subseteq,\Cal N(\mu))$, as claimed.
}%end of proof of 524G

\leader{524H}{Corollary} Let $\kappa$ be an infinite cardinal, and $\mu$
a \Mth\ Radon probability measure with Maharam type $\kappa$.
Then $(\frak B_{\kappa}^+,\Bsupseteqshort^{\strprime},
  [\frak B_{\kappa}^+]^{\le\omega})$,
$(\ell^1(\kappa),
  \le^{\strprime},\penalty-100[\ell^1(\kappa)]^{\le\omega})$,
$(\kappa^{\Bbb N},\subseteq^*,\Cal S_{\kappa})$
and $(\Cal N(\mu),\subseteq\nobreak,\Cal N(\mu))$ are
Galois-Tukey equivalent.

\proof{ By 512Gb, 524D and 524B,

\Centerline{$(\frak B_{\kappa}^-,\Bsubseteqshort^{\strprime},
  [\frak B_{\kappa}^-]^{<\omega_1})
  \prGT(\frak B_{\kappa}^-,\Bsubseteqshort^{\strprime},
  [\frak B_{\kappa}^-]^{<\omega})
  \prGT(\ell^1(\kappa),\le,\ell^1(\kappa))$,}

\Centerline{$(\Cal N(\mu),\subseteq,\Cal N(\mu))
\prGT(\frak B_{\kappa}^-,\Bsubseteqshort,\frak B_{\kappa}^-)$.}

\noindent So

$$\eqalignno{(\frak B_{\kappa}^+,\Bsupseteqshort^{\strprime},
  [\frak B_{\kappa}^+]^{\le\omega})
&\cong(\frak B_{\kappa}^-,\Bsubseteqshort^{\strprime},
  [\frak B_{\kappa}^-]^{\le\omega})
=(\frak B_{\kappa}^-,\Bsubseteqshort^{\strprime},
  [\frak B_{\kappa}^-]^{<\omega_1})\cr
&\prGT(\ell^1(\kappa),\le^{\strprime},[\ell^1(\kappa)]^{<\omega_1})\cr
\displaycause{512Gd}
&=(\ell^1(\kappa),\le^{\strprime},[\ell^1(\kappa)]^{\le\omega})
\prGT(\kappa^{\Bbb N},\subseteq^*,\Cal S_{\kappa})\cr
\displaycause{524E}
&\prGT(\Cal N(\mu),\subseteq,\Cal N(\mu))\cr
\displaycause{524G}
&\equivGT(\Cal N(\mu),\subseteq^{\strprime},[\Cal N(\mu)]^{\le\omega})\cr
\displaycause{513Id again}
&\prGT(\frak B_{\kappa}^-,\Bsubseteqshort^{\strprime},
  [\frak B_{\kappa}^-]^{\le\omega})\cr}$$

\noindent by 512Gb again.
}%end of proof of 524H

\leader{524I}{Corollary} Let $\mu$ be a \Mth\ Radon
probability measure with infinite Maharam type $\kappa$.   Then

\Centerline{$\add\Cal N(\mu)
=\add\Cal N_{\kappa}=\add_{\omega}\ell^1(\kappa)$,}

\Centerline{$\cf\Cal N(\mu)=\cf\Cal N_{\kappa}=\cf\ell^1(\kappa)$.}

\proof{ By 524H and 512Db,

$$\eqalign{\add(\ell^1(\kappa),\le^{\strprime},
  [\ell^1(\kappa)]^{\le\omega})
&=\add(\Cal N(\mu),\subseteq,\Cal N(\mu))\cr
&=\add(\Cal N(\nu_{\kappa}),\subseteq,\Cal N(\nu_{\kappa}))
=\add(\Cal N_{\kappa},\subseteq,\Cal N_{\kappa}).\cr}$$

\noindent But
$\add(\ell^1(\kappa),\le^{\strprime},[\ell^1(\kappa)]^{\le\omega})
=\add_{\omega}\ell^1(\kappa)$ (513Ia), while
$\add(\Cal N(\mu),\subseteq,\Cal N(\mu))=\add\Cal N(\mu)$ and
$\add(\Cal N_{\kappa},\subseteq,\Cal N_{\kappa})=\add\Cal N_{\kappa}$
(512Ea).   So

\Centerline{$\add_{\omega}\ell^1(\kappa)
=\add\Cal N(\mu)=\add\Cal N_{\kappa}$.}

On the other side, 512Da tells us that

\Centerline{$\cov(\Cal N_{\kappa},\subseteq,\Cal N_{\kappa})
=\cov(\Cal N(\mu),\subseteq,\Cal N(\mu))
=\cov(\ell^1(\kappa),\le^{\strprime},
  [\ell^1(\kappa)]^{\le\omega})$.}

\noindent But

\Centerline{$\cov(\Cal N_{\kappa},\subseteq,\Cal N_{\kappa})
=\cf\Cal N_{\kappa}$,
\quad$\cov(\Cal N(\mu),\subseteq,\Cal N(\mu))
=\cf\Cal N(\mu)$}

\noindent (512Ea).   Next, $\cf\ell^1(\kappa)>\omega$.   \Prf\ If
$\sequencen{x_n}$ is any sequence in $\ell^1(\kappa)$, then (because
$\kappa$ is infinite)
$F_n=\{x:x\le x_n\}$ is nowhere dense (for the norm topology) for any
$n\in\Bbb N$, so $\sequencen{F_n}$ cannot cover $\ell^1(\kappa)$ (4A2Ma)
and $\{x_n:n\in\Bbb N\}$ cannot be cofinal.\ \QeD\   So 512Gf tells us
that

\Centerline{$\cov(\ell^1(\kappa),\le^{\strprime},
  [\ell^1(\kappa)]^{\le\omega})
=\cov(\ell^1(\kappa),\le,\ell^1(\kappa))=\cf\ell^1(\kappa)$.}

\noindent Putting these together,

\Centerline{$\cf\Cal N(\mu)=\cf\Cal N_{\kappa}=\cf\ell^1(\kappa)$}

\noindent as required.
}%end of proof of 524I

\leader{524J}{Theorem} Let $(X,\frak T,\Sigma,\mu)$ and
$(Y,\frak S,\Tau,\nu)$ be
Radon measure spaces with non-zero measure and isomorphic measure
algebras.

(a) $\Cal N(\mu)$ and $\Cal N(\nu)$ are Tukey equivalent,
so $\add\mu=\add\Cal N(\mu)=\add\Cal N(\nu)=\add\nu$ and
$\cf\Cal N(\mu)=\cf\Cal N(\nu)$.

(b) $(X,\in,\Cal N(\mu))$ and $(Y,\in,\Cal N(\nu))$ are Galois-Tukey
equivalent, so
$\cov\Cal N(\mu)=\cov\Cal N(\nu)$ and $\non\Cal N(\mu)=\non\Cal N(\nu)$.

\proof{{\bf (a)} Let $\frak A$, $\frak B$ be the measure algebras of
$\mu$ and $\nu$.   Let $\familyiI{a_i}$ be a partition of unity in
$\frak A^+$ such that all the principal ideals
$\frak A_{a_i}$ are homogeneous and totally finite, and
$\familyiI{b_i}$ a matching family in $\frak B$, so that
$\frak A_{a_i}\cong\frak B_{b_i}$ for every $i$.   Because
$(X,\Sigma,\mu)$ and $(Y,\Tau,\nu)$ are strictly localizable (416B),
there are decompositions $\familyiI{X_i}$ and $\familyiI{Y_i}$ of $X$,
$Y$ respectively such that $X_i^{\ssbullet}=a_i$ and
$Y_i^{\ssbullet}=b_i$ for every $i$ (322M).   Write $\mu_{X_i}$,
$\nu_{Y_i}$ for the corresponding subspace measures;  of course these
are Radon measures (416Rb).   Then
$\Cal N(\mu_{X_i})$ and $\Cal N(\nu_{Y_i})$ are Tukey equivalent for
every $i$.   \Prf\ If the common Maharam type of $\frak A_{a_i}$ and
$\frak B_{b_i}$ is infinite, this is a consequence of 524H.   If
$\frak A_{a_i}=\{0,a_i\}$, then $\mu_{X_i}$ is purely atomic and there
is a single point $x$ of $X_i$ such that $\mu\{x\}=\mu X_i$ (414G).   In
this case $\Cal N(\mu_{X_i})$ has a greatest member $X_i\setminus\{x\}$,
and similarly $\Cal N(\nu_{\kappa_i})$ has a greatest member, so they have
Tukey equivalent cofinal subsets and are Tukey equivalent (513E(d-ii)).\
\Qed

Now $E\mapsto\familyiI{E\cap X_i}$ is a partially-ordered-set
isomorphism between $\Cal N(\mu)$ and $\prod_{i\in I}\Cal N(\mu_{X_i})$.
Similarly, $\Cal N(\nu)$ is isomorphic to
$\prod_{i\in I}\Cal N(\nu_{Y_i})$.   It now follows from 513Eg
that $\Cal N(\mu)$ and $\Cal N(\nu)$ are Tukey equivalent.   Accordingly
$\add\Cal N(\mu)=\add\Cal N(\nu)$ and $\cf\Cal N(\mu)=\cf\Cal N(\nu)$.
By 521Ad, $\add\mu=\add\Cal N(\mu)$ and $\add\nu=\add\Cal N(\nu)$.

\medskip

{\bf (b)} Immediate from 521La, applied in both directions.
}%end of proof of 524J

\leader{524K}{Corollary} Let $(X,\frak T,\Sigma,\mu)$ and
$(Y,\frak S,\Tau,\nu)$ be Radon measure spaces with measure algebras
$\frak A$, $\frak B$ respectively.   If $\frak A$ can be regularly
embedded in $\frak B$, then $\Cal N(\mu)\prT\Cal N(\nu)$.

\proof{ As usual, write $\bar\mu$ and $\bar\nu$ for the functionals on
$\frak A$, $\frak B$ respectively defined from $\mu$ and $\nu$, and let
$\pi:\frak A\to\frak B$ be a regular embedding, that is, an
order-continuous injective Boolean homomorphism.

\medskip

{\bf (a)} Consider first the case in which $\mu$ is totally finite
and $\pi$ is measure-preserving for $\bar\mu$ and $\bar\nu$.  Let
$(\tilde{X},\tilde{\frak T},\tilde{\Sigma},\tilde{\mu})$ and
$(\tilde{Y},\tilde{\frak S},\tilde{\Tau},\tilde{\nu})$ be the Stone spaces
of $(\frak A,\bar\mu)$ and $(\frak B,\bar\nu)$ respectively.   Then $\pi$
corresponds to a continuous function $f:\tilde Y\to\tilde X$ (312Q).
By 418I, the image measure $\tilde\nu f^{-1}$ is a Radon measure on
$\tilde X$.
If $a\in\frak A$ and $\widehat{a}$ is the corresponding open-and-closed set
in $\tilde X$, then

\Centerline{$\tilde\nu f^{-1}[\widehat{a}]=\tilde\nu(\widehat{\pi a})
=\bar\nu(\pi a)=\bar\mu a=\tilde\mu\widehat{a}$.}

\noindent By 415H(v), $\tilde\nu f^{-1}=\tilde\mu$.   By 521Hb,
$\Cal N(\tilde\mu)\prT\Cal N(\tilde\nu)$.   But now 524Ja tells us that

\Centerline{$\Cal N(\mu)\equivT\Cal N(\tilde\mu)\prT\Cal N(\tilde\nu)
\equivT\Cal N(\nu)$.}

\medskip

{\bf (b)} Next, consider the case in which $\mu$ and $\nu$ are
totally finite but $\pi$ is not necessarily measure-preserving.
As it is (sequentially) order-continuous, we have a measure $\mu'$ on $X$
defined by saying that $\mu'E=\bar\nu(\pi E^{\ssbullet})$ for $E\in\Sigma$,
and $\Cal N(\mu')=\Cal N(\mu)$.   Because $\mu'$ is absolutely continuous
with respect to $\mu$, it is an indefinite-integral
measure over $\mu$ (234O)
and is a Radon measure on $X$ (416S).   Taking
$\bar\mu'$ to be the corresponding functional on $\frak A$,
$(\frak A,\bar\mu')$ is the measure algebra of $\mu'$ and $\pi$ is
measure-preserving for $\bar\mu'$ and $\bar\nu$.   So (a) tells us that

\Centerline{$\Cal N(\mu)=\Cal N(\mu')\prT\Cal N(\nu)$.}

\medskip

{\bf (c)} Thirdly, suppose that $\mu$ is totally finite, but $\nu$ might
not be.   Set $\frak B^f=\{b:b\in\frak B$, $\bar\nu b<\infty\}$.   For
$b\in\frak B^f$, set $c_b=\sup\{a:a\in\frak A$, $b\Bcap\pi a=0\}$;  then
$b\Bcap\pi c_b=0$, because $\pi$ is order-continuous.
If $a\in\frak A\setminus\{0\}$, there is a
$b\in\frak B^f$ such that $b\Bcap\pi a\ne 0$, so that $a\notBsubseteq c_b$.
Accordingly $\sup_{b\in\frak B^f}1\Bsetminus c_b=1$ in $\frak A$;
as $\frak A$ is ccc, there is a sequence $\sequencen{b_n}$ in $\frak B^f$ such that
$\sup_{n\in\Bbb N}1\Bsetminus c_{b_n}=1$, that is,
$\bar\nu(a\Bcap\sup_{n\in\Bbb N}b_n)>0$ for every non-zero $a\in\frak A$.

For each $n\in\Bbb N$, choose $F_n\in\Tau$ such that $F_n^{\ssbullet}=b_n$
in $\frak B$, and set $Y'=\bigcup_{n\in\Bbb N}F_n$.   The subspace measure
$\nu_{Y'}$ is $\sigma$-finite, so there is a totally finite measure
$\nuprime$ on $Y'$, an indefinite-integral measure over $\nu_{Y'}$,
with the same null ideal as $\nu_{Y'}$ (use 215B(ix)).
The measures $\nu_{Y'}$ and $\nuprime$ are
both Radon measures (416Rb, 416S).   Setting $b=\sup_{n\in\Bbb N}b_n$ in
$\frak B$, the principal ideal $\frak B_b$ can be identified with the
measure algebra of $\nu_{Y'}$ (322I) and $\nuprime$.   Moreover, the map
$a\mapsto b\Bcap\pi a:\frak A\to\frak B_b$ is an injective order-continuous
Boolean homomorphism.   By (b) and 521Fa,

\Centerline{$\Cal N(\mu)\prT\Cal N(\nuprime)
=\Cal N(\nu_{Y'})\prT\Cal N(\nu)$.}

\medskip

{\bf (d)} For the general case, let $\familyiI{a_i}$ be a partition of
unity in $\frak A$ such that $\bar\mu a_i$ is finite for every $i$, and
set $b_i=\pi a_i$ for each $i$, so that $\familyiI{b_i}$ is a partition of
unity in $\frak B$.   As in the proof of 524J, we have corresponding
partitions $\familyiI{X_i}$, $\familyiI{Y_i}$ of $X$, $Y$ into measurable
sets;  as before, 322M tells us that $\Cal N(\mu)$ and $\Cal N(\nu)$ can be
identified with $\prod_{i\in I}\Cal N(\mu_{X_i})$ and
$\prod_{i\in I}\Cal N(\nu_{Y_i})$ respectively.
Now, for each $i$, we can identify the principal ideals
$\frak A_{a_i}$, $\frak B_{b_i}$ with the measure algebras of the subspace
measures $\mu_{X_i}$ and $\nu_{Y_i}$, and $\pi\restrp\frak A_{a_i}$ is an
order-continuous embedding of $\frak A_{a_i}$ in $\frak B_{b_i}$.   So (c)
tells us that $\Cal N(\mu_{X_i})\prT\Cal N(\nu_{Y_i})$.   Accordingly

\Centerline{$\Cal N(\mu)\cong\prod_{i\in I}\Cal N(\mu_{X_i})
\prT\prod_{i\in I}\Cal N(\nu_{Y_i})\cong\Cal N(\nu)$}

\noindent (513Eg again), and the proof is complete.
}%end of proof of 524K

\leader{524L}{}\cmmnt{ So far we have been looking at cardinals
defined from null ideals.   Of course there is an equally important series
based on measurable algebras, which turns out to be similarly strongly
associated with the cardinal functions of the ideals $\Cal N_{\kappa}$.
I have already developed a good deal of the machinery in the arguments
of this section.   But for `linking numbers' we need a new idea, which
is most clearly expressed in the context of homogeneous algebras.

\medskip

\noindent}{\bf Proposition}\cmmnt{ ({\smc Dow \& Stepr\=ans 94})} Let
$\kappa$ be an infinite cardinal.   Then for any $n\ge 2$ the
$n$-linking number $\link_n(\frak B_{\kappa})$ is the least $\lambda$
such that $\kappa\le 2^{\lambda}$.

\proof{ Let $\lambda$ be the least cardinal such that
$\kappa\le 2^{\lambda}$.

\medskip

{\bf (a)} By 514Cb, $\frak B_{\kappa}$ is isomorphic, as partially
ordered set, to a subset of
$\Cal P(\link(\frak B_{\kappa}))$, so we must have

\Centerline{$2^{\link(\frak B_{\kappa})}
\ge\#(\frak B_{\kappa})\ge\kappa$}

\noindent and $\link(\frak B_{\kappa})\ge\lambda$.   It follows at once
that $\link_n(\frak B_{\kappa})\ge\lambda$ for every $n\ge 2$
(511Ia).

\medskip

{\bf (b)} Now let $n\ge 2$, and take an injective function
$\phi:\kappa\to\{0,1\}^{\lambda}$.
Let $\Cal C$ be the family of measurable cylinders in
$\{0,1\}^{\kappa}$, that is, sets of the form
$\{x:x\in\{0,1\}^{\kappa},\,x\restr I=z\}$, where $I\subseteq\kappa$ is
finite and $z\in\{0,1\}^I$.    For each
$E\in\Tau_{\kappa}\setminus\Cal N_{\kappa}$ we can find
disjoint finite sets $I'_E$, $I''_E$, $J_E\subseteq\kappa$ and
$G_E\in\Tau_{\kappa}$ such that

\inset{setting $C_E=\{x:x\in\{0,1\}^{\kappa}$, $x(\xi)=0$ for $\xi\in I'_E$
and $x(\xi)=1$ for $\xi\in I''_E\}$, and $k_E=\#(I'_E)+\#(I''_E)$,
$\nu_{\kappa}(C_E\setminus E)\le\Bover1{4n}\nu_{\kappa}C_E
=\Bover1{4n}\cdot 2^{-k_E}$;

$G_E$ is determined by coordinates in $J_E$ and
$\nu_{\kappa}(C_E\cap(E\symmdiff G_E))\le\Bover1{4n}\cdot 2^{-nk_E}$;

$\nu_{\kappa}G_E\ge 1-\Bover1{2n}$.}

\noindent\Prf\
By 254Fe, there is a set $W$, expressible as the union of finitely
many measurable cylinders, such that
$\nu_{\kappa}(E\symmdiff W)\le\Bover1{5n}\nu_{\kappa}E$.   Now
$\nu_{\kappa}W\ge\bover9{10}\nu_{\kappa}E$ so
$\nu_{\kappa}(W\setminus E)\le\Bover1{4n}\nu_{\kappa}W$.   $W$ is
determined by coordinates in a finite set, so is expressible as a
disjoint union of non-empty measurable cylinders, and for at least one
of these we must have
$\nu_{\kappa}(C\setminus E)\le\Bover1{4n}\nu_{\kappa}C$;  take such a
one for $C_E$.
Express $C_E$ as $\{x:x\restr I_E=z_E\}$, where $I_E\subseteq\kappa$ is
finite and $z_E\in\{0,1\}^{I_E}$, and set
$I'_E=\{\xi:\xi\in I_E,\,z_E(\xi)=0\}$ and
$I''_E=\{\xi:\xi\in I_E,\,z_E(\xi)=1\}$;  then $\nu_{\kappa}C_E=2^{-k_E}$
and $\nu_{\kappa}(C_E\setminus E)\le\Bover14n\cdot 2^{-k_E}$.

Next, take a set $W'\subseteq\{0,1\}^{\kappa}$, determined by
coordinates in a finite subset $J$ of $\kappa$, such that
$\nu_{\kappa}(E\symmdiff W')\le\Bover1{4n}\cdot 2^{-nk_E}$.   Set

\Centerline{$G_E
=\{x:x\in\{0,1\}^{\kappa},\,\Exists y\in W'\cap C_E,\,
  x\restr\kappa\setminus I_E=y\restr\kappa\setminus I_E\}$,}

\noindent so that $G_E$ is determined by coordinates in
$J_E=J\setminus I_E$ and $G_E\cap C_E=W'\cap C_E$;  accordingly

\Centerline{$\nu_{\kappa}(C_E\cap(E\symmdiff G_E))
=\nu_{\kappa}(C_E\cap(E\symmdiff W'))
\le\nu_{\kappa}(E\symmdiff W')\le\Bover1{4n}\cdot 2^{-nk_E}$.}

Note that $G_E$ and $C_E$ are stochastically independent, so that

$$\eqalign{\nu_{\kappa}C_E(1-\nu_{\kappa}G_E)
&=\nu_{\kappa}(C_E\setminus G_E)
\le\nu_{\kappa}(C_E\setminus E)+\nu_{\kappa}(C_E\cap(E\setminus G_E))\cr
&\le\Bover1{4n}\nu_{\kappa}C_E+\Bover1{4n}(\nu_{\kappa}C_E)^n
\le\Bover1{2n}\nu_{\kappa}C_E\cr}$$

\noindent and $\nu_{\kappa}G_E\ge 1-\Bover1{2n}$.\ \Qed

\medskip

{\bf (c)} Let $Q$ be the set of all quadruples $(k,U,V,W)$ where
$k\in\Bbb N$ and $U$, $V$, $W$ are disjoint open-and-closed subsets of
$\{0,1\}^{\lambda}$ in its usual topology.   For $q=(k,U,V,W)\in Q$, set

\Centerline{$\Cal E_q
=\{E:E\in\Tau_{\kappa}\setminus\Cal N_{\kappa},\,
k_E=k,\,\phi[I'_E]\subseteq U,\,
\phi[I''_E]\subseteq V,\,\phi[J_E]\subseteq W\}$.}

\noindent For any $E\in\Tau_{\kappa}\setminus\Cal N_{\kappa}$, $I'_E$,
$I''_E$ and $J_E$, as chosen in (b) above, are disjoint finite sets, so
$\phi[I'_E]$,
$\phi[I''_E]$ and $\phi[J_E]$ also are, and there is a $q\in Q$ such
that $E\in\Cal E_q$.   Now if $q=(k,U,V,W)\in Q$ and $E_i\in\Cal E_q$
for $i<n$, then $\nu_{\kappa}(\bigcap_{i<n}E_i)>0$.   \Prf\ Set
$I'=\bigcup_{i<n}I'_{E_i}$, $I''=\bigcup_{i<n}I''_{E_i}$ and
$J=\bigcup_{i<n}J_{E_i}$.   Then $\phi[I']\subseteq U$,
$\phi[I'']\subseteq V$ and $\phi[J]\subseteq W$, so that $I'$, $I''$ and
$J$ must be disjoint.   Set

\Centerline{$C=\bigcap_{i<n}C_{E_i}
=\{x:x\in\{0,1\}^{\kappa},\,x(\xi)=0$ for $\xi\in I'$, $x(\xi)=1$ for
$\xi\in I''\}$;}

\noindent  then $\nu_{\kappa}C=2^{-\#(I'\cup I'')}\ge 2^{-nk}$.
Next, setting $G=\bigcap_{i<n}G_{E_i}$,

\Centerline{$\nu_{\kappa}G\ge 1-\sum_{i=0}^{n-1}(1-\nu_{\kappa}G_{E_i})
\ge\Bover12$,}

\noindent and $G$ is stochastically independent of $C$, so that
$\nu_{\kappa}(C\cap G)\ge 2^{-nk-1}$.   Finally,

\Centerline{$\nu_{\kappa}(C\cap G\setminus E_i)
\le\nu_{\kappa}(C_{E_i}\cap G_{E_i}\setminus E_i)
\le\Bover1{4n}\cdot 2^{-nk}$}

\noindent for each $i$, so

\Centerline{$\nu_{\kappa}(C\cap G\setminus\bigcap_{i<n}E_i)
\le 2^{-nk-2}<\nu_{\kappa}(C\cap G)$}

\noindent and $\nu_{\kappa}(\bigcap_{i<n}E_i)>0$.\ \Qed

\medskip

{\bf (d)} This means that if we set $A_q=\{E^{\ssbullet}:E\in\Cal E_q\}$
for each $q\in Q$, then every $A_q$ is an $n$-linked set in
$\frak B_{\kappa}$ and $\bigcup_{q\in Q}A_q=\frak B_{\kappa}^+$.
Because $\{0,1\}^{\lambda}$ is a compact topological space with a
subbase of size
$\lambda\ge\omega$, it has $\lambda$ open-and-closed sets and
$\#(Q)=\lambda$.   So $\family{q}{Q}{A_q}$ witnesses that
$\link_n(\frak B_{\kappa})\le\lambda$, and the proof is complete.
}%end of proof of 524L

\leader{524M}{\bf Theorem} Let $(\frak A,\bar\mu)$ be a semi-finite
measure algebra.   Let $K$ be the set of infinite
cardinals $\kappa$ such that $\frak A$ has a homogeneous principal ideal
with Maharam type $\kappa$.

\vskip-\baselineskip

$$\leqalignno{\#(\frak A)
&=2^{c(\frak A)}\text{ if }\frak A\text{ is finite},
  &\indent\text{(a)}\cr
&=\tau(\frak A)^{\omega}\text{ if $\frak A$ is ccc and infinite}.\cr
\wdistr(\frak A)
&=\infty\text{ if }\frak A\text{ is purely atomic},
  &\indent\text{(b)}\cr
&=\add\Cal N\text{ if }K=\{\omega\},\cr
&=\omega_1\text{ otherwise}.\cr
\pi(\frak A)
&=c(\frak A)\text{ if }\frak A\text{ is purely atomic},
  &\indent\text{(c)}\cr
&=\max(c(\frak A),\cf\Cal N,\sup_{\kappa\in K}\cff[\kappa]^{\le\omega})
   \text{ otherwise}.\cr
\frak m(\frak A)
&=\infty\text{ if }\frak A\text{ is purely atomic},
  &\indent\text{(d)}\cr
&=\min_{\kappa\in K}\cov\Cal N_{\kappa}\text{ otherwise}.\cr
d(\frak A)
&=c(\frak A)\text{ if }\frak A\text{ is purely atomic},
  &\indent\text{(e)}\cr
&=\max(c(\frak A),\sup_{\kappa\in K}\non\Cal N_{\kappa})
  \text{ otherwise}.\cr
&&\indent\text{(f) For }2\le n<\omega,\cr
\link_n(\frak A)
&=c(\frak A)\text{ if }\frak A\text{ is purely atomic},\cr
&=\max(c(\frak A),\min\{\lambda:\tau(\frak A)\le 2^{\lambda}\})
   \text{ otherwise}.\cr}$$

\proof{ The case $\frak A=\{0\}$ is trivial, so I shall assume
henceforth that $\frak A\ne\{0\}$.   Let $\familyiI{a_i}$ be a partition
of unity in $\frak A^+$ such that all the principal ideals
$\frak A_{a_i}$ are homogeneous and totally finite.   For each $i\in I$,
set $\kappa_i=\tau(\frak A_{a_i})$, so that
$\frak A_{a_i}\cong\frak B_{\kappa_i}$, and let $(Z_i,\lambda_i)$ be the
Stone space of $(\frak A_{a_i},\bar\mu\restrp\frak A_{a_i})$.
Let $(\widehat{\frak A},\tilde\mu)$ be the localization of
$(\frak A,\bar\mu)$ (322Q).
$\frak A$ can be identified with
an order-dense Boolean subalgebra of $\widehat{\frak A}$, so that
$\familyiI{a_i}$ is still a partition of unity in $\widehat{\frak A}$.
Because $\frak A^f=\widehat{\frak A}^f$ (322P),
$\frak A_{a_i}$ is still
a principal ideal of $\widehat{\frak A}$, and $\widehat{\frak A}$ can be
identified with the simple product $\prod_{i\in I}\frak A_{a_i}$ (315F).

\medskip

{\bf (a)} This is elementary if $\frak A$ is finite (see 511Ic).   If
$\frak A$ is infinite, then 515Ma tells us that
$\#(\frak A)=\tau(\frak A)^{\omega}$.

\medskip

{\bf (b)}

$$\eqalignno{\wdistr(\frak A)
&=\wdistr(\widehat{\frak A})\cr
\displaycause{514Ee}
&=\min_{i\in I}\wdistr(\frak A_{a_i})\cr
\displaycause{514Ef}
&=\min_{i\in I}\wdistr(\frak B_{\kappa_i})
=\min_{i\in I}\add(\Cal N(\lambda_i))\cr
\displaycause{514Be, because $\Cal N(\lambda_i)$ is the ideal of nowhere
dense subsets of $Z$, by 322R}
&=\min_{i\in I}\add(\Cal N_{\kappa_i})\cr
\displaycause{524Ja}
&=\infty\text{ if }K=\emptyset,\cr
&=\add\Cal N\text{ if }K=\{\omega\},\cr
&=\omega_1\text{ otherwise}\cr}$$

\noindent (523E).

\medskip

{\bf (c)(i)} Consider first an algebra $\frak B_{\kappa}$, where
$\kappa\ge\omega$.   Then $\ci\frak B_{\kappa}^+>\omega$.   \Prf\ If
$\sequencen{b_n}$ is any sequence in $\frak B_{\kappa}^+$, then (because
$\frak B_{\kappa}$ is atomless) we can choose $c_n\Bsubseteq b_n$ such
that $0<\bar\nu_{\kappa}c_n\le 2^{-n-2}$ for each $n\in\Bbb N$.   Set
$c=\sup_{n\in\Bbb N}c_n$, $b=1\Bsetminus c$;  then $b\ne 0$ and
$b_n\notBsubseteq b$ for every $n$, so $\{b_n:n\in\Bbb N\}$ is not
coinitial with $\frak B_{\kappa}^+$.\ \Qed

It follows that

$$\eqalignno{\ci\frak B_{\kappa}^+
&=\cov(\frak B_{\kappa}^+,\Bsupseteqshort,\frak B_{\kappa}^+)
=\cov(\frak B_{\kappa}^+,\Bsupseteqshort^{\strprime},
  [\frak B_{\kappa}^+]^{\le\omega})\cr
\displaycause{512Gf}
&=\cov(\Cal N_{\kappa},\subseteq,\Cal N_{\kappa})\cr
\displaycause{524H}
&=\cf\Cal N_{\kappa}.\cr}$$

\medskip

\quad{\bf (ii)} If $\frak A$ is purely atomic, then
$\frak A_{a_i}=\{0,a_i\}$ for every
$i$, and $\pi(\frak A)=\#(I)=c(\frak A)$.   Otherwise,

$$\eqalignno{\max(c(\frak A),\sup_{i\in I}\pi(\frak A_{a_i}))
&\le\pi(\frak A)\cr
\displaycause{514Da, 514Ed}
&\le\max(\omega,c(\frak A),\sup_{i\in I}\pi(\frak A_{a_i}))\cr
\displaycause{514Ef}
&=\max(c(\frak A),\sup_{\kappa\in K}\pi(\frak B_{\kappa}))
=\max(c(\frak A),\sup_{\kappa\in K}\cf\Cal N_{\kappa})\cr
\displaycause{by (i)}
&=\max(c(\frak A),\cf\Cal N,
  \sup_{\kappa\in K}\cff[\kappa]^{\le\omega}))\cr}$$

\noindent by 523N.

\medskip

{\bf (d)} If $\frak A$ is purely atomic, then $\frak m(\frak A)=\infty$
(511If).   Otherwise,

$$\eqalignno{\frak m(\frak A)
&=\frak m(\widehat{\frak A})\cr
\displaycause{517Id}
&=\min_{i\in I}\frak m(\frak A_{a_i})
=\min_{i\in I}n(Z_i)\cr
\displaycause{517N}
&=\min_{i\in I}\cov\Cal N(\lambda_i)\cr
\displaycause{again because $\Cal N(\lambda_i)$ is the ideal of nowhere
dense subsets of $Z_i$}
&=\min_{i\in I}\cov\Cal N_{\kappa_i}\cr
\displaycause{524Jb}
&=\min_{\kappa\in K}\cov\Cal N_{\kappa},\cr}$$

\noindent as claimed.

\medskip

{\bf (e)(i)} I note first that $d(\frak A_{a_i})=\non\Cal N_{\kappa_i}$
for each $i$.   \Prf\ Let
$A\in\Cal PZ_i\setminus\Cal N(\lambda_i)$ be a set with cardinal
$\non\Cal N(\lambda_i)$.   Then $H=\interior\overline{A}$ is not empty.
Let $a\in\frak A_{a_i}^+$ be such that the corresponding open-and-closed
set $\widehat{a}$ is included in $H$.   Then $\widehat{a}$ can be
identified with the Stone space of $\frak A_a$ (312T);   because
$\frak A_{a_i}$ is homogeneous, and $A\cap\widehat{a}$ is dense in
$\widehat{a}$,

$$\eqalignno{d(\frak A_{a_i})
&=d(\frak A_a)=d(\widehat{a})\cr
\displaycause{514Bd}
&\le\#(A\cap\widehat{a})
\le\non\Cal N(\lambda_i)
=\non\Cal N_{\kappa_i}\cr
\displaycause{524Jb}
&\le d(Z_i)\cr
\displaycause{because $\Cal N(\lambda_i)$ is the ideal of nowhere dense
subsets of $Z_i$, so surely contains no dense set}
&=d(\frak A_{a_i})\cr}$$

\noindent by 514Bd again.\ \Qed

\wheader{524M}{6}{2}{2}{48pt}

\quad{\bf (ii)} If $\frak A$ is purely atomic, $d(\frak A)=c(\frak A)$.
Otherwise,

$$\eqalignno{\max(c(\frak A),\sup_{i\in I}d(\frak A_{a_i}))
&\le d(\frak A)\cr
\displaycause{514Da, 514Ed}
&=d(\widehat{\frak A})\cr
\displaycause{514Ee}
&\le\max(\omega,c(\frak A),\sup_{i\in I}d(\frak A_{a_i}))\cr
\displaycause{514Ef}
&=\max(c(\frak A),\sup_{i\in I}\non\Cal N_{\kappa_i})
=\max(c(\frak A),\sup_{\kappa\in K}\non\Cal N_{\kappa}).\cr}$$

\medskip

{\bf (f)} If $\frak A$ is purely atomic, this is elementary, since any
linked subset of $\frak A^+$ can contain at most one atom.   Otherwise,
set

\Centerline{$\theta
=\max(c(\frak A),\min\{\lambda:\tau(\frak A)\le 2^{\lambda}\})$,
\quad$\theta'=\link_n(\frak A)$.}

\noindent For any $i\in I$, $\kappa_i\le\tau(\frak A)$
(514Ed), so $\kappa_i\le 2^{\theta}$ and
$\link_n(\frak A_{a_i})=\link_n(\frak B_{\kappa_i})\le\theta$ (524L;  of
course the case $\kappa_i=0$ is trivial here).   Accordingly

$$\eqalignno{\theta'
&=\text{link}_n(\widehat{\frak A})\cr
\displaycause{514Ee}
&\le\max(\omega,c(\frak A),\sup_{i\in I}\text{link}_n(\frak A_{a_i}))\cr
\displaycause{514Ef}
&\le\theta.\cr}$$

On the other hand, $c(\frak A)\le\theta'$ (514Da).   For each $i\in I$,
$\link_n(\frak A_i)\le\theta'$ (514Ed), so $\kappa_i\le 2^{\theta'}$
(524L, in the other direction).   Let $A_i$ be a $\tau$-generating
subset of $\frak A_{a_i}$ of size $\kappa_i$.   Now the order-closed
subalgebra of $\frak A$ generated by
$A=\{a_i:i\in I\}\cup\bigcup_{i\in I}A_i$ is $\frak A$, so

\Centerline{$\tau(\frak A)\le\#(A)
=\max(c(\frak A),\sup_{i\in I}\kappa_i)
\le\max(\theta',2^{\theta'})=2^{\theta'}$.}

\noindent But this means that $\theta\le\theta'$ and the two are equal.
}%end of proof of 524M

\cmmnt{\medskip

\noindent{\bf Remark} For the corresponding calculation of $\tau(\frak A)$,
when $(\frak A,\bar\mu)$ is localizable, see 332S.
}

\leader{524N}{Corollary (a)} If $(X,\Sigma,\mu)$ is a semi-finite locally
compact measure space, with $\mu X>0$, then
$\cov\Cal N(\mu)\ge\frak m_{\sigma\text{-linked}}$.

(b) If $\frak A$ is any measurable algebra, then
$\frak m(\frak A)\ge\frak m_{\sigma\text{-linked}}$.

\proof{{\bf (a)} Because $\mu$ is semi-finite and $\mu X>0$, there is an
$E\in\Sigma$ such that $0<\mu E<\infty$.   The subspace measure $\mu_E$ on
$E$ is compact, so $\nu=\Bover1{\mu E}\mu_E$ is a
compact probability measure.   Set $\kappa=\max(\omega,\tau(\nu))$.
Because $\nu$ is a compact measure, there is
a function $f:\{0,1\}^{\kappa}\to E$ which is \imp\ for $\nu_{\kappa}$ and
$\nu$ (343Cd).   Now

$$\eqalignno{\frak m_{\sigma\text{-linked}}
&\le\frak m(\frak B_{\frakc})\cr
\displaycause{because $\frak B_{\frakc}$ is $\sigma$-linked, by 524Mf}
&=\cov\Cal N_{\frakc}\cr
\displaycause{524Md}
&\le\cov\Cal N_{\kappa}\cr
\displaycause{523F}
&\le\cov\Cal N(\nu)\cr
\displaycause{521Ha}
&=\cov\Cal N(\mu_E)
\le\cov\Cal N(\mu)\cr}$$

\noindent (521Fb).

\medskip

{\bf (b)} This is now immediate from 524Md.
}%end of proof of 524N

\leader{524O}{Freese-Nation \dvrocolon{numbers}}\cmmnt{ I spell out
those facts about Freese-Nation numbers of measure algebras which can be
read off from the results in \S518.

\medskip

\noindent}{\bf Proposition} (a) Let $(\frak A,\bar\mu)$ be an infinite
measure algebra.   Then $\FN(\frak A)\ge\FN(\Cal P\Bbb N)$.

(b) Let $\frak A$ be a measurable algebra.

\quad(i) $\FN(\frak A)\le\frak c^+$.

\quad(ii) If $\tau(\frak A)\le\frak c$ then
$\FN(\frak A)\le\FN(\Cal P\Bbb N)$.

\quad(iii) If

\inset{($\alpha$) $\cf([\lambda]^{\le\omega})\le\lambda^+$ for every
cardinal $\lambda\le\tau(\frak A)$,

($\beta$) $\square_{\lambda}$ is true for every uncountable
cardinal $\lambda\le\tau(\frak A)$ of countable cofinality,}

\noindent then $\FN(\frak A)\le\FN^*(\Cal P\Bbb N)$.

(c) Suppose that the continuum hypothesis and
CTP$(\omega_{\omega+1},\omega_{\omega})$ are both true.   If $\frak A$ is
a measurable algebra, then

$$\eqalign{\FN(\frak A)&=\frak c=\omega_1
   \text{ if }\omega\le\tau(\frak A)<\omega_{\omega},\cr
&=\frak c^+=\omega_2\text{ otherwise }.\cr}$$

\proof{{\bf (a)} This is a special case of 518Ca.

\medskip

{\bf (b)(i)} Consider first the case $\frak A=\frak B_{\kappa}$ for some
cardinal $\kappa$.
For $I\subseteq\kappa$, let $\frak C_I$ be the closed subalgebra of
$\frak B_{\kappa}$ consisting of those
$a\in\frak B_{\kappa}$ expressible in the form $E^{\ssbullet}$ for some
measurable $E\subseteq\{0,1\}^{\kappa}$ determined by coordinates in
$I$.   For
$a\in\frak B_{\kappa}$, there is a smallest subset $I_a$ of $\lambda$
such that $a\in\frak C_I$ (325M again);  $I_a$ is always countable.

For each $a\in\frak B_{\kappa}$, set

\Centerline{$f(a)=\{b:I_b\subseteq I_a\}$.}

\noindent Then $\#(f(a))\le\frak c$.   If $a\Bsubseteq b$, then there is
a $c\in\frak B_{\kappa}$ such that
$a\Bsubseteq c\Bsubseteq b$ and $I_c\subseteq I_a\cap I_b$
(325M(b-ii)).   So $f$ is a Freese-Nation function.   This shows that
$\FN(\frak B_{\kappa})\le\frak c^+$.

In general, $\frak A$ is either $\{0\}$ or isomorphic to
a closed subalgebra of $\frak B_{\kappa}$ where
$\kappa=\max(\omega,\tau(\frak A))$, so
$\FN(\frak A)\le\FN(\frak B_{\kappa})\le\frak c^+$ by 518Cc.

\medskip

\quad{\bf (ii)}  $\frak A$ is
$\sigma$-linked (524Mf), so 518D(iii) tells us that
$\FN(\frak A)\le\FN(\Cal P\Bbb N)$.

\medskip

\quad{\bf (iii)} If $\frak B\subseteq\frak A$ is a countably generated
order-closed subalgebra, then $\FN(\frak B)\le\FN(\Cal P\Bbb N)$, by
(ii);  so 518I tells us that $\FN(\frak A)\le\FN^*(\Cal P\Bbb N)$.

\medskip

{\bf (c)} If $\tau(\frak A)<\omega_{\omega}$ then
$\cff[\lambda]^{\le\omega}=\lambda$ for
$\omega_1\le\lambda\le\tau(\frak A)$ (5A1E(e-iv)),
so we can use (a) and (b-iii);  otherwise use (b-i) and 518K.
}%end of proof of 524O

\leader{524P}{The Maharam \dvrocolon{classification}}\cmmnt{ If the
cardinal functions of a Radon measure space are determined by its
measure algebra, there ought to be some way of calculating them directly
from the classification of measure algebras in \S332.   In many cases
this is straightforward.

\wheader{524P}{6}{2}{2}{72pt}
\noindent}{\bf Theorem} Let $(X,\frak T,\Sigma,\mu)$ be a
Radon measure space, and $\frak A$ its measure algebra.
Let $K$ be the set of infinite cardinals $\kappa$ such that the
Maharam-type-$\kappa$ component of $\frak A$ is non-zero.

\vskip-\baselineskip

\newif\iftempa\tempafalse
\newif\iftempb\tempbfalse
\newif\iftempc\tempcfalse
\newif\iftempd\tempdfalse
\newif\iftempe\tempefalse
\pagetotalplus=\the\pagetotal
\advance\pagetotalplus by 120pt
\ifdim\pagetotalplus>\pagegoal\tempatrue\else
\advance\pagetotalplus by 36pt
\ifdim\pagetotalplus>\pagegoal\tempbtrue\else
\advance\pagetotalplus by 36pt
\ifdim\pagetotalplus>\pagegoal\tempctrue\else
\advance\pagetotalplus by 36pt
\ifdim\pagetotalplus>\pagegoal\tempdtrue\else
\advance\pagetotalplus by 36pt
\ifdim\pagetotalplus>\pagegoal\tempetrue
\fi\fi\fi\fi\fi
%\showthe\pagetotal   \showthe\pagetotalplus  \showthe\pagegoal

$$\leqalignno{\add\mu=\add\Cal N(\mu)&=\infty\text{ if }K=\emptyset,
  &\indent\text{(a)}\cr
&=\add\Cal N\text{ if }K=\{\omega\},\cr
&=\omega_1\text{ otherwise}.\cr
\iftempa\noalign{\break}\fi
\pi(\mu)=\pi(\frak A)
&=c(\frak A)\text{ if }K=\emptyset,
  &\indent\text{(b)}\cr
&=\max(c(\frak A),\cf\Cal N,\sup_{\kappa\in K}\cff[\kappa]^{\le\omega})
\text{ otherwise}.\cr
\iftempb\noalign{\break}\fi
\cov\Cal N(\mu)
&=1\text{ if }\frak A=\{0\},
  &\indent\text{(c)}\cr
&=\infty\text{ if }\frak A\text{ has an atom},\cr
&=\cov\Cal N_{\min K}\text{ otherwise}.\cr
\iftempc\noalign{\break}\fi
\non\Cal N(\mu)
&=\infty\text{ if }\frak A=\{0\},
  &\indent\text{(d)}\cr
&=1\text{ if }\frak A\text{ has an atom},\cr
&=\non\Cal N_{\min K}\text{ otherwise}.\cr
\iftempd\noalign{\break}\fi
\shr\Cal N(\mu)
&=0\text{ if }\frak A=\{0\},
  &\indent\text{(e)}\cr
&=1\text{ if }\frak A\text{ has an atom},\cr
&\ge\shr\Cal N\text{ otherwise}.\cr
\iftempe\noalign{\break}\fi
&&\indent\text{(f) If }\mu\text{ is }\sigma\text{-finite},\cr
\cf\Cal N(\mu)
&=1\text{ if }K=\emptyset,\cr
&=\max(\cf\Cal N,\cff[\tau(\frak A)]^{\le\omega})\text{ otherwise}.\cr}$$

\proof{ If $\mu X=0$ all these results are trivial, so let us suppose
henceforth that $\mu X>0$.   As in part (a) of the proof of 524J, there
is a decomposition $\familyiI{X_i}$ of $X$ such that the subspace
measures $\mu_{X_i}$ are all \Mth\ and non-zero.   Note that
$\max(\omega,\#(I))=\max(\omega,c(\frak A))$ (332E).   For each
$i\in I$, let $\kappa_i$ be the Maharam type of $\mu_{X_i}$.

\medskip

{\bf (a)} By 521Ad, $\add\mu=\add\Cal N(\mu)$.
The map $E\mapsto\familyiI{E\cap X_i}$ identifies
$\Cal N(\mu)$, as partially ordered set, with the product of the family
$\familyiI{\Cal N(\mu_{X_i})}$.   So
$\add\Cal N(\mu)=\min_{i\in I}\add\Cal N(\mu_{X_i})$ (511Hg).
Now if $i\in I$ and
$\kappa_i=0$, $X_i$ is an atom of $(X,\Sigma,\mu)$,
so there is an $x_i\in X_i$ such that $\mu(X_i\setminus\{x_i\})=0$
(414G again).
In this case, $X_i\setminus\{x_i\}$ is the largest member of
$\Cal N(\mu_{X_i})$ and $\add\Cal N(\mu_{X_i})=\infty$.   If $\kappa_i$
is infinite, then $\add\Cal N(\mu_{X_i})=\add\Cal N_{\kappa_i}$, by 524I
applied to a scalar multiple of $\mu_{X_i}$.   So
$\add\Cal N(\mu)=\min_{\kappa\in K}\add\Cal N_{\kappa}$,
interpreting this as $\infty$ if $K=\emptyset$.   But we know from 523E
that $\add\Cal N_{\kappa}=\omega_1$ if $\kappa>\omega$, while of course
$\add\Cal N_{\omega}=\add\Cal N$.   It follows at once that

$$\eqalignno{\add\Cal N(\mu)=\min_{i\in I}\add\Cal N(\mu_{X_i})
&=\infty\text{ if }K=\emptyset,\cr
&=\add\Cal N\text{ if }K=\{\omega\},\cr
&=\omega_1\text{ otherwise}.\cr}$$

\medskip

{\bf (b)} By 521Dd, $\pi(\mu)=\pi(\frak A)$;
and 524Mc gives us the formula for $\pi(\frak A)$.

\medskip

{\bf (c)} If $\Cal E$ is a cover of $X$ by negligible sets, and
$i\in I$, then
$\{E\cap X_i:E\in\Cal E\}$ is a cover of $X_i$ by negligible sets;  thus
$\cov\Cal N(\mu)\ge\sup_{i\in I}\cov\Cal N(\mu_{X_i})$.   By 524Jb,
$\cov\Cal N(\mu)\ge\sup_{i\in I}\cov\Cal N_{\kappa_i}$.   If any of the
$\kappa_i$ is zero, that is, if $\frak A$ has an atom, this is $\infty$,
and we can stop.

Otherwise, for each $i\in I$,

\Centerline{$\cov\Cal N(\mu_{X_i})=\cov\Cal N_{\kappa_i}
\le\cov\Cal N_{\min K}=\lambda$}

\noindent say, by 523B.   So we have a family
$\ofamily{\xi}{\lambda}{E_{i\xi}}$ of negligible subsets of $X_i$
covering $X_i$;  setting $E_{\xi}=\bigcup_{i\in I}E_{i\xi}$ for each
$\xi$, we have a family $\ofamily{\xi}{\lambda}{E_{\xi}}$ in
$\Cal N(\mu)$ covering $X$, so $\cov\Cal N(\mu)\le\cov\Cal N_{\min K}$.
But we already know that

\Centerline{$\cov\Cal N(\mu)\ge\sup_{i\in I}\cov\Cal N_{\kappa_i}
\ge\cov\Cal N_{\min K}$,}

\noindent so $\cov\Cal N(\mu)=\cov\Cal N_{\min K}$.

\medskip

{\bf (d)} A set $A\subseteq X$ is non-negligible iff $A\cap X_i$ is
non-negligible for some $i\in I$.   It follows at once that
$\non\Cal N(\mu)=\min_{i\in I}\non\Cal N(\mu_{X_i})$.
If any of the $X_i$ is an atom, it contains a point of non-zero measure,
so that $\non\Cal N(\mu)=1$.   If $\kappa_i\ge\omega$ for every $i$,
then we have

\Centerline{$\non\Cal N(\mu)=\min_{i\in I}\non\Cal N_{\kappa_i}
=\non\Cal N_{\min K}$}

\noindent by 524Jb and 523B again.

\medskip

{\bf (e)} If $\frak A$ is purely atomic, then $\mu$ is point-supported, so
$\shr\Cal N(\mu)=1$.   Otherwise,
let $E$ be a measurable set of non-zero finite measure such that the
subspace measure $\mu_E$ is atomless;  let $\nu$ be the normalized subspace
measure $\Bover1{\mu E}\mu_E$;  then $\nu$, like $\mu_E$, is a Radon
measure.   By 343Cb, there is a
function $f:E\to\{0,1\}^{\omega}$ which
is \imp\ for $\nu$ and $\nu_{\omega}$;   because $\{0,1\}^{\omega}$ is
separable and metrizable, $\nu f^{-1}$ is a Radon measure
(451O, or
418I-418J) and must be equal to $\nu_{\omega}$ (416Eb).   By 521Fd and
521Hb,

\Centerline{$\shr\Cal N(\mu)\ge\shr\Cal N(\mu_E)=\shr\Cal N(\nu)
\ge\shr\Cal N(\nu_{\omega})=\shr\Cal N$.}

\medskip

{\bf (f)(i)} If $K=\emptyset$ then (a) tells us that $\Cal N(\mu)$ has a
greatest member, so that $\cf\Cal N(\mu)=1$.

\medskip

\quad{\bf (ii)} Now suppose that $K$ is not empty.   Then 524Fb tells us
that there is a family
$\ofamily{\xi}{\tau(\frak A)}{E_{\xi}}$ in $\Cal N(\mu)$ such that
$\{\xi:E_{\xi}\subseteq E\}$ is countable for every $E\in\Cal N(\mu)$.
In this case, $J\mapsto\bigcup_{\xi\in J}E_{\xi}:
   [\tau(\frak A)]^{\le\omega}\to\Cal N(\mu)$ is a Tukey function,
so $\cf\Cal N(\mu)\ge\cff[\tau(\frak A)]^{\le\omega}$.
At the same time, there is an $i\in I$ such that $\kappa_i\ge\omega$.
The identity map from $\Cal N(\mu_{X_i})$
to $\Cal N(\mu)$ is a Tukey function;  but this means that

$$\eqalignno{\cf\Cal N(\mu)
&\ge\cf\Cal N(\mu_{X_i})=\cf\Cal N_{\kappa_i}\cr
\displaycause{524I again}
&\ge\cf\Cal N_{\omega}
=\cf\Cal N\cr}$$

\noindent (523B).   Thus
$\cf\Cal N(\mu)\ge\max(\cf\Cal N,\cff[\tau(\frak A)]^{\le\omega})$.

\medskip

\quad{\bf (iii)}
In the other direction, we know from 524H (again, applied to a scalar
multiple of $\mu_{X_i}$) that
$(\Cal N(\mu_{X_i}),\subseteq,\Cal N(\mu_{X_i})
\equivGT(\kappa_i^{\Bbb N},\subseteq^*,\Cal S_{\kappa_i})$ whenever
$\kappa_i$ is infinite.   Now $\tau(\frak A)\ge\kappa_i$, so the maps

\Centerline{identity$:\kappa_i^{\Bbb N}\to\tau(\frak A)^{\Bbb N}$,
\quad$S\mapsto S\cap(\Bbb N\times\kappa_i):
  \Cal S_{\tau(\frak A)}\to\Cal S_{\kappa_i}$}

\noindent form a Galois-Tukey connection from
$(\kappa_i^{\Bbb N},\subseteq^*,\Cal S_{\kappa_i})$ to
$(\tau(\frak A)^{\Bbb N},\subseteq^*,\Cal S_{\tau(\frak A)})$.
Accordingly we have

$$\eqalign{(\Cal N(\mu_{X_i}),\subseteq,\Cal N(\mu_{X_i}))
&\equivGT(\kappa_i^{\Bbb N},\subseteq^*,\Cal S_{\kappa_i})\cr
&\prGT(\tau(\frak A)^{\Bbb N},\subseteq^*,\Cal S_{\tau(\frak A)})
\equivGT(\Cal N_{\tau(\frak A)},\subseteq,\Cal N_{\tau(\frak A)}),\cr}$$

\noindent and $\Cal N(\mu_{X_i})\prT\Cal N_{\tau(\frak A)}$.

The arguments quoted assume that $\kappa_i$ is infinite;  but of course
it is still true that
$\Cal N(\mu_{X_i})\prT\Cal N_{\tau(\frak A)}$ when $\kappa_i=0$, since
then any constant function from $\Cal N(\mu_{X_i})$ to
$\Cal N_{\tau(\frak A)}$ is a Tukey function.   It follows that

\Centerline{$\Cal N(\mu)\cong\prod_{i\in I}\Cal N(\mu_{X_i})
\prT\Cal N_{\tau(\frak A)}^I$}

\noindent (513Eg once more).

\medskip

\quad{\bf (iv)}
At this point observe that as we are assuming that $K\ne\emptyset$,
$\tau(\frak A)$ is infinite;  and as $\mu$ is supposed to be
$\sigma$-finite, $I$ is countable.   So we can find a disjoint family
$\familyiI{F_i}$ of measurable subsets of $\{0,1\}^{\tau(\frak A)}$ such
that all the subspace measures $(\nu_{\tau(\frak A)})_{F_i}$ are
isomorphic to scalar multiples of $\nu_{\tau(\frak A)}$.   (Take
$F_i=\{x:x(n_i)=1$, $x(m)=0$ for $m<n_i\}$ where
$i\mapsto n_i:I\to\Bbb N$ is injective.)   In this case, the map

\Centerline{$\familyiI{E_i}\mapsto\bigcup_{i\in I}E_i:
\prod_{i\in I}\Cal N((\nu_{\tau(\frak A)})_{F_i})
\to\Cal N(\nu_{\tau(\frak A)})$}

\noindent  is a Tukey function, while $\Cal N_{\tau(\frak A)}^I$ is
isomorphic to
$\prod_{i\in I}\Cal N((\nu_{\tau(\frak A)})_{F_i})$.   Putting these
together,

\Centerline{$\Cal N(\mu)\prT\Cal N_{\tau(\frak A)}^I
\cong\prod_{i\in I}\Cal N((\nu_{\tau(\frak A}))_{F_i})
\prT\Cal N_{\tau(\frak A)}$.}

It follows that

\Centerline{$\cf\Cal N(\mu)\le\cf\Cal N_{\tau(\frak A)}
=\max(\Cal N,\cff[\tau(\frak A)]^{\le\omega})$.}

\noindent So we have inequalities in both directions and
$\cf\Cal N(\mu)=\max(\Cal N,\cff[\tau(\frak A)]^{\le\omega})$, as
claimed.
}%end of proof of 524P

\leader{*524Q}{}\cmmnt{ I do not know how to calculate
$\cf\Cal N(\mu)$ for non-$\sigma$-finite
Radon measures $\mu$ without special
assumptions.   In the presence of GCH, however, we have the following
result.

\medskip

\noindent}{\bf Proposition} Suppose that the generalized continuum
hypothesis is true.    Let $(X,\frak T,\Sigma,\mu)$ be a Radon
measure space and $(\frak A,\bar\mu)$ its measure algebra.   For each
cardinal $\kappa$, write $e_{\kappa}$ for the Maharam-type-$\kappa$
component of $\frak A$, and $\frak C_{\kappa}$ for the principal ideal
of $\frak A$ generated by $\sup_{\kappa'>\kappa}e_{\kappa'}$;  set
$\lambda=\sup\{\kappa:e_{\kappa}\ne 0\}$.   Then
$\cf\Cal N(\mu)=\max(c(\frak C_0)^+,\lambda^+)$ unless
$\lambda>c(\frak C_0)$ and there is some $\gamma<\lambda$ such that
$\cf\lambda>c(\frak C_{\gamma})$, in which case
$\cf\Cal N(\mu)=\lambda$.

\proof{{\bf (a)} Write

$$\eqalign{\theta
&=\lambda\text{ if }\lambda>c(\frak C_0)\text{ and }
\cf\lambda>\min_{\gamma<\lambda}c(\frak C_{\gamma}),\cr
&=\max(\lambda^+,c(\frak C_0)^+)\text{ otherwise}.\cr}$$

\noindent If $\mu$ is purely atomic, it is point-supported, so
$\lambda=0$ and $\frak C_0=\{0\}$ and $\theta=1=\cf\Cal N(\mu)$.   So
let us suppose henceforth that $\mu$ is not purely atomic, that is,
$\frak C_0\ne\{0\}$ and $\lambda\ge\omega$.   As in the proofs of
524J and 524P,
there is a decomposition $\familyiI{X_i}$ of $X$ such that the subspace
measures $\mu_{X_i}$ are all \Mth\ and non-zero.   Let
$\kappa_i$ be the Maharam type of $\mu_{X_i}$ for each $i$, so that
$\lambda=\sup_{i\in I}\kappa_i$.   Now
$\Cal N(\mu)\cong\prod_{i\in I}\Cal N(\mu_{X_i})$ (see the proof of
524Ja).   For $i\in I$, $\cf\Cal N(\mu_{X_i})=1$ if $\kappa_i=0$, and
otherwise is
$\max(\cf\Cal N,\cff[\kappa_i]^{\le\omega})
=\max(\omega_1,\cff[\kappa_i]^{\le\omega})$ (524Ja, 523N).   By 5A6Ab,

$$\eqalign{\cf\Cal N(\mu_{X_i})
&=1\text{ if }\kappa_i=0,\cr
&=\kappa_i\text{ if }\cf\kappa_i>\omega,\cr
&=\kappa_i^+\text{ if }\cf\kappa_i=\omega.\cr}$$

\medskip

{\bf (b)} For each cardinal $\kappa$, set
$J_{\kappa}=\{i:i\in I$, $\cf\Cal N(\mu_{X_i})>\kappa\}$, and set

$$\eqalign{\lambda_1=\sup_{i\in I}\cf\Cal N(\mu_{X_i})
&=\lambda^+\text{ if there is an }i\in I
\text{ such that }\kappa_i=\lambda\text{ and }\cf\kappa_i=\omega,\cr
&=\lambda\text{ otherwise}.\cr}$$

\noindent Then 513J tells us that
if $\lambda_1>\#(J_1)$ and there is some $\gamma<\lambda_1$ such that
$\cf\lambda_1>\#(J_{\gamma})$, then $\cf\Cal N(\mu)=\lambda_1$, and that
otherwise $\cf\Cal N(\mu)=\max(\#(J_1)^+,\lambda_1^+)$.   As we are
supposing that $\mu$ is not purely atomic,
$c(\frak C_0)\ge\omega$ and $c(\frak C_0)=\max(\omega,\#(J_1))$;  also
$\lambda^+\ge\lambda_1\ge\lambda\ge\omega$.

\medskip

\quad{\bf case 1} Suppose $\lambda_1\le\#(J_1)$.   Then $J_1$ is
infinite, so $c(\frak C_0)=\#(J_1)\ge\lambda$, and

\Centerline{$\cf\Cal N(\mu)=\#(J_1)^+=c(\frak C_0)^+=\theta$}

\noindent as required.

\medskip

\quad{\bf case 2} Suppose $\lambda_1>\max(\lambda,\#(J_1))$.   Then
there must be some $i\in I$ such that
$\cf\Cal N(\mu_{X_i})>\lambda\ge\omega$, in which case
$\kappa_i=\lambda$ has countable cofinality and $\lambda_1=\lambda^+$.
In this case, $\cf\lambda_1=\lambda_1>\#(J_1)$, so
$\cf\Cal N(\mu)=\lambda_1$.   If $\gamma<\lambda$, then
$\frak C_{\gamma}$ is non-trivial, and
$\cf\lambda=\omega\le c(\frak C_{\gamma})$;  so

\Centerline{$\theta=\max(\lambda^+,\#(J_1)^+)=\lambda_1
=\cf\Cal N(\mu)$.}

\medskip

\quad{\bf case 3} Suppose $\lambda_1=\lambda>\#(J_1)$ has countable
cofinality.   In this case we must have $\kappa_i<\lambda_1$ for every
$i$, so $\#(J_{\gamma})\ge\omega=\cf\lambda_1$ for every
$\gamma<\lambda_1$, and $\cf\Cal N(\mu)=\lambda_1^+$.   At the same time,
$\cf\lambda=\omega\le c(\frak C_{\gamma})$ for every $\gamma<\lambda$,
so

\Centerline{$\theta
=\max(\lambda^+,\#(J_1)^+)=\max(\lambda_1^+,\#(J_1)^+)=\cf\Cal N(\mu)$.}

\medskip

\quad{\bf case 4} Suppose $\lambda_1=\lambda>\#(J_1)$ has uncountable
cofinality.   In this case we have
$\lambda>\max(\omega,\#(J_1))=c(\frak C_0)$, so

$$\eqalign{\cf\Cal N(\mu)=\lambda_1
&\iff\#(J_{\gamma})<\cf\lambda_1\text{ for some }\gamma<\lambda_1\cr
&\iff\max(\omega,\#(J_{\gamma}))<\cf\lambda_1
   \text{ for some }\gamma<\lambda_1\cr
&\iff c(\frak C_{\gamma})<\cf\lambda
   \text{ for some }\gamma<\lambda\cr
&\iff\theta=\lambda
\iff\theta=\lambda_1,\cr}$$

\noindent and otherwise

\Centerline{$\cf\Cal N(\mu)
=\lambda_1^+=\max(\lambda^+,c(\frak C_0)^+)=\theta$.}

\noindent Thus $\cf\Cal N(\mu)=\theta$ in all cases.
}%end of proof of 524Q

\leader{524R}{}\cmmnt{ The results above show that most of the most
important cardinal functions of measurable algebras and Radon
measures are readily calculable from the cardinal functions of the
ideals $\Cal N_{\kappa}$ studied in \S523.   There are no
such simple formulae for other classes of space such as compact
or quasi-Radon measures (524Xj, 524Xk).    However I can give a handful of partial
results, as follows.

\medskip

\noindent}{\bf Proposition} Let $(X,\Sigma,\mu)$ be a countably compact
$\sigma$-finite measure space with Maharam type $\kappa$.   Then
$[\kappa]^{\le\omega}\prT\Cal N(\mu)$.   Consequently
$\cff[\kappa]^{\le\omega}\le\cf\Cal N(\mu)$, and if $\kappa$ is
uncountable then $\add\Cal N(\mu)=\omega_1$ and
$\cf\Cal N(\mu)\ge\cf\Cal N_{\kappa}$.

\proof{ If $\ofamily{\xi}{\kappa}{E_{\xi}}$ is a family as in 524Fb, then
$I\mapsto\bigcup_{\xi\in I}E_{\xi}:[\kappa]^{\le\omega}\to\Cal N(\mu)$
is a Tukey function, if both $[\kappa]^{\le\omega}$ and $\Cal N(\mu)$
are given their natural partial orderings of inclusion.   By 513Ee,
$\cff[\kappa]^{\le\omega}\le\cf\Cal N(\mu)$ and
$\add[\kappa]^{\le\omega}\ge\add\Cal N(\mu)$.   But if $\kappa$ is
uncountable, $\add[\kappa]^{\le\omega}=\omega_1$ so $\add\Cal N(\mu)$ is
also $\omega_1$.   At the same time, $\cf\Cal N(\mu)\ge\cf\Cal N$ (521K),
so

\Centerline{$\cf\Cal N(\mu)
\ge\max(\cf\Cal N,\cff[\kappa]^{\le\omega})=\cf\Cal N_{\kappa}$.}
}%end of proof of 524R

\leader{524S}{}\cmmnt{ In a different direction, there is something we
can say about quasi-Radon measures.

\medskip

\noindent}{\bf Proposition} Let $(X,\frak T,\Sigma,\mu)$ be a Radon
measure space, with $\mu X>0$, and $(Y,\frak S,\Tau,\nu)$ a quasi-Radon
measure space such that the measure algebras of $\mu$ and $\nu$ are
isomorphic.   Then

(a) $\Cal N(\nu)\prT\Cal N(\mu)$, so
$\add\nu=\add\Cal N(\nu)\ge\add\Cal N(\mu)=\add\mu$ and
$\cf\Cal N(\nu)\le\cf\Cal N(\mu)$;

(b) $(Y,\in,\Cal N(\nu))\prGT(X,\in,\Cal N(\mu))$, so
$\cov\Cal N(\nu)\le\cov\Cal N(\mu)$ and
$\non\Cal N(\nu)\ge\non\Cal N(\mu)$.

\proof{{\bf (a)} Let $(Z,\frak U,\Lambda,\lambda)$ be the Stone space of
the measure algebra $\frak B$ of $(Y,\Tau,\nu)$, and
$R\subseteq Z\times Y$ the relation described in 415Q/416V, so that
$R^{-1}[F]\in\Cal N(\lambda)$ for every $F\in\Cal N(\nu)$.   Let
$W\subseteq Z$ be the union of the open sets of finite measure.   Then
the subspace measure $\lambda_W$ is a Radon measure
and its measure algebra
is isomorphic to the measure algebras of $\nu$ and $\mu$
(411Pf).  %from 2005

Now $F\mapsto W\cap R^{-1}[F]:\Cal N(\nu)\to\Cal N(\lambda_W)$ is a
Tukey function.   \Prf\Quer\ Otherwise, there is a family
$\Cal A\subseteq\Cal N(\nu)$ such that $\bigcup\Cal A\notin\Cal N(\nu)$
but $\{W\cap R^{-1}[A]:A\in\Cal A\}$ is bounded above in
$\Cal N(\lambda_W)$.   Because $W$ is conegligible,
$B=\bigcup_{A\in\Cal A}R^{-1}[A]$ is negligible in $Z$.
Let $E\in\Tau$ be a measurable envelope of $\bigcup\Cal A$ (213J/213L).
Then the open-and-closed set $E^*\subseteq Z$ corresponding to
$E^{\ssbullet}\in\frak B$ is not negligible;  as $\lambda$ is inner
regular with respect to the open-and-closed sets (411Pb), there must be
a non-empty open-and-closed set $V\subseteq E^*$ which is disjoint from
$\bigcup_{A\in\Cal A}R^{-1}[A]$.   Express $V$ as $F^*$ where
$F\in\Tau$.   Then $R[V]=R[F^*]$ is disjoint from $\bigcup\Cal A$.
But $R[F^*]$ is measurable and $F\setminus R[F^*]$ is negligible
(415Qb), while $F\setminus E$ must also be negligible, so
$E\cap R[F^*]$ is a non-negligible measurable subset of
$E\setminus\bigcup\Cal A$, which is impossible.\ \Bang\Qed

This shows that $\Cal N(\nu)\prT\Cal N(\lambda_W)$.   But $\lambda_W$
and $\mu$ are Radon measures with isomorphic non-zero measure algebras,
so $\Cal N(\lambda_W)\equivT\Cal N(\mu)$ (524J) and
$\Cal N(\nu)\prT\Cal N(\mu)$.   Accordingly
$\add\Cal N(\nu)\ge\add\Cal N(\mu)$ and
$\cf\Cal N(\nu)\le\cf\Cal N(\mu)$

\medskip

{\bf (b)}   This is a special case of 521La.
}%end of proof of 524S

\leader{524T}{Corollary} Let $(Y,\frak S,\Tau,\nu)$ be a
quasi-Radon measure space, and $\frak B$ its measure algebra.
Let $K$ be the set of infinite cardinals $\kappa$ such that the
Maharam-type-$\kappa$ component of $\frak B$ is non-zero.

\vskip-\baselineskip

\newif\iftempa\tempafalse
\newif\iftempb\tempbfalse
\newif\iftempc\tempcfalse
\newif\iftempd\tempdfalse
\pagetotalplus=\the\pagetotal
\advance\pagetotalplus by 120pt
\ifdim\pagetotalplus>\pagegoal\tempatrue\else
\advance\pagetotalplus by 36pt
\ifdim\pagetotalplus>\pagegoal\tempbtrue\else
\advance\pagetotalplus by 36pt
\ifdim\pagetotalplus>\pagegoal\tempctrue\else
\advance\pagetotalplus by 36pt
\ifdim\pagetotalplus>\pagegoal\tempdtrue\else
\fi\fi\fi\fi
%\showthe\pagetotal   \showthe\pagetotalplus  \showthe\pagegoal


$$\leqalignno{\add\nu=\add\Cal N(\nu)&=\infty\text{ if }K=\emptyset,
  &\indent\text{(a)}\cr
&\ge\add\Cal N\text{ if }K=\{\omega\}.\cr
\iftempa\noalign{\break}\fi
\pi(\nu)=\pi(\frak B)
&=c(\frak B)\text{ if }K=\emptyset,&\indent\text{(b)}\cr
&=\max(c(\frak B),\cf\Cal N,\sup_{\kappa\in K}\cff[\kappa]^{\le\omega})
\text{ otherwise}.\cr
\iftempb\noalign{\break}\fi
\cov\Cal N(\nu)
&=1\text{ if }\frak B=\{0\},&\indent\text{(c)}\cr
&=\infty\text{ if }\frak B\text{ has an atom},\cr
&\le\cov\Cal N_{\min K}\text{ otherwise}.\cr
\iftempc\noalign{\break}\fi
\non\Cal N(\nu)
&=\infty\text{ if }\frak B=\{0\},&\indent\text{(d)}\cr
&=1\text{ if }\frak B\text{ has an atom},\cr
&\ge\non\Cal N_{\min K}\text{ otherwise}.\cr
\iftempd\noalign{\break}\fi
&&\indent\text{(e) If }\nu\text{ is }\sigma\text{-finite},\cr
\cf\Cal N(\nu)
&=1\text{ if }K=\emptyset,\cr
&\le\max(\cf\Cal N,\cff[\tau(\frak B)]^{\le\omega})
  \text{ otherwise}.\cr}$$

\proof{ Parts (a), (c), (d) and (e) are mostly a matter of putting 524P
and 524S together.   If there are atoms for $\mu$, they may no longer
include singletons of non-zero measure;  but they do include minimal
non-negligible closed sets, so there are non-negligible singletons and
$\cov\Cal N(\mu)$, $\non\Cal N(\mu)$ are $\infty$ and $1$ respectively.
As for (b), the proof of 524Pb still works.
}%end of proof of 524T

\exercises{\leader{524X}{Basic exercises (a)}
%\spheader 524Xa
Suppose that $(\frak A,\bar\mu)$ is a probability
algebra and that $\kappa=\link_n(\frak A)$, where $2\le n<\omega$.
Show that there are families $\ofamily{\xi}{\kappa}{A_{\xi}}$ in
$\frak A\setminus\{0\}$ and $\ofamily{\xi}{\kappa}{\epsilon_{\xi}}$ in
$\ocint{0,1}$ such that $\bar\mu(\inf I)\ge\epsilon_{\xi}$ whenever
$I\in[A_{\xi}]^n$ and
$\bigcup_{\xi<\kappa}A_{\xi}=\frak A\setminus\{0\}$.   \Hint{proof of
524L.}
%524M

\spheader 524Xb Let $(X,\Sigma,\mu)$ be a semi-finite measure
space with measure algebra $\frak A$, and $\Cal A$ a family of
non-negligible (not necessarily measurable) subsets of $X$ such that
every non-negligible member of $\Sigma$ includes a member of $\Cal A$.
Show that $\#(\Cal A)\ge\pi(\frak A)$.
%524P

\spheader 524Xc Show that if $\kappa$ is uncountable, there is no
function $f:[0,1]^{\kappa}\to\{0,1\}^{\kappa}$ which is almost
continuous and \imp\ for the usual measures on these spaces.   \Hint{if
$K\subseteq[0,1]^{\kappa}$ is a zero set, any continuous function from
$K$ to $\{0,1\}^{\kappa}$ is determined by coordinates in a countable
set.}
%524P

\spheader 524Xd Let $I^{\|}$ be the split interval and $\mu$ its usual
measure (343J).   Show that there are $f:\{0,1\}^{\omega}\to I^{\|}$ and
$g:I^{\|}\to\{0,1\}^{\omega}$ such that $\mu=\nu_{\omega}f^{-1}$ and
$\nu_{\omega}=\mu g^{-1}$.   \Hint{let $A\subseteq[0,1]$ be a
non-measurable set;  define
$f_0:[0,1]^2\to I^{\|}$ by setting $f_0(x,y)=y^+$ if $x\in A$, $y^-$
otherwise.}
%524P

\spheader 524Xe Let $(Z,\mu)$ be the Stone space of
$(\frak B_{\omega},\bar\nu_{\omega})$.   Show that there is no
$f:\{0,1\}^{\omega}\to Z$ such that $\mu=\nu_{\omega}f^{-1}$.
\Hint{use 515J and 322Ra to show that every non-negligible measurable
subset of $Z$ has cardinal $2^{\frak c}$.}
%524P

\spheader 524Xf Let $X$ be a Hausdorff space with a compact
topological probability measure $\mu$ with Maharam type $\kappa$, and
suppose that $w(X)<\cov\Cal N_{\kappa}$.   (i) Show
that there is an equidistributed sequence for $\mu$.  \Hint{491Eb.}
(ii) Show that if $\mu$ is strictly positive then $X$ is separable.
%524P

\spheader 524Xg Let $(X,\frak T,\Sigma,\mu)$ be a Radon probability
space with a strong lifting, and $(Z,\nu)$ the Stone space of its
measure algebra.   Show that $\shr\Cal N(\mu)\le\shr\Cal N(\nu)$
and $\shr^+\Cal N(\mu)\le\shr^+\Cal N(\nu)$.
\Hint{453Mb.}
%524P

\spheader 524Xh Let $(X,\frak T,\Sigma,\mu)$ be a Radon measure space,
and $K$ the set of infinite cardinals $\kappa$ such that the
Maharam-type-$\kappa$ component of its measure algebra $\frak A$ is
non-zero.   Show that

\Centerline{$\min\{\#(A):A\subseteq X$ has full outer measure$\}
=\sup(\{c(\frak A)\}\cup\{\non\Cal N_{\kappa}:\kappa\in K\})$.}

\spheader 524Xi Show that for any $\sigma$-ideal $\Cal I$ of sets there
is a compact probability measure $\mu$ such that $\Cal I=\Cal N(\mu)$.
\Hint{set $X=\bigcup\Cal I\cup\{x_0\}$.}
%524R

\spheader 524Xj Show that for any non-zero measurable algebra $\frak B$
and any cardinal $\kappa$, there is a complete compact probability
measure $\mu$ such that the measure algebra of $\mu$ is isomorphic to
$\frak B$, $\add\Cal N(\mu)=\omega_1$ and $\cf\Cal N(\mu)\ge\kappa$.
\Hint{524Xi.}
%524Xi 524R

\spheader 524Xk Suppose that
$\non\Cal N_{\frakc}=\cov\Cal N_{\frakc}=\cf\frak c=\frak c$.   Show that
there is a quasi-Radon probability measure $\mu$ with Maharam type
$\frak c$ such that $\add\Cal N(\mu)=\frak c$.
%524T mt52bits   \frak m=\frak c  does it; 524Md, 517Rb + 5A1Ed, 528L

\leader{524Y}{Further exercises (a)}
%\spheader 524Ya
Show that if $m\ge 2$ and
$\familyiI{\frak A_i}$ is a family of $\sigma$-$m$-linked
Boolean algebras, with $\#(I)\le\frak c$, then the free product of
$\familyiI{\frak A_i}$ is $\sigma$-$m$-linked.
%524L  %out of order query

\spheader 524Yb
Let $\frak A$ be a measurable algebra with Maharam type $\lambda$.
Show that there is a family
$\Cal V\subseteq[\lambda]^{\le\frakc}$, cofinal with
$[\lambda]^{\le\frakc}$, such that
$\#(\{A\cap V:V\in\Cal V\})<\FN^*(\frak A)$
for every countable set $A\subseteq\lambda$.
%524O 518J mt52bits

\spheader 524Yc For a Boolean algebra $\frak A$ and a
cardinal $\theta$, write $\psi_{\theta}(\frak A)$ for the smallest
size of any subalgebra $\frak C$ of $\frak A$ such that
$d(\frak C)\ge\theta$.   (If $\theta>d(\frak A)$ set
$\psi_{\theta}(\frak A)=\infty$.)
(i) Show that if $Z$ is the Stone space of $\frak A$, $\Cal I$ is the ideal
of nowhere dense sets in $Z$, and $\theta\ge 2$ then
$\psi_{\theta}(\frak A)\le\cov([Z]^{<\theta},\subseteq,\Cal I)$.
(ii) Show that if $(X,\Sigma,\mu)$ is a \Mth\ compact probability
space with Maharam type $\kappa$, and $\theta$ is uncountable, then

\Centerline{$\psi_{\theta}(\frak B_{\kappa})
=\cov([X]^{<\theta},\subseteq,\Cal N(\mu))
=\add(\Sigma\setminus\Cal N(\mu),\text{\tt{meet}},[X]^{<\theta})$,}

\noindent where
$\text{\tt{meet}}$ is the relation $\{(A,B):A\cap B\ne\emptyset\}$.
\Hint{start with $\mu=\nu_{\kappa}$.}
(iii) Show that if $(X,\Sigma,\mu)$ is a semi-finite locally compact
measure space with measure algebra $\frak A$ then
$\psi_{\omega_1}(\frak A)\le\cf([\cov\Cal N(\mu)]^{\le\omega})$.
(iv) Show that if $(X,\Sigma,\mu)$ is any probability space, with measure
algebra $\frak A$, and $\lambda$ is the product probability measure
on $X^{\Bbb N}$, then
$\cov\Cal N(\lambda)\le\psi_{\omega_1}(\frak A)$.
(v) Show that $\psi_{\add\Cal M}(\frak B_{\omega})\le\non\Cal M$, where
$\Cal M$ is the ideal of meager subsets of $\Bbb R$.
%mt52bits
}%end of exercises

\leader{524Z}{Problems (a)}
%\spheader 524Za
Let $(Z,\mu)$ be the Stone space of
$(\frak B_{\omega},\bar\nu_{\omega})$.   Is $\shr\Cal N(\mu)$
necessarily equal to $\shr\Cal N$?

\spheader 524Zb Can there be a quasi-Radon probability measure $\mu$ with
Maharam type greater than $\frak c$ such that $\add\Cal N(\mu)>\omega_1$?

\endnotes{
\Notesheader{524} The ideas of this section are derived primarily from
{\smc Bartoszy\'nski 84}, {\smc Fremlin 84b} and {\smc Fremlin 91}.
Of course it is not necessary to pass through both $\ell^1(\kappa)$ and
the $\kappa$-localization relation
$(\kappa^{\Bbb N},\subseteq^*,\Cal S_{\kappa})$.   I bring
$\ell^1(\kappa)$ into the argument (following {\smc Bartoszy\'nski 84})
because it will be useful when we come to look at other structures in
later in the chapter, and $\Cal S_{\kappa}$ because it echoes the ideas
of \S522.   But note that 524G seems to need a new idea (the
family $\ofamily{\xi}{\kappa}{E_{\xi}}$ from 524F) not required in 522M.

The difficulties of the work above arise from the fact that while there
are many \imp\ functions between Radon measure spaces, immediately
linking covering numbers and uniformities, there are far fewer
continuous \imp\ functions;  for instance, there is no almost continuous
\imp\ function from the unit interval to the split interval, let alone
to the Stone space of its measure algebra.   And the straightforward
Tukey functions between the ideals $\Cal N_{\kappa}$ of \S523 depend on
measures being images of each other, which is something we can rely
on only when our functions are almost continuous.   (But see 524Xd.)   I
do not know of any
direct construction of a Tukey function from the null
ideal of the Stone space of the Lebesgue measure algebra to $\Cal N$,
for instance.   This is why there is nearly nothing about
shrinking numbers in this section (see 524Za).

There is a significant gap in the calculations in 524P;  for the
cofinality of the null ideal I need to assume that the measure is
$\sigma$-finite.   I have no useful general recipe for $\cf\Cal N(\mu)$,
valid in ZFC, when $\mu$ is a
non-$\sigma$-finite Radon measure.   The point is
that although we can identify $\Cal N(\mu)$ with the product of a family
$\Cal N(\mu_{X_i})$ of partially ordered sets to which the arguments of
this section apply (524Q), this is not in itself
enough to determine its cofinality in the absence of special axioms.
% ccc forcing raises $2^{\omega_1}$ but not $\cf(\omega_1^{\omega_1})$
}%end of notes

\discrpage
