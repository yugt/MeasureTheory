\frfilename{mt254.tex}
\versiondate{23.2.16}
\copyrightdate{2002}

\def\chaptername{Product measures}
\def\sectionname{Infinite products}

\newsection{254}

I come now to the second basic idea of this
chapter:  the description of a product measure on the product of a
(possibly large) family of probability spaces.   The section begins with
a construction on similar lines to that of \S251 (254A-254F) and its
defining property in terms of \imp\ functions (254G).   I discuss the
usual measure on $\{0,1\}^I$
(254J-254K), subspace measures (254L) and various properties of
subproducts (254M-254T), including a study of the associated conditional
expectation operators (254R-254T).

\leader{254A}{Definitions (a)} Let $\familyiI{(X_i,\Sigma_i,\mu_i)}$ be
a family of probability
spaces.   Set $X=\prod_{i\in I}X_i$, the family of functions $x$ with
domain $I$ such that $x(i)\in X_i$ for every $i\in I$.   In this
context, I will say that a {\bf measurable cylinder} is a subset of $X$
expressible in the form

\Centerline{$C=\prod_{i\in I}C_i,$}

\noindent where $C_i\in\Sigma_i$ for every $i\in I$ and
$\{i:C_i\ne X_i\}$ is finite.   Note that for a non-empty $C\subseteq X$
this expression is unique.   \prooflet{\Prf\ Suppose that
$C=\prod_{i\in I}C_i=\prod_{i\in I}C'_i$.   For each $i\in I$ set

\Centerline{$D_i=\{x(i):x\in C\}.$}

\noindent Of course $D_i\subseteq C_i$.   Because $C\ne\emptyset$, we
can fix on some $z\in C$.   If $i\in I$ and $t\in C_i$,
consider $x\in X$ defined by setting

\Centerline{$x(i)=t$, \quad$x(j)=z(j)\text{ for }j\ne i;$}

\noindent then $x\in C$ so $t=x(i)\in D_i$.   Thus $D_i=C_i$ for
$i\in I$.   Similarly, $D_i=C'_i$.\ \Qed}

\header{254Ab}{\bf (b)} We can therefore define a functional
$\theta_0:\Cal C\to[0,1]$, where $\Cal C$ is the set of measurable
cylinders, by setting

\Centerline{$\theta_0C=\prod_{i\in I}\mu_iC_i$}

\noindent whenever  $C_i\in\Sigma_i$ for every $i\in I$ and
$\{i:C_i\ne X_i\}$ is finite\cmmnt{,
noting that only finitely many terms in the product can differ from $1$,
so that it can safely be treated as a finite product.   If
$C=\emptyset$, one of the $C_i$ must be empty, so $\theta_0C$ is surely
$0$, even though the expression of $C$ as $\prod_{i\in I}C_i$
is no longer unique}.

\header{254Ac}{\bf (c)} Now define $\theta:\Cal PX\to[0,1]$ by setting

\Centerline{$\theta A
=\inf\{\sum_{n=0}^{\infty}\theta_0C_n:C_n\in\Cal C
  \text{ for every }n\in\Bbb N$, $A\subseteq\bigcup_{n\in\Bbb N}C_n\}$.}

\leader{254B}{Lemma} The functional $\theta$ defined in 254Ac is always
an outer measure on $X$.

\proof{ Use exactly the same arguments as those in 251B above.}

\leader{254C}{Definition} Let
$\langle(X_i,\Sigma_i,\mu_i)\rangle_{i\in I}$ be any indexed family of
probability spaces, and $X$ the Cartesian
product $\prod_{i\in I}X_i$.   The {\bf product measure} on $X$ is the
measure defined by \Caratheodory's method\cmmnt{ (113C)} from the
outer measure $\theta$ defined in 254A.

\cmmnt{
\leader{254D}{Remarks (a)} In 254Ab, I asserted that if $C\in\Cal C$ and
no $C_i$ is empty, then nor is $C=\prod_{i\in I}C_i$.   This is the
`Axiom of Choice':  the product of
any family $\langle C_i\rangle_{i\in I}$ of
non-empty sets is non-empty, that is, there is a `choice function'
$x$ with domain $I$ picking out a distinguished member $x(i)$ of each
$C_i$.   In this volume I have not attempted to be scrupulous in
indicating uses of the axiom of choice.   In fact the use here is not an
absolutely vital one;  I mean, the theory of infinite products, even
uncountable products, of probability spaces does not change character
completely in the absence of the full axiom of choice (provided, that
is, that we allow ourselves to use the countable axiom of choice).   The
point is that all we really need, in the present context, is that
$X=\prod_{i\in I}X_i$ should be non-empty;  and in many contexts we can
prove this, for the particular cases of interest, without using the
axiom of choice, by actually exhibiting a member of $X$.   The simplest
case in which this is difficult is when the $X_i$ are uncontrolled Borel
subsets of $[0,1]$;  and even then, if they are presented with coherent
descriptions, we may, with appropriate labour, be able to construct a
member of $X$.   But clearly such a process is liable to slow us down a
good deal, and for the moment I think there is no great virtue in taking
so much trouble.

\header{254Db}{\bf (b)}
I have given this section the title `infinite products', but it
is useful to be able to apply the ideas to finite $I$;  I should mention
in particular the cases $\#(I)\le 2$.

\medskip

\quad {\bf (i)}  If $I=\emptyset$, $X$ consists of the unique function
with domain $I$, the empty function.  If we identify a function with its
graph, then $X$ is actually $\{\emptyset\}$;  in any case, $X$ is to be
a singleton set, with $\lambda X=1$.

\medskip

\quad{\bf (ii)} If $I$ is a singleton $\{i\}$, then we can identify $X$
with $X_i$;  $\Cal C$ becomes identified with $\Sigma_i$ and $\theta_0$
with $\mu_i$, so that $\theta$ can be identified with $\mu_i^*$
and the `product measure' becomes the measure on $X_i$
defined from $\mu_i^*$, that is, the completion of $\mu_i$\cmmnt{
(see 213Xa(iv))}.

\medskip

\quad{\bf (iii)} If $I$ is a doubleton $\{i,j\}$, then we can identify
$X$ with $X_i\times X_j$;  in this case the definitions of 254A and 254C
match exactly with those of
251A and 251C, so that $\lambda$ here can be identified with the
primitive product measure as defined in 251C.   Because $\mu_i$ and
$\mu_j$ are both totally finite, this agrees with the c.l.d.\ product
measure of 251F.
}%end of comment


\leader{254E}{Definition} Let $\langle X_i\rangle_{i\in I}$ be any
family of sets, and $X=\prod_{i\in I}X_i$.   If $\Sigma_i$ is a
$\sigma$-subalgebra of subsets of $X_i$ for each $i\in I$, I write
$\Tensorhat_{i\in I}\Sigma_i$ for the $\sigma$-algebra of subsets of $X$
generated by

\Centerline{$\{\{x:x\in X$, $x(i)\in E\}: i\in I$, $E\in\Sigma_i\}$.}

\cmmnt{\noindent (Compare 251D.)}

\leader{254F}{Theorem} Let $\langle(X_i,\Sigma_i,\mu_i)\rangle_{i\in I}$
be a family of probability spaces, and let $\lambda$ be the product
measure on $X=\prod_{i\in I}X_i$\cmmnt{ defined as in 254C};  let
$\Lambda$ be its domain.

(a) $\lambda X=1$.

(b) If $E_i\in\Sigma_i$ for every $i\in I$, and $\{i:E_i\ne X_i\}$ is
countable, then $\prod_{i\in I}E_i\in\Lambda$, and
$\lambda(\prod_{i\in I}E_i)=\prod_{i\in I}\mu_iE_i$.   In particular,
$\lambda C=\theta_0C$
for every measurable cylinder $C$,\cmmnt{ as defined in 254A,} and if
$j\in I$ then $x\mapsto x(j):X\to X_j$ is inverse-measure-preserving.

(c) $\Tensorhat_{i\in I}\Sigma_i\subseteq\Lambda$.

(d) $\lambda$ is complete.

(e) For every $W\in\Lambda$ and $\epsilon>0$ there is a finite family
$C_0,\ldots,C_n$ of measurable cylinders such that
$\lambda(W\symmdiff\bigcup_{k\le n}C_k)\penalty-100\le\epsilon$.

(f) For every $W\in\Lambda$ there are $W_1$,
$W_2\in\Tensorhat_{i\in I}\Sigma_i$ such that
$W_1\subseteq W\subseteq W_2$ and $\lambda(W_2\setminus W_1)=0$.

\cmmnt{\medskip

\noindent{\bf Remark} Perhaps I should pause to interpret the
product $\prod_{i\in I}\mu_iE_i$.   Because all the $\mu_iE_i$ belong to
$[0,1]$, this is simply $\inf_{J\subseteq I,J\text{ is
finite}}\prod_{i\in J}\mu_iE_i$, taking the empty product to be $1$.
}%end of comment

\proof{ Throughout this proof, define $\Cal C$, $\theta_0$
and $\theta$ as in 254A.   I will write out an argument which applies to
finite $I$ as well as infinite $I$, but you may reasonably
prefer to assume that $I$ is infinite on first reading.

\medskip

{\bf (a)} Of course $\lambda X=\theta X$, so I have to show that
$\theta X=1$.   Because $X$, $\emptyset\in\Cal C$ and
$\theta_0X=\prod_{i\in I}\mu_iX_i=1$ and $\theta_0\emptyset=0$,

\Centerline{$\theta X\le\theta_0X+\theta_0\emptyset+\ldots=1$.}

\noindent   I therefore have to show
that $\theta X\ge 1$.   \Quer\  Suppose, if possible, otherwise.

\medskip

\quad{\bf (i)} There is a sequence $\sequencen{C_n}$ in $\Cal C$,
covering $X$, such that $\sum_{n=0}^{\infty}\theta_0C_n<1$.    For each
$n\in\Bbb N$, express $C_n$ as $\{x:x(i)\in E_{ni}\Forall i\in I\}$,
where every $E_{ni}\in\Sigma_i$ and $J_n=\{i:E_{ni}\ne X_i\}$ is finite.
No $J_n$ can be empty, because
$\theta_0C_n<1=\theta_0X$;  set $J=\bigcup_{n\in\Bbb N}J_n$.   Then $J$
is a countable non-empty subset of $I$.   Set $K=\Bbb N$ if $J$ is
infinite, $\{k:0\le k<\#(J)\}$ if $J$ is finite;   let
$k\mapsto i_k:K\to J$ be a bijection.

For each $k\in K$, set $L_k=\{i_j:j<k\}\subseteq J$, and set
$\alpha_{nk}=\prod_{i\in I\setminus L_k}\mu_iE_{ni}$ for $n\in\Bbb N$,
$k\in K$.   If $J$ is finite, then we can identify $L_{\#(J)}$ with $J$,
and set $\alpha_{n,\#(J)}=1$ for every
$n$.   We have $\alpha_{n0}=\theta_0C_n$ for each $n$, so
$\sum_{n=0}^{\infty}\alpha_{n0}<1$.   For $n\in\Bbb N$, $k\in K$ and
$t\in X_{i_k}$ set

$$\eqalign{f_{nk}(t)&=\alpha_{n,k+1}\text{ if }t\in E_{n,i_k},\cr
&=0\text{ otherwise}.\cr}$$

\noindent Then

\Centerline{$\int f_{nk}d\mu_{i_k}
=\alpha_{n,k+1}\mu_{i_k}E_{n,i_k}=\alpha_{nk}$.}

\medskip

\quad{\bf (ii)} Choose $t_k\in X_{i_k}$ inductively, for $k\in K$, as
follows.   The inductive hypothesis will be that
$\sum_{n\in M_k}\alpha_{nk}<1$, where
$M_k=\{n:n\in \Bbb N$, $t_j\in E_{n,i_j}\Forall j<k\}$;  of course
$M_0=\Bbb N$, so the induction starts.   Given that

\Centerline{$1>\sum_{n\in M_k}\alpha_{nk}
=\sum_{n\in M_k}\int f_{nk}d\mu_{i_k}
=\int(\sum_{n\in M_k}f_{nk})d\mu_{i_k}$}

\noindent (by B.Levi's theorem), there must be a $t_k\in X_{i_k}$ such
that $\sum_{n\in M_k}f_{nk}(t_{k})<1$.   Now for such a choice of
$t_k$, $\alpha_{n,k+1}=f_{nk}(t_k)$ for every $n\in M_{k+1}$, so that
$\sum_{n\in M_{k+1}}\alpha_{n,k+1}<1$, and the induction continues,
unless $J$ is finite and $k+1=\#(J)$.   In this last case we must just
have $M_{\#(J)}=\emptyset$, because $\alpha_{n,\#(J)}=1$ for every $n$.

\medskip

\quad{\bf (iii)} If $J$ is infinite, we obtain a full sequence
$\sequence{k}{t_k}$;  if $J$ is finite, we obtain just a finite sequence
$\langle t_k\rangle_{k<\#(J)}$.   In either case, there is an $x\in X$
such that $x(i_k)=t_k$ for each $k\in K$.   Now there must be some
$m\in\Bbb N$ such that $x\in C_m$.   Because $J_m=\{i:E_{mi}\ne X_i\}$
is finite, there is a $k\in\Bbb N$ such that $J_m\subseteq L_k$
(allowing $k=\#(J)$ if $J$ is finite).   Now $m\in M_k$, so in fact we
cannot have $k=\#(J)$, and
$\alpha_{mk}=1$, so $\sum_{n\in M_k}\alpha_{nk}\ge 1$, contrary to the
inductive hypothesis.\ \Bang

This contradiction shows that $\theta X=1$.

\medskip

{\bf (b)(i)} I take the particular case first.
Suppose that $j\in I$ and
$E\in\Sigma_j$, and let $C\in\Cal C$;  set $W=\{x:x\in X$, $x(j)\in
E\}$;
then $C\cap W$ and $C\setminus W$ both belong to $\Cal C$, and $\theta_0
C=\theta_0(C\cap W)+\theta_0(C\setminus W)$.   \Prf\ If
$C=\prod_{i\in I}C_i$, where $C_i\in\Sigma_i$ for each $i$, then
$C\cap W=\prod_{i\in I}C_i'$, where $C_i'=C_i$ if $i\ne j$, and
$C_j'=C_j\cap E$;  similarly,
$C\setminus W=\prod_{i\in I}C_i''$, where $C_i''=C_i$ if $i\ne j$, and
$C_j''=C_j\setminus E$.   So both belong to $\Cal C$, and

\Centerline{$\theta_0(C\cap W)+\theta_0(C\setminus W)
=(\mu_j(C_j\cap E)+\mu_j(C_j\setminus E))\prod_{i\ne j}\mu C_i
=\prod_{i\in I}\mu C_i
=\theta_0C$.  \Qed}

\medskip

\quad{\bf (ii)} Now suppose that $A\subseteq X$ is any set, and
$\epsilon>0$.   Then there is a sequence $\sequencen{C_n}$ in $\Cal C$
such that $A\subseteq\bigcup_{n\in\Bbb N}C_n$ and
$\sum_{n=0}^{\infty}\theta_0C_n\le\theta A+\epsilon$.   In this case

\Centerline{$A\cap W\subseteq\bigcup_{n\in\Bbb N}C_n\cap W$,\quad
$A\setminus W\subseteq\bigcup_{n\in\Bbb N}C_n\setminus W,$}

\noindent so

\Centerline{$\theta(A\cap W)\le\sum_{n=0}^{\infty}\theta_0(C_n\cap
W),\quad
\theta(A\setminus W)\le\sum_{n=0}^{\infty}\theta_0(C_n\setminus W),$}

\noindent and

\Centerline{$\theta(A\cap W)+\theta(A\setminus W)
\le\sum_{n=0}^{\infty}\theta_0(C_n\cap W)+\theta_0(C_n\setminus W)
=\sum_{n=0}^{\infty}\theta_0C_n
\le\theta A+\epsilon.$}
\noindent As $\epsilon$ is arbitrary, $\theta(A\cap W)+\theta(A\setminus
W)\le\theta A$;  as $A$ is arbitrary, $W\in\Lambda$.

\medskip

\quad{\bf (iii)}
I show next that if $J\subseteq I$ is finite and $C_i\in\Sigma_i$ for
each $i\in J$, and $C=\{x:x\in X$, $x(i)\in C_i\Forall i\in J\}$,
then $C\in\Lambda$ and  $\lambda C=\prod_{i\in J}\mu_iC_i$.   \Prf\
Induce on $\#(J)$.   If $\#(J)=0$, that is, $J=\emptyset$, then $C=X$
and this is part (a).   For the inductive step to $\#(J)=n+1$, take any
$j\in J$ and set $J'=J\setminus\{j\}$,

\Centerline{$C'=\{x:x\in X$, $x(i)\in C_i\Forall i\in J'\},$}

\Centerline{$C''=C'\setminus C=\{x:x\in C'$, $x(j)\in X_j\setminus
C_j\}.$}

\noindent Then $C$, $C'$, $C''$ all belong to $\Cal C$, and $\theta_0
C'=\prod_{i\in J'}\mu_iC_i=\alpha$ say, $\theta_0 C=\alpha\mu_jC_j$,
$\theta_0 C''=\alpha(1-\mu_jC_j)$.   Moreover, by the inductive
hypothesis, $C'\in\Lambda$ and $\alpha=\lambda C'=\theta C'$.   So
$C=C'\cap\{x:x(j)\in C_j\}\in\Lambda$ by (ii), and $C''=C'\setminus
C\in\Lambda$.

We surely have $\lambda C=\theta C\le\theta_0 C$, $\lambda
C''\le\theta_0 C''$;   but also

\Centerline{$\alpha=\lambda C'=\lambda C+\lambda C''\le\theta_0
C+\theta_0C''=\alpha,$}

\noindent so in fact

\Centerline{$\lambda C=\theta_0 C=\alpha\mu_jC_j
=\prod_{i\in J}\mu C_i$,}

\noindent and the induction proceeds.\ \Qed

\medskip

\quad{\bf (iv)} Now let us return to the general case of a set $W$ of
the form $\prod_{i\in I}E_i$ where $E_i\in\Sigma_i$ for each $i$, and
$K=\{i:E_i\ne X_i\}$ is countable.   If $K$ is finite then
$W=\{x:x(i)\in E_i\Forall i\in K\}$ so $W\in\Lambda$ and

\Centerline{$\lambda W
=\prod_{i\in K}\mu_iE_i=\prod_{i\in I}\mu_iE_i$.}

\noindent    Otherwise, let
$\sequencen{i_n}$ be an enumeration of $K$.   For each $n\in\Bbb N$ set
$W_n=\{x:x\in X$, $x(i_k)\in E_{i_k}\Forall k\le n\}$;  then  we know
that $W_n\in\Lambda$ and that
$\lambda W_n=\prod_{k=0}^n\mu_{i_k}E_{i_k}$.   But $\sequencen{W_n}$ is
a non-increasing sequence with intersection $W$, so $W\in\Lambda$ and

\Centerline{$\lambda W=\lim_{n\to\infty}\lambda W_n
=\prod_{i\in K}\mu_iE_i
=\prod_{i\in I}\mu_iE_i.$}

\medskip

{\bf (c)} is an immediate consequence of (b) and the definition of
$\Tensorhat_{i\in I}\Sigma_i$.

\medskip

{\bf (d)} Because $\lambda$ is constructed by \Caratheodory's method it
must be complete.

\medskip

{\bf (e)} Let $\sequencen{C_n}$ be a sequence in $\Cal C$ such that
$W\subseteq\bigcup_{n\in\Bbb N}C_n$ and
$\sum_{n=0}^{\infty}\theta_0C_n\le\theta W+\bover12\epsilon$.   Set
$V=\bigcup_{n\in\Bbb N}C_n$;  by (b), $V\in\Lambda$.   Let $n\in\Bbb N$
be
such that $\sum_{i=n+1}^{\infty}\theta_0C_i\le\bover12\epsilon$, and
consider $W'=\bigcup_{k\le n}C_k$.
Since $V\setminus W'\subseteq\bigcup_{i>n}C_i$,

$$\eqalign{\lambda(W\symmdiff W')
&\le\lambda(V\setminus W)+\lambda(V\setminus W')
=\lambda V-\lambda W+\lambda(V\setminus W')
=\theta V-\theta W+\theta(V\setminus W')\cr
&\le\sum_{i=0}^{\infty}\theta_0C_i-\theta W+\sum_{i=n+1}\theta_0C_i
\le\Bover12\epsilon+\Bover12\epsilon
=\epsilon.\cr}$$

\medskip

{\bf (f)(i)} If $W\in\Lambda$ and $\epsilon>0$ there is a
$V\in\Tensorhat_{i\in I}\Sigma_i$ such that $W\subseteq V$ and $\lambda
V\le\lambda W+\epsilon$.   \Prf\ Let $\sequencen{C_n}$ be a sequence in
$\Cal C$ such that $W\subseteq\bigcup_{n\in\Bbb N}C_n$ and
$\sum_{n=0}^{\infty}\theta_0C_n\le\theta W+\epsilon$.   Then
$C_n\in\Tensorhat_{i\in I}\Sigma_i$ for each $n$, so
$V=\bigcup_{n\in\Bbb N}C_n\in\Tensorhat_{i\in I}\Sigma_i$.   Now
$W\subseteq V$, and

\Centerline{$\lambda V=\theta V\le\sum_{n=0}^{\infty}\theta_0C_n
\le\theta W+\epsilon=\lambda W+\epsilon$.\ \Qed}

\medskip

\quad{\bf (ii)} Now, given $W\in\Lambda$, let $\sequencen{V_n}$ be a
sequence of sets in $\Tensorhat_{i\in I}\Sigma_i$ such that $W\subseteq
V_n$ and $\lambda V_n\le\lambda W+2^{-n}$ for each $n$;  then
$W_2=\bigcap_{n\in\Bbb N}V_n$ belongs to $\Tensorhat_{i\in I}\Sigma_i$
and
$W\subseteq W_2$ and $\lambda W_2=\lambda W$.   Similarly, there is a
$W'_2\in\Tensorhat_{i\in I}\Sigma_i$ such that $X\setminus W\subseteq
W'_2$ and $\lambda W'_2=\lambda(X\setminus W)$, so we may take
$W_1=X\setminus W'_2$ to complete the proof.
}

\leader{254G}{}\cmmnt{ The following is a fundamental, indeed
defining, property of product measures.   (Compare 251L.)

\medskip

\noindent}{\bf Theorem} Let $\familyiI{(X_i,\Sigma_i,\mu_i)}$ be a
family
of probability spaces with product $(X,\Lambda,\lambda)$.   Let
$(Y,\Tau,\nu)$ be a complete probability space and $\phi:Y\to X$ a
function.
Suppose that $\nu^*\phi^{-1}[C]\le\lambda C$ for every measurable
cylinder $C\subseteq X$.   Then $\phi$ is \imp.
In particular, $\phi$ is \imp\ iff $\phi^{-1}[C]\in\Tau$ and
$\nu\phi^{-1}[C]=\lambda C$ for every measurable cylinder
$C\subseteq X$.

\cmmnt{\medskip

\noindent{\bf Remark} By $\nu^*$ I mean the usual outer measure defined
from $\nu$ as in \S132.
}

\proof{{\bf (a)} First note that, writing $\theta$ for the outer measure
of 254A, $\nu^*\phi^{-1}[A]\le\theta A$ for every $A\subseteq X$.
\Prf\ Given $\epsilon>0$, there is a sequence $\sequencen{C_n}$ of
measurable cylinders such that $A\subseteq\bigcup_{n\in\Bbb N}C_n$ and
$\sum_{n=0}^{\infty}\theta_0C_n\le\theta A+\epsilon$, where $\theta_0$
is the functional of 254A.   But we know that $\theta_0C=\lambda C$ for
every measurable cylinder $C$ (254Fb), so

\Centerline{$\nu^*\phi^{-1}[A]
\le\nu^*(\bigcup_{n\in\Bbb N}\phi^{-1}[C_n])
\le\sum_{n=0}^{\infty}\nu^*\phi^{-1}[C_n]
\le\sum_{n=0}^{\infty}\lambda C_n
\le\theta A+\epsilon$.}

\noindent As $\epsilon$ is arbitrary, $\nu^*\phi^{-1}[A]\le\theta A$.\
\Qed

\medskip

{\bf (b)} Now take any $W\in\Lambda$.   Then there are $F$, $F'\in\Tau$
such that

\Centerline{$\phi^{-1}[W]\subseteq F$, \quad$\phi^{-1}[X\setminus
W]\subseteq F'$,}

\Centerline{$\nu F=\nu^*\phi^{-1}[W]\le\theta W=\lambda W$,
\quad$\nu F'\le\lambda[X\setminus W]$.}

\noindent We have

\Centerline{$F\cup F'\supseteq\phi^{-1}[W]\cup\phi^{-1}[X\setminus
W]=Y$,}

\noindent so

\Centerline{$\nu(F\cap F')=\nu F+\nu F'-\nu(F\cup F')
\le\lambda W+\lambda(X\setminus W)-1=0$.}

\noindent Now

\Centerline{$F\setminus\phi^{-1}[W]\subseteq F\cap\phi^{-1}[X\setminus
W]\subseteq F\cap F'$}

\noindent is $\nu$-negligible.   Because $\nu$ is complete,
$F\setminus\phi^{-1}[W]\in\Tau$ and
$\phi^{-1}[W]=F\setminus(F\setminus\phi^{-1}[W])$ belongs to $\Tau$.
Moreover,

\Centerline{$1=\nu F+\nu F'\le\lambda W+\lambda(X\setminus W)=1$,}

\noindent so we must have $\nu F=\lambda W$;  but this means that
$\nu\phi^{-1}[W]=\nu W$.   As $W$ is arbitrary, $\phi$ is \imp.
}%end of proof of 254G

\leader{254H}{Corollary} Let
$\langle(X_i,\Sigma_i,\mu_i)\rangle_{i\in I}$
and $\langle(Y_i,\Tau_i,\nu_i)\rangle_{i\in I}$ be two families of
probability spaces, with products $(X,\Lambda,\lambda)$ and
$(Y,\Lambda',\lambda')$.   Suppose that for each $i\in I$ we are given
an \imp\ function $\phi_i:X_i\to Y_i$.   Set
$\phi(x)=\familyiI{\phi_i(x(i))}$ for $x\in X$.   Then $\phi:X\to Y$ is
\imp.

\proof{ If $C=\prod_{i\in I}C_i$ is a measurable cylinder in $Y$, then
$\phi^{-1}[C]=\prod_{i\in I}\phi_i^{-1}[C_i]$ is a measurable cylinder
in
$X$, and

\Centerline{$\lambda\phi^{-1}[C]
=\prod_{i\in I}\mu_i\phi_i^{-1}[C_i]
=\prod_{i\in I}\nu_i C_i
=\lambda'C$.}

\noindent Since $\lambda$ is a complete probability measure, 254G tells
us that $\phi$ is \imp.
}%end of proof of 254H

\leader{254I}{}\cmmnt{ Corresponding to 251T we have the following.

\medskip

\noindent}{\bf Proposition} Let
$\langle(X_i,\Sigma_i,\mu_i)\rangle_{i\in I}$ be a family of probability
spaces, $\lambda$ the product measure
on $X=\prod_{i\in I}X_i$, and $\Lambda$ its domain.   Then $\lambda$ is
also the product of the
completions $\hat\mu_i$ of the $\mu_i$\cmmnt{ (212C)}.

\proof{ Write $\hat\lambda$ for the product of the $\hat\mu_i$, and
$\hat\Lambda$ for its domain.   (i) The identity map from $X_i$ to
itself is \imp\ if regarded as a map from $(X_i,\hat\mu_i)$ to
$(X_i,\mu_i)$, so
the identity map on $X$ is \imp\ if regarded as a map from
$(X,\hat\lambda)$ to $(X,\lambda)$, by 254H;  that is,
$\Lambda\subseteq\hat\Lambda$ and $\lambda=\hat\lambda\restr\Lambda$.
(ii) If $C$ is a measurable cylinder for $\langle\hat\mu_i\rangle_{i\in
I}$, that is, $C=\prod_{i\in I}C_i$ where $C_i\in\hat\Sigma_i$ for every
$i$ and $\{i:C_i\ne X_i\}$ is
finite, then for each $i\in I$ we can find a $C'_i\in\Sigma_i$ such that
$C_i\subseteq C'_i$ and $\mu_iC'_i=\hat\mu_iC_i$;  setting
$C'=\prod_{i\in I}C'_i$, we get

\Centerline{$\lambda^*C\le\lambda C'=\prod_{i\in I}\mu_iC'_i
=\prod_{i\in I}\hat\mu_iC_i=\hat\lambda C$.}

\noindent By 254G, $\lambda W$ must be defined and equal to
$\hat\lambda W$ whenever $W\in\hat\Lambda$.   Putting this
together with (i), we see that $\lambda=\hat\lambda$.
}%end of proof of 254I

\leader{254J}{}{\bf The product measure on $\{0,1\}^I$ (a)} Perhaps the
most important of all examples of infinite product measures is the case
in which each factor $X_i$ is just $\{0,1\}$ and each $\mu_i$ is the
`fair-coin' probability measure, setting

\Centerline{$\mu_i\{0\}=\mu_i\{1\}=\Bover12$.}

\noindent In this case, the product $X=\{0,1\}^I$ has a family
$\langle E_i\rangle_{i\in I}$ of measurable sets such that,
writing $\lambda$ for the product measure on $X$,

\Centerline{$\lambda(\bigcap_{i\in J}E_i)=2^{-\#(J)}$ if $J\subseteq I$
is finite.}

\noindent\prooflet{(Just take $E_i=\{x:x(i)=1\}$ for each $i$.)  }I
will call this
$\lambda$ the {\bf usual measure} on $\{0,1\}^I$.   Observe that if $I$
is finite then $\lambda\{x\}=2^{-\#(I)}$ for each
$x\in X$\cmmnt{ (using
254Fb)}.   On the other hand, if $I$ is infinite, then $\lambda\{x\}=0$
for every $x\in X$\prooflet{ (because, again using 254Fb,
$\lambda^*\{x\}\le 2^{-n}$ for every $n$)}.

\spheader 254Jb There is a natural bijection between $\{0,1\}^I$ and
$\Cal PI$, matching $x\in\{0,1\}^I$ with $\{i:i\in I$, $x(i)=1\}$.
So we get a standard measure $\tilde\lambda$ on $\Cal PI$, which I will
call the {\bf usual measure on $\Cal PI$}.   Note that for any finite
$b\subseteq I$ and any $c\subseteq b$ we have

\Centerline{$\tilde\lambda\{a:a\cap b=c\}=\lambda\{x:x(i)=1$ for $i\in
c$, $x(i)=0$ for $i\in b\setminus c\}=2^{-\#(b)}$.}

\spheader 254Jc Of course we can apply 254G to these measures; if
$(Y,\Tau,\nu)$ is a complete probability space, a function
$\phi:Y\to\{0,1\}^I$ is \imp\ iff

\Centerline{$\nu\{y:y\in Y$, $\phi(y)\restr J=z\}=2^{-\#(J)}$}

\noindent whenever $J\subseteq I$ is finite and
$z\in\{0,1\}^J$\prooflet{;  this is because the measurable cylinders in
$\{0,1\}^I$ are precisely the sets of the form $\{x:x\restr J=z\}$ where
$J\subseteq I$ is finite}.

\spheader 254Jd\dvAformerly{2{}54Xe}
Define addition on $X$ by setting
$(x+y)(i)=x(i)+_2y(i)$ for every $i\in I$, $x$, $y\in X$, where
$0+_20=1+_21=0$, $0+_21=1+_20=1$.   If $y\in X$, the
map $x\mapsto x+y:X\to X$ is\cmmnt{ \imp.   \prooflet{\Prf\ If $J\subseteq I$
is finite and $z\in\{0,1\}^J$, set $z'=\family{j}{J}{z(j)+_2y(j)}$;  then

\Centerline{$\lambda\{x:(x+y)\restr J=z\}
=\lambda\{x:x\restr J=z'\}=2^{-\#(J)}$.}

\noindent As $J$ is arbitrary, (c) tells us that $x\mapsto x+y$ is \imp.\
\Qed}    Now since

\Centerline{$(x+y)+y=x+(y+y)=x+0=x$}

\noindent for every $x$, the map $x\mapsto x+y:X\to X$ is bijective and
equal to its inverse, so it is actually} a measure space automorphism
of $(X,\lambda)$.

%\def\spheader#1#2#3#4#5{\header{#1#2#3#4#5}{\bf (#5)}}
%\def\header#1{
%     \wheader{#1}{10}{4}{4}{24pt}}

\header{254Je}{\bf *(e)}\dvAnew{2016}
Just because all the factors $(X_i,\mu_i)$ are the same, we
have another class of automorphisms of $(X,\lambda)$, corresponding to
permutations of $I$.   If $\pi:I\to I$ is any permutation, then
we have a corresponding function $x\mapsto x\pi:X\to X$.   \cmmnt{If
$J\subseteq I$ is finite and $z\in\{0,1\}^J$, set $J'=\pi[J]$ and
$z'=z\pi^{-1}\in\{0,1\}^{J'}$;  then

\Centerline{$\lambda\{x:(x\pi)\restr J=z\}=\lambda\{x:x\restr J'=z'\}
=2^{-\#(J')}=2^{-\#(J)}$.}

\noindent So $x\mapsto x\pi$ is \imp.   This time, its inverse is
$x\mapsto x\pi^{-1}$, which is again \imp;  so} $x\mapsto x\pi$ is a measure
space automorphism.

\leader{254K}{}\cmmnt{In the case of countably infinite $I$, we
have a very important
relationship between the usual product measure of $\{0,1\}^I$ and
Lebesgue measure on $[0,1]$.

\medskip

\noindent}{\bf Proposition} Let $\lambda$ be the usual measure on
$X=\{0,1\}^{\Bbb N}$, and let $\mu$ be Lebesgue measure on $[0,1]$;
write $\Lambda$ for the domain of $\lambda$ and $\Sigma$ for the domain
of $\mu$.

(i) For $x\in X$ set $\phi(x)=\sum_{i=0}^{\infty}2^{-i-1}x(i)$.   Then

\qquad $\phi^{-1}[E]\in\Lambda$ and $\lambda\phi^{-1}[E]=\mu E$
for every $E\in\Sigma$;

\qquad $\phi[F]\in\Sigma$ and $\mu\phi[F]=\lambda F$ for
every $F\in\Lambda$.

(ii) There is a bijection $\tilde\phi:X\to[0,1]$ which is equal to
$\phi$ at all but countably many points, and any such bijection is an
isomorphism between $(X,\Lambda,\lambda)$ and $([0,1],\Sigma,\mu)$.

\proof{{\bf (a)} The first point to observe is that $\phi$ itself
is nearly a bijection.   Setting

\Centerline{$H=\{x:x\in X$, $\exists\,m\in\Bbb
N$, $x(i)=x(m)\Forall i\ge m\}$,}

\Centerline{$H'=\{2^{-n}k:n\in\Bbb N$, $k\le 2^n\}$,}

\noindent then $H$ and $H'$ are countable and $\phi\restr X\setminus H$
is a bijection between $X\setminus H$ and $[0,1]\setminus H'$.   (For
$t\in[0,1]\setminus H'$, $\phi^{-1}(t)$ is the binary expansion of $t$.)
Because $H$ and $H'$ are countably infinite, there is a bijection
between them;  combining this with $\phi\restr X\setminus H$, we have a
bijection between $X$ and $[0,1]$ equal to $\phi$ except at countably
many points.   For the rest of this proof, let $\tilde\phi$ be any such
bijection.   Let $M$ be the countable set $\{x:x\in
X$, $\phi(x)\ne\tilde\phi(x)\}$, and $N$ the countable set
$\phi[M]\cup\tilde\phi[M]$;  then
$\phi[A]\symmdiff\tilde\phi[A]\subseteq N$ for every $A\subseteq X$.

\medskip

{\bf (b)} To see that $\lambda\tilde\phi^{-1}[E]$ exists and is equal to
$\mu E$ for every $E\in\Sigma$, I consider successively more complex
sets $E$.

\medskip

\quad\grheada\ If $E=\{t\}$ then
$\lambda\tilde\phi^{-1}[E]=\lambda\{\tilde\phi^{-1}(t)\}$ exists and is
zero.

\medskip

\quad\grheadb\ If $E$ is of the form
$\coint{2^{-n}k,2^{-n}(k+1)}$, where $n\in\Bbb N$ and $0\le k<2^n$, then
$\phi^{-1}[E]$ differs by at most two points from a set of the form
$\{x:x(i)=z(i)\Forall i<n\}$, so $\tilde\phi^{-1}[E]$ differs from
this by a countable set, and

\Centerline{$\lambda\tilde\phi^{-1}[E]=2^{-n}=\mu E$.}

\medskip

\quad\grheadc\ If $E$ is of the form
$\coint{2^{-n}k,2^{-n}l}$, where $n\in\Bbb N$ and $0\le k<l\le 2^n$,
then
\discrcenter{468pt}
{$E=\bigcup_{k\le i<l}\coint{2^{-n}i,2^{-n}(i+1)}$, }so

\Centerline{$\lambda\tilde\phi^{-1}[E]=2^{-n}(l-k)=\mu E$.}

\medskip

\quad\grheadd\ If $E$ is of the form $\coint{t,u}$, where
$0\le t<u\le 1$, then for each $n\in\Bbb N$ set
$k_n=\lfloor 2^nt\rfloor$, the integer
part of $2^nt$, $l_n=\lfloor 2^nu\rfloor$ and
$E_n=\coint{2^{-n}(k_n+1),2^{-n}l_n}$;  then $\sequencen{E_n}$ is a
non-decreasing sequence and $\bigcup_{n\in\Bbb N}E_n$ is $\ooint{t,u}$.
So (using ($\alpha$))

$$\eqalign{\lambda\tilde\phi^{-1}[E]
&=\lambda\tilde\phi^{-1}[\bigcup_{n\in\Bbb N}E_n]
=\lim_{n\to\infty}\lambda\tilde\phi^{-1}[E_n]\cr
&=\lim_{n\to\infty}\mu E_n
=\mu E.\cr}$$

\medskip

\quad\grheade\ If $E\in\Sigma$, then for any
$\epsilon>0$ there is a sequence $\sequencen{I_n}$ of
half-open subintervals of $\coint{0,1}$ such that
$E\setminus\{1\}\subseteq\bigcup_{n\in\Bbb N}I_n$ and
$\sum_{n=0}^{\infty}\mu I_n\le\mu E+\epsilon$;  now
$\tilde\phi^{-1}[E]
\subseteq \{\tilde\phi^{-1}(1)\}\cup\bigcup_{n\in\Bbb N}\phi^{-1}[I_n]$,
so

\Centerline{$\lambda^*\tilde\phi^{-1}[E]
\le\lambda(\bigcup_{n\in\Bbb N}\tilde\phi^{-1}[I_n])
\le \sum_{n=0}^{\infty}\lambda\tilde\phi^{-1}[I_n]
=\sum_{n=0}^{\infty}\mu I_n
\le\mu E+\epsilon$.}

\noindent As $\epsilon$ is arbitrary,
$\lambda^*\tilde\phi^{-1}[E]\le\mu E$,  and there
is a $V\in\Lambda$ such that $\tilde\phi^{-1}[E]\subseteq V$ and
$\lambda V\le\mu E$.

\medskip

\quad\grheadz\ Similarly, there is a $V'\in\Lambda$ such
that $V'\supseteq\tilde\phi^{-1}[[0,1]\setminus E]$ and
$\lambda V'\le\mu([0,1]\setminus E)$.   Now $V\cup V'=X$, so

\Centerline{$\lambda(V\cap V')=\lambda V+\lambda V'-\lambda(V\cup V')
\le\mu E+(1-\mu E)-1=0$}

\noindent and

\Centerline{$\tilde\phi^{-1}[E]=(X\setminus V')
\cup(V\cap V'\cap\tilde\phi^{-1}[E])$}

\noindent belongs to $\Lambda$, with

\Centerline{$\lambda\tilde\phi^{-1}[E]\le\lambda V\le\mu E$;}

\noindent at the same time,

\Centerline{$1-\lambda\tilde\phi^{-1}[E]\le\lambda V'\le 1-\mu E$}

\noindent so $\lambda\tilde\phi^{-1}[E]=\mu E$.

\medskip

{\bf (c)} Now suppose that $C\subseteq X$ is a measurable cylinder of
the special form $\{x:x(0)=\epsilon_0,\ldots,x(n)=\epsilon_n\}$ for some
$\epsilon_0,\ldots,\epsilon_n\in\{0,1\}$.   Then
$\phi[C]=[t,t+2^{-n-1}]$ where $t=\sum_{i=0}^n2^{-i-1}\epsilon_i$, so
that $\mu\phi[C]=\lambda C$.   Since
$\tilde\phi[C]\symmdiff\phi[C]\subseteq N$ is countable,
$\mu\tilde\phi[C]=\lambda C$.

If $C\subseteq X$ is any measurable cylinder, then it is of the form
$\{x:x\restr J=z\}$ for some finite $J\subseteq\Bbb N$;  taking $n$ so
large that $J\subseteq\{0,\ldots,n\}$, $C$ is expressible as a disjoint
union of $2^{n+1-\#(J)}$ sets of the form just considered, being just
those in which $\epsilon_i=z(i)$ for $i\in J$.   Summing their measures,
we again get $\mu\tilde\phi[C]=\lambda C$.   Now 254G tells us that
$\tilde\phi^{-1}:[0,1]\to X$ is \imp, that is, $\tilde\phi[W]$ is
Lebesgue measurable, with measure $\lambda W$, for every $W\in\Lambda$.

Putting this together with (b), $\tilde\phi$ must be an isomorphism
between $(X,\Lambda,\lambda)$ and $([0,1],\Sigma,\mu)$, as claimed in
(ii) of the proposition.

\medskip

{\bf (d)} As for (i), if $E\in\Sigma$ then
$\phi^{-1}[E]\symmdiff\tilde\phi^{-1}[E]\subseteq M$ is countable, so
$\lambda\phi^{-1}[E]=\lambda\tilde\phi^{-1}[E]=\mu E$.   While if
$W\in\Lambda$, $\phi[F]\symmdiff\tilde\phi[W]\subseteq N$ is countable,
so $\mu\phi[W]=\mu\tilde\phi[W]=\lambda W$.
}%end of proof of 254K

\leader{254L}{\dvrocolon{Subspaces}}\cmmnt{ Just as in 251Q, we can
consider the product of
subspace measures.   There is a simplification in the form of
the result because in the present context we are restricted to
probability measures.

\medskip

\noindent}{\bf Theorem} Let
$\langle(X_i,\Sigma_i,\mu_i)\rangle_{i\in I}$
be a family of probability spaces, and $(X,\Lambda,\lambda)$ their
product.

(a) For each $i\in I$, let $A_i\subseteq X_i$ be a set of full outer
measure, and write $\tilde\mu_i$ for the subspace measure on
$A_i$\cmmnt{ (214B)}.
Let $\tilde\lambda$ be the product measure on $A=\prod_{i\in I}A_i$.
Then $\tilde\lambda$ is the subspace measure on $A$ induced by
$\lambda$.

(b) $\lambda^*(\prod_{i\in I}A_i)=\prod_{i\in I}\mu_i^*A_i$ whenever
$A_i\subseteq X_i$ for every $i$.

\proof{{\bf (a)} Write $\lambda_A$ for the subspace measure on $A$
defined from $\lambda$, and $\Lambda_A$ for its domain;  write
$\tilde\Lambda$ for the domain of $\tilde\lambda$.

\medskip

\quad{\bf (i)} Let $\phi:A\to X$ be the identity map.   If $C\subseteq
X$ is a measurable cylinder, say $C=\prod_{i\in I}C_i$ where
$C_i\in\Sigma_i$ for each $i$, then $\phi^{-1}[C]=\prod_{i\in I}(C_i\cap
A_i)$ is a measurable cylinder in $A$, and

\Centerline{$\tilde\lambda\phi^{-1}[C]=\prod_{i\in I}\tilde\mu_i(C_i\cap
A_i)\le\prod_{i\in I}\mu_iC_i=\mu C$.}

\noindent By 254G, $\phi$ is \imp, that is, $\tilde\lambda(A\cap
W)=\lambda W$ for every $W\in\Lambda$.   But this
means that $\tilde\lambda V$ is defined and equal to
$\lambda_AV=\lambda^*V$ for every $V\in\Lambda_A$, since
for any such $V$ there is a $W\in\Lambda$ such that $V=A\cap
W$ and $\lambda W=\lambda_AV$.   In particular, $\lambda_AA=1$.

\medskip

\quad{\bf (ii)} Now regard $\phi$ as a function from the measure space
$(A,\Lambda_A,\lambda_A)$ to $(A,\tilde\Lambda,\tilde\lambda)$.   If $D$
is a measurable
cylinder in $A$, we can express it as $\prod_{i\in I}D_i$ where every
$D_i$ belongs to the domain of $\tilde\mu_i$ and $D_i=A_i$ for all but
finitely many $i$.   Now for each $i$ we can find $C_i\in\Sigma_i$ such
that $D_i=C_i\cap A_i$ and $\mu C_i=\tilde\mu_iD_i$, and we can suppose
that $C_i=X_i$ whenever $D_i=A_i$.   In this case $C=\prod_{i\in
I}C_i\in\Lambda$ and

\Centerline{$\lambda C=\prod_{i\in I}\mu_iC_i=\prod_{i\in
I}\tilde\mu_iD_i=\tilde\lambda D$.}

\noindent Accordingly

\Centerline{$\lambda_A\phi^{-1}[D]=\lambda_A(A\cap C)\le\lambda C
=\tilde\lambda D$.}

\noindent By 254G again, $\phi$ is \imp\ in this manifestation, that is,
$\lambda_AV$ is defined and equal to $\tilde\lambda V$ for every
$V\in\tilde\Lambda$.   Putting this together with (i), we
have $\lambda_A=\tilde\lambda$, as claimed.

\medskip

{\bf (b)} For each $i\in I$, choose a set $E_i\in\Sigma_i$ such that
$A_i\subseteq E_i$ and $\mu_iE_i=\mu^*_iA_i$;  do this in such a way
that $E_i=X_i$ whenever $\mu^*_iA_i=1$.   Set
$B_i=A_i\cup(X_i\setminus E_i)$, so that $\mu^*_iB_i=1$ for each $i$ (if
$F\in\Sigma_i$ and
$F\supseteq B_i$ then $F\cap E_i\supseteq A_i$, so

\Centerline{$\mu_iF=\mu_i(F\cap E_i)+\mu_i(F\setminus E_i)
=\mu_i E_i+\mu_i(X_i\setminus E_i)=1$.)}

\noindent By (a), we can identify the subspace measure $\lambda_B$ on
$B=\prod_{i\in I}B_i$ with the product of the subspace measures
$\tilde\mu_i$ on $B_i$.   In particular, $\lambda^*B=\lambda_BB=1$.
Now $A_i=B_i\cap E_i$ so (writing $A=\prod_{i\in I}A_i$),
$A=B\cap\prod_{i\in I}E_i$.

If $\prod_{i\in I}\mu_i^*A_i=0$, then for every $\epsilon>0$ there is a
finite $J\subseteq I$ such that $\prod_{i\in J}\mu_i^*A_i\le\epsilon$;
consequently (using 254Fb)

\Centerline{$\lambda^*A\le\lambda\{x:x(i)\in E_i$ for every
$i\in J\}=\prod_{i\in J}\mu_iE_i\le\epsilon$.}

\noindent As $\epsilon$ is arbitrary, $\lambda^*A=0$.  If
$\prod_{i\in I}\mu_i^*A_i>0$, then for every $n\in\Bbb N$ the set
$\{i:\mu^*A_i\le 1-2^{-n}\}$ must be finite, so

\Centerline{$J=\{i:\mu^*A_i<1\}=\{i:E_i\ne X_i\}$}

\noindent is countable.   By 254Fb again,
applied to $\langle E_i\cap B_i\rangle_{i\in I}$ in the product
$\prod_{i\in I}B_i$,

$$\eqalign{\lambda^*(\prod_{i\in I}A_i)
&=\lambda_B(\prod_{i\in I}A_i)
=\lambda_B\{x:x\in B,\,x(i)\in E_i\cap B_i\text{ for every }i\in J\}\cr
&=\prod_{i\in J}\tilde\mu_{i}(E_i\cap B_i)
=\prod_{i\in I}\mu_i^*A_i,\cr}$$

\noindent as required.
}%end of proof of 254L

\leader{254M}{}\cmmnt{ I now turn to the basic results which make it
possible to use these product measures effectively.   First, I offer a
vocabulary for
dealing with subproducts.}   Let $\langle X_i\rangle_{i\in I}$ be a
family of sets, with product $X$.

\spheader 254Ma For $J\subseteq I$, write $X_J$ for
$\prod_{i\in J}X_i$.   We have a canonical bijection
$x\mapsto(x\restr J,x\restr I\setminus J):
X\to X_I\times X_{I\setminus J}$.   Associated with this
we have the map  $x\mapsto\pi_J(x)=x\restr J:X\to X_J$.   Now I will say
that a set $W\subseteq X$ is {\bf determined by coordinates in $J$} if
there is a $V\subseteq X_J$ such that $W=\pi_J^{-1}[V]$;  that is, $W$
corresponds to
$V\times X_{I\setminus J}\subseteq X_J\times X_{I\setminus J}$.

\cmmnt{It is easy to see that}

$$\eqalign{W&\text{ is determined by coordinates in }J\cr
&\quad\iff x'\in W\text{ whenever }x\in W,\,x'\in X\text{ and }x'\restr
J=x\restr J\cr
&\quad\iff W=\pi_J^{-1}[\pi_J[W]].\cr}$$

\noindent It follows that if $W$ is determined by coordinates in $J$,
and $J\subseteq K\subseteq I$, $W$ is also determined by coordinates in
$K$.   The family $\Cal W_J$ of subsets of $X$ determined by coordinates
in $J$ is closed under complementation and arbitrary unions and
intersections.
\prooflet{\Prf\ If $W\in\Cal W_J$, then

\Centerline{$X\setminus W=X\setminus\pi_J^{-1}[\pi_J[W]]
=\pi_J^{-1}[X_J\setminus\pi_J[W]]\in\Cal W_J$.}

\noindent If $\Cal V\subseteq\Cal W_J$, then

\Centerline{$\bigcup\Cal V
=\bigcup_{V\in\Cal V}\pi_J^{-1}[\pi_J[V]]
=\pi_J^{-1}[\bigcup_{V\in\Cal V}\pi_J[V]]
\in\Cal W_J$.  \Qed}
}%end of prooflet

\spheader 254Mb \cmmnt{It follows that}

\Centerline{$\Cal W=\bigcup\{\Cal W_J:J\subseteq I$ is
countable$\}$\cmmnt{,}}

\noindent\cmmnt{the family of subsets of $X$ determined by coordinates
in some countable
set, }is a $\sigma$-algebra of subsets of $X$.
\prooflet{\Prf\ (i) $X$ and $\emptyset$ are determined by coordinates in
$\emptyset$ (recall that $X_{\emptyset}$ is a singleton, and that
$X=\pi_{\emptyset}^{-1}[X_{\emptyset}]$,
$\emptyset=\pi^{-1}_{\emptyset}[\emptyset]$).   (ii) If $W\in\Cal W$,
there is a countable $J\subseteq I$ such that $W\in\Cal W_J$;  now

\Centerline{$X\setminus W=\pi_J^{-1}[X_J\setminus\pi_J[W]]\in\Cal
W_J\subseteq\Cal W$.}

\noindent (iii) If $\sequencen{W_n}$
is a sequence in $\Cal W$, then for each $n\in\Bbb N$ there is a
countable $J_n\subseteq I$ such that $W\in\Cal W_{J_n}$.   Now
$J=\bigcup_{n\in\Bbb N}J_n$ is a countable subset of $I$,
and every $W_n$ belongs to $\Cal W_J$, so

\Centerline{$\bigcup_{n\in\Bbb N}W_n\in\Cal W_J\subseteq\Cal W$.  \Qed}
}%end of prooflet

\spheader 254Mc If $i\in I$ and $E\subseteq X_i$ then
$\{x:x\in X$, $x(i)\in E\}$ is determined by the single coordinate $i$,
so surely belongs to $\Cal W$;  accordingly $\Cal W$ must include
$\Tensorhat_{i\in I}\Cal PX_i$.
\cmmnt{{\it A fortiori}, if $\Sigma_i$ is a $\sigma$-algebra of
subsets of $X_i$ for each $i$,
$\Cal W\supseteq\Tensorhat_{i\in I}\Sigma_i$;  that
is, every member of $\Tensorhat_{i\in I}\Sigma_i$ is determined
by coordinates in some countable set.}

\vleader{72pt}{254N}{\bf Theorem} Let
$\langle(X_i,\Sigma_i,\mu_i)\rangle_{i\in I}$ be
a family of probability spaces and $\langle K_j\rangle_{j\in J}$ a
partition of $I$.   For each $j\in J$ let $\lambda_j$ be the product
measure on $Z_j=\prod_{i\in K_j}X_i$, and write $\lambda$ for the
product measure on $X=\prod_{i\in I}X_i$.   Then the natural bijection

\Centerline{$x\mapsto\phi(x)
=\langle x\restr K_j\rangle_{j\in J}:X\to\prod_{j\in J}Z_j$}

\noindent identifies $\lambda$ with the product of the family
$\langle\lambda_j\rangle_{j\in J}$.

In particular, if $K\subseteq I$ is any set, then $\lambda$ can be
identified with the c.l.d.\ product of the product measures on
$\prod_{i\in K}X_i$ and $\prod_{i\in I\setminus K}X_i$.

\proof{ (Compare 251N.) Write $Z=\prod_{j\in J}Z_j$ and
$\tilde\lambda$ for the product measure on $Z$;  let $\Lambda$,
$\tilde\Lambda$ be the domains of $\lambda$ and $\tilde\lambda$.

\medskip

{\bf (a)} Let $C\subseteq Z$ be a measurable cylinder.   Then
$\lambda^*\phi^{-1}[C]\le\tilde\lambda C$.   \Prf\ Express $C$ as
$\prod_{j\in J}C_j$ where $C_j\subseteq Z_j$ belongs to the domain
$\Lambda_j$ of $\lambda_j$ for each $j$.
Set $L=\{j:C_j\ne Z_j\}$, so that $L$ is finite.   Let $\epsilon>0$.
For each $j\in L$ let $\sequencen{C_{jn}}$ be a sequence of measurable
cylinders in $Z_j=\prod_{i\in K_j}X_i$ such that
$C_j\subseteq\bigcup_{n\in\Bbb N}C_{jn}$ and
$\sum_{n=0}^{\infty}\lambda_jC_{jn}\le\lambda C_j+\epsilon$.   Express
each $C_{jn}$ as $\prod_{i\in K_j}C_{jni}$ where $C_{jni}\in\Sigma_i$
for $i\in K_j$ (and $\{i:C_{jni}\ne X_i\}$ is finite).

For $f\in\BbbN^L$, set

\Centerline{$D_f
=\{x:x\in X$, $x(i)\in C_{j,f(j),i}$ whenever $j\in L$, $i\in K_j\}$.}

\noindent Because $\bigcup_{j\in L}\{i:C_{j,f(j),i}\ne X_i\}$ is finite,
$D_f$ is a measurable cylinder in $X$, and

\Centerline{$\lambda D_f=\prod_{j\in L}\prod_{i\in K_j}\mu_iC_{j,f(j),i}
=\prod_{j\in L}\lambda_jC_{j,f(j)}$.}

\noindent Also

\Centerline{$\bigcup\{D_f:f\in\BbbN^L\}\supseteq\phi^{-1}[C]$}

\noindent because if $\phi(x)\in C$ then $\phi(x)(j)\in C_j$ for each
$j\in L$, so there must be an $f\in\BbbN^L$ such that $\phi(x)(j)\in
C_{j,f(j)}$ for every $j\in L$.   But (because $\BbbN^L$ is countable)
this means that

$$\eqalign{\lambda^*\phi^{-1}[C]
&\le\sum_{f\in\BbbN^L}\lambda D_f
=\sum_{f\in \BbbN^L}\prod_{j\in L}\lambda_jC_{j,f(j)}\cr
&=\prod_{j\in L}\sum_{n=0}^{\infty}\lambda_jC_{jn}
\le\prod_{j\in L}(\lambda_jC_j+\epsilon).\cr}$$

\noindent As $\epsilon$ is arbitrary,

\Centerline{$\lambda^*\phi^{-1}[C]\le\prod_{j\in L}\lambda_jC_j
=\tilde\lambda C$.  \Qed}

By 254G, it follows that $\lambda\phi^{-1}[W]$ is defined, and equal to
$\tilde\lambda W$, whenever $W\in\tilde\Lambda$.
\medskip

{\bf (b)} Next, $\tilde\lambda\phi[D]=\lambda D$ for every measurable
cylinder $D\subseteq X$.   \Prf\ This is easy.   Express $D$ as
$\prod_{i\in I}D_i$ where $D_i\in\Sigma_i$ for every $i\in I$ and
$\{i:D_i\ne\Sigma_i\}$ is finite.   Then $\phi[D]=\prod_{j\in J}\tilde
D_j$, where $\tilde D_j=\prod_{i\in K_j}D_i$ is a measurable cylinder
for each $j\in J$.   Because $\{j:\tilde D_j\ne Z_j\}$ must also be
finite (in fact, it cannot have more members than the finite set
$\{i:D_i\ne X_i\}$), $\prod_{j\in J}\tilde D_j$ is itself a measurable
cylinder in $Z$, and

\Centerline{$\tilde\lambda\phi[D]
=\prod_{j\in J}\lambda_j\tilde D_j
=\prod_{j\in J}\prod_{i\in K_j}\mu D_i
=\lambda D$.   \Qed}

Applying 254G to $\phi^{-1}:Z\to X$, it follows that
$\tilde\lambda\phi[W]$ is defined, and equal to $\lambda W$, for every
$W\in\Lambda$.   But together with (a) this means that
for any $W\subseteq X$,

\inset{if $W\in\Lambda$ then $\phi[W]\in\tilde\Lambda$ and
$\tilde\lambda\phi[W]=\lambda W$,}

\inset{if $\phi[W]\in\tilde\Lambda$ then $W\in\Lambda$ and $\lambda
W=\tilde\lambda\phi[W]$.}

\noindent And of course this is just what is meant by saying that $\phi$
is an isomorphism between $(X,\Lambda,\lambda)$ and
$(Z,\tilde\Lambda,\tilde\lambda)$.
}%end of proof of 254N

\leader{254O}{Proposition} Let
$\langle(X_i,\Sigma_i,\mu_i)\rangle_{i\in I}$ be a family of probability
spaces.   For each $J\subseteq I$ let
$\lambda_J$ be the product probability measure on
$X_J=\prod_{i\in J}X_i$, and $\Lambda_J$ its domain;  write $X=X_I$,
$\lambda=\lambda_I$ and $\Lambda=\Lambda_I$.   For $x\in X$ and
$J\subseteq I$ set $\pi_J(x)=x\restr J\in X_J$.

(a) For every $J\subseteq I$, $\lambda_J$ is the image measure
$\lambda\pi_J^{-1}$\cmmnt{ (234D)};  in particular,
$\pi_J:X\to X_J$ is \imp\ for $\lambda$
and $\lambda_J$.

(b) If $J\subseteq I$ and $W\in\Lambda$ is
determined by coordinates in $J$\cmmnt{ (254M)}, then
$\lambda_J\pi_J[W]$ is defined and equal to $\lambda W$.   Consequently
there are $W_1$, $W_2$ belonging to the $\sigma$-algebra of subsets of
$X$ generated by

\Centerline{$\{\{x:x(i)\in E\}:i\in J$, $E\in\Sigma_i\}$}

\noindent such that $W_1\subseteq W\subseteq W_2$ and
$\lambda(W_2\setminus W_1)=0$.

(c) For every $W\in\Lambda$, we can find a
countable set $J$ and $W_1$, $W_2\in\Lambda$, both
determined by coordinates in $J$, such that
$W_1\subseteq W\subseteq W_2$ and $\lambda(W_2\setminus W_1)=0$.

(d) For every $W\in\Lambda$, there is a
countable set $J\subseteq I$ such that $\pi_J[W]\in\Lambda_J$ and
$\lambda_J\pi_J[W]=\lambda W$;  so that
$W'=\pi_J^{-1}[\pi_J[W]]$ belongs to $\Lambda$, and
$\lambda(W'\setminus W)=0$.

\proof{{\bf (a)(i)} By 254N, we can identify $\lambda$ with the
product of $\lambda_J$ and $\lambda_{I\setminus J}$ on
$X_J\times X_{I\setminus J}$.
Now $\pi_J^{-1}[E]\subseteq X$ corresponds to
$E\times X_{I\setminus J}\subseteq X_J\times X_{I\setminus J}$, so

\Centerline{$\lambda(\pi^{-1}[E])
=\lambda_JE\cdot\lambda_{I\setminus J}X_{I\setminus J}=\lambda_JE$,}

\noindent by 251E or 251Ia, whenever $E\in\Lambda_J$.   This shows that
$\pi_J$ is \imp.

\medskip

\quad{\bf (ii)} To see that $\lambda_J$ is actually the image measure,
suppose that $E\subseteq X_J$ is such that $\pi_J^{-1}[E]\in\Lambda$.
Identifying $\pi^{-1}_J[E]$ with $E\times X_{I\setminus J}$, as before,
we are supposing that $E\times X_{I\setminus J}$ is measured by the
product measure on $X_J\times X_{I\setminus J}$.   But this means that
for $\lambda_{I\setminus J}$-almost every
$z\in X_{I\setminus J}$, $E_z=\{y:(y,z)\in E\times X_{I\setminus J}\}$
belongs to $\Lambda_J$ (252D(ii), because $\lambda_J$ is complete).
Since $E_z=E$ for every $z$, $E$ itself belongs to $\Lambda_J$, as
claimed.

\medskip

{\bf (b)} If $W\in\Lambda$ is determined by coordinates
in $J$, set $H=\pi_J[W]$;  then $\pi_J^{-1}[H]=W$, so $H\in\Lambda_J$ by
(a) just above.   By 254Ff, there are $H_1$,
$H_2\in\Tensorhat_{i\in J}\Sigma_i$ such that
$H_1\subseteq H\subseteq H_2$ and $\lambda_J(H_2\setminus H_1)=0$.

Let $\Tau_J$ be the $\sigma$-algebra of subsets of $X$ generated by sets
of the form $\{x:x(i)\in E\}$ where $i\in J$ and $E\in\Sigma_J$.
Consider $\Tau'_J=\{G:G\subseteq X_J$, $\pi_J^{-1}[G]\in\Tau_J\}$.
This
is a $\sigma$-algebra of subsets of $X_J$, and it contains $\{y:y\in
X_J$, $y(i)\in E\}$ whenever $i\in J$, $E\in\Sigma_J$ (because

\Centerline{$\pi_J^{-1}[\{y:y\in X_J$, $y(i)\in E\}]=\{x:x\in X$,
$x(i)\in
E\}$}

\noindent whenever $i\in J$, $E\subseteq X_i$).   So $\Tau'_J$ must
include $\Tensorhat_{i\in J}\Sigma_i$.   In particular, $H_1$ and $H_2$
both belong to $\Tau'_J$, that is, $W_k=\pi_J^{-1}[H_k]$ belongs to
$\Tau_J$ for both $k$.   Of course $W_1\subseteq W\subseteq W_2$,
because $H_1\subseteq H\subseteq H_2$, and

\Centerline{$\lambda(W_2\setminus W_1)=\lambda_J(H_2\setminus H_1)=0$,}
\noindent as required.

\medskip

{\bf (c)} Now take any $W\in\Lambda$.   By
254Ff, there are $W_1$ and $W_2\in\Tensorhat_{i\in I}\Sigma_i$ such that
$W_1\subseteq W\subseteq W_2$ and $\lambda(W_2\setminus W_1)=0$.   By
254Mc, there are countable sets $J_1$, $J_2\subseteq I$ such that, for
each $k$, $W_k$ is determined by coordinates in $J_k$.   Setting
$J=J_1\cup J_2$, $J$ is a countable subset of $I$ and both $W_1$ and
$W_2$ are determined by coordinates in $J$.

\medskip

{\bf (d)} Continuing the argument from (c), $\pi_J[W_1]$,
$\pi_J[W_2]\in\Lambda_J$, by (b), and
$\lambda_J(\pi_J[W_2]\setminus\pi_J[W_1])=0$.   Since
$\pi_J[W_1]\subseteq\pi_J[W]\subseteq\pi_J[W_2]$, it follows that
$\pi_J[W]\in\Lambda_J$, with
$\lambda_J\pi_J[W]=\lambda_J\pi_J[W_2]$;  so that, setting
$W'=\pi_J^{-1}[\pi_J[W]]$, $W'\in\Lambda$, and

\Centerline{$\lambda W'=\lambda_J\pi_J[W]=\lambda_J\pi_J[W_2]
=\lambda\pi_J^{-1}[\pi_J[W_2]]=\lambda W_2=\lambda W$.}
}%end of proof of 254O

\leader{254P}{Proposition} Let
$\langle(X_i,\Sigma_i,\mu_i)\rangle_{i\in I}$
be a family of probability spaces, and for each $J\subseteq I$ let
$\lambda_J$ be the product probability measure on
$X_J=\prod_{i\in J}X_i$, and $\Lambda_J$ its domain;  write $X=X_I$,
$\Lambda=\Lambda_I$ and $\lambda=\lambda_I$.   For $x\in X$ and
$J\subseteq I$ set $\pi_J(x)=x\restr J\in X_J$.

(a) If $J\subseteq I$ and $g$ is a real-valued function defined on a
subset of $X_J$, then $g$ is $\Lambda_J$-measurable iff $g\pi_J$ is
$\Lambda$-measurable.

(b) Whenever $f$ is a $\Lambda$-measurable real-valued function defined
on a $\lambda$-conegligible subset of $X$, we can find a countable set
$J\subseteq I$ and a $\Lambda_J$-measurable function $g$ defined on a
$\lambda_J$-conegligible subset of $X_J$ such that $f$ extends $g\pi_J$.

\proof{{\bf (a)(i)} If $g$ is $\Lambda_J$-measurable and $a\in\Bbb R$,
there is an $H\in\Lambda_J$ such that $\{y:y\in\dom g$, $g(y)\ge
a\}=H\cap\dom g$.   Now $\pi_J^{-1}[H]\in\Lambda$, by 254Oa, and
$\{x:x\in\dom g\pi_J$, $g\pi_J(x)\ge a\}=\pi_J^{-1}[H]\cap\dom g\pi_J$.
So
$g\pi_J$ is $\Lambda$-measurable.

\medskip

\quad{\bf (ii)} If $g\pi_J$ is $\Lambda$-measurable and $a\in\Bbb R$,
then there is a $W\in\Lambda$ such that
$\{x:g\pi_J(x)\ge a\}=W\cap\dom g\pi_J$.   As in the proof of 254Oa, we
may identify $\lambda$ with the
product of $\lambda_J$ and $\lambda_{I\setminus J}$, and 252D(ii) tells
us that, if we identify $W$ with the corresponding subset of
$X_J\times X_{I\setminus J}$, there is at least one
$z\in X_{I\setminus J}$ such
that $W_z=\{y:y\in X_J$, $(y,z)\in W\}$ belongs to $\Lambda_J$.   But
since (on this convention) $g\pi_J(y,z)=g(y)$ for every $y\in X_J$, we
see that $\{y:y\in\dom g$, $g(y)\ge a\}=W_z\cap\dom g$.   As $a$ is
arbitrary, $g$ is $\Lambda_J$-measurable.

\medskip

{\bf (b)} For rational numbers $q$, set
$W_q=\{x:x\in\dom f$, $f(x)\ge q\}$.   By 254Oc we can find for each $q$
a countable set
$J_q\subseteq I$ and sets $W'_q$, $W_q''$, both determined by
coordinates
in $J_q$, such that $W'_q\subseteq W_q\subseteq W_q''$ and
$\lambda(W_q''\setminus W_q')=0$.   Set $J=\bigcup_{q\in\Bbb Q}J_q$,
$V=X\setminus\bigcup_{q\in\Bbb Q}(W_q''\setminus W_q')$;  then $J$ is a
countable subset of $I$ and $V$ is a conegligible subset of $X$;
moreover, $V$ is determined by coordinates in $J$ because all the
$W_q'$, $W_q''$ are.

For every $q\in\Bbb Q$, $W_q\cap V=W_q'\cap V$, because
$V\cap(W_q\setminus W_q')\subseteq V\cap(W_q''\setminus
W_q')=\emptyset$;  so $W_q\cap V$ is determined by coordinates in $J$.
Consequently $V\cap\dom f=\bigcup_{q\in\Bbb Q}V\cap W_q$ also is
determined by coordinates in $J$.   Also

\Centerline{$\{x:x\in V\cap\dom f$, $f(x)\ge a\}
=\bigcap_{q\le a}V\cap W_q$}

\noindent is determined by coordinates in $J$.   What this means is that
if $x$, $x'\in V$ and $\pi_Jx=\pi_Jx'$, then $x\in\dom f$ iff $x'\in\dom
f$ and in this case $f(x)=f(x')$.   Setting $H=\pi_J[V\cap\dom f]$, we
have $\pi_J^{-1}[H]=V\cap\dom f$ a conegligible subset of $X$, so
(because $\lambda_J=\lambda\pi_J^{-1}$) $H$ is conegligible in $X_J$.
Also, for
$y\in H$, $f(x)=f(x')$ whenever $\pi_Jx=\pi_Jx'=y$, so there is a
function $g:H\to\Bbb R$ defined by saying that $g\pi_J(x)=f(x)$ whenever
$x\in V\cap\dom f$.   Thus $g$ is defined almost everywhere in $X_J$ and
$f$ extends $g\pi_J$.   Finally, for any $a\in\Bbb R$,

\Centerline{$\pi_J^{-1}[\{y:g(y)\ge a\}]
=\{x:x\in V\cap\dom f$, $f(x)\ge a\}\in\Lambda$;}

\noindent by 254Oa, $\{y:g(y)\ge a\}\in\Lambda_J$;  as $a$ is arbitrary,
$g$ is measurable.
}%end of proof of 254P

\leader{254Q}{Proposition} Let $\langle(X_i,\Sigma_i,\mu_i)\rangle_{i\in
I}$ be a family of probability spaces, and for each $J\subseteq I$ let
$\lambda_J$ be the product probability measure on $X_J=\prod_{i\in
J}X_i$;  write $X=X_I$, $\lambda=\lambda_I$.   For $x\in X$, $J\subseteq
I$ set $\pi_J(x)=x\restr J\in X_J$.

(a) Let $\eusm S$ be the linear subspace of $\Bbb R^X$ spanned by
$\{\chi C:C\subseteq X$ is a measurable cylinder$\}$.   Then for every
$\lambda$-integrable real-valued function $f$ and every $\epsilon>0$
there is a $g\in\eusm S$ such that $\int|f-g|d\lambda\le\epsilon$.

(b) Whenever $J\subseteq I$ and $g$ is a real-valued function defined on
a subset of $X_J$, then $\int g\,d\lambda_J=\int g\pi_Jd\lambda$ if
either integral is defined in $[-\infty,\infty]$.

(c) Whenever $f$ is a $\lambda$-integrable real-valued function, we can
find a countable set $J\subseteq X$ and a
$\lambda_J$-integrable function $g$ such that $f$ extends $g\pi_J$.

\proof{{\bf (a)(i)} Write $\overline{\eusm S}$ for the set of functions
$f$ satisfying the assertion, that is, such that for every $\epsilon>0$
there is a $g\in\eusm S$ such that $\int|f-g|\le\epsilon$.   Then
$f_1+f_2$ and $cf_1\in\overline{\eusm S}$ whenever $f_1$,
$f_2\in\overline{\eusm S}$.   \Prf\ Given $\epsilon>0$ there are $g_1$,
$g_2\in\eusm S$ such that $\int|f_1-g_1|\le\bover{\epsilon}{2+|c|}$,
$\int|f_2-g_2|\le\bover{\epsilon}2$;  now $g_1+g_2$, $cg_1\in\eusm S$
and $\int|(f_1+f_2)-(g_1+g_2)|\le\epsilon$,
$\int|cf_1-cg_1|\le\epsilon$.\ \QeD\  Also, of course,
$f\in\overline{\eusm S}$ whenever $f_0\in\overline{\eusm S}$ and
$f\eae f_0$.


\medskip

\quad{\bf (ii)} Write $\Cal W$ for
$\{W:W\subseteq X$, $\chi W\in\overline{\eusm S}\}$, and $\Cal C$ for
the
family of measurable
cylinders in $X$.   Then it is plain from the definition in 254A that
$C\cap C'\in\Cal C$ for all $C$, $C'\in\Cal C$, and of course
$C\in\Cal W$ for every $C\in\Cal C$, because $\chi C\in\eusm S$.   Next,
$W\setminus V\in\Cal W$ whenever $W$, $V\in\Cal W$ and $V\subseteq W$,
because then $\chi(W\setminus V)=\chi W-\chi V$.   Thirdly,
$\bigcup_{n\in\Bbb N}W_n\in\Cal W$ for every non-decreasing sequence
$\sequencen{W_n}$ in $\Cal W$.   \Prf\ Set $W=\bigcup_{n\in\Bbb N}W_n$.
Given $\epsilon>0$, there is an $n\in\Bbb N$ such that
$\lambda(W\setminus W_n)\le\bover{\epsilon}2$.   Now there is a
$g\in\eusm S$ such that $\int|\chi W_n-g|\le\bover{\epsilon}2$, so that
$\int|\chi W-g|\le\epsilon$.\ \QeD\  Thus $\Cal W$ is a Dynkin class of
subsets of $X$.

By the Monotone Class Theorem (136B), $\Cal W$ must include the
$\sigma$-algebra of subsets of $X$ generated by $\Cal C$, which is
$\Tensorhat_{i\in I}\Sigma_i$.   But this means that $\Cal W$ contains
every measurable subset of $X$, since by 254Ff any measurable set
differs by a negligible set from some member of $\Tensorhat_{i\in
I}\Sigma_i$.

\medskip

\quad{\bf (iii)} Thus $\overline{\eusm S}$ contains the indicator
function of any measurable subset of $X$.   Because it is closed under
addition and scalar multiplication, it contains all simple functions.
But this means that it must contain all integrable functions.   \Prf\ If
$f$ is a real-valued function which is integrable over $X$, and
$\epsilon>0$, there is a simple function $h:X\to\Bbb R$ such that
$\int|f-h|\le\bover{\epsilon}2$ (242M), and now there is a $g\in\eusm S$
such that $\int|h-g|\le\bover{\epsilon}2$, so that
$\int|f-g|\le\epsilon$.\ \Qed

This proves part (a) of the proposition.

\medskip

{\bf (b)} Put 254Oa and 235J together.

\medskip

{\bf (c)} By 254Pb, there are a countable $J\subseteq I$ and a
real-valued function $g$ defined on a conegligible subset of $X_J$ such
that $f$ extends $g\pi_J$.   Now $\dom(g\pi_J)=\pi_J^{-1}[\dom g]$ is
conegligible, so $f\eae g\pi_J$ and $g\pi_J$ is $\lambda$-integrable.
By (b), $g$ is $\lambda_J$-integrable.
}%end of proof of 254Q

\leader{254R}{Conditional
expectations \dvrocolon{again}}\cmmnt{ Putting the ideas of 253H
together with the work above, we obtain some results which are important
not only for their direct applications but for the light they throw on
the structures here.

\medskip

\noindent}{\bf Theorem} Let
$\langle(X_i,\Sigma_i,\mu_i)\rangle_{i\in I}$ be a family of probability
spaces with product
$(X,\Lambda,\lambda)$.   For $J\subseteq I$ let
$\Lambda_J\subseteq\Lambda$ be the $\sigma$-subalgebra of sets
determined by coordinates in $J$\cmmnt{ (254Ma)}.   Then we may regard
$L^0(\lambda\restr\Lambda_J)$ as a subspace of
$L^0(\lambda)$\cmmnt{ (242Jh)}.    Let
$P_J:L^1(\lambda)\to L^1(\lambda\restr\Lambda_J)\subseteq L^1(\lambda)$
be the corresponding
conditional expectation operator\cmmnt{ (242Jd)}.   Then

(a) for any $J$, $K\subseteq I$, $P_{K\cap J}=P_KP_J$;

(b) for any $u\in L^1(\lambda)$, there is a countable set
$J^*\subseteq I$ such that $P_Ju=u$ iff $J\supseteq J^*$;

(c) for any $u\in L^0(\lambda)$, there is a unique smallest set
$J^*\subseteq I$ such that $u\in L^0(\lambda\restr\Lambda_{J^*})$, and
this $J^*$ is countable;

(d) for any $W\in\Lambda$ there is a unique smallest set
$J^*\subseteq I$
such that $W\symmdiff W'$ is negligible for some $W'\in\Lambda_{J^*}$,
and this $J^*$ is countable;

(e) for any $\Lambda$-measurable real-valued function $f:X\to\Bbb R$
there is a unique smallest set $J^*\subseteq I$ such that $f$ is equal
almost everywhere to a $\Lambda_J^*$-measurable function, and this $J^*$
is countable.

\proof{ For $J\subseteq I$, write $X_J=\prod_{i\in J}X_i$, let
$\lambda_J$ be the product measure on $X_J$, and set
$\pi_J(x)=x\restr J$ for $x\in X$.   Write $L^0_J$ for
$L^0(\lambda\restr\Lambda_J)$, regarded as a subset of $L^0=L^0_I$, and
$L^1_J$ for $L^1(\lambda\restr\Lambda_J)=L^1(\lambda)\cap L^0_J$, as in
242Jb;  thus $L^1_J$ is the set of values of the projection $P_J$.

\medskip

{\bf (a)(i)} Let $C\subseteq X$ be a measurable cylinder, expressed as
$\prod_{i\in I}C_i$ where $C_i\in\Sigma_i$ for every $i$ and
$L=\{i:C_i\ne X_i\}$ is finite.   Set

\Centerline{$C'_i=C_i$ for $i\in J$, $X_i$ for $i\in I\setminus J$,
\quad$C'=\prod_{i\in I}C'_i$,
\quad$\alpha=\prod_{i\in I\setminus J}\mu_iC_i$.}

\noindent Then $\alpha\chi C'$ is a conditional
expectation of $\chi C$ on $\Lambda_J$.   \Prf\ By
254N, we can identify $\lambda$ with the product of $\lambda_J$ and
$\lambda_{I\setminus J}$.   This identifies $\Lambda_J$ with
$\{E\times X_{I\setminus J}:E\in\dom\lambda_J\}$.   By 253H we have a
conditional expectation $g$ of $\chi C$ defined by setting

\Centerline{$g(y,z)=\int\chi C(y,t)\lambda_{I\setminus J}(dt)$}

\noindent for $y\in X_J$, $z\in X_{I\setminus J}$.   But $C$ is
identified with $C_J\times C_{I\setminus J}$, where
$C_J=\prod_{i\in J}C_i$, so that $g(y,z)=0$ if $y\notin C_J$ and
otherwise is
$\lambda_{I\setminus J}C_{I\setminus J}=\alpha$.   Thus
$g=\alpha\chi(C_J\times X_{I\setminus J})$.   But the identification
between $X_I\times X_{I\setminus J}$ and $X$ matches
$C_J\times X_{I\setminus J}$ with $C'$, as described above.   So $g$
becomes
identified with $\alpha\chi C'$ and $\alpha\chi C'$ is a conditional
expectation of $\chi C$.\ \Qed

\medskip

\quad{\bf (ii)} Next, setting

\Centerline{$C''_i=C'_i$ for $i\in K$, $X_i$ for $i\in I\setminus K$,
\quad$C''=\prod_{i\in I}C''_i$,}

\Centerline{$\beta=\prod_{i\in I\setminus K}\mu_iC'_i
=\prod_{i\in I\setminus(J\cup K)}\mu_iC_i$,}

\noindent the same arguments show that $\beta\chi C''$ is a conditional
expectation of $\chi C'$ on $\Lambda_K$.   So we have

\Centerline{$P_KP_J(\chi C)^{\ssbullet}
=\beta\alpha(\chi C'')^{\ssbullet}$.}

\noindent But if we look at $\beta\alpha$, this is just
$\prod_{i\in I\setminus(K\cap J)}\mu_iC_i$, while $C''_i=C_i$ if
$i\in K\cap J$,
$X_i$ for other $i$.   So $\beta\alpha\chi C''$ is a conditional
expectation of $\chi C$ on $\Lambda_{K\cap J}$, and

\Centerline{$P_KP_J(\chi C)^{\ssbullet}
=P_{K\cap J}(\chi C)^{\ssbullet}$.}

\medskip

\quad{\bf (iii)} Thus we see that the operators $P_KP_J$, $P_{K\cap J}$
agree on elements of the form $\chi C^{\ssbullet}$ where $C$ is a
measurable cylinder.   Because they are both linear, they agree on
linear combinations of these, that is, $P_KP_Jv=P_{K\cap J}v$ whenever
$v=g^{\ssbullet}$ for some $g$ in the space $\eusm S$ of 254Q.   But if
$u\in L^1(\lambda)$ and $\epsilon>0$, there is a $\lambda$-integrable
function  $f$ such that $f^{\ssbullet}=u$ and there is a $g\in\eusm S$
such that $\int|f-g|\le\epsilon$ (254Qa), so that
$\|u-v\|_1\le\epsilon$, where $v=g^{\ssbullet}$.   Since $P_J$, $P_K$
and $P_{K\cap J}$ are all linear operators of norm $1$,

\Centerline{$\|P_KP_Ju-P_{K\cap J}u\|_1
\le2\|u-v\|_1+\|P_KP_Jv-P_{K\cap J}v\|_1\le 2\epsilon$.}

\noindent As $\epsilon$ is arbitrary, $P_KP_Ju=P_{K\cap J}u$;  as $u$
is arbitrary, $P_KP_J=P_{K\cap J}$.

\medskip

{\bf (b)} Take $u\in L^1(\lambda)$.   Let $\Cal J$ be the family of all
subsets $J$ of $I$ such that $P_Ju=u$.   By (a), $J\cap K\in\Cal J$ for
all $J$, $K\in\Cal J$.   Next, $\Cal J$ contains a countable set $J_0$.
\Prf\ Let $f$ be a $\lambda$-integrable function such that
$f^{\ssbullet}=u$.   By 254Qc, we can find a countable set
$J_0\subseteq I$ and a $\lambda_{J_0}$-integrable function $g$ such that
$f\eae g\pi_{J_0}$.   Now $g\pi_{J_0}$ is $\Lambda_{J_0}$-measurable
and $u=(g\pi_{J_0})^{\ssbullet}$ belongs to $L^1_{J_0}$, so
$J_0\in\Cal J$.\ \Qed

Write $J^*=\bigcap\Cal J$, so that $J^*\subseteq J_0$ is countable.
Then $J^*\in\Cal J$.   \Prf\ Let $\epsilon>0$.   As in the proof of (a)
above, there is a $g\in\eusm S$ such that $\|u-v\|_1\le\epsilon$, where
$v=g^{\ssbullet}$.   But because $g$ is a finite linear combination of
indicator functions of measurable cylinders, each determined by
coordinates in some
finite set, there is a finite $K\subseteq I$ such that
$g$ is $\Lambda_K$-measurable, so that $P_Kv=v$.
Because $K$ is finite, there must be $J_1,\ldots,J_n\in\Cal J$ such that
$J^*\cap K=\bigcap_{1\le i\le n}J_i\cap K$;  but as $\Cal J$ is closed
under finite intersections, $J=J_1\cap\ldots\cap J_n\in\Cal J$, and
$J^*\cap K=J\cap K$.

Now we have

\Centerline{$P_{J^*}v=P_{J^*}P_Kv=P_{J^*\cap K}v=P_{J\cap K}v=P_JP_Kv
=P_Jv$,}

\noindent using (a) twice.   Because both $P_J$ and $P_{J^*}$ have norm
$1$,

$$\eqalign{\|P_{J^*}u-u\|_1
&\le\|P_{J^*}u-P_{J^*}v\|_1
  +\|P_{J^*}v-P_Jv\|_1
  +\|P_Jv-P_Ju\|_1
  +\|P_Ju-u\|_1\cr
&\le\|u-v\|_1+0+\|u-v\|_1+0
\le 2\epsilon.\cr}$$

\noindent As $\epsilon$ is arbitrary, $P_{J^*}u=u$ and
$J^*\in\Cal J$.\ \Qed

Now, for any $J\subseteq I$,

$$\eqalign{P_Ju=u
&\Longrightarrow J\in\Cal J
\Longrightarrow J\supseteq J^*\cr
&\Longrightarrow P_Ju=P_JP_{J^*}u=P_{J\cap J^*}u=P_{J^*}u=u.\cr}$$

\noindent Thus $J^*$ has the required properties.

\medskip

{\bf (c)} Set $e=(\chi X)^{\ssbullet}$, $u_n=(-ne)\vee(u\wedge ne)$ for
each $n\in\Bbb N$.   Then, for any $J\subseteq I$, $u\in L^0_J$ iff
$u_n\in L^0_J$ for every $n$.   \Prf\ ($\alpha$) If $u\in L^0_J$, then
$u$ is expressible as $f^{\ssbullet}$ for some $\Lambda_J$-measurable
$f$;  now $f_n=(-n\chi X)\vee(f\wedge n\chi X)$ is
$\Lambda_J$-measurable, so $u_n=f_n^{\ssbullet}\in L^0_J$ for every
$n$.   ($\beta$) If $u_n\in L^0_J$ for each $n$, then for each $n$ we
can find a $\Lambda_J$-measurable function $f_n$ such that
$f_n^{\ssbullet}=u_n$.   But there is also a $\Lambda$-measurable
function $f$ such that $u=f^{\ssbullet}$, and we must have
$f_n\eae(-n\chi X)\vee(f\wedge n\chi X)$ for each $n$, so that
$f\eae\lim_{n\to\infty}f_n$ and
$u=(\lim_{n\to\infty}f_n)^{\ssbullet}$.   Since $\lim_{n\to\infty}f_n$
is $\Lambda_J$-measurable and defined on a
$\mu\restr\Lambda_J$-conegligible set, $u\in L^0_J$.\ \Qed

As every $u_n$ belongs to $L^1$, we know that

\Centerline{$u_n\in L^0_J\iff u_n\in L^1_J\iff P_Ju_n=u_n$.}

\noindent By (b), there is for each $n$ a countable $J_n^*$ such that
$P_{J}u_n=u_n$ iff $J\supseteq J_n^*$.   So we see that $u\in L^0_J$ iff
$J\supseteq J_n^*$ for every $n$, that is,
$J\supseteq\bigcup_{n\in\Bbb N}J_n^*$.   Thus
$J^*=\bigcup_{n\in\Bbb N}J_n^*$ has the property claimed.

\medskip

{\bf (d)} Applying (c) to $u=(\chi W)^{\ssbullet}$, we have a
(countable) unique smallest $J^*$ such that $u\in L^0_{J^*}$.   But if
$J\subseteq I$, then there is a $W'\in\Lambda_J$ such that
$W'\symmdiff W$ is negligible iff $u\in L^0_J$.   So this is the $J^*$
we are looking for.

\medskip

{\bf (e)} Again apply (c), this time to $f^{\ssbullet}$.
}%end of proof of 254R

\leader{254S}{Proposition} Let
$\langle(X_i,\Sigma_i,\mu_i)\rangle_{i\in I}$ be a family of probability
spaces, with product $(X,\Lambda,\lambda)$.

(a) If $A\subseteq X$ is
determined by coordinates in $I\setminus\{j\}$ for every $j\in I$,
then its outer measure $\lambda^*A$ must be either $0$ or $1$.

(b) If $W\in\Lambda$ and $\lambda W>0$, then for every $\epsilon>0$
there are a $W'\in\Lambda$ and a finite set $J\subseteq I$ such that
$\lambda W'\ge 1-\epsilon$ and for every $x\in W'$ there is a $y\in W$
such that $x\restr I\setminus J=y\restr I\setminus J$.

\proof{ For $J\subseteq I$ write $X_J$ for $\prod_{i\in J}X_i$ and
$\lambda_J$ for the product measure on $X_J$.

\medskip

{\bf (a)} Let $W$ be a measurable envelope of $A$.   By 254Rd, there is
a smallest $J\subseteq I$ for which
there is a $W'\in\Lambda$, determined by coordinates in $J$, with
$\lambda(W\symmdiff W')=0$.   Now $J=\emptyset$.   \Prf\ Take any
$j\in I$.   Then $A$ is determined by coordinates in $I\setminus\{j\}$,
that is, can be regarded as $X_j\times A'$ for some
$A'\subseteq X_{I\setminus\{j\}}$.   We can also think of
$\lambda$ as the product of $\lambda_{\{j\}}$ and
$\lambda_{I\setminus\{j\}}$ (254N).   Let $\Lambda_{I\setminus\{j\}}$ be
the domain of $\lambda_{I\setminus\{j\}}$.   By 251S,

\Centerline{$\lambda^*A
=\lambda_{\{j\}}^*X_j\cdot\lambda_{I\setminus\{j\}}^*A'
=\lambda_{I\setminus\{j\}}^*A'$.}

\noindent Let $V\in\Lambda_{I\setminus\{j\}}$ be measurable envelope of
$A'$. Then $W'=X_j\times V$ belongs to
$\Lambda$, includes $A$ and has measure $\lambda^*A$, so
$\lambda(W\cap W')=\lambda W=\lambda W'$ and $W\symmdiff W'$ is
negligible.   At the same time, $W'$ is determined by coordinates in
$I\setminus\{j\}$.   This means that $J$ must be included in
$I\setminus\{j\}$.   As $j$ is arbitrary, $J=\emptyset$.\ \Qed

But the only subsets of $X$ which are determined by coordinates in
$\emptyset$ are $X$ and $\emptyset$.   Since $W$ differs from one of
these by a negligible set, $\lambda^*A=\lambda W\in\{0,1\}$, as claimed.

\medskip

{\bf (b)} Set $\eta=\bover12\min(\epsilon,1)\lambda W$.   By 254Fe,
there is a measurable set $V$, determined by coordinates in a finite
subset $J$ of $I$, such that $\lambda(W\symmdiff V)\le\eta$.   Note that

\Centerline{$\lambda V\ge\lambda W-\eta\ge\Bover12\lambda W>0$,}

\noindent so

\Centerline{$\lambda(W\symmdiff V)\le\Bover12\epsilon\lambda W
\le\epsilon\lambda V$.}

\noindent We may identify
$\lambda$ with the c.l.d.\ product of $\lambda_J$ and
$\lambda_{I\setminus J}$ (254N).   Let $\tilde W$,
$\tilde V\subseteq X_I\times X_{I\setminus J}$ be the sets corresponding
to $W$, $V\subseteq X$.   Then $\tilde V$ can be expressed as
$U\times X_{I\setminus J}$ where $\lambda_JU=\lambda V>0$.  Set
$U'=\{z:z\in X_{I\setminus J}$, $\lambda_J\tilde W^{-1}[\{z\}]=0\}$.
Then $U'$ is measured by $\lambda_{I\setminus J}$ (252D(ii) again,
because both $\lambda_J$ and $\lambda_{I\setminus J}$ are complete), and

$$\eqalignno{\lambda_JU\cdot\lambda_{I\setminus J}U'
&\le\int\lambda_J(\tilde W^{-1}[\{z\}]\symmdiff U)
  \lambda_{I\setminus J}(dz)\cr
\displaycause{because if $z\in U'$ then
$\lambda_J(\tilde W^{-1}[\{z\}]\symmdiff U)=\lambda_JU$}
&=\int\lambda_J(\tilde W\symmdiff\tilde V)^{-1}[\{z\}]
  \lambda_{I\setminus J}(dz)\cr
&=(\lambda_J\times\lambda_{I\setminus J})(\tilde W\symmdiff\tilde V)\cr
\displaycause{252D once more}
&=\lambda(W\symmdiff V)
\le\epsilon\lambda V
=\epsilon\lambda_JU.\cr}$$

\noindent This means that $\lambda_{I\setminus J}U'\le\epsilon$.   Set
$W'=\{x:x\in X$, $x\restr I\setminus J\notin U'\}$;  then
$\lambda W'\ge 1-\epsilon$.   If $x\in W'$, then
$z=x\restr I\setminus J\notin U'$, so $\tilde W^{-1}[\{z\}]$ is not
empty, that is, there is a $y\in W$ such that $y\restr I\setminus J=z$.
So this $W'$ has the required properties.
}%end of proof of 254S

\leader{254T}{Remarks} \cmmnt{It is important to understand that the
results above apply to $L^0$ and $L^1$ and
measurable-sets-up-to-a-negligible-set, not to sets and functions
themselves.   One idea does apply to sets and functions, whether
measurable or not.

\header{254Ta}}{\bf (a)} Let $\langle X_i\rangle_{i\in I}$ be a family
of sets with Cartesian product $X$.   For each $J\subseteq I$ let
$\Cal W_J$
be the set of subsets of $X$ determined by coordinates in $J$.   Then
$\Cal W_J\cap\Cal W_K=\Cal W_{J\cap K}$ for all $J$, $K\subseteq I$.
\prooflet{\Prf\ Of course
$\Cal W_J\cap\Cal W_K\supseteq\Cal W_{J\cap K}$, because
$\Cal W_J\supseteq\Cal W_{J'}$ whenever $J'\subseteq J$.   On the
other hand, suppose $W\in\Cal W_J\cap\Cal W_K$, $x\in W$, $y\in X$
and $x\restr J\cap K=y\restr J\cap K$.   Set $z(i)=x(i)$ for $i\in J$,
$y(i)$ for $i\in I\setminus J$.   Then $z\restr J=x\restr J$ so $z\in
W$.   Also $y\restr K=z\restr K$ so $y\in W$.   As $x$, $y$ are
arbitrary, $W\in\Cal W_{J\cap K}$;  as $W$ is arbitrary,
$\Cal W_J\cap\Cal W_K\subseteq\Cal W_{J\cap K}$.\ \QeD}
Accordingly, for any $W\subseteq X$, $\Cal F=\{J:W\in\Cal W_J\}$ is a
filter on $I$ (unless $W=X$ or $W=\emptyset$, in which case
$\Cal F=\Cal PX$).   \cmmnt{But $\Cal F$ does not necessarily have a
least element, as the following example shows.}

\spheader 254Tb Set $X=\{0,1\}^{\Bbb N}$,

\Centerline{$W=\{x:x\in X$, $\lim_{i\to\infty}x(i)=0\}$.}

\noindent Then for every $n\in\Bbb N\,\,W$ is determined by coordinates
in
$J_n=\{i:i\ge n\}$.   But $W$ is not determined by coordinates in
$\bigcap_{n\in\Bbb N}J_n=\emptyset$.   \cmmnt{Note that

\Centerline{$W=\bigcup_{n\in\Bbb N}\bigcap_{i\ge n}\{x:x(i)=0\}$}

\noindent is measured by the usual measure on $X$.   But it is also
negligible (since it is countable);  in 254Rd we have $J^*=\emptyset$,
$W'=\emptyset$.}


\leader{*254U}{}\cmmnt{ I am now in a position to describe a
counter-example answering a natural question arising out of \S251.

\medskip

\noindent}{\bf Example} There are a localizable measure space
$(X,\Sigma,\mu)$ and a
probability space $(Y,\Tau,\nu)$ such that the c.l.d.\ product measure
$\lambda$ on $X\times Y$ is not localizable.

\proof{{\bf (a)} Take $(X,\Sigma,\mu)$ to be the space of 216E, so that
$X=\{0,1\}^I$, where $I=\Cal PC$ for some set $C$ with cardinal greater
than $\frak c$.   For each $\gamma\in C$ write $E_{\gamma}$ for
$\{x:x\in X$, $x(\{\gamma\})=1\}$ (that is, $G_{\{\gamma\}}$ in the
notation of 216Ec);  then $E_{\gamma}\in\Sigma$ and $\mu E_{\gamma}=1$;
also every measurable set of non-zero measure meets some $E_{\gamma}$ in
a set of non-zero measure, while $E_{\gamma}\cap E_{\delta}$ is
negligible for all distinct $\gamma$, $\delta$ (see 216Ee).

Let $(Y,\Tau,\nu)$ be $\{0,1\}^C$ with the usual measure (254J).   For
$\gamma\in C$, let $F_{\gamma}$ be $\{y:y\in Y$, $y(\gamma)=1\}$, so
that $\nu F_{\gamma}=\bover12$.
Let $\lambda$ be the c.l.d.\ product measure on $X\times Y$, and
$\Lambda$ its domain.

\medskip

{\bf (b)} Consider the family $\Cal W=\{E_{\gamma}\times
F_{\gamma}:\gamma\in C\}\subseteq\Lambda$.   \Quer\ Suppose, if
possible, that $V$ were an essential supremum of $\Cal W$ in $\Lambda$
in the sense of 211G.   For $\gamma\in C$ write
$H_{\gamma}=\{x:V[\{x\}]\symmdiff F_{\gamma}$ is negligible$\}$.
Because $F_{\gamma}\symmdiff F_{\delta}$ is non-negligible,
$H_{\gamma}\cap H_{\delta}=\emptyset$ for all $\gamma\ne\delta$.

Now $E_{\gamma}\setminus H_{\gamma}$ is $\mu$-negligible for every
$\gamma\in C$.   \Prf\
$\lambda((E_{\gamma}\times F_{\gamma})\setminus V)=0$, so
$F_{\gamma}\setminus V[\{x\}]$ is negligible for almost every
$x\in E_{\gamma}$, by 252D.   On the other hand, if we set
$F'_{\gamma}=Y\setminus F_{\gamma}$,
$W_{\gamma}=(X\times Y)\setminus(E_{\gamma}\times F'_{\gamma})$, then we
see that

\Centerline{$(E_{\gamma}\times F'_{\gamma})\cap(E_{\gamma}\times
F_{\gamma})=\emptyset$,
\quad $E_{\gamma}\times F_{\gamma}\subseteq W_{\gamma}$,}

\Centerline{$\lambda((E_{\delta}\times F_{\delta})\setminus W_{\gamma})
=\lambda((E_{\gamma}\times F'_{\gamma})
  \cap(E_{\delta}\times F_{\delta}))
\le\mu(E_{\gamma}\cap E_{\delta})
=0$}

\noindent for every $\delta\ne\gamma$, so $W_{\gamma}$ is an essential
upper bound for $\Cal W$ and
$V\cap(E_{\gamma}\times F'_{\gamma})=V\setminus W_{\gamma}$ must be
$\lambda$-negligible.    Accordingly
$V[\{x\}]\setminus F_{\gamma}=V[\{x\}]\cap F'_{\gamma}$
is $\nu$-negligible for $\mu$-almost every $x\in E_{\gamma}$.   But this
means that $V[\{x\}]\symmdiff F_{\gamma}$ is $\nu$-negligible for
$\mu$-almost every $x\in E_{\gamma}$, that is,
$\nu(E_{\gamma}\setminus H_{\gamma})=0$.\ \Qed

Now consider the family
$\langle E_{\gamma}\cap H_{\gamma}\rangle_{\gamma\in C}$.   This is a
disjoint family of sets of
finite measure in $X$.   If $E\in\Sigma$ has non-zero measure, there is
a $\gamma\in C$ such that
$\mu(E_{\gamma}\cap H_{\gamma}\cap E)=\nu(E_{\gamma}\cap E)>0$.   But
this means that $\Cal E=\{E_{\gamma}\cap H_{\gamma}:\gamma\in C\}$
satisfies the conditions of
213Oa, and $\mu$ must be strictly localizable;  which it isn't.\ \Bang

\medskip

{\bf (c)} Thus we have found a family $\Cal W\subseteq\Lambda$ with no
essential supremum in $\Lambda$, and $\lambda$ is not localizable.
}%end of proof of 254U

\cmmnt{\medskip

\noindent{\bf Remark} If $(X,\Sigma,\mu)$ and $(Y,\Tau,\nu)$ are any
localizable measure spaces with a non-localizable c.l.d.\ product
measure, then their c.l.d.\ versions are still localizable (213Hb) and
still have a
non-localizable product (251T), which cannot be strictly localizable;
so that at least one of the factors is not strictly localizable (251O).
Thus any
example of the type here must involve a complete locally determined
localizable space which is not strictly localizable, as in 216E.
}%end of comment

\leader{*254V}{}\cmmnt{ Corresponding to 251U and 251Wo, we have the
following result on countable powers of atomless probability spaces.

\medskip

\noindent}{\bf Proposition} Let $(X,\Sigma,\mu)$ be an atomless
probability space and $I$ a countable set.   Let $\lambda$ be the
product probability measure on $X^I$.   Then
$\{x:x\in X^I$, $x$ is injective$\}$ is $\lambda$-conegligible.
%used in 495

\proof{ For any pair $\{i,j\}$ of distinct elements of $X$, the set
$\{z:z\in X^{\{i,j\}}$, $z(i)=z(j)\}$ is negligible for the product
measure on $X^{\{i,j\}}$, by 251U.   By 254Oa,
$\{x:x\in X$, $x(i)=x(j)\}$ is $\lambda$-negligible.   Because $I$ is
countable, there are only countably many such pairs $\{i,j\}$, so
$\{x:x\in X$, $x(i)=x(j)$ for some distinct $i$, $j\in I\}$ is
negligible, and its complement is conegligible;  but this complement is
just the set of injective functions from $I$ to $X$.
}%end of proof of 254V

\exercises{
\leader{254X}{Basic exercises (a)}
%\spheader 254Xa
Let $\langle(X_i,\Sigma_i,\mu_i)\rangle_{i\in I}$ be any family of
probability spaces, with product $(X,\Lambda,\mu)$.    Write $\Cal E$
for the family of subsets of $X$ expressible as the union of a finite
disjoint family of measurable cylinders.   (i) Show that if
$C\subseteq X$
is a measurable cylinder then $X\setminus C\in\Cal E$.   (ii) Show that
$W\cap V\in\Cal E$ for all $W$, $V\in\Cal E$.   (iii) Show that
$X\setminus W\in\Cal E$ for every $W\in\Cal E$.   (iv) Show that
$\Cal E$ is an algebra of subsets of $X$.   (v) Show that for any
$W\in\Lambda$, $\epsilon>0$ there is a $V\in\Cal E$ such that
$\lambda(W\symmdiff V)\le\epsilon^2$.   (vi) Show that for any
$W\in\Lambda$ and $\epsilon>0$ there are disjoint measurable cylinders
$C_0,\ldots,C_n$ such that
$\lambda(W\cap C_j)\ge(1-\epsilon)\lambda C_j$ for every $j$ and
$\lambda(W\setminus\bigcup_{j\le n}C_j)\le\epsilon$.
\Hint{select the $C_j$ from the
measurable cylinders composing a set $V$ as in (v).}   (vii) Show that
if $f$, $g$ are $\lambda$-integrable functions and $\int_Cf\le\int_Cg$
for every measurable cylinder $C\subseteq X$, then $f\leae g$.
\Hint{show that $\int_Wf\le\int_Wf$ for every $W\in\Lambda$.}
%254F

\sqheader 254Xb Let $\langle (X_i,\Sigma_i,\mu_i)$ be
a family of probability spaces, with product $(X,\Lambda,\lambda)$.
Show that the outer measure $\lambda^*$ defined by $\lambda$ is exactly
the outer measure $\theta$ described in 254A, that is, that $\theta$ is
a regular outer measure.
%254F

\spheader 254Xc Let $\langle (X_i,\Sigma_i,\mu_i)$ be
a family of probability
spaces, with product $(X,\Lambda,\lambda)$.   Write $\lambda_0$ for the
restriction of $\lambda$ to $\Tensorhat_{i\in I}\Sigma_i$, and $\Cal C$
for the family of measurable cylinders in $X$.   Suppose that
$(Y,\Tau,\nu)$ is a probability space and $\phi:Y\to X$ a function.
(i) Show that $\phi$ is \imp\ when regarded as a function from
$(Y,\Tau,\nu)$ to $(X,\Tensorhat_{i\in I}\Sigma_i,\lambda_0)$ iff
$\phi^{-1}[C]$ belongs to $\Tau$ and $\nu\phi^{-1}[C]=\lambda_0C$ for
every $C\in\Cal C$.   (ii) Show that $\lambda_0$ is the only measure on
$X$ with this property.   \Hint{136C.}
%254G

\sqheader 254Xd Let $I$ be a set and $(Y,\Tau,\nu)$ a complete
probability space.   Show that a function $\phi:Y\to\{0,1\}^I$ is \imp\
for $\nu$ and the usual measure on $\{0,1\}^I$ iff
$\nu\{y:\phi(y)(i)=1$ for every $i\in J\}=2^{-\#(J)}$ for every finite
$J\subseteq I$.
%254J

\sqheader 254Xe Let $I$ be any set and $\lambda$ the usual measure on
$X=\{0,1\}^I$.   Define addition on $X$ as in 254Jd.   Show that
the map $(x,y)\mapsto x+y:X\times X\to X$ is \imp, if $X\times X$ is
given its product measure.
%254J

\sqheader 254Xf Let $I$ be any set and $\lambda$ the usual measure on
$\Cal PI$.   (i) Show that the map
$a\mapsto a\symmdiff b:\Cal PI\to\Cal PI$ is \imp\ for any
$b\subseteq I$;  in particular, $a\mapsto I\setminus a$ is \imp.  (ii)
Show that the map
$(a,b)\mapsto a\symmdiff b:\Cal PI\times\Cal PI\to\Cal PI$ is \imp.
%254J

\sqheader 254Xg Show that for any $q\in[0,1]$ and any set $I$ there is a
measure $\lambda$ on $\Cal PI$ such that
$\lambda\{a:J\subseteq a\}=q^{\#(J)}$ for every finite $J\subseteq I$.
%254J

\sqheader 254Xh Let $(Y,\Tau,\nu)$ be a complete probability
space, and write $\mu$ for Lebesgue measure on $[0,1]$.   Suppose that
$\phi:Y\to[0,1]$ is a function such that $\nu\phi^{-1}[I]$ exists and is
equal to $\mu I$ for every interval $I$ of the form
$[2^{-n}k,2^{-n}(k+1)]$, where $n\in\Bbb N$ and $0\le k<2^n$.   Show
that $\phi$ is \imp\ for $\nu$ and $\mu$.
%254K

\spheader 254Xi Show that if $\tilde\phi:\{0,1\}^{\Bbb N}\to[0,1]$ is
any bijection constructed by the method of 254K, then
$\{\tilde\phi^{-1}[E]:E\subseteq[0,1]$ is a Borel set$\}$ is just the
$\sigma$-algebra of subsets of $\{0,1\}^{\Bbb N}$ generated by the sets
$\{x:x(i)=1\}$ for $i\in\Bbb N$.
%254K

\spheader 254Xj Let $\langle X_i\rangle_{i\in I}$ be a family of
sets, and for each $i\in I$ let $\Sigma_i$ be a $\sigma$-algebra of
subsets of $X_i$.   Show that for every
$E\in\Tensorhat_{i\in I}\Sigma_i$ there is a countable set
$J\subseteq I$ such that $E$ is
expressible as $\pi_J^{-1}[F]$ for some $F\in\Tensorhat_{i\in J}X_j$,
writing $\pi_J(x)=x\restr J\in\prod_{i\in J}X_i$ for
$x\in\prod_{i\in I}X_i$.
%254M

\spheader 254Xk(i) Let $\nu$ be the usual measure on
$X=\{0,1\}^{\Bbb N}$.   Show that for any $k\ge 1$, $(X,\nu)$ is
isomorphic to $(X^k,\nu_k)$, where $\nu_k$ is the measure on $X^k$ which
is the product measure obtained by giving each factor $X$ the measure
$\nu$.    (ii) Writing $\mu_{[0,1]}$ for Lebesgue measure on
$[0,1]$, etc., show that for any $k\ge 1$, $([0,1]^k,\mu_{[0,1]^k})$ is
isomorphic to $([0,1],\mu_{[0,1]})$.
%254K, 254N

\spheader 254Xl(i) Writing $\mu_{[0,1]}$ for Lebesgue measure on
$[0,1]$, etc., show that $([0,1],\mu_{[0,1]})$ is isomorphic to
$(\coint{0,1},\mu_{\coint{0,1}})$.    (ii) Show that for any $k\ge
1$, $(\coint{0,1}^k,\mu_{\coint{0,1}^k})$ is isomorphic to
$(\coint{0,1},\mu_{\coint{0,1}})$.   (iii) Show that for any $k\ge
1$, $(\Bbb R,\mu_{\Bbb R})$ is isomorphic to $(\BbbR^k,\mu_{\Bbb
R^k})$.
%254K, 254N

\spheader 254Xm Let $\mu$ be Lebesgue measure on $[0,1]$ and
$\lambda$ the product measure on $[0,1]^{\Bbb N}$.   Show that
$([0,1],\mu)$ and $([0,1]^{\Bbb N},\lambda)$ are isomorphic.
%254K, 254N

\spheader 254Xn Let $\langle(X_i,\Sigma_i,\mu_i)\rangle_{i\in I}$
be a family of complete probability spaces and $\lambda$ the product
measure on $\prod_{i\in I}X_i$, with domain $\Lambda$.   Suppose that
$A_i\subseteq X_i$ for
each $i\in I$.   Show that $\prod_{i\in I}A_i\in\Lambda$ iff either (i)
$\prod_{i\in I}\mu_i^*A_i=0$ or (ii) $A_i\in\Sigma_i$ for
every $i$ and $\{i:A_i\ne X_i\}$ is countable.   \Hint{assemble
ideas from 252Xc, 254F, 254L and 254N.}
%254F, 254L, 254N

\spheader 254Xo Let $\familyiI{(X_i,\Sigma_i,\mu_i)}$ be a family of
probability spaces with product $(X,\Lambda,\lambda)$.   (i) Show that,
for any $A\subseteq X$,

\Centerline{$\lambda^*A=\min\{\lambda_J^*\pi_J[A]:J\subseteq I$ is
countable$\}$,}

\noindent where for $J\subseteq I$ I write $\lambda_J$ for the product
probability measure on $X_J=\prod_{i\in J}X_i$ and $\pi_J:X\to X_J$ for
the canonical map.   (ii) Show that if $J$, $K\subseteq I$ are disjoint
and $A$, $B\subseteq X$ are determined by coordinates in $J$, $K$
respectively, then $\lambda^*(A\cap B)=\lambda^*A\cdot\lambda^*B$.
%254O

\spheader 254Xp Let $\familyiI{(X_i,\Sigma_i,\mu_i)}$ be a family of
probability spaces with product $(X,\Lambda,\lambda)$.   Let $\eusm S$
be the linear span of the set of indicator functions of measurable
cylinders in $X$, as in 254Q.   Show that
$\{f^{\ssbullet}:f\in\eusm S\}$ is dense in $L^p(\mu)$ for every
$p\in\coint{1,\infty}$.
%254Q

\spheader 254Xq Let $\langle(X_i,\Sigma_i,\mu_i)\rangle_{i\in I}$ be a
family of probability spaces, and $(X,\Lambda,\lambda)$ their product;
for $J\subseteq I$ let $\Lambda_J$ be the $\sigma$-algebra of members of
$\Lambda$ determined by coordinates in $J$ and
$P_J:L^1=L^1(\lambda)\to L^1_J=L^1(\lambda\restr\Lambda_J)$ the
corresponding conditional expectation.   (i) Show that if $u\in L^1_J$
and $v\in L^1_{I\setminus J}$ then $u\times v\in L^1$ and
$\int u\times v=\int u\cdot\int v$.
\Hint{253D.}   (ii) Show that if $\Cal J\subseteq\Cal PI$ is
non-empty, with $J^*=\bigcap\Cal J$, then
$L^1_{J^*}=\bigcap_{J\in\Cal J}L^1_J$.
%254R

\spheader 254Xr(i) Let $I$ be any set and $\lambda$ the usual measure on
$\Cal PI$.   Let $A\subseteq\Cal PI$ be such that $a\symmdiff b\in A$
whenever $a\in A$ and $b\subseteq I$ is finite.   Show that $\lambda^*A$
must be either $0$ or $1$.
(ii) Let $\lambda$ be the usual measure on $\{0,1\}^{\Bbb N}$, and
$\Lambda$ its domain.   Let $f:\{0,1\}^{\Bbb N}\to\Bbb R$ be a function
such that, for $x$, $y\in\{0,1\}^{\Bbb N}$,
$f(x)=f(y)\iff\{n:n\in\Bbb N$, $x(n)\ne y(n)\}$ is finite.
Show that $f$ is not $\Lambda$-measurable.
\Hint{for any $q\in\Bbb Q$, $\lambda^*\{x:f(x)\le q\}$ is either $0$ or
$1$.}
%254S

\spheader 254Xs Let $\familyiI{X_i}$ be any family of sets and
$A\subseteq B\subseteq\prod_{i\in I}X_i$.   Suppose that $A$ is
determined by coordinates in $J\subseteq I$ and that $B$ is determined
by coordinates in $K$.   Show that there is a set $C$ such that
$A\subseteq C\subseteq B$ and $C$ is determined by coordinates in
$J\cap K$.
%254T

\leader{254Y}{Further exercises (a)}
%\spheader 254Ya
Let $\familyiI{(X_i,\Sigma_i,\mu_i)}$ be a family of
probability spaces, and for $J\subseteq I$ let $\lambda_J$ be the
product
measure on $X_J=\prod_{i\in J}X_i$;  write $X=X_I$, $\lambda=\lambda_I$
and $\pi_J(x)=x\restr J$ for $x\in X$ and $J\subseteq I$.

\quad (i) Show that for $K\subseteq J\subseteq I$ we have a natural
linear, order-preserving and
norm-preserving map $T_{JK}:L^1(\lambda_K)\to L^1(\lambda_J)$ defined by
writing $T_{JK}(f^{\ssbullet})=(f\pi_{KJ})^{\ssbullet}$ for every
$\lambda_K$-integrable function $f$, where $\pi_{KJ}(y)=y\restr K$ for
$y\in X_J$.

\quad (ii) Write $\Cal K$ for the set of finite subsets of $I$.   Show
that if  $W$ is any Banach space and $\langle T_K\rangle_{K\in\Cal K}$
is a family such that ($\alpha$) $T_K$ is a bounded linear operator from
$L^1(\lambda_K)$ to $W$ for every $K\in\Cal K$ ($\beta$)
$T_K=T_JT_{JK}$ whenever $K\subseteq J\in \Cal K$ ($\gamma$)
$\sup_{K\in\Cal K}\|T_K\|<\infty$, then there is a unique bounded
linear operator $T:L^1(\lambda)\to W$ such that $T_K=TT_{IK}$ for every
$K\in\Cal K$.

\quad (iii) Write $\Cal J$ for the set of countable subsets of $I$.
Show that $L^1(\lambda)=\bigcup_{J\in\Cal J}T_{IJ}[L^1(\lambda_J)]$.
%254F

\spheader 254Yb\dvAnew{2010} Let $\familyiI{(X_i,\Sigma_i,\mu_i)}$
be any family of measure spaces.
Set $X=\prod_{i\in I}X_i$ and let $\Cal F$ be a filter on the
set $[I]^{<\omega}$ of finite subsets of $I$ such that
$\{J:i\in J\in[I]^{<\omega}\}\in\Cal F$ for every $i\in I$.   Show that
there is a complete locally determined
measure $\lambda$ on $X$ such that $\lambda(\prod_{i\in I}E_i)$
is defined and equal to $\lim_{J\to\Cal F}\prod_{i\in J}\mu_iE_i$ whenever
$E_i\in\Sigma_i$ for every $i\in I$ and
$\lim_{J\to\Cal F}\prod_{i\in J}\mu_iE_i$ is defined in
$\coint{0,\infty}$.
\Hint{{\smc Baker 04}.}  %n10313.tex
%254F

\spheader 254Yc
Let $\familyiI{(X_i,\Sigma_i,\mu_i)}$ be a family of probability
spaces, and $\lambda$ a complete measure on $X=\prod_{i\in I}X_i$.
Suppose that for every complete probability space $(Y,\Tau,\nu)$ and
function
$\phi:Y\to X$, $\phi$ is \imp\ for $\nu$ and $\lambda$ iff
$\nu\phi^{-1}[C]$ is defined and
equal to $\theta_0 C$ for every measurable cylinder $C\subseteq X$,
writing $\theta_0$ for the functional of 254A.    Show that
$\lambda$ is the product measure on $X$.
%254G

\spheader 254Yd Let $I$ be a set, and $\lambda$ the usual
measure on $\{0,1\}^I$.   Show that $L^1(\lambda)$ is separable, in its
norm topology, iff $I$ is countable.
%254J

\spheader 254Ye\dvAnew{2013} Let $f:[0,1]\to[0,1]^2$ be a function
which is \imp\ for Lebesgue planar measure on $[0,1]^2$ and
Lebesgue linear measure on $[0,1]$, as in 134Yl;  let $f_1$, $f_2$ be the
coordinates of $f$.   Define $g:[0,1]\to[0,1]^{\Bbb N}$ by setting
$g(t)=\sequencen{f_1f_2^n(t)}$ for $0\le t\le 1$.   Show that $g$ is
\imp.   \Hint{show that $g_n:[0,1]\to[0,1]^{n+1}$ is \imp\ for every
$n\ge 1$, where
$g_n(t)=(f_1(t),f_1f_2(t),\ldots,\penalty-100f_1f_2^{n-1}(t),f_2^n(t))$
for $t\in[0,1]$.}
%254N 254G out of order query

\spheader 254Yf Let $I$ be a set, and $\lambda$ the usual measure on
$\Cal PI$.   Show that if $\Cal F$ is a non-principal ultrafilter on $I$
then $\lambda^*\Cal F=1$.   \Hint{254Xr, 254Xf.}
%254J, 254S

\spheader 254Yg Let $(X,\Sigma,\mu)$, $(Y,\Tau,\nu)$ and $\lambda$ be as
in 254U.   Set $A=\{x_{\gamma}:\gamma\in C\}$ as defined in 216E.   Let
$\mu_A$ be the subspace measure on $A$, and $\tilde\lambda$ the c.l.d.\
product measure of $\mu_A$ and $\nu$ on $A\times Y$.   Show that
$\tilde\lambda$ is a proper extension of the subspace measure
$\lambda_{A\times Y}$.   \Hint{consider
$\tilde W=\{(x_{\gamma},y):\gamma\in C$, $y\in F_{\gamma}\}$, in the
notation of 254U.}
%254U

\spheader 254Yh Let $(X,\Sigma,\mu)$ be an atomless probability space,
$I$ a set with cardinal at most $\#(X)$, and $A$ the set of injective
functions from $I$ to $X$.   Show that $A$ has full outer measure for
the product measure on $X^I$.
%254V

}%end of exercises

\endnotes{
\Notesheader{254} While there are many reasons for studying infinite
products of probability spaces, one stands pre-eminent, from the point
of view of abstract measure theory:  they provide constructions of
essentially new kinds of measure space.   I cannot describe the nature
of this `newness' effectively without venturing into the territory of
Volume 3.   But the function spaces of Chapter 24 do give at least a
form of words we can use:  these are the first {\it probability} spaces
$(X,\Lambda,\lambda)$ we have seen for which $L^1(\lambda)$ need not be
separable for its norm topology (254Yd).

The formulae of 254A, like those of 251A, lead
very naturally to measures;  the point at
which they become more than a curiosity is when we find that the product
measure $\lambda$ is a
probability measure (254Fa), which must be regarded as the crucial
argument of this section, just as 251E is the essential basis of
\S251.   It is I think remarkable that it makes no difference to the
result here whether $I$ is finite, countably infinite or uncountable.
If you write out the proof for the case $I=\Bbb N$, it will seem natural
to expand the sets $J_n$ until they are initial segments of $I$ itself,
thereby avoiding altogether the auxiliary set $K$;  but this is a
misleading simplification, because it hides an essential feature of the
argument, which is that any sequence in $\Cal C$ involves only countably
many coordinates, so that as long as we are dealing with only one such
sequence the uncountability of the whole set $I$ is irrelevant.
This general principle naturally permeates the whole of the section;  in
254O I have tried to spell out the way in which many of the questions we
are interested in can be expressed in terms of countable subproducts of
the factor spaces $X_i$.   See also the exercises 254Xj, 254Xn and
254Ya(iii).

There is a slightly paradoxical side to this principle:  even the
best-behaved subsets $E_i$ of $X_i$ may fail to have measurable products
$\prod_{i\in I}E_i$ if $E_i\ne X_i$ for uncountably many $i$.   For
instance, $\ooint{0,1}^I$ is not a measurable subset of $[0,1]^I$ if $I$
is uncountable (254Xn).   It has full outer measure and its
own product measure is just the subspace measure (254L), but any
measurable subset must have measure zero.   The point is that the empty
set is the only member of $\Tensorhat_{i\in I}\Sigma_i$, where
$\Sigma_i$ is the algebra of Lebesgue measurable subsets of $[0,1]$ for
each $i$, which is included in $\ooint{0,1}^I$ (see 254Xj).

As in \S251, I use a construction which automatically produces a
complete measure on the product space.   I am sure that this is the best
choice for `the' product measure.   But there are occasions when its
restriction to the $\sigma$-algebra generated by the measurable
cylinders is worth looking at;  see 254Xc.

Lemma 254G is a result of a type which will be commoner in Volume 3
than in the present volume.   It describes the product measure in terms
not of what it {\it is} but of what it {\it does};
specifically, in terms of a property of the associated family of
\imp\ functions.  It
is therefore a `universal mapping theorem'.   (Compare 253F.)
Because this description is sufficient to determine the product measure
completely (254Yc), it is not surprising that I use it repeatedly.

The `usual measure' on $\{0,1\}^I$ (254J) is sometimes called
`coin-tossing measure' because it can be used to model the concept of
tossing a coin arbitrarily many times indexed by the set $I$, taking an
$x\in\{0,1\}^I$ to represent the outcome in which the coin is `heads'
for just those $i\in I$ for which $x(i)=1$.   The sets, or `events',
in the class $\Cal C$ are those which can be specified by declaring
the outcomes of
finitely many tosses, and the probability of any particular sequence of
$n$ results is $1/2^n$, regardless of which tosses we look at or in
which order.   In Chapter 27 I will return to the use of product
measures to represent probabilities involving independent events.

In 254K I come to the first case in this treatise of a non-trivial
isomorphism between two measure spaces.   If you have been brought up on
a conventional diet of modern abstract pure mathematics based on algebra
and topology, you may already have been struck by the absence of
emphasis on any concept of `homomorphism' or `isomorphism'.   Here
indeed I start to speak of `isomorphisms' between measure spaces
without even troubling to define them;  I hope it really is obvious that
an isomorphism between measure spaces $(X,\Sigma,\mu)$ and
$(Y,\Tau,\nu)$ is a bijection $\phi:X\to Y$ such that
$\Tau=\{F:F\subseteq Y$, $\phi^{-1}[F]\in\Sigma\}$ and
$\nu F=\mu\phi^{-1}[F]$ for every $F\in\Tau$, so that $\Sigma$ is
necessarily $\{E:E\subseteq X$, $\phi[E]\in\Tau\}$ and
$\mu E=\nu\phi[E]$
for every $E\in\Sigma$.   Put like this, you may, if you worked through
the exercises of Volume 1, be reminded of some constructions of
$\sigma$-algebras in 111Xc-111Xd and of the `image measures' in
234C-234D.   The result in 254K (see also 134Yo) naturally leads to two
distinct notions of
`homomorphism' between two measure spaces $(X,\Sigma,\mu)$ and
$(Y,\Tau,\nu)$:

\inset{(i) a function $\phi:X\to Y$ such that $\phi^{-1}[F]\in\Sigma$
and $\mu\phi^{-1}[F]=\nu F$ for every $F\in\Tau$,}

\inset{(ii) a function $\phi:X\to Y$ such that $\phi[E]\in \Tau$ and
$\nu\phi[E]=\mu E$ for every $E\in\Sigma$.}

\noindent On either definition, we find that a bijection $\phi:X\to Y$
is an isomorphism iff $\phi$ and $\phi^{-1}$ are both homomorphisms.
(Also, of course, the composition of homomorphisms will be a
homomorphism.)   My own view is that (i) is the more important, and
in this treatise I study such functions at length, calling
them `\imp'.   But both have their uses.   The
function $\phi$ of 254K not only satisfies both definitions, but is also
`nearly' an isomorphism in several different ways, of which possibly
the most important is that there are conegligible sets
$X'\subseteq\{0,1\}^{\Bbb N}$,
$Y'\subseteq [0,1]$ such that $\phi\restr X'$ is an
isomorphism between $X'$ and $Y'$ when both are
given their subspace measures.

Having once established the isomorphism between $[0,1]$ and
$\{0,1\}^{\Bbb N}$, we are led immediately to many more;  see
254Xk-254Xm.   In fact Lebesgue measure on $[0,1]$ is isomorphic to a
large proportion of the probability spaces arising in applications.   In
Volumes 3 and 4 I will discuss these isomorphisms at length.

The general notion of `subproduct' is associated with some of the
deepest and most characteristic results in the theory of product
measures.   Because we are looking at products of arbitrary families of
probability spaces, the definition must ignore any possible structure in
the index set $I$ of 254A-254C.   But many applications, naturally
enough, deal with index sets with favoured subsets or partitions, and
the first essential step is the `associative law' (254N;  compare
251Xe-251Xf and 251Wh).   This is, for instance, the tool by which
we can apply Fubini's theorem within infinite products.   The natural
projection maps from $\prod_{i\in I}X_i$ to $\prod_{i\in J}X_i$, where
$J\subseteq I$, are related in a way which has already been used as the
basis of theorems in \S235;  the product measure on $\prod_{i\in J}X_i$
is precisely the image of the product measure on $\prod_{i\in I}X_i$
(254Oa).   In 254O-254Q I explore the consequences of this fact and the
fact already noted that all measurable sets in the product are
`essentially' determined by coordinates in some countable set.

In 254R I go more deeply into this notion of a set
$W\subseteq\prod_{i\in I}X_i$ `determined by coordinates in' a set
$J\subseteq I$.   In its primitive form this is a purely set-theoretic
notion (254M, 254Ta).   I think that even a three-element set $I$ can
give us surprises;  I invite you to try to visualize subsets of
$[0,1]^3$ which are determined by pairs of coordinates.   But the
interactions of this with measure-theoretic ideas, and in particular
with a willingness to add or discard negligible sets, lead to much
more, and in particular to the unique minimal sets of coordinates
associated with measurable sets and functions (254R).   Of course
these results can be elegantly and effectively described in terms of
$L^1$ and $L^0$ spaces, in which negligible sets are swept out of sight
as the spaces are constructed.   The basis of all this is the fact that
the conditional expectation operators associated with subproducts
multiply together in the simplest possible way (254Ra);  but some
further idea is needed to show that if $\Cal J$ is a non-empty family of
subsets of $I$, then $L^0_{\bigcap\Cal J}=\bigcap_{J\in\Cal J}L^0_J$
(see part (b) of the proof of 254R, and 254Xq(iii)).

254Sa is a version of the `zero-one law' (272O below).   254Sb is a
strong version of the principle that measurable sets in a
product must be approximable by sets determined by a {\it finite} set of
coordinates (254Fe, 254Qa, 254Xa).   Evidently it is not a coincidence
that the set $W$ of 254Tb is negligible.   In \S272 I will revisit many
of the ideas of 254R-254S and 254Xq, in particular, in the more
general context of `independent $\sigma$-algebras'.

Finally, 254U and 254Yg hardly belong to this section at all;  they are
unfinished business from \S251.   They are here because the construction
of 254A-254C is the simplest way to produce an adequately complex
probability space $(Y,\Tau,\nu)$.
}%end of notes

\discrpage
