\frfilename{mt565.tex}
\versiondate{25.4.14}
\copyrightdate{2006}

\def\chaptername{Choice and determinacy}
\def\sectionname{Lebesgue measure without choice}

\newsection{565}

I come now to the construction of specific
non-trivial Borel-coded measures.
Primary among them is of course Lebesgue measure on $\BbbR^r$;  we also
have Hausdorff measures (565N-565O).   For Lebesgue
measure I begin, as in \S115, with
half-open intervals.   The corresponding `outer measure' may no longer be
countably subadditive, so I call it `Lebesgue submeasure'.
Carath\'eodory's method no longer seems quite appropriate, as it smudges
the distinction between `negligible' and `outer measure zero', so I use
563H to show that there is a Borel-coded measure agreeing with Lebesgue
submeasure on open sets (565C-565D);  it is the completion of this
Borel-coded measure which I will call Lebesgue measure.
We have a version of Vitali's theorem for well-orderable families
(in particular, for countable families)  of balls (565F).   From this we
can prove the Fundamental Theorem of Calculus in essentially its standard
form (565M).

\leader{565A}{Definitions} Throughout this section, except when otherwise
stated, $r\ge 1$ will be a
fixed integer.   \cmmnt{As in \S115, }I will say that a 
{\bf half-open interval} in $\BbbR^r$ is a set of the form

\Centerline{$\coint{a,b}=\{x:x\in\BbbR^r$, $a(i)\le x(i)<b(i)$ for
$i<r\}$}

\noindent where $a$, $b\in\BbbR^r$.
For a half-open interval $I$, set $\lambda I=0$ if $I=\emptyset$
and otherwise $\lambda I=\prod_{i=0}^{r-1}b(i)-a(i)$ where $I=\coint{a,b}$.
Now for $A\subseteq\BbbR^r$ set

\Centerline{$\theta A=\inf\{\sum_{j=0}^{\infty}\lambda I_j:
\sequence{j}{I_j}
\text{ is a sequence of half-open intervals covering }A\}$.}

\leader{565B}{Proposition} In the notation of 565A,

(a) the function
$\theta:\Cal P\BbbR^r\to[0,\infty]$ is a submeasure,

(b) $\theta I=\lambda I$ for every half-open interval $I\subseteq\BbbR^r$.

\proof{{\bf (a)} As in parts (a-i) to (a-iii) of the proof of 115D,
$\theta\emptyset=0$ and $\theta A\le\theta B$ whenever $A\subseteq B$.
If $A$, $B\subseteq\BbbR^r$ and $\epsilon>0$, we have sequences
$\sequencen{I_n}$ and $\sequencen{J_n}$ of half-open intervals such that

\Centerline{$A\subseteq\bigcup_{n\in\Bbb N}I_n$,
\quad$B\subseteq\bigcup_{n\in\Bbb N}J_n$,}

\Centerline{$\sum_{n=0}^{\infty}\lambda I_n\le\theta A+\epsilon$,
\quad$\sum_{n=0}^{\infty}\lambda J_n\le\theta B+\epsilon$.}

\noindent Set $K_{2n}=I_n$, $K_{2n+1}=J_n$ for $n\in\Bbb N$;  then
$A\cup B\subseteq\bigcup_{n\in\Bbb N}K_n$ so

\Centerline{$\theta(A\cup B)
\le\sum_{n=0}^{\infty}\lambda K_n
=\sum_{n=0}^{\infty}\lambda I_n+\sum_{n=0}^{\infty}\lambda J_n
\le\theta A+\theta B+2\epsilon$.}

\noindent As $\epsilon$ is arbitrary,
$\theta(A\cup B)\le\theta A+\theta B$.

\medskip

{\bf (b)} The arguments of 114B/115B/115Db nowhere called on 
any form of the axiom of choice, so can be used unchanged.
}%end of proof of 565B

\medskip

\noindent{\bf Definition} I will call the submeasure $\theta$ {\bf Lebesgue
submeasure} on $\BbbR^r$.

\leader{565C}{Lemma} Let $\Cal I$ be the family of half-open intervals in
$\BbbR^r$;  let $\theta$ be Lebesgue submeasure, and set

\Centerline{$\Sigma
=\{E:E\subseteq X$, $\theta A=\theta(A\cap E)+\theta(A\setminus E)$
for every $A\subseteq X\}$,
\quad$\nu=\theta\restr\Sigma$\dvro{.}{}}

\cmmnt{\noindent (563G).}

(a) Let $\sequencen{I_n}$ be a disjoint sequence in $\Cal I$.   Then
$E=\bigcup_{n\in\Bbb N}I_n$ belongs to $\Sigma$
and $\nu E=\sum_{n=0}^{\infty}\nu I_n$.

(b) Every open set in $\BbbR^r$ belongs to $\Sigma$.

(c) If $G$, $H\subseteq\BbbR^r$ are open, then
$\nu G+\nu H=\nu(G\cap H)+\nu(G\cup H)$.

(d) If $\sequencen{G_n}$ is a non-decreasing
sequence of open sets then
$\nu(\bigcup_{n\in\Bbb N}G_n)=\lim_{n\to\infty}G_n$.

\proof{{\bf (a)(i)} If $i<r$ and $\alpha\in\Bbb R$ then
$\{x:x\in\BbbR^r$, $x(i)<\alpha\}\in\Sigma$, as in 115F.   So every
half-open interval belongs to $\Sigma$.   By 565Bb,
$\nu I=\theta I=\lambda I$ for every $I\in\Cal I$.

\medskip

\quad{\bf (ii)}\grheada\ $\theta E=\sum_{n=0}^{\infty}\nu I_n$.
\Prf\ Because
$\sequencen{I_n}$ is a sequence in $\Cal I$ covering $E$,
$\theta E$ is at most
$\sum_{n=0}^{\infty}\lambda I_n\le\sum_{n=0}^{\infty}\nu I_n$.
In the other direction,

\Centerline{$\theta E
\ge\sup_{n\in\Bbb N}\theta(\bigcup_{i\le n}I_i)
=\sup_{n\in\Bbb N}\nu(\bigcup_{i\le n}I_i)
=\sup_{n\in\Bbb N}\sum_{i\le n}\nu I_i
=\sum_{n=0}^{\infty}\nu I_n$.  \Qed}

\medskip

\qquad\grheadb\ $E\in\Sigma$.   \Prf\ Let
$A\subseteq\BbbR^r$ be such that $\theta A$ is finite, and $\epsilon>0$.
We have a sequence $\sequence{m}{J_m}$ in $\Cal I$ such
that $A\subseteq\bigcup_{m\in\Bbb N}J_m$ and
$\sum_{m=0}^{\infty}\lambda J_m\le\theta A+\epsilon$ is finite.
Let $m$ be such that $\sum_{j=m+1}^{\infty}\lambda J_j\le\epsilon$,
and set $K=\bigcup_{j\le m}J_j$;   then

\Centerline{$A\cap E\subseteq(K\cap E)\cup\bigcup_{j>m}J_j$,}

\noindent so

\Centerline{$\theta(A\cap E)
\le\theta(K\cap E)+\theta(\bigcup_{j>m}J_j)
\le\theta(K\cap E)+\sum_{j=m+1}^{\infty}\lambda J_j
\le\theta(K\cap E)+\epsilon$.}

\noindent Similarly,
$\theta(A\setminus E)\le\theta(K\setminus E)+\epsilon$.   Next, by
($\alpha$) applied to $\sequence{i}{J_j\cap I_i}$ or
otherwise, $\sum_{i=0}^{\infty}\nu(J_j\cap I_i)$ is finite for every $j$,
so there is an $n\in\Bbb N$ such that
$\sum_{j=0}^m\sum_{i=n+1}^{\infty}\nu(J_j\cap I_i)\le\epsilon$.   Set
$L=\bigcup_{i\le n}I_i$;  then

\Centerline{$K\cap E
\subseteq(K\cap L)\cup\bigcup_{j\le m,i>n}J_j\cap I_i$,}

$$\eqalign{\theta(K\cap E)
&\le\theta(K\cap L)+\theta(\bigcup_{j\le m,i>n}J_j\cap I_i)\cr
&\le\theta(K\cap L)+\sum_{j=0}^m\sum_{i=n+1}^{\infty}\nu(J_j\cap I_i)
\le\theta(K\cap L)+\epsilon.\cr}$$

\noindent Assembling these,

$$\eqalignno{\theta A
&\le\theta(A\cap E)+\theta(A\setminus E)
\le\theta(K\cap E)+\theta(K\setminus E)+2\epsilon\cr
&\le\theta(K\cap L)+\theta(K\setminus L)+3\epsilon
=\theta K+3\epsilon\cr
\displaycause{because we know that $L\in\Sigma$}
&\le\theta A+3\epsilon.\cr}$$

\noindent As $\epsilon$ is arbitrary,
$\theta A=\theta(A\cap E)+\theta(A\setminus E)$.   This was on the
assumption that $\theta A$ was finite;  but of course it is also true if
$\theta A=\infty$.   As $A$ is arbitrary, $E\in\Sigma$.\ \Qed

\medskip

\qquad\grheadc\ Accordingly $\nu E=\theta E=\sum_{n=0}^{\infty}\nu I_n$.

\medskip

{\bf (b)} Let $\Cal I_0$ be the family of dyadic half-open intervals in
$\BbbR^r$ of the form $\coint{2^{-k}z,2^{-k}(z+\tbf{1})}$ where
$k\in\Bbb N$, $z\in\BbbZ^r$ and $\tbf{1}=(1,\ldots,1)$.   Note that
$\Cal I_0$ is countable and that if $I$, $J\in\Cal I_0$ then either
$I\subseteq J$ or $J\subseteq I$ or $I\cap J=\emptyset$.   Also any
non-empty subset of $\Cal I_0$ has a maximal element.

If $G\subseteq\BbbR^r$ is open, set
$\Cal J=\{I:I\in\Cal I_0$, $I\subseteq G\}$ and let $\Cal J'$ be the set of
maximal elements of $\Cal J$.   Then $\Cal J'$ is disjoint and countable,
so by (a-ii) $G=\bigcup\Cal J=\bigcup\Cal J'$ belongs to $\Sigma$.

\medskip

{\bf (c)} Because $\nu$ is additive on $\Sigma$,

\Centerline{$\nu(G\cup H)+\nu(G\cap H)
=\nu G+\nu(H\setminus G)+\nu(H\cap G)
=\nu G+\nu H$}

\noindent for all open sets $G$, $H\subseteq\BbbR^r$.

\medskip

{\bf (d)} This time, let $\Cal J$ be
$\bigcup_{n\in\Bbb N}\{I:I\in\Cal I_0$, $I\subseteq G_n\}$;  again, let
$\Cal J'$ be the set of maximal elements of $\Cal J$.   Then
$G=\bigcup\Cal J=\bigcup\Cal J'$, so

\Centerline{$\nu G=\sum_{J\in\Cal J'}\nu J
=\sup_{\Cal K\subseteq\Cal J'\text{ is finite}}\sum_{J\in\Cal K}\nu J
\le\sup_{n\in\Bbb N}\nu G_n=\lim_{n\to\infty}\nu G_n\le\nu G$}

\noindent because $\sequencen{G_n}$ is non-decreasing.
}%end of proof of 565C

\leader{565D}{Definition} Let $\theta$ and $\nu$ be as in 565C.
By 563H, there is a unique Borel-coded measure
$\mu$ on $\BbbR^r$ such that $\mu G=\nu G=\theta G$ for every open set
$G\subseteq\BbbR^r$.   I will say that {\bf Lebesgue measure} on $\BbbR^r$
is the completion $\mu_L$ of $\mu$;
the sets it measures will be {\bf Lebesgue measurable}.

\leader{565E}{Proposition} Let
$\Cal I$, $\theta$, $\Sigma$, $\nu$, $\mu$ and $\mu_L$ be as in
565A-565D.

(a) $\mu$ is the restriction of $\theta$ to the
algebra $\Cal B_c(\BbbR^r)$ of codable Borel sets.

(b) For every $A\subseteq\BbbR^r$,

\Centerline{$\theta A
=\inf\{\mu_L E:E\supseteq A$ is Lebesgue measurable$\}
=\inf\{\mu G:G\supseteq A$ is open$\}$.}

(c) $E\in\Sigma$ and $\mu_L E=\nu E=\theta E$
whenever $E$ is Lebesgue measurable.

(d) $\mu_L$ is inner regular with respect to the compact sets and outer
regular with respect to the open sets.

\proof{{\bf (a)} If $E\in\Cal B_c(\BbbR^r)$, then

$$\eqalignno{\mu E
&=\inf\{\mu G:G\supseteq E\text{ is open}\}\cr
\displaycause{563Fd}
&=\inf\{\theta G:G\supseteq E\text{ is open}\}
\ge\theta E.\cr}$$

\noindent Next, if $I\subseteq\BbbR^r$ is a half-open interval, it is a
codable Borel set and

$$\eqalign{\lambda I
&=\inf\{\lambda J:J\in\Cal I,\,I\subseteq\interior J\}
\ge\inf\{\theta(\interior J):J\in\Cal I,\,I\subseteq\interior J\}\cr
&\ge\inf\{\theta G:G\supseteq I\text{ is open}\}
=\mu I
\ge\theta I
=\lambda I.\cr}$$

\noindent So $\mu$ and $\lambda$ agree on $\Cal I$.   If now
$E$ is a codable Borel set and $\epsilon>0$, there is a sequence
$\sequencen{I_n}$ in $\Cal I$ such that $E\subseteq\bigcup_{n\in\Bbb N}I_n$
and $\sum_{n=0}^{\infty}\lambda I_n\le\theta E+\epsilon$.   But every $I_n$
is resolvable (because it belongs to the algebra of sets generated by the
open sets), so $\sequencen{I_n}$ is a codable sequence (562J) and

\Centerline{$\mu E
\le\mu(\bigcup_{n\in\Bbb N}I_n)
\le\sum_{n=0}^{\infty}\mu I_n
=\sum_{n=0}^{\infty}\lambda I_n
\le\theta E+\epsilon$.}

\noindent As $E$ and $\epsilon$ are arbitrary,
$\mu=\theta\restr\Cal B_c(\BbbR^r)$.

\medskip

{\bf (b)} Suppose that $A\subseteq\BbbR^r$.   If $E\supseteq A$ is
Lebesgue measurable, there are $F$, $H\in\Cal B_c(\BbbR^r)$ such that
$E\symmdiff F\subseteq H$ and $\mu H=0$, so that
$E\subseteq F\cup H$ and

\Centerline{$\theta A\le\theta(F\cup H)=\mu(F\cup H)
=\mu_L E$.}

\noindent So we have

\Centerline{$\theta A
\le\inf\{\mu_L E:E\supseteq A$ is Lebesgue measurable$\}
\le\inf\{\mu G:G\supseteq A$ is open$\}$.}

\noindent In the other direction, given $\epsilon>0$ there is a sequence
$\sequencen{I_n}$ in $\Cal I$, covering $A$, such that
$\sum_{n=0}^{\infty}\lambda I_n\le\theta A+\epsilon$.   As in (a) just
above, $E=\bigcup_{n\in\Bbb N}I_n$ is a codable Borel set and
$\mu E\le\sum_{n=0}^{\infty}\lambda I_n$;  now there is an open
$G\supseteq E$ such that $\mu G\le\mu E+\epsilon\le\theta A+2\epsilon$.
As $\epsilon$ is arbitrary,

\Centerline{$\inf\{\mu G:G\supseteq A$ is open$\}\le\theta A$}

\noindent and we have the equalities.

\medskip

{\bf (c)} Suppose that $E$ is Lebesgue measurable, $A\subseteq\BbbR^r$ and
$\epsilon>0$.   By (b), there is an open set $G\supseteq A$ such that
$\mu G\le\theta A+\epsilon$.   Now

$$\eqalignno{\theta(A\cap E)+\theta(A\setminus E)
&\le\theta(G\cap E)+\theta(G\setminus E)
\le\mu_L(G\cap E)+\mu_L(G\setminus E)\cr
\displaycause{by (b)}
&=\mu_L G
=\mu G
\le\theta A+\epsilon.\cr}$$

\noindent As usual, this is enough to ensure that $E\in\Sigma$.   Now (b)
again tells us that $\mu_L E=\theta E=\nu E$.

\medskip

{\bf (d)} Of course $\mu$ is locally finite, while $\BbbR^r$
is a regular topological space.   So 563F(d-ii) tells us that $\mu$ is
inner regular with respect to the closed sets and outer regular with
respect to the open sets;  it follows that $\mu_L$ also is.
Next, every closed set is K$_{\sigma}$, while compact sets are
resolvable and all sequences of compact sets are codable, so
$\mu F=\sup\{\mu K:K\subseteq F$ is compact$\}$ for every closed set
$F\subseteq\BbbR^r$;  consequently
 $\mu_L$ is inner regular with respect to the compact sets.
}%end of proof of 565E

\leader{565F}{Vitali's Theorem} Let $\Cal C$ be a well-orderable
family of non-singleton closed balls
in $\BbbR^r$.   For $\Cal I\subseteq\Cal C$ set

\Centerline{$A_{\Cal I}
=\bigcap_{\delta>0}\bigcup\{C:C\in\Cal I$, $\diam C\le\delta\}$.}

\noindent Let $\frak T$ be the family of open subsets of $\BbbR^r$.
Then there are functions
$\Psi:\Cal P\Cal C\to\Cal P\Cal C$ and
$\Theta:\Cal P\Cal C\times\Bbb N\to\frak T$ such that
$\Psi(\Cal I)\subseteq\Cal I$, $\Psi(\Cal I)$ is disjoint and countable,
$\mu_L(\Theta(\Cal I,k))\le 2^{-k}$ and
$A_{\Cal I}\subseteq\bigcup\Psi(\Cal I)\cup\Theta(\Cal I,k)$ whenever
$\Cal I\subseteq\Cal C$ and $k\in\Bbb N$.   In particular,

\Centerline{$A_{\Cal I}\setminus\bigcup\Psi(\Cal I)
\subseteq\bigcap_{k\in\Bbb N}\Theta(\Cal I,k)$}

\noindent is negligible.

\proof{ We use the greedy algorithm of 221A/261B, but watching more
carefully.   Start by fixing on a well-ordering $\preccurlyeq$ of
$\Cal C\cup\{\emptyset\}$.
Next, for each $n\in\Bbb N$, set $U_n=\{x:x\in\BbbR^r$, $n<\|x\|<n+1\}$,
where $\|\,\|$ is the Euclidean norm on $\BbbR^r$.
It will be convenient to fix at this point on a family
$\langle G_{kn}\rangle_{k,n\in\Bbb N}$ of open sets such that
$\mu_L G_{kn}\le 2^{-k-n-2}$ and $\{x:\|x\|=n\}\subseteq G_{kn}$ for all $k$
and $n$;  for instance, $G_{kn}$ could be an open shell with rational inner
and outer radii (except for $G_{k0}$, which should be an open
ball).\footnote{Of course we can still use the similarity argument
from part (g) of the proof of 261B to check that thin shells have small
measure.}

\woddheader{565F}{0}{0}{0}{60pt}

Now define $C_{\Cal Inm}$, for $\Cal I\subseteq\Cal C$ and $m$,
$n\in\Bbb N$, by saying that

\inset{\noindent given $\ofamily{i}{m}{C_{\Cal Ini}}$, $C_{\Cal Inm}$ is to
be the $\preccurlyeq$-first member of
$\{\emptyset\}\cup(\Cal I\cap\Cal PU_n)$ which is disjoint from
$\bigcup_{i<m}C_{\Cal Ini}$ and has diameter at least

\Centerline{$\Bover12\sup\{\diam C:
C\in\{\emptyset\}\cup(\Cal I\cap\Cal PU_n)$ is disjoint from
$\bigcup_{i<m}C_{\Cal Ini}\}$.}}

\noindent (I take the diameter of the empty set to be $0$, as usual.)
Set

\Centerline{$\Psi(\Cal I)
=\{C_{\Cal Inm}:n$, $m\in\Bbb N\}\setminus\{\emptyset\}$.}

Because the $U_n$ are disjoint, $\Psi(\Cal I)$ is a disjoint subfamily of
$\Cal I$, and of course it is countable.
Just as in 261B, we find that for each $\Cal I\subseteq\Cal C$ and
$n\in\Bbb N$ we have

\Centerline{$A_{\Cal I}\cap U_n
\subseteq\bigcup_{i<m}C_{\Cal Ini}\cup\bigcup_{i\ge m}C'_{\Cal Ini}$,}

\noindent where for $C\in\Cal C$ I write $C'$ for the {\it open} ball
with the same centre and {\it six} times the radius;
$\emptyset'$ will be $\emptyset$.   Just as in 261B,
$\sum_{m=0}^{\infty}\mu_L C'_{\Cal Inm}\le 6^r\mu_L B(\tbf{0},n+1)$
is finite.
So, for each $k$ and $n$, we can take the first $m_{kn}$ such that
$\sum_{i=m_{kn}}^{\infty}\mu_L C'_{\Cal Ini}\le 2^{-n-k-2}$.   Now set

\Centerline{$\Theta(\Cal I,k)
=\bigcup_{n\in\Bbb N}G_{kn}
  \cup\bigcup_{n\in\Bbb N,i\ge m_{kn}}C'_{\Cal Ini}$;}

\noindent we shall have
$A\setminus\bigcup\Psi(\Cal I)\subseteq\Theta(\Cal I,k)$ and

\Centerline{$\mu_L\Theta(\Cal I,k)\le\sum_{n=0}^{\infty}\mu_L G_{kn}
  +\sum_{n=0}^{\infty}\sum_{i=m_{kn}}^{\infty}
     \mu_L C'_{\Cal Ini}\le 2^{-k}$,}

\noindent as required.
}%end of proof of 565F

\leader{565G}{Proposition} Let $A\subseteq\BbbR^r$ be any set.   Then its
Lebesgue submeasure is

\Centerline{$\theta A
=\inf\{\sum_{n=0}^{\infty}\mu_L B_n:\sequencen{B_n}$ is a sequence of
closed balls covering $A\}$.}

\proof{ Let $\epsilon>0$.   Then there is a (non-empty) open set
$G\supseteq A$ with
$\mu_L G\le\theta A+\epsilon$.   Use Vitali's theorem, with $\Cal C$
the family of closed balls with rational centres and non-zero
rational radii, to see that there is a
disjoint sequence $\sequencen{C_n}$ of balls included in $G$ such that
$\mu_L(A\setminus\bigcup_{n\in\Bbb N}C_n)=0$ and
$\sum_{n\in\Bbb N}\mu_L C_n\le\theta A+\epsilon$.   Next, cover
$A\setminus\bigcup_{n\in\Bbb N}C_n$ by a sequence of half-open intervals
with measures summing to not more than $\epsilon$, and expand these to
balls with measures summing to not more than $\epsilon r^{r/2}$.   
Interleaving this sequence with the $C_n$, we get a
sequence $\sequencen{B_n}$ of balls, covering $A$, with
$\sum_{n=0}^{\infty}\mu_L B_n\le\theta A+(1+r^{r/2})\epsilon$.   So

\Centerline{$\theta A
\ge\inf\{\sum_{n=0}^{\infty}\mu_L B_n:\sequencen{B_n}$ is a sequence of
closed balls covering $A\}$.}

\noindent The reverse inequality is elementary (563C(a-ii)).
}%end of proof of 565G

\leader{565H}{Corollary} Lebesgue measure is invariant under isometries.

\proof{ We can see from its definition that Lebesgue submeasure is
translation-invariant, so Lebesgue measure also is.   Consequently two
balls with the same radii have the same measure.
Isometries of $\BbbR^r$ take closed balls to closed balls with the
same radii, so 565G gives the result.
}%end of proof of 565H

\leader{565I}{Lemma} (a) Writing $C_k(\BbbR^r)$ for the space of
continuous real-valued functions on $\BbbR^r$ with compact support,
$C_k(\BbbR^r)\subseteq\eusm L^1(\mu_L)$.

(b) There is a countable set $D\subseteq C_k(\BbbR^r)$ such that
$\{g^{\ssbullet}:g\in D\}$ is norm-dense in $L^1(\mu_L)$.

\proof{{\bf (a)} This is elementary;  every continuous function is
resolvable, therefore a codable Borel function and belongs to $\eusm L^0$;
if in addition it has compact support it is dominated by an integrable
function and is integrable, by 564E(c-i).

\medskip

{\bf (b)(i)} Let $\Cal U$ be a countable base for the topology of
$\BbbR^r$, consisting of bounded sets and closed under finite unions.
Let $D_0$ be the set of functions of the form
$x\mapsto\max(0,1-2^k\rho(x,\BbbR^r\setminus U))$ for $U\in\Cal U$ and
$k\in\Bbb N$, where $\rho$ is the Euclidean metric on $\BbbR^r$,
and $D$ the set of rational linear combinations of members of
$D_0$;  then $D$ is a countable subset of $C_k(\BbbR^r)$.

\medskip

\quad{\bf (ii)}
If $E\subseteq\BbbR^r$ is a codable Borel set of finite
measure, and $\epsilon>0$, then by 565Ed
there are a compact set $K\subseteq E$ and
an open set $G\supseteq E$ such that $\mu_L(G\setminus K)\le\epsilon$.
Now there are a $U\in\Cal U$ such that $K\subseteq U\subseteq G$ and
a $g\in D_0$ such that $\chi K\le g\le\chi U$, so that
$\int|g-\chi E|\le\epsilon$.

\medskip

\quad{\bf (iii)} It follows that whenever $f$ is a simple
codable Borel function, in the sense of 564Aa, and $\epsilon>0$ there is
a $g\in D$ such that $\int|f-g|\le\epsilon$.

\medskip

\quad{\bf (iv)} If $f\in\eusm L^1$ and $\epsilon>0$ there are a simple
codable Borel function $g$ and an $h\in D$ such that 
$\int|f-g|\le\bover12\epsilon$ such that $\int|g-h|\le\bover12\epsilon$, so that
$\int|f-h|\le\epsilon$.
}%end of proof of 565I

\leader{565J}{Lemma} Suppose that $f$ is an integrable function on
$\BbbR^r$, and that $\int_If\ge 0$ for every
half-open interval $I\subseteq\BbbR^r$.   Then $f(x)\ge 0$ for almost every
$x\in\BbbR^r$.

\proof{{\bf (a)} Note first that any finite union $E$
of half-open intervals is
expressible as a finite disjoint union of half-open intervals.
So $\int_Ef\ge 0$.

\medskip

{\bf (b)} Suppose that $g$ is a simple codable Borel function such that
$\int_Eg\le\epsilon$ whenever $E$ is a finite union of
half-open intervals.   Then $\int g^+\le\epsilon$.   \Prf\ Set
$F=\{x:g(x)>0\}$, and take any $\eta>0$.   Then there are a compact
$K\subseteq F$ and an open $G\supseteq F$ such that
$\mu_L(G\setminus K)\le\eta$.   There is a set $E$,
a finite union of half-open intervals, such that $K\subseteq E\subseteq G$.
In this case,

\Centerline{$\int g^+-\int_Eg\le\int|g\times\chi(E\symmdiff F)|
\le\|g\|_{\infty}\mu_L(G\setminus K)$,
\quad$\int g^+\le\epsilon+\eta\|g\|_{\infty}$;}

\noindent as $\eta$ is arbitrary, we have the result.\ \Qed

\medskip

{\bf (c)} We know that there is a codable
sequence $\sequencen{g_n}$ of simple codable
Borel functions such that $f\eae\lim_{n\to\infty}g_n$ and
\ifnum\stylenumber=12 the sum \else\fi
$\sum_{n=0}^{\infty}\int|g_{n+1}-g_n|$ is finite.   Set
$\epsilon_n=\sum_{i=n}^{\infty}\int|g_{i+1}-g_i|$ for each $n$;  then
$\int|f-g_n|\le\epsilon_n$, because $\int|g_m-g_n|\le\epsilon_n$ for every
$m\ge n$.   So if $E$ is a finite union of half-open intervals,

\Centerline{$\int_Eg_n
=\int g_n\times\chi E\ge\int f\times\chi E-\int|f-g_n|
\ge-\epsilon_n$;}

\noindent by (a), applied to $-g_n$,
$\int g_n^-\le\epsilon_n$.   By 564Be,

\Centerline{$f^-\eae\lim_{n\to\infty}g_n^-\eae
\liminf_{n\to\infty}g_n^-=0$}

\noindent almost everywhere, as required.
}%end of proof of 565J

\leader{565K}{Theorem} A monotonic function $f:\Bbb R\to\Bbb R$ is
differentiable almost everywhere.

\cmmnt{\medskip

\noindent{\bf Remark} Of course `almost everywhere'
here is with respect to Lebesgue
measure on $\Bbb R$;  in this result and the next two I am taking $r=1$.}

\proof{ We can use the ideas in 222A if we refine them using 565F.   First,
$\Cal C$ will be the set of closed non-trivial intervals with rational
endpoints;  take $\Psi$ and $\Theta$ as in 565F.   It will be enough to
deal with the case of non-decreasing $f$.   For $a<b$ in $\Bbb R$, set
$f^*([a,b])=[f(a),f(b)]$.   I shall repeatedly use the fact that if
$\Cal I\subseteq\Cal C$ is disjoint, then

\Centerline{$\mu_L(\bigcup_{C\in\Cal I}f^*(C))
=\sum_{C\in\Cal I}\mu_L f^*(C)$,}

\noindent because $\Cal I$ is countable and
$f^*(C)\cap f^*(C')$ contains at most one point for any distinct
$C$, $C'\in\Cal I$, and we can use 563C(a-iv).

\wheader{565K}{6}{2}{2}{72pt}
{\bf (a)} Again set

\Centerline{$\DiniD f(x)=\limsup_{h\to 0}{1\over h}(f(x+h)-f(x))$,
\quad$\Dinid f(x)=\liminf_{h\to 0}{1\over h}(f(x+h)-f(x))$}
     
\noindent for $x\in\Bbb R$.   To see that $\DiniD f<\infty$ a.e., set
$E_m=\{x:|x|<m$, $\DiniD f(x)>2^m(1+f(m)-f(-m))\}$ and

$$\eqalign{\Cal I_m
&=\{[\alpha,\beta]:\alpha,\,\beta\in\Bbb Q,\,-m<\alpha<\beta<m,\cr
&\mskip 200mu f(\beta)-f(\alpha)>2^m(1+f(m)-f(-m))(\beta-\alpha)\}\cr
&=\{C:C\in\Cal C,\,C\subseteq\ooint{-m,m},
\mu_L f^*(C)>2^m(1+f(m)-f(-m))\mu_L C\}\cr}$$

\noindent for each $m$.   Then, in the language of 565F,
$E_m\subseteq A_{\Cal I_m}$.   (If $x\in E_m$ and $\delta>0$, then
$x$ is an endpoint of a non-trivial
closed interval $[\alpha,\beta]\subseteq\ooint{-m,m}$, of length less than
$\delta$, such that
$f(\beta)-f(\alpha)>2^m(1+f(m)-f(-m))(\beta-\alpha)$.
Now we can expand $[\alpha,\beta]$
slightly to get an interval $[\alpha',\beta']\in\Cal I_m$ of length at most
$\delta$.)   So $E_m\subseteq\bigcup\Psi(\Cal I_m)\cup\Theta(\Cal I_m,k)$
for each $k$.   $\Psi(\Cal I_m)$ is a countable family of closed
sets, and

$$\eqalign{2^m(1+f(m)-f(-m))\sum_{C\in\Psi(\Cal I_m)}\mu_L C
&\le\sum_{C\in\Psi(\Cal I_m)}\mu_L f^*(C)\cr
&=\mu_L\bigl(\bigcup_{C\in\Psi(\Cal I_m)}f^*(C)\bigr)
\le f(m)-f(-m).\cr}$$

\noindent So $\sum_{C\in\Psi(\Cal I_m)}\mu_L C\le 2^{-m}$.
Setting 
$H_m=\Theta(\Cal I_m,m)\cup\bigcup\{\interior C:C\in\Psi(\Cal I_m)\}$,
$H_m$ is open, $\mu H_m\le 2^{-m+1}$ and $E_m\setminus\Bbb Q\subseteq H_m$.

Set $E=\{x:\DiniD f(x)=\infty\}$, and take any $n\in\Bbb N$.
Then

\Centerline{$E\setminus\Bbb Q
\subseteq\bigcup_{m\ge n}E_m\setminus\Bbb Q
\subseteq\bigcup_{m\ge n}H_m$.}

\noindent Now 563C(a-ii) tells us that

\Centerline{$\mu_L(\bigcup_{m\ge n}H_m)
\le\sum_{m=n}^{\infty}\mu H_m\le 2^{-n+1}$}

\noindent for each $n$, so that
$E\setminus\Bbb Q$ is included in a negligible G$_{\delta}$ set and
$\mu_L E=\mu_L(E\setminus\Bbb Q)=0$.   Thus $\DiniD f$ is finite a.e.

\medskip

{\bf (b)} To see that $\DiniD f\leae\Dinid f$, we use similar ideas, but with
an extra layer of complexity, corresponding to the double use of
Vitali's theorem.   Set $F=\{x:\Dinid f(x)<\DiniD f(x)\}$.   Take any
$\epsilon>0$;  because $\Bbb Q$ is countable, there is a family
$\langle\epsilon_{mqq'}\rangle_{m\in\Bbb N,q,q'\in\Bbb Q}$
of strictly positive numbers such that
$\sum_{m\in\Bbb N,q,q'\in\Bbb Q}\epsilon_{mqq'}\le\bover12\epsilon$.
For $q$, $q'\in\Bbb Q$ and $m$, $k\in\Bbb N$ let $\Cal I_{mqk}$,
$\Cal J_{mqk}$ be

\Centerline{$\{C:C\in\Cal C$, $C\subseteq\ooint{-m,m}$,
$\mu_L C\le 2^{-k}$, $\mu_L f^*(C)\ge q\mu_L C\}$,}

\Centerline{$\{C:C\in\Cal C$, $C\subseteq\ooint{-m,m}$,
$\mu_L C\le 2^{-k}$, $\mu_L f^*(C)\le q\mu_L C\}$}

\noindent respectively.  For $m$, $k\in\Bbb N$ and $q$, $q'\in\Bbb Q$ set

\Centerline{$G_{mqq'k}=\bigcup\{\interior C:C\in\Cal I_{mq'k}\}
   \cap\bigcup\{\interior C:C\in\Cal J_{mqk}$\};}

\noindent then $\sequence{k}{G_{mqq'k}}$ is a
non-increasing sequence of open sets of finite measure.
So, setting $F_{mqq'}=\bigcap_{k\in\Bbb N}G_{mqq'k}$, we can find a family
$\langle k(m,q,q')\rangle_{m\in\Bbb N,q,q'\in\Bbb Q}$ in $\Bbb N$ such that

\Centerline{$\mu_L(G_{m,q,q',k(m,q,q')}\setminus F_{mqq'})
\le\min(1,\Bover{q'-q}{q})\epsilon_{mqq'}$}

\noindent whenever $m\in\Bbb N$, $q$,
$q'\in\Bbb Q$ and $0<q<q'$ (563C(b-ii)).   Write $H_{mqq'}$ for
$G_{m,q,q',k(m,q,q')}$.

If $m\in\Bbb N$ and $0<q<q'$ in $\Bbb Q$, set

\Centerline{$\Cal J'_{mqq'}=\{C:C\in\Cal C$,
$C\subseteq H_{mqq'}$, $\mu_L f^*(C)\le q\mu_L C\}$.}

\noindent Then $F_{mqq'}\subseteq H_{mqq'}$,
so every point of $F_{mqq'}$ belongs to the interiors of arbitrarily small
intervals belonging to $\Cal J'_{mqq'}$;  accordingly
$F_{mqq'}\setminus\bigcup\Psi(\Cal J'_{mqq'})$ is negligible.

Now let $\Cal I'_{mqq'}$ be the set

\Centerline{$\{C:C\in\Cal C$, $C\subseteq C'$ for some
$C'\in\Psi(\Cal J'_{mqq'})$, $\mu_L f^*(C)\ge q'\mu_L C\}$.}

\noindent Then every point of
$F_{mqq'}\cap\bigcup\Psi(\Cal J'_{mqq'})\setminus\Bbb Q$ belongs to
arbitrarily small members of $\Cal I'_{mqq'}$, so
$F_{mqq'}\setminus\bigcup\Psi(\Cal I'_{mqq'})$ is negligible.

Now we come to the calculation at the heart of the proof.   If $m\in\Bbb N$
and $0<q<q'$ in $\Bbb Q$,

$$\eqalignno{q'\mu_L F_{mqq'}
&\le q'\mu_L(\bigcup\Psi(\Cal I'_{mqq'}))
=q'\sum_{C\in\Psi(\Cal I'_{mqq'})}\mu_L C\cr
&\le\sum_{C\in\Psi(\Cal I'_{mqq'})}\mu_L f^*(C)
=\mu_L\bigl(\bigcup_{C\in\Psi(\Cal I'_{mqq'})}f^*(C)\bigr)
\le\mu_L\bigl(\bigcup_{C\in\Psi(\Cal J'_{mqq'})}f^*(C)\bigr)\cr
\displaycause{because every member of $\Cal I'_{mqq'}$ is included in a
member of $\Psi(\Cal J'_{mqq'})$}
&=\sum_{C\in\Psi(\Cal J'_{mqq'})}\mu_L f^*(C)
\le q\sum_{C\in\Psi(\Cal J'_{mqq'})}\mu_L C\cr
&=q\mu_L(\bigcup\Psi(\Cal J'_{mqq'})
\le q\mu_L H_{mqq'}
\le q\mu_L F_{mqq'}+(q'-q)\epsilon_{mqq'},\cr}$$

\noindent and $\mu_L F_{mqq'}\le\epsilon_{mqq'}$,
$\mu_L H_{mqq'}\le 2\epsilon_{mqq'}$.   But this means that

\Centerline{$F\setminus\Bbb Q
\subseteq\bigcup_{m\in\Bbb N,q,q'\in\Bbb Q,0<q<q'}F_{mqq'}
\subseteq\bigcup_{m\in\Bbb N,q,q'\in\Bbb Q,0<q<q'}H_{mqq'}$,}

\noindent which has measure at most

\Centerline{$\sum_{m\in\Bbb N,q,q'\in\Bbb Q,0<q<q'}\mu_L H_{mqq'}
\le 2\sum_{m\in\Bbb N,q,q'\in\Bbb Q,0<q<q'}\epsilon_{mqq'}
\le\epsilon$.}

The process described here gives a recipe, starting from $\epsilon>0$, for
finding an open set of measure at most $\epsilon$ including
$F\setminus\Bbb Q$.   So we can repeat this for each term of a sequence
converging to $0$ to define a negligible G$_{\delta}$ set including
$F\setminus\Bbb Q$, and $F$ must be negligible, as required.
}%end of proof of 565K

\leader{565L}{Lemma} Suppose that $F:\Bbb R\to\Bbb R$ is a bounded
non-decreasing function.   Then $\int F'$ is defined and is at most
$\lim_{x\to\infty}F(x)-\lim_{x\to-\infty}F(x)$.

\wheader{565L}{0}{0}{0}{48pt}

\proof{ I copy the ideas of 222C.
For each $n\in\Bbb N$, define $g_n:\Bbb R\to\Bbb R$ by setting
$a_{nk}=2^{-n+1}k(n+1)-n$ for $k\le 2^n$,

$$\eqalign{g_n(x)
&=\Bover{2^{n-1}}{n+1}(F(a_{n,k+1})-F(a_{nk}))\text{ if }k<2^n
  \text{ and }a_{nk}\le x<a_{n,k+1},\cr
&=0\text{ if }x<-n\text{ or }x\ge n+2.\cr}$$

\noindent Then $g_n$ is a simple Borel function
and $F'(x)=\lim_{n\to\infty}g_n(x)$ whenever
$F'(x)$ is defined, which is almost everywhere, by 565K.
Also $\int g_n=F(n+2)-F(-n)$.   Because the $g_n$ are resolvable,
$\sequencen{g_n}$ is codable;  by Fatou's Lemma (564Fb),

\Centerline{$\int F'
\le\liminf_{n\to\infty}\int g_n
=\lim_{x\to\infty}F(x)-\lim_{x\to-\infty}F(x)$.}
}%end of proof of 565L

\leader{565M}{Theorem} Let $F:\Bbb R\to\Bbb R$ be a function.   Then the
following are equiveridical:

\inset{(i) there is an integrable function $f$ such that
$F(x)=\int_{\ooint{-\infty,x}}f$ for every $x\in\Bbb R$,

(ii) $F$ is of bounded variation, absolutely continuous on every bounded
interval, and
\ifdim\pagewidth>467pt\break\fi
$\lim_{x\to-\infty}F(x)=0$,}

\noindent and in this case $F'\eae f$.

\proof{{\bf (a)} If $f$ is integrable and
$F(x)=\int_{\ooint{-\infty,x}}f$ for every $x\in\Bbb R$, take any
$\epsilon>0$.   Then there is a $g\in C_k(\Bbb R)$ such that
$\int|f-g|\le\epsilon$ (565Ib).
Let $x_0$ be such that $g(x)=0$ for $x\le x_0$;  then

\Centerline{$|F(x)|\le\int|f-g|\le\epsilon$}

\noindent whenever $x\le x_0$.   Set
$\delta=\Bover{\epsilon}{1+\|g\|_{\infty}}$.   If
$a_0\le b_0\le\ldots\le a_n\le b_n$ and $\sum_{i=0}^nb_i-a_i\le\delta$,
then

$$\eqalign{\sum_{i=0}^n|F(b_i)-F(a_i)|
&=\sum_{i=0}^n|\int_{\coint{a_i,b_i}}f|
\le\sum_{i=0}^n\int_{\coint{a_i,b_i}}|g|+\int_{\coint{a_i,b_i}}|f-g|\cr
&\le\sum_{i=0}^n(b_i-a_i)\|g\|_{\infty}+\int|f-g|
\le\delta\|g\|_{\infty}+\epsilon
\le 2\epsilon.\cr}$$

\noindent As $\epsilon$ is arbitrary, $\lim_{x\to-\infty}F(x)=0$ and $F$ is
absolutely continuous on every bounded interval.   As for the variation of
$F$, if $a_0\le a_1\le\ldots\le a_n$ then

$$\eqalign{\sum_{i=1}^n|F(a_i)-F(a_{i-1})|
&=\sum_{i=1}^n|\int_{\coint{a_{i-1},a_i}}f|
\le\sum_{i=1}^n\int_{\coint{a_{i-1},a_i}}|f|\cr
&=\int_{\coint{a_0,a_n}}|f|
\le\int|f|,\cr}$$

\noindent so $\Var(F)\le\int|f|$ is finite.

Thus (i)$\Rightarrow$(ii).

\medskip

{\bf (b)} Moreover, under the conditions of (a),
$F'\eae f$.   \Prf\ Because $f$ is the difference of two
non-negative integrable functions, it is enough to consider the case
$f\ge 0$ a.e., so that $F$ is non-decreasing.
In this case, applying 565L to the function $x\mapsto\med(F(a),F(x),F(b))$,
we see that $\int_{\coint{a,b}}F'\le\int_{\coint{a,b}}f$ whenever $a\le b$
in $\Bbb R$;  also, applying 565L to $F$ itself, $F'$ is integrable.
Applying 565J to $f-F'$, we see that $F'\leae f$.

Recall that there is a countable subset $D$ of $C_k(\Bbb R)$
approximating all integrable functions in mean (565Ib).
So there is a sequence  $\sequencen{g_n}$ in $D$ such that
$\sum_{n=0}^{\infty}\int|g_n-f|$ is finite.   Set
$\tilde g_n=\sup_{i\le n}g_i^+$ for $n\in\Bbb N$;  then all the
$\tilde g_n$ are continuous, therefore resolvable, and
$\sequencen{\tilde g_n}$ is a codable sequence of integrable functions.
By 564Fa, $g=\lim_{n\to\infty}\tilde g_n$ is defined a.e.\ and integrable.
Let $G_n$, $G$ be the indefinite integrals of $\tilde g_n$, $g$
respectively.   Then the arguments just used show that $G'\leae g$.
But note that each $G_n$, being the indefinite integral of a continuous
function, has $G'_n=\tilde g_n$ exactly,
while $G'_n\le G'$ whenever $G'$ is defined.   So

\Centerline{$g\eae\lim_{n\to\infty}\tilde g_n=\lim_{n\to\infty}G'_n
\leae G'$,}

\noindent and $g\eae G'$.

At this point observe that $\int\liminf_{n\to\infty}|g_n-f|=0$, by 564Fb,
so $f\leae g$, while $G-F$ is the indefinite
integral of the essentially non-negative integrable function $g-f$.
So $G'-F'\leae g-f\eae G'-f$ and $f\leae F'$.   So actually $f\eae F'$,
as hoped for.\ \Qed

\medskip

{\bf (c)} Now suppose that $F:\Bbb R\to\Bbb R$
is of bounded variation and absolutely continuous on every bounded
interval, and that $\lim_{x\to-\infty}F(x)=0$.   By 224D and 565L,
$F'$ is integrable;  set $G(x)=\int_{\ooint{-\infty,x}}F'$
and $H(x)=F(x)-G(x)$ for $x\in\Bbb R$.   By (b), $H'=F'-G'$ is zero a.e.,
while $H$, like $F$ and $G$, is absolutely continuous on every bounded
interval.   But this means that $H$ is constant.   \Prf\ Suppose that
$a<b$ in $\Bbb R$ and $\epsilon>0$.   Let $\delta\in\ooint{0,b-a}$
be such that $\sum_{i=0}^n|H(b_i)-H(a_i)|\le\epsilon$ whenever
$a\le a_0\le b_0\le a_1\le b_1\le\ldots\le a_n\le b_n\le b$ and
$\sum_{i=0}^nb_i-a_i\le\delta$   Set
$E=\{x:x\in\ooint{a,b}$, $H'(x)=0\}$.   Let $\Cal C$ be the family of
non-trivial closed subintervals $[c,d]$ of $\ooint{a,b}$ with rational
endpoints such that $|H(d)-H(c)|\le\epsilon(d-c)$;  then every point of $E$
belongs to arbitrarily small members of $\Cal C$.   By Vitali's theorem
(565F) there is a disjoint countable
family $\Cal I\subseteq\Cal C$ such that
$E\setminus\bigcup\Cal I$ is negligible, so that

\Centerline{$\sum_{I\in\Cal I}\mu_LI=\mu_L(\bigcup\Cal I)=b-a$.}

\noindent Let $\Cal J\subseteq\Cal I$ be a finite subset such that
$\sum_{I\in\Cal J}\mu_LI\ge b-a-\delta$;  express $\Cal J$ as
$\ofamily{i}{n}{[b_i,a_{i+1}]}$ where $\ofamily{i}{n}{b_i}$ is strictly
increasing.   Setting $a_0=a$ and $b_n=b$, we have
$a=a_0\le b_0\le\ldots\le a_n\le b_n=b$ and $\sum_{i=0}^nb_i-a_i\le\delta$.
So

$$\eqalign{|H(b)-H(a)|
&\le\sum_{i=0}^n|H(b_i)-H(a_i)|+\sum_{i=0}^{n-1}|H(a_{i+1})-H(b_i)|\cr
&\le\epsilon+\epsilon\sum_{i=0}^{n-1}(a_{i+1}-b_i)
\le\epsilon(1+b-a).\cr}$$

\noindent As $a$, $b$ and $\epsilon$ are arbitrary, $H$ is constant.\
\Qed

So $F-G$ is constant.   As both $F$ and $G$ tend to $0$ at $-\infty$,
they are equal.   Thus $F(x)=\int_{\ooint{-\infty,x}}F'$ for every $x$, and
$F$ is an indefinite integral.
}%end of proof of 565M

\leader{565N}{Hausdorff measures} Let $(X,\rho)$ be a metric space and
$s\in\ooint{0,\infty}$.   As in \S471, we can define {\bf Hausdorff
$s$-dimensional submeasure} $\theta_s:\Cal PX\to[0,\infty]$ by
writing

$$\eqalign{\theta_sA
=\sup_{\delta>0}\inf\{\sum_{n=0}^{\infty}(\diam D_n)^s:
  \sequencen{D_n}
&\text{ is a sequence of subsets of }X\text{ covering }A,\cr
&\qquad\qquad\qquad\quad
  \diam D_n\le\delta\text{ for every }n\in\Bbb N\},\cr}$$

\noindent counting $\diam\emptyset$ as $0$ and
$\inf\emptyset$ as $\infty$.
\cmmnt{As with Lebesgue submeasure,} $\theta_s$ is a submeasure.

\leader{565O}{Theorem} Let $(X,\rho)$ be a second-countable metric
space, and $s>0$.   Then there is a Borel-coded measure $\mu$ on $X$ such
that $\mu K=\theta_sK$ whenever $K\subseteq X$ is compact and $\theta_sK$
is finite.

%If $\theta_sX$ is finite, then there is a Borel-coded
%measure $\mu$ on $X$ such that $\mu G=\theta_sG$ for every open set
%$G\subseteq X$.

\proof{{\bf (a)} To begin with, suppose that $X$ is compact and
$\theta_sX$ is finite.

\medskip

\quad{\bf (i)} Let $\Cal U$ be a countable base for the topology of $X$
closed under finite unions;  let
$\preccurlyeq$ be a well-ordering of $\Cal U$.
Then for any compact $K\subseteq X$,
$\delta>0$ and $\epsilon>0$, there are $U_0,\ldots,U_n\in\Cal U$ such that
$K\subseteq\bigcup_{i\le n}U_i$, $\diam U_i\le\delta$ for every $i$ and
$\sum_{i=0}^n(\diam U_i)^s\le\theta_sK+\epsilon$.   \Prf\ There is a sequence
$\sequencen{A_n}$ of subsets of $X$ such that
$K\subseteq\bigcup_{n\in\Bbb N}A_n$, $\diam A_n\le\bover12\delta$ for every
$n$ and $\sum_{n=0}^{\infty}(\diam A_n)^s\le\theta_sK+\bover12\epsilon$.
Let $\sequencen{\eta_n}$ be a sequence in $\ooint{0,\bover14\delta}$ such
that $\sum_{n=0}^{\infty}(2\eta_n+\diam A_n)^s\le\theta_sK+\epsilon$.

For each $n\in\Bbb N$, set $G_n
=\{x:\rho(x,A_n)<\eta_n\}$.   Then $\overline{A}_n$ is a compact subset 
of $G_n$ so there is a $\preccurlyeq$-first $U_n\in\Cal U$ such that
$\overline{A}_n\subseteq U_n\subseteq G_n$.   Now
$\diam U_n\le\min(\delta,\diam A_n+2^{-n-2}\epsilon)$ for each $n$ so
$\sum_{n=0}^{\infty}(\diam U_n)^s\le\theta_sK+\epsilon$.   But as $K$ is
compact there is an $n$ such that $K\subseteq\bigcup_{i\le n}U_i$.\ \Qed

\medskip

\quad{\bf (ii)} As in the proof of 471Da, $\theta_s$ is a `metric
submeasure', that is, $\theta_s(A\cup B)=\theta_sA+\theta_sB$ whenever $A$,
$B\subseteq X$ and $\rho(A,B)>0$.   (It will be convenient here to say
that $\rho(A,B)=\infty$ if either $A$ or $B$ is empty.)   It follows that
$\theta_s(\bigcup_{n\in\Bbb N}K_n)=\sum_{n=0}^{\infty}\theta_sK_n$ whenever
$\sequencen{K_n}$ is a disjoint sequence of compact subsets of $X$.   \Prf\
Recall that $\rho(K,K')>0$ whenever $K$, $K'$ are
disjoint compact subsets of $X$;  this is because $K\times K'$ is compact
and $\rho:X\times X\to\Bbb R$ is continuous.   So

\Centerline{$\theta_s(\bigcup_{n\in\Bbb N}K_n)
\ge\theta_s(\bigcup_{i\le n}K_i)
=\sum_{i=0}^n\theta_sK_i$}

\noindent for every $n\in\Bbb N$, and
$\theta_s(\bigcup_{n\in\Bbb N}K_n)\ge\sum_{n=0}^{\infty}\theta_sK_n$.   In
the other direction, let $\epsilon>0$.   Let $\preccurlyeq'$ be a
well-ordering of $\bigcup_{n\in\Bbb N}\Cal U^n$.   Then for each
$n\in\Bbb N$ there is a $\preccurlyeq'$-first finite sequence
$U_{n0},\ldots,U_{nm_n}$ in $\Cal U$ such that
$K_n\subseteq\bigcup_{i\le m_n}U_{ni}$, $\diam U_{ni}\le\epsilon$ for every
$i$ and $\sum_{i=0}^{m_n}(\diam U_{ni})^s\le\theta_sK_n+2^{-n}\epsilon$.
Now $\langle U_{ni}\rangle_{n\in\Bbb N,i\le m_n}$ witnesses that

$$\eqalign{\sum_{n=0}^{\infty}\theta_sK_n+2\epsilon
&\ge\inf\{\sum_{j=0}^{\infty}(\diam D_j)^s:
  \sequence{j}{D_j}\text{ is a sequence of subsets of }X\cr
&\mskip150mu\text{ covering }\bigcup_{n\in\Bbb N}K_n,\,
  \diam D_j\le\epsilon\text{ for every }j\in\Bbb N\}.\cr}$$

\noindent As $\epsilon$ is arbitrary,
$\theta_s(\bigcup_{n\in\Bbb N}K_n)\le\sum_{n=0}^{\infty}\theta_s(K_n)$.\
\Qed

\medskip

\quad{\bf (iii)} If $G\subseteq X$ is open and $\epsilon>0$, there is a compact
set $K\subseteq G$ such that $\theta_s(G\setminus K)\le\epsilon$.   \Prf\
Set $K_0=\{x:\rho(x,X\setminus G)\ge 1\}$ and for $n\ge 1$ set

\Centerline{$K_n=\{x:2^{-n}\le\rho(x,X\setminus G)\le 2^{n+1}\}$.}

\noindent Then $\sum_{n=0}^{\infty}\theta_sK_{2n}$ and
$\sum_{n=0}^{\infty}\theta_sK_{2n+1}$ are both bounded by
$\theta_sX<\infty$, so there is an $n\in\Bbb N$ such that
$\sum_{i=n}^{\infty}\theta_sK_n\le\epsilon$.   But this means that
$\theta_s(\bigcup_{i\ge n}K_n)\le\epsilon$ (apply (ii) to the odd and even
terms separately).   Set $K=\bigcup_{i\le n}K_i$;  this works.\ \Qed

\medskip

\quad{\bf (iv)} Writing $\frak T$ for the topology of $X$,
$\theta_s\restrp\frak T$ satisfies the conditions of 563H.   \Prf\ Of course
it is zero at $\emptyset$, monotonic and locally finite.   If $G$,
$H\in\frak T$ and $\epsilon>0$, let $K\subseteq G$, $L\subseteq H$ be
compact sets such
that $\theta_s(G\setminus K)+\theta_s(H\setminus L)\le\epsilon$.   Then
$K\setminus H$, $K\cap L$ and $L\setminus G$ are disjoint compact sets and

\Centerline{$(G\cup H)
\setminus((K\setminus H)\cup(K\cap L)\cup(L\setminus G))$,
\quad$(G\cap H)\setminus(K\cap L)$,}

\Centerline{$G\setminus((K\setminus H)\cup(K\cap L))$,
\quad$H\setminus((L\setminus G)\cup(K\cap L))$}

\noindent are all included in $(G\setminus K)\cup(H\setminus L)$, so all
have submeasure at most $\epsilon$.   But this means that
$\theta_s(G\cup H)+\theta_s(G\cap H)$ and $\theta_sG+\theta_sH$ both differ
from $\theta_s(K\setminus H)+2\theta_s(K\cap L)+\theta_s(L\setminus G)$ by
at most $2\epsilon$ (upwards)
and differ from each other by at most $2\epsilon$ also.
As $\epsilon$ is arbitrary, we have the modularity condition.

As for the sequential order-continuity, this is elementary;  if
$\sequencen{G_n}$ is a non-decreasing sequence with union $G$, and
$\epsilon>0$, there is a compact $K\subseteq G$ such that
$\theta_s(G\setminus K)\le\epsilon$;  now $K\subseteq G_n$ for some $n$,
and $\theta_sG\le\theta_sG_n+\epsilon$.\ \Qed

\medskip

\quad{\bf (v)} So 563H tells us that there is a Borel-coded
measure $\mu$ on $X$
extending $\theta_s\restrp\frak T$.   Now $\mu K=\theta_sK$ for every
compact $K\subseteq X$.   \Prf\

\Centerline{$\mu K+\mu(X\setminus K)=\mu X
=\theta_sX\le\theta_sK+\theta_s(X\setminus K)
=\theta_sK+\mu(X\setminus K)$,}

\noindent so $\mu K\le\theta_sK$.   On the other hand, given $\epsilon>0$,
there is a compact $L\subseteq X\setminus K$ such that
$\theta_sL\ge\theta_s(X\setminus K)-\epsilon$, and now

\Centerline{$\theta_sK=\theta_s(K\cup L)-\theta_sL
\le\theta_sX-\theta_s(X\setminus K)+\epsilon
=\mu K+\epsilon$;}

\noindent as $\epsilon$ is arbitrary, $\mu K=\theta_sK$.\ \Qed

\medskip

\quad{\bf (vi)} Note that 563H tells us that $\mu$ is the only Borel-coded
measure extending $\theta_s\restrp\frak T$, and must therefore be the only
Borel-coded measure agreeing with $\theta_s$ on the compact sets.

\medskip

{\bf (b)} For the general case, let $\Cal K$ be
$\{K:K\subseteq X$ is compact, $\theta_sK<\infty\}$.
Then (a) tells us that for every $K\in\Cal K$ there is a unique Borel-coded measure
$\mu_K$ on $K$ agreeing with $\theta_s$ on the compact subsets of $K$.
If $K$, $L\in\Cal K$ and $K\subseteq L$, $\mu_L\restr\Cal B_c(K)$ is a
Borel-coded measure on $K$ (563Fa)
agreeing with $\theta_s$ on the compact subsets of $K$, so $\mu_L$ extends
$\mu_K$.   We therefore have a Borel-coded measure $\mu$ on $X$
defined by setting $\mu E=\sup_{K\in\Cal K}\mu_K(E\cap K)$ for every
$E\in\Cal B_c(X)$ (cf.\ 563E),
and $\mu$ agrees with $\theta_s$ on $\Cal K$, as required.
}%end of proof of 565O

%any hope of Howroyd's theorem?

\exercises{\leader{565X}{Basic exercises (a)}%
%\spheader 565Xa
(i) Show that Lebesgue submeasure $\theta$ and Lebesgue measure are
translation-invariant.   (ii) Show that if $A\subseteq\BbbR^r$ and
$\alpha\ge 0$ then $\theta(\alpha A)=\alpha^r\theta A$.  (iii)
Show that if $E\subseteq\BbbR^r$ is measurable and
$\alpha\in\Bbb R$ then $\alpha E$ is measurable.
%565C

\spheader 565Xb Suppose that there is a sequence $\sequencen{A_n}$ of
countable subsets of $[0,1]$ with union $[0,1]$.   (i) Set
$A=\bigcup_{m\le n}A_m+n$.    Show that $A$ belongs to the algebra $\Sigma$
of 565C, that the Lebesgue submeasure of $A$ is $\infty$,
but that $A\cap[0,n]$ is Lebesgue negligible for every $n$.
(ii) Set $B=\{2^{-n}x:n\in\Bbb N$, $x\in A_n\}$.   Show that $B$ has
Lebesgue submeasure $0$, but is not Lebesgue negligible.
%565D

\spheader 565Xc Let $g:\Bbb R\to\Bbb R$ be a non-decreasing function.   For
half-open intervals $I\subseteq\Bbb R$ define $\lambda_gI$ by setting

\Centerline{$\lambda_g\emptyset=0$,
\quad$\lambda_g\coint{a,b}
=\lim_{x\uparrow b}g(x)-\lim_{x\uparrow a}g(x)$}

\noindent if $a<b$.   For any set $A\subseteq\Bbb R$ set

\Centerline{$\theta_gA=\inf\{\sum_{j=0}^{\infty}\lambda_gI_j:
\sequence{j}{I_j}
\text{ is a sequence of }\text{half-open intervals covering }A$.}

\noindent Show that $\theta_g$ is a submeasure on $\Cal P\Bbb R$.
Show that there is a Borel-coded measure
$\mu_g$ on $\Bbb R$ agreeing with $\theta_g$ on open sets.
%565D

\spheader 565Xd Apply 564N to relate Lebesgue measure on $\BbbR^2$ to
Lebesgue measure on $\Bbb R$.
%565D

\spheader 565Xe Suppose that there is a sequence $\sequencen{A_n}$ of
countable sets with union $[0,1]$.   Show that there is a set
$A\subseteq[0,1]^2$, with two-dimensional Lebesgue submeasure zero, such
that all the vertical sections $A[\{x\}]$, for $x\in[0,1]$, have non-zero
one-dimensional Lebesgue measure.
%565L

\spheader 565Xf Confirm that the principal
results of \S281 can be proved without the axiom of choice.

\leader{565Y}{Further exercises (a)}
%\spheader 565Ya
Show that if $X$ is a second-countable space and $\mu$ is a
codably $\sigma$-finite Borel-coded
measure on $X$, then there is a non-decreasing function
$g:\Bbb R\to\Bbb R$ such that the Lebesgue-Stieltjes measure $\mu_g$
of 565Xc has measure algebra isomorphic to that of $\mu$.
%565D 565Xc
}%end of exercises


\endnotes{
\Notesheader{565} In these five sections I have tried to indicate,
without succumbing to the temptation to re-write the whole treatise, a
possible version of Lebesgue's theory which can be used in plain ZF.
With the Fundamental Theorem of Calculus (565M), the
Radon-Nikod\'ym theorem (564L), Fubini's theorem (564N) and at least some
infinite product measures (564O), it is clear that most of the ideas of
Volume 2 should be expressible in forms not relying on the axiom of choice.
We must expect restrictions of the type already found in the convergence
theorems (564F);  for versions of the Central Limit Theorem or the strong
law of large numbers or Koml\'os' theorem, for instance, we should
certainly start by changing any hypothesis `let $\sequencen{f_n}$ be a
sequence of random variables' into `let $\sequencen{f_n}$ be a codable
sequence of codable Borel functions'.   I am not sure how to approach
martingales, but the best chance of positive results will be with `codable
martingales' in which we have a full set of Borel codes for countable sets
generating each of the $\sigma$-algebras involved, along the lines of
564Md.   If you glance at the formulae of
Chapter 28, you will see that while there are many appeals to the
convergence theorems, they are generally applied to sequences of the form
$\sequencen{f\times g_n}$ where $f$ is integrable and the $g_n$ are
continuous;  but this means that $\sequencen{g_n}$ is necessarily codable
(562Qa, 562Sc) 
so that $\sequencen{f\times g_n}$ will be a codable sequence if $f$
itself is a codable function.

In Volumes 3 and 4 we encounter much more solid obstacles, and I see no way
in which Maharam's theorem, or the Lifting Theorem, can be made to work
without something approaching the full axiom of choice, or a strong
hypothesis declaring the existence of a well-orderable set at a crucial
point.   I give an example of such a hypothesis in the statement of
Vitali's theorem (565F).   But in the applications of Vitali's theorem
later in this section, we can always work with a countable family of balls,
for which well-orderability is not an issue.   Separability and
second-countability hypotheses can be expected to act in similar ways;  so
that, for instance, we have 565Ya, which is a kind of primitive case of
Maharam's theorem.


}%end of notes

\discrpage


