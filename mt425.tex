\frfilename{mt425.tex}
\versiondate{9.8.13}
\copyrightdate{2009}

\def\chaptername{Descriptive set theory}
\def\sectionname{Realization of automorphisms}

\newsection{*425\dvAnew{2009}}

In \S344 I presented some results on the representation of a countable
semigroup of Boolean homomorphisms in a measure algebra by a semigroup of
functions on the measure space underlying the algebra.   \S424 provides us
with the
tools needed for a remarkable extension, in the case of the Lebesgue
measure algebra, to groups of cardinal $\omega_1$
(Theorem 425D).   The expression of the ideas is made smoother by using the
language of group actions\cmmnt{ (4A5B-4A5C)}.

\leader{425A}{}\dvAnew{2011}
I begin with what amounts to a special case of the main
theorem, with some refinements which will be useful elsewhere.

\medskip

\noindent{\bf Proposition} (a) Let $(X,\Sigma)$ and $(Y,\Tau)$ be
non-empty standard Borel
spaces, and $\Cal I$, $\Cal J\,\,\sigma$-subalgebras of $\Sigma$, $\Tau$
respectively;  write $\frak A=\Sigma/\Cal I$ and $\frak B=\Tau/\Cal J$ for
the quotient algebras.   For $E\in\Sigma$, $F\in\Tau$ write $\Sigma_E$,
$\Tau_F$ for the subspace $\sigma$-algebras on $E$, $F$ respectively.

(a) If $\pi:\frak A\to\frak B$ is a sequentially order-continuous
Boolean
homomorphism, there is a $(\Tau,\Sigma)$-measurable $f:Y\to X$ which
represents $\pi$ in the sense that
$\pi E^{\ssbullet}=f^{-1}[E]^{\ssbullet}$ for every $E\in\Sigma$.

(b) If $\pi:\frak A\to\frak B$ is a Boolean
isomorphism, there are $G\in\Cal I$, $H\in\Cal J$ and a bijection
$h:Y\setminus H\to X\setminus G$ which is an isomorphism between
$(Y\setminus H,\Tau_{Y\setminus H})$ and
$(X\setminus G,\Sigma_{X\setminus G})$, and represents $\pi$ in the sense that
$\pi E^{\ssbullet}=h^{-1}[E\setminus G]^{\ssbullet}$ for every
$E\in\Sigma$.

(c) If $\pi:\frak A\to\frak A$ is a Boolean automorphism, there is a
bijection $h:X\to X$ which is an
automorphism of $(X,\Sigma)$ and represents $\pi$ in the sense of (a).

(d) If $\#(X)=\#(Y)=\frak c$, $\frak A$ and $\frak B$ are ccc, and
$\pi:\frak A\to\frak B$ is a Boolean isomorphism, there is a bijection
$h:Y\to X$ which is an isomorphism between $(Y,\Tau)$ and $(X,\Sigma)$, and
represents $\pi$ in the sense of (a).

\proof{{\bf (a)(i)} If either $\frak A$ or $\frak B$ is $\{0\}$, so
is the other, and we can take $f$ to be a constant function.   So
henceforth let us suppose that $X\notin\Cal I$ and $Y\notin\Cal J$.

\medskip

\quad{\bf (ii)} If $X\subseteq\Bbb N$ and $\Sigma=\Cal PX$, set
$Z=\{n:n\in X$, $\{n\}\notin\Cal I\}$.   For $n\in Z$, set
$a_n=\{n\}^{\ssbullet}$ and $b_n=\pi a_n$, and choose $F_n\in\Tau$ such
that $F_n^{\ssbullet}=b_n$.   Since $\sup_{n\in Z}a_n=1$ in
$\frak A$, $\sup_{n\in Z}b_n=1$ in $\frak B$ and
$Y\setminus\bigcup_{n\in Z}F_n\in\Cal J$.   Define $f:Y\to X$ by saying
that

$$\eqalign{f(y)
&=\min\{n:n\in Z,\,y\in F_n\}\text{ if }y\in\bigcup_{n\in Z}F_n,\cr
&=\min \text{ otherwise}.\cr}$$

\noindent Then $f$ represents $\pi$ in the required sense.

\medskip

\quad{\bf (iii)} If $X=\{0,1\}^{\Bbb N}$ and $\Sigma$ is
its Borel $\sigma$-algebra, then for each $n\in\Bbb N$ set
$E_n=\{x:x(n)=1\}$ and $e_n=E_n^{\ssbullet}$
and choose $F_n\in\Tau$ such that
$F_n^{\ssbullet}=\pi e_n$.
Set $f(y)=\sequencen{\chi F_n(y)}$ for $y\in Y$.
Then $f^{-1}[E_n]^{\ssbullet}=\pi E_n^{\ssbullet}$ for every $n$;  as
$\{E:f^{-1}[E]^{\ssbullet}=\pi E^{\ssbullet}\}$ is a $\sigma$-subalgebra of
$\Sigma$ containing every $E_n$, it is the whole of $\Sigma$, and again $f$
represents $\pi$.

\medskip

\quad{\bf (iv)} By 424C, any standard Borel space is isomorphic to either
that in (iii) or one of those in (ii), so these cases together
are sufficient to prove the general result.

\medskip

{\bf (b)} By (a), we have $f:Y\to X$ and $g:X\to Y$ representing
$\pi$, $\pi^{-1}$ respectively.   Now we see that $gf:Y\to Y$ represents
$\pi\pi^{-1}:\frak B\to\frak B$, that is,
$F\symmdiff(gf)^{-1}[F]\in\Cal J$ for every $F\in\Tau$.   Let
$\sequencen{F_n}$ be a sequence in $\Tau$ which separates the points of
$Y$, and set $H_0=\bigcup_{n\in\Bbb N}F_n\symmdiff(gf)^{-1}[F_n]$;
then $H_0\in\Cal J$ and $g(f(y))=y$ for every $y\in Y\setminus H_0$.

Similarly, there is a $G_0\in\Cal I$ such that $f(g(x))=x$ for every
$x\in X\setminus G_0$.   Set $G=G_0\cup g^{-1}[H_0]$ and
$H=H_0\cup f^{-1}[G_0]$.   Then $g[X\setminus G]\subseteq Y\setminus H$.
\Prf\ If $x\in X\setminus G$ then $g(x)\notin H_0$;  moreover,
$f(g(x))=x\notin G_0$ so $g(x)\notin f^{-1}[G_0]$ and $g(x)\notin H$.\
\QeD\  Similarly, $f(y)\in X\setminus G$ for every $y\in Y\setminus H$.
As $f(g(x))=x$ for $x\in X\setminus G$ and $g(f(y))=y$ for
$y\in Y\setminus H$, $h=f\restr Y\setminus H$ is a bijection with inverse
$g\restr X\setminus G$.   Because $f$ is $(\Tau,\Sigma)$-measurable,
$h$ is
$(\Tau_{Y\setminus H},\Sigma_{X\setminus G})$-measurable;
because $g$ is $(\Sigma,\Tau)$-measurable, $h^{-1}$ is
$(\Sigma_{X\setminus G},\Tau_{Y\setminus H})$-measurable, and $h$ is an
isomorphism between $(Y\setminus H,\Tau_{Y\setminus H})$
and $(X\setminus G,\Sigma_{X\setminus G})$.
Finally, if $E\in\Sigma$,

\Centerline{$\pi(E^{\ssbullet})=\pi((E\setminus G)^{\ssbullet})
=(f^{-1}[E\setminus G])^{\ssbullet}
=(h^{-1}[E\setminus G])^{\ssbullet}$,}

\noindent so $h$ represents $\pi$ in the sense declared.

\medskip

{\bf (c)} We can repeat the proof of (b) with an additional idea.   The
point is that there is an element $E_0$ of $\Cal I$ with maximal
cardinality.   \Prf\ Because $(G,\Sigma_G)$ is a standard Borel space
(424G), $\#(G)$ is either $\frak c$ or countable (424Db), 
for every $G\in\Cal I$.
If $\Cal I$ contains arbitrarily large finite sets, it must contain an
infinite set, because it is a $\sigma$-algebra.   So the supremum
$\sup\{\#(G):G\in\Cal I\}$ is attained.\ \Qed

Now, in (b), take $(Y,\Tau)=(X,\Sigma)$ and $\frak B=\frak A$, and choose
$f$, $H_0$, $g$ and $G_0$ as before;  but this time set
$G=(G_0\cup E_0)\cup g^{-1}[H_0\cup E_0]$ and
$H=(H_0\cup E_0)\cup f^{-1}[G_0\cup E_0]$.   The same arguments as before
tell us that $h_0=f\restr Y\setminus H$ is an isomorphism between
$(Y\setminus H,\Tau_{Y\setminus H})$ and
$(X\setminus G,\Sigma_{X\setminus G})$ representing $\pi$.   We now,
however, have $G$, $H\in\Cal I$ and both include $E_0$;  so
we must have $\#(G)=\#(E_0)=\#(H)$.   By 424Da, there is an isomorphism
$h_1$ between $(H,\Sigma_H)$ and $(G,\Sigma_G)$.   So if we set

$$\eqalign{h(y)
&=h_0(y)\text{ if }y\in X\setminus H,\cr
&=h_1(y)\text{ if }y\in H,\cr}$$

\noindent then $h:X\to X$ is a bijection which is an automorphism
of $(X,\Sigma)$, and

\Centerline{$h^{-1}[E]^{\ssbullet}
=(H\cap h^{-1}[E])^{\ssbullet}
=(h^{-1}[E\cap G])^{\ssbullet}
=(h_0^{-1}[E\cap G])^{\ssbullet}
=\pi E^{\ssbullet}$}

\noindent for every $E\in\Sigma$, as required.

\medskip

{\bf (d)} An adaptation of the ideas of (c) works in this case too.
First note that as $\#(X)=\frak c$,
$(X,\Sigma)\cong(\BbbR^2,\Cal B(\BbbR^2))$ and there is a partition
of $X$ into $\frak c$ members of $\Sigma$ all of cardinal $\frak c$.
As $\frak A$ is ccc, all but countably many
of these must belong to $\Cal I$, and we have an $E_0\in\Cal I$ with
$\#(E_0)=\frak c$.   Similarly, there is an $F_0\in\Cal J$ with
$\#(F_0)=\frak c$.

Now choose
$f$, $H_0$, $g$ and $G_0$ as in (b);  but this time set
$G=(G_0\cup E_0)\cup g^{-1}[H_0\cup F_0]$ and
$H=(H_0\cup F_0)\cup f^{-1}[G_0\cup E_0]$.   Again,
$h_0=f\restr Y\setminus H$ is an isomorphism between
$(Y\setminus H,\Tau_{Y\setminus H})$ and
$(X\setminus G,\Sigma_{X\setminus G})$ representing $\pi$.   As
$\#(G)=\frak c=\#(H)$, there is an isomorphism
$h_1$ between $(H,\Tau_H)$ and $(G,\Sigma_G)$.   So if we set

$$\eqalign{h(y)
&=h_0(y)\text{ if }y\in Y\setminus H,\cr
&=h_1(y)\text{ if }y\in H,\cr}$$

\noindent we get a bijection $h:Y\to X$ which is an isomorphism between
$(Y,\Tau)$ and $(X,\Sigma)$, and represents $\pi$, just as in (c).
}%end of proof of 425A
%should be at the beginning of the section;  useful for (c) in the proof
%of 425D

\leader{425B}{Lemma} Let $G$ be a group, $G_0$ a
subgroup of $G$, $H$ another group, and $X$, $Z$ sets;  let
$\action_r$ be the right shift action of $H$ on $Z^H$\cmmnt{ (4A5C(c-ii))}.
Suppose we are given a group homomorphism $\theta:G\to H$,
an injective function $f:\Bbb N\times Z^H\to X$
and an action $\action_0$ of $G_0$ on $X$ such that
$\pi\action_0f(n,z)=f(n,\theta(\pi)\action_rz)$ whenever
$n\in\Bbb N$ and $z\in Z^H$.

(a) If $\#(X\setminus f[\Bbb N\times Z^H])\le\#(Z)$, there is an
action $\action$ of $G$ on $X$ extending $\action_0$.

(b) Suppose moreover 
that $H$ is countable, $X$ and $Z$ are Polish spaces, and
$f$ is Borel measurable when $\Bbb N\times Z^H$ is given the product
topology.   If $x\mapsto\pi\action_0x$ is
Borel measurable for every $\pi\in G_0$, then $\action$ can be chosen in
such a way that $x\mapsto\psi\action x$ is Borel measurable
for every $\psi\in G$.

\proof{{\bf (a)(i)}
Let $D\subseteq H$ be a selector for the left cosets of the subgroup
$\theta[G_0]$,
so that every member of $H$ is uniquely expressible as $\psi\theta(\pi)$
where $\psi\in D$ and $\pi\in G_0$.   Observe that if $\pi\in G_0$ and
$x\in f[\Bbb N\times Z^H]$, then
$\pi^{-1}\action_0x\in f[\Bbb N\times Z^H]$;  consequently, setting
$Y=X\setminus f[\Bbb N\times Z^H]$, $\pi\action_0y\in Y$ for
every $y\in Y$,
because $\pi^{-1}\action_0(\pi\action_0y)=y$.
Let $g_0:Y\to Z$ be any injection, and define
$g:Y\to Z^H$ by setting
$g(y)(\psi\theta(\pi))=g_0(\pi\action_0y)$ whenever $\psi\in D$,
$\pi\in G_0$ and $y\in Y$.   In this case,
$\theta(\pi)\action_rg(y)=g(\pi\action_0y)$ whenever $y\in Y$ and
$\pi\in G_0$.   \Prf\ Take any $\psi\in D$ and $\phi\in G_0$.
Then

$$\eqalign{(\theta(\pi)\action_rg(y))(\psi\theta(\phi))
&=g(y)(\psi\theta(\phi)\theta(\pi))
=g(y)(\psi\theta(\phi\pi))\cr
&=g_0(\phi\pi\action_0y)
=g_0(\phi\action_0(\pi\action_0y))
=g(\pi\action_0y)(\psi\theta(\phi)).\cr}$$

\noindent As $D\theta[G_0]=H$, this shows that
$\theta(\pi)\action_rg(y)=g(\pi\action_0y)$.  \Qed

Note that as $g(y)(\psi)=g_0(y)$ whenever $\psi\in D$ and $y\in Y$,
$g$ also is injective.

\medskip

\quad{\bf (ii)} Now define $h:\Bbb N\times Z^H\to X$ by setting

$$\eqalignno{h(n,z)
&=g^{-1}(z)\text{ if }n=0\text{ and }z\in g[Y]\cr
&\mskip150mu
  \text{matching }\{0\}\times g[Y]\text{ with }Y,\cr
&=f(n-1,z)\text{ if }n\ge 1\text{ and }z\in g[Y]\cr
&\mskip150mu
  \text{matching }(\Bbb N\setminus\{0\})\times g[Y]\text{ with }
    f[\Bbb N\times g[Y]],\cr
&=f(n,z)\text{ if }n\in\Bbb N\text{ and }z\notin g[Y]\cr
&\mskip150mu
  \text{matching }\Bbb N\times(Z^H\setminus g[Y])
    \text{ with }f[\Bbb N\times(Z^H\setminus g[Y])].\cr}$$

\noindent Clearly $h$ is a bijection.   If $n\in\Bbb N$, $z\in Z^H$ and
$\pi\in G_0$, then $h(n,\theta(\pi)\action_rz)=\pi\action_0h(n,z)$.   \Prf\
If $z\in g[Y]$, then

\Centerline{$\theta(\pi)\action_rz
=\theta(\pi)\action_rg(g^{-1}(z))
=g(\pi\action_0g^{-1}(z))\in g[Y]$,}

\noindent so

\Centerline{$h(0,\theta(\pi)\action_rz)
=\pi\action_0g^{-1}(z)=\pi\action_0h(0,z)$,}

\noindent while if $n\ge 1$, then

\Centerline{$h(n,\theta(\pi)\action_rz)
=f(n-1,\theta(\pi)\action_rz)
=\pi\action_0f(n-1,z)
=\pi\action_0h(n,z)$.}

\noindent On the other hand,
if $n\in\Bbb N$ and $z\in Z^H\setminus g[Y]$, then
$\theta(\pi)\action_rz\notin g[Y]$, because
$\theta(\pi^{-1})\action_r(\theta(\pi)\action_rz)\notin g[Y]$;   so

\Centerline{$h(n,\theta(\pi)\action_rz)
=f(n,\theta(\pi)\action_rz)
=\pi\action_0f(n,z)
=\pi\action_0h(n,z)$.  \Qed}

\medskip

\quad{\bf (iii)} Try

\Centerline{$\psi\action x=h(n,\theta(\psi)\action_rz)$}

\noindent whenever $x\in X$, $h^{-1}(x)=(n,z)$ and $\psi\in G$.
If $x\in X$, $\psi$, $\psi'\in G$ and $\pi\in G_0$,
express $h^{-1}(x)$ as $(n,z)$;  then, writing $\iota$ for the identity of
$G$,

$$\eqalign{\iota\action x
&=h(n,\theta(\iota)\action_rz)
=h(n,z)
=x,\cr
\psi'\psi\action x
&=h(n,\theta(\psi'\psi)\action_rz)
=h(n,\theta(\psi')\action_r(\theta(\psi)\action_rz))
=\psi'\action h(n,\theta(\psi)\action_rz)
=\psi'\action(\psi\action x),\cr
\pi\action x
&=h(n,\theta(\pi)\action_rz)
=\pi\action_0h(n,z)
=\pi\action_0x,\cr}$$

\noindent so we have an action
of $G$ on $X$ extending $\action_0$, as required.

\medskip

{\bf (b)} Under the topological hypotheses, we follow the same line of
argument, but taking time for checks at each stage.
Because $Z$, and therefore $\Bbb N\times Z^H$, are Polish spaces,
and $f$ is a measurable injection,
$f[\Bbb N\times Z^H]$ and $Y$ are Borel sets (423Ib).
In (i), we must of
course take $g_0$ to be Borel measurable;  since all the
functions $x\mapsto\pi\action_0x$ are Borel measurable,
all the functions
$y\mapsto g(y)(\psi)$ will be Borel measurable, and
$g:Y\to Z^H$ will be Borel measurable.
Consequently all the sets $\{n\}\times g[Y]$ will be Borel,
and $h$ will be Borel measurable, therefore a Borel
isomorphism between $\Bbb N\times Z^H$ and $X$.
Finally, because $z\mapsto\theta(\psi)\action_rz:Z^H\to Z^H$
is Borel measurable for every $\psi\in G$,
$(n,z)\mapsto(n,\theta(\psi)\action_rz)$ is Borel measurable for every
$n$, and $x\mapsto\psi\action x$ is Borel measurable,
for every $\psi\in G$.   So $\action$ is measurable in the
required sense.
}%end of proof of 425B

\leader{425C}{Master actions (a)} For each $R\subseteq\BbbN^2$,
consider the family $F_R$ of injective functions $f$ from countable
ordinals to $\Bbb N$ such that

\Centerline{for every $\beta\in\dom f$, $f(\beta)$ is the unique member of
$\Bbb N$ such that
$R[\{f(\beta)\}]=f[\beta]$.}

\noindent\cmmnt{If $f$, $g\in F_R$ have domains $\alpha$, $\alpha'$
respectively
where $\alpha\le\alpha'$, then $f=g\restr\alpha$.
\prooflet{\Prf\ If $\beta<\alpha$
is such that $f\restr\beta=g\restr\beta$, then both $f(\beta)$ and
$g(\beta)$ are the unique $m\in\Bbb N$ such that
$R[\{m\}]=f[\beta]$.\ \QeD}   Consequently }%
$f_R=\bigcup F_R$ is the unique maximal element of $F_R$.
\cmmnt{(Compare 2A1B.)}

For a countable ordinal $\alpha$, let
$\Cal R_{\alpha}$ be the set of those $R\subseteq\BbbN^2$ such that
$\alpha\le\dom f_R$.   Note that if $f:\alpha\to\Bbb N$ is injective, then
there is an $R\in\Cal R_{\alpha}$ such that
$f=f_R\restr\alpha$\prooflet{ (set
$R=\{(f(\beta),f(\gamma)):\gamma<\beta<\alpha\}
  \cup((\Bbb N\setminus f[\alpha])\times\Bbb N)$)}.

\medskip

{\bf (b)} Let $\maltese$ be the family of group operations $\star$ on 
$\Bbb N$.   We are going to need the natural Borel structure on
$\maltese$ corresponding to the identification of each $\star\in\maltese$,
which is a function from $\Bbb N\times\Bbb N$ to $\Bbb N$, with the set

\Centerline{$\{(i,j,k):i\star j=k\}\subseteq\BbbN^3$.}

\noindent\cmmnt{ So we can think of $\maltese$ as the set of subsets 
$\star$ of $\BbbN^3$ such that

\inset{for all $i$, $j\in\Bbb N$ there is just one $k\in\Bbb N$ such that
$(i,j,k)\in\star$,

if $i$, $j$, $k$, $l$, $m$, $n\in\Bbb N$, and $(i,j,l)$,
$(l,k,n)$, $(j,k,m)$ belong to $\star$, then $(i,m,n)\in\star$,

$(0,i,i)$, $(i,0,i)\in\star$ for every $i\in\Bbb N$,

for every $i\in\Bbb N$ there is a $j\in\Bbb N$ such that $(i,j,0)$ and
$(j,i,0)$ belong to $\star$.}

\noindent} For $\star\in\maltese$, let $\action^{\star}_r$ be the 
corresponding
right shift action of $\Bbb N$ on $\BbbR^{\Bbb N}$, so that
$(m\action^{\star}_rz)(i)=z(i\star m)$
whenever $z\in\BbbR^{\Bbb N}$ and $i$, $m\in\Bbb N$.

\medskip

{\bf (c)} Let $G$ be a group, of cardinal $\omega_1$, with
identity $\iota$;  let $\ofamily{\alpha}{\omega_1}{\pi_{\alpha}}$
enumerate $G$, with $\pi_0=\iota$.   Let $F\subseteq\omega_1$ be the
set of those $\alpha$ such that $G_{\alpha}=\{\pi_{\beta}:\beta<\alpha\}$
is a subgroup of $G$.
%note $1=\min F$
For $\alpha\in F$, set

$$\eqalign{\Cal S_{\alpha}
&=\{(R,\star):R\in \Cal R_{\alpha},\,\star\in\maltese\text{ and }
f_R(\beta)\star f_R(\gamma)=f_R(\delta)\cr
&\mskip160mu
\text{whenever }\beta,\,\gamma,\,\delta<\alpha\text{ and }
\pi_{\beta}\pi_{\gamma}=\pi_{\delta}\}\dvro{.}{;}\cr}$$

\cmmnt{\noindent so that if $(R,\star)\in\Cal S_{\alpha}$,
$f_R\restr\alpha$ codes a group homomorphism from
$G_{\alpha}$ to $(\Bbb N,\star)$.}

\medskip

{\bf (d)} For $\alpha\in F$, set

\Centerline{$\Cal M_{\alpha}
=\{(R,\star,z):(R,\star)\in\Cal S_{\alpha}$, $z\in\BbbR^{\Bbb N}\}$.}

\noindent Then
we have an action $\action'_{\alpha}$ of $G_{\alpha}$ on $\Cal M_{\alpha}$
defined by saying that

\Centerline{$\pi_{\beta}\action'_{\alpha}(R,\star,z)
=(R,\star,f_R(\beta)\action^{\star}_rz)$}

\noindent whenever $(R,\star)\in\Cal S_{\alpha}$, $z\in\BbbR^{\Bbb N}$
and $\beta<\alpha$.   \prooflet{\Prf\
$\action'_{\alpha}$ is well-defined as a
function on $G_{\alpha}\times\Cal M_{\alpha}$ because
$f_R(\beta)=f_R(\gamma)$ whenever $(R,\star)\in\Cal S_{\alpha}$ and
$\pi_{\beta}=\pi_{\gamma}$.
If $\beta$, $\gamma$, $\delta<\alpha$ and
$\pi_{\delta}=\pi_{\beta}\pi_{\gamma}$, then

$$\eqalignno{
\pi_{\beta}\action'_{\alpha}(\pi_{\gamma}\action'_{\alpha}(R,\star,z))
&=\pi_{\beta}\action'_{\alpha}(R,\star,f_R(\gamma)\action^{\star}_rz)\cr
&=(R,\star,f_R(\beta)\action^{\star}_r(f_R(\gamma)\action^{\star}_rz))
=(R,\star,f_R(\delta)\action^{\star}_rz)\cr
\displaycause{because $(f_R(\beta),f_R(\gamma),f_R(\delta))\in\star$}
&=\pi_{\delta}\action'_{\alpha}(R,\star,z),\cr
\iota\action'_{\alpha}(R,\star,z)
&=\pi_0\action'_{\alpha}(R,\star,z)
=(R,\star,f_R(0)\action^{\star}_rz)
=(R,\star,z).  \text{ \Qed}\cr}$$
}

\medskip

{\bf (e)} If $\alpha$, $\beta\in F$ and $\alpha\le\beta$, then\cmmnt{ it
is elementary to check that}
$\Cal R_{\beta}\subseteq \Cal R_{\alpha}$,
$\Cal S_{\beta}\subseteq\Cal S_{\alpha}$,
$\Cal M_{\beta}\subseteq\Cal M_{\alpha}$ and
$\pi\action'_{\beta}(R,\star,z)=\pi\action'_{\alpha}(R,\star,z)$ whenever
$(R,\star,z)\in\Cal M_{\beta}$ and $\pi\in G_{\alpha}$.
If $\beta\in F$ and $\beta=\sup(\beta\cap F)$, then
$\Cal R_{\beta}=\bigcap_{\alpha\in\beta\cap F}\Cal R_{\alpha}$ and
$\Cal M_{\beta}=\bigcap_{\alpha\in\beta\cap F}\Cal M_{\alpha}$.

\medskip

{\bf (f)(i)} For $\alpha<\omega_1$, $\Cal R_{\alpha}$ belongs to
the Borel $\sigma$-algebra $\Cal B(\Cal P\BbbN^2)$\cmmnt{ when
$\Cal P(\BbbN^2)$ is
given its usual compact Hausdorff topology\cmmnt{ (4A2Ud)}},
and\cmmnt{ moreover}
$\{(R,m):R\subseteq\BbbN^2$, $(\beta,m)\in f_R\}
\in\Cal B(\Cal P\BbbN^2\times\Bbb N)$ whenever $m\in\Bbb N$ and
$\beta<\alpha$.   \prooflet{\Prf\
Induce on $\alpha$.   We start with $\Cal R_0=\Cal P(\BbbN^2)$.
For the inductive step to a successor ordinal $\alpha+1$, given
$R\subseteq\BbbN^2$ and $m\in\Bbb N$, then

$$\eqalign{(\alpha,m)\in f_R
&\iff R\in\Cal R_{\alpha}\text{ and }m\text{ is the unique member of }
\BbbN\cr
&\mskip100mu\text{ such that }R[\{m\}]=f_R[\alpha]\cr
&\iff R\in\Cal R_{\alpha},\,R[\{m\}]\ne R[\{n\}]\text{ for every }
  n\ne m,\cr
&\mskip100mu\Forall i\in\Bbb N,\,(m,i)\in R\iff\Exists\beta<\alpha,
\,(\beta,i)\in f_R.\cr}$$

\noindent So $\{(R,m):(\alpha,m)\in f_R\}$ is

$$\eqalign{(\Cal R_{\alpha}\times\Bbb N)
&\cap\bigcap_{n\in\Bbb N}\bigcup_{i\in\Bbb N}
  \{(R,m):n=m\text{ or }(m,i)\in R\,\&\,(n,i)\notin R
  \text{ or }(m,i)\notin R\,\&\,(n,i)\in R\}\cr
&\cap\bigcap_{i\in\Bbb N}\bigcup_{\beta<\alpha}
  \{(R,m):(m,i)\notin R\text{ or }(\beta,i)\in f_R\}\cr
&\cap\bigcap_{i\in\Bbb N}\bigcap_{\beta<\alpha}
  \{(R,m):(m,i)\in R\text{ or }(\beta,i)\notin f_R\},\cr}$$

\noindent and is a Borel set.   Now

\Centerline{$\Cal R_{\alpha+1}
=\bigcup_{m\in\Bbb N}\{R:(\alpha,m)\in f_R\}
\in\Cal B(\Cal P\BbbN^2)$.}

\noindent For the inductive step to a countable limit
ordinal $\alpha>0$,
$\Cal R_{\alpha}=\bigcap_{\gamma<\alpha}\Cal R_{\gamma}$.\ \Qed}

\medskip

\quad{\bf (ii)} $\maltese$ is a Borel subset of $\Cal P(\BbbN^3)$,\cmmnt{ so}
$\Cal S_{\alpha}$ is a Borel subset of
$\Cal P(\BbbN^2)\times\Cal P(\BbbN^3)$,
and $\Cal M_{\alpha}$ is a Borel subset of
$\Cal P(\BbbN^2)\times\Cal P(\BbbN^3)\times\BbbR^{\Bbb N}$, for every
$\alpha\in F$.   \cmmnt{Next,}
$(R,\star,z)\mapsto\pi\action'_{\alpha}(R,\star,z):
  \Cal M_{\alpha}\to\Cal M_{\alpha}$ is Borel
measurable whenever $\alpha\in F$ and $\pi\in G_{\alpha}$.
\prooflet{\Prf\ Let
$\beta<\alpha$ be such that $\pi=\pi_{\beta}$.   If
$E_0\subseteq\Cal P(\BbbN^2)\times\Cal P(\BbbN^3)$ is a Borel set, and
$E=\{(R,\star,z):(R,\star,z)\in\Cal M_{\alpha}$, $(R,\star)\in E_0\}$,
then

\Centerline{$\{(R,\star,z):\pi\action'_{\alpha}(R,\star,z)\in E\}=E$}

\noindent is Borel;  if $n\in\Bbb N$ and $E_1\subseteq\Bbb R$ is a Borel
set, and $E=\{(R,\star,z):(R,\star,z)\in\Cal M_{\alpha}$, $z(n)\in E_1\}$, then

$$\eqalign{\{(R,\star,z):\pi\action'_{\alpha}(R,\star,z)\in E\}
&=\{(R,\star,z):(f_R(\beta)\action^{\star}_rz)(n)\in E_1\}\cr
&=\bigcup_{i,j\in\Bbb N}\{(R,\star,z):f_R(\beta)=i,\,(n,i,j)\in\star,\,
   z(j)\in E_1\}\cr}$$

\noindent is Borel.   Since
$\Cal B(\Cal P\BbbN^2\times\Cal P\BbbN^3\times\BbbR^{\Bbb N})$ is the
product $\sigma$-algebra

\Centerline{$\Cal B(\Cal P\BbbN^2)\tensorhat\Cal B(\Cal P\BbbN^3)
  \tensorhat\Tensorhat_{\Bbb N}\Cal B(\Bbb R)$,}

\noindent $(R,\star,z)\mapsto\pi\action'_{\alpha}(R,\star,z)$ is 
Borel measurable.\ \Qed}

\leader{425D}{T\"ornquist's theorem}\cmmnt{ ({\smc T\"ornquist 11})}
Let $(X,\Sigma)$ be a standard Borel space and
$\Cal I$ a $\sigma$-ideal of $\Sigma$ containing an uncountable
set.   Let $\frak A$ be the quotient algebra $\Sigma/\Cal I$, and
$G\subseteq\Aut\frak A$ a subgroup of cardinal at most $\omega_1$.
Then there is an action $\action$ of $G$ on $X$ which represents $G$ in the
sense that $\pi\action E\cmmnt{\mskip5mu=\{\pi\action x:x\in E\}}$
belongs to $\Sigma$, and
$(\pi\action E)^{\ssbullet}=\pi(E^{\ssbullet})$,
for every $E\in\Sigma$ and $\pi\in G$.

\proof{{\bf (a)} It may help if I try to describe the line of argument I
mean to follow.   The important case is when $G$ has an enumeration
$\ofamily{\alpha}{\omega_1}{\pi_{\alpha}}$;  let
$F$ be the set of those $\alpha<\omega_1$ such that
$G_{\alpha}=\{\pi_{\beta}:\beta<\alpha\}$ is a subgroup of $G$.
For each $\pi\in G$, choose $g_{\pi}:X\to X$ representing $\pi$.   
For $\alpha\in F$, set

\Centerline{$Y_{\alpha}=\{x:x\in X$,
$g_{\phi}(g_{\pi}(x))=g_{\pi\phi}(x)\in X\setminus M$ for all $\pi$,
$\phi\in G_{\alpha}\}$,}

\noindent where $M$ is an uncountable member of $\Cal I$.
Choose $\family{\alpha}{F}{\action_{\alpha}}$ inductively such
that $\action_{\alpha}$ is an action of $G_{\alpha}$ on $X$,
$\pi\action_{\alpha}x=g_{\pi^{-1}}(x)$ whenever $\pi\in G_{\alpha}$ and
$x\in Y_{\alpha}$, and $\pi\action_{\alpha}x=\pi\action_{\beta}x$ whenever
$\beta<\alpha$, $\pi\in G_{\beta}$ and $x\in X$.   At the end of the
construction, set $\action=\bigcup_{\alpha\in F}\action_{\alpha}$.

It is straightforward to show that $X\setminus Y_{\alpha}$
always belongs to $\Cal I$ (part (c) of the proof);
consequently $\action$
will represent $G$.   The non-trivial part of the proof is in the
extension of a given action of $G_{\alpha}$ to an action of $G_{\beta}$
where $\beta$ is the next element of $F$ above $\alpha$, and this is where
we shall need 425B-425C.

Now for the details.

\medskip

{\bf (b)} Give $X$ a Polish topology for which $\Sigma$ is the Borel
$\sigma$-algebra $\Cal B(X)$.   
For nearly the whole of the proof (down to the end of
(g) below), suppose that $\#(G)=\omega_1$, that
$X=\{0,1\}^{\Bbb N}$ and that
$\Sigma=\Cal B(X)$ is the Borel $\sigma$-algebra of $X$.
Enumerate $G$ as
$\ofamily{\alpha}{\omega_1}{\pi_{\alpha}}$ starting with $\pi_0$ equal to
the identity $\iota$.   We need to know that, setting 
$G_{\alpha}=\{\pi_{\beta}:\beta<\alpha\}$, the set
$F=\{\alpha:\alpha<\omega_1$, $G_{\alpha}$ is a subgroup of $G\}$ is
a closed cofinal subset of $\omega_1$.   \Prf\ This is elementary.
If $\alpha\in\overline{F}$, then
$\{G_{\beta}:\beta\in F$, $\beta\le\alpha\}$ is a
non-empty upwards-directed family
of subgroups of $G$, so
$G_{\alpha}=\bigcup_{\beta\in F,\beta\le\alpha}G_{\beta}$ is a subgroup of
$G$, and $\alpha\in F$.   If $\alpha<\omega_1$, let $\sequencen{\alpha_n}$
be a non-decreasing sequence in $\omega_1$ such that
$\alpha_0=\max(1,\alpha)$ and, for each $n\in\Bbb N$,

\inset{----- for all $\beta$, $\gamma<\alpha_n$ there is a
$\delta<\alpha_{n+1}$ such that $\pi_{\delta}=\pi_{\beta}\pi_{\gamma}$,

----- for every $\beta<\alpha_n$ there is a $\delta<\alpha_{n+1}$ such that
$\pi_{\delta}=\pi_{\beta}^{-1}$.}

\noindent Setting $\alpha^*=\sup_{n\in\Bbb N}\alpha_n$, we see that
$\alpha\le\alpha^*\in F$.   So $F$ is cofinal with $\omega_1$.\ \Qed

Of course $1=\min F$ and $G_1=\{\iota\}$, because we started with
$\pi_0=\iota$.

\medskip

{\bf (c)} Next, for every $\pi\in\Aut\frak A$, we can choose a
Borel measurable $g_{\pi}:X\to X$ representing $\pi$ in the
sense that $\pi E^{\ssbullet}=g_{\pi}^{-1}[E]^{\ssbullet}$ for every
$E\in\Cal B(X)$ (425Ac).   Of course when $\pi=\iota$ we take
$g_{\iota}(x)=x$ for every $x\in X$.   Fix an uncountable $M\in\Cal I$, and
for $\alpha\in F$ set

\Centerline{$Y_{\alpha}=\{x:x\in X$,
$g_{\phi}(g_{\pi}(x))=g_{\pi\phi}(x)\in X\setminus M$ for all $\pi$,
$\phi\in G_{\alpha}\}$,}

\noindent as declared in (a).   If $\alpha\in F$ and $\pi\in G_{\alpha}$,
$X\setminus Y_{\alpha}\in\Cal I$ and
$g_{\pi}\restr Y_{\alpha}$ is a permutation of $Y_{\alpha}$.
\Prf\ (Compare the proof of 344B.)
There is a sequence $\sequence{k}{E_k}$ in $\Cal B(X)$ separating the
points of $X$.   So, for any $\psi$, $\phi\in G_{\alpha}$, the set

\Centerline{$\{x:g_{\psi}g_{\phi}(x)\ne g_{\phi\psi}(x)\}
=\bigcup_{k\in\Bbb N}g_{\phi}^{-1}[g_{\psi}^{-1}[E_k]]
   \symmdiff g_{\phi\psi}^{-1}[E_k]$}

\noindent is Borel, and moreover, transferring the formulae
to the quotient algebra,

$$\eqalign{\{x:g_{\psi}g_{\phi}(x)\ne g_{\phi\psi}(x)\}^{\ssbullet}
&=\sup_{k\in\Bbb N}g_{\phi}^{-1}[g_{\psi}^{-1}[E_k]]^{\ssbullet}
   \Bsymmdiff g_{\phi\psi}^{-1}[E_k]^{\ssbullet}\cr
&=\sup_{k\in\Bbb N}\phi(\psi E_k^{\ssbullet})
   \Bsymmdiff(\phi\psi)E_k^{\ssbullet}
=0,\cr}$$

\noindent so $\{x:g_{\psi}g_{\phi}(x)\ne g_{\phi\psi}(x)\}\in\Cal I$.
Accordingly

\Centerline{$X\setminus Y_{\alpha}
=\bigcup_{\psi,\phi\in G_{\alpha}}
     \{x:g_{\psi}g_{\phi}(x)\ne g_{\phi\psi}(x)\}
   \cup\bigcup_{\psi\in G_{\alpha}}g_{\psi}^{-1}[M]$}

\noindent belongs to $\Cal I$.

If $\pi\in G_{\alpha}$ and $x\in Y_{\alpha}$, then

\Centerline{$g_{\psi}g_{\phi}g_{\pi}(x)=g_{\psi}g_{\pi\phi}(x)
  =g_{\pi\phi\psi}(x)=g_{\phi\psi}g_{\pi}(x)$}

\noindent and $g_{\pi\phi\psi}(x)\notin M$, for all
$\phi$, $\psi\in G_{\alpha}$;  so $g_{\pi}(x)\in Y_{\alpha}$.
Similarly, $g_{\pi^{-1}}[Y_{\alpha}]\subseteq Y_{\alpha}$.
Moreover,

\Centerline{$g_{\pi}g_{\pi^{-1}}(x)=g_{\iota}(x)=x
=g_{\pi^{-1}}g_{\pi}(x)$}

\noindent for every $x\in Y_{\alpha}$, so that $g_{\pi}\restr Y_{\alpha}$
must be a permutation of $Y_{\alpha}$.\ \Qed

\medskip

{\bf (d)} From the group $G$ and the enumeration
$\ofamily{\alpha}{\omega_1}{\pi_{\alpha}}$, construct $\maltese$ and
families $\family{R}{\Cal P\BbbN^2}{f_R}$,
$\ofamily{\alpha}{\omega_1}{\Cal R_{\alpha}}$, 
$\family{\star}{\maltese}{\action^{\star}_r}$ and
$\family{\alpha}{F}{(\Cal S_{\alpha},\Cal M_{\alpha},\action'_{\alpha})}$
as in 425C.   Let $h$ be a Borel isomorphism from
$\Bbb R\times\Bbb N\times\Cal M_1$ to $M$ (424Da again).
Fix a family $\family{\delta}{F}{t_{\delta}}$ of distinct members of
$\Bbb R$, and set $J_{\alpha}=\Bbb R\setminus\{t_{\delta}:\delta<\alpha\}$
for $\alpha\in F$.

\medskip

{\bf (e)} (The key.  Some readers may wish at this point to provide
themselves with
coffee and a large scratch-pad.)  Let $\alpha<\beta<\gamma$ be members of 
$F$, and suppose that
$\action_0$ is an action of $G_{\alpha}$ on $X$ such that

\inset{$x\mapsto\pi\action_0x$ is Borel measurable
for every $\pi\in G_{\alpha}$,

$\pi\action_0x=g_{\pi^{-1}}(x)$ whenever $x\in Y_{\alpha}$ and
$\pi\in G_{\alpha}$,

$\pi\action_0h(t,n,q)=h(t,n,\pi\action'_{\beta}q)$ whenever
$t\in J_{\alpha}$,
$n\in\Bbb N$, $q\in\Cal M_{\beta}$ and $\pi\in G_{\alpha}$.}

\noindent Then there is an action $\action_1$ of $G_{\beta}$ on $X$ such
that

\inset{$x\mapsto\pi\action_1x$ is Borel measurable
for every $\pi\in G_{\beta}$,

$\pi\action_1x=g_{\pi^{-1}}(x)$ whenever $x\in Y_{\beta}$ and
$\pi\in G_{\beta}$,

$\pi\action_1h(t,n,q)=h(t,n,\pi\action'_{\gamma}q)$ whenever
$t\in J_{\beta}$, $n\in\Bbb N$, $q\in\Cal M_{\gamma}$ and
$\pi\in G_{\beta}$,

$\pi\action_1x=\pi\action_0x$ whenever $\pi\in G_{\alpha}$ and $x\in X$.}

\noindent\Prf\ Choose $\star\in\maltese$ such that
there is an injective group homomorphism $\theta$ from
$G_{\beta}$ to $(\Bbb N,\star)$,
and set $f'(\delta)=\theta(\pi_{\delta})$ for
$\delta<\beta$;  let $R\in \Cal R_{\beta}$ be such that
$f_R\restr\beta=f'$.   (In the normal case,
when $\beta\ge\omega$, we can choose $f'$
first, as an arbitrary bijection from $\beta$ to $\Bbb N$, and use this to
define $\theta$, $\star$ and $R$.   If $\beta<\omega$, the first step is to
take an injective group homomorphism from $G_{\beta}$ to a countably
infinite group, e.g., $G_{\beta}\times\Bbb Z$.)   
Then $(R,\star)\in\Cal S_{\beta}$.
Define $h_0:\Bbb N\times\BbbR^{\Bbb N}\to M$ by setting
$h_0(n,z)=h(t_{\alpha},n,(R,\star,z))$ for $n\in\Bbb N$ and
$z\in\BbbR^{\Bbb N}$.   Then $h_0$ is injective and Borel
measurable, and if $\pi\in G_{\alpha}$ then

$$\eqalignno{h_0(n,\theta(\pi)\action^{\star}_rz)
&=h(t_{\alpha},n,(R,\star,\theta(\pi)\action^{\star}_rz))
=h(t_{\alpha},n,(R,\star,f_{R}(\delta)\action^{\star}_rz))\cr
\displaycause{where $\delta<\alpha$ is such that $\pi=\pi_{\delta}$}
&=h(t_{\alpha},n,\pi\action'_{\alpha}(R,\star,z))
=h(t_{\alpha},n,\pi\action'_{\beta}(R,\star,z))\cr
&=\pi\action_0h(t_{\alpha},n,(R,\star,z))
=\pi\action_0h_0(n,z)
\cr}$$

\noindent for every $n\in\Bbb N$ and $z\in\BbbR^{\Bbb N}$.
Set

\Centerline{$V=h[J_{\beta}\times\Bbb N\times\Cal M_{\gamma}]
\subseteq M\subseteq X\setminus Y_{\beta}$,
\quad$X'=X\setminus(Y_{\beta}\cup V)$.}

\noindent  Note that
if $\pi\in G_{\alpha}$, $t\in J_{\beta}$, $n\in\Bbb N$ and
$q\in\Cal M_{\gamma}$, then

\Centerline{$\pi\action_0h(t,n,q)
=h(t,n,\pi\action'_{\beta}q)=h(t,n,\pi\action'_{\gamma}q)\in V$;}

\noindent thus $V$ is invariant under the action $\action_0$.
The same is true of $Y_{\beta}$, because 
$g_{\pi^{-1}}\restr Y_{\beta}$ is
a permutation of $Y_{\beta}$ for every $\pi\in G_{\alpha}$,
therefore also of $X'$.   Note that as $t_{\alpha}\notin J_{\beta}$,
$h_0[\Bbb N\times\BbbR^{\Bbb N}]\subseteq X'$, while
there is certainly a Borel measurable injection from
$X'\setminus h_0[\Bbb N\times\BbbR^{\Bbb N}]$ into $\Bbb R$.   So
425Bb tells us that there is an action $\hat{\action}_1$ of $G_{\beta}$
on $X'$, extending $\action_0\restr G_{\alpha}\times X'$, such that
$x\mapsto\pi\hat{\action}_1x:X'\to X'$ is
Borel measurable for every $\pi\in G_{\beta}$.

We can therefore define $\action_1$ by setting

$$\eqalign{\pi\action_1x
&=g_{\pi^{-1}}(x)\text{ if }x\in Y_{\beta},\cr
&=h(t,n,\pi\action'_{\gamma}q)\text{ whenever }t\in J_{\beta},\,
   n\in\Bbb N,\,q\in\Cal M_{\gamma}\text{ and }x=h(t,n,q),\cr
&=\pi\hat{\action}_1x\text{ if }x\in X'\cr}$$

\noindent for every $\pi\in G_{\beta}$.   It is easy to check that
$\action_1$ is a function from $G_{\beta}\times X$ to $X$ extending
$\action_0$, and that $x\mapsto\pi\action_1x$ is
Borel measurable for every $\pi\in G_{\beta}$.   If $\pi$,
$\phi\in G_{\beta}$ and $x\in X$, then

$$\eqalign{\pi\action_1(\phi\action_1x)
&=\pi\action_1g_{\phi^{-1}}(x)
=g_{\pi^{-1}}g_{\phi^{-1}}(x)
=g_{(\pi\phi)^{-1}}(x)
=(\pi\phi)\action_1x\text{ if }x\in Y_{\beta},\cr
&=\pi\action_1h(t,n,\phi\action'_{\gamma}q)
=h(t,n,\pi\action'_{\gamma}(\phi\action'_{\gamma}q))
=h(t,n,(\pi\phi)\action'_{\gamma}q)\cr
&\mskip100mu=(\pi\phi)\action_1h(t,n,q)
=(\pi\phi)\action_1x\cr
&\mskip50mu\text{ whenever }t\in J_{\beta},\,
   n\in\Bbb N,\,q\in\Cal M_{\gamma}\text{ and }x=h(t,n,q),\cr
&=\pi\hat{\action}_1(\phi\hat{\action}_1x)
=(\pi\phi)\hat{\action}_1x
=(\pi\phi)\action_1x
\text{ if }x\in X'.\cr}$$

\noindent So $\action_1$ is an action of $G_{\beta}$ on $X$, as required.\
\Qed

\medskip

{\bf (f)} Accordingly we can build $\family{\alpha}{F}{\action_{\alpha}}$
inductively, as follows.   The
inductive hypothesis will be that, for each $\alpha\in F$,

\inset{$\action_{\alpha}$ is an action of $G_{\alpha}$ on $X$,

$x\mapsto\pi\action_{\alpha}x$ is Borel measurable for every
$\pi\in G_{\alpha}$,

$\pi\action_{\alpha}x=g_{\pi^{-1}}(x)$ whenever $x\in Y_{\alpha}$ and
$\pi\in G_{\alpha}$,

$\pi\action_{\alpha}h(t,n,q)=h(t,n,\pi\action'_{\beta}q)$ whenever
$t\in J_{\alpha}$, $n\in\Bbb N$, $q\in\Cal M_{\beta}$,
$\beta\in F$, $\beta>\alpha$ and $\pi\in G_{\alpha}$,

$\action_{\delta}=\action_{\alpha}\restr G_{\delta}\times X$ whenever
$\delta\in F\cap\alpha$.
}

\noindent The induction starts with $G_1=\{\iota\}$ and $\iota\action_1x=x$
for every $x\in X$.

Given $\alpha\in F$ and $\action_{\alpha}$, let $\beta$ be the next
element of $F$ above $\alpha$ and $\gamma$ the next element of $F$ above
$\beta$.   By (d), we have an action
$\action_{\beta}$ of $G_{\beta}$ on $X$ such that

\inset{$x\mapsto\pi\action_{\beta}x$ is Borel measurable for every
$\pi\in G_{\beta}$,

$\pi\action_{\beta}x=g_{\pi^{-1}}(x)$ whenever $x\in Y_{\beta}$ and
$\pi\in G_{\beta}$,

$\pi\action_{\beta}h(t,n,q)=h(t,n,\pi\action'_{\gamma}q)$ whenever
$t\in J_{\beta}$, $n\in\Bbb N$, $q\in\Cal M_{\gamma}$ and
$\pi\in G_{\alpha}$,

$\action_{\alpha}=\action_{\beta}\restr G_{\alpha}\times X$.}

\noindent It follows at once that if $\delta\in F$ and $\delta\le\alpha$,

\Centerline{$\pi\action_{\delta}x=\pi\action_{\alpha}x
=\pi\action_{\beta}x$}

\noindent whenever $\pi\in G_{\delta}$ and $x\in X$;  on the other side,
if $\delta\in F$ and $\delta\ge\gamma$,

\Centerline{$\pi\action_{\beta}h(t,n,q)=h(t,n,\pi\action'_{\gamma}q)
=h(t,n,\pi\action'_{\delta}q)$}

\noindent whenever $t\in J_{\beta}$, $n\in\Bbb N$, $q\in\Cal M_{\delta}$
and $\pi\in G_{\beta}$.   So the induction continues to the next step.

If $\alpha\in F$ and $\alpha=\sup(F\cap\alpha)$, then
$G_{\alpha}=\bigcup_{\beta\in F\cap\alpha}G_{\beta}$, so we have an action
$\action_{\alpha}=\bigcup_{\beta\in F\cap\alpha}\action_{\beta}$ of
$G_{\alpha}$ on $X$;  and it is elementary to check that the inductive
hypothesis is satisfied at the new level.

\medskip

{\bf (g)} At the end of the induction,
$\action=\bigcup_{\alpha\in F}\action_{\alpha}$ will be an action of $G$ on
$X$ such that
$x\mapsto\pi\action x$ is Borel measurable for every
$\pi\in G$.   Moreover,
$\pi E^{\ssbullet}=(\pi\action E)^{\ssbullet}$ for every Borel set
$E\subseteq X$ and
$\pi\in G$.   \Prf\ Let $\alpha\in F$ be such that $\pi\in G_{\alpha}$.
Then

$$\eqalignno{\pi(E^{\ssbullet})
&=\pi((E\cap Y_{\alpha})^{\ssbullet})\cr
\displaycause{because $X\setminus Y_{\alpha}\in\Cal I$}
&=(g_{\pi}^{-1}[E\cap Y_{\alpha}])^{\ssbullet}
=(Y_{\alpha}\cap g_{\pi}^{-1}[E\cap Y_{\alpha}])^{\ssbullet}
=(g_{\pi^{-1}}[E\cap Y_{\alpha}])^{\ssbullet}\cr
\displaycause{because $g_{\pi}\restr Y_{\alpha}$ is a permutation with
inverse $g_{\pi^{-1}}\restr Y_{\alpha}$}
&=(\pi\action(E\cap Y_{\alpha}))^{\ssbullet}
\Bsubseteq(\pi\action E)^{\ssbullet}.\cr}$$

\noindent(Because $x\mapsto\pi\action x$ is a Borel measurable
permutation of $X$, $\pi\action E$ is certainly a Borel set.)
Since equally we must have

\Centerline{$\pi((X\setminus E)^{\ssbullet})
\Bsubseteq(\pi\action(X\setminus E))^{\ssbullet}$,}

\noindent while
$\pi(E^{\ssbullet})\Bcup\pi((X\setminus E)^{\ssbullet})=1_{\frak A}$ and
$(\pi\action E)^{\ssbullet}
  \Bcap(\pi\action(X\setminus E))^{\ssbullet}=0_{\frak A}$,
both the inclusions here are equalities, and
$\pi E^{\ssbullet}=(\pi\action E)^{\ssbullet}$.\ \Qed

\medskip

{\bf (h)} As for the elementary case in which $G$ is countable, we can use
arguments already presented, as follows.   For each $\pi\in G$,
choose $g_{\pi}$ representing $\pi$.   This time, go
straight to
$Y=\{x:g_{\pi\phi}(x)=g_{\phi}g_{\pi}(x)$ for all $\pi$, $\phi\in G\}$;
as in (c), $X\setminus Y\in\Cal I$ and
$g_{\pi}\restr Y$ is a permutation of $Y$ for every $\pi\in G$.
So if we set

$$\eqalign{\pi\action x
&=g_{\pi^{-1}}(x)\text{ for }x\in Y,\cr
&=x\text{ for }x\in X\setminus Y,\cr}$$

\noindent $\action$ will be an appropriate action of $G$ on $X$.
}%end of proof of 425D

\leader{425E}{Scholium}
The theorem here applies to groups of cardinal at
most $\omega_1$.   So it is worth noting that in the context of 425D the
whole group $\Aut\frak A$ has cardinal at most $\frak c$.
\prooflet{\Prf\ Since
$\Sigma$ is countably $\sigma$-generated, there is a countable set
$D\subseteq\frak A$ countably $\sigma$-generating $\frak A$.   If $\pi$,
$\phi\in\Aut\frak A$ and $\pi\restr D=\phi\restr D$, then $\pi=\phi$;  so
$\#(\Aut\frak A)\le\#(\frak A^D)$.   As
$\#(\frak A)\le\#(\Sigma)\le\frak c$ (424Db), $\#(\Aut\frak A)$ is at most
$\frak c$ (4A1A(c-ii)).\ \Qed}

We therefore have\cmmnt{ a corollary of 425D, as follows:}

\inset{Suppose the continuum hypothesis is true.   Let $(X,\Sigma)$ be a
standard Borel space and
$\Cal I$ a $\sigma$-ideal of subsets of $\Sigma$ containing an uncountable
set.   Then there is an action $\action$ of
$\Aut(\Sigma/\Cal I)$ on $X$ such that
$\pi E^{\ssbullet}=(\pi\action E)^{\ssbullet}$
whenever $E\in\Sigma$ and $\pi\in\Aut(\Sigma/\Cal I)$.}

\exercises{\leader{425X}{Basic exercises (a)}
%\sqheader 425Xa
(Cf.\ 382Xc.)
Let $(X,\Sigma)$ be a standard Borel space, $\Cal I$ a
$\sigma$-ideal of subsets of $X$ and $\frak A=\Sigma/\Cal I$ the quotient
algebra.   (i) Show that every member of
$\Aut\frak A$ has a separator (definition:  382Aa).   (ii) Show that
if $G$ is a countably full subgroup of $\Aut\frak A$, then every
member of $G$ is expressible as the product of at most three involutions
belonging to $G$.
%425- 382M

\spheader 425Xb Let $(X,\Sigma)$ be a standard Borel space and set
$\frak A=\Sigma/[X]^{\le\omega}$.
(i)\dvAformerly{4{}24Xk} Show that the Boolean
algebra $\frak A$ is homogeneous (definition:  316N).
(ii) Show that $\Aut\frak A$ is simple.
\Hint{382Yc.}
%425Xa

\spheader 425Xc Let $(X,\Sigma)$ be a standard Borel
space and $\Cal I$ a $\sigma$-subalgebra of $\Sigma$ with associated
quotient algebra $\frak A=\Sigma/\Cal I$.
Suppose that $G$ is a countable semigroup of sequentially
order-continuous Boolean homomorphisms from $\frak A$ to itself.
Show that there is a family
$\family{\pi}{G}{f_{\pi}}$ of $(\Sigma,\Sigma)$-measurable functions from
$X$ to itself such that ($\alpha$)
$\pi E^{\ssbullet}=f_{\pi}^{-1}[E]^{\ssbullet}$
for every $\pi\in G$ and $E\in\Sigma$ ($\beta$)
$f_{\pi\phi}=f_{\phi}f_{\pi}$ for all $\pi$, $\phi\in G$.
%425A
%should be early in 425X, see 425Xd

\spheader 425Xd Let $X$ be a set,
$\Sigma$ a $\sigma$-algebra of subsets of $X$, $\Cal I$ a $\sigma$-ideal of
$\Sigma$ and $\frak A$ the quotient $\Sigma/\Cal I$.
Suppose that
$(X,\Sigma)$ is countably separated in the sense that there is a
countable subset of $\Sigma$ separating the points of $X$.
Let $G$ be a countable subsemigroup of the semigroup of
Boolean homomorphisms from $\frak A$ to itself such that
for every $\pi\in G$
there is a $(\Sigma,\Sigma)$-measurable $g:X\to X$ such that
$\pi E^{\ssbullet}=g^{-1}[E]^{\ssbullet}$ for every $E\in\Sigma$.
Show that there is a
family $\family{\pi}{G}{f_{\pi}}$ of $(\Sigma,\Sigma)$-measurable functions
from $X$ to itself such that
$f_{\pi}^{-1}[E]^{\ssbullet}=\pi E^{\ssbullet}$ and
$f_{\pi\phi}=f_{\phi}f_{\pi}$ whenever $\pi$, $\phi\in G$ and $E\in\Sigma$.
\Hint{344B.}
%425A

\spheader 425Xe Show that the set $\maltese$ in 425Cb is a G$_{\delta}$ set
in the compact metrizable space $\Cal P(\BbbN^3)$.
%425C

\spheader 425Xf Give an example of a standard Borel space $(X,\Sigma)$, a
$\sigma$-ideal $\Cal I$ of $\Sigma$, a finite subgroup $G$ of
$\Aut(\Sigma/\Cal I)$, a subgroup $H$ of $G$ and an action $\action_0$ of
$H$ on $X$ such that
$\pi E^{\ssbullet}=(\pi\action_0E)^{\ssbullet}$ whenever
$\pi\in H$ and $x\in X$, but there is no action $\action$ of $G$ on $X$,
extending $\action_0$, such that
$\pi E^{\ssbullet}=(\pi\action E)^{\ssbullet}$ whenever
$\pi\in G$ and $x\in X$.  \Hint{$\#(X)=6$, $\#(H)=2$.}
%\Aut\frak A\cong S_4, H\subseteq A_4 mt42bits
%425D

\spheader 425Xg Let $I^{\|}$ be the split interval with its usual topology
and measure, and $\frak A$ its measure algebra.   Let $G$ be a subgroup of
$\Aut\frak A$ of cardinal at most $\omega_1$.   Show that there is
an action $\action$ of $G$ on $I^{\|}$ such that
$\pi\action E$ is a Borel set and
$(\pi\action E)^{\ssbullet}=\pi E^{\ssbullet}$
for every $E\in\Cal B(I^{\|})$ and $\pi\in G$.
%425D take \pi\action(t^{+/-})=(\pi\action't)^{+/-}

\leader{425Y}{Further exercises (a)}
%\spheader 425Ya
({\smc T\"ornquist 11}) Let $X$ be a set,
$\Sigma$ a $\sigma$-algebra of subsets of $X$, $\Cal I$ a $\sigma$-ideal of
$\Sigma$ and $\frak A$ the quotient $\Sigma/\Cal I$.   Suppose that
$(X,\Sigma,\Cal I)$ is countably separated in the sense that there is a
countable subset of $\Sigma$ separating the points of $X$, and complete in
the sense that $A\in\Cal I$ whenever $A\subseteq B\in\Cal I$.
Let $G\subseteq\Aut\frak A$ be a group of size
at most $\omega_1$ such that for every $\pi\in G$ there is a
$(\Sigma,\Sigma)$-measurable function $g:X\to X$ such that
$\pi E^{\ssbullet}=g^{-1}[E]^{\ssbullet}$ for every $E\in\Sigma$.
Show that if $\Cal I$ contains a set of cardinal $\frak c$
there is an action $\action$ of $G$ on $X$ such that
$\pi E^{\ssbullet}=(\pi\action E)^{\ssbullet}$ for every $\pi\in G$ and
$E\in\Sigma$.
%425D mt42bits

\spheader 425Yb
Let $(X,\Sigma,\mu)$ be a countably separated
perfect complete strictly localizable measure space,
$\frak A$ its measure algebra and $G$ a subgroup of $\Aut\frak A$ of
cardinal at most $\omega_1$.   Show that there is an action
$\action$ of $G$ on $X$ such that
$\pi\action E\in\Sigma$ and
$(\pi\action E)^{\ssbullet}=\pi(E^{\ssbullet})$ whenever
$\pi\in G$ and $E\in\Sigma$.
%425Ya 425D

}%end of exercises

\leader{425Z}{Problems (a)} %425Za
Suppose that $(X,\Sigma)$ is an uncountable
standard Borel space and $\Cal I$ the ideal $[X]^{\le\omega}$\cmmnt{ of
countable
subsets of $X$}.   Which subgroups $G$ of $\Aut(\Sigma/\Cal I)$ can be
represented by actions of $G$ on $X$?

\spheader 425Zb Let $(\frak A,\bar\mu)$ be the measure algebra of Lebesgue
measure $\mu$ on $[0,1]$, and $G$ a semigroup of measure-preserving Boolean
homomorphisms from $\frak A$ to itself with $\#(G)=\omega_1$.
Must there be a family $\family{\pi}{G}{f_{\pi}}$ of \imp\ functions from
$[0,1]$ to itself such that $f_{\pi\phi}=f_{\phi}f_{\pi}$ for all $\pi$,
$\phi\in G$ and $f_{\pi}$ represents $\pi$, in the sense
of 425A, for every $\pi\in G$?   \cmmnt{(See 344B.)}

\spheader 425Zc Let $(\frak B_{\omega_1},\bar\nu_{\omega_1})$ be the
measure algebra of the usual measure on $\{0,1\}^{\omega_1}$, and $G$ a
group of measure-preserving automorphisms of $\frak B_{\omega_1}$ with
$\#(G)=\omega_1$.   Must there be a family $\family{\pi}{G}{f_{\pi}}$ of
\imp\ functions from
$\{0,1\}^{\omega_1}$ to itself such that $f_{\pi\phi}=f_{\phi}f_{\pi}$ for
all $\pi$,
$\phi\in G$ and $f_{\pi}$ represents $\pi$ for every $\pi\in G$?
\cmmnt{(See 344E.)}

\endnotes{
\Notesheader{425} I have starred this section because (apart from 425A)
it deals with a very
special topic.   From the point of view of measure theory, 425D 
applies only
to copies of Borel measures on $\Bbb R$ (and not quite all of those),
and the limitation to groups of
cardinal $\omega_1$ means that we need to assume the continuum hypothesis,
or at least $\frak p=\frak c$ (535Yd\Latereditions),
to get a theorem we really want.   However the result
connects naturally with an
important theme from Chapter 34, and the general question of simultaneous
representation of many automorphisms has significant implications for
the ergodic theory treated in Chapter 38 and \S494 of this volume.

What makes 425D difficult is the ambitious target:  we
want to represent the automorphisms in $G$ by
a consistent family of Borel measurable
functions.   We know from Chapter 34 that we can hope to handle countable
groups, so it is natural to start by expressing $G$ as an inductive limit
of a family $\family{\alpha}{F}{G_{\alpha}}$, as in parts (a)-(b) of the
proof of 425D, and to try to define the action of $G$ from actions of the
$G_{\alpha}$.   Since any $\pi\in G$ must eventually determine a
Borel measurable function on $X$, we are going to have to
freeze its action at some point;  we could afford to change it once or
twice, or even countably often, as the induction continued, but sometime
we must stop tinkering, and really we want to have
$\pi\action x=\pi\action_{\alpha}x$ whenever $x\in X$ and
$\pi\in G_{\alpha}$.   In this case, $\action_{\beta}$ will have to be a
direct extension of $\action_{\alpha}$ whenever $\beta>\alpha$.
We are in a context in which arbitrary actions are not always
extensible (425Xf), and something like Lemma 425B is going to be needed.
This demands a plentiful supply of copies of shift actions,
at the very least including representations of all the shift actions on
$X^{G_{\alpha}}$, which will have to be built in from the very beginning.
The trouble is that these have to be assembled in a way which will
give us Borel measurable functions.   Now while $X$ can be taken to be a
fixed Polish space, the groups $G_{\alpha}$ are more or less arbitrary
countable groups.   They can all be represented by group structures on
$\Bbb N$, but if we go by that road, we seem to need to choose
injective functions from countable ordinals into $\Bbb N$.   (Of course
each $G_{\alpha}$ comes with a bijection between it and the ordinal
$\alpha$.)
No direct enumeration of these is going to lead to Borel structures.
Instead, we have to look at the whole set of group structures on
$\Bbb N$, the set $\maltese$ of 425Cb, and nearly all injective
functions from countable ordinals to $\Bbb N$, coded by subsets of
$\BbbN^2$, as in 425Ca.   Fortunately it does not matter that there is a
great deal of redundancy in this coding;  it can all be fitted naturally
into a standard Borel structure, and we just need to include, in the
hypotheses of 425D, a negligible set of size $\frak c$.
When $\Cal I\subseteq[X]^{\le\omega}$, the problem changes (425Za).

In order to ensure that each of the actions $\action_{\alpha}$ correctly
represents the action of $G_{\alpha}$ on $\frak A$, we can use essentially
the same method as that of 344B (part (c) of the proof of 425D).
This means, of course, that 425B has to be applied to a carefully chosen
fragment of $X$, the set $X'$ of part (d) of the proof of 425D.
There is an awkward shift here between the representation of an
automorphism $\pi$ by the function $g_{\pi}$, where I follow the
conventions used in Chapter 34 and 425A here with the contravariant formula
$\pi(E^{\ssbullet})=(g_{\pi}^{-1}[E])^{\ssbullet}$,
and the representation in the statement of this theorem, 
with the covariant formula
$\pi E^{\ssbullet}=(\pi\action E)^{\ssbullet}$.   The latter is
forced by the rule of 4A5Ba that
$(\pi\phi)\action x=\pi\action(\phi\action x)$.   By allowing `reverse
actions', in which $(\pi\phi)\action x=\phi\action(\pi\action x)$, we could
escape this conflict, and in the formulae of parts (d)-(e)
of the proof of
425D we should have $\pi\action_{\alpha}x=g_{\pi}(x)$ for $x\in Y_{\alpha}$
and $\pi\in G_{\alpha}$.   This would be essential if we wanted to use
the formulae here on semigroups of Boolean homomorphisms, as in \S344,
so that the $g_{\pi}$ were no longer injective on conegligible sets.   But
there seem to be more substantial obstacles (425Zb).

The proof of 425D appeals repeatedly to the special properties of standard
Borel spaces.   But conceivably enough of it can be applied to the Baire
$\sigma$-algebras of powers of $\{0,1\}$
to give a similar result for other important probability spaces (425Zc).
If we change the rules, and assume that $\Cal I$ is an ideal of $\Cal PX$
as well as of $\Sigma$,
we can dispense with the ideas of 425C and work directly from 425Ba,
using copies of
$\Bbb N\times X^{G_{\alpha}}$ inside a negligible set $M$ (425Ya);
this gives us an approach to use on complete measure spaces (425Yb,
535Yd).

Another way of looking at 425D is to think of it as a kind of
lifting theorem.   Let $\Phi$ be the group of 
$(\Sigma,\Sigma)$-bimeasurable
$\Cal I$-invariant permutations of $X$.   Then each $f\in\Phi$ induces an
automorphism $f^{\smallcirc}$ of $\frak A$ defined by saying that
$f^{\smallcirc}(E^{\ssbullet})=(f[E])^{\ssbullet}$ for each $E\in\Sigma$.
(I am using the push-forward rather than the pull-back representation
here.)   425Ac is enough to show that
the group homomorphism
$f\mapsto f^{\smallcirc}:\Phi\to\Aut\frak A$ is surjective.   Under the
conditions of 425D, the action $\action$ corresponds to a group
homomorphism $\theta:G\to\Phi$ such that $(\theta\pi)^{\smallcirc}=\pi$ for
every $\pi\in G$;  and subject to the continuum hypothesis, we have a
lifting for the whole of $\Aut\frak A$.   The word `lifting' in this
context should remind you of the Lifting Theorem of measure theory (341K).
That theorem demands a complete measure space, and does not
ordinarily apply to Borel measures.   However, subject to the continuum
hypothesis, there is a lifting theorem applicable to a variety of
non-complete measure spaces, including any $(X,\Sigma,\mu)$
where $(X,\Sigma)$ is a standard Borel space and $\mu$ is $\sigma$-finite
(535E(b-i) of Volume 5).   I am not sure that it is really helpful to think
of 425E and 535E together;  certainly the manoeuvres of 425C have no
analogues in the lifting theorems of measure theory.   But the
correspondence is striking and suggests directions of enquiry which may be
worth exploring.
}

\discrpage

