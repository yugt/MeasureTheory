\frfilename{mt473.tex}
\versiondate{25.7.11}
\copyrightdate{2000}

\def\chaptername{Geometric measure theory}
\def\sectionname{Poincar\'e's inequality}

\newsection{473}

In this section I embark on the main work of the first half of the
chapter, leading up to
the Divergence Theorem in \S475.   I follow the method in 
{\smc Evans \& Gariepy 92}.   The first step is to add some minor results on
differentiable and Lipschitz functions to those already set out in \S262
(473B-473C).   Then we need to know something about convolution products
(473D), extending ideas in \S\S256 and 444;  in particular, it will be
convenient to have a fixed sequence $\sequencen{\tilde h_n}$ of
smoothing functions with some useful special properties (473E).

The new ideas of the section begin with the Gagliardo-Nirenberg-Sobolev
inequality, relating $\|f\|_{r/(r-1)}$ to $\int\|\grad f\|$.   In its
simplest form (473H) it applies only to functions with compact support;
we need to work much harder to get results which we can use to estimate
$\int_B|f|^{r/(r-1)}$ in terms of $\int_B\|\grad f\|$ and $\int_B|f|$ for
balls $B$ (473I, 473K).

\vleader{48pt}{473A}{Notation} For the next three sections, 
$r\ge 2$ will be a fixed integer.   For $x\in\BbbR^r$ and $\delta\ge 0$,
$B(x,\delta)=\{y:\|y-x\|\le\delta\}$ will be the closed ball with centre
$x$ and radius $\delta$.
I will write $\partial B(x,\delta)$ for the boundary of $B(x,\delta)$,
the sphere  $\{y:\|y-x\|=\delta\}$.
$S_{r-1}=\partial B(\tbf{0},1)$ will be the unit sphere.
As in Chapter 26, I will use Greek letters to represent coordinates of
vectors, so that $x=(\xi_1,\ldots,\xi_r)$, etc.

$\mu$ will always be Lebesgue measure on $\BbbR^r$.
$\beta_r=\mu B(\tbf{0},1)$ will be the $r$-dimensional volume of the
unit ball, that is,

$$\eqalign{\beta_r
&=\Bover{\pi^k}{k!}\text{ if }r=2k\text{ is even},\cr
&=\Bover{2^{2k+1}k!\pi^k}{(2k+1)!}\text{ if }r=2k+1\text{ is odd}\cr}$$

\noindent (252Q).
$\nu$ will be normalized Hausdorff $(r-1)$-dimensional measure on
$\BbbR^r$, that is,
$\nu=2^{-r+1}\beta_{r-1}\mu_{H,r-1}$, where $\mu_{H,r-1}$ is
$(r-1)$-dimensional Hausdorff measure on $\BbbR^r$ as described in \S264.
Recall from 265F and 265H that
$\nu S_{r-1}=2\pi\beta_{r-2}=r\beta_r$ (counting $\beta_0$ as $1$).

\leader{473B}{Differentiable functions (a)} Recall from \S262 that a
function $\phi$ from a subset of $\BbbR^r$ to $\BbbR^s$ (where
$s\ge 1$) is differentiable at $x\in\BbbR^r$, with
derivative an $s\times r$ matrix $T$, if for every $\epsilon>0$ there
is a $\delta>0$ such that $\|\phi(y)-\phi(x)-T(y-x)\|\le\epsilon\|y-x\|$
whenever $\|y-x\|\le\delta$;  this includes the assertion that
$B(x,\delta)\subseteq\dom\phi$.   In this case, the coefficients of $T$
are the partial derivatives $\Pd{\phi_j}{\xi_i}(x)$ at $x$, where
$\phi_1,\ldots,\phi_s$ are the coordinate functions of $\phi$, and
$\Pd{}{\xi_i}$ represents partial differentiation with respect to the
$i$th coordinate\cmmnt{ (262Ic)}.

\spheader 473Bb When $s=1$\cmmnt{, so that we have a real-valued
function $f$
defined on a subset of $\BbbR^r$}, I will write $(\grad f)(x)$ for the
derivative of $f$ at $x$, the {\bf gradient} of $f$.   \cmmnt{If we
strictly adhere to the language of (a), $\grad f$ is a $1\times r$
matrix $\Matrix{\Pd{f}{\xi_1}&\ldots&\Pd{f}{\xi_r}}$;  but it is
convenient to
treat it as a vector, so that $\grad f(x)$ (when defined) belongs to
$\BbbR^r$, and we can speak of $y\dotproduct\grad f(x)$ rather
than $(\grad f(x))(y)$, etc.}

\spheader 473Bc {\bf Chain rule for functions of many
variables}\cmmnt{ I find that I have not written out the
following basic fact.}   Let $\phi:A\to\BbbR^s$ and
$\psi:B\to\BbbR^p$ be functions, where $A\subseteq\BbbR^r$ and
$B\subseteq\BbbR^s$.   If $x\in A$ is such that $\phi$ is
differentiable at $x$, with derivative $S$, and $\psi$ is differentiable
at $\phi(x)$, with derivative $T$, then the composition $\psi\phi$ is
differentiable at $x$, with derivative $TS$.

\prooflet{\Prf\ Recall that if we regard $S$ and $T$ as linear
operators, they have finite norms (262H).   Given $\epsilon>0$, let
$\eta>0$ be such that  $\eta\|T\|+\eta(\|S\|+\eta)\le\epsilon$.   Let
$\delta_1$, $\delta_2>0$ be such that
$\phi(y)$ is defined and $\|\phi(y)-\phi(x)-S(y-x)\|\le\eta\|y-x\|$
whenever $\|y-x\|\le\delta_1$, and $\psi(z)$ is defined and
$\|\psi(z)-\psi\phi(x)-T(z-\phi(x))\|\le\eta\|z-\phi(x)\|$ whenever
$\|z-\phi(x)\|\le\delta_2$.   Set
$\delta=\min(\delta_1,\Bover{\delta_2}{\eta+\|S\|})>0$.   If
$\|y-x\|\le\delta$, then $\phi(y)$ is defined and

\Centerline{$\|\phi(y)-\phi(x)\|\le\|S(y-x)\|+\|\phi(y)-\phi(x)-S(y-x)\|
\le(\|S\|+\eta)\|y-x\|\le\delta_2$,}

\noindent so $\psi\phi(y)$ is defined and

$$\eqalign{\|\psi\phi(y)&-\psi\phi(x)-TS(y-x)\|\cr
&\le\|\psi\phi(y)-\psi\phi(x)-T(\phi(y)-\phi(x))\|
     +\|T\|\|\phi(y)-\phi(x)-S(y-x)\|\cr
&\le\eta\|\phi(y)-\phi(x)\|+\|T\|\eta\|y-x\|\cr
&\le\eta(\|S\|+\eta)\|y-x\|+\|T\|\eta\|y-x\|
\le\epsilon\|y-x\|;\cr}$$

\noindent as $\epsilon$ is arbitrary, $\psi\phi$ is differentiable at
$x$ with derivative $TS$.\ \Qed}%end of prooflet

\spheader 473Bd It follows that if $f$ and $g$ are real-valued functions
defined on a
subset of $\BbbR^r$, and $x\in\dom f\cap\dom g$ is such that
$(\grad f)(x)$ and $(\grad g)(x)$ are both defined, then
$\grad(f\times g)(x)$ is defined and equal to
$f(x)\grad g(x)+g(x)\grad f(x)$.   \prooflet{\Prf\ Set
$\phi(y)=\Matrix{f(y)\\g(y)}$ for $y\in\dom f\cap\dom g$;  then $\phi$
is differentiable at $x$ with derivative the $2\times r$ matrix
$\Matrix{\grad f(x)\\ \grad g(x)}$ (262Ib).   Set
$\psi(z)=\zeta_1\zeta_2$ for $z=(\zeta_1,\zeta_2)\in\BbbR^2$;  then
$\psi$ is differentiable everywhere, with derivative the $1\times 2$
matrix $\Matrix{\zeta_2&\zeta_1}$.   So $f\times g=\psi\phi$ is
differentiable at $x$ with derivative

\Centerline{$\Matrix{g(x)&f(x)}\Matrix{\grad f(x)\\ \grad g(x)}
=g(x)\grad f(x)+f(x)\grad g(x)$.  \Qed}
}%end of prooflet

\spheader 473Be Let $D$ be a subset of $\BbbR^r$ and $\phi:D\to\BbbR^s$
any function.   Set $D_0=\{x:x\in D,\,\phi$ is differentiable at $x\}$.
Then the derivative of $\phi$, regarded as a function from $D_0$ to
$\BbbR^{rs}$, is (Lebesgue) measurable.
\prooflet{\Prf\ Use 262P;  the point is that, writing $T(x)$ for the
derivative of $\phi$ at $x$, $T(x)$ is surely a derivative of
$\phi\restr D_0$, relative to $D_0$, at every point of $D_0$.\ \QeD\
}%end of prooflet
\cmmnt{(See also 473Ya.)}

\spheader 473Bf A function $\phi:\BbbR^r\to\BbbR^s$ is {\bf smooth} if
it is differentiable arbitrarily often;  that is, if all its repeated
partial derivatives

\Centerline{$\Bover{\partial^m\phi_j}{\partial
\xi_{i_1}\ldots\partial\xi_{i_m}}$}

\noindent are defined and continuous everywhere on $\BbbR^r$.
I will write $\eusm D$ for the family of real-valued functions
on $\BbbR^r$ which are smooth and have compact support.

\leader{473C}{Lipschitz functions (a)} If $f$ and
$g$ are bounded real-valued Lipschitz functions, defined on any subsets
of $\BbbR^r$, then $f\times g$, defined on $\dom f\cap\dom g$, is
Lipschitz.   \prooflet{\Prf\ Let $\gamma_f$, $M_f$, $\gamma_g$ and $M_g$
be such that $|f(x)|\le M_f$ and $|f(x)-f(y)|\le\gamma_f\|x-y\|$ for all
$x$, $y\in\dom f$, while $|g(x)|\le M_g$ and
$|g(x)-g(y)|\le\gamma_g\|x-y\|$ for all $x$, $y\in\dom g$.   Then for
any $x$, $y\in\dom f\cap\dom g$,

$$\eqalign{|f(x)g(x)-f(y)g(y)|
&\le|f(x)||g(x)-g(y)|+|g(y)||f(x)-f(y)|\cr
&\le (M_f\gamma_g+M_g\gamma_f)\|x-y\|.\cr}$$

\noindent So $M_f\gamma_g+M_g\gamma_f$ is a Lipschitz constant for
$f\times g$.\ \Qed}

\leaveitout{It follows that if $f$ and $g$ are bounded real-valued Lipschitz functions defined on $\BbbR^r$, and $g$ has compact support, then $f\times g$ is Lipschitz.}

\spheader 473Cb Suppose that $F_1$, $F_2\subseteq\BbbR^r$ are closed
sets with convex union $C$.   Let $f:C\to\Bbb R$ be a function such that
$f\restr F_1$ and $f\restr F_2$ are both Lipschitz.   Then $f$ is
Lipschitz.   \prooflet{\Prf\ For each $j$, let $\gamma_j$ be a Lipschitz
constant for $f\restr F_j$, and set $\gamma=\max(\gamma_1,\gamma_2)$, so
that $\gamma$ is a Lipschitz constant for both $f\restr F_1$ and
$f\restr F_2$.   Take any $x$, $y\in C$.   If both belong to the same
$F_j$, then
$|f(x)-f(y)|\le\gamma\|x-y\|$.   If $x\in F_j$ and $y\notin F_j$, then
$y$ must belong to $F_{3-j}$, and $(1-t)x+ty\in F_1\cup F_2$ for every
$t\in[0,1]$, because $C$ is convex.   Set
$t_0=\sup\{t:t\in[0,1],\,(1-t)x+ty\in F_j\}$,
$z=(1-t_0)x+t_0y$;  then $z\in F_1\cap F_2$, because both are closed, so

\Centerline{$|f(x)-f(y)|\le|f(x)-f(z)|+|f(z)-f(y)|
\le\gamma\|x-z\|+\gamma\|z-y\|=\gamma\|x-y\|$.}

\noindent As $x$ and $y$ are arbitrary, $\gamma$ is a Lipschitz constant
for $f$.\ \Qed}

\spheader 473Cc Suppose that $f:\BbbR^r\to\Bbb R$ is Lipschitz.
\cmmnt{Recall that by Rademacher's theorem (262Q),} $\grad f$ is
defined almost everywhere.   \cmmnt{All the partial
derivatives of $f$ are (Lebesgue) measurable, by 473Be, so} $\grad f$ is
(Lebesgue) measurable on its domain.   If $\gamma$ is a Lipschitz
constant for $f$, $\|\grad f(x)\|\le\gamma$ whenever $\grad f(x)$ is
defined.   \prooflet{\Prf\ If $z\in\BbbR^r$, then

\Centerline{$\lim_{t\downarrow 0}
  \Bover1t|f(x+tz)-f(x)-tz\dotproduct\grad f(x)|=0$,}

\noindent so

\Centerline{$|z\dotproduct\grad f(x)|
=\lim_{t\downarrow 0}\Bover1t|f(x+tz)-f(x)|
\le\gamma\|z\|$;}

\noindent as $z$ is arbitrary, $\|\grad f(x)\|\le\gamma$.\ \Qed}

\spheader 473Cd Conversely, if $f:\BbbR^r\to\Bbb R$ is differentiable
and $\|\grad f(x)\|\le\gamma$ for every $x$, then $\gamma$ is a
Lipschitz constant for $f$.   \prooflet{\Prf\ Take $x$, $y\in\BbbR^r$.   Set $g(t)=f((1-t)x+ty)$ for $t\in\Bbb R$.   The function
$t\mapsto (1-t)x+ty:\Bbb R\to\BbbR^r$ is everywhere differentiable, with
constant derivative $y-x$, so by 473Bc $g$ is differentiable, with
derivative $g'(t)=(y-x)\dotproduct\grad f((1-t)x+ty)$ for every $t$;
in particular, $|g'(t)|\le\gamma\|y-x\|$ for every $t$.   Now, by the Mean Value Theorem, there is a $t\in[0,1]$ such that
$g(1)-g(0)=g'(t)$, so that $|f(y)-f(x)|=|g'(t)|\le\gamma\|y-x\|$.   As $x$ and $y$ are
arbitrary, $f$ is $\gamma$-Lipschitz.\ \Qed}

\spheader 473Ce Note that if
$f\in\eusm D$ then\cmmnt{ all its partial derivatives are continuous
functions with compact support, so are bounded (436Ia), and} $f$ is
Lipschitz as well as bounded\cmmnt{, by (d) here}.

\spheader 473Cf{\bf (i)} If $D\subseteq\BbbR^r$ is bounded and 
$f:D\to\Bbb R$ is Lipschitz, then there is a Lipschitz function 
$g:\BbbR^r\to\Bbb R$, with compact support,
extending $f$.   \prooflet{\Prf\ By 262Bb there is a Lipschitz function
$f_1:\BbbR^r\to\Bbb R$ which extends $f$.   Let $\gamma>0$ be such that
$D\subseteq B(\tbf{0},\gamma)$ and $\gamma$ is a Lipschitz constant for
$f_1$;  set $M=|f_1(0)|+\gamma^2$;  then $|f_1(x)|\le M$ for every
$x\in D$, so if we set
$f_2(x)=\med(-M,f_1(x),M)$ for $x\in\BbbR^r$, $f_2$ is a bounded
Lipschitz function extending $f$.   Set
$f_3(x)=\med(0,1+\gamma-\|x\|,1)$ for $x\in\BbbR^r$;  then
$f_3$ is a bounded Lipschitz function with compact support.   By (a),
$g=f_3\times f_2$ is Lipschitz, and $g:\BbbR^r\to\Bbb R$ is a function
with compact support extending $f$.\ \Qed}

\medskip

\quad{\bf (ii)}
It follows that if $D\subseteq\BbbR^r$ is bounded and $f:D\to\Bbb R^s$ is 
Lipschitz, then there is a Lipschitz function $g:\BbbR^r\to\BbbR^s$, with 
compact support,
extending $f$.   \prooflet{\Prf\ By 262Ba, we need only apply (i) to each 
coordinate of $f$.\ \Qed}

\leader{473D}{Smoothing by \dvrocolon{convolution}}\cmmnt{ We shall
need a miscellany of facts, many of them special cases of results in
\S\S255 and 444, concerning convolutions on $\BbbR^r$.   Recall that I
write $(f*g)(x)=\int f(y)g(x-y)\mu(dy)$ whenever $f$ and $g$ are
real-valued functions defined almost everywhere in $\BbbR^r$ and the
integral is defined, and that $f*g=g*f$ (255Fb, 444Og).   Now we
have the following.

\medskip

\noindent}{\bf Lemma} Suppose that $f$ and $g$ are Lebesgue measurable
real-valued functions defined $\mu$-almost everywhere in $\BbbR^r$.

(a) If $f$ is integrable and $g$ is essentially bounded, then their
convolution $f*g$ is defined everywhere in $\BbbR^r$ and uniformly
continuous, and $\|f*g\|_{\infty}\le\|f\|_1\esssup|g|$.

(b) If $f$ is locally integrable and $g$ is bounded and has compact
support, then $f*g$ is defined everywhere in $\BbbR^r$ and is
continuous.

(c) If $f$ and $g$ are defined everywhere in $\BbbR^r$ and
$x\in\BbbR^r\setminus(\{y:f(y)\ne 0\}+\{z:g(z)\ne 0\})$, then $(f*g)(x)$
is defined and equal to $0$.

(d) If $f$ is integrable and $g$ is bounded, Lipschitz and defined
everywhere, then $f*\grad g$ and $\grad(f*g)$ are defined everywhere and
equal, where $f*\grad g=(f*\Pd{g}{\xi_1},\ldots,f*\Pd{g}{\xi_r})$.
Moreover, $f*g$ is Lipschitz.

(e) If $f$ is locally integrable, and $g\in\eusm D$,
then $f*g$ is defined everywhere and is smooth.

(f) If $f$ is essentially bounded and $g\in\eusm D$, then all the
derivatives of $f*g$ are bounded, and $f*g$ is Lipschitz.

(g) If $f$ is integrable and $\phi:\BbbR^r\to\BbbR^r$ is a bounded
measurable function with components $\phi_1,\ldots,\phi_r$, and we write
$(f*\phi)(x)=((f*\phi_1)(x),\ldots,(f*\phi_r)(x))$, then
$\|(f*\phi)(x)\|\le\|f\|_1\sup_{y\in\BbbR^r}\|\phi(y)\|$ for every
$x\in\BbbR^r$.

\proof{{\bf (a)} See 255K.

\medskip

{\bf (b)} Suppose that $g(y)=0$ when $\|y\|\ge n$.   Given
$x\in\BbbR^r$, set $\tilde f=f\times\chi B(x,n+1)$.   Then $\tilde f*g$
is defined everywhere and continuous, by (a), while
$(f*g)(z)=(\tilde f*g)(z)$ whenever $z\in B(x,1)$;  so $f*g$ is defined
everywhere in $B(x,1)$ and is continuous at $x$.

\medskip

{\bf (c)} We have only to note that $f(y)g(x-y)=0$ for every $y$.

\medskip

{\bf (d)} Let $\gamma$ be a Lipschitz constant for $g$.   We know that
$\grad g$ is defined almost everywhere, is measurable, and that
$\|\grad g(x)\|\le\gamma$ whenever it is defined (473Cc);  so
$(f*\grad g)(x)$ is defined for every $x$, by (a) here.   Fix
$x\in\BbbR^r$.   If $y$, $z\in\BbbR^r$ set

\Centerline{$\theta(y,z)
=\Bover1{\|z\|}\bigl(g(x-y+z)-g(x-y)
  -z\dotproduct\grad g(x-y)\bigr)$}

\noindent whenever this is defined.   Then $|\theta(y,z)|\le 2\gamma$
whenever it is defined.   Now suppose that $\sequencen{z_n}$ is a
sequence in $\BbbR^r\setminus\{\tbf{0}\}$ converging to $\tbf{0}$.
Then $\lim_{n\to\infty}\theta(y,z_n)=0$ whenever $\grad g(x-y)$ is
defined, which almost everywhere.   So
$\lim_{n\to\infty}\int f(y)\theta(y,z_n)\mu(dy)=0$, by Lebesgue's
Dominated Convergence Theorem.   But this means that

\Centerline{$\Bover1{\|z_n\|}
  \bigl((f*g)(x+z_n)-(f*g)(x)-((f*\grad g)(x))\dotproduct z_n\bigr)
\to 0$}

\noindent as $n\to\infty$.   As $\sequencen{z_n}$ is arbitrary,
$\grad(f*g)(x)$ is defined and is equal to $(f*\grad g)(x)$.

Now $\grad g$ is bounded, because $g$ is Lipschitz, so
$\grad(f*g)=f*\grad g$ also is bounded, by (a), and $f*g$ must be
Lipschitz (473Cd).

\medskip

{\bf (e)} By (b), $f*g$ is defined everywhere and is continuous.   Now,
for any $i\le r$, $\Pd{}{\xi_i}(f*g)=f*\Pd{g}{\xi_i}$ everywhere.
\Prf\ Let $n\in\Bbb N$ be such that $g(y)=0$ if $\|y\|\ge n$.   Given
$x\in\BbbR^r$, set $\tilde f=f\times\chi B(x,n+1)$.   Then
$(f*g)(z)=(\tilde f*g)(z)$ for every $z\in B(x,1)$, so that

$$\eqalignno{\Pd{(f*g)}{\xi_i}(x)
&=\Pd{(\tilde f*g)}{\xi_i}(x)
=(\tilde f*\Pd{g}{\xi_i})(x)\cr
\displaycause{by (d)}
&=(f*\Pd{g}{\xi_i})(x)\cr}$$

\noindent (because of course $\Pd{g}{\xi_i}$ is also zero outside
$B(\tbf{0},n)$).\ \QeD\   Inducing on $k$,

$$\bover{\partial^k}{\partial\xi_{i_1}\ldots\partial\xi_{i_k}}(f*g)(x)
=(f*\bover{\partial^kg}{\partial\xi_{i_1}\ldots\partial\xi_{i_k}})(x)$$

\noindent for every $x\in\BbbR^r$ and every $i_1,\ldots,i_k$;  so we
have the result.

\medskip

{\bf (f)} The point is just that all the partial derivatives of $g$,
being smooth functions with compact support, are integrable, and that

\Centerline{$|\Pd{}{\xi_i}(f*g)(x)|
=|(f*\Pd{g}{\xi_i})(x)|
\le\|f\|_{\infty}\|\Pd{g}{\xi_i}\|_1$}

\noindent for every $x$ and every $i\le r$.   Inducing on the order of $D$,
we see that $D(f*g)=f*Dg$ and
$\|D(f*g)\|_{\infty}\le\|f\|_{\infty}\|Dg\|_1$, so that $D(f*g)$ is
bounded, for any partial differential operator $D$.   In particular,
$\grad(f*g)$ is bounded, so that $f*g$ is Lipschitz, by 473Cd.

\medskip

{\bf (g)} If $x$, $z\in\BbbR^r$, then

$$\eqalign{z\dotproduct(f*\phi)(x)
&=\sum_{i=1}^r\zeta_i(f*\phi_i)(x)
=\int f(y)\sum_{i=1}^r\zeta_i\phi_i(x-y)\mu(dy)\cr
&\le\int|f(y)||\sum_{i=1}^r\zeta_i\phi_i(x-y)|\mu(dy)\cr
&\le\int|f(y)|\|z\|\|\phi(x-y)\|\mu(dy)
\le\|z\|\|f\|_1\sup_{y\in\BbbR^r}\|\phi(y)\|.\cr}$$

\noindent As $z$ is arbitrary,
$\|(f*\phi)(x)\|\le\|f\|_1\sup_{y\in\BbbR^r}\|\phi(y)\|$.
}%end of proof of 473D

\leader{473E}{Lemma} (a) Define $h:\Bbb R\to[0,1]$ by
setting $h(t)=\exp(\Bover{1}{t^2-1})$ for $|t|<1$, $0$ for
$|t|\ge 1$.   Then $h$ is smooth, and $h'(t)\le 0$ for $t\ge 0$.

(b) For $n\ge 1$, define $\tilde h_n:\BbbR^r\to\Bbb R$ by setting

\Centerline{$\alpha_n=\int h((n+1)^2\|x\|^2)\mu(dx)$,
\quad$\tilde h_n(x)=\Bover1{\alpha_n}h((n+1)^2\|x\|^2)$}

\noindent for every $x\in\BbbR^r$.   Then
$\tilde h_n\in\eusm D$, $\tilde h_n(x)\ge 0$ for every $x$,
$\tilde h_n(x)=0$ if
$\|x\|\ge\Bover1{n+1}$, and $\int\tilde h_nd\mu=1$.

(c) If $f\in\eusm L^0(\mu)$, then
$\lim_{n\to\infty}(f*\tilde h_n)(x)=f(x)$ for every $x\in\dom f$ at
which $f$ is continuous.

(d) If $f:\BbbR^r\to\Bbb R$ is uniformly continuous (in particular, if
it is either Lipschitz or a continuous function with compact support),
then $\lim_{n\to\infty}\|f-f*\tilde h_n\|_{\infty}=0$.

(e) If $f\in\eusm L^0(\mu)$ is locally integrable, then
$f(x)=\lim_{n\to\infty}(f*\tilde h_n)(x)$ for $\mu$-almost every
$x\in\BbbR^r$.

(f) If $f\in\eusm L^p(\mu)$, where $1\le p<\infty$, then
$\lim_{n\to\infty}\|f-f*\tilde h_n\|_p=0$.

\proof{{\bf (a)} Set $h_0(t)=\exp(-\bover1t)$ for $t>0$, $0$ for
$t\le 0$.
A simple induction on $n$ shows that the $n$th derivative $h_0^{(n)}$ of
$h_0$ is of the form

$$\eqalign{h_0^{(n)}(t)&=q_n(\Bover1t)\exp(-\Bover1t)\text{ for }t>0\cr
&=0\text{ for }t\le 0,\cr}$$

\allowmorestretch{468}{\noindent where each $q_n$ is a polynomial of degree $2n$;  the
inductive hypothesis depends on the fact that
$\lim_{s\to\infty}q(s)e^{-s}=0$ for every polynomial $q$.   So $h_0$ is
smooth.   Now $h(t)=h_0(1-t^2)$ so $h$ also is smooth.   If $0\le t<1$
then}

\Centerline{$h'(t)=-\exp(\Bover1{t^2-1})\cdot\Bover{2t}{(t^2-1)^2}<0$;}

\noindent if $t>1$ then $h'(t)=0$;  since $h'$ is continuous,
$h'(t)\le 0$ for every $t\ge 0$.

\medskip

{\bf (b)} We need only observe that

\Centerline{$x\mapsto(n+1)^2\|x\|^2=(n+1)^2\sum_{i=1}^r\xi_i^2$}

\noindent is smooth
and that the composition of smooth functions is smooth (using 473Bc).

\medskip

{\bf (c)} If $f$ is continuous at $x$ and $\epsilon>0$, let
$n_0\in\Bbb N$ be such that $|f(y)-f(x)|\le\epsilon$ whenever
$y\in\dom f$ and  $\|y-x\|\le\Bover1{n_0+1}$.   Then for any $n\ge n_0$,

$$\eqalignno{|(f*\tilde h_n)(x)-f(x)|
&=|\int f(x-y)\tilde h_n(y)\mu(dy)-\int f(x)\tilde h_n(y)\mu(dy)|\cr
&\le\int|f(x-y)-f(x)|\tilde h_n(y)\mu(dy)
\le\int\epsilon\tilde h_n(y)\mu(dy)
=\epsilon.\cr}$$

\noindent As $\epsilon$ is arbitrary, we have the result.

\medskip

{\bf (d)} Repeat the argument of (c), but `uniformly in $x$';  that is,
given $\epsilon>0$, take $n_0$ such that $|f(y)-f(x)|\le\epsilon$
whenever $x$, $y\in\BbbR^r$ and $\|y-x\|\le\Bover1{n_0+1}$, and see that
$|(f*\tilde h_n)(x)-f(x)|\le\epsilon$ for every $n\ge n_0$ and every
$x$.

\medskip

{\bf (e)} We know from 472Db or 261E that, for almost every $x\in\BbbR^r$,

\Centerline{$\lim_{\delta\downarrow 0}\Bover1{\mu B(x,\delta)}
  \int_{B(x,\delta)}|f(y)-f(x)|\mu(dy)=0$.}

\noindent Take any such $x$.   Set $\gamma=f(x)$,
Set $g(y)=|f(y)-\gamma|$ for every $y\in\dom f$.   Let $\epsilon>0$.   Then there is some $\delta>0$ such that
$\Bover{q(t)}{\beta_rt^r}\le\epsilon$ whenever $0<t\le\delta$, where

\Centerline{$q(t)=\int_{B(x,t)}g\,d\mu
=\int_0^t\int_{\partial B(x,s)}g(y)\nu(dy)dt$}

\noindent by 265G, so $q'(t)=\int_{\partial B(y,t)}g\,d\nu$ for almost
every $t\in[0,\delta]$, by 222E.   If $n+1\ge\Bover1{\delta}$, then

$$\eqalignno{(g*\tilde h_n)(x)
&=\int g(y)\tilde h_n(x-y)\mu(dy)
=\int_{B(x,\delta)}g(y)\tilde h_n(x-y)\mu(dy)\cr
&=\Bover1{\alpha_n}\int_0^{\delta}\int_{\partial B(x,t)}
  g(y)h((n+1)^2t^2)\nu(dy)dt\cr
\displaycause{265G again}
&=\Bover1{\alpha_n}\int_0^{\delta}h((n+1)^2t^2)q'(t)dt\cr
&=-\Bover1{\alpha_n}\int_0^{\delta}2(n+1)^2th'((n+1)^2t^2)q(t)dt\cr
\displaycause{integrating by parts (225F), because
$q(0)=h((n+1)^2\delta^2)=0$ and both $q$ and $h$ are absolutely
continuous}
&\le-\Bover{\epsilon}{\alpha_n}\int_0^{\delta}
  2(n+1)^2th'((n+1)^2t^2)\beta_rt^rdt\cr
\displaycause{because $0\le q(t)\le\epsilon\beta_rt^r$ and
$h'((n+1)^2t^2)\le 0$ for $0\le t\le\delta$}
&=\epsilon\cr}$$

\noindent (applying the same calculations with $\chi\BbbR^r$ in place of
$g$).   But now, since $(\gamma\chi\BbbR^r*\tilde h_n)(x)=\gamma$ for
every $n$,

\Centerline{$|(f*\tilde h_n)(x)-\gamma|
=|\int(f(y)-\gamma)\tilde h_n(x-y)\mu(dy)|
\le\int|f(y)-\gamma|\tilde h_n(x-y)\mu(dy)
\le\epsilon$}

\noindent whenever $n+1\ge\Bover1{\delta}$.  As $\epsilon$ is arbitrary,
$f(x)=\gamma=\lim_{n\to\infty}(f*\tilde h_n)(x)$;  and this is true for
$\mu$-almost every $x$.

\medskip

{\bf (f)} Apply 444T to the indefinite-integral measure
$\tilde h_n\mu$ over $\mu$ defined by $\tilde h_n$;
use 444Pa for the identification of
$(\tilde h_n\mu)*f$ with $\tilde h_n*f=f*\tilde h_n$.
}%end of proof of 473E

\leader{473F}{Lemma} For any measure space $(X,\Sigma,\lambda)$ and any
non-negative $f_1,\ldots,f_k\in\eusm L^0(\lambda)$,

\Centerline{$\int\prod_{i=1}^kf_i^{1/k}d\lambda
\le\prod_{i=1}^k\bigl(\int f_id\lambda\bigr)^{1/k}$.}

\proof{ Induce on $k$.   Note that we can suppose that every
$f_i$ is integrable;  for if any
$\int f_i$ is zero, then $f_i=0$ a.e.\ and the result is trivial;  and
if all the $\int f_i$ are greater than zero and any of them is infinite,
the result is again trivial.

The induction starts with the trivial case $k=1$.   For the inductive
step to $k\ge 2$, we have

$$\eqalignno{\int\prod_{i=1}^kf_i^{1/k}d\lambda
&\le\|\prod_{i=1}^{k-1}f_i^{1/k}\|_{k/(k-1)}\|f_k^{1/k}\|_k\cr
\displaycause{by H\"older's inequality, 244E}
&=\bigl(\int\prod_{i=1}^{k-1}f_i^{1/(k-1)}d\lambda\bigr)^{(k-1)/k}
   \bigl(\int f_kd\lambda\bigr)^{1/k}\cr
&\le\bigl(\prod_{i=1}^{k-1}(\int f_id\lambda)^{1/(k-1)}\bigr)^{(k-1)/k}
   \bigl(\int f_kd\lambda\bigr)^{1/k}\cr
\displaycause{by the inductive hypothesis}
&=\prod_{i=1}^k(\int f_id\lambda)^{1/k},\cr}$$

\noindent as required.
}%end of proof of 473F

\leader{473G}{Lemma} Let $(X,\Sigma,\lambda)$ be a $\sigma$-finite
measure space and
$k\ge 2$ an integer.   Write $\lambda_k$ for the product measure on
$X^k$.   For
$x=(\xi_1,\ldots,\xi_k)\in X^k$, $t\in X$ and $1\le i\le k$ set
$S_i(x,t)=(\xi'_1,\ldots,\xi'_k)$ where $\xi'_i=t$ and $\xi'_j=\xi_j$
for $j\ne i$.   Then if $h\in\eusm L^1(\lambda_k)$ is non-negative, and
we set $h_i(x)=\int h(S_i(x,t))\lambda(dt)$ whenever this is defined in
$\Bbb R$, we have

\Centerline{$\int(\prod_{i=1}^kh_i)^{1/(k-1)}d\lambda_k
\le(\int h\,d\lambda_k)^{k/(k-1)}$.}

\proof{ Induce on $k$.

\medskip

{\bf (a)} If $k=2$, we have

$$\eqalign{\int h_1\times h_2d\lambda_k
&=\iint\bigl(\int h(\tau_1,\xi_2)d\tau_1\bigr)
  \bigl(\int h(\xi_1,\tau_2)d\tau_2\bigr)d\xi_1d\xi_2\cr
&=\iiiint h(\tau_1,\xi_2)h(\xi_1,\tau_2)d\tau_1d\tau_2d\xi_1d\xi_2\cr
&=\iint h(\tau_1,\xi_2)d\tau_1d\xi_2
  \cdot\iint h(\xi_1,\tau_2)d\tau_2d\xi_1
=\bigl(\int h\,d\lambda_2\bigr)^2\cr}$$

\noindent by Fubini's theorem (252B) used repeatedly, because (by 253D)
\discrcenter{390pt}{$(\tau_1,\tau_2,\xi_1,\xi_2)\mapsto
h(\xi_1,\tau_2)h(\tau_1,\xi_2)$ }is $\lambda_4$-integrable.
(See 251W for a sketch of the manipulations
needed to apply 252B, as stated, to the integrals above.)

\medskip

{\bf (b)} For the inductive step to $k\ge 3$, argue as follows.
For $y\in X^{k-1}$, set $g(y)=\int h(y,t)dt$
whenever this is defined in $\Bbb R$, identifying $X^k$ with
$X^{k-1}\times X$, so that $g(y)=h_k(y,t)$ whenever either is defined.
If $1\le i<k$, we can consider $S_i(y,t)$ for $y\in X^{k-1}$ and
$t\in X$, and we have

\Centerline{$\int g(S_i(y,t))dt
=\iint h(S_i(y,t),u)dudt
=\int h_i(y,t)dt$}

\noindent for almost every $y\in X^{k-1}$.   So

$$\eqalignno{\int(\prod_{i=1}^kh_i)^{1/(k-1)}d\lambda_k
&=\iint(\prod_{i=1}^{k-1}h_i(y,t))^{1/(k-1)}g(y)^{1/(k-1)}
  dt\,\lambda_{k-1}(dy)\cr
&=\int g(y)^{1/(k-1)}\int(\prod_{i=1}^{k-1}h_i(y,t))^{1/(k-1)}
  dt\,\lambda_{k-1}(dy)\cr
&\le\int g(y)^{1/(k-1)}
  \prod_{i=1}^{k-1}\bigl(\int h_i(y,t)dt\bigr)^{1/(k-1)}
  \lambda_{k-1}(dy)\cr
\displaycause{473F}
&=\int g(y)^{1/(k-1)}
  \prod_{i=1}^{k-1}g_i(y)^{1/(k-1)}\lambda_{k-1}(dy)\cr
\displaycause{where $g_i$ is defined from $g$ in the same way as $h_i$
is defined from $h$}
&\le\bigl(\int g(y)\lambda_{k-1}(dy)\bigr)^{1/(k-1)}
  \bigl(\int\prod_{i=1}^{k-1}g_i(y)^{1/(k-2)}
      \lambda_{k-1}(dy)\bigr)^{(k-2)/(k-1)}\cr
\displaycause{by H\"older's inequality again, this time with
$\Bover1{k-1}+\Bover{k-2}{k-1}=1$}
&\le\bigl(\int g(y)\lambda_{k-1}(dy)\bigr)^{1/(k-1)}
  \cdot\int g(y)\lambda_{k-1}(dy)\cr
\displaycause{by the inductive hypothesis}
&=\bigl(\int g(y)\lambda_{k-1}(dy)\bigr)^{k/(k-1)}
=\bigl(\int h(x)\lambda_k(dx)\bigr)^{k/(k-1)},\cr}$$

\noindent and the induction proceeds.
}%end of proof of 473G

\leader{473H}{Gagliardo-Nirenberg-Sobolev inequality} Suppose that
$f:\BbbR^r\to\Bbb R$ is a Lipschitz function with compact support.
Then $\|f\|_{r/(r-1)}\le\int\|\grad f\|d\mu$.

\proof{ By 473Cc, $\grad f$ is measurable and bounded, so $\|\grad f\|$
also is;  since it must have compact support, it is integrable.

For $1\le i\le r$,
$x=(\xi_1,\ldots,\xi_r)\in\BbbR^r$ and $t\in\Bbb R$ write
$S_i(x,t)=(\xi'_1,\ldots,\xi'_r)$ where $\xi'_i=t$ and $\xi'_j=\xi_j$
for $j\ne i$.   Set
$h_i(x)=\int_{-\infty}^{\infty}\|\grad f(S_i(x,t))\|dt$ when this is
defined,
which will be the case for almost every $x$.   Now, whenever $h_i(x)$ is
defined,

\Centerline{$|f(x)|=|f(S_i(x,\xi_i))|
=|\int_{-\infty}^{\xi_i}\pd{}{t}f(S_i(x,t))dt|
\le h_i(x)$.}

\noindent (Use 225E and the fact that a Lipschitz function on any
bounded interval in $\Bbb R$ is absolutely continuous.)
So $|f|\leae h_i$ for every $i\le r$.   Accordingly

\Centerline{$\int|f(x)|^{r/(r-1)}\mu(dx)
\le\int\prod_{i=1}^rh_i(x)^{1/(r-1)}\mu(dx)
\le\bigl(\int\|\grad f(x)\|\mu(dx)\bigr)^{r/(r-1)}$}

\noindent by 473G.   Raising both sides to the power $(r-1)/r$ we
have the result.
}%end of proof of 473H

\leader{473I}{Lemma} For any Lipschitz function
$f:B(\tbf{0},1)\to\Bbb R$,

\Centerline{$\int_{B(\tbf{0},1)}|f|^{r/(r-1)}d\mu
\le \bigl(2^{r+4}\sqrt r
  \int_{B(\tbf{0},1)}\|\grad f\|+|f|d\mu\bigr)^{r/(r-1)}$.}

\proof{{\bf (a)} Set $g(x)=\max(0,2\|x\|^2-1)$ for $x\in B(\tbf{0},1)$.
Then $\grad g$ is defined at
every point $x$ such that $\|x\|<1$ and $\|x\|\ne\Bover1{\sqrt{2}}$, and at all such points $\pd{g}{\xi_i}$ is either $0$ or $4\xi_i$ for each $i$, so that $\|\grad g(x)\|\le 4\|x\|\le 4$.   Hence
(or otherwise) $g$ is
Lipschitz.   So $f_1=f\times g$ is Lipschitz (473Ca).

By Rademacher's theorem again, $\grad f_1$ is defined almost everywhere
in $B(\tbf{0},1)$.   Now

$$\eqalignno{\int_{B(\tbf{0},1)}\|\grad f_1\|d\mu
&=\int_{B(\tbf{0},1)}\|f(x)\grad g(x)+g(x)\grad f(x)\|\mu(dx)\cr
\displaycause{473Bd}
&\le\int_{B(\tbf{0},1)}4|f|+\|\grad f\|d\mu.\cr}$$

\medskip

{\bf (b)} It will be convenient to have an elementary fact out in the
open.   Set $\phi(x)=\Bover{x}{\|x\|^2}$ for
$x\in\BbbR^r\setminus\{\tbf{0}\}$;  note that $\phi^2(x)=x$.   Then
$\phi\restr\{x:\|x\|\ge\delta\}$ is Lipschitz, for any $\delta>0$.
\Prf\ If $\|x\|=\alpha\ge\delta$, $\|y\|=\beta\ge\delta$, then  we have

$$\eqalign{\|\phi(x)-\phi(y)\|^2
&=\Bover1{\alpha^4}\|x\|^2-\Bover2{\alpha^2\beta^2}x\dotproduct y
  +\Bover1{\beta^4}\|y\|^2\cr
&=\Bover1{\alpha^2\beta^2}
  \bigl(\|y\|^2-2x\dotproduct y+\|x\|^2\bigr)
\le\Bover1{\delta^4}\|x-y\|^2,\cr}$$

\noindent so $\Bover1{\delta^2}$ is a Lipschitz constant for
$\phi\restrp\BbbR^r\setminus B(\tbf{0},\delta)$.\ \Qed

\medskip

{\bf (c)} Set $f_2(x)=f(x)$ if $\|x\|\le 1$, $f_1\phi(x)$ if
$\|x\|\ge 1$.   Then $f_2$ is
well-defined (because $f_1(x)=f(x)$ if $\|x\|=1$), is zero outside
$B(\tbf{0},\sqrt{2})$ (because $g(x)=0$ if $\|x\|\le\Bover1{\sqrt 2}$),
and is Lipschitz.   \Prf\ By 473Cb, it will be enough to show that
$f_2\restr F$ is Lipschitz, where $F=\{x:\|x\|\ge 1\}$.   But (b) shows
that $\phi\restr F$ is $1$-Lipschitz, so any Lipschitz constant
for $f_1$ is also a Lipschitz constant for
$f_2\restr F$.\ \Qed

If $\|x\|>1$, then, for any $i\le r$,

$$\eqalign{\Pd{f_2}{\xi_i}(x)
&=\sum_{j=1}^r\Pd{f_1}{\xi_j}(\phi(x))
  \cdot\Pd{}{\xi_i}(\Bover{\xi_j}{\|x\|^2})
=\Pd{f_1}{\xi_i}(\phi(x))\cdot\Bover1{\|x\|^2}
   -2\sum_{j=1}^r\Pd{f_1}{\xi_j}(\phi(x))
      \cdot\Bover{\xi_i\xi_j}{\|x\|^4}\cr
&=\Pd{f_1}{\xi_i}(\phi(x))\cdot\Bover1{\|x\|^2}
   -\Bover{2\xi_i}{\|x\|^4}x\dotproduct\grad f(\phi(x))\cr}$$

\noindent wherever the right-hand side is defined, that is, wherever all
the partial derivatives $\Pd{f_1}{\xi_j}(\phi(x))$ are defined.   But
$H=B(\tbf{0},1)\setminus\dom(\grad f_1)$ is negligible, and does not
meet $\{x:\|x\|<\Bover1{\sqrt 2}\}$, so $\phi\restr H$ is Lipschitz and
$\phi[H]=\phi^{-1}[H]$ is
negligible (262D);  while $\grad f_1(\phi(x))$ is defined whenever
$\|x\|>1$
and $x\notin\phi^{-1}[H]$.   So the formula here is valid for almost
every $x\in F$, and

$$\eqalign{|\Pd{f_2}{\xi_i}(x)|
&\le\|\grad f_1(\phi(x))\|\cdot\Bover1{\|x\|^2}
   +\Bover{2|\xi_i|}{\|x\|^4}\|\grad f_1(\phi(x))\|\|x\|\cr
&=\|\grad f_1(\phi(x))\|\Bover{\|x\|+2|\xi_i|}{\|x\|^3}
\le 3\|\grad f_1(\phi(x))\|\cr}$$

\noindent for almost every $x\in F$.   But (since we know that
$\grad f_2$ is
defined almost everywhere, by Rademacher's theorem, as usual) we have

\Centerline{$\|\grad f_2(x)\|
\le 3\sqrt r\|\grad f_1(\phi(x))\|$}

\noindent for almost every $x\in F$.

\medskip

{\bf (d)} We are now in a position to estimate

$$\eqalignno{\int_F\|\grad f_2\|d\mu
&=\int_{B(\tbf{0},\sqrt 2)}\|\grad f_2\|d\mu
  -\int_{B(\tbf{0},1)}\|\grad f_2\|d\mu\cr
\displaycause{because $f_2(x)=0$ if $\|x\|\ge\sqrt 2$}
&=\int_1^{\sqrt 2}\int_{\partial B(\tbf{0},t)}
  \|\grad f_2(x)\|\nu(dx)dt\cr
\displaycause{265G, as usual}
&\le 3\sqrt r\int_1^{\sqrt 2}\int_{\partial B(\tbf{0},t)}
   \|\grad f_1(\Bover1{t^2}x)\|\nu(dx)dt\cr
\displaycause{by (b) above}
&\le 3\sqrt r\int_1^{\sqrt 2}\int_{\partial B(\tbf{0},1/t)}
   t^{2r-2}\|\grad f_1(y)\|\nu(dy)dt\cr}$$

\noindent substituting $x=t^2y$ in the inner integral;  the point
being that as the function $y\mapsto t^2y$ changes all distances by a
scalar multiple $t^2$, it must transform Hausdorff $(r-1)$-dimensional
measure by a multiple $t^{2r-2}$.   But now, substituting $s=\bover1t$
in the outer integral, we have

$$\eqalign{\int_F\|\grad f_2\|d\mu
&\le 3\sqrt r\int_{1/\sqrt 2}^1\Bover1{s^{2r}}
  \int_{\partial B(\tbf{0},s)}\|\grad f_1(y)\|\nu(dy)ds\cr
&\le 2^r\cdot 3\sqrt r\int_{1/\sqrt 2}^1
    \int_{\partial B(\tbf{0},s)}\|\grad f_1(y)\|\nu(dy)ds\cr
&=2^r\cdot 3\sqrt r\int_{B(\tbf{0},1)}\|\grad f_1\|d\mu\cr
&\le 2^{r+2}\sqrt r\int_{B(\tbf{0},1)}4|f|+\|\grad f\|d\mu\cr}$$

\noindent by (a) above.

\medskip

{\bf (e)} Accordingly

$$\eqalign{\int_{\Bbb R^r}\|\grad f_2\|d\mu
&=\int_{B(\tbf{0},1)}\|\grad f\|d\mu+\int_F\|\grad f_2\|d\mu\cr
&\le 2^{r+4}\sqrt r\int_{B(\tbf{0},1)}|f|+\|\grad f\|d\mu.\cr}$$

\noindent But now we can apply 473H to see that

$$\eqalign{\int_{B(\tbf{0},1)}|f|^{r/(r-1)}d\mu
&\le\int|f_2|^{r/r-1}d\mu
\le(\int\|\grad f_2\|d\mu)^{r/(r-1)}\cr
&\le\bigl(2^{r+4}\sqrt r
  \int_{B(\tbf{0},1)}|f|+\|\grad f\|d\mu\bigr)^{r/(r-1)},\cr}$$

\noindent as claimed.
}%end of proof of 473I

\leader{473J}{Lemma} Let $f:\BbbR^r\to\Bbb R$ be a Lipschitz function.
Then

\Centerline{$\int_{B(y,\delta)}|f(x)-f(z)|\mu(dx)
\le \Bover{2^r}r\delta^r\int_{B(y,\delta)}
  \|\grad f(x)\|\|x-z\|^{1-r}\mu(dx)$}

\noindent whenever $y\in\BbbR^r$, $\delta>0$ and $z\in B(y,\delta)$.
%OK for other convex sets of finite diameter?

\proof{{\bf (a)} To begin with, suppose that $f$ is smooth.   In this
case, for any $x$, $z\in B(y,\delta)$,

$$\eqalign{|f(x)-f(z)|
&=\bigl|\int_0^1\Bover{d}{dt}f(z+t(x-z))dt\bigr|\cr
&=\bigl|\int_0^1(x-z)\dotproduct\grad f(z+t(x-z))dt\bigr|\cr
&\le\|x-z\|\int_0^1\|\grad f(z+t(x-z))\|dt.\cr}$$

\noindent So, for $\eta>0$,

$$\eqalignno{\int_{B(y,\delta)\cap\partial B(z,\eta)}
  &|f(x)-f(z)|\nu(dx)\cr
&\le\eta\int_0^1\int_{B(y,\delta)\cap\partial B(z,\eta)}
  \|\grad f(z+t(x-z))\|\nu(dx)dt\cr
\displaycause{$\grad f$ is continuous and bounded, and the subspace
measure induced by $\nu$ on $\partial B(z,\eta)$ is a
(quasi-\nobreak)Radon measure (471E, 471Dh), so
its product with Lebesgue measure also is (417T), and there is no
difficulty with the change in order of integration}
&\le\eta\int_0^1\Bover1{t^{r-1}}
  \int_{B(y,\delta)\cap\partial B(z,t\eta)}
  \|\grad f(w)\|\nu(dw)dt\cr
\displaycause{because if $\phi(x)=z+t(x-z)$, then
$\nu\phi^{-1}[E]=\Bover1{t^{r-1}}\nu E$ whenever $\nu$ measures $E$ and
$t>0$, while $\phi(x)\in B(y,\delta)$ whenever $x\in B(y,\delta)$}
&=\eta^r\int_0^1\int_{B(y,\delta)\cap\partial B(z,t\eta)}
  \|\grad f(w)\|\|w-z\|^{1-r}\nu(dw)dt\cr
&=\eta^{r-1}\int_0^{\eta}\int_{B(y,\delta)\cap\partial B(z,s)}
  \|\grad f(w)\|\|w-z\|^{1-r}\nu(dw)ds\cr
\displaycause{substituting $s=t\eta$}
&=\eta^{r-1}\int_{B(y,\delta)\cap B(z,\eta)}
  \|\grad f(w)\|\|w-z\|^{1-r}\mu(dw).\cr}$$

\noindent So

$$\eqalign{\int_{B(y,\delta)}|f(x)-f(z)|\mu(dx)
&=\int_0^{2\delta}\int_{B(y,\delta)\cap\partial B(z,\eta)}
  |f(x)-f(z)|\nu(dx)d\eta\cr
&\le\int_0^{2\delta}\eta^{r-1}\int_{B(y,\delta)\cap B(z,\eta)}
  \|\grad f(w)\|\|w-z\|^{1-r}\mu(dw)d\eta\cr
&\le\int_0^{2\delta}\eta^{r-1}\int_{B(y,\delta)}
  \|\grad f(w)\|\|w-z\|^{1-r}\mu(dw)d\eta\cr
&=\bover{2^r}r\delta^r\int_{B(y,\delta)}
  \|\grad f(w)\|\|w-z\|^{1-r}\mu(dw).\cr}$$

\medskip

{\bf (b)} Now turn to the general case in which $f$ is not necessarily
differentiable everywhere, but is known to be Lipschitz and bounded.   We need to know that
$\int_{B(y,\delta)}\|x-z\|^{1-r}\mu(dx)$ is finite;  this is because

$$\eqalignno{\int_{B(y,\delta)}\|x-z\|^{1-r}\mu(dx)
&\le\int_{B(z,2\delta)}\|x-z\|^{1-r}\mu(dx)
=\int_0^{2\delta}t^{1-r}\nu(\partial B(z,t))dt\cr
&=\int_0^{2\delta}t^{1-r}r\beta_rt^{r-1}dt
=2\delta r\beta_r.\cr}$$

\noindent Take the sequence $\sequencen{\tilde h_n}$ from 473E.
Then $\sequencen{f*\tilde h_n}$ converges uniformly to $f$ (473Ed),
while $\sequencen{\grad(f*\tilde h_n)}=\sequencen{\tilde h_n*\grad f}$
(473Dd) is uniformly bounded (473Cc, 473Dg) and
converges almost everywhere to $\grad f$ (473Ee).   But this means
that, setting $f_n=f*\tilde h_n$,

$$\eqalignno{\int_{B(y,\delta)}|f(x)-f(z)|\mu(dx)
&=\lim_{n\to\infty}\int_{B(y,\delta)}|f_n(x)-f_n(z)|\mu(dx)\cr
&\le\lim_{n\to\infty}\Bover{2^r}r\delta^r\int_{B(y,\delta)}
  \|\grad f_n(x)\|\|x-z\|^{1-r}\mu(dx)\cr
\displaycause{because every $f_n$ is smooth, by 473De}
&=\Bover{2^r}r\delta^r\int_{B(y,\delta)}
  \|\grad f(x)\|\|x-z\|^{1-r}\mu(dx)\cr}$$

\noindent by Lebesgue's Dominated Convergence Theorem.

\medskip

{\bf (c)} Finally, if $f$ is not bounded on the whole of $\BbbR^r$,
it is surely bounded on $B(y,\delta)$, so we can apply (b) to the function
$x\mapsto\med(-M,f(x),M)$ for a suitable $M\ge 0$ to get the result as
stated.
}%end of proof of 473J

\vleader{48pt}{473K}{Poincar\'e's inequality for balls} Let $B\subseteq\BbbR^r$
be a non-trivial closed ball, and $f:B\to\Bbb R$ a Lipschitz function.
Set $\gamma=\Bover1{\mu B}\int_Bfd\mu$.   Then

\Centerline{$\bigl(\int_B|f-\gamma|^{r/(r-1)}d\mu\bigr)^{(r-1)/r}
\le c\int_B\|\grad f\|d\mu$,}

\noindent where $c=2^{r+4}\sqrt r(1+2^{r+1})$.

\proof{{\bf (a)} To begin with (down to the end of (b)) suppose that
$B$ is the unit ball $B(\tbf{0},1)$.
Then, for any $x\in B$,

$$\eqalign{|f(x)-\gamma|
&=\Bover1{\mu B}
  \bigl|\int_Bf(x)-f(z)\mu(dz)\bigr|\cr
&\le\Bover1{\mu B}\int_B|f(x)-f(z)|\mu(dz)\cr
&\le\Bover{2^r}r\cdot\Bover1{\mu B}
  \int_B\|\grad f(z)\|\|x-z\|^{1-r}\mu(dz),\cr}$$

\noindent by 473J.   Also, for any $z\in B$,

$$\eqalignno{\int_{B(z,2)}\|x-z\|^{1-r}\mu(dx)
&=\int_0^2\int_{\partial B(z,t)}\|x-z\|^{1-r}\nu(dx)dt\cr
&=\int_0^2t^{1-r}\nu(\partial B(z,t))dt
=\int_0^2t^{1-r}t^{r-1}\nu S_{r-1}dt
=2r\beta_r.\cr}$$

\noindent So

$$\eqalign{\int_B|f(x)-\gamma|\mu(dx)
&\le\Bover{2^r}r\cdot\Bover1{\mu B}
  \int_B\int_B\|\grad f(z)\|\|x-z\|^{1-r}\mu(dz)\mu(dx)\cr
&=\Bover{2^r}{r\beta_r}\int_B\int_B
  \|\grad f(z)\|\|x-z\|^{1-r}\mu(dx)\mu(dz)\cr
&\le\Bover{2^r}{r\beta_r}
  \int_B\|\grad f(z)\|\int_{B(z,2)}
  \|x-z\|^{1-r}\mu(dx)\mu(dz)\cr
&\le 2^{r+1}\int_B\|\grad f(z)\|\mu(dz).\cr}$$

\medskip

{\bf (b)} Now apply 473I to $g=f(x)-\gamma$.   We have

$$\eqalignno{\int_B|f-\gamma|^{r/(r-1)}d\mu
&\le\bigl(2^{r+4}\sqrt r
  \int_B\|\grad f\|+|g|d\mu\bigr)^{r/(r-1)}\cr
&\le\bigl(2^{r+4}\sqrt r(1+2^{r+1})
  \int_B\|\grad f\|d\mu\bigr)^{r/(r-1)}\cr
\displaycause{by (a)}
&=\bigl(c\int_B\|\grad f\|d\mu\bigr)^{r/(r-1)}.\cr}$$

\medskip

{\bf (c)} For the general case, express $B$ as $B(y,\delta)$, and set
$h(x)=f(y+\delta x)$ for
$x\in B(\tbf{0},1)$.   Then $\grad h(x)=\delta\grad f(y+\delta x)$ for
almost every $x\in B(\tbf{0},1)$.
Now

\Centerline{$\int_{B(\tbf{0},1)}h\,d\mu
=\Bover1{\delta^r}\int_{B(y,\delta)}fd\mu$,}

\noindent so

\Centerline{$\Bover1{\mu B(\tbf{0},1)}\int_{B(\tbf{0},1)}h\,d\mu
=\Bover1{\mu B(y,\delta)}\int_{B(y,\delta)}fd\mu
=\gamma$.}

\noindent We therefore have

$$\eqalignno{\int_{B(y,\delta)}|f-\gamma|^{r/(r-1)}d\mu
&=\delta^r\int_{B(\tbf{0},1)}|h-\gamma|^{r/(r-1)}d\mu\cr
&\le\delta^r
  \bigl(c\int_{B(\tbf{0},1)}\|\grad h\|d\mu\bigr)^{r/(r-1)}\cr
\displaycause{by (a)-(b) above}
&=\delta^r\bigl(\Bover{\delta c}{\delta^r}
      \int_{B(y,\delta)}\|\grad f\|d\mu\bigr)^{r/(r-1)}\cr
&=\bigl(c\int_{B(y,\delta)}\|\grad f\|d\mu\bigr)^{r/(r-1)}.\cr}$$

\noindent Raising both sides to the power $(r-1)/r$ we have the result
as stated.
}%end of proof of 473K

\cmmnt{\medskip

\noindent{\bf Remark} As will be plain from the way in which the proof
here is constructed, there is no suggestion that the formula offered for
$c$ gives anything near the best possible value.
}%end of comment

\leader{473L}{Corollary} Let $B\subseteq\BbbR^r$ be a non-trivial
closed ball, and $f:B\to[0,1]$ a Lipschitz function.   Set

\Centerline{$F_0=\{x:x\in B,\,f(x)\le\bover14\}$,
\quad$F_1=\{x:x\in B,\,f(x)\ge\bover34\}$.}

\noindent Then

\Centerline{$\bigl(\min(\mu F_0,\mu F_1)\bigr)^{(r-1)/r}
\le 4c\int_B\|\grad f\|d\mu$,}

\noindent where
$c=2^{r+4}\sqrt r(1+2^{r+1})$.

\proof{ Setting $\gamma=\Bover1{\mu B}\int_Bfd\mu$,

$$\eqalign{\int_B|f-\gamma|^{r/(r-1)}d\mu
&\ge\Bover1{4^{r/(r-1)}}\mu F_0\text{ if }\gamma\ge\Bover12,\cr
&\ge\Bover1{4^{r/(r-1)}}\mu F_1\text{ if }\gamma\le\Bover12.\cr}$$

\noindent So 473K tells us that

\Centerline{$\Bover14\bigl(\min(\mu F_0,\mu F_1)\bigr)^{(r-1)/r}
\le c\int_B\|\grad f\|d\mu$,}

\noindent as required.
}%end of proof of 473L

\leader{473M}{The case $r=1$}\cmmnt{ The general rubric for this
section declares that $r$ is taken to be at least $2$, which is clearly
necessary for the formula in 473K to be appropriate.   For the sake of
an application in the next section, however, I mention the elementary
corresponding result when $r=1$.}   In this case, $B$ is just a closed
interval, and \cmmnt{$\grad f$ is the ordinary derivative of $f$;
interpreting $(\int_B|f-\gamma|^{r/(r-1)})^{(r-1)/r}$ as
$\|f\times\chi B-\gamma\chi B\|_{r/(r-1)}$, it is natural to look at}

\Centerline{$\|f\times\chi B-\gamma\chi B\|_{\infty}
=\sup_{x\in B}|f(x)-\gamma|
\le\sup_{x,y\in B}|f(x)-f(y)|
\le\int_B|f'|d\mu$,}

\noindent giving a version of 473K for $r=1$.   \cmmnt{We see that
the formula for $c$ remains valid in the case $r=1$, with a good deal to
spare.}   As for 473L, if $\int_B|f'|<\bover12$ then at least one
of $F_0$, $F_1$ must be empty.

\exercises{\leader{473X}{Basic exercises (a)}
%\spheader 473Xa
Set $f(x)=\max(0,-\ln\|x\|)$, $f_k(x)=\min(k,f(x))$ for
$x\in\BbbR^2\setminus\{0\}$, $k\in\Bbb N$.   Show that
$\lim_{k\to\infty}\|f-f_k\|_2
=\lim_{k\to\infty}\|\grad f-\grad f_k\|_1=0$, so that all the
inequalities 473H-473L %473H 473I 473J 473K 473L
are valid for $f$.
%473L

\spheader 473Xb Let $k\in[1,r]$ be an integer, and set
$m=\Bover{(r-1)!}{(k-1)!(r-k)!}$.   Let $e_1,\ldots,e_r$ be the standard orthonormal basis of $\BbbR^r$ and $\Cal J$ the family of subsets of $\{1,\ldots,r\}$ with $k$ members.   For $J\in\Cal J$ let
$V_J$ be the linear span of $\{e_i:i\in J\}$, $\pi_J:\BbbR^r\to V_J$ the orthogonal projection and $\nu_J$ the normalized $k$-dimensional Hausdorff measure on $V_J$.   Show that if
$A\subseteq\BbbR^r$ then
$(\mu^*A)^m\le\prod_{J\in\Cal J}\nu_J^*\pi_J[A]$.
\Hint{start with $A\subseteq[0,1]^r$ and note that $([0,1]^r)^m$ can be identified with $\prod_{J\in\Cal J}[0,1]^J$.}
%not much to do with anything

\leader{473Y}{Further exercises (a)}
%\spheader 473Ya
Let $D\subseteq\BbbR^r$ be any set and
$\phi:D\to\BbbR^s$ any function.   Show that $D_0=\{x:x\in D,\,\phi$ is
differentiable at $x\}$ is a Borel subset of $\BbbR^r$, and that the
derivative of $\phi$ is a Borel measurable function.
(Compare 225J.)

}%end of exercises

\endnotes{
\Notesheader{473} The point of all the inequalities
473H-473L %473H 473I 473J 473K 473L
is that they bound some measure of variance of a function $f$ by the
integral of $\|\grad f\|$.   If $r=2$, indeed, we are looking at
$\|f\|_2$ (473H) or $\int_B|f|^2$ (473I) or something essentially equal
to the variance of probability theory (473K).   In higher dimensions we
need to look at $\|\,\|_{r/(r-1)}$ in place of $\|\,\|_2$, and when
$r=1$ we can interpret the inequalities in terms of the supremum norm
$\|\,\|_{\infty}$ (473M).   In all cases we want to develop inequalities
which will enable us to discuss a function in terms of its first
derivative.   In one dimension, this is the familiar Fundamental Theorem
of Calculus (Chapter 22).   We find there a
straightforward criterion (`absolute continuity') to determine whether a
given function of one variable
is an indefinite integral, and that if so it is the indefinite
integral of its own derivative.   Even in two dimensions, this
simplicity disappears.   The essential problem is that a function can be
the indefinite integral of an integrable gradient function without being
bounded (473Xa).   The principal results of this section are stated for
Lipschitz functions, but in fact they apply much more widely.   The
argument suggested in 473Xa involves approximating the unbounded
function $f$ by Lipschitz functions $f_k$ in a sharp enough sense to
make it possible to read off all the inequalities for $f$ from the
corresponding inequalities for the $f_k$.   This idea leads naturally to
the concept of `Sobolev space', which I leave on one side for the
moment;  see {\smc Evans \& Gariepy 92}, chap.\ 4, for details.
}%end of notes

\discrpage


