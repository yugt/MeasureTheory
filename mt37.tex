\frfilename{mt37.tex}
\versiondate{6.11.03/10.4.08}
\copyrightdate{1996}

\def\chaptername{Linear operators between function spaces}
\def\sectionname{Introduction}

\newchapter{37}

As everywhere in functional analysis, the function spaces of measure
theory cannot be properly understood without investigating linear
operators between them.   In this chapter I have collected a number of
results which rely on, or illuminate, the measure-theoretic aspects of
the theory.   \S371 is devoted to a fundamental property of linear
operators on $L$-spaces, if considered abstractly, that is, of
$L^1$-spaces, if considered in the language of Chapter 36, and to an
introduction to the class $\Cal T$ of operators which are
norm-decreasing for both $\|\,\|_1$ and $\|\,\|_{\infty}$.   This makes
it possible to prove a version of Birkhoff's Ergodic Theorem for
operators which need not be positive (372D).   In \S372 I give various
forms of this theorem, for linear operators between function spaces, for
measure-preserving Boolean homomorphisms between measure algebras, and
for \imp\ functions between measure spaces, with an excursion into the
theory of continued fractions.   In \S373 I make a fuller analysis of
the class $\Cal T$, with a complete characterization of those $u$, $v$
such that $v=Tu$ for some $T\in\Cal T$.   Using this we can describe 
`rearrangement-invariant' function spaces and extended Fatou norms
(\S374).   Returning to ideas left on one side in \S\S364 and 368, I
investigate positive linear operators defined on $L^0$ spaces (\S375).
In the penultimate section of the chapter (\S376), 
I look at operators which
can be defined in terms of kernels on product spaces.   Finally, in \S377,
I examine the function spaces of reduced products, projective limits
and inductive limits of probability algebras.

\discrpage

