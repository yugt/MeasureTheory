\frfilename{mt443.tex}
\versiondate{14.1.13}
\copyrightdate{2001}

\def\lti{left-\vthsp translation-\vthsp invariant}

\def\chaptername{Topological groups}
\def\sectionname{Further properties of Haar measure}

\newsection{443}

I devote a section to filling in some details of the general theory of
Haar measures before turning to the special topics dealt with in the
rest of the chapter.   The first question concerns the left and right
shift operators acting on sets, on elements of the measure algebra, on
measurable functions and on function spaces.   All these operations can
be regarded as group actions, and, if appropriate topologies are
assigned, they
are continuous actions (443C, 443G).   As an immediate consequence
of this I give an important result about product sets
$\{ab:a\in A,\,b\in B\}$ in a
topological group carrying Haar measures (443D).

The second part of the section revolves around a basic structure
theorem:  all the Haar measures considered here can be reduced to Haar
measures on locally compact Hausdorff groups (443L).   The argument
involves two steps:  the reduction to the Hausdorff case, which is
elementary, and the completion of a Hausdorff topological group.   Since
a group carries more than one natural uniform structure we must take
care to use the correct one, which in this context is the `bilateral'
uniformity (443H-443I, 443K).   On the way I pick up an essential fact
about the approximation of Haar measurable sets by Borel sets (443J).
Finally, I give Halmos' theorem that Haar measures are completion
regular (443M) and a note on the complementary nature of the meager and
null ideals for atomless Haar measure (443O).

In the third part of the section I turn to the special properties of
quotient groups of locally compact groups and the corresponding actions,
following A.Weil.   If $X$ is a locally compact Hausdorff group and $Y$
is a closed subgroup of $X$, then $Y$ is again a locally compact
Hausdorff group, so has Haar measures and a modular function;  at the
same time, we have a natural action of $X$ on the set of left cosets of
$Y$.   It turns out that there is an invariant Radon measure for this
action if and only if the modular function of $Y$ matches that of $X$
(443R).   In this case we can express a left Haar measure of $X$ as an
integral of measures supported by the cosets of $Y$ (443Q).   When $Y$
is a normal subgroup, so that $X/Y$ is itself a locally compact
Hausdorff group, we can relate the modular functions of $X$ and $X/Y$
(443T).   We can apply these results whenever we
have a continuous transitive action of a compact group on a compact
space (443U).


\leader{443A}{Haar measurability}\cmmnt{ I recall and expand on some
facts already noted in 442H.}  Let $X$ be a topological group carrying
Haar measures.

\spheader 443Aa All Haar measures on $X$, whether left or right, have
the same domain
$\Sigma$,\cmmnt{ which I call} the algebra of `Haar measurable' sets,
and the same null ideal $\Cal N$,\cmmnt{ which I call} the ideal
of `Haar negligible' sets.   The corresponding quotient algebra
$\frak A=\Sigma/\Cal N$, the `Haar measure algebra', is the Boolean
algebra underlying
the measure algebra of any Haar measure.   \cmmnt{Because Haar
measures are (by the definition in 441D) quasi-Radon, therefore complete
and strictly localizable
(415A), }$\Sigma_G$ is closed under Souslin's operation\cmmnt{ (431A)} and
$\frak A$ is Dedekind complete\cmmnt{ (322Be)}.   Recall that
any semi-finite measure on $\frak A$\cmmnt{, and in particular any
Haar measure on $X$,} gives rise to the same
measure-algebra topology and uniformity on
$\frak A$\cmmnt{ (324H)}, so we may speak of `the' topology and
uniformity of $\frak A$.

\cmmnt{Because $\Sigma$ is the domain of a left Haar measure,}
$xE\in\Sigma$ whenever $E\in\Sigma$ and $x\in X$;  \cmmnt{because
$\Sigma$ is the domain of a right Haar measure,} $Ex\in\Sigma$ whenever
$E\in\Sigma$ and $x\in X$.
\cmmnt{Similarly,} $xE$ and $Ex$ are Haar negligible whenever $E$ is
Haar negligible and $x\in X$.
\cmmnt{Moreover,} $E^{-1}\cmmnt{\mskip5mu =\{x^{-1}:x\in E\}}$ is
Haar measurable or Haar negligible whenever $E$ is.

Note that $\Sigma$ and $\Cal N$ are invariant in the strong sense that if
$\phi:X\to X$ is any group automorphism which
is also a homeomorphism, then $\Sigma=\{\phi[E]:E\in\Sigma\}$ and
$\Cal N=\{\phi[E]:E\in\Cal N\}$.   \prooflet{\Prf\ If $\mu$ is a left
Haar measure on $X$, let $\nu$ be the image measure $\mu\phi^{-1}$.
Because $\phi$ is a homeomorphism, $\nu$ is a non-zero quasi-Radon measure.
If $\nu$ measures $E$ and $a\in X$, then

\Centerline{$\nu(aE)=\mu\phi^{-1}[aE]=\mu((\phi^{-1}a)(\phi^{-1}[E]))
=\mu\phi^{-1}[E]=\nu E$.}

\noindent So $\nu$ is again a left Haar measure, and has domain $\Sigma$
and null ideal $\Cal N$.   But
$\dom\nu=\{E:\phi^{-1}[E]\in\Sigma\}=\{\phi[E]:E\in\Sigma\}$ and
$\nu^{-1}[\{0\}]=\{E:\phi^{-1}[E]\in\Cal N\}=\{\phi[E]:E\in\Cal N$.\ \Qed}

\spheader 443Ab We\cmmnt{ even} have a symmetric notion of `measurable
envelope' in $X$:  for any $A\subseteq X$, there is a Haar measurable
set $E\supseteq A$ such that $\mu(E\cap F)=\mu^*(A\cap F)$ for any Haar
measurable $F\subseteq X$ and any Haar measure $\mu$ on $X$.
\prooflet{\Prf\ Start with a fixed Haar measure $\mu_0$.   Then $A$ has
a measurable envelope $E$ for $\mu_0$, by 213J and 213L.
Now to say that `$E$
is a measurable envelope for $A$' means just that (i)
$A\subseteq E\in\Sigma$ (ii) if $F\in\Sigma$ and $F\subseteq E\setminus A$
then
$F\in\Cal N$, so $E$ is also a measurable envelope for $A$ for any other
Haar measure on $X$.\ \Qed
}%end of prooflet

In this context I will call $E$ a {\bf Haar measurable envelope} of $A$.

\spheader 443Ac Similarly, we have a notion of {\bf full outer Haar
measure}:  a subset $A$ of $X$ is of full outer Haar measure if $X$ is a
Haar measurable envelope of $A$\cmmnt{, that is, $A\cap E\ne\emptyset$
whenever $E$ is a Haar measurable set which is not Haar negligible, that
is, $A$ is of full outer measure for any (left or right) Haar measure on
$X$}.

\spheader 443Ad For any Haar measure $\mu$ on $X$, we can identify
$L^{\infty}(\mu)$ with $L^{\infty}(\frak A)$\cmmnt{ (363I)} and
$L^0(\mu)$ with $L^0(\frak A)$\cmmnt{ (364Ic)}.
Thus these constructions are independent of $\mu$.   The topology of
convergence in measure of $L^0$ is\cmmnt{ determined by its Riesz
space structure (367T);  so that this also is} independent of the
particular Haar measure we may select.     Of course the same is true of
the norm of $L^{\infty}$.   \cmmnt{Note however that the spaces $L^p$,
for $1\le p<\infty$, are different for left and right Haar measures on
any group which is not unimodular (442Xg), and that even for left
Haar measures $\mu$ the norm on $L^p(\mu)$ changes if $\mu$ is
re-normalized.}

\spheader 443Ae \cmmnt{When it seems appropriate,} I will use the
phrases {\bf
Haar measurable function}, meaning a function measurable with respect to
the $\sigma$-algebra of Haar measurable sets, and {\bf Haar almost
everywhere},
meaning `on the complement of a Haar negligible set'.   Note that we can
identify $L^0(\frak A)$ with the set of equivalence classes in the space
$\eusm L^0$, where $\eusm L^0$ is the space of Haar measurable
real-valued functions defined Haar-a.e.\ in $X$, and $f\sim g$ if $f=g$
Haar-a.e.   \cmmnt{In the language of \S241,
$\eusm L^0=\eusm L^0(\mu)$ for any Haar measure $\mu$ on $X$.}

\spheader 443Af \dvro{We}{Because $E^{-1}\in\Sigma$ for every
$E\in\Sigma$, and $E^{-1}\in\Cal N$ whenever $E\in\Cal N$, we} have a
canonical automorphism $a\mapsto\Reverse{a}:\frak A\to\frak A$ defined by
writing $(E^{\ssbullet})\ssplrarrow=(E^{-1})^{\ssbullet}$ for every
$E\in\Sigma$.   Being an automorphism, this must be a homeomorphism for
the measure-algebra topology of $\frak A$\cmmnt{ (324G)}.
\dvro{If}{In the same way, if} $f\in\eusm L^0$ then
$\Reverse{f}\in\eusm L^0$, where $\Reverse{f}(x)=f(x^{-1})$ whenever this
is defined\cmmnt{ (4A5C(c-ii)), since
$\{x:x\in\dom\Reverse{f},\,\Reverse{f}(x)>\alpha\}
=\{x:x\in\dom f,\,f(x)>\alpha\}^{-1}$ belongs to $\Sigma$ for every
$\alpha$};  and we can define an $f$-algebra automorphism
$u\mapsto\Reverse{u}:L^0\to L^0$ by saying that
$(f^{\ssbullet})\ssplrarrow=(\Reverse{f})^{\ssbullet}$ for $f\in\eusm L^0$.   If
we identify $L^0$ with $L^0(\frak A)$\cmmnt{ rather than with a set of
equivalence classes in $\eusm L^0$}, then we can define the map
$u\mapsto\Reverse{u}$ as the Riesz homomorphism associated with the Boolean
homomorphism $a\mapsto\Reverse{a}:\frak A\to\frak A$\cmmnt{, as in
364P}.
\cmmnt{Note that $\|\Reverse{u}\|_{\infty}=\|u\|_{\infty}$ for every
$u\in L^{\infty}$, but that (unless $X$ is unimodular) other $L^p$
spaces are not invariant under the involution $\ssplrarrow$ (442Xg again).}

\spheader 443Ag If $X$ carries any totally finite\cmmnt{ (left or
right)} Haar measure, it is unimodular, and has a
unique, two-sided, Haar probability measure\cmmnt{ (442Ie)}.
\cmmnt{(In particular, this is
the case whenever $X$ is compact.)}   For such
groups we have $L^p$-spaces, for $1\le p\le\infty$, defined by the group
structure, with canonical norms.

\leader{443B}{Lemma} Let $X$ be a topological group and $\mu$ a left
Haar measure on $X$.   If $E\subseteq X$ is measurable and
$\mu E<\infty$, then for any $\epsilon>0$ there is a neighbourhood $U$
of the
identity $e$ such that $\mu(E\symmdiff xEy)\le\epsilon$ whenever $x$,
$y\in U$.

\proof{ Set $\delta=\Bover{\min(1,\epsilon)}{10+3\mu E}>0$.
Write $\Cal U$ for the
family of open neighbourhoods of $e$.   Because $\mu$ is effectively
locally finite, there is an open set $G_0$ of finite measure such that
$\mu(E\setminus G_0)\le\delta$.   Let $F\subseteq G_0\setminus E$ be a
closed set such that $\mu F\ge\mu(G_0\setminus E)-\delta$, and set
$G=G_0\setminus F$, so that $\mu(G\setminus E)\le\delta$
and $\mu(E\setminus G)\le\delta$.   For
$U\in\Cal U$ set $H_U=\interior\{x:UxU\subseteq G\}$.   Then $\Cal
H=\{H_U:U\in\Cal U\}$ is upwards-directed, and has union $G$, because if
$x\in G$ there is a $U\in\Cal U$ such that $UxUU\subseteq G$, so that
$x\in H_U$.   So there is a $V\in\Cal U$ such that
$\mu(G\setminus H_V)\le\delta$.
Recall that the left modular function $\Delta$ of $X$ is continuous
(442J).   So there is a $U\in\Cal U$ such that $U\subseteq V$ and
$|\Delta(y)-1|\le\delta$ for every $y\in U$.

Now suppose that $x$, $y\in U$.   Set $E_1=E\cap H_V$.   Then
$xE_1y\subseteq G$, so

\Centerline{$\mu(E_1\cup xE_1y)\le\mu G\le\mu E+\delta$.}

\noindent   On the other hand,

\Centerline{$\mu E_1\ge\mu E-\mu(E\setminus G)-\mu(G\setminus H_V)
\ge\mu E-2\delta$,}

\Centerline{$\mu(xE_1y)=\Delta(y)\mu E_1
\ge(1-\delta)(\mu E-2\delta)\ge\mu E-(2+\mu E)\delta$.}

\noindent So

\Centerline{$\mu(E\cap xEy)\ge\mu(E_1\cap xE_1y)
=\mu E_1+\mu(xE_1y)-\mu(E_1\cup xE_1y)
\ge\mu E-(5+\mu E)\delta$.}

\noindent At the same time,

\Centerline{$\mu(xEy)=\Delta(y)\mu E\le(1+\delta)\mu E$.}

\noindent So

\Centerline{$\mu(E\symmdiff xEy)=\mu E+\mu(xEy)-2\mu(E\cap xEy)
\le(10+3\mu E)\delta\le\epsilon$,}

\noindent as required.
}%end of proof of 443B

\leader{443C}{Theorem}  Let $X$ be a topological group carrying Haar
measures, and $\frak A$ its Haar measure algebra.   Then we have
continuous actions of $X$ on $\frak A$ defined by writing

\Centerline{$x\action_lE^{\ssbullet}=(xE)^{\ssbullet}$,
\quad$x\action_rE^{\ssbullet}=(Ex^{-1})^{\ssbullet}$,
\quad$x\action_cE^{\ssbullet}=(xEx^{-1})^{\ssbullet}$}

\noindent for Haar measurable sets $E\subseteq X$ and $x\in X$.

\proof{{\bf (a)} The functions $\action_l$, $\action_r$ and $\action_c$
are all well defined because the maps $E\mapsto xE$, $E\mapsto Ex^{-1}$ and
$E\mapsto xEx^{-1}$ are all Boolean automorphisms of the algebra
$\Sigma$ of Haar measurable sets preserving the ideal of Haar negligible
sets (442G).   It is elementary to check that they are actions of
$X$ on $\frak A$.

Fix a left Haar measure $\mu$ on $X$ and let $\bar\mu$ be the
corresponding measure on $\frak A$.   Then the topology of $\frak A$ is
defined by the pseudometrics $\rho_a$, for $\bar\mu a<\infty$, where
$\rho_a(b,c)=\bar\mu(a\Bcap(b\Bsymmdiff c))$.

\medskip

{\bf (b)} Now suppose that $x_0\in X$, $b_0\in\frak A$, $\bar\mu
a<\infty$ and $\epsilon>0$.   Let $E$, $F_0\in\Sigma$ be such that
$E^{\ssbullet}=a$ and $F_0^{\ssbullet}=b_0$;  set
$\delta=\bover14\epsilon>0$.   Note that

\Centerline{$\mu(x_0^{-1}E\cap F_0)\le\mu(x_0^{-1}E)=\mu E<\infty$.}

\noindent Let $U$ be a neighbourhood of the identity $e$ such that

\Centerline{$\mu(E\symmdiff yE)\le\delta$,
\quad$\mu((x_0^{-1}E\cap F_0)\symmdiff y(x_0^{-1}E\cap F_0))\le\delta$}

\noindent whenever $y\in U$ (443B).   Set

\Centerline{$a'=x_0^{-1}\action_la=(x_0^{-1}E)^{\ssbullet}$.}

\noindent Now suppose that
$x\in Ux_0\cap x_0U^{-1}$ and that $\rho_{a'}(b,b_0)\le\delta$.   Then
$\rho_a(x\action_lb,x_0\action_lb_0)\le\epsilon$.   \Prf\ Let
$F\in\Sigma$ be such that $F^{\ssbullet}=b$.   Then $xx_0^{-1}$ and
$x^{-1}x_0$ both belong to $U$, so

$$\eqalignno{\rho_a(x\action_lb,x_0\action_lb_0)
&=\mu(E\cap(xF\symmdiff x_0F_0))\cr
&=\mu(E\cap xF)
    +\mu(E\cap x_0F_0)
    -2\mu(E\cap xF\cap x_0F_0)\cr
&=\mu(x^{-1}E\cap F)
    +\mu(E\cap x_0F_0)
    -2\mu(x^{-1}x_0(x_0^{-1}E\cap F_0)\cap F)\cr
&\le\mu(x_0^{-1}E\cap F)
    +\mu(x^{-1}E\symmdiff x_0^{-1}E)
    +\mu(x_0^{-1}E\cap F_0)\cr
  &\qquad -2\mu(x_0^{-1}E\cap F_0\cap F)
    +2\mu(x^{-1}x_0(x_0^{-1}E\cap F_0)\symmdiff(x_0^{-1}E\cap F_0))\cr
&\le\mu(x_0^{-1}E\cap(F\symmdiff F_0))
    +\mu(E\symmdiff xx_0^{-1}E)
    +2\delta\cr
&\le\delta+\delta+2\delta
=4\delta=\epsilon. \text{ \Qed}\cr}$$

As $x_0$, $b_0$, $a$ and $\epsilon$ are arbitrary, $\action_l$ is
continuous.

\medskip

{\bf (c)} The same arguments, using a right Haar measure to provide
pseudometrics defining the topology of $\frak A$, show that $\action_r$
is continuous.   (Or use the method of 443X(d-ii).)

\medskip

{\bf (d)} Accordingly the map $(x,y,a)\mapsto x\action_l(y\action_ra)$
is continuous.   So

\Centerline{$(x,a)\mapsto x\action_l(x\action_ra)=x\action_ca$}

\noindent is continuous.
}%end of proof of 443C

\leader{443D}{Proposition} Let $X$ be a topological group carrying Haar
measures.
If $E\subseteq X$ is Haar measurable but not Haar negligible, and
$A\subseteq X$ is not Haar negligible, then

(a) there are $x$, $y\in X$ such that $A\cap xE$, $A\cap Ey$ are not
Haar negligible;

(b) $EA$ and $AE$ both have non-empty interior;

(c) $E^{-1}E$ and $EE^{-1}$ are neighbourhoods of the identity.

\proof{{\bf (a)(i)} Let $\mu$ be any left Haar measure on $X$, and for
Borel sets $F\subseteq X$ set

\Centerline{$\nu F=\sup\{\mu(F\cap IE):I\subseteq X$ is finite$\}$.}

\noindent It is easy to check that $\nu$ is an effectively locally
finite $\tau$-additive Borel measure, inner regular with respect to the
closed sets, because $\{IE:I\in[X]^{<\omega}\}$ is upwards-directed and
$\mu$ is quasi-Radon.   Moreover,

\Centerline{$\nu(xF)=\sup_{I\in[X]^{<\omega}}\mu(xF\cap IE)
=\sup_{I\in[X]^{<\omega}}\mu(F\cap x^{-1}IE)
=\nu F$}

\noindent for every Borel set $F\subseteq X$ and every $x\in X$.
Accordingly the c.l.d.\ version $\tilde\nu$ of $\nu$ is a
left-translation-invariant quasi-Radon measure on $X$ (415Cb);  since
$\nu E>0$, $\tilde\nu$ is non-zero and is itself a left Haar measure.
Consequently $A$ is not
$\tilde\nu$-negligible.   Let $H$ be a measurable envelope of $A$ for
Haar measure (443Ab).   Then $H$ is not Haar negligible, so there is a
closed set $F\subseteq H$ which is not Haar negligible, and
$\nu F=\tilde\nu F>0$.   Thus there is an $x\in X$ such that

\Centerline{$0<\mu(F\cap xE)\le\mu(H\cap xE)=\mu^*(A\cap xE)$,}

\noindent and $A\cap xE$ is not Haar negligible.

\medskip

\quad{\bf (ii)} Applying the same arguments, but starting with a right
Haar measure $\mu$, we see that there is a $y\in X$ such that $A\cap Ey$
is not Haar negligible.

\medskip

{\bf (b)} Let $\mu$ be a left Haar measure on $X$, and $F$ a Haar
measurable envelope of $A$.   The function
$x\mapsto(xE^{-1})^{\ssbullet}:X\to\frak A$ is continuous, where
$\frak A$ is the Haar measure algebra of $X$ (443C), so

\Centerline{$H=\{x:\mu^*(A\cap xE^{-1})>0\}
=\{x:F^{\ssbullet}\cap(xE^{-1})^{\ssbullet}\ne 0\}$}

\noindent is open.   Now

\Centerline{$H\subseteq\{x:A\cap xE^{-1}\ne\emptyset\}=AE$,}

\noindent so $H\subseteq\interior AE$;  and $E^{-1}$ is Haar measurable
and not Haar negligible, so $H\ne\emptyset$, by (a).   Thus $\interior
AE\ne\emptyset$.

Similarly, using a right Haar measure (or observing that
$EA=(A^{-1}E^{-1})^{-1}$), we see that $EA$ has non-empty interior.

\medskip

{\bf (c)} Again taking a left Haar measure $\mu$, $\mu$ is semi-finite, so
there is an $F\subseteq E$ such that $0<\mu F<\infty$.   By 443B, there is
a neighbourhood $U$ of the identity such that $\mu(F\symmdiff xFy)<\mu F$
for all $x$, $y\in U$.   In particular, if $x\in U$,
$\mu(F\setminus xF)<\mu F$, so $F\cap xF\ne\emptyset$ and
$x\in FF^{-1}\subseteq EE^{-1}$;  at the same time,
$\mu(F\setminus Fx)<\mu F$, $F\cap Fx\ne\emptyset$ and
$x\in F^{-1}F\subseteq E^{-1}E$.   So $E^{-1}E$ and $EE^{-1}$ both include
$U$ and are neighbourhoods of the identity.
}%end of proof of 443D

%look:  really need to know that if $E$ has non-zero inner measure,
%$D$ is dense then $D+E$ is conegligible

\vleader{72pt}{443E}{Corollary} Let $X$ be a Hausdorff topological group
carrying Haar measures.   Then the following are equiveridical:

(i) $X$ is locally compact;

(ii) every Haar measure on $X$ is a Radon measure;

(iii) there is some compact subset of $X$ which is not Haar negligible.

\proof{{\bf (i)$\Rightarrow$(ii)} Haar measures are locally finite
quasi-Radon measures (441D, 442Aa), so on locally compact Hausdorff
spaces must be Radon measures (416G).

\medskip

{\bf (ii)$\Rightarrow$(iii)} is obvious, just because
Haar measures are non-zero and any Radon measure
is tight (that is, inner regular with respect to the
closed compact sets).

\medskip

{\bf (iii)$\Rightarrow$(i)} If $K\subseteq X$ is a compact set which is
not Haar negligible, then $KK$ is a compact set with non-empty interior,
so $X$ is locally compact (4A5Eg).
}%end of proof of 443E

\leader{443F}{}\cmmnt{ Later in the chapter we shall need the
following straightforward fact.

\medskip

\noindent}{\bf Lemma} Let $X$ be a topological group carrying Haar
measures, and $Y$ an open subgroup of $X$.   If $\mu$ is a left Haar
measure on $X$, then the subspace measure $\mu_Y$ is a left Haar measure
on $Y$.   Consequently a subset of $Y$ is Haar measurable or Haar
negligible, when regarded as a subset of the topological group $Y$, iff
it is Haar measurable or Haar negligible when regarded as a subset of
the topological group $X$.

\proof{ By 415B, $\mu_Y$ is a quasi-Radon measure;  because $\mu$ is
strictly positive, $\mu_Y$ is non-zero, and it is easy to check that
it is \lti.   So
it is a Haar measure on $Y$.   The rest follows at once from 442H/443A.
}%end of proof of 443F

\leader{443G}{}\cmmnt{ We can repeat the ideas of 443C in terms of
function spaces, as follows.

\medskip

\noindent}{\bf Theorem} Let $X$ be a topological group with a left Haar
measure $\mu$.   Let $\Sigma$ be the domain of $\mu$,
$\eusm L^0=\eusm L^0(\mu)$ the space of $\Sigma$-measurable real-valued
functions defined
almost everywhere in $X$, and $L^0=L^0(\mu)$ the corresponding space of
equivalence classes\cmmnt{ (\S241)}.

(a) $a\action_lf$, $a\action_rf$ and
$a\action_cf$\cmmnt{ (definitions:  4A5C(c-ii))} belong to $\eusm L^0$ for
every $f\in\eusm L^0$ and $a\in X$.

(b) If $a\in X$, then
$\esssup|a\action_lf|=\esssup|a\action_rf|=\esssup|f|$ for every
$f\in\eusm L^{\infty}=\eusm L^{\infty}(\mu)$\cmmnt{, where
$\esssup|f|$ is the essential supremum of $|f|$ (243Da)}.   For
$1\le p<\infty$, $\|a\action_lf\|_p=\|f\|_p$ and
$\|a\action_rf\|_p=\Delta(a)^{-1/p}\|f\|_p$ for every
$f\in\eusm L^p=\eusm L^p(\mu)$, where $\Delta$ is the left modular
function of $X$.

(c) We have shift actions of $X$ on $L^0$ defined by setting

\Centerline{$a\action_lf^{\ssbullet}=(a\action_lf)^{\ssbullet}$,
\quad$a\action_rf^{\ssbullet}=(a\action_rf)^{\ssbullet}$,
\quad$a\action_cf^{\ssbullet}=(a\action_cf)^{\ssbullet}$}

\noindent for $a\in X$ and $f\in\eusm L^0$.   If
$\Reverse{\phantom{x}}$ is
the reversal operator on $L^0$ defined in 443Af, we have

\Centerline{$a\action_l\Reverse{u}=(a\action_ru)\ssplrarrow$,
\quad$a\action_c\Reverse{u}=(a\action_cu)\ssplrarrow$}

\noindent for every $a\in X$ and $u\in L^0$.

(d)\dvAnew{2010} If we give $L^0$ its
topology of convergence in measure these three actions, and also the
reversal operator $\ssplrarrow$, are continuous.

(e) For $1\le p\le\infty$ the formulae of (c) define actions of $X$ on
$L^p=L^p(\mu)$, and $\|a\action_lu\|_p=\|u\|_p$ for every $u\in L^p$,
$a\in X$;  interpreting $\Delta(a)^{-1/\infty}$ as $1$ if necessary,
$\|a\action_ru\|_p=\Delta(a)^{-1/p}\|u\|_p$ whenever $u\in L^p$ and
$a\in X$.

(f)\dvAformerly{4{}43Ge}
For $1\le p<\infty$ these actions are continuous.

\proof{{\bf (a)} Let $f\in\eusm L^0$.   Then $F=\dom f$ is conegligible,
so $aF=\dom a\action_lf$ and $Fa^{-1}=\dom a\action_rf$ are conegligible
(442G).
For any $\alpha\in\Bbb R$, set $E_{\alpha}=\{x:x\in F,\,f(x)<\alpha\}$;
then $\{x:(a\action_lf)(x)<\alpha\}=aE_{\alpha}$ and
$\{x:(a\action_rf)(x)<\alpha\}=E_{\alpha}a^{-1}$ are measurable, so
$a\action_lf$ and
$a\action_rf$ are measurable.   Thus $a\action_lf$ and $a\action_rf$
belong to $\eusm L^0$.   It follows at once that
$a\action_cf=a\action_l(a\action_rf)$ belongs to $\eusm L^0$.

\medskip

{\bf (b)(i)} For $\alpha\ge 0$,

$$\eqalign{\esssup|f|\le\alpha
&\iff|f(x)|\le\alpha\text{ for almost all }x\cr
&\iff|(a\action_lf)(x)|\le\alpha\text{ for almost all }x\cr
&\iff|(a\action_rf)(x)|\le\alpha\text{ for almost all }x\cr}$$

\noindent because the null ideal of $\mu$ is invariant under both
left and right translations.   So
$\esssup|f|=\esssup|a\action_lf|=\esssup|a\action_rf|$.

\medskip

\quad{\bf (ii)} For $1\le p<\infty$,

$$\eqalignno{\|a\action_lf\|_p^p
&=\int|(a\action_lf)(x)|^p\mu(dx)
=\int|f(a^{-1}x)|^p\mu(dx)
=\int|f(x)|^p\mu(dx)\cr
\noalign{\noindent (441J)}
&=\|f\|_p^p,\cr
\|a\action_rf\|_p^p
&=\int|(a\action_rf)(x)|^p\mu(dx)
=\int|f(xa)|^p\mu(dx)
=\Delta(a^{-1})\int|f(x)|^p\mu(dx)\cr
\noalign{\noindent (442Kc)}
&=(\Delta(a)^{-1/p}\|f\|_p)^p.\cr}$$

\medskip

{\bf (c)(i)} I have already checked that $a\action_lf$, $a\action_rf$
and $a\action_cf$ belong to
$\eusm L^0$ whenever $f\in\eusm L^0$ and $a\in X$.   If $f$,
$g\in\eusm L^0$ and $f\eae g$, let $E$ be the conegligible set
$\{x:x\in\dom f\cap\dom g,\,f(x)=g(x)\}$;  then $aE$ and $Ea^{-1}$ and
$aEa^{-1}$ are conegligible and

\Centerline{$(a\action_lf)(x)=(a\action_lg)(x)$ for every $x\in aE$,
\quad$(a\action_rf)(x)=(a\action_rg)(x)$ for every $x\in Ea^{-1}$,}

\Centerline{$(a\action_cf)(x)=(a\action_cg)(x)$ for every
$x\in aEa^{-1}$,}

\noindent so $a\action_lf\eae a\action_lg$,
$a\action_rf\eae a\action_rg$ and
$a\action_cf\eae a\action_cg$.   Accordingly the formulae given
define functions $\action_l$, $\action_r$ and $\action_c$ from
$X\times L^0$ to $L^0$.   They are actions just because the original
$\action_l$, $\action_r$ and $\action_c$ are actions of $X$ on
$\eusm L^0$ (4A5Cc-4A5Cd).

\medskip

\quad{\bf (ii)} If $f\in\eusm L^0$, then

\Centerline{$(a\action_l\Reverse{f})(x)
=\Reverse{f}(a^{-1}x)
=f(x^{-1}a)
=(a\action_rf)(x^{-1})
=(a\action_rf)\ssplrarrow(x)$}

\noindent when any of these is defined, which is almost everywhere, so
$a\action_lu=(a\action_ru)\ssplrarrow$ for every $u\in L^0$.
Similarly,

\Centerline{$(a\action_c\Reverse{f})(x)
=\Reverse{f}(a^{-1}xa)
=f(a^{-1}x^{-1}a)
=(a\action_cf)(x^{-1})
=(a\action_cf)\ssplrarrow(x)$}

\noindent and $a\action_c\Reverse{u}=(a\action_cu)\ssplrarrow$.

\medskip

{\bf (d)(i)} In 367T there is a description of convergence in measure on
$L^0$ in terms of its Riesz space structure.   As $\ssplrarrow$ is a Riesz
space automorphism of $L^0$, it must also be a homeomorphism for the
topology of convergence in measure.

\medskip

\quad{\bf (ii)} To see that $\action_l$ is continuous,
it will be convenient to work
with the space $\eusm L^0_{\Sigma}$ of Haar measurable
real-valued functions
defined on the whole of $X$.   I will use a characterization of
convergence in measure from 245F:  a subset $W$ of $L^0$ is open iff
whenever $f_0^{\ssbullet}\in W$ there are a set $E$ of finite measure
and an $\epsilon>0$ such that $f^{\ssbullet}\in W$ whenever
$\mu\{x:x\in E,\,|f(x)-f_0(x)|>\epsilon\}\le\epsilon$.   Now if $E$ is a
measurable
set of finite measure, $f\in\eusm L^0_{\Sigma}$ and $\epsilon>0$,
there is a neighbourhood $U$ of the identity $e$ of $X$ such that
$\mu\{x:x\in E,\,|f(ax)-f(x)|\ge\epsilon\}\le\epsilon$ for every $a\in U$.
\Prf\ Let $m\ge 1$ be such that
$\mu\{x:x\in E,\,|f(x)|\ge m\epsilon\}\le\bover12\epsilon$.   For
$-m\le k<m$, set
$E_k=\{x:x\in E,\,k\epsilon\le f(x)<(k+1)\epsilon\}$.   By 443B, there
is a neighbourhood $U$ of $e$ such that

\Centerline{$\mu(E_k\symmdiff a^{-1}E_k)\le\Bover{\epsilon}{4m}$}

\noindent whenever $a\in U$ and $-m\le k<m$.   Now, for $a\in U$,

\Centerline{$\{x:x\in E,\,|f(ax)-f(x)|\ge\epsilon\}
\subseteq\{x:x\in E,\,|f(x)|\ge m\epsilon\}
  \cup\bigcup_{k=-m}^{m-1}(E_k\symmdiff a^{-1}E_k)$}

\noindent has measure at most

\Centerline{$\Bover{\epsilon}2+2m\Bover{\epsilon}{4m}=\epsilon$.  \Qed}

\medskip

\quad{\bf (iii)} Let $E$ be a measurable set of finite measure, $a_0\in X$,
$f_0\in\eusm L^0_{\Sigma}$ and $\epsilon>0$.
Set $\delta=\epsilon/3>0$.   Note that $\mu(a_0^{-1}E)=\mu E$ is finite.
Let $U$ be a neighbourhood of $e$ such that

\Centerline{$\mu\{x:x\in a_0^{-1}E,\,
|f_0(yx)-f_0(x)|\ge\delta\}\le\delta$,
\quad$\mu(yE\symmdiff E)\le\delta$}

\noindent whenever $y\in U$.

Now suppose that $a\in Ua_0\cap a_0U^{-1}$
and that $f\in\eusm L^0_{\Sigma}$ is such that
$\mu\{x:x\in a_0^{-1}E,\,|f(x)-f_0(x)|\ge\delta\}\le\delta$.
In this case,

$$\eqalign{\{x:x&\in E,\,|f(a^{-1}x)-f_0(a_0^{-1}x)|\ge\epsilon\}\cr
&\subseteq\{x:x\in E,\,|f(a^{-1}x)-f_0(a^{-1}x)|\ge\delta\}\cr
  &\qquad\qquad  \cup\{x:x\in E,\,
     |f_0(a^{-1}x)-f_0(a_0^{-1}x)|\ge\delta\}\cr
&\subseteq(E\symmdiff aa_0^{-1}E)
  \cup\{x:x\in aa_0^{-1}E,\,
     |f(a^{-1}x)-f_0(a^{-1}x)|\ge\delta\}\cr
  &\qquad\qquad  \cup a_0\{w:w\in a_0^{-1}E,\,
     |f_0(a^{-1}a_0w)-f_0(w)|\ge\delta\}\cr
&\subseteq(E\symmdiff aa_0^{-1}E)
  \cup a\{w:w\in a_0^{-1}E,\,\,
     |f(w)-f_0(w)|\ge\delta\}\cr
  &\qquad\qquad  \cup a_0\{w:w\in a_0^{-1}E,\,
     |f_0(a^{-1}a_0w)-f_0(w)|\ge\delta\}\cr}$$

\noindent has measure at most $3\delta=\epsilon$
because $aa_0^{-1}$, $e$ and $a^{-1}a_0$ all belong to $U$.
Because $E$ and $\epsilon$ are arbitrary, the function
$(a,u)\mapsto a\action_lu$ is continuous at
$(a_0,f_0^{\ssbullet})$;
as $a_0$ and $f_0$ are arbitrary, $\action_l$ is continuous.

\medskip

\quad{\bf (iv)} Now

\Centerline{$(a,u)\mapsto a\action_ru
=(a\action_l\Reverse{u})\ssplrarrow$}

\noindent must also be continuous.   It follows at once that $\action_c$
is continuous, since
$a\action_cu=a\action_l(a\action_ru)$.

\medskip

{\bf (e)} follows at once from (b) and (c).

\medskip

{\bf (f)} Fix $p\in\coint{1,\infty}$.

\medskip

\quad{\bf (i)} If $u\in L^p$ and $\epsilon>0$, there is a neighbourhood
$U$ of $e$ such that
$\|u-y\action_l(z\action_ru)\|_p\le\epsilon$ whenever $y$, $z\in U$.
\Prf\ When $u$ is of the form $(\chi E)^{\ssbullet}$, where
$\mu E<\infty$, we have

\Centerline{$y\action_l(z\action_ru)=\chi(yEz^{-1})^{\ssbullet}$,
\quad$\|u-y\action_l(z\action_ru)\|_p=\mu(E\symmdiff yEz^{-1})^{1/p}$,}

\noindent so the result is immediate from 443B.   If
$u=\sum_{i=0}^n\alpha_i(\chi E_i)^{\ssbullet}$, where every $E_i$ has
finite measure, then, setting $u_i=(\chi E_i)^{\ssbullet}$ for each $i$,

\Centerline{$\|u-y\action_l(z\action_ru)\|_p
\le\sum_{i=0}^n|\alpha_i|\|u_i-y\action_l(z\action_ru_i)\|_p
\le\epsilon$}

\noindent whenever $y$ and $z$ are close enough to $e$.
In general, there is a
$v$ of this form such that $\|u-v\|_p\le\bover14\epsilon$.   If we take
a neighbourhood $U$ of $e$ such that
$\|v-y\action_l(z\action_rv)\|_p\le\bover14\epsilon$ and
$\Delta(z)^{-1/p}\le 2$ whenever $y$, $z\in U$, then

\Centerline{$\|y\action_l(z\action_ru)-y\action_l(z\action_rv)\|_p
=\Delta(z)^{-1/p}\|u-v\|_p\le \Bover12\epsilon$}

\noindent whenever $z\in U$, so

\Centerline{$\|u-y\action_l(z\action_ru)\|_p
\le\|u-v\|_p+\|v-y\action_l(z\action_r v)\|_p
  +\|y\action_l(z\action_rv)-y\action_l(z\action_ru)\|_p
\le\epsilon$}

\noindent whenever $y$, $z\in U$.\ \Qed

\medskip

\quad{\bf (ii)} Now suppose that $u_0\in L^p$, $a_0$, $b_0\in X$ and
$\epsilon>0$.   Set $v_0=a_0\action_l(b_0\action_ru_0)$ and
$\delta=\epsilon/(1+2\Delta(b_0)^{-1/p})>0$.   Let $U$ be a
neighbourhood of $e$ such that
$\Delta(y)^{-1/p}\le 2$ and
$\|v_0-y\action_l(z\action_rv_0)\|_p\le\delta$ whenever $y$, $z\in U$.
If $a\in Ua_0$, $b\in Ub_0$ and $\|u-u_0\|_p\le\delta$, then

$$\eqalign{\|a\action_l(b\action_ru)-v_0\|_p
&\le\|a\action_l(b\action_ru)-a\action_l(b\action_ru_0)\|_p
 +\|a\action_l(b\action_ru_0)-v_0\|_p\cr
&=\Delta(b)^{-1/p}\|u-u_0\|_p
  +\|aa_0^{-1}\action_l(bb_0^{-1}\action_rv_0)
       -v_0\|_p\cr
&\le\Delta(bb^{-1}_0)^{-1/p}\Delta(b_0)^{-1/p}\delta
  +\delta
\le\delta(1+2\Delta(b_0)^{-1/p})
=\epsilon.\cr}$$

\noindent As $\epsilon$ is arbitrary,
$(a,b,u)\mapsto a\action_l(b\action_ru)$ is continuous at
$(a_0,b_0,u_0)$.

As in (c), this is enough to show that $\action_l$, $\action_r$ and
$\action_c$ are all continuous actions.
}%end of proof of 443G

\cmmnt{\medskip

\noindent{\bf Remark} I have written this out for a left Haar measure
$\mu$, since the spaces $L^p(\mu)$ depend on this;  if $\nu$ is a right
Haar measure, and $X$ is not unimodular, then $L^p(\mu)\ne L^p(\nu)$ for
$1\le p<\infty$.   But recall that the topology of convergence in
measure on $L^0$ is the same for all Haar measures (443Ad),
so (c) above, and the case $p=\infty$ of (b) and (d), are two-sided;
they belong to the theory of the Haar measure algebra.
}%end of comment

\leader{443H}{Theorem} Let $X$ be a topological group carrying Haar
measures.   Then there is a neighbourhood of the identity which is
totally bounded for the bilateral uniformity on $X$.

\proof{ Let $\mu$ be a left Haar measure on $X$.   Let $V_0$ be a
neighbourhood of the identity $e$ such that $\mu V_0<\infty$ (442Aa).
Let $V$ be a neighbourhood of $e$ such that $VV\subseteq V_0$ and
$V^{-1}=V$.

\Quer\ Suppose, if possible, that $V$ is not totally bounded for the
bilateral uniformity on $X$.   By 4A5Oa, one of the following must
occur:

\medskip

{\bf case 1} There is an open neighbourhood $U$ of $e$ such that
$V\not\subseteq IU$ for any finite set $I\subseteq X$.   In this case,
we may choose a sequence $\sequencen{x_n}$ in $V$ inductively such that
$x_n\notin x_iU$ whenever $i<n$.   Let $U_1$ be an open neighbourhood of
$e$ such that $U_1\subseteq V$ and $U_1U_1^{-1}\subseteq U$;  then
$\sequencen{x_nU_1}$ is disjoint.   Since $\mu(x_nU_1)=\mu U_1>0$ for
every $n$ (by the other clause in 442Aa),
$\mu(\bigcup_{n\in\Bbb N}x_nU_1)=\infty$;  but
$x_nU_1\subseteq V_0$ for every $n$, so this is impossible.

\medskip

{\bf case 2} There is an open neighbourhood $U$ of $e$ such that
$V\not\subseteq UI$ for any finite set $I\subseteq X$.   So we may
choose a sequence $\sequencen{x_n}$ in $V$ inductively such that
$x_n\notin Ux_i$ whenever $i<n$.   Let $U_1$ be an open neighbourhood of
$e$ such that $U_1\subseteq V$ and $U_1U_1^{-1}\subseteq U$;  then
$\sequencen{U_1^{-1}x_n}$ is disjoint, so $\sequencen{x_n^{-1}U_1}$ is
also disjoint.   Since $\mu(x_n^{-1}U_1)=\mu U_1>0$ for every $n$,
$\mu(\bigcup_{n\in\Bbb N}x_n^{-1}U_1)=\infty$;  but
$x_n^{-1}U_1\subseteq V_0$ for every $n$, so this also is
impossible.\ \Bang

Thus $V$ is totally bounded for the bilateral uniformity on $X$, and we
have the required totally bounded neighbourhood of $e$.
}%end of proof of 443H

\leader{443I}{Corollary}
Let $X$ be a topological group.   If $A\subseteq X$
is totally bounded for the bilateral
uniformity of $X$, it has finite outer measure for any (left or right)
Haar measure on $X$.

\proof{ If $\mu$ is a Haar measure on $X$, let $U$ be an open
neighbourhood of the identity $e$ of finite measure.   There is a finite
set $I$ such that $A\subseteq IU\cap UI$ (4A5Oa again), so that
$\mu^*A\le\#(I)\mu U$ is finite.
}%end of proof of 443I

\vleader{48pt}{443J}{Proposition}
Let $X$ be a topological group carrying Haar measures, and $\frak A$ its
Haar measure algebra.

(a) There is an open-and-closed subgroup $Y$ of $X$ such that, for any
Haar measure $\mu$ on $X$, $Y$ can be covered by countably many open
sets of finite measure.

(b)(i) If $E\subseteq X$ is any Haar measurable set, there are an
F$_{\sigma}$ set $E'\subseteq E$ and a G$_{\delta}$ set $E''\supseteq E$
such that $E''\setminus E'$ is Haar negligible.

\quad(ii) Every Haar negligible set is included in a Haar negligible
Borel set, and for every Haar measurable set $E$ there is a Borel set
$F$ such that $E\symmdiff F$ is Haar negligible.

\quad(iii) The Haar measure algebra $\frak A$ of $X$ may be identified
with $\Cal B/\Cal I$, where $\Cal B$ is the Borel $\sigma$-algebra of
$X$ and
$\Cal I$ is the ideal of Haar negligible Borel sets.

\quad(iv) Every member of $L^0(\frak A)$ can be identified with the
equivalence class of some Borel measurable function from $X$ to
$\Bbb R$.
Every member of $L^{\infty}(\frak A)$ can be identified with the
equivalence class of a bounded Borel measurable function from $X$ to
$\Bbb R$.

\proof{{\bf (a)} Let $V$ be an open neighbourhood of the identity which
is totally bounded for the bilateral uniformity of $X$ (443H);  we may
suppose that $V^{-1}=V$.   Set $Y=V\cup VV\cup VVV\cup VVVV\cup\ldots$.
Then $Y$ is an open subgroup of $X$, therefore also closed (4A5Ek).
By 4A5Ob, every power of $V$ is totally bounded, so $Y$ is a countable
union of totally bounded sets.   If $\mu$
is any left Haar measure on $X$, then any totally bounded set has finite
outer measure for $\mu$ (443I).   Thus $Y$ is a countable union of
sets of finite measure for $\mu$.   The same argument applies to right
Haar measures, so $Y$ is a subgroup of the required form.

\medskip

{\bf (b)(i)} Let $E\subseteq X$ be a Haar measurable set, and fix a left
Haar measure $\mu$ on $X$.   Take the open subgroup $Y$ of (a), and
index the set of its left cosets as $\familyiI{Y_i}$;  because any
translate of a totally bounded set is totally bounded (4A5Ob again), each
$Y_i$ is an open set expressible as $\bigcup_{n\in\Bbb N}H_{in}$, where
every $H_{in}$ is a totally bounded open set, so that $\mu H_{in}$ is
finite.

For $i\in I$ and $m$, $n\in\Bbb N$ there is a closed set
$F_{imn}\subseteq E\cap H_{im}$ such that
$\mu F_{imn}\ge\mu(E\cap H_{im})-2^{-n}$.   Set
$F_{mn}=\bigcup_{i\in I}F_{imn}$ for each $m$, $n\in\Bbb N$;   then
$F_{mn}$ is closed (4A2Bb).   So $E'=\bigcup_{m,n\in\Bbb N}F_{mn}$ is
F$_{\sigma}$.   For each $i\in I$,

\Centerline{$(E\setminus E')\cap Y_i
\subseteq\bigcup_{m\in\Bbb N}
\bigl(E\cap H_{im}\setminus\bigcup_{n\in\Bbb N}F_{imn}\bigr)$}

\noindent is negligible;  thus
$\{G:G\subseteq X$ is open, $\mu(G\cap E\setminus E')=0\}$ covers $X$
and $E\setminus E'$ must be negligible (414Ea).

In the same way, there is an F$_{\sigma}$ set $F^*\subseteq X\setminus
E$ such that $(X\setminus E)\setminus F^*$ is negligible;  now
$E''=X\setminus F^*$ is G$_{\delta}$ and $E''\setminus E$ is negligible,
so $E''\setminus E'$ also is.   (I am speaking here as if `negligible'
meant `$\mu$-negligible'.   But of course this is the same thing as the
`Haar negligible' of the statement of the proposition.)

\medskip

\quad{\bf (ii), (iii), (iv)} follow at once.
}%end of proof of 443J

\vleader{48pt}{443K}{Theorem} 
Let $X$ be a Hausdorff topological group carrying
Haar measures.   Then the completion $\widehat{X}$ of $X$ under its
bilateral uniformity is a locally compact Hausdorff group, and $X$ is of
full outer Haar measure in $\widehat{X}$.   Any\cmmnt{ (left or
right)} Haar
measure on $X$ is the subspace measure corresponding to a Haar measure
\cmmnt{(of the same chirality)} on $\widehat{X}$.

\proof{{\bf (a)} By 443H and 4A5N,
$\widehat{X}$ is a locally compact Hausdorff
group in which $X$ is embedded as a dense subgroup.

\medskip

{\bf (b)} Let $\mu$ be a left Haar measure on $X$.   Then there is a
Radon measure $\lambda$ on $\widehat{X}$ such that $\mu$ is the
subspace measure $\lambda_X$.   \Prf\ For Borel sets
$E\subseteq\widehat{X}$, set $\nu E=\mu(X\cap E)$.   Then $\nu$ is a
Borel measure, and it is $\tau$-additive because $\mu$ is.
Any point of $\widehat{X}$
has a compact neighbourhood $V$ in $\widehat{X}$;  now $V$ must be
totally bounded
for the bilateral uniformity of $\widehat{X}$ (4A2Je), so $V\cap X$ is
totally bounded for the bilateral uniformity of $X$ (4A5Ma), and
$\nu V=\mu(V\cap X)$ is
finite (443I).   Thus $\nu$ is locally finite.   If $\nu E>0$, there is
an open set $H\subseteq X$ such that $\mu H<\infty$ and
$\mu(H\cap X\cap E)>0$, because $\mu$ is effectively locally finite;
now there is an
open set $G\subseteq\widehat{X}$ such that $H=X\cap G$, so that
$\nu G<\infty$ and $\nu(G\cap E)>0$.   Thus $\nu$ is effectively locally
finite.

By 416H, the c.l.d.\ version $\lambda$ of $\nu$ is a Radon measure on
$\widehat{X}$.   Since $\lambda K=\nu K=0$ whenever
$K\subseteq\widehat{X}\setminus X$ is compact, $X$ is of full outer
measure for $\lambda$.   Accordingly

\Centerline{$\lambda_X(G\cap X)=\lambda G=\nu G=\mu(G\cap X)$}

\noindent for every open set $G\subseteq\widehat{X}$, and
$\lambda_X=\mu$, because they are quasi-Radon measures agreeing on the
open sets (415B, 415H(iii)).\ \Qed

\medskip

{\bf (c)} Continuing the argument of (b), $\lambda$ is a left Haar
measure on $\widehat{X}$.   \Prf\ Let $G\subseteq\widehat{X}$ be open,
and $z\in\widehat{X}$.   If $K\subseteq zG$ is compact, then
$z^{-1}K\subseteq G$, and $\{w:w\in\widehat{X},\,wK\subseteq G\}$ is a
non-empty open set (4A5Ei), so meets $X$.   Take $x\in X$ such that
$xK\subseteq G$;  then

\Centerline{$\lambda G=\mu(X\cap G)\ge\mu(X\cap xK)
=\mu(x(X\cap K))=\mu(X\cap K)=\lambda K$.}

\noindent As $K$ is arbitrary, $\lambda(zG)\le\lambda G$.   By
441Ba, $\lambda$ is invariant under the left action of $\widehat{X}$
on itself, that is, is a left Haar measure.\ \Qed

We know that $X$ is of full outer measure for $\lambda$, so this shows
that it has full outer Haar measure in $\widehat{X}$.

\medskip

{\bf (d)} The same arguments, looking at $Gz$ and $Kz^{-1}$ in (c), show
that if $\mu$ is a right Haar measure on $X$ it is the subspace measure
$\lambda_X$ for a right Haar measure $\lambda$ on $\widehat{X}$.
}%end of proof of 443K

\leader{443L}{Corollary}\discrversionA{\footnote{Expanded 2009.}}{}
Let $X$ be any topological group with a Haar
measure $\mu$.   Then we can find $Z$, $\lambda$ and $\phi$ such that

(i) $Z$ is a locally compact Hausdorff topological group;

(ii) $\lambda$ is a Haar measure on $Z$;

(iii) $\phi:X\to Z$ is a continuous homomorphism,
\imp\ for $\mu$ and $\lambda$;

(iv) $\mu$ is inner regular with respect to
$\{\phi^{-1}[K]:K\subseteq Z$ is compact$\}$;

(v) if $E\subseteq X$ is Haar measurable, we can find a Haar measurable set
$F\subseteq Z$ such that $\phi^{-1}[F]\subseteq E$ and
$E\setminus\phi^{-1}[F]$ is Haar negligible;

(vi) a set $G\subseteq X$ is an open set in $X$ iff it is of the
form $\phi^{-1}[H]$ for some open set $H\subseteq Z$;

(vii) a set $G\subseteq X$ is a regular open set in $X$ iff it is of the
form $\phi^{-1}[H]$ for some regular open set $H\subseteq Z$;

(viii) a set $A\subseteq X$ is nowhere dense in $X$ iff $\phi[A]$ is
nowhere dense in $Z$.

%\query How far are (Z,\lambda,\phi)  unique?
%indeed:  if  G  is the subgroup of  \Aut\frak A  generated by the
%action of  X , how far does  (\Aut\frak A,G)  determine a
%locally compact Hausdorff topological group  X ?

\proof{{\bf (a)} Let $Y\subseteq X$ be the closure of $\{e\}$, where $e$
is the identity of $X$.   Then $Y$ is a closed normal subgroup of $X$,
and if $\phi:X\to X/Y$ is the quotient map, every open (or closed)
subset of $X$ is of the form $\phi^{-1}[H]$ for some open (or closed)
set $H\subseteq Y$ (4A5Kb).

Consider the image measure $\nu=\mu\phi^{-1}$ on $X/Y$.   This is
quasi-Radon.   \Prf\ Because $\mu$ is a complete $\tau$-additive
topological measure, so is $\nu$.   If $F\in\dom\nu$ and $\nu F>0$,
there is an open set $G\subseteq X$ such that $\mu G<\infty$ and
$\mu(G\cap\phi^{-1}[F])>0$;  now $G=\phi^{-1}[H]$ for some open set
$H\subseteq X/Y$, and $\nu H=\mu G$ is finite, while $\nu(H\cap
F)=\mu(G\cap\phi^{-1}[F])>0$.   Thus $\nu$ is effectively locally finite
(therefore semi-finite).   Again, if $F\in\dom\nu$ and $\nu F>\gamma$,
there is a closed set $E\subseteq\phi^{-1}[F]$ such that $\mu
E\ge\gamma$;  now $E$ is expressible as $\phi^{-1}[H]$ for some closed
set $H\subseteq X/Y$;  because $\phi$ is surjective, $H\subseteq F$, and
$\nu H=\mu E\ge\gamma$.   Thus $\nu$ is inner regular with respect to
the closed sets.   Finally, suppose that $F\subseteq X/Y$ is such that
$F\cap F'\in\dom\nu$ whenever $\nu F'<\infty$.  If $E\subseteq X$ is a
closed set of finite measure, it is of the form $\phi^{-1}[F']$ where
$\nu F'=\mu E<\infty$, so $F'\cap F\in\dom\nu$ and
$E\cap\phi^{-1}[F]\in\dom\mu$;  by 412Ja, we can conclude that
$\phi^{-1}[F]\in\dom\mu$ and $F\in\dom\nu$.   Thus $\nu$ is locally
determined and is a quasi-Radon measure.\ \Qed

We find also that $\nu$ is a left Haar measure.   \Prf\ If $z\in X/Y$
and $F\in\dom\nu$, express $z$ as $\phi(x)$ where $x\in X$;  then
$\phi^{-1}[zF]=x\phi^{-1}[F]$, so

\Centerline{$\nu(zF)=\mu(x\phi^{-1}[F])=\mu\phi^{-1}[F]=\nu F$. \Qed}

\medskip

{\bf (b)} Thus $X/Y$ is a topological group with a left Haar measure
$\nu$.   Because $Y$ is closed, $X/Y$ is Hausdorff (4A5J(b-ii-$\alpha$)).  
We can
therefore form its completion $Z=\widehat{X/Y}$, a locally compact
Hausdorff group, and find a left Haar measure $\lambda$ on $Z$ such
that $\nu$ is the corresponding subspace measure on $X/Y$, which is of
full outer measure for $\lambda$ (443K).
The embedding $X/Y\embedsinto Z$ is
therefore \imp\ for $\nu$ and $\lambda$, so that $\phi$, regarded as a
map from $X$ to $Z$, is \imp\ for $\mu$ and $\lambda$.   Also, of
course, $\phi:X\to Z$ is a continuous homomorphism.

If $E\subseteq X$ and $\mu E>\gamma$, there is a closed set
$E'\subseteq E$ such that $\mu E'>\gamma$.
Now $E'$ is of the form $\phi^{-1}[F]$
where $F\subseteq X/Y$ is closed and $\nu F=\mu E'>\gamma$.   Next, $F$
is of the form $(X/Y)\cap F'$ where $F'\subseteq Z$ is closed and
$\lambda F'=\nu F>\gamma$.   So there is a compact set $K\subseteq F'$
such that $\lambda K\ge\gamma$, and we have

\Centerline{$\phi^{-1}[K]\subseteq\phi^{-1}[F']
=\phi^{-1}[F]\subseteq E$,
\quad$\mu\phi^{-1}[K]=\nu(K\cap(X/Y))=\lambda K\ge\gamma$.}

\noindent As $E$ and $\gamma$ are arbitrary, $\mu$ is inner regular with
respect to $\{\phi^{-1}[K]:K\subseteq Z$ is compact$\}$.
So (i)-(iv) are true.

\medskip

{\bf (c)} If $E\subseteq X$ is Haar measurable, then by 443J(b-i) there
is an F$_{\sigma}$ set $E'\subseteq E$ such that $E\setminus E'$ is Haar
negligible.   Now there are an F$_{\sigma}$ set $H\subseteq X/Y$ such
that $E'=\phi^{-1}[H]$ and an F$_{\sigma}$ set $F\subseteq Z$ such that
$H=(X/Y)\cap F$, in which case $E'=\phi^{-1}[F]$, and
$E\setminus\phi^{-1}[F]$ is Haar negligible.   This deals with (v).

\medskip

{\bf (d)} Concerning (vi)-(viii), we just have to put 4A2B and 4A5Kb
together.   If $G$, $A\subseteq X$, then

$$\eqalignno{G\text{ is open in }X
&\iff\text{ there is an open }V\subseteq X/Y
  \text{ such that }G=\phi^{-1}[V]\cr
&\iff\text{ there are a }V\subseteq X/Y\text{ and an open }H\subseteq Z\cr
&\mskip100mu
  \text{ such that }G=\phi^{-1}[V]\text{ and }V=(X/Y)\cap H\cr
&\iff\text{ there is an open }H\subseteq Z
  \text{ such that }G=\phi^{-1}[H];}$$

$$\eqalignno{G\text{ is a regular open set in }X
&\iff\phi[G]\text{ is a regular open subset of }X/Y\cr
\displaycause{4A5K(b-iii), 4A2B(f-iii)}
&\iff\text{ there is a regular open }H\subseteq Z\cr
&\mskip100mu
  \text{ such that }\phi[G]=(X/Y)\cap H\cr
\displaycause{4A2B(j-ii)}
&\iff\text{ there is a regular open }H\subseteq Z\cr
&\mskip100mu
  \text{ such that }G=\phi^{-1}[H];}$$

$$\eqalignno{A\text{ is nowhere dense in }X
&\iff\phi[A]\text{ is nowhere dense in }X/Y\cr
\displaycause{4A5K(iv)-(v)}
&\iff\phi[A]\text{ is nowhere dense in }Z\cr}$$

\noindent(4A2B(j-i)).
}%end of proof of 443L

\leader{443M}{Theorem}\cmmnt{ ({\smc Halmos 50})} Let $X$ be a
topological group and $\mu$ a Haar measure on $X$.   Then $\mu$ is
completion regular.

\proof{{\bf (a)} Suppose first that $\mu$ is a left Haar measure and
that $X$ is locally compact and Hausdorff.   In this case any
self-supporting compact set $K\subseteq X$ is a zero set.   \Prf\ For
each $n\in\Bbb N$, there is an open neighbourhood $U_n$ of the identity
such that $\mu(K\symmdiff xK)\le 2^{-n}$ for every $x\in U_n$ (443B);
we may suppose that $\overline{U}_{\mskip-3mu n+1}\subseteq U_n$
for each $n$.
Each set $U_nK$ is open (4A5Ed), so $\bigcap_{n\in\Bbb N}U_nK$ is a
G$_{\delta}$
set.   \Quer\ If $K\ne\bigcap_{n\in\Bbb N}U_nK$, there is an
$x\in\bigcap_{n\in\Bbb N}U_nK\setminus K$.   For each $n\in\Bbb N$,
there are $y_n\in U_n$, $z_n\in K$ such that $x=y_nz_n$.   Let $\Cal F$
be any non-principal ultrafilter on $\Bbb N$.   Then
$z=\lim_{n\to\Cal F}z_n$ is defined in $K$, so

\Centerline{$xz^{-1}=\lim_{n\to\Cal F}xz_n^{-1}=\lim_{n\to\Cal F}y_n$}

\noindent is defined in $X$;  because $y_n\in\overline{U}_{\mskip-3mu i}$
for every $i\le n$,

\Centerline{$xz^{-1}
\in\bigcap_{i\in\Bbb N}\overline{U}_{\mskip-3mu i}
=\bigcap_{i\in\Bbb N}U_i$.}

\noindent Consequently $\mu(xz^{-1}K\symmdiff K)=0$;  because $\mu$ is
left-translation-invariant, $\mu(K\setminus zx^{-1}K)=0$.   But as
$x\notin K$, $z\in K\setminus zx^{-1}K$ and $K\cap(X\setminus zx^{-1}K)$
is non-empty.   And $zx^{-1}K$ is closed, so $X\setminus zx^{-1}K$ is
open and $K$ is not self-supporting, contrary to hypothesis.\ \Bang

Thus $K=\bigcap_{n\in\Bbb N}U_nK$ is a G$_{\delta}$ set.   Being a
compact G$_{\delta}$ set in a completely regular Hausdorff space, it is
a zero set (4A2F(h-v)).\ \Qed

Since $\mu$ is surely inner regular with respect to the compact
self-supporting sets (414F), it is inner regular with respect to the
zero sets, and is completion regular.

\medskip

{\bf (b)} Now suppose that $\mu$ is a left Haar measure on an arbitrary
topological group $X$.   By 443L, we can find a locally compact
Hausdorff topological group $Z$, a continuous homomorphism $\phi:X\to Z$
and a left Haar measure $\lambda$ on $Z$ such that $\phi$ is \imp\ for
$\mu$ and $\lambda$ and $\mu$ is inner regular with respect to
$\{\phi^{-1}[K]:K\subseteq Z$ is compact$\}$.   Now if $E\in\dom\mu$ and
$\gamma<\mu E$, there is a compact set $K\subseteq Z$ such that
$\phi^{-1}[K]\subseteq E$ and $\nu K>\gamma$.   Next, there is a zero
set $L\subseteq K$ such that $\nu L\ge\gamma$;  in which case
$\phi^{-1}[L]\subseteq E$ is a zero set and $\mu\phi^{-1}[L]\ge\gamma$.
Thus $\mu$ is inner regular with respect to the zero sets and is
completion regular.

\medskip

{\bf (c)} Finally, if $\mu$ is a right Haar measure on a topological
group $X$, let $\Reverse{\mu}$ be the corresponding left Haar measure, setting
$\Reverse{\mu} E=\mu E^{-1}$ for Haar measurable sets $E$.   Then
$\Reverse{\mu}$ is
inner regular with respect to the zero sets;  because
$x\mapsto x^{-1}:X\to X$ is a homeomorphism, so is $\mu$.
}%end of proof of 443M

\leader{443N}{}\cmmnt{ I give a simple result showing how the
measure-theoretic properties of groups carrying Haar measures have
topological consequences which might not be expected.

\medskip

\noindent}{\bf Proposition} Let $X$ be a topological group carrying
Haar measures\cmmnt{ (for instance, $X$ might be any locally
compact Hausdorff group)}.

(i) Let $G$ be a regular open subset of $X$.   Then $G$ is a cozero set.

(ii) Let $F$ be a nowhere dense subset of $X$.   Then $F$ is included in
a nowhere dense zero set.

\proof{{\bf (a)} Suppose to begin with that $X$ is locally compact,
$\sigma$-compact and Hausdorff.   Let $\mu$ be a left
Haar measure on $X$;  then $\mu$ is $\sigma$-finite, because $X$ is
covered by a sequence of compact sets, which must all have finite
measure.

\medskip

\quad{\bf (i)} Write $\Cal U$ for the
family of open neighbourhoods of
the identity $e$ of $X$.   For each $U\in\Cal U$, set
$H_U=\interior\{x:xU\subseteq G\}$;   then $\{H_U:U\in\Cal U\}$ is an
upwards-directed family of open sets with union $G$, as in the proofs of
442Ab and 442B, so $G^{\ssbullet}=\sup_{U\in\Cal U}H_U^{\ssbullet}$ in
the measure algebra $\frak A$ of $\mu$.   Because $\mu$ is
$\sigma$-finite, $\frak A$ is ccc (322G) and there is a sequence
$\sequencen{U_n}$ in $\Cal U$ such that
$G^{\ssbullet}=\sup_{n\in\Bbb N}H_{U_n}^{\ssbullet}$ (316E).   In this
case, $G\setminus\bigcup_{n\in\Bbb N}H_{U_n}$ is negligible, so must
have empty interior.

By 4A5S, there is a closed normal subgroup $Y$ of $X$, included in
$\bigcap_{n\in\Bbb N}U_n$, such that $X/Y$ is metrizable.   Let
$\pi:X\to X/Y$ be the canonical map.

For each $n\in\Bbb N$, $Q_n=\pi[H_{U_n}]$ is open (4A5J(a-i)), and

\Centerline{$H_{U_n}\subseteq\pi^{-1}[Q_n]=H_{U_n}Y
\subseteq H_{U_n}U_n\subseteq G$.}

\noindent So

\Centerline{$\overline{G}
=\overline{\bigcupop_{n\in\Bbb N}H_{U_n}}
=\overline{\bigcupop_{n\in\Bbb N}\pi^{-1}[Q_n]}$.}

\noindent Setting $Q=\interior\overline{\bigcup_{n\in\Bbb N}Q_n}$, and
using 4A5J(a-i) and 4A2B(f-ii), we see that

\Centerline{$\pi^{-1}[Q]
=\interior\overline{\bigcupop_{n\in\Bbb N}\pi^{-1}[Q_n]}
=\interior\overline{G}=G$}

\noindent (this is where I use the hypothesis that $G$ is a regular open
set).   But $Q$, being an open set in a metrizable space, is a cozero
set (4A2Lc), so $G=\pi^{-1}[Q]$ is a cozero set (4A2C(b-iv)), as
required by (i).

\medskip

\quad{\bf (ii)} Now consider the nowhere dense set $F\subseteq X$.
This
time, let $\Cal G$ be a maximal disjoint family of cozero subsets of
$X\setminus F$.   Then $\Cal G$ is countable, again because $\mu$ is
$\sigma$-finite, and $\bigcup\Cal G$ is dense, because the topology of
$X$ is completely regular.   So $X\setminus\bigcup\Cal G$ is a nowhere
dense zero set including $F$.

\medskip

{\bf (b)} Next, suppose that $X$ is any locally compact Hausdorff
topological group.   Then $X$ has a $\sigma$-compact open subgroup $X_0$
(4A5El).   By (a), any
regular open set in $X_0$ is a cozero set in $X_0$.   The same applies
to all the (left) cosets of $X_0$, because these are homeomorphic to
$X_0$.

If $C$ is any coset
of $X_0$, then $G\cap C$ is a regular open set in $C$, so is a cozero
set in $C$.   But as the left cosets of $X_0$ form a partition of $X$
into open sets, $G$ is also a cozero set in $X$ (4A2C(b-vii)).

Similarly, $F\cap C$ is nowhere dense in $C$ for every left coset $C$ of
$X_0$, so is included in a nowhere dense zero set in $C$, and the union
of these will be a nowhere dense zero set in $X$ including $F$.

\medskip

{\bf (c)} Now suppose that $X$ is any group carrying Haar
measures.   Let $Z$, $\lambda$ and $\phi:X\to Z$ be as in 443L.
Then $G$ is expressible as $\phi^{-1}[H]$ for some regular open set
$H\subseteq Z$ (443L(vii));  by (b), $H$ is a cozero set, so $G$ also is
(4A2C(b-iv) again).   As for $F$, $\phi[F]$ is nowhere dense
in $Z$, by 443L(viii).   Let $F'\supseteq\phi[F]$ be a nowhere dense zero
set;  then $\phi^{-1}[F']\supseteq F$ is a zero set, and
$\phi[\phi^{-1}[F']]\subseteq F'$ is nowhere dense, so $\phi^{-1}[F']$ is
nowhere dense.
}%end of proof of 443N

\leader{443O}{}\discrversionA{\footnote{New 2009.}}
{}\cmmnt{ An expected result, well known for Lebesgue measure,
but which seems to need a little attention for the non-metrizable case, is
the following.

\medskip

\noindent}{\bf Proposition} Let $X$ be a topological group and $\mu$ a left
Haar measure on $X$.   Then the following are equiveridical:

(i) $\mu$ is not purely atomic;

(ii) $\mu$ is atomless;

(iii) there is a non-negligible nowhere dense subset of $X$;

(iv) $\mu$ is inner regular with respect to the nowhere dense sets;

(v) there is a conegligible meager subset of $X$;

(vi) there is a negligible comeager subset of $X$.

\noindent If $X$ is Hausdorff, we can add

(vii) the topology of $X$ is not discrete.

\proof{ Write $\Sigma$ for the domain of $\mu$.

\medskip

{\bf (a)(i)$\Rightarrow$(ii)} If $\mu$ is not purely atomic, let
$E\in\Sigma$ be a non-negligible set not including any atom.   If
$F\in\Sigma$ is any other non-negligible set, then there is an $x\in X$
such that $F\cap xE$ is not negligible, by 443Da.
Now the subspace measures on
$F\cap xE$ and $x^{-1}F\cap E$ are isomorphic, and the latter is atomless,
so $F\cap xE$ is not an atom and $F$ is not an atom.   As $F$ is arbitrary,
$\mu$ is atomless.

\medskip

{\bf (b)(iii)$\Rightarrow$(iv)} The argument is similar.   Suppose that $A$
is a non-negligible nowhere dense subset of $X$;  then $E=\overline{A}$ is
a non-negligible closed nowhere dense set.   If $F\in\Sigma$ is
non-negligible, there is an $x\in X$ such that $F\cap xE$ is
non-negligible;  as $y\mapsto xy:X\to X$ is a homeomorphism, $xE$ and
$F\cap xE$ are nowhere dense.   Thus every non-negligible measurable set
includes a nowhere dense non-negligible measurable set;  as the family of
nowhere dense sets is an ideal, $\mu$ is inner regular with respect to the
nowhere dense sets (412Aa).

\medskip

{\bf (c)(iv)$\Rightarrow$(i)} Suppose that $\mu$ is inner regular with
respect to the nowhere dense sets.
Let $U$ be an open neighbourhood of the identity $e$ of $X$ with
finite measure (442Aa once more),
and $V$ an open neighbourhood of $e$ such that
$VV^{-1}V\subseteq U$.   Then $\mu V>0$, by the other half of 442Aa.

\Quer\ If there is a $\mu$-atom $E\subseteq V$, let $F_0\subseteq E$ be a
non-negligible measurable nowhere dense set, $F_1\subseteq F_0$ a
non-negligible closed set and $F\subseteq F_1$ a non-empty self-supporting
closed set (414F).   Because $y\mapsto xy$ is a measure-preserving
automorphism, $xF$ is a self-supporting closed set, and an atom for $\mu$,
for every
$x\in X$.   So if $x$, $y\in X$ and $xF\cap yF$ is non-negligible,
then $xF\setminus yF$ is negligible and $xF\subseteq yF$;  similarly,
$yF\subseteq xF$, so $xF=yF$.   Now no finite number of translates of
the nowhere dense set $F$
can cover the non-empty open set $V$, while
$V\subseteq\bigcup_{x\in X}xF$, so we must have a sequence
$\sequencen{x_n}$ in $X$ such that
$x_nF\cap V\setminus\bigcup_{i<n}x_iF\ne\emptyset$ for every $n\in\Bbb N$.
In this case, $x_iF\cap x_nF$ is negligible whenever $i<n$, so

\Centerline{$\mu(\bigcup_{n\in\Bbb N}x_nF)
=\sum_{n\in\Bbb N}\mu(x_nF)=\infty$.}

\noindent However $V\cap x_nF\ne\emptyset$, so
$x_n\in VF^{-1}\subseteq VV^{-1}$ and $x_nF\subseteq VV^{-1}V\subseteq U$
for every $n\in\Bbb N$;  and $U$ is supposed to have finite measure.\ \Bang

Accordingly $V$ does not include any atom and $\mu$ cannot be purely
atomic.

\medskip

{\bf (d)(iv)$\Rightarrow$(v)} Suppose that $\mu$ is inner regular with
respect to the nowhere dense sets.   Let $\Cal G$ be a maximal disjoint
family of open sets of non-zero finite measure.   Then
$\interior(X\setminus\bigcup\Cal G)$ is negligible, because $\mu$ is
effectively locally finite, so must be empty, and
$X\setminus\bigcup\Cal G$ is nowhere dense.   For each $G\in\Cal G$, let
$\sequencen{F_{Gn}}$ be a sequence of nowhere dense measurable
subsets of $G$ such that $\mu G=\lim_{n\to\infty}\mu F_{Gn}$.   For
$n\in\Bbb N$, set $A_n=\bigcup_{G\in\Cal G}F_{Gn}$.   Then $A_n$ is nowhere
dense.   \Prf\ If $H\subseteq X$ is open and not empty, either
$H\cap A_n=\emptyset$ or there is a $G\in\Cal G$ such that
$H\cap G\ne\emptyset$, in which case
$H\cap G\setminus\overline{F}_{\mskip-3mu Gn}\subseteq H\setminus A_n$
is open and non-empty.\ \QeD\   So
$D=(X\setminus\bigcup\Cal G)
  \cup\bigcup_{n\in\Bbb N}\overline{A}_{\mskip-3mu n}$ is
meager.   \Quer\ If $D$ is not conegligible, let $H$ be an open set of
finite measure such that $H\setminus D$ is non-negligible.   As
$H\setminus D\subseteq\bigcup\Cal G$, there is a $G\in\Cal G$ such that
$\mu(H\cap G\setminus D)>0$;  but
$H\cap G\setminus D\subseteq G\setminus\bigcup_{n\in\Bbb N}F_{Gn}$.\ \Bang

Thus $D$ is a conegligible meager set.

\medskip

{\bf (e)(v)$\Leftrightarrow$(vi)} The complement of a witness for (v)
witnesses (vi), and conversely.

\medskip

{\bf (f)(v)$\Rightarrow$(iii)} is elementary, since $\mu$ is non-zero.

\medskip

{\bf (g)(ii)$\Rightarrow$(iii)} Suppose that $\mu$ is atomless.
I proceed through an expanding series of special cases,
as in 443N.

\medskip

\quad\grheada\ Suppose to begin with that
$X$ is locally compact, Hausdorff and
$\sigma$-compact.   In this case $X$ has a closed negligible normal
subgroup $Y$ such that $X/Y$ is separable and metrizable.   \Prf\ Since
$\mu$ is atomless, $\{e\}$ must be negligible.   Since $\mu$ is locally
finite and inner regular with respect to the closed sets,
there must be a sequence $\sequencen{U_n}$ of open neighbourhoods of $e$
such that $\inf_{n\in\Bbb N}\mu U_n=0$.   By 4A5S again,
there is a closed normal
subgroup $Y\subseteq\bigcap_{n\in\Bbb N}U_n$ such that $X/Y$ is metrizable,
and of course $Y$ is negligible.   Since the canonical map from $X$ onto
$X/Y$ is continuous (4A5J(a-i) again),
$X/Y$ is $\sigma$-compact, therefore separable (4A2Hd, 4A2Pd).\ \Qed

Write $\pi:X\to X/Y$ for the canonical map.
Let $D\subseteq X/Y$ be a countable dense set, and consider $\pi^{-1}[D]$.
This is a countable union of
translates of $Y$, so is negligible;  let
$F\subseteq X\setminus\pi^{-1}[D]$
be a closed non-negligible set.   Then $\pi[F]$ does not meet $D$.
Because $\pi$ is an open mapping (4A5Ja once more),
$\interior F=\emptyset$ and $F$ is nowhere dense.

Thus in this case we have a non-negligible nowhere dense set, as required.

\medskip

\quad\grheadb\ Now suppose just that $X$ is locally compact and Hausdorff.   In
this case it has an open $\sigma$-compact subgroup $X_0$ say.
The subspace measure $\mu_{X_0}$
on $X_0$ is a left Haar measure on $X_0$ (443F), and
is atomless;  by ($\alpha$),
there is a nowhere dense set $F\subseteq X_0$ such
that $0<\mu_{X_0}F=\mu F$.   So in this case too we have a non-negligible
nowhere dense set.

\medskip

\quad\grheadc\
For the general case, let $Z$, $\lambda$ and $\phi:X\to Z$ be as
in 443L.   Then $\lambda$ is atomless.   \Prf\ 443L(v) implies that
the measure-preserving Boolean
homomorphism from the measure algebra of $\lambda$ to the measure
algebra of $\mu$ induced by $\phi$ is surjective, therefore an isomorphism;
so both measure algebras are atomless and $\lambda$ is atomless (322Bg).\
\Qed

By ($\beta$), there is a nowhere dense non-negligible subset $H$ of $Z$;
replacing $H$ by its closure, if necessary, we may suppose that $H$ is
closed.   Set $F=\phi^{-1}[H]$;  then $F\subseteq X$ is closed and
non-negligible because $\phi$ is continuous and \imp.   Since
$\phi[F]\subseteq H$ is nowhere dense, so is $F$ (443L(viii)).
Thus we have the required non-negligible nowhere dense set in the general
case also.

\medskip

{\bf (h)} Now suppose that $X$ is Hausdorff.   If $\mu$ is atomless, then
$\mu\{x\}=0$ for every $x$, so $\{x\}$ is never open and the topology is
not discrete.   If $\mu$ has an atom $E$, let $F\subseteq E$ be a closed
self-supporting set of non-zero measure;  then $F$ also is an atom, so
cannot have two disjoint non-empty relatively open sets, and must be a
singleton.   Thus we have an $x_0$ such that $\mu\{x_0\}>0$;  as $\mu$ is
\lti, $\mu\{x\}=\mu\{x_0\}$ for every $x\in X$.
We know also that there is an open set $G$ of non-zero finite measure,
which must be finite;  so every singleton subset of $G$ is open.   It
follows that every singleton subset of $X$ is open, and $X$ has its
discrete topology.   Thus (ii)$\Leftrightarrow$(vii) when $X$ is Hausdorff.
}%end of proof of 443O


\leader{443P}{Quotient
\dvrocolon{spaces}}\discrversionA{\footnote{Formerly 4{}43O.}}
{}\cmmnt{ I come now to the
relationship between the modular functions of \S442, normal subgroups
and quotient spaces.

\medskip

\noindent}{\bf Lemma} Let $X$ be a locally compact Hausdorff topological
group and $Y$ a closed subgroup of $X$.   Let $Z=X/Y$ be the set of left
cosets of $Y$ in $X$ with the quotient topology and $\pi:X\mapsto Z$ the
canonical map, so that $Z$ is a
locally compact Hausdorff space and we have a continuous action of $X$
on $Z$ defined by writing $a\action\pi x=\pi(ax)$ for $a$,
$x\in X$\cmmnt{ (4A5J(b-iii))}.   Let $\nu$ be a left Haar measure on $Y$
and write $C_k(X)$, $C_k(Z)$ for the spaces of continuous
real-valued functions with compact supports on $X$, $Z$ respectively.

(a) We have a positive linear operator $T:C_k(X)\to C_k(Z)$
defined by writing

\Centerline{$(Tf)(\pi x)=\int_Yf(xy)\nu(dy)$}

\noindent for every $f\in C_k(X)$ and $x\in X$.   If $f>0$ in $C_k(X)$
then $Tf>0$ in $C_k(Z)$.
If $h\ge 0$ in $C_k(Z)$ then there is an $f\ge 0$ in $C_k(X)$ such
that $Tf=h$.

(b) If $a\in X$ and $f\in C_k(X)$, then
$T(a\action_lf)(z)=(Tf)(a^{-1}\action z)$ for every $z\in Z$.

(c) Now suppose that $a$ belongs to the normalizer of
$Y$\cmmnt{ (that is, $aYa^{-1}=Y$)}.   In this case, we can define
$\psi(a)\in\ooint{0,\infty}$ by the formula

\Centerline{$\nu(aFa^{-1})=\psi(a)\nu F$ for every $F\in\dom\nu$,}

\noindent and

\Centerline{$T(a\action_rf)(\pi x)=\psi(a)\cdot(Tf)(\pi(xa))$}

\noindent for every $x\in X$ and $f\in C_k(X)$.

\proof{ I should begin by remarking that because $Y$ is a closed
subgroup of a locally compact Hausdorff group, it is itself a locally
compact Hausdorff group, so does have a left Haar measure, which is a
Radon measure.

\medskip

{\bf (a)(i)} The first thing to check is that if $f\in C_k(X)$ then $Tf$
is well-defined as a member of $\BbbR^{Z}$.   \Prf\ ($\alpha$) If $x\in
X$, then $y\mapsto f(xy):Y\to\Bbb R$ is a continuous function with
compact support, so $\int f(xy)\nu(dy)$ is defined in $\Bbb R$.
($\beta$) If
$x_1$, $x_2\in X$ and $\pi x_1=\pi x_2$, then $x_1^{-1}x_2\in Y$, and

\Centerline{$\int f(x_2y)\nu(dy)
=\int f(x_1(x_1^{-1}x_2y))\nu(dy)
=\int f(x_1y)\nu(dy)$,}

\noindent applying 441J to the function $y\mapsto f(x_1y)$ and
the left action of $Y$ on itself.   Thus we can safely write
$(Tf)(\pi x)=\int f(xy)\nu(dy)$ for every $x\in X$, and $Tf$ will be a
real-valued function on $Z$.\ \Qed

\medskip

\quad{\bf (ii)} Now $Tf$ is continuous for every $f\in C_k(X)$.   \Prf\
Given $z_0\in Z$, take $x_0\in X$ such that $z=\pi x_0$.   We have an
$h\in C_k(X)^+$ such that for every $\epsilon>0$ there is an open set
$U_{\epsilon}$ containing $x_0$ such that
$|f(x_0y)-f(xy)|\le\epsilon h(y)$ whenever $x\in U_{\epsilon}$ and
$y\in X$ (4A5Pb).   In this case,

\Centerline{$|(Tf)(\pi x)-(Tf)(\pi x_0)|
=|\int f(xy)-f(x_0y)\nu(dy)|
\le\epsilon\int_Yh(y)\nu(dy)$}

\noindent for every $x\in U_{\epsilon}$, so that
$|(Tf)(z)-(Tf)(z_0)|\le\epsilon\int_Y h\,d\nu$ for every
$z\in\pi[U_{\epsilon}]$.   Since each $\pi[U_{\epsilon}]$ is an open
neighbourhood of $z_0$ (4A5J(a-i), as always),
$Tf$ is continuous at $z_0$;  as $z_0$
is arbitrary, $Tf$ is continuous.\ \Qed

\medskip

\quad{\bf (iii)} Since

$$\eqalign{\{z:(Tf)(z)\ne 0\}
&=\{\pi x:\int f(xy)\nu(dy)\ne 0\}
\subseteq\{\pi x:f(xy)\ne 0\text{ for some }y\in Y\}\cr
&=\{\pi x:f(x)\ne 0\}
\subseteq\pi[\overline{\{x:f(x)\ne 0\}}]\cr}$$

\noindent is relatively compact, $Tf\in C_k(Z)$ for every
$f\in C_k(X)$.

\medskip

\quad{\bf (iv)} The formula for $Tf$ makes it plain that
$T:C_k(X)\to C_k(Z)$ is a positive linear operator.

\medskip

\quad{\bf (v)} If $f\in C_k(X)^+$ and $x\in X$ are such that $f(x)>0$, then
$\{y:y\in Y,\,f(xy)>0\}$ is a non-empty open set in $Y$;  because $\nu$
is strictly positive,

\Centerline{$(Tf)(\pi x)=\int f(xy)\nu(dy)>0$.}

\noindent In particular, $Tf>0$ if $f>0$.   Moreover, if $z\in Z$
there is an $f\in C_k(X)^+$ such that $(Tf)(z)>0$.   Now

\Centerline{$\{\{z:(Tf)(z)>0\}:f\in C_k(X)^+\}$}

\noindent is an upwards-directed family of open subsets of $Z$, so if
$L\subseteq Z$ is any compact set there is an $f\in C_k(X)^+$ such
that $(Tf)(z)>0$ for every $z\in L$.

\medskip

\quad{\bf (vi)} Now suppose that $h\in C_k(Z)^+$.   By (v), there is
an $f_0\in C_k(X)^+$ such that $(Tf_0)(z)>0$ whenever
$z\in\overline{\{w:h(w)\ne 0\}}$.   Setting $h'(z)=h(z)/(Tf_0)(z)$ when
$h(z)\ne 0$, $0$ for other $z\in Z$, we have $h'\in C_k(Z)$ and
$h=h'\times Tf_0$.   Set

\Centerline{$f(x)=f_0(x)h'(\pi x)\ge 0$}

\noindent for every $x\in X$.   Because $h'$ and $\pi$ are continuous,
$f\in C_k(X)$.   For any $x\in X$,

$$\eqalign{(Tf)(\pi x)
&=\int f_0(xy)h'(\pi(xy))\nu(dy)
=h'(\pi x)\int f_0(xy)\nu(dy)\cr
&=h'(\pi x)(Tf_0)(\pi x)
=h(\pi x).\cr}$$

\noindent Thus $Tf=h$.

\medskip

{\bf (b)} If $z=\pi x$, then $a^{-1}\action z=\pi(a^{-1}x)$, so

\Centerline{$(Tf)(a^{-1}\action z)
=\int f(a^{-1}xy)\nu(dy)
=\int (a\action_lf)(xy)\nu(dy)
=T(a\action_lf)(z)$.}

\medskip

{\bf (c)} Define $\phi:Y\to Y$ by writing $\phi(y)=a^{-1}ya$ for
$y\in Y$.   Because $\phi$ is a homeomorphism, the image measure
$\nu\phi^{-1}$ is a Radon
measure on $Y$;  because $\phi$ is a group automorphism, $\nu\phi^{-1}$
is a left Haar measure.   (If $F\in\dom\nu\phi^{-1}$ and $y\in Y$, then

\Centerline{$\nu\phi^{-1}[yF]=\nu(ayFa^{-1})=\nu(Fa^{-1})
=\nu(aFa^{-1})=\nu\phi^{-1}[F]$.)}

\noindent $\nu\phi^{-1}$ must therefore be a multiple of $\nu$;  say
$\nu\phi^{-1}=\psi(a)\nu$.

If $g\in C_k(Y)$, then

\Centerline{$\int g(a^{-1}ya)\nu(dy)
=\int g\phi\,d\nu
=\int g\,d(\nu\phi^{-1})
=\psi(a)\int g\,d\nu$.}

\noindent Now take $f\in C_k(X)$.   Then

$$\eqalignno{T(a\action_rf)(\pi x)
&=\int(a\action_rf)(xy)\nu(dy)
=\int f(xya)\nu(dy)\cr
&=\int f(xa(a^{-1}ya))\nu(dy)
=\psi(a)\int f(xay)\nu(dy)\cr
\noalign{\noindent (using the remark above with $g(y)=f(xay)$)}
&=\psi(a)(Tf)(\pi(xa))\cr}$$

\noindent for every $x\in X$, as claimed.
}%end of proof of 443P

\leader{443Q}{Theorem}\discrversionA{\footnote{Formerly 4{}43P.}}
{} Let $X$ be a locally compact Hausdorff
topological group and $Y$ a closed subgroup of $X$.   Let $Z=X/Y$ be the
set of left cosets of $Y$ in $X$ with the quotient topology, and
$\pi:X\to Z$ the canonical map, so that $Z$ is
a locally compact Hausdorff space and we have a continuous action of $X$
on $Z$ defined by writing $a\action\pi x=\pi(ax)$ for $a$,
$x\in X$.   Let $\nu$ be a left Haar measure on $Y$.   Suppose that
$\lambda$ is a non-zero $X$-invariant Radon measure on $Z$.

(a) For each $z\in Z$, we have a Radon measure $\nu_z$ on $X$ defined
by the formula

\Centerline{$\nu_zE=\nu(Y\cap x^{-1}E)$}

\noindent whenever $\pi x=z$ and the right-hand side is defined.   In
this case, for a real-valued function $f$ defined on a subset of $X$,

\Centerline{$\int f\,d\nu_z=\int f(xy)\nu(dy)$}

\noindent whenever either side is defined in $[-\infty,\infty]$.

(b) We have a left Haar measure $\mu$ on $X$ defined by the formulae

\Centerline{$\int f\,d\mu=\biggeriint f\,d\nu_z\lambda(dz)$}

\noindent for every $f\in C_k(X)$, and

\Centerline{$\mu G=\int\nu_zG\,\lambda(dz)$}

\noindent for every open set $G\subseteq X$.

(c) If $D\subseteq Z$, then $D\in\dom\lambda$ iff
$\pi^{-1}[D]\subseteq X$ is
Haar measurable, and $\lambda D=0$ iff $\pi^{-1}[D]$ is Haar negligible.

(d) If $\nu Y=1$, then $\lambda$ is the image measure $\mu\pi^{-1}$.

(e) Suppose now that $X$ is $\sigma$-compact.   Then
$\mu E=\int\nu_zE\,\lambda(dz)$ for every Haar measurable set
$E\subseteq X$.   If $f\in\eusm L^1(\mu)$, then
$\int f\,d\mu=\iint f\,d\nu_z\lambda(dz)$.

(f) Still supposing that $X$ is $\sigma$-compact, take
$f\in\eusm L^1(\mu)$, and for $a\in X$ set $f_a(y)=f(ay)$ whenever
$y\in Y$ and
$ay\in\dom f$.  Then $Q_f=\{a:a\in X,\,f_a\in\eusm L^1(\nu)\}$ is
$\mu$-conegligible, and the function
$a\mapsto f_a^{\ssbullet}:Q_f\to L^1(\nu)$ is almost continuous.

\proof{{\bf (a)} First, we do have a function $\nu_z$ depending only on
$z$, because if $z=\pi x_1=\pi x_2$ then
$x_2^{-1}x_1\in Y$, so

\Centerline{$\nu(Y\cap x_1^{-1}E)
=\nu(x_2^{-1}x_1(Y\cap x_1^{-1}E))
=\nu(Y\cap x_2^{-1}E)$}

\noindent whenever either side is defined.   Of course $\nu_z$, being
the image of the Radon measure $\nu$ under
the continuous map $y\mapsto xy:Y\to X$ whenever $\pi x=z$, is always
a Radon measure on $X$ (418I).   We also have

\Centerline{$\int_Yf(xy)\nu(dy)=\int_Xf\,d\nu_{\pi x}$}

\noindent whenever $x\in X$ and $f$ is a real-valued function such that
either side is defined in $[-\infty,\infty]$, by 235J.

I remark here that if $z\in Z$ then the coset $C=\pi^{-1}[\{z\}]$ is
$\nu_z$-conegligible, because if $\pi x=z$ then $Y=Y\cap x^{-1}C$.

\medskip

{\bf (b)(i)} Let $T:C_k(X)\to C_k(Z)$ be the positive linear operator
of 443P;  that is,

\Centerline{$(Tf)(z)=\int f(xy)\nu(dy)=\int f\,d\nu_z$}

\noindent whenever $f\in C_k(X)$, $x\in X$ and $z=\pi x$.   Then we have
a positive linear functional $\theta:C_k(X)\to\Bbb R$ defined by setting
$\theta(f)=\int Tf\,d\lambda$ for every $f\in C_k(X)$.   By the Riesz
Representation Theorem (436J), there is a Radon measure $\mu$ on $X$
defined by saying that $\int f\,d\mu=\theta(f)$ for every $f\in C_k(X)$.
Note that $\mu$ is non-zero.   \Prf\ Because $\lambda$ is non-zero,
there is some $h\in C_k(Z)^+$ such that $\int h\,d\lambda\ne 0$;  now
there is some $f\in C_k(X)$ such that $Tf=h$, by 443Pa, and
$\int f\,d\mu\ne 0$.\ \Qed

\medskip

\quad{\bf (ii)} $\mu$ is a left Haar measure.   \Prf\ If $f\in C_k(X)$
and $a\in X$, then we have
$T(a\action_lf)(z)=(Tf)(a^{-1}\action z)$ for every $z\in Z$, by 443Pb.
So

$$\eqalignno{\int a\action_lfd\mu
&=\int T(a\action_lf)d\lambda
=\int Tf(a^{-1}\action z)\lambda(dz)
=\int Tf(z)\lambda(dz)\cr
\noalign{\noindent (by 441J or 441L, because $\lambda$ is
$X$-invariant)}
&=\int f\,d\mu.\cr}$$

\noindent By 441L in the other direction, $\mu$ is invariant under the
left action of $X$ on itself, that is, is a left Haar measure.\ \Qed

\medskip

\quad{\bf (iii)} If $G\subseteq X$ is open then
$\mu G=\int\nu_zG\,\lambda(dz)$.   \Prf\ Set
$A=\{f:f\in C_k(X),\,0\le f\le\chi G\}$.   Then
$\mu G=\sup_{f\in A}\int f\,d\mu$ and

\Centerline{$\nu_zG=\sup_{f\in A}\int f\,d\nu_z
=\sup_{f\in A}(Tf)(z)$}

\noindent for every $z\in Z$, by 414Ba, because $\nu_z$ is
$\tau$-additive.   But as $Tf$ is continuous
for every $f\in A$, and $\lambda$ is $\tau$-additive, we also have

\Centerline{$\int\nu_zG\,\lambda(dz)
=\sup_{f\in A}\int Tf\,d\lambda
=\sup_{f\in A}\int f\,d\mu
=\mu G$.  \Qed}

\medskip

{\bf (c)(i)} Let $\Cal A$ be the family of those sets $A\subseteq X$
such that $\mu A$ and $\int\nu_z(A)\lambda(dz)$ are defined in
$[0,\infty]$ and equal.   Then $\bigcup_{n\in\Bbb N}A_n$ belongs to
$\Cal A$ whenever $\sequencen{A_n}$ is a non-decreasing sequence in
$\Cal A$, and $A\setminus B\in\Cal A$ whenever $A$, $B\in\Cal A$,
$B\subseteq A$ and $\mu A<\infty$.   Moreover, if $A\in\Cal A$ and
$\mu A=0$, then every subset of $A$ belongs to $\Cal A$, since $A$ must
be $\nu_z$-negligible for $\lambda$-almost every $z$.   We also know
from (b) that every open set belongs to $\Cal A$.

Applying the Monotone Class Theorem (136B) to
$\{A:A\in\Cal A,\,A\subseteq G\}$, we see that if $E\subseteq X$ is a
Borel set
included in an open set $G$ of finite measure, then $E\in\Cal A$.   So
if $E$ is a relatively compact Haar measurable set, $E\in\Cal A$ (using
443J(b-i), or otherwise).

\medskip

\quad{\bf (ii)} If $D\in\dom\lambda$ then $\pi^{-1}[D]\in\dom\mu$.
\Prf\ Let $K\subseteq X$ be compact.   Then $\pi[K]\subseteq Z$ is
compact, so there are Borel sets $F_1$, $F_2\subseteq Z$ such that
$F_1\subseteq D\cap\pi[K]\subseteq F_2$ and $F_2\setminus F_1$ is
$\lambda$-negligible.   Now
$\nu_z(K\cap\pi^{-1}[F_2\setminus F_1])=0$ whenever
$z\notin F_2\setminus F_1$, by the remark added to the proof of (a)
above, so

\Centerline{$\mu(K\cap\pi^{-1}[F_2\setminus F_1])
=\int\nu_z(K\cap\pi^{-1}[F_2\setminus F_1])\lambda(dz)
=0$.}

\noindent Since

\Centerline{$K\cap\pi^{-1}[F_1]\subseteq K\cap\pi^{-1}[D]
\subseteq K\cap\pi^{-1}[F_2]$,}

\noindent $K\cap\pi^{-1}[D]\in\dom\mu$.   As $K$ is arbitrary,
$\pi^{-1}[D]$ is measured by $\mu$, so is Haar measurable.\ \Qed

If $\lambda D=0$ then the same arguments show that
$\mu(K\cap\pi^{-1}[D])=0$ for every compact $K\subseteq X$, so that
$\mu\pi^{-1}[D]=0$ and $\pi^{-1}[D]$ is Haar negligible.

\medskip

\quad{\bf (iii)} Now suppose that $D\subseteq Z$ is such that
$\pi^{-1}[D]\in\dom\mu$.   Let $L\subseteq Z$ be compact.   Then there
is a relatively compact open set $G\subseteq X$ such that
$\pi[G]\supseteq L$ (because $\{\pi[G]:G\subseteq X$ is open and
relatively compact$\}$ is an upwards-directed family of open sets
covering $Z$).   In this case,

\Centerline{$\int\nu_z(G\cap\pi^{-1}[D\cap L])\lambda(dz)
=\mu(G\cap\pi^{-1}[D]\cap\pi^{-1}[L])$}

\noindent is well-defined, by (i).   But  if $z=\pi x$ then

$$\eqalign{\nu_z(G\cap\pi^{-1}[D\cap L])
&=0\text{ if }z\notin D\cap L,\cr
&=\nu_zG=\nu(Y\cap x^{-1}G)>0\text{ if }z\in D\cap L,\cr}$$

\noindent because if $z\in L$ then $\pi x\in\pi[G]$ and
$Y\cap x^{-1}G\ne\emptyset$.   So

\Centerline{$D\cap L
=\{z:\nu_z(G\cap\pi^{-1}[D\cap L])>0\}$}

\noindent is measured by $\lambda$.   As $L$ is arbitrary, and $\lambda$
is a Radon measure, $D\in\dom\lambda$.

\medskip

\quad{\bf (iv)} If $\pi^{-1}[D]$ is Haar negligible, then, in (iii) above,
we shall have $\int\nu_z(G\cap\pi^{-1}[D\cap L])\lambda(dz)=0$, so
that $\lambda(D\cap L)=0$;  as $L$ is arbitrary, $\lambda D=0$,
by 412Ib or 412Jc.

\medskip

{\bf (d)} If $\nu Y=1$, then, for any open set $H\subseteq Z$,
$\nu_z\pi^{-1}[H]=1$ if $z\in H$, $0$ otherwise.   So

\Centerline{$\mu\pi^{-1}[H]
=\int\nu_z(\pi^{-1}[H])\lambda(dz)
=\lambda H$.}

\noindent Thus $\lambda$ and the image measure $\mu\pi^{-1}$ agree on
the open sets and, being Radon measures (418I again), must be equal
(416E(b-iii)).

\medskip

{\bf (e)} If $X$ is actually $\sigma$-compact, then (c)(i) of this proof
tells us that
$\mu E=\int\nu_zE\,\lambda(dz)$ for every Haar measurable set
$E\subseteq X$, since $E$ is the union of an increasing sequence of
relatively compact measurable sets.   Consequently
$\int f\,d\mu=\iint f\,d\nu_z\lambda(dz)$ for every $\mu$-simple
function $f$.
Now suppose that $f$ is a non-negative
$\mu$-integrable function.   Then
there is a non-decreasing sequence $\sequencen{f_n}$ of non-negative
$\mu$-simple functions converging to $f$ everywhere in $\dom f$.   If we
set $A=\{z:\nu_z^*(X\setminus\dom f)>0\}$, then

\Centerline{$\int\nu_z(X\setminus\dom f)\lambda(dz)
=\mu(X\setminus\dom f)=0$,}

\noindent so $\lambda A=0$.  Since
$\int f\,d\nu_z=\lim_{n\to\infty}\int f_nd\nu_z$ for every
$z\in Z\setminus A$,

$$\eqalign{\int f\,d\mu
&=\lim_{n\to\infty}\int f_n\,d\mu\cr
&=\lim_{n\to\infty}\iint f_nd\nu_z\lambda(dz)
=\iint f\,d\nu_z\lambda(dz).\cr}$$

\noindent Applying this to the positive and negative parts of $f$, we
see that the same formula is valid for any $\mu$-integrable function
$f$.

\medskip

{\bf (f)(i)} If $f\in\eusm L^1(\mu)$ and $a\in X$, then

\Centerline{$\int f_ad\nu=\int f(ay)\nu(dy)
=\int f\,d\nu_{\pi a}$}

\noindent if any of these are defined.   So if $f$,
$g\in\eusm L^1(\mu)$, $\|f_a-g_a\|_1=\int|f-g|d\nu_{\pi a}$ if either is
defined.

\medskip

\quad{\bf (ii)} Let $\Phi$ be the set of all almost continuous functions
from $\mu$-conegligible subsets of $X$ to $L^1(\nu)$, where $L^1(\nu)$
is given its norm topology.   (In terms of the
definition in 411M, a member $\phi$ of $\Phi$ is to be almost continuous
with respect to the subspace measure on $\dom\phi$.)   If $\phi\in\Phi$
and $\psi$ is a function from a conegligible subset of $X$ to $L^1(\nu)$
which is equal almost everywhere to $\phi$, then $\psi\in\Phi$.   If
$\sequencen{\phi_n}$ is a sequence in $\Phi$ converging $\mu$-almost
everywhere to $\psi$, then $\psi\in\Phi$ (418F).

\medskip

\quad{\bf (iii)} For $f\in\eusm L^1(\mu)$, set
$\phi_f(a)=f_a^{\ssbullet}$ whenever this is defined in $L^1(\nu)$.
Set $M=\{f:f\in\eusm L^1(\mu),\,\phi_f\in\Phi\}$.   If
$\sequencen{f^{(n)}}$ is a sequence in $M$, $f\in\eusm L^1(\mu)$ and
$\|f^{(n)}-f\|_1\le 4^{-n}$ for every $n$, then $f\in M$.   \Prf\ Set
$g=\sum_{n=0}^{\infty}2^n|f^{(n)}-f|$, defined on

\Centerline{$\{x:x\in\dom f\cap\bigcap_{n\in\Bbb N}\dom f^{(n)},
\,\sum_{n=0}^{\infty}2^n|f^{(n)}(x)-f(x)|<\infty\}$;}

\noindent then $g\in\eusm L^1(\mu)$.   Now

\Centerline{$D=\{z:z\in Z,\,g$ is $\nu_z$-integrable$\}$}

\noindent is $\lambda$-conegligible, by (e), and
$E=\{a:a\in X,\,\pi a\in D\}$ is $\mu$-conegligible, by (c).

If $a\in E$, then

\Centerline{$|f^{(n)}_a-f_a|\le 2^{-n}g\,\,\nu_{\pi a}$-a.e.}

\noindent for every $n\ge 1$.   So

$$\eqalign{\|\phi_f(a)-\phi_{f^{(n)}}(a)\|_1
&=\int_Y|f_a-f_a^{(n)}|d\nu
=\int_X|f-f^{(n)}|d\nu_{\pi a}\cr
&\le 2^{-n}\int g\,d\nu_{\pi a}
\to 0\cr}$$

\noindent as $n\to\infty$.   Thus
$\phi_f=\lim_{n\to\infty}\phi_{f^{(n)}}$ almost everywhere, and
$\phi\in\Phi$, by (ii).\ \Qed

\medskip

\quad{\bf (iv)} $C_k(X)\subseteq M$.   \Prf\ If $f\in C_k(X)$ and
$a_0\in X$, then there is an $h\in C_k(X)^+$ such that for every
$\epsilon>0$ there is an open set $G_{\epsilon}$ containing $a_0$ such
that $|f(a_0y)-f(ay)|\le\epsilon h(y)$ whenever
$a\in G_{\epsilon}$ and $y\in X$ (4A5Pb again).   In this case,

\Centerline{$\|\phi_f(a_0)-\phi_f(a)\|_1
=\int|f(a_0y)-f(ay)|\nu(dy)
\le\epsilon\int_Yh\,d\nu$}

\noindent whenever $a\in G_{\epsilon}$.   As $\epsilon$ is arbitrary,
$\phi_f$ is continuous at $a_0$;  as $a_0$ is arbitrary, $\phi_f$ is
continuous, and $f\in M$.\ \Qed

\medskip

\quad{\bf (v)} Now take any $f\in\eusm L^1(\mu)$.   Then for each
$n\in\Bbb N$ we can find $f^{(n)}\in C_k(X)$ such that
$\|f^{(n)}-f\|_1\le 4^{-n}$ (416I), so $f\in M$, by (iii).
This completes the proof.
}%end of proof of 443Q

\vleader{72pt}{443R}{Theorem}\dvAformerly{4{}43Q}
Let $X$ be a locally compact Hausdorff
topological group and $Y$ a closed subgroup of $X$.   Let $Z=X/Y$ be the
set of left cosets of $Y$ in $X$ with the quotient topology, so that
$Z$ is a locally compact Hausdorff space and we have a continuous
action of $X$ on $Z$ defined by writing $a\action(xY)=axY$ for $a$,
$x\in X$.   Let $\Delta_X$ be the left modular function
of $X$ and $\Delta_Y$ the left modular function of $Y$.   Then the
following are equiveridical:

(i) there is a non-zero $X$-invariant Radon measure $\lambda$ on $Z$;

(ii) $\Delta_Y$ is the restriction of $\Delta_X$ to $Y$.

\proof{ Fix a left Haar measure $\nu$ on $Y$, and let
$T:C_k(X)\to C_k(Z)$ be the corresponding linear operator as defined in
443Pa.

\medskip

{\bf (a)(i)$\Rightarrow$(ii)} Suppose that $\lambda$ is a
non-zero $X$-invariant Radon measure on $Z$.   Construct a left Haar
measure $\mu$ on $X$ as in 443Q.
In the notation of part (b-i) of the proof of 443Q, we have

\Centerline{$\int fd\mu=\iint fd\nu_z\lambda(dz)=\int Tfd\lambda$}

\noindent for every $f\in C_k(X)$.

Suppose that $a\in Y$.   In this case, $a$ surely belongs to the
normalizer of $Y$, and, in the language of 443Pc, we have
$\nu(aFa^{-1})=\psi(a)\nu F$ for every $F\in\dom\nu$.   But as

\Centerline{$\nu(aFa^{-1})=\nu(Fa^{-1})=\Delta_Y(a^{-1})\nu F$,}

\noindent we must have $\psi(a)=\Delta_Y(a^{-1})$.

Fix some $f>0$ in $C_k(X)$.   We have

$$\eqalignno{T(a\action_rf)(\pi x)
&=\psi(a)\cdot(Tf)(\pi(xa))
=\psi(a)\int f(xay)\nu(dy)
=\psi(a)\int f(xy)\nu(dy)\cr
\displaycause{because $a\in Y$}
&=\psi(a)\cdot(Tf)(\pi x)\cr}$$

\noindent for every $x$, so that (using 442Kc) we have

$$\eqalign{\Delta_X(a^{-1})\int f\,d\mu
&=\int a\action_rfd\mu
=\int T(a\action_rf)d\lambda\cr
&=\psi(a)\int Tf\,d\lambda
=\psi(a)\int f\,d\mu
=\Delta_Y(a^{-1})\int f\,d\mu.\cr}$$

\noindent As $\int f\,d\mu>0$, we must have
$\Delta_X(a^{-1})=\Delta_Y(a^{-1})$;  as $a$ is arbitrary,
$\Delta_Y=\Delta_X\restrp Y$, as required by (ii).

\medskip

{\bf (b)(ii)$\Rightarrow$(i)} Now suppose that
$\Delta_Y=\Delta_X\restrp Y$.   This time, start
with a left Haar measure $\mu$ on $X$.

\medskip

\quad\grheada\ (The key.)  If $f\in C_k(X)$ is such that $Tf\ge 0$ in
$C_k(Z)$, then $\int f\,d\mu\ge 0$.   \Prf\ There is an
$h\in C_k(Z)^+$ such that $h(z)=1$ whenever $z\in Z$ and $(Tf)(z)\ne 0$;
by 443Pa, we can
find a $g\in C_k(X)^+$ such that $Tg=h$.   Now observe that
$x\mapsto(Tf)(\pi x)$ is a
non-negative continuous real-valued function on $X$, so

$$\eqalignno{0
&\le\int_Xg(x)(Tf)(\pi x)\mu(dx)\cr
&=\int_Xg(x)\int_Yf(xy)\nu(dy)\mu(dx)
=\int_Y\int_Xg(x)f(xy)\mu(dx)\nu(dy)\cr
\noalign{\noindent (by 417Ha or 417Hb, because
$(x,y)\mapsto g(x)f(xy):X\times Y\to\Bbb R$ is a continuous function
with compact support)}
&=\int_Y\Delta_X(y^{-1})\int_Xg(xy^{-1})f(x)\mu(dx)\nu(dy)\cr
\noalign{\noindent (applying 442Kc to the function
$x\mapsto g(xy^{-1})f(x)$)}
&=\int_Xf(x)\int_Y\Delta_X(y^{-1})g(xy^{-1})\nu(dy)\mu(dx)\cr
\noalign{\noindent (because $(x,y)\mapsto\Delta_X(y^{-1})g(xy^{-1})f(x)$
is continuous and has compact support)}
&=\int_Xf(x)\int_Y\Delta_Y(y^{-1})g(xy^{-1})\nu(dy)\mu(dx)\cr
\noalign{\noindent (because $\Delta_X\restrp Y=\Delta_Y$, by
hypothesis)}
&=\int_Xf(x)\int_Yg(xy)\nu(dy)\mu(dx)\cr
\noalign{\noindent (applying 442K(b-ii) to the function
$y\mapsto g(xy)$)}
&=\int_Xf(x)(Tg)(\pi x)\mu(dx)
=\int_Xf(x)\mu(dx)\cr}$$

\noindent because $(Tg)(\pi x)=1$ whenever $f(x)\ne 0$.\ \Qed

\medskip

\quad\grheadb\ Applying this to $f$ and $-f$, we see that
$\int f\,d\mu=0$ whenever $Tf=0$, so that $\int f\,d\mu=\int g\,d\mu$
whenever $f$, $g\in C_k(X)$ and $Tf=Tg$.   Accordingly (because $T$ is
surjective)
we have a functional $\theta:C_k(Z)\to\Bbb R$ defined by saying that
$\theta(Tf)=\int f\,d\mu$ whenever $f\in C_k(X)$, and $\theta$ is
positive and linear.   By the Riesz Representation Theorem again, there
is a Radon measure $\lambda$ on $Z$ such that
$\theta(h)=\int h\,d\lambda$ for every $h\in C_k(Z)$.

If $a\in X$ and $h\in C_k(Z)$ take $f\in C_k(X)$ such that $Tf=h$.
Then, for any $x\in X$,

\Centerline{$(Tf)(a\action\pi x)=(Tf)(\pi(ax))
=\int f(axy)\nu(dy)=\int (a^{-1}\action_lf)(xy)\nu(dy)
=T(a^{-1}\action_lf)(\pi x)$.}

\noindent So

$$\eqalign{\int h(a\action z)\lambda(dz)
&=\int (Tf)(a\action z)\lambda(dz)
=\int T(a^{-1}\action_lf)(z)\lambda(dz)\cr
&=\int a^{-1}\action_lfd\mu
=\int f\,d\mu
=\int h(z)\lambda(dz).\cr}$$

\noindent By 441L again, $\lambda$ is $X$-invariant.   Also $\lambda$ is
non-zero because there is surely some $f$ such that $\int f\,d\mu\ne 0$.
So we have the required non-zero $X$-invariant Radon measure on $Z$.
}%end of proof of 443R

\leader{443S}{Applications}\discrversionA{\footnote{Formerly 4{}43R.}}
{}\cmmnt{ This theorem applies in a variety
of cases.}   Let $X$ be a locally compact Hausdorff
topological group and $Y$ a closed subgroup of $X$.

\spheader 443Sa If $Y$ is a normal subgroup of $X$, then
$\Delta_Y=\Delta_X\restrp Y$.
\prooflet{\Prf\ $X/Y$ has a group structure under which it is a locally
compact Hausdorff group (4A5J(b-ii)).   It therefore has a left Haar
measure, which is surely $X$-invariant in the sense of 443R.\ \Qed
}%end of prooflet

Note that in this context any of the invariant measures $\lambda$ of 443Q
must be left Haar measures on the quotient group.

\spheader 443Sb If $Y$ is compact, then $\Delta_Y=\Delta_X\restrp Y$.
\prooflet{\Prf\ Both $\Delta_Y$ and $\Delta_X\restrp Y$ are
continuous homomorphisms from $Y$ to $\ooint{0,\infty}$;  since the only
compact subgroup of $\ooint{0,\infty}$ is $\{1\}$, they are both
constant with value $1$.\ \QeD}   So\cmmnt{ 443R tells us that} we
have an $X$-invariant Radon measure $\lambda$ on $X/Y$.   Since $Y$ has
a Haar probability measure, $\lambda$ will be the image of
a left Haar measure under the canonical map\cmmnt{ (443Qd)}.

\spheader 443Sc If, in (b), $Y$ is a normal subgroup,
then\cmmnt{ we find that, for $W\in\dom\lambda$ and $x\in X$,

\Centerline{$\lambda(W\cdot\pi x)=\mu(\pi^{-1}[W]\cdot x)
=\Delta_X(x)\lambda W$,}

\noindent so that} $\Delta_{X/Y}\pi=\Delta_X$, writing
$\pi:X\to X/Y$ for the canonical map.   \cmmnt{This is
a special case of 443T below, because (in the terminology there)
$\psi(a)=\nu Y/\nu Y=1$ for every $a\in X$.}

\spheader 443Sd If $Y$ is open, $\Delta_Y=\Delta_X\restrp Y$.
\prooflet{\Prf\ If $\mu$ is a left Haar measure on $X$, then the
subspace measure $\mu_Y$ is a left Haar measure on $Y$ (443F).   There
is an open set $G\subseteq Y$ such that $0<\mu G<\infty$, and now
$\mu(Gy)=\Delta_X(y)\mu G=\Delta_Y(y)\mu G$ for every
$y\in Y$.\ \QeD}   \cmmnt{This time, $X/Y$ is discrete, so counting
measure is an $X$-invariant Radon measure on $X/Y$.}

\leader{443T}{Theorem}\discrversionA{\footnote{Formerly 4{}43S.}}
{} Let $X$ be a locally compact Hausdorff
topological group and
$Y$ a closed normal subgroup of $X$;  let $Z=X/Y$ be the quotient group,
and $\pi:X\to Z$ the canonical map.   Write $\Delta_X$,
$\Delta_{Z}$ for the left modular functions of $X$, $Z$
respectively.   Define $\psi:X\to\ooint{0,\infty}$ by the formula

\Centerline{$\nu(aFa^{-1})=\psi(a)\nu F$ whenever $F\in\dom\nu$ and
$a\in X$,}

\noindent where $\nu$ is a left Haar measure on
$Y$\cmmnt{ (cf.\ 443Pc)}.   Then

\Centerline{$\Delta_{Z}(\pi a)=\psi(a)\Delta_X(a)$}

\noindent for every $a\in X$.

\proof{ Let $T:C_k(X)\to C_k(Z)$ be the map defined in 443P, and
$\lambda$ a
left Haar measure on $Z$;  then, as in 443Qb, we have a left Haar
measure $\mu$ on $X$ defined by the formula
$\int f\,d\mu=\int Tf\,d\lambda$ for every $f\in C_k(X)$.   Fix on some
$f>0$ in $C_k(X)$ and $a\in X$, and set $w=\pi a$.   By 443Pc, we have

\Centerline{$T(a\action_rf)(z)=\psi(a)(Tf)(\pi(xa))
=\psi(a)(Tf)(zw)$}

\noindent \noindent whenever $\pi x=z$.   So

$$\eqalignno{\Delta_X(a^{-1})\int f\,d\mu
&=\int a\action_rfd\mu\cr
\noalign{\noindent (442Kc once more)}
&=\int T(a\action_rf)d\lambda=\psi(a)\int(Tf)(zw)\lambda(dz)\cr
&=\psi(a)\Delta_{Z}(w^{-1})\int Tf(z)\lambda(dz)
=\psi(a)\Delta_{Z}(w^{-1})\int f\,d\mu.\cr}$$

\noindent Thus $\Delta_X(a^{-1})=\psi(a)\Delta_{Z}(w^{-1})$;  because
both $\Delta_X$ and $\Delta_{Z}$ are multiplicative,

\Centerline{$\Delta_{Z}(\pi a)=\Delta_{Z}(w)=\psi(a)\Delta_X(a)$.}
}%end of proof of 443T

\leader{443U}{Transitive
\dvrocolon{actions}}\discrversionA{\footnote{Formerly 4{}43T.} }
{}\cmmnt{ All the results
from 443P onwards have been expressed in terms of groups acting on
quotient groups.   But the same structures can appear if we start from a
group action.   To simplify the hypotheses, I give the following result
for compact groups only.

\medskip

\noindent}{\bf Theorem} Let $X$ be a compact Hausdorff topological
group, $Z$ a non-empty compact Hausdorff space, and $\action$ a
transitive continuous action of $X$ on $Z$.
Write $\pi_z(x)=x\action z$ for $z\in Z$ and $x\in X$.

(a) For every $z\in Z$, $Y_z=\{x:x\in X,\,x\action z=z\}$ is a compact
subgroup of $X$.   If we give the set $X/Y_z$ of left cosets of $Y_z$ in
$X$ its quotient topology, we have a homeomorphism $\phi_z:X/Y_z\to Z$
defined by the formula $\phi_z(xY_z)=x\action z$ for every $x\in X$.

(b) Let $\mu$ be a Haar probability measure on $X$.   Then the image
measure $\mu\pi^{-1}_z$ is an $X$-invariant Radon probability measure on
$Z$, and $\mu\pi^{-1}_w=\mu\pi^{-1}_z$ for all $w$, $z\in Z$.

(c) Every non-zero $X$-invariant Radon measure on $Z$ is of the form
$\mu\pi^{-1}_z$ for a Haar measure $\mu$ on $X$ and some (therefore any)
$z\in Z$.

(d) There is a strictly positive $X$-invariant Radon probability measure
on $Z$, and any two non-zero $X$-invariant Radon measures on $Z$ are
scalar multiples of each other.

(e) Take any $z\in Z$, and let $\nu$ be the Haar probability measure of
$Y_z$.   If $\mu$ is a Haar measure on $X$, then

\Centerline{$\mu E=\int\nu(Y_z\cap x^{-1}E)\mu(dx)$}

\noindent whenever $E\subseteq X$ is Haar measurable.

\proof{{\bf (a)} Because $\action$ is an action of $X$ on $Z$,
$Y_z$ is always a subgroup;  because $\action$ is continuous, $Y_z$ is
closed, therefore compact.
Given $z\in Z$, then for $x$, $y\in X$ we have

\Centerline{$x\action z=y\action z
\iff x^{-1}y\in Y_z\iff xY_z=yY_z$.}

\noindent So the formula given for $\phi_z$ defines an injection from
$Z/Y_z$ to $Z$, which is surjective because $\action$ is transitive.
To see that $\phi_z$ is continuous, take any open set
$H\subseteq Z$.   Then

\Centerline{$\{x:xY_z\in\phi_z^{-1}[H]\}=\{x:x\action z\in H\}
=\pi_z^{-1}[H]$}

\noindent is open in $X$ (because $\action$ is continuous), so
$\phi_z^{-1}[H]$ is open in $X/Y_z$.
Because $X/Y_z$ is compact and $\phi_z$ is a bijection, $\phi_z$ is a
homeomorphism (3A3Dd).

\medskip

{\bf (b)} Because $X$ is compact, therefore unimodular (442Ic), we can
speak of `Haar measures' on $X$ without specifying `left' or `right'.
If $\mu$ is the Haar probability measure on $X$, then the image measure
$\mu\pi_z^{-1}$ is a Radon probability measure on $Z$ (418I once more).
To see
that the measures $\mu\pi_z^{-1}$ are $X$-invariant, take any Borel set
$H\subseteq Z$ and $y\in X$, and consider

$$\eqalign{(\mu\pi_z^{-1})(y^{-1}\action H)
&=\mu\{x:x\action z\in y^{-1}\action H\}
=\mu\{x:yx\action z\in H\}
=\mu\{x:yx\in\pi_z^{-1}[H]\}\cr
&=\mu(y^{-1}\pi_z^{-1}[H])
=\mu(\pi_z^{-1}[H])
=(\mu\pi_z^{-1})(H).\cr}$$

\noindent By 441B, this is enough to ensure that $\mu\pi_z^{-1}$ is
invariant.

If $w$, $z\in Z$ and $H\subseteq Z$ is a Borel set, then there is a
$y\in X$ such that $y\action w=z$, and now

\Centerline{$\pi_w^{-1}[H]=\{x:x\action w\in H\}
=\{x:(xy^{-1})\action z\in H\}=\{x:xy^{-1}\in\pi_z^{-1}[H]\}
=(\pi_z^{-1}[H])y$.}

\noindent But $\mu$ is a two-sided Haar measure, so

\Centerline{$(\mu\pi_w^{-1})(H)=\mu(\pi_w^{-1}[H])
=\mu((\pi_z^{-1}[H])y)=\mu(\pi_z^{-1}[H])=(\mu\pi_z^{-1})(H)$.}

\noindent Thus $\mu\pi_w^{-1}$ and $\mu\pi_z^{-1}$ agree on the Borel
sets and must be equal (416Eb).

\medskip

{\bf (c)} Now we come to the interesting bit.   Suppose that $\lambda$
is a non-zero $X$-invariant Radon measure on $Z$.   Take any $z\in Z$
and consider the Radon measure $\lambda'$ on $X/Y_z$ got by setting
$\lambda'H=\lambda\phi_z[H]$ whenever $H\subseteq X/Y_z$ and $\phi_z[H]$
is measured by $\lambda$.   In this case, if $x\in X$ and
$H\subseteq X/Y_z$ is measured by $\lambda'$,

$$\eqalignno{\lambda'(x\action H)
&=\lambda\{\phi_z(x\action w):w\in H\}
=\lambda\{\phi_z(x\action yY_z):y\in X,\,yY_z\in H\}\cr
&=\lambda\{\phi_z(xyY_z):y\in X,\,yY_z\in H\}
=\lambda\{xy\action z:y\in X,\,yY_z\in H\}\cr
&=\lambda\{x\action(y\action z):y\in X,\,yY_z\in H\}
=\lambda(x\action\phi_z[H])
=\lambda\phi_z[H]\cr
\displaycause{because $\lambda$ is $X$-invariant}
&=\lambda'H.\cr}$$

\noindent So $\lambda'$ is $X$-invariant.

Now let $\nu$ be the Haar probability measure on the compact Hausdorff
group $Y_z$.   By 443Qb, we have a (left) Haar measure $\mu$ on $X$
defined by the formula
$\mu G=\int\nu_wG\,\lambda'(dw)$ for every open $G\subseteq X$, where
$\nu_{xY_z}G=\nu(Y_z\cap x^{-1}G)$ for every $y\in X$ and every open
$G\subseteq X$.   Let $H\subseteq Z$ be an open set.   Then for any
$x\in X$,

$$\eqalign{\nu_{xY_z}(\pi_z^{-1}[H])
&=\nu\{y:y\in Y_z,\,xy\in\pi_z^{-1}[H]\}
=\nu\{y:y\in Y_z,\,xy\action z\in H\}\cr
&=\nu\{y:y\in Y_z,\,x\action(y\action z)\in H\}
=\nu\{y:y\in Y_z,\,x\action z\in H\}\cr
&=\nu Y_z
=1\text{ if }x\action z\in H,\cr
&=\nu\emptyset=0\text{ otherwise}.\cr}$$

\noindent So

\Centerline{$\mu(\pi_z^{-1}[H])
=\lambda'\{xY_z:x\in X,\,x\action z\in H\}
=\lambda'\phi_z^{-1}[H]
=\lambda H$.}

\noindent As $H$ is arbitrary, the image measure $\mu\pi_z^{-1}$ agrees
with $\lambda$ on the open subsets of $Z$;  as they are both Radon
measures, $\mu\pi_z^{-1}=\lambda$, as required.

\medskip

{\bf (d)} This is now easy.   $X$ carries a non-zero Haar measure, so by
(b) there is an $X$-invariant Radon probability measure on $Z$.
If $\lambda_1$ and $\lambda_2$ are non-zero $X$-invariant Radon measures
on $Z$, then they are of the form $\mu_1\pi_w^{-1}$ and
$\mu_2\pi_z^{-1}$ where $\mu_1$ and $\mu_2$ are Haar measures on $X$ and
$w$, $z\in Z$.
By (b) again, $\mu_1\pi_w^{-1}=\mu_1\pi_z^{-1}$, and since $\mu_1$ and
$\mu_2$ are multiples of each other (442B), so are $\lambda_1$ and
$\lambda_2$.

To see that the invariant probability measure $\lambda$ on $X$ is
strictly positive, take any non-empty open set $H\subseteq Z$.   Then
$Z$ is
covered by the open sets $x\action H$, as $x$ runs over $X$.
Because $Z$ is compact, it is covered by finitely many of these, so at
least one of them has non-zero measure.   But they all have the same
measure as $H$, so $\lambda H>0$.

\medskip

{\bf (e)} Write $\theta:X\to X/Y_z$ for the canonical map.   For
$w\in X/Y_z$ we have a Radon measure $\nu_w$ on $X$ defined by setting
$\nu_wE=\nu(Y_z\cap x^{-1}E)$ whenever $\theta x=w$ and the right-hand
side is defined (443Qa).   By (a)-(b) above, or otherwise, there is a
non-zero $X$-invariant Radon measure $\lambda$ on $X/Y_z$;  re-scaling
if necessary, we may suppose that $\lambda(X/Y_z)=\mu X$.   By 443Qe, we
have a Haar measure $\mu'$ on $X$ defined by setting
$\mu'E=\int\nu_wE\,\lambda(dw)$ for every Haar measurable $E$;  since

\Centerline{$\mu'X=\int\nu_wX\,\lambda(dw)
=\int\nu Y_z\,\lambda(dw)=\lambda(X/Y_z)=\mu X$,}

\noindent $\mu'=\mu$.   Moreover, $\lambda=\mu'\theta^{-1}$ (443Qd).
So

$$\eqalign{\mu E
&=\mu'E
=\int\nu_wE\,\lambda(dw)\cr
&=\int\nu_{\theta x}E\,\mu'(dx)
=\int\nu(Y_z\cap x^{-1}E)\mu(dx)\cr}$$

\noindent for every Haar measurable $E\subseteq X$.
}%end of proof of 443U

\exercises{\leader{443X}{Basic exercises $\pmb{>}$(a)}%
%\sqheader 443Xa
\dvAformerly{4{}43Xb} Let $X$ be a topological group, $\mu$ a left
Haar measure on $X$ and $\lambda$ the corresponding quasi-Radon product
measure on $X\times X$.   (i) Show that the maps
$(x,y)\mapsto(y,x)$,
$(x,y)\mapsto(x,xy)$, $(x,y)\mapsto(y^{-1}x,y)$ are automorphisms of the
measure space $(X\times X,\lambda)$.   \Hint{use 417C(iv) to show that
they preserve the measures of open sets.}   (ii) Show that the maps
$(x,y)\mapsto(y^{-1},xy)$, $(x,y)\mapsto(yx,x^{-1})$,
$(x,y)\mapsto(y^3x,x^{-1}y^{-2})$ are automorphisms
of $(X\times X,\lambda)$.   \Hint{express them as compositions of maps
of the forms in (ii).}
%- %

\spheader 443Xb
Let $X$ be a topological group carrying Haar measures
and $A\subseteq X$.   (i) Show that
$A$ is self-supporting (definition:  411Na) for one Haar
measure on $X$ iff it is self-supporting for every Haar measure on $X$.
(ii) Show that $A$ has non-zero inner measure for one
Haar measure on $X$ iff it has non-zero inner measure for every Haar
measure on $X$.
%443A

\spheader 443Xc Let $X$ be a topological group, $\mu$ a Haar measure on
$X$, and $E$, $F$ measurable subsets of $X$.   Show that
$(x,y,w,z)\mapsto\mu(xEy\cap wFz):X^4\to[0,\infty]$ is lower
semi-continuous.
%443B

\spheader 443Xd Let $X$ be a topological group carrying Haar measures
and $\frak A$ its Haar measure algebra.
(i) Show that we have a continuous
action of $X\times X$ on $\frak A$ defined by the formula
$(x,y)\action E^{\ssbullet}=(xEy^{-1})^{\ssbullet}$ for $x$, $y\in X$
and Haar measurable sets $E\subseteq X$.
(ii) Show that if $x\in X$ and
$a\in\frak A$ then
$x\action_ra=(x\action_l\Reverse{a})\ssplrarrow$, where $\Reverse{a}$ is as
defined in 443Af.
%443C

\spheader 443Xe Let $(\frak A,\bar\mu)$ be a measure algebra.   Show that
it is isomorphic to the measure algebra of a topological group with a Haar
measure iff it is localizable and quasi-homogeneous in the sense of
374G-374H.
%443D

\spheader 443Xf Let $X$ be a topological group with a left Haar measure
$\mu$.   (i) Show that if $Y$ is a subgroup of $X$ such that $\mu_*Y>0$,
then $Y$ is open.   In particular, any non-negligible closed subgroup of
$X$ is open.
(ii) Let $Y$ be any subgroup of $X$ which is not Haar negligible.   Show
that the subspace measure $\mu_Y$ is a left Haar measure on $Y$.   Show
that $\overline{Y}$ is a Haar measurable envelope of $Y$.   \Hint{apply
443Db inside the topological group $\overline{Y}$.}
%443D %443F

\spheader 443Xg Write out a version of 443G for right Haar measures.
%443G

\spheader 443Xh Let $X$ be a topological group carrying Haar measures,
and $L^0$ the space of equivalence classes of Haar measurable functions,
as in 443A;  let $u\mapsto\Reverse{u}:L^0\to L^0$ be the operator of 443Af.
Show that if
$\mu$ is a left Haar measure on $X$ and $\nu$ is a right Haar measure,
$p\in[1,\infty]$ and $u\in L^0$, then $\Reverse{u}\in L^p(\nu)$ iff
$u\in L^p(\mu)$.
%443G

\spheader 443Xi Let $X$ be a topological group carrying Haar measures,
and $\frak A$ its Haar measure algebra.   Show that, in the language of
443C and 443G, $\chi(x\action_la)=x\action_l\chi a$ and
$\chi(x\action_ra)=x\action_r\chi a$ for every $x\in X$ and
$a\in\frak A$.
%443G

\spheader 443Xj Let $X$ be a topological group carrying Haar measures.
Show that $X$ is totally bounded for its bilateral uniformity
iff $X$ is totally bounded for its right uniformity\cmmnt{ (definition:
4A5Ha)} iff its Haar measures are totally finite.
%443H

\spheader 443Xk Let $X$ be a topological group, $\mu$ a left Haar
measure on $X$, and $A\subseteq X$ a set which is self-supporting for
$\mu$.   Show that the following are equiveridical:  (i) for every
neighbourhood $U$ of the identity $e$, there is a countable set
$I\subseteq X$ such that $A\subseteq UI$;  (ii) for every neighbourhood
$U$ of $e$, there is a countable set $I\subseteq X$ such that
$A\subseteq IU$;
(iii) for every neighbourhood $U$ of $e$, there is a countable set
$I\subseteq X$ such that $A\subseteq IUI$;   (iv) $A$ can be covered by
countably many sets of finite measure for $\mu$;  (v) $A$ can be covered
by countably many open sets of finite measure for $\mu$;  (vi) $A$ can
be covered by countably many sets which are totally bounded for the
bilateral uniformity on $X$.
%443I

\sqheader 443Xl Let $X$ be a topological group carrying Haar measures.
(i) Show that the following are equiveridical:  ($\alpha$) $X$ is ccc;
($\beta$) $X$ has a $\sigma$-finite Haar measure;  ($\gamma$) every Haar
measure on $X$ is $\sigma$-finite.
(ii) Show that if $X$ is locally compact and Hausdorff, we can add
($\delta$) $X$ is $\sigma$-compact.
%443Xk 443I

\spheader 443Xm Let $X$ be a topological group carrying Haar measures.
Show that every subset of $X$ has a Haar measurable envelope which is a
Borel set.
%443J

\spheader 443Xn In 443L, show that (i) $\phi[A]$ is Haar negligible in $Z$
whenever $A$ is Haar negligible in $X$ (ii) $\Delta_X=\Delta_Z\phi$, where
$\Delta_X$, $\Delta_Z$ are the left modular functions of $X$, $Z$
respectively (iii) $\phi[X]$ is dense in $Z$ (iv) $Z$ is unimodular iff $X$
is unimodular.
%443L

\sqheader 443Xo Let $X$ and $Y$ be topological groups with left Haar
measures $\mu$ and $\nu$.   Show that the c.l.d.\ and quasi-Radon
product measures of $\mu$ and $\nu$ on $X\times Y$ coincide.
\Hint{start with locally compact Hausdorff spaces, and show that a
compact G$_{\delta}$ set in $X\times Y$ belongs to
$\Cal B(X)\tensorhat\Cal B(Y)$, where $\Cal B(X)$ and $\Cal B(Y)$ are
the Borel $\sigma$-algebras of $X$ and $Y$;  now use 441Xi and 443L.}
%443M

\spheader 443Xp Let $\familyiI{X_i}$ be a family of topological groups
and $X=\prod_{i\in I}X_i$ their product.   Suppose that each $X_i$ has a
Haar probability measure $\mu_i$.   Show that the ordinary and
quasi-Radon product measures on $X$ coincide.
%443M

\spheader 443Xq Let $X$ be a locally compact Hausdorff group and $Y$ a
closed subgroup of $X$;  write $X/Y$ for the space of left cosets of $Y$
in $X$, with its quotient topology.   Show that if $\lambda_1$ and
$\lambda_2$ are non-zero $X$-invariant Radon measures on $X/Y$, then
each is a multiple of the other.   \Hint{look at the Haar measures they
define on $X$.}
%443Q

\sqheader 443Xr Write $S^1$ for the circle group
$\{s:s\in\Bbb C,\,|s|=1\}$, and set $X=S^1\times S^1$, where the first copy of $S^1$
is given its usual topology and the second copy is given its discrete
topology, so that $X$ is an abelian locally compact Hausdorff group.
Set $E=\{(s,s):s\in S^1\}$.
(i) Show that $E$ is a closed Haar negligible subset of $X$.
(ii) Set $Y=\{(1,s):s\in S^1\}$;  check that $Y$ is a closed
normal subgroup of $X$, and that the quotient group $X/Y$ can be
identified with $S^1$ with its usual topology;  let $\lambda$ be the
Haar probability measure of $X/Y$.   Let $\nu$ be counting measure on
$Y$.   Show that, in the language of 443Q, $\nu_zE=1$ for every
$z\in X/Y$, so that $\mu E\ne\int\nu_zE\,\lambda(dz)$.
(iii) Setting $f=\chi E$,
show that the map $a\mapsto f_a^{\ssbullet}$ described in 443Qf is not
almost continuous.
%443Q

\spheader 443Xs(i) In 443P, suppose that $G\subseteq X$ is an open set
such that $GY=X$.   Show that for every $h\in C_k(Z)^+$ there is an
$f\in C_k(X)^+$ such that $Tf=h$
and $\overline{\{x:f(x)>0\}}\subseteq G$.
(ii) In 443Pc, show that $\psi$ is multiplicative.
(iii) In 443R, suppose that
there is an open set $G\subseteq X$
such that $GY$ has finite measure for the left Haar measures
of $X$.   Show that $Z$ has an $X$-invariant Radon probability measure.
\Hint{$Y$ is totally bounded for its right uniformity.}
%443R

\spheader 443Xt Let $X$ be a locally compact Hausdorff group.   Show
that it has a closed normal subgroup $Y$ such that $Y$ and $X/Y$ are
both unimodular.   \Hint{take $Y=\{x:\Delta(x)=1\}$.}
%443Sa mt44bits

\sqheader 443Xu Let $X=\BbbR^2$ be the example of 442Xf.   (i) Let
$Y_1$ be the subgroup $\{(\xi,0):\xi\in\Bbb R\}$.   Describe the left
cosets of $Y_1$ in $X$.   Show that there is no non-trivial
$X$-invariant Radon measure on the set $X/Y_1$ of these left cosets.
Find a base $\Cal U$ for the topology of $X/Y_1$ such that you can
identify the sets $x\action U$, where $x\in X$ and $U\in\Cal U$, with
sufficient precision to explain why the hypothesis (iii) of 441C is not
satisfied.  (ii) Let $Y_2$ be the normal subgroup
$\{(0,\xi):\xi\in\Bbb R\}$.   Find
the associated function $\psi:X\to\ooint{0,\infty}$ as described in
443Pc and 443T.
%443T, 442Xf

\spheader 443Xv Let $X$ be a locally compact Hausdorff group and $Y$ a
compact normal subgroup of $X$.   Show that $X$ is unimodular iff the
quotient group $X/Y$ is unimodular.   \Hint{the function $\psi$ of 443T
must be constant.}
%443T

\spheader 443Xw Take any integer $r\ge 1$, and let $G$ be the isometry
group of $\BbbR^r$ with its topology of pointwise convergence (441G).
(i) Show that $G$ is metrizable and locally compact.   \Hint{441Xp.}
(ii) Let $H\subseteq G$ be the set of translations.   Show that
$H$ is an abelian closed normal subgroup of $G$, and that Lebesgue
measure on $\BbbR^r$ can be regarded as a Haar measure on $H$.   (iii)
Show that the quotient group $G/H$ is compact.   (iv) Show that $G$ is
unimodular.   \Hint{the function $\psi$ of 443T is constant.}
%443T

\sqheader 443Xx Set $X=\BbbR^3$ with the operation

\Centerline{$(\xi_1,\xi_2,\xi_3)*(\eta_1,\eta_2,\eta_3)
=(\xi_1+\eta_1,\xi_2+e^{\xi_1}\eta_2,\xi_3+e^{-\xi_1}\eta_3)$.}

\noindent (i) Show that (with the usual topology of $\BbbR^3$) $X$ is a
topological group.   (ii) Show that it is unimodular.   \Hint{Lebesgue
measure is a two-sided Haar measure.}
(iii) Show that $X$ has both a closed subgroup and a
Hausdorff quotient group which are not unimodular.
%443T, 442Xf

\sqheader 443Xy Let $(X,\rho)$ be a non-empty compact metric space such
that the group $G$ of isometries of $X$ is transitive.   Show that any
two non-zero
$G$-invariant Radon measures on $X$ must be multiples of each other.
\Hint{441Gb, 443U.}
%443U

\sqheader 443Xz Show that 443G is equally valid if we take functions to be
complex-valued rather than real-valued, and work with $L^p_{\Bbb C}$
rather than $L^p$.
%used in \S445

\leader{443Y}{Further exercises (a)}
%\spheader 443Ya
Let $X$ be a topological group carrying Haar measures
and $\frak A$ its Haar measure algebra.   Show that two principal ideals
of $\frak A$ are isomorphic (as Boolean algebras) iff they have the same
cellularity.
%443D

\spheader 443Yb Let $X$ be a topological group carrying Haar measures,
and $E\subseteq X$ a Haar measurable set such that $E\cap U$ is not Haar
negligible for any neighbourhood $U$ of the identity.   Show that for
any $A\subseteq X$ the set
$A'=\{x:x\in A,\,A\cap xE$ is Haar negligible$\}$ is Haar negligible.
\leaveitout{otherwise,
take a closed self-supporting non-negligible set $F$ of finite measure,
which is a Haar measurable envelope of $F\cap A'$,
and show that $F\cap xE$ is
negligible for every $x\in F$.   Consider the set
$\{(x,y):x\in F,\,y\in E,\,xy\in F\}$;  show that its vertical
sections are negligible but some of its horizontal sections are not.}
%443D

\spheader 443Yc Let $X$ be a locally compact Hausdorff group.   Show
that we have continuous shift
actions $\action_l$, $\action_r$ and $\action_c$
of $X$ on the Banach space $C_0(X)$ defined by formulae corresponding to
those of 443G.
%443G

\spheader 443Yd Let $X$ be a compact Hausdorff topological group and
$\frak A$ its Haar measure algebra.   Let $Y$ be a subgroup of $X$;
for $y\in Y$, define $\hat y\in\Aut\frak A$ by setting
$\hat y(a)=y\action_la$ for $a\in\frak A$.
Show that $\{\hat y:y\in Y\}$ is ergodic (definition:
395Ge) iff $Y$ is dense in $X$.
%443G

\spheader 443Ye Let $\frak A$ be a Boolean algebra, $G$ a group, and
$\action$ an action of $G$ on $\frak A$ such that $a\mapsto g\action a$ is
a Boolean automorphism for every $g\in G$.   (i) Show that we have a
corresponding action of $G$ on $L^{\infty}=L^{\infty}(\frak A)$
defined by saying that, for every $g\in G$,
$g\action\chi a=\chi(g\action a)$ for $a\in\frak A$ and
$u\mapsto g\action u$ is a positive linear operator on
$L^{\infty}$.   (ii) Show that if $\frak A$ is
Dedekind $\sigma$-complete,
this action on $L^{\infty}$ extends to an action on
$L^0=L^0(\frak A)$ defined by saying that
$\Bvalue{g\action u>\alpha}=g\action\Bvalue{u>\alpha}$ for $g\in G$,
$u\in L^0$ and $\alpha\in\Bbb R$.
%443G

\spheader 443Yf Let $(\frak A,\bar\mu)$ be a measure algebra,
$G$ a topological group, and
$\action$ a continuous action of $G$ on $\frak A$ (when $\frak A$ is
given its measure-algebra topology) such that $a\mapsto g\action a$ is
a measure-preserving Boolean automorphism for every $g\in G$.
(i) Show that the corresponding action of $G$ on $L^0=L^0(\frak A)$, as
defined in
443Ye, is continuous when $L^0$ is given the topology of convergence in
measure, and induces continuous actions of $G$ on
$L^p=L^p(\frak A,\bar\mu)$ for $1\le p<\infty$.
(ii) Show that if we give the
unit ball $B$ of $L^{\infty}=L^{\infty}(\frak A)$ the topology induced by
$\frak T_s(L^{\infty},L^1)$, then the action of $G$ on $L^0$ induces a
continuous action of $G$ on $B$.
%443Ye 443G

\spheader 443Yg Let $X$ be a topological group with a left Haar measure
$\mu$, and $A\subseteq X$.   Show that the following are equiveridical:
(i) $A$ is totally bounded for the bilateral uniformity of $X$ (ii)
there are non-empty open sets $G$, $H\subseteq X$ such that $\mu(AG)$,
$\mu(A^{-1}H)$ are both finite.
%443I

\spheader 443Yh
Give an example of a locally compact Hausdorff group, with left Haar
measure $\mu$, such that no open normal subgroup can be covered by a
sequence of sets of finite measure for $\mu$.
%443J mt43bits

\spheader 443Yi Let $X$ be a topological group.   Let $\Sigma$ be the
family of subsets of $X$ expressible in the form $\phi^{-1}[F]$ for some
Borel subset $F$ of a separable metrizable topological group $Y$ and
some continuous homomorphism $\phi:X\to Y$.   Show that $\Sigma$ is a
$\sigma$-algebra of subsets of $X$ and that multiplication, regarded as
a function from $X\times X$ to $X$, is
$(\Sigma\tensorhat\Sigma,\Sigma)$-measurable.   Show that any compact
G$_{\delta}$ set belongs to $\Sigma$.   Show that if $X$ is
$\sigma$-compact, then $\Sigma$ is the Baire $\sigma$-algebra of $X$.
%443L

\spheader 443Yj Let $X$ be any Hausdorff topological group of cardinal
greater than $\frak c$.   Let $\Cal B$ be the Borel $\sigma$-algebra of
$X$.   Show that $(x,y)\mapsto xy$ is not
$(\Cal PX\tensorhat\Cal PX,\Cal B)$-measurable.
\leaveitout{show that for
any sequence $\sequencen{A_n}$ in $\Cal PX$ there are distinct $x$,
$y\in X$ such that $\{n:x\in A_n\}=\{n:y\in A_n\}$;  consequently
$\{(x,y):xy^{-1}=e\}\notin\Cal PX\tensorhat\Cal PX$.}
%443Yi (443L)

\spheader 443Yk Let $X$ be a topological group and $\mu$ a left Haar
measure on $X$.   Show that $\mu$ is inner regular with respect to the
family of closed sets $F\subseteq X$ such that
$F=\bigcap_{n\in\Bbb N}FU_n$ for some sequence $\sequencen{U_n}$ of
neighbourhoods of the identity.   \leaveitout{use 443M
for the locally compact
case, and 443L to extend the result to general Haar measures.}
%443L, 443M

\spheader 443Yl
Let $X$ be a topological group carrying Haar measures.   Let
$E\subseteq X$ be a Haar
measurable set such that $E\cap U$ is not Haar negligible for any
neighbourhood $U$ of the identity.   Show that there is a sequence
$\sequencen{x_n}$ in $E$ such that
$x_{i_0}x_{i_1}\ldots x_{i_n}\in E$ whenever $n\in\Bbb N$ and
$i_0<i_1<\ldots<i_n$ in $\Bbb N$.   (See {\smc Plewik \& Voigt 91}.)
%443C, 443Yb, 443L, 443M, 443Yk mt43bits

\spheader 443Ym Let $X$ be a topological group and $\mu$ a Haar
measure on $X$.   Show that any closed self-supporting subset of $X$ is
a zero set.   \leaveitout{start with the case in which $X$ is
locally compact and $\sigma$-compact.}
%443M

\spheader 443Yn Find a compact Hausdorff space $X$ with a strictly
positive Radon measure such that there is a regular open set
$G\subseteq X$ which is not a cozero set.
%443N %mt44bits 419A

\spheader 443Yo Let $X$ be a locally compact Hausdorff topological
group which is not discrete (as topological space).   (i) Show that there
is a Haar negligible zero set containing the identity of $X$.   (ii) Show
that if $X$ is $\sigma$-compact, it
has a Haar negligible compact normal subgroup $Y$ which is a zero
set in $X$, so that $X/Y$ is metrizable.   (iii) Show that there is a Haar
negligible set $A\subseteq X$ such that $AA$ is not Haar measurable.
%419Xe, 443O(v) for (iii)

\spheader 443Yp Find a
non-discrete locally compact Hausdorff topological group $X$ such
that if $Y$ is a normal subgroup of $X$ which is a zero set in $X$ then $Y$
is open.
%443Yo 443O

\spheader 443Yq Show that there is a subgroup $X$ of the additive group
$\BbbR^2$ such that $X$ has full outer Lebesgue measure but
$\{\xi:(\xi,0)\in X\}=\Bbb Q$.   \leaveitout{enumerate the family
$\Cal K$ of
compact subsets of $\BbbR^2$ of positive Lebesgue measure as
$\langle K_{\xi}\rangle_{\xi<\frak c}$ (cf.\ 4A3Fa).   Show that
$\#(\pi_2[K])=\frak c$ for every $K\in\Cal K$, where
$\pi_2(\alpha_1,\alpha_2)=\alpha_2$ (Fubini's
theorem and 419H).   Choose $x_{\xi}\in K_{\xi}$, for $\xi<\frak c$,
such that $\pi_2x_{\xi}$ is never a rational linear combination of
$\{\pi_2x_{\eta}:\eta<\xi\}$.   Let $X$ be the group generated by
$\{x_{\xi}:\xi<\frak c\}\cup\{(q,0):q\in\Bbb Q\}$.}
Show that $X$ carries Haar measures, but that its closed subgroup
$X\cap(\Bbb R\times\{0\})$ does not.
%443P

\spheader 443Yr Show that if $X$ is a locally compact Hausdorff group,
$Y$ a compact subgroup of $X$, $Z=X/Y$ the set of left cosets of $Y$ and
$\pi:X\to Z$ the canonical map, and $Z$ is given its quotient
topology, then $R=\{(\pi x,x):x\in X\}$ is an usco-compact relation in
$Z\times X$ and $R[L]=\pi^{-1}[L]$ is compact
for every compact $L\subseteq Z$.
%443S

\spheader 443Ys Let $G$ be the isometry group of $\BbbR^r$, as in
443Xw.   (i) Show that
if we set $\rho(g,h)=\sup_{\|x\|\le 1}\|g(x)-h(x)\|$, then $\rho$ is a
metric on $G$ defining its topology.   (ii) Describe Haar measures
on $G$ ($\alpha$) in terms of Hausdorff measure of an appropriate
dimension for the metric $\rho$
($\beta$) in terms of a parametrization of $G$ and
Lebesgue measure on a suitable Euclidean space.
%443Xw, 443T

\spheader 443Yt Let $r\ge 1$ be an integer, and $G$ the group of
invertible affine transformations of $\BbbR^r$, with the topology of
pointwise convergence inherited from $(\BbbR^r)^{\BbbR^r}$.   (i) Show
that $G$ is a locally compact Hausdorff topological group.
(ii) Show that $G$ is
not unimodular, and find its modular functions.
%443Xw (443T) 443Ys 442Yb 443Xu

\spheader 443Yu Let $X$ be a topological group with a left Haar measure
$\mu$ and left modular function $\Delta$;  suppose that $X$ is not
unimodular.   Show that
$\mu\{x:\alpha<\Delta(x)<\beta\}=\infty$ whenever $\alpha<\beta$
and $\{x:\alpha<\Delta(x)<\beta\}$ is non-empty.
%443T %443L
}%end of exercises

\endnotes{
\Notesheader{443} Most of us, by the time we come to study measures on
general topological groups, have come to trust our intuition concerning
the behaviour of Lebesgue measure on $\Bbb R$, and the principal
discipline imposed by the subject is the search for the true path
between hopelessness and over-confidence when extending this intuition
to the general setting.   The biggest step is the loss of commutativity,
especially as the non-abelian groups of elementary courses in group
theory are mostly finite, and are therefore untrustworthy guides to the
concerns of this chapter.   Accordingly we find ourselves going rather
slowly and carefully through the calculations in such results as 443C.
Note, for instance, that in an abelian group the actions I call
$\action_l$ and $\action_r$ are still different;  $x\action_lE$
corresponds to $x+E=E+x$, but $x\action_rE$ corresponds to $E-x$.   (The
action $\action_c$ becomes trivial, of course.)   When we come to
translate the formulae of Fourier analysis in the next section,
manoeuvres of this kind will often be necessary.   In the present
section, I have done my best to give results in `symmetric' forms;  you
may therefore take it that when the words `left' or `right' appear in
the statement of a proposition, there is some real need to break the
symmetry.   Subject to these remarks, such results as 443C and 443G are
just a matter of careful conventional analysis.   I see that I have used
slightly different techniques in the two cases.   Of course 443C can be
thought of as a special case of 443Gc-443Gd, if we remember that
$\chi:\frak A\to L^0(\frak A)$ embeds $\frak A$ topologically as a
subspace of $L^0$ (367R).

443B, 443D and 443E belong to a different family;  they deal with actual
sets rather than with members of a measure algebra or a function space.
I suppose it is 443D which will most often be quoted.
%check when possible!  443D (2) 443B (0) 443E (0) 28.10.01
Its corollary 443E deals with the obvious question of when Haar measures
are tight.

Readers who have previously encountered the theory of Haar measures on
locally compact groups will have been struck by how closely the more
general theory here
is able to follow it.   The explanation lies in 443K-443L;  my general
Haar measures are really just subspace measures on subgroups of full
Haar measure in locally compact groups.   Knowing this, it is easy to
derive all the results above from the locally compact theory.   I use
this method only in 443M-443O, %443M 443N 443O
because (following {\smc Halmos 50}) I feel
that from the point of view of pure measure theory the methods show
themselves more clearly if we do not use ideas depending on compactness,
but instead rely directly on $\tau$-additivity.   But I note that 443Xo
and 443Yi also go faster with the aid of 443L.

{\smc Halmos 50} goes a little farther, with a theory of groups carrying
translation-invariant measures for which the operation $(x,y)\mapsto xy$
is $(\Sigma\tensorhat\Sigma,\Sigma)$-measurable, where $\Sigma$ is the
domain of the measure.   The essence of the theory here, and my reason
for insisting on $\tau$-additive measures, is that for these we
have a suitable theory of product measures.   If we start with
quasi-Radon measures and use the quasi-Radon product measure, then
multiplication is measurable just because it is continuous.   In order
to achieve similar results without either assuming metrizability or
using the theory of $\tau$-additive product measures, we have to
restrict the measure to something smaller than the Borel
$\sigma$-algebra, as in 443Yg.   (See 443Yj.)
443J and 443Xo show that certain disconcerting features of general
Radon measures (419C, 419E) cannot arise in the case of Haar measures.

When I come to look at subgroups and quotient groups, I do specialize to
the locally compact case.   One obstacle to generalization is the fact
that a closed subgroup
of a group carrying Haar measures need not itself carry Haar measures
(443Yq).   443P and 443R are taken from {\smc Federer 69}, who goes
farther, with many other applications.   I give a fairly detailed
analysis of the relationship between a Haar measure on a group $X$ and a
corresponding $X$-invariant measure on a family of left cosets (443Q)
for the sake of applications in \S447.   One of the challenges here is
to distinguish clearly those results which apply to all locally compact
groups from those which rely on $\sigma$-compactness or some such
limitation.   There is a standard example (443Xr) which provides a
useful test in such questions.   You may recognise this as a version of
a fundamental example related to Fubini's theorem, given in 252K.

Most of the results of this section begin with a topological group $X$.
But starting from \S441 it is equally natural to start with a group
action.   If we have a topological group $X$ acting continuously on a
topological space $Z$, and $X$ carries Haar measures, then we have a
good chance of finding an invariant measure on $Z$.   In the simplest
case, in which $X$ and $Z$ are both compact and the action is
transitive, there is a unique invariant Radon probability measure on $Z$
(443U).
}%end of notes

\discrpage



