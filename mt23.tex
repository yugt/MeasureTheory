\frfilename{mt23.tex}
\versiondate{17.11.04}
\copyrightdate{1995}


\def\chaptername{The Radon-Nikod\'ym Theorem}
\newchapter{23}
\def\chaptername{The Radon-Nikod\'ym theorem}

In Chapter 22, I discussed the indefinite integrals of
integrable functions on $\Bbb R$, and gave what I hope you feel are
satisfying descriptions both of
the functions which are indefinite integrals (the absolutely
continuous functions) and of how to find which functions they are
indefinite integrals of (you differentiate them).   For general measure
spaces, we have no structure present which can give such simple
formulations;  but nevertheless the same questions can be asked and, up
to a point, answered.

The first section of this chapter introduces the basic machinery needed,
the concept of `countably additive' functional and its decomposition
into positive and negative parts.   The main theorem takes up the second
section:  indefinite integrals are the `truly continuous' additive
functionals;  on $\sigma$-finite spaces, these are the `absolutely
continuous' countably additive functionals.   In \S233 I discuss the
most important single application of the theorem, its use in providing a
concept of `conditional expectation'.   This is one of the central
concepts of probability theory -- as you very likely know;  but the form
here is a dramatic generalization of the elementary concept of the
conditional probability of one event given another, and needs the whole
strength of the general theory of measure and integration as developed
in Volume 1 and this chapter.   I include some notes on convex
functions, up to and including versions of Jensen's inequality
(233I-233J).

While we are in the area of `pure' measure theory, I take the opportunity 
to discuss some further topics.
I begin with some essentially elementary constructions,
image measures, sums of measures and indefinite-integral measures;
I think the details need a little attention,
and I work through them in \S234.
Rather deeper ideas are needed to deal with `measurable
transformations'.   In \S235 I set out the techniques necessary to provide an
abstract basis for a general method of integration-by-substitution, with
a detailed account of sufficient conditions for a formula of the type

\Centerline{$\int g(y)dy=\int g(\phi(x))J(x)dx$}

\noindent to be valid.

\discrpage
