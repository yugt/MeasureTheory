\frfilename{mt216.tex}
\versiondate{25.9.04}
\copyrightdate{1994}

\def\chaptername{Taxonomy of measure spaces}
\def\sectionname{Examples}

\newsection{216}

It is common practice -- and, in my view, good
practice -- in books on pure mathematics, to provide discriminating
examples;  I mean that whenever we are given a list of new concepts, we
expect to be provided with examples to show that we have a fair picture
of the relationships between them, and in particular that we are not
being kept ignorant of some startling implication.   Concerning the
concepts listed in
211A-211K, %211A 211B 211C 211D 211E 211F 211G 211H 211I 211J 211K
we have ten different properties which some, but
not all, measure spaces possess, giving a conceivable total of $2^{10}$
different types of measure space, classified according to which of these
ten properties they have.   The list of basic relationships in 211L
reduces these 1024 possibilities to 72.   Observing that a space can be
simultaneously atomless and purely atomic only when the measure of the
whole space is $0$, we find ourselves with 56 possibilities, being two
trivial cases with $\mu X=0$ (because such a measure may or may not be
complete) together with $9\times 2\times 3$ cases, corresponding to the
nine classes

\quad probability spaces,

\quad spaces which are totally finite, but not probability spaces,

\quad spaces which are $\sigma$-finite, but not totally finite,

\quad spaces which are strictly localizable, but not $\sigma$-finite,

\quad spaces which are localizable and locally determined, but not
strictly localizable,

\quad spaces which are localizable, but not locally determined,

\quad spaces which are locally determined, but not localizable,

\quad spaces which are semi-finite, but neither locally determined nor
localizable,

\quad spaces which are not semi-finite;

\noindent the two classes

\quad spaces which are complete,

\quad spaces which are not complete;

\noindent and the three classes

\quad spaces which are atomless, not of measure $0$,

\quad spaces which are purely atomic, not of measure $0$,

\quad spaces which are neither atomless nor purely atomic.

I do not propose to give a complete set of fifty-six examples,
particularly as rather fewer than fifty-six different ideas are
required.   However, I do think that for a proper understanding of
abstract measure spaces it is necessary to have seen realizations of
some of the critical combinations of properties.   I therefore take a
few paragraphs to describe three special examples to add to those of
211M-211R. %211M 211N 211O 211P 211Q 211R

\leader{216A}{Lebesgue measure}\cmmnt{ Before turning to the new ideas, let me mention Lebesgue measure again.
As remarked in 211M, 211P and 211Qa,

}(a) Lebesgue measure $\mu$ on $\Bbb R$ is complete, atomless and
$\sigma$-finite, therefore
strictly localizable, localizable and locally determined.

(b) The subspace measure $\mu_{[0,1]}$ on $[0,1]$ is a complete,
atomless probability measure.

(c) The restriction $\mu\restr\Cal B$ of $\mu$ to the Borel
$\sigma$-algebra $\Cal B$ of $\Bbb R$ is atomless,
$\sigma$-finite and not complete.

\cmmnt{
\leader{216B}{}I now embark on the description of three
`counter-examples';  meaning
spaces built specifically for the purpose of showing that there are no
unexpected implications among the ten properties under consideration
here.   Even by the standards of this chapter these must be regarded as
dispensable by the student who wants to get on with the real business of
understanding the big theorems of the subject.   Neither the existence
of these examples, nor the techniques needed in constructing them, are
vital for anything else we shall look at before Volume 5.   But if you
are going to take abstract measure theory seriously at all, sooner or
later you will need to form some kind of mental picture of the nature of
the spaces possessing the different properties here, and a minimal
requirement of such a picture is that it should include the
discriminations witnessed by these examples.
}%end of comment

\leader{*216C}{A complete, localizable, non-locally-determined
space}\cmmnt{ The
first example hardly needs an idea beyond what we already have, but it
does call for more manipulations than it seems fair to set as an
exercise, and may therefore be useful as a demonstration of technique.

\medskip

}{\bf (a)} Let $I$ be any uncountable set, and set $X=\{0,1\}\times I$.
For $E\subseteq X$, $y\in\{0,1\}$ set
$E[\{y\}]=\{i:(y,i)\in E\}\subseteq I$.   Set

\Centerline{$\Sigma=\{E:E\subseteq X,\,E[\{0\}]\symmdiff E[\{1\}]$ is
countable$\}$.}

\noindent Then $\Sigma$ is a $\sigma$-algebra of subsets of $X$.
\prooflet{\Prf\ (i)
$\emptyset[\{0\}]\symmdiff\emptyset[\{1\}]=\emptyset$ is countable, so
$\emptyset\in\Sigma$.   (ii) If $E\in\Sigma$ then

\Centerline{$(X\setminus E)[\{0\}]
\symmdiff(X\setminus E)[\{1\}]=E[\{0\}]\symmdiff E[\{1\}]$}

\noindent is countable.   (iii) If $\sequencen{E_n}$ is a
sequence in
$\Sigma$ and $E=\bigcup_{n\in\Bbb N}E_n$, then

\Centerline{$E[\{0\}]\symmdiff E[\{1\}]
\subseteq\bigcup_{n\in\Bbb N}E_n[\{0\}]\symmdiff E_n[\{1\}]$}

\noindent is countable.\ \Qed}

For $E\in\Sigma$, set $\mu E=\#(E[\{0\}])$ if this is finite, $\infty$
otherwise;  then $(X,\Sigma,\mu)$ is a measure space.

\medskip

{\bf (b)} $(X,\Sigma,\mu)$ is complete.   \prooflet{\Prf\ If $A\subseteq
E\in\Sigma$ and $\mu E=0$, then $(0,i)\notin E$ for every $i$.   So

\Centerline{$A[\{0\}]\symmdiff A[\{1\}]=A[\{1\}]\subseteq E[\{1\}]
=E[\{1\}]\symmdiff E[\{0\}]$}

\noindent must be countable, and $A\in\Sigma$.\ \Qed}

\medskip

{\bf (c)} $(X,\Sigma,\mu)$ is semi-finite.   \prooflet{\Prf\ If $E\in\Sigma$
and $\mu E>0$, there is an $i\in I$ such that $(0,i)\in E$;  now
$F=\{(0,i)\}\subseteq E$ and $\mu F=1$.\ \Qed}

\medskip

{\bf (d)} $(X,\Sigma,\mu)$ is localizable.   \prooflet{\Prf\ Let $\Cal E$ be
any subset of $\Sigma$.   Set

\Centerline{$J=\bigcup_{E\in\Cal E}E[\{0\}]$,
\quad $G=\{0,1\}\times J$.}

\noindent Then $G\in\Sigma$.   If $H\in\Sigma$, then

$$\eqalign{\mu(E\setminus H)&=0 \text{ for every }E\in\Cal E\cr
&\iff E[\{0\}]\subseteq H[\{0\}]\text{ for every }E\in\Cal E\cr
&\iff (0,i)\in H\text{ for every }i\in J\cr
&\iff \mu(G\setminus H)=0.\cr}$$

\noindent Thus $G$ is an essential supremum for $\Cal E$ in $\Sigma$;  as
$\Cal E$ is arbitrary, $\mu$ is localizable.\ \Qed}

\medskip

{\bf (e)} $(X,\Sigma,\mu)$ is not locally determined.   \prooflet{\Prf\
Consider $H=\{0\}\times I$.   Then $H\notin\Sigma$ because
$H[\{0\}]\symmdiff H[\{1\}]=I$ is uncountable.   But let $E\in\Sigma$ be any
set such that $\mu E<\infty$.   Then

\Centerline{$(E\cap H)[\{0\}]\symmdiff (E\cap H)[\{1\}]
=(E\cap H)[\{0\}]\subseteq E[\{0\}]$}

\noindent is finite, so $E\cap H\in\Sigma$.   As $E$ is arbitrary, $H$
witnesses that $\mu$ is not locally determined.\ \Qed}
\medskip

{\bf (f)} $(X,\Sigma,\mu)$ is
purely atomic.   \prooflet{\Prf\ Let $E\in\Sigma$ be any set of non-zero
measure.   Let $i\in I$ be such that $(0,i)\in E$.   Then $(0,i)\in E$ and
$F=\{(0,i)\}$ is a set of measure $1$, included in $E$;  because $F$
is a singleton set, it must be an atom for $\mu$;  as $E$ is arbitrary,
$\mu$ is purely atomic.\ \Qed}

\cmmnt{\medskip

{\bf (g)} Thus the construction here yields a complete, localizable, purely
atomic, non-locally-determined space.
}%end of comment

\leader{*216D}{A complete, locally determined space which is not
localizable}\cmmnt{ The next construction requires a little set theory.}
We need two sets $I$, $J$ such that $I$ is uncountable\cmmnt{ (more
strictly, $I$
cannot be expressed as the union of countably many countable sets)},
$I\subseteq J$ and $J$ cannot be expressed as $\bigcup_{i\in I}K_i$
where every $K_i$ is countable.   \cmmnt{The most natural way of doing
this, subject to the axiom of choice, is to take $I=\omega_1$, the first
uncountable ordinal, and $J$ to be $\omega_2$, the first ordinal from
which there is no injection into $\omega_1$ (see 2A1Fc);  but in case
you prefer other formulations (e.g., $I=\{\{x\}:x\in\Bbb R\}$ and
$J=\Cal P\Bbb R$), I will write the following argument in
terms of $I$ and $J$, and you can pick your own pair.}%end of comment

\medskip

{\bf (a)} Let $\Tau$ be the countable-cocountable $\sigma$-algebra
of $J$ and $\nu$ the countable-cocountable measure on
$J$\cmmnt{ (211R)}.   Set $X=J\times J$ and for $E\subseteq X$ set

\Centerline{$E[\{\xi\}]=\{\eta:(\xi,\eta)\in E\}$,\quad
$E^{-1}[\{\xi\}]=\{\eta:(\eta,\xi)\in E\}$}

\noindent for every $\xi\in J$.   Set

\Centerline{$\Sigma=\{E:E[\{\xi\}]$ and $E^{-1}[\{\xi\}]$ belong to
$\Tau$ for every $\xi\in J\}$,}

\Centerline{$\mu E=\sum_{\xi\in J}\nu E[\{\xi\}]
+\sum_{\xi\in J}\nu E^{-1}[\{\xi\}]$}

\noindent for every $E\in\Sigma$.   \cmmnt{It is easy to check that}
$\Sigma$ is a $\sigma$-algebra and\cmmnt{ that} $\mu$ is a measure.

\medskip

{\bf (b)} $(X,\Sigma,\mu)$ is complete.   \prooflet{\Prf\ If $A\subseteq
E\in\Sigma$ and $\mu E=0$, then all the sets
$E[\{\xi\}]$ and $E^{-1}[\{\xi\}]$ are countable, so the same is true of
all the sets $A[\{\xi\}]$ and $A^{-1}[\{\xi\}]$, and $A\in\Sigma$.\ \Qed}

\medskip

{\bf (d)} $(X,\Sigma,\mu)$ is semi-finite.   \prooflet{\Prf\ For each
$\zeta\in J$, set

\Centerline{$G_{\zeta}=\{\zeta\}\times J$,\quad
$\tilde G_{\zeta}=J\times\{\zeta\}$.}

\noindent Then all the sections $G_{\zeta}[\{\xi\}]$,
$G_{\zeta}^{-1}[\{\xi\}]$, $\tilde G_{\zeta}[\{\xi\}]$ and
$\tilde  G_{\zeta}^{-1}[\{\xi\}]$ are either $J$ or $\emptyset$ or
$\{\zeta\}$, so belong to $\Tau$, and all the $G_{\zeta}$,
$\tilde G_{\zeta}$ belong to $\Sigma$, with $\mu$-measure $1$.

Suppose that $E\in\Sigma$ is a set of strictly positive measure.   Then
there must be some $\xi\in J$ such that

\Centerline{$0<\nu E[\{\xi\}]+\nu E^{-1}[\{\xi\}]
=\mu(E\cap G_{\xi})+\mu(E\cap \tilde G_{\xi})<\infty$,}

\noindent and one of the sets $E\cap G_{\xi}$, $E\cap\tilde G_{\xi}$ is
a set of non-zero finite measure included in $E$.\ \Qed}

\medskip

{\bf (e)} $(X,\Sigma,\mu)$ is locally determined.   \prooflet{\Prf\ Suppose
that $H\subseteq X$ is such that $H\cap E\in\Sigma$ whenever
$E\in\Sigma$ and $\mu E<\infty$.   Then, in particular, $H\cap
G_{\zeta}$ and $H\cap \tilde G_{\zeta}$ belong to $\Sigma$, so

\Centerline{$H[\{\zeta\}]
=(H\cap \tilde G_{\zeta})[\{\zeta\}]\in\Tau$,}

\Centerline{$H^{-1}[\{\zeta\}]
=(H\cap G_{\zeta})^{-1}[\{\zeta\}]\in\Tau$,}

\noindent for every $\zeta\in J$.   This shows that $H\in\Sigma$.   As
$H$ is arbitrary, $\mu$ is locally determined.\ \Qed}

\medskip

{\bf (f)} $(X,\Sigma,\mu)$ is not localizable.
\prooflet{\Prf\ Set $\Cal E=\{G_{\zeta}:\zeta\in J\}$.
\Quer\ Suppose, if possible, that
$G\in\Sigma$ is an essential supremum for $\Cal E$.   Then

\Centerline{$\nu(J\setminus G[\{\xi\}])
=\mu(G_{\xi}\setminus G)=0$}

\noindent and $J\setminus G[\{\xi\}]$ is countable, for every $\xi\in
J$.   Consequently $J\ne\bigcup_{\xi\in I}(J\setminus G[\{\xi\}])$, and
there is an $\eta$ belonging to $J\setminus\bigcup_{\xi\in I}(J\setminus
G[\{\xi\}])=\bigcap_{\xi\in I}G[\{\xi\}]$.
This means just that $(\xi,\eta)\in G$ for every $\xi\in I$, that is,
that $I\subseteq G^{-1}[\{\eta\}]$.   Accordingly
$G^{-1}[\{\eta\}]$ is uncountable, so that
$\nu G^{-1}[\{\eta\}]=\mu(G\cap\tilde G_{\eta})=1$.   But observe that
$\mu(G_{\xi}\cap \tilde G_{\eta})=\mu\{(\xi,\eta)\}=0$ for every $\xi\in
J$.   This means that, setting $H=X\setminus \tilde G_{\eta}$,
$E\setminus H$ is negligible, for every $E\in\Cal E$;  so that we must
have $0=\mu(G\setminus H)=\mu(G\cap\tilde G_{\eta})=1$, which is
absurd.\ \Bang

Thus $\Cal E$ has no essential supremum in $\Sigma$, and $\mu$ cannot be
localizable.\ \Qed}
\medskip

{\bf (g)} $(X,\Sigma,\mu)$ is purely atomic.   \prooflet{\Prf\ If
$E\in\Sigma$ has
non-zero measure, there must be some $\xi\in J$ such that one of
$E[\{\xi\}]$, $E^{-1}[\{\xi\}]$ is not countable;  that is, such that
one of $E\cap G_{\xi}$, $E\cap \tilde G_{\xi}$ is not negligible.   But
if now $H\in\Sigma$ and $H\subseteq E\cap G_{\xi}$,
either $H[\{\xi\}]$ is countable, and
$\mu H=0$, or $J\setminus H[\{\xi\}]$ is countable, and
$\mu(G_{\xi}\setminus H)=0$;  similarly, if
$H\subseteq E\cap\tilde G_{\xi}$,
one of $\mu H$, $\mu (\tilde G_{\xi}\setminus H)$ must be $0$,
according to whether $H^{-1}[\{\xi\}]$ is countable or not.   Thus
$E\cap G_{\xi}$ and $E\cap\tilde G_{\xi}$, if not negligible, must be
atoms, and $E$ must include an atom.   As $E$ is arbitrary, $\mu$ is purely
atomic.\ \Qed}

\cmmnt{\medskip

{\bf (h)} Thus $(X,\Sigma,\mu)$ is complete, locally determined and
purely atomic, but is not localizable.
}%end of comment

\ifdim\pagewidth>467pt\fontdimen3\tenbf=3pt\fontdimen4\tenbf=2pt\fi
\leader{*216E}{}{\bf A complete, locally determined, localizable space
which is not strictly localizable}\cmmnt{ For the last, and most
interesting, construction, we need a non-trivial result in infinitary
combinatorics, which I have written out in 2A1P:  if $I$ is any set, and
$\langle f_{\alpha}\rangle_{\alpha\in A}$ is a family in $\{0,1\}^I$, the set of
functions from $I$ to $\{0,1\}$, with $\#(A)$ strictly greater than
$\frak c$, the cardinal of the continuum, and if
$\langle K_{\alpha}\rangle_{\alpha\in A}$ is any family of countable subsets of
$I$, then there must be distinct $\alpha$, $\beta\in A$ such that
$f_{\alpha}$ and $f_{\beta}$ agree on $K_{\alpha}\cap K_{\beta}$.
\fontdimen3\tenbf=1.92pt
\fontdimen4\tenbf=1.28pt

Armed with this fact, I proceed as follows.

\medskip

}{\bf (a)} Let $C$ be any set with cardinal greater than $\frak c$.   Set
$I=\Cal PC$ and $X=\{0,1\}^I$.   For
$\gamma\in C$, define $x_{\gamma}\in X$ by saying that
$x_{\gamma}(\Gamma)=1$ if $\gamma\in\Gamma\subseteq C$ and
$x_{\gamma}(\Gamma)=0$ if $\gamma\notin\Gamma\subseteq C$.   Let $\Cal
K$ be the family of countable subsets of $I$, and for $K\in\Cal K$,
$\gamma\in C$ set

\Centerline{$F_{\gamma K}=\{x:x\in X,\,x\restr K=x_{\gamma}\restr
K\}\subseteq X$.}

\noindent Let

$$\eqalign{\Sigma_{\gamma}=\{E:E\subseteq X,
&\text{ either there is a }K\in\Cal K\text{ such that }F_{\gamma
K}\subseteq E\cr
&\text{ or there is a }K\in\Cal K\text{ such that }F_{\gamma K}\subseteq
X\setminus E\}.\cr}$$

\noindent Then $\Sigma_{\gamma}$ is a $\sigma$-algebra of subsets of
$X$.   \prooflet{\Prf\ (i) $F_{\gamma\emptyset}\subseteq
X\setminus\emptyset$ so
$\emptyset\in\Sigma_{\gamma}$.   (ii) The definition of
$\Sigma_{\gamma}$ is symmetric between $E$ and $X\setminus E$, so
$X\setminus E\in\Sigma_{\gamma}$ whenever $E\in\Sigma_{\gamma}$.   (iii)
Let $\sequencen{E_n}$ be a sequence in $\Sigma_{\gamma}$, with union
$E$.   ($\alpha$) If there are $n\in\Bbb N$, $K\in\Cal K$ such that
$F_{\gamma K}\subseteq E_n$, then $F_{\gamma K}\subseteq E$, so
$E\in\Sigma_{\gamma}$.   ($\beta$) Otherwise, there is for each
$n\in\Bbb N$ a $K_n\in\Cal K$ such that $F_{\gamma,K_n}\subseteq
X\setminus E_n$.   Set $K=\bigcup_{n\in\Bbb N}K_n\in\Cal K$.   Then

$$\eqalign{F_{\gamma K}
&=\{x:x\restr K=x_{\gamma}\restr K\}
=\{x:x\restr K_n=x_{\gamma}\restr K_n\text{ for every }n\in\Bbb N\}\cr
&=\bigcap_{n\in\Bbb N}F_{\gamma,K_n}
\subseteq\bigcap_{n\in\Bbb N}X\setminus E_n
=X\setminus E,\cr}$$

\noindent so again $E\in\Sigma_{\gamma}$.
As $\sequencen{E_n}$ is arbitrary, $\Sigma_{\gamma}$ is a
$\sigma$-algebra.\ \Qed}

\medskip

{\bf (b)} Set

\Centerline{$\Sigma=\bigcap_{\gamma\in C}\Sigma_{\gamma}$;}

\noindent then $\Sigma$\cmmnt{, being an intersection of
$\sigma$-algebras,} is
a $\sigma$-algebra of subsets of $X$\cmmnt{ (see 111Ga)}.
Define $\mu:\Sigma\to[0,\infty]$ by setting

$$\eqalign{\mu E&=\#(\{\gamma:x_{\gamma}\in E\})
\text{ if this is finite},\cr
&=\infty\text{ otherwise};\cr}$$

\noindent then $\mu$ is a measure.

\medskip

{\bf (c)} It will be convenient later to know something about the sets

\Centerline{$G_D=\{x:x\in X,\,x(D)=1\}$}

\noindent for $D\subseteq C$.   In particular, every $G_D$ belongs to
$\Sigma$.   \prooflet{\Prf\  If $\gamma\in D$, then $x_{\gamma}(D)=1$ so
$G_D=F_{\gamma,\{D\}}\in\Sigma_{\gamma}$.   If $\gamma\in C\setminus D$,
then $x_{\gamma}(D)=0$ so $G_D=X\setminus
F_{\gamma,\{D\}}\in\Sigma_{\gamma}$.\ \QeD}
Also\cmmnt{, of course,} $\{\gamma:x_{\gamma}\in G_D\}=D$.

\medskip

{\bf (d)} $(X,\Sigma,\mu)$ is complete.   \prooflet{\Prf\ Suppose that
$A\subseteq
E\subseteq\Sigma$ and that $\mu E=0$.   For every $\gamma\in C$,
$E\in\Sigma_{\gamma}$ and $x_{\gamma}\notin E$, so $F_{\gamma
K}\not\subseteq E$ for any $K\in\Cal K$ and there is a $K\in\Cal K$ such
that
\Centerline{$F_{\gamma K}\subseteq X\setminus E\subseteq X\setminus A$.}

\noindent Thus $A\in\Sigma_{\gamma}$;  as $\gamma$ is arbitrary,
$A\in\Sigma$.   As $A$ is arbitrary, $\mu$ is complete.\ \Qed}

\medskip

{\bf (e)} $(X,\Sigma,\mu)$ is semi-finite.   \prooflet{\Prf\ Let
$E\in\Sigma$ be a
set of positive measure.   Then there must be some $\gamma\in C$ such
that $x_{\gamma}\in E$.   Consider $E'=E\cap G_{\{\gamma\}}$.   As
$x_{\gamma}\in E'$, $\mu E'\ge 1>0$.   On the other hand, $\mu
G_{\{\gamma\}}=\#(\{\delta:\delta\in\{\gamma\}\})=1$, so $\mu E'=1$.
As $E$ is arbitrary, $\mu$ is
semi-finite.\ \Qed}

\medskip

{\bf (f)} $(X,\Sigma,\mu)$ is localizable.   \prooflet{\Prf\ Let $\Cal E$ be
any subset of $\Sigma$.   Set $D=\{\delta:\delta\in
C,\,x_{\delta}\in\bigcup\Cal E\}$.   Consider $G_D$.
For $H\in\Sigma$,

$$\eqalign{\mu(E\setminus H)&=0\text{ for every }E\in\Cal E\cr
&\iff x_{\gamma}\notin E\setminus H\text{ for every }E\in\Cal E,
\gamma\in C\cr
&\iff x_{\gamma}\in H\text{ for every }\gamma\in D\cr
&\iff x_{\gamma}\notin G_D\setminus H\text{ for every }\gamma\in C\cr
&\iff \mu(G_D\setminus H)=0.\cr}$$

\noindent Thus $G_D$ is an essential supremum for $\Cal E$ in
$\Sigma$.   As $\Cal E$ is arbitrary, $\mu$ is localizable.\ \Qed}

\medskip

{\bf (g)} $(X,\Sigma,\mu)$ is not strictly localizable.
\prooflet{\Prf\Quer\
Suppose, if possible, that $\langle X_j\rangle_{j\in J}$ is a
decomposition of $(X,\Sigma,\mu)$.   Set $J'=\{j:j\in J,\,\mu X_j>0\}$.
For each $j\in J'$, the set $C_j=\{\gamma:x_{\gamma}\in X_j\}$ must be
finite and non-empty.   Moreover, for each $\gamma\in C$, there must be
some $j\in J$ such that $\mu(G_{\{\gamma\}}\cap X_j)>0$, and in this
case $j\in J'$ and $\gamma\in C_j$.   Thus $C=\bigcup_{j\in J'}C_j$.
Because $\#(C)>\frak c$, $\#(J')>\frak c$ (2A1Ld).

For each $j\in J'$, choose $\gamma_j\in C_j$.   Then

\Centerline{$x_{\gamma_j}\in X_j\in\Sigma\subseteq\Sigma_{\gamma_j}$,}

\noindent so there must be a $K_j\in\Cal K$ such that
$F_{\gamma_j,K_j}\subseteq X_j$.

At this point I finally turn to the result cited at the start of this
example.   Because $\#(J')>\frak c$, there must be distinct $j$, $k\in
J'$ such that $x_{\gamma_j}$ and $x_{\gamma_k}$ agree on $K_j\cap K_k$.
We may therefore define $x\in X$ by saying that

$$\eqalign{x(\delta)&=x_{\gamma_j}(\delta)\text{ if }\delta\in K_j,\cr
&=x_{\gamma_k}(\delta)\text{ if }\delta \in K_k,\cr
&=0\text{ if }\delta\in C\setminus(K_j\cup K_j).\cr}$$

\noindent Now

\Centerline{$x\in F_{\gamma_j,K_j}\cap F_{\gamma_k,K_k}
\subseteq X_j\cap X_k$,}

\noindent and $X_j\cap X_k\ne\emptyset$;  contradicting the assumption
that the $X_j$ formed a decomposition of $X$.\ \Bang\Qed}

\medskip

{\bf (h)} $(X,\Sigma,\mu)$ is purely atomic.   \prooflet{\Prf\ If
$E\in\Sigma$ and
$\mu E>0$, then (as remarked in (e) above) there is a $\gamma\in C$ such
that $\mu(E\cap G_{\{\gamma\}})=1$;  now $E\cap G_{\{\gamma\}}$ must be
an atom.\ \Qed}

\cmmnt{\medskip

{\bf (i)} Accordingly $(X,\Sigma,\mu)$ is a complete, locally
determined, localizable, purely atomic measure space which is not
strictly localizable.
}%end of comment

\exercises{
\leader{216X}{Basic exercises (a)}
%\spheader 216Xa
In the construction of 216C, show that the
subspace measure on $\{1\}\times I$ is not semi-finite.
%216C

\spheader 216Xb Suppose, in 216D, that $I=\omega_1$.   (i) Show that the set $\{(\xi,\eta):\xi\le\eta<\omega_1\}$ is
measured by the measure
constructed by \Caratheodory's method from
$\mu^*\restrp\Cal P(I\times I)$,
but not by the subspace measure on $I\times I$.   (ii) Hence, or otherwise,
show that the subspace measure on $I\times I$ is not locally determined.
%216D

\spheader 216Xc In 216Ya, 252Yq and 252Ys below, I indicate how to
construct atomless versions of 216C, 216D and 216E, that is, atomless
complete measure
spaces of which the first is localizable but not locally determined, the
second is locally determined spaces but not localizable,
and the third is locally determined and localizable
but not strictly localizable.   Show how direct sums of these,
together with
counting measure and the examples described in this chapter, can be
assembled to provide all 56 examples called for by the discussion in
the introduction to this section.
%216E

\leader{216Y}{Further exercises (a)}
%\spheader 216Ya
Let $\lambda$ be Lebesgue measure on $[0,1]$, and $\Lambda$ its domain.
Set $Y=[0,1]\times\{0,1\}$ and write

\Centerline{$\Tau=\{F:F\subseteq Y,\,F^{-1}[\{0\}]\in\Lambda\}$,}

\Centerline{$\nu F=\lambda F^{-1}[\{0\}]$ for every $F\in\Tau$.}

\noindent Set

\Centerline{$\Tau_0=\{F:F\in\Tau,\,F^{-1}[\{0\}]\symmdiff
F^{-1}[\{1\}]$ is $\lambda$-negligible$\}$.}

\noindent Let $I$ be an uncountable set.   Set $X=Y\times I$,

\Centerline{$\Sigma=\{E:E\subseteq X,\,E^{-1}[\{i\}]\in\Tau$ for every $i\in
I$, $\{i:E^{-1}[\{i\}]\notin \Tau_0\}$ is countable$\}$,}

\Centerline{$\mu E=\sum_{i\in I}\nu E^{-1}[\{i\}]$ for $E\in\Sigma$.}

\noindent (i) Show that $(Y,\Tau,\nu)$ and $(Y,\Tau_0,\nu\restrp\Tau_0)$ are
complete probability spaces, and that for every $F\in\Tau$ there is an
$F'\in\Tau_0$ such
that $\nu(F\symmdiff F')=0$.   (ii) Show that $(X,\Sigma,\mu)$ is an
atomless complete localizable measure space which is not locally determined.
%216C

\spheader 216Yb Define a measure $\mu$ on $X=\omega_2\times\omega_2$
as follows.   Take $\Sigma$ to be the $\sigma$-algebra of subsets of $X$
generated by

\Centerline{$\{A\times\omega_2:A\subseteq\omega_2\}
\cup\{\omega_2\times\alpha:\alpha<\omega_2\}$.}

\noindent For $E\in\Sigma$ set

\Centerline{$W(E)=\{\xi:\xi<\omega_2$, $\sup E[\{\xi\}]=\omega_2\}$,}

\noindent and set $\mu E=\#(W(E))$ if this is finite, $\infty$ otherwise.
Show that $\mu$ is a measure on $X$, is localizable and locally determined,
but does not have locally determined negligible sets.   Find a subspace $Y$
of $X$ such that the subspace measure on $Y$ is not semi-finite.
%216D

\spheader 216Yc Show that in the space described in 216E every set has
a measurable envelope, but that this is not true in the spaces of 216C and 216D.
%216E

\spheader 216Yd Set $X=\omega_1\times\omega_2$.   For $E\subseteq X$ set

\Centerline{$A(E)=\{\zeta:$ for some $\xi$, just one of
$(\xi,\zeta)$, $(\xi,\zeta+1)$ belongs to $E\}$,}

\ifdim\pagewidth>400pt
\Centerline{$B(E)=\{\zeta:$ there are $\xi$, $\zeta'$ such that
$\zeta<\zeta'<\omega_2$ and just one of
$(\xi,\zeta)$, $(\xi,\zeta')$ belongs to $E\}$,}
\else
$$\eqalign{B(E)
&=\{\zeta:\text{ there are }\xi,\,\zeta'\text{ such that }
\zeta<\zeta'<\omega_2\cr&
\mskip100mu\text{ and just one of }
(\xi,\zeta),\,(\xi,\zeta')\text{ belongs to }E\},
\cr}$$
\fi

\Centerline{$W(E)=\{\xi:\#(E[\{\xi\}])=\omega_2\}$.}

\noindent Let $\Sigma$ be the set of subsets $E$ of $X$ such that $A(E)$ is
countable and $\#(B(E))\le\omega_1$.   For $E\in\Sigma$, set
$\mu E=\#(W(E))$ if this is finite, $\infty$ otherwise.   (i) Show that
$(X,\Sigma,\mu)$ is a measure space.   (ii) Show that if $\hat\mu$ is the
completion of $\mu$, then its domain is the set of subsets $E$ of $X$ such
that $A(E)$ is countable, and $\hat\mu$ is strictly localizable.   (iii)
Show that $\mu$ is not strictly localizable.
%216Yb 216D

\spheader 216Ye\dvAnew{2015} Show that there is a
complete atomless semi-finite measure space
with a singleton subset which is not negligible.   
\Hint{set $X=(\omega_1\times[0,1])\cup\{\omega_1\}$ and let
$\Sigma$ be the $\sigma$-algebra of subsets of $X$ generated by
$\{\{\xi\}\times E:\xi<\omega_1$, 
$E\subseteq[0,1]$ is Lebesgue measurable$\}$}.
%215E

}%end of exercises

\endnotes{
\Notesheader{216} The examples 216C-216E %216C 216D 216E
are designed to
form, with Lebesgue measure, a basis for constructing a complete set of
examples for the concepts listed in
211A-211K.   %211A 211B 211C 211D 211E 211F 211G 211H 211I 211J 211K
One does not really
expect to encounter these phenomena in applications, but a clear
understanding of the possibilities demonstrated by these examples is
part of a proper appreciation of their rarity.   Of course, if we add
further properties to our list -- for instance, the property of having
locally determined negligible sets (213I), or the property that every
subset should have a measurable envelope
(213Xl) -- then there are further positive results to complement 211L, and more examples to hunt for, like 216Yb.
But it is
time, perhaps past time, that we returned to the classical theorems
which apply to the measure spaces at the centre of the subject.
}%end of comment

\discrpage

