\frfilename{mt431.tex}
\versiondate{9.4.05}
\copyrightdate{2000}

\def\chaptername{Topological measure spaces II}
\def\sectionname{Souslin's operation}

\newsection{431}

I begin the chapter with a short section on Souslin's
operation\cmmnt{ (\S421)}.   The basic facts we need to know are that
(in a complete locally determined measure space) the family of
measurable sets is closed under Souslin's operation (431A), and that the
kernel of a Souslin scheme can be approximated from within in measure
(431D).

\leader{431A}{Theorem} Let $(X,\Sigma,\mu)$ be a complete locally
determined measure space.   Then $\Sigma$ is closed under Souslin's
operation.

\proof{ (I follow the notation of 421A-421B.)
Let $\family{\sigma}{S^*}{E_{\sigma}}$ be a Souslin scheme
in $\Sigma$ with kernel $A$.   Write
$S=\bigcup_{k\in\BbbN}\BbbN^k$.
If $F\in\Sigma$ and $\mu F<\infty$, then $A\cap F\in\Sigma$.
\Prf\ For each $\sigma\in S$, set

\Centerline{$A_{\sigma}
=\bigcup_{\phi\in\BbbN^{\Bbb N},\phi\supseteq\sigma}
  \bigcap_{n\ge 1}E_{\phi\restr n}$,}

\noindent and let $G_{\sigma}$ be a measurable envelope of
$A_{\sigma}\cap F$.   Because $A_{\sigma}\subseteq E_{\sigma}$ (writing
$E_{\emptyset}=X$), we may suppose that $G_{\sigma}\subseteq
E_{\sigma}\cap F$.   Now, for any $\sigma\in S$,

\Centerline{$A_{\sigma}\cap F
=\bigcup_{i\in\BbbN}A_{\sigma^{\smallfrown}\fraction{i}}\cap F
\subseteq\bigcup_{i\in\BbbN}G_{\sigma^{\smallfrown}\fraction{i}}$,}

\noindent so

\Centerline{$H_{\sigma}
=G_{\sigma}\setminus\bigcup_{i\in\Bbb N}
   G_{\sigma^{\smallfrown}\fraction{i}}$}

\noindent is negligible.

Set $H=\bigcup_{\sigma\in S}H_{\sigma}$, so that $H$ is negligible.
Take any $x\in G_{\emptyset}\setminus H$.   Choose
$\sequence{i}{\phi(i)}$
inductively, as follows.   Given that
$\sigma=\langle\phi(i)\rangle_{i<k}$
has been chosen and $x\in G_{\sigma}$, then $x\notin H_{\sigma}$, so
there must be some $j\in\BbbN$ such that $x\in G_{\sigma^{\smallfrown}\fraction{j}}$;
set $\phi(k)=j$, and continue.   Now

\Centerline{$x\in\bigcap_{k\ge 1}G_{\phi\restr k}
\subseteq\bigcap_{k\ge 1}E_{\phi\restr k}
\subseteq A$.}

Thus we see that $G_{\emptyset}\setminus H\subseteq A$;  as
$G_{\emptyset}\subseteq F$, $G_{\emptyset}\setminus H\subseteq A\cap F$.
On the other hand, $A\cap F\subseteq G_{\emptyset}$.   Because $H$ is
negligible and $\mu$ is complete,
$A\cap F\in\Sigma$.\ \Qed

Because $\mu$ is locally determined, it follows that
$A\in\Sigma$.   As $\family{\sigma}{S^*}{E_{\sigma}}$ is arbitrary,
$\Sigma$ is closed under Souslin's operation.
}%end of proof of 431A

\leader{431B}{Corollary} If $(X,\frak T,\Sigma,\mu)$ is a complete
locally determined topological measure space, every Souslin-F set in
$X$\cmmnt{ (definition:  421K)} is measurable.

\leader{431C}{Corollary} Let $X$ be a set and $\theta$ an outer measure
on $X$.   Let $\mu$ be the measure defined by \Caratheodory's method,
and $\Sigma$ its domain.   Then $\Sigma$ is closed under Souslin's
operation.

\proof{ Let $\family{\sigma}{S^*}{E_{\sigma}}$ be a Souslin scheme in
$\Sigma$ with kernel $A$.   Take any $C\subseteq X$ such that
$\theta C<\infty$.   Then $\theta_C=\theta\restrp\Cal PC$ is an outer
measure on $C$;  let $\mu_C$ be the measure on $C$ defined from
$\theta_C$ by \Caratheodory's method, and $\Sigma_C$ its domain.   If
$\sigma\in S^*$ and $D\subseteq C$ then

$$\eqalign{\theta_C(D\cap C\cap E_{\sigma})
  +\theta_C(D\setminus(C\cap E_{\sigma}))
&=\theta(D\cap E_{\sigma})+\theta(D\setminus E_{\sigma})\cr
&=\theta D=\theta_CD;\cr}$$

\noindent as $D$ is arbitrary, $C\cap E_{\sigma}\in\Sigma_C$.   $\mu_C$
is a complete totally finite measure, so 431A tells us that the kernel
of the Souslin scheme $\family{\sigma}{S^*}{C\cap E_{\sigma}}$ belongs to
$\Sigma_C$.   But this is just $C\cap A$ (applying 421Cb to the identity
map from $C$ to $X$).   So

\Centerline{$\theta(C\cap A)+\theta(C\setminus A)
=\theta_C(C\cap A)+\theta_C(C\setminus A)
=\theta_CC=\theta C$.}

\noindent As $C$ is arbitrary, $A\in\Sigma$ (113D).   As
$\family{\sigma}{S^*}{E_{\sigma}}$ is arbitrary, we have the result.
}%end of proof of 431C

\leader{431D}{Theorem} Let $(X,\Sigma,\mu)$ be a complete locally
determined measure space, and $\family{\sigma}{S^*}{E_{\sigma}}$ a Souslin
scheme in $\Sigma$ with kernel $A$.   Then

$$\eqalign{\mu A
&=\sup\{\mu(\bigcup_{\phi\in K}\bigcap_{n\ge 1}E_{\phi\restr n}):
  K\subseteq\BbbN^{\BbbN}\text{ is compact}\}\cr
&=\sup\{\mu(\bigcup_{\phi\le\psi}\bigcap_{n\ge 1}E_{\phi\restr n}):
  \psi\in\BbbN^{\BbbN}\},\cr}$$

\noindent writing $\phi\le\psi$ if $\phi(i)\le\psi(i)$ for every
$i\in\Bbb N$.

\proof{{\bf (a)} By 431A, $A$ is measurable.   For
$K\subseteq\BbbN^{\Bbb N}$, set
$H_K=\bigcup_{\phi\in K}\bigcap_{n\ge 1}E_{\phi\restr n}$.   Of course
$H_K\subseteq A$, and we know from 421M (or otherwise) that
$H_K\in\Sigma$ if $K$ is compact.   So surely $\mu A\ge\mu H_K$ for
every compact $K\subseteq \BbbN^{\BbbN}$.
If $\psi\in\BbbN^{\BbbN}$, then
$\{\phi:\phi\le\psi\}=\prod_{i\in\Bbb N}(\psi(i)+1)$ is compact.   We
therefore have

$$\eqalignno{\mu A
&\ge\sup\{\mu(\bigcup_{\phi\in K}\bigcap_{n\ge 1}E_{\phi\restr
n}):K\subseteq\BbbN^{\BbbN}\text{ is compact}\}\cr
&\ge\sup\{\mu(\bigcup_{\phi\le\psi}\bigcap_{n\ge 1}E_{\phi\restr
n}):\psi\in\BbbN^{\BbbN}\}.\cr}$$

\noindent So what I need to prove is that

\Centerline{$\mu A\le\sup\{\mu(\bigcup_{\phi\le\psi}\bigcap_{n\ge
1}E_{\phi\restr n}):\psi\in\BbbN^{\BbbN}\}$.}

\medskip

{\bf (b)} Fix on a set $F\in\Sigma$ of finite measure.   For $\sigma\in
S=\bigcup_{k\in\BbbN}\BbbN^k$ set

\Centerline{$A_{\sigma}=\bigcup_{\phi\in\BbbN^{\Bbb
N},\phi\supseteq\sigma}\bigcap_{n\ge 1}E_{\phi\restr n}$.}

\noindent We need to know that $A_{\sigma}$ belongs to $\Sigma$;  this
follows from 431A, because writing $E'_{\tau}=E_{\tau}$ if
$\tau\subseteq\sigma$ or $\sigma\subseteq\tau$, $\emptyset$ otherwise,

\Centerline{$A_{\sigma}
=\bigcup_{\phi\in\BbbN^{\BbbN}}\bigcap_{n\ge 1}E'_{\phi\restr n}
\in\Cal S(\Sigma)=\Sigma$,}

\noindent writing $\Cal S$ for Souslin's operation, as in \S421.

Let $\epsilon>0$, and take a family
$\family{\sigma}{S}{\epsilon_{\sigma}}$ of strictly positive real
numbers such that $\sum_{\sigma\in S}\epsilon_{\sigma}\le\epsilon$.
For each $\sigma\in S$ we have $A_{\sigma}=\bigcup_{i\in\Bbb
N}A_{\sigma^{\smallfrown}\fraction{i}}$, so there is an $m_{\sigma}\in\BbbN$ such
that

\Centerline{$\mu(F\cap A_{\sigma}\setminus\bigcup_{i\le
m_{\sigma}}A_{\sigma^{\smallfrown}\fraction{i}})\le\epsilon_{\sigma}$.}

\noindent Define $\psi\in\BbbN^{\BbbN}$ by saying that

\Centerline{$\psi(k)=\max\{m_{\sigma}:\sigma\in\Bbb
N^k,\,\sigma(i)\le\psi(i)$ for every $i<k\}$}

\noindent for each $k\in\BbbN$.  Set

\Centerline{$H=\bigcup_{\phi\in\BbbN^{\BbbN},\phi\le\psi}\bigcap_{n\ge
1}E_{\phi\restr n}$.}

\medskip

{\bf (c)} Set

\Centerline{$G=\bigcup_{\sigma\in S}F\cap
A_{\sigma}\setminus\bigcup_{i\le m_{\sigma}}A_{\sigma^{\smallfrown}\fraction{i}}$,}

\noindent so that $\mu G\le\epsilon$, by the choice of the
$\epsilon_{\sigma}$ and the $m_{\sigma}$.   Then
$F\cap A\setminus G\subseteq H$.
\Prf\ If $x\in F\cap A\setminus G$, choose
$\sequence{i}{\phi(i)}$ inductively, as follows.   Given that
$\phi(i)\le\psi(i)$ for $i<k$ and $x\in A_{\sigma}$, where
$\sigma=\langle\phi(i)\rangle_{i<k}$, then $x\notin
A_{\sigma}\setminus\bigcup_{j\le m_{\sigma}}A_{\sigma^{\smallfrown}\fraction{j}}$,
so there must be some $j\le m_{\sigma}$ such that
$x\in A_{\sigma^{\smallfrown}\fraction{j}}$;  set $\phi(k)=j$;  because
$\sigma\in\prod_{i<k}(\psi(i)+1)$, $j\le m_{\sigma}\le\psi(k)$, and the
induction continues.   At the end of the induction, $\phi\le\psi$ and

\Centerline{$x\in\bigcap_{n\ge 1}A_{\phi\restr n}\subseteq\bigcap_{n\ge
1}E_{\phi\restr n}\subseteq H$.   \Qed}

\medskip

{\bf (d)} It follows that

\Centerline{$\mu(A\cap F)\le\mu G+\mu H\le\epsilon+\mu H$.}

\noindent As $F$ and $\epsilon$ are arbitrary, and $\mu$ is semi-finite,

\Centerline{$\mu A\le\sup\{\mu(\bigcup_{\phi\le\psi}\bigcap_{n\ge
1}E_{\phi\restr n}):\psi\in\BbbN^{\BbbN}\}$,}

\noindent and the proof is complete.
}%end of proof of 431D

\leader{431E}{Corollary} If $(X,\frak T,\Sigma,\mu)$ is a topological
measure space and $E\subseteq X$ is a Souslin-F set with finite outer
measure, then $\mu^*E=\sup\{\mu F:F\subseteq E$ is closed$\}$.

\proof{ Let $\tilde\mu$ be the c.l.d.\ version of
$\mu$\cmmnt{ (213E)}.   Let $\family{\sigma}{S^*}{E_{\sigma}}$ be a
Souslin scheme of closed sets with kernel $E$.   Then 213Fb and 431D
tell us that

\Centerline{$\mu^*E=\tilde\mu E
=\sup_{K\subseteq\NN\text{ is compact}}\mu F_K$,}

\ifdim\pagewidth>467pt\fontdimen3\tenrm=2.5pt\fi
\ifdim\pagewidth>467pt\fontdimen4\tenrm=1.67pt\fi
\noindent where
$F_K=\bigcup_{\phi\in K}\bigcap_{n\ge 1}E_{\phi\restr n}$ for
$K\subseteq\NN$.   But every $F_K$ is closed, by 421M.   So
$\mu^*E\penalty-100\le\penalty-100
\sup_{F\subseteq E\text{ is closed}}\mu F$;  as the reverse
inequality is trivial, we have the result.
\fontdimen3\tenrm=1.67pt
\fontdimen4\tenrm=1.11pt
}%end of proof of 431E

\leader{*431F}{}\cmmnt{ Two further versions of the ideas in
431A will be useful.   The first is topological.

\medskip

\noindent}{\bf Theorem} Let $X$ be any topological space, and
$\widehat{\Cal B}$ its Baire-property algebra.

(a) For any $A\subseteq X$, there is a Baire-property envelope of $A$,
that is, a set $E\in\widehat{\Cal B}$ such that
$A\subseteq E$ and $E\setminus F$ is meager whenever
$A\subseteq F\in\widehat{\Cal B}$.

(b) $\widehat{\Cal B}$ is closed under Souslin's operation.

\proof{{\bf (a)} By 4A3Ra, there is an open set $H\subseteq X$ such that
$A\setminus H$ is meager and $H\cap G$ is empty whenever $G\subseteq X$
is open and $A\cap G$ is meager.   Set $E=A\cup H$;  then $E\supseteq A$
and $E\symmdiff H=A\setminus H$ is meager, so $E\in\widehat{\Cal B}$.
If $A\subseteq F\in\widehat{\Cal B}$, let $G$ be an open set such that
$G\symmdiff(X\setminus F)$ is meager.   Then $G\cap A\subseteq G\cap F$
is meager, so $G\cap H$ is empty and
$E\setminus F\subseteq(E\symmdiff H)\cup(G\symmdiff(X\setminus F))$ is
meager.  Thus $E$ is a Baire-property envelope of $A$.

\medskip

{\bf (b)} Let $\family{\sigma}{S^*}{E_{\sigma}}$ be a Souslin scheme
in $\widehat{\Cal B}$ with kernel $A$.   Write

\Centerline{$S=\bigcup_{k\in\BbbN}\BbbN^k=S^*\cup\{\emptyset\}$.}

\noindent For each $\sigma\in S$, set

\Centerline{$A_{\sigma}
=\bigcup_{\phi\in\BbbN^{\Bbb N},\phi\supseteq\sigma}
  \bigcap_{n\ge 1}E_{\phi\restr n}$,}

\noindent and let $G_{\sigma}$ be a Baire-property envelope of
$A_{\sigma}$ as described in (a).   Because
$A_{\sigma}\subseteq E_{\sigma}$ (writing
$E_{\emptyset}=X$), we may suppose that
$G_{\sigma}\subseteq E_{\sigma}$.   Now, for any $\sigma\in S$,

\Centerline{$A_{\sigma}
=\bigcup_{i\in\BbbN}A_{\sigma^{\smallfrown}\fraction{i}}
\subseteq\bigcup_{i\in\BbbN}G_{\sigma^{\smallfrown}\fraction{i}}$,}

\noindent so

\Centerline{$H_{\sigma}
=G_{\sigma}\setminus\bigcup_{i\in\Bbb N}G_{\sigma^{\smallfrown}\fraction{i}}$}

\noindent is meager.

Set $H=\bigcup_{\sigma\in S}H_{\sigma}$, so that $H$ is meager.
Take any $x\in G_{\emptyset}\setminus H$.   Choose
$\sequence{i}{\phi(i)}$
inductively, as follows.   Given that
$\sigma=\langle\phi(i)\rangle_{i<k}$
has been chosen and $x\in G_{\sigma}$, then $x\notin H_{\sigma}$, so
there must be some $j\in\BbbN$ such that
$x\in G_{\sigma^{\smallfrown}\fraction{j}}$;   set $\phi(k)=j$, and continue.   Now

\Centerline{$x\in\bigcap_{k\ge 1}G_{\phi\restr k}
\subseteq\bigcap_{k\ge 1}E_{\phi\restr k}
\subseteq A$.}

Thus we see that $G_{\emptyset}\setminus H\subseteq A$.
On the other hand, $A\subseteq G_{\emptyset}$, so
$G_{\emptyset}\symmdiff A$ is meager and $A\in\widehat{\Cal B}$.   As
$\family{\sigma}{S^*}{E_{\sigma}}$ is arbitrary,
$\widehat{\Cal B}$ is closed under Souslin's operation.
}%end of proof of 431F

\vleader{48pt}{*431G}{}\cmmnt{ The second relies on a countable chain condition
to give the same envelope property.

\medskip

\noindent}{\bf Theorem}\dvAnew{2008}
Let $X$ be a set, $\Sigma$ a $\sigma$-algebra of
subsets of $X$ and $\Cal I\subseteq\Sigma$ a $\sigma$-ideal of subsets
of $X$.   If $\Sigma/\Cal I$ is ccc then $\Sigma$ is closed under
Souslin's operation.

\proof{{\bf (a)} As before, the essential fact is that for every
$A\subseteq X$
there is an $E\in\Sigma$ such that $A\subseteq E$ and $F\in\Cal I$
whenever $F\in\Sigma$ and
$F\subseteq E\setminus A$.   \Prf\ Let $\Cal E$ be a maximal disjoint
family of members of $\Sigma\setminus\Cal I$ disjoint from $A$.
Because $\Sigma/\Cal I$ is
ccc, $\Cal E$ is countable (316C), so $E=X\setminus\bigcup\Cal E$
belongs to $\Sigma$;  now it is easy to see that this $E$ serves.\ \Qed

In this case I will call $E$ a `measurable envelope' of $A$.

\medskip

{\bf (b)} Now we can argue as in 431A or 431F.   Let
$\family{\sigma}{S^*}{E_{\sigma}}$ be a Souslin scheme
in $\Sigma$ with kernel $A$;  for
$\sigma\in S$, set

\Centerline{$A_{\sigma}
=\bigcup_{\phi\in\BbbN^{\Bbb N},\sigma\subseteq\phi}
  \bigcap_{n\ge 1}E_{\phi\restr n}$,}

\noindent and let $G_{\sigma}\subseteq E_{\sigma}$
be a measurable envelope of $A_{\sigma}$.   Setting

\Centerline{$H=\bigcup_{\sigma\in S}
  (G_{\sigma}\setminus\bigcup_{i\in\Bbb N}
    G_{\sigma^{\smallfrown}\fraction{i}})$,}

\noindent $H\in\Cal I$ and $G_{\emptyset}\symmdiff A\subseteq H$, so
$A\in\Sigma$.
As $\family{\sigma}{S^*}{E_{\sigma}}$ is arbitrary, $\Sigma$ is closed under
Souslin's operation.
}%end of proof of 431G

\exercises{
\leader{431X}{Basic exercises (a)}
%\spheader 431Xa
Let $(X,\frak T,\Sigma,\mu)$ be a complete locally
determined topological measure space, $Y$ a topological space,
and $f:X\to Y$ a measurable function.   Let $\Cal B(Y)$ be the Borel
algebra of $Y$.   Show that $f^{-1}[B]\in\Sigma$ for every
$B\in\Cal S(\Cal B(Y))$.
%431A

\spheader 431Xb  Let $X$ be a topological space and $\mu$ a semi-finite
topological measure on $X$ which is inner regular with respect to the
Souslin-F sets.   Show that $\mu$ is inner regular with respect to the
closed sets.
%431D

\spheader 431Xc Let $(X,\Sigma,\mu)$ be a complete locally determined
measure space, and $\family{\sigma}{S^*}{E_{\sigma}}$ a regular Souslin
scheme in $\Sigma$ (definition:  421Xm) with kernel $A$.   Show that

\Centerline{$\mu A
=\sup_{\phi\in\BbbN^{\BbbN}}\mu(\bigcap_{n\ge 1}E_{\phi\restr n})$.}
%431D

\sqheader 431Xd Let $(X,\Sigma,\mu)$ be a measure space with locally
determined negligible sets (definition:  213I), and
$\family{\sigma}{S^*}{E_{\sigma}}$ a Souslin scheme in $\Sigma$
with kernel $A$.   Show that

\Centerline{$\mu^*A
=\sup_{\psi\in\BbbN^{\Bbb N}}
  \mu(\bigcup_{\phi\le\psi}\bigcap_{n\ge 1}E_{\phi\restr n})$.}
%431D

\sqheader 431Xe Let $(X,\Sigma,\mu)$ be a semi-finite measure space, and
$\family{\sigma}{S^*}{E_{\sigma}}$ a Souslin scheme in $\Sigma$
with kernel $A$.   Show that

\Centerline{$\mu_*A
=\sup_{\psi\in\BbbN^{\Bbb N}}
  \mu(\bigcup_{\phi\le\psi}\bigcap_{n\ge 1}E_{\phi\restr n})$.}
%431D

\leader{431Y}{Further exercises (a)}
%\spheader 431Ya
Let $(X,\Sigma,\mu)$ be a complete measure space with
the measurable envelope property (213Xl).   Show that $\Sigma$ is closed
under Souslin's operation.
%431A

\spheader 431Yb Let $X$ be a set, $\Sigma$ a $\sigma$-algebra of subsets
of $X$, and $\Cal I$ a $\sigma$-ideal of subsets of $X$ such that
$\Cal I\subseteq\Sigma$.   Suppose that for every $A\subseteq X$ there
is an $F\in\Sigma$ such that
$A\subseteq F$ and $F\setminus E\in\Cal I$ whenever
$A\subseteq E\in\Sigma$.   Show that $\Sigma$ is closed under Souslin's
operation.
%431F

\spheader 431Yc Let $X$ be a set, $\Sigma$ a $\sigma$-algebra of subsets
of $X$, and $\Cal I$ an $\omega_1$-saturated $\sigma$-ideal of $\Sigma$;
suppose that $A\in\Sigma$ whenever $A\subseteq B\in\Cal I$.    Show that
$\Sigma$ is closed under Souslin's operation.
%431Yb 431F
}%end of exercises

\endnotes{\Notesheader{431}
From the point of view of measure theory, the most important property of
Souslin's operation, after its idempotence, is the fact that (for many
measure spaces) the family of measurable sets is closed under the
operation (431A).   The proof I give here is based on the concept of
measurable envelope, which can be used in other cases of great interest
(431Yb, 431Yc).   But for some applications it is also very important to
know that if $A$ is the kernel of a Souslin scheme
$\family{\sigma}{S^*}{E_{\sigma}}$,
then $A$ can be approximated from inside by sets of the form
$H=\bigcup_{\phi\le\psi}\bigcap_{n\ge 1}E_{\phi\restr n}$ (431D, 431Xd),
which belong to the $\sigma$-algebra generated by the $E_{\sigma}$
(421M).   A typical application of this idea is when every $E_{\sigma}$
is a Borel subset of $\Bbb R$;  then we find not only that $A$ is
Lebesgue measurable
(indeed, measured by every Radon measure on $\Bbb R$) but that (for
any given Radon measure $\mu$) the Souslin scheme itself provides
Borel subsets $H$ of $A$ of measure approximating the measure of $A$.

Let me repeat that the essence of descriptive set theory is that we are
not satisfied merely to know that a set of a certain type exists.   We
want also to know how to build it, because we expect that an explicit
construction will be valuable later on.   For instance, the construction
given in 431D shows that if the Souslin scheme consists of closed
compact sets, the sets $H$ will be compact (421Xn).

I mention 431B as a typical application of 431A, even though it is both
obvious and obviously less than what can be said.   The algebras
$\Sigma$ of this section are algebras closed under Souslin's operation.
In a complete locally determined topological measure space, the algebra
$\Sigma$ of measurable sets includes the open sets (by definition),
therefore the Borel algebra $\Cal B$, therefore $\Cal S(\Cal B)$;  but
now we can take the algebra $\Cal A_1$ generated by $\Cal S(\Cal B)$,
and $\Cal A_1$ and $\Cal S(\Cal A_1)$ will also be included in $\Sigma$,
so that $\Sigma$ will included the algebra $\Cal A_2$ generated by
$\Cal S(\Cal A_1)$, and so on.   (Note that $\Cal S(\Cal A_1)$ includes
the $\sigma$-algebra generated by $\Cal A_1$, by 421F, so I do not need
to mention that separately.)   We have to run through all the countable
ordinals before we can be sure of getting to the smallest algebra
$\Cal A_{\omega_1}=\bigcup_{\xi<\omega_1}\Cal A_{\xi}$ which contains
every open set and is closed under Souslin's operation, and we shall
then have $\Cal A_{\omega_1}\subseteq\Sigma$.

The result in 431D is one of the special features of measures.   (A
similar result, based on rather different hypotheses, is in 432K.)   But
the argument of 431A can be applied in many other cases;  see
431Yb-431Yc.   A striking one is 431F, which will be useful in Volume 5.
%5{}14 5{}27
}%end of notes

\discrpage

