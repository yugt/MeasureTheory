\frfilename{mt323.tex}
\versiondate{20.7.06}
\copyrightdate{1999}

\def\chaptername{Measure algebras}
\def\sectionname{The topology of a measure algebra}

\newsection{323}

I take a short section to discuss one of the fundamental tools for
studying totally finite measure algebras, the natural metric that each
carries.   The same ideas, suitably adapted, can be applied to an
arbitrary
measure algebra, where we have a topology corresponding closely to the
topology of convergence in measure on the function space $L^0$.   Most
of the section consists of an analysis of the relations between this
topology and the order structure of the measure algebra.

\leader{323A}{The pseudometrics $\rho_a$ (a)} Let $(\frak A,\bar\mu)$ be
a measure algebra.   Write $\frak A^f=\{a:a\in\frak A,\,\bar\mu
a<\infty\}$.   For $a\in\frak A^f$ and $b$, $c\in\frak A$, write
$\rho_a(b,c)=\bar\mu(a\Bcap(b\Bsymmdiff c))$.
\cmmnt{Then} $\rho_a$ is a pseudometric on $\frak A$.
\prooflet{\Prf\ (i)
Because $\bar\mu a<\infty$, $\rho_a$ takes values in $\coint{0,\infty}$.
(ii) If $b$, $c$, $d\in\frak A$ then
$b\Bsymmdiff d\Bsubseteq(b\Bsymmdiff c)\Bcup(c\Bsymmdiff d)$, so

$$\eqalign{\rho_a(b,d)
&=\bar\mu(a\Bcap(b\Bsymmdiff d))
\le\bar\mu((a\Bcap(b\Bsymmdiff c))\Bcup(a\Bcap(c\Bsymmdiff d)))\cr
&\le\bar\mu(a\Bcap(b\Bsymmdiff c))+\bar\mu(a\Bcap(c\Bsymmdiff d))
=\rho_a(b,c)+\rho_a(c,d).\cr}$$

\noindent (iii) If $b$, $c\in\frak A$ then

\Centerline{$\rho_a(b,c)
=\bar\mu(a\Bcap(b\Bsymmdiff c))
=\bar\mu(a\Bcap(c\Bsymmdiff b))
=\rho_a(c,b)$.  \Qed}
}%end of prooflet

\header{323Ab}{\bf (b)} Now the {\bf measure-algebra topology} of the
measure algebra $(\frak A,\bar\mu)$ is that generated by the family
$\Rho=\{\rho_a:a\in\frak A^f\}$ of pseudometrics on $\frak A$.
Similarly the {\bf measure-algebra uniformity} on $\frak A$ is that
generated by $\Rho$.   \cmmnt{For the rest of this section I will take
it that every measure algebra is endowed with its measure-algebra topology
and uniformity.}

\cmmnt{(For a general discussion of topologies defined by pseudometrics,
see 2A3F {\it et seq.}   For the associated uniformities see \S3A4.)}

\header{323Ac}{\bf (c)}\cmmnt{ Note that} $\Rho$ is
upwards-directed\cmmnt{, since
$\rho_{a\Bcup a'}\ge\max(\rho_a,\rho_{a'})$ for all $a$,
$a'\in\frak A^f$}.

\spheader 323Ad\discrversionA{\footnote{Revised 2006.}}{} 
On the ideal $\frak A^f$ we have an actual metric $\rho$
defined by saying that $\rho(a,b)=\bar\mu(a\Bsymmdiff b)$ for $a$,
$b\in\frak A^f$\prooflet{ (to see that $\rho$ is a metric, repeat the
formulae of (a) above)};  this is the {\bf measure metric}\cmmnt{ or
{\bf Fr\'echet-Nikod\'ym metric}}. %Plebanek 97
I will call the topology it generates the {\bf strong measure-algebra
topology} on $\frak A^f$.

When $\bar\mu$ is totally finite\cmmnt{, that is, $\frak A^f=\frak A$},
$\rho=\rho_1$ defines the measure-algebra topology and uniformity of
$\frak A$.

\leader{323B}{Proposition} Let $(\frak A,\bar\mu)$ be any measure
algebra, and give $\frak A$ its measure-algebra topology.

(a) The operations $\Bcup$, $\Bcap$, $\Bsetminus$ and $\Bsymmdiff$ are
all uniformly continuous.

(b)\dvAnew{2009} $\frak A^f$ is dense in $\frak A$.

\proof{{\bf (a)} 
The point is that for any $b$, $c$, $b'$, $c'\in\frak A$ we have

\Centerline{$(b*c)\Bsymmdiff(b'*c')\Bsubseteq(b\Bsymmdiff
b')\Bcup(c\Bsymmdiff c')$}

\noindent for any of the operations $*=\Bcup$, $\Bcap$ etc.;  so that if
$a\in\frak A^f$ then

\Centerline{$\rho_a(b*c,b'*c')\le\rho_a(b,b')+\rho_a(c,c')$.}

\noindent Consequently the operation $*$ must be uniformly continuous.

\medskip

{\bf (b)} Given $b\in\frak A$, $a\in\frak A^f$ and
$\epsilon>0$, then $a\Bcap b\in\frak A^f$ and $\rho_a(b,a\Bcap b)=0$.
Because the family $\{\rho_a:a\in\frak A^f\}$ is upwards-directed, this is
enough to show that every neighbourhood of $b$ meets $\frak A^f$;  as $b$
is arbitrary, $\frak A^f$ is dense.
}%end of proof of 323B

\leader{323C}{Proposition} (a) Let $(\frak A,\bar\mu)$ be a totally
finite measure algebra.   Then $\bar\mu:\frak A\to\coint{0,\infty}$ is
uniformly continuous.

(b) Let $(\frak A,\bar\mu)$ be a semi-finite measure algebra.   Then
$\bar\mu:\frak A\to[0,\infty]$ is lower semi-continuous.

(c) Let $(\frak A,\bar\mu)$ be any measure algebra.   If $a\in\frak A$
and $\bar\mu a<\infty$, then
$b\mapsto\bar\mu(b\Bcap a):\frak A\to\Bbb R$ is uniformly continuous.

\proof{{\bf (a)} For any $a$, $b\in\frak A$,

\Centerline{$|\bar\mu a-\bar\mu b|
\le\bar\mu(a\Bsymmdiff b)=\rho_1(a,b)$.}

\medskip

{\bf (b)} Suppose that $b\in\frak A$ and $\bar\mu b>\alpha\in\Bbb R$.
Then there is an $a\Bsubseteq b$ such that $\alpha<\bar\mu a<\infty$
(322Eb).   If $c\in\frak A$ is such that $\rho_a(b,c)<\bar\mu a-\alpha$,
then

\Centerline{$\bar\mu c\ge\bar\mu(a\Bcap c)
=\bar\mu a-\bar\mu(a\Bcap(b\Bsetminus c))>\alpha$.}

\noindent Thus $\{b:\bar\mu b>\alpha\}$ is open;  as $\alpha$ is
arbitrary, $\bar\mu$ is lower semi-continuous.

\medskip

{\bf (c)}
$|\bar\mu(a\Bcap b)-\bar\mu(a\Bcap c)|\le\rho_a(b,c)$ for all
$b$, $c\in\frak A$.
}%end of proof of 323C

\vleader{108pt}{323D}{}\cmmnt{ The following facts are basic to any
understanding
of the relationship between the order structure and topology of a
measure algebra.

\medskip

\noindent}{\bf Lemma} Let $(\frak A,\bar\mu)$ be a measure algebra.

(a) Let $B\subseteq\frak A$ be a non-empty upwards-directed set.   For
$b\in B$ set $F_b=\{c:b\Bsubseteq c\in B\}$.

\quad (i) $\{F_b:b\in B\}$ generates a Cauchy filter
$\Cal F(B\closeuparrow)$ on $\frak A$.

\quad (ii) If $\sup B$ is defined in $\frak A$, then it is a
topological limit of $\Cal F(B\closeuparrow)$;  \cmmnt{in particular,}
it belongs to the topological closure of $B$.

(b) Let $B\subseteq\frak A$ be a non-empty downwards-directed set.   For
$b\in B$ set $F'_b=\{c:b\Bsupseteq c\in B\}$.

\quad (i) $\{F'_b:b\in B\}$ generates a Cauchy filter
$\Cal F(B\closedownarrow)$ on $\frak A$.

\quad (ii) If $\inf B$ is defined in $\frak A$, then it is a
topological limit of $\Cal F(B\closedownarrow)$;  \cmmnt{in
particular,} it
belongs to the topological closure of $B$.

(c)(i) Closed subsets of $\frak A$ are order-closed\cmmnt{ in the
sense of 313Da}.

\quad (ii) An order-dense subalgebra of $\frak A$ must be dense in the
topological sense.

(d) Now suppose that $(\frak A,\bar\mu)$ is semi-finite.

\quad (i) The sets $\{b:b\Bsubseteq c\}$, $\{b:b\Bsupseteq c\}$ are
closed for every $c\in\frak A$.

\quad (ii) If $B\subseteq\frak A$ is non-empty and upwards-directed and
$e$ is a cluster point of $\Cal F(B\closeuparrow)$, then $e=\sup B$.

\quad (iii) If $B\subseteq\frak A$ is non-empty and downwards-directed
and $e$ is a cluster point of $\Cal F(B\closedownarrow)$, then 
$e=\inf B$.

\proof{ I use the notations $\frak A^f$, $\rho_a$ from 323A.

\medskip

{\bf (a)(i)} ($\alpha$) If $b$, $c\in B$ then there is a $d\in B$ such
that $b\Bcup c\Bsubseteq d$, so that $F_d\subseteq F_b\cap F_c$;
consequently

\Centerline{$\Cal F(B\closeuparrow)
=\{F:F\subseteq \frak A,\,\exists\,b\in B,\,F_b\subseteq F\}$}

\noindent is a filter on $\frak A$.  ($\beta$) Let $a\in\frak A^f$,
$\epsilon>0$.   Then there is a $b\in B$ such that
$\bar\mu(a\Bcap c)\le\bar\mu(a\Bcap b)+\bover12\epsilon$ for every
$c\in B$, and $F_b\in\Cal F(B\closeuparrow)$.   If now $c$, $c'\in F_b$,
$c\Bsymmdiff c'\Bsubseteq(c\Bsetminus b)\Bcup(c'\Bsetminus b)$, so

\Centerline{$\rho_a(c,c')
\le\bar\mu(a\Bcap c\Bsetminus b)+\bar\mu(a\Bcap c'\Bsetminus b)
=\bar\mu(a\Bcap c)+\bar\mu(a\Bcap c')-2\bar\mu(a\Bcap b)
\le\epsilon$.}

\noindent As $a$ and $\epsilon$ are arbitrary, $\Cal F(B\closeuparrow)$
is Cauchy.

\medskip

\quad{\bf (ii)} Suppose that $e=\sup B$ is defined in $\frak A$.
Let $a\in\frak A^f$, $\epsilon>0$.   By 313Ba,
$a\Bcap e=\sup_{b\in B}a\Bcap b$;
but $\{a\Bcap b:b\in B\}$ is upwards-directed,
so  $\bar\mu(a\Bcap e)=\sup_{b\in B}\bar\mu(a\Bcap b)$, by 321D.
Let $b\in B$ be such that
$\bar\mu(a\Bcap b)\ge\bar\mu(a\Bcap e)-\epsilon$.
Then for any $c\in F_b$, $e\Bsymmdiff c\Bsubseteq e\Bsetminus b$, so

\Centerline{$\rho_a(e,c)
=\bar\mu(a\Bcap(e\Bsymmdiff c))
\le\bar\mu(a\Bcap(e\Bsetminus b))
=\bar\mu(a\Bcap e)-\bar\mu(a\Bcap b)
\le\epsilon$.}

\noindent As $a$ and $\epsilon$ are arbitrary,
$\Cal F(B\closeuparrow)\to e$.

Because $B\in\Cal F(B\closeuparrow)$, $e$ surely belongs to the
topological closure of $B$.

\medskip

{\bf (b)} Either repeat the arguments above, with appropriate
inversions, using 321F in place of 321D, or apply (a) to the set
$\{1\Bsetminus b:b\in B\}$.

\medskip

{\bf (c)(i)} This follows at once from (a) and (b) and the definition in
313Da.

\medskip

\quad{\bf (ii)} If $\frak B\subseteq\frak A$ is an order-dense
subalgebra
and $a\in\frak A$, then $B=\{b:b\in\frak B,\,b\Bsubseteq a\}$ is
upwards-directed and has supremum $a$ (313K);  by (a-ii),
$a\in\overline{B}\subseteq\overline{\frak B}$.   As $a$ is arbitrary,
$\frak B$ is topologically dense.

\medskip

{\bf (d)(i)} Set $F=\{b:b\Bsubseteq c\}$.   If $d\in\frak A\setminus F$,
then (because $(\frak A,\bar\mu)$ is semi-finite) there is an $a\in\frak
A^f$ such that $\delta=\bar\mu(a\Bcap d\Bsetminus c)>0$;  now if $b\in
F$,

\Centerline{$\rho_a(d,b)\ge\bar\mu(a\Bcap d\Bsetminus b)\ge\delta$,}

\noindent so that $d$ cannot belong to the closure of $F$.   As $d$ is
arbitrary, $F$ is closed.   Similarly, $\{b:b\Bsupseteq c\}$ is closed.

\medskip

\quad{\bf (ii)} ($\alpha$) If $b\in B$, then $e\in\overline{F_b}$,
because $F_b\in\Cal F(B\closeuparrow)$;  but
$\{c:b\Bsubseteq c\}$ is a closed set including $F_b$, so
contains $e$, and $b\Bsubseteq e$.   As $b$ is arbitrary, $e$ is an
upper bound for $B$.   ($\beta$)
If $d$ is an upper bound of $B$, then $\{c:c\Bsubseteq d\}$
is a closed set belonging to $\Cal F(B\closeuparrow)$, so
contains $e$.   As $d$ is arbitrary, this shows that $e$ is the
supremum of $B$, as claimed.

\medskip

\quad{\bf (iii)} Use the same arguments as in (ii), but inverted.
}%end of proof of 323D

\leader{323E}{Corollary} Let $(\frak A,\bar\mu)$ be a measure algebra.

(a) If $\sequencen{b_n}$ is a non-decreasing sequence in $\frak A$ with
supremum $b$, then $\sequencen{b_n}$ converges topologically to $b$.

(b) If $\sequencen{b_n}$ is a non-increasing sequence in $\frak A$ with
infimum $b$, then $\sequencen{b_n}$ converges topologically to $b$.

\proof{ I call this a `corollary' because it is the special case of
323Da-323Db in which $B$ is the set of terms of a monotonic sequence;
but it is probably easier to work directly from the definition in 323A,
and use 321Be or 321Bf to see that $\lim_{n\to\infty}\rho_a(b_n,b)=0$
whenever $\bar\mu a<\infty$.
}%end of proof of 323E


\leader{323F}{}\cmmnt{ The following is a useful calculation.

\medskip

\noindent}{\bf Lemma} Let $(\frak A,\bar\mu)$ be a measure algebra and
$\sequencen{c_n}$ a sequence in $\frak A$ such that \dvro{}{the sum}
$\sum_{n=0}^{\infty}\bar\mu(c_n\Bsymmdiff c_{n+1})$ is finite.   Set
$d_0=\sup_{n\in\Bbb N}\inf_{m\ge n}c_m$, $d_1=\inf_{n\in\Bbb
N}\sup_{m\ge n}c_m$.   Then $d_0=d_1$ and, writing $d$ for their common
value, $\lim_{n\to\infty}\bar\mu (c_n\Bsymmdiff d)=0$.

\proof{ Write $\alpha_n=\bar\mu(c_n\Bsymmdiff c_{n+1})$,
$\beta_n=\sum_{k=n}^{\infty}\alpha_k$ for $n\in\Bbb N$;  we are
supposing that $\lim_{n\to\infty}\beta_n=0$.   Set $b_n=\sup_{m\ge
n}c_m\Bsymmdiff c_{m+1}$;  then

\Centerline{$\bar\mu b_n\le\sum_{m=n}^{\infty}\bar\mu(c_m\Bsymmdiff
c_{m+1})=\beta_n$}

\noindent for each $n$.   If $m\ge n$, then

\Centerline{$c_m\Bsymmdiff c_n\Bsubseteq\sup_{n\le k<m}c_k\Bsymmdiff
c_{k+1}\Bsubseteq b_n$,}

\noindent so

\Centerline{$c_n\Bsetminus b_n\Bsubseteq c_m\Bsubseteq c_n\Bcup b_n$.}
\noindent Consequently

\Centerline{$c_n\Bsetminus b_n\Bsubseteq\inf_{k\ge m}c_k\Bsubseteq
\sup_{k\ge m}c_k\Bsubseteq c_n\Bcup b_n$}

\noindent for every $m\ge n$, and

\Centerline{$c_n\Bsetminus b_n\Bsubseteq d_0\Bsubseteq d_1\Bsubseteq
c_n\Bcup b_n$,}

\noindent so that

\Centerline{$c_n\Bsymmdiff d_0\Bsubseteq b_n$,
\quad$c_n\Bsymmdiff d_1\Bsubseteq b_n$,
\quad$d_1\Bsetminus d_0\Bsubseteq b_n$.}

\noindent As this is true for every $n$,

\Centerline{$\lim_{n\to\infty}\bar\mu(c_n\Bsymmdiff d_i)
\le\lim_{n\to\infty}\bar\mu b_n=0$}

\noindent for both $i$, and

\Centerline{$\bar\mu(d_1\Bsymmdiff d_0)\le\inf_{n\in\Bbb N}\bar\mu
b_n=0$,}

\noindent so that $d_1=d_0$.
}%end of proof of 323F

\leader{323G}{The classification of measure algebras:  Theorem} Let
$(\frak A,\bar\mu)$ be a measure algebra, $\frak T$ its measure-algebra
topology and $\Cal U$ its measure-algebra uniformity.

(a) $(\frak A,\bar\mu)$ is semi-finite iff $\frak T$ is Hausdorff.

(b) $(\frak A,\bar\mu)$ is $\sigma$-finite iff $\frak T$ is metrizable,
and in this case $\Cal U$ also is metrizable.

(c) $(\frak A,\bar\mu)$ is localizable iff $\frak T$ is Hausdorff and
$\frak A$ is complete under $\Cal U$.

\proof{ I use the notations $\frak A^f$, $\rho_a$ from 323A.

\medskip

{\bf (a)(i)} Suppose that $(\frak A,\bar\mu)$ is semi-finite and that
$b$, $c$ are distinct members of $\frak A$.   Then there is an
$a\Bsubseteq
b\Bsymmdiff c$ such that $0<\bar\mu a<\infty$, and now $\rho_a(b,c)>0$.
As $b$ and $c$ are arbitrary, $\frak T$ is Hausdorff (2A3L).

\medskip

\quad{\bf (ii)} Suppose that $\frak T$ is Hausdorff and that $b\in\frak
A$ has $\bar\mu b=\infty$.   Then $b\ne 0$ so there must be an
$a\in\frak
A^f$ such that $\bar\mu(a\Bcap b)=\rho_a(0,b)>0$;   in which case
$a\Bcap b\Bsubseteq b$ and $0<\bar\mu(a\Bcap b)<\infty$.   As $b$ is
arbitrary, $\bar\mu$ is semi-finite.

\medskip

{\bf (b)(i)} Suppose that $\bar\mu$ is $\sigma$-finite.   Let
$\sequencen{a_n}$ be a non-decreasing sequence in $\frak A^f$ with
supremum $1$.   Set

$$\eqalign{\rho(b,c)
=\sum_{n=0}^{\infty}\bover{\rho_{a_n}(b,c)}{1+2^n\bar\mu a_n}}$$

\noindent for $b$, $c\in\frak A$.   Then $\rho$ is a metric on
$\frak A$, because if $\rho(b,c)=0$ then $a_n\Bcap(b\Bsymmdiff c)=0$ for
every $n$, so $b\Bsymmdiff c=0$ and $b=c$.

If $a\in\frak A^f$ and $\epsilon>0$, take $n$ such that
$\bar\mu(a\Bsetminus
a_n)\le\bover12\epsilon$.   If $b$, $c\in\frak A$ and
$\rho(b,c)\le\epsilon/2(1+2^n\bar\mu a_n)$, then

$$\eqalign{\rho_a(b,c)
&=\rho_{a\Bsetminus a_n}(b,c)+\rho_{a\Bcap a_n}(b,c)
\le\bar\mu(a\Bsetminus a_n)+\rho_{a_n}(b,c)\cr
&\le\Bover12\epsilon+(1+2^n\bar\mu a_n)\rho(b,c)
\le\epsilon.\cr}$$

In the other direction, given $\epsilon>0$, take $n\in\Bbb N$ such that
$2^{-n}\le\bover12\epsilon$;   then $\rho(b,c)\le\epsilon$ whenever
$\rho_{a_n}(b,c)\le\epsilon/2(n+1)$.

This shows that $\Cal U$ is the same as the metrizable uniformity
defined by $\{\rho\}$;  accordingly $\frak T$ also is defined by $\rho$.

\medskip

\quad{\bf (ii)} Now suppose that $\frak T$ is metrizable, and let $\rho$
be a metric defining $\frak T$.   For each $n\in\Bbb N$ there must be
$a_{n0},\ldots,a_{nk_n}\in\frak A^f$ and $\delta_n>0$ such that

\Centerline{$\rho_{a_{ni}}(b,1)\le\delta_n$ for every $i\le
k_n\Longrightarrow\rho(b,1)\le 2^{-n}$.}

\noindent Set $d=\sup_{n\in\Bbb N,i\le k_n}a_{ni}$.   Then
$\rho_{a_{ni}}(d,1)=0$ for every $n$ and $i$, so $\rho(d,1)\le 2^{-n}$ for
every $n$ and $d=1$.   Thus $1$ is the supremum of countably
many elements of finite measure and $(\frak A,\bar\mu)$ is
$\sigma$-finite.

\medskip

{\bf (c)(i)} Suppose that $(\frak A,\bar\mu)$ is localizable.   Then
$\frak T$ is Hausdorff, by (a).   Let $\Cal F$ be a Cauchy filter on
$\frak A$.   For each $a\in\frak A^f$, choose a sequence
$\sequencen{F_n(a)}$ in $\Cal F$ such that
$\rho_a(b,c)\le 2^{-n}$ whenever $b$, $c\in F_n(a)$ and $n\in\Bbb N$.
Choose $c_{an}\in\bigcap_{k\le n}F_k(a)$ for each $n$;  then
$\rho_a(c_{an},c_{a,n+1})\le 2^{-n}$ for each $n$.   Set
$d_a=\sup_{n\in\Bbb N}\inf_{k\ge n}a\Bcap c_{ak}$.   Then

\Centerline{$\lim_{n\to\infty}\rho_a(d_a,c_{an})
=\lim_{n\to\infty}\bar\mu(d_a\Bsymmdiff(a\Bcap c_{an}))=0$,}

\noindent by 323F.

If $a$, $b\in\frak A^f$ and $a\Bsubseteq b$, then $d_a=a\Bcap d_b$.
\Prf\ For each $n\in\Bbb N$, $F_n(a)$ and $F_n(b)$ both belong to
$\Cal F$, so must have a point $e$ in common;  now

$$\eqalign{\rho_a(d_a,d_b)
&\le\rho_a(d_a,c_{an})+\rho_a(c_{an},e)
  +\rho_a(e,c_{bn})+\rho_a(c_{bn},d_b)\cr
&\le\rho_a(d_a,c_{an})+\rho_a(c_{an},e)
  +\rho_b(e,c_{bn})+\rho_b(c_{bn},d_b)\cr
&\le\rho_a(d_a,c_{an})+2^{-n}+2^{-n}+\rho_b(c_{bn},d_b)\cr
&\to 0\text{ as }n\to\infty.\cr}$$

\noindent Consequently $\rho_a(d_a,d_b)=0$, that is,

\Centerline{$d_a=a\Bcap d_a=a\Bcap d_b$.   \Qed}

Set $d=\sup\{d_b:b\in\frak A^f\}$;  this is defined because $\frak A$ is
Dedekind complete.   Then $\Cal F\to d$.   \Prf\  If
$a\in\frak A^f$ and $\epsilon>0$, then

\Centerline{$a\Bcap d
=\sup_{b\in\frak A^f}a\Bcap d_b
=\sup_{b\in\frak A^f}a\Bcap b\Bcap d_{a\Bcup b}
=\sup_{b\in\frak A^f}a\Bcap b\Bcap d_{a}
=a\Bcap d_a$.}

\noindent So if we choose $n\in\Bbb N$ such that
$2^{-n}+\rho_a(c_{an},d_a)\le\epsilon$, then for any $e\in F_n(a)$ we
shall have

\Centerline{$\rho_a(e,d)
\le\rho_a(e,c_{an})+\rho_a(c_{an},d)
\le 2^{-n}+\rho_a(c_{an},d_a)
\le\epsilon$.}

\noindent   Thus

\Centerline{$\{e:\rho_a(d,e)\le \epsilon\}\supseteq F_n(a)\in\Cal F$.}

\noindent As $a$, $\epsilon$ are arbitrary, $\Cal F$ converges to $d$.\
\QeD\  As $\Cal F$ is arbitrary, $\frak A$ is complete.

\medskip

\quad{\bf (ii)} Now suppose that $\frak T$ is Hausdorff and that 
$\frak A$ is complete under $\Cal U$.   By (a), $(\frak A,\bar\mu)$ is
semi-finite.
Let $B$ be any non-empty subset of $\frak A$, and set
$B'=\{b_0\Bcup\ldots\Bcup b_n:b_0,\ldots,b_n\in B\}$, so that $B'$ is
upwards-directed and has the same upper bounds as $B$.   By 323Da, we
have a Cauchy filter $\Cal F(B'\backstep3\uparrow)$;  because $\frak A$
is complete, this is convergent;  and because $(\frak A,\bar\mu)$ is
semi-finite, its limit must be $\sup B'=\sup B$, by 323Dd.
As $B$ is arbitrary, $\frak A$ is Dedekind complete, so
$(\frak A,\bar\mu)$ is localizable.
}%end of proof of 323G

\leader{323H}{Closed\dvrocolon{ subalgebras}}\cmmnt{ The ideas used in
the proof of
(c) above have many other applications, of which one of the most
important is the following.   You may find it helpful to read the next
theorem first on the assumption that $(\frak A,\bar\mu)$ is a
probability algebra.

\medskip

\noindent}{\bf Theorem} Let $(\frak A,\bar\mu)$ be a localizable measure
algebra, and $\frak B$ a subalgebra of $\frak A$.   Then it is
topologically closed iff it is order-closed.

\proof{{\bf (a)} If $\frak B$ is closed, it must be order-closed, by
323Dc.

\medskip

{\bf (b)} Now suppose that $\frak B$ is order-closed.   I repeat the
ideas of part (c-i) of the proof of 323G.   Let
$e$ be any member of the closure of $\frak B$ in $\frak A$.   For each
$a\in\frak A^f$ and $n\in\Bbb N$ choose $c_{an}\in \frak B$ such that
$\rho_a(c_{an},e)\le 2^{-n}$.   Then

$$\eqalign{\sum_{n=0}^{\infty}\bar\mu((a\Bcap c_{an})\Bsymmdiff(a\Bcap
c_{a,n+1}))
&=\sum_{n=0}^{\infty}\rho_a(c_{an},c_{a,n+1})\cr
&\le\sum_{n=0}^{\infty}\rho_a(c_{an},e)+\rho_a(e,c_{a,n+1})
<\infty.\cr}$$

\noindent So if we set $e_a=\sup_{n\in\Bbb N}\inf_{k\ge n}c_{ak}$, then

\Centerline{$\rho_a(e_a,c_{an})=\rho_a(a\Bcap e_a,a\Bcap c_{an})\to 0$}

\noindent as $n\to\infty$, by 323F, and $\rho_a(e,e_a)=0$, that is,
$a\Bcap e_a=a\Bcap e$.   Also, because $\frak B$ is order-closed,
$\inf_{k\ge n}c_{ak}\in\frak B$ for every $n$, and $e_a\in\frak B$.

Because $\frak A$ is Dedekind complete, we can set

\Centerline{$e'_a=\inf\{e_b:b\in\frak A^f,\,a\Bsubseteq b\}$;}

\noindent then $e'_a\in\frak B$ and

\Centerline{$e'_a\Bcap a
=\inf_{b\Bsupseteq a}e_b\Bcap a
=\inf_{b\Bsupseteq a}e_b\Bcap b\Bcap a
=\inf_{b\Bsupseteq a}e\Bcap b\Bcap a
=e\Bcap a$.}

\noindent Now $e'_a\Bsubseteq e'_b$ whenever $a\Bsubseteq b$, so
$B=\{e'_a:a\in\frak A^f\}$ is upwards-directed, and

\Centerline{$\sup B=\sup\{e'_a\Bcap a:a\in\frak A^f\}=\sup\{e\Bcap
a:a\in\frak A^f\}=e$}

\noindent because $(\frak A,\bar\mu)$ is semi-finite.   Accordingly
$e\in\frak B$.   As $e$ is arbitrary, $\frak B$ is closed, as claimed.
}%end of proof of 323H

\leader{323I}{Notation} In the context of 323H, I will
say\cmmnt{ simply} that
$\frak B$ is a {\bf closed subalgebra} of $\frak A$.

\leader{323J}{Proposition} If $(\frak A,\bar\mu)$ is a localizable
measure algebra and $\frak B$ is a subalgebra of $\frak A$, then the
topological
closure $\overline{\frak B}$ of $\frak B$ in $\frak A$ is precisely the
order-closed subalgebra of $\frak A$ generated by $\frak B$.

\proof{ Write $\frak B_{\tau}$ for the smallest order-closed subset of
$\frak A$ including $\frak B$.   By 313Gc, $\frak B_{\tau}$ is a
subalgebra of $\frak A$, and is the order-closed subalgebra of $\frak A$
generated by $\frak B$.   Being an order-closed subalgebra of $\frak A$,
it is topologically closed, by 323H, and must include
$\overline{\frak B}$.   On the other hand, $\overline{\frak B}$, being
topologically
closed, is order-closed (323D(c-i)), so includes $\frak B_{\tau}$.
Thus $\overline{\frak B}=\frak B_{\tau}$ is the order-closed subalgebra
of $\frak A$ generated by $\frak B$.
}%end of proof of 323J

\leader{323K}{}\cmmnt{ I note some simple results for future
reference.

\medskip

\noindent}{\bf Lemma} If $(\frak A,\bar\mu)$ is a localizable measure
algebra and $\frak B$ is a closed subalgebra of $\frak A$, then for any
$a\in\frak A$ the subalgebra\cmmnt{ $\frak C$} of $\frak A$ generated
by $\frak B\cup\{a\}$ is closed.

\proof{ By 314Ja, $\frak C$ is order-closed.
}%end of proof of 323K

\leader{323L}{Proposition} Let $\familyiI{(\frak A_i,\bar\mu_i)}$ be a
family of measure algebras with simple product
$(\frak A,\bar\mu)$\cmmnt{ (322K)}.
Then the measure-algebra topology on
$\frak A=\prod_{i\in I}\frak A_i$ defined by $\bar\mu$ is\cmmnt{ just}
the product of the measure-algebra topologies of the $\frak A_i$.

\proof{ I use the notations $\frak A^f$, $\rho_a$ from 323A.   Write
$\frak T$ for the topology of $\frak A$ and $\frak S$
for the product topology.
For $i\in I$ and $d\in\frak A_i^f$ define a pseudometric $\tilde\rho_{di}$
on $\frak A$ by setting

\Centerline{$\tilde\rho_{di}(b,c)=\rho_{d}(b(i),c(i))$}

\noindent whenever $b$, $c\in\frak A$;  then $\frak S$ is defined by
$\Rho=\{\tilde\rho_{di}:i\in I,\,a\in\frak A_i^f\}$ (3A3Ig).   Now
each $\tilde\rho_{di}$ is one of the defining pseudometrics for
$\frak T$, since

\Centerline{$\tilde\rho_{di}(b,c)=\bar\mu(\tilde d\Bcap(b\symmdiff c))$}

\noindent where $\tilde d(i)=d$, $\tilde d(j)=0$ for $j\ne i$.   So
$\frak S\subseteq\frak T$.

Now suppose that $a\in\frak A^f$ and $\epsilon>0$.   Then $\sum_{i\in
I}\bar\mu_ia(i)=\bar\mu a$ is finite, so there is a finite set
$J\subseteq I$ such that $\sum_{i\in I\setminus
J}\bar\mu_ia(i)\le\bover12\epsilon$.   For each $j\in J$,
$\tau_j=\tilde\rho_{a(j),j}$ belongs to $\Rho$, and

$$\eqalign{\rho_a(b,c)
&=\sum_{i\in I}\bar\mu_i(a(i)\Bcap(b(i)\Bsymmdiff c(i)))\cr
&\le\sum_{j\in J}\bar\mu_j(a(j)\Bcap(b(j)\Bsymmdiff c(j)))
   +\Bover12\epsilon
=\sum_{j\in J}\tau_j(b,c)+\Bover12\epsilon
\le\epsilon\cr}$$

\noindent whenever $b$, $c$ are such that
$\tau_j(b,c)\le\epsilon/(1+2\#(J))$ for every $j\in J$.   By 2A3H,
the identity map from $(\frak A,\frak S)$ to $(\frak A,\frak T)$ is
continuous, that is, $\frak T\subseteq\frak S$.

Putting these together, we see that $\frak S=\frak T$, as claimed.
}%end of proof of 323L

\leader{*323M}{}\dvAformerly{3{}23Xg}\cmmnt{ In this volume we shall have 
little need to
consider the measure metric on $\frak A^f$, but the following facts are
sometimes useful.

\medskip

\noindent}{\bf Proposition} Let $(\frak A,\bar\mu)$ be a measure algebra,
and give $\frak A^f$ its measure metric.   

(a) The Boolean operations $\Bsymmdiff$, $\Bcap$, $\Bcup$ and $\Bsetminus$
on $\frak A^f$ are uniformly continuous.

(b) $\bar\mu\restrp\frak A^f:\frak A^f\to\coint{0,\infty}$ is
$1$-Lipschitz, therefore uniformly continuous.

(c) $\frak A^f$ is complete.

\proof{{\bf (a)} Writing $\rho$ for the measure metric on $\frak A^f$,
then, just as in the proof of 323Ba,

\Centerline{$\rho(b*c,b'*c')\le\rho(b,b')+\rho(c,c')$}

\noindent for all $b$, $c$, $b'$, $c'\in\frak A^f$ and any of the Boolean
operations $*=\Bsymmdiff$, $\Bcap$, $\Bcup$ and $\Bsetminus$.

\medskip

{\bf (b)} If $a$, $b\in\frak A^f$ then 

\Centerline{$|\bar\mu a-\bar\mu b|
\le|\bar\mu a-\bar\mu(a\Bcap b)|+|\bar\mu b-\bar\mu(a\Bcap b)|
=\bar\mu(a\Bsetminus b)+\bar\mu(b\Bsetminus a)
=\rho(a,b)$.}

\medskip

{\bf (c)} If $\sequencen{a_n}$ is a sequence in $\frak A^f$ such that
$\sum_{n=0}^{\infty}\rho(a_n,a_{n+1})<\infty$, set 
$d=\sup_{n\in\Bbb N}\inf_{m\ge n}a_m$.   By 323F, 
$\lim_{n\to\infty}\bar\mu(d\Bsymmdiff a_n)=0$.   In particular, there is
some $n\in\Bbb N$ such that $\bar\mu(d\Bsetminus a_n)$ is finite, so
$d\in\frak A^f$ and $\lim_{n\to\infty}\rho(d,a_n)=0$.   As in 2A4E, this is
enough to show that $\frak A^f$ is complete.
}%end of proof of 323M

\exercises{
\leader{323X}{Basic exercises $\pmb{>}$(a)}
%\sqheader 323Xa
Let $(X,\Sigma,\mu)$ be a measure space, and
$(\frak A,\bar\mu)$ its measure algebra.   (i) Show that we have an
injection $\chi:\frak A\to L^0(\mu)$ (see \S241) given by setting
$\chi(E^{\ssbullet})=(\chi E)^{\ssbullet}$ for every $E\in\Sigma$.
(ii) Show that $\chi$ is a homeomorphism between $\frak A$ and its image
if $\frak A$ is given its measure-algebra topology and $L^0(\mu)$ is
given its topology of convergence in measure (245A).
%323A

\sqheader 323Xb\dvAformerly{3{}23Xg} 
Let $(\frak A,\bar\mu)$ be a measure algebra and $\rho$
the measure metric on the ideal $\frak A^f$ of elements of finite
measure.   (i)
Show that the embedding $\frak A^f\subseteq\frak A$ is
uniformly continuous for
the measure-algebra uniformity on $\frak A$.   (ii) In the context of
323Xa, show that $\chi:\frak A^f\to L^0(\mu)$ is an isometry between
$\frak A^f$ and a subset of $L^1(\mu)$.
%323Xa, 323B

\spheader 323Xc
Let $(\frak A,\bar\mu)$ be a semi-finite measure algebra.   Show
that the set $\{(a,b):a\Bsubseteq b\}$ is a closed set in
$\frak A\times\frak A$.
%323B

\sqheader 323Xd Let $(X,\Sigma,\mu)$ be a $\sigma$-finite measure space
and $(\frak A,\bar\mu)$ its measure algebra.   (i) Show that if $\Tau$
is a $\sigma$-subalgebra of $\Sigma$, then $\{F^{\ssbullet}:F\in\Tau\}$
is a closed subalgebra of $\frak A$.   (ii) Show that if $\frak B$ is a
closed subalgebra of $\frak A$, then
$\{F:F\in\Sigma,\,F^{\ssbullet}\in\frak B\}$ is a $\sigma$-subalgebra of
$\Sigma$.
%323H 

\spheader 323Xe Let $(\frak A,\bar\mu)$ be a localizable measure
algebra, and $C\subseteq\frak A$ a set such that $\sup A$, $\inf A$
belong to $C$ for all non-empty subsets $A$ of $C$.   Show that $C$ is
topologically closed.
%323H

\spheader 323Xf Show that if $(\frak A,\bar\mu)$ is any measure
algebra and $\frak B$ is a subalgebra of $\frak A$, then its topological
closure $\overline{\frak B}$ is again a subalgebra.
%323J

\spheader 323Xg Let $(\frak A,\bar\mu)$ be a measure algebra, and
$e\in\frak A$;  let $\frak A_e$ be the principal ideal of $\frak A$
generated by $e$, and $\bar\mu_e$ its measure (322H).  (i) Show that the
measure-algebra 
topology on $\frak A_e$ defined by $\bar\mu_e$ is just the subspace
topology induced by the measure-algebra topology of $\frak A$.
(ii)\dvAnew{2011} Show that the
measure-algebra uniformity on $\frak A_e$ is the subspace
uniformity induced by the measure-algebra uniformity of $\frak A$.
(iii) Show that the strong
measure-algebra topology on $\frak A_e^f$ is the subspace
topology induced by the strong measure-algebra topology of $\frak A^f$.
%323L

\spheader 323Xh Let $(\frak A,\bar\mu)$ be a measure algebra.  Show that
its localization (322P) can be identified with its completion under its
measure-algebra uniformity.
%323C out of order query

\leader{323Y}{Further exercises (a)}
%\spheader 323Ya
Let $(\frak A,\bar\mu)$ be a $\sigma$-finite
measure algebra.   Show that a set $F\subseteq\frak A$ is
topologically closed iff $e\in F$ whenever there are non-empty sets
$B$, $C\subseteq \frak A$ such that $B$ is upwards-directed,
$C$ is downwards-directed, $\sup B=\inf C=e$ and
$[b,c]\cap F\ne\emptyset$ for every $b\in B$, $c\in C$, writing
$[b,c]=\{d:b\Bsubseteq d\Bsubseteq c\}$.
%323H

\spheader 323Yb Give an example to show that (a) is false for
general localizable measure algebras.   %mt32bits
%323Ya, 323H

\spheader 323Yc Give an example of a semi-finite measure algebra
$(\frak A,\bar\mu)$ with an order-closed subalgebra which is not
topologically closed.   %mt32bits
%323H

\spheader 323Yd Let $(\frak A,\bar\mu)$ be a probability algebra and
write $\Bbb B$ for the family of closed subalgebras of $\frak A$.   For
$\frak B$, $\frak C\in\Bbb B$ set
$\rho(\frak B,\frak C)
=\max(\sup_{b\in\frak B}\inf_{c\in\frak C}\bar\mu(b\Bsymmdiff c),
\sup_{c\in\frak C}\inf_{b\in\frak B}\bar\mu(b\Bsymmdiff c))$.   Show
that $(\Bbb B,\rho)$ is a complete metric space.   (Cf.\ 246Yb, 4A2T.)
%323+

\spheader 323Ye Let $(\frak A,\bar\mu)$ be the measure algebra of
Lebesgue measure on $\Bbb R$.   Show that it is separable in its
measure-algebra topology.   \Hint{245Yj.}

}%end of exercises

\cmmnt{
\Notesheader{323} The message of this section is that the topology
of a measure algebra is essentially defined by its order and algebraic
structure;  see also 324F-324H %324F 324G 324H
below.   Of course the results are really
about semi-finite measure algebras, and indeed this whole volume, like
the rest of measure theory, has little of interest to say about others;
they are included only because they arise occasionally and it is not
absolutely essential to exclude them.   We therefore expect to be able
to describe such things as closed subalgebras and continuous
homomorphisms in terms of the ordering, as in 323H and 324G.
For $\sigma$-finite algebras, indeed, there is an easy description of
the topology in terms of the order (323Ya).   I think the result of this
section on which I shall most often depend is 323H:  in most
contexts, there is no need to distinguish between `topologically
closed subalgebra' and `order-closed subalgebra'.
%323H not especially often quoted by name;  but phrase `closed subalgebra'
%very common.
However a $\sigma$-subalgebra of
a localizable measure algebra need not be topologically sequentially
closed;  there is an example in {\smc Fremlin Pagter \& Ricker 05}.

It is also the case that the topology of a measure algebra corresponds
very closely indeed to the topology of convergence in measure.   A
description of this correspondence is in 323Xa.  Indeed all the results
of this section have analogues in the theory of topological Riesz
spaces.   I will enlarge on the
idea here in \S367.   For the moment, however, if you look back to
Chapter 24, you will see that 323B and 323G are closely paralleled by
245D and 245E, while 323Ya is related to 245L.

It is I think natural to ask whether there are any other topological
Boolean algebras with the properties 323B-323D.   %323B 323C 323D
In fact there are;  see 393G and 393Xf below.
}%end of notes

\discrpage

