\frfilename{mt3a3.tex}
\versiondate{14.12.07}
\copyrightdate{2004}

\def\Bourbaki{{\smc Bourbaki 66}}
%\def\Bushaw{{\smc Bushaw 63}}
%\def\Christenson{{\smc Christenson \& Voxmann 77}}
\def\Dugundji{{\smc Dugundji 66}}
\def\Engelking{{\smc Engelking 89}}
\def\Gaal{{\smc Gaal 64}}
\def\James{{\smc James 87}}
%\def\Kelley{{\smc Kelley 55}}
\def\Schubert{{\smc Schubert 68}}

\def\chaptername{Appendix}
\def\sectionname{General topology}

\newsection{3A3}

In \S2A3, I looked at a selection of topics in general topology in some
detail, giving proofs;  the point was that an ordinary elementary course
in the subject would surely go far beyond what we needed there, and at
the same time might omit some of the results I wished to quote.
It seemed therefore worth taking a bit of space to cover the requisite
material, giving readers the option of delaying a proper study of the
subject until a convenient opportunity arose.   In the context of the
present volume, this approach is probably no longer appropriate, since
we need a much greater proportion of the fundamental ideas, and by the
time you have reached familiarity with the topics here you will be
well able to find your way about one of the many excellent textbooks on
the subject.   This time round, therefore, I give most of the results
without proofs (as in \S\S2A1 and 3A1), hoping that some of the
references I
offer will be accessible in all senses.   I do, however, give a full set
of definitions, partly to avoid ambiguity (since even in this relatively
mature subject, there are some awkward divergences remaining in the
usage of different authors), and partly because many of the proofs are
easy enough for even a novice to fill in with a bit of thought, once the
meaning of the words is clear.   \cmmnt{In fact this happens so often
that I will mark with a $*$ those points where a proof needs an idea not
implicit in the preceding work.}

\leader{3A3A}{Taxonomy of topological \dvrocolon{spaces}}\cmmnt{ I
begin with the handful of definitions we need in order to classify the
different types of topological space used in this volume.   A couple
have already been introduced in Volume 2, but I repeat them because the
list would look so odd without them.

\medskip

\noindent}{\bf Definitions} Let $(X,\frak T)$ be a topological space.

\spheader 3A3Aa $X$ is {\bf T$_{\pmb{1}}$} if singleton subsets of $X$ are
closed.

\spheader 3A3Ab $X$ is {\bf Hausdorff} if for any
distinct points $x$,
$y\in X$ there are disjoint open sets $G$, $H\subseteq X$ such that
$x\in G$ and $y\in H$.

\spheader 3A3Ac $X$ is {\bf regular} if whenever $F\subseteq X$ is
closed and $x\in X\setminus F$ there are disjoint open sets $G$,
$H\subseteq X$ such that $x\in G$ and $F\subseteq H$.   \cmmnt{(Note
that in this definition I do not require $X$ to be Hausdorff, following
\James\leaveitout{ and \Christenson} but not \Engelking, \Bourbaki,
\Dugundji,
\Schubert\ or \Gaal\leaveitout{ or \Bushaw}.)}

\spheader 3A3Ad $X$ is {\bf completely regular} if whenever
$F\subseteq X$ is closed and $x\in X\setminus F$ there is a
continuous function
$f:X\to[0,1]$ such that $f(x)=1$ and $f(y)=0$ for every $y\in F$.
\cmmnt{(Note that many authors restrict the phrase `completely
regular' to Hausdorff spaces.)}

\spheader 3A3Ae $X$ is {\bf zero-dimensional} if whenever $G\subseteq X$
is an open set and $x\in G$ then there is an open-and-closed set $H$
such that $x\in H\subseteq G$.

\spheader 3A3Af $X$ is {\bf extremally disconnected} if the closure of
every open set in $X$ is open.

\spheader 3A3Ag $X$ is {\bf compact} if every open cover of $X$
has a finite subcover.

\spheader 3A3Ah $X$ is {\bf locally compact} if for every $x\in X$ there
is a set $K\subseteq X$ such that $x\in\interior K$ and $K$ is
compact\cmmnt{ (in its subspace topology, as defined in 2A3C)}.

\spheader 3A3Ai If every subset of $X$ is open, we call $\frak T$ the
{\bf discrete topology} on $X$.

\leader{3A3B}{Elementary relationships (a)} A completely regular space
is regular.  \prooflet{(\Engelking, p.\ 39;  \Dugundji, p.\ 154;
\Schubert, p.\ 104.)}

\spheader 3A3Bb A locally compact Hausdorff space is completely
regular\cmmnt{, therefore regular}.   \cmmnt{$*$}
\prooflet{(\Engelking,
3.3.1;  \Dugundji, p.\ 238;  \Gaal, p.\ 149\leaveitout{;  Christenson,
p.\ 123}.)}

\spheader 3A3Bc A compact Hausdorff space is locally compact\cmmnt{,
therefore completely regular and regular}.

\spheader 3A3Bd A regular extremally disconnected space is
zero-dimensional.  \prooflet{(\Engelking, 6.2.25.)}

\spheader 3A3Be
Any topology defined by pseudometrics\cmmnt{ (2A3F), in particular
the weak topology of a normed space (2A5I),} is completely
regular\cmmnt{, therefore regular}.  \prooflet{(\Bourbaki, IX.1.5;
\Dugundji, p.\ 200.)}

\spheader 3A3Bf If $X$ is a completely regular Hausdorff
space\cmmnt{ (in particular, if $X$ is (locally) compact and
Hausdorff),} and $x$, $y$ are distinct points in $X$, then there is a
continuous function $f:X\to\Bbb R$ such that $f(x)\ne f(y)$.
\prooflet{(Apply 3A3Ad with $F=\{y\}$, which is closed because $X$ is
Hausdorff.)}

\spheader 3A3Bg An open set in a locally compact Hausdorff space is
locally compact in its subspace topology.
\prooflet{(\Engelking, 3.3.8;  \Bourbaki, I.9.7.)}

\leader{3A3C}{Continuous functions} Let $(X,\frak T)$ and $(Y,\frak S)$
be topological spaces.

\spheader 3A3Ca If $f:X\to Y$ is a function and $x\in X$, we say that
$f$ is {\bf continuous at $x$} if $x\in\interior f^{-1}[H]$ whenever
$H\subseteq Y$ is an open set containing $f(x)$.

\spheader 3A3Cb Now a function from $X$ to $Y$ is continuous iff it is
continuous at every point of $X$.  \prooflet{(\Bourbaki, I.2.1;
\Dugundji, p.\ 80;  \Schubert, p.\ 24;  \Gaal, p.\ 183;
\James, p.\ 26\leaveitout{;  \Christenson, p.\ 15}.)}

\spheader 3A3Cc If $f:X\to Y$ is continuous at $x\in X$, and
$A\subseteq X$ is such that $x\in\overline{A}$, then
$f(x)\in\overline{f[A]}$.
\prooflet{(\Bourbaki, I.2.1;  \Schubert, p.\ 23.)}

\spheader 3A3Cd If $f:X\to Y$ is continuous, then
$f[\overline{A}]\subseteq\overline{f[A]}$ for every $A\subseteq X$.
\prooflet{(\Engelking, 1.4.1;  \Bourbaki, I.2.1;  \Dugundji, p.\ 80;
\Schubert, p.\ 24;  \Gaal, p.\ 184;  \James, p.\ 27\leaveitout{;
\Christenson, p.\ 31;  \Bushaw, p.\ 38}.)}

\spheader 3A3Ce A function $f:X\to Y$ is a {\bf homeomorphism} if it is
a continuous bijection and its inverse is also
continuous\cmmnt{;  that is, if $\frak S=\{f[G]:G\in\frak T\}$ and
$\frak T=\{f^{-1}[H]:H\in\frak S\}$}.

\spheader 3A3Cf A function $f:X\to[-\infty,\infty]$ is
{\bf lower semi-continuous} if $\{x:x\in X,\,f(x)>\alpha\}$ is open for
every $\alpha\in\Bbb R$.  \cmmnt{(Cf.\ 225H.)}

\leader{3A3D}{Compact spaces} \cmmnt{Any extended series of
applications of general topology
is likely to involve some new features of compactness.   I start with
the easy bits, continuing from 2A3Nb.

\medskip

}{\bf (a)}\cmmnt{ The first is just a definition of
compactness in terms of closed sets instead of open sets.}    A family
$\Cal F$ of sets has the {\bf finite intersection property} if
$\bigcap\Cal F_0$ is non-empty for every finite
$\Cal F_0\subseteq\Cal F$.   Now a topological space $X$ is compact iff
$\bigcap\Cal F\ne\emptyset$ whenever $\Cal F$ is a family of closed subsets of $X$
with the finite intersection property.   \prooflet{(\Engelking, 3.1.1;
\Bourbaki, I.9.1;  \Dugundji, p.,\ 223;  \Schubert, p.\ 68;  \Gaal, p.\
127\leaveitout{;  \Christenson, p.\ 67;  \Bushaw, p.\ 77}.)}

\spheader 3A3Db\cmmnt{ A marginal generalization of this is the
following.}  Let $X$ be a topological space and $\Cal F$ a family of
closed subsets of $X$ with the finite intersection property.   If
$\Cal F$ contains a compact set then $\bigcap\Cal F\ne\emptyset$.
\prooflet{(Apply (a) to $\{K\cap F:F\in\Cal F\}$ where $K\in\Cal F$ is
compact.)}

\spheader 3A3Dc In a Hausdorff space, compact subsets are closed.
\prooflet{(\Engelking, 3.1.8;  \Bourbaki, I.9.4;  \Dugundji, p.\ 226;
\Schubert, p.\ 70;  \Gaal, p.\ 138;  \James, p.\ 77\leaveitout{;
\Bushaw, p.\ 73}.)}

\spheader 3A3Dd If $X$ is compact, $Y$ is Hausdorff and $\phi:X\to Y$ is
continuous and injective, then $\phi$ is a homeomorphism between $X$ and
$\phi[X]$\cmmnt{ (where $\phi[X]$ is given the subspace topology)}.
\prooflet{(\Engelking, 3.1.12;  \Bourbaki, I.9.4;
\Dugundji, p.\ 226;  \Schubert, p.\ 71;  \Gaal, p.\ 207\leaveitout{;
\Bushaw, p.\ 73}.)}

\spheader 3A3De Let $X$ be a regular topological space and $A$ a subset
of $X$.   Then the following are equiveridical:  (i) $A$ is relatively
compact in $X$\cmmnt{ (that is, $A$ is included in some compact subset
of $X$, as in 2A3Na)};  (ii) $\overline{A}$ is compact;  (iii) every
ultrafilter on $X$ which contains $A$ has a limit in $X$.
%better would be:  every filter containing A has a cluster point
\prooflet{\Prf (ii)$\Rightarrow$(i) is trivial, and
(i)$\Rightarrow$(iii) is a consequence of 2A3R;  neither of these
requires $X$ to be regular.
Now assume (iii) and let $\Cal F$ be an ultrafilter on $X$ containing
$\overline{A}$.   Set

\Centerline{$\Cal H=\{B:B\subseteq X$, there is an open set $G\in\Cal F$
such that $A\cap G\subseteq B\}$.}

\noindent Then $\Cal H$ does not contain $\emptyset$ and
$B_1\cap B_2\in\Cal H$ whenever $B_1$, $B_2\in\Cal H$, so $\Cal H$ is a
filter on $X$,
and it contains $A$.   Let $\Cal H^*\supseteq\Cal H$ be an ultrafilter
(2A1O).   By hypothesis, $\Cal H^*$ has a limit $x$ say.   Because
$A\in\Cal H^*$, $X\setminus\overline{A}$ is an open set not belonging to
$\Cal H^*$, and cannot be a neighbourhood of $x$;  thus $x$ must belong
to $\overline{A}$.   Let $G$ be an open set containing $x$.   Then there
is an open set $H$ such that $x\in H\subseteq\overline{H}\subseteq G$
(this is where I use the hypothesis that $X$ is regular).   Because
$\Cal H^*\to x$, $H\in\Cal H^*$ so $X\setminus\overline{H}$ does not
belong to $\Cal H^*$ and therefore does not belong to $\Cal H$.   But
$X\setminus\overline{H}$ is open, so by the definition of $\Cal H$ it
cannot belong to $\Cal F$.   As $\Cal F$ is an ultrafilter,
$\overline{H}\in\Cal F$ and $G\in\Cal F$.   As $G$ is arbitrary,
$\Cal F\to x$.   As $\Cal F$ is arbitrary, $\overline{A}$ is compact
(2A3R).   Thus (iii)$\Rightarrow$(ii).\ \Qed
}%end of prooflet

\leader{3A3E}{Dense sets} \cmmnt{Recall that a set $D$ in a
topological space $X$ is {\bf dense} if $\overline{D}=X$, and that $X$
is {\bf separable} if it has a countable dense subset (2A3Ud).

\medskip

}{\bf (a)} If $X$ is a topological space, $D\subseteq X$ is
dense and $G\subseteq X$ is dense and open, then $G\cap D$ is dense.
\prooflet{(\Engelking, 1.3.6.)}  Consequently the intersection of
finitely many dense open sets is always dense.

\spheader 3A3Eb If $X$ and $Y$ are topological spaces,
$D\subseteq A\subseteq X$, $D$ is dense in $A$ and
$f:X\to Y$ is a continuous
function, then $f[D]$ is dense in $f[A]$.   \prooflet{(Use 3A3Cd.)}

\vleader{48pt}{3A3F}{Meager sets} Let $X$ be a topological space.

\spheader 3A3Fa A set $A\subseteq X$ is {\bf nowhere dense}
if $\interior\overline{A}=\emptyset$\cmmnt{, that is,
$\interior(X\setminus A)=X\setminus\overline{A}$ is dense, that
is, for every non-empty open set $G$ there is a non-empty open set
$H\subseteq G\setminus A$}.

\spheader 3A3Fb A set $M\subseteq X$ is {\bf meager}
if it is expressible as the union of a sequence of nowhere
dense sets.   A subset of $X$ is {\bf comeager} if its complement is
meager.

\spheader 3A3Fc Any subset of a nowhere dense set is nowhere dense;  the
union of finitely many nowhere dense sets is nowhere dense.
\prooflet{(3A3Ea.)}
%\Bourbaki, IX.5.1;  \Gaal, p.\ 287.

\spheader 3A3Fd Any subset of a meager set is meager;  the union of
countably many meager sets is meager.   \prooflet{(314L.)}

\leader{3A3G}{Baire's theorem for locally compact Hausdorff spaces} Let
$X$ be a locally compact Hausdorff space and $\sequencen{G_n}$ a
sequence of dense open
subsets of $X$.   Then $\bigcap_{n\in\Bbb N}G_n$ is dense.
\cmmnt{$*$} \prooflet{(\Engelking, 3.9.4;  \Bourbaki, IX.5.3;
\Dugundji, p.\ 249;  \Schubert, p.\ 148.)}
Consequently every comeager subset of $X$ is dense.

\leader{3A3H}{Corollary} (a) Let $X$ be a compact Hausdorff space.
Then a non-empty open subset of $X$ cannot be meager.
\prooflet{(\Dugundji, p.\ 250;  \Schubert, p.\ 147.)}

(b) Let $X$ be a non-empty locally compact Hausdorff space.   If
$\sequencen{A_n}$ is a sequence of sets covering $X$, then there is some
$n\in\Bbb N$ such that $\interior\overline{A}_n$ is non-empty.
\prooflet{(\Dugundji, p.\ 250.)}

\leader{3A3I}{Product spaces (a) Definition} Let $\familyiI{X_i}$
be a family of topological spaces, and
$X=\prod_{i\in I}X_i$ their Cartesian product.   We say that a set
$G\subseteq X$ is open for the {\bf product topology} if for every $x\in
G$ there are a finite $J\subseteq I$ and a family $\langle
G_j\rangle_{j\in J}$ such that every $G_j$ is an open set in the
corresponding $X_j$ and

\Centerline{$\{y:y\in X,\,y(j)\in G_j$ for every $j\in J\}$}

\noindent contains $x$ and is included in $G$.

\cmmnt{(Of course we must check that this does indeed define a
topology;  see \Engelking, 2.3.1;  \Bourbaki, I.4.1;  \Schubert, p.\
38;  \Gaal, p.\ 144\leaveitout{;  \Christenson, p.\ 162;  \Bushaw, p.\
56}.)}

\spheader 3A3Ib If $\familyiI{X_i}$ is a family of
topological spaces, with product $X$, and $Y$ another topological
space, a function $\phi:Y\to X$ is continuous iff $\pi_i\phi$ is
continuous for every $i\in I$, where $\pi_i(x)=x(i)$ for $x\in X$ and
$i\in I$.   \prooflet{(\Engelking, 2.3.6;  \Bourbaki, I.4.1;  \Dugundji,
p.\ 101;  \Schubert, p.\ 62;  \James, p.\ 31.)}

\spheader 3A3Ic Let $\familyiI{X_i}$ be any family of
non-empty topological spaces, with product $X$.   If $\Cal F$ is a
filter on $X$ and $x\in X$, then $\Cal F\to x$ iff $\pi_i[[\Cal F]]\to
x(i)$ for every $i$, where $\pi_i(y)=y(i)$ for $y\in X$\cmmnt{, and
$\pi_i[[\Cal F]]$ is the image filter on $X_i$ (2A1Ib)}.
\prooflet{(\Bourbaki, I.7.6;  \Schubert, p.\ 61;  \James, p.\ 32.}

\spheader 3A3Id The product of any family of Hausdorff spaces is
Hausdorff.   \prooflet{(\Engelking, 2.3.11;  \Bourbaki, I.8.2;
\Schubert, p.\ 62;  \James, p.\ 87.)}

\spheader 3A3Ie Let $\familyiI{X_i}$ be any family of
topological spaces.   If $D_i$ is a dense subset of $X_i$ for each $i$,
then $\prod_{i\in I}D_i$ is dense in $\prod_{i\in I}X_i$.
\prooflet{(\Engelking, 2.3.5.)}.

\spheader 3A3If Let $\familyiI{X_i}$ be any family of
topological spaces.   If $F_i$ is a closed subset of $X_i$ for each $i$,
then $\prod_{i\in I}F_i$ is closed in $\prod_{i\in I}X_i$.
\prooflet{(\Engelking, 2.3.4;  \Bourbaki, I.4.3.)}

\spheader 3A3Ig Let $\familyiI{(X_i,\frak T_i)}$ be a family of
topological spaces with product $(X,\frak T)$.   Suppose that each
$\frak T_i$ is defined by a family $\Rho_i$ of pseudometrics on
$X_i$\cmmnt{ (2A3F)}.   Then $\frak T$ is defined by the family
$\Rho=\{\tilde\rho_i:i\in I,\rho\in\Rho_i\}$ of pseudometrics on $X$,
where I write $\tilde\rho_i(x,y)=\rho(\pi_i(x),\pi_i(y))$ whenever
$i\in I$, $\rho\in\Rho_i$ and $x$, $y\in X$\cmmnt{, taking $\pi_i$ to
be the coordinate map from $X$ to $X_i$, as in (b)-(c)}.
\prooflet{\Prf\ \cmmnt{(Compare 2A3Tb).}   (i) It is easy to check
that every $\tilde\rho_i$ is a pseudometric on $X$.   Write
$\frak T_{\Rho}$ for the topology generated by $\Rho$.   (ii) If
$x\in G\in\frak T_{\Rho}$, let $\Rho'\subseteq\Rho$ and $\delta>0$ be
such that $\Rho'$
is finite and $\{y:\tau(y,x)\le\delta$ for every $\tau\in\Rho'\}$ is
included in $G$.   Express $\Rho'$ as
$\{\tilde\rho_j:j\in J,\,\rho\in\Rho'_j\}$ where $J\subseteq I$ is
finite and $\Rho'_j\subseteq\Rho_j$ is finite for each $j\in J$.   Set

\Centerline{$G_j=\{t:t\in X_j,\,\rho(t,\pi_j(x))<\delta$ for every
$\rho\in\Rho'_j\}$}

\noindent for every $j\in J$.   Then $G'=\{y:\pi_j(y)\in G_j$ for every
$j\in J\}$ contains $x$, belongs to $\frak T$ and is included in $G$.
As $x$ is arbitrary, $G\in\frak T$;  as $G$ is arbitrary,
$\frak T_{\Rho}\subseteq\frak T$.   (iii) Every $\pi_i$ is
$(\frak T_{\Rho},\frak T_i)$-continuous, by 2A3H;  by (b) above, the
identity
map from $X$ to itself is $(\frak T_{\Rho},\frak T)$-continuous, that
is, $\frak T_{\Rho}\subseteq\frak T$ and $\frak T_{\Rho}=\frak T$, as
claimed.\ \Qed
}%end of prooflet

\spheader 3A3Ih Let $\familyiI{X_i}$ be a family of
topological spaces with product $X$, and $Y$ another topological space.
Then a function $f:X\to Y$ is {\bf separately continuous} if for every
$j\in I$ and $z\in\prod_{i\in I\setminus\{j\}}X_i$ the function
$t\mapsto f(z^{\smallfrown}\fraction{t}):X_j\to Y$ is continuous, where
$z^{\smallfrown}\fraction{t}$ is
the member of $X$ extending $z$ and such that
$(z^{\smallfrown}\fraction{t})(j)=t$.

\leader{3A3J}{Tychonoff's theorem} The product of any family of compact
topological spaces is compact.

\proof{\Engelking, 3.2.4;  \Bourbaki, I.9.5;  \Dugundji, p.\ 224;
\Schubert, p.\ 72;   \Gaal, p.\ 146 and p.\ 272;  \James, p.\
67\leaveitout{; \Christenson, p.\ 164;  \Bushaw, p.\ 79}.
}%end of proof of 3A3J

\leader{3A3K}{The spaces $\{0,1{\delimiter"5267309}^I$, $\BbbR^I$} For any set $I$, we
can think of $\{0,1\}^I$ as the product $\prod_{i\in I}X_i$ where
$X_i=\{0,1\}$ for each $i$.   If we endow each $X_i$ with its discrete
topology, the product topology is the {\bf usual topology} on
$\{0,1\}^I$.   Being a product of Hausdorff spaces, it is Hausdorff;  by
Tychonoff's theorem, it is compact.
A subset $G$ of $\{0,1\}^I$ is open iff for every $x\in G$ there is a
finite $J\subseteq I$ such that
$\{y:y\in\{0,1\}^I,\,y\restr J=x\restr J\}\subseteq G$.

Similarly, the `usual topology' of $\BbbR^I$ is the product topology
when each factor is given its Euclidean topology\cmmnt{ (cf.\
2A3Tc)}.

\leader{3A3L}{Cluster points of filters (a)} Let $X$ be a topological
space and $\Cal F$ a filter on $X$.   A point $x$ of $X$ is a {\bf
cluster point} of $\Cal F$ if $x\in\overline{A}$ for every $A\in\Cal F$.

\spheader 3A3Lb For any topological space $X$, filter $\Cal F$ on $X$
and $x\in X$, $x$ is a cluster point
of $\Cal F$ iff there is a filter $\Cal G\supseteq\Cal F$ such that
$\Cal G\to x$.  \prooflet{(\Engelking, 1.6.8;  \Bourbaki, I.7.2;  \Gaal,
p.\ 260;  \James, p.\ 22\leaveitout{;  \Christenson, p.\ 286}.)}

\leaveitout{
\spheader 3A3Lc For any topological space $X$, filter $\Cal F$ on $X$
and $x\in X$, the following are equiveridical:  (i) $\Cal F\to x$;  (ii)
$x$ is a cluster point of every filter $\Cal G\supseteq\Cal F$;  (iii)
$\Cal G\to x$ for every ultrafilter $\Cal G\supseteq\Cal F$.
\prooflet{\Prf (i)$\Rightarrow$(iii) is elementary, and
(iii)$\Rightarrow$(ii) is a consequence of (b) and the Ultrafilter
Theorem.   Now assume (ii).   \Quer\ If $\Cal F\not\to x$, there is an
open set $G$ containing $x$ and not belonging to $\Cal F$.   Set $\Cal
G=\{B:\,\exists\, F\in\Cal F,\,H\supseteq F\setminus G\}$.   Then $\Cal
G$ is a filter including $\Cal F$ and $X\setminus G$ is a closed set,
belonging to $\Cal G$, and not containing $x$, so $x$ is not a cluster
point of $\Cal G$.\ \Bang\  So (i) is true.   Compare \Dugundji, p.\
214.\ \Qed
}%end of prooflet
}%end of leaveitout

\spheader 3A3Lc If $\sequencen{\alpha_n}$ is a sequence
in $\Bbb R$, $\alpha\in\Bbb R$ and $\lim_{n\to\Cal H}\alpha_n=\alpha$
for every non-principal ultrafilter
$\Cal H$ on $\Bbb N$\cmmnt{ (definition:  2A3Sb)}, then
$\lim_{n\to\infty}\alpha_n=\alpha$.
\prooflet{\Prf\Quer\ If $\sequencen{\alpha_n}\not\to\alpha$, there is
some $\epsilon>0$ such that
$I=\{n:|\alpha_n-\alpha|\ge\epsilon\}$ is infinite.   Now $\Cal
F_0=\{F:F\subseteq\Bbb N,\,I\setminus F$ is finite$\}$ is a filter on
$\Bbb N$, so there is an ultrafilter $\Cal F\supseteq\Cal F_0$.   But
now $\alpha$ cannot be $\lim_{n\to\Cal F}\alpha_n$.\ \Bang\Qed}

\leader{3A3M}{Topology bases (a)} If $X$ is a set and $\Bbb T$ is any
non-empty family of topologies on $X$, $\bigcap\Bbb T$ is a topology on
$X$.   So if $\Cal A$ is any
family of subsets of $X$, the intersection of all the topologies on $X$
including $\Cal A$ is a topology on $X$;  this is the
{\bf topology generated by $\Cal A$}.

\header{3A3Mb}{\bf (b)} If $X$ is a set and $\frak T$ is a topology on
$X$, a {\bf base} for $\frak T$ is a set $\Cal U\subseteq\frak T$ such
that whenever $x\in G\in\frak T$ there is a $U\in\Cal U$ such that
$x\in U\subseteq G$\cmmnt{;  that is, such that
$\frak T=\{\bigcup\Cal G:\Cal G\subseteq\Cal U\}$}.
In this case\cmmnt{, of course,} $\Cal U$ generates $\frak T$.

\header{3A3Mc}{\bf (c)} If $X$ is a set and $\Cal E$ is a family of
subsets of $X$, then $\Cal E$ is a base for a topology on $X$ iff (i)
whenever $E_1$, $E_2\in\Cal E$ and $x\in E_1\cap E_2$ then there is an
$E\in\Cal E$ such that $x\in E\subseteq E_1\cap E_2$ (ii) $\bigcup\Cal
E=X$.  \prooflet{(\Engelking, p.\ 12\leaveitout{;  \Christenson, p.\ 16;
\Bushaw, p.\ 47}.)}

\leader{3A3N}{Uniform convergence (a)} Let $X$ be a set, $(Y,\rho)$ a
metric space and
$\sequencen{f_n}$ a sequence of functions from $X$ to $Y$.   We say
that $\sequencen{f_n}$ converges {\bf uniformly} to a function
$f:X\to Y$ if for every $\epsilon>0$ there is an $n_0\in\Bbb N$ such
that $\rho(f_n(x),f(x))\le\epsilon$ whenever $n\ge n_0$ and $x\in X$.

\spheader 3A3Nb Let $X$ be a topological space and $(Y,\rho)$ a metric
space.   Suppose that $\sequencen{f_n}$ is a
sequence of continuous functions from $X$ to $Y$ converging uniformly
to $f:X\to Y$.   Then $f$ is continuous.  \cmmnt{$*$}
\prooflet{(\Engelking, 1.4.7/4.2.19;  \Gaal, p.\ 202.)}

\leader{3A3O}{One-point compactifications} Let $(X,\frak T)$ be a
locally compact Hausdorff space.   Take any object $x_{\infty}$ not
belonging to $X$ and set $X^*=X\cup\{x_{\infty}\}$.   Let $\frak T^*$ be
the family of those sets $H\subseteq X^*$ such that $H\cap X\in\frak T$
and either $x_{\infty}\notin H$ or $X\setminus H$ is compact (for
$\frak T$).   Then $\frak T^*$ is the unique compact Hausdorff topology on
$X^*$ inducing $\frak T$ as the subspace topology on $X$;
$(X^*,\frak T^*)$ is the {\bf one-point compactification}
of $(X,\frak T)$.   \prooflet{(\Engelking, 3.5.11;
\Bourbaki, I.9.8;  \Dugundji, p.\ 246.)}

\leader{3A3P}{Topologies defined from a sequential convergence:
Proposition}
(a) Let $X$ be a set and $\to^*$ a relation between $X^{\Bbb N}$ and $X$
such that whenever $\sequencen{x_n}\in X^{\Bbb N}$, $x\in X$,
$\sequencen{x_n}\to^*x$ and $\sequencen{x'_n}\in X^{\Bbb N}$ is a
subsequence of $\sequencen{x_n}$ then $\sequencen{x'_n}\to^*x$.
Then there is a unique topology on $X$ for which a set
$F\subseteq X$ is closed iff $x\in F$ whenever $\sequencen{x_n}$ is a
sequence in $F$ and $\sequencen{x_n}\to^*x$.   Moreover, if
$\sequencen{x_n}\to^*x$ then $\sequencen{x_n}$ converges to $x$ for this
topology.

(b) Let $X$ and $Y$ be sets, and suppose that
$\to^*_X\mskip5mu\subseteq X^{\Bbb N}\times X$,
$\to^*_Y\mskip5mu\subseteq Y^{\Bbb N}\times Y$ are relations with the
subsequence property described in (a).   Give $X$ and $Y$ the corresponding
topologies.   If $f:X\to Y$ is a function such that
$\sequencen{f(x_n)}\to^*_Yf(x)$ whenever
$\sequencen{x_n}\to_X^*x$, then $f$ is continuous.

\proof{{\bf (a)(i)} Let $\Cal F$ be the family of those
$F\subseteq X$ which are closed under *-convergence, that is, such that
$x\in F$ whenever $\sequencen{x_n}$ is a sequence in $F$ and
$\sequencen{x_n}\to^*x$.   Of course $\emptyset$ and
$X$ belong to $\Cal F$;  also the intersection of any non-empty subset
of $\Cal F$ belongs to $\Cal F$.
The point is that the union of two members of $\Cal F$ belongs to
$\Cal F$.   \Prf\ Suppose that $F_1$, $F_2\in\Cal F$
and that $\sequencen{x_n}$ is a sequence in $F_1\cup F_2$
*-converging to $x$.   Then there
is a subsequence $\sequencen{x'_n}$ of $\sequencen{x_n}$ which lies
entirely in one of the sets;  say $x'_n\in F_j$ for
every $n$.   By hypothesis, $\sequencen{x'_n}\to^*x$, so
$x\in F_j\subseteq F_1\cup F_2$.   As
$\sequencen{x_n}$ and $x$ are arbitrary, $F_1\cup F_2\in\Cal F$.\ \QeD\
Taking complements, we see that
$\{X\setminus F:F\in\Cal F\}$ is a topology on $X$ for which $\Cal F$ is
the family of closed sets;  and of course
there can be only one such topology.

\medskip

\quad{\bf (ii)} Now suppose that $\sequencen{x_n}\to^*x$.   \Quer\ If
$\sequencen{x_n}$ does not converge topologically to $x$, then there is
an open set $G$ containing $x$ such that $\{n:x_n\notin G\}$ is
infinite, that is, there is a subsequence $\sequencen{x'_n}$ of
$\sequencen{x_n}$ such that $x'_n\notin G$ for every $n$.   Now
$\sequencen{x'_n}\to^*x$;  but $X\setminus G$ is supposed to be closed
under *-convergence, so this is impossible.\ \Bang

\medskip

{\bf (b)} Let $H\subseteq Y$ be open;  set $F=Y\setminus H$ and
let $\sequencen{x_n}$ be a sequence in $f^{-1}[F]$ such that
$\sequencen{x_n}\to^*_Xx$.   Then $\sequencen{f(x_n)}\to^*_Yf(x)$, so
$f(x)\in F$ and $x\in f^{-1}[F]$,   As $\sequencen{x_n}$ and $x$ are
arbitrary, $f^{-1}[F]$ is closed and $f^{-1}[H]$ is open;  as $H$ is
arbitrary, $f$ is continuous.
}%end of proof of 3A3P

\leader{3A3Q}{Miscellaneous definitions} Let $X$ be a topological space.

\spheader 3A3Qa A subset of $X$ is a {\bf zero set} if it is of the
form $f^{-1}[\{0\}]$ for some continuous function $f:X\to\Bbb R$.   A
subset of $X$ is a {\bf cozero set} if its complement is a zero set.
A subset of $X$ is a {\bf G$_{\delta}$ set}
if it is expressible as the intersection of a sequence of open sets.

\spheader 3A3Qb An {\bf isolated point} of $X$ is a point $x\in X$ such
that the singleton set $\{x\}$ is open.

\discrpage

