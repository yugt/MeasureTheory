\frfilename{mt351.tex}
\versiondate{16.10.07}
\copyrightdate{1996}

\def\chaptername{Riesz spaces}
\def\sectionname{Partially ordered linear spaces}
\def\pols{partially ordered linear space}
\def\polss{partially ordered linear spaces}

\newsection{351}

I begin with an account of the most basic structures which involve an
order relation on a linear space, \polss.   As often in this volume, I
find myself impelled to do some of the work in very much greater
generality than is strictly required, in order to show more clearly the
nature of the arguments being used.   I give the definition (351A) and
most elementary properties (351B-351L) of \polss;  then I describe a
general representation theorem for arbitrary \polss\ as subspaces of
reduced powers of $\Bbb R$ (351M-351Q).   I end with a brief note on
Archimedean \polss\ (351R).


\leader{351A}{Definition}\cmmnt{ I repeat a definition mentioned in
241E.}   A {\bf \pols} is a linear space $(U,+,\cdot)$ over
$\Bbb R$ together with a partial order $\le$ on $U$ such that

\Centerline{$u\le v\Longrightarrow u+w\le v+w$,}

\Centerline{$u\ge 0$, $\alpha\ge 0\Longrightarrow \alpha u\ge 0$}

\noindent for $u$, $v$, $w\in U$ and $\alpha\in\Bbb R$.

\leader{351B}{Elementary facts} Let $U$ be a \pols.  \cmmnt{ We have
the following elementary consequences of the definition above,
corresponding to the familiar rules for manipulating inequalities among
real numbers.}

\spheader 351Ba For $u$, $v\in U$,

\dvro{\Centerline{$u\le v\iff 0\le v-u\iff -v\le-u$.}}
{\Centerline{$u\le v\Longrightarrow 0=u+(-u)\le v+(-u)=v-u
\Longrightarrow u=0+u\le v-u+u=v$,}

\Centerline{$u\le v\Longrightarrow -v=u+(-v-u)\le v+(-v-u)=-u$.}}

\spheader 351Bb Suppose that $u$, $v\in U$ and $u\le v$.   Then
$\alpha u\le\alpha v$ for every $\alpha\ge 0$ and $\alpha v\le\alpha u$ for
every $\alpha\le 0$.   \prooflet{\Prf\ (i) If $\alpha\ge 0$, then
$\alpha(v-u)\ge 0$ so $\alpha v\ge\alpha u$.   (ii) If $\alpha\le 0$
then $(-\alpha)u\le(-\alpha)v$ so

\Centerline{$\alpha v=-(-\alpha)v\le -(-\alpha u)=u$.   \Qed}
}

\spheader 351Bc If $u\ge 0$ and $\alpha\le\beta$ in $\Bbb R$,
then\cmmnt{ $(\beta-\alpha)u\ge 0$, so} $\alpha u\le\beta u$.
If $0\le u\le v$ in $U$ and $0\le\alpha\le\beta$ in $\Bbb R$, then
$\alpha u\cmmnt{\mskip5mu\le\beta u}\le\beta v$.

\leader{351C}{Positive cones} Let $U$ be a \pols.

\spheader 351Ca I will write $U^+$ for the {\bf positive cone} of $U$,
the set $\{u:u\in U,\,u\ge 0\}$.

\spheader 351Cb \cmmnt{By 351Ba, the ordering is determined by the
positive cone $U^+$, in the sense that}
$u\le v\iff v-u\in U^+$.

\spheader 351Cc \cmmnt{It is easy to characterize positive cones.}
If $U$ is a real linear space,  a set $C\subseteq U$ is the positive
cone for some ordering rendering $U$ a \pols\ iff

\Centerline{$u+v\in C$,
\quad$\alpha u\in C$ whenever $u$, $v\in C$ and $\alpha\ge 0$,}

\Centerline{$0\in C$,\quad $u\in C\,\&\,-u\in C\Longrightarrow u=0$.}

\prooflet{\noindent \Prf\ (i) If $C=U^+$ for some \pols\ ordering $\le$
of $U$, then

\Centerline{$u$, $v\in C\Longrightarrow 0\le u\le u+v\Longrightarrow u+v\in C$,}

\Centerline{$u\in C$, $\alpha\ge 0\Longrightarrow \alpha u\ge 0$, i.e.,
$\alpha u\in C$,}

\Centerline{$0\le 0$ so $0\in C$,}

\Centerline{$u$, $-u\in C\Longrightarrow u=0+u\le(-u)+u=0\le u\Longrightarrow
u=0$.}

\noindent (ii) On the other hand, if $C$ satisfies the conditions,
define the relation $\le$ by writing $u\le v\iff v-u\in C$;  then

\Centerline{$u-u=0\in C$ so $u\le u$ for every $u\in U$,}

\Centerline{if $u\le v$ and $v\le w$ then $w-u=(w-v)+(v-u)\in C$ so
$u\le w$,}

\Centerline{if $u\le v$ and $v\le u$ then $u-v$, $v-u\in C$ so $u-v=0$
and $u=v$}

\noindent and $\le$ is a partial order;  moreover,

\Centerline{if $u\le v$ and $w\in U$ then $(v+w)-(u+w)=v-u\in C$ and
$u+w\le v+w$,}

\Centerline{if $u$, $\alpha\ge 0$ then $\alpha u\in C$ and $\alpha u\ge
0$,}

\Centerline{$u\ge 0\iff u\in C$.}

\noindent So $\le$ makes $U$ a \pols\ in which $C$ is the positive
cone.\ \Qed}

\spheader 351Cd \cmmnt{An incidental useful fact.}  Let $U$ be a
\pols, and $u\in U$.   Then $u\ge 0$ iff $u\ge -u$.   \prooflet{\Prf\ If
$u\ge 0$ then $0\ge -u$ so $u\ge -u$.   If $u\ge -u$ then $2u\ge 0$ so
$u=\bover12\cdot 2u\ge 0$.\ \Qed}

\spheader 351Ce \cmmnt{I have called $U^+$ a `positive cone'
without defining the term `cone'.  I think this is something we can
pass by for the moment;  but it will be useful to recognise that} $U^+$
is always
convex\cmmnt{, for if $u$, $v\in U^+$ and $\alpha\in[0,1]$ then
$\alpha u$, $(1-\alpha)v\ge 0$ and $\alpha u+(1-\alpha)v\in U^+$, so is a
`convex cone' as defined in 3A5Ba}.

\leader{351D}{Suprema and infima}  Let $U$ be a \pols.

\spheader 351Da \cmmnt{The
definition of \lq\pols' implies that} $u\mapsto u+w$ is always an
order-isomorphism;  \cmmnt{on the other hand,} $u\mapsto -u$ is
order-reversing\cmmnt{, by 351Ba}.

\spheader 351Db \dvro{If}{It follows that if} $A\subseteq U$ and $v\in U$
then

\Centerline{$\sup_{u\in A}(v+u)=v+\sup A$ if either side is defined,}

\Centerline{$\inf_{u\in A}(v+u)=v+\inf A$ if either side is defined,}

\Centerline{$\sup_{u\in A}(v-u)=v-\inf A$ if either side is defined,}

\Centerline{$\inf_{u\in A}(v-u)=v-\sup A$ if either side is defined.}

\spheader 351Dc \dvro{If}{Moreover, we find that if} $A$, $B\subseteq U$
and $\sup A$ and $\sup B$ are defined, then $\sup(A+B)$ is defined and
equal to $\sup A+\sup B$\cmmnt{, writing $A+B=\{u+v:u\in A,\,v\in B\}$
as usual}.   \prooflet{\Prf\ Set $u_0=\sup A$, $v_0=\sup B$.
Using (b), we have

$$\eqalign{u_0+v_0
&=\sup_{u\in A}(u+v_0)\cr
&=\sup_{u\in A}(\sup_{v\in B}(u+v))
=\sup(A+B).\text{ \Qed}\cr}$$
}

\noindent Similarly, if $A$, $B\subseteq U$ and $\inf A$, $\inf B$ are
defined then $\inf(A+B)=\inf A+\inf B$.
%end of 351Dc

\spheader 351Dd If $\alpha>0$ then\cmmnt{ $u\mapsto\alpha u$ is an
order-isomorphism, so we have} $\sup(\alpha A)=\alpha\sup A$ if either
side is defined;  similarly, $\inf(\alpha A)=\alpha\inf A$.

\leader{351E}{Linear subspaces} If $U$ is a \pols, and $V$ is any linear
subspace of $U$, then $V$, with the induced linear and order structures,
is a \pols\cmmnt{;  this is obvious from the definition}.

\leader{351F}{Positive linear operators} Let $U$ and $V$ be \polss, and
write $\eurm L(U;V)$ for the linear space of all linear operators from
$U$ to $V$.
For $S$, $T\in\eurm L(U;V)$ say that $S\le T$ iff $Su\le Tu$ for every
$u\in U^+$.   Under this ordering, $\eurm L(U;V)$ is a \pols;  its
positive cone is
$\{T:Tu\ge 0$ for every $u\in U^+\}$.   \prooflet{\Prf\ This is an
elementary verification.\ \Qed}   \dvro{For}{Note that, for}
$T\in\eurm L(U;V)$,\cmmnt{

$$\eqalign{T\ge 0
&\Longrightarrow Tu\le Tu+T(v-u)=Tv\text{ whenever }u\le v\text{ in }U\cr
&\Longrightarrow 0=T0\le Tu\text{ for every }u\in U^+\cr
&\Longrightarrow T\ge 0,\cr}$$

\noindent so that} $T\ge 0$ iff $T$ is order-preserving.   In this case
we say that $T$ is a {\bf positive} linear operator.

\cmmnt{Clearly }$ST$ is a positive linear operator whenever $U$, $V$
and $W$ are \polss\ and $T:U\to V$, $S:V\to W$ are positive linear
operators\cmmnt{ (cf.\ 313Ia)}.
%end of 351F

\leader{351G}{Order-continuous positive linear operators:  Proposition}
Let $U$ and $V$ be \polss\ and $T:U\to V$ a positive linear operator.

(a) The following are equiveridical:

\quad(i) $T$ is order-continuous;

\quad(ii) $\inf T[A]=0$ in $V$ whenever $A\subseteq U$ is a non-empty
downwards-directed set with infimum $0$ in $U$;

\quad(iii) $\sup T[A]=Tw$ in
$V$ whenever $A\subseteq U^+$ is a non-empty upwards-directed set with
supremum $w$ in $U$.

(b) The following are equiveridical:

\quad(i) $T$ is sequentially
order-continuous;

\quad(ii) $\inf_{n\in\Bbb N}Tu_n=0$ in $V$ whenever
$\sequencen{u_n}$ is a non-increasing sequence in $U$ with infimum $0$
in $U$;

\quad(iii) $\sup_{n\in\Bbb N}Tu_n=Tw$ in $V$ whenever
$\sequencen{u_n}$ is a non-decreasing sequence in $U^+$ with supremum
$w$ in $U$.

\proof{{\bf (a)(i)$\Rightarrow$(iii)} is trivial.

\medskip

\quad{\bf (iii)$\Rightarrow$(ii)} Assuming (iii), and given that $A$ is
non-empty, downwards-directed and has infimum $0$, take any $u_0\in A$
and consider $A'=\{u:u\in A,\,u\le u_0\}$,
$B=u_0-A'$.   Then $A'$ is non-empty, downwards-directed and has infimum
$0$, so $B$ is non-empty, upwards-directed and has supremum $u_0$ (using
351Db);  by (iii), $\sup T[B]=Tu_0$ and (inverting again)

\Centerline{$\inf T[A']=\inf T[u_0-B]=\inf(Tu_0-T[B])=Tu_0-\sup T[B]=0$.}

\noindent But (because $T$ is positive) $0$ is surely a lower bound for
$T[A]$, so it is also the infimum of $T[A]$.   As $A$ is arbitrary, (ii)
is true.

\medskip

\quad{\bf (ii)$\Rightarrow$(i)} Suppose now that (ii) is true.
\grheada\ If $A\subseteq U$ is non-empty, downwards-directed and has
infimum $w$, then $A-w$ is non-empty, downwards-directed and has infimum
$0$, so

\Centerline{$\inf T[A-w]=0$,\quad$\inf T[A]=\inf(T[A-w]+Tw)
=Tw+\inf T[A-w]=Tw$.}

\noindent\grheadb\ If $A\subseteq U$ is non-empty, upwards-directed and
has supremum $w$, then $-A$ is non-empty, downwards-directed and has
infimum $-w$, so

\Centerline{$\sup T[A]=-\inf(-T[A])=-\inf T[-A]=-T(-w)=Tw$.}

\noindent Putting these together, $T$ is order-continuous.

\medskip

{\bf (b)} The arguments are identical, replacing each directed set by an
appropriate sequence.
}%end of proof of 351G

\leader{351H}{Riesz homomorphisms (a)}\cmmnt{ For the sake of a
representation theorem below (351Q), I introduce the
following definition.}  Let $U$, $V$ be \polss.   A {\bf Riesz
homomorphism} from $U$ to $V$ is a linear operator $T:U\to V$ such that
whenever $A\subseteq U$ is a finite non-empty set and $\inf A=0$ in $U$,
then $\inf T[A]=0$ in $V$.   \cmmnt{The following facts are now nearly
obvious.}

\spheader 351Hb Any Riesz homomorphism is a positive linear operator.
\prooflet{(For if $T$ is a Riesz homomorphism and $u\ge 0$, then
$\inf\{0,u\}=0$ so $\inf\{0,Tu\}=0$ and $Tu\ge 0$.)}

\spheader 351Hc Let $U$ and $V$ be \polss\ and $T:U\to V$ a Riesz
homomorphism.   Then

\Centerline{$\inf T[A]$ exists $=T(\inf A)$, \quad $\sup T[A]$ exists
$=T(\sup A)$}

\noindent whenever $A\subseteq U$ is a finite non-empty set and
$\inf A$, $\sup A$ exist.   \prooflet{(Apply the definition in (a) to

\Centerline{$A'=\{u-\inf A:u\in A\}$,\quad $A''=\{\sup A-u:u\in A\}$.)}
}

\spheader 351Hd If $U$, $V$ and $W$ are \polss\ and $T:U\to V$,
$S:V\to W$
are Riesz homomorphisms then $ST:U\to W$ is a Riesz homomorphism.

\leader{351I}{Solid sets} Let $U$ be a \pols.   I will say
that a subset $A$ of $U$ is {\bf solid} if

\dvro{\Centerline{$A=\bigcup_{u\in A}[-u,u]$.}}
{\Centerline{$A=\{v:v\in U,\,-u\le v\le u$ for some $u\in A\}
=\bigcup_{u\in A}[-u,u]$}}

\cmmnt{\noindent in the notation of 2A1Ab.   (I should perhaps remark
that while this definition
is well established in the case of Riesz spaces (\S352), the extension
to general \polss\ is not standard.   See 351Yb for a warning.)}

\leader{351J}{Proposition} Let $U$ be a \pols\ and $V$ a solid linear
subspace of $U$.   Then the quotient linear space $U/V$ has a \pols\
structure defined by either of the rules

\inset{$u^{\ssbullet}\le w^{\ssbullet}$ iff there is some $v\in V$ such
that $u\le v+w$,}

\inset{$(U/V)^+=\{u^{\ssbullet}:u\in U^+\}$,}

\noindent and for this partial order on $U/V$ the map
$u\mapsto u^{\ssbullet}:U\to U/V$ is a Riesz homomorphism.

\proof{{\bf (a)} I had better start by giving priority to one of the
descriptions of the relation $\le$ on $U/V$;  I choose the first.   To
see that this makes $U/V$ a \pols, we have to check the following.

(i) $0\in V$ and $u\le u+0$, so $u^{\ssbullet}\le u^{\ssbullet}$ for
every $u\in U$.

(ii) If $u_1$, $u_2$, $u_3\in U$ and
$u_1^{\ssbullet}\le u_2^{\ssbullet}$,
$u_2^{\ssbullet}\le u_3^{\ssbullet}$ then there are
$v_1$, $v_2\in V$ such that $u_1\le u_2+v_1$, $u_2\le u_3+v_2$;  in
which case $v_1+v_2\in V$ and $u_1\le u_3+v_1+v_2$, so
$u_1^{\ssbullet}\le u_3^{\ssbullet}$.

(iii) If $u$, $w\in U$ and $u^{\ssbullet}\le w^{\ssbullet}$,
$w^{\ssbullet}\le u^{\ssbullet}$ then there are $v$, $v'\in V$ such that
$u\le w+v$, $w\le u+v'$.   Now there are $v_0$, $v'_0\in V$ such that
$-v_0\le v\le v_0$, $-v'_0\le v'\le v'_0$, and in this case $v_0$,
$v'_0\ge 0$ (351Cd), so

\Centerline{$-v_0-v'_0\le -v'\le u-w\le v\le v_0+v'_0\in V$,}.

\noindent Accordingly  $u-w\in V$ and $u^{\ssbullet}=w^{\ssbullet}$.
Thus $U/V$ is a partially ordered set.

(iv) If $u_1$, $u_2$, $w\in U$ and $u_1^{\ssbullet}\le u_2^{\ssbullet}$,
then there is a $v\in V$ such that $u_1\le u_2+v$, in which case
$u_1+w\le u_2+w+v$ and $u_1^{\ssbullet}+w^{\ssbullet}\le
u_2^{\ssbullet}+w^{\ssbullet}$.

(v) If $u\in U$, $\alpha\in \Bbb R$, $u^{\ssbullet}\ge 0$ and
$\alpha\ge 0$ then there is a $v\in V$ such that $u+v\ge 0$;
now $\alpha v\in V$ and $\alpha u+\alpha v\ge 0$, so
$\alpha u^{\ssbullet}=(\alpha u)^{\ssbullet}\ge 0$.

Thus $U/V$ is a \pols.

\medskip

{\bf (b)} Now $(U/V)^+=\{u^{\ssbullet}:u\ge 0\}$.   \Prf\ If $u\ge 0$
then of course $u^{\ssbullet}\ge 0$ because $0\in V$ and $u+0\ge 0$.
On the other hand, if we have any element $p$ of $(U/V)^+$, there are
$u\in U$, $v\in V$ such that $u^{\ssbullet}=p$ and $u+v\ge 0$;  but now
$p=(u+v)^{\ssbullet}$ is of the required form.\ \Qed

\medskip

{\bf (c)} Finally, $u\mapsto u^{\ssbullet}$ is a Riesz homomorphism.
\Prf\ Suppose that $A\subseteq U$ is a non-empty finite set and that
$\inf A=0$ in $U$.   Then $u^{\ssbullet}\ge 0$ for every $u\in A$, that
is, $0$ is a lower bound for $\{u^{\ssbullet}:u\in A\}$.   Let $p\in
U/V$ be any other lower bound for
$\{u^{\ssbullet}:u\in A\}$.   Express $p$ as $w^{\ssbullet}$ where $w\in
U$.   For each $u\in A$, $w^{\ssbullet}\le u^{\ssbullet}$ so there is a
$v_u\in V$ such that $w\le u+v_u$.   Next, there is a $v'_u\in V$ such
that $-v'_u\le v_u\le v'_u$.   Set $v^*=\sum_{u\in A}v'_u\in V$.   Then
$v_u\le v'_u\le v^*$ so $w\le u+v^*$ for every $u\in A$, and $w-v^*$ is
a lower bound for $A$ in $U$.   Accordingly $w-v^*\le 0$, $w\le0 + v^*$
and $p=w^{\ssbullet}\le 0$.   As $p$ is arbitrary,
$\inf\{u^{\ssbullet}:u\in A\}=0$;  as $A$ is arbitrary, $u\mapsto
u^{\ssbullet}$ is a Riesz homomorphism.\ \Qed
}%end of proof of 351J

\leader{351K}{Lemma} Suppose that $U$ is a \pols, and that $W$, $V$ are
solid linear subspaces of $U$ such that $W\subseteq V$.   Then
$V_1=\{v^{\ssbullet}:v\in V\}$ is a solid linear subspace of $U/W$.

\proof{ (i) Because the map $u\mapsto u^{\ssbullet}$ is linear, $V_1$ is
a linear subspace of $U/W$.   (ii) If $p\in V_1$, there is a $v\in V$
such that $p=v^{\ssbullet}$;  because $V$ is solid in $U$, there is a
$v_0\in V$ such that $-v_0\le v\le v_0$;  now $v_0^{\ssbullet}\in V_1$
and $-v_0^{\ssbullet}\le p\le v_0^{\ssbullet}$.   (iii) If $p\in V_1$,
$q\in U/W$ and $-p\le q\le p$, take $v_0\in V$, $u\in U$ such that
$v_0^{\ssbullet}=p$ and $u^{\ssbullet}=q$.   Because
$-v_0^{\ssbullet}\le u^{\ssbullet}\le v_0^{\ssbullet}$, there are $w$,
$w'\in W$ such that
$-v_0-w\le u\le v_0+w'$.   Now $-v_0-w$, $v_0+w'$ both belong to $V$,
which is solid, so $u\in V$
and $q=u^{\ssbullet}\in V_1$.   (iv) Putting (ii) and (iii) together,
$V_1$ is solid.
}%end of proof of 351K

\leader{351L}{Products} If $\langle U_i\rangle_{i\in I}$ is any family
of \polss, we have a product linear space $U=\prod_{i\in I}U_i$;  if we
set $u\le v$ in $U$ iff $u(i)\le v(i)$ for every $i\in I$, $U$ becomes a
\pols, with positive cone $\{u:u(i)\ge 0$ for every $i\in I\}$.   For
each $i\in I$ the map $u\mapsto u(i):U\to U_i$ is an order-continuous
Riesz
homomorphism\cmmnt{ (in fact, it preserves arbitrary suprema and
infima)}.

\leader{351M}{Reduced powers of $\Bbb R$ (a)} Let $X$ be any set.   Then
$\Bbb R^X$ is a \pols\ if we say that $f\le g$ means that $f(x)\le g(x)$
for every $x\in X$\cmmnt{, as in 351L}.   If now $\Cal F$ is a filter
on $X$, we have a corresponding set

\Centerline{$V=\{f:f\in\Bbb R^X,\,\{x:f(x)=0\}\in\Cal F\}$;}

\noindent\cmmnt{it is easy to see that }$V$ is a linear subspace of
$\Bbb R^X$, and is solid\cmmnt{ because $f\in V$ iff $|f|\in V$}.
By the {\bf reduced power} $\Bbb R^X|\Cal F$ I shall mean the quotient
\pols\ $\Bbb R^X/V$.

\spheader 351Mb \dvro{For}{Note that for} $f\in\Bbb R^X$,

\Centerline{$f^{\ssbullet}\ge 0$ in
$\Bbb R^X|\Cal F\iff\{x:f(x)\ge 0\}\in \Cal F$.}

\prooflet{\noindent \Prf\ (i) If $f^{\ssbullet}\ge 0$, there is a
$g\in V$ such that $f+g\ge 0$;  now

\Centerline{$\{x:f(x)\ge 0\}\supseteq\{x:g(x)=0\}\in\Cal F$.}

\noindent (ii) If $\{x:f(x)\ge 0\}\in\Cal F$, then
$\{x:(|f|-f)(x)=0\}\in\Cal F$, so
$f^{\ssbullet}=|f|^{\ssbullet}\ge 0$.\ \Qed}

\leader{351N}{}\cmmnt{ On the way to the next theorem, the main result
(in terms of mathematical content)
of this section, we need a string of lemmas.

\medskip

\noindent}{\bf Lemma} Let $U$ be a \pols.   If $u$,
$v_0,\ldots,v_n\in U$ are such that $u\ne 0$ and $v_0,\ldots,v_n\ge 0$
then there is a
linear functional $f:U\to\Bbb R$
such that $f(u)\ne 0$ and $f(v_i)\ge 0$ for every $i$.

\proof{ The point is that at most one of $u$, $-u$ can belong to the
convex cone $C$ generated by $\{v_0,\ldots,v_n\}$, because this is
included in the convex cone set $U^+$, and since $u\ne 0$ at most one of
$u$, $-u$ can belong to $U^+$.

Now however the Hahn-Banach theorem, in the form 3A5D, tells us
that if $u\notin C$ there is a linear functional $f:U\to\Bbb R$ such
that $f(u)<0$ and $f(v_i)\ge 0$ for every $i$;  while if $-u\notin C$ we
can get $f(-u)<0$ and $f(v_i)\ge 0$ for every $i$.   Thus in either case
we have a functional of the required type.
}%end of proof of 351N

\leader{351O}{Lemma} Let $U$ be a \pols, and $u_0$ a non-zero member of
$U$.   Then there is a solid linear subspace $V$ of $U$ such that
$u_0\notin V$ and whenever $A\subseteq U$ is finite, non-empty and has
infimum $0$ then $A\cap V\ne\emptyset$.

\proof{{\bf (a)} Let $\Cal W$ be the family of all solid linear
subspaces of $U$ not containing $u_0$.   Then any non-empty totally
ordered $\Cal V\subseteq\Cal W$ has an upper bound $\bigcup\Cal V$ in
$\Cal W$.    By Zorn's Lemma, $\Cal W$ has a maximal element $V$ say.
This is surely a solid linear subspace of $U$ not containing $u_0$.

\medskip

{\bf (b)} Now
for any $w\in U^+\setminus V$ there are $\alpha\ge 0$, $v\in V^+$ such
that $-\alpha w-v\le u_0\le \alpha w+v$.   \Prf\ Let $V_1$ be

\Centerline{$\{u:u\in U$, there are $\alpha\ge 0$, $v\in V^+$ such that
$-\alpha w-v\le u\le\alpha w+v\}$.}

\noindent Then it is easy to check that $V_1$ is a solid linear subspace
of $U$, including $V$, and containing $w$;  because $w\notin V$,
$V_1\ne V$, so $V_1\notin\Cal W$ and $u\in V_1$, as claimed.\ \Qed

\medskip

{\bf (c)} It follows that if $A\subseteq U$ is finite and non-empty and
$\inf A=0$ in $U$ then $A\cap V\ne\emptyset$.   \Prf\Quer\ Otherwise, for every
$w\in A$ there must be $\alpha_w\ge 0$, $v_w\in V^+$ such that
$-\alpha_ww-v_w\le u_0\le\alpha_ww+v_w$.   Set
$\alpha=1+\sum_{w\in A}\alpha_w$, $v=\sum_{w\in A}v_w\in V$;  then
$-\alpha w-v\le u_0\le\alpha w+v$ for every $w\in A$.   Accordingly
$\bover1{\alpha}(u_0-v)\le w$ for
every $w\in A$ and $\bover1{\alpha}(u_0-v)\le 0$, so $u_0\le v$.
Similarly,
$-\bover1{\alpha}(v+u_0)\le w$ for every $w\in A$ and $-v\le u_0$.   But
(because $V$ is solid) this means that $u_0\in V$, which is not so.\
\Bang\Qed

Accordingly $V$ has the required properties.
}%end of proof of 351O

\leader{351P}{Lemma} Let $U$ be a \pols\ and $u$ a non-zero element of $U$, and suppose that $A_0,\ldots,A_n$ are finite non-empty subsets of $U$ such that $\inf A_j=0$ for every $j\le n$.   Then there is a linear
functional $f:U\to\Bbb R$ such that $f(u)\ne 0$ and $\min f[A_j]=0$ for
every $j\le n$.

\proof{ By 351O, there is a solid linear subspace $V$ of $U$ such that
$u\notin V$ and $A_j\cap V\ne 0$ for every $j\le n$.   Give the quotient
space $U/V$ its standard partial ordering (351J), and in $U/V$ set
$C=\{v^{\ssbullet}:v\in\bigcup_{j\le n}A_j\}$.   Then $C$ is a finite
subset of $(U/V)^+$, while $u^{\ssbullet}\ne 0$, so by 351N there is a
linear functional $g:U/V\to\Bbb R$ such that $g(u^{\ssbullet})\ne 0$ but
$g(p)\ge 0$ for every $p\in C$.   Set $f(v)=g(v^{\ssbullet})$ for $v\in
U$;  then $f:U\to\Bbb R$ is linear, $f(u)\ne 0$ and $f(v)\ge 0$ for
every $v\in\bigcup_{j\le n}A_j$.   But also, for each $j\le n$, there is
a $v_j\in A_j\cap V$, so that $f(v_j)=0$;  and this means that
$\min f[A_j]$ must be $0$, as required.
}%end of proof of 351P

\leader{351Q}{}\cmmnt{ Now we are ready for the theorem.

\medskip

\noindent}{\bf Theorem} Let $U$ be any \pols.   Then we can find a set
$X$, a filter $\Cal F$ on $X$ and an injective Riesz homomorphism from
$U$ to the reduced power
$\Bbb R^X|\Cal F$\cmmnt{ described in 351M}.

\proof{ Let $X=U'$ be the set of all linear functionals $f:U\to\Bbb R$;
for $u\in U$ define $\hat u\in\Bbb R^X$ by setting
$\hat u(f)=f(u)$ whenever $f\in X$ and $u\in U$.   Let $\Cal A$ be the
family of non-empty finite sets $A\subseteq U$ such that $\inf A=0$.
For $A\in\Cal A$ let $F_A$ be the set of those $f\in X$ such that
$\min f[A]=0$.   Since $0\in F_A$ for every $A\in\Cal A$, the set

\Centerline{$\Cal F=\{F:F\subseteq X$, there are
$A_0,\ldots,A_n\in\Cal A$ such that
$F\supseteq\bigcap_{j\le n}F_{A_j}\}$}

\noindent is a filter on $X$.   Set
$\psi(u)=\hat u^{\ssbullet}\in \Bbb R^X|\Cal F$ for $u\in U$.
Then $\psi:U\to\Bbb R^X|\Cal F$ is an injective Riesz homomorphism.

\Prf\ (i) $\psi$ is linear because $u\mapsto\hat u:U\to\BbbR^X$ and
$h\mapsto h^{\ssbullet}:\Bbb R^X\to\BbbR^X|\Cal F$ are linear.
(ii) If $A\in\Cal A$,
then $F_A\in\Cal F$.   So, first, if $v\in A$, then
$\{f:\hat v(f)\ge 0\}\in\Cal F$, so that
$\psi(v)=\hat v^{\ssbullet}\ge 0$ in
$\Bbb R^X|\Cal F$
(351Mb).   Next, if $w\in\Bbb R^X|\Cal F$ and $w\le\psi(v)$ for every
$v\in A$, we can express $w$ as $h^{\ssbullet}$ where
$h^{\ssbullet}\le\hat v^{\ssbullet}$ for every $v\in A$, that is,
$H_v=\{f:h(f)\le\hat v(f)\}\in\Cal F$ for every $v\in A$.   But now
$H=F_A\cap\bigcap_{v\in A}H_v\in\Cal F$, and for $f\in H$ we have
$h(f)\le\min_{v\in A}f(v)=0$.   This means that $w=h^{\ssbullet}\le 0$.
As $w$ is arbitrary, $\inf\psi[A]=0$.   As $A$ is arbitrary, $\psi$ is a
Riesz homomorphism.   (iii) Finally, \Quer\ suppose, if possible, that
there is a non-zero $u\in U$ such that $\psi(u)=0$.   Then
$F=\{f:f(u)=0\}\in\Cal F$, and there are $A_0,\ldots,A_n\in\Cal A$
such that $F\supseteq\bigcap_{j\le n}F_{A_j}$.   By 351P, there is an
$f\in\bigcap_{j\le n}F_{A_j}$ such that $f(u)\ne 0$;  which is
impossible.\ \Bang\
Accordingly $\psi$ is injective, as claimed.\ \Qed
}%end of proof of 351Q

\vleader{72pt}{351R}{Archimedean spaces (a)} For a \pols\ $U$, the following
are equiveridical:  (i) if $u$, $v\in U$ are such that $nu\le v$ for every
$n\in\Bbb N$ then $u\le 0$ (ii) if $u\ge 0$ in $U$ then
$\inf_{\delta>0}\delta u=0$.   \prooflet{\Prf{\bf (i)$\Rightarrow$(ii)} If
(i) is true and $u\ge 0$, then of course $\delta u\ge 0$ for every
$\delta>0$;  on the other hand, if $v\le\delta u$ for every $\delta>0$,
then $nv\le n\cdot\bover1nu=u$ for every $n\ge 1$, while of course
$0v=0\le u$, so $v\le 0$.   Thus $0$ is the greatest lower bound of
$\{\delta u:\delta>0\}$.   {\bf (ii)$\Rightarrow$(i)} If (ii) is true
and $nu\le v$ for every $n\in\Bbb N$, then $0\le v$ and $u\le\bover1nv$
for every $n\ge 1$.   If now $\delta>0$, then there is an $n\ge 1$ such
that $\bover1n\le\delta$, so that $u\le\bover1nv\le\delta v$ (351Bc).
Accordingly $u$ is a lower bound for $\{\delta v:\delta>0\}$ and
$u\le 0$.\ \Qed}

\spheader 351Rb I will say that \polss\ satisfying the equiveridical
conditions of (a) above are {\bf Archimedean}.

\spheader 351Rc Any linear subspace of an Archimedean \pols, with the
induced \pols\ structure, is Archimedean.

\spheader 351Rd Any product of Archimedean \polss\ is Archimedean.
\prooflet{\Prf\ If $U=\prod_{i\in I}U_i$ is a product of Archimedean
spaces, and $nu\le v$ in $U$ for every $n\in\Bbb N$, then for each $i\in
I$ we must have $nu(i)\le v(i)$ for every $n$, so that $u(i)\le 0$;
accordingly $u\le 0$.\ \Qed}
In particular, $\Bbb R^X$ is Archimedean for any set $X$.

\exercises{\leader{351X}{Basic exercises $\pmb{>}$(a)}
Let $\zeta$ be any ordinal.   The {\bf lexicographic ordering} of
$\Bbb R^{\zeta}$ is defined by saying that $f\le g$ iff either $f=g$
or there
is a $\xi<\zeta$ such that $f(\eta)=g(\eta)$ for $\eta<\xi$ and
$f(\xi)<g(\xi)$.   Show that this is a total order on $\BbbR^{\zeta}$
which renders $\BbbR^{\zeta}$ a \pols.
%351A

\spheader 351Xb Let $U$ be a \pols\ and $V$ a linear subspace of $U$.
Show that the formulae of 351J define a \pols\ structure on the quotient
$U/V$ iff $V$ is {\bf order-convex}, that is, $u\in V$ whenever $v_1$,
$v_2\in V$ and $v_1\le u\le v_2$.
%351J

\spheader 351Xc Let $\langle U_i\rangle_{i\in I}$ be a family of \polss\
with product $U$.   For $i\in I$, define $T_i:U_i\to U$ by setting
$T_ix=u$ where
$u(i)=x$, $u(j)=0$ for $j\ne i$.   Show that $T_i$ is an injective
order-continuous Riesz homomorphism.
%351L

\sqheader 351Xd Let $U$ be a \pols\ and $\langle V_i\rangle_{i\in I}$ a
family of \polss\ with product $V$.   Show that $\eurm L(U;V)$ can be
identified, as \pols, with $\prod_{i\in I}\eurm L(U;V_i)$.
%351L

\sqheader 351Xe Show that if $U$, $V$ are \polss\ and $V$ is
Archimedean, then $\eurm L(U;V)$ is Archimedean.
%351R

\leader{351Y}{Further exercises (a)}
%\spheader 351Ya
Give an example of two \polss\ $U$ and $V$ and a bijective
Riesz homomorphism $T:U\to V$ such that $T^{-1}:V\to U$ is not a Riesz
homomorphism.
%351H mt35bits

\spheader 351Yb(i) Let $U$ be a \pols.   Show that $U$ is a solid
subset of itself (on the definition 351I) iff $U=U^+-U^+$.   (ii) Give
an example of a \pols\ $U$ satisfying this condition with an element
$u\in U$ such that the intersection of the solid sets containing $u$ is
not solid.
%351I mt35bits

\spheader 351Yc Show that a reduced power $\Bbb R^X|\Cal F$, as
described in 351M, is totally ordered iff $\Cal F$ is an ultrafilter,
and in this case has a natural structure as a totally ordered field.
%351M

\spheader 351Yd Let $U$ be a \pols, and suppose that $A$, $B\subseteq U$
are two non-empty finite sets such that ($\alpha$) $u\vee v=\sup\{u,v\}$
is defined for every $u\in A$, $v\in B$ ($\beta$) $\inf A$ and $\inf B$
and $(\inf A)\vee(\inf B)$ are defined.   Show that
$\inf\{u\vee v:u\in A,\,v\in B\}=(\inf A)\vee(\inf B)$.   ({\it Hint\/}:
show that this is true if $U=\Bbb R$, if
$U=\Bbb R^X$ and if $U=\Bbb R^X|\Cal F$, and use 351Q.)
%351Q

\spheader 351Ye Show that a reduced power $\Bbb R^X|\Cal F$, as
described in 351M, is Archimedean iff $\bigcap_{n\in\Bbb N}F_n\in\Cal F$
whenever $\sequencen{F_n}$ is a sequence in $\Cal F$.
%351R
}%end of exercises

\cmmnt{\Notesheader{351} The idea of `\pols' is a very natural
abstraction from the elementary examples of $\Bbb R^X$ and its
subspaces, and the only possible difficulty lies in guessing the exact
boundary at which one's standard manipulations with such familiar spaces
cease to be valid in the general case.   (For instance, most people's
favourite examples are Archimedean, in the sense of 351R, so it is
prudent to check your intuitions against a non-Archimedean space like
that of 351Xa.)   There is really no room for any deep idea to appear in
351B-351F.   When I come to what I call `Riesz homomorphisms',
however (351H), there are some more interesting possibilities in the
background.

I shall not discuss the applications of Theorem 351Q to general \polss;
it is here for the sake of its application to Riesz spaces in the next
section.   But I think it is a very striking fact that not only does any
\pols\ $U$ appear as a linear subspace of some reduced power of
$\Bbb R$, but the embedding can be taken to preserve any suprema and infima of
finite sets which exist in $U$.   This is in a sense a result of the
same kind as the Stone representation theorem for Boolean algebras;  it
gives us a chance to confirm that an intuition valid for $\Bbb R$ or
$\BbbR^X$ may in fact apply to arbitrary \polss.   If you like, this
provides a metamathematical foundation for such results as those in
351B.   I have to say that for \polss\ it is generally quicker to find a
proof directly from the definition than to trace through an argument
relying on 351Q;  but this is not always the case for Riesz spaces.   I
offer 351Yd as an example of a result where a direct proof does at least
call for a moment's thought, while the argument through 351Q is
straightforward.

`Reduced powers' are of course of great importance for other reasons;
I mention 351Yc as a hint of what can be done.

}%end of notes
\discrpage

