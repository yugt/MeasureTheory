\frfilename{mt33.tex} 
\versiondate{17.11.10} 
\copyrightdate{1995} 
 
\def\chaptername{Maharam's theorem} 
\def\sectionname{Introduction} 
 
\newchapter{33} 
 
We are now ready for the astonishing central fact about measure 
algebras:  there are very few of them.   Any localizable measure algebra 
has a canonical expression as a simple product of measure algebras of 
easily described types.   This complete classification necessarily 
dominates all further discussion of measure algebras;  to the point that 
all the results of Chapter 32 have to be regarded as `elementary', since 
however complex their formulation they have been proved by techniques 
not involving, nor providing, any particular insight into the special 
nature of measure algebras.   The proof depends, of course, on 
developing methods of defining measure-preserving homomorphisms and 
isomorphisms;  I give a number of results, progressively more elaborate, 
but all based on the same idea.   These techniques are of great power, 
leading, for instance, to an effective classification of closed 
subalgebras and their embeddings. 
 
`Maharam's theorem' itself, the classification of localizable measure 
algebras, is in \S332.   I devote \S331 to the definition and 
description of `homogeneous' probability algebras.   In \S333 I turn to 
the problem of describing pairs $(\frak A,\frak C)$ where $\frak A$ is a 
probability algebra and $\frak C$ is a closed subalgebra.   Finally, in 
\S334, I give some straightforward results on the classification of free 
products of probability algebras. 
 
 
 
 
\discrpage 
 
 
