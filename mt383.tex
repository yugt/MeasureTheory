\frfilename{mt383.tex}
\versiondate{30.7.03/24.10.03}
\copyrightdate{2003}

\def\cycleii#1#2#3{\cycle{#1\,_{#2}\,#3}}
\def\magnitude{\mathop{\text{mag}}}

\def\chaptername{Automorphisms}
\def\sectionname{Automorphism groups of measure algebras}

\newsection{383}

I turn now to the group of measure-preserving automorphisms of a measure
algebra, seeking to apply the results of the last section.   The
principal theorems are 383D, which is a straightforward special case of
382N, and 383I, corresponding to 382S.   I give another example of
the use of 382R to describe the normal subgroups of $\AmuA$
(383J).   I conclude with an important fact about conjugacy in $\AmuA$
and $\Aut\frak A$ (383L).

\leader{383A}{Definition} Let $(\frak A,\bar\mu)$ be a measure algebra.
I will write $\AmuA$ for the set of all measure-preserving automorphisms
of $\frak A$.   This is a group\cmmnt{, being a subgroup of the group
$\Aut\frak A$ of all Boolean automorphisms of $\frak A$}.

\vleader{60pt}{383B}{Lemma} Let $(\frak A,\bar\mu)$ be a measure algebra, and
$\langle a_i\rangle_{i\in I}$, $\langle b_i\rangle_{i\in I}$ two
partitions of unity in $\frak A$.   Assume

\quad{\it either} that $I$ is countable

\quad{\it or} that $(\frak A,\bar\mu)$ is localizable.

\noindent Suppose that for each $i\in I$ we have a
measure-preserving isomorphism $\pi_i:\frak A_{a_i}\to\frak A_{b_i}$
between the corresponding principal ideals.   Then there is a unique
$\pi\in\AmuA$ such that $\pi c=\pi_ic$ whenever $i\in I$ and
$c\Bsubseteq a_i$.

\proof{ (Compare 381C.)   By 322L, we may identify $\frak A$ with each
of the simple products $\prod_{i\in I}\frak A_{a_i}$,
$\prod_{i\in I}\frak A_{b_i}$;   now $\pi$ corresponds to the
isomorphism between the two products induced by the $\pi_i$.
}%end of proof of 383B

\leader{383C}{Corollary} If $(\frak A,\bar\mu)$ is a localizable measure
algebra, then\cmmnt{, in the language of 381Be,} $\AmuA$ is a full
subgroup of $\Aut\frak A$.

\leader{383D}{Theorem} Let $(\frak A,\bar\mu)$ be a localizable measure
algebra.   Then every measure-preserving automorphism of $\frak A$ is
expressible as the product of at most three measure-preserving
involutions.

\proof{ This is immediate from 383C and 382N.
}%end of proof of 383D

\leader{383E}{Lemma} If $(\frak A,\bar\mu)$ is a homogeneous semi-finite
measure algebra, it is $\sigma$-finite, therefore localizable.

\proof{ If $\frak A=\{0\}$, this is trivial.   Otherwise there is an
$a\in\frak A$ such that $0<\bar\mu a<\infty$.   The principal ideal
$\frak A_a$ is ccc (322G), so $\frak A$ also is, and $(\frak A,\bar\mu)$
must be $\sigma$-finite, by 322G in the opposite direction.
}%end of proof of 383E

\leader{383F}{Lemma} Let $(\frak A,\bar\mu)$ be a homogeneous
semi-finite measure algebra.

(a) If $\langle a_i\rangle_{i\in I}$, $\langle b_i\rangle_{i\in I}$ are
partitions of unity in $\frak A$ with $\bar\mu a_i=\bar\mu b_i$ for
every $i$, there is a $\pi\in\AmuA$ such that $\pi a_i=b_i$ for each
$i$.

(b) If $(\frak A,\bar\mu)$ is totally finite, then whenever $\langle
a_i\rangle_{i\in I}$, $\langle b_i\rangle_{i\in I}$ are disjoint
families in $\frak A$ with $\bar\mu a_i=\bar\mu b_i$ for every $i$,
there is a $\pi\in\AmuA$ such that $\pi a_i=b_i$ for each $i$.

\proof{{\bf (a)} By 383E, $(\frak A,\bar\mu)$ is $\sigma$-finite,
therefore localizable.   For each $i\in I$, the principal ideals
$\frak A_{a_i}$, $\frak A_{b_i}$ are homogeneous, of the same measure
and the same Maharam type
(being $\tau(\frak A)$ if $a_i\ne 0$, $0$ if $a_i=0$).   Because they
are ccc, they are of the same magnitude, as defined in 332Ga, and there
is a measure-preserving isomorphism
$\pi_i:\frak A_{a_i}\to\frak A_{b_i}$ (332J).   By 383B there is a
measure-preserving automorphism
$\pi:\frak A\to\frak A$ such that $\pi d=\pi_id$ for every $i\in I$,
$d\Bsubseteq a_i$;  and this $\pi$ serves.

\medskip

{\bf (b)} Set $a^*=1\Bsetminus\sup_{i\in I}a_i$,
$b^*=1\Bsetminus\sup_{i\in I}b_i$.  We must have

\Centerline{$\bar\mu a^*=\bar\mu 1-\sum_{i\in I}\bar\mu a_i
=\bar\mu 1-\sum_{i\in I}\bar\mu b_i=\bar\mu b^*$,}

\noindent so adding $a^*$, $b^*$ to the families we obtain partitions of
unity to which we can apply the result of (a).
}%end of proof of 383F

\leader{383G}{Lemma} (a) If $(\frak A,\bar\mu)$ is an atomless
semi-finite measure algebra, then $\Aut\frak A$ and $\AmuA$ have many
involutions.

(b) If $(\frak A,\bar\mu)$ is an atomless localizable measure algebra,
then every element of $\frak A$ is the support of some involution in
$\AmuA$.

\proof{{\bf (a)} If $a\in\frak A\setminus\{0\}$, then by 332A
there is a non-zero $b\Bsubseteq a$, of finite measure, such that the
principal ideal $\frak A_b$ is
(Maharam-\vthsp type-\nobreak)homogeneous.   Now because
$\frak A$ is atomless, there is a $c\Bsubseteq b$ such that
$\bar\mu c=\bover12\bar\mu b$ (331C), so that $\frak A_c$ and
$\frak A_{b\Bsetminus c}$ are isomorphic measure algebras.   If
$\theta:\frak A_c\to\frak A_{b\Bsetminus c}$ is any measure-preserving
isomorphism, then
$\pi=\cycleii{c}{\theta}{b\Bsetminus c}$ is an involution in $\AmuA$
(and therefore in $\Aut\frak A$) supported by $a$.

\medskip

{\bf (b)} Use 383C, (a) and 382Q.
}%end of proof of 383G

\leader{383H}{Corollary} Let $(\frak A,\bar\mu)$ be an atomless
localizable measure algebra.   Then

(a) the lattice of normal subgroups of $\Aut\frak A$ is isomorphic to
the lattice of $\Aut\frak A$-invariant ideals of $\frak A$;

(b) the lattice of normal subgroups of $\AmuA$ is isomorphic to the
lattice of $\AmuA$-invariant ideals of $\frak A$.

\proof{ Use 382R.   Taking $G$ to be either $\Aut\frak A$ or $\AmuA$,
and $\Cal I$ to be the family of $G$-invariant ideals in $\frak A$, we
have a map $I\mapsto H_I=\{\pi:\pi\in G,\,\supp\pi\in I\}$ from $\Cal I$
to the family $\Cal H$ of normal subgroups of $G$.   Of course this map
is order-preserving;  382R tells us that it is surjective;  and 383Gb
tells us that it is injective and its inverse is order-preserving, since
if $a\in I\setminus J$ there is a $\pi\in G$ with $\supp\pi=a$, so that
$\pi\in H_I\setminus H_J$.   Thus we have an order-isomorphism between
$\Cal H$ and $\Cal I$.
}%end of proof of 383H

\leader{383I}{Normal subgroups of $\Aut\frak A$ and
\dvrocolon{$\AmuA$}}\cmmnt{ 382R provides the machinery for a full
description of the normal
subgroups of $\Aut\frak A$ and $\AmuA$ when $(\frak A,\bar\mu)$ is an
atomless localizable measure algebra, as we know that they correspond
exactly to the invariant ideals of $\frak A$.   The general case is
complicated.   But the following are easy enough.

\medskip

\noindent}{\bf Theorem} Let $(\frak A,\bar\mu)$ be a homogeneous
semi-finite measure algebra.

(a) $\Aut\frak A$ is simple.

(b) If $(\frak A,\bar\mu)$ is totally finite, $\AmuA$ is simple.

(c) If $(\frak A,\bar\mu)$ is not totally finite, $\AmuA$
has exactly one non-trivial proper normal subgroup.

\proof{{\bf (a)} $\frak A$ is Dedekind complete (383E), so this is a
special case of 382S.

\medskip

{\bf (b)-(c)} The point is that the only possible $\AmuA$-invariant
ideals of $\frak A$ are $\{0\}$, $\frak A^f$ and $\frak A$.   \Prf\ If
$\frak A$ is $\{0\}$ or $\{0,1\}$ this is trivial.   Otherwise,
$\frak A$ is atomless.   Let $I\normalsubgroup\frak A$ be
an invariant ideal.

(i) If $I\not\subseteq\frak A^f$, take $a\in I$ with $\bar\mu a=\infty$.
By 383E, $\frak A$ is $\sigma$-finite, so $a$ has the same magnitude
$\omega$ as $1$.   By
332I, there is a partition of unity $\sequencen{e_n}$ in $\frak A$ with
$\bar\mu e_n=1$ for every $n$;   setting $b=\sup_{n\in\Bbb N}e_{2n}$,
$b'=1\Bsetminus b$, we see that both $b$ and $b'$ are of infinite
measure.    Similarly we can divide $a$ into $c$ and $c'$, both of
infinite measure.    Now by 332J the principal ideals $\frak A_b$,
$\frak A_{b'}$, $\frak A_c$, $\frak A_{1\Bsetminus c}$ are all
isomorphic as
measure algebras, so that there are automorphisms $\pi$, $\phi\in\AmuA$
such that

\Centerline{$\pi c=b$,\quad $\phi c=b'$.}

\noindent But this means that both $b$ and $b'$ belong to $I$, so that
$1=b\Bcup b'\in I$ and $I=\frak A$.

(ii) If $I\subseteq\frak A^f$ and $I\ne\{0\}$, take any non-zero $a\in
I$.   If $b$ is any member of $\frak A$, then (because $\frak A$ is
atomless) $b$ can be partitioned into $b_0,\ldots,b_n$, all of measure
at most $\bar\mu a$.   Then for each $i$ there is a $b_i'\Bsubseteq a$
such that $\bar\mu b'_i=\bar\mu b_i$;  since this common measure is
finite, $\bar\mu(1\Bsetminus b'_i)=\bar\mu(1\Bsetminus b_i)$.   By 332J
and 383Fa, there is a
$\pi_i\in\AmuA$ such that $\pi_ib'_i=b_i$, so that $b_i$ belongs to $I$.
Accordingly $b\in I$.   As $b$ is arbitrary, $I=\frak A^f$.

Thus the only invariant ideals of $\frak A$ are $\{0\}$, $\frak A^f$ and
$\frak A$.\ \Qed

By 383Hb we therefore have either one, two or three normal subgroups of
$\AmuA$, according to whether $\bar\mu 1$ is zero, finite and not zero,
or infinite.
}%end of proof of 383I

\cmmnt{\medskip

\noindent{\bf Remark} For the Lebesgue probability algebra, (b) is due
to {\smc Fathi 78}.   The extension to algebras of uncountable Maharam
type is from {\smc Choksi \& Prasad 82}.
}

\leader{383J}{}\cmmnt{ The language of \S352 offers a way of
describing another case.

\medskip

\noindent}{\bf Proposition} Let $(\frak A,\bar\mu)$ be an atomless
totally finite measure algebra.   For each infinite cardinal $\kappa$,
let $e_{\kappa}$ be the Maharam-type-$\kappa$ component of $\frak A$,
and let $K$ be $\{\kappa:e_{\kappa}\ne 0\}$.   Let $\Cal H$ be the
lattice of normal subgroups of $\AmuA$.   Then

(i) if $K$ is finite, $\Cal H$ is isomorphic, as partially ordered set,
to $\Cal PK$;

(ii) if $K$ is infinite, then $\Cal H$ is isomorphic, as partially
ordered set, to the lattice of solid linear subspaces of
$\ell^{\infty}$.

\proof{{\bf (a)} Let $\Cal I$ be the family of $\AmuA$-invariant ideals
of $\frak A$, so that $\Cal H\cong\Cal I$, by 383Hb.   For $a$,
$b\in\frak A$, say that $a\preceq b$ if there is some $k\in\Bbb N$ such
that $\bar\mu(a\Bcap e_{\kappa})\le k\bar\mu(b\Bcap e_{\kappa})$ for
every $\kappa\in K$.   Then an ideal $I$ of $\frak A$ is
$\AmuA$-invariant iff
$a\in I$ whenever $a\preceq b\in I$.   \Prf\ ($\alpha$) Suppose
that $I$ is $\AmuA$ invariant and that $b\in I$, $\bar\mu(a\Bcap
e_{\kappa})\le k\bar\mu(b\Bcap e_{\kappa})$ for every $\kappa\in K$.
Then for each $\kappa$ we can find $a_{\kappa 1},\ldots,a_{\kappa k}$
such that $a\Bcap e_{\kappa}=\sup_{i\le k}a_{\kappa i}$ and $\bar\mu
a_{\kappa i}\le\bar\mu(b\Bcap e_{\kappa})$ for every $i$.   Now there
are measure-preserving automorphisms $\pi_{\kappa i}$ of the principal
ideal $\frak A_{e_{\kappa}}$ such that $\pi_{\kappa i}a_{\kappa
i}\Bsubseteq
b$.   Setting $\pi_id=\sup_{\kappa\in K}\pi_{\kappa i}(d\Bcap
e_{\kappa})$ for every $d\in\frak A$, and $a_i=\sup_{\kappa\in
K}a_{\kappa i}$, we have $\pi_i\in\AmuA$ and $\pi_ia_i\Bsubseteq b$, so
$a_i\in I$ for each $i$;  also $a=\sup_{i\le k}a_i$, so $a\in I$.
($\beta$) On the other hand, if $a\in\frak A$ and $\pi\in\AmuA$, then

\Centerline{$\bar\mu(\pi a\Bcap e_{\kappa})
=\bar\mu\pi(a\Bcap e_{\kappa})
=\bar\mu(a\Bcap e_{\kappa})$}

\noindent for every $\kappa\in K$, because $\pi e_{\kappa}=e_{\kappa}$,
so that $\pi
a\preceq a$.   So if $I$ satisfies the condition, $\pi[I]\subseteq I$
for every $\pi\in\AmuA$ and $I\in\Cal I$.\ \Qed\

\medskip

{\bf (b)} Consequently, for $I\in\Cal I$ and $\kappa\in K$,
$e_{\kappa}\in I$ iff there is some $a\in I$ such that $a\Bcap
a_{\kappa}\ne 0$, since in this case $e_{\kappa}\preceq a$.   (This is
where I use the hypothesis that $(\frak A,\bar\mu)$ is totally finite.)
It follows that if $K$ is finite, any $I\in\Cal I$ is the principal
ideal generated by $\sup\{e_{\kappa}:e_{\kappa}\in I\}$.   Conversely,
of course, all such ideals are $\AmuA$-invariant.   Thus $\Cal I$ is in
a natural order-preserving correspondence with $\Cal PK$, and $\Cal
H\cong\Cal PK$.

\medskip

{\bf (c)} Now suppose that $K$ is infinite;  enumerate it as
$\sequencen{\kappa_n}$.   Define $\theta:\frak A\to\ell^{\infty}$ by
setting 

\Centerline{$\theta a
=\sequencen{\bar\mu(a\Bcap e_{\kappa_n})/\bar\mu(e_{\kappa_n})}$}
%\Centerline needed for Lulu version

\noindent for $a\in\frak A$;  so that

\Centerline{$a\preceq b$ iff there is some $k$ such that $\theta a\le
k\theta b$,}

\Centerline{$\theta a\le\theta(a\Bcup b)\le\theta a+\theta b\le
2\theta(a\Bcup b)$}

\noindent for all $a$, $b\in\frak A$, while $\theta(1_{\frak A})$ is the
standard order unit $\chi\Bbb N$ of $\ell^{\infty}$.   
Let $\Cal U$ be the
family of solid linear subspaces of $\ell^{\infty}$ and define functions
$I\mapsto V_I:\Cal I\to\Cal U$, $U\mapsto J_U:\Cal U\to\Cal I$ by saying

\Centerline{$V_I=\{f:f\in\ell^{\infty},\,|f|\le k\theta a$ for some
$a\in I$, $k\in\Bbb N\}$,}

\Centerline{$J_U=\{a:a\in\frak A,\,\theta a\in U\}$.}

\noindent The properties of $\theta$ just listed ensure that
$V_I\in\Cal U$ and  $J_U\in\Cal I$ for every $I\in\Cal I$, $U\in\Cal U$.
Of course
both $I\mapsto V_I$ and $U\mapsto J_U$ are order-preserving.   If
$I\in\Cal I$, then

\Centerline{$J_{V_I}=\{a:\exists\,b\in I,\,a\preceq b\}=I$.}

\noindent Finally, $V_{J_U}=U$ for every $U\in\Cal U$.    \Prf\

\Centerline{$V_{J_U}=\{f:\exists\,a\in\frak A$, $k\in\Bbb N$, $|f|\le
k\theta a\in U\}\subseteq U$}

\noindent because $U$ is a solid linear subspace.   But also, given
$g\in U$, there is an $a\in\frak A$ such that $\bar\mu(a\Bcap
e_{\kappa_n})=\min(1,|g(n)|)\bar\mu(e_{\kappa_n})$ for every $n$
(because $\frak A$ is atomless);  in which case

\Centerline{$\theta a\le|g|\le\max(1,\|g\|_{\infty})\theta a$}

\noindent so $a\in J_U$ and $g\in V_{J_U}$.   Thus $U=V_{J_U}$.\ \QeD\
So the functions $I\mapsto V_I$ and $U\mapsto J_U$ are the two halves of
an order-isomorphism between $\Cal I$ and $\Cal U$, and
$\Cal H\cong\Cal I\cong\Cal U$, as claimed.
}%end of proof of 383J

\leader{383K}{}\cmmnt{ Later in this chapter I will give a good deal
of space to the question of when two automorphisms of a measure algebra
are conjugate.   Because, on any measure algebra $(\frak A,\bar\mu)$, we
have two groups $\Aut\frak A$ and $\AmuA$ with claims on our attention,
we have two different conjugacy relations to examine.   To clear the
ground, I give a result showing that in a significant number of cases
the two coincide.

\medskip

\noindent}{\bf Proposition} Let $(\frak A,\bar\mu)$ be a totally finite
measure algebra and $\pi:\frak A\to\frak A$ an ergodic
measure-preserving Boolean homomorphism.   If $\phi\in\Aut\frak A$ is
such that $\phi\pi\phi^{-1}$ is measure-preserving, then $\phi$ is
measure-preserving.

\proof{ Consider the functional $\nu:\frak A\to\Bbb R$ defined by saying
that $\nu a=\bar\mu(\phi a)$ for every $a\in\frak A$.   Because
$\bar\mu$ is completely additive (321F) and strictly positive, so is
$\nu$.   We therefore have a $c=\Bvalue{\nu>\bar\mu}$ in $\frak A$ such
that $\nu a>\bar\mu a$ whenever $0\ne a\Bsubseteq c$ and
$\nu a\le\bar\mu a$ whenever
$a\Bcap c=0$ (326T).   Now $\pi c=c$.   \Prf\Quer\ Otherwise, because
$\pi$ is measure-preserving,

\Centerline{$\bar\mu(\pi c\Bsetminus c)
=\bar\mu(\pi c)-\bar\mu(c\Bcap\pi c)=\bar\mu c-\bar\mu(c\Bcap\pi c)
=\bar\mu(c\Bsetminus\pi c)=\Bover12\bar\mu(c\Bsymmdiff\pi c)>0$.}

\noindent Next,

\Centerline{$\nu\pi c=\bar\mu(\phi\pi c)
=\bar\mu(\phi\pi\phi^{-1}\phi c)=\nu c$,}

\noindent so we also have
$\nu(\pi c\Bsetminus c)=\nu(c\Bsetminus\pi c)$.   But now observe that

\Centerline{$\nu(\pi c\Bsetminus c)\le\bar\mu(\pi c\Bsetminus c)$,
\quad$\nu(c\Bsetminus\pi c)>\bar\mu(c\Bsetminus\pi c$}

\noindent by the choice of $c$, which is impossible.\ \Bang\Qed

Because $\pi$ is ergodic, $c$ must be $0$ or $1$ (372Pa).   But as
$\nu\pi 1=\nu 1=\bar\mu 1$, we cannot have $0\ne 1\Bsubseteq c$, so
$c=0$.   This means that $\nu a\le\bar\mu a$ for every $a\in\frak A$;
once again, $\nu 1=\bar\mu 1$, so in fact $\nu a=\bar\mu a$ for every
$a$, that is, $\phi$ is measure-preserving.
}%end of proof of 383K

\leader{383L}{Corollary} Let $(\frak A,\bar\mu)$ be a totally finite
measure algebra, and $\pi_1$, $\pi_2\in\AmuA$ two ergodic
measure-preserving automorphisms.   If they are conjugate in
$\Aut\frak A$ then they are conjugate in $\AmuA$.

\proof{ There is a $\phi\in\Aut\frak A$ such that
$\phi\pi_1\phi^{-1}=\pi_2$;  now 383K tells us that $\phi\in\AmuA$.
}%end of proof of 383L

\exercises{
\leader{383X}{Basic exercises (a)}
%\spheader 383Xa
Let $(X,\Sigma,\mu)$ be a countably separated measure space, and write
$\Aut_{\mu}\Sigma$ for the group of automorphisms $\phi:\Sigma\to\Sigma$
such that $\mu\phi(E)=\mu E$ for every $E\in\Sigma$.   Show that every
member of $\Aut_{\mu}\Sigma$ is expressible as a product of at most
three involutions belonging to $\Aut_{\mu}\Sigma$.   \Hint{382Xc.}
%383D

\sqheader 383Xb
Let $(\frak A,\bar\mu)$ be a localizable measure
algebra.   For each infinite cardinal $\kappa$, let $e_{\kappa}$ be the
Maharam-type-$\kappa$ component of $\frak A$.   (i) Show that $\AmuA$ is
a simple group iff {\it either} there is just one infinite cardinal
$\kappa$ such that $e_{\kappa}\ne 0$, that $e_{\kappa}$ has
finite measure and all the atoms of $\frak A$ (if any) have different
measures {\it or} $\frak A$ is purely atomic and there is just one pair
of atoms of the same measure {\it or} $\frak A$ is purely atomic and all
its atoms have different measures.   (ii) Show that $\Aut\frak A$ is a
simple group iff {\it either} $(\frak A,\bar\mu)$ is $\sigma$-finite and
there is just one infinite cardinal
$\kappa$ such that $e_{\kappa}\ne 0$ and $\frak A$ has at most one atom
{\it or} $\frak
A$ is purely atomic and has at most two atoms.
%383I

\spheader 383Xc Let $(\frak A,\bar\mu)$ be a localizable measure
algebra.   (i) Show that $\AmuA$ is simple iff it is isomorphic to one
of the groups $\{\iota\}$, $\Bbb Z_2$ or
$\Aut_{\bar\nu_{\kappa}}\frak B_{\kappa}$ where $\kappa$ is an infinite
cardinal and $(\frak B_{\kappa},\bar\nu_{\kappa})$ is the measure
algebra of the usual
measure on $\{0,1\}^{\kappa}$.   (ii) Show that $\Aut\frak A$ is simple
iff it is isomorphic to one of the groups $\{\iota\}$, $\Bbb Z_2$ or
$\Aut\frak B_{\kappa}$.
%383I

\spheader 383Xd Show that if $(\frak A,\bar\mu)$ is a semi-finite
measure algebra of magnitude greater than $\frak c$, its automorphism
group $\AmuA$ is not simple.
%383I

\spheader 383Xe
Let $(\frak A,\bar\mu)$ be an atomless localizable
measure algebra.    For each infinite cardinal $\kappa$ write
$e_{\kappa}$ for the
Maharam-type-$\kappa$ component of $\frak A$.   For $\pi$,
$\psi\in\AmuA$ show that $\pi$ belongs to the normal subgroup of $\AmuA$
generated by $\psi$ iff there is a $k\in\Bbb N$ such that

\Centerline{$\magnitude(e_{\kappa}\Bcap\supp\pi)\le
k\magnitude(e_{\kappa}\Bcap\supp\psi)$ for every infinite cardinal
$\kappa$,}

\noindent writing $\magnitude a$ for the magnitude of $a$, and setting
$k\zeta=\zeta$ if $k>0$ and $\zeta$ is an infinite cardinal.
%383J

\sqheader 383Xf Let $(\frak A,\bar\mu)$ be the measure algebra of
Lebesgue measure on $\Bbb R$.   For $n\in\Bbb N$ set
$e_n=[-n,n]^{\ssbullet}\in\frak A$.   Let $G\le\AmuA$ be the group
consisting of measure-preserving automorphisms $\pi$ such that
$\supp\pi\Bsubseteq e_n$ for some $n$.   Show that $G$ is simple.
({\it Hint\/}:  show that $G$ is the union of an increasing sequence of
simple subgroups.)
%383J

\spheader 383Xg Let $(\frak A,\bar\mu)$ be an atomless totally finite
measure algebra.   Let $\Cal H$ be the lattice of normal subgroups of
$\Aut\frak A$.   Show that $\Cal H$ is isomorphic, as partially ordered
set, to $\Cal PK$ for some countable set $K$.
%383J

\spheader 383Xh Let $(\frak A,\bar\mu)$ be an atomless localizable
measure algebra which is not $\sigma$-finite, and suppose that
$\tau(\frak A_a)=\tau(\frak A_b)$ whenever $a$, $b\in\frak A$ and
$0<\bar\mu a\le\bar\mu b<\infty$.   Let $\kappa$ be the magnitude of
$\frak A$.   (i) Show that the lattice $\Cal H$ of normal subgroups of
$\AmuA$ is well-ordered, with least member $\{\iota\}$ and one
member $H_{\zeta}$ for each infinite cardinal $\zeta$ less than or equal
to $\kappa^+$, setting

\Centerline{$H_{\zeta}=\{\pi:\pi\in \AmuA$,
$\magnitude(\supp\pi)<\zeta\}$,}

\noindent where $\magnitude a$ is the magnitude of $a$.   (ii) Show that
the lattice $\Cal H'$ of normal subgroups of $\Aut\frak A$ is
well-ordered, with least member $\{\iota\}$ and one member $H'_{\zeta}$
for each uncountable cardinal $\zeta$ less than or equal to $\kappa^+$,
setting

\Centerline{$H'_{\zeta}=\{\pi:\pi\in \Aut\frak A$,
$\magnitude(\supp\pi)<\zeta\}$.}
%383J

\spheader 383Xi Let $(\frak A,\bar\mu)$ be the measure algebra of
Lebesgue measure on $[0,1]$.   Give an example of two measure-preserving
automorphisms of $\frak A$ which are conjugate in $\Aut\frak A$ but not
in $\AmuA$.
%383L

\spheader 383Xj Let $(\frak A,\bar\mu)$ be a probability algebra.   For
$\pi$, $\phi\in\AmuA$ set

\Centerline{$\rho(\pi,\phi)
=\sup_{a\in\frak A}\bar\mu(\pi a\Bsymmdiff\phi a)$,
\quad$\sigma(\pi,\phi)=\bar\mu(\supp(\pi^{-1}\phi))$.}

\noindent(i) Show that $\rho$ and $\sigma$ are metrics on $\AmuA$, and that
$\rho\le\sigma\le\bover32\rho$.   \Hint{382Eb.}   (ii) Show that
$\rho(\psi\pi,\psi\phi)=\rho(\pi\psi,\phi\psi)=\rho(\pi,\phi)$, 
$\rho(\pi^{-1},\phi^{-1})=\rho(\pi,\phi)$,
$\rho(\pi\psi,\phi\theta)\le\rho(\pi,\phi)+\rho(\psi,\theta)$,
$\sigma(\psi\pi,\psi\phi)=\sigma(\pi\psi,\phi\psi)=\sigma(\pi,\phi)$,
$\sigma(\pi^{-1},\phi^{-1})=\sigma(\pi,\phi)$,
$\sigma(\pi\psi,\phi\theta)\le\sigma(\pi,\phi)+\sigma(\psi,\theta)$
for all $\pi$, $\phi$, $\psi$, $\theta\in\AmuA$.   
(iii) Show that $\AmuA$ is complete
under $\rho$ and $\sigma$.

\spheader 383Xk\dvAnew{2009} Let $(X,\Sigma,\mu)$ be a measure space and 
$(\frak A,\bar\mu)$ its measure algebra.   Let $S$ be the set of functions
which are isomorphisms between conegligible measurable
subsets of $X$ with their
subspace measures.   (i) Show that the composition of two members of $S$
belongs to $S$.  (ii) Show that there is a map
$f\mapsto\pi_f:S\to\AmuA$ defined by saying that
$\pi_f(E^{\ssbullet})=f^{-1}[E]^{\ssbullet}$ for every $E\in\Sigma$, and
that $\pi_{fg}=\pi_f\pi_g$, $\pi_f^{-1}=\pi_{f^{-1}}$ 
for all $f$, $g\in S$.
(iii) Show that $\{\pi_f:f\in S\}$ is a countably full subgroup of
$\AmuA$.

\sqheader 383Xl\dvAnew{2009} Let $(X,\Sigma,\mu)$ be a measure space and
$(\frak A,\bar\mu)$ its measure algebra.   Let $\Phi$ be the group of
measure space automorphisms of $(X,\Sigma,\mu)$.   For $f\in\Phi$,
let $\pi_f\in\AmuA$ be the corresponding automorphism, defined
by setting $\pi_f(E^{\ssbullet})=(f^{-1}[E])^{\ssbullet}$ for every
$E\in\Sigma$.   (i) Show that $f\mapsto\pi_f$ is a group homomorphism from
$\Phi$ to $\AmuA$.   (ii) Show that if $F\subseteq\Phi$ and the subgroup of
$\Phi$ generated by $F$ is $\Psi$, then the subgroup of $\AmuA$ generated
by $\{\pi_f:f\in F\}$ is $\{\pi_f:f\in\Psi\}$.   (iii) Show that if
$(X,\Sigma,\mu)$ is countably separated (definition: 343D) and
$F\subseteq\Phi$ is countable, then the full subgroup of $\AmuA$ generated
by $\{\pi_f:f\in F\}$ is $\{\pi_g:g\in F^*\}$, where

\Centerline{$F^*
=\{g:g\in\Phi$, $g(x)\in\{f(x):x\in F\}$ for every $x\in X\}$.}

\leader{383Y}{Further exercises (a)}
%\spheader 383Ya
Let $(\frak A,\bar\mu)$ be an
atomless totally finite measure algebra.   Show that $\AmuA$ and
$\Aut\frak A$ have the same (cardinal) number of normal subgroups.

\spheader 383Yb Let $X$ be a set.   Show that $\Aut\Cal PX$ has one
normal subgroup if $\#(X)\le 1$, two if $\#(X)=2$, three if $\#(X)=3$ or
$5\le\#(X)\le\omega$, four if $\#(X)=4$ or $\#(X)=\omega$, five if
$\#(X)=\omega_1$.
}%end of exercises

\cmmnt{
\Notesheader{383} This section is short because there are no substantial
new techniques to be developed.  383D is simply a matter of checking
that the hypotheses of 382N are satisfied (and these hypotheses were of
course chosen with 383D in mind), and 383I is similarly direct from
382R.   383I-383J, 383Xe and 383Xh are variations on a theme.   In a
general Boolean algebra $\frak A$ with a group $G$ of automorphisms, we
have a transitive, reflexive relation $\preceq_G$ defined by saying that
$a\preceq_Gb$ if there are $\pi_1,\ldots,\pi_k\in G$ such that
$a\Bsubseteq\sup_{i\le k}\pi_ib$;  the point about localizable measure
algebras is that the functions `Maharam type' and `magnitude'
enable us to describe this relation when $G=\AmuA$, and the essence of
382R is that in that context $\pi$ belongs to the normal subgroup of $G$
generated by $\psi$ iff $\supp\pi\preceq_G\supp\psi$.

Some of the most interesting questions concerning automorphism groups of
measure algebras can be expressed in the form `how can we determine when
a given pair of automorphisms are conjugate?'   Generally, people have
concentrated on conjugacy in $\AmuA$.   But the same question can be
asked in $\Aut\frak A$.   In particular, it is possible for two members
of $\AmuA$ to be conjugate in $\Aut\frak A$ but not in $\AmuA$ (383Xi).
However this phenomenon does not occur for {\it ergodic} automorphisms,
or even for ergodic measure-preserving Boolean homomorphisms
(383K-383L).

Most of the work of this chapter is focused on atomless measure
algebras.   There are various extra complications which appear if we
allow atoms.   The most striking are in the next section;  here I
mention only 383Xb and 383Yb.
}%end of notes

\discrpage

