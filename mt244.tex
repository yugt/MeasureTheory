\frfilename{mt244.tex}
\versiondate{6.3.09}
\copyrightdate{1995}

\def\chaptername{Function spaces}
\def\sectionname{$L^{p}$}

\newsection{244}

Continuing with our tour of the classical Banach spaces, we come to
the $L^p$ spaces for $1<p<\infty$.   The case $p=2$ is more important
than all the others put together, and it would be reasonable, perhaps
even advisable, to read this section first with this case alone in mind.
But the other spaces provide instructive examples and remain a basic
part of the education of any functional analyst.

\leader{244A}{Definitions}
Let $(X,\Sigma,\mu)$ be any measure space, and $p\in\ooint{1,\infty}$.
Write $\eusm L^p=\eusm L^p(\mu)$ for the set of functions
$f\in\eusm L^0=\eusm L^0(\mu)$ such that $|f|^p$ is integrable,
and $L^p=L^p(\mu)$ for
$\{f^{\ssbullet}:f\in\eusm L^p\}\subseteq L^0=L^0(\mu)$.

\cmmnt{Note that if $f\in\eusm L^p$, $g\in\eusm L^0$ and $f\eae g$,
then $|f|^p\eae|g|^p$ so $|g|^p$ is integrable and $g\in\eusm L^p$;
thus} $\eusm L^p=\{f:f\in\eusm L^0,\,f^{\ssbullet}\in L^p\}$.

\cmmnt{Alternatively, we can define $u^p$ whenever $u\in L^0$,
$u\ge 0$ by writing $(f^{\ssbullet})^p=(f^p)^{\ssbullet}$ for every
$f\in\eusm L^0$ such that $f(x)\ge 0$ for every $x\in\dom f$ (compare
241I), and say that $L^p=\{u:u\in L^0,\,|u|^p\in L^1(\mu)\}$.
}

\leader{244B}{Theorem} Let $(X,\Sigma,\mu)$ be any measure space, and
$p\in[1,\infty]$.

(a) $L^p=L^p(\mu)$ is a linear subspace of $L^0=L^0(\mu)$.

(b) If $u\in L^p$, $v\in L^0$ and $|v|\le |u|$, then $v\in L^p$.
Consequently $|u|$, $u\vee v$ and $u\wedge v$ belong to $L^p$ for all
$u$, $v\in L^p$.

\proof{ The cases $p=1$, $p=\infty$ are covered by 242B, 242C and 243B;
so I suppose that $1<p<\infty$.

\medskip

{\bf (a)(i)}  Suppose that $f$, $g\in\eusm L^p=\eusm L^p(\mu)$.   If
$a$, $b\in\Bbb R$ then $|a+b|^p\le
2^p\max(|a|^p,|b|^p)$, so $|f+g|^p\leae 2^p(|f|^p\vee|g|^p)$;  now
$|f+g|^p\in\eusm L^0$ and
$2^p(|f|^p\vee|g|^p)\in\eusm L^1$ so $|f+g|^p\in\eusm L^1$.   Thus
$f+g\in\eusm L^p$ for all $f$, $g\in \eusm L^p$;  it follows at once
that $u+v\in L^p$ for all $u$, $v\in L^p$.

\medskip
\quad{\bf (ii)} If $f\in \eusm L^p$ and $c\in\Bbb R$ then
$|cf|^p=|c|^p|f|^p\in\eusm L^1$, so $cf\in\eusm L^p$.
Accordingly $cu\in L^p$ whenever $u\in L^p$ and $c\in\Bbb R$.

\medskip

{\bf (b)(i)} Express $u$ as $f^{\ssbullet}$ and $v$ as $g^{\ssbullet}$,
where $f\in\eusm L^p$ and $g\in\eusm L^0$.   Then $|g|\leae|f|$, so
$|g|^p\leae|f|^p$ and $|g|^p$ is integrable;   accordingly
$g\in\eusm
L^p$ and $v\in L^p$.

\medskip

\quad{\bf (ii)} Now $|\,|u|\,|=|u|$ so $|u|\in L^p$ whenever $u\in L^p$.
Finally $u\vee v=\bover12(u+v+|u-v|)$ and
$u\wedge v=\bover12(u+v-|u-v|)$
belong to $L^p$ for all $u$, $v\in L^p$.
}

\leader{244C}{The order structure of $L^p$} Let $(X,\Sigma,\mu)$ be any
measure space, and $p\in[1,\infty]$.   Then\cmmnt{ 244B is enough to
ensure that} the partial order inherited from $L^0(\mu)$ makes
$L^p(\mu)$ a Riesz space\cmmnt{ (compare 242C, 243C)}.

\leader{244D}{The norm of $L^p$} Let $(X,\Sigma,\mu)$ be a measure
space, and $p\in\ooint{1,\infty}$.

\spheader 244Da For $f\in\eusm L^p=\eusm L^p(\mu)$, set
$\|f\|_p=(\int|f|^p)^{1/p}$.
If $f$, $g\in\eusm L^p$ and $f\eae g$ then
$|f|^p\eae|g|^p$ so $\|f\|_p=\|g\|_p$.   Accordingly we may define
$\|\,\|_p:L^p=L^p(\mu)\to\coint{0,\infty}$ by writing
$\|f^{\ssbullet}\|_p=\|f\|_p$ for every $f\in\eusm L^p$.

\cmmnt{Alternatively, we can say just that $\|u\|_p=(\int|u|^p)^{1/p}$
for every $u\in L^p=L^p(\mu)$.}

\spheader 244Db\cmmnt{ The notation $\|\,\|_p$ carries a
promise that it is
a norm on $L^p$;  this is indeed so, but I hold the proof over to 244F
below. For the moment, however, let us note just that}
$\|cu\|_p=|c|\|u\|_p$ for
all $u\in L^p$ and $c\in\Bbb R$, and\cmmnt{ that} if $\|u\|_p=0$
then\cmmnt{ $\int|u|^p=0$ so $|u|^p=0$ and} $u=0$.

\spheader 244Dc If $|u|\le |v|$ in $L^p$
then $\|u\|_p\le\|v\|_p$\cmmnt{;  this is because $|u|^p\le|v|^p$}.

\leader{244E}{}\cmmnt{ I now work through the lemmas required to show
that $\|\,\|_p$ is a norm on $L^p$ and, eventually, that the normed
space dual of $L^p$ may be identified with a suitable $L^q$.

\medskip

\noindent}{\bf Lemma} Suppose $(X,\Sigma,\mu)$ is a measure space, and
that
$p$, $q\in\ooint{1,\infty}$ are such that ${1\over p}+{1\over q}=1$.

(a) $ab\le{1\over p}a^p+{1\over q}b^q$ for all real $a$, $b\ge 0$.

(b)(i) $f\times g$ is integrable and

\Centerline{$|\int f\times g|\le\int|f\times g|\le\|f\|_p\|g\|_q$}

\noindent for all $f\in \eusm L^p=\eusm L^p(\mu)$,
$g\in\eusm L^q=\eusm L^q(\mu)$;

\quad(ii) $u\times v\in L^1=L^1(\mu)$ and

\Centerline{$|\int u\times v|\le\|u\times v\|_1\le\|u\|_p\|v\|_q$}

\noindent for all $u\in L^p=L^p(\mu)$, $v\in L^q=L^q(\mu)$.


\proof{{\bf (a)} If either $a$ or $b$ is $0$, this is trivial.
If both are non-zero, we may argue as follows.   The function
$x\mapsto x^{1/p}:\coint{0,\infty}\to\Bbb R$ is concave, with second
derivative
strictly less than $0$, so lies entirely below any of its tangents;  in
particular, below its tangent at the point $(1,1)$, which has equation
$y=1+{1\over p}(x-1)$.   Thus we have

\Centerline{$x^{1/p}\le\Bover1px+1-\Bover1p=\Bover1px+\Bover1q$}

\noindent for every $x\in\coint{0,\infty}$.   So if $c$, $d>0$, then

\Centerline{$(\Bover{c}{d})^{1/p}\le
\Bover1p\Bover{c}{d}+\Bover1q$;}

\noindent multiplying both sides by $d$,

\Centerline{$c^{1/p}d^{1/q}\le\Bover1pc+\Bover1qd$;}

\noindent setting $c=a^p$ and $d=b^q$, we get

\Centerline{$ab\le\Bover1pa^p+\Bover1qb^q$,}

\noindent as claimed.

\medskip

{\bf (b)(i)}($\alpha$) Suppose first that $\|f\|_p=\|g\|_q=1$.
For every $x\in\dom f\cap\dom g$ we have

\Centerline{$|f(x)g(x)|\le\Bover1p|f(x)|^p+\Bover1q|g(x)|^q$}

\noindent by (a).   So

\Centerline{$|f\times g|\leae
\Bover1p|f|^p+\Bover1q|g|^q\in\eusm L^1(\mu)$}

\noindent and $f\times g$ is integrable;  also

\Centerline{$\int|f\times g|\le\Bover1p\int|f|^p+\Bover1q\int|g|^q
=\Bover1p\|f\|_p^p+\Bover1q\|g\|_q^q
=\Bover1p+\Bover1q=1$.}

\noindent ($\beta$) If $\|f\|_p=0$, then $\int|f|^p=0$ so
$|f|^p\eae\tbf{0}$, $f\eae\tbf{0}$, $f\times g\eae\tbf{0}$ and

\Centerline{$\int|f\times g|=0=\|f\|_p\|g\|_q$.}

\noindent Similarly, if $\|g\|_q=0$, then $g\eae\tbf{0}$ and again

\Centerline{$\int|f\times g|=0=\|f\|_p\|g\|_q$.}

\noindent($\gamma$) Finally, for general $f\in\eusm L^p$,
$g\in\eusm L^q$ such that $c=\|f\|_p$ and $d=\|g\|_q$ are both non-zero,
we have $\|\bover1cf\|_p=\|\bover1dg\|_q=1$ so

\Centerline{$f\times g=cd(\Bover1cf\times\Bover1dg)$}

\noindent is integrable, and

\Centerline{$\int|f\times g|
=cd\int|\Bover1cf\times\Bover1dg|\le cd$,}

\noindent as required.

\medskip

\quad{\bf (ii)} Now if $u\in L^p$, $v\in L^q$ take $f\in\eusm L^p$,
$g\in \eusm L^q$ such that $u=f^{\ssbullet}$ and $v=g^{\ssbullet}$;
$f\times g$ is integrable, so $u\times v\in L^1$, and

\Centerline{$|\int u\times v|\le\|u\times v\|_1
=\int|f\times g|\le\|f\|_p\|g\|_q=\|u\|_p\|v\|_q$.}
}

\cmmnt{\medskip

\noindent{\bf Remark} Part (b) is `{\bf H\"older's
inequality}'.   In the case $p=q=2$ it is `{\bf Cauchy's inequality}'.
}

\leader{244F}{Proposition} Let $(X,\Sigma,\mu)$ be a measure space and
$p\in\ooint{1,\infty}$.   Set $q=p/(p-1)$, so that
${1\over p}+{1\over q}=1$.

(a) For every $u\in L^p=L^p(\mu)$,
$\|u\|_p=\max\{\int u\times v:v\in L^q(\mu),\,\|v\|_q\le 1\}$.

(b) $\|\,\|_p$ is a norm on $L^p$.

\proof{{\bf (a)} For $u\in L^p$, set

\Centerline{$\tau(u)
=\sup\{\int u\times v:v\in L^q(\mu),\,\|v\|_q\le 1\}$.}

\noindent By 244E(b-ii), $\|u\|_p\ge\tau(u)$.   If
$\|u\|_p=0$ then surely

\Centerline{$0=\|u\|_p=\tau(u)
=\max\{\int u\times v:v\in L^q(\mu),\,\|v\|_q\le 1\}$.}

\noindent   If $\|u\|_p=c>0$, consider

\Centerline{$v=c^{-p/q}\sgn u\times|u|^{p/q}$,}

\noindent where for $a\in\Bbb R$ I write $\sgn a=|a|/a$ if $a\ne 0$, $0$
if $a=0$, so that $\sgn:\Bbb R\to\Bbb R$ is a Borel measurable function;
for $f\in\eusm L^0$ I write $(\sgn f)(x)=\sgn(f(x))$ for $x\in\dom f$,
so that $\sgn f\in\eusm L^0$;  and for $f\in\eusm L^0$ I write
$\sgn(f^{\ssbullet})=(\sgn f)^{\ssbullet}$ to define a function
$\sgn:L^0\to L^0$ (cf.\ 241I).   Then $v\in L^q=L^q(\mu)$ and

\Centerline{$\|v\|_q=(\int|v|^q)^{1/q}=c^{-p/q}(\int|u|^p)^{1/q}
=c^{-p/q}c^{p/q}=1$.}

\noindent So

$$\eqalign{\tau(u)\ge\int u\times v
&=c^{-p/q}\int\sgn u\times|u|\times\sgn u\times|u|^{p/q}\cr
&=c^{-p/q}\int|u|^{1+{p\over q}}
=c^{-p/q}\int|u|^p
=c^{p-{p\over q}}
=c,\cr}$$

\noindent recalling that $1+{p\over q}=p$, $p-{p\over q}=1$.
Thus $\tau(u)\ge\|u\|_p$ and

\Centerline{$\tau(u)=\|u\|_p=\int u\times v$.}

\medskip

{\bf (b)} In view of the remarks in 244Db, I have only to check that
$\|u+v\|_p\le\|u\|_p+\|v\|_p$ for all $u$, $v\in L^p$.   But given $u$
and $v$, let $w\in L^q$ be such that $\|w\|_q=1$ and $\int (u+v)\times
w=\|u+v\|_p$.   Then

\Centerline{$\|u+v\|_p=\int(u+v)\times w
=\int u\times w+\int v\times w\le\|u\|_p+\|v\|_p$,}

\noindent as required.
}%end of proof of 244F

\cmmnt{\medskip

\noindent{\bf Remark} The triangle
inequality `$\|u+v\|_p\le\|u\|_p+\|v\|_p$'
is {\bf Minkowski's inequality}.}

\leader{244G}{Theorem} Let $(X,\Sigma,\mu)$ be any measure space, and
$p\in[1,\infty]$.   Then $L^p=L^p(\mu)$ is a Banach lattice under
its norm $\|\,\|_p$.

\proof{ The cases $p=1$, $p=\infty$ are covered by 242F and 243E,
so let us suppose that $1<p<\infty$.    We know already that
$\|u\|_p\le\|v\|_p$ whenever $|u|\le|v|$, so
it remains only to show that $L^p$ is complete.

Let $\sequencen{u_n}$ be a sequence in
$L^p$ such that $\|u_{n+1}-u_n\|_p\le 4^{-n}$ for every $n\in\Bbb N$.
Note that

\Centerline{$\|u_n\|_p\le\|u_0\|_p+\sum_{k=0}^{n-1}\|u_{k+1}-u_k\|_p
\le\|u_0\|_p+\sum_{k=0}^{\infty}4^{-k}\le\|u_0\|_p+2$}

\noindent for every $n$.   For each $n\in\Bbb N$, choose
$f_n\in\eusm L^p$ such that $f_0^{\ssbullet}=u_{0}$,
$f_n^{\ssbullet}=u_{n}-u_{n-1}$ for $n\ge 1$;  do this in such a way
that $\dom f_n=X$ and $f_n$ is $\Sigma$-measurable (241Bk).   Then
$\|f_n\|_p\le 4^{-n+1}$ for $n\ge 1$.

For $m$, $n\in\Bbb N$, set

\Centerline{$E_{mn}=\{x:|f_m(x)|\ge 2^{-n}\}\in\Sigma$.}

\noindent Then $|f_m(x)|^p\ge 2^{-np}$ for $x\in E_{mn}$, so

\Centerline{$2^{-np}\mu E_{mn}\le\int|f_m|^p<\infty$}

\noindent and  $\mu E_{mn}<\infty$.
So $\chi E_{mn}\in\eusm L^q=\eusm L^q(\mu)$ and

\Centerline{$\int_{E_{mn}}|f_k|=\int|f_k|\times\chi E_{mn}
\le\|f_k\|_p\|\chi E_{mn}\|_q$}

\noindent for each $k$, by 244E(b-i).   Accordingly

\Centerline{$\sum_{k=0}^{\infty}\int_{E_{mn}}|f_k|
\le\|\chi E_{mn}\|_q\sum_{k=0}^{\infty}\|f_k\|_p<\infty$,}

\noindent and $\sum_{k=0}^{\infty}f_k(x)$ exists for
almost every $x\in E_{mn}$, by 242E.   This is true for all $m$,
$n\in\Bbb N$.   But if
$x\in X\setminus\bigcup_{m,n\in\Bbb N}E_{mn}$, $f_n(x)=0$ for every $n$,
so $\sum_{k=0}^{\infty}f_k(x)$ certainly exists.   Thus
$g(x)=\sum_{k=0}^{\infty}f_k(x)$ is defined in $\Bbb R$ for almost every
$x\in X$.

Set $g_n=\sum_{k=0}^nf_k$;  then $g_n^{\ssbullet}=u_{n}\in L^p$ for each
$n$, and $g(x)=\lim_{n\to\infty}g_n(x)$ is defined for almost every $x$.
Now consider $|g|^p\eae\lim_{n\to\infty}|g_n|^p$.   We know that

\Centerline{$\liminf_{n\to\infty}\int|g_n|^p
=\liminf_{n\to\infty}\|u_{n}\|_p^p\le(2+\|u_{0}\|_p)^p<\infty$,}

\noindent so by Fatou's Lemma

\Centerline{$\int|g|^p\le\liminf_{k\to\infty}\int|g_k|^p<\infty$.}

\noindent Thus $u=g^{\ssbullet}\in L^p$.   Moreover, for any $m\in\Bbb
N$,

$$\eqalign{\int|g-g_m|^p
&\le\liminf_{n\to\infty}\int|g_n-g_m|^p
=\liminf_{n\to\infty}\|u_{n}-u_{m}\|_p^p\cr
&\le\liminf_{n\to\infty}\sum_{k=m}^{n-1}4^{-kp}
=\sum_{k=m}^{\infty}4^{-kp}
=4^{-mp}/(1-4^{-p}).\cr}$$

\noindent So

\Centerline{$\|u-u_{m}\|_p=(\int|g-g_m|^p)^{1/p}
\le 4^{-m}/(1-4^{-p})^{1/p}\to 0$}

\noindent as $m\to\infty$.    Thus $u=\lim_{m\to\infty}u_m$ in $L^p$.
As $\sequencen{u_n}$ is arbitrary, $L^p$ is complete.
}

\leader{244H}{}\cmmnt{ Following 242M and 242O, I note that $L^p$
behaves like $L^1$ in respect of certain dense subspaces.

\medskip

\noindent}{\bf Proposition} (a) Let $(X,\Sigma,\mu)$ be any measure
space, and $p\in\coint{1,\infty}$.
Then the space $S$ of equivalence classes of
$\mu$-simple functions is a dense linear subspace of $L^p=L^p(\mu)$.

(b) Let $X$ be any subset of $\BbbR^r$, where $r\ge 1$, and let
$\mu$ be the subspace measure on $X$
induced by Lebesgue measure on $\BbbR^r$.   Write $C_k$ for the
set of (bounded) continuous functions $g:\BbbR^r\to\Bbb R$
such that $\{x:g(x)\ne 0\}$ is bounded, and $S_0$ for the space
of linear combinations of functions of the form $\chi I$, where
$I\subseteq\BbbR^r$ is a bounded half-open interval.   Then
$\{(g\restr X)^{\ssbullet}:g\in C_k\}$ and
$\{(h\restr X)^{\ssbullet}:h\in S_0\}$ are dense in
$L^p(\mu)$.

\proof{{\bf (a)} I repeat the argument of 242M with a tiny modification.

\medskip

\quad{\bf (i)} Suppose that $u\in L^p(\mu)$, $u\ge 0$ and
$\epsilon>0$.   Express $u$ as $f^{\ssbullet}$ where
$f:X\to\coint{0,\infty}$ is a measurable function.   Let $g:X\to\Bbb R$
be a simple function such that $0\le g\le f^p$ and
$\int g\ge\int f^p-\epsilon^p$.   Set $h=g^{1/p}$.
Then $h$ is a simple function and $h\le f$.   Because $p>1$,
$(f-h)^p+h^p\le f^p$ and

\Centerline{$\int(f-h)^p\le \int f^p-g\le\epsilon^p$,}

\noindent so

\Centerline{$\|u-h^{\ssbullet}\|_p
=(\int|f-h|^p)^{1/p}\le\epsilon$,}

\noindent while $h^{\ssbullet}\in S$.

\medskip

\quad{\bf (ii)} For general $u\in L^p$, $\epsilon>0$, $u$ can be
expressed as
$u^+-u^-$ where $u^+=u\vee 0$, $u^-=(-u)\vee 0$ belong to $L^p$ and are
non-negative.   By (i), we can find $v_1$, $v_2\in S$ such that
$\|u^+-v_1\|_p\le\bover12\epsilon$ and $\|u^--v_2\|_p\le\bover12\epsilon$,
so that $v=v_1-v_2\in S$ and $\|u-v\|_p\le\epsilon$.   As $u$ and
$\epsilon$ are arbitrary, $S$ is dense.

\medskip

{\bf (b)} Again, all the ideas are to be found in 242O;  the changes
needed are in the formulae, not in the method.   They will go more
easily if I note at the outset that whenever $f_1$,
$f_2\in\eusm L^p(\mu)$ and
$\int|f_1|^p\le\epsilon^p$, $\int|f_2|^p\le\delta^p$ (where $\epsilon$,
$\delta\ge 0$), then $\int|f_1+f_2|^p\le(\epsilon+\delta)^p$;  this is
just the triangle inequality for $\|\,\|_p$ (244Fb).   Also I will
regularly express the target relationships in the form
`$\int_X|f-g|^p\le\epsilon^p$', `$\int_X|f-g|^p\le\epsilon^p$'.
Now let me run
through the argument of 242Oa, rather more briskly than before.

\medskip

\quad{\bf (i)} Suppose first that $f=\chi I\restr X$ where
$I\subseteq\BbbR^r$ is a bounded half-open interval.
As before, we can set $h=\chi I$ and get $\int_X|f-h|^p=0$.
This time, use the same construction to find an interval $J$ and a
function $g\in C_k$ such
that $\chi I\le g\le \chi J$ and $\mu_r(J\setminus I)\le\epsilon^p$;
this will ensure that $\int_X|f-g|^p\le\epsilon^p$.

\medskip

\quad{\bf (ii)} Now suppose that $f=\chi(X\cap E)$ where
$E\subseteq\BbbR^r$ is a
set of finite measure.   Then, for the same reasons as before, there is
a disjoint family
$I_0,\ldots,I_n$ of half-open intervals such that
$\mu_r(E\symmdiff\bigcup_{j\le n}I_j)\le(\bover12\epsilon)^p$.
Accordingly $h=\sum_{j=0}^n\chi I_j\in S_0$ and
$\int_X|f-h|^p\le(\bover12\epsilon)^p$.
And (i) tells us that there is for each
$j\le n$ a $g_j\in C_k$ such that
$\int_X|g_j-\chi I_j|^p\le(\epsilon/2(n+1))^p$, so that
$g=\sum_{j=0}^ng_j\in C_k$ and $\int_X|f-g|^p\le\epsilon^p$.

\medskip

\quad{\bf (iii)} The move to simple functions, and thence to arbitrary
members of $\eusm L^p(\mu)$, is just as before, but using $\|f\|_p$ in
place of $\int_X|f|$.   Finally, the translation from $\eusm L^p$ to
$L^p$ is again direct -- remembering, as before, to check that
$g\restr X$, $h\restr X$ belong to $\eusm L^p(\mu)$ whenever
$g\in C_k$ and $h\in S_0$.
}%end of proof of 244H

\leader{*244I}{Corollary} In the context of 244Hb, $L^p(\mu)$ is
separable.

\proof{ Let $A$ be the set

\Centerline{$\{(\sum_{j=0}^nq_j\chi(\coint{a_j,b_j}\cap X))^{\ssbullet}:
n\in\Bbb N$, $q_0,\ldots,q_n\in\Bbb Q$,
$a_0,\ldots,a_n,b_0,\ldots,b_n\in\BbbQ^r\}$.}

\noindent Then $A$ is a countable subset of $L^p(\mu)$, and its closure
must contain
$(\sum_{j=0}^nc_j\chi(\coint{a_j,b_j}\cap X))^{\ssbullet}$ whenever
$c_0,\ldots,c_n\in\Bbb R$ and
$a_0,\ldots,a_n,b_0,\ldots,b_n\in\BbbR^r$;  that is, $\overline A$ is
a closed set including $\{(h\restr X)^{\ssbullet}:h\in S_0\}$, and is
the whole of $L^p(\mu)$, by 244Hb.
}%end of proof of 244I

\leader{244J}{Duality in $L^p$ spaces} Let $(X,\Sigma,\mu)$ be any
measure space, and $p\in\ooint{1,\infty}$.
Set $q=p/(p-1)$\cmmnt{;  note that ${1\over p}+{1\over q}=1$ and that
$p=q/(q-1)$;  the relation between $p$ and $q$ is symmetric}.   Now
$u\times v\in L^1(\mu)$ and $\|u\times v\|_1\le\|u\|_p\|v\|_q$ whenever
$u\in L^p=L^p(\mu)$ and $v\in L^q=L^q(\mu)$\cmmnt{ (244E)}.
Consequently we have a bounded linear operator $T$ from $L^q$
to the normed space dual $(L^p)^*$ of $L^p$, given by writing

\Centerline{$(Tv)(u)=\int u\times v$}

\noindent for all $u\in L^p$ and $v\in L^q$\cmmnt{, exactly as in
243F}.

\leader{244K}{Theorem} Let $(X,\Sigma,\mu)$ be a measure space, and
$p\in\ooint{1,\infty}$;  set $q=p/(p-1)$.   Then the canonical map
$T:L^q(\mu)\to L^p(\mu)^*$\cmmnt{, described in 244J,} is a normed
space isomorphism.

\cmmnt{\medskip

\noindent{\bf Remark} I should perhaps remind anyone who is reading this
section to learn about $L^2$ that the basic theory of Hilbert spaces
yields this theorem in the case $p=q=2$ without any need for the more
generally applicable argument given below (see 244N, 244Yk).
}%end of comment

\proof{ We know that $T$ is a bounded linear operator of
norm at most $1$;  I need to show (i) that $T$ is actually an isometry
(that is, that $\|Tv\|=\|v\|_q$ for every $v\in L^q$), which will show
incidentally that $T$ is injective (ii) that $T$ is surjective, which is
the really substantial part of the theorem.

\medskip

{\bf (a)} If $v\in L^q$, then by 244Fa
(recalling that $p=q/(q-1)$) there
is a $u\in L^p$ such that $\|u\|_p\le 1$ and $\int u\times v=\|v\|_q$;
thus $\|Tv\|\ge (Tv)(u)=\|v\|_q$.   As we know already that
$\|Tv\|\le\|v\|_q$, we have $\|Tv\|=\|v\|_q$ for every $v$, and $T$ is
an isometry.

\medskip

{\bf (b)} The rest of the proof, therefore, will be devoted to showing
that $T:L^q\to(L^p)^*$ is surjective.   Fix $h\in (L^p)^*$ with
$\|h\|=1$.

I need to show that $h$ `lives on' a countable union of sets of
finite measure in $X$, in the following sense:  there is a
non-decreasing
sequence $\sequencen{E_n}$ of sets of finite measure such that
$h(f^{\ssbullet})=0$ whenever $f\in\eusm L^p$ and $f(x)=0$ for
$x\in\bigcup_{n\in\Bbb N}E_n$.   \Prf\ Choose a sequence
$\sequencen{u_n}$ in $L^p$ such that $\|u_n\|_p\le 1$ for every $n$ and
$\lim_{n\to\infty}h(u_n)=\|h\|=1$.
For each $n$, express $u_n$ as $f_n^{\ssbullet}$, where $f_n:X\to\Bbb R$
is a measurable function.   Set

\Centerline{$E_n=\{x:\sum_{k=0}^n|f_k(x)|^p\ge 2^{-n}\}$}

\noindent for $n\in \Bbb N$;   because $|f_k|^p$ is measurable and
integrable and has domain $X$ for every $k$, $E_n\in \Sigma$ and $\mu
E_n<\infty$ for each $n$.

Now suppose that $f\in\eusm L^p(X)$ and that $f(x)=0$ for
$x\in\bigcup_{n\in\Bbb N}E_n$;  set $u=f^{\ssbullet}\in L^p$.   \Quer\
Suppose, if possible, that $h(u)\ne 0$, and consider $h(cu)$, where

\Centerline{$\sgn c=\sgn h(u)$,\quad
$0<|c|<(p\,|h(u)|\,\|u\|_p^{-p})^{1/(p-1)}$.}

\noindent (Of course $\|u\|_p\ne 0$ if $h(u)\ne 0$.)   For each $n$, we
have

\Centerline{$\{x:f_n(x)\ne 0\}\subseteq\bigcup_{m\in\Bbb
N}E_m\subseteq\{x:f(x)=0\}$,}

\noindent so $|f_n+cf|^p=|f_n|^p+|cf|^p$ and

\Centerline{$h(u_n+cu)\le\|u_n+cu\|_p=(\|u_n\|_p^p+\|cu\|_p^p)^{1/p}
\le(1+|c|^p\|u\|_p^p)^{1/p}$.}

\noindent Letting $n\to\infty$,

\Centerline{$1+ch(u)\le(1+|c|^p\|u\|_p^p)^{1/p}$.}

\noindent
Because $\sgn c=\sgn h(u)$, $ch(u)=|c||h(u)|$ and we have

\Centerline{$1+p|c||h(u)|\le(1+ch(u))^p\le 1+|c|^p\|u\|_p^p$,}

\noindent so that

\Centerline{$p|h(u)|\le |c|^{p-1}\|u\|_p^p<p|h(u)|$}

\noindent by the choice of $c$;  which is impossible.\ \Bang

This means that $h(f^{\ssbullet})=0$ whenever $f:X\to\Bbb R$ is
measurable, belongs to $\eusm L^q$, and is zero on $\bigcup_{n\in\Bbb
N}E_n$.\ \Qed

\medskip

{\bf (c)} Set $H_n=E_n\setminus\bigcup_{k<n}E_k$ for each $n\in\Bbb N$;
then $\sequencen{H_n}$ is a disjoint sequence of sets of finite measure.
Now $h(u)=\sum_{n=0}^{\infty}h(u\times(\chi H_n)^{\ssbullet})$ for every
$u\in L^p$.   \Prf\ Express $u$ as $f^{\ssbullet}$, where $f:X\to\Bbb R$
is measurable.   Set $f_n=f\times\chi H_n$ for each $n$,
$g=f\times\chi(X\setminus\bigcup_{n\in\Bbb N}H_n)$;  then
$h(g^{\ssbullet})=0$, by (a), because $\bigcup_{n\in\Bbb
N}H_n=\bigcup_{n\in\Bbb N}E_n$.   Consider

\Centerline{$g_n=g+\sum_{k=0}^nf_k\in \eusm L^p$}

\noindent for each $n$.   Then $\lim_{n\to\infty}f-g_n=\tbf{0}$, and

\Centerline{$|f-g_n|^p\le|f|^p\in\eusm L^1$}

\noindent for every $n$, so by either Fatou's Lemma
or Lebesgue's Dominated Convergence Theorem

\Centerline{$\lim_{n\to\infty}\int|f-g_n|^p=0$,}

\noindent and

$$\eqalign{\lim_{n\to\infty}\|u-g^{\ssbullet}-\sum_{k=0}^nu\times(\chi
H_k)^{\ssbullet}\|_p
&=\lim_{n\to\infty}\|u-g_n^{\ssbullet}\|_p\cr
&=\lim_{n\to\infty}\bigl(\int|f-g_n|^p\bigr)^{1/p}
=0,\cr}$$

\noindent that is,

\Centerline{$u
=g^{\ssbullet}+\sum_{k=0}^{\infty}u\times\chi H_k^{\ssbullet}$}

\noindent in $L^p$.   Because $h:L^p\to\Bbb R$ is linear and
continuous, it follows that

\Centerline{$h(u)
=h(g^{\ssbullet})+\sum_{k=0}^{\infty}h(u\times\chi H_k^{\ssbullet})
=\sum_{k=0}^{\infty}h(u\times\chi H_k^{\ssbullet})$,}

\noindent as claimed.\ \Qed

\medskip

{\bf (d)} For each $n\in\Bbb N$, define $\nu_n:\Sigma\to\Bbb R$ by
setting

\Centerline{$\nu_nF=h(\chi(F\cap H_n)^{\ssbullet})$}

\noindent for every $F\in\Sigma$.   (Note that $\nu_nF$ is always
defined because $\mu(F\cap H_n)<\infty$, so that

\Centerline{$\|\chi(F\cap H_n)\|_p=\mu(F\cap H_n)^{1/p}<\infty$.)}

\noindent   Then $\nu_n\emptyset=h(0)=0$,
and if $\sequence{k}{F_k}$ is a disjoint sequence in $\Sigma$,

\Centerline{$\|\chi(\bigcup_{k\in\Bbb N}H_n\cap F_k)
-\sum_{k=0}^m\chi(H_n\cap F_k)\|_p
=\mu(H_n\cap\bigcup_{k=m+1}^{\infty}F_k)^{1/p}\to 0$}

\noindent as $m\to\infty$, so

\Centerline{$\nu_n(\bigcup_{k\in\Bbb N}F_k)
=\sum_{k=0}^{\infty}\nu_nF_k$.}

\noindent   So $\nu_n$ is countably additive.
Further, $|\nu_nF|\le\mu(H_n\cap F)^{1/p}$ for every $F\in\Sigma$,
so $\nu_n$ is truly continuous in the sense of 232Ab.

There is therefore an integrable function $g_n$ such that
$\nu_nF=\int_Fg_n$ for every $F\in\Sigma$;  let us suppose that
$g_n$ is measurable and defined on the whole of $X$.   Set
$g(x)=g_n(x)$ whenever $n\in\Bbb N$ and $x\in H_n$,
$g(x)=0$ for $x\in X\setminus\bigcup_{n\in\Bbb N}H_n$.

\medskip

{\bf (e)} $g=\sum_{n=0}^{\infty}g_n\times\chi H_n$ is measurable
and has the property that
$\int_Fg=h(\chi F^{\ssbullet})$
whenever $n\in\Bbb N$ and $F$ is a measurable subset of $H_n$;
consequently $\int_Fg=h(\chi F^{\ssbullet})$ whenever $n\in\Bbb N$
and $F$ is a measurable subset of $E_n=\bigcup_{k\le n}H_k$.   Set
$G=\{x:g(x)>0\}\subseteq\bigcup_{n\in\Bbb N}E_n$.   If $F\subseteq G$
and $\mu F<\infty$, then

\Centerline{$\lim_{n\to\infty}\int g\times\chi(F\cap E_n)
\le\sup_{n\in\Bbb N}h(\chi(F\cap E_n)^{\ssbullet})
\le\sup_{n\in\Bbb N}\|\chi(F\cap E_n)\|_p
=(\mu F)^{1/p}$,}

\noindent so by B.Levi's theorem

\Centerline{$\int_Fg=\int g\times\chi F
=\lim_{n\to\infty}\int g\times\chi(F\cap E_n)$}

\noindent exists.   Similarly, $\int_Fg$ exists if
$F\subseteq\{x:g(x)<0\}$ has finite measure;  while obviously
$\int_Fg$ exists if $F\subseteq\{x:g(x)=0\}$.
Accordingly $\int_Fg$ exists for every set $F$ of finite measure.
Moreover, by Lebesgue's Dominated Convergence Theorem,

\Centerline{$\int_Fg=\lim_{n\to\infty}\int_{F\cap E_n}g
=\lim_{n\to\infty}h(\chi(F\cap E_n)^{\ssbullet})
=\sum_{n=0}^{\infty}h(\chi(F\cap H_n)^{\ssbullet})
=h(\chi F^{\ssbullet})$}

\noindent for such $F$, by (c) above.
It follows at once that

\Centerline{$\int g\times f=h(f^{\ssbullet})$}

\noindent for every simple function $f:X\to\Bbb R$.

\medskip

{\bf (f)} Now $g\in L^q$.   \Prf\  (i)  We already know
that $|g|^q:X\to\Bbb R$ is measurable, because $g$ is
measurable and $a\mapsto|a|^q$ is continuous.    (ii)  Suppose that
$f$ is a non-negative simple function and $f\leae|g|^q$.   Then
$f^{1/p}$ is a simple function,  and $\sgn g$ is measurable and
takes only the values $0$, $1$ and $-1$, so
$f_1=f^{1/p}\times\sgn g$ is simple.   We see that $\int|f_1|^p=\int f$,
so $\|f_1\|_p=(\int f)^{1/p}$.   Accordingly

$$\eqalignno{(\int f)^{1/p}&\ge h(f_1^{\ssbullet})
=\int g\times f_1=\int|g\times f^{1/p}|\cr
&\ge\int f^{1/q}\times f^{1/p}\cr
\noalign{\noindent (because $0\le f^{1/q}\leae|g|$)}
&=\int f,\cr}$$

\noindent and we must have $\int f\le 1$.   (iii) Thus

\Centerline{$\sup\{\int f: f$ is a non-negative simple function,
$f\leae |g|^q\}\le 1<\infty$.}

\noindent  But now observe that if $\epsilon>0$ then

\Centerline{$\{x:|g(x)|^q\ge\epsilon\}
=\bigcup_{n\in\Bbb N}\{x:x\in E_n,\,|g(x)|^q\ge\epsilon\}$,}

\noindent and for each $n\in\Bbb N$

\Centerline{$\mu\{x:x\in E_n,\,|g(x)|^q\ge\epsilon\}
\le{1\over{\epsilon}}$,}

\noindent because $f=\epsilon\chi\{x:x\in E_n,\,|g(x)|^q\ge\epsilon\}$
is a simple function less than or equal to
$|g|^q$, so has integral at most $1$.   Accordingly

\Centerline{$\mu\{x:|g(x)|^q\ge\epsilon\}
=\sup_{n\in\Bbb N}\mu\{x:x\in E_n,\,|g(x)|^q\ge\epsilon\}
\le{1\over{\epsilon}}<\infty$.}

\noindent Thus $|g|^q$ is integrable, by the criterion in 122Ja.\ \Qed

\medskip

{\bf (g)} We may therefore speak of
$h_1=T(g^{\ssbullet})\in (L^p)^*$, and we know that it agrees with $h$
on members of $L^p$ of the
form $f^{\ssbullet}$ where $f$ is a simple function.   But these form a
dense subset of $L^p$, by 244Ha, and both $h$ and $h_1$ are continuous,
so $h=h_1$, by 2A3Uc, and $h$ is a value of $T$.
The argument as written so
far has assumed that $\|h\|=1$.   But of course any
non-zero member of $(L^p)^*$ is a scalar multiple of an element of norm
$1$, so is a value of $T$.   So $T:L^q\to(L^p)^*$ is indeed surjective,
and is therefore an isometric isomorphism, as claimed.
}%end of proof of 244K

\vleader{48pt}{244L}{}\cmmnt{ Continuing with the same topics as in
\S\S242 and 243, I turn to the order-completeness of $L^p$.

\medskip

\noindent}{\bf Theorem} Let $(X,\Sigma,\mu)$ be any measure space, and
$p\in\coint{1,\infty}$.   Then $L^p=L^p(\mu)$ is Dedekind complete.

\proof{ I use 242H.   Let $A\subseteq L^p$ be a non-empty set
which is bounded above in $L^p$.   Fix $u_0\in A$ and set

\Centerline{$A'=\{u_0\vee u:u\in A\}$,}

\noindent so that $A'$ has the same upper bounds as $A$ and is bounded
below by $u_0$.   Fixing an upper bound $w_0$ of $A$ in $L^p$, then
$u_0\le u\le w_0$ for every $u\in A'$.   Set

\Centerline{$B=\{(u-u_0)^p:u\in A'\}$.}

\noindent Then

\Centerline{$0\le v\le(w_0-u_0)^p\in L^1=L^1(\mu)$}

\noindent for every $v\in B$, so $B$ is a non-empty subset of $L^1$
which is bounded above in $L^1$, and therefore has a least upper bound
$v_1$ in $L^1$.   Now $v_1^{1/p}\in L^p$;  consider $w_1=u_0+v_1^{1/p}$.
If $u\in A'$ then $(u-u_0)^p\le v_1$ so $u-u_0\le v_1^{1/p}$ and $u\le
w_1$;  thus $w_1$ is an upper bound for $A'$.   If $w\in L^p$ is an
upper bound for $A'$, then $u-u_0\le w-u_0$ and $(u-u_0)^p\le(w-u_0)^p$
for every $u\in A'$, so $(w-u_0)^p$ is an upper bound for $B$ and
$v_1\le(w-u_0)^p$, $v_1^{1/p}\le w-u_0$ and $w_1\le w$.   Thus $w=\sup
A'=\sup A$ in $L^p$.   As $A$ is arbitrary, $L^p$ is Dedekind complete.
}%end of proof of 244L

\leader{244M}{}\cmmnt{ As in the last two sections, the theory of
conditional expectations is worth revisiting.

\medskip

\noindent}{\bf Theorem} Let $(X,\Sigma,\mu)$ be a probability space, and
$\Tau$ a $\sigma$-subalgebra of $\Sigma$.   Take $p\in[1,\infty]$.
Regard $L^0(\mu\restrp\Tau)$
as a subspace of $L^0=L^0(\mu)$\cmmnt{, as in 242Jh},
so that $L^p(\mu\restrp\Tau)$ becomes
$L^p(\mu)\cap L^0(\mu\restrp\Tau)$.   Let
$P:L^1(\mu)\to L^1(\mu\restrp\Tau)$ be the conditional expectation
operator\cmmnt{, as described in 242Jd}.   Then whenever
$u\in L^p=L^p(\mu)$,
$|Pu|^p\le P(|u|^p)$, so $Pu\in L^p(\mu\restrp\Tau)$ and
$\|Pu\|_p\le\|u\|_p$.    Moreover, $P[L^p]=L^p(\mu\restrp\Tau)$.

\proof{ For $p=\infty$, this is 243Jb, so I assume henceforth that
$p<\infty$.   Concerning the identification of $L^p(\mu\restrp\Tau)$ with
$L^p\cap L^0(\mu\restrp\Tau)$, if $S:L^0(\mu\restrp\Tau)\to L^0$ is the
canonical embedding described in 242J, we have $|Su|^p=S(|u|^p)$ for every
$u\in L^0(\mu\restrp\Tau)$, so that $Su\in L^p$ iff
$|u|^p\in L^1(\mu\restrp\Tau)$ iff $u\in L^p(\mu\restrp\Tau)$.

Set $\phi(t)=|t|^p$ for $t\in\Bbb R$;  then $\phi$ is a
convex function (because it is absolutely continuous on any bounded
interval, and its derivative is non-decreasing), and $|u|^p=\bar\phi(u)$
for every
$u\in L^0=L^0(\mu)$, where $\bar\phi$ is defined as in 241I.   Now
if $u\in L^p=L^p(\mu)$, we surely have $u\in L^1$ (because
$|u|\le|u|^p\vee(\chi X)^{\ssbullet}$, or otherwise);  so 242K tells us
that $|Pu|^p\le P|u|^p$.   But this means that
$Pu\in L^p\cap L^1(\mu\restrp\Tau)=L^p(\mu\restrp\Tau)$, and

\Centerline{$\|Pu\|_p=(\int|Pu|^p)^{1/p}\le(\int P|u|^p)^{1/p}
=(\int|u|^p)^{1/p}=\|u\|_p$,}

\noindent as claimed.    If $u\in L^p(\mu\restrp\Tau)$, then
$Pu=u$, so $P[L^p]$ is the whole of $L^p(\mu\restrp\Tau)$.
}%end of proof of 244M


\leader{244N}{The space $L^2$ (a)}\cmmnt{ As I have already remarked,
the really
important function spaces are $L^0$, $L^1$, $L^2$ and $L^{\infty}$.}
$L^2$ has the special property of being an inner product space;  if
$(X,\Sigma,\mu)$ is any measure space and $u$, $v\in L^2=L^2(\mu)$ then
$u\times v\in L^1(\mu)$\cmmnt{, by 244Eb}, and we may write
$\innerprod{u}{v}=\int u\times v$.
This makes $L^2$ a real inner product space\prooflet{ (because

\Centerline{$\innerprod{u_1+u_2}{v}
=\innerprod{u_1}{v}+\innerprod{u_2}{v}$,
\quad
$\innerprod{cu}{v}=c\innerprod{u}{v}$,
\quad
$\innerprod{u}{v}=\innerprod{v}{u}$,}

\Centerline{$\innerprod{u}{u}\ge 0$,
\quad $u=0$ whenever $\innerprod{u}{u}=0$}

\noindent for all $u$, $u_1$, $u_2$, $v\in L^2$ and $c\in\Bbb R$)} and
its norm $\|\,\|_2$ is the associated norm\cmmnt{ (because
$\|u\|_2=\sqrt{\innerprod{u}{u}}$
whenever $u\in L^2$)}.   \cmmnt{Because} $L^2$\cmmnt{ is complete
(244G), it} is
a real Hilbert space.   \cmmnt{The fact that it may be identified with
its own dual (244K) can of course be deduced from this.}

I will use the phrase `{\bf square-integrable}' to describe functions
in $\eusm L^2(\mu)$.

\spheader 244Nb \cmmnt{Conditional expectations take a special form in
the case of $L^2$.}   Let $(X,\Sigma,\mu)$ be a probability space,
$\Tau$ a $\sigma$-subalgebra of $\Sigma$, and
$P:L^1=L^1(\mu)\to L^1(\mu\restrp\Tau)\subseteq L^1$ the corresponding
conditional expectation operator.   Then $P[L^2]\subseteq L^2$, where
$L^2=L^2(\mu)$\cmmnt{ (244M)}, so we have an operator
$P_2=P\restr L^2$
from $L^2$ to itself.   Now $P_2$ is an orthogonal projection and its
kernel is $\{u:u\in L^2,\,\int_Fu=0$ for every $F\in\Tau\}$.
\prooflet{\Prf\ (i) If $u\in L^1$ then $Pu=0$ iff $\int_Fu=0$ for every
$F\in\Tau$ (cf.\ 242Je);  so surely the kernel of $P_2$ is the set
described.   (ii) Since $P^2=P$, $P_2$ also is a projection;  because
$P_2$ has norm at most $1$ (244M), and is therefore continuous,

\Centerline{$U=P_2[L^2]=L^2(\mu\restrp\Tau)
=\{u:u\in L^2,\,P_2u=u\}$,\quad$V=\{u:P_2u=0\}$}

\noindent are closed linear subspaces of $L^2$ such that
$U\oplus V=L^2$.
(iii) Now suppose that $u\in U$ and $v\in V$.   Then $P|v|\in L^2$, so
$u\times P|v|\in L^1$ and $P(u\times v)=u\times Pv$, by 242L.
Accordingly

\Centerline{$\innerprod{u}{v}=\int u\times v
=\int P(u\times v)=\int u\times Pv=0$.}

\noindent Thus $U$ and $V$ are orthogonal subspaces of $L^2$, which is
what we mean by saying that $P_2$ is an orthogonal projection.   (Some
readers will know that every projection of norm at most $1$ on an inner
product space is orthogonal.)\ \Qed
}%end of prooflet

\leader{*244O}{}\dvAnew{2009}\cmmnt{ This is not the place for a detailed
discussion
of the geometry of $L^p$ spaces.   However there is a particularly
important fact about the shape of the unit ball which is
accessible by the methods available to us here.

\medskip

\noindent}{\bf Theorem}\cmmnt{ ({\smc Clarkson 36})}
Suppose that $p\in\ooint{1,\infty}$ and $(X,\Sigma,\mu)$ is a
measure space.   Then $L^p=L^p(\mu)$ is uniformly
convex\cmmnt{ (definition:  2A4K)}.

\proof{({\smc Hanner 56}, {\smc Naor 04})

\medskip

{\bf (a)(i)} For $0<t\le 1$ and $a$, $b\in\Bbb R$, set

\Centerline{$\phi_0(t)=(1+t)^{p-1}+(1-t)^{p-1}$,}

\Centerline{$\phi_1(t)=\Bover{(1+t)^{p-1}-(1-t)^{p-1}}{t^{p-1}}
=(\Bover1t+1)^{p-1}-(\Bover1t-1)^{p-1}$,}

\Centerline{$\psi_{ab}(t)=|a|^p\phi_0(t)+|b|^p\phi_1(t)$,}

\Centerline{$\phi_2(b)=(1+b)^p+|1-b|^p$.}

\medskip

\quad{\bf (ii)} We have

$$\eqalignno{\phi_0'(t)
&=(p-1)((1+t)^{p-2}-(1-t)^{p-2}),\text{ which has the same sign as }p-2,\cr
\noalign{\noindent (of course it is zero if $p=2$),}
\phi_1'(t)
&=-\Bover{p-1}{t^2}((\Bover1t-1)^{p-2}-(\Bover1t-1)^{p-2})\cr
&=-\Bover{p-1}{t^p}((1+t)^{p-2}-(1-t)^{p-2})
=-\Bover1{t^p}\phi_0'(t)\cr}$$

\noindent for every $t\in\ooint{0,1}$.   Accordingly $\phi_0'-\phi_1'$ has the
same sign as $p-2$ everywhere on $\ooint{0,1}$.   Also

\Centerline{$\phi_0(1)=2^{p-1}=\phi_1(1)$,}

\noindent so $\phi_0-\phi_1$ has the same sign as $2-p$ everywhere on
$\ocint{0,1}$.

\medskip

\quad{\bf (iii)} $\phi_2$ is strictly increasing on $\coint{0,\infty}$.
\Prf\ For $b>0$,

$$\eqalign{\phi_2'(b)
&=p((1+b)^{p-1}-(1-b)^{p-1})>0\text{ if }b\le 1,\cr
&=p((1+b)^{p-1}+(b-1)^{p-1})>0\text{ if }b\ge 1.  \text{ \Qed}\cr}$$

\medskip

\quad{\bf (iv)} If $0<b\le a$, then

$$\eqalignno{\psi_{ab}(\Bover{b}{a})
&=a^p\phi_0(\Bover{b}{a})+b^p\phi_1(\Bover{b}{a})\cr
&=a^p(1+\Bover{b}{a})^{p-1}+a^p(1-\Bover{b}{a})^{p-1}
   +b^p(\Bover{a}{b}+1)^{p-1}-b^p(\Bover{a}{b}-1)^{p-1}\cr
&=a(a+b)^{p-1}+a(a-b)^{p-1}+b(a+b)^{p-1}-b(a-b)^{p-1}\cr
&=(a+b)^p+(a-b)^p
=(a+b)^p+|a-b|^p.&(\dagger)\cr}$$

\noindent Also $\psi_{ab}'(t)=(a^p-\Bover{b^p}{t^p})\phi_0'(t)$ has the
sign of $2-p$ if $0<t<\Bover{b}{a}$ and the sign of $p-2$ if
$\Bover{b}{a}<t<1$.   Accordingly

\inset{----- if $1<p\le 2$,
$\psi_{ab}(t)\le\psi_{ab}(\Bover{b}{a})=(a+b)^p+|a-b|^p$ for every
$t\in\ocint{0,1}$,

----- if $p\ge 2$,
$\psi_{ab}(t)\ge\psi_{ab}(\Bover{b}{a})=(a+b)^p+|a-b|^p$ for every
$t\in\ocint{0,1}$.}

\medskip

\quad{\bf (v)} Now consider the case $0<a\le b$.   If $1<p\le 2$,

$$\eqalignno{\psi_{ab}(t)
&=a^p\phi_0(t)+b^p\phi_1(t)
\le a^p\phi_0(t)+b^p\phi_1(t)+(b^p-a^p)(\phi_0(t)-\phi_1(t))\cr
\displaycause{by (ii)}
&=b^p\phi_0(t)+a^p\phi_1(t)
\le(b+a)^p+(b-a)^p
=(a+b)^p+|a-b|^p\cr}$$

\noindent for every $t\in\ocint{0,1}$.   If $p\ge 2$, on the other hand,

$$\eqalignno{\psi_{ab}(t)
&=a^p\phi_0(t)+b^p\phi_1(t)
\ge a^p\phi_0(t)+b^p\phi_1(t)+(b^p-a^p)(\phi_0(t)-\phi_1(t))\cr
&=b^p\phi_0(t)+a^p\phi_1(t)
\ge(a+b)^p+|a-b|^p\cr}$$

\noindent for every $t$.

\medskip

\quad{\bf (vi)} Thus we have the inequalities

$$\eqalign{\psi_{ab}(t)
&\le|a+b|^p+|a-b|^p\text{ if }p\in\ocint{1,2},\cr
&\ge|a+b|^p+|a-b|^p\text{ if }p\in\coint{2,\infty}\cr}\eqno{(*)}$$

\noindent whenever $t\in\ocint{0,1}$ and
$a$, $b\in\ooint{0,\infty}$.   Since
$(a,b)\mapsto\psi_{ab}(t)$ is continuous for every $t$,
the same inequalities are valid for all $a$, $b\in\coint{0,\infty}$.
And since

\Centerline{$\psi_{ab}(t)=\psi_{|a|,|b|}(t)$,
\quad$|a+b|^p+|a-b|^p=\big||a|+|b|\bigr|^p+\bigl||a|-|b|\bigr|^p$}

\noindent for all $a$, $b\in\Bbb R$ and $t\in\ocint{0,1}$, the inequalities
(*) are valid for all $a$, $b\in\Bbb R$ and $t\in\ocint{0,1}$.

\medskip

{\bf (b)} Suppose that $p\ge 2$.

\medskip

\quad{\bf (i)}

\Centerline{$\|u+v\|_p^p+\|u-v\|_p^p
\le(\|u\|_p+\|v\|_p)^p+|\|u\|_p-\|v\|_p|^p$}

\noindent for all $u$, $v\in L^p$.   \Prf\ First consider the
case $0<\|v\|_p\le\|u\|_p$.   Let $f$, $g:X\to\Bbb R$ be
$\Sigma$-measurable functions such that $f^{\ssbullet}=u$ and
$g^{\ssbullet}=v$.   Then for any $t\in\ocint{0,1}$,

$$\eqalignno{\|u+v\|_p^p+\|u-v\|_p^p
&=\int|f(x)+g(x)|^p+|f(x)-g(x)|^p\mu(dx)\cr
&\le\int\psi_{f(x),g(x)}(t)\mu(dx)\cr
\displaycause{by the second inequality in (*)}
&=\int|f(x)|^p\phi_0(t)+|g(x)|^p\phi_1(t)\mu(dx)
=\|u\|_p^p\phi_0(t)+\|v\|_p^p\phi_1(t).\cr}$$

\noindent In particular, taking $t=\|v\|_p/\|u\|_p$, and applying
($\dagger$) from (a-iv),

\Centerline{$\|u+v\|_p^p+\|u-v\|_p^p
\le(\|u\|_p+\|v\|_p)^p+|\|u\|_p-\|v\|_p|^p$.}

\noindent Of course the result will also be true if $0<\|u\|_p\le\|v\|_p$,
and the case in which either $u$ or $v$ is zero is trivial.\ \Qed

\medskip

\quad{\bf (ii)} Let $\epsilon\in\ocint{0,2}$.   Set
$\delta=2-(2^p-\epsilon^p)^{1/p}>0$.   If $u$, $v\in L^p$,
$\|u\|_p=\|v\|_p=1$ and $\|u-v\|_p\ge\epsilon$, then

\Centerline{$\|u+v\|_p^p+\epsilon^p
\le\|u+v\|_p^p+\|u-v\|_p^p
\le(\|u\|_p+\|v\|_p)^p+|\|u\|_p-\|v\|_p|^p
=2^p$,}

\noindent so $\|u+v\|_p\le(2^p-\epsilon^p)^{1/p}=2-\delta$.
As $u$, $v$ and $\epsilon$ are arbitrary, $L^p$ is uniformly convex.

\wheader{*244O}{6}{2}{2}{36pt}

{\bf (c)} Next suppose that $p\in\ocint{1,2}$.

\medskip

\quad{\bf (i)}

\Centerline{$(\|u\|_p+\|v\|_p)^p+|\|u\|_p-\|v\|_p|^p
\le\|u+v\|_p^p+\|u-v\|_p^p$}

\noindent for all $u$, $v\in L^p$.   \Prf\ We can repeat all the ideas,
and most of the formulae, of (b-i).   As before, start with the
case $0<\|v\|_p\le\|u\|_p$.   Let $f$, $g:X\to\Bbb R$ be
$\Sigma$-measurable functions such that $f^{\ssbullet}=u$ and
$g^{\ssbullet}=v$.   Taking $t=\|v\|_p/\|u\|_p$,

$$\eqalignno{\|u+v\|_p^p+\|u-v\|_p^p
&=\int|f(x)+g(x)|^p+|f(x)-g(x)|^p\mu(dx)\cr
&\ge\int\psi_{f(x),g(x)}(t)\mu(dx)\cr
\displaycause{by the first inequality in (*)}
&=\|u\|_p^p\phi_0(t)+\|v\|_p^p\phi_1(t)
=(\|u\|_p+\|v\|_p)^p+|\|u\|_p-\|v\|_p|^p.\cr}$$

\noindent Similarly if $0<\|u\|_p\le\|v\|_p$,
and the case in which either $u$ or $v$ is zero is trivial.\ \Qed

\medskip

\quad{\bf (ii)} Let $\epsilon>0$.   Set $\gamma=\phi_2(\bover{\epsilon}2)>2$
(see (a-iii) above) and
$\delta=2\bigl(1-(\Bover{2}{\gamma})^{1/p}\bigr)>0$.
Now suppose that $\|u\|_p=\|v\|_p=1$ and
$\|u-v\|_p\ge\epsilon$.   Then $\|u+v\|_p\le 2-\delta$.   \Prf\ If
$u+v=0$ this is trivial.   Otherwise, set $a=\|u+v\|_p$ and $b=\|u-v\|_p$.
Then $a\le 2$ and $b\ge\epsilon$, so

$$\eqalignno{a^p\gamma
&=a^p\phi_2(\Bover{\epsilon}2)
\le a^p\phi_2(\Bover{b}{a})\cr
\displaycause{by (a-iii) again}
&=(a+b)^p+|a-b|^p
=(\|u+v\|_p+\|u-v\|_p)^p+|\|u+v\|_p-\|u-v\|_p|^p\cr
&\le\|2u\|_p^p+\|2v\|_p^p\cr
\displaycause{by (i) here}
&=2^{p+1}\cr}$$

\noindent
and $a\le 2\bigl(\Bover{2}{\gamma}\bigr)^{1/p}=2-\delta$.\ \QeD\  As $u$,
$v$ and $\epsilon$ are arbitrary, $L^p$ is uniformly convex.
}%end of proof of 244O?

\cmmnt{\medskip

\noindent{\bf Remark} The inequalities in (b-i) and (c-i) of the proof are
called {\bf Hanner's inequalities}.}

\leader{244P}{Complex $L^p$}\dvAformerly{2{}44O}
Let $(X,\Sigma,\mu)$ be any measure space.

\header{244Pa}{\bf (a)} For any $p\in\ooint{1,\infty}$, set

\Centerline{$\eusm L^p_{\Bbb C}=\eusm L^p_{\Bbb C}(\mu)
=\{f:f\in\eusm L^0_{\Bbb C}(\mu),\,|f|^p$ is integrable$\}$,}

$$\eqalign{L^p_{\Bbb C}=L^p_{\Bbb C}(\mu)
&=\{f^{\ssbullet}:f\in\eusm L^p_{\Bbb C}\}\cr
&=\{u:u\in L^0_{\Bbb C}(\mu),\,\Real(u)\in L^p(\mu)\text{ and }
\Imag(u)\in L^p(\mu)\}\cr
&=\{u:u\in L^0_{\Bbb C}(\mu),\,|u|\in L^p(\mu)\}.\cr}$$

\noindent Then $L^p_{\Bbb C}$ is a linear subspace of $L^0_{\Bbb C}(\mu)$.
Set $\|u\|_p=\||u|\|_p=(\int|u|^p)^{1/p}$ for $u\in L^p_{\Bbb C}$.

\header{244Pb}{\bf  (b)} The proof of 244E(b-i) applies unchanged to
complex-valued functions, so taking $q=p/(p-1)$ we get

\Centerline{$\|u\times v\|_1\le\|u\|_p\|v\|_q$}

\noindent for all $u\in L^p_{\Bbb C}$, $v\in L^q_{\Bbb C}$.   244Fa
becomes

\inset{for every $u\in L^p_{\Bbb C}$ there is a $v\in L^q_{\Bbb C}$ such
that $\|v\|_q\le 1$ and

\Centerline{$\int u\times v=|\int u\times v|=\|u\|_p$\dvro{.}{;}}}

\noindent\cmmnt{the same proof works, if you omit all mention of the
functional $\tau$, and allow me to write
$\sgn a=|a|/a$ for all non-zero complex numbers -- it would perhaps be
more natural to
write $\overline{\sgn}a$ in place of $\sgn a$.   So, just as
before, we find that} $\|\,\|_p$ is a norm.   \cmmnt{We can use the
argument of 244G to show that} $L^p_{\Bbb C}$ is complete.
\cmmnt{(Alternatively,
note that a sequence $\sequencen{u_n}$ in $L^0_{\Bbb C}$ is
Cauchy, or convergent, iff its real and imaginary parts are.)}
The space $S_{\Bbb C}$ of equivalence classes of `complex-valued
simple functions' is dense in $L^p_{\Bbb C}$.   If $X$ is a subset of
$\BbbR^r$ and $\mu$ is Lebesgue measure on $X$, then the space
of equivalence classes of continuous complex-valued
functions on $X$ with bounded support is dense in $L^p_{\Bbb C}$.

\header{244Pc}{\bf (c)} The canonical map $T:L^q_{\Bbb C}\to (L^p_{\Bbb
C})^*$, defined by writing $(Tv)(u)=\int u\times v$, is
surjective\cmmnt{ because $T\restr L^q:L^q\to (L^p)^*$ is surjective;}
and\cmmnt{ it is} an
isometry\cmmnt{ by the remarks in (b) just above}.   Thus we can still
identify $L^q_{\Bbb C}$ with $(L^p_{\Bbb C})^*$.

\header{244Pd}{\bf (d)}\cmmnt{ When we come to the complex form of
Jensen's inequality, it seems that a new idea is needed.   I have
relegated this
to 242Yk-242Yl.   But for the complex form of 244M a simpler argument
will suffice.}   If $(X,\Sigma,\mu)$ is a probability space, $\Tau$ is a
$\sigma$-subalgebra of $\Sigma$ and
$P:L^1_{\Bbb C}(\mu)\to L^1_{\Bbb C}(\mu\restrp\Tau)$
is the corresponding
conditional expectation operator, then for
any $u\in L^p_{\Bbb C}$\cmmnt{ we shall have

\Centerline{$|Pu|^p\le(P|u|)^p\le P(|u|^p)$,}

\noindent applying 242Pc and 244M.
So} $\|Pu\|_p\le\|u\|_p$\cmmnt{, as before}.

\header{244Pe}{\bf (e)}\cmmnt{ There is a special point arising with
$L^2_{\Bbb C}$.}   We now have to define

\Centerline{$\innerprod{u}{v}=\int u\times\bar v$}

\noindent for $u$, $v\in L^2_{\Bbb C}$\cmmnt{, so that
$\innerprod{u}{u}=\int|u|^2=\|u\|_2^2$ for every $u$};  \cmmnt{this
means that} $\innerprod{v}{u}$ is the complex conjugate of
$\innerprod{u}{v}$.

\exercises{
\leader{244X}{Basic exercises $\pmb{>}$(a)}
%\spheader 244Xa
Let $(X,\Sigma,\mu)$ be a measure space, and $(X,\hat\Sigma,\hat\mu)$
its completion.   Show that $\eusm L^p(\hat\mu)=\eusm L^p(\mu)$ and
$L^p(\hat\mu)=L^p(\mu)$ for every $p\in[1,\infty]$.
%244A used in 465F

\spheader 244Xb Let $(X,\Sigma,\mu)$ be a measure space, and
$1\le p\le q\le r\le\infty$.   Show that
$L^p(\mu)\cap L^r(\mu)\subseteq L^q(\mu)
\subseteq L^p(\mu)+L^r(\mu)\subseteq L^0(\mu)$.   (See also 244Yh.)
%244A

\spheader 244Xc Let $(X,\Sigma,\mu)$ be a measure space.   Suppose that
$p$, $q$, $r\in[1,\infty]$ and that $\bover1p+\bover1q=\bover1r$,
setting
$\bover1{\infty}=0$ as usual.   Show that $u\times v\in L^r(\mu)$ and
$\|u\times v\|_r\le\|u\|_p\|v\|_q$ whenever $u\in L^p(\mu)$ and
$v\in L^q(\mu)$.   \Hint{if $r<\infty$ apply H\"older's inequality to
$|u|^r\in L^{p/r}$, $|v|^r\in L^{q/r}$.}
%244E

\sqheader 244Xd(i) Let $(X,\Sigma,\mu)$ be a probability space.   Show
that if $1\le p\le r\le\infty$ then $\|f\|_p\le\|f\|_r$ for every
$f\in{\eusm L}^r(\mu)$.   \Hint{use H\"older's inequality to show that
$\int|f|^p\le\||f|^p\|_{r/p}$.}   In particular,
${\eusm L}^p(\mu)\supseteq {\eusm L}^r(\mu)$.   (ii) Let
$(X,\Sigma,\mu)$ be a measure space such
that $\mu E\ge 1$ whenever $E\in\Sigma$ and $\mu E>0$.   (This happens,
for instance, when $\mu$ is `counting measure' on $X$.)  Show that
if $1\le p\le r\le\infty$ then $L^p(\mu)\subseteq L^r(\mu)$ and
$\|u\|_p\ge\|u\|_r$ for every $u\in L^p(\mu)$.   \Hint{look first at the
case $\|u\|_p=1$.}
%244F

\sqheader 244Xe\dvAformerly{2{}44Xf} 
Let $(X,\Sigma,\mu)$ be a semi-finite measure space,
and $p$, $q\in[1,\infty]$ such that $\bover1p+\bover1q=1$.   Show that
if $u\in L^0(\mu)\setminus L^p(\mu)$ then
there is a $v\in L^q(\mu)$ such that $u\times v\notin L^1(\mu)$.
\Hint{reduce to the case $u\ge 0$.   Show that in this case there is for
each $n\in\Bbb N$ a $u_n\le u$ such that $4^n\le\|u_n\|_p<\infty$;  take
$v_n\in L^q$ such that $\|v_n\|_q\le 2^{-n}$ and
$\int u_n\times v_n\ge 2^n$, and set $v=\sum_{n=0}^{\infty}v_n$.}
%244F

\spheader 244Xf Let $\langle(X_i,\Sigma_i,\mu_i)\rangle_{i\in I}$ be a
family of measure spaces, and $(X,\Sigma,\mu)$ their direct sum (214L).
Take any $p\in\coint{1,\infty}$.   Show that the canonical isomorphism
between $L^0(\mu)$ and $\prod_{i\in I}L^0(\mu_i)$ (241Xd) induces an
isomorphism between $L^p(\mu)$ and the subspace

\Centerline{$\{u:u\in\prod_{i\in I}L^p(\mu_i),\,
\|u\|=\bigl(\sum_{i\in I}\|u(i)\|_p^p)^{1/p}<\infty\}$}

\noindent of $\prod_{i\in I}L^p(\mu_i)$.
%244F

\spheader 244Xg Let $(X,\Sigma,\mu)$ be a measure space.   Set
$M^{\infty,1}=L^1(\mu)\cap L^{\infty}(\mu)$.   Show that for
$u\in M^{\infty,1}$ the function
$p\mapsto\|u\|_p:\coint{1,\infty}\to\coint{0,\infty}$ is continuous, and
that $\|u\|_{\infty}=\lim_{p\to\infty}\|u\|_p$.   \Hint{consider first
the case in which $u$ is the equivalence class of a simple function.}
%244F

\spheader 244Xh Let $\mu$ be counting measure on $X=\{1,2\}$, so that
$\eusm L^0(\mu)=\BbbR^2$ and $L^p(\mu)=L^0(\mu)$ can be identified with
$\BbbR^2$ for every $p\in[1,\infty]$.   Sketch the unit balls
$\{u:\|u\|_p\le 1\}$ in $\BbbR^2$ for $p=1$, $\bover32$, $2$, $3$ and
$\infty$.
%244F

\spheader 244Xi Let $\mu$ be counting measure on $X=\{1,2,3\}$, so that
$\eusm L^0(\mu)=\BbbR^3$ and $L^p(\mu)=L^0(\mu)$ can be identified with
$\BbbR^3$ for every $p\in[1,\infty]$.   Describe the unit balls
$\{u:\|u\|_p\le 1\}$ in $\BbbR^3$ for $p=1$, $2$ and $\infty$.
%244F

\spheader 244Xj At which points does the argument of
244Hb break down if we try to apply it to $L^{\infty}$ with
$\|\,\|_{\infty}$?
%244H

\spheader 244Xk Let $p\in\coint{1,\infty}$.   (i) Show that
$|a^p-b^p|\ge|a-b|^p$ for all $a$,
$b\ge 0$.   \Hint{for $a>b$, differentiate both sides with respect to
$a$.}   (ii) Let $(X,\Sigma,\mu)$ be a measure space and $U$ a linear
subspace of $L^0(\mu)$ such that ($\alpha$) $|u|\in U$ for every
$u\in U$ ($\beta$) $u^{1/p}\in U$ for every $u\in U$ ($\gamma$)
$U\cap L^1$ is dense in $L^1=L^1(\mu)$.   Show that $U\cap L^p$ is dense
in $L^p=L^p(\mu)$.   \Hint{check first that
$\{u:u\in U\cap L^1$, $u\ge 0\}$ is dense in
$\{u:u\in L^1$, $u\ge 0\}$.}   (iii) Use this to prove 244H from 242M
and 242O.
%244H

\spheader 244Xl For any measure space $(X,\Sigma,\mu)$ write
$M^{1,\infty}=M^{1,\infty}(\mu)$ for $\{v+w:v\in L^1(\mu),\,w\in
L^{\infty}(\mu)\}\subseteq L^0(\mu)$.    Show that $M^{1,\infty}$ is a
linear subspace of $L^0$ including $L^p$ for every $p\in[1,\infty]$, and
that if $u\in L^0$, $v\in M^{1,\infty}$ and $|u|\le |v|$ then $u\in
M^{1,\infty}$.  \Hint{$u=v\times w$ where $|w|\le\chi X^{\ssbullet}$.}
%244M

\spheader 244Xm Let $(X,\Sigma,\mu)$ and $(Y,\Tau,\nu)$ be two measure
spaces, and let $\Cal T^+$ be the set of linear operators
$T:M^{1,\infty}(\mu)\to M^{1,\infty}(\nu)$ such that ($\alpha$)
$Tu\ge 0$
whenever $u\ge 0$ in $M^{1,\infty}(\mu)$ ($\beta$) $Tu\in L^1(\nu)$ and
$\|Tu\|_1\le\|u\|_1$ whenever $u\in L^1(\mu)$ ($\gamma$)
$Tu\in L^{\infty}(\nu)$ and $\|Tu\|_{\infty}\le\|u\|_{\infty}$ whenever
$u\in L^{\infty}(\mu)$.   (i) Show that if $\phi:\Bbb R\to\Bbb R$ is a
convex function such that $\phi(0)=0$, and $u\in M^{1,\infty}(\mu)$ is
such that $\bar\phi(u)\in M^{1,\infty}(\mu)$ (interpreting
$\bar\phi:L^0(\mu)\to L^0(\mu)$ as in 241I), then
$\bar\phi(Tu)\in M^{1,\infty}(\nu)$ and $\bar\phi(Tu)\le T(\bar\phi(u))$
for every $T\in\Cal T^+$.   (ii) Hence show that if $p\in[1,\infty]$ and
$u\in L^p(\mu)$, $Tu\in L^p(\nu)$ and $\|Tu\|_p\le\|u\|_p$ for every
$T\in\Cal T^+$.
%244M

\sqheader 244Xn Let $X$ be any set, and let $\mu$ be counting measure on
$X$.   In this case it is customary to
write $\ell^p(X)$ for ${\eusm L}^p(\mu)$, and to identify it with
$L^p(\mu)$.  In particular, $L^2(\mu)$ becomes identified with
$\ell^2(X)$, the
space of square-summable functions on $X$.   Write out statements
and proofs of the results of this section adapted to this special case.
%244N

\spheader 244Xo Let $(X,\Sigma,\mu)$ and $(Y,\Tau,\nu)$ be measure
spaces and $\phi:X\to Y$ an \imp\ function.   Show that the map
$g\mapsto g\phi:\eusm L^0(\nu)\to\eusm L^0(\mu)$ (241Xg) induces a
norm-preserving map from $L^p(\nu)$ to $L^p(\mu)$ for every
$p\in[1,\infty]$, and also a map from $M^{1,\infty}(\nu)$ to
$M^{1,\infty}(\mu)$ which belongs to the class $\Cal T^+$ of 244Xm.
%244N

\leader{244Y}{Further exercises (a)}
%\spheader 244Ya
Let $(X,\Sigma,\mu)$ be a measure space, and
$(X,\tilde\Sigma,\tilde\mu)$ its c.l.d.\ version.   Show that
$\eusm L^p(\mu)\subseteq\eusm L^p(\tilde\mu)$ and that this embedding
induces a Banach lattice
isomorphism between $L^p(\mu)$ and $L^p(\tilde\mu)$, for every
$p\in\coint{1,\infty}$.
%244A

\spheader 244Yb Let $(X,\Sigma,\mu)$ be any measure space, and
$p\in\coint{1,\infty}$.   Show that $L^p(\mu)$ has the countable sup
property in the sense of 241Ye.   \Hint{242Yh.}
%244A

\spheader 244Yc\dvAnew{2009.} Suppose that $(X,\Sigma,\mu)$ is a measure
space, and
that $p\in\ooint{0,1}$, $q<0$ are such that $\bover1p+\bover1q=1$.
(i) Show that
$ab\ge\bover1pa^p+\bover1qb^q$ for all real $a\ge 0$, $b>0$.
\Hint{set $p'=\bover1p$, $q'=\bover{p'}{p'-1}$,
$c=(ab)^p$, $d=b^{-p}$ and apply 244Ea.}
(ii) Show that if $f$, $g\in\eusm L^0(\mu)$ are non-negative and
$E=\{x:x\in\dom g$, $g(x)>0\}$, then

\Centerline{$(\int_Ef^p)^{1/p}(\int_Eg^q)^{1/q}
\le\int f\times g$.}

\noindent(iii) Show that
if $f$, $g\in\eusm L^0(\mu)$ are non-negative, then

\Centerline{$(\int f^p)^{1/p}+(\int g^p)^{1/p}
\le(\int(f+g)^p)^{1/p}$.}
%244E

\spheader 244Yd\dvAformerly{2{}44Yc.}
Let $(X,\Sigma,\mu)$ be a measure space, and $Y$ a subset
of $X$;  write $\mu_Y$ for the subspace measure on $Y$.   Show that the
canonical map $T$ from $L^0(\mu)$ onto $L^0(\mu_Y)$ (241Yg) includes a
surjection from $L^p(\mu)$ onto $L^p(\mu_Y)$ for every $p\in[1,\infty]$,
and also a map from $M^{1,\infty}(\mu)$ to $M^{1,\infty}(\mu_Y)$ which
belongs to the class $\Cal T^+$ of 244Xm.   Show that the
following are equiveridical:  (i)  there is some $p\in\coint{1,\infty}$
such that $T\restr L^p(\mu)$ is
injective;  (ii) $T:L^p(\mu)\to L^p(\mu_Y)$ is
norm-preserving for every $p\in\coint{1,\infty}$;    (iii)
$F\cap Y\ne\emptyset$ whenever $F\in\Sigma$ and $0<\mu F<\infty$.
%244F

\spheader 244Ye\dvAformerly{2{}44Yd.}
 Let $(X,\Sigma,\mu)$ be any measure space, and
$p\in\coint{1,\infty}$.   Show that the norm $\|\,\|_p$ on $L^p(\mu)$ is
order-continuous in the sense of 242Yg.
%244F

\spheader 244Yf\dvAformerly{2{}44Ye.}
 Let $(X,\Sigma,\mu)$ be any measure space, and
$p\in[1,\infty]$.   Show that if $A\subseteq L^p(\mu)$ is
upwards-directed and norm-bounded, then it is bounded above.
\Hint{242Yf.}
%244F

\spheader 244Yg\dvAformerly{2{}44Yf.}
 Let $(X,\Sigma,\mu)$ be any measure space, and
$p\in[1,\infty]$.   Show that if a non-empty set $A\subseteq L^p(\mu)$
is upwards-directed and has a supremum in $L^p(\mu)$, then
$\|\sup A\|_p\le\sup_{u\in A}\|u\|_p$.   \Hint{consider first the case
$0\in A$.}
%244F

\spheader 244Yh\dvAformerly{2{}44Yg.}
 Let $(X,\Sigma,\mu)$ be a measure space and
$u\in L^0(\mu)$.   (i) Show that
$I=\{p:p\in\coint{1,\infty},\,u\in L^p(\mu)\}$ is
an interval.   Give examples to show that it may be open, closed or
half-open.   (ii) Show that $p\mapsto p\ln\|u\|_p:I\to\Bbb R$ is convex.
\Hint{if $p<q$ and $t\in\ooint{0,1}$, observe that
$\int|u|^{tp+(1-t)q}\le\||u|^{pt}\|_{1/t}\||u|^{q(1-t)}\|_{1/(1-t)}$.}
(iii) Show that if $p\le q\le r$ in $I$, then
$\|u\|_q\le\max(\|u\|_p,\|u\|_r)$.
%244F mt24bits

\spheader 244Yi\dvAformerly{2{}44Yh.}
 Let $[a,b]$ be a non-trivial closed interval in $\Bbb R$
and $F:[a,b]\to\Bbb R$ a function;  take $p\in\ooint{1,\infty}$.   Show
that the following are equiveridical:   (i) $F$ is absolutely continuous
and its derivative $F'$ belongs to $\eusm L^p(\mu)$, where $\mu$ is
Lebesgue measure on $[a,b]$  (ii)

\Centerline{$c=\sup\{\sum_{i=1}^n\Bover{|F(a_i)-F(a_{i-1})|^p}
{(a_i-a_{i-1})^{p-1}}:a\le a_0<a_1<\ldots<a_n\le b\}$}

\noindent is finite, and that in this case $c=\|F'\|_p$.   \Hint{(i) if
$F$ is absolutely continuous and $F'\in\eusm L^p$, use H\"older's
inequality to show that
$|F(b')-F(a')|^p\le(b'-a')^{p-1}\int_{a'}^{b'}|F'|^p$ whenever
$a\le a'\le b'\le b$.   (ii) If $F$ satisfies the condition, show that
$(\sum_{i=0}^n|F(b_i)-F(a_i)|)^p\le c(\sum_{i=0}^n(b_i-a_i))^{p-1}$
whenever $a\le a_0\le b_0\le a_1\le\ldots\le b_n\le b$, so that $F$ is
absolutely continuous.   Take a sequence $\sequencen{F_n}$ of polygonal
functions approximating $F$;  use 223Xj to show that $F'_n\to F'$ a.e.,
so that $\int|F'|^p\le\sup_{n\in\Bbb N}\int|F'_n|^p\le c^p$.}
%244F

\spheader 244Yj\dvAformerly{2{}44Yi.}
 Let $G$ be an open set in $\BbbR^r$ and write $\mu$ for
Lebesgue measure on $G$.   Let $C_k(G)$ be the set of
continuous functions $f:G\to\Bbb R$ such that
$\inf\{\|x-y\|:x\in G,\,f(x)\ne 0,\,y\in\BbbR^r\setminus G\}>0$
(counting $\inf\emptyset$ as $\infty$).
Show that for any $p\in\coint{1,\infty}$ the set
$\{f^{\ssbullet}:f\in C_k(G)\}$ is a dense linear subspace of
$L^p(\mu)$.
%244H

\spheader 244Yk\dvAformerly{2{}44Yj.}
 Let $U$ be any Hilbert space.   (i) Show that if
$C\subseteq U$ is convex (that is, $tu+(1-t)v\in C$ whenever
$u$, $v\in C$ and $t\in[0,1]$;  see 233Xd), closed and not empty, and
$u\in U$, then there is a unique $v\in C$ such that
$\|u-v\|=\inf_{w\in C}\|u-w\|$, and $\innerprod{u-v}{v-w}\ge 0$ for
every $w\in C$.   (ii) Show that if $h\in U^*$ there is a unique
$v\in U$
such that $h(w)=\innerprod{w}{v}$ for every $w\in U$.   \Hint{apply (i)
with $C=\{w:h(w)=1\}$, $u=0$.}   (iii) Show that if $V\subseteq U$ is a
closed linear subspace then there is a unique linear projection $P$ on
$U$ such that $P[U]=V$ and $\innerprod{u-Pu}{v}=0$ for all $u\in U$,
$v\in V$ ($P$ is `orthogonal').   \Hint{take $Pu$ to be the point of
$V$ nearest to $u$.}
%244N

\spheader 244Yl\dvAformerly{2{}44Yk.}
 Let $(X,\Sigma,\mu)$ be a probability space, and $\Tau$
a $\sigma$-subalgebra of $\Sigma$.   Use part (iii) of 244Yk to show
that there is an orthogonal projection
$P:L^2(\mu)\to L^2(\mu\restrp\Tau)$ such
that $\int_FPu=\int_Fu$ for every $u\in L^2(\mu)$ and $F\in\Tau$.   Show
that $Pu\ge 0$ whenever $u\ge 0$ and that $\int Pu=\int u$ for every
$u$, so that $P$ has a unique extension to a continuous operator from
$L^1(\mu)$ onto $L^1(\mu\restrp\Tau)$.   Use this to develop the theory
of conditional expectations without using the Radon-Nikod\'ym theorem.
%244N

\spheader 244Ym ({\smc Roselli \& Willem 02})\dvAnew{2009}
(i) Set $C=\coint{0,\infty}^2\subseteq\BbbR^2$.   Let
$\phi:C\to\Bbb R$ be a continuous function
such that $\phi(tz)=t\phi(z)$ for all $z\in C$.
Show that $\phi$ is convex (definition:  233Xd) iff
$t\mapsto\phi(1,t):\coint{0,\infty}\to\Bbb R$ is convex.
(ii) Show that if $p\in\ooint{1,\infty}$ and $q=\bover{p}{p-1}$ then
$(s,t)\mapsto -s^{1/p}t^{1/q}$,
$(s,t)\mapsto -(s^{1/p}+t^{1/p})^p:C\to\Bbb R$ are convex.
(iii) Show that if $p\in[1,2]$ then
$(s,t)\mapsto|s^{1/p}+t^{1/p}|^p+|s^{1/p}-t^{1/p}|^p$ is convex.
(iv) Show that if $p\in\coint{2,\infty}$ then
$(s,t)\mapsto -|s^{1/p}+t^{1/p}|^p-|s^{1/p}-t^{1/p}|^p$ is convex.
(v) Use (ii) and 233Yj to prove H\"older's and Minkowski's inequalities.
(vi) Use (iii) and (iv) to prove Hanner's inequalities.
(vii) Adapt the method to answer (ii) and (iii) of 244Yc.
%244O

\spheader 244Yn\dvAnew{2009}(i) Show that any inner product space is
uniformly convex.
(ii) Let $U$ be a uniformly convex Banach space, $C\subseteq U$ a non-empty
closed convex set, and $u\in U$.   Show that there is a unique $v_0\in C$
such that $\|u-v_0\|=\inf_{v\in C}\|u-v\|$.
(iii) Let $U$ be a uniformly convex Banach space, and $A\subseteq U$ a
non-empty bounded set.   Set $\delta_0=\inf\{\delta:$ there is some
$u\in U$ such that $A\subseteq B(u,\delta)=\{v:\|v-u\|\le\delta\}\}$.
Show that there is a
unique $u_0\in U$ such that $A\subseteq B(u_0,\delta_0)$.
%244O

\spheader 244Yo\dvAnew{2009} Let $(X,\Sigma,\mu)$ be a measure space, and
$u\in L^0(\mu)$.   Suppose that $\sequencen{p_n}$ is a sequence in
$[1,\infty]$ with limit $p\in[1,\infty]$.   Show that if
$\limsup_{n\to\infty}\|u\|_{p_n}$ is finite then
$\lim_{n\to\infty}\|u\|_{p_n}$ is defined and is equal to $\|u\|_p$.
%244O
}%end of exercises

\endnotes{
\Notesheader{244} At this point I feel we must leave the investigation
of further function spaces.   The next stage would have to be a systematic
abstract analysis of general Banach lattices.   The $L^p$ spaces give a
solid foundation for such an analysis, since they introduce the basic
themes of norm-completeness, order-completeness and identification of
dual spaces.   I have tried in the exercises to suggest the importance
of the next layer of concepts:  order-continuity of norms and the
relationship between norm-boundedness and
order-boundedness.   What I have not had space to discuss is the subject
of order-preserving linear operators between Riesz spaces, which is the
key to understanding the order structure of the dual spaces here.   (But
you can make a start by re-reading the theory of conditional expectation
operators in 242J-242L, %242J 242K 242L
243J and 244M.)     All these topics are treated
in {\smc Fremlin 74} and in Chapters 35 and 36 of the next volume.

I remember that one of my early teachers of analysis said that the $L^p$
spaces (for $p\ne 1$, $2$, $\infty$)
had somehow got into the syllabus and had never been
got out again.   I would myself call them classics, in the sense that
they have been part
of the common experience of all functional analysts since functional
analysis began;  and while you are at liberty to dislike them, you can
no more ignore them than you can ignore Milton if you are studying
English poetry.   H\"older's inequality, in particular, has a wealth of
applications;  not only 244F and 244K, but also 244Xc-244Xd and
244Yh-244Yi, for instance.

The $L^p$ spaces, for $1\le p\le \infty$, form a kind of continuum.   In
terms of the concepts dealt with here, there is no distinction to be
drawn between different $L^p$ spaces for $1<p<\infty$ except the
observation that the norm of $L^2$ is an inner product norm,
corresponding to a Euclidean geometry on its finite-dimensional
subspaces.   To discriminate between the other $L^p$ spaces we need much
more refined concepts in the geometry of normed spaces.

In terms of the theorems given here, $L^1$ seems closer to the middle
range of $L^p$ for $1<p<\infty$ than $L^{\infty}$ does;  thus, for all
$1\le p<\infty$, we have $L^p$ Dedekind complete (independent of the
measure space involved), the space $S$ of equivalence classes of simple
functions is dense in $L^p$ (again, for every measure space), and the
dual $(L^p)^*$ is (almost) identifiable as another function space.   All
of these should be regarded as consequences in one way or another of the
order-continuity of the norm of $L^p$ for $p<\infty$.   The chief
obstacle to the universal identification of $(L^1)^*$ with $L^{\infty}$
is that for non-$\sigma$-finite measure spaces the space $L^{\infty}$
can be inadequate, rather than any pathology in the $L^1$ space itself.
(This point, at least, I mean to return to in Volume 3.)
There is also the point that for a non-semi-finite measure space the
purely infinite sets can contribute to $L^{\infty}$ without any
corresponding contribution to $L^1$.   For $1<p<\infty$, neither of these
problems can arise.   Any member of any such $L^p$ is supported entirely
by a $\sigma$-finite part of the measure space, and the same applies to
the dual -- see part (c) of the proof of 244K.

Of course $L^1$ does have a markedly different geometry from the other
$L^p$ spaces.   The first sign of this is that it is not reflexive as a
Banach space (except when it is finite-dimensional), whereas for
$1<p<\infty$ the identifications of $(L^p)^*$ with $L^q$ and of $(L^q)^*$
with $L^p$, where $q=p/(p-1)$, show that the canonical embedding of $L^p$
in $(L^p)^{**}$ is surjective, that is, that $L^p$ is reflexive.   But
even when $L^1$ is
finite-dimensional the unit balls of $L^1$ and $L^{\infty}$ are clearly
different in kind from the unit balls of $L^p$ for $1<p<\infty$;   they
have corners instead of being smoothly rounded (244Xh-244Xi).
A direct expression of the
difference is in 244O.   As usual, the case $p=2$ is both much more
important than the general case and enormously easier (244Yn(i));  and note
how Hanner's inequalities reverse at $p=2$.   (See 244Yc for the reversal
of H\"older's and Minkowski's inequalities at $p=1$.)   There are
occasions on which it is useful to know that $\|\,\|_1$ and
$\|\,\|_{\infty}$ can be approximated, in an exactly describable way,
by uniformly convex norms (244Yo).   I have written out a proof of 244O
based on ingenuity and advanced calculus, like that of 244E.
With a bit more about convex sets and functions, sketched in
233Yf-233Yj, %233Yf 233Yg 233Yh 233Yi 233Yj
there is a striking alternative proof (244Ym).   Of course the proof of
244Ea also uses convexity, upside down.

The proof of 244K, identifying $(L^p)^*$, is a fairly long haul, and it
is natural to ask whether we really have to work so hard, especially
since in the case of $L^2$ we have a much easier argument (244Yk).   Of
course we can go faster if we know a bit more about Banach lattices
(\S369 in Volume 3 has the relevant facts), though this route uses some
theorems quite as hard as 244K as given.   There are alternative routes
using the geometry of the $L^p$ spaces, following the ideas of 244Yk,
but I do not think they are any easier, and the argument I have
presented here at least has the virtue of using some of the same ideas
as the identification of $(L^1)^*$ in 243G.   The difference is that
whereas in
243G we may have to piece together a large family of functions $g_F$
(part (b-v) of the proof), in 244K there are only countably many $g_n$;
consequently we can make the argument work for arbitrary measure spaces,
not just localizable ones.

The geometry of Hilbert space gives us an approach to conditional
expectations which does not depend on the Radon-Nikod\'ym theorem
(244Yl).   To turn these ideas into a proof of the Radon-Nikod\'ym
theorem itself, however, requires qualities of determination and
ingenuity which can be better employed elsewhere.

The convexity arguments of 233J/242K can be used on many operators
besides conditional expectations (see 244Xm).   The class `$\Cal T^+$'
described there is not in fact the largest for which these arguments
work;  I take the ideas farther in Chapter 37.   There is also a great
deal more to be said if you put an arbitrary pair of $L^p$ spaces in
place of $L^1$ and $L^{\infty}$ in 244Xl.   244Yh is a start, but for
the real thing (the `Riesz convexity theorem') I refer you to
{\smc Zygmund 59}, XII.1.11 or {\smc Dunford \& Schwartz 57}, VI.10.11.
}%end of notes


\discrpage

