\wheader{}{10}{4}{4}{100pt}
     
Introduction to Volume 3 \vtmpb{31.12.01}\pagereference{11}{}
     
%wheader parameters:  #1 new paragraph name
 % #2#3#4  \vskip parameters if page break not forced (no. of points)
 % #5  spare height demanded if page break not to be forced
\wheader{}{10}{4}{4}{100pt}
     
 Chapter 31:  Boolean algebras
     
\chapintrosection{31.12.01}{13}{}
     
\section{311}{Boolean algebras}{15.10.08}{13}{}
{Boolean rings and algebras;  ideals and ring homomorphisms to
$\Bbb{Z}_2$;  Stone's theorem;  the operations $\Bcup$, $\Bcap$,
$\Bsymmdiff$, $\Bsetminus$ and the relation $\Bsubseteq$;  partitions of
unity;  topology of
the Stone space;   Boolean algebras as complemented distributive
lattices.}
     
\section{312}{Homomorphisms}{29.5.07}{21}{}
{Subalgebras;  ideals;  Boolean homomorphisms;  the ordering
determines the ring structure;  quotient algebras;  extension of
homomorphisms;  homomorphisms and Stone spaces.}
     
\section{313}{Order-continuity}{8.6.11}{29}{}
{General distributive laws;  order-closed sets;  order-closures;
Monotone Class Theorem;  order-preserving functions;  order-continuity;
order-dense sets;  order-continuous Boolean homomorphisms;  and Stone
spaces;  regularly embedded subalgebras;  upper envelopes.}
     
\section{314}{Order-completeness}{26.7.07}{39}{}
{Dedekind completeness and $\sigma$-completeness;  quotients,
subalgebras, principal ideals;  order-continuous homomorphisms;
extension of homomorphisms;  Loomis-Sikorski representation of a
$\sigma$-complete algebra as a quotient of a $\sigma$-algebra of sets;
regular open algebras;  Stone spaces;  Dedekind completion of a Boolean
algebra.}
     
\section{315}{Products and free products}{13.11.12}{49}{}
{Simple product of Boolean algebras;  free product of Boolean
algebras;  algebras of sets and their quotients;  projective and inductive
limits.}
     
\section{316}{Further topics}{26.1.09}{59}{}
{The countable chain condition;  weak
$(\sigma,\infty)$-distributivity;  Stone spaces;  atomic and atomless
Boolean algebras;  homogeneous Boolean algebras.}
     
\wheader{}{10}{4}{4}{100pt}
     
 Chapter 32:  Measure algebras
     
\chapintrosection{6.1.02}{68}{}
     
\section{321}{Measure algebras}{3.1.11}{68}{}
{Measure algebras;  elementary properties;  the measure algebra of
a measure space;  Stone spaces.}
     
\section{322}{Taxonomy of measure algebras}{24.4.06}{71}{}
{Totally finite, $\sigma$-finite, semi-finite and localizable
measure algebras;  relation to corresponding types of measure space;
completions and c.l.d.\ versions of measures;  semi-finite measure
algebras are \wsid;  subspace measures and indefinite-integral measures;  
simple products of measure
algebras;  Stone spaces of localizable measure algebras;  localizations
of semi-finite measure algebras.}
     
\section{323}{The topology of a measure algebra}{20.7.06}{81}{}
{Defining a topology and uniformity on a measure algebra;
continuity of algebraic operations;  order-closed sets;  Hausdorff and
metrizable topologies, complete uniformities;  closed subalgebras;
products.}
     
\section{324}{Homomorphisms}{6.2.02}{87}{}
{Homomorphisms induced by measurable functions;  order-continuous
and continuous homomorphisms;  the topology of a semi-finite measure
algebra is determined by the algebraic structure;  measure-preserving
homomorphisms.}
     
\section{325}{Free products and product measures}{30.8.06}{93}{}
{The measure algebra of a product measure;  the
localizable measure algebra free product of two semi-finite measure
algebras;  the measure algebra of a product of probability measures;
the probability algebra free product of probability algebras;
factorizing through subproducts.}
     
\section{326}{Additive functionals on Boolean algebras}{21.5.11}{102}{}
{Additive, countably additive and completely additive functionals;
Jordan decomposition;  Hahn decomposition;  Liapounoff's convexity 
theorem;  the region $\Bvalue{\mu>\nu}$.}
     
\section{327}{Additive functionals on measure algebras}{13.7.11}{114}{}
{Absolutely continuous and continuous additive functionals;
Radon-Nikod\'ym theorem;  the standard extension of a continuous
additive functional on a closed subalgebra.}

\section{*328}{Reduced products and other constructions}{2.6.09}{120}{}
{Reduced products of probability algebras;  inductive and projective
limits;  converting homomorphisms into automorphisms.}
     
\wheader{}{10}{4}{4}{100pt}
     
 Chapter 33:  Maharam's theorem
     
\chapintrosection{13.1.02}{127}{}
     
\section{331}{Maharam types and homogeneous measure 
algebras}{1.2.05}{127}{}
{Relatively atomless algebras;  one-step extension of
measure-preserving homomorphisms;  Maharam type of a measure algebra;
\Mth\ probability algebras of the same Maharam type are
isomorphic;  the measure algebra of $\{0,1\}^{\kappa}$ is homogeneous.}
     
\section{332}{Classification of localizable measure
algebras}{19.3.05}{135}{}
{Any localizable measure algebra is isomorphic to a simple product
of homogeneous totally finite algebras;  complete description of
isomorphism types;  closed subalgebras.}
     
\section{333}{Closed subalgebras}{27.6.08}{145}{}
{Relative Maharam types;  extension of measure-preserving Boolean
homomorphisms;  complete classification of closed subalgebras of
probability algebras as triples $(\frak{A},\bar\mu,\frak{C})$;  fixed-point
subalgebras.}
     
\section{334}{Products}{26.9.08}{160}{}
{Maharam types of product measures;  infinite powers of probability spaces
are Maharam-type-homogeneous.}
     
\wheader{}{10}{4}{4}{100pt}
     
 Chapter 34:  Liftings
     
\chapintrosection{19.1.02}{163}{}
     
\section{341}{The lifting theorem}{9.4.10}{163}{}
{Liftings and lower densities;  strictly localizable spaces have
lower densities;  construction of a lifting from a density;  complete
strictly localizable spaces have liftings;  liftings and Stone spaces.}
     
\section{342}{Compact measure spaces}{9.7.10}{174}{}
{Inner regular measures;  compact classes;  compact and locally
compact measures;  perfect measures.}
     
\section{343}{Realization of homomorphisms}{17.11.10}{181}{}
{Representing homomorphisms between measure algebras by functions;
possible when target measure space is locally compact;  countably
separated measures and uniqueness of representing functions;  the split
interval;  perfect measures.}
     
\section{344}{Realization of automorphisms}{22.3.06}{189}{}
{Simultaneously representing groups of automorphisms of
measure algebras by functions -- Stone spaces,
countably separated measure spaces,
measures on $\{0,1\}^I$;  characterization of Lebesgue measure as a
measure space;  strong homogeneity of usual measure on $\{0,1\}^I$.}
     
\section{345}{Translation-invariant liftings}{27.6.06}{198}{}
{Translation-invariant liftings on $\Bbb{R}^r$ and $\{0,1\}^I$;
there is no t.-i.\ Borel lifting on $\Bbb{R}$.}
     
\section{346}{Consistent liftings}{17.12.10}{206}{}
{Liftings of product measures which respect the product structure;
translation-invariant liftings on $\{0,1\}^I$;  products of \Mth\
probability spaces;  lower densities respecting product
structures;  consistent liftings;  the Stone space of Lebesgue measure.}
     
\wheader{}{10}{4}{4}{0pt}
     
% Concordance to part I \pagereference{216}{}

%213 pp
