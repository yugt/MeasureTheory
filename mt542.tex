\frfilename{mt542.tex}
\versiondate{8.7.13}
\copyrightdate{2004}

\def\chaptername{Real-valued-measurable cardinals}
\def\sectionname{Quasi-measurable cardinals}

\newsection{542}

As is to be expected, the results of \S541 take especially dramatic
forms when we look at $\omega_1$-saturated $\sigma$-ideals.   542B-542C
spell out the application of the most important ideas from \S541 to this
special case.   In addition, we can us Shelah's pcf theory to give us
some remarkable combinatorial results concerning cardinal arithmetic
(542E) and cofinalities of partially ordered sets (542I-542J).

\leader{542A}{Definition} A cardinal $\kappa$ is {\bf quasi-measurable}
if $\kappa$ is regular and uncountable and there is an
$\omega_1$-saturated normal ideal on $\kappa$.

\leader{542B}{Proposition} If $X$ is a set and $\Cal I$ is a proper
$\omega_1$-saturated $\sigma$-ideal of $\Cal PX$ containing singletons,
then $\add\Cal I$ is quasi-measurable.

\proof{ This is immediate from the special case $\lambda=\omega_1$ of 541J.
}%end of proof of 542B

\leader{542C}{Proposition} If $\kappa$ is a \qm\ cardinal, then $\kappa$
is weakly inaccessible, the set of weakly inaccessible cardinals less
than $\kappa$ is stationary in $\kappa$, and either
$\kappa\le\frak c$ or $\kappa$ is \2vm.

\proof{ By 541L, $\kappa$ is weakly inaccessible and the set of weakly
inaccessible cardinals less than $\kappa$ is stationary in $\kappa$.
Let $\Cal I$ be an $\omega_1$-saturated normal ideal on $\kappa$
and $\frak A=\Cal P\kappa/\Cal I$.   Then 541P tells us that either
$\frak A$ is atomless and $\kappa\le\frak c$ or $\frak A$ is purely
atomic and $\kappa$ is \2vm.
}%end of proof of 542C

\leader{542D}{Proposition} %RVMC 7K
Let $\kappa$ be a quasi-measurable cardinal.

(a) Let $\langle\theta_{\zeta}\rangle_{\zeta<\lambda}$ be a family
such that $\lambda<\kappa$ is a cardinal, every $\theta_{\zeta}$
is a regular infinite cardinal and
$\lambda<\theta_{\zeta}<\kappa$ for every $\zeta<\lambda$.   Then
$\cf(\prod_{\zeta<\lambda}\theta_{\zeta})<\kappa$.

(b) If $\alpha$ and $\gamma$ are cardinals less than $\kappa$, then
$\Theta(\alpha,\gamma)$\cmmnt{ (definition: 5A2Db)} is less than
$\kappa$.

(c) If $\alpha$, $\beta$, $\gamma$ and $\delta$ are cardinals, with
$\alpha<\kappa$, $\gamma\le\beta$ and $\delta\ge\omega_1$, then
$\covSh(\alpha,\beta,\gamma,\delta)$\cmmnt{ (definition: 5A2Da)}
is less than $\kappa$.

(d) $\Theta(\kappa,\kappa)=\kappa$.

\proof{ Fix an $\omega_1$-saturated normal ideal $\Cal I$ on $\kappa$.

\medskip

{\bf (a)} \Quer\ Suppose, if possible, otherwise.   Then $\lambda$ is
surely infinite.   By 5A2Bc, 
there is an ultrafilter $\Cal F$ on $\lambda$ such that
$\cf(\prod_{\zeta<\lambda}\theta_{\zeta})
=\cf(\prod_{\zeta<\lambda}\theta_{\zeta}|\Cal F)$, where the reduced
product $\prod_{\zeta<\lambda}\theta_{\zeta}|\Cal F$ is defined in 5A2A;
by 5A2C, 
there is a family $\ofamily{\zeta}{\lambda}{\theta'_{\zeta}}$ of regular
cardinals such that $\lambda<\theta'_{\zeta}\le\theta_{\zeta}$ for each
$\zeta$ and $\cf(\prod_{\zeta<\lambda}\theta'_{\zeta}|\Cal F)=\kappa$.
Let $\ofamily{\xi}{\kappa}{p_{\xi}}$ be a cofinal family in
$P=\prod_{\zeta<\lambda}\theta'_{\zeta}|\Cal F$.
For each $\xi<\kappa$ we can find $q_{\xi}\in P$ such that
$q_{\xi}\not\le p_{\eta}$ for any $\eta\le\xi$;  because $P$ is
upwards-directed, we can suppose that $q_{\xi}\ge p_{\xi}$, so that
$\{q_{\xi}:\xi<\kappa\}$ also is cofinal with $P$.    Choose
$f_{\xi}\in\prod_{\zeta<\lambda}\theta'_{\zeta}$ such that
$f_{\xi}^{\ssbullet}=q_{\xi}$ for each $\xi$.

For each $\zeta<\lambda$, $\theta'_{\zeta}<\kappa$, so there is a
countable set $M_{\zeta}\subseteq\theta'_{\zeta}$ such that
$I_{\zeta}=\kappa\setminus\{\xi:f_{\xi}(\zeta)\in M_{\zeta}\}$ belongs
to $\Cal I$ (541E).   As

\Centerline{$\omega\le\lambda<\theta'_{\zeta}=\cf\theta'_{\zeta}$,}

\noindent we can find $g(\zeta)$ such that
$M_{\zeta}\subseteq g(\zeta)<\theta'_{\zeta}$.   Consider
$g^{\ssbullet}$ in $\prod_{\zeta<\lambda}\theta'_{\zeta}|\Cal F$.
There is an $\eta<\kappa$ such that $g^{\ssbullet}\le p_{\eta}$, so that
$f_{\xi}^{\ssbullet}\not\le g^{\ssbullet}$ for every $\xi\ge\eta$.
On the other hand, $\eta\cup\bigcup_{\zeta<\lambda}I_{\zeta}$ belongs to
$\Cal I$, so there is a $\xi\ge\eta$ such that
$f_{\xi}(\zeta)\in M_{\zeta}$ for every $\zeta<\lambda$;  in which case
$f_{\xi}\le g$, which is impossible.\ \Bang

\medskip

{\bf (b)} \Quer\ Suppose, if possible, otherwise.   Of course we can
suppose that $\gamma$ is infinite.   For each
$\xi<\kappa$ there must be a family
$\langle\theta_{\xi\zeta}\rangle_{\zeta<\lambda_{\xi}}$
of regular cardinals less than $\alpha$ such that
$\lambda_{\xi}<\gamma$, $\omega\le\lambda_{\xi}<\theta_{\xi\zeta}$
for every $\zeta<\lambda_{\xi}$
and $\cf(\prod_{\zeta<\lambda_{\xi}}\theta_{\xi\zeta})\ge\xi$.
Let $\lambda$ be such that
$A=\{\xi:\xi<\kappa$, $\lambda_{\xi}=\lambda\}\notin\Cal I$.
By 541Ra, applied to the function
$I\mapsto\{\theta_{\xi\zeta}:\xi\in A\cap I$, $\zeta<\lambda_{\xi}\}:
[\kappa]^{<\omega}\to[\alpha]^{<\lambda^+}$,
there are $C\in\Cal I$ and $M\in[\alpha]^{\le\lambda}$ such that
$\theta_{\xi\zeta}\in M$ whenever
$\xi\in A\setminus C$ and $\zeta<\lambda$.
Let $\langle\theta_{\zeta}\rangle_{\zeta<\lambda'}$ enumerate
$\{\theta:\theta\in M$ is a regular cardinal, $\theta>\lambda\}$.
By (a), there is a cofinal set
$F\subseteq\prod_{\zeta<\lambda'}\theta_{\zeta}$ with $\#(F)<\kappa$.
Let $\xi\in A\setminus C$ be such that $\xi>\#(F)$.
For each $f\in F$ define
$g_f\in\prod_{\zeta<\lambda}\theta_{\xi\zeta}$ by setting

\Centerline{$g_f(\zeta)=f(\zeta')$
whenever $\theta_{\xi\zeta}=\theta_{\zeta'}$.}

\noindent Then $\{g_f:f\in F\}$ is cofinal with
$\prod_{\zeta<\lambda}\theta_{\xi\zeta}$, because if
$h\in\prod_{\zeta<\lambda}\theta_{\xi\zeta}$
there is an $f\in F$ such that

\Centerline{$f(\zeta')\ge\sup\{h(\zeta):\zeta<\lambda$,
$\theta_{\xi\zeta}=\theta_{\zeta'}\}$}

\noindent for every $\zeta'<\lambda'$, and in this case $h\le g_f$.   So

\Centerline{$\#(F)<\xi\le\cf(\prod_{\zeta<\lambda}\theta_{\xi\zeta})
\le\#(F)$,}

\noindent which is absurd.\ \Bang

\medskip

{\bf (c)} This is trivial if any of the cardinals $\alpha$, $\beta$ or
$\gamma$ is finite;  let us take it that they are all infinite.   Then

\Centerline{$\covSh(\alpha,\beta,\gamma,\delta)
\le\covSh(\alpha,\gamma,\gamma,\omega_1)
\le\max(\omega,\alpha,\Theta(\alpha,\gamma))<\kappa$}

\noindent by 5A2D, 5A2G and (b) above.

\medskip

{\bf (d)} Because $\kappa$ is an uncountable limit cardinal,
$\kappa\le\Theta(\kappa,\kappa)$.   (If $\omega\le\theta<\kappa$, then
$\cf\theta^+\le\Theta(\kappa,\kappa)$.)
On the other hand, let $\lambda<\kappa$ be an infinite cardinal and
$\langle\theta_{\zeta}\rangle_{\zeta<\lambda}$ a family of infinite
regular cardinals such that $\lambda<\theta_{\zeta}<\kappa$ for every
$\zeta<\lambda$.   Then
$\alpha=\sup_{\zeta<\lambda}\theta_{\zeta}^+<\kappa$ and

\Centerline{$\cf(\prod_{\zeta<\lambda}\theta_{\zeta})
\le\Theta(\alpha,\lambda^+)<\kappa$.}

\noindent As $\langle\theta_{\zeta}\rangle_{\zeta<\lambda}$ is
arbitrary, $\Theta(\kappa,\kappa)=\kappa$.
}%end of proof of 542D

\leader{542E}{Theorem}\cmmnt{ ({\smc Gitik \& Shelah 93})}
%RVMC 7P
If $\kappa\le\frak c$ is a \qm\ cardinal, then

\Centerline{$\{2^{\gamma}:\omega\le\gamma<\kappa\}$}

\noindent is finite.

\proof{ \Quer\ Suppose, if possible, otherwise.

\medskip

{\bf (a)} Define a sequence $\langle\gamma_n\rangle_{n\in\Bbb N}$ of
cardinals by setting

\Centerline{$\gamma_0=\omega$,
\quad$\gamma_{n+1}=\min\{\gamma:2^{\gamma}>2^{\gamma_n}\}$ for
$n\in\Bbb N$.}

\noindent Then we are supposing that $\gamma_n<\kappa$ for every $n$, so
by 541S, %\gamma_n regular,
  %2^{\gamma_{n+1}}=\covSh(2^{\gamma_n},\kappa,\gamma_{n+1}^+,\omega_1)
5A2D, %\covSh(2^{\gamma_n},\kappa,\gamma_{n+1}^+,\omega_1)
  %\le\covSh(2^{\gamma_n},\gamma_{n+1}^+,\gamma_{n+1}^+,\omega_1)
5A2G %\covSh(2^{\gamma_n},\gamma_{n+1}^+,\gamma_{n+1}^+,\omega_1)
  %\le\max(2^{\gamma_n},\Theta(2^{\gamma_n},\gamma_{n+1}^+))
and 5A2F %\Theta(2^{\gamma_n},\gamma_{n+1}^+)\le 2^{\gamma_{n+1}}
$\gamma_n$ is regular and

$$\eqalign{2^{\gamma_{n+1}}
&=\covSh(2^{\gamma_n},\kappa,\gamma_{n+1}^+,\omega_1)
\le\covSh(2^{\gamma_n},\gamma_{n+1}^+,\gamma_{n+1}^+,\omega_1)\cr
&\le\max(2^{\gamma_n},\Theta(2^{\gamma_n},\gamma_{n+1}^+))
\le 2^{\gamma_{n+1}}\cr}$$

\noindent for every $n\in\Bbb N$.

\medskip

{\bf (b)} Now $\Theta(2^{\gamma_n},\gamma)=\Theta(\frak c,\gamma)$
whenever $n\in\Bbb N$ and $\gamma$ is a regular cardinal with
$\gamma_n<\gamma<\kappa$.
\Prf\ Induce on $n$.   For $n=0$ we have $\frak c=2^{\gamma_0}$.   For
the inductive step to $n+1$, if $\gamma$ is regular and
$\gamma_{n+1}<\gamma<\kappa$, then
$\frak c\ge\kappa>\Theta(\gamma,\gamma)$ (542Db), so

$$\eqalignno{\Theta(2^{\gamma_{n+1}},\gamma)
&=\Theta(\Theta(2^{\gamma_n},\gamma_{n+1}^+),\gamma)\cr
\displaycause{by (a)}
&\le\Theta(\Theta(2^{\gamma_n},\gamma),\gamma)\cr
\displaycause{because $\gamma\ge\gamma_{n+1}^+$ and $\Theta$ is
order-preserving}
&=\Theta(\Theta(\frak c,\gamma),\gamma)\cr
\displaycause{by the inductive hypothesis}
&\le\Theta(\frak c,\gamma)\cr
\displaycause{5A2H}    
&\le\Theta(2^{\gamma_{n+1}},\gamma)\cr}$$

\noindent (because $2^{\gamma_{n+1}}\ge\frak c$).\ \QeD\  In particular,

\Centerline{$2^{\gamma_{n+1}}=\Theta(2^{\gamma_n},\gamma_{n+1}^+)
=\Theta(\frak c,\gamma_{n+1}^+)$}

\noindent for every $n\in\Bbb N$.

\medskip

{\bf (c)} For each $n\in\Bbb N$, let $\lambda_n$ be the least infinite
cardinal such that $\Theta(\lambda_n,\gamma_n^+)>\frak c$.   Then
$\langle\lambda_n\rangle_{n\in\Bbb N}$ is non-increasing;  also
$\lambda_1\le\frak c$, because

\Centerline{$\Theta(\frak c,\gamma_1^+)=2^{\gamma_1}>\frak c$,}

\noindent so there are $n\ge 1$, $\lambda\le\frak c$ such that
$\lambda_m=\lambda$
for every $m\ge n$.   Now for $m\ge n$ we have

$$\eqalignno{\frak c
<\Theta(\lambda,\gamma_m^+)
&\le\max(\lambda,
(\sup_{\lambda'<\lambda}
  \Theta(\lambda',\gamma_m^+))^{\cf\lambda})\cr
\displaycause{5A2I}   
&\le\max(\lambda,\frak c^{\cf\lambda})=2^{\cf\lambda}.\cr}$$

\noindent Also we still have
$\lambda\ge\kappa>\Theta(\gamma_n^+,\gamma_n^+)$ because
$\Theta(\lambda',\gamma_n^+)<\kappa\le\frak c$
for every $\lambda'<\kappa$.   Using 5A2H again,

\Centerline{$2^{\gamma_n}
=\Theta(\frak c,\gamma_n^+)
\le\Theta(\Theta(\lambda,\gamma_n^+),\gamma_n^+)
\le\Theta(\lambda,\gamma_n^+)
\le 2^{\cf\lambda}$;}

\noindent consequently

\Centerline{$2^{\gamma_n}<2^{\gamma_{n+1}}\le 2^{\cf\lambda}$}

\noindent and $\cf\lambda>\gamma_n$.   But 5A2Ia now tells us that

\Centerline{$\Theta(\lambda,\gamma_n^+)
\le\max(\lambda,
   \sup_{\lambda'<\lambda}\Theta(\lambda',\gamma_n^+))
\le\frak c$,}

\noindent which is absurd.\ \Bang

This contradiction proves the theorem.
}%end of proof of 542E

\leader{542F}{Corollary} Let $\kappa\le\frak c$ be a \qm\ cardinal.

(a) There is a regular infinite cardinal $\gamma<\kappa$ such that
$2^{\gamma}=2^{\delta}$ for every cardinal $\delta$ such that
$\gamma\le\delta<\kappa$;  that is, $\#([\kappa]^{<\kappa})=2^{\gamma}$.

(b) Let $\Cal I$ be any proper $\omega_1$-saturated $\kappa$-additive
ideal of $\Cal P\kappa$ containing singletons, and
$\frak A=\Cal P\kappa/\Cal I$.
If $\gamma$ is as in (a), then the
cardinal power $\tau(\frak A)^{\gamma}$ is equal to $2^{\kappa}$.

%\quad(ii) $\tau(\frak A)^{\omega}\le 2^{\kappa}\le 2^{\tau(\frak A)}$.

\proof{{\bf (a)} By 542E, there is a first $\gamma<\kappa$ such that
$2^{\delta}=2^{\gamma}$ whenever $\gamma\le\delta<\kappa$.   Of course
$\gamma$ is infinite;  by 5A1Eh it is regular.
Because $\kappa$ is regular,

\Centerline{$[\kappa]^{<\kappa}=\bigcup_{\xi<\kappa}\Cal P\xi$,
\quad
$\#([\kappa]^{<\kappa})=\max(\kappa,\sup_{\delta<\kappa}2^{\delta})
=\sup_{\delta<\kappa}2^{\delta}=2^{\gamma}$.}

\medskip

{\bf (b)} Of course $\tau(\frak A)\le\#(\frak A)\le\#(\Cal P\kappa)$, so
$\tau(\frak A)^{\gamma}\le(2^{\kappa})^{\gamma}=2^{\kappa}$.   In the
other direction, we have an injective function
$\phi_{\xi}:\Cal P\xi\to\Cal P\gamma$ for each $\xi<\kappa$.   For
$A\subseteq\kappa$ and $\eta<\gamma$ set

\Centerline{$d_{A\eta}
=\{\xi:\xi<\kappa$, $\eta\in\phi_{\xi}(A\cap\xi)\}^{\ssbullet}
\in\frak A$.}

\noindent If $A$, $B\subseteq\kappa$ are distinct then there is a
$\zeta<\kappa$ such that $\phi_{\xi}(A\cap\xi)\ne\phi_{\xi}(B\cap\xi)$
for every $\xi\ge\zeta$, that is,

\Centerline{$\bigcup_{\eta<\gamma}
  \{\xi:\eta\in\phi_{\xi}(A\cap\xi)\symmdiff\phi_{\xi}(B\cap\xi)\}
\supseteq\kappa\setminus\zeta\notin\Cal I$.}

\noindent Because $\Cal I$ is $\kappa$-additive and $\gamma<\kappa$,
there is an $\eta<\gamma$ such that
$\{\xi:\eta\in\phi_{\xi}(A\cap\xi)\symmdiff\phi_{\xi}(B\cap\xi)\}
\notin\Cal I$, that is, $d_{A\eta}\ne d_{B\eta}$.   Thus
$A\mapsto\ofamily{\eta}{\gamma}{d_{A\eta}}:
\Cal P\kappa\to\frak A^{\gamma}$ is injective, and
$2^{\kappa}\le\#(\frak A)^{\gamma}$.   But $\frak A$ is ccc, so
$\#(\frak A)\le\max(4,\tau(\frak A)^{\omega})$ (514De) and

\Centerline{$2^{\kappa}\le(\tau(\frak A)^{\omega})^{\gamma}
=\tau(\frak A)^{\gamma}$.}
}%end of proof of 542F

\leader{542G}{Corollary} Suppose that $\kappa$ is a \qm\ cardinal.

(a) If $\kappa\le\frak c<\kappa^{(+\omega_1)}$\cmmnt{ (notation:
5A1E(a-ii))}, then
$2^{\lambda}\le\frak c$ for every cardinal $\lambda<\kappa$.

(b) Let $\Cal I$ be any proper $\omega_1$-saturated $\kappa$-additive
ideal of $\Cal P\kappa$ containing singletons, and
$\frak A=\Cal P\kappa/\Cal I$.   If $2^{\lambda}\le\frak c$ for every
cardinal $\lambda<\kappa$, then $\#(\frak A)=2^{\kappa}$.

\proof{{\bf (a)} \Quer\ Otherwise, taking $\gamma$ as in 542Fa,
$2^{\gamma}>\frak c$.   Let $\gamma_1\le\gamma$ be the first cardinal
such that $2^{\gamma_1}>\frak c$;  note that, using 542E and 5A1Eh,
we can be sure that $\gamma_1$ is regular.   Next,

\Centerline{$\covSh(\kappa,\kappa,\gamma_1^+,\gamma_1)
\le\cff[\kappa]^{<\kappa}=\kappa$,}

\noindent by 5A2Ea.   So
$\covSh(\alpha,\kappa,\gamma_1^+,\gamma_1)\le\alpha$ whenever
$\kappa\le\alpha<\kappa^{(+\gamma_1)}$ (induce on $\alpha$, using
5A2Eb).
In particular, $\covSh(\frak c,\kappa,\gamma_1^+,\gamma_1)\le\frak c$.
But 541S tells us that
$\covSh(\frak c,\kappa,\gamma_1^+,\gamma_1)=2^{\gamma_1}$.\ \Bang

\medskip

{\bf (b)} By 541B, $\frak A$ is Dedekind complete;  by 515L,
$\#(\frak A)^{\omega}=\#(\frak A)$;  so 542Fb tells us that

\Centerline{$2^{\kappa}=\tau(\frak A)^{\omega}\le\#(\frak A)
\le\#(\Cal P\kappa)=2^{\kappa}$.}
}%end of proof of 542G

\leader{542H}{Lemma} %RVMC S7A
Let $\kappa$ be a \qm\ cardinal and
$\langle\alpha_i\rangle_{i\in I}$ a countable family of ordinals less
than $\kappa$ and of cofinality at least $\omega_2$.   Then there is a
set $F\subseteq P=\prod_{i\in I}\alpha_i$ such that

(i) $F$ is cofinal with $P$;

(ii) if $\langle f_{\xi}\rangle_{\xi<\omega_1}$
is a non-decreasing family in $F$ then
$\sup_{\xi<\omega_1}f_{\xi}\in F$;

(iii) $\#(F)<\kappa$.

\proof{ Note that $\add P=\min_{i\in I}\cf\alpha_i>\omega_1$ (I am
passing over the trivial case $I=\emptyset$), so
$\sup_{\xi<\omega_1}f_{\xi}$ is defined in $P$ for every family
$\ofamily{\xi}{\omega_1}{f_{\xi}}$ in $P$.   We have

\Centerline{$\cf P
=\cf(\prod_{i\in I}\cf\alpha_i)
\le\Theta(\sup_{i\in I}(\cf\alpha_i)^+,\omega_1)
<\kappa$,}

\noindent by 542Db.   So
we can find a cofinal set $F_0\subseteq P$ of
cardinal less than $\kappa$.   Now for $0<\zeta\le\omega_2$ define
$F_{\zeta}$ by saying that

\Centerline{$F_{\zeta+1}=\{\sup_{\xi<\omega_1}f_{\xi}:
\langle f_{\xi}\rangle_{\xi<\omega_1}$ is a non-decreasing family in
$F_{\zeta}\}$,}

\Centerline{$F_{\zeta}=\bigcup_{\eta<\zeta}F_{\eta}$ for non-zero limit
ordinals $\zeta\le\omega_2$.}

\noindent Then $\#(F_{\zeta})<\kappa$ for every $\zeta$.   \Prf\ Induce
on $\zeta$.   For the inductive step to $\zeta+1$, \Quer\ suppose, if
possible, that $\#(F_{\zeta})<\kappa$ but $\#(F_{\zeta+1})\ge\kappa$.
Then there is a proper $\omega_1$-saturated $\kappa$-additive
$\Cal I\normalsubgroup\Cal PF_{\zeta+1}$ containing singletons.
For each $h\in F_{\zeta+1}$ choose a non-decreasing family
$\langle f_{h\xi}\rangle_{\xi<\omega_1}$ in $F_{\zeta}$ with supremum
$h$.   The set $h[I]$ of values of $h$ is a countable subset of
$Y=\bigcup_{i\in I}\alpha_i$, and $\#(Y)<\kappa$.   By 541D, there is a
set $H\subseteq F_{\zeta+1}$, with cardinal $\kappa$, such that
$M=\bigcup_{h\in H}h[I]$ is countable.   Now, for each $h\in H$, there
is a $\gamma(h)<\omega_1$ such that whenever $i\in I$ and $\beta\in M$
then $h(i)>\beta$ iff
$f_{h,\gamma(h)}(i)>\beta$.  If $g$, $h\in H$ and $i\in I$ and
$g(i)<h(i)$,
then $f_{g,\gamma(g)}(i)\le g(i)<f_{h,\gamma(h)}(i)$, because
$g(i)\in M$.
Thus $h\mapsto f_{h,\gamma(h)}:H\to F_{\zeta}$ is injective;  but
$\#(F_{\zeta})<\kappa=\#(H)$.\ \Bang

Thus $\#(F_{\zeta+1})<\kappa$ if $\#(F_{\zeta})<\kappa$.   At limit
ordinals $\zeta$ the induction proceeds without difficulty because
$\cf\kappa>\zeta$.\ \Qed

So $\#(F_{\omega_2})<\kappa$ and we may take $F=F_{\omega_2}$.
}%end of proof of 542H

\leader{542I}{Theorem}\cmmnt{ ({\smc Shelah 96})} %RVMC S7B
Let $\kappa$ be a \qm\ cardinal.

(a) For any cardinal $\theta$, $\cff[\kappa]^{<\theta}\le\kappa$.

(b) For any cardinal $\lambda<\kappa$, and any $\theta$,
$\cff[\lambda]^{<\theta}<\kappa$.

\proof{{\bf (a)(i)} Consider first the case $\theta=\omega_1$.
Write $G_1$ for the set of ordinals less than $\kappa$ of cofinality
less than or equal to $\omega_1$;  for $\delta\in G_1$  let
$\psi_{\delta}:\cf\delta\to\delta$ enumerate
a cofinal subset of $\delta$.   Next, write
$G_2$ for $\kappa\setminus G_1$, and for every countable set
$A\subseteq G_2$ let
$F(A)\subseteq\prod_{\alpha\in A}\alpha$ be a cofinal set, with cardinal
less than $\kappa$, closed under
suprema of non-decreasing families of length $\omega_1$;  such exists by
542H above.

\medskip

\quad{\bf (ii)} It is worth observing at this point that if
$\langle A_{\zeta}\rangle_{\zeta<\omega_1}$ is any family of countable
subsets of $G_2$, $D=\bigcup_{\zeta<\omega_1}A_{\zeta}$, and
$g\in\prod_{\alpha\in D}\alpha$, then there is an
$f\in\prod_{\alpha\in D}\alpha$ such that
$f\ge g$ and $f\restr A_{\zeta}\in F(A_{\zeta})$ for every
$\zeta<\omega_1$.   \Prf\ Let $\langle\phi(\xi)\rangle_{\xi<\omega_1}$
run over $\omega_1$ with cofinal repetitions.   Choose a
non-decreasing family $\langle f_{\xi}\rangle_{\xi<\omega_1}$ in
$\prod_{\alpha\in D}\alpha$ in such a way that $f_0=g$ and
$f_{\xi+1}\restr A_{\phi(\xi)}\in F(A_{\phi(\xi)})$ for every $\xi$;
this is possible because $\add(\prod_{\alpha\in D})\ge\omega_2$ and
$F(A)$ is cofinal with $\prod_{\alpha\in A}\alpha$ for every $A$.   Set
$f=\sup_{\xi<\omega_1}f_{\xi}$;  this
works because every $F(A)$ is closed under suprema of non-decreasing
families of length $\omega_1$.\ \Qed

\medskip

\quad{\bf (iii)} We can now find a family $\Cal A$ of countable subsets
of $\kappa$ such that

\qquad$(\alpha)$ $\{\alpha\}\in\Cal A$ for every $\alpha<\kappa$;

\qquad$(\beta)$ whenever $A$, $A'\in\Cal A$ and $\zeta<\omega_1$ then
$A\cup A'$, $A\cap G_2$ and $\{\psi_{\alpha}(\xi):
\alpha\in A\cap G_1,\,\xi<\min(\zeta,\cf\alpha)\}$ all belong to
$\Cal A$;

\qquad$(\gamma)$ whenever $A\in\Cal A\cap[G_2]^{\le\omega}$ and
$f\in F(A)$ then $f[A]\in\Cal A$;

\qquad$(\delta)$ $\#(\Cal A)\le\kappa$.

\medskip

\quad{\bf (iv)} \Quer\ Suppose, if possible, that
$\cff[\kappa]^{\le\omega}>\kappa$.   Because
$[\kappa]^{\le\omega}=\bigcup_{\lambda<\kappa}[\lambda]^{\le\omega}$,
there is a cardinal $\lambda<\kappa$ such that
$\cff[\lambda]^{\le\omega}>\kappa$.   We can therefore choose
inductively a family $\langle a_{\xi}\rangle_{\xi<\kappa}$ of countable
subsets of $\lambda$ such that

\Centerline{$a_{\xi}\not\subseteq\bigcup_{\eta\in A\cap\xi}a_{\eta}$}

\noindent whenever $\xi<\kappa$ and $A\in\Cal A$.   By 541D, there is a
set $W\subseteq\kappa$, with cardinal $\kappa$, such that
$\bigcup_{\xi\in W}a_{\xi}$ is countable.   Let $\delta<\kappa$ be such
that $W\cap\delta$ is cofinal with $\delta$ and of order type
$\omega_1$;  then $\delta\in G_1$.

\medskip

\quad{\bf (v)} I choose a family
$\langle A_{k\zeta}\rangle_{\zeta<\omega_1,k\in\Bbb N}$ in $\Cal A$ as
follows.    Start by setting
$A_{0\zeta}=\psi_{\delta}[\zeta]$ for every $\zeta<\omega_1$;  then
$A_{0\zeta}\in\Cal A$ by (iii)($\alpha$-$\beta$).   Given
$\langle A_{k\zeta}\rangle_{\zeta<\omega_1}$, set
$A'_{k\zeta}=A_{k\zeta}\cap G_2$ for each $\zeta<\omega_1$.   For
$\alpha\in D_k=\bigcup_{\zeta<\omega_1}A'_{k\zeta}$, set
$g_k(\alpha)=\sup(\alpha\cap W\cap\delta)<\alpha$;
choose $f_k\in\prod_{\alpha\in D_k}\alpha$ such that $g_k\le f_k$ and
$f_k\restr A'_{k\zeta}\in F(A'_{k\zeta})$ for every $\zeta$;  this is
possible by (ii) above.   Set

\Centerline{$A_{k+1,\zeta}
=A_{k\zeta}\cup f_k[A'_{k\zeta}]\cup\{\psi_{\alpha}(\xi):
\alpha\in A_{k\zeta}\cap G_1,\,\xi<\min(\zeta,\cf\alpha)\}
\in\Cal A$}

\noindent for each $\zeta<\omega_1$, and continue.
An easy induction on $k$ shows that
$\langle A_{k\zeta}\rangle_{\zeta<\omega_1}$ is non-decreasing for every
$k$;  also, every $A_{k\zeta}$ is a subset of $\delta$.

\medskip

\quad{\bf (vi)} Set $V_k=\bigcup_{\zeta<\omega_1}A_{k\zeta}$,
$b_k=\bigcup\{a_{\xi}:\xi\in W\cap V_k\}$;  then $b_k$ is countable and
there is a $\beta(k)<\omega_1$ such that
$b_k=\bigcup\{a_{\xi}:\xi\in W\cap A_{k,\beta(k)}\}$.   Now
$\bigcup_{k\in\Bbb N}A_{k,\beta(k)}$ is a
countable subset of $\delta$, so there is a member $\gamma$ of
$W\cap\delta$ greater than its supremum.   We have

\Centerline{$a_{\gamma}
\not\subseteq\bigcup\{a_{\eta}:\eta\in A_{k,\beta(k)}\}$,}

\noindent so $a_{\gamma}\not\subseteq b_k$ and
$\gamma\notin V_k$, for each $k$.

Set $V=\bigcup_{k\in\Bbb N}V_k$.   We have just seen that
$W\cap\delta\not\subseteq V$;  set
$\gamma_0=\min(W\cap\delta\setminus V)$.   Because
$V_0=\psi_{\delta}[\omega_1]$ is cofinal with $\delta$,
$V\setminus\gamma_0\ne\emptyset$;  let $\gamma_1$ be its least member.
Then $\gamma_1>\gamma_0$.   There must be $k\in\Bbb N$ and
$\zeta<\omega_1$ such that $\gamma_1\in A_{k\zeta}$.   Observe
that if $\alpha\in V\cap G_1$ then $V\cap\alpha$ is cofinal with
$\alpha$;  but $V\cap\gamma_1\subseteq\gamma_0$, so $\gamma_1\notin G_1$
and $\gamma_1\in A'_{k\zeta}\subseteq D_k$.   But now
$f_k(\gamma_1)\in A_{k+1,\zeta}\subseteq V$ and
$\gamma_0\le g_k(\gamma_1)\le f_k(\gamma_1)<\gamma_1$, so
$\gamma_1\ne\min(V\setminus\gamma_0)$.\ \Bang

\medskip

\quad{\bf (vii)} This contradiction shows that
$\cff[\kappa]^{\le\omega}\le\kappa$.   Now consider
$\cff[\kappa]^{\le\delta}$, where $\delta<\kappa$ is an infinite
cardinal.   Then

$$\eqalignno{\covSh(\kappa,\delta^+,\delta^+,\omega_1)
&=\max(\kappa,
  \sup_{\lambda<\kappa}\covSh(\lambda,\delta^+,\delta^+,\omega_1))\cr
\displaycause{5A2Eb}
&\le\kappa\cr}$$

\noindent by 542Dc.   So there
is a family $\Cal B\subseteq [\kappa]^{\le\delta}$, with cardinal at most
$\kappa$,  such that every member of $[\kappa]^{\le\delta}$ is covered
by a sequence in $\Cal B$.   But now
$\cff[\Cal B]^{\le\omega}\le\kappa$, so there is a family $\frak C$ of
countable subsets of $\Cal B$ which is cofinal with
$[\Cal B]^{\le\omega}$ and with cardinal at most $\kappa$;  setting
$\Cal D=\{\bigcup\Cal C:\Cal C\in\frak C\}$, we have $\Cal D$ cofinal
with $[\kappa]^{\le\delta}$ and with cardinal at most $\kappa$.   So
$\cff[\kappa]^{\le\delta}\le\kappa$.

Finally, of course,
$[\kappa]^{<\theta}=\bigcup_{\delta<\theta}[\kappa]^{\le\delta}$, so

\Centerline{$\cff[\kappa]^{<\theta}
\le\sup_{\delta<\theta}\cff[\kappa]^{\le\delta}\le\kappa$}

\noindent whenever $\theta\le\kappa$.    For
$\theta>\kappa$ we have $\cff[\kappa]^{<\theta}=1$, so
$\cff[\kappa]^{<\theta}\le\kappa$ for every $\theta$.

\medskip

{\bf (b)} If $\Cal A$ is cofinal with $[\kappa]^{<\theta}$ then
$\{A\cap\lambda:A\in\Cal A\}$ is cofinal with $[\lambda]^{<\theta}$, so
$\cff[\lambda]^{<\theta}\le\cff[\kappa]^{\theta}
\le\kappa$, by (a).

\Quer\ If $\cff[\lambda]^{<\theta}=\kappa$, there is a cofinal family
$\ofamily{\xi}{\kappa}{A_{\xi}}$ in $[\lambda]^{<\theta}$ such that
$A_{\xi}\not\subseteq A_{\eta}$ for any $\eta<\xi<\kappa$.   Of course
$\omega<\theta\le\lambda<\kappa=\cf\kappa$, so we may suppose that every
$A_{\xi}$ is infinite.   So there is an infinite $\delta<\theta$ such
that $E=\{\xi:\xi<\kappa$, $\#(A_{\xi})=\delta\}$ has cardinal $\kappa$.
Next, by 541D applied to a suitable ideal of subsets of $E$, there is a
set $M\in[\lambda]^{\le\delta}$ such that
$F=\{\xi:\xi\in E$, $A_{\xi}\subseteq M\}$ has cardinal $\kappa$.   But
now there must be an $\eta<\kappa$ such that $M\subseteq A_{\eta}$, and
a $\xi\in F$ such that $\xi>\eta$;  which is impossible.\ \Bang

Thus $\cff[\lambda]^{<\theta}<\kappa$, as claimed.
}%end of proof of 542I

\leader{542J}{Corollary} %RVMC S7C
Let $\kappa$ be a \qm\ cardinal.   Let
$\langle P_{\zeta}\rangle_{\zeta<\lambda}$ be a family of partially
ordered sets
such that $\lambda<\add P_{\zeta}\le\cf P_{\zeta}<\kappa$ for every
$\zeta<\lambda$.   Then $\cf(\prod_{\zeta<\lambda}P_{\zeta})<\kappa$.

\proof{ For each $\zeta<\lambda$ let $Q_{\zeta}$ be a
cofinal subset of $P_{\zeta}$ with cardinal less than $\kappa$.   Set
$P=\prod_{\zeta<\kappa}P_{\zeta}$, $Z=\bigcup_{\zeta<\lambda}Q_{\zeta}$;
then $\#(Z)<\kappa$ so $\cff[Z]^{\le\lambda}<\kappa$, by 542Ib.   Let
$\Cal A$ be a cofinal subset of $[Z]^{\le\lambda}$ with
$\#(\Cal A)<\kappa$.   For each $A\in\Cal A$ choose $f_A\in P$ such that
$f_A(\zeta)$ is an upper bound for $A\cap P_{\zeta}$ for every $\zeta$;
this is possible because $\add P_{\zeta}>\#(A)$.   Set
$F=\{f_A:A\in\Cal A\}$.

If $g\in P$, then there is an $h\in\prod_{\zeta<\lambda}Q_{\zeta}$ such
that $g\le h$.   Now $h[\lambda]\in[Z]^{\le\lambda}$ so there is an
$A\in\Cal A$ such that $h[\lambda]\subseteq A$.   In this case
$h\le f_A$.   Accordingly $F$ is cofinal with $P$ and
$\cf P\le\#(F)<\kappa$, as required.
}%end of proof of 542J


\exercises{\leader{542X}{Basic exercises (a)}
%\spheader 542Xa
Let $\kappa$ be a quasi-measurable cardinal, and $\theta$ a cardinal
such that $2\le\theta\le\kappa$.   Show that
$\cff[\kappa]^{<\theta}=\kappa$.
%542I

\leader{542Y}{Further exercises (a)}
%\spheader 542Ya
Let $X$ be a \hwtr\ topological space such that there is no
quasi-measurable cardinal less than or equal to the weight of $X$, and
$\mu$ a totally finite Maharam submeasure on the Borel $\sigma$-algebra of
$X$.   (i) Show that $\mu$ is $\tau$-subadditive in the sense that if
whenever $\Cal G$ is a non-empty
upwards-directed family of open sets in $X$ with union $H$, then
$\inf_{G\in\Cal G}\mu(H\setminus G)=0$.
(ii) Show that if $X$ is Hausdorff and K-analytic, then the completion of
$\mu$ is a Radon submeasure on $X$.
%4{}38M 4{}96J

}%end of exercises

\endnotes{
\Notesheader{542} The arguments of this section have taken on a certain
density, and I ought to explain what they are for.   The cardinal
arithmetic of 542E-542G is relevant to one of the most important
questions in this chapter, to be treated in the next section:  supposing
that there is an extension of Lebesgue measure to a measure $\mu$
defined on all subsets of $\Bbb R$, what can we say about the Maharam
type of $\mu$?   And 542I-542J will tell us something about the
cofinalities of our favourite partially ordered sets under the same
circumstances.

Let me draw your attention to a useful trick, used twice above.   If
$\kappa$ is a quasi-measurable cardinal, and $X$ is any set with cardinal
at least $\kappa$, there is a non-trivial
$\omega_1$-saturated $\sigma$-ideal of subsets of $X$.   This is the
basis of the proof of 542H (taking $X=F_{\zeta+1}$ in the inductive
step) and the final step in the proof of 542I (taking $X=E$).   Exposed
like this, the idea seems obvious.   In the thickets of an argument it
sometimes demands an imaginative jump.

}%end of notes

\discrpage



