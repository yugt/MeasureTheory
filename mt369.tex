\frfilename{mt369.tex}
\versiondate{5.11.03}
\copyrightdate{1996}
     
\def\chaptername{Function spaces}
\def\sectionname{Banach function spaces}
     
\newsection{369}
     
In this section I continue the work of \S368 with results which involve
measure algebras.   The first step is a modification of the basic
representation theorem for Archimedean Riesz spaces.   If $U$ is any
Archimedean Riesz space, it can be represented as a subspace of
$L^0=L^0(\frak A)$, where $\frak A$ is its band algebra (368E);
now if $U^{\times}$ separates the points of $U$, there is a measure
rendering $\frak A$ a localizable measure algebra
(369A\cmmnt{, 369Xa}).
Moreover, we get a simultaneous representation of $U^{\times}$ as a
subspace of $L^0$ (369C-369D), the duality between $U$ and $U^{\times}$ corresponding exactly to the familiar duality between $L^p$ and $L^q$.
In particular, every $L$-space can be represented as an $L^1$-space (369E).
     
Still drawing inspiration from the classical $L^p$ spaces, we have a
general theory of `associated Fatou norms' (369F-369M, 369R).   I
include notes on the spaces $M^{1,\infty}$, $M^{\infty,1}$ and $M^{1,0}$
(369N-369Q), which will be particularly useful in the next chapter.
     
\leader{369A}{Theorem} Let $U$ be a Riesz space such that $U^{\times}$
separates the points of $U$.   Then $U$ can be embedded as an
order-dense Riesz subspace of $\cmmnt{L^0=}L^0(\frak A)$ for some 
localizable measure algebra $(\frak A,\bar\mu)$.
     
\proof{{\bf (a)} Consider the canonical map $S:U\to U^{\times\times}$.
We know that this is a Riesz homomorphism onto an order-dense Riesz
subspace of $U^{\times\times}$ (356I).   Because $U^{\times}$
separates the points of $U$, $S$ is injective.   Let $\frak A$ be the
band algebra of $U^{\times\times}$ and 
$T:U^{\times\times}\to L^0$ an injective Riesz homomorphism onto an 
order-dense Riesz subspace $V$ of $L^0$, as in 368E.   The composition 
$TS:U\to L^0$ is now an injective Riesz homomorphism, so embeds $U$ as a Riesz subspace of $L^0$, which is order-dense because $V$ is order-dense in $L^0$ and $TS[U]$ is order-dense in $V$ (352Nc).
Thus all that we need to find is a measure $\bar\mu$ on $\frak A$
rendering it a localizable measure algebra.
     
\medskip
     
{\bf (b)}
Note that $V$ is isomorphic, as Riesz space, to $U^{\times\times}$,
which is Dedekind complete (356B), so $V$ must be solid in $L^0$
(353K).   Also $V^{\times}$ must separate the points of $V$
(356L).
     
Let $D$ be the set of those $d\in\frak A$ such that the principal ideal
$\frak A_d$ is measurable in the sense that there is some $\bar\nu$ for
which $(\frak A_d,\bar\nu)$ is a totally finite measure algebra.   Then
$D$
is order-dense in $\frak A$.   \Prf\ Take any non-zero $a\in\frak A$.
Because $V$ is order-dense, there is a non-zero $v\in V$ such that
$v\le\chi a$.   Take $h\ge 0$ in $V^{\times}$ such that $h(v)>0$.   Then
there is a $v'$ such that $0<v'\le v$ and $h(w)>0$ whenever $0<w\le v'$
in $V$ (356H).
Let $\alpha>0$ be such that $d=\Bvalue{v'>\alpha}\ne 0$.   Then $\chi
b\le\bover1{\alpha}v'\in V$ whenever $b\in\frak A_d$.   Set $\bar\nu
b=h(\chi b)\in\coint{0,\infty}$ for $b\in\frak A_d$.   Because the map
$b\mapsto\chi b:\frak A\to L^0$ is additive and order-continuous, the
map $b\mapsto\chi b:\frak A_d\to V$ also is, and $\bar\nu=h\chi$ must be
additive and order-continuous;  in particular,
$\bar\nu(\sup_{n\in\Bbb N}b_n)=\sum_{n=0}^{\infty}\bar\nu b_n$ whenever
$\sequencen{b_n}$ is a
disjoint sequence in $\frak A_d$.   Moreover, if $b\in\frak A_d$ is
non-zero, then $0<\alpha\chi b\le v'$, so $\bar\nu b=h(\chi b)>0$.
Thus $(\frak A_d,\bar\nu)$ is a totally finite measure algebra, and
$d\in D$, while $0\ne d\Bsubseteq a$.   As $a$ is arbitrary, $D$ is
order-dense.\ \Qed
     
\medskip
     
{\bf (c)} By 313K, there is a partition of unity $C\subseteq D$.   For
each $c\in C$, let $\bar\nu_c:\frak A_c\to\coint{0,\infty}$ be a
functional such that $(\frak A_c,\bar\nu_c)$ is a totally finite measure
algebra.   Define $\bar\mu:\frak A\to[0,\infty]$ by setting 
$\bar\mu a=\sum_{c\in C}\bar\nu_c(a\Bcap c)$ for every $a\in\frak A$.   Then $(\frak A,\bar\mu)$ is a localizable measure algebra.   \Prf\ (i)
$\bar\mu 0=\sum_{c\in C}\bar\nu 0=0$.   (ii) If $\sequencen{a_n}$ is a
disjoint sequence in $\frak A$ with supremum $a$, then
     
\Centerline{$\bar\mu a
=\sum_{c\in C}\bar\nu_c(a\Bcap c)
=\sum_{c\in C,n\in\Bbb N}\bar\nu_c(a_n\Bcap c)
=\sum_{n=0}^{\infty}\bar\mu a_n$.}
     
\noindent (iii) If $a\in\frak A\setminus\{0\}$, then there is a $c\in C$
such that $a\Bcap c\ne 0$, so that $\bar\mu a\ge\bar\nu_c(a\Bcap c)>0$.
Thus $(\frak A,\bar\mu)$ is a measure algebra.   (iv) Moreover, in
(iii), $\bar\mu(a\Bcap c)=\bar\nu_c(a\Bcap c)$ is finite.   So $(\frak
A,\bar\mu)$ is semi-finite.   (v) $\frak A$ is Dedekind complete, being
a band algebra (352Q), so $(\frak A,\bar\mu)$ is localizable.\ \Qed
}%end of proof of 369A
     
\leader{369B}{Corollary} Let $U$ be a Banach lattice with
order-continuous norm.   Then $U$ can be embedded as an order-dense
solid linear subspace of $L^0(\frak A)$ for some localizable measure
algebra $(\frak A,\bar\mu)$.
     
\proof{ By 356Dd, $U^{\times}=U^*$, which separates the points of
$U$, by the Hahn-Banach theorem (3A5Ae).   So 369A tells us that $U$ can
be embedded as an order-dense Riesz subspace of an appropriate
$L^0(\frak A)$.    But also $U$ is Dedekind complete (354Ee), so its
copy in $L^0(\frak A)$ must be solid, as in 368H.
}%end of proof of 369B
     
\leader{369C}{}\cmmnt{ The representation in 369A is complemented by
the following result, which is a kind of generalization of 365M and
366Dc.
     
\medskip
     
\noindent}{\bf Theorem} Let $(\frak A,\bar\mu)$ be a semi-finite measure
algebra, and $U\subseteq L^0=L^0(\frak A)$ an order-dense Riesz
subspace.   Set
     
\Centerline{$V=\{v:v\in L^0,\,v\times u\in L^1$ for every $u\in U\}$,}
     
\noindent writing $L^1$ for $L^1(\frak A,\bar\mu)\subseteq L^0$.   Then
$V$ is a solid linear subspace of $L^0$, and we have an order-continuous
injective Riesz homomorphism $T:V\to U^{\times}$ defined by setting
     
\Centerline{$(Tv)(u)=\int u\times v$ for all 
$u\in U$, $v\in V$.}
     
\noindent The image of $V$ is order-dense in $U^{\times}$.   If 
$(\frak A,\bar\mu)$ is localizable, then $T$ is surjective, so is a Riesz space isomorphism between $V$ and $U^{\times}$.
     
\proof{{\bf (a)(i)} Because $\times:L^0\times L^0\to L^0$ is bilinear
and $L^1$ is a linear subspace of $L^0$, $V$ is a linear subspace of
$L^0$.   If $u\in U$, $v\in V$, $w\in L^0$ and $|w|\le|v|$, then
     
\Centerline{$|w\times u|=|w|\times|u|\le|v|\times|u|=|v\times u|\in
L^1$;}
     
\noindent  as $L^1$ is solid, $w\times u\in L^1$;  as $u$ is arbitrary,
$w\in V$;  this shows that $V$ is solid.
     
\medskip
     
\quad{\bf (ii)} By the definition of $V$, $(Tv)(u)$ is defined in $\Bbb
R$ for all $u\in U$, $v\in V$.   Because $\times$ is bilinear and $\int$
is linear, $Tv:U\to\Bbb R$ is linear for every $v\in V$, and $T$ is a
linear functional from $V$ to the space of linear operators from $U$ to
$\Bbb R$.
     
\medskip
     
\quad{\bf (iii)}
If $u\ge 0$ in $U$ and $v\ge 0$ in $V$, then $u\times v\ge 0$ in $L^1$
and $(Tv)(u)=\int u\times v\ge 0$.   This shows that $T$ is a positive
linear operator from $V$ to $U^{\sim}$.
     
\medskip
     
\quad{\bf (iv)}
If $v\ge 0$ in $V$ and $A\subseteq U$ is a non-empty downwards-directed
set with infimum $0$ in $U$, then $\inf A=0$ in $L^0$, because $U$ is
order-dense (352Nb).   Consequently $\inf_{u\in A}u\times v=0$ in
$L^0$ and in $L^1$ (364Ba, 353Oa), and
     
\Centerline{$\inf_{u\in A}(Tv)(u)=\inf_{u\in A}\int u\times v=0$}
     
\noindent (because $\int$ is order-continuous).   As $A$ is arbitrary,
$Tv$ is order-continuous.   As $v$ is arbitrary,
$T[V]\subseteq U^{\times}$.
     
\medskip
     
\quad{\bf (v)}
If $v\in V$ and $u_0\ge 0$ in $U$, set $a=\Bvalue{v>0}$.   Then
$v^+=v\times\chi a$.   Set $A=\{u:u\in U,\,0\le u\le u_0\times\chi a\}$.
Because $U$ is order-dense in $L^0$, $u_0\times\chi a=\sup A$ in $L^0$.
Because $\times$ and $\int$ are order-continuous,
     
$$\eqalign{(Tv)^+(u_0)
&\ge\sup_{u\in A}(Tv)(u)
=\sup_{u\in A}\int v\times u\cr
&=\int v\times u_0\times\chi a
=\int v^+\times u_0
=(Tv^+)(u_0).\cr}$$
     
\noindent As $u_0$ is arbitrary, $(Tv)^+\ge Tv^+$.   But because $T$ is
a positive linear operator, we must have $Tv^+\ge(Tv)^+$, so that
$Tv^+=(Tv)^+$.   As $v$ is arbitrary, $T$ is a Riesz homomorphism.
     
\medskip
     
\quad{\bf (vi)} Now $T$ is injective.   \Prf\ If $v\ne 0$ in $V$, there
is a $u>0$ in $U$ such that $u\le|v|$, because $U$ is order-dense.   In
this case $u\times|v|>0$ so $\int u\times|v|>0$.   Accordingly
$|Tv|=T|v|\ne 0$ and $Tv\ne 0$.\ \Qed
     
\medskip
     
{\bf (b)} Putting (a-i) to (a-vi) together, we see that $T$ is an
injective Riesz homomorphism from $V$ to $U^{\times}$.   All this is
easy.    The point of the theorem is the fact that $T[V]$ is order-dense
in $U^{\times}$.
     
\Prf\  Take $h>0$ in $U^{\times}$.
Let $U_1$ be the solid linear subspace of $L^0$ generated by $U$.   Then
$U$ is an order-dense Riesz subspace of $U_1$, $h:U\to\Bbb R$ is an
order-continuous positive linear functional, and 
$\sup\{h(u):u\in U,\,0\le u\le v\}$ is defined in $\Bbb R$ for every 
$v\ge 0$ in $U_1$;  so we have an 
extension $\tilde h$ of $h$ to $U_1$ such that $\tilde h\in U_1^{\times}$ 
(355F).
     
Set $S_1=S(\frak A)\cap U_1$;  then $S_1$ is an order-dense Riesz
subspace of $U_1$, because $S(\frak A)$ is order-dense in $L^0$ and
$U_1$ is solid in $L^0$.   Note
that $S_1$ is the linear span of $\{\chi c:c\in I\}$, where
$I=\{c:c\in\frak A,\,\chi c\in U_1\}$, and that $I$ is an ideal in
$\frak A$.
     
Because $h\ne 0$, $\tilde h\ne 0$;  there must therefore be a 
$u_0\in S_1$ such that $\tilde h(u_0)>0$, and a $d\in I$ such that $\tilde h(\chi d)>0$.   For $a\in\frak A$, set 
$\nu a=\tilde h\chi(d\Bcap a)$.
Because $\Bcap$, $\chi$ and $\tilde h$ are all order-continuous, so is
$\nu$, and $\nu:\frak A\to\Bbb R$ is a non-negative completely additive
functional.
     
By 365Ea, there is a $v\in L^1$ such that
     
\Centerline{$\int_av=\nu a=\tilde h\chi(d\Bcap a)$}
     
\noindent for every $a\in\frak A$;  of course $v\ge 0$.   We have
$\int u\times v\le \tilde h(u)$ whenever $u=\chi a$ for $a\in I$, and
therefore for every $u\in S_1^+$.   If $u\in U^+$, then
$A=\{u':u'\in S_1,\,0\le u'\le u\}$ is upwards-directed, $\sup A=u$ and
     
\Centerline{$\sup_{u'\in A}\int v\times u'\le\sup_{u'\in A}\tilde h(u')
=\tilde h(u)=h(u)$}
     
\noindent is finite, so $v\times u=\sup_{u\in A'}v\times u'$ belongs to
$L^1$ (365Df) and $\int v\times u\le h(u)$.   As $u$
is arbitrary, $v\in V$ and $Tv\le h$.   At the same time,
$\int_dv=\tilde h(\chi d)>0$, so $Tv>0$.   As $h$ is arbitrary, $T[V]$
is order-dense.\ \Qed
     
It follows that $T$ is order-continuous (352Nb), as can also be easily
proved by the argument of (a-iv) above.
     
\medskip
     
{\bf (c)} Now suppose that $(\frak A,\bar\mu)$ is localizable, that is,
that $\frak A$ is Dedekind complete.   $T^{-1}:T[V]\to V$ is a Riesz
space isomorphism, so
certainly an order-continuous Riesz homomorphism;  because $V$ is a
solid linear subspace of $L^0$, $T^{-1}$ is still an injective
order-continuous Riesz homomorphism when regarded as a map from $T[V]$
to $L^0$.   Since $T[V]$ is order-dense in $U^{\times}$, $T^{-1}$ has an
extension to an order-continuous Riesz homomorphism
$Q:U^{\times}\to L^0$ (368B).
But $Q[U^{\times}]\subseteq V$.   \Prf\ Take $h\ge 0$ in $U^{\times}$
and $u\ge 0$ in $U$.   Then $B=\{g:g\in T[V],\,0\le g\le h\}$ is
upwards-directed and has supremum $h$.   For $g\in B$, we know that
$u\times T^{-1}g\in L^1$ and $\int u\times T^{-1}g=g(u)$, by the
definition of $T$.   But this means that
     
\Centerline{$\sup_{g\in B}\int u\times T^{-1}g
=\sup_{g\in B}g(u)=h(u)<\infty$.}
     
\noindent Since $\{u\times T^{-1}g:g\in B\}$ is upwards-directed, it
follows that
     
\Centerline{$u\times Qh=\sup_{g\in B}u\times Qg
=\sup_{g\in B}u\times T^{-1}g\in L^1$}
     
\noindent by 365Df again.  As $u$ is arbitrary, $Qh\in V$.   As $h$
is arbitrary (and $Q$ is linear), $Q[U^{\times}]\subseteq V$.\ \Qed
     
Also $Q$ is injective.   \Prf\ If $h\in U^{\times}$ is non-zero, there
is a
$v\in V$ such that $0<Tv\le|h|$, so that
     
\Centerline{$|Qh|=Q|h|\ge QTv=v>0$}
     
\noindent and $Qh\ne 0$.\ \QeD\   Since $QT$ is the identity on $V$, $Q$
and $T$ must be the two halves of a Riesz space isomorphism between $V$
and $U^{\times}$.
}%end of proof of 369C
     
\leader{369D}{Corollary} Let $U$ be any Riesz space such that
$U^{\times}$ separates the points of $U$.   Then there is a localizable
measure algebra $(\frak A,\bar\mu)$ such that the pair $(U,U^{\times})$
can be represented by a pair $(V,W)$ of order-dense Riesz subspaces of
$L^0=L^0(\frak A)$ such that 
$W=\{w:w\in L^0,\,v\times w\in L^1$ for every $v\in V\}$, writing $L^1$ for $L^1(\frak A,\bar\mu)$.
In this case, $U^{\times\times}$ becomes represented by $\tilde
V=\{v:v\in L^0,\,v\times w\in L^1$ for every $w\in W\}\supseteq V$.
     
\proof{ Put 369A and 369C together.   The construction of 369A finds
$(\frak A,\bar\mu)$ and an order-dense $V$ which is isomorphic to $U$,
and 369C identifies $W$ with $V^{\times}$ and $W^{\times}$ with 
$\tilde V$.   To check that $W$ is order-dense, take any $u>0$ in $L^0$.    There is a $v\in V$ such that
$0<v\le u$.   There is an $h\in(V^{\times})^+$ such that $h(v)>0$, so
there is a $w\in W^+$ such that $w\times v\ne 0$, that is, 
$w\wedge v\ne 0$.
But now $w\wedge v\in W$, because $W$ is solid, and $0<w\wedge v\le u$.
}%end of proof of 369D
     
\cmmnt{\medskip
     
\noindent{\bf Remark} Thus the canonical embedding of $U$ in
$U^{\times\times}$ (356I) is represented by the embedding
$V\embedsinto\tilde V$;  $U$, or $V$, is `perfect' iff $V=\tilde V$.
}%end of comment
     
\leader{369E}{Kakutani's theorem}\cmmnt{ ({\smc Kakutani 41})} If $U$
is any $L$-space, there is a
localizable measure algebra $(\frak A,\bar\mu)$ such that $U$ is
isomorphic, as Banach lattice, to $\cmmnt{L^1=}L^1(\frak A,\bar\mu)$.
     
\proof{ $U$ is a perfect Riesz space, and $U^{\times}=U^*$ has an order
unit $\int$ defined by saying that $\int u=\|u\|$ for $u\ge 0$ (356P).
By 369D, we can find a localizable measure algebra $(\frak A,\bar\mu)$
and an
identification of the pair $(U,U^{\times})$, as dual Riesz spaces, with
a pair $(V,W)$ of subspaces of $L^0=L^0(\frak A)$;  and $V$ will be
$\{v:v\times w\in L^1$ for every $w\in W\}$.   But $W$, like
$U^{\times}$, must have an order unit;  call it $e$.   Because $W$ is
order-dense, $\Bvalue{e>0}$ must be $1$ and $e$ must have a
multiplicative inverse
$\bover1e$ in $L^0$ (364N).   This means that $V$ must be 
$\{v:v\times e\in L^1\}$, so that $v\mapsto v\times e$ is a Riesz space isomorphism
between $V$ and $L^1$, which gives a Riesz space isomorphism between $U$
and $L^1$.
Moreover, if we write $\|\,\|'$ for the norm on $V$ corresponding to the
norm of $U$, we have
     
\Centerline{$\|u\|=\int|u|$ for $u\in U$,
\quad$\|v\|'=\int|v|\times e=\int|v\times e|$ for $v\in V$.}
     
\noindent Thus the Riesz space isomorphism between $U$ and $L^1$ is
norm-preserving, and $U$ and $L^1$ are isomorphic as Banach lattices.
}%end of proof of 369E
     
\leader{369F}{}\cmmnt{ The $L^p$ spaces are leading examples for a
general theory of normed subspaces of $L^0$, which I proceed to sketch
in the rest of the section.
     
\medskip
     
\noindent}{\bf Definition} Let $\frak A$ be a Dedekind $\sigma$-complete
Boolean algebra.   An {\bf extended Fatou norm} on $L^0=L^0(\frak A)$ is
a function $\tau:L^0\to[0,\infty]$ such that
     
(i) $\tau(u+v)\le\tau(u)+\tau(v)$ for all $u$, $v\in L^0$;
     
(ii) $\tau(\alpha u)=|\alpha|\tau(u)$ for all $u\in L^0$, 
$\alpha\in\Bbb R$\cmmnt{ (counting $0\cdot\infty$ as $0$, as usual)};
     
(iii) $\tau(u)\le\tau(v)$ whenever $|u|\le|v|$ in $L^0$;
     
(iv) $\sup_{u\in A}\tau(u)=\tau(v)$ whenever $A\subseteq (L^0)^+$ is a
non-empty upwards-directed set with supremum $v$ in $L^0$;
     
(v) $\tau(u)>0$ for every non-zero $u\in L^0$;
     
(vi) whenever $u>0$ in $L^0$ there is a $v\in L^0$ such that $0<v\le u$
and $\tau(v)<\infty$.
     
\leader{369G}{Proposition} Let $\frak A$ be a Dedekind
$\sigma$-complete Boolean algebra and $\tau$ an extended Fatou norm on
$L^0=L^0(\frak A)$.
Then $L^{\tau}=\{u:u\in L^0,\,\tau(u)<\infty\}$ is an order-dense solid
linear subspace of $L^0$, and $\tau$, restricted to $L^{\tau}$, is a
Fatou norm under which $L^{\tau}$ is a Banach lattice.   If
$\sequencen{u_n}$ is a non-decreasing norm-bounded sequence in
$(L^{\tau})^+$, then it has a supremum in $L^{\tau}$;  if
$\frak A$ is Dedekind complete, then $L^{\tau}$ has the Levi property.
     
\proof{{\bf (a)} By (i), (ii) and (iii) of 369F, $L^{\tau}$ is a solid
linear subspace of $L^0$;  by (vi), it is order-dense.   Hypotheses (i),
(ii), (iii) and (v) show that $\tau$ is a Riesz norm on $L^{\tau}$,
while (iv) shows that it is a Fatou norm.
     
\medskip
     
{\bf (b)(i)} Suppose that $\sequencen{u_n}$ is a non-decreasing
norm-bounded sequence in $(L^{\tau})^+$.   Then $u=\sup_{n\in\Bbb N}u_n$
is defined in $L^0$.   \Prf\Quer\ Otherwise, there is a $v>0$ in $L^0$
such that $kv=\sup_{n\in\Bbb N}kv\wedge u_n$ for every $k\in\Bbb N$
(368A).   By (v)-(vi) of 369F, there is a $v'$ such that $0<v'\le v$
and $0<\tau(v')<\infty$.   Now $kv'=\sup_{n\in\Bbb N}kv'\wedge u_n$ for
every $k$, so
     
\Centerline{$k\tau(v')=\tau(kv')=\sup_{n\in\Bbb N}\tau(kv'\wedge
u_n)\le\sup_{n\in\Bbb N}\tau(u_n)$}
     
\noindent for every $k$, using 369F(iv), and $\sup_{n\in\Bbb
N}\tau(u_n)=\infty$, contrary to hypothesis.\ \Bang\QeD\   By 369F(iv)
again, $\tau(u)=\sup_{n\in\Bbb N}\tau(u_n)<\infty$, so that 
$u\in L^{\tau}$ and $u=\sup_{n\in\Bbb N}u_n$ in $L^{\tau}$.
     
\medskip
     
\quad{\bf (ii)} It follows that $L^{\tau}$ is complete under $\tau$.
\Prf\ Let $\sequencen{u_n}$ be a sequence in $L^{\tau}$ such that
$\tau(u_{n+1}-u_n)\le 2^{-n}$ for every $n\in\Bbb N$.   Set
$v_{mn}=\sum_{i=m}^n|u_{i+1}-u_i|$ for $m\le n$;  then 
$\tau(v_{mn})\le 2^{-m+1}$ for every $n$, so by (i) just above $v_m=\sup_{n\in\Bbb N}v_{mn}$ is defined in $L^{\tau}$, and $\tau(v_m)\le 2^{-m+1}$.   Now
$v_m=|u_{m+1}-u_m|+v_{m+1}$ for each $m$, so $\sequence{m}{u_m-v_m}$ is
non-decreasing and $\sequence{m}{u_m+v_m}$ is non-increasing, while
$u_m-v_m\le u_m\le u_m+v_m$ for every $m$.   Accordingly
$u=\sup_{m\in\Bbb N}u_m-v_m$ is defined in $L^{\tau}$ and 
$|u-u_m|\le v_m$ for every $m$.   But this means that
$\lim_{m\to\infty}\tau(u-u_m)\le\lim_{m\to\infty}\tau(v_m)=0$ and
$u=\lim_{m\to\infty}u_m$ in $L^{\tau}$.   As $\sequencen{u_n}$ is
arbitrary, $L^{\tau}$ is complete.\ \Qed
     
\medskip
     
{\bf (c)} Now suppose that $\frak A$ is Dedekind complete and
$A\subseteq (L^{\tau})^+$ is a non-empty upwards-directed norm-bounded
set in $L^{\tau}$.   By the argument of (b-i) above, using the other
half of 368A, $\sup A$ is defined in $L^0$ and belongs to $L^{\tau}$.
As $A$ is arbitrary, $L^{\tau}$ has the Levi property.
}%end of proof of 369G
     
\leader{369H}{Associate norms:  Definition} Let $(\frak A,\bar\mu)$ be a
semi-finite measure algebra, and $\tau$ an extended Fatou norm on
$L^0=L^0(\frak A)$.   Define $\tau':L^0\to[0,\infty]$ by setting
     
\Centerline{$\tau'(u)=\sup\{\|u\times v\|_1:v\in L^0,\,\tau(v)\le 1\}$}
     
\noindent for every $u\in L^0$;  then $\tau'$ is the {\bf associate} of
$\tau$.   \cmmnt{(The word suggests a symmetric relationship;  it is
justified by the next theorem.)}
     
\leader{369I}{Theorem} Let $(\frak A,\bar\mu)$ be a semi-finite measure
algebra, and $\tau$ an extended Fatou norm on $L^0=L^0(\frak A)$.   Then
     
(i) its associate $\tau'$ is also an extended Fatou norm on $L^0$;
     
(ii) $\tau$ is the associate of $\tau'$;
     
(iii) $\|u\times v\|_1\le\tau(u)\tau'(v)$ for all $u$, $v\in L^0$.
     
\proof{{\bf (a)} Before embarking on the proof that $\tau'$ is an
extended Fatou seminorm on $L^0$, I give the greater part of the
argument needed to show that $\tau=\tau''$, where
     
\Centerline{$\tau''(u)=\sup\{\|u\times w\|_1:w\in L^0,\,
  \tau'(w)\le 1\}$}
     
\noindent for every $u\in L^0$.
          
\medskip
     
\quad\grheada\ Set
     
\Centerline{$B=\{u:u\in L^1,\,\tau(u)\le 1\}$,}
     
\noindent writing $L^1$ for $L^1(\frak A,\bar\mu)$.   Then $B$ is a convex set in $L^1$ and is
closed for the norm topology of $L^1$.   \Prf\ Suppose that $u$ belongs
to the closure of $B$ in $L^1$.   Then for each $n\in\Bbb N$ we can
choose $u_n\in B$ such that $\|u-u_n\|_1\le 2^{-n}$.   Set
$v_{mn}=\inf_{m\le i\le n}|u_i|$ for $m\le n$, and
     
\Centerline{$v_m=\inf_{n\ge m}v_{mn}=\inf_{n\ge m}|u_n|\le|u|$}
     
\noindent for $m\in\Bbb N$.  The sequence $\sequence{m}{v_m}$ is
non-decreasing, $\tau(v_m)\le\tau(u_m)\le 1$ for every $m$, and
     
\Centerline{$\||u|-v_m\|_1\le\sup_{n\ge m}\||u|-v_{mn}\|_1
\le\sum_{i=m}^{\infty}\||u|-|u_i|\|_1
\le\sum_{i=m}^{\infty}\|u-u_i\|_1\to 0$}
     
\noindent as $m\to\infty$.   So $|u|=\sup_{m\in\Bbb N}v_m$ in $L^0$,
     
\Centerline{$\tau(u)=\tau(|u|)=\sup_{m\in\Bbb N}\tau(v_m)\le 1$}
     
\noindent and $u\in B$.\ \Qed
     
\medskip
     
\quad\grheadb\ Now take any $u_0\in L^0$ such that $\tau(u_0)>1$.
Then, writing $\frak A^f$ for $\{a:\bar\mu a<\infty\}$,
     
\Centerline{$A=\{u:u\in S(\frak A^f),\,0\le u\le u_0\}$}
\noindent is an upwards-directed set with supremum $u_0$ (this is where
I use the hypothesis that $(\frak A,\bar\mu)$ is semi-finite, so that
$S(\frak A^f)$ is order-dense in $L^0$), and $\sup_{u\in
A}\tau(u)=\tau(u_0)>1$.   Take $u_1\in A$ such that $\tau(u_1)>1$, that
is, $u_1\notin B$.   By the Hahn-Banach theorem (3A5Cc), there is a
continuous linear functional $f:L^1\to\Bbb R$ such that $f(u_1)>1$ but
$f(u)\le 1$ for every $u\in B$.   Because $(L^1)^*=(L^1)^{\sim}$
(356Dc), $|f|$ is defined in $(L^1)^*$, and of course
     
\Centerline{$|f|(u_1)\ge f(u_1)>1$,
\quad$|f|(u)=\sup\{f(v):|v|\le u\}\le 1$}
     
\noindent whenever $u\in B$ and $u\ge 0$.   Set $c=\Bvalue{u_1>0}$, so
that $\bar\mu c<\infty$, and define
     
\Centerline{$\nu a=|f|(\chi(a\Bcap c))$}
     
\noindent for every $a\in\frak A$.   Then $\nu$ is a completely additive
real-valued functional on $\frak A$, so there is a $w\in L^1$ such that
$\nu a=\int_aw$ for every $a\in\frak A$ (365Ea).  Because 
$\nu a\ge 0$ for every $a$, $w\ge 0$.   Now
     
\Centerline{$\int_aw=|f|(\chi a\times\chi c)$}
     
\noindent for every $a\in\frak A$, so
     
\Centerline{$\int w\times u=|f|(u\times\chi c)\le|f|(u)\le 1$}
     
\noindent for every $u\in S(\frak A)^+\cap B$.   But if $\tau(v)\le 1$,
then
     
\Centerline{$A_v=\{u:u\in S(\frak A)^+\cap B,\,u\le|v|\}$}
     
\noindent is an upwards-directed set with supremum $|v|$, so that
     
\Centerline{$\|w\times v\|_1=\sup_{u\in A_v}\int w\times u\le 1$.}
     
\noindent Thus $\tau'(w)\le 1$.   On the other hand,
     
\Centerline{$\|w\times u_0\|_1\ge\int w\times u_0
\ge\int w\times u_1=|f|(u_1)>1$,}
     
\noindent so $\tau''(u_0)>1$.
     
\medskip
     
\quad\grheadc\ This shows that, for $u\in L^0$,
     
\Centerline{$\tau''(u)\le 1\Longrightarrow\tau(u)\le 1$.}
     
\medskip
     
{\bf (c)} Now I return to the proof that $\tau'$ is an extended Fatou
norm.   It is easy to check that it satisfies conditions (i)-(iv) of
369F;  in effect, these depend only on the fact that $\|\,\|_1$ is an
extended Fatou norm.   For (v)-(vi), take $v>0$ in $L^0$.   Then there
is a $u$ such that $0\le u\le v$ and $0<\tau(u)<\infty$;   set
$\alpha=1/\tau(u)$.   Then $\tau(2\alpha u)>1$, so that 
$\tau''(2\alpha u)>1$ and there is a $w\in L^0$ such that 
$\tau'(w)\le 1$, $\|2\alpha u\times w\|_1>1$.   But now set $v_1=v\wedge|w|$;  then
     
\Centerline{$v\ge v_1\ge u\wedge|w|>0$,}
     
\noindent while $\tau'(v_1)<\infty$.   Also $v\wedge\alpha u\ne 0$ so
     
\Centerline{$\tau'(v)\ge\|v\times\alpha u\|_1>0$.}
     
\noindent As $v$ is arbitrary, $\tau'$ satisfies 369F(v)-(vi).
     
\medskip
     
{\bf (d)} Accordingly $\tau''$ also is an extended Fatou norm.   Now in
(a) I showed that
     
\Centerline{$\tau''(u)\le 1\Longrightarrow\tau(u)\le 1$.}
     
\noindent It follows easily that $\tau(u)\le\tau''(u)$ for every $u$
(since otherwise there would be some $\alpha$ such that
     
\Centerline{$\tau''(\alpha
u)=\alpha\tau''(u)<1<\alpha\tau(u)=\tau(\alpha u)$).}
     
\noindent On the other hand, we surely have
     
\Centerline{$\tau(u)\le 1
\Longrightarrow\|u\times v\|_1\le 1\text{ whenever }\tau'(v)\le 1
\Longrightarrow\tau''(u)\le 1$,}
     
\noindent so we must also have $\tau''(u)\le\tau(u)$ for every $u$.
Thus $\tau''=\tau$, as claimed.
     
\medskip
     
{\bf (e)} Of course we have $\|u\times v\|_1\le 1$ whenever 
$\tau(u)\le 1$ and $\tau'(v)\le 1$.   It follows easily that 
$\|u\times v\|_1\le\tau(u)\tau'(v)$ whenever $u$, $v\in L^0$ and both $\tau(u)$, $\tau'(v)$ are non-zero.   But if one of them is zero, then $u\times v=0$,
because both $\tau$ and $\tau'$ satisfy (v) of 369F, so the result is
trivial.
}%end of proof of 369I
     
\leader{369J}{Theorem} Let $(\frak A,\bar\mu)$ be a semi-finite
measure algebra, and $\tau$ an extended Fatou norm on 
$L^0=L^0(\frak A)$, with associate $\theta$.   Then
     
\Centerline{$L^{\theta}=\{v:v\in L^0,\,
  u\times v\in L^1(\frak A,\bar\mu)$ for every $u\in L^{\tau}\}$.}
     
\proof{{\bf (a)} If $v\in L^{\theta}$ and $u\in L^{\tau}$, then
$\|u\times v\|_1$ is finite, by 369I(iii), so 
$u\times v\in L^1=L^1(\frak A,\bar\mu)$.
     
\medskip
     
{\bf (b)} If $v\notin L^{\theta}$ then for every $n\in\Bbb N$ there is a
$u_n$ such that $\tau(u_n)\le 1$ and $\|u_n\times v\|_1\ge 2^n$.   Set
$w_n=\sum_{i=0}^n2^{-i}|u_i|$ for each $n$.   Then $\sequencen{w_n}$ is
a non-decreasing sequence and $\tau(w_n)\le 2$ for each $n$, so
$w=\sup_{n\in\Bbb N}w_n$ is defined in $L^{\tau}$, by 369G;  now 
$\int w\times|v|\ge n+1$ for every $n$, so $w\times v\notin L^1$.
}%end of proof of 369J
     
     
\leader{369K}{Corollary} Let $(\frak A,\bar\mu)$ be a localizable
measure algebra, and $\tau$ an extended Fatou norm on 
$\cmmnt{L^0=}L^0(\frak A)$, with associate $\theta$.   Then $L^{\theta}$ may be identified, as normed Riesz space, with
$(L^{\tau})^{\times}\subseteq(L^{\tau})^*$, and $L^{\tau}$ is a perfect
Riesz space.
     
\proof{ Putting 369J and 369C together, we have an identification
between $L^{\theta}$ and $(L^{\tau})^{\times}$.   Now 369I tells us that
$\tau$ is the associate of $\theta$, so that we can identify $L^{\tau}$
with $(L^{\theta})^{\times}$, and $L^{\tau}$ is perfect, as in 369D.
     
By the definition of $\theta$, we have, for any $v\in L^{\theta}$,
     
$$\eqalign{\theta(v)
&=\sup_{\tau(u)\le 1}\|u\times v\|_1\cr
&=\sup_{\tau(u)\le 1,\|w\|_{\infty}\le 1}\int u\times v\times w
=\sup_{\tau(u)\le 1}\int u\times v,\cr}$$
     
\noindent which is the norm of the linear functional on $L^{\tau}$
corresponding to $v$.
}%end of proof of 369K
     
\leader{369L}{$\pmb{L^p}$}\cmmnt{ I remarked above that the $L^p$ spaces are
leading examples for this theory;  perhaps I should spell out the
details.}   Let $(\frak A,\bar\mu)$ be a semi-finite measure algebra and
$p\in[1,\infty]$.   Then $\|\,\|_p$ is an extended Fatou norm.
\prooflet{\Prf\ Conditions (i)-(iii) and (v) of 369F are true just because 
$L^p=L^p_{\bar\mu}$ is a solid linear subspace of $L^0(\frak A)$
on which $\|\,\|_p$ is a Riesz norm, (iv) is
true because $\|\,\|_p$ is a Fatou norm with the Levi property (363Ba,
365C, 366D), and (vi) is true because $S(\frak A^f)$ is included in
$L^p$ and order-dense in $L^0=L^0(\frak A)$ (364K).\ \Qed}
     
As usual, set $q=p/(p-1)$ if $1<p<\infty$, $\infty$ if $p=1$, and $1$ if
$p=\infty$.   Then $\|\,\|_q$ is the associate extended Fatou norm of
$\|\,\|_p$.   \prooflet{\Prf\ By 365Mb and 366C,
$\|v\|_q=\sup\{\|u\times v\|_1:\|u\|_p\le 1\}$ for every 
$v\in L^q=L^q_{\bar\mu}$.   But as $L^q$ is order-dense in $L^0$,
     
$$\eqalign{\|v\|_q
=\sup_{w\in L^q,|w|\le v}\|w\|_q
&=\sup\{\int|u|\times|w|:w\in L^q,\,w\le|v|,\,\|u\|_p\le 1\}\cr
&=\sup\{\int|u|\times|v|:\|u\|_p\le 1\}\cr}$$
     
\noindent for every $v\in L^0$.\ \Qed}
     
\vleader{108pt}{369M}{Proposition} Let $(\frak A,\bar\mu)$ be a semi-finite
measure algebra and $\tau$ an extended Fatou norm on 
$L^0=L^0(\frak A)$.   Then
     
(a) the embedding $L^{\tau}\subseteq L^0$ is continuous for the norm
topology of $L^{\tau}$ and the topology of convergence in measure on
$L^0$;
     
(b) $\tau:L^0\to[0,\infty]$ is lower semi-continuous\cmmnt{, that is, all the balls $\{u:\tau(u)\le\gamma\}$ are closed for the topology
of convergence in measure};
     
(c) if $\sequencen{u_n}$ is a sequence in $L^0$ which is
order*-convergent to $u\in L^0$\cmmnt{ (definition:  367A)}, then
$\tau(u)$ is at most $\liminf_{n\to\infty}\tau(u_n)$.
     
\proof{{\bf (a)} This is a special case of 367O.
     
\medskip
     
{\bf (b)} Set $B_{\gamma}=\{u:\tau(u)\le\gamma\}$.   If 
$u\in L^0\setminus B_{\gamma}$, then
     
\Centerline{$A=\{|u|\times\chi a:a\in\frak A^f\}$}
     
\noindent is an upwards-directed set with supremum $|u|$, so there is an
$a\in\frak A^f$ such that $\tau(u\times\chi a)>\gamma$.   \Quer\ If $u$
is in the closure of $B_{\gamma}$ for the topology of convergence in
measure, then for every $k\in\Bbb N$ there is a $v_k\in B_{\gamma}$ such
that $\bar\mu(a\Bcap\Bvalue{|u-v_k|>2^{-k}})\le 2^{-k}$ (see the
formulae in 367L).   Set
     
\Centerline{$v'_k=|u|\wedge\inf_{i\ge k}|v_i|$}
     
\noindent for each $k$, and $v^*=\sup_{k\in\Bbb N}v'_k$.   Then
$\tau(v'_k)\le\tau(v_k)\le\gamma$ for each $k$, and $\sequence{k}{v_k}$
is non-decreasing, so $\tau(v^*)\le\gamma$.   But
     
\Centerline{$a\Bcap\Bvalue{|u|-v^*>2^{-k}}
\Bsubseteq a\Bcap\sup_{i\ge k}\Bvalue{|u-v_i|>2^{-k}}$}
     
\noindent has measure at most $\sum_{i=k}^{\infty}2^{-i}$ for each $k$,
so $a\Bcap\Bvalue{|u|-v^*>0}$ must be $0$, that is, $|u|\times\chi a\le
v^*$ and $\tau(|u|\times\chi a)\le\gamma$;  contrary to the choice of
$a$.\ \BanG\   Thus $u$ cannot belong to the closure of $B_{\gamma}$.
As $u$ is arbitrary, $B_{\gamma}$ is closed.
     
\medskip
     
{\bf (c)} If $\sequencen{u_n}$ order*-converges to $u$, it converges
in measure (367Ma).   If $\gamma>\liminf_{n\to\infty}\tau(u_n)$, there
is a subsequence of $\sequencen{u_n}$ in $B_{\gamma}$, and
$\tau(u)\le\gamma$,
by (b).   As $\gamma$ is arbitary,
$\tau(u)\le\liminf_{n\to\infty}\tau(u_n)$.
}%end of proof of 369M
     
\leader{369N}{}\cmmnt{ I now turn to another special case which we
have already had occasion to consider in other contexts.
     
\medskip
     
\noindent}{\bf Definition} Let $(\frak A,\bar\mu)$ be a measure algebra.
Set
     
\Centerline{$M^{\infty,1}_{\bar\mu}=M^{\infty,1}(\frak A,\bar\mu)
=L^1(\frak A,\bar\mu)\cap L^{\infty}(\frak A)$,}
     
\Centerline{$M^{1,\infty}_{\bar\mu}=M^{1,\infty}(\frak A,\bar\mu)
=L^1(\frak A,\bar\mu)+L^{\infty}(\frak A)$,}
     
\noindent and
     
\Centerline{$\|u\|_{\infty,1}=\max(\|u\|_1,\|u\|_{\infty})$}
     
\noindent for $u\in L^0(\frak A)$.
     
\cmmnt{\medskip
     
\noindent{\bf Remark} I hope that the notation I have chosen here will
not completely overload your short-term memory.   The idea is that in
$M^{p,q}$ the symbol $p$ is supposed to indicate the `local' nature
of the space, that is, the nature of $u\times\chi a$ where 
$u\in M^{p,q}$ and $\bar\mu a<\infty$, while $q$ indicates the nature of
$|u|\wedge\chi 1$ for $u\in M^{p,q}$.   Thus $M^{1,\infty}$ is the space
of $u$ such that $u\times\chi a\in L^1$ for every $a\in\frak A^f$ amd
$|u|\wedge\chi 1\in L^{\infty}$;  in $M^{1,0}$ we demand further that
$|u|\wedge\chi 1\in M^0$ (366F);  while in $M^{\infty,1}$ we ask that
$|u|\wedge\chi 1\in L^1$ and that $u\times\chi a\in L^{\infty}$ for every $a\in\frak A^f$.
}%end of comment
     
\leader{369O}{Proposition} Let $(\frak A,\bar\mu)$ be a semi-finite
measure algebra.
     
(a) $\|\,\|_{\infty,1}$ is an extended Fatou norm on $L^0=L^0(\frak A)$.
     
(b) Its associate $\|\,\|_{1,\infty}$ may be defined by the formulae
     
$$\eqalign{\|u\|_{1,\infty}
&=\min\{\|v\|_1+\|w\|_{\infty}:v\in L^1,\,w\in L^{\infty},\,v+w=u\}\cr
&=\min\{\alpha+\int(|u|-\alpha\chi 1)^+:\alpha\ge 0\}\cr
&=\int_0^{\infty}\min(1,\bar\mu\Bvalue{|u|>\alpha})d\alpha\cr}$$
     
\noindent for every $u\in L^0$, writing $L^1=L^1(\frak A,\bar\mu)$,
$L^{\infty}=L^{\infty}(\frak A)$.
     
(c)
     
\Centerline{$\{u:u\in L^0,\,\|u\|_{1,\infty}<\infty\}
=M^{1,\infty}=M^{1,\infty}(\frak A,\bar\mu)$,}
     
\Centerline{$\{u:u\in L^0,\,\|u\|_{\infty,1}<\infty\}=M^{\infty,1}
=M^{\infty,1}(\frak A,\bar\mu)$.}
     
(d) Writing $\frak A^f=\{a:\bar\mu a<\infty\}$, $S(\frak A^f)$ is
norm-dense in $M^{\infty,1}$ and $S(\frak A)$ is norm-dense in
$M^{1,\infty}$.
     
(e) For any $p\in[1,\infty]$,
     
\Centerline{$\|u\|_{1,\infty}\le\|u\|_p\le\|u\|_{\infty,1}$}
     
\noindent for every $u\in L^0$.
     
\cmmnt{\medskip
     
\noindent{\bf Remark} By writing `min' rather than `inf' in the
formulae of part (b) I mean to assert that the infima are attained.
}
     
\proof{{\bf (a)} This is easy;  all we need to know is that $\|\,\|_1$
and $\|\,\|_{\infty}$ are extended Fatou norms.
     
\medskip
     
{\bf (b)} We have four functionals on $L^0$ to look at;  let me give
them names:
     
\Centerline{$\tau_1(u)=\sup\{\|u\times v\|_1:\|v\|_{\infty,1}\le 1\}$,}
     
\Centerline{$\tau_2(u)=\inf\{\|u'\|_1+\|u''\|_{\infty}:u=u'+u''\}$,}
     
\Centerline{$\tau_3(u)
=\inf_{\alpha\ge 0}(\alpha+\int(|u|-\alpha\chi 1)^+)$,}
     
\Centerline{$\tau_4(u)
=\int_0^{\infty}\min(1,\bar\mu\Bvalue{|u|>\alpha})d\alpha$.}
     
\noindent (I write `$\inf$' here to avoid the question of attainment
for the moment.)   Now we have the following.
     
\medskip
     
\quad{\bf (i)} $\tau_1(u)\le\tau_2(u)$.   \Prf\ If
$\|v\|_{\infty,1}\le 1$ and $u=u'+u''$, then
     
\Centerline{$\|u\times v\|_1
\le\|u'\times v\|_1+\|u''\times v\|_1
\le\|u'\|_1\|v\|_{\infty}+\|u''\|_{\infty}\|v\|_1
\le\|u'\|_1+\|u''\|_{\infty}$.}
     
\noindent Taking the supremum over $v$ and the infinum over $u'$ and
$u''$, $\tau_1(u)\le\tau_2(u)$.\ \Qed
     
\medskip
     
\quad{\bf (ii)} $\tau_2(u)\le\tau_4(u)$.   \Prf\ If $\tau_4(u)=\infty$
this is trivial.   Otherwise, take $w$ such that $\|w\|_{\infty}\le 1$
and $u=|u|\times w$.   Set
$\alpha_0=\inf\{\alpha:\bar\mu\Bvalue{|u|>\alpha}\le 1\}$, and try
     
\Centerline{$u'=w\times(|u|-\alpha_0\chi 1)^+$,
\quad$u''=w\times(|u|\wedge\alpha_0\chi 1)$.}
     
\noindent Then $u=u'+u''$,
     
$$\eqalign{\|u'\|_1
&=\int_0^{\infty}\bar\mu\Bvalue{|u'|>\alpha}d\alpha\
=\int_0^{\infty}\bar\mu\Bvalue{|u|>\alpha+\alpha_0}d\alpha\cr
&=\int_{\alpha_0}^{\infty}\bar\mu\Bvalue{|u|>\alpha}d\alpha
=\int_{\alpha_0}^{\infty}
   \min(1,\bar\mu\Bvalue{|u|>\alpha})d\alpha,\cr}$$
     
\Centerline{$\|u''\|_{\infty}
\le\alpha_0
=\int_0^{\alpha_0}\min(1,\Bvalue{|u|>\alpha})d\alpha$,}
     
\noindent so
     
\Centerline{$\tau_2(u)\le\|u'\|_1+\|u''\|_{\infty}\le\tau_4(u)$.
\Qed}
     
\medskip
     
\quad{\bf (iii)} $\tau_4(u)\le\tau_3(u)$.   \Prf\ For any $\alpha\ge 0$,
     
$$\eqalign{\tau_4(u)
&=\int_0^{\alpha}\min(1,\bar\mu\Bvalue{|u|>\beta})d\beta
   +\int_{\alpha}^{\infty}\min(1,\bar\mu\Bvalue{|u|>\beta})d\beta\cr
&\le\alpha+\int_0^{\infty}\bar\mu\Bvalue{|u|>\alpha+\beta}d\beta\cr
&=\alpha
   +\int_0^{\infty}\bar\mu\Bvalue{(|u|-\alpha\chi 1)^+>\beta}d\beta
=\alpha+\int(|u|-\alpha\chi 1)^+.\cr}$$
     
\noindent Taking the infimum over $\alpha$,
$\tau_4(u)\le\tau_3(u)$.\ \Qed
     
\medskip
     
\quad{\bf (iv)} $\tau_3(u)\le\tau_1(u)$.
     
\medskip
     
\quad\Prf\grheada\  It is enough to consider the case
$0<\tau_1(u)<\infty$,
because if $\tau_1(u)=0$ then $u=0$ and evidently $\tau_3(0)=0$, while
if $\tau_1(u)=\infty$ the required inequality is trivial.   Furthermore,
since $\tau_3(u)=\tau_3(|u|)$ and $\tau_1(u)=\tau_1(|u|)$, it is enough
to consider the case $u\ge 0$.
     
\medskip
     
\qquad\grheadb\ Note next that if $\bar\mu a<\infty$, then
$\|\bover1{\max(1,\bar\mu a)}\chi a\|_{\infty,1}\le 1$, so that
$\int_au\le\max(1,\bar\mu a)\tau_1(u)$.
     
\medskip
     
\qquad\grheadc\ Set $c=\Bvalue{u>2\tau_1(u)}$.   If $a\Bsubseteq c$ and
$\bar\mu a<\infty$, then
     
\Centerline{$2\tau_1(u)\bar\mu a\le\int_au
\le\max(1,\bar\mu a)\tau_1(u)$,}
     
\noindent so $\bar\mu a\le\bover12$.   As $(\frak A,\bar\mu)$ is
semi-finite, it follows that $\bar\mu c\le\bover12$ (322Eb).
     
\medskip
     
\qquad\grheadd\ I may therefore write
     
\Centerline{$\alpha_0=\inf\{\alpha:\alpha\ge
0,\,\bar\mu\Bvalue{u>\alpha}\le 1\}$.}
     
\noindent Now
$\Bvalue{u>\alpha_0}=\sup_{\alpha>\alpha_0}\Bvalue{u>\alpha}$, so
     
     
\Centerline{$\bar\mu\Bvalue{u>\alpha_0}
=\sup_{\alpha>\alpha_0}\bar\mu\Bvalue{u>\alpha}\le 1$.}
     
\medskip
     
\qquad\grheade\ If $\alpha\ge\alpha_0$ then
     
\Centerline{$(u-\alpha_0\chi 1)^+
\le(\alpha-\alpha_0)\chi\Bvalue{u>\alpha_0}+(u-\alpha\chi 1)^+$,}
\noindent so
     
$$\eqalign{\alpha_0+\int(u-\alpha_0\chi 1)^+
&\le\alpha_0+(\alpha-\alpha_0)\bar\mu\Bvalue{u>\alpha_0}
   +\int(u-\alpha\chi 1)^+\cr
&\le\alpha+\int(u-\alpha\chi 1)^+.\cr}$$
     
\noindent If $0\le\alpha<\alpha_0$ then, for every
$\beta\in\coint{0,\alpha_0-\alpha}$,
     
\Centerline{$(u-\alpha_0\chi 1)^++\beta\Bvalue{u>\alpha+\beta}
\le(u-\alpha\chi 1)^+$,}
     
\noindent while $\bar\mu\Bvalue{u>\alpha+\beta}>1$, so
     
\Centerline{$\int(u-\alpha_0\chi 1)^++\beta+\alpha
\le\alpha+\int(u-\alpha\chi 1)^+$;}
     
\noindent taking the supremum over $\beta$,
     
\Centerline{$\alpha_0+\int(u-\alpha_0\chi 1)^+
\le\alpha+\int(u-\alpha\chi 1)^+$.}
     
\noindent Thus $\alpha_0+\int(u-\alpha_0\chi 1)^+=\tau_3(u)$.
     
\medskip
     
\qquad{\bf ($\pmb{\zeta}$)} If $\alpha_0=0$, take $v=\chi\Bvalue{u>0}$;
then $\|v\|_{\infty,1}=\bar\mu\Bvalue{u>0}\le 1$ and
     
\Centerline{$\tau_3(u)=\int u=\|u\times v\|_1\le\tau_1(u)$.}
     
\qquad{\bf ($\pmb{\eta}$)} If $\alpha_0>0$, set
$\gamma=\bar\mu\Bvalue{u>\alpha_0}$.
Take any $\beta\in\coint{0,\alpha_0}$.   Then
$\bar\mu(\Bvalue{u>\beta}\Bsetminus\Bvalue{u>\alpha_0})>1-\gamma$, so
there is a $b\Bsubseteq\Bvalue{u>\beta}\Bsetminus\Bvalue{u>\alpha_0}$
such that $1-\gamma<\bar\mu b<\infty$.   Set
$v=\chi\Bvalue{u>\alpha_0}+\bover{1-\gamma}{\bar\mu b}\chi b$.   Then
$\|v\|_{\infty,1}=1$ so
     
\Centerline{$\tau_1(u)\ge\int u\times v
\ge\int(u-\alpha_0\chi 1)^+
+\alpha_0\gamma+\beta\Bover{1-\gamma}{\bar\mu b}\bar\mu b
=\tau_3(u)-(1-\gamma)(\alpha_0-\beta)$.}
     
\noindent As $\beta$ is arbitrary, $\tau_1(u)\ge\tau_3(u)$ in this case
also.\ \Qed
     
\medskip
     
\quad{\bf (v)} Thus $\tau_1(u)=\tau_2(u)=\tau_3(u)=\tau_4(u)$ for every
$u\in L^0$, and I may write $\|u\|_{1,\infty}$ for their common value;
being the associate of $\|\,\|_{\infty,1}$, $\|\,\|_{1,\infty}$ is an
extended Fatou norm.   As for the attainment of the infima, the argument
of (iv-$\epsilon$) above shows that, at least when
$0<\|u\|_{1,\infty}<\infty$, there is an $\alpha_0$ such that
$\alpha_0+\int(|u|-\alpha_0)^+=\|u\|_{1,\infty}$.   This omits the cases
$\|u\|_{1,\infty}\in\{0,\infty\}$;  but in either of these cases we can
set $\alpha_0=0$ to see that the infimum is attained for trivial
reasons.
For the other infimum, observe that the argument of (ii) produces $u'$,
$u''$ such that $u=u'+u''$ and $\|u'\|_1+\|u''\|_{\infty}\le\tau_4(u)$.
     
\medskip
     
{\bf (c)} This is now obvious from the definition of $\|\,\|_{\infty,1}$
and the characterization of $\|\,\|_{1,\infty}$ in terms of $\|\,\|_1$
and $\|\,\|_{\infty}$.
     
\medskip
     
{\bf (d)} To see that $S=S(\frak A)$ is norm-dense in $M^{1,\infty}$, we
need only note that $S$ is dense in $L^{\infty}$ and $S\cap L^1$ is
dense in $L^1$;  so that given $v\in L^1$, $w\in L^{\infty}$ and
$\epsilon>0$ there are $v'$, $w'\in S$ such that
     
\Centerline{$\|(v+w)-(v'+w')\|_{1,\infty}\le
\|v-v'\|_1+\|w-w'\|_{\infty}\le\epsilon$.}
     
\noindent As for $M^{\infty,1}$, if $u\ge 0$ in $M^{\infty,1}$ and
$r\in\Bbb N$, set $v_r=\sup_{k\in\Bbb
N}2^{-r}k\chi\Bvalue{u>2^{-r}k}$;
then each $v_r\in S^f=S(\frak A^f)$,
$\|u-v_r\|_{\infty}\le 2^{-r}$, and $\sequence{r}{v_r}$ is a
non-decreasing sequence with supremum $u$, so that
$\lim_{r\to\infty}\int v_r=\int u$ and
$\lim_{r\to\infty}\|u-v_r\|_{\infty,1}=0$.   Thus $(S^f)^+$ is dense in
$(M^{\infty,1})^+$.   As usual, it follows that $S^f=(S^f)^+-(S^f)^+$ is
dense in $M^{\infty,1}=(M^{\infty,1})^+-(M^{\infty,1})^+$.
     
\medskip
     
{\bf (e)(i)} If $p=1$ or $p=\infty$ this is immediate from the
definition
of $\|\,\|_{\infty,1}$ and the characterization of $\|\,\|_{1,\infty}$
in
(b).   So suppose henceforth that $1<p<\infty$.
     
\medskip
     
\quad{\bf (ii)} If $\|u\|_{\infty,1}\le 1$ then $\|u\|_p\le 1$.   \Prf\
Because $\|u\|_{\infty}\le 1$, $|u|^p\le|u|$, so that
$\int|u|^p\le\|u\|_1\le 1$ and $\|u\|_p\le 1$.\ \Qed
     
On considering scalar multiples of $u$, we see at once that
$\|u\|_p\le\|u\|_{\infty,1}$ for every $u\in L^0$.
     
\medskip
     
\quad{\bf (ii)} Now set $q=p/(p-1)$.   Then
     
$$\eqalignno{\|u\|_p&=\sup\{\|u\times v\|_1:\|v\|_q\le 1\}\cr
\noalign{\noindent (369L)}
&\ge\sup\{\|u\times v\|_1:\|v\|_{\infty,1}\le 1\}
=\|u\|_{1,\infty}\cr}$$
     
\noindent because $\|\,\|_{1,\infty}$ is the associate of
$\|\,\|_{\infty,1}$.   This completes the proof.
}%end of proof of 369O
     
\leader{369P}{}\cmmnt{ In preparation for some ideas in \S372, I go a
little farther with $M^{1,0}$, as defined in 366F.
     
\medskip
     
\noindent}{\bf Proposition} Let $(\frak A,\bar\mu)$ be a measure
algebra.
     
(a) $M^{1,0}=M^{1,0}(\frak A,\bar\mu)$ is a norm-closed solid linear
subspace of $\cmmnt{M^{1,\infty}=}M^{1,\infty}(\frak A,\bar\mu)$.
     
(b) The norm $\|\,\|_{1,\infty}$ is order-continuous on
$M^{1,0}$.
     
(c) $S(\frak A^f)$ and $L^1(\frak A,\bar\mu)$ are norm-dense and
order-dense in $M^{1,0}$.
     
\proof{{\bf (a)} Of course $M^{1,0}$, being a solid linear subspace
of $L^0=L^0(\frak A)$ included in $M^{1,\infty}$, is a solid linear subspace of
$M^{1,\infty}$.   To see that it is norm-closed, take any point $u$ of
its closure.   Then for any $\epsilon>0$ there is a $v\in M^{1,0}$ such
that $\|u-v\|_{1,\infty}\le\epsilon$;  now 
$(|u-v|-\epsilon\chi 1)^+\in L^1=L^1_{\bar\mu}$, so 
$\Bvalue{|u-v|>2\epsilon}$ has finite measure;  also
$\Bvalue{|v|>\epsilon}$ has finite measure, so
     
\Centerline{$\Bvalue{|u|>3\epsilon}
\Bsubseteq\Bvalue{|u-v|>2\epsilon}\Bcup\Bvalue{|v|>\epsilon}$}
     
\noindent (364Ea) has finite measure.   As $\epsilon$ is
arbitrary, $u\in M^{1,0}$;  as $u$ is arbitrary, $M^{1,0}$ is closed.
     
\medskip
     
{\bf (b)} Suppose that $A\subseteq M^{1,0}$ is non-empty and
downwards-directed and has infimum $0$.   Let $\epsilon>0$.   Set
$B=\{(u-\epsilon\chi 1)^+:u\in A\}$.   Then $B\subseteq L^1$ (by 366Gc);
$B$ is
non-empty and downwards-directed and has infimum $0$.   Because
$\|\,\|_1$ is order-continuous (365C), $\inf_{v\in B}\|v\|_1=0$ and
there is a $u\in A$ such that $\|(u-\epsilon\chi 1)^+\|_1\le\epsilon$,
so that $\|u\|_{1,\infty}\le 2\epsilon$.   As $\epsilon$ is arbitrary,
$\inf_{u\in A}\|u\|_{1,\infty}=0$;  as $A$ is arbitrary,
$\|\,\|_{1,\infty}$ is order-continuous on $M^{1,0}$.
     
\medskip
     
{\bf (c)} By 366Gb, $S(\frak A^f)$ is order-dense in $M^{1,0}$.
Because the norm of $M^{1,0}$ is order-continuous, $S(\frak A^f)$ is
also norm-dense (354Ef).   Now 
$S(\frak A^f)\subseteq L^1\subseteq M^{1,0}$, so $L^1$ must also be norm-dense and order-dense.
}%end of proof of 369P
     
\leader{369Q}{Corollary} Let $(\frak A,\bar\mu)$ be a localizable
measure algebra.   Set $M^{1,\infty}=M^{1,\infty}(\frak A,\bar\mu)$,
etc.
     
(a) $(M^{1,\infty})^{\times}$ and $(M^{1,0})^{\times}$ can both be
identified with $M^{\infty,1}$.
     
(b) $(M^{\infty,1})^{\times}$ can be identified with $M^{1,\infty}$;
$M^{1,\infty}$ and $M^{\infty,1}$ are perfect Riesz spaces.
     
\proof{ Everything is covered by 369O and 369K except the identification
of $(M^{1,0})^{\times}$ with $M^{\infty,1}$.   For this I return to
369C.   Of course $M^{1,0}$ is order-dense in $L^0$, because it includes
$L^1$, or otherwise.   Setting
     
\Centerline{$V=\{v:v\in L^0,\,u\times v\in L^1$ for every $u\in
M^{1,0}\}$,}
     
\noindent 369C identifies $V$ with $(M^{1,0})^{\times}$.   Of course
$M^{\infty,1}\subseteq V$ just because $M^{1,0}\subseteq M^{1,\infty}$.
     
Also $V\subseteq M^{\infty,1}$.   \Prf\ Because $L^1\subseteq M^{1,0}$ and $\|\,\|_{\infty}$ is the associate of $\|\,\|_1$, 
$V\subseteq L^{\infty}$.   \Quer\ If there is a $v\in V\setminus L^1$,
then (because $(\frak A,\bar\mu)$ is semi-finite,
so that $|v|=\sup_{a\in\frak A^f}|v|\times\chi a$) 
$\sup_{a\in\frak A^f}\int_a|v|=\infty$.   For each $n\in\Bbb N$ choose $a_n\in\frak A^f$
such that $\int_{a_n}|v|\ge 4^n$, and set 
$u=\sup_{n\in\Bbb N}2^{-n}\chi a_n\in M^{1,0}$;  then 
$\int u\times|v|\ge 2^n$ for each $n$, so again
$v\notin V$.\ \BanG\  Thus $V\subseteq L^1$ and 
$V\subseteq M^{\infty,1}$.\ \Qed
     
So $M^{\infty,1}=V$ can be identified with $(M^{1,0})^{\times}$.
}%end of proof of 369Q
     
\leader{369R}{}\cmmnt{ The detailed formulae of 369O are of course
special to the norms $\|\,\|_1$, $\|\,\|_{\infty}$, but the general
phenomenon is not.
     
\medskip
     
\noindent}{\bf Theorem} Let $(\frak A,\bar\mu)$ be a localizable measure
algebra, and $\tau_1$, $\tau_2$ two extended Fatou norms on
$L^0=L^0(\frak A)$ with associates $\tau'_1$, $\tau'_2$.   Then we have
an extended Fatou norm $\tau$ defined by the formula
     
\Centerline{$\tau(u)=\min\{\tau_1(v)+\tau_2(w):v,\,w\in L^0,\,v+w=u\}$}
     
\noindent for every $u\in L^0$, and its associate $\tau'$ is given by
the formula
     
\Centerline{$\tau'(u)=\max(\tau'_1(u),\tau'_2(u))$}
     
\noindent for every $u\in L^0$.   Moreover, the corresponding function
spaces are
     
\Centerline{$L^{\tau}=L^{\tau_1}+L^{\tau_2}$,
\quad$L^{\tau'}=L^{\tau'_1}\cap L^{\tau'_2}$.}
     
\proof{{\bf (a)} For the moment, define $\tau$ by setting
     
\Centerline{$\tau(u)=\inf\{\tau_1(v)+\tau_2(w):v+w=u\}$}
     
\noindent for $u\in L^0$.   It is easy to check that, for $u$,
$u'\in L^0$ and $\alpha\in\Bbb R$,
     
\Centerline{$\tau(u+u')\le\tau(u)+\tau(u')$,
\quad$\tau(\alpha u)=|\alpha|\tau(u)$,
\quad$\tau(u)\le\tau(u')$ if $|u|\le|u'|$.}
     
\noindent (For the last, remember that in this case $u=u'\times z$ where
$\|z\|_{\infty}\le 1$.)
     
\medskip
     
{\bf (b)} Take any non-empty, upwards-directed set $A\subseteq(L^0)^+$,
with supremum $u_0$.   Suppose that
$\gamma=\sup_{u\in A}\tau(u)<\infty$.    For $u\in A$ and $n\in\Bbb N$ set
     
\Centerline{$C_{un}=\{v:v\in L^0,\,0\le v\le u_0,\,
\tau_1(v)+\tau_2(u-v)^+\le\gamma+2^{-n}\}$.}
     
\noindent Then
     
\medskip
     
(i) every $C_{un}$ is non-empty (because $\tau(u)\le\gamma$);
     
(ii) every $C_{un}$ is convex (because if $v_1$, $v_2\in C_{un}$ and
$\alpha\in[0,1]$ and $v=\alpha v_1+(1-\alpha)v_2$, then
     
\Centerline{$(u-v)^+=(\alpha(u-v_1)+(1-\alpha)(u-v_2))^+
\le\alpha(u-v_1)^++(1-\alpha)(u-v_2)^+$,}
     
\noindent so
     
$$\eqalign{\tau_1(v)+\tau_2(u-v)^+
&\le\alpha\tau_1(v_1)+(1-\alpha)\tau_1(v_2)+\alpha\tau_2(u-v_1)^+
+(1-\alpha)\tau_2(u-v_2)^+\cr
&\le\gamma+2^{-n});\cr}$$
     
(iii) if $u$, $u'\in A$, $m$, $n\in\Bbb N$ and $u\le u'$, $m\le n$ then
$C_{u'n}\subseteq C_{um}$;
     
(iv) every $C_{un}$ is closed for the topology of convergence in
measure.   \Prf\Quer\ Suppose otherwise.   Then we can find a $v$ in the
closure of $C_{un}$ for the topology of convergence in measure, but such
that $\tau_1(v)+\tau_2(u-v)^+>\gamma+2^{-n}$.   In this case
     
\Centerline{$\tau_1(v)=\sup\{\tau_1(v\times\chi a):a\in\frak A^f\}$,
\quad$\tau_2(u-v)^+=\sup\{\tau_2((u-v)^+\times\chi a):a\in\frak A^f\}$,}
     
\noindent so there is an $a\in\frak A^f$ such that
     
\Centerline{$\tau_1(v\times\chi a)+\tau_2((u-v)^+\times\chi a)
>\gamma+2^{-n}$.}
     
\noindent Now there is a sequence $\sequence{k}{v_k}$ in $C_{un}$ such
that $\bar\mu(a\Bcap\Bvalue{|v-v_k|\ge2^{-k}})\le 2^{-k}$ for every $k$.
Setting
     
\Centerline{$v'_k=\inf_{i\ge k}v_i$,
\quad$w_k=\inf_{i\ge k}(u-v_i)^+$}
     
\noindent we have
     
\Centerline{$\tau_1(v'_k)+\tau_2(w_k)\le\tau_1(v_k)+\tau_2(u-v_k)^+
\le\gamma+2^{-n}$}
     
\noindent for each $k$, and $\sequence{k}{v'_k}$, $\sequence{k}{w_k}$
are non-decreasing.   So setting $v^*=\sup_{k\in\Bbb N}v\wedge v'_k$,
$w^*=\sup_{k\in\Bbb N}(u-v)^+\wedge w_k$, we get
     
\Centerline{$\tau_1(v^*)+\tau_2(w^*)\le\gamma+2^{-n}$.}
     
\noindent But $v^*\ge v\times\chi a$ and $w^*\ge(u-v)^+\times\chi a$,
so
     
\Centerline{$\tau_1(v\times\chi a)+\tau_2((u-v)^+\times\chi a)\le
\gamma+2^{-n}$,}
     
\noindent contrary to the choice of $a$.\ \Bang\Qed
     
Applying 367V, we find that $\bigcap_{u\in A,n\in\Bbb N}C_{un}$ is
non-empty.   If $v$ belongs to the intersection, then
     
\Centerline{$\tau_1(v)+\tau_2(u-v)^+\le\gamma$}
     
\noindent for every $u\in A$;  since $\{(u-v)^+:u\in A\}$ is an
upwards-directed set with supremum $(u_0-v)^+$, and $\tau_2$ is an
extended Fatou norm,
     
\Centerline{$\tau_1(v)+\tau_2(u_0-v)^+\le\gamma$.}
     
\medskip
     
{\bf (c)} This shows both that the infimum in the definition of
$\tau(u)$ is always attained (since this is trivial if $\tau(u)=\infty$,
and otherwise we consider $A=\{|u|\}$), and also that $\tau(\sup
A)=\sup_{u\in A}\tau(u)$ whenever $A\subseteq (L^0)^+$ is a non-empty
upwards-directed set with a supremum.   Thus $\tau$ satisfies conditions
(i)-(iv) of 369F.   Condition (vi) there is trivial, since (for
instance) $\tau(v)\le\tau_1(v)$ for every $v$.
As for 369F(v), suppose that $u>0$ in $L^0$.   Take $u_1$ such that
$0<u_1\le u$ and $\tau'_1(u_1)\le 1$, $u_2$ such that $0<u_2\le u_1$ and
$\tau'_2(u_2)\le 1$.   In this case, if $u_2=v+w$, we must have
     
\Centerline{$\tau_1(v)+\tau_2(w)
\ge\|v\times u_1\|_1+\|w\times u_2\|_1
\ge\|u_2\times u_2\|_1$;}
     
\noindent so that
     
\Centerline{$\tau(u)\ge\|u_2\times u_2\|_1>0$.}
     
Thus all the conditions of 369F are satisfied, and $\tau$ is an extended
Fatou norm on $L^0$.
     
\medskip
     
{\bf (d)} The calculation of $\tau'$ is now very easy.   Since surely we
have $\tau\le\tau_i$ for both $i$, we must have $\tau'\ge\tau'_i$ for
both $i$.   On the other hand, if $u$, $z\in L^0$, then there are $v$,
$w$ such that $u=v+w$ and $\tau(u)=\tau_1(v)+\tau_2(w)$, so that
     
\Centerline{$\|u\times z\|_1\le\|v\times z\|_1+\|w\times z\|_1
\le\tau_1(v)\tau'_1(z)+\tau_2(w)\tau'_2(z)
\le\tau(u)\max(\tau'_1(z),\tau'_2(z))$;}
     
\noindent as $u$ is arbitrary, $\tau'(z)\le\max(\tau'_1(z),\tau'_2(z))$.
So $\tau'=\max(\tau'_1,\tau'_2)$, as claimed.
     
\medskip
     
{\bf (e)} Finally, it is obvious that
     
\Centerline{$L^{\tau'}=\{z:\tau'(z)<\infty\}=\{z:\tau'_1(z)<\infty,\,
\tau'_2(z)<\infty\}=L^{\tau'_1}\cap L^{\tau'_2}$,}
     
\noindent while the fact that the infimum in the definition of $\tau$ is
always attained means that $L^{\tau}\subseteq L^{\tau_1}+L^{\tau_2}$, so
that we have equality here also.
}%end of proof of 369R
         
\exercises{\leader{369X}{Basic exercises $\pmb{>}$(a)}
%\spheader 369Xa
Let $\frak A$ be a Dedekind $\sigma$-complete Boolean algebra.  Show
that the following are equiveridical:  (i) there is a function $\bar\mu$
such that $(\frak A,\bar\mu)$ is a semi-finite measure algebra;  (ii)
$(L^{\infty})^{\times}$ separates the points of
$L^{\infty}=L^{\infty}(\frak A)$;   (iii) for every non-zero 
$a\in\frak A$ there is a completely additive functional 
$\nu:\frak A\to\Bbb R$ such
that $\nu a\ne 0$;   (iv) there is some order-dense Riesz subspace $U$
of $L^0=L^0(\frak A)$ such that $U^{\times}$ separates the points of
$U$;  (v) for every order-dense Riesz subspace $U$ of $L^0$ there is an
order-dense Riesz subspace $V$ of $U$ such that $V^{\times}$ separates
the points of $V$.
%369A
     
\spheader 369Xb Let us say that a function
$\phi:\Bbb R\to\ocint{-\infty,\infty}$ is {\bf convex} if 
$\phi(\alpha x+(1-\alpha)y)\ge\alpha\phi(x)+(1-\alpha)\phi(y)$ for all $x$, $y\in\Bbb R$
and $\alpha\in[0,1]$, interpreting $0\cdot\infty$ as $0$, as usual.
For any convex function $\phi:\Bbb R\to\ocint{-\infty,\infty}$ which is
not always infinite, set $\phi^*(y)=\sup_{x\in\Bbb R}xy-\phi(x)$ for
every $y\in\Bbb R$.   (i) Show that 
$\phi^*:\Bbb R\to\ocint{-\infty,\infty}$ is convex and lower
semi-continuous and not always infinite.   \Hint{233Xh.}   (ii) Show that if $\phi$ is
lower semi-continuous then $\phi=\phi^{**}$.   \Hint{It is easy to check
that $\phi^{**}\le\phi$.   For the reverse inequality, set
$I=\{x:\phi(x)<\infty\}$, and consider $x\in\interior I$,
$x\in I\setminus\interior I$ and $x\notin I$ separately;   233Ha is
useful for the first.}
     
\sqheader 369Xc For the purposes of this exercise and the next, say that
a {\bf Young's function} is a
non-negative non-constant lower semi-continuous convex function
$\phi:\coint{0,\infty}\to[0,\infty]$ such that $\phi(0)=0$ and $\phi(x)$
is finite for some $x>0$.   ({\bf Warning!} the phrase `Young's
function' has other meanings.)   (i) Show that in this case $\phi$ is
non-decreasing and continuous on the left and $\phi^*$, defined by
saying that $\phi^*(y)=\sup_{x\ge 0}xy-\phi(x)$ for every $y\ge 0$, is
again a Young's function.   (ii) Show that $\phi^{**}=\phi$.   Say that
$\phi$ and $\phi^*$ are {\bf complementary}.
(iii) Compute $\phi^*$ in the cases ($\alpha$) $\phi(x)=x$ ($\beta$)
$\phi(x)=\max(0,x-1)$ ($\gamma$) $\phi(x)=x^2$ ($\delta$) $\phi(x)=x^p$
where $1<p<\infty$.
     
\sqheader 369Xd  Let $\phi$, $\psi=\phi^*$ be complementary Young's
functions in the sense of 369Xc, and
$(\frak A,\bar\mu)$ a semi-finite measure algebra.   Set
     
\Centerline{$B=\{u:u\in L^0,\,\int\bar\phi(|u|)\le 1\}$,
\quad$C=\{v:v\in L^0,\,\int\bar\psi(|v|)\le 1\}$.}
     
\noindent (For finite-valued $\phi$, $\bar\phi:(L^0)^+\to L^0$ is given
by 364H.   Devise an appropriate convention for the case in which
$\phi$ takes the value $\infty$.)   (i) Show that $B$ and $C$ are
order-closed solid convex sets, and that $\int|u\times v|\le 2$ for all
$u\in B$, $v\in C$.   \Hint{for `order-closed', use 364Xg(iv).}
(ii) Show that there is a unique extended Fatou norm $\tau_{\phi}$ on
$L^0$ for which $B$ is the unit ball.  (iii) Show that if
$u\in L^0\setminus B$ there is a $v\in C$ such that $\int|u\times v|>1$.
\Hint{start with the case in which
$u\in S(\frak A)^+$.}   (iv) Show that
$\tau_{\psi}\le\tau'_{\phi}\le 2\tau_{\psi}$, where $\tau_{\psi}$ is the
extended Fatou norm corresponding to $\psi$ and $\tau'_{\phi}$ is the
associate of $\tau_{\phi}$, so that $\tau_{\psi}$ and $\tau'_{\phi}$ can
be interpreted as equivalent norms on the same Banach space.
     
($U$ and $V$ are complementary {\bf Orlicz spaces};  I will call
$\tau_{\phi}$, $\tau_{\psi}$ {\bf Orlicz norms}.)
%369F, 369H
     
\spheader 369Xe Let $U$ be a Riesz space such that $U^{\times}$
separates the points of $U$, and suppose that $\|\,\|$ is a Fatou norm
on $U$.  (i) Show that there is a localizable measure algebra
$(\frak A,\bar\mu)$ with an extended Fatou norm $\tau$ on $L^0(\frak A)$
such that $U$ can be identified, as normed Riesz space, with an
order-dense Riesz subspace of $L^{\tau}$.    (ii) Hence, or otherwise,
show that $\|u\|=\sup_{f\in U^{\times},\,\|f\|\le 1}|f(u)|$ for every
$u\in U$.
(iii) Show that if $U$ is Dedekind complete and has the Levi property,
then $U$ becomes identified with $L^{\tau}$ itself, and in particular is
a Banach lattice (cf.\ 354Xn).
%369I
     
\spheader 369Xf Let $(\frak A,\bar\mu)$ be a semi-finite measure
algebra, and $\tau$ an extended Fatou norm on $L^0(\frak A)$.   Show
that the norm of $L^{\tau}$ is order-continuous iff the norm topology of
$L^{\tau}$ agrees with the topology of convergence in measure on any
order-bounded subset of $L^{\tau}$.
%369M
     
\spheader 369Xg Let $(\frak A,\bar\mu)$ be a $\sigma$-finite measure
algebra of countable Maharam type, and $\tau$ an extended Fatou norm on
$L^0(\frak A)$ such that the norm of $L^{\tau}$ is order-continuous.
Show that $L^{\tau}$ is separable in its norm topology.
%369Xf
     
\spheader 369Xh Let $(\frak A,\bar\mu)$ be an atomless semi-finite
measure algebra.   Show that
$\|u\|_{1,\infty}=\max\{\int_a|u|:a\in\frak A,\,\bar\mu a\le 1\}$ for
every $u\in L^0(\frak A)$.   \Hint{take
$a\Bsupseteq\Bvalue{|u|>\alpha_0}$ in part (b-iv) of the proof of 369O.}
%369O
     
\spheader 369Xi Let $(\frak A,\bar\mu)$ be any semi-finite measure
algebra.   Show that if $\tau_{\phi}$ is any Orlicz
norm on $L^0=L^0(\frak A)$, then there is a $\gamma>0$ such that
$\|u\|_{1,\infty}\le\gamma\tau_{\phi}(u)\le\gamma^2\|u\|_{\infty,1}$ for
every $u\in L^0$, so that
$M^{\infty,1}_{\bar\mu}\subseteq L^{\tau_{\phi}}
\subseteq M^{1,\infty}_{\bar\mu}$.
%369O
     
\spheader 369Xj Let $(\frak A,\bar\mu)$ be a semi-finite measure
algebra.   Show that the subspaces $M^{1,\infty}_{\bar\mu}$,
$M^{\infty,1}_{\bar\mu}$ of $L^0(\frak A)$ can be expressed as a
complementary pair
of Orlicz spaces, and that the norm $\|\,\|_{\infty,1}$ can be
represented as an Orlicz norm, but $\|\,\|_{1,\infty}$ cannot.
%369O
     
\sqheader 369Xk Let $(\frak A,\bar\mu)$ be a measure algebra and $U$ a
Banach space.   (i) Suppose that $\nu:\frak A\to U$ is an additive
function such that $\|\nu a\|\le\min(1,\bar\mu a)$ for every $a\in\frak
A$.   Show that there is a unique bounded linear operator
$T:M^{1,\infty}_{\bar\mu}\to U$ such that $T(\chi a)=\nu a$ for
every $a\in\frak A$.   (ii) Suppose that $\nu:\frak A^f\to U$ is an
additive function such that $\|\nu a\|\le\max(1,\bar\mu a)$ for every
$a\in\frak A^f$.   Show that there is a unique bounded linear operator
$T:M^{\infty,1}_{\bar\mu}\to U$ such that $T(\chi a)=\nu a$ for
every $a\in\frak A^f$.
%369O
     
\spheader 369Xl Let $(\frak A,\bar\mu)$ and $(\frak B,\bar\nu)$ be
semi-finite measure algebras, and $\pi:\frak A^f\to\frak B^f$ a
measure-preserving ring homomorphism, as in 366H, with associated maps
$T:M^0_{\bar\mu}\to M^0_{\bar\nu}$ and
$P:M^{1,0}_{\bar\nu}\to M^{1,0}_{\bar\mu}$.   Show that 
$\|Tu\|_{\infty,1}=\|u\|_{\infty,1}$ for every 
$u\in M^{\infty,1}_{\bar\mu}$ and $\|Pv\|_{\infty,1}\le\|v\|_{\infty,1}$ 
for every $v\in M^{\infty,1}_{\bar\nu}$.
%369N
     
\spheader 369Xm Let $(\frak A,\bar\mu)$ and $(\frak B,\bar\nu)$ be
measure algebras, and $\pi:\frak A\to\frak B$ a measure-preserving
Boolean homomorphism.   (i) Show that there is a unique Riesz
homomorphism $T:M^{1,\infty}_{\bar\mu}\to M^{1,\infty}_{\bar\nu}$ such
that $T(\chi a)=\chi(\pi a)$ for every $a\in\frak A$
and $\|Tu\|_{1,\infty}=\|u\|_{1,\infty}$ for every 
$u\in M^{1,\infty}_{\bar\mu}$.   (ii) Now suppose that 
$(\frak A,\bar\mu)$ is localizable and $\pi$ is order-continuous.   
Show that there is a unique positive linear operator 
$P:M^{1,\infty}_{\bar\nu}\to M^{1,\infty}_{\bar\mu}$ such that
$\int_aPv=\int_{\pi a}v$ for every $a\in\frak A^f$ and
$v\in M^{1,\infty}_{\bar\nu}$, and that
$\|Pv\|_{\infty}\le\|v\|_{\infty}$ for every 
$v\in L^{\infty}(\frak B)$, %referred in 366 notes
$\|Pv\|_{1,\infty}\le\|v\|_{1,\infty}$ for
every $v\in M^{1,\infty}_{\bar\nu}$.   (Compare 365P.)
%369N
     
\spheader 369Xn Let $(\frak A,\bar\mu)$ and $(\frak B,\bar\nu)$ be
semi-finite measure algebras, and $\phi:\coint{0,\infty}\to[0,\infty]$ a
Young's function;  write $\tau_{\phi}$ for the
corresponding
Orlicz norm on either $L^0(\frak A)$ or $L^0(\frak B)$.
Let $\pi:\frak A\to\frak B$ be a
measure-preserving Boolean homomorphism, with associated map
$T:M^{1,\infty}_{\bar\mu}\to M^{1,\infty}_{\bar\nu}$, as
in 369Xm.   (i) Show that $\tau_{\phi}(Tu)=\tau_{\phi}(u)$ for every
$u\in M^{1,\infty}_{\bar\mu}$.
(ii) Show that if $(\frak A,\bar\mu)$ is localizable, $\pi$ is
order-continuous and 
$P:M^{1,\infty}_{\bar\nu}\to M^{1,\infty}_{\bar\mu}$ is the map of 369Xm(ii), then $\tau_{\phi}(Pv)\le\tau_{\phi}(v)$ for every 
$v\in M^{1,\infty}_{\bar\nu}$.
\Hint{365R.}
%369N
     
\sqheader 369Xo Let $(\frak A,\bar\mu)$ be any semi-finite measure
algebra and $\tau_1$, $\tau_2$ two extended Fatou norms on 
$L^0(\frak A)$.   Show that $u\mapsto\max(\tau_1(u),\tau_2(u))$ is an extended Fatou norm.
%369O
     
\spheader 369Xp Let $(\frak A,\bar\mu)$ be a semi-finite measure
algebra, and $(\widehat{\frak A},\tilde\mu)$ its localization (322Q).
Show that the Dedekind completion of $M^{1,\infty}(\frak A,\bar\mu)$ can
be identified with $M^{1,\infty}(\widehat{\frak A},\tilde\mu)$.
%369Q
     
\spheader 369Xq Let $(\frak A,\bar\mu)$ be a localizable measure
algebra.   (i) Show that if $\frak B$ is any closed subalgebra of 
$\frak A$ such that $\sup\{b:b\in\frak B,\,\bar\mu b<\infty\}=1$ in 
$\frak A$, we have an order-continuous positive linear operator 
$P_{\frak B}:
M^{1,\infty}_{\bar\mu}\to M^{1,\infty}_{\bar\mu\restrp\frak B}$ 
such that $\int_bP_{\frak B}u=\int_bu$ whenever 
$u\in M^{1,\infty}_{\bar\mu}$, $b\in\frak B$ and $\bar\mu b<\infty$.   
(ii) Show that if $\sequencen{\frak B_n}$ is a
non-decreasing sequence of closed subalgebras of $\frak A$ such that
$\sup\{b:b\in\frak B_0,\,\bar\mu b<\infty\}=1$ in $\frak A$, and 
$\frak B$ is the closure of $\bigcup_{n\in\Bbb N}\frak B_n$, then
$\sequencen{P_{\frak B_n}u}$ is order*-convergent to $P_{\frak B}u$
for every $u\in M^{1,\infty}_{\bar\mu}$.   (Cf.\ 367J.)
%369Q
     
\spheader 369Xr Let $\phi_1$ and $\phi_2$ be Young's functions
and $(\frak A,\bar\mu)$ a semi-finite measure algebra.   Set
$\phi(x)=\max(\phi_1(x),\penalty-100\phi_2(x))$ for $x\in\coint{0,\infty}$.   (i)
Show that $\phi$ is a Young's function.   (ii) Writing $\tau_{\phi_1}$
$\tau_{\phi_2}$, $\tau_{\phi}$ for the corresponding extended Fatou
norms on $L^0(\frak A)$ (369Xd), show that
$\tau_{\phi}\ge\max(\tau_{\phi_1},\tau_{\phi_2})\ge\bover12\tau_{\phi}$,
so that $L^{\tau_{\phi}}=L^{\tau_{\phi_1}}\cap L^{\tau_{\phi_2}}$ and
$L^{\tau_{\phi^*}}=L^{\tau_{\phi_1^*}}+L^{\tau_{\phi_2^*}}$, writing
$\phi^*$ for the Young's function complementary to $\phi$.   (iii)
Repeat with $\psi=\phi_1+\phi_2$ in place of $\phi$.
%369R
     
\leader{369Y}{Further exercises (a)}
%\spheader 369Ya
Let $(\frak A,\bar\mu)$ be a localizable measure algebra
and $A\subseteq L^0=L^0(\frak A)$ a countable set.   Show that the solid
linear subspace $U$ of $L^0$ generated by $A$ is a perfect Riesz space.
\Hint{reduce to the case in which $U$ is order-dense.   If
$A=\{u_n:n\in\Bbb N\}$, $w\in(L^0)^+\setminus U$ find $v_n\in(L^0)^+$
such that $\int v_n\times w\ge 2^n\ge 4^n\int v_n\times|u_i|$ for every
$i\le n$.   Show that $v=\sup_{n\in\Bbb N}v_n$ is defined in $L^0$ and
corresponds to a member of $U^{\times}$.}
%369C
     
\spheader 369Yb Let $U$ be a Banach lattice and suppose that
$p\in\coint{1,\infty}$ is such that $\|u+v\|^p=\|u\|^p+\|v\|^p$ whenever
$u$, $v\in U$ and $|u|\wedge|v|=0$.   Show that $U$ is isomorphic, as
Banach lattice, to $L^p_{\bar\mu}$ for some localizable measure
algebra $(\frak A,\bar\mu)$.   \Hint{start by using 354Yb to show that
the norm of $U$ is order-continuous, as in 354Yk.}
%369E
     
\spheader 369Yc
Let $\phi:\coint{0,\infty}\to\coint{0,\infty}$ be a strictly increasing
Young's function such that $\sup_{t>0}\phi(2t)/\phi(t)$
is finite.   Show that the associated Orlicz norms $\tau_{\phi}$
are always order-continuous on their function spaces.
%369Xd, 369H
     
\spheader 369Yd
Let $\phi:\coint{0,\infty}\to[0,\infty]$ be a Young's function, and
suppose that the corresponding Orlicz norm on
$L^0(\frak A_L)$, where $(\frak A_L,\bar\mu_L)$ is the measure algebra
of Lebesgue measure on $\Bbb R$, is order-continuous on its function
space $L^{\tau_{\phi}}$.   Show that there is an $M\ge 0$ such that
$\phi(2t)\le M\phi(t)$ for every $t\ge 0$.
%369Xd, 369H
     
\spheader 369Ye Let $(\frak A,\bar\mu)$ be a semi-finite measure algebra
and $\tau_{\phi}$ an Orlicz norm which is order-continuous on
$L^{\tau_{\phi}}$.
Show that if $\Cal F$ is a filter on $L^{\tau_{\phi}}$, then
$\Cal F\to u\in L^{\tau_{\phi}}$ for the norm $\tau_{\phi}$ iff (i)
$\Cal F\to u$ for the topology of convergence in measure (ii)
$\limsup_{v\to\Cal F}\tau_{\phi}(v)\le\tau_{\phi}(u)$.   (Compare
245Xl.)
%369M
     
\spheader 369Yf Give an example of an extended Fatou norm $\tau$ on
$L^0(\frak A_L)$, where $(\frak A_L,\bar\mu_L)$ is the measure algebra
of Lebesgue measure on $[0,1]$, such that (i) $\tau$ gives rise to an
order-continuous norm on its function space $L^{\tau}$ (ii) there is a
sequence $\sequencen{u_n}$ in $L^{\tau}$, converging in measure to $u\in
L^{\tau}$, such that $\lim_{n\to\infty}\tau(u_n)=\tau(u)$ but
$\sequencen{u_n}$ does not converge to $u$ for the norm on $L^{\tau}$.
%369M %mt36bits
     
\spheader 369Yg Let $(\frak A,\bar\mu)$ be a semi-finite measure
algebra, and $\tau$ an Orlicz norm on $L^0(\frak A)$.   Show that
$L^{\tau}$ has the Levi property, whether or not $\frak A$ is Dedekind
complete.
%369O
     
\spheader 369Yh Let $(\frak A,\bar\mu)$ be any measure algebra.   Show
that $(M^{1,0}_{\bar\mu})^{\times}$ can be identified with
$M^{\infty,1}_{\bar\mu}$.   \Hint{show that neither $M^{1,0}$ nor
$M^{\infty,1}$ is changed by moving first to the semi-finite version of
$(\frak A,\bar\mu)$, as described in 322Xa, and then to its
localization.}
%369Q
     
\spheader 369Yi Give an example to show that the result of 369R may fail
if $(\frak A,\bar\mu)$ is only semi-finite, not localizable.
}%end of exercises
     
\endnotes{
\Notesheader{369} The representation theorems 369A-369D give a
concrete form to the notion of `perfect' Riesz space:  it is just one
which can be expressed as a subspace of $L^0(\frak A)$, for some
localizable measure algebra $(\frak A,\bar\mu)$, in such a way that it
is its own second dual, where the duality here is between subspaces of
$L^0$, taking $V=\{v:u\times v\in L^1$ for every $u\in U\}$.   (I see
that in this expression I ought somewhere to mention that both $U$ and
$V$ are assumed to be order-dense in $L^0$.)   Indeed I believe that
the original perfect spaces were the `vollkommene R\"aume' of
G.K\"othe, which were subspaces of $\BbbR^{\Bbb N}$, corresponding to
the measure algebra $\Cal P\Bbb N$ with counting measure, so that $V$
or $U^{\times}$ was $\{v:u\times v\in\ell^1$ for every $u\in U\}$.
     
I have presented Kakutani's theorem on the representation of $L$-spaces
as a corollary of 369A and 369C.   As usual in such things, this is a
reversal of the historical relationship;  Kakutani's theorem was one of
the results
which led to the general theory.   If we take the trouble to re-work the
argument of 369A in this context, we find that the $L$-space condition
`$\|u+v\|=\|u\|+\|v\|$ whenever $u$, $v\ge 0$' can be relaxed to
`$\|u+v\|=\|u\|+\|v\|$ whenever $u\wedge v=0$' (369Yb).   The complete
list of localizable measure algebras provided by Maharam's theorem
(332B, 332J) now gives us a complete list of $L$-spaces.
     
Just as perfect Riesz spaces come in dual pairs, so do some of the most
important Banach lattices:  those with Fatou norms and the Levi property
for which the order-continuous dual separates the points.   (Note that
the dual of any space with a Riesz norm has these properties;  see
356Da.)   I leave the details of representing such spaces to you
(369Xe).   The machinery of 369F-369K gives a solid basis for studying
such pairs.
     
Among the extended Fatou norms of 369F the Orlicz norms (369Xd,
369Yc-369Ye) form a
significant subfamily.   Because they are defined in a way which is to
some extent independent of the measure algebra involved, these spaces
have some of the same properties as $L^p$ spaces in relation to
measure-preserving homomorphisms (369Xm-369Xn).   In \S\S373-374 I will
elaborate on these ideas.   Among the Orlicz spaces, we have a
largest and a smallest;  these are just $M^{1,\infty}=L^1+L^{\infty}$
and $M^{\infty,1}=L^1\cap L^{\infty}$ (369N-369O, 369Xi, 369Xj).   Of
course these two are particularly important.
     
There is an interesting phenomenon here.   It is easy to see that
$\|\,\|_{\infty,1}=\max(\|\,\|_1,\|\,\|_{\infty})$ is an extended Fatou
norm and that the corresponding Banach lattice is $L^1\cap L^{\infty}$;
and that the same ideas work for any pair of extended Fatou norms
(369Xo).   To check that the dual of $L^1\cap L^{\infty}$ is precisely
the linear sum $L^{\infty}+L^1$ a little more is needed, and the
generalization of this fact to other extended Fatou norms (369R) seems
to go quite deep.   In view of our ordinary expectation that properties
of these normed function spaces should be reflected in perfect Riesz
spaces in general, I mention that I believe I have found an example,
dependent on the continuum hypothesis, of
two perfect Riesz subspaces $U$, $V$ of $\BbbR^{\Bbb N}$ such that
their linear sum $U+V$ is not perfect.
%mt36bits
}%end of comment
     
\frnewpage

