\frfilename{mt371.tex}
\versiondate{13.12.06}
\copyrightdate{1995}

\def\chaptername{Linear operators between function spaces}
\def\sectionname{The Chacon-Krengel theorem}

\newsection{371}

The first topic I wish to treat is a remarkable property of $L$-spaces:
if $U$ and $V$ are $L$-spaces, then every continuous linear operator
$T:U\to V$ is order-bounded, and $\||T|\|=\|T\|$ (371D).   This
generalizes in
various ways to other $V$ (371B, 371C).   I apply the result to a
special type of operator between $M^{1,0}$ spaces which will be
conspicuous in the next section (371F-371H).

\leader{371A}{Lemma} Let $U$ be an $L$-space, $V$ a Banach lattice and
$T:U\to V$ a bounded linear operator.   Take $u\ge 0$ in $U$ and set

\Centerline{$B=\{\sum_{i=0}^n|Tu_i|:u_0,\ldots,u_n\in U^+,\,
\sum_{i=0}^nu_i=u\}\subseteq V^+$.}

\noindent Then $B$ is upwards-directed and
$\sup_{v\in B}\|v\|\le\|T\|\|u\|$.

\proof{{\bf (a)} Suppose that $v$, $v'\in B$.   Then we have
$u_0,\ldots,u_m,u'_0,\ldots,u'_n\in U^+$ such that
$\sum_{i=0}^mu_i=\sum_{j=0}^nu'_j=u$, $v=\sum_{i=0}^m|Tu_i|$ and
$v'=\sum_{j=0}^n|Tu'_j|$.   Now there are $v_{ij}\ge 0$ in $U$, for
$i\le m$ and $j\le n$, such that $u_i=\sum_{j=0}^nv_{ij}$ for $i\le m$
and $u'_j=\sum_{i=0}^mv_{ij}$ for $j\le n$ (352Fd).   We have
$u=\sum_{i=0}^m\sum_{j=0}^nv_{ij}$, so that
$v''=\sum_{i=0}^m\sum_{j=0}^n|Tv_{ij}|\in B$.   But

\Centerline{$v=\sum_{i=0}^m|Tu_i|=\sum_{i=0}^m|T(\sum_{j=0}^nv_{ij})|
\le\sum_{i=0}^m\sum_{j=0}^m|Tv_{ij}|=v''$,}

\noindent and similarly $v'\le v''$.   As $v$ and $v'$ are arbitrary,
$B$ is upwards-directed.

\medskip

{\bf (b)} The other part is easy.   If $v\in B$ is expressed as
$\sum_{i=0}^n|Tu_i|$ where $u_i\ge 0$ for every $i$
and $\sum_{i=0}^nu_i=u$, then

\Centerline{$\|v\|\le\sum_{i=0}^n\|Tu_i\|\le\|T\|\sum_{i=0}^n\|u_i\|
=\|T\|\|u\|$}

\noindent because $U$ is an $L$-space.
}%end of proof of 371A

\leader{371B}{Theorem} Let $U$ be an
$L$-space and $V$ a Dedekind complete Banach
lattice $U$ with a Fatou norm.   Then the Riesz space
$\eurm L^{\sim}(U;V)=\eurm L^{\times}(U;V)$ is a closed linear
subspace of the Banach space $\eurm B(U;V)$ and is in itself a Banach
lattice with a Fatou norm.

\proof{{\bf (a)} I start by noting that
$\eurm L^{\sim}(U;V)=\eurm L^{\times}(U;V)\subseteq\eurm B(U;V)$ just
because $V$ has a Riesz norm
and $U$ is a Banach lattice with an order-continuous norm (355Kb, 355C).

\medskip

{\bf (b)} The first new step is to check that $\||T|\|\le\|T\|$ for
any $T\in\eurm L^{\sim}(U;V)$.   \Prf\ Start with any $u\in U^+$.
Set

\Centerline{$B=\{\sum_{i=0}^n|Tu_i|:u_0,\ldots,u_n\in U^+,\,
\sum_{i=0}^nu_i=u\}\subseteq V^+$,}

\noindent as in 371A.   If $u_0,\ldots,u_n\ge 0$ are such that
$\sum_{i=0}^nu_i=u$, then $|Tu_i|\le|T|u_i$ for each $i$, so that
$\sum_{i=0}^n|Tu_i|\le\sum_{i=0}^n|T|u_i=|T|u$;  thus $B$ is bounded
above by $|T|u$ and $\sup B\le|T|u$.   On the other hand, if $|v|\le u$
in $U$, then $v^++v^-+(u-|v|)=u$, so $|Tv^+|+|Tv^-|+|T(u-|v|)|\in B$
and

\Centerline{$|Tv|=|Tv^++Tv^-|\le|Tv^+|+|Tv^-|\le\sup B$.}

\noindent As $v$ is arbitrary, $|T|u\le\sup B$ and $|T|u=\sup B$.
Consequently

\Centerline{$\||T|u\|\le\|\sup B\|=\sup_{w\in B}\|w\|\le\|T\|\|u\|$}

\noindent because $V$ has a Fatou norm and $B$ is upwards-directed.

For general $u\in U$,

\Centerline{$\||T|u\|\le\||T||u|\|\le\|T\|\||u|\|=\|T\|\|u\|$.}

\noindent This shows that $\||T|\|\le\|T\|$.  \Qed

\medskip

{\bf (c)} Now if $|S|\le|T|$ in $\eurm L^{\sim}(U;V)$, and $u\in U$,
we must have

\Centerline{$\|Su\|\le\||S||u|\|\le\||T||u|\|\le\||T|\|\||u|\|
\le\|T\|\|u\|$;}

\noindent as $u$ is arbitrary, $\|S\|\le\|T\|$.   This shows that the
norm of $\eurm L^{\sim}(U;V)$, inherited from $\eurm B(U;V)$, is a
Riesz norm.

\medskip

{\bf (d)} Suppose next that $T\in\eurm B(U;V)$ belongs to the
norm-closure of $\eurm L^{\sim}(U;V)$.   For each $n\in\Bbb N$ choose
$T_n\in\eurm L^{\sim}(U;V)$ such that $\|T-T_n\|\le 2^{-n}$.   Set
$S_n=|T_{n+1}-T_n|\in\eurm L^{\sim}(U;V)$ for each $n$.   Then

\Centerline{$\|S_n\|=\|T_{n+1}-T_n\|\le 3\cdot 2^{-n-1}$}

\noindent for each $n$, so $S=\sum_{n=0}^{\infty}S_n$ is defined in the
Banach space $\eurm B(U;V)$.   But if $u\in U^+$, we surely have

\Centerline{$Su=\sum_{n=0}^{\infty}S_nu\ge 0$}

\noindent in $V$.   Moreover, if $u\in U^+$ and $|v|\le u$, then for
any $n\in\Bbb N$

\Centerline{$|T_{n+1}v-T_0v|
=|\sum_{i=0}^n(T_{i+1}-T_i)v|
\le\sum_{i=0}^nS_iu
\le Su$,}

\noindent and $T_0v-Su\le T_{n+1}v\le T_0v+Su$;  letting $n\to\infty$,
we see that

\Centerline{$-|T_0|u-Su\le T_0v-Su\le Tv\le T_0v+Su\le|T_0|u+Su$.}

\noindent So $|Tv|\le|T_0|u+Su$ whenever $|v|\le u$.   As
$u$ is arbitrary, $T\in\eurm L^{\sim}(U;V)$.

This shows that $\eurm L^{\sim}(U;V)$ is closed in $\eurm B(U;V)$
and is therefore a Banach space in its own right;  putting this together
with (b), we see that it is a Banach lattice.

\medskip

{\bf (e)} Finally, the norm of $\eurm L^{\sim}(U;V)$ is a Fatou norm.
\Prf\ Let $A\subseteq\eurm L^{\sim}(U;V)^+$ be a non-empty,
upwards-directed set with supremum $T_0\in\eurm L^{\sim}(U;V)$.   For
any $u\in U$,

\Centerline{$\|T_0u\|=\||T_0u|\|\le\|T_0|u|\|=\|\sup_{T\in A}T|u|\|$}

\noindent by 355Ed.   But $\{T|u|:T\in A\}$ is upwards-directed and
the norm of $V$ is a Fatou norm, so

\Centerline{$\|T_0u\|\le\sup_{T\in A}\|T|u|\|\le\sup_{T\in
A}\|T\|\|u\|$.}

\noindent As $u$ is arbitrary, $\|T_0\|\le\sup_{T\in A}\|T\|$.   As $A$
is arbitrary, the norm of $\eurm L^{\sim}(U;V)$ is Fatou.  \Qed
}%end of proof of 371B

\leader{371C}{Theorem} Let $U$ be an $L$-space and $V$ a Dedekind
complete Banach
lattice with a Fatou norm and the Levi property.   Then $\eurm
B(U;V)=\eurm L^{\sim}(U;V)=\eurm L^{\times}(U;V)$ is a Dedekind complete
Banach lattice with a Fatou norm
and the Levi property.   In particular, $|T|$ is defined and
$\||T|\|=\|T\|$ for every $T\in\eurm B(U;V)$.

\proof{{\bf (a)} Let $T:U\to V$ be any bounded linear operator.   Then
$T\in\eurm L^{\sim}(U;V)$.   \Prf\ Take any $u\ge 0$ in $U$.   Set

\Centerline{$B=\{\sum_{i=0}^n|Tu_i|:u_0,\ldots,u_n\in U^+,\,
\sum_{i=0}^nu_i=u\}\subseteq V^+$}

\noindent as in 371A.   Then 371A tells us that $B$ is
upwards-directed and norm-bounded.   Because $V$ has the Levi property,
$B$ is bounded above.   But just as in part (b) of the proof of 371B,
any upper bound of $B$ is also an upper bound of $\{Tv:|v|\le u\}$.   As
$u$ is arbitrary, $T\in\eurm L^{\sim}(U;V)$.
\Qed

\medskip

{\bf (b)} Accordingly $\eurm L^{\sim}(U;V)=\eurm B(U;V)$.   By 371B,
this is a Banach lattice with a Fatou norm, and equal to $\eurm
L^{\times}(U;V)$.   To see that it also has
the Levi property, let $A\subseteq\eurm L^{\sim}(U;V)$ be any
non-empty norm-bounded upwards-directed set.   For $u\in U^+$,
$\{Tu:T\in A\}$ is non-empty, norm-bounded and upwards-directed in $V$,
so is bounded above in $V$.  By 355Ed, $A$ is bounded above in
$\eurm L^{\sim}(U;V)$.
}%end of proof of 371C

\leader{371D}{Corollary} Let $U$ and $V$ be $L$-spaces.   Then
$\eurm L^{\sim}(U;V)=\eurm L^{\times}(U;V)=\eurm B(U;V)$ is a Dedekind
complete Banach lattice with a Fatou norm and the Levi property.

\cmmnt{
\leader{371E}{Remarks} Note that both these theorems show that
$\eurm L^{\sim}(U;V)$ is a Banach lattice with properties similar to
those of $V$ whenever $U$ is an $L$-space.   They can therefore be
applied repeatedly, to give facts about
$\eurm L^{\sim}(U_1;\eurm L^{\sim}(U_2;V))$ where $U_1$, $U_2$ are
$L$-spaces and $V$ is a Banach
lattice, for instance.   I hope that this formula will recall some of
those in the theory of bilinear operators and tensor products (see
253Xa-253Xb).
}%end of comment

\leader{371F}{The class \dvrocolon{$\Cal T^{(0)}$}}\cmmnt{ For the
sake of applications in the next section, I introduce now a class of
operators of great intrinsic interest.

\medskip

\noindent}{\bf Definition} Let $(\frak A,\bar\mu)$, $(\frak B,\bar\nu)$
be measure algebras.   \cmmnt{Recall that $M^{1,0}(\frak A,\bar\mu)$
is the space of those $u\in L^1(\frak A,\bar\mu)+L^{\infty}(\frak A)$
such that $\bar\mu\Bvalue{|u|>\alpha}<\infty$ for every $\alpha>0$
(366F-366G, 369P).}
Write $\Cal T^{(0)}=\Cal T^{(0)}_{\bar\mu,\bar\nu}$ for the set of all
linear operators
$T:M^{1,0}(\frak A,\bar\mu)\to M^{1,0}(\frak B,\bar\nu)$ such that
$Tu\in L^1(\frak B,\bar\nu)$ and
$\|Tu\|_1\le\|u\|_1$ for every
$u\in L^1(\frak A,\bar\mu)$, $Tu\in L^{\infty}(\frak B)$ and
$\|Tu\|_{\infty}\le\|u\|_{\infty}$ for
every $u\in L^{\infty}(\frak A)\cap M^{1,0}(\frak A,\bar\mu)$.

\leader{371G}{Proposition} Let $(\frak A,\bar\mu)$ and
$(\frak B,\bar\nu)$ be measure algebras.

(a) $\Cal T^{(0)}=\Cal T^{(0)}_{\bar\mu,\bar\nu}$
is a convex set in the unit ball of
$\eurm B(M^{1,0}(\frak A,\bar\mu);M^{1,0}(\frak B,\bar\nu))$.   If
$T_0:L^1(\frak A,\bar\mu)\to L^1(\frak B,\bar\nu)$ is a linear operator
of norm at most $1$, and $T_0u\in L^{\infty}(\frak B)$ and
$\|T_0u\|_{\infty}\le\|u\|_{\infty}$ for
every $u\in L^1(\frak A,\bar\mu)\cap L^{\infty}(\frak A)$, then $T_0$
has a unique extension to a member of $\Cal T^{(0)}$.

(b) If $T\in\Cal T^{(0)}$ then $T$ is order-bounded and $|T|$, taken in

\Centerline{$\eurm L^{\sim}(M^{1,0}(\frak A,\bar\mu);
  M^{1,0}(\frak B,\bar\nu))
=\eurm L^{\times}(M^{1,0}(\frak A,\bar\mu);M^{1,0}(\frak B,\bar\nu))$,}

\noindent also belongs to $\Cal T^{(0)}$.

(c) If $T\in\Cal T^{(0)}$ then $\|Tu\|_{1,\infty}\le\|u\|_{1,\infty}$
for every $u\in M^{1,0}(\frak A,\bar\mu)$.

(d) If $T\in\Cal T^{(0)}$, $p\in\coint{1,\infty}$ and
$w\in L^p(\frak A,\bar\mu)$ then
$Tw\in L^p(\frak B,\bar\nu)$ and $\|Tw\|_p\le\|w\|_p$.

(e) If $(\frak C,\bar\lambda)$ is another measure algebra then
$ST\in\Cal T^{(0)}_{\bar\mu,\bar\lambda}$ whenever
$T\in\Cal T^{(0)}_{\bar\mu,\bar\nu}$ and
$S\in\Cal T^{(0)}_{\bar\nu,\bar\lambda}$.

\proof{ I write $M^{1,0}_{\bar\mu}$, $L^p_{\bar\nu}$ for
$M^{1,0}_{\bar\mu}$, $L^p(\frak B,\bar\nu)$, etc.

\medskip

{\bf (a)(i)} If $T\in\Cal T^{(0)}$ and $u\in M^{1,0}_{\bar\mu}$ then
there are $v\in L^1_{\bar\mu}$,
$w\in L^{\infty}_{\bar\mu}$ such that $u=v+w$ and
$\|v\|_1+\|w\|_{\infty}=\|u\|_{1,\infty}$ (369Ob);  so that

\Centerline{$\|Tu\|_{1,\infty}\le\|Tv\|_1+\|Tw\|_{\infty}
\le\|v\|_1+\|w\|_{\infty}\le\|u\|_{1,\infty}$.}

\noindent As $u$ is arbitrary, $T$ is in the unit ball of
$\eurm B(M^{1,0}_{\bar\mu};M^{1,0}_{\bar\nu})$.

\medskip

\quad{\bf (ii)} Because the unit balls of
$\eurm B(L^1_{\bar\mu};L^1_{\bar\nu})$ and
$\eurm B(L^{\infty}_{\bar\mu};L^{\infty}_{\bar\nu})$ are convex, so is
$\Cal T^{(0)}$.

\medskip

\quad{\bf (iii)} Now suppose that $T_0:L^1_{\bar\mu}\to L^1_{\bar\nu}$
is a linear operator of norm at most $1$ such that
$\|T_0u\|_{\infty}\le\|u\|_{\infty}$ for every
$u\in L^1_{\bar\mu}\cap L^{\infty}_{\bar\mu}$.   By the argument of (i),
$T_0$ is a bounded operator for
the $\|\,\|_{1,\infty}$ norms;  since $L^1_{\bar\mu}$ is dense in
$M^{1,0}_{\bar\mu}$ (369Pc), $T_0$ has a unique extension to a
bounded linear operator $T:M^{1,0}_{\bar\mu}\to M^{1,0}_{\bar\nu}$.   Of
course $\|Tu\|_1=\|T_0u\|_1\le\|u\|_1$ for every $u\in L^1_{\bar\mu}$.

Now suppose that $u\in L^{\infty}_{\bar\mu}\cap M^{1,0}_{\bar\mu}$;  set
$\gamma=\|u\|_{\infty}$.   Let $\epsilon>0$, and set

\Centerline{$v=(u^+-\epsilon\chi 1)^+-(u^--\epsilon\chi 1)^+$;}

\noindent then $|v|\le|u|$ and $\|u-v\|_{\infty}\le\epsilon$ and
$v\in L^1_{\bar\mu}\cap L^{\infty}_{\bar\mu}$.   Accordingly

\Centerline{$\|Tu-Tv\|_{1,\infty}\le\|u-v\|_{1,\infty}\le\epsilon$,
\quad$\|Tv\|_{\infty}=\|T_0v\|_{\infty}
\le\|v\|_{\infty}\le\gamma$.}
\noindent So if we set $w=(|Tu-Tv|-\epsilon\chi 1)^+\in L^1_{\bar\nu}$,
$\|w\|_1\le\epsilon$;  while

\Centerline{$|Tu|\le|Tv|+w+\epsilon\chi 1\le(\gamma+\epsilon)\chi 1+w$,}

\noindent so

\Centerline{$\|(|Tu|-(\gamma+\epsilon)\chi 1)^+\|_1
\le\|w\|_1\le\epsilon$.}

\noindent As $\epsilon$ is arbitrary, $|Tu|\le\gamma\chi 1$, that
is, $\|Tu\|_{\infty}\le\|u\|_{\infty}$.   As $u$ is arbitrary,
$T\in\Cal T^{(0)}$.

\medskip

{\bf (b)} Because $M^{1,0}_{\bar\mu}$ has an order-continuous norm
(369Pb), $\eurm L^{\sim}(M^{1,0}_{\bar\mu};M^{1,0}_{\bar\nu})
=\eurm L^{\times}(M^{1,0}_{\bar\mu};M^{1,0}_{\bar\nu})$ (355Kb).   Take
any $T\in\Cal T^{(0)}$ and consider
$T_0=T\restr L^1_{\bar\mu}:L^1_{\bar\mu}\to L^1_{\bar\nu}$.   This is an
operator of norm at most $1$.   By
371D, $T_0$ is order-bounded, and $\||T_0|\|\le 1$, where $|T_0|$
is taken in $\eurm L^{\sim}(L^1_{\bar\mu};L^1_{\bar\mu})
=\eurm B(L^1_{\bar\mu};L^1_{\bar\nu})$.   Now if
$u\in L^1_{\bar\mu}\cap L^{\infty}_{\bar\nu}$,

\Centerline{$||T_0|u|\le|T_0||u|
=\sup_{|u'|\le|u|}|T_0u'|\le\|u\|_{\infty}\chi 1$,}

\noindent so $\||T_0|u\|_{\infty}\le\|u\|_{\infty}$.   By (a), there is
a unique $S\in\Cal T^{(0)}$ extending $|T_0|$.   Now $Su^+\ge 0$ for
every $u\in L^1_{\bar\mu}$, so $Su^+\ge 0$ for every
$u\in M^{1,0}_{\bar\mu}$ (since the function
$u\mapsto (Su^+)^+-Su^+:M^{1,0}_{\bar\mu}\to M^{1,0}_{\bar\nu}$ is
continuous and zero on the dense set $L^1_{\bar\mu}$), that is, $S$ is a
positive operator;  also $S|u|\ge|Tu|$ for every $u\in L^1_{\bar\mu}$,
so $Sv\ge S|u|\ge|Tu|$ whenever $u$, $v\in M^{1,0}_{\bar\mu}$ and
$|u|\le v$.   This means
that $T:M^{1,0}_{\bar\mu}\to M^{1,0}_{\bar\nu}$ is order-bounded.
Because $M^{1,0}_{\bar\nu}$ is
Dedekind complete (366Ga), $|T|$ is defined in
$\eurm L^{\sim}(M^{1,0}_{\bar\mu};M^{1,0}_{\bar\mu})$.

If $v\ge 0$ in $L^1_{\bar\mu}$, then

\Centerline{$|T|v=\sup_{|u|\le v}Tu=\sup_{|u|\le v}T_0u=|T_0|v=Sv$.}

\noindent Thus $|T|$ agrees with $S$ on $L^1_{\bar\mu}$.   Because
$M^{1,0}_{\bar\mu}$ is a
Banach lattice (or otherwise), $|T|$ is a bounded operator, therefore
continuous (2A4Fc), so $|T|=S\in\Cal T^{(0)}$, which is what we needed
to know.

\medskip

{\bf (c)} We can express $u$ as $v+w$ where
$\|v\|_1+\|w\|_{\infty}=\|u\|_{1,\infty}$;  now
$w=u-v\in M^{1,0}_{\bar\mu}$, so we can speak of $Tw$, and

\Centerline{$\|Tu\|_{1,\infty}=\|Tv+Tw\|_{1,\infty}
\le\|Tv\|_1+\|Tw\|_{\infty}\le\|v\|_1+\|w\|_{\infty}=\|u\|_{1,\infty}$,}

\noindent as required.

\medskip

{\bf (d)} (This is a modification of 244M.)

\medskip

\quad{\bf (i)} Suppose that $T$, $p$, $w$ are as described, and that in
addition $T$ is positive.   The function $t\mapsto|t|^p$ is convex
(233Xc), so we can find families $\langle\beta_q\rangle_{q\in\Bbb Q}$,
$\langle\gamma_q\rangle_{q\in\Bbb Q}$ of real numbers such that
$|t|^p=\sup_{q\in\Bbb Q}\beta_q+\gamma_q(t-q)$ for every $t\in\Bbb R$
(233Hb).   Then
$|u|^p=\sup_{q\in\Bbb Q}\beta_q\chi 1+\gamma_q(u-q\chi 1)$ for every
$u\in L^0$.   (The easiest way to check this is perhaps to think of
$L^0$ as a quotient of a space of functions, as in 364C;  it is also a
consequence of 364Xg(iii).)   We know that
$|w|^p\in L^1_{\bar\mu}$, so we may speak of $T(|w|^p)$;  while
$w\in M^{1,0}_{\bar\mu}$ (366Ga), so we may speak of $Tw$.

For any $q\in\Bbb Q$, $0^p\ge\beta_q-q\gamma_q$, that is,
$q\gamma_q-\beta_q\ge 0$, while
$\gamma_qw-|w|^p\le(q\gamma_q-\beta_q)\chi 1$ and
$\|(\gamma_qw-|w|^p)^+\|_{\infty}\le q\gamma_q-\beta_q$.   Now this
means that

$$\eqalign{T(\gamma_qw-|w|^p)
&\le T(\gamma_qw-|w|^p)^+
\le\|T(\gamma_qw-|w|^p)^+\|_{\infty}\chi 1\cr
&\le\|(\gamma_qw-|w|^p)^+\|_{\infty}\chi 1
\le (q\gamma_q-\beta_q)\chi 1.\cr}$$

\noindent Turning this round again,

\Centerline{$\beta_q\chi 1+\gamma_q(Tw-q\chi 1)\le T(|w|^p)$.}

\noindent Taking the supremum over $q$, $|Tw|^p\le T(|w|^p)$, so that
$\int|Tw|^p\le\int|w|^p$ (because $\|Tv\|_1\le\|v\|_1$ for every
$v\in L^1$).   Thus $Tw\in L^p$ and $\|Tw\|_p\le\|w\|_p$.

\medskip

\quad{\bf (ii)} For a general $T\in\Cal T^{(0)}$, we have
$|T|\in\Cal T^{(0)}$, by
(b), and $|Tw|\le|T||w|$, so that $\|Tw\|_p\le\||T||w|\|_p\le\|w\|_p$,
as required.

\medskip

{\bf (e)} This is elementary, because

\Centerline{$\|STu\|_1\le\|Tu\|_1\le\|u\|_1$,
\quad$\|STv\|_{\infty}\le\|Tu\|_{\infty}\le\|u\|_{\infty}$}

\noindent whenever $u\in L^1_{\bar\mu}$ and
$v\in L^{\infty}_{\bar\mu}\cap M^{1,0}_{\bar\mu}$.
}%end of proof of 371G

\cmmnt{
\leader{371H}{Remark} In the context of 366H,
$T_{\pi}\restr M^{1,0}_{\bar\mu}\in\Cal T^{(0)}_{\bar\mu,\bar\nu}$,
while $P_{\pi}\in\Cal T^{(0)}_{\bar\nu,\bar\mu}$.   Thus 366H(a-iv) and
366H(b-iii) are special cases of 371Gd.
}

\exercises{\leader{371X}{Basic exercises $\pmb{>}$(a)}
%\spheader 371Xa
Let $U$ be an $L$-space, $V$ a Banach lattice with an order-continuous
norm and $T:U\to V$ a bounded linear operator.   Let $B$ be the unit
ball of $U$.   Show that $|T|[B]\subseteq\overline{T[B]}$.
%371A

\spheader 371Xb Let $U$ and $V$ be Banach spaces.  (i) Show that the
space $\eurm K(U;V)$ of compact linear operators from $U$ to
$V$ (definition:  3A5La) is a closed
linear subspace of $\eurm B(U;V)$.   (ii) Show that if $U$ is an
$L$-space and $V$ is a Banach lattice with an order-continuous norm,
then $\eurm K(U;V)$ is a norm-closed Riesz subspace of
$\eurm L^{\sim}(U;V)$.  (See {\smc Krengel 63}.)
%371A, 371Xa

\spheader 371Xc Let $(\frak A,\bar\mu)$ be a semi-finite measure algebra
and set $U=L^1(\frak A,\bar\mu)$.   Show that
$\eurm L^{\sim}(U;U)=\eurm B(U;U)$ is a Banach lattice with a Fatou norm
and the Levi property.
Show that its norm is order-continuous iff $\frak A$ is finite.
\Hint{consider operators $u\mapsto u\times\chi a$, where $a\in\frak A$.}
%371C

\sqheader 371Xd Let $U$ be a Banach lattice, and $V$ a Dedekind complete
$M$-space.   Show that $\eurm L^{\sim}(U;V)=\eurm B(U;V)$ is a Banach
lattice with a Fatou norm and the Levi property.

\spheader 371Xe Let $U$ and $V$ be Riesz spaces, of which $V$ is
Dedekind complete, and let $T\in\eurm L^{\sim}(U;V)$.   Define
$T'\in\eurm L^{\sim}(V^{\sim};U^{\sim})$ by writing $T'(h)=hT$ for
$h\in V^{\sim}$.   (i) Show that $|T|'\ge|T'|$ in
$\eurm L^{\sim}(V^{\sim};U^{\sim})$.   (ii) Show that $|T|'h=|T'|h$ for
every $h\in V^{\times}$.   ({\it Hint\/}:  show that if $u\in U^+$ and
$h\in(V^{\times})^+$ then $(|T'|h)(u)$ and $h(|T|u)$ are both equal to
$\sup\{\sum_{i=0}^ng_i(Tu_i):|g_i|\le h,\,u_i\ge 0,
\,\sum_{i=0}^nu_i=u\}$.)
%371C

\sqheader 371Xf Using 371D, but nothing about uniformly integrable sets
beyond the definition (354P), show that if $U$ and $V$ are $L$-spaces,
$A\subseteq U$ is uniformly integrable in $U$, and $T:U\to V$ is a
bounded linear operator, then $T[A]$ is uniformly integrable in $V$.
%371D

\leader{371Y}{Further exercises (a)}
%\spheader 371Ya
Let $U$ and $V$ be Banach spaces.   (i) Show that the
space $\eurm K_w(U;V)$ of weakly compact linear operators from
$U$ to $V$ (definition:  3A5Lb) is a closed linear subspace of
$\eurm B(U;V)$.   (ii) Show
that if $U$ is an $L$-space and $V$ is a Banach lattice with an
order-continuous norm, then $\eurm K_w(U;V)$ is a norm-closed Riesz
subspace of $\eurm L^{\sim}(U;V)$.
%371A, 371Xa

\spheader 371Yb Let $(\frak A,\bar\mu)$ be a measure algebra, $U$ a
Banach space, and $T:L^1(\frak A,\bar\mu)\to U$ a bounded linear
operator.   Show that $T$ is a compact linear operator iff
$\{\Bover1{\bar\mu a}T(\chi a):a\in\frak A,\,0<\bar\mu a<\infty\}$ is
relatively compact in $U$.
%could be anywhere;  needs only that unit ball is
%closed convex hull of these

\spheader 371Yc Let $(\frak A,\bar\mu)$ be a probability algebra, and
set $L^1=L^1(\frak A,\bar\mu)$.   Let $\sequencen{a_n}$ be a
stochastically
independent sequence of elements of $\frak A$ of measure $\bover12$, and
define $T:L^1\to\BbbR^{\Bbb N}$ by setting
$Tu(n)=\int u-2\int_{a_n}u$ for each $n$.   Show that
$T\in\eurm B(L^1;\pmb{c}_0)\setminus\eurm L^{\sim}(L^1;\pmb{c}_0)$,
where $\pmb{c}_0$
is the Banach lattice of sequences converging to $0$.   (See
272Ye\formerly{2{}72Yd}.)
%371B

\spheader 371Yd Regarding $T$ of 371Yc as a map from $L^1$ to
$\ell^{\infty}$, show that $|T'|\ne|T|'$ in $\eurm
L^{\sim}((\ell^{\infty})^*,L^{\infty}(\frak A))$.
%371Yc, 371B

\spheader 371Ye(i) In $\ell^2$ define $e_i$ by setting $e_i(i)=1$,
$e_i(j)=0$ if $j\ne i$.   Show that if
$T\in\eurm L^{\sim}(\ell^2;\ell^2)$ then
$\innerprod{|T|e_i}{e_j}=|\innerprod{Te_i}{e_j}|$ for all $i$,
$j\in\Bbb N$.   (ii) Show that for each $n\in\Bbb N$ there is an
orthogonal
$(2^n\times 2^n)$-matrix $\tbf{A}_n$ such that every coefficient of
$\tbf{A}_n$ has modulus $2^{-n/2}$.   ({\it Hint\/}:  $\tbf{A}_{n+1}
=\Bover1{\sqrt 2}\Matrix{\tbf{A}_n&\tbf{A}_n\\-\tbf{A}_n&\tbf{A}_n}$.)
(iii) Show that there is a linear isometry $S:\ell^2\to\ell^2$ such that
$|\innerprod{Se_i}{e_j}|=2^{-n/2}$ if $2^n\le i,\,j<2^{n+1}$.   (iv)
Show that $S\notin\eurm L^{\sim}(\ell^2;\ell^2)$.
%371C

}%end of exercises

\endnotes{
\Notesheader{371}
The `Chacon-Krengel theorem', properly speaking ({\smc Chacon \&
Krengel 64}), is 371D in the case in which $U=L^1(\mu)$, $V=L^1(\nu)$;
of course no new ideas are required in the generalizations here, which I
have copied from {\smc Fremlin 74a}.

Anyone with a training in functional analysis will automatically seek to
investigate properties of operators $T:U\to V$ in terms of properties of
their adjoints $T':V^*\to U^*$, as in 371Xe and 371Yd.   When $U$ is an
$L$-space, then $U^*$ is a Dedekind complete $M$-space, and it is easy
to see that this forces $T'$ to be order-bounded, for any Banach lattice
$V$ (371Xd).   But since no important $L$-space is reflexive, this
approach cannot reach 371B-371D without a new idea of some kind.   It
can however be adapted to the special case in 371Gb
({\smc Dunford \& Schwartz 57}, VIII.6.4).

In fact the results of 371B-371C are characteristic of
$L$-spaces ({\smc Fremlin 74b}).   To see that they fail in the simplest
cases in which $U$ is not an $L$-space and $V$ is not an $M$-space, see
371Yc-371Ye.

}%end of comment

\discrpage

