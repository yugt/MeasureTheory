\frfilename{mt459.tex}
\versiondate{7.12.10}
\copyrightdate{2003}

\def\chaptername{Perfect measures, disintegrations and processes}
\def\sectionname{Symmetric measures and exchangeable random variables}

\newsection{459}

Among the relatively independent families of random variables discussed
in 458K,
it is natural to give extra attention to those which are `relatively
identically distributed'.   It turns out that these have a particularly
appealing characterization as the `exchangeable' families (459C).   In
the same way, among the measures on a product space $X^I$ there is a
special place for those which are invariant under permutations of
coordinates (459E, 459H).   A more abstract kind of permutation-invariance
is examined in 495I-495J.

\leader{459A}{}\cmmnt{ The following elementary fact seems to have
gone unmentioned so far.

\medskip

\noindent}{\bf Lemma} Let $(X,\Sigma,\mu)$ and $(Y,\Tau,\nu)$ be
probability spaces and $\phi:X\to Y$ an \imp\ function;  set
$\Sigma_0=\{\phi^{-1}[F]:F\in\Tau\}$.   Let $\Tau_1$ be a
$\sigma$-subalgebra of $\Tau$ and
$\Sigma_1=\{f^{-1}[F]:F\in\Tau_1\}$.   If $g\in\eusm L^1(\nu)$ and $h$
is a conditional expectation of $g$ on $\Tau_1$, then $h\phi$ is a
conditional expectation of $g\phi$ on $\Sigma_1$.

\proof{ $h$ is $\nu\restr\Tau_1$-integrable and $\phi$ is \imp\ for
$\mu\restr\Sigma_1$ and $\nu\restr\Tau_1$, so $h\phi$ is
$\mu\restr\Sigma_1$-integrable (235G).
If $E\in\Sigma_1$ then there is an
$F\in\Tau_1$ such that $E=\phi^{-1}[F]$, and now

\Centerline{$\int_Eg\phi\,d\mu=\int_{f^{-1}[F]}g\phi\,d\mu=\int_Fg\,d\nu
=\int_Fh\,d\nu=\int_Eh\phi\,d\mu$.}

\noindent As $E$ is arbitrary, $h\phi$ is a conditional expectation of
$g\phi$ on $\Sigma_1$.
}%end of proof of 459A

\vleader{72pt}{459B}{Theorem} Let $(X,\Sigma,\mu)$ be
a probability space, $Z$ a set, $\Upsilon$ a $\sigma$-algebra of subsets
of $Z$ and $\familyiI{f_i}$ an infinite family of
$(\Sigma,\Upsilon)$-measurable functions from $X$ to $Z$.   For each
$i\in I$, set $\Sigma_i=\{f_i^{-1}[H]:H\in\Upsilon\}$.
Then the following are equiveridical:

\inset{(i) whenever $i_0,\ldots,i_r\in I$ are distinct,
$j_0,\ldots,j_r\in I$ are distinct, and $H_k\in\Upsilon$ for each
$k\le r$, then
$\mu(\bigcap_{k\le r}f_{i_k}^{-1}[H_k])
=\mu(\bigcap_{k\le r}f_{j_k}^{-1}[H_k])$;

(ii) there is a $\sigma$-subalgebra $\Tau$ of $\Sigma$ such that

\quad($\alpha$) $\familyiI{\Sigma_i}$ is relatively independent over
$\Tau$,

\quad($\beta$) whenever $i$, $j\in I$, $H\in\Upsilon$ and $F\in\Tau$,
then $\mu(F\cap f_i^{-1}[H])=\mu(F\cap f_j^{-1}[H])$.}

\noindent Moreover, if $I$ is totally ordered by $\le$, we can add

\inset{(iii) whenever $i_0<\ldots<i_r\in I$, $j_0<\ldots<j_r\in I$ and
$H_k\in\Upsilon$ for each $k\le r$, then
$\mu(\bigcap_{k\le r}f_{i_k}^{-1}[H_k])
=\mu(\bigcap_{k\le r}f_{j_k}^{-1}[H_k])$.}

\proof{{\bf (a)} Since there is always some total
order on $I$, we may assume that we have one from the start.
Of course (i)$\Rightarrow$(iii).   Also (ii)$\Rightarrow$(i).
\Prf\ Suppose that (ii) is true.   Then (ii-$\beta$) tells us that for
each $H\in\Upsilon$ there is a $\Tau$-measurable function
$g_H:X\to[0,1]$ which is a conditional expectation of
$\chi(f_i^{-1}[H])$ on $\Tau$ for every $i\in I$.   Now suppose that
$i_0,\ldots,i_r\in I$ are distinct, $j_0,\ldots,j_r\in I$ are distinct,
and $H_k\in\Upsilon$ for each $k\le r$.   Then

\Centerline{$\mu(\bigcap_{k\le r}f_{i_k}^{-1}[H_k])
=\int(\prod_{k=0}^rg_{H_k})d\mu
=\mu(\bigcap_{k\le r}f_{j_k}^{-1}[H_k])$}

\noindent because $\familyiI{\Sigma_i}$ is relatively independent over
$\Tau$.   So (i) is true.\ \Qed

So henceforth I will
suppose that (iii) is true and seek to prove (ii).

\medskip

{\bf (b)} Suppose first that $I=\Bbb N$ with its usual ordering.

\medskip

\quad\grheada\
For each $n$, $r\in\Bbb N$, let $\Sigma_{nr}$ be the $\sigma$-subalgebra
of $\Sigma$ generated by $\bigcup_{n\le i\le n+r}\Sigma_i$;  let
$\Tau_n$ be the
$\sigma$-algebra generated by $\bigcup_{r\in\Bbb N}\Sigma_{nr}$, and
$\Tau=\bigcap_{n\in\Bbb N}\Tau_n$.   For $n\in\Bbb N$ and
$H\in\Upsilon$, let $g_{nH}:X\to\Bbb R$ be a $\Tau$-measurable
function which is a conditional expectation of
$\chi f_n^{-1}[H]$ on $\Tau$.

\medskip

\quad\grheadb\ (The key.)  For any $n\in\Bbb N$ and Borel set
$H\subseteq\Bbb R$, $g_{nH}$ is a conditional expectation of
$\chi f_n^{-1}[H]$ on $\Tau_{n+1}$.
\Prf\ For $m$, $r\in\Bbb N$, let $h_{mr}:X\to[0,1]$ be a
$\Sigma_{mr}$-measurable function which is a conditional
expectation of $\chi f_n^{-1}[H]$ on $\Sigma_{mr}$;  for $m\in\Bbb N$,
set $h_m=\lim_{r\to\infty}h_{mr}$ where this is defined.
By L\'evy's martingale theorem (275I) $h_m$ is defined almost everywhere
and is a conditional expectation of $\chi f_n^{-1}[H]$ on $\Tau_m$.

For $m$, $r\in\Bbb N$, define $F_{mr}:X\to Z^{r+2}$ by setting
$F_{mr}(x)=(f_n(x),f_m(x),f_{m+1}(x),\ldots,\penalty-100f_{m+r}(x))$ for
$x\in X$.   At this point, examine the hypothesis (iii).   This implies
that if $m>n$ and $r\in\Bbb N$ then

$$\eqalign{\mu(F_{mr}^{-1}[H'\times H_0\times\ldots\times H_r])
&=\mu(f_n^{-1}[H']\cap\bigcap_{k\le r}f_{m+k}^{-1}[H_k])\cr
&=\mu(f_n^{-1}[H']\cap\bigcap_{k\le r}f_{m+1+k}^{-1}[H_k])\cr
&=\mu(F_{m+1,r}^{-1}[H'\times H_0\times\ldots\times H_r])\cr}$$

\noindent for all $H'$, $H_0,\ldots,H_r\in\Upsilon$.   By the Monotone
Class Theorem (136C), the image measures $\mu F_{mr}^{-1}$ and
$\mu F_{m+1,r}^{-1}$ agree on the $\sigma$-algebra
$\Tensorhat_{r+2}\Upsilon$ of subsets of $Z^{r+2}$ generated by
measurable cylinders;  set

\Centerline{$\lambda=\mu F_{mr}^{-1}\restr\Tensorhat_{r+2}\Upsilon
=\mu F_{m+1,r}^{-1}\restr\Tensorhat_{r+2}\Upsilon$.}

\noindent Let $\Lambda$ be the $\sigma$-subalgebra of
$\Tensorhat_{r+2}\Upsilon$ generated by sets of the form
$Z\times H_0\times\ldots\times H_r$ where $H_0,\ldots,H_r\in\Upsilon$,
and let $h$ be a conditional
expectation of $\chi(H\times Z^{r+1})$ on $\Lambda$ with respect to
$\lambda$.   Then 459A tells us that $hF_{mr}$ is a conditional
expectation of
$\chi(f_n^{-1}[H])$ on $\Sigma_{mr}$, and is therefore equal almost
everywhere to $h_{mr}$.
Similarly, $hF_{m+1,r}\eae h_{m+1,r}$, and this is true for every
$r\in\Bbb N$.   But as $F_{mr}$ and $F_{m+1,r}$
are both \imp\ for $\mu$ and $\lambda$, this means that $h_{mr}$, $h$
and $h_{m+1,r}$ all have the same distribution.
In particular, $\int h_{mr}^2d\mu=\int h_{m+1,r}^2d\mu$.   Now
$\sequence{r}{h_{mr}}$ and $\sequence{r}{h_{m+1,r}}$ converge almost
everywhere to $h_m$ and $h_{m+1}$ respectively, so

\Centerline{$\int h_m^2d\mu=\lim_{r\to\infty}\int h_{mr}^2d\mu
=\lim_{r\to\infty}\int h_{m+1,r}^2d\mu=\int h_{m+1}^2d\mu$.}

\noindent On the other hand, $\Tau_{m+1}\subseteq\Tau_m$, so
$h_{m+1}$ is a conditional expectation of
$h_m$ on $\Tau_{m+1}$ (233Eh).   This means that

\Centerline{$\int h_m\times h_{m+1}d\mu
=\int h_{m+1}\times h_{m+1}d\mu$}

\noindent (233Eg).   A direct calculation tells us that
$\int(h_m-h_{m+1})^2d\mu=0$, so that $h_m\eae h_{m+1}$.   Inducing on
$r$, we see that $h_m\eae h_r$  whenever $n<m\le r$.

Now the reverse martingale theorem (275K) tells us that
$\lim_{m\to\infty}h_m$ is
defined almost everywhere and is a conditional expectation of
$\chi f_n^{-1}[H]$ on $\Tau$, that is, is equal almost everywhere to
$g_{nH}$.   Since the $h_m$, for $m>n$, are equal almost everywhere,
they are all equal to $g_{nH}$ a.e.   In particular, $g_{nH}$ is equal
a.e.\ to $h_{n+1}$, and is a
conditional expectation of $\chi f_n^{-1}[H]$ on $\Tau_{n+1}$.\ \Qed

\medskip

\quad\grheadc\ If $n\in\Bbb N$ and $H_0,\ldots,H_r\in\Upsilon$, then
$\prod_{i=0}^rg_{n+i,H_i}$ is a conditional expectation of
$\chi(\bigcap_{i\le r}f_{n+i}^{-1}[H_i])$ on $\Tau$.   \Prf\ Induce on
$r$.   For $r=0$ this is just the definition of $g_{nH_0}$.   For
the inductive step to $r\ge 1$, observe that
$g_{nH_0}\times\prod_{i=1}^r\chi f_{n+i}^{-1}[H_i]$ is a
conditional expectation of $\prod_{i=0}^r\chi f_{n+i}^{-1}[H_i]$
on $\Tau_{n+1}$, by 233Eg or 233K, because $g_{nH_0}$ is a
conditional expectation of $\chi f_n^{-1}[H_0]$ on $\Tau_{n+1}$ and
$\prod_{i=1}^r\chi f_{n+i}^{-1}[H_i]$ is
$\Tau_{n+1}$-measurable.   But as (by the inductive hypothesis)
$\prod_{i=1}^rg_{n+i,H_i}$ is a conditional expectation of
$\prod_{i=1}^r\chi f_{n+i}^{-1}[H_i]$ on $\Tau$, while $g_{nH_0}$ is
$\Tau$-measurable,
$\prod_{i=0}^rg_{n+i,H_i}$ is a conditional expectation of
$\prod_{i=0}^r\chi f_{n+i}^{-1}[H_i]$ on $\Tau$, by 233Eg/233K
again.\ \Qed

\medskip

\quad\grheadd\ In particular, $\prod_{i=0}^rg_{iH_i}$ is a conditional
expectation of $\chi(\bigcap_{i\le r}f_i^{-1}[H_i])$ on $\Tau$ for every
$r\in\Bbb N$ and $H_0,\ldots,H_r\in\Upsilon$.   This
shows that $\sequencen{\Sigma_n}$ is relatively independent over $\Tau$.

\medskip

\quad\grheade\ Now consider part ($\beta$) of the condition (ii).   For
this, observe
that if $m>0$, $H\in\Upsilon$ and $H_i\in\Upsilon$ for $i\le r$, then

\Centerline{$\mu(f_0^{-1}[H]\cap\bigcap_{i\le r}f_{m+i+1}^{-1}[H_i])
=\mu(f_m^{-1}[H]\cap\bigcap_{i\le r}f_{m+i+1}^{-1}[H_i])$.}

\noindent By the Monotone Class Theorem,

\Centerline{$\mu(F\cap f_0^{-1}[H])=\mu(F\cap f_m^{-1}[H])$}

\noindent for any $F\in\Tau_{m+1}$ and in particular for any
$F\in\Tau$.
Thus (ii) is true.

\medskip

{\bf (c)} Now suppose that there is a strictly increasing sequence
$\sequence{k}{j_k}$ in $I$.   For each $n$, let $\Tau_n$ be the
$\sigma$-algebra generated by $\bigcup_{k\ge n}\Sigma_{j_k}$, and set
$\Tau=\bigcap_{n\in\Bbb N}\Tau_n$.   Then (b), applied to
$\sequence{k}{f_{j_k}}$, tells us that $\sequence{k}{\Sigma_{j_k}}$ is
relatively independent over $\Tau$ and that for each $H\in\Upsilon$
there is a function $g_H$ which is a conditional expectation of
$\chi(f_{j_k}^{-1}[H])$ on $\Tau$ for every $k\in\Bbb N$.

\medskip

\quad\grheada\ If $i_0,\ldots,i_r\in I$ are distinct and
$H_0,\ldots,H_r\in\Upsilon$, then
$\mu(\bigcap_{k\le r}f_{i_k}^{-1}[H_k])
=\mu(\bigcap_{k\le r}f_{j_k}^{-1}[H_k])$.   \Prf\ Let $\rho$ be the
permutation of $\{0,\ldots,r\}$ such that
$i_{\rho(0)}<i_{\rho(1)}<\ldots<i_{\rho(r)}$.   Then

$$\eqalign{\mu(\bigcap_{k\le r}f_{i_k}^{-1}[H_k])
&=\mu(\bigcap_{k\le r}f_{i_{\rho(k)}}^{-1}[H_{\rho(k)}])
=\mu(\bigcap_{k\le r}f_{j_k}^{-1}[H_{\rho(k)}])\cr
&=\int(\prod_{k=0}^rg_{H_{\rho(k)}})d\mu
=\int(\prod_{k=0}^rg_{H_k})d\mu
=\mu(\bigcap_{k\le r}f_{j_k}^{-1}[H_k]).  \text{ \Qed}\cr}$$

\medskip

\quad\grheadb\ Now suppose that $i_0,\ldots,i_r\in I$ are distinct.
Then there is some $m\in\Bbb N$ such that
$j_k\ne i_l$ for any $l\le r$ and $k\ge m$.   In this case, consider the
sequence $f_{i_0},\ldots,f_{i_r},f_{j_m},f_{j_{m+1}},\ldots$.   By
($\alpha$) here, this sequence satisfies the condition (iii).   We can
therefore apply the construction of (b).   But observe that the tail
$\sigma$-algebra obtained from $f_{i_0},\ldots,f_{i_r},f_{j_m},\ldots$
is precisely $\Tau$, as defined from $\sequence{k}{f_{j_k}}$ just above.
So $\langle\Sigma_{i_k}\rangle_{k\le r}$ is relatively independent over
$\Tau$.   As $i_0,\ldots,i_r$ are arbitrary, $\familyiI{\Sigma_i}$ is
relatively independent over $\Tau$.   At the same time we see that if
$H\in\Upsilon$ then all the $\chi(f_{i_k}^{-1}[H])$ have the same
conditional expectations over $\Tau$ as $\chi(f_{j_m}^{-1}[H])$.   So
(ii-$\beta$) is satisfied.

\medskip

{\bf (d)} Finally, if there is no strictly increasing sequence in $I$,
then $(I,\ge)$ is well-ordered;  since $I$ is infinite, the
well-ordering starts with an initial segment of order type $\omega$, that is,
a sequence $\sequence{k}{j_k}$ such that $j_0>j_1>\ldots$.   But note
now that (iii) tells us that
$\mu(\bigcap_{k\le r}f_{i_k}^{-1}[H_k])
=\mu(\bigcap_{k\le r}f_{j_k}^{-1}[H_k])$ whenever
$i_0>\ldots>i_r$ and $j_0>\ldots>j_r$ and $H_k\in\Upsilon$ for every
$k$.   So we can apply (c) to
$(I,\ge)$ to get the result in this case also.
}%end of proof of 459B

\vleader{72pt}{459C}{Exchangeable random \dvrocolon{variables}}\cmmnt{ I
spell out the leading special case of this theorem.

\medskip

\noindent}{\bf De Finetti's theorem} Let $(X,\Sigma,\mu)$ be
a probability space, and $\familyiI{f_i}$ an infinite family in
$\eusm L^0(\mu)$.   Then the following are equiveridical:

\inset{(i) the joint distribution of $(f_{i_0},f_{i_1},\ldots,f_{i_r})$
is the same as the joint distribution of
$(f_{j_0},f_{j_1},\ldots,f_{j_r})$
whenever $i_0,\ldots,i_r\in I$ are distinct and $j_0,\ldots,j_r\in I$
are distinct;
%overfull hbox reported;  no problem

(ii) there is a $\sigma$-subalgebra $\Tau$ of $\Sigma$ such that
$\familyiI{f_i}$ is relatively independent over $\Tau$ and all the
$f_i$ have the same relative distributions over $\Tau$.}

\noindent Moreover, if $I$ is totally ordered by $\le$, we can add

\inset{(iii) the joint distribution of
$(f_{i_0},f_{i_1},\ldots,f_{i_r})$ is the
same as the joint distribution of $(f_{j_0},f_{j_1},\ldots,f_{j_r})$
whenever $i_0<\ldots<i_r$ and $j_0<\ldots<j_r$ in $I$.}
%overfull hbox reported;  no problem

\cmmnt{\medskip

\noindent{\bf Remark} Families of random variables
satisfying the condition in (i) are called {\bf exchangeable}.   The
equivalence of (i) and (ii) can be expressed by
saying that `an exchangeable family of random variables is a mixture
of independent identically distributed families'.
}%end of comment

\proof{ Changing each $f_i$ on a negligible set will
not change either their joint distributions (271De) or their relative
distributions over $\Tau$ or their relative independence;   so we may
suppose that every $f_i$ is a $\Sigma$-measurable
function from $X$ to $\Bbb R$.   Now look at 459B, taking
$(Z,\Upsilon)$ to be $\Bbb R$ with its Borel $\sigma$-algebra.   The
condition 459B(i) reads

\inset{\noindent whenever $i_0,\ldots,i_r\in I$ are distinct,
$j_0,\ldots,j_r\in I$ are distinct, and $H_k\in\Upsilon$ for each
$k\le r$, then
$\mu(\bigcap_{k\le r}f_{i_k}^{-1}[H_k])
=\mu(\bigcap_{k\le r}f_{j_k}^{-1}[H_k])$,}

\noindent matching (i) here, by 271B;  similarly, (iii) of 459B
matches (iii) here.   Equally, condition (ii) here is just a
re-phrasing of 459B(ii) in the language of 458A and 458I.
So 459B gives the result.
}%end of proof of 459C

\leader{459D}{}\cmmnt{ Specializing 459B in another direction, we
have the case in which $X$ is actually the product $Z^I$.   In this
case, the condition 459B(i) corresponds to a strong kind of symmetry in
the measure $\mu$.   It now makes sense to look for subsets of $X=Z^I$
which are essentially invariant under permutations, and we have the
following result.

\medskip

\noindent}{\bf Proposition} Let $Z$ be a set, $\Upsilon$ a
$\sigma$-algebra of subsets of $Z$, $I$ an infinite set and $\mu$ a
measure on $Z^I$ with domain the $\sigma$-algebra $\Tensorhat_I\Upsilon$
generated by
$\{\pi_i^{-1}[H]:i\in I$, $H\in\Upsilon\}$, taking $\pi_i(x)=x(i)$ for
$x\in Z^I$ and $i\in I$.   For each permutation $\rho$ of $I$, define
$\hat\rho:Z^I\to Z^I$ by setting $\hat\rho(x)=x\rho$ for $x\in Z^I$.
Suppose that $\mu=\mu\hat\rho^{-1}$ for every $\rho$.   Let $\Cal E$ be
the family of those sets $E\in\Tensorhat_I\Upsilon$ such that
$\mu(E\symmdiff\hat\rho^{-1}[E])=0$ for every permutation $\rho$ of $I$,
and $\Cal V$ the family of those sets $V\in\Tensorhat_I\Upsilon$ such
that $V$ is determined by coordinates in $I\setminus\{i\}$ for every
$i\in I$.

(a) $\Cal E$ is a $\sigma$-subalgebra of $\Tensorhat_I\Upsilon$.

(b) $\Cal V$ is a $\sigma$-subalgebra of $\Cal E$.

(c) If $E\in\Cal E$ and $J\subseteq I$ is infinite, then there is a
$V\in\Cal V$, determined by coordinates in $J$, such that
$\mu(E\symmdiff V)=0$.

(d) Setting $\Sigma_i=\{\pi_i^{-1}[H]:H\in\Upsilon\}$ for each $i\in I$,

\inset{($\alpha$) $\familyiI{\Sigma_i}$ is relatively independent over
$\Cal E$,

($\beta$) for every $H\in\Upsilon$ there is an $\Cal E$-measurable
function $g_H:Z^I\to[0,1]$ which is a conditional expectation of
$\chi(\pi_i^{-1}[H])$ on $\Cal E$ for every $i\in I$.}

\proof{{\bf (a)} is elementary.

\medskip

{\bf (b)} Let $V\in\Cal V$.   Suppose that $\rho:I\to I$ is a
permutation, $J\subseteq I$ is finite and
$H_j\in\Upsilon$ for
every $j\in J$.   Then there is a permutation $\sigma:I\to I$ such that
$\sigma(j)=\rho(j)$ for every $j\in J$ and $J'=\{i:\sigma(i)\ne i\}$
is finite.   By 254Ta, $V$ is determined by coordinates in
$I\setminus J'$, so $\hat\sigma^{-1}[V]=V$.   Now

$$\eqalign{\mu(\hat\rho^{-1}[V]\cap\bigcap_{j\in J}\pi_j^{-1}[H_j])
&=\mu(V\cap\bigcap_{j\in J}\pi_{\rho(j)}^{-1}[H_j])
=\mu(V\cap\bigcap_{j\in J}\pi_{\sigma(j)}^{-1}[H_j])\cr
&=\mu(\hat\sigma^{-1}[V]\cap\bigcap_{j\in J}\pi_j^{-1}[H_j])
=\mu(V\cap\bigcap_{j\in J}\pi_j^{-1}[H_j]).\cr}$$

\noindent By the Monotone Class Theorem, as usual,
$\mu(E\cap\hat\rho^{-1}[V])=\mu(E\cap V)$ for every
$E\in\Tensorhat_I\Upsilon$.   In particular, taking $E=V$ and $E=Z^I\setminus V$, we see that
$V\symmdiff\hat\rho^{-1}[V]$ is negligible.   As $\rho$ is arbitrary,
$V\in\Cal E$.

This shows that $\Cal V\subseteq\Cal E$.   Of course $\Cal V$ is a
$\sigma$-algebra, since it is just the intersection of the
$\sigma$-algebras $\{V:V\in\Tensorhat_I\Sigma$, $V$ is determined by
coordinates in $I\setminus\{i\}\}$.

\medskip

{\bf (c)} For each $n\in\Bbb N$, there is a set
$E_n\in\Tensorhat_I\Sigma$, determined by a finite set $J_n$ of
coordinates, such that $\mu(E\symmdiff E_n)\le 2^{-n}$.   Choose
permutations $\rho_n$ of $I$ such that $\sequencen{\rho_n[J_n]}$ is a
disjoint sequence of subsets of $J$.   Set $F_n=\hat\rho_n^{-1}[E_n]$;
then $F_n$ is determined by coordinates in $\rho_n[J_n]$ for each
$n\in\Bbb N$, so $V=\bigcap_{n\in\Bbb N}\bigcup_{m\ge n}F_m$ belongs to
$\Cal V$ and is determined by coordinates in $J$.   Also

\Centerline{$\mu(E\symmdiff F_n)=\mu(\hat\rho[E]\symmdiff E_n)
=\mu(E\symmdiff E_n)\le 2^{-n}$}

\noindent for each $n$, so $\mu(E\symmdiff V)=0$, as required.

\medskip

{\bf (d)} Let $\sequencen{j_n}$ be any sequence of distinct points of
$I$.   Set $J=\{j_n:n\in\Bbb N\}$.   For $n\in\Bbb N$ let $\Tau_n$ be
the $\sigma$-algebra generated by $\bigcup_{k\ge n}\Sigma_{j_k}$, and
set $\Tau=\bigcap_{n\in\Bbb N}\Tau_n$, so that $\Tau=\{V:V\in\Cal V$,
$V$ is determined by coordinates in $J\}$.   \Prf\ Of course
$\Tau\subseteq\Cal V$ and every member of $\Tau$ is determined by
coordinates in $J$, because every member of $\Tau_0$ is.   On the other hand, if $V\in\Cal V$ is determined by coordinates in $J$, then
fix some $w\in Z^{I\setminus J}$.   In this case, identifying $Z^I$ with
$Z^J\times Z^{I\setminus J}$, the set $V_1=\{z:z\in Z^J$, $(z,w)\in V\}$
must belong to $\Tensorhat_J\Upsilon$, so $V=V_1\times Z^{I\setminus J}$
belongs to $\Tau_0$.   Applying the same idea to
$J\setminus\{j_k:k<n\}$, we see that $V\in\Tau_n$ for every $n$, so that
$V\in\Tau$.\ \Qed

Part (c) of the proof of 459B tells us that $\familyiI{\Sigma_i}$ is
relatively independent over $\Tau$ and that for every $H\in\Upsilon$
there is a $\Tau$-measurable $g_H$ which is a conditional expectation of
$\chi(\pi_i^{-1}[H])$ on $\Tau$ for every $i\in I$.   Now (c) here tells
us that $g_H$ is a conditional expectation of $\chi(\pi_i^{-1}[H])$ on
$\Cal E$;  and examining the definition in 458Aa, we see that
$\familyiI{\Sigma_i}$ is relatively independent over $\Cal E$, as
claimed.
}%end of proof of 459D

\leader{459E}{}\cmmnt{ If $\mu$ is countably
compact, we have a strong disintegration theorem, as follows.

\medskip

\noindent}{\bf Theorem} Let $Z$ be a set, $\Upsilon$ a
$\sigma$-algebra of subsets of $Z$, $I$ an infinite set, and $\mu$ a
countably compact probability measure on $Z^I$ with domain the
$\sigma$-algebra $\Tensorhat_I\Upsilon$\cmmnt{ generated by
$\{\pi_i^{-1}[H]:i\in I$, $H\in\Upsilon\}$, taking $\pi_i(x)=x(i)$ for
$x\in Z^I$ and $i\in I$}.   Then the following are equiveridical:

\inset{(i) for every permutation $\rho$ of $I$,
$x\mapsto x\rho:Z^I\to Z^I$ is \imp\ for $\mu$;

(ii) for every transposition $\rho$ of two elements of $I$,
$x\mapsto x\rho:Z^I\to Z^I$ is \imp\ for $\mu$;

(iii) for each $n\in\Bbb N$ and any two injective functions $p$,
$q:n\to I$ the maps $x\mapsto xp:Z^I\to Z^n$, $x\mapsto xq:Z^I\to Z^n$
induce the same measure on $Z^n$;

(iv) there are a probability space $(Y,\Tau,\nu)$ and a family
$\family{y}{Y}{\lambda_y}$ of probability measures on $Z$ such that
$\family{y}{Y}{\lambda_y^I}$ is a disintegration of $\mu$ over $\nu$,
writing $\lambda_y^I$ for the product of copies of $\lambda_y$.}

\noindent Moreover, if $I$ is totally ordered, we can add

\inset{(v) for each $n\in\Bbb N$ and any two strictly increasing
functions $p$, $q:n\to I$ the maps $x\mapsto xp:Z^I\to Z^n$,
$x\mapsto xq:Z^I\to Z^n$ induce the same measure on $Z^n$.}

\noindent If the conditions (i)-(v) are satisfied, then there is a
countably compact measure $\lambda$, with domain $\Upsilon$, which is
the common marginal measure of $\mu$ on every coordinate;  and
if $\Cal K$ is a countably compact class of subsets of $Z$, closed under
finite unions and countable intersections, such that $\lambda$ is inner
regular with respect to $\Cal K$, then

\inset{(iv)$'$ there are a probability space $(Y,\Tau,\nu)$ and a family
$\family{y}{Y}{\lambda_y}$ of complete probability measures on $Z$, all
with domains including $\Cal K$ and inner regular with respect to
$\Cal K$, such that $\family{y}{Y}{\lambda_y^I}$ is a disintegration of
$\mu$ over $\nu$.}

\proof{{\bf (a)} Since any set $I$ can be totally ordered, we may
suppose from the outset that we have been given a total ordering $\le$
of $I$.   I start with the easy bits.

\medskip

\quad{\bf (iv)$'\mskip-5mu\Rightarrow$(iv)} is trivial, at least if
there is a common countably compact marginal measure on $Z$.

\medskip

\quad{\bf (iv)$\Rightarrow$(i)} If (iv) is true and $\rho:I\to I$ is a
permutation, take any $E\in\Tensorhat_I\Sigma$ and set
$E'=\{x:x\in Z^I$, $x\rho\in E\}$.   For any $y\in Y$, $x\mapsto x\rho$
is an isomorphism of the measure space $(Z^I,\lambda_y^I)$, so

\Centerline{$\mu E'=\int\lambda_y^IE'\,\nu(dy)
=\int\lambda_y^IE\,\nu(dy)=\mu E$.}

\noindent As $E$ is arbitrary, (i) is true.

\medskip

\quad{\bf (i)$\Rightarrow$(ii)} is trivial.

\medskip

\quad{\bf (ii)$\Rightarrow$(iii)} There is a permutation $\rho$ of $I$
such that $q=\rho p$ and $\rho$ moves only finitely many points of $I$,
that is, $\rho$ is a product of transpositions.   By (ii),
$x\mapsto x\rho$ and $x\mapsto x\rho^{-1}$ are \imp\ for $\mu$, that is, are isomorphisms of $(Z^I,\mu)$.   But this means that
$x\mapsto xp$ and $x\mapsto x\rho p=xq$ must induce the same measure on
$Z^n$.

\medskip

\quad{\bf (iii)$\Rightarrow$(v)} is trivial.

\medskip

{\bf (b)} So for the rest of the proof I assume that (v) is true.
Taking $n=1$ in the statement of (v), we see that there is a common
image measure $\lambda=\mu\pi_i^{-1}$ for every $i\in I$.   By 452R,
$\lambda$ is countably compact.   Let $\Cal K\subseteq\Cal PZ$ be a
countably compact class, closed under finite unions and countable
intersections, such that $\lambda$ is inner regular with respect to
$\Cal K$.

In 459B, set $X=Z^I$ and $\Sigma=\Tensorhat_I\Upsilon$ and
$f_i=\pi_i:X\to Z$ for $i\in I$.   Then (v) here corresponds to (iii) of
459B, so (translating (ii) of 459B) we have a $\sigma$-subalgebra
$\Tau$ of $\Tensorhat_I\Upsilon$ and a family
$\family{H}{\Upsilon}{g_H}$ of $\Tau$-measurable functions from $Z^I$ to
$[0,1]$ such that

\Centerline{$\mu(\bigcap_{i\in J}\pi_i^{-1}[H_i])
=\int(\prod_{i\in J}g_{H_i})d\mu$}

\noindent whenever $J\subseteq I$ is finite and not empty and
$H_i\in\Upsilon$ for $i\in J$.
In particular, $g_H$ is a conditional expectation of
$\chi(\pi_i^{-1}[H])$ on $\Tau$ whenever $H\in\Upsilon$ and $i\in I$.

Fix $i^*\in I$ for the moment.   Set $\nu=\mu\restr\Tau$.
The \imp\ function $\pi_{i^*}$ from $(X,\mu)$ to $(Z,\lambda)$ gives us
an integral-preserving Riesz homomorphism
$T_0:L^{\infty}(\lambda)\to L^{\infty}(\mu)$ defined by setting
$T_0h^{\ssbullet}=(h\pi_{i^*})^{\ssbullet}$ for every
$h\in\eusm L^{\infty}(\lambda)$.   Let $P:L^1(\mu)\to L^1(\nu)$ be the
conditional expectation operator;  then
$T=PT_0:L^{\infty}(\lambda)\to L^{\infty}(\nu)$ is an
integral-preserving positive linear operator, and
$T(\chi Z^{\ssbullet})=\chi X^{\ssbullet}$.

By 452H, we have a family $\family{x}{X}{\lambda_x}$ of complete
probability measures on $Z$, all with domains including $\Cal K$ and
inner regular with respect to $\Cal K$, such that
$\int_Fh\pi_{i^*}d\mu=\int_F\int_Zh\,d\lambda_x\nu(dx)$ for every
$h\in\eusm L^{\infty}(\lambda)$ and $F\in\Tau$.   In particular, setting
$g'_H(x)=\lambda_xH$ whenever $H\in\Upsilon$ and $x\in X$ are such that
$H\in\dom\lambda_x$, then $g'_H$ will be a conditional expectation of
$\chi\pi_{i^*}^{-1}[H]$ on $\Tau$, and will be equal $\nu$-almost
everywhere to $g_H$.

This means that if $J\subseteq I$ is finite and not empty and
$H_i\in\Upsilon$ for $i\in J$,

$$\eqalign{\int_X\lambda_x^I(\bigcap_{i\in J}\pi_i^{-1}[H_i])\nu(dx)
&=\int_X\prod_{i\in J}\lambda_xH_i\,\nu(dx)
=\int_X\prod_{i\in J}g'_{H_i}d\nu\cr
&=\int_X\prod_{i\in J}g_{H_i}d\nu
=\int_X\prod_{i\in J}g_{H_i}d\mu
=\mu(\bigcap_{i\in J}\pi_i^{-1}[H_i]).\cr}$$

\noindent Thus the family $\Cal W$ of sets $E\subseteq X$ such that
$\int\lambda_x^IE\,\nu(dx)$ and $\mu E$ are defined and equal contains
all measurable cylinders.   As $\Cal W$ is a Dynkin class it includes
$\Tensorhat_I\Upsilon$.   But this says exactly that
$\family{x}{X}{\lambda_x^I}$ is a disintegration of $\mu$ over $\nu$, as
required by (iv)$'$.

Thus (v)$\Rightarrow$(iv)$'$ and the proof is complete.
}%end of proof of 459E

\leader{459F}{Lemma}\dvAnew{2010}
Let $X$ be a Hausdorff space and $P_{\text{R}}(X)$
the space of Radon probability measures on $X$ with its narrow
topology\cmmnt{ (definition:  437Jd)}.
If $\sequencen{K_n}$ is a disjoint sequence of compact subsets of $X$,
then $A=\{\mu:\mu\in P_R(X)$, $\mu(\bigcup_{n\in\Bbb N}K_n)=1\}$ is a
K-analytic subset of $P_R(X)$.

\proof{ (Recall that $P_{\text{R}}(X)$ is Hausdorff, by 437R(a-ii).)
For each $n\in\Bbb N$, let $C_n$ be the set of Radon measures on
$K_n$ with magnitude at most $1$;  by 437R(f-ii), $C_n$ is compact in its
narrow topology.   Let $C$ be the compact space $\prod_{n\in\Bbb N}C_n$;
for the rest of this
proof, I will use the formula $\pmb{\mu}=\sequencen{\mu_n}$ to describe
the coordinates of members of $C$.   Define
$\psi:C\to[0,1]^{\Bbb N}$ by setting $\psi(\pmb{\mu})(n)=\mu_nK_n$ for
$n\in\Bbb N$ and $\pmb{\mu}\in C$.
Then $\psi$ is continuous.   Since
$B=\{\sequencen{\alpha_n}:\sum_{n=0}^{\infty}\alpha_n=1\}$ is a
Borel subset
of $[0,1]^{\Bbb N}$, therefore a Baire set (4A3Kb),
$D=\psi^{-1}[B]$ is a Baire subset of $C$ (4A3Kc), therefore
Souslin-F (421L) and K-analytic (422Hb).

For $\pmb{\mu}\in D$, define a function $\phi(\pmb{\mu})$ by saying that

\Centerline{$\phi(\pmb{\mu})(E)
=\sum_{n=0}^{\infty}\mu_n(E\cap K_n)\text{ if }E\subseteq X\text{ and }
  \mu_n\text{ measures }E\cap K_n\text{ for every }n$}

\noindent and is undefined otherwise.   It is easy to check that
$\phi(\pmb{\mu})\in P_{\text{R}}(X)$.   Also $\phi:D\to P_{\text{R}}(X)$
is continuous.   \Prf\ If $G\subseteq X$ is open, then
$\nu\mapsto\nu(G\cap K_n):C_n\to[0,1]$ and therefore
$\pmb{\mu}\mapsto\mu_n(G\cap K_n):D\to[0,1]$ are lower semi-continuous for
each $n$ (4A2B(d-ii)), so
$\pmb{\mu}\mapsto\phi(\pmb{\mu})(G)$ is lower semi-continuous (4A2B(d-iii),
4A2B(d-v)), and $\{\pmb{\mu}:\phi(\pmb{\mu})(G)>\alpha\}$ is open for every
$\alpha$;  by 4A2B(a-iii), $\phi$ is continuous.\ \Qed

Consequently $A=\phi[D]$ is K-analytic (422Gd).
}%end of proof of 459F

\leader{459G}{Lemma}\dvAnew{2010}
Let $X$ be a topological space,
$(Y,\frak S,\Tau,\nu)$ a totally finite quasi-Radon measure space,
$y\mapsto\mu_y$ a continuous function from $Y$ to the space
$M^+_{\text{qR}}(X)$ of totally finite quasi-Radon measures on $X$ with its
narrow topology, and $\Cal U$ a base for the topology of $X$,
containing $X$ and closed under finite intersections.   If
$\mu\in M^+_{\text{qR}}(X)$ is such that $\mu U=\int\mu_yU\,\nu(dy)$ for
every $U\in\Cal U$, then $\family{y}{Y}{\mu_y}$ is a disintegration of
$\mu$ over $\nu$.

\proof{{\bf (a)} Let $\Cal E$ be the family of subsets $E$ of $X$ such that
$\mu E$ and $\int\mu_yE\,\nu(dy)$ are defined and equal.   Because
$X\in\Cal E$, $\Cal E$ is
a Dynkin class;  as $\Cal U$ is included in $\Cal E$ and is closed under
finite intersections, the $\sigma$-algebra of sets generated by $\Cal U$ is
included in $\Cal E$, and in particular any finite union of members of
$\Cal U$ belongs to $\Cal E$.

\medskip

{\bf (b)} In fact every open subset of $X$ belongs to $\Cal E$.   \Prf\ If
$G\subseteq X$ is open, set
$\Cal H=\{H:H\subseteq G$ is a finite union of members of $\Cal U\}$.
Then $\Cal H$ is upwards-directed and has union $G$.   Set
$f_H(y)=\mu_yH$ for $y\in Y$ and $H\in\Cal H$.   Since
$\lambda\mapsto\lambda H:M^+_{\text{qR}}(X)\to\Bbb R$ is lower
semi-continuous (by the definition of the narrow topology) and
$y\mapsto\mu_y$ is continuous, $f_H:Y\to\Bbb R$ is lower semi-continuous
(4A2B(d-ii) again).   
Moreover, $\{f_H:H\in\Cal H\}$ is an upwards-directed
family of functions with supremum $f_G$, where $f_G(y)=\mu_yG$ for each
$y$, because every $\mu_y$ is $\tau$-additive.   Now

$$\eqalignno{\mu G
&=\sup_{H\in\Cal H}\mu H=\sup_{H\in\Cal H}\int f_Hd\nu=\int f_Gd\nu\cr
\displaycause{414Ba}
&=\int\mu_yG\,\nu(dy)\cr}$$

\noindent and $G\in\Cal E$.\ \Qed

\medskip

{\bf (c)} It follows that every Borel subset of $X$ belongs to $\Cal E$,
that is, that $\family{y}{Y}{\mu_y}$ is a disintegration of the restriction
$\mu_{\Cal B}$ to the Borel $\sigma$-algebra of $X$.   Since every $\mu_y$
is complete, $\family{y}{Y}{\mu_y}$ is also a disintegration over
$\nu$ of the completion of $\mu_{\Cal B}$ (452B(a-ii)), which is
$\mu$.
}%end of proof of 459G

\leader{459H}{Theorem} Let $Z$ be a Hausdorff space, $I$ an infinite
set, and $\tilde\mu$ a quasi-Radon probability measure on $Z^I$ such
that the marginal measures on each copy of $Z$ are Radon measures.
%is this the right way of putting it?  we just need a filling sequence
%of compact sets.
Write $P_{\text{R}}(Z)$ for the set of Radon probability measures on $Z$
with its narrow topology.   Then the following are equiveridical:

\inset{(i) for every permutation $\rho$ of $I$,
$w\mapsto w\rho:Z^I\to Z^I$ is \imp\ for $\tilde\mu$;

(ii) for every transposition $\rho$ of two elements of $I$,
$w\mapsto w\rho:Z^I\to Z^I$ is \imp\ for $\tilde\mu$;

(iii) for each $n\in\Bbb N$ and any two injective functions $p$,
$q:n\to I$ the maps $w\mapsto wp:Z^I\to Z^n$ and
$w\mapsto wq:Z^I\to Z^n$ induce the same measure on $Z^n$;

(iv) there are a probability space $(Y,\Tau,\nu)$ and a family
$\family{y}{Y}{\mu_y}$ of $\tau$-additive Borel probability measures on
$Z$ such that $\family{y}{Y}{\tilde\mu_y^I}$ is a disintegration of
$\tilde\mu$ over $\nu$, writing $\tilde\mu_y^I$ for the
$\tau$-additive product of copies of $\mu_y$;

(v) there is a Radon probability measure $\tilde\nu$ on
$P_{\text{R}}(Z)$ such that
$\family{\theta}{P_{\text{R}}(Z)}{\tilde\theta^I}$ is disintegration of
$\tilde\mu$ over $\tilde\nu$, writing $\tilde\theta^I$ for the
quasi-Radon product of copies of $\theta$.}

\noindent Moreover, if $I$ is totally ordered, we can add

\inset{(vi) for each $n\in\Bbb N$ and any two strictly increasing
functions $p$, $q:n\to I$ the maps $w\mapsto wp:Z^I\to Z^n$ and
$w\mapsto wq:Z^I\to Z^n$ induce the same measure on $Z^n$.}

\proof{{\bf (a)} As in 459E, we need consider only the case in which
$I$ is totally ordered, and the implications

\Centerline{(v)$\Rightarrow$(iv)$\Rightarrow$(i)$
\Rightarrow$(ii)$\Rightarrow$(iii)$\Rightarrow$(vi)}

\noindent are elementary.   So henceforth I will suppose that (vi) is
true and seek to prove (v).

\medskip

{\bf (b)} We are going to need a second topology on the set $Z$, so I will
call the original topology $\frak T$, and for the rest of this proof
I will declare the
topology on which each topological concept or construction is based.
Write $\mu$ for
$\tilde\mu\restr\Tensorhat_I\Cal B(Z,\frak T)$, where $\Cal B(Z,\frak T)$
is the Borel $\sigma$-algebra of $Z$ for the topology $\frak T$.
Then (vi) is also true of $\mu$.
(Strictly speaking, we ought to check that the different images of
$\mu$ all have the same domain.   But this is true, because the
image of $\mu$ corresponding to a strictly increasing function
$p:r\to I$ has domain $\Tensorhat_r\Cal B(Z,\frak T)$.)
The (unique) marginal
measure $\lambda$ of $\mu$ is the restriction to $\Cal B(Z,\frak T)$ of
the $\frak T$-Radon measure $\tilde\lambda$ which is the marginal of
$\tilde\mu$, so is a $\frak T$-tight $\frak T$-Borel
measure, therefore countably compact.   By 454A(b-ii), $\mu$ is countably
compact.   So 459E, with $\Cal K$ the family of $\frak T$-compact subsets
of $Z$,
tells us that there are a probability space $(Y_0,\Tau_0,\nu_0)$ and a
family $\family{y}{Y_0}{\mu_y}$ in $P_{\text{R}}(Z,\frak T)$ such that
$\family{y}{Y_0}{\mu_y^I}$ is a disintegration of $\mu$ over
$\nu_0$, writing $\mu_y^I$ for the ordinary product of copies of
$\mu_y$.   We can of course suppose that $\nu_0$ is complete.
Note also that $\family{y}{Y_0}{\mu_y}$ is a
disintegration of $\lambda$;  this is clearly achieved by the proof of
459E, and it is necessarily true if $\family{y}{Y_0}{\mu_y^I}$ is to be
a disintegration of $\mu$.   Because every $\mu_y$ is complete,
$\family{y}{Y_0}{\mu_y}$ is also a
disintegration of the completion $\tilde\lambda$ of $\lambda$.

\medskip

{\bf (c)}
Let $\sequencen{K_n}$ be a disjoint sequence of $\frak T$-compact
subsets of $Z$ such that $\sum_{n=0}^{\infty}\tilde\lambda K_n=1$
(412Aa).   Let $\frak S$ be

\Centerline{$\{H:H\subseteq Z$, $Z\setminus(H\cap K_n)\in\frak T$ for every
$n\in\Bbb N\}$.}

\noindent Then $\frak S$ is a locally compact topology on $Z$ finer than
$\frak T$.   (If you like, $\frak S$ is the disjoint union topology
corresponding to the partition
$\{K_n:n\in\Bbb N\}\cup\{\{z\}:z\in Z\setminus\bigcup_{n\in\Bbb N}K_n\}$.)
Note that the subspace topologies on any $K_n$ induced by $\frak S$ and
$\frak T$ are the same, so that a $\frak T$-compact subset of $K_n$ is
$\frak S$-compact.   Because $\frak S$ is finer than $\frak T$,
$P_{\text{R}}(Z,\frak S)\subseteq P_{\text{R}}(Z,\frak T)$ (use 418I).
If $\theta\in P_{\text{R}}(Z,\frak T)$ and
$\theta(\bigcup_{n\in\Bbb N}K_n)=1$,
then, from the standpoint of the topology $\frak S$, $\theta$ is a
complete topological probability
measure inner regular with respect to the compact sets, so belongs to
$P_{\text{R}}(Z,\frak S)$.   In particular,
$\tilde\lambda\in P_{\text{R}}(Z,\frak S)$.

We shall need to know that the family $\Cal V$
of $\frak T$-Borel $\frak S$-cozero subsets of $Z$
is a base for $\frak S$.   \Prf\ If $z\in H\in\frak S$, then if
$z\notin\bigcup_{n\in\Bbb N}K_n$ the singleton $\{z\}$ belongs to
$\Cal V$.   If $n\in\Bbb N$ and $z\in K_n$, then $H\cap K_n\in\frak S$;
as $\frak S$ is locally compact, there is an $\frak S$-cozero set $G$ such
that $z\in G\subseteq H\cap K_n$, and now $G$ is $\frak T$-relatively open
in the $\frak T$-compact set $K_n$, so $G$ is $\frak T$-Borel.\ \QeD\

\medskip

{\bf (d)} We know that

\Centerline{$\int\mu_y(\bigcup_{n\in\Bbb N}K_n)\nu_0(dy)
=\tilde\lambda(\bigcup_{n\in\Bbb N}K_n)=1$;}

\noindent since $\mu_yZ=1$ for every $y$, the set
$Y=\{y:y\in Y_0,\,\mu_y(\bigcup_{n\in\Bbb N}K_n)=1\}$ must be
$\nu_0$-conegligible.   Let $\nu$ be the subspace measure induced by
$\nu_0$ on $Y$.   Then $\family{y}{Y}{\mu_y}$ is a disintegration of
$\tilde\lambda$ over $\nu$, and $\mu_y\in P_{\text{R}}(Z,\frak S)$ for
every $y\in Y$, by (c).

\medskip

{\bf (e)} By 459F, the set

\Centerline{$A=\{\theta:\theta\in P_{\text{R}}(Z,\frak S)$,
$\theta(\bigcup_{n\in\Bbb N}K_n=1\}$}

\noindent is K-analytic in its narrow
topology, while $\mu_y\in A$ for every $y\in Y$.
If $G\in\Cal V$ and $\alpha>0$,
$\{y:y\in Y_0$, $\mu_yG>\alpha\}\in\Tau_0$, so
$\{y:y\in Y$, $\mu_yG>\alpha\}$ is measured by $\nu$.   By 432I,
applied to the map $y\mapsto\mu_y:Y\to A$, there is a
Radon probability measure $\tilde\nu_A$ on $A$ such that

\Centerline{$\int h\,d\tilde\nu_A=\int h(\mu_y)\nu(dy)$}

\noindent for every bounded continuous $h:A\to\Bbb R$.

\medskip

{\bf (f)}
Now suppose that $f:Z\to\Bbb R$ is bounded and $\frak S$-continuous.
Then $\theta\mapsto\int fd\theta:P_{\text{R}}(Z,\frak S)\to\Bbb R$ is
continuous (437K), so that

\Centerline{$\iint fd\theta\,\tilde\nu_A(d\theta)
=\iint fd\mu_y\nu(dy)$.}

\noindent If $G\in\Cal V$, there is a
non-decreasing sequence $\sequencen{f_n}$ of non-negative
$\frak S$-continuous functions with supremum $\chi G$, so

$$\eqalign{\int\theta G\,\tilde\nu_A(d\theta)
&=\sup_{n\in\Bbb N}\iint f_nd\theta\,\tilde\nu_A(d\theta)
=\sup_{n\in\Bbb N}\iint f_nd\mu_y\nu(dy)\cr
&=\int\mu_yG\,\nu(dy)
=\lambda G
=\tilde\lambda G.\cr}$$

\noindent So we can apply 459G to the identity map from
$A$ to itself and the family
$\family{\theta}{A}{\theta}$ to see that
$\family{\theta}{A}{\theta}$ is a disintegration of
$\tilde\lambda$ over $\tilde\nu_A$.

It follows that if $E\subseteq Z$ is $\tilde\lambda$-negligible, then
$\theta E=0$ for $\tilde\nu_A$-almost every $\theta$.
Moreover, since $\family{y}{Y}{\mu_y}$ is a disintegration of
$\tilde\lambda$ over $\nu$, $\mu_yE=0$ for $\nu$-almost every $y$.

\medskip

{\bf (g)} If $J\subseteq I$ is finite, $G_j\in\frak T$ for
$j\in J$,  and $W=\{w:w\in Z^I$, $w(j)\in G_j$ for $j\in J\}$, then

\Centerline{$\tilde\mu W=\int\theta^IW\tilde\nu_A(d\theta)$.}

\Prf\ Because $\Cal V$ is a base for $\frak S$ closed under countable
unions, and $\tilde\lambda$ is $\frak S$-Radon, there is for each $j\in J$
a $G'_j\in\Cal V$, included in $G_j$, such that
$\tilde\lambda G'_J=\tilde\lambda G_j$.
Set $W'=\{w:w\in Z^I$, $w(j)\in G'_j$ for $j\in J\}$.
We have

\Centerline{$W\setminus W'
\subseteq\bigcup_{j\in J}\{w:w(j)\in G_j\setminus G'_j\}$,}

\noindent while

\Centerline{$\tilde\mu\{w:w(j)\in G_j\setminus G'_j\}
=\tilde\lambda(G_j\setminus G'_j)=0$}

\noindent for each $j$, so $\tilde\mu W'$ is defined and equal to
$\tilde\mu W=\mu W$.   Note that the same calculation shows that
$\theta^IW=\theta^IW'$ whenever $\theta\in A$ is such
that $\theta G'_j=\theta G_j$ for every $j$, that is, for
$\tilde\nu_A$-almost every $\theta$.   Now, for each $j\in J$,
we have a non-decreasing sequence $\sequencen{f_{jn}}$ of non-negative
$\frak S$-continuous real-valued functions with supremum $\chi G'_j$.
Set $g_n(w)=\prod_{j\in J}f_{jn}(w(j))$ for $w\in Z^I$ and $n\in\Bbb N$.
(I suppose you should take $g_n(w)=1$ if $J$ is empty.)   Then each
$g_n$ is $\frak S^I$-continuous, so if we set
$h_n(\theta)=\int g_nd\theta^I$ for $\theta\in A$,
$h_n$ is continuous (put 437Mb and 437Kb together).
Also $\sequencen{g_n}$ is a
non-decreasing sequence with supremum $\chi W'$, so

\Centerline{$\theta^IW'=\sup_{n\in\Bbb N}\int g_nd\theta^I
=\sup_{n\in\Bbb N}h_n(\theta)$}

\noindent for $\theta\in A$.   Accordingly

$$\eqalign{\tilde\mu W
&=\tilde\mu W'
=\int\mu_y^IW'\nu(dy)
=\sup_{n\in\Bbb N}\int h_n(\mu_y)\nu(dy)\cr
&=\sup_{n\in\Bbb N}\int h_nd\tilde\nu_A
=\sup_{n\in\Bbb N}\iint g_nd\theta^I\,\tilde\nu_A(d\theta)\cr
&=\int\theta^IW'\,\tilde\nu_A(d\theta)
=\int\theta^IW\,\tilde\nu_A(d\theta),\cr}$$

\noindent as required.\ \Qed

\medskip

{\bf (h)} We are nearly ready to dispense with the topology $\frak S$.
Since the embeddings
$A\embedsinto P_{\text{R}}(Z,\frak S)\embedsinto P_{\text{R}}(Z,\frak T)$
are continuous
(437Jh), we have an image Radon probability measure $\tilde\nu$ on
$P_{\text{R}}(Z,\frak T)$, and

\Centerline{$\int_{P_{\text{R}}(Z,\frak T)}h\,d\tilde\nu
=\int_Ah\,d\tilde\nu_A$}

\noindent for every $h:P_{\text{R}}(Z,\frak T)\to\Bbb R$ such that
$\int_Ah\,d\tilde\nu_A$ is defined.

In particular, if we take $\Cal W$ to be the family of
$\frak T^I$-open cylinder sets expressible as
$\{w:w\in Z^I$, $w(j)\in G_j$ for $j\in J\}$ where $J\subseteq I$ is finite
and $G_j\in\frak T$ for each $j$, (g) tells us that

\Centerline{$\tilde\mu W
=\int\theta^IW\,\tilde\nu_A(d\theta)
=\int\theta^IW\,\tilde\nu(d\theta)
=\int\tilde\theta^IW\,\tilde\nu(d\theta)$}

\noindent for every $W\in\Cal W$,
where I now write $\frak T^I$ for the product topology on $Z^I$
corresponding to the topology $\frak T$ on $Z$, and
$\tilde\theta^I$ for the
$\frak T^I$-quasi-Radon product measure on $Z^I$ corresponding to the
$\frak T$-Radon measure $\theta$ (417R).   Now turn again to 437Mb and
459G;  $\theta\mapsto\tilde\theta^I$ is a continuous function from
$P_{\text{R}}(Z,\frak T)$ to the space $P_{\text{qR}}(Z^I,\frak T^I)$ of
$\frak T^I$-quasi-Radon probability measures on $Z^I$, and $\Cal W$
is a base for the topology $\frak T^I$, so
$\family{\theta}{P_{\text{R}}(Z,\frak T)}{\tilde\theta^I}$ is a
disintegration of $\tilde\mu$ over $\tilde\nu$, which is what I set out to
prove.
}%end of proof of 459H

\leader{459I}{}\cmmnt{ I come now to a lemma based on ideas in
{\smc Tao 07}.   It is in a form more elaborate than is required for the
elementary application here (459J), but which will be needed in \S497.

\medskip

\noindent}{\bf Lemma}\dvAnew{2010}
Let $(X,\Sigma,\mu)$ be a probability space and
$I$ a set.   For a family $\Bbb T$ of subalgebras of $\Cal PX$,
write $\bigvee\Bbb T$ for the $\sigma$-algebra generated by
$\bigcup\Bbb T$\cmmnt{, as in 458Ad}.
Let $G$ be the group of permutations $\phi$ of $I$ such that
$\{i:\phi(i)\ne i\}$ is finite.   Suppose that $\action$ is an action of
$G$ on $X$ such that $x\mapsto\phi\action x$ is \imp\ for each $\phi\in G$;
set $\phi\action A=\{\phi\action x:x\in A\}$ for $\phi\in G$ and
$A\subseteq X$\cmmnt{, as in 441Aa}.
Let $\langle\Sigma_J\rangle_{J\subseteq I}$ be a family of
$\sigma$-subalgebras of $\Sigma$ such that

\inset{(i) for every $J\subseteq I$,
$\Sigma_J$ is the $\sigma$-algebra generated by
$\bigcup_{K\subseteq J\text{ is finite}}\Sigma_K$;

(ii) if $J\subseteq I$, $E\in\Sigma_J$ and $\phi\in G$, then
$\phi\action E\in\Sigma_{\phi[J]}$;

(iii) if $J\subseteq I$, $E\in\Sigma_J$ and $\phi\in G$ is such that
$\phi(i)=i$ for every $i\in J$, then $\phi\action E=E$.}

\noindent Suppose that $\Cal J^*$ is a filter on $I$ not containing any
infinite set, and that $K\subseteq I$, $\Cal K\subseteq\Cal PI$ and
$\Cal J\subseteq\Cal J^*$ are such that for every $K'\in\Cal K$ there is a
$J\in\Cal J$ such that $K\cap K'\subseteq J$.
Then $\Sigma_K$ and $\bigvee_{K'\in\Cal K}\Sigma_{K'}$
are relatively independent over $\bigvee_{J\in\Cal J}\Sigma_J$.

\proof{{\bf (a)(i)} Let us note straight away that condition (i)
above implies
that $\Sigma_K\subseteq\Sigma_J$ whenever $K\subseteq J\subseteq I$.

\medskip

\quad{\bf (ii)}
For any $\sigma$-subalgebra $\Tau$ of $\Sigma$, I will
(slightly abusing notation, as in 242Jh)
write $L^2(\mu\restr\Tau)$ for the $\|\,\|_2$-closed linear
subspace of $L^2(\mu)$ consisting of
equivalence classes of $\mu$-square-integrable $\Tau$-measurable
real-valued functions defined on $X$, and
$P_{\Tau}:L^2(\mu)\to L^2(\mu\restr\Tau)$ for the corresponding
conditional-expectation operator (244M).   Note that $P_{\Tau}$ is an
orthogonal projection (244Nb).

\medskip

\quad{\bf (iii)} We have an action of $G$ on $L^2(\mu)$,
defined by saying that

\Centerline{$(\phi\action f)(x)=f(\phi^{-1}\action x)$ for $\phi\in G$,
$x\in X$ and $f\in\Bbb R^X$}

\noindent(4A5C(c-i)),

\Centerline{$\phi\action f^{\ssbullet}=(\phi\action f)^{\ssbullet}$ for
$\phi\in G$ and $f\in\eusm L^2(\mu)\cap\Bbb R^X$}

\noindent(441Kc).

\medskip

\quad{\bf (iv)} If $\Bbb T$ is the family of $\sigma$-algebras of subsets
of $X$, we have an action of $G$ on $\Bbb T$ defined by setting

\Centerline{$\phi\action\Tau=\{\phi\action E:E\in\Tau\}$}

\noindent for $\Tau\in\Bbb T$ and $\phi\in G$.   If
$\family{\gamma}{\Gamma}{\Tau_{\gamma}}$ is a family in $\Bbb T$, then
$\phi\action\bigvee_{\gamma\in\Gamma}\Tau_{\gamma}
=\bigvee_{\gamma\in\Gamma}\phi\action\Tau_{\gamma}$ for every
$\phi\in G$, just because $E\mapsto\phi\action E$ is an automorphism of the
Boolean algebra $\Cal PX$.

\medskip

\quad{\bf (v)} If $\phi\in G$ and $L\subseteq I$, then
$\phi\action\Sigma_L=\Sigma_{\phi[L]}$.   \Prf\ Condition (ii) of this
lemma says just that
$\phi\action\Sigma_L=\{\phi\action E:E\in\Sigma_L\}$ is included in
$\Sigma_{\phi[L]}$;  and now of course

\Centerline{$\Sigma_{\phi[L]}=\phi\action\phi^{-1}\action\Sigma_L
\subseteq\phi\action\Sigma_{\phi^{-1}[\phi[L]]}
=\phi\action\Sigma_L$.\  \Qed}

\medskip

\quad{\bf (vi)} If $\phi\in G$ and $\Tau$ is a $\sigma$-subalgebra of
$\Sigma$, then $\phi\action(P_{\Tau}u)=P_{\phi\action\Tau}(\phi\action u)$
for every $u\in L^2(\mu)$.    \Prf\ I should of course note that
$\phi\action\Sigma=\Sigma$ because $x\mapsto\phi\action x$ is an
automorphism of $(X,\Sigma,\mu)$, so $\phi\action\Tau\subseteq\Sigma$ and
we can speak of $P_{\phi\action\Tau}$.   Let $f:X\to\Bbb R$ be a
$\Sigma$-measurable function such that $f^{\ssbullet}=u$, and
$g:X\to\Bbb R$ a $\Tau$-measurable function which is a conditional
expectation of $f$ on $\Tau$.   In this case, for any $\alpha\in\Bbb R$,

$$\eqalign{\{x:(\phi\action g)(x)>\alpha\}
&=\{x:g(\phi^{-1}\action x)>\alpha\}
=\{\phi\action x:g(x)>\alpha\}\cr
&=\phi\action\{x:g(x)>\alpha\}
\in\phi\action\Tau,\cr}$$

\noindent so $\phi\action g$ is $(\phi\action\Tau)$-measurable.   Next, for
any $F\in\phi\action\Tau$,

$$\eqalignno{\int_F\phi\action g\,d\mu
&=\int_Fg(\phi^{-1}\action x)\mu(dx)
=\int_{\phi^{-1}\action F}g(x)\mu(dx)\cr
\displaycause{applying 235G to the \imp\ function
$x\mapsto\phi\action x:X\to X$ and the integrable function
$x\mapsto g(\phi^{-1}\action x)$}
&=\int_{\phi^{-1}\action F}f\,d\mu\cr
\displaycause{because $\phi^{-1}\action F\in\Tau$}
&=\int_F\phi\action fd\mu.}$$

\noindent As $F$ is arbitrary, $\phi\action g$ is a conditional expectation
of $\phi\action f$ on $\phi\action\Tau$, and

\Centerline{$\phi\action(P_{\Tau}u)=\phi\action g^{\ssbullet}
=(\phi\action g)^{\ssbullet}
=P_{\phi\action\Tau}(\phi\action f)^{\ssbullet}
=P_{\phi\action\Tau}(\phi\action u)$. \Qed}

\medskip

{\bf (b)(i)}
Let $\family{\gamma}{\Gamma}{J_{\gamma}}$ be a non-empty finite
family of subsets of $I$ with infinite intersection, and set
$\Lambda=\bigvee_{\gamma\in\Gamma}\Sigma_{J_{\gamma}}$.   Suppose that $K$,
$\family{\gamma}{\Gamma}{K_{\gamma}}$ are such that

\Centerline{$K\in[I]^{<\omega}$,
\quad$K_{\gamma}\in[I]^{<\omega}$ and
$K\cap K_{\gamma}\subseteq J_{\gamma}$ for every $\gamma\in\Gamma$.}

\noindent Take $E\in\Sigma_K$ and
$F_{\gamma}\in\Sigma_{K_{\gamma}}$ for every $\gamma\in\Gamma$, and set
$F=\bigcap_{\gamma\in\Gamma}F_{\gamma}$.   Let $g$, $h:X\to[0,1]$ be
$\Lambda$-measurable functions which are conditional expectations of
$\chi E$, $\chi F$ respectively on $\Lambda$.   Let $\epsilon>0$.

\medskip

\quad{\bf (ii)} For $L\subseteq I$ set
$\Lambda_L=\bigvee_{\gamma\in\Gamma}\Sigma_{J_{\gamma}\cap L}
\subseteq\Lambda$.   For any
$u\in L^2(\mu)$ there is a finite $L\subseteq I$ such that
$\|P_{\Tau}u-P_{\Lambda}u\|_2\le\epsilon$ whenever $\Tau$ is a
$\sigma$-subalgebra of $\Lambda$ including $\Lambda_L$.
\Prf\ By condition (i) of this lemma,
$\Lambda$ is the $\sigma$-algebra generated by

\Centerline{$\bigcup_{L\subseteq I\text{ is finite}}
\bigcup_{\gamma\in\Gamma}\Sigma_{J_{\gamma}\cap L}$,}

\noindent so $\{\Lambda_L:L\in[I]^{<\omega}\}$ is an upwards-directed
family of $\sigma$-algebras whose union $\sigma$-generates $\Lambda$, and
$\bigcup_{L\subseteq I\text{ is finite}}L^2(\mu\restr\Lambda_L)$
is norm-dense in
$L^2(\mu\restr\Lambda)$.   There are therefore a finite
$L\subseteq I$ and a $v\in L^2(\mu\restr\Lambda_L)$ such that
$\|v-P_{\Lambda}u\|_2\le\epsilon$.  If now
$\Lambda_L\subseteq\Tau\subseteq\Lambda$,
$v\in L^2(\mu\restr\Tau)$, while
$P_{\Tau}$ is the orthogonal projection onto $L^2(\mu\restr\Tau)$, so

\Centerline{$\|P_{\Tau}u-P_{\Lambda}u\|_2
=\|P_{\Tau}P_{\Lambda}u-P_{\Lambda}u\|_2\le\|v-P_{\Lambda}u\|_2
\le\epsilon$.  \Qed}

\medskip

\quad{\bf (iii}
Set $u=\chi E^{\ssbullet}$ and $v=\chi F^{\ssbullet}$,
so that $g^{\ssbullet}=P_{\Lambda}u$ and $h^{\ssbullet}=P_{\Lambda}v$.
By (b), there is an $L_0\in[I]^{<\omega}$ such that

\Centerline{$\|P_{\Tau}u-P_{\Lambda}u\|_2\le\epsilon$,
\quad$\|P_{\Tau}v-P_{\Lambda}v\|_2\le\epsilon$,
\quad$\|P_{\Tau}(u\times v)-P_{\Lambda}(u\times v)\|_2\le\epsilon$}

\noindent whenever $\Tau$ is a $\sigma$-subalgebra of $\Lambda$ including
$\Lambda_{L_0}$.   We can suppose that
$L_0\supseteq K\cup\bigcup_{\gamma\in\Gamma}K_{\gamma}$.   Write $\Tau_0$
for $\Lambda_{L_0}$.   We have

$$\eqalignno{\|P_{\Lambda}u\times P_{\Lambda}v
  &-P_{\Tau_0}u\times P_{\Tau_0}v\|_2
  \cr
&\le\|P_{\Lambda}u\times(P_{\Lambda}v-P_{\Tau_0}v)\|_2
   +\|(P_{\Lambda}u-P_{\Tau_0}u)\times P_{\Tau_0}v\|_2\cr
&\le\|P_{\Lambda}v-P_{\Tau_0}v\|_2
   +\|P_{\Lambda}u-P_{\Tau_0}u\|_2\cr
\displaycause{because $\|P_{\Lambda}u\|_{\infty}$ and
$\|P_{\Tau_0}v\|_{\infty}$ are both at most $1$}
&\le 2\epsilon.\cr}$$

\medskip

\quad{\bf (iv)} Let
$L_1\subseteq\bigcap_{\gamma\in\Gamma}J_{\gamma}\setminus L_0$
be a set of size $\#(L_0\setminus K)$;  let $\phi\in G$ be such that
$\phi[L_0\setminus K]=L_1$, $\phi^2$ is the identity and $\phi(i)=i$ for
$i\in I\setminus(L_1\cup(L_0\setminus K))$.   In this case,
$\phi(i)=i$ for $i\in K$, so
$\phi[L]\subseteq(L\cap K)\cup\bigcap_{\gamma\in\Gamma}J_{\gamma}$
for every $L\subseteq L_0$.   Setting
$M_{\gamma}=(L_0\cap J_{\gamma})\cup\phi[L_0\cap J_{\gamma}]$, we have

\Centerline{$L_0\cap J_{\gamma}\subseteq M_{\gamma}=\phi[M_{\gamma}]
\subseteq J_{\gamma}$,
\quad$\phi[K_{\gamma}]\subseteq J_{\gamma}$}

\noindent for each $\gamma\in\Gamma$.   (This is where we need to know that
$K\cap K_{\gamma}\subseteq J_{\gamma}$.)

Now

\Centerline{$\phi\action u
=\phi\action(\chi E^{\ssbullet})=\chi(\phi\action E)^{\ssbullet}
=\chi E^{\ssbullet}=u$}

\noindent by condition (iii) of this lemma;  also

\Centerline{$\|\phi\action(P_{\Tau_0}u)-P_{\Lambda}u\|_2
\le 3\epsilon$.}

\noindent\Prf\ By (a-iv) and (a-v),

$$\eqalign{\phi\action\Tau_0
&=\phi\action\bigvee_{\gamma\in\Gamma}\Sigma_{L_0\cap J_{\gamma}}
=\bigvee_{\gamma\in\Gamma}\phi\action\Sigma_{L_0\cap J_{\gamma}}\cr
&=\bigvee_{\gamma\in\Gamma}\Sigma_{\phi[L_0\cap J_{\gamma}]}
\subseteq\bigvee_{\gamma\in\Gamma}\Sigma_{M_{\gamma}}
\subseteq\bigvee_{\gamma\in\Gamma}\Sigma_{J_{\gamma}}
=\Lambda.\cr}$$

\noindent Set $\Tau=\Tau_0\vee\phi\action\Tau_0$;  then
$\Tau_0\subseteq\Tau=\phi[\Tau]\subseteq\Lambda$.   But now

\Centerline{$\phi\action(P_{\Tau}u)=P_{\phi\action\Tau}(\phi\action u)
=P_{\Tau}u$}

\noindent(see (a-vi)), so

$$\eqalign{\|\phi\action(P_{\Tau_0}u)-P_{\Lambda}u\|_2
&\le\|\phi\action(P_{\Tau_0}u)-\phi\action(P_{\Tau}u)\|_2
    +\|P_{\Tau}u-P_{\Lambda}u\|_2\cr
&=\|P_{\Tau}u-P_{\Tau_0}u\|_2+\|P_{\Tau}u-P_{\Lambda}u\|_2\cr
&\le\|P_{\Lambda}u-P_{\Tau_0}u\|_2+2\|P_{\Tau}u-P_{\Lambda}u\|_2
\le 3\epsilon.  \text{ \Qed}\cr}$$

\medskip

\quad{\bf (v)} Set

\Centerline{$\Tau^*
=\bigvee_{\gamma\in\Gamma}\Sigma_{K_{\gamma}\cup M_{\gamma}}$.}

\noindent Because $L_0\cap J_{\gamma}\subseteq M_{\gamma}$ for every
$\gamma$, $\Tau^*$ and

\Centerline{$\phi\action\Tau^*
=\bigvee_{\gamma\in\Gamma}\Sigma_{\phi[K_{\gamma}]\cup M_{\gamma}}$}

\noindent include $\Lambda_{L_0}=\Tau_0$,
while $\phi\action\Tau^*\subseteq\Lambda$ because
$\phi[K_{\gamma}]\cup M_{\gamma}\subseteq J_{\gamma}$ for every $\gamma$.
Also $F\in\Tau^*$, because
$F_{\gamma}\in\Sigma_{K_{\gamma}}\subseteq\Tau^*$ for every $\gamma$.
Now

\noindent and

$$\eqalignno{\|P_{\Tau_0}(u\times v)
  -P_{\Tau_0}u\times P_{\Tau_0}v\|_2
&=\|P_{\Tau_0}P_{\Tau^*}(u\times v)
  -P_{\Tau_0}u\times P_{\Tau_0}v\|_2\cr
\displaycause{because $T_0\subseteq T^*$}
&=\|P_{\Tau_0}(v\times P_{\Tau^*}u)
  -P_{\Tau_0}(v\times P_{\Tau_0}u)\|_2\cr
\displaycause{because $v\in L^2(\mu\restr\Tau^*)$ and
$P_{\Tau_0}u\in L^2(\mu\restr\Tau_0)$, see 242L}
&\le\|v\times P_{\Tau^*}u-v\times P_{\Tau_0}u\|_2
\le\|P_{\Tau^*}u-P_{\Tau_0}u\|_2\cr
\displaycause{because $\|v\|_{\infty}\le 1$}
&=\|\phi\action(P_{\Tau^*}u)-\phi\action(P_{\Tau_0}u)\|_2\cr
&=\|P_{\phi\action\Tau^*}(\phi\action u)
  -\phi\action(P_{\Tau_0}u)\|_2\cr
&\le\|P_{\phi\action\Tau^*}u-P_{\Lambda}u\|_2
  +\|P_{\Lambda}u-\phi\action(P_{\Tau_0}u)\|_2\cr
&\le\epsilon+3\epsilon
=4\epsilon.\cr}$$

\medskip

\quad{\bf (vi)} Putting these together,

$$\eqalign{\|P_{\Lambda}(u\times v)-P_{\Lambda}u\times P_{\Lambda}v\|_2
&\le\|P_{\Lambda}(u\times v)-P_{\Tau_0}(u\times v)\|_2\cr
&\mskip100mu
  +\|P_{\Tau_0}(u\times v)-P_{\Tau_0}u\times P_{\Tau_0}v\|_2\cr
&\mskip100mu
  +\|P_{\Lambda}u\times P_{\Lambda}v-P_{\Tau_0}u\times P_{\Tau_0}v\|_2\cr
&\le\epsilon+4\epsilon+2\epsilon
=7\epsilon.\cr}$$

\medskip

\quad{\bf (vii)} As $\epsilon$ is arbitrary,
$P_{\Lambda}(u\times v)=P_{\Lambda}u\times P_{\Lambda}v$, that is,
$g\times h$ is a conditional expectation of $\chi(E\cap F)$ on $\Lambda$,
and $E$ and $F$ are relatively independent over $\Lambda$.

\medskip

{\bf (c)} It follows that $\Sigma_K$ and
$\bigvee_{\gamma\in\Gamma}\Sigma_{K_{\gamma}}$ are relatively independent
over $\Lambda$.  \Prf\ Suppose that $E\in\Sigma_K$, and consider the set

\Centerline{$\Cal E=\{F:F\in\Sigma$,
$P_{\Lambda}\chi(E\cap F)^{\ssbullet}
=P_{\Lambda}(\chi E^{\ssbullet})\times P_{\Lambda}(\chi F^{\ssbullet})\}$.}

\noindent Then $\Cal E$ is a Dynkin class, and by (b) above it contains

\Centerline{$\Cal E_0
=\{\bigcap_{\gamma\in\Gamma}F_{\gamma}:F_{\gamma}\in\Sigma_{K_{\gamma}}$
for every $\gamma\in\Gamma\}$,}

\noindent which is closed under $\cap$.
Accordingly $\Cal E$ includes the $\sigma$-algebra generated by
$\Cal E_0$, which is $\bigvee_{\gamma\in\Gamma}\Sigma_{K_{\gamma}}$.
Thus

\Centerline{$P_{\Lambda}\chi(E\cap F)^{\ssbullet}
=P_{\Lambda}(\chi E^{\ssbullet})\times P_{\Lambda}(\chi F^{\ssbullet})$}

\noindent for every $E\in\Sigma_K$ and
$F\in\bigvee_{\gamma\in\Gamma}\Sigma_{K_{\gamma}}$, and
$\Sigma_K$ and
$\bigvee_{\gamma\in\Gamma}\Sigma_{K_{\gamma}}$ are relatively independent
over $\Lambda$.\ \Qed

\medskip

{\bf (d)} Now suppose that $\family{\gamma}{\Gamma}{J_{\gamma}}$ is a
non-empty
finite family of subsets of $I$ with infinite intersection.   As before,
write $\Lambda$ for $\bigvee_{\gamma\in\Gamma}\Sigma_{J_{\gamma}}$.
Suppose that $K\subseteq I$ and that
$\family{\gamma}{\Gamma}{K_{\gamma}}$ is a family of subsets of $I$ such
that $K\cap K_{\gamma}\subseteq J_{\gamma}$ for every $\gamma\in\Gamma$.
Then $\Sigma_K$ and $\bigvee_{\gamma\in\Gamma}\Sigma_{K_{\gamma}}$ are
relatively independent over $\Lambda$.
\Prf\ Set $\Tau=\bigcup\{\Sigma_L:L\in[K]^{<\omega}\}$ and
for $\gamma\in\Gamma$ set
$\Tau_{\gamma}=\bigcup\{\Sigma_L:L\in[K_{\gamma}]^{<\omega}\}$.
Then (b)-(c) tell us that $\Tau$ and the algebra
$\Tau'\,\,\sigma$-generated by
$\bigcup_{\gamma\in\Gamma}\Tau_{\gamma}$ are relatively independent over
$\Lambda$.
Since $\Sigma_K$ is the $\sigma$-algebra
generated by $\Tau$, while $\bigvee_{\gamma\in\Gamma}\Sigma_{K_{\gamma}}$
is the $\sigma$-algebra generated by $\Tau'$,
458Da-458Db tell us that $\Sigma_K$ and
$\bigvee_{\gamma\in\Gamma}\Sigma_{K_{\gamma}}$ are relatively independent
over $\Lambda$.\ \Qed

\medskip

{\bf (e)} At last we are ready to approach the sets $K$, $\Cal K$ and
$\Cal J$ of the statement of this lemma.   The case $\Cal J=\emptyset$
is trivial (as then $\Cal K$ must also be empty), so suppose that $\Cal J$
is non-empty.

\medskip

\quad{\bf (i)} To begin with, suppose that $\Cal J$ and $\Cal K$ are
finite.   In this case, we can find finite families
$\family{\gamma}{\Gamma}{J_{\gamma}}$ and
$\family{\gamma}{\Gamma}{K_{\gamma}}$ running over $\Cal J$,
$\Cal K\cup\{\emptyset\}$
respectively such that $K\cap K_{\gamma}\subseteq J_{\gamma}$ for every
$\gamma$.   So (d) tells us that $\Sigma_K$ and
$\bigvee_{K'\in\Cal K}\Sigma_{K'}\vee\Sigma_{\emptyset}$ are relatively
independent over $\bigvee_{J\in\Cal J}\Sigma_J$.

\medskip

\quad{\bf (ii)} If $\Cal K$ is finite but $\Cal J$ is infinite, then
let $\Cal J_0\subseteq\Cal J$ be a finite set such that for every
$K'\in\Cal K$ there is a $J\in\Cal J_0$ including $K\cap K'$.
Then for any finite $\Cal J'\subseteq\Cal J$ including $\Cal J_0$,
$\Sigma_K$ and $\bigvee_{K'\in\Cal K}\Sigma_{K'}$
are relatively independent over $\bigvee_{J\in\Cal J'}\Sigma_J$.   Since

\Centerline{$\{\bigvee_{J\in\Cal J'}\Sigma_J:
\Cal J_0\subseteq\Cal J'\in[\Cal J]^{<\omega}\}$}

\noindent is an upwards-directed family of $\sigma$-algebras whose union
$\sigma$-generates $\bigvee_{J\in\Cal J}\Sigma_J$, 458C tells us that
$\Sigma_K$ and $\bigvee_{K'\in\Cal K}\Sigma_{K'}$ are relatively independent
over $\bigvee_{J\in\Cal J}\Sigma_J$.

\medskip

\quad{\bf (iii)} Finally, for the general case, (ii) tells us that
$\Sigma_K$ and $\bigvee_{K'\in\Cal K'}\Sigma_{K'}$
are relatively independent
over $\bigvee_{J\in\Cal J}\Sigma_J$ for every finite
$\Cal K'\subseteq\Cal K$, so $\Sigma_K$ and
$\bigvee_{K'\in\Cal K}\Sigma_{K'}$ are relatively independent
over $\bigvee_{J\in\Cal J}\Sigma_J$, by 458D again.
}%end of proof of 459I

\leader{459J}{Corollary}\dvAnew{2010}
Let $(X,\Sigma,\mu)$ be a probability space and
$I$ a set.   Let $G$ be the group of permutations $\phi$ of $I$ such that
$\{i:\phi(i)\ne i\}$ is finite.   Suppose that $\action$ is an action of
$G$ on $X$ such that $x\mapsto\phi\action x$ is \imp\ for each $\phi\in G$.
Let $\langle\Sigma_J\rangle_{J\subseteq I}$ be a family of
$\sigma$-subalgebras of $\Sigma$ such that

\inset{(i) for every $J\subseteq I$,
$\Sigma_J$ is the $\sigma$-algebra generated by
$\bigcup_{K\subseteq J\text{ is finite}}\Sigma_K$;

(ii) if $J\subseteq I$, $E\in\Sigma_J$ and $\phi\in G$, then
$\phi\action E\in\Sigma_{\phi[J]}$;

(iii) if $J\subseteq I$, $E\in\Sigma_J$ and $\phi\in G$ is such that
$\phi(i)=i$ for every $i\in J$, then $\phi\action E=E$.}

\noindent Then if $J\subseteq I$ is infinite and
$\family{\gamma}{\Gamma}{K_{\gamma}}$ is a family of subsets of $I$ such
that $K_{\gamma}\cap K_{\delta}\subseteq J$ for all distinct $\gamma$,
$\delta\in\Gamma$, $\family{\gamma}{\Gamma}{\Sigma_{K_{\gamma}}}$ is
relatively independent over $\Sigma_J$.

\proof{ By 459I, $\Sigma_{K_{\gamma}}$ and
$\bigvee_{\delta\in\Delta}\Sigma_{K_{\delta}}$ are
relatively independent over $\Sigma_J$ whenever $\Delta\subseteq\Gamma$ and
$\gamma\in\Gamma\setminus\Delta$.   Now
458Hb tells us that we can induce on $\#(\Delta)$ to see that
$\family{\gamma}{\Delta}{\Sigma_{K_{\gamma}}}$ is relatively independent
over $\Sigma_J$ for every finite $\Delta\subseteq\Gamma$, and it follows at
once that $\family{\gamma}{\Gamma}{\Sigma_{K_{\gamma}}}$
is relatively independent over $\Sigma_J$, as remarked in 458Ac.
}%end of proof of 459J
\proof{ Note first that if $G$ is the group of permutations $\phi$ of $I$
such that $\{i:\phi(i)\ne i\}$ is finite, then
any $\phi\in G$ is expressible as the
product of finitely many transpositions, so $w\mapsto w\phi$ is an
automorphism of $(X^I,\mu)$.   Let $\action$ be the action of $G$ on $X^I$
defined by saying that $\phi\action w=w\phi^{-1}$ for $x\in X^I$ and
$\phi\in G$.   Then $w\mapsto\phi\action w$ is \imp\ for every $\phi$.

If $L\subseteq I$ then $\Sigma_L$ is the $\sigma$-algebra of subsets of
$X^I$ generated by sets of the form $\{x:x(i)\in E\}$ where $i\in L$ and
$E\in\Sigma$.   So $\Sigma_L$ is the $\sigma$-algebra generated by
$\bigcup\{\Sigma_K:K\in[L]^{<\omega}\}$.

If $i\in I$, $E\in\Sigma$ and $\phi\in G$, then

\Centerline{$\phi\action\{x:x(i)\in E\}=\{\phi\action x:x(i)\in E\}
=\{x:(\phi^{-1}\action x)(i)\in E\}
=\{x:x(\phi(i))\in E\}$.}

\noindent So if $L\subseteq I$ and $\phi\in G$,
$\{W:\phi\action W\in\Sigma_{\phi[L]}\}$ is a $\sigma$-algebra of subsets
of $X^I$ containing $\{x:x(i)\in E\}$ whenever $i\in L$ and $E\in\Sigma$,
therefore including $\Sigma_L$;  that is,
$\phi\action W\in\Sigma_{\phi[L]}$ whenever $W\in\Sigma_L$.

If $L\subseteq I$ and $\phi\in G$ is such that $\phi(i)=i$
for every $i\in L$, then $\{W:\phi\action W=W\}$ is a $\sigma$-algebra of
subsets of $X^I$ containing $\{x:x(i)\in E\}$ whenever $i\in L$ and
$E\in\Sigma$, so $\phi\action W=W$ for every $W\in\Sigma_L$.

Thus the conditions of 459I are satisfied, and the result follows at once.
}%end of proof of 459J

\leader{459K}{}\cmmnt{ Following the results of \S452 (especially
452Ye), we do not generally expect to find disintegrations of measures
which are not countably compact.   It may however illuminate the
constructions here if I give a specific example related to the contexts
of 459E and 459H.

\medskip

\noindent}{\bf Example}\cmmnt{ ({\smc Dubins \& Freedman 79})}
There are a separable metrizable space $Z$ and a quasi-Radon measure on
$Z^{\Bbb N}$, invariant under permutations of coordinates, which cannot
be disintegrated into powers of measures on $Z$.

\proof{{\bf (a)} Let $\lambda$ be Lebesgue measure on $[0,1]$.
$Q=[0,1]\times[0,1]^{\Bbb N}$, with its usual topology, is a compact
metrizable space, so has just $\frak c$ Borel sets (4A3F).   Let
$\ofamily{\xi}{\frakc}{W_{\xi}}$ enumerate the Borel subsets of
$Q$ with non-zero measure for the product
measure $\lambda\times\lambda^{\Bbb N}$.
(Remember that $\lambda\times\lambda^{\Bbb N}$ is a Radon measure,
by 416U.)
For each $\xi$, we have
$0<(\lambda\times\lambda^{\Bbb N})(W_{\xi})
=\int\lambda^{\Bbb N}(W_{\xi}[\{t\}])\lambda(dt)$, so
$A_{\xi}=\{t:W_{\xi}[\{t\}]\ne\emptyset\}$ has cardinal $\frak c$
(419H);  we can therefore choose $\ofamily{\xi}{\frakc}{t_{\xi}}$ in
$[0,1]$ such that $t_{\xi}\in A_{\xi}\setminus\{t_{\eta}:\eta<\xi\}$
for every $\xi<\frak c$.   Now choose $t_{\xi n}$, for $\xi<\frak c$ and
$n\in\Bbb N$, such that $(t_{\xi},\sequencen{t_{\xi n}})\in W_{\xi}$.
Set
$Z=\{(t_{\xi},t_{\xi n}):\xi<\frak c$, $n\in\Bbb N\}\subseteq[0,1]^2$.

\medskip

{\bf (b)} Set $X=([0,1]^2)^{\Bbb N}$ and define $\phi:Q\to X$ by setting
$\phi(t,\sequencen{t_n})=\sequencen{(t,t_n)}$ for $t$, $t_n\in[0,1]$.
Then $\phi$ is a homeomorphism between $Q$ and $\phi[Q]$, so there is a
unique Radon measure $\mu^{\#}$ on $X$ such that $\phi$ is \imp\ for
$\lambda\times\lambda^{\Bbb N}$ and $\mu^{\#}$.   Now $\mu^{\#}$ is
invariant under permutations of coordinates, because if
$\rho:\Bbb N\to\Bbb N$ is a permutation and $\hat\rho(x)=x\rho$ for
$x\in X$, then $\hat\rho\phi=\phi\bar\rho$, where
$\bar\rho(t,\sequencen{t_n})=(t,\sequencen{t_{\rho(n)}})$;  and as
$\bar\rho:Q\to Q$ is \imp, so is $\hat\rho:X\to X$.

Also $Z^{\Bbb N}$ has full outer measure for $\mu^{\#}$.   \Prf\ If
$\mu^{\#}W>0$, then $(\lambda\times\lambda^{\Bbb N})\phi^{-1}[W]>0$, so
there is some $\xi<\frak c$ such that $W_{\xi}\subseteq\phi^{-1}[W]$.
Now $\sequencen{(t_{\xi},t_{\xi n})}\in Z^{\Bbb N}\cap W$.\ \QeD\
Accordingly the subspace measure $\tilde\mu$ on $Z$ is a probability
measure.   Because $\mu^{\#}$ is invariant under permutations of
coordinates, so is $\tilde\mu$;  because $\mu^{\#}$ is a Radon measure,
$\tilde\mu$ is a quasi-Radon measure (416Ra).

\medskip

{\bf (c)} \Quer\ Suppose, if possible, that there are a probability
space $(Y,\Tau,\nu)$ and a family $\family{y}{Y}{\mu_y}$ of probability
measures on $Z$ such that $\tilde\mu E=\int\mu_y^{\Bbb N}E\,\nu(dy)$ for
every Borel set $E\subseteq Z^{\Bbb N}$.   (The argument to follow will not depend on which product measure is used in forming the
$\mu_y^{\Bbb N}$.)   Looking at sets of the form
$(Z\cap H)\times Z\times Z\times\ldots$, where
$H\subseteq[0,1]^2$ is a Borel set, we see that $\mu_y(Z\cap H)$ must be
defined for almost every $y$;  as $Z$ is second-countable, $\mu_y$ must
be a topological measure for almost every $y$.
Looking at sets of the form
$(Z\cap(G_0\times[0,1]))\times(Z\cap(G_1\times[0,1]))\times
Z\times\ldots$, where $G_0$ and $G_1$ are disjoint Borel subsets of
$[0,1]$, we see that
$\mu_y(Z\cap(G_0\times[0,1]))\cdot\mu_y(Z\cap(G_1\times[0,1]))=0$ for
almost every $y$;  as $[0,1]$ is second-countable and Hausdorff, there
must be, for almost every $y\in Y$, an $s_y\in[0,1]$ such that
$\mu_y(Z\cap(\{s_y\}\times[0,1]))=1$.

Next, if $G\subseteq[0,1]$ is a Borel set, then
$\mu_y(Z\cap([0,1]\times G))=\lambda G$ for almost every $y$.   \Prf\

\Centerline{$h(y)
=\mu_y^{\Bbb N}((Z\cap([0,1]\times G))\times Z\times\ldots)
=\mu_y(Z\cap([0,1]\times G))$}

\noindent is defined for almost every $y$, and $h$ is $\nu$-integrable,
with

\Centerline{$\int h\,d\nu
=\tilde\mu((Z\cap([0,1]\times G))\times Z\times\ldots)
=\mu^{\#}(([0,1]\times G)\times[0,1]^2\times\ldots)=\lambda G$.}

\noindent At the same time,

$$\eqalign{\int h(y)(1-h(y))\nu(dy)
&=\tilde\mu((Z\cap([0,1]\times G))
\times(Z\cap([0,1]\times([0,1]\setminus G)))\times Z\times\ldots)\cr
&=\mu^{\#}(([0,1]\times G)
\times([0,1]\times([0,1]\setminus G))\times[0,1]^2\times\ldots)\cr
&=\lambda G(1-\lambda G).\cr}$$

\noindent Rearranging, we see that $\int h^2d\nu=(\int h)^2$.   But this
means that $\int(h(y)-\int h)^2\nu(dy)=0$ and $h(y)=\lambda G$ for
almost every $y$.\ \Qed

It follows that, for at least some $y$,
$\mu_y(Z\cap(\{s_y\}\times G))=\lambda G$ for every interval
$G\subseteq[0,1]$ with rational endpoints.   But this is impossible,
because all the vertical sections of $Z$ are countable.\ \Bang

Thus there is no such disintegration, as claimed.
}%end of proof of 459K

\exercises{\leader{459X}{Basic exercises $\pmb{>}$(a)}
%\sqheader 459Xa
Let $(X,\Sigma,\mu)$ be a probability space and
$\sequencen{f_n}$ an exchangeable sequence of real-valued random
variables on $X$ all with finite expectation.   Use 459C and 273I to
show that $\sequencen{\Bover1{n+1}\sum_{i=0}^nf_i}$ converges a.e.
(Compare 276Xg\formerly{2{}76Xe}.)
%459C

\spheader 459Xb Let $(X,\Sigma,\mu)$ be a probability space and
$\sequencen{f_n}$ an exchangeable sequence of real-valued random
variables on $X$ all with finite variance, such that
$\lim_{n\to\infty}\Bover1{n+1}\sum_{i=0}^nf_i=0$ a.e.   Show that
$\sequencen{\Pr(\sum_{i=0}^nf_i\ge\alpha\sqrt{n+1})}$ is
convergent for every $\alpha\in\Bbb R$.   \Hint{274I.}
%459C

\spheader 459Xc\dvAformerly{4{}59F}
Let $X$ be a completely regular topological space,
$(Y,\frak S,\Tau,\nu)$ a totally finite quasi-Radon measure space, and
$y\mapsto\mu_y$ a continuous function from $Y$ to the space
$M^+_{\text{qR}}(X)$ of totally finite quasi-Radon measures on $X$ with its
narrow topology.   Show that if
$\mu\in M^+_{\text{qR}}(X)$ is such that
$\int fd\mu=\iint fd\mu_y\,\nu(dy)$ for
every $f\in C_b(X)$, then $\family{y}{Y}{\mu_y}$ is a disintegration of
$\mu$ over $\nu$.
%459G

\sqheader 459Xd
({\smc Diaconis \& Freedman 80}) Let $Z$ be a non-empty
compact Hausdorff space and $I$ an infinite set including $\Bbb N$.
Let $\tilde\mu$ be a Radon probability measure on $Z^I$ invariant
under permutations of $I$.   For $k\le n$ let $D_{nk}\subseteq n^k$ be the set of injective functions from $k$ to $n$ and $\Omega_{nk}$ the set
$Z^I\times n^k\times D_{nk}$, endowed with the product $\lambda_{nk}$ of
$\tilde\mu$ and the uniform probability measures on the finite sets
$n^k$ and $D_{nk}$.   Define $\phi_{nk}:\Omega_{nk}\to Z^k$ and
$\psi_{nk}:\Omega_{nk}\to Z^k$ by setting

$$\eqalign{\phi_{nk}(w,p,q)&=wp,\cr
\psi_{nk}(w,p,q)&=wp\text{ if }p\in D_{nk},\cr
&=wq\text{ otherwise.}\cr}$$

\noindent (i) Show that there is a disintegration
$\family{w}{Z^I}{\mu_{nw}^k}$ of the image measure
$\lambda_{nk}\phi_{nk}^{-1}$
over $\tilde\mu$ where each $\mu_{nw}$ is a suitable point-supported
measure on $Z^k$.
(ii) Show that the image measure $\lambda_{nk}\psi_{nk}^{-1}$
is the image measure $\tilde\mu_k=\tilde\mu\tilde\pi_k^{-1}$, where
$\tilde\pi_k(w)=w\restr k$ for $w\in Z^I$.    (iii) Show that if $n>0$ then
$|\tilde\mu_kW-\int\mu_{nw}^kW\,\tilde\mu(dw)|
\le\Bover{k(k-1)}{2n}$ for every Baire set $W\subseteq Z^k$.
(iv) Show that there is a Radon probability measure $\tilde\nu_n$ on
$P_{\text{R}}(Z)$ for which $w\mapsto\mu_{nw}$ is \imp.   (v) Show that
if $\tilde\nu$ is any cluster point of
$\sequencen{\tilde\nu_n}$ in $P_{\text{R}}(Z)$ then
$\family{\theta}{P_{\text{R}}(Z)}{\tilde\theta^I}$ is a
disintegration of $\tilde\mu$ over $\tilde\nu$,
writing $\tilde\theta^I$
for the Radon product of copies of any $\theta\in P_{\text{R}}(Z)$.
%459H

\sqheader 459Xe ({\smc Hewitt \& Savage 55}) Let $X$ be a
non-empty compact Hausdorff space and $I$
an infinite set.   Let $Q$ be the set of Radon probability measures on
$X^I$ which are invariant under permutations of $I$.   Show that (i) $Q$
is a closed convex subset of the set $P_{\text{R}}(X^I)$ of all Radon
probability measures on $X^I$ with its narrow topology;  (ii) $Q$ is
isomorphic, as topological convex structure, to
$P_{\text{R}}(P_{\text{R}}(X))$;  (iii) the
extreme points of $Q$ are just the powers of Radon probability measures
on $X$.
%459H

\spheader 459Xf Let $X$, $I$ be sets, $\Sigma$ a
$\sigma$-algebra of subsets of $X$ and $\mu$ a probability measure with
domain $\Tensorhat_I\Sigma$ which is transposition-invariant
in the sense that for every
transposition $\tau:I\to I$ the function $x\mapsto x\tau:X^I\to X^I$ is
\imp.   For $J\subseteq I$, let $\Sigma_J$ be the $\sigma$-algebra

\Centerline{$W:W\in\Tensorhat_I\Sigma$,
$W$ is determined by coordinates in $J\}$.}

\noindent Show that if $J\subseteq I$ is infinite and
$\family{\gamma}{\Gamma}{K_{\gamma}}$ is a family of subsets of $I$ such
that $K_{\gamma}\cap K_{\delta}\subseteq J$ for all distinct $\gamma$,
$\delta\in\Gamma$, $\family{\gamma}{\Gamma}{\Sigma_{K_{\gamma}}}$ is
relatively independent over $\Sigma_J$ (i) using 459D (ii) using 459J.
%459D 459J

\leaveitout{\spheader 459X?
Let $X$ be a non-empty Hausdorff space.   Let $Q$ be the
set of Radon probability measures on
$X^{\Bbb N}$ which are invariant under permutations of $\Bbb N$.   Show
that (i) $Q$ is a closed convex subset of the set
$P_{\text{R}}(X^{\Bbb N})$ of all Radon
probability measures on $X^{\Bbb N}$ with its narrow topology;  (ii) $Q$
is isomorphic, as topological convex structure, to
$P_{\text{R}}(P_{\text{R}}(X))$;  (iii) the
extreme points of $Q$ are just the powers of Radon probability measures
on $X$.
%459H 459Xe

\query is (ii) true? (see 459Z) and if so, should it be in 459Y?  see
459Ya.
}%end of leaveitout

\leader{459Y}{Further exercises (a)}%
%\spheader 459Ya
\dvAnew{2010} Let $X$ be a
topological space and $I$ an infinite set.   Write $P_{\tau}(X)$,
$P_{\tau}(X^I)$ and
$P_{\tau}(P_{\tau}(X))$ for the spaces of $\tau$-additive Borel probability
measures in $X$, $X^I$ and $P_{\tau}(X)$ respectively,
with their narrow topologies.
(i) For $\theta\in P_{\tau}(X)$ write
$\tilde\theta^I$ for the $\tau$-additive Borel measure on $X^I$
corresponding to $\theta$, that is, the restriction to the Borel
$\sigma$-algebra of $X^I$ of the $\tau$-additive product measure described
in 417G.   Show that
$\theta\mapsto\tilde\theta^I:P_{\tau}(X)\mapsto P_{\tau}(X^I)$ is
continuous.   (ii) Show that
if $\nu\in P_{\tau}(P_{\tau}(X))$ there is a unique
$\mu_{\nu}\in P_{\tau}(X^I)$ such that
$\family{\theta}{P_{\tau}{(X)}}{\tilde\theta^I}$ is a disintegration of
$\mu_{\nu}$ over $\nu$, where $\tilde\theta^I$ is the $\tau$-additive
Borel product
measure on $X^I$ corresponding to $\theta\in P_{\tau}(X)$.
(iii) Show that $\nu\mapsto\mu_{\nu}$ is a homeomorphism between
$P_{\tau}(P_{\tau}(X))$ and its image in $P_{\tau}(X^I)$.
%mt45bits 459H

\spheader 459Yb
Discuss the problems which arise in
459B, 459C, 459E and 459H if the index set $I$ is finite.
%459H
}%end of exercises

\leaveitout{
\leader{459Z}{Problem}\dvAnew{2010} Let $X$ be a topological space,
$P_{\text{qR}}(X)$ the space of quasi-Radon
probability measures on $X$ with its narrow
topology, and $\nu$ a quasi-Radon probability measure on
$P_{\text{qR}}(X)$.   Is there necessarily a quasi-Radon probability
measure $\mu$ on $X$ such that $\family{\theta}{P_{\text{qR}}(X)}{\theta}$
is a disintegration of $\mu$ over $\nu$?
%459Ya (Yes if $X$ is regular.)
}

\endnotes{
\Notesheader{459} As I have presented this material, the centre of the
argument of 459A-459H %459A 459B 459C 459D 459E 459F 459G 459H
lies in the martingales in part (b-$\beta$) of the proof of 459B.
We are
trying to resolve the functions $f_i$ into `common' and `independent'
parts.   The `common' part is given by the conditional expectations of
the $f_i$ over an appropriate $\sigma$-algebra $\Tau$, and we approach
these by looking at the conditional expectations of each $f_i$ on
$\sigma$-algebras $\Tau_n$ generated by `distant' $f_j$.   All the most
important ideas are already exhibited when the index set $I$ is equal to
$\Bbb N$.   Note in particular that in the basic hypothesis that all
finite strings $(f_{i_0},\ldots,f_{i_r})$ have the same joint
distribution, it is enough to look at increasing strings.   But there is
a striking phenomenon which appears in sharper relief with uncountable
sets $I$:  any sequence $\sequence{k}{j_k}$ of distinct elements of $I$
can be used to generate an adequate $\sigma$-algebra, because while the
tail $\sigma$-algebra of sets depends on the choice of the $j_k$, they
all lead to the same closed subalgebra of the measure algebra (459D).

Perhaps I should emphasize at this point that $I$ really does have to be
infinite, though for large finite $I$ there are approximations to the
results here.

The proof of 459B is one of the standard proofs of De Finetti's theorem,
with trifling modifications.   In the case of real-valued random
variables we have a notion of relative distribution (458I) which gives a
quick way of saying that all the $f_i$ have the same conditional
expectations over $\Tau$, as in 459C(ii).   For variables taking values
in other spaces the situation may be different (459K), unless (as in
\S452) we have a countably compact measure (459E).

Specializing to the case $X=Z^I$ in 459B, we find ourselves examining
symmetric measures on infinite product spaces, which are of great
interest in themselves.   Note that while in the hypothesis of 459E I
have asked for the measure $\mu$ on the product space $Z^I$ to be
countably compact, what is actually necessary is that the marginal
measure on $Z$ should be countably compact.   By 454Ab, this comes to the
same thing.

As in 452O, we can look for a disintegration consisting of Radon
measures, provided of course that the marginal measure is a Radon
measure.   What we have to work harder for is a direct expression in
terms of an integral $\int\tilde\theta^I\tilde\nu(d\theta)$ where
$\tilde\nu$ is itself a Radon probability measure on the space of Radon
probability measures $\theta$ (459H).   But most of the extra work
consists of finding the correct reduction to the case of
locally compact spaces.
For compact spaces we can approach by a completely different route
(459Xd).   I will not go farther with this idea here, but I note that
the method can be used in a wide variety of problems involving symmetric
structures.

Lemma 459I is entirely different.   I include it here because it gives
another approach to relative independence and looks at
permutation-invariant measures, though in a more abstract setting which
does not bind us to the product spaces which are their most natural
expressions.   Its power lies precisely in the fact that in its hypotheses
we do {\it not} suppose that
$\Sigma_{J\cup K}=\Sigma_J\vee\Sigma_K$ for $J$, $K\subseteq I$, so the
$\sigma$-algebras $\bigvee_{K'\in\Cal K}\Sigma_{K'}$ and
$\bigvee_{J\in\Cal J}\Sigma_J$ have to be handled with special care.
}%end of notes


\discrpage


