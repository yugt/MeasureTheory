\frfilename{mt387.tex}
\versiondate{18.12.03}
\copyrightdate{1997}

\def\barln{\mathop{\bar{\text{ln}}}}

\def\chaptername{Automorphisms}
\def\sectionname{Ornstein's theorem}

\newsection{387}

I come now to the most important of the handful of theorems known which
enable us to describe automorphisms of measure algebras up to
isomorphism:  two two-sided Bernoulli shifts (on algebras of countable Maharam type) of the same entropy are isomorphic (387I, 387K).   This is hard work.
%, and I know of no printed version which
%attempts to cover it in the detail I present here.
It requires both
delicate $\epsilon$-$\delta$ analysis and substantial skill with the
manipulation of measure-preserving homomorphisms.   The proof is based
on two difficult lemmas (387C and 387F), and includes Sina\v\i's theorem
(387E, 387L), describing the Bernoulli shifts which arise as
factors of a given ergodic automorphism.

\leader{387A}{}\cmmnt{ The %3{8}6A
following definitions offer a
language in which to express the ideas of this section.

\medskip

\noindent}{\bf Definitions} Let $(\frak A,\bar\mu)$ be a probability
algebra and $\pi:\frak A\to\frak A$ a measure-preserving Boolean
homomorphism.

\spheader 387Aa A {\bf Bernoulli partition} for $\pi$ is a partition of
unity $\langle a_i\rangle_{i\in I}$ such that

\Centerline{$\bar\mu(\inf_{j\le k}\pi^ja_{i(j)})
=\prod_{j=0}^k\bar\mu a_{i(j)}$}

\noindent whenever $k\in\Bbb N$ and $i(0),\ldots,i(k)\in I$.

\spheader 387Ab If $\pi$ is an automorphism, a Bernoulli partition
$\langle a_i\rangle_{i\in I}$ for $\pi$ is {\bf (two-sidedly)
generating}
if the closed subalgebra generated by $\{\pi^ja_i:i\in I,\,j\in\Bbb Z\}$
is $\frak A$ itself.

\spheader 387Ac A {\bf factor} of $(\frak A,\bar\mu,\pi)$ is a triple
$(\frak B,\bar\mu\restrp\frak B,\pi\restrp\frak B)$ where $\frak B$ is a
closed subalgebra of $\frak A$ such that $\pi[\frak B]=\frak B$.

\leader{387B}{Remarks} Let %3{8}6B
$(\frak A,\bar\mu)$ be a probability algebra,
$\pi:\frak A\to\frak A$ a measure-preserving Boolean homomorphism and
$\familyiI{a_i}$ a Bernoulli partition for $\pi$.

\spheader 387Ba $\sequence{k}{\pi^k[\frak A_0]}$ is independent, where
$\frak A_0$ is the closed subalgebra of $\frak A$ generated by
$\{a_i:i\in I\}$.   \prooflet{\Prf\ Suppose that
$c_j\in\pi^j[\frak A_0]$ for $j\le k$.   Then each
$\pi^{-j}c_j\in\frak A_0$ is expressible as
$\sup_{i\in I_j}a_i$ for some $I_j\subseteq I$.   Now

$$\eqalign{\bar\mu(\inf_{j\le k}c_j)
&=\bar\mu(\sup_{i_0\in I_0,\ldots,i_k\in I_k}\inf_{j\le k}
  \pi^ja_{i_j})\cr
&=\sum_{i_0\in I_0,\ldots,i_k\in I_k}\bar\mu(\inf_{j\le k}
  \pi^ja_{i_j})
=\sum_{i_0\in I_0,\ldots,i_k\in I_k}\prod_{j=0}^k\bar\mu a_{i_j}\cr
&=\prod_{j=0}^k\sum_{i\in I_j}\bar\mu a_i
=\prod_{j=0}^k\bar\mu(\sup_{i\in I_j}a_i)
=\prod_{j=0}^k\bar\mu c_j.\cr}$$

\noindent As $c_0,\ldots,c_k$ are arbitrary,
$\sequence{k}{\pi^k[\frak A_0]}$ is independent.\ \Qed
}%end of prooflet

\spheader 387Bb If $\pi$ is an automorphism,  then
$\family{k}{\Bbb Z}{\pi^k[\frak A_0]}$ is independent\cmmnt{, by
385Sf}.

\spheader 387Bc Setting $A=\{a_i:i\in I\}\setminus\{0\}$, we have
$h(\pi,A)=H(A)$\cmmnt{, as in part (a) of the proof of 385R}, so
$h(\pi)\ge H(A)$.

\spheader 387Bd If $H(A)>0$, then $\frak A$ is atomless.
\prooflet{\Prf\ As $A$ contains at least two elements of non-zero
measure, $\gamma=\max_{a\in A}\bar\mu a<1$.   Because $\familyiI{a_i}$
is a Bernoulli partition, every member of $D_k(A,\pi)$ has measure at
most $\gamma^k$, for any $k\in\Bbb N$.
Thus any atom of $\frak A$ could have measure at most $\inf_{k\in\Bbb
N}\gamma^k=0$.\ \Qed}

\spheader 387Be If $\frak B$ is any closed subalgebra of $\frak A$ such
that $\pi[\frak B]\subseteq\frak B$, then
$h(\pi\restrp\frak B)\le h(\pi)$\cmmnt{, just because
$h(\pi\restrp\frak B)$ is calculated from
the action of $\pi$ on a smaller set of partitions}.   If $\frak C^+$ is
the closed subalgebra of $\frak A$ generated by
$\{\pi^ja_i:i\in I,\,j\in\Bbb N\}$, then
$\pi[\frak C^+]\subseteq\frak C^+$\cmmnt{ (compare
386Nb),} and $\pi\restrp\frak C^+$ is a one-sided Bernoulli shift with
root algebra $\frak A_0$ and entropy $H(A)$, so that

\Centerline{$H(A)=h(\pi\restrp\frak C^+)$\dvro{.}{}}

\cmmnt{\noindent by the Kolmogorov-Sina\v\i\ theorem (385P, 385R).}

\spheader 387Bf If $\pi$ is an automorphism, and $\frak C$ is the closed
subalgebra of $\frak A$ generated by $\{\pi^ja_i:i\in I,\,j\in\Bbb Z\}$,
then $\pi[\frak C]=\frak C$\cmmnt{ (386Nb)} and $\pi\restrp\frak C$ is
a two-sided Bernoulli shift with root algebra $\frak A_0$.

\spheader 387Bg Thus every Bernoulli partition for $\pi$ gives rise
to a factor of $(\frak A,\bar\mu,\pi)$ which is a one-sided Bernoulli
shift, and if $\pi$ is an automorphism we can extend this to the
corresponding two-sided Bernoulli shift.   If
$\pi$ has a generating Bernoulli partition then it is itself a Bernoulli
shift.

\spheader 387Bh Now suppose that $(\frak B,\bar\nu)$ is another
probability algebra, $\phi:\frak B\to\frak B$ is a measure-preserving
Boolean homomorphism, and $\familyiI{b_i}$ is a Bernoulli partition for
$\phi$ such that $\bar\nu b_i=\bar\mu a_i$ for every $i$.   We have a
unique measure-preserving Boolean
homomorphism $\theta^+:\frak C^+\to\frak B$ such that
$\theta^+(\pi^ja_i)=\phi^jb_i$ for every $i\in I$, $j\in\Bbb N$.
\cmmnt{(Apply 324P.)}   Now $\theta^+\pi=\phi\theta^+$.
\prooflet{(The set $\{a:\theta^+\pi a=\phi\theta^+ a\}$ is a closed
subalgebra of $\frak C^+$ containing every $\pi^ja_i$.)}

\spheader 387Bi If, in (h) above, $\pi$ and $\phi$ are both
automorphisms, then\cmmnt{ the same arguments show that} we have a
unique measure-preserving Boolean homomorphism
$\theta:\frak C\to\frak B$ such that $\theta a_i=b_i$ for every $i\in I$
and $\theta\pi=\phi\theta$.

\leader{387C}{Lemma} %3{8}6C
Let $(\frak A,\bar\mu)$ be an atomless probability algebra and
$\pi:\frak A\to\frak A$ an ergodic measure-preserving automorphism.
Let $\sequence{i}{a_i}$ be a partition of unity in $\frak A$, of finite
entropy, and $\sequence{i}{\gamma_i}$ a sequence of
non-negative real numbers such that

\Centerline{$\sum_{i=0}^{\infty}\gamma_i=1$,
\quad$\sum_{i=0}^{\infty}q(\gamma_i)\le h(\pi)$,}

\noindent where $q$ is the function of 385A.   Then for any $\epsilon>0$
we can find a partition $\sequence{i}{a'_i}$ of unity in $\frak A$ such
that

\inset{(i) $\{i:a'_i\ne 0\}$ is finite,}

\inset{(ii) $\sum_{i=0}^{\infty}|\gamma_i-\bar\mu a'_i|\le\epsilon$,}

\inset{(iii) $\sum_{i=0}^{\infty}\bar\mu(a'_i\Bsymmdiff a_i)
\le\epsilon+6\sqrt{\sum_{i=0}^{\infty}|\bar\mu a_i-\gamma_i|
  +\sqrt{2(H(A)-h(\pi,A))}}$}

\noindent where $A=\{a_i:i\in\Bbb N\}\setminus\{0\}$,

\inset{(iv) $H(A')\le h(\pi,A')+\epsilon$}

\noindent where $A'=\{a'_i:i\in\Bbb N\}\setminus\{0\}$.

\proof{{\bf (a)} Of course $h(\pi,A)\le H(A)$, as remarked in 385M, so the square root $\sqrt{2(H(A)-h(\pi,A))}$ gives no difficulty.   Set
$\beta=\sqrt{\sum_{i=0}^{\infty}|\bar\mu a_i-\gamma_i|
  +\sqrt{2(H(A)-h(\pi,A))}}$,
$\delta=\min(\bover14,\bover1{24}\epsilon)$.

There is a sequence $\sequence{i}{\bar\gamma_i}$ of non-negative real
numbers such that $\{i:\bar\gamma_i>0\}$ is finite,
$\sum_{i=0}^{\infty}\bar\gamma_i=1$,
$\sum_{i=0}^{\infty}|\bar\gamma_i-\gamma_i|\le 2\delta^2$ and
$\sum_{i=0}^{\infty}q(\bar\gamma_i)\le h(\pi)$.   \Prf\ Take
$k\in\Bbb N$
such that $\sum_{i=k}^{\infty}\gamma_i\le\delta^2$, and set
$\bar\gamma_i=\gamma_i\text{ for }i<k$,
$\bar\gamma_k=\sum_{i=k}^{\infty}\gamma_i$ and $\bar\gamma_i=0$ for
$i>k$;   then $q(\bar\gamma_k)\le\sum_{i=k}^{\infty}q(\gamma_i)$
(385Ab), so

\Centerline{$\sum_{i=0}^{\infty}q(\bar\gamma_i)
\le\sum_{i=0}^{\infty}q(\gamma_i)\le h(\pi)$,}

\noindent while

\Centerline{$\sum_{i=0}^{\infty}|\bar\gamma_i-\gamma_i|
\le\bar\gamma_k+\sum_{i=k}^{\infty}\gamma_i\le 2\delta^2$.  \Qed}

Because $\sum_{i=0}^{\infty}q(\bar\gamma_i)$ is finite, there is a
partition of unity $C$ in $\frak A$, of finite entropy, such
that $\sum_{i=0}^{\infty}q(\bar\gamma_i)\le h(\pi,C)+3\delta$;
replacing
$C$ by $C\vee A$ if need be (note that $C\vee A$ still has finite
entropy, by 385H), we may suppose that $C$ refines $A$.

There is a sequence $\sequence{i}{\gamma'_i}$ of non-negative real
numbers such that $\sum_{i=0}^{\infty}\gamma'_i=1$,
$\{i:\gamma'_i>0\}$ is finite,
$\sum_{i=0}^{\infty}|\gamma'_i-\gamma_i|\le 4\delta^2$ and

\Centerline{$\sum_{i=0}^{\infty}q(\gamma'_i)=h(\pi,C)+3\delta$.}

\noindent\Prf\ Take $k\in\Bbb N$ such that $\bar\gamma_i=0$ for $i>k$.
Take $r\ge 1$ such that
$\delta^2\ln(\bover{r}{\delta^2})\ge h(\pi,C)+3\delta$ and set

$$\eqalign{\tilde\gamma_i
&=(1-\delta^2)\bar\gamma_i\text{ for }i\le k\cr
&=\Bover1r\delta^2\text{ for }k+1\le i\le k+r\cr
&=0\text{ for }i>k+r.\cr}$$

\noindent Then

\Centerline{$\sum_{i=0}^{\infty}|\tilde\gamma_i-\bar\gamma_i|
=2\delta^2$,
\quad$\sum_{i=0}^{\infty}|\tilde\gamma_i-\gamma_i|\le 4\delta^2$,}

\Centerline{$\sum_{i=0}^{k+r}q(\bar\gamma_i)
\le h(\pi,C)+3\delta
\le\delta^2\ln(\Bover{r}{\delta^2})
=rq(\Bover{\delta^2}{r})
\le\sum_{i=0}^{k+r}q(\tilde\gamma_i)$.}

\noindent Now the function

\Centerline{$\alpha\mapsto\sum_{i=0}^{k+r}q(\alpha\bar\gamma_i
+(1-\alpha)\tilde\gamma_i):[0,1]\to\Bbb R$}

\noindent is continuous, so there is some $\alpha\in[0,1]$ such that
\Centerline{$\sum_{i=0}^{k+r}q(\alpha\bar\gamma_i
+(1-\alpha)\tilde\gamma_i)
=h(\pi,C)+3\delta$,}

\noindent and we can set $\gamma'_i
=\alpha\bar\gamma_i+(1-\alpha)\tilde\gamma_i$ for every $i$;
of course

\Centerline{$\sum_{i=0}^{\infty}|\gamma'_i-\gamma_i|
\le\alpha\sum_{i=0}^{\infty}|\bar\gamma_i-\gamma_i|
   +(1-\alpha)\sum_{i=0}^{\infty}|\tilde\gamma_i-\gamma_i|
\le 4\delta^2$.\ \Qed}

Set $M=\{i:\gamma'_i\ne 0\}$, so that $M$ is finite.

\medskip

{\bf (b)} Let $\eta\in\ocint{0,\delta}$ be so small that

\inset{(i) $|q(s)-q(t)|\le\Bover{\delta}{1+\#(M)}$ whenever $s$,
$t\in[0,1]$ and $|s-t|\le 3\eta$,}

\inset{(ii) $\sum_{c\in C}q(\min(\bar\mu c,2\eta))\le\delta$,}

\inset{(iii) $\eta\le\Bover16$.}

\noindent (Actually, (iii) is a consequence of (i).   For (ii) we must
of course rely on the fact that $\sum_{c\in C}q(\bar\mu c)$ is finite.)

Let $\nu$ be the probability measure on $M$ defined by saying that
$\nu\{i\}=\gamma'_i$ for every $i\in M$, and $\lambda$ the product
measure on $M^{\Bbb N}$.   Define $X_{ij}:M^{\Bbb N}\to\{0,1\}$, for
$i\in M$ and $j\in\Bbb N$, and $Y_j:M^{\Bbb N}\to\Bbb R$, for $j\in\Bbb N$,
by setting

$$\eqalign{X_{ij}(\omega)&=1\text{ if }\omega(j)=i,\cr
&=0\text{ otherwise},\cr
Y_j(\omega)
&=\ln(\gamma'_{\omega(j)})\text{ for every }\omega\in M^{\Bbb N}.\cr}$$

\noindent Then, for each $i\in M$, $\sequence{j}{X_{ij}}$ is an
independent sequence of random variables, all with expectation
$\gamma'_i$, and $\sequence{j}{Y_j}$ also is an independent sequence of
random variables, all with expectation

\Centerline{$\sum_{i\in M}\gamma'_i\ln\gamma'_i
=-\sum_{i=0}^{\infty}q(\gamma'_i)=-h(\pi,C)-3\delta$.}

Let $n\ge 1$ be so large that

\inset{(iv) $\bar\mu\Bvalue{w_n-h(\pi,C)\chi 1\ge\delta}<\eta$,
where

\Centerline{$w_n=\Bover1n\sum_{d\in
D_n(C,\pi)}\ln(\Bover1{\bar\mu d})\chi d$;}
}

\inset{(v)

\Centerline{$\Pr\bigl(\sum_{i\in M}
  |\Bover1n\sum_{j=0}^{n-1}X_{ij}-\gamma'_i|\le\eta\bigr)\ge 1-\delta$,}

\Centerline{$\Pr\bigl(|\Bover1n\sum_{j=0}^{n-1}Y_j+h(\pi,C)+3\delta|
  \le\delta\bigr)\ge 1-\delta$;}
}

\inset{(vi) $e^{n\delta}\ge 2$,
\quad$\Bover{1}{n+1}\le\eta$,
\quad$q(\Bover1{n+1})+q(\Bover{n}{n+1})\le\delta$;}

\noindent these will be true for all sufficiently large $n$, using the
Shannon-McMillan-Breiman theorem (386E) for (iv) and the strong law of
large numbers (in any of the forms 273D, 273H or 273I) for (v).

\medskip

{\bf (c)} There is a family $\langle b_{ji}\rangle_{j<n,i\in M}$
such that

\quad($\alpha$) for each $j<n$, $\langle b_{ji}\rangle_{i\in M}$ is a
partition of unity in $\frak A$,

\quad($\beta$) $\bar\mu(\inf_{j<n}b_{j,i(j)})
=\prod_{j=0}^{n-1}\gamma'_{i(j)}$ for every
$i(0),\ldots,i(n-1)\in M$,

\quad($\gamma$) $\sum_{i\in M}\bar\mu(b_{ji}\Bcap\pi^ja_i)
\ge 1-\beta^2-4\delta^2$ for every $j<n$.

\noindent\Prf\ Construct $\langle b_{ji}\rangle_{i\in M}$ for $j=n-1$,
$n-2,\ldots,0$, as follows.   Given $b_{ji}$, for $k<j<n$, such that

\Centerline{$\bar\mu(\inf_{j\le k}\pi^ja_{i(j)}
   \Bcap\inf_{k<j<n}b_{j,i(j)})
=\bar\mu(\inf_{j\le k}\pi^ja_{i(j)})
   \cdot\prod_{j=k+1}^{n-1}\gamma'_{i(j)}$}

\noindent for every $i(0),\ldots,i(n-1)\in M$ (of course this
hypothesis is trivial for $k=n-1$), let $B_k$ be the set of atoms of the
(finite) subalgebra of $\frak A$ generated by
$\{b_{ji}:i\in M,\,k<j<n\}$.   Then
$\bar\mu(b\Bcap d)=\bar\mu b\cdot\bar\mu d$ for every $b\in B_k$,
$d\in D_{k+1}(A,\pi)$.

Now

$$\eqalignno{\sum_{i=0}^{\infty}
&\sum_{c\in D_k(A,\pi)}|\bar\mu(\pi^ka_i\Bcap c)-\gamma'_i\bar\mu c|\cr
&\le\sum_{i=0}^{\infty}\sum_{c\in D_k(A,\pi)}
    |\bar\mu(\pi^ka_i\Bcap c)-\bar\mu a_i\cdot\bar\mu c|
  +\sum_{i=0}^{\infty}|\bar\mu a_i-\gamma'_i|
    \sum_{c\in D_k(A,\pi)}\bar\mu c\cr
&\le\sum_{i=0}^{\infty}|\gamma_i-\gamma'_i|
  +\sum_{i=0}^{\infty}|\bar\mu a_i-\gamma_i|
  +\sum_{i=0}^{\infty}\sum_{c\in D_k(A,\pi)}
    |\bar\mu(\pi^ka_i\Bcap c)-\bar\mu a_i\cdot\bar\mu c|\cr
&\le 4\delta^2
  +\sum_{i=0}^{\infty}|\bar\mu a_i-\gamma_i|
  +\sqrt{2\bigl(H(\pi^k[A])+H(D_k(A,\pi))-H(D_{k+1}(A,\pi))\bigr)}\cr
\noalign{\noindent (by 386I, because
$D_{k+1}(A,\pi)=\pi^k[A]\vee D_k(A,\pi)$)}
&\le 4\delta^2
  +\sum_{i=0}^{\infty}|\bar\mu a_i-\gamma_i|
  +\sqrt{2(H(A)-h(\pi,A))}\cr
\noalign{\noindent (because
$h(\pi,A)\le H(D_{k+1}(A,\pi))-H(D_k(A,\pi))$, by 386Lc)}
&=\beta^2+4\delta^2.\cr}$$

Choose a partition of unity $\langle b_{ki}\rangle_{i\in M}$ such that,
for each $c\in D_{k}(A,\pi)$, $b\in B_k$ and $i\in M$,

\Centerline{$\bar\mu(b_{ki}\Bcap b\Bcap c)=\gamma_i'\bar\mu(b\Bcap c)$,}

\Centerline{if
$\bar\mu(\pi^ka_i\Bcap b\Bcap c)\ge\gamma'_i\bar\mu(b\Bcap c)$ then $b_{ki}\Bcap b\Bcap c\Bsubseteq\pi^ka_i$,}

\Centerline{if
$\bar\mu(\pi^ka_i\Bcap b\Bcap c)\le\gamma'_i\bar\mu(b\Bcap c)$ then $\pi^ka_i\Bcap b\Bcap c\Bsubseteq b_{ki}$.}

\noindent (This is where I use the hypothesis that $\frak A$ is
atomless.)   Note that in these formulae we always have

\Centerline{$\bar\mu(b\Bcap c)=\bar\mu b\cdot\bar\mu c$,
\quad$\bar\mu(\pi^ka_i\Bcap b\Bcap c)
=\bar\mu(\pi^ka_i\Bcap c)\cdot\bar\mu b$.}

\noindent Consequently

$$\eqalignno{\sum_{i\in M}\bar\mu(\pi^ka_i\Bcap b_{ki})
&=\sum_{b\in B_k}\sum_{c\in D_k(A,\pi)}\sum_{i\in M}
  \bar\mu(b\Bcap c\Bcap(\pi^ka_i\Bcap b_{ki}))\cr
&=\sum_{b\in B_k}\sum_{c\in D_k(A,\pi)}\sum_{i=0}^{\infty}
  \min(\bar\mu(b\Bcap c\Bcap\pi^ka_i),\gamma'_i\bar\mu(b\Bcap c))\cr
&\ge\sum_{b\in B_k}\sum_{c\in D_k(A,\pi)}\sum_{i=0}^{\infty}
  \bar\mu(b\Bcap c\Bcap\pi^ka_i)
  -|\bar\mu(b\Bcap c\Bcap\pi^ka_i)-\gamma'_i\bar\mu(b\Bcap c)|\cr
&=1-\sum_{b\in B_k}\sum_{c\in D_k(A,\pi)}\sum_{i=0}^{\infty}
  |\bar\mu(b\Bcap c\Bcap\pi^ka_i)-\gamma'_i\bar\mu(b\Bcap c)|\cr
&=1-\sum_{b\in B_k}\sum_{c\in D_k(A,\pi)}\sum_{i=0}^{\infty}
  \bar\mu b\cdot|\bar\mu(c\Bcap\pi^ka_i)-\gamma'_i\bar\mu c|\cr
&=1-\sum_{c\in D_k(A,\pi)}\sum_{i=0}^{\infty}
  |\bar\mu(c\Bcap\pi^ka_i)-\gamma'_i\bar\mu c|
\ge 1-\beta^2-4\delta^2.\cr}$$

\noindent Also we have

\Centerline{$\bar\mu(b_{ki}\Bcap b\Bcap c)
=\gamma'_i\bar\mu b\cdot\bar\mu c=\bar\mu(b_{ki}\Bcap b)\cdot\bar\mu c$}

\noindent for every $b\in B_k$, $c\in D_k(A,\pi)$ and $i\in M$, so the
(downwards) induction proceeds.\ \Qed

\medskip

{\bf (d)} Let $B$ be the set of atoms of the algebra generated by
$\{b_{ji}:j<n,\,i\in M\}$.   For $b\in B$ and $d\in D_n(C,\pi)$ set

\Centerline{$I_{bd}
=\{j:j<n,\,\exists\,i\in M,\,b\Bsubseteq b_{ji},
  \,d\Bsubseteq \pi^ja_i\}$.}

\noindent Then, for any $j<n$,

\Centerline{$\sup\{b\Bcap d:b\in B,\,d\in D_n(C,\pi),\,j\in I_{bd}\}
=\sup_{i\in M}b_{ji}\Bcap\pi^ja_i$,}

\noindent because $C$ refines $A$, so every $\pi^ja_i$ is a supremum of
members of $D_n(C,\pi)$.   Accordingly

$$\eqalign{\sum_{b\in B,d\in D_n(C,\pi)}\#(I_{bd})\bar\mu(b\Bcap d)
&=\sum_{j=0}^{n-1}\sum_{i\in M}
  \bar\mu(b_{ji}\Bcap\pi^ja_i)
\ge n(1-\beta^2-4\delta^2).\cr}$$

\noindent Set

\Centerline{$e_0=\sup\{b\Bcap d:b\in B,\,d\in D_n(C,\pi),\,\#(I_{bd})
\ge n(1-\beta-4\delta)\}$;}

\noindent then $\bar\mu e_0\ge 1-\beta-\delta$.

\medskip

{\bf (e)} Let $B'\subseteq B$ be the set of those $b\in B$ such that

\Centerline{$\bar\mu b\le e^{-n(h(\pi,C)+2\delta)}$,
\quad
$\sum_{i\in M}|\gamma'_i-\Bover1n\#(\{j:j<n,\,b\Bsubseteq b_{ji}\})|
\le\eta$.}

\noindent Then $\bar\mu(\sup B')\ge 1-2\delta$.   \Prf\ Set

$$\eqalign{B'_1
&=\{b:b\in B,\,\bar\mu b\le e^{-n(h(\pi,C)+2\delta)}\}\cr
&=\{b:b\in B,\,h(\pi,C)+2\delta+\Bover1n\ln(\bar\mu b)\le 0\}\cr
&=\{\inf_{j<n}b_{j,i(j)}:i(0),\ldots,i(n-1)\in M,\,
  h(\pi,C)+2\delta+\Bover1n\sum_{j=0}^{n-1}\ln\gamma'_{i(j)}
  \le 0\}.\cr}$$

\noindent Then

$$\eqalign{\bar\mu(\sup B'_1)
&=\Pr(h(\pi,C)+2\delta+\Bover1n\sum_{j=0}^{n-1}Y_j\le 0)\cr
&\ge\Pr(|h(\pi,C)+3\delta+\Bover1n\sum_{j=0}^{n-1}Y_j|\le\delta)
\ge 1-\delta\cr}$$

\noindent by the choice of $n$.   On the other hand, setting

$$\eqalign{B'_2
&=\{b:b\in B,\,\sum_{i\in M}|\gamma'_i
  -\Bover1n\#(\{j:j<n,\,b\Bsubseteq b_{ji}\})|\le\eta\}\cr
&=\{\inf_{j<n}b_{j,i(j)}:i(0),\ldots,i(n-1)\in M,\,
  \sum_{i\in M}|\gamma'_i-\Bover1n\#(\{j:i(j)=i\})|\le\eta\},\cr}$$

\noindent we have

\Centerline{$\bar\mu(\sup B'_2)
=\Pr(\sum_{i\in M}|\gamma'_i-\Bover1n\sum_{j=0}^{n-1}X_{ij}|\le\eta)
\ge 1-\delta$}

\noindent by the other half of clause (b-v).   Since $B'=B'_1\cap B'_2$,
$\bar\mu(\sup B)\ge 1-2\delta$.\ \Qed

Let $D_0'$ be the set of those $d\in D_n(C,\pi)$ such that

\Centerline{$\Bover1n\ln(\Bover1{\bar\mu d})\le h(\pi,C)+\delta$,
\quad i.e.,
\quad$\bar\mu d\ge e^{-n(h(\pi,C)+\delta)}$;}

\noindent by (b-iv), $\bar\mu(\sup D_0')>1-\eta$.   Let
$D'\subseteq D_0'$ be a finite set such that
$\bar\mu(\sup D')\ge 1-\eta$.   If $d\in D'$ and $b\in B'$ then

\Centerline{$\bar\mu d\ge e^{-n(h(\pi,C)+\delta)}
\ge e^{n\delta}\bar\mu b\ge 2\bar\mu b$.}

\noindent Since $\bar\mu(\sup D')\le 1\le 2\bar\mu(\sup B')$ (remember
that $\delta\le\bover14$), $\#(D')\le\#(B')$.

Set $e_1=e_0\Bcap\sup B'$, so that $\bar\mu e_1\ge 1-\beta-3\delta$, and

\Centerline{$D''=\{d:d\in D',\,\bar\mu(d\Bcap e_1)
\ge\Bover12\bar\mu d\}$;}

\noindent then

\Centerline{$\bar\mu(\sup(D'\setminus D''))
\le 2\bar\mu(1\Bsetminus e_1)
\le 2\beta+6\delta$,}

\noindent so

\Centerline{$\bar\mu(\sup D'')\ge 1-2\beta-6\delta-\eta
\ge 1-2\beta-7\delta$.}

\medskip

{\bf (f)} If $d_1,\ldots,d_k\in D''$ are distinct,

\Centerline{$\bar\mu(\sup_{1\le i\le k}d_i\Bcap e_1)
\ge\Bover{k}2\inf_{i\le k}\bar\mu d_i
\ge k\sup_{b\in B'}\bar\mu b$,}

\noindent and
\Centerline{$\#(\{b:b\in B',\,b\Bcap e_0\Bcap\sup_{1\le i\le k}d_i\}\ne
0)\ge k$.}

\noindent By the Marriage Lemma (3A1K), there is an injective function
$f_0:D''\to B'$ such that $d\Bcap f_0(d)\Bcap e_0\ne 0$ for every
$d\in D''$.   Because $\#(D')\le\#(B')$, we can extend $f_0$ to an
injective function $f:D'\to B'$.

\medskip

{\bf (g)} By the Halmos-Rokhlin-Kakutani lemma, in the strong form
386C(iv), there is an $a\in\frak A$ such that
$a,\pi^{-1}a,\ldots,\discretionary{}{}{}\pi^{-n+1}a$ are disjoint
and $\bar\mu(a\Bcap d)=\bover{1}{n+1}\bar\mu d$ for every
$d\in D'\cup\{1\}$.
Set $\tilde e=\sup\{\pi^{-j}(a\Bcap d):j<n,\,d\in D'\}$.   Because
$\langle\pi^{-j}(a\Bcap d)\rangle_{j<n,d\in D'}$ is disjoint,

\Centerline{$\bar\mu\tilde e
=\sum_{j=0}^{n-1}\sum_{d\in D'}\bar\mu(a\Bcap d)
=\Bover{n}{n+1}\sum_{d\in D'}\bar\mu d
\ge(1-\eta)^2
\ge 1-2\eta$.}

\medskip

{\bf (h)} For $i\in M$, set

\Centerline{$a'_i
=\sup\{\pi^{-j}(a\Bcap d):j<n,\,d\in D',\,f(d)\Bsubseteq b_{ji}\}$.}

\noindent Then the $a'_i$ are disjoint.   \Prf\ Suppose that $i$,
$i'\in M$ are distinct.   If $j$, $j'<n$ and $d$, $d'\in D'$ and
$f(d)\Bsubseteq b_{ji}$, $f(d')\Bsubseteq b_{j'i'}$, then either
$j\ne j'$ or $j=j'$.   In the former case,

\Centerline{$\pi^{-j}(a\Bcap d)\Bcap\pi^{-j'}(a\Bcap d')
\Bsubseteq\pi^{-j}a\Bcap\pi^{-j'}a=0$.}

\noindent In the latter case, $b_{ji}\Bcap b_{j'i'}=0$, so
$f(d)\ne f(d')$ and $d\ne d'$ and

\Centerline{$\pi^{-j}(a\Bcap d)\Bcap\pi^{-j'}(a\Bcap d')
\Bsubseteq\pi^{-j}(d\Bcap d')=0$.  \Qed}

\noindent Observe that

\Centerline{$\sup_{i\in M}a'_i
=\sup_{j<n,d\in D'}\pi^{-j}(a\Bcap d)=\tilde e$}

\noindent because if $j<n$ and $d\in D'$ there must be some $i\in M$
such that $f(d)\Bsubseteq b_{ji}$.   Take any $m\in\Bbb N\setminus M$
and set $a'_m=1\Bsetminus\tilde e$, $a'_i=0$
for $i\in\Bbb N\setminus(M\cup\{m\})$;  then $\sequence{i}{a'_i}$ is a
partition of unity.   Now

$$\eqalignno{\sum_{i\in M}|\bar\mu a'_i-\gamma'_i|
&\le\sum_{i\in M}\gamma'_i|1-n\bar\mu(a\Bcap\sup D')|
  +\sum_{i\in M}|\bar\mu a'_i-n\gamma'_i\bar\mu(a\Bcap\sup D')|\cr
&\le 1-\Bover{n}{n+1}\bar\mu(\sup D')\cr
&\qquad+\sum_{i\in M}
 |\sum_{j=0}^{n-1}\sum_{\Atop{d\in D'}{f(d)\Bsubseteq b_{ji}}}
 \bar\mu(\pi^{-j}(a\Bcap d))-n\gamma'_i\sum_{d\in D'}\bar\mu(a\Bcap d)|
 \cr
&\le 1 - (1-\eta)^2\cr
&\qquad+\sum_{d\in D'}\sum_{i\in M}
     |\bar\mu(a\Bcap d)\cdot\#(\{j:j<n,\,f(d)\Bsubseteq b_{ji}\})
         -n\gamma'_i\bar\mu(a\Bcap d)|\cr
&\le 1-(1-\eta)^2
  +\sum_{d\in D'}\bar\mu(a\Bcap d)n\eta\cr
\noalign{\noindent (see the definition of $B'$ in (e) above)}
&\le 2\eta+n\eta\bar\mu a\le 3\eta.\cr}$$

\noindent So

$$\eqalign{\sum_{i=0}^{\infty}|\bar\mu a'_i-\gamma_i|
&\le\bar\mu a'_m
  +\sum_{i\in M}|\bar\mu a'_i-\gamma'_i|
  +\sum_{i=0}^{\infty}|\gamma'_i-\gamma_i|\cr
&\le 2\eta+3\eta+4\delta^2
\le 6\delta\le\epsilon.\cr}$$

We shall later want to know that $|\bar\mu a'_i-\gamma'_i|\le 3\eta$ for
every $i$;  for $i\in M$ this is covered by the formulae above, for
$i=m$ it is true because $\bar\mu a'_m=1-\bar\mu\tilde e\le 2\eta$ (see (g)),
and for other $i$ it is trivial.

\medskip

{\bf (i)} The next step is to show that
$\sum_{i=0}^{\infty}\bar\mu(a'_i\Bcap a_i)\ge 1-3\beta-12\delta$.
\Prf\ It is enough to consider the case in which $3\beta+12\delta<1$.
We know that

$$\eqalign{\sup_{i\in\Bbb N}a'_i\Bcap a_i
&\Bsupseteq\sup\{\pi^{-j}(a\Bcap d):j<n,\,d\in D',\cr
&\qquad\qquad\qquad\exists\,i\in M\text{ such that }
 f(d)\Bsubseteq b_{ji}\text{ and }d\Bsubseteq \pi^ja_i\}\cr
&=\sup\{\pi^{-j}(a\Bcap d):d\in D',\,j\in I_{f(d),d}\}\cr}$$

\noindent (see (d) for the definition of $I_{bd}$) has measure at least
$\sum_{d\in D'}\#(I_{f(d),d})\bar\mu(a\Bcap d)$.

For $d\in D''$, we arranged that $d\Bcap f(d)\Bcap e_0\ne 0$.   This
means that there must be some $b\in B$ and $d'\in D_n(C,\pi)$ such that
$d\Bcap f(d)\Bcap b\Bcap f(d')\ne 0$ and
$\#(I_{bd'})\ge n(1-\beta-4\delta)$;  of
course $b=f(d)$ and $d'=d$ (because $f$ is injective), so that $\#(I_{f(d),d})$ must be at least
$n(1-\beta-4\delta)$.   Accordingly

$$\eqalign{\sum_{i=0}^{\infty}\bar\mu(a'_i\Bcap a_i)
&\ge\sum_{d\in D''}n(1-\beta-4\delta)\bar\mu(a\Bcap d)
=n(1-\beta-4\delta)\Bover{1}{n+1}\bar\mu(\sup D'')\cr
&\ge(1-\eta)(1-\beta-4\delta)(1-2\beta-7\delta)
\ge 1-3\beta-12\delta. \text{ \Qed}\cr}$$

\noindent But this means that

\Centerline{$\sum_{i=0}^{\infty}\bar\mu(a'_i\Bsymmdiff a_i)
=2(1-\sum_{i=0}^{\infty}\bar\mu(a'_i\Bcap a_i))
\le 6\beta+24\delta\le\epsilon+6\beta$}

\noindent (using 386J for the equality).

\medskip

{\bf (j)} Finally, we need to estimate $H(A')$ and $h(\pi,A')$, where
$A'=\{a'_i:i\in\Bbb N\}\setminus\{0\}$.   For the former, we have
$H(A')\le h(\pi,C)+4\delta$.   \Prf\
$|\bar\mu a'_i-\gamma'_i|\le 3\eta$ for every $i$, by (h) above.
So by (b-i),

\Centerline{$H(A')
=\sum_{i\in M\cup\{m\}}q(\bar\mu a'_i)
\le \delta+\sum_{i=0}^{\infty}q(\gamma'_i)
=h(\pi,C)+4\delta$.  \Qed}

\medskip

{\bf (k)} Consider the partition of unity

\Centerline{$A''=A'\vee\{a,1\Bsetminus a\}$.}

\noindent Let $\frak D$ be the closed subalgebra of $\frak A$ generated
by $\{\pi^jc:j\in\Bbb Z,\,c\in A''\}$.

\medskip

\quad{\bf (i)} $a\Bcap d\in\frak D$ for every $d\in D'$.   \Prf\ Of
course $a\Bcap\tilde e\in\frak D$, because $1\Bsetminus\tilde e=a'_m$.   If
$d'\in D'$ and $d'\ne d$, then (because
$f$ is injective) $f(d)\ne f(d')$;  there must therefore be some $k<n$
and distinct $i$, $i'\in M$ such that $f(d)\Bsubseteq b_{ki}$ and
$f(d')\Bsubseteq b_{ki'}$.   But this means that
$\pi^{-k}(a\Bcap d)\Bsubseteq a'_i$ and
$\pi^{-k}(a\Bcap d')\Bsubseteq a'_{i'}$, so that
$a\Bcap d\Bsubseteq\pi^ka'_i$ and $a\Bcap d'\Bcap \pi^ka'_i=0$.

What this means is that if we set

\Centerline{$\tilde d=a\Bcap\tilde e\Bcap\inf\{\pi^ka'_i:k<n,\,i\in M,\,a\Bcap
d\Bsubseteq \pi^ka'_i\}$,}

\noindent we get a member of $\frak D$ (because every $a'_i\in\frak D$,
and $\pi[\frak D]=\frak D$) including $a\Bcap d$ and disjoint from
$a\Bcap d'$ whenever $d'\in D'$ and $d'\ne d$.   But as
$a\Bcap\pi^{-j}a=0$ if $0<j<n$, $a\Bcap\tilde e$ must be
$\sup\{a\Bcap d':d'\in D'\}$, and
$a\Bcap d=\tilde d$ belongs to $\frak D$.\ \Qed

\medskip

\quad{\bf (ii)} Consequently $c\Bcap\tilde e\in\frak D$ for every $c\in C$.
\Prf\ We have
$$\eqalignno{c\Bcap\tilde e
&=\sup\{c\Bcap \pi^{-j}(a\Bcap d):j<n,\,d\in D'\}\cr
&=\sup\{\pi^{-j}(\pi^jc\Bcap a\Bcap d):j<n,\,d\in D'\}\cr
&=\sup\{\pi^{-j}(a\Bcap d):j<n,\,d\in D',\,d\Bsubseteq \pi^jc\}\cr
\noalign{\noindent (because if $d\in D'$ and $j<n$ then either
$d\Bsubseteq\pi^jc$ or $d\Bcap\pi^jc=0$)}
&\in\frak D\cr}$$

\noindent because $a\Bcap d\in\frak D$ for every $d\in D'$ and
$\pi^{-1}[\frak D]=\frak D$.\ \Qed

\medskip

\quad{\bf (iii)} It follows that $h(\pi,A'')\ge h(\pi,C)-\delta$.
\Prf\ For any $c\in C$,

\Centerline{$\rho(c,\frak D)\le\bar\mu(c\Bsymmdiff(c\Bcap\tilde e))
=\bar\mu(c\Bsetminus\tilde e)\le\min(\bar\mu c,2\eta)\le\Bover13$.}

\noindent So

$$\eqalignno{h(\pi,C)
&\le h(\pi\restrp\frak D)+H(C|\frak D)\cr
\noalign{\noindent (386Ld, because $\pi[\frak D]=\frak D$)}
&\le h(\pi,A'')+\sum_{c\in C}q(\rho(c,\frak D))\cr
\noalign{\noindent (by the Kolmogorov-Sina\v\i\ theorem (385P) and
386Mb)}
&\le h(\pi,A'')+\sum_{c\in C}q(\min(\bar\mu c,2\eta))\cr
\noalign{\noindent (because $q$ is monotonic on $[0,\bover13]$)}
&\le h(\pi,A'')+\delta\cr}$$

\noindent by the choice of $\eta$.\ \Qed

\medskip

\quad{\bf (iv)} Finally, $h(\pi,A')\ge h(\pi,C)-2\delta$.   \Prf\ Using
386Lb,

$$\eqalign{h(\pi,C)-\delta
&\le h(\pi,A'')
\le h(\pi,A')+H(\{a,1\Bsetminus a\})\cr
&=h(\pi,A')+q(\bar\mu a)+q(1-\bar\mu a)\cr
&=h(\pi,A')+q(\Bover1{n+1})+q(\Bover{n}{n+1})
\le h(\pi,A')+\delta\cr}$$

\noindent by the choice of $n$.  \Qed

\medskip

{\bf (l)} Putting these together,

\Centerline{$H(A')\le h(\pi,C)+4\delta\le h(\pi,A')+6\delta
\le h(\pi,A')+\epsilon$,}

\noindent and the proof is complete.
}%end of proof of 387C

\leader{387D}{Corollary} %3{8}6D
Let $(\frak A,\bar\mu)$ be an atomless
probability algebra and
$\pi:\frak A\to\frak A$ an ergodic measure-preserving automorphism.
Let $\sequence{i}{a_i}$ be a partition of unity in
$\frak A$, of finite entropy, and $\sequence{i}{\gamma_i}$ a sequence of
non-negative real numbers such that

\Centerline{$\sum_{i=0}^{\infty}\gamma_i=1$,
\quad$\sum_{i=0}^{\infty}q(\gamma_i)\le h(\pi)$.}

\noindent Then for any $\epsilon>0$
we can find a Bernoulli partition $\sequence{i}{a^*_i}$ for $\pi$ such
that $\bar\mu a^*_i=\gamma_i$ for every $i\in\Bbb N$ and

\Centerline{$\sum_{i=0}^{\infty}\bar\mu(a^*_i\Bsymmdiff a_i)
\le\epsilon+6\sqrt{\sumop_{i=0}^{\infty}|\bar\mu a_i-\gamma_i|
  +\sqrt{2(H(A)-h(\pi,A))}}$,}

\noindent writing $A=\{a_i:i\in\Bbb N\}\setminus\{0\}$.

\proof{{\bf (a)} Set $\beta=\sqrt{\sum_{i=0}^{\infty}
|\bar\mu a_i-\gamma_i|+\sqrt{2(H(A)-h(\pi,A))}}$.
Let $\sequencen{\epsilon_n}$ be a sequence of strictly positive real
numbers such that

\Centerline{$\sum_{n=0}^{\infty}\epsilon_n
+6\sqrt{\epsilon_n+\sqrt{2\epsilon_n}}\le\epsilon$.}

\noindent Using 387C, we can choose inductively, for $n\in\Bbb N$,
partitions of unity $\sequence{i}{a_{ni}}$ such that, for each
$n\in\Bbb N$,

\Centerline{$\sum_{i=0}^{\infty}|\gamma_i-\bar\mu a_{ni}|
\le\epsilon_n$,}

\Centerline{$H(A_n)\le h(\pi,A_n)+\epsilon_n<\infty$}

\noindent (writing $A_n=\{a_{ni}:i\in\Bbb N\}\setminus\{0\}$),

\Centerline{$\sum_{i=0}^{\infty}\bar\mu(a_{n+1,i}\Bsymmdiff a_{ni})
   \le \epsilon_{n+1}+6\sqrt{\epsilon_n+\sqrt{2\epsilon_n}}$,}

\noindent while

\Centerline{$\sum_{i=0}^{\infty}\bar\mu(a_{0i}\Bsymmdiff a_i)
\le\epsilon_0+6\beta$.}

On completing the induction, we see that

\Centerline{$\sum_{n=0}^{\infty}\sum_{i=0}^{\infty}
  \bar\mu(a_{n+1,i}\Bsymmdiff a_{ni})
\le \sum_{n=1}^{\infty}\epsilon_n
+\sum_{n=0}^{\infty}6\sqrt{\epsilon_n+\sqrt{2\epsilon_n}}
<\infty$.}

\noindent In particular, given $i\in\Bbb N$,
$\sum_{n=0}^{\infty}\bar\mu(a_{n+1,i}\Bsymmdiff a_{ni})$ is finite, so
$\sequencen{a_{ni}}$ is a Cauchy sequence in the complete metric space
$\frak A$ (323Gc), and has a limit $a^*_i$, with

\Centerline{$\bar\mu a^*_i=\lim_{n\to\infty}\bar\mu a_{ni}=\gamma_i$}

\noindent (323C).   If $i\ne j$,

\Centerline{$a^*_i\Bcap a^*_j=\lim_{n\to\infty}a_{ni}\Bcap a_{nj}=0$}

\noindent (using 323B), so $\sequence{i}{a^*_i}$ is disjoint;  since

\Centerline{$\sum_{i=0}^{\infty}\bar\mu
a^*_i=\sum_{i=0}^{\infty}\gamma_i=1$,}

\noindent $\sequence{i}{a^*_i}$ is a partition of unity.   We also have

$$\eqalign{\sum_{i=0}^{\infty}\bar\mu(a^*_i\Bsymmdiff a_i)
&\le\sum_{i=0}^{\infty}\bar\mu(a_{0i}\Bsymmdiff a_i)
  +\sum_{n=0}^{\infty}\sum_{i=0}^{\infty}
    \bar\mu(a_{n+1,i}\Bsymmdiff a_{ni})\cr
&\le\epsilon_0+6\beta
  +\sum_{n=1}^{\infty}\epsilon_n
  +\sum_{n=0}^{\infty}6\sqrt{\epsilon_n+\sqrt{2\epsilon_n}}
\le\epsilon+6\beta.\cr}$$

\medskip

{\bf (b)} Now take any $i(0),\ldots,i(k)\in\Bbb N$.    For each $j<k$,
$n\in\Bbb N$,

\Centerline{$H(\pi^j[A_n])+H(D_j(A_n,\pi))-H(D_{j+1}(A_n,\pi))
\le H(A_n)-h(\pi,A_n)\le\epsilon_n$}

\noindent (using 386Lc).   But this means that

$$\sum_{d\in D_j(A_n,\pi)}\sum_{i=0}^{\infty}|\bar\mu(d\Bcap\pi^ja_{ni})
-\bar\mu d\cdot\bar\mu a_{ni}|\le\sqrt{2\epsilon_n},$$

\noindent by 386I.   {\it A fortiori},

\Centerline{$|\bar\mu(d\Bcap\pi^ja_{ni})
-\bar\mu d\cdot\bar\mu a_{ni}|\le\sqrt{2\epsilon_n}$}

\noindent for each $d\in D_j(A_n,\pi)$, $i\in\Bbb N$.   Inducing on
$r$, we see that

\Centerline{$|\bar\mu(\inf_{j\le r}\pi^ja_{n,i(j)})
  -\prod_{j=0}^r\bar\mu a_{n,i(j)}|
\le r\sqrt{2\epsilon_n}\to 0$}

\noindent as $n\to\infty$, for any $r\le k$.   Because $\bar\mu$,
$\Bcap$ and $\pi$ are all continuous (323C, 323B, 324Kb),

$$\eqalign{\bar\mu(\inf_{j\le k}\pi^ja^*_{i(j)})
&=\lim_{n\to\infty}\bar\mu(\inf_{j\le k}\pi^ja_{n,i(j)})\cr
&=\lim_{n\to\infty}\prod_{j=0}^k\bar\mu a_{n,i(j)}
=\prod_{j=0}^k\gamma_{i(j)}.\cr}$$

\noindent As $i(0),\ldots,i(k)$ are arbitrary, $\sequence{i}{a^*_i}$ is
a Bernoulli partition for $\pi$.
}%end of proof of 387D

\leader{387E}{Sina\v\i's %3{8}6E
theorem (atomic case)}\cmmnt{ ({\smc
Sina\v\i\ 62})} Let $(\frak A,\bar\mu)$ be
an atomless probability algebra and $\pi:\frak A\to\frak A$ an ergodic
measure-{\vthsp}preserving automorphism.   Let $\sequence{i}{\gamma_i}$
be a sequence of non-negative real numbers such that
$\sum_{i=0}^{\infty}\gamma_i=1$ and
$\sum_{i=0}^{\infty}q(\gamma_i)\le h(\pi)$.   Then there is a Bernoulli
partition $\sequence{i}{a_i^*}$ for
$\pi$ such that $\bar\mu a_i^*=\gamma_i$ for every $i\in\Bbb N$.

\proof{ Apply 387D from any starting point, e.g., $a_0=1$, $a_i=0$
for $i>0$.
}%end of proof of 387E

\leader{387F}{Lemma} %3{8}6F
Let $(\frak A,\bar\mu)$ be an atomless probability
algebra and $\pi$ a measure-{\vthsp}preserving automorphism of
$\frak A$.   Let $\sequence{i}{b_i}$ and $\sequence{i}{c_i}$ be Bernoulli partitions for $\pi$, of the same finite entropy, and write $\frak B$, $\frak C$ for the closed subalgebras of $\frak A$
generated by $\{\pi^jb_i:i\in\Bbb N,\,j\in\Bbb Z\}$ and
$\{\pi^jc_i:i\in\Bbb N,\,j\in\Bbb Z\}$.   Suppose that
$\frak C\subseteq\frak B$.   Then for
any $\epsilon>0$ we can find a Bernoulli partition $\sequence{i}{d_i}$
for $\pi$ such that

(i) $d_i\in\frak C$ for every $i\in\Bbb N$,

(ii) $\bar\mu d_i=\bar\mu b_i$ for every $i\in\Bbb N$,

(iii) $\bar\mu(\phi c_i\Bsymmdiff c_i)\le\epsilon$ for every
$i\in\Bbb N$,

\noindent where $\phi:\frak B\to\frak C$ is the measure-preserving
Boolean homomorphism such that $\phi b_i=d_i$ for every $i$ and
$\pi\phi=\phi\pi$\cmmnt{ (387Bi)}.

\proof{{\bf (a)} Set $B=\{b_i:i\in\Bbb N\}\setminus\{0\}$,
$C=\{c_i:i\in\Bbb N\}\setminus\{0\}$.   If only one $c_i$ is
non-zero, then $H(C)=0$, so $H(B)=0$
and $\frak B=\{0,1\}$, in which case $\frak B=\frak C$ and we take
$d_i=b_i$ and stop.    Otherwise, $\frak C$ is atomless (387Bd).

For $k\in\Bbb N$, let $\frak B_k\subseteq\frak B$ be the closed
subalgebra of
$\frak A$ generated by $\{\pi^jb_i:i\le k,\,|j|\le k\}$.
Because $\frak C\subseteq\frak B$, there is an $m\in\Bbb N$ such that

\Centerline{$\rho(c_i,\frak B_m)\le\Bover14\epsilon$ for every
$i\in\Bbb N$}

\noindent (386K).   Let $\eta$, $\xi>0$ be such that

\Centerline{$\eta+6\root{4}\of{2\eta}\le\Bover{\epsilon}{4(2m+1)}$,
\quad$\xi\le\min(\Bover{\epsilon}4,\Bover16)$,
\quad$\sum_{i=0}^{\infty}q(\min(2\xi,\bar\mu c_i))\le\eta$.}

\noindent (The last is achievable because
$\sum_{i=0}^{\infty}q(\bar\mu c_i)$ is finite.)   Let $r\ge m$ be such
that

\Centerline{$\rho(c_i,\frak B_r)\le\xi$ for every $i\in\Bbb N$.}

\noindent Let $n\ge r$ be such that

\Centerline{$\Bover{2r+1}{2n+2}\le\xi$,
\quad$\bar\mu c_i\le\xi$ for every $i>n$.}

\medskip

{\bf (b)} Let
$\sequence{i}{b'_i}$ be a partition of unity in $\frak C$ such that
$\bar\mu b'_i=\bar\mu b_i$ for every $i\in\Bbb N$.
Let $U$ be the set of atoms of the subalgebra of $\frak B$
generated by
$\{\pi^jb_i:i\le n,\,|j|\le n\}\cup\{\pi^jc_i:i\le n,\,|j|\le n\}$, and
$V$ the set of atoms of the subalgebra of $\frak C$ generated by
$\{\pi^jb'_i:i\le n,\,|j|\le n\}\cup\{\pi^jc_i:i\le n,\,|j|\le n\}$.
For each $v\in V$, choose a disjoint family $\family{u}{U}{d_{vu}}$ in
$\frak C$ such that $\sup_{u\in U}d_{vu}=v$ and
$\bar\mu d_{vu}=\bar\mu(v\Bcap u)$ for every $u\in U$.   By 386C(iv),
there is an $a\in\frak C$ such that
$a,\pi a,\ldots,\pi^{2n}a$ are disjoint and
$\bar\mu(a\Bcap d_{vu})=\Bover1{2n+2}\bar\mu(d_{vu})$ for every
$u\in U$ and $v\in V$.   ($\pi\restrp\frak C$ is a Bernoulli shift,
therefore ergodic, by 385Se, therefore aperiodic, by 386D.)   Set
$e=\sup_{|j|\le n}\pi^ja$,
$\tilde e=\sup_{|j|\le n-r}\pi^ja$;  then

\Centerline{$\bar\mu e=(2n+1)\bar\mu a=\Bover{2n+1}{2n+2}$,
\quad$\bar\mu\tilde e=(2(n-r)+1)\bar\mu a=1-\Bover{2r+1}{2n+2}$.}

\noindent Let $\frak C_{\tilde e}$ be the principal ideal of $\frak C$
generated by $\tilde e$.

\medskip

{\bf (c)} The family
$\langle\pi^{-j}(a\Bcap d_{vu})\rangle_{|j|\le n,u\in U,v\in V}$ is
disjoint.   \Prf\ All we have to note is that the families
$\langle d_{vu}\rangle_{u\in U,v\in V}$ and

\Centerline{$\langle\pi^{-j}a\rangle_{|j|\le n}
=\langle\pi^{-n}(\pi^{n+j}a)\rangle_{|j|\le n}$}

\noindent are disjoint.\ \QeD\  Consequently, if we set

\Centerline{$\hat b_i
=\sup_{|j|\le n}\sup_{v\in V}\sup_{u\in U,u\Bsubseteq\pi^jb_i}
  \pi^{-j}(a\Bcap d_{vu})\in\frak C$}

\noindent for $i\in\Bbb N$, $\sequence{i}{\hat b_i}$ is disjoint, since
a given triple $(j,u,v)$ can contribute to at most one $\hat b_i$.

Of course $\hat b_i\Bsubseteq\sup_{|j|\le n}\pi^{-j}a=e$ for every
$i$.   If $i\le n$, we also have
$\bar\mu\hat b_i=\bar\mu e\cdot\bar\mu b_i$.   \Prf\ For $|j|\le n$,
$\pi^jb_i$ is a union of members of $U$, so

$$\eqalignno{\bar\mu\hat b_i
&=\sum_{j=-n}^n\sum_{v\in V}\sum_{u\in U,u\Bsubseteq\pi^jb_i}
  \bar\mu(\pi^{-j}(a\Bcap d_{vu}))\cr
\displaycause{because
$\langle\pi^{-j}(a\Bcap d_{vu})\rangle_{|j|\le n,u\in U,v\in V}$ is
disjoint}
&=\sum_{j=-n}^n\sum_{v\in V}\sum_{u\in U,u\Bsubseteq\pi^jb_i}
  \bar\mu(a\Bcap d_{vu})
=\Bover1{2n+2}\sum_{j=-n}^n\sum_{v\in V}
  \sum_{u\in U,u\Bsubseteq\pi^jb_i}
  \bar\mu d_{vu}\cr
\displaycause{by the choice of $a$}
&=\Bover1{2n+2}\sum_{j=-n}^n\sum_{v\in V}
  \sum_{u\in U,u\Bsubseteq\pi^jb_i}
  \bar\mu(v\Bcap u)\cr
\displaycause{by the choice of $d_{vu}$}
&=\Bover1{2n+2}\sum_{j=-n}^n
  \sum_{u\in U,u\Bsubseteq\pi^jb_i}
  \bar\mu u
=\Bover1{2n+2}\sum_{j=-n}^n
    \bar\mu(\pi^jb_i)\cr
\displaycause{because $\pi^jb_i$ is a disjoint union of members of $U$
when $i\le n$, $|j|\le n$}
&=\Bover{2n+1}{2n+2}\bar\mu b_i
=\bar\mu e\cdot\bar\mu b_i. \text{ \Qed}\cr}$$

\noindent Again because $\frak C$ is atomless, we can choose a partition
of unity $\sequence{i}{b^*_i}$ in $\frak C$ such that
$\bar\mu b^*_i=\bar\mu b_i$ for every $i$,
while $b^*_i\Bsupseteq\hat b_i$ and $b^*_i\Bcap e=\hat b_i$ for
$i\le n$.

\medskip

{\bf (d)} Let $\frak E$ be the finite subalgebra of $\frak B$ generated
by $\{\pi^jb_i:i\le n,\,|j|\le r\}\cup\{\pi^jc_i:i\le n,\,|j|\le r\}$.
Define $\theta:\frak E\to\frak C_{\tilde e}$ by setting

\Centerline{$\theta b=\sup_{|j|\le n-r}\sup_{v\in V}\sup_{u\in
U,u\Bsubseteq\pi^jb}\pi^{-j}(a\Bcap d_{vu})$}

\noindent for $b\in\frak E$.

\medskip

\quad{\bf (i)} $\theta$ is a Boolean homomorphism.
\Prf\  The point is
that if $|j|\le n-r$ and $b\in\frak E$, then $\pi^jb$ belongs to the
algebra generated by
$\{\pi^kb_i:i\le n,\,|k|\le n\}\cup\{\pi^kc_i:i\le n,\,|k|\le n\}$, so
is a union of members of $U$.   Since each map

$$b\mapsto
\pi^{-j}(a\Bcap d_{vu})\text{ if }u\Bsubseteq\pi^jb,
\quad 0\text{ otherwise}$$

\noindent is a Boolean homomorphism from $\frak E$ to the principal
ideal generated by $\pi^{-j}(a\Bcap d_{vu})$, and
\discrcenter{468pt}
{$\langle\pi^{-j}(a\Bcap d_{vu})\rangle_{|j|\le n-r,u\in U,v\in V}$ }is a
partition of unity in $\frak C_{\tilde e}$, $\theta$ also is a Boolean
homomorphism.\ \QeD\

\medskip

\quad{\bf (ii)} $\bar\mu(\theta b)\le\bar\mu b$
for every $b\in\frak E$.   \Prf\ (Compare (c) above.)

$$\eqalign{\bar\mu(\theta b)
&=\sum_{j=-n+r}^{n-r}\sum_{v\in V}\sum_{u\in U,u\Bsubseteq\pi^jb}
  \bar\mu\pi^{-j}(a\Bcap d_{vu})\cr
&=\Bover1{2n+2}\sum_{j=-n+r}^{n-r}\sum_{v\in V}
  \sum_{u\in U,u\Bsubseteq\pi^jb}
  \bar\mu(v\Bcap u)
=\Bover{2n-2r+1}{2n+2}\bar\mu b
\le\bar\mu b. \text{ \Qed}\cr}$$

\medskip

\quad{\bf (iii)} $\theta(\pi^kb_i)=\tilde e\Bcap\pi^k b^*_i$
for $i\le n$, $|k|\le r$.   \Prf\ Of course $\pi^kb_i\in\frak
E$.   If $|j|\le n-r$, then $|j+k|\le n$, so

$$\eqalign{\pi^{-j}a\Bcap\theta(\pi^kb_i)
&=\sup_{v\in V}\sup_{u\in U,u\Bsubseteq\pi^{j+k}b_i}
   \pi^{-j}(a\Bcap d_{vu})\cr
&=\pi^k\bigl(\sup_{v\in V}\sup_{u\in U,u\Bsubseteq\pi^{j+k}b_i}
   \pi^{-j-k}(a\Bcap d_{vu})\bigr)\cr
&=\pi^k(\pi^{-j-k}a\Bcap\hat b_i)
=\pi^{-j}a\Bcap\pi^k(e\Bcap b^*_i)
=\pi^{-j}a\Bcap\pi^k b^*_i\cr}$$

\noindent because $\pi^{-j}a\Bsubseteq\pi^ke$.   Taking the supremum of
these pieces we have

\Centerline{$\theta(\pi^kb_i)
=\sup_{|j|\le n-r}\pi^{-j}a\Bcap\theta(\pi^kb_i)
=\sup_{|j|\le n-r}\pi^{-j}a\Bcap\pi^kb^*_i
=\tilde e\Bcap\pi^kb^*_i$.  \Qed}

It follows that

\Centerline{$\theta(\pi^k(1\Bsetminus\sup_{i\le l}b_i))
=\tilde e\Bcap\pi^k(1\Bsetminus\sup_{i\le l}b_i^*))$}

\noindent if $l\le n$ and $|k|\le r$.

\medskip

\quad{\bf (iv)} Finally, $\theta c_i=c_i\Bcap\tilde e$ for every
$i\le n$.  \Prf\  If $|j|\le n-r$ and $v\in V$ then either $v\Bsubseteq\pi^jc_i$
or $v\Bcap\pi^jc_i=0$.   In the former case,

\Centerline{$d_{vu}=v\Bcap u=0$ whenever $u\in U$ and
$u\notBsubseteq\pi^jc_i$,}

\noindent so that
\Centerline{$v=\sup_{u\in U}d_{vu}=\sup_{u\in
U,u\Bsubseteq\pi^jc_i}d_{vu}$;}

\noindent in the latter case, $d_{vu}=v\Bcap u=0$ whenever $u\Bsubseteq
\pi^jc_i$.   So we have

\Centerline{$v\Bcap\pi^jc_i=\sup_{u\in U,u\Bsubseteq\pi^jc_i}d_{vu}$}

\noindent for every $v\in V$, and

$$\eqalignno{\theta c_i
&=\sup_{|j|\le n-r}\sup_{v\in V}
    \sup_{u\in U,u\Bsubseteq\pi^jc_i}\pi^{-j}(a\Bcap d_{vu})\cr
&=\sup_{|j|\le n-r}\pi^{-j}(a\Bcap\sup_{v\in V}
    \sup_{u\in U,u\Bsubseteq\pi^jc_i}d_{vu})\cr
&=\sup_{|j|\le n-r}\pi^{-j}(a\Bcap
    \sup_{v\in V}(v\Bcap\pi^jc_i))\cr
&=\sup_{|j|\le n-r}\pi^{-j}(a\Bcap\pi^jc_i)
=c_i\Bcap\sup_{|j|\le n-r}\pi^{-j}a
=c_i\Bcap\tilde e.\text{ \Qed}\cr}$$

\medskip

{\bf (e)} Let $\frak B^*$ be the closed subalgebra of
$\frak A$ generated by $\{\pi^jb^*_i:i\in\Bbb N,\,|j|\in\Bbb Z\}$.
Then for every $b\in\frak B_r$ there is a $b^*\in\frak B^*$ such that
$\theta b=b^*\Bcap\tilde e$.   \Prf\ The set of $b$ for which this is
true is a
subalgebra of $\frak A$ containing $\pi^kb_i$ for $i\le r$ and
$|k|\le r$, by (d-iii).\ \QeD\  It follows that

\Centerline{$\rho(c_i,\frak B^*)\le 2\xi$ for $i\in\Bbb N$.}

\noindent\Prf\ If $i>n$ this is trivial, because $\bar\mu c_i\le\xi$, by
the choice of $n$.
Otherwise, $c_i\in\frak E$.   Take $b\in\frak B_r$
such that $\bar\mu(b\Bsymmdiff c_i)=\rho(c_i,\frak B_r)\le\xi$ (386Ma).
Let $b^*\in\frak B^*$ be such
that $\theta b=b^*\Bcap\tilde e$.   Then

$$\eqalignno{\rho(c_i,\frak B^*)
&\le\bar\mu(c_i\Bsymmdiff b^*)
\le 1-\bar\mu\tilde e+\bar\mu(\tilde e\Bcap(c_i\Bsymmdiff b^*))\cr
&=\Bover{2r+1}{2n+2}
  +\bar\mu((\tilde e\Bcap c_i)\Bsymmdiff \theta b)
=\Bover{2r+1}{2n+2}
  +\bar\mu(\theta c_i\Bsymmdiff\theta b)\cr
\noalign{\noindent (by (d-iv))}
&=\Bover{2r+1}{2n+2}
  +\bar\mu(\theta(c_i\Bsymmdiff b))
\le\Bover{2r+1}{2n+2}
  +\bar\mu(c_i\Bsymmdiff b)\cr
\noalign{\noindent (by (d-ii))}
&\le 2\xi\cr}$$

\noindent by the choice of $n$.\ \Qed

\medskip

{\bf (f)} Set $B^*=\{b^*_i:i\in\Bbb N\}\setminus\{0\}$.   Then
$H(B^*)=h(\pi,C)\le h(\pi,B^*)+\eta$.  \Prf\

$$\eqalignno{H(B^*)
&=H(B)=H(C)\cr
\noalign{\noindent (because $\bar\mu b^*_i=\bar\mu b_i$ for every $i$,
and we supposed from the beginning that $H(C)=H(B)$)}
&=h(\pi,C)\cr
\noalign{\noindent (because $C$ is a Bernoulli partition, see 387Bc)}
&\le h(\pi\restrp\frak B^*)+H(C|\frak B^*)\cr
\noalign{\noindent (386Ld)}
&\le h(\pi\restrp\frak B^*)
  +\sum_{i=0}^{\infty}q(\rho(c_i,\frak B^*))\cr
\displaycause{386Mb}
&\le h(\pi,B^*)+\sum_{i=0}^{\infty}q(\min(2\xi,\bar\mu c_i))\cr
\noalign{\noindent (by the Kolmogorov-Sina\v\i\ theorem, 385P(ii), and
(e) above, recalling that $\xi\le\bover16$, so that $q$ is monotonic on
$[0,2\xi]$)}
&\le h(\pi,B^*)+\eta\cr}$$

\noindent by the choice of $\xi$.\ \Qed

Note also that $H(B^*)=h(\pi,C)\le h(\pi)$.

\medskip

{\bf (g)} By 387D, applied to $\pi\restrp\frak C$ and the partition
$\sequence{i}{b_i^*}$ of unity in $\frak C$ and the sequence
$\sequence{i}{\gamma_i}=\sequence{i}{\bar\mu b_i^*}$, we have a
Bernoulli partition $\sequence{i}{d_i}$ in
$\frak C$ such that $\bar\mu d_i=\bar\mu b^*_i=\bar\mu b_i$ for every
$i\in\Bbb N$ and

\Centerline{$\sum_{i=0}^{\infty}\bar\mu(d_i\Bsymmdiff b^*_i)
\le\eta+6\root{4}\of{2\eta}\le\Bover{\epsilon}{4(2m+1)}$.}

\noindent Let $\frak D\subseteq\frak C$ be the closed subalgebra of
$\frak A$ generated by $\{\pi^jd_i:i\in\Bbb N,\,j\in\Bbb Z\}$.   Then
$(\frak B,\pi\restrp\frak B,\sequence{i}{b_i})$ is isomorphic to
$(\frak D,\pi\restrp\frak D,\sequence{i}{d_i})$, with an isomorphism
$\phi:\frak B\to\frak D$ such that $\phi\pi=\pi\phi$ and $\phi b_i=d_i$
for every $i\in\Bbb N$ (387Bi).

\medskip

{\bf (h)} Set

\Centerline{$e^*
=\tilde e\Bsetminus\sup_{|j|\le m,i\in\Bbb N}
  \pi^j(d_i\Bsymmdiff b^*_i)$.}

\noindent Then $\phi(\pi^jb_i)\Bcap e^*=\theta(\pi^jb_i)\Bcap e^*$
whenever $i\le m$ and $|j|\le m$.   \Prf\

$$\eqalign{\phi(\pi^jb_i)\Bcap e^*
&=\pi^j(\phi b_i)\Bcap e^*
=\pi^jd_i\Bcap e^*\cr
&=\pi^jb^*_i\Bcap e^*
=\pi^jb^*_i\Bcap\tilde e\Bcap e^*
=\theta(\pi^jb_i)\Bcap e^*\cr}$$

\noindent by (d-iii), because $i$ and $|j|$ are both at most
$m\le r\le n$.\ \QeD\
Since $b\mapsto\phi b\Bcap e^*:\frak A\to\frak A_{e^*}$,
$b\mapsto\theta b\Bcap e^*:\frak E\to\frak A_{e^*}$ are Boolean
homomorphisms, $\phi b\Bcap e^*=\theta b\Bcap e^*$ for every
$b\in\frak B_m$.

Now $\bar\mu(c_i\Bsymmdiff\phi c_i)\le\epsilon$ for every $i\in\Bbb N$.
\Prf\ If $i>n$ then of course

\Centerline{$\bar\mu(\phi c_i\Bsymmdiff c_i)\le 2\bar\mu c_i
\le 2\xi\le\epsilon$.}

\noindent If $i\le n$, then (by the choice of $m$) there is a
$b\in\frak B_m$ such that
$\bar\mu(c_i,b)\le\bover14\epsilon$.   So

$$\eqalign{\phi c_i\Bsymmdiff c_i
&\Bsubseteq(\phi c_i\Bsymmdiff\phi b)
  \Bcup(\phi b\Bsymmdiff\theta b)
  \Bcup(\theta b\Bsymmdiff\theta c_i)
  \Bcup(\theta c_i\Bsymmdiff c_i)\cr
&\Bsubseteq\phi(c_i\Bsymmdiff b)
  \Bcup(1\Bsetminus e^*)
  \Bcup\theta(b\Bsymmdiff c_i)\cr}$$

\noindent (using the definition of $e^*$ and (d-iv))
has measure at most

$$\eqalignno{\bar\mu(c_i\Bsymmdiff b)&+\bar\mu(1\Bsetminus e^*)
  +\bar\mu(b\Bsymmdiff c_i)\cr
\displaycause{by (d-ii), since $b$ and $c_i$ both belong to $\frak E$}
&\le 2\bar\mu(c_i\Bsymmdiff b)+\bar\mu(1\Bsetminus\tilde e)
  +(2m+1)\sum_{i=0}^{\infty}\bar\mu(d_i\Bsymmdiff b^*_i)\cr
&\le\Bover{\epsilon}2+\Bover{2r+1}{2n+2}+\Bover{\epsilon}4
\le\epsilon,\cr}$$

\noindent as required.\ \Qed
}%end of proof of 387F

\leader{387G}{Lemma} %3{8}6G
Let $(\frak A,\bar\mu)$ be an atomless probability
algebra and $\pi$ a measure-preserving automorphism of
$\frak A$.   Let
$\sequence{i}{b_i}$ and $\sequence{i}{c_i}$ be Bernoulli partitions for
$\pi$, of the same finite entropy, and write $\frak B$, $\frak C$ for
the closed subalgebras generated by
$\{\pi^jb_i:i\in\Bbb N,\,j\in\Bbb Z\}$
and $\{\pi^jc_i:i\in\Bbb N,\,j\in\Bbb Z\}$.   Suppose that
$\frak C\subseteq\frak B$.   Then for any $\epsilon>0$ we can find a
Bernoulli partition $\sequence{i}{d_i}$ for $\pi$ such that

(i) $\bar\mu d_i=\bar\mu c_i$ for every $i\in\Bbb N$,

(ii) $\bar\mu(d_i\Bsymmdiff c_i)\le\epsilon$ for every $i\in\Bbb N$,

(iii) writing $\frak D$ for the closed subalgebra of $\frak A$ generated
by $\{\pi^jd_i:i\in\Bbb N,\,j\in\Bbb Z\}$,
$\rho(b_i,\frak D)\le\epsilon$ for every $i\in\Bbb N$.

\proof{{\bf (a)} By 387F, there is a Bernoulli partition
$\sequence{i}{b^*_i}$
for $\pi$ such that $b^*_i\in\frak C$ for every $i\in\Bbb N$,
$\bar\mu b^*_i=\bar\mu b_i$ for every $i\in\Bbb N$, and
$\bar\mu(\phi c_i\Bsymmdiff c_i)\le\bover14\epsilon$ for every
$i\in\Bbb N$, where $\phi:\frak B\to\frak C$ is
the measure-preserving Boolean homomorphism such that $\phi b_i=b^*_i$
for every $i$ and $\pi\phi=\phi\pi$.   Note that this implies that
$\pi^{-1}\phi=\phi\pi^{-1}$, and generally that $\pi^j\phi=\phi\pi^{-j}$
for every $j\in\Bbb Z$;  so
$\phi[\frak B]\subseteq\frak C$ is the closed subalgebra of $\frak A$
generated by
$\{\phi\pi^jb_i:i\in\Bbb N,\,j\in\Bbb Z\}
=\{\pi^jb_i^*:i\in\Bbb N,\,j\in\Bbb Z\}$ (324L), and is invariant under
the action of $\pi$ and $\pi^{-1}$.

Let $m\in\Bbb N$ be such that

\Centerline{$\rho(c_i,\frak B_m)\le\bover14\epsilon$ for every
$i\in\Bbb N$,}

\noindent where $\frak B_m$ is the closed subalgebra of $\frak A$
generated by $\{\pi^jb_i:i\in\Bbb N,\,|j|\le m\}$ (386K).   Let
$\eta\in\ocint{0,\epsilon}$ be such that

\Centerline{$(2m+1)\sum_{i=0}^{\infty}\min(\eta,2\bar\mu b_i)
\le\bover14\epsilon$.}

\noindent By 387F again, applied to $\pi\restrp\frak C$, there is a
Bernoulli partition $\sequence{i}{c^*_i}$ for $\pi$ such that
$c^*_i\in\phi[\frak B]$, $\bar\mu c^*_i=\bar\mu c_i$ and
$\bar\mu(\psi b^*_i\Bsymmdiff b^*_i)\le\eta$ for every $i\in\Bbb N$,
where $\psi:\frak C\to\frak C$ is
the measure-preserving Boolean homomorphism such that $\psi c_i=c^*_i$
for every $i\in\Bbb N$ and $\psi\pi=\pi\psi$.   Once again,
$\psi[\frak C]$ will be the closed subalgebra of $\frak A$ generated
by $\{\psi\pi^jc_i:i\in\Bbb N,\,j\in\Bbb Z\}
=\{\pi^jc_i^*:i\in\Bbb N,\,j\in\Bbb Z\}$;  because every $c_i^*$ belongs
to $\phi[\frak B]$, $\psi[\frak C]\subseteq\phi[\frak B]$.

\medskip

{\bf (b)} Now $\bar\mu(c^*_i\Bsymmdiff\phi c_i)\le\epsilon$ for every
$i\in\Bbb N$.   \Prf\ There is a $b\in\frak B_m$ such that
$\bar\mu(c_i\Bsymmdiff b)\le\bover14\epsilon$.   We know that
$\phi[\frak B_m]$ is
the closed subalgebra of $\frak A$ generated by
$\{\phi\pi^jb_i:i\in\Bbb N,\,|j|\le m\}
=\{\pi^j b^*_i:i\in\Bbb N,\,|j|\le m\}$, and contains $\phi b$.
Because

\Centerline{$\psi(\phi b)\Bsymmdiff\phi b
\Bsubseteq\sup_{i\in\Bbb N,|j|\le m}\psi(\pi^jb^*_i)\Bsymmdiff\pi^jb^*_i
=\sup_{|j|\le m}\pi^j(\sup_{i\in\Bbb N}\psi b^*_i\Bsymmdiff b^*_i)$,}

\noindent we have

$$\eqalign{\bar\mu(\psi\phi b\Bsymmdiff\phi b)
&\le(2m+1)\sum_{i=0}^{\infty}\bar\mu(\psi b^*_i\Bsymmdiff b^*_i)\cr
&\le(2m+1)\sum_{i=0}^{\infty}\min(\eta,2\bar\mu b_i)
\le\Bover14\epsilon.\cr}$$

\noindent But this means that

$$\eqalign{\bar\mu(c^*_i\Bsymmdiff\phi c_i)
&=\bar\mu(\psi c_i\Bsymmdiff\phi c_i)
\le\bar\mu(\psi c_i\Bsymmdiff\psi\phi b)
  +\bar\mu(\psi\phi b\Bsymmdiff\phi b)
  +\bar\mu(\phi b\Bsymmdiff\phi c_i)\cr
&\le\bar\mu(c_i\Bsymmdiff\phi b)
  +\Bover{\epsilon}4
  +\bar\mu(b\Bsymmdiff c_i)
\le\bar\mu(c_i\Bsymmdiff\phi c_i)
  +\bar\mu(\phi c_i\Bsymmdiff\phi b)
  +\Bover{\epsilon}2\cr
&\le\Bover{\epsilon}4
  +\bar\mu(c_i\Bsymmdiff b)
  +\Bover{\epsilon}2
\le\epsilon. \text{ \Qed}\cr}$$

\medskip

{\bf (c)} Set $d_i=\phi^{-1}c^*_i$ for each $i$;  this is well-defined
because $\phi$ is injective and $c^*_i\in\phi[\frak B]$.   Write
$\frak D$ for the closed subalgebra of $\frak A$
generated by $\{\pi^jd_i:i\in\Bbb N,\,j\in\Bbb Z\}
=\{\phi^{-1}\psi\pi^jc_i:i\in\Bbb N,\,j\in\Bbb Z\}$;  note that
$\frak D=\phi^{-1}[\psi[\frak C]]$, by 324L again, because
$\phi^{-1}:\phi[\frak B]\to\frak B$ is a measure-preserving
homomorphism.   Then
$\bar\mu d_i=\bar\mu c^*_i=\bar\mu c_i$ for every $i\in\Bbb N$, and
$\sequence{i}{d_i}$ is a Bernoulli partition for $\pi$.
\Prf\ If $i(0),\ldots,i(n)\in\Bbb N$, then

$$\eqalign{\bar\mu(\inf_{j\le n}\pi^jd_{i(j)})
&=\bar\mu(\inf_{j\le n}\pi^j\phi^{-1}c^*_{i(j)})
=\bar\mu(\phi(\inf_{j\le n}\pi^j\phi^{-1}c^*_{i(j)}))\cr
&=\bar\mu(\inf_{j\le n}\pi^jc^*_{i(j)}))
=\prod_{j=0}^n\bar\mu c^*_{i(j)}
=\prod_{j=0}^n\bar\mu d_{i(j)}.  \text{ \Qed}\cr}$$

Next,

\Centerline{$\bar\mu(c_i\Bsymmdiff d_i)
=\bar\mu(\phi c_i\Bsymmdiff\phi d_i)
=\bar\mu(\phi c_i\Bsymmdiff c^*_i)\le\epsilon$}

\noindent for every $i$, by (b).   Finally, if $i\in\Bbb N$, then
$\psi b^*_i$ belongs to $\psi[\frak C]$, while
$\frak D=\phi^{-1}[\psi[\frak C]]$, so

\Centerline{$\rho(b_i,\frak D)
=\rho(\phi b_i,\psi[\frak C])
\le\bar\mu(\phi b_i\Bsymmdiff\psi b^*_i)
=\bar\mu(b^*_i\Bsymmdiff\psi b^*_i)\le\eta\le\epsilon$.}

\noindent This completes the proof.
}%end of proof of 387G

\vleader{84pt}{387H}{Lemma} %3{8}6H
Let $(\frak A,\bar\mu)$ be an atomless probability
algebra and $\pi$ a measure-preserving automorphism of $\frak A$.   Let
$\sequence{i}{b_i}$, $\sequence{i}{c_i}$ be Bernoulli partitions for
$\pi$, of the same finite entropy, and write $\frak B$, $\frak C$ for
the closed subalgebras generated by
$\{\pi^jb_i:i\in\Bbb N,\,j\in\Bbb Z\}$
and $\{\pi^jc_i:i\in\Bbb N,\,j\in\Bbb Z\}$.   Suppose that
$\frak C\subseteq\frak B$.   Then for any $\epsilon>0$ we can find a
Bernoulli partition $\sequence{i}{d_i}$ for $\pi$ such that

(i) $\bar\mu d_i=\bar\mu c_i$ for every $i\in\Bbb N$,

(ii) $\bar\mu(d_i\Bsymmdiff c_i)\le\epsilon$ for every $i\in\Bbb N$,

(iii) the closed subalgebra of $\frak A$ generated
by $\{\pi^jd_i:i\in\Bbb N,\,j\in\Bbb Z\}$ is $\frak B$.

\proof{{\bf (a)} To begin with (down to the end of (c) below) suppose
that $\frak A=\frak B$.   Choose $\sequencen{\epsilon_n}$,
$\sequencen{\delta_n}$, $\sequencen{r_n}$ and
$\sequencen{\sequence{i}{d_{ni}}}$ inductively, as follows.   Start with
$r_0=0$ and $d_{0i}=c_i$ for every $i$.   Given that $\sequence{i}{d_{ni}}$
is a Bernoulli partition with $\bar\mu d_{ni}=\bar\mu c_i$ for every
$i$, take $\epsilon_n$, $\delta_n>0$ such that

\Centerline{$(2r_m+1)\epsilon_n\le 2^{-n}$ for every $m\le n$,}

\Centerline{$\delta_n\le 2^{-n-1}\epsilon$,
\quad$\sum_{i=0}^{\infty}\min(\delta_n,2\bar\mu c_i)\le\epsilon_n$,}

\noindent and use 387G to find a Bernoulli partition
$\sequence{i}{d_{n+1,i}}$ for $\pi$ such that

\Centerline{$\bar\mu d_{n+1,i}=\bar\mu c_i$,
\quad$\bar\mu(d_{n+1,i}\Bsymmdiff d_{ni})\le\delta_n$,
\quad$\rho(b_i,\frak D^{(n+1)})\le 2^{-n-1}$}

\noindent for every $i\in\Bbb N$, where $\frak D^{(n+1)}$ is the closed
subalgebra of $\frak A$ generated by
$\{\pi^jd_{n+1,i}:i\in\Bbb N,\,j\in\Bbb Z\}$.   Let $r_{n+1}$ be such that

\Centerline{$\rho(b_i,\frak D^{(n+1)}_{r_{n+1}})\le 2^{-n}$}

\noindent for every $i\in\Bbb N$, where $\frak D^{(n+1)}_{r_{n+1}}$ is
the closed subalgebra of $\frak A$ generated by
$\{\pi^jd_{n+1,i}:i\in\Bbb N,\,|j|\le r_{n+1}\}$.   Continue.

\medskip

{\bf (b)} For any $i\in\Bbb N$,

\Centerline{$\sum_{n=0}^{\infty}\bar\mu(d_{n+1,i}\Bsymmdiff d_{ni})
\le\sum_{n=0}^{\infty}\delta_n\le\epsilon$,}

\noindent so $\sequencen{d_{ni}}$ has a limit $d_i$ in $\frak A$.   Of
course

\Centerline{$\bar\mu(c_i\Bsymmdiff d_i)
\le\sum_{n=0}^{\infty}\bar\mu(d_{n+1,i}\Bsymmdiff d_{ni})\le\epsilon$}

\noindent for every $i$.   We must have

\Centerline{$\bar\mu d_i=\lim_{n\to\infty}\bar\mu d_{ni}=\bar\mu c_i$}

\noindent for each $i$, and if $i\ne j$ then

\Centerline{$d_i\Bcap d_j=\lim_{n\to\infty}d_{ni}\Bcap d_{nj}=0$;}

\noindent since $\sum_{i=0}^{\infty}\bar\mu c_i=1$,
$\sequence{i}{d_i}$ is a partition of unity in $\frak A$.   For any
$i(0),\ldots,i(k)$ in $\Bbb N$,

\Centerline{$\bar\mu(\inf_{j\le k}\pi^jd_{i(j)})
=\lim_{n\to\infty}\bar\mu(\inf_{j\le k}\pi^jd_{n,i(j)})
=\lim_{n\to\infty}\prod_{j=0}^k\bar\mu d_{n,i(j)}
=\prod_{j=0}^k\bar\mu d_{i(j)}$,}

\noindent so $\sequence{i}{d_i}$ is a Bernoulli partition.

\medskip

{\bf (c)} Let $\frak D$ be the closed subalgebra of $\frak A$ generated
by $\{\pi^jd_i:i\in\Bbb N,\,j\in\Bbb Z\}$.   Then $b_j\in\frak D$ for
every $j\in\Bbb N$.   \Prf\ Fix $m\in\Bbb N$.   Then
$\rho(b_j,\frak D^{(m+1)}_{r_{m+1}})\le 2^{-m}$, so there is a
$b\in\frak D^{(m+1)}_{r_{m+1}}$ such that
$\bar\mu(b_j\Bsymmdiff b)\le 2^{-m}$.   Now

$$\eqalign{\sum_{i=0}^{\infty}\rho(d_{m+1,i},\frak D)
&\le\sum_{i=0}^{\infty}\bar\mu(d_{m+1,i}\Bsymmdiff d_i)
\le\sum_{i=0}^{\infty}\sum_{k=m+1}^{\infty}
  \bar\mu(d_{k+1,i}\Bsymmdiff d_{ki})\cr
&\le\sum_{k=m+1}^{\infty}\sum_{i=0}^{\infty}
  \min(2\bar\mu c_i,\delta_k)
\le\sum_{k=m+1}^{\infty}\epsilon_k.\cr}$$

\noindent So

$$\eqalignno{\rho(b,\frak D)
&\le(2r_{m+1}+1)\sum_{i=0}^{\infty}\rho(d_{m+1,i},\frak D)\cr
\displaycause{386Nc}
&\le\sum_{k=m+1}^{\infty}(2r_{m+1}+1)\epsilon_k
\le\sum_{k=m+1}^{\infty}2^{-k}
=2^{-m},\cr}$$

\noindent and

\Centerline{$\rho(b_j,\frak D)
\le\bar\mu(b_j\Bsymmdiff b)+\rho(b,\frak D)
\le 2^{-m}+2^{-m}=2^{-m+1}$.}

\noindent As $m$ is arbitrary, $\rho(b_j,\frak D)=0$ and
$b_j\in\frak D$.\ \Qed

\medskip

{\bf (d)} This completes the proof if $\frak A=\frak B$.   For the
general case, apply the arguments above to
$(\frak B,\bar\mu\restrp\frak B,\pi\restrp\frak B)$.
}%end of proof of 387H

\leader{387I}{Ornstein's theorem (finite entropy case)} %3{8}6I
Let $(\frak A,\bar\mu)$ and $(\frak B,\bar\nu)$ be probability algebras,
and $\pi:\frak A\to\frak A$,
$\phi:\frak B\to\frak B$ two-sided Bernoulli shifts of the same finite
entropy.   Then $(\frak A,\bar\mu,\pi)$ and $(\frak B,\bar\nu,\phi)$ are
isomorphic.

\proof{{\bf (a)} Let $\sequence{i}{a_i}$, $\sequence{i}{b_i}$ be
(two-sided) generating Bernoulli partitions in $\frak A$, $\frak B$
respectively.   By the Kolmogorov-Sina\v\i\ theorem,
$\sequence{i}{a_i}$ and $\sequence{i}{b_i}$ both have entropy equal to
$h(\pi)=h(\phi)$.   If this entropy is zero, then $\frak A$ and
$\frak B$ are both $\{0,1\}$, and the result is trivial;
so let us assume that
$h(\pi)>0$, so that $\frak A$ is atomless (387Bd).

\medskip

{\bf (b)} By Sina\v\i's theorem (387E), there is a Bernoulli partition
$\sequence{i}{c_i}$ for $\pi$ such that $\bar\mu c_i=\bar\nu b_i$ for
every $i\in\Bbb N$.   By 387H, there is a Bernoulli partition
$\sequence{i}{d_i}$ for $\pi$ such that $\bar\mu d_i=\bar\mu c_i$ for
every $i$ and the closed subalgebra of $\frak A$ generated
by $\{\pi^jd_i:i\in\Bbb N,\,j\in\Bbb Z\}$ is $\frak A$.   But now
$(\frak
A,\bar\mu,\pi,\sequence{i}{d_i})$ is isomorphic to $(\frak
B,\bar\nu,\phi,\sequence{i}{b_i})$,
so $(\frak A,\bar\mu,\pi)$ and $(\frak B,\bar\nu,\phi)$ are
isomorphic.
}%end of proof of 387I

\leader{387J}{}\cmmnt{ Using %3{8}6J
the same methods, we can extend the last
result to the case of Bernoulli shifts of infinite entropy.   The first
step uses the ideas of 387C, as follows.

\medskip

\noindent}{\bf Lemma} Let $(\frak A,\bar\mu)$ be a probability algebra and
$\pi:\frak A\to\frak A$ an ergodic measure-preserving automorphism.
Suppose that $\familyiI{a_i}$ is a finite Bernoulli partition for $\pi$,
with $\#(I)=r\ge 1$ and $\bar\mu a_i=1/r$ for every $i\in I$, and that
$h(\pi)\ge\ln 2r$.   Then for any $\epsilon>0$ there is a Bernoulli
partition $\langle b_{ij}\rangle_{i\in I,j\in\{0,1\}}$ for $\pi$ such
that

\Centerline{$\bar\mu(a_i\Bsymmdiff(b_{i0}\Bcup b_{i1}))\le\epsilon$,
\quad$\bar\mu b_{i0}=\bar\mu b_{i1}=\Bover1{2r}$}
\noindent for every $i\in I$.

\proof{{\bf (a)} Let $\delta>0$ be such that

\Centerline{$\delta+6\sqrt{4\delta}\le\epsilon$.}

\noindent Let $\eta>0$ be such that

\Centerline{$\eta<\ln 2$,
\quad$\sqrt{8\eta}\le\delta$}

\noindent and

\Centerline{$|t-\bover12|\le\delta$ whenever $t\in[0,1]$ and
$q(t)+q(1-t)\ge\ln 2-4\eta$}

\noindent (385Ad).   We have

\Centerline{$H(A)=rq(\Bover1r)=\ln r$,}

\noindent and $\bar\mu d=r^{-n}$ whenever $n\in\Bbb N$,
$d\in D_n(A,\pi)$.

Note that $\frak A$ is atomless.
\Prf\Quer\ If $a\in\frak A$ is an atom, then
$\sup_{j\in\Bbb Z}\pi^ja=1$ (because $\pi$ is ergodic, 372Pb),
and $\frak A$ is purely atomic, with atoms all of
the same size as $a$;  but this means that $H(C)\le\ln(\bover1{\bar\mu a})$ for every partition of unity $C\subseteq\frak A$, so that

\Centerline{$h(\pi,C)=\lim_{n\to\infty}\Bover1n H(D_n(C,\pi))
\le\lim_{n\to\infty}\Bover1n\ln(\Bover1{\bar\mu a})=0$}

\noindent for every partition of unity $C$, and

\Centerline{$0=h(\pi)\ge\ln 2r\ge\ln 2$.  \Bang\Qed}

\medskip

{\bf (b)} There is a finite partition of unity $C\subseteq\frak A$ such
that

\Centerline{$h(\pi,C)=\ln 2r-\eta$,}

\noindent and $C$ refines $A=\{a_i:i\in I\}\setminus\{0\}$.   \Prf\
Because $h(\pi)\ge\ln 2r$, there is a finite partition of unity $C'$ such that
$h(\pi,C')\ge\ln 2r-\eta$;  replacing $C'$ by $C'\vee A$ if need be, we
may suppose that $C'$ refines $A$;  take such a $C'$ of minimal size.
Because $H(C')\ge h(\pi,C')>H(A)$, there must be distinct $c_0$, $c_1\in
C'$ included in the same member of $A$.   Because $\frak A$ is atomless,
the principal ideal generated by $c_1$ has a closed subalgebra
isomorphic, as measure algebra, to the measure algebra of Lebesgue
measure on $[0,1]$, up to a scalar multiple of the measure;  and in
particular there is a family $\family{t}{[0,1]}{d_t}$ such that
$d_s\Bsubseteq d_t$ whenever $s\le t$, $d_1=c_1$ and $\bar\mu
d_t=t\bar\mu c_1$ for every $t\in[0,1]$.   Let $D_t$ be the partition of
unity

\Centerline{$(C'\setminus\{c_0,c_1\})
  \cup\{c_0\Bcup d_t,c_1\Bsetminus d_t\}$}

\noindent for each $t\in[0,1]$.   Then

\Centerline{$h(\pi,D_1)
=h(\pi,(C'\setminus\{c_0,c_1\})\cup\{c_0\Bcup c_1\})<\ln 2r-\eta$,}

\noindent by the minimality of $\#(C')$, while

\Centerline{$h(\pi,D_0)=h(\pi,C')
\ge \ln 2r-\eta$.}

\noindent Using 385N, we also have, for any $s$, $t\in[0,1]$ such that
$|s-t|\le\bover1e$,

$$\eqalignno{h(\pi,D_s)-h(\pi,D_t)
&\le H(D_s|\frak D_t)\cr
\noalign{\noindent (where $\frak D_t$ is the closed subalgebra generated
by $D_t$)}
&\le q(\rho(c_0\Bcup d_s,\frak D_t))
  +q(\rho(c_1\Bsetminus d_s,\frak D_t))\cr
\displaycause{by 386Mb, because
$D_s\setminus\frak D_t\subseteq\{c_0\Bcup d_s,c_1\Bsetminus d_s\}$}
&\le q(\bar\mu((c_0\Bcup d_s)\Bsymmdiff(c_0\Bcup d_t)))
  +q(\bar\mu((c_1\Bsetminus d_s)\Bsymmdiff(c_1\Bsetminus d_t)))\cr
&=2q(\bar\mu(d_s\Bsymmdiff d_t))
=2q(|s-t|\bar\mu c_1)\cr}$$

\noindent because $q$ is monotonic on $[0,|s-t|\bar\mu c_1]$.
But this means that $t\mapsto h(\pi,D_t)$
is continuous and there must be some $t$ such that
$h(\pi,D_t)=\ln 2r-\eta$;  take $C=D_t$.\ \Qed

\medskip

{\bf (c)} Let $\xi>0$ be such that

\Centerline{$\xi\le\eta$,
\quad$\xi\le\Bover16$,
\quad$q(2\xi)+q(1-2\xi)\le\eta$,
\quad$\sum_{c\in C}q(\min(2\xi,\bar\mu c))\le\eta$.}

\noindent Let $n\in\Bbb N$ be such that

\Centerline{$\Bover1{n+1}\le\xi$,
\quad$q(\Bover1{n+1})+q(\Bover{n}{n+1})\le\eta$,
\quad$\bar\mu\Bvalue{w_n-h(\pi,C)\chi 1\ge\eta}\le\xi$,}

\noindent where

\Centerline{$w_n=\Bover1n\sum_{d\in D_n(C,\pi)}\ln(\Bover1{\bar\mu
d})\chi d$.}

\noindent (The
Shannon-McMillan-Breiman theorem, 386E, assures us that any sufficiently
large $n$ has these properties.)

\medskip

{\bf (d)} Let $D$ be the set of those $d\in D_n(C,\pi)$ such that

\Centerline{$\bar\mu d\ge (2r)^{-n}$,
\quad i.e.,\quad $\Bover1n\ln(\Bover1{\bar\mu d})\le\ln 2r$.}
\noindent Then $\bar\mu(\sup D)\ge 1-\xi$, by the choice of $n$, because
$h(\pi,C)=\ln 2r-\eta$.    Note that every member of $D$ is included in
some member of $D_n(A,\pi)$, because $C$ refines $A$.   If
$b\in D_n(A,\pi)$,
then $\bar\mu b=r^{-n}$, so $\#(\{d:d\in D,\,d\Bsubseteq b\})\le 2^n$;
we can therefore find a function $f:D\to\{0,1\}^n$ such that $f$ is
injective on $\{d:d\in D,\,d\Bsubseteq b\}$ for every $b\in D_n(A,\pi)$.

\medskip

{\bf (e)} By 386C(iv), as usual, there is an $a\in\frak A$ such that
$a,\pi^{-1}a,\ldots,\pi^{-n+1}a$ are disjoint and
$\bar\mu(a\Bcap d)=\bover1{n+1}\bar\mu d$ for every $d\in D_n(C,\pi)$.
Set

\Centerline{$e=\sup_{d\in D,j<n}\pi^{-j}(a\Bcap d)$;}

\noindent then

\Centerline{$\bar\mu e
=\sum_{j=0}^{n-1}\sum_{d\in D}\bar\mu(a\Bcap d)
=\Bover{n}{n+1}\bar\mu(\sup D)\ge(1-\xi)^2\ge 1-2\xi$.}

\medskip

{\bf (f)} Set

\Centerline{$c^*=\sup\{\pi^{-j}(a\Bcap d):j<n,\,d\in D,\,f(d)(j)=1\}$.}

\noindent (I am identifying members of $\{0,1\}^n$ with functions from
$\{0,\ldots,n-1\}$ to $\{0,1\}$.)   Set

\Centerline{$A^*=A\vee\{c^*,1\Bsetminus c^*\}$,
\quad$A'=A^*\vee\{a,1\Bsetminus a\}\vee\{e,1\Bsetminus e\}$,}

\noindent and let $\frak A'$ be the closed subalgebra of $\frak A$
generated by $\{\pi^ja:a\in A',\,j\in\Bbb Z\}$.   Then
$a\Bcap d\in\frak A'$ for every $d\in D$.   \Prf\ Set

\Centerline{$\tilde d=\upr(a\Bcap d,\frak A')
=\inf\{c:c\in\frak A',\,c\Bsupseteq a\Bcap d\}\in\frak A'$.}

\noindent Let $b$ be the element of $D_n(A,\pi)$ including $d$.  Because
$a$, $b$, $e\in\frak A'$,

\Centerline{$\tilde d\Bsubseteq a\Bcap b\Bcap e
=\sup_{d'\in D}a\Bcap b\Bcap d'
=\sup\{a\Bcap d':d'\in D,\,d'\Bsubseteq b\}$.}

\noindent Now if $d'\in D$, $d'\Bsubseteq b$ and $d'\ne d$, then
$f(d')\ne f(d)$.   Let $j$ be such that $f(d')(j)\ne f(d)(j)$;  then
$\pi^{-j}(a\Bcap d)$ is included in one of $c^*$, $1\Bsetminus c^*$ and
$\pi^{-j}(a\Bcap d')$ in the other.   This means that one of $\pi^jc^*$,
$1\Bsetminus\pi^jc^*$ is a member of $\frak A'$ including $a\Bcap d$ and
disjoint from $a\Bcap d'$, so that $\tilde d\Bcap d'=0$.   Thus
$\tilde d$ must be actually equal to $a\Bcap d$, and $a\Bcap d\in\frak
A'$.\ \Qed

Next, $c\Bcap e\in\frak A'$ for every $c\in C$.   \Prf\
$\langle \pi^{-j}(a\Bcap d)\rangle_{j<n,\,d\in D}$ is a disjoint family
in $\frak A'$ with supremum $e$.   But whenever $d\in D$ and $j<n$ we must have $d\Bsubseteq\pi^jc'$ for some $c'\in C$, so either
$d\Bsubseteq\pi^jc$ or $d\Bcap\pi^jc=0$;  thus $\pi^{-j}(a\Bcap d)$ must
be either included in $c$ or disjoint from it.   Accordingly

\Centerline{$c\Bcap e
=\sup\{\pi^{-j}(a\Bcap d):j<n,\,d\in D,\,d\Bsubseteq\pi^jc\}
\in\frak A'$.   \Qed}

Consequently $h(\pi,A')\ge\ln 2r-2\eta$.   \Prf\ For any $c\in C$,

\Centerline{$\rho(c,\frak A')\le\bar\mu(c\Bsymmdiff(c\Bcap e))
=\bar\mu(c\Bsetminus e)\le\min(\bar\mu c,2\xi)\le\Bover13$,}

\noindent so

$$\eqalignno{\ln 2r-\eta
&=h(\pi,C)
\le h(\pi\restrp\frak A')+H(C|\frak A')\cr
\noalign{\noindent (386Ld)}
&\le h(\pi,A')+\sum_{c\in C}q(\rho(c,\frak A'))\cr
\noalign{\noindent (by the Kolmogorov-Sina\v\i\ theorem and
386Mb)}
&\le h(\pi,A')+\sum_{c\in C}q(\min(\bar\mu c,2\xi))
\le h(\pi,A')+\eta\cr}$$

\noindent by the choice of $\xi$.\ \Qed

Finally, $h(\pi,A^*)\ge\ln 2r-4\eta$.
\Prf\

$$\eqalignno{\ln 2r-2\eta
&\le h(\pi,A')
\le h(\pi,A^*)+H(\{a,1\Bsetminus a\})+H(\{e,1\Bsetminus e\})\cr
\noalign{\noindent (applying 386Lb twice)}
&=h(\pi,A^*)+q(\bar\mu a)+q(1-\bar\mu a)+q(\bar\mu e)+q(1-\bar\mu
e)\cr
&\le h(\pi,A^*)+q(\Bover1n)+q(\Bover{n}{n+1})+q(2\xi)+q(1-2\xi)\cr
&\le h(\pi,A^*)+\eta+\eta
=h(\pi,A^*)+2\eta. \text{ \Qed}\cr}$$

\medskip

{\bf (g)} We have

$$\eqalign{\ln 2r-4\eta
&\le h(\pi,A^*)
\le H(A^*)\cr
&\le H(A)+H(\{c^*,1\Bsetminus c^*\})
=\ln r+H(\{c^*,1\Bsetminus c^*\})
\le\ln 2r,\cr}$$

\noindent so

\Centerline{$q(\bar\mu c^*)+q(1-\bar\mu c^*)
=H(\{c^*,1\Bsetminus c^*\})\ge\ln 2-4\eta$.}

\noindent By the choice of $\eta$, $|\bar\mu c^*-\bover12|\le\delta$.

Next,

\Centerline{$\sum_{i\in I}|\bar\mu(a_i\Bcap c^*)-\Bover1{2r}|
+|\bar\mu(a_i\Bsetminus c^*)-\Bover1{2r}|\le 3\delta$.}

\noindent\Prf\ By 386I,

$$\eqalign{\sum_{i\in I}|\bar\mu(a_i\Bcap c^*)-\Bover1{r}\bar\mu c^*|
&+|\bar\mu(a_i\Bsetminus c^*)-\Bover1{r}\bar\mu(1\Bsetminus c^*)|\cr
&\le\sqrt{2(H(A)+H(\{c^*,1\Bsetminus c^*\})-H(A^*))}\cr
&\le\sqrt{2(\ln r+\ln 2-\ln 2r+4\eta)}
=\sqrt{8\eta}\le\delta.\cr}$$

\noindent So

$$\eqalign{\sum_{i\in I}|\bar\mu(&a_i\Bcap c^*)-\Bover1{2r}|
  +|\bar\mu(a_i\Bsetminus c^*)-\Bover1{2r}|\cr
&\le\sum_{i\in I}\bigl(|\bar\mu(a_i\Bcap c^*)-\Bover1{r}\bar\mu c^*|
  +\Bover1r|\bar\mu c^*-\Bover12|\cr
&\qquad\qquad\qquad
  +|\bar\mu(a_i\Bsetminus c^*)-\Bover1{r}\bar\mu(1\Bsetminus c^*)|
  +\Bover1r|\bar\mu(1\Bsetminus c^*)-\Bover12|\bigr)\cr
&\le\delta+|\bar\mu c^*-\Bover12|+|\bar\mu(1\Bsetminus c^*)-\Bover12|
\le 3\delta.   \text{ \Qed}\cr}$$


\medskip

{\bf (h)} Now apply 387D to the partition of unity $A^*$, indexed
as $\langle a^*_{ij}\rangle_{i\in I,j\in\{0,1\}}$, where
$a^*_{i1}=a_i\Bcap c^*$ and $a^*_{i0}=a_i\Bsetminus c^*$, and
$\langle\gamma_{ij}\rangle_{i\in I,j\in\{0,1\}}$, where
$\gamma_{ij}=\bover1{2r}$ for all $i$, $j$.   We have

\Centerline{$\sum_{i\in I,j\in\{0,1\}}|\bar\mu a^*_{ij}-\gamma_{ij}|\le
3\delta$}

\noindent by (g), while

\Centerline{$H(A^*)-h(\pi,A^*)\le\ln 2r-\ln 2r+4\eta=4\eta$,}

\noindent so

\Centerline{$\sum_{i\in I,j\in\{0,1\}}|\bar\mu a^*_{ij}-\gamma_{ij}|
+\sqrt{2(H(A^*)-h(\pi,A^*))}\le 3\delta+\sqrt{8\eta}\le 4\delta$.}

\noindent Also

\Centerline{$\sum_{i\in I,j\in\{0,1\}}q(\gamma_{ij})=\ln 2r\le h(\pi)$.
}

\noindent So 387D tells us that there is a Bernoulli partition $\langle
b_{ij}\rangle_{i\in I,j\in\{0,1\}}$ for $\pi$ such that $\bar\mu
b^*_{ij}=\bover1{2r}$ for all $i$, $j$ and

\Centerline{$\sum_{i\in I,j\in\{0,1\}}\bar\mu(b_{ij}\Bsymmdiff a^*_{ij})
\le\delta+6\sqrt{4\delta}\le\epsilon$.}

\noindent Now of course

$$\eqalign{\sum_{i\in I}\bar\mu(a_i\Bsymmdiff(b_{i0}\Bcup b_{i1}))
&\le\sum_{i\in I}\bar\mu((a_i\Bcap c^*)\Bsymmdiff b_{i1})
  +\bar\mu((a_i\Bsetminus c^*)\Bsymmdiff b_{i0})\cr
&=\sum_{i\in I,j\in\{0,1\}}\bar\mu(a^*_{ij}\Bsymmdiff b_{ij})
\le\epsilon,\cr}$$

\noindent as required.
}%end of proof of 387J

\leader{387K}{Ornstein's theorem (infinite entropy case)} %3{8}6K
Let $(\frak A,\bar\mu)$ be a probability algebra of countable Maharam
type, and $\pi:\frak A\to\frak A$ a two-sided Bernoulli shift of infinite
entropy.   Then $(\frak A,\bar\mu,\pi)$ is isomorphic to
$(\frak B_{\Bbb Z},\bar\nu_{\Bbb Z},\phi)$, where
$(\frak B_{\Bbb Z},\bar\nu_{\Bbb Z})$ is the measure
algebra of the usual measure on
$[0,1]^{\Bbb Z}$, and $\phi$ is the standard two-sided Bernoulli shift on
$\frak B_{\Bbb Z}$\cmmnt{ (385Sb)}.

\proof{{\bf (a)} We have to find a root algebra $\frak E$ for $\pi$
which is
isomorphic to the measure algebra of Lebesgue measure on $[0,1]$.
The materials we have to start with are a root algebra
$\frak A_0\subseteq\frak A$ such that {\it either} $\frak A_0$ is not purely atomic {\it or} $H(A_0)=\infty$, where $A_0$ is the set of atoms of $\frak A_0$.

Because $\frak A$ has countable Maharam type, there is a sequence
$\sequencen{d_n}$ in $\frak A_0$ such that $\{d_n:n\in\Bbb N\}$ is dense
in the metric of $\frak A_0$.

\medskip

{\bf (b)} There is a sequence $\sequencen{C_n}$ of partitions of unity
in $\frak A_0$ such that $C_{n+1}$ refines $C_n$, $H(C_n)=n\ln 2$ and
$d_n$ is a union of members of $C_{n+1}$ for every $n$.
\Prf\ We have

\Centerline{$\sup\{H(C):
  C\subseteq\frak A_0$ is a partition of unity$\}=\infty$}

\noindent (385J).   Choose the $C_n$ inductively, as follows.
Start with $C_0=\{0,1\}$.   Given $C_n$ with $H(C_n)=n\ln 2$, set
$C'_n=C_n\vee\{d_n,1\Bsetminus d_n\}$;  then

\Centerline{$H(C'_n)\le H(C_n)+H(\{d_n,1\Bsetminus d_n\})
\le(n+1)\ln 2$.}

\noindent By 386O, there is a partition of unity $C_{n+1}$, refining
$C'_n$, such that $H(C_{n+1})=(n+1)\ln 2$.   Continue.\ \Qed

\medskip

{\bf (c)} For each $n\in\Bbb N$, let $\frak C_n$ be the closed
subalgebra
of $\frak A$ generated by $\{\pi^ja:a\in C_n,\,j\in\Bbb Z\}$.   Then
$\sequencen{\frak C_n}$ is increasing.   For each $n$,
$\pi[\frak C_n]=\frak C_n$;  because $C_n\subseteq\frak A_0$, $\pi\restrp\frak C_n$ is a Bernoulli shift with generating partition $C_n$.   Accordingly

\Centerline{$h(\pi\restrp\frak C_n)=h(\pi,C_n)=H(C_n)=n\ln 2$.}

\noindent Of course $d_n\in\frak C_{n+1}$ for every $n$.

Choose inductively, for each $n\in\Bbb N$, $\epsilon_n>0$,
$r_n\in\Bbb N$
and a Bernoulli partition $\family{\sigma}{\{0,1\}^n}{b_{n\sigma}}$ in
$\frak C_n$, as follows.   Start with $b_{0\emptyset}=1$.   (See 3A1H
for the notation I am using here.)   Given that
$\family{\sigma}{\{0,1\}^n}{b_{n\sigma}}$ is a Bernoulli partition for
$\pi$ which generates $\frak C_n$, in the sense that $\frak C_n$ is the
closed subalgebra of $\frak A$ generated by
$\{\pi^jb_{n\sigma}:\sigma\in\{0,1\}^n,\,j\in\Bbb Z\}$, and
$\bar\mu b_{n\sigma}=2^{-n}$ for every $\sigma$, take $\epsilon_n>0$
such that

\Centerline{$(2r_m+1)\epsilon_n\le 2^{-n}$ for every $m<n$.}

\noindent We know that

\Centerline{$h(\pi\restrp\frak C_{n+1})=(n+1)\ln 2
=\ln(2\cdot 2^n)$.}

\noindent So we can apply 387J to
$(\frak C_{n+1},\pi\restrp\frak C_{n+1})$
to see that there is a Bernoulli partition
$\family{\tau}{\{0,1\}^{n+1}}{b'_{n\tau}}$ for $\pi$ such that

\Centerline{$b'_{n\tau}\in\frak C_{n+1}$,
\quad$\bar\mu b'_{n\tau}=2^{-n-1}$}

\noindent for every $\tau\in\{0,1\}^{n+1}$,

\Centerline{$\bar\mu(b_{n\sigma}\Bsymmdiff
  (b'_{n,\sigma^{\smallfrown}\fraction{0}}
    \Bcup b'_{n,\sigma^{\smallfrown}\fraction{1}}))
\le 2^{-n}\epsilon_n$}

\noindent for every $\sigma\in\{0,1\}^n$.   By 387H (with
$\frak B=\frak C=\frak C_{n+1}$), there is a Bernoulli partition
$\family{\tau}{\{0,1\}^{n+1}}{b_{n+1,\tau}}$ for
$\pi\restrp\frak C_{n+1}$ such that
the closed subalgebra generated by
$\{\pi^jb_{n+1,\tau}:\tau\in\{0,1\}^{n+1},\,j\in\Bbb Z\}$ is
$\frak C_{n+1}$, $\bar\mu b_{n+1,\tau}=2^{-n-1}$ for every
$\tau\in\{0,1\}^{n+1}$, and

\Centerline{$\sum_{\tau\in\{0,1\}^{n+1}}
  \bar\mu(b_{n+1,\tau}\Bsymmdiff b'_{n\tau})
\le\epsilon_n$.}

\noindent For each $k\in\Bbb N$, let $\frak B^{(n+1)}_k$ be the closed
subalgebra of $\frak C_{n+1}$ generated by
$\{\pi^jb_{n+1,\tau}:\tau\in\{0,1\}^{n+1},\,|j|\le k\}$.   Since
$d_m\in\frak C_{m+1}\subseteq\frak C_{n+1}$ for every $m\le n$, there is
an $r_n\in\Bbb N$ such that

\Centerline{$\rho(d_m,\frak B^{(n+1)}_{r_n})\le 2^{-n}$ for every
$m\le n$.}

\noindent Continue.

\medskip

{\bf (d)} Fix $m\le n\in\Bbb N$ for the moment.   For
$\sigma\in\{0,1\}^m$, set

\Centerline{$b_{n\sigma}=\sup\{b_{n\tau}:\tau\in\{0,1\}^n,\,\tau$
extends $\sigma\}$.}

\noindent (If $n=m$, then of course $\sigma$ is the unique member of
$\{0,1\}^m$ extending itself, so this formula is safe.)   Then

\Centerline{$\bar\mu b_{n\sigma}
=2^{-n}\#(\{\tau:\tau\in\{0,1\}^n,\,\tau$ extends $\sigma\})
=2^{-n}2^{n-m}=2^{-m}$.}

\noindent Next, if $\sigma$, $\sigma'\in\{0,1\}^m$ are distinct, there
is no member of $\{0,1\}^n$ extending both, so
$b_{n\sigma}\Bcap b_{n\sigma'}=0$;  thus
$\family{\sigma}{\{0,1\}^m}{b_{n\sigma}}$ is a
partition of unity.   If $\sigma(0),\ldots,\sigma(k)\in\{0,1\}^m$, then

$$\eqalignno{\bar\mu(\inf_{j\le k}\pi^jb_{n,\sigma(j)})
&=\bar\mu(\sup_{\Atop{\tau(0),\ldots,\tau(k)\in\{0,1\}^n}
    {\tau(j)\supseteq\sigma(j)\forall j\le k}}
  \inf_{j\le k}\pi^jb_{n,\tau(j)})\cr
&=\sum_{\Atop{\tau(0),\ldots,\tau(k)\in\{0,1\}^n}
    {\tau(j)\supseteq\sigma(j)\forall j\le k}}
  \bar\mu(\inf_{j\le k}\pi^jb_{n,\tau(j)})\cr
&=\sum_{\Atop{\tau(0),\ldots,\tau(k)\in\{0,1\}^n}
    {\tau(j)\supseteq\sigma(j)\forall j\le k}}
  (2^{-n})^{k+1}\cr
&=(2^{n-m})^{k+1}(2^{-n})^{k+1}
=(2^{-m})^{k+1}
=\prod_{j=0}^k\bar\mu b_{n,\sigma(j)},\cr}$$

\noindent so $\family{\sigma}{\{0,1\}^m}{b_{n\sigma}}$ is a Bernoulli
partition.

\medskip

{\bf (e)}  If $m\le n\in\Bbb N$, then

\Centerline{$\sum_{\sigma\in\{0,1\}^m}
  \bar\mu(b_{n\sigma}\Bsymmdiff b_{n+1,\sigma})
\le 2\epsilon_n$.}

\noindent \Prf\ We have

$$\eqalignno{\sum_{\sigma\in\{0,1\}^m}
  \bar\mu(b_{n\sigma}\Bsymmdiff b_{n+1,\sigma})
&\le\sum_{\tau\in\{0,1\}^n}\bar\mu(b_{n\tau}\Bsymmdiff b_{n+1,\tau})\cr
&=\sum_{\tau\in\{0,1\}^n}
\bar\mu(b_{n\tau}\Bsymmdiff
  (b_{n+1,\tau^{\smallfrown}\fraction{0}}
    \Bcup b_{n+1,\tau^{\smallfrown}\fraction{1}}))\cr
&\le\sum_{\tau\in\{0,1\}^n}
\bar\mu(b_{n\tau}\Bsymmdiff
  (b'_{n,\tau^{\smallfrown}\fraction{0}}
    \Bcup b'_{n,\tau^{\smallfrown}\fraction{1}}))
+\sum_{\upsilon\in\{0,1\}^{n+1}}
  \bar\mu(b'_{n\upsilon}\Bsymmdiff b_{n+1,\upsilon})\cr
&\le\sum_{\tau\in\{0,1\}^n}2^{-n}\epsilon_n+\epsilon_n
=2\epsilon_n. \text{ \Qed}\cr}$$

\medskip

{\bf (f)} In particular, for any $m\in\Bbb N$ and $\sigma\in\{0,1\}^m$,

\Centerline{$\sum_{n=m}^{\infty}
  \bar\mu(b_{n\sigma}\Bsymmdiff b_{n+1,\sigma})
\le\sum_{n=m}^{\infty}2\epsilon_n<\infty$.}

\noindent So we can define $b_{\sigma}=\lim_{n\to\infty}b_{n\sigma}$ in
$\frak A$.   We have

\Centerline{$\bar\mu b_{\sigma}=\lim_{n\to\infty}\bar\mu b_{n\sigma}
=2^{-m}$;}

\noindent and if $\sigma$, $\sigma'\in\{0,1\}^m$ are distinct, then

\Centerline{$b_{\sigma}\Bcap b_{\sigma'}
=\lim_{n\to\infty}b_{n\sigma}\Bcap b_{n\sigma'}=0$,}

\noindent so $\family{\sigma}{\{0,1\}^m}{b_{\sigma}}$ is a partition of
unity in $\frak A$.   If $\sigma(0),\ldots,\sigma(k)\in\{0,1\}^m$, then

$$\eqalign{\bar\mu(\inf_{j\le k}\pi^jb_{\sigma(j)})
&=\lim_{n\to\infty}\bar\mu(\inf_{j\le k}\pi^jb_{n,\sigma(j)})\cr
&=\lim_{n\to\infty}\prod_{j=0}^k\bar\mu b_{n,\sigma(j)}
=\prod_{j=0}^k\bar\mu b_{\sigma(j)},\cr}$$

\noindent so $\family{\sigma}{\{0,1\}^m}{b_{\sigma}}$ is a Bernoulli
partition for $\pi$.   If $\sigma\in\{0,1\}^m$, then

\Centerline{$b_{\sigma^{\smallfrown}\fraction{0}}
  \Bcup b_{\sigma^{\smallfrown}\fraction{1}}
=\lim_{n\to\infty}b_{n,\sigma^{\smallfrown}\fraction{0}}
  \Bcup b_{n,\sigma^{\smallfrown}\fraction{1}}
=\lim_{n\to\infty}b_{n,\sigma}
=b_{\sigma}$.}

\medskip

{\bf (g)} Let $\frak E$ be the closed subalgebra of $\frak A$ generated
by $\bigcup_{m\in\Bbb N}\{b_{\sigma}:\sigma\in\{0,1\}^m\}$.   Then
$\frak E$ is atomless and countably generated, so
$(\frak E,\bar\mu\restrp\frak E)$ is isomorphic to the measure algebra of
Lebesgue measure on $[0,1]$.
Now $\bar\mu(\inf_{j\le k}\pi^je_j)=\prod_{j=0}^k\bar\mu e_j$ for all
$e_0,\ldots,e_k\in\frak E$.   \Prf\ Let $\epsilon>0$.   For
$m\in\Bbb N$,
let $\frak E_m$ be the subalgebra of $\frak E$ generated by
$\{b_{\sigma}:\sigma\in\{0,1\}^m\}$.   $\sequence{m}{\frak E_m}$ is
non-decreasing, so $\overline{\bigcup_{m\in\Bbb N}\frak E_m}$ is a
closed subalgebra of $\frak A$, and must be $\frak E$.   Now the
function

\Centerline{$(a_0,\ldots,a_k)
\to\bar\mu(\inf_{j\le k}\pi^ja_j)-\prod_{j=0}^k\bar\mu a_j
:\frak A^{k+1}\to\Bbb R$}

\noindent is continuous and zero on $\frak E_m^{k+1}$ for every $m$, by
387Bb, so is zero on $\frak E^{k+1}$, and in particular is zero at
$(e_0,\ldots,e_k)$, as required.\ \Qed

By 385Sf, $\family{j}{\Bbb Z}{\pi^j[\frak E]}$ is independent.

\medskip

{\bf (h)} Let $\frak B^*$ be the closed subalgebra of $\frak A$
generated by $\{\pi^jb_{\sigma}:
  \sigma\in\bigcup_{m\in\Bbb N}\{0,1\}^m,\,j\in\Bbb Z\}$;  then
$\frak B^*$ is the closed
subalgebra of $\frak A$ generated by
$\bigcup_{j\in\Bbb Z}\pi^j[\frak E]$.    It follows from (e) that, for
any $m\in\Bbb N$,

$$\eqalign{\sum_{\sigma\in\{0,1\}^m}\rho(b_{m\sigma},\frak B^*)
&\le\sum_{\sigma\in\{0,1\}^m}
  \bar\mu(b_{m\sigma}\Bsymmdiff b_{\sigma})\cr
&\le\sum_{\sigma\in\{0,1\}^m}\sum_{n=m}^{\infty}
  \bar\mu(b_{n\sigma}\Bsymmdiff b_{n+1,\sigma})
\le 2\sum_{n=m}^{\infty}\epsilon_n.\cr}$$

\noindent So if $b\in\frak B^{(m+1)}_{r_m}$,

$$\eqalignno{\rho(b,\frak B^*)
&\le(2r_m+1)\sum_{\sigma\in\{0,1\}^{m+1}}
  \rho(b_{m+1,\sigma},\frak B^*)\cr
\displaycause{386Nc}
&\le 2(2r_m+1)\sum_{n=m+1}^{\infty}\epsilon_n
\le 2\sum_{n=m+1}^{\infty}2^{-n}
=2^{-m+1}.\cr}$$

\noindent It follows that, whenever $m\le n$ in $\Bbb N$,

\Centerline{$\rho(d_m,\frak B^*)
\le\rho(d_m,\frak B^{(n+1)}_{r_n})+2^{-n+1}
\le 2^{-n}+2^{-n+1}$}

\noindent by the choice of $r_n$.   Letting $n\to\infty$, we see that
$\rho(d_m,\frak B^*)=0$, that is, $d_m\in\frak B^*$, for every
$m\in\Bbb N$.   But this means that $\frak A_0\subseteq\frak B^*$, by
the choice of
$\sequence{m}{d_m}$.   Accordingly $\pi^j[\frak A_0]\subseteq\frak B^*$
for every $j$ and $\frak B^*$ must be the whole of $\frak A$.

\medskip

{\bf (i)} Thus $\pi$ is a two-sided Bernoulli shift with root algebra
$\frak E$;  by 385Sc, $(\frak A,\bar\mu,\pi)$ is isomorphic to
$(\frak B_{\Bbb Z},\bar\nu_{\Bbb Z},\phi)$.
}%end of proof of 387K

\leader{387L}{Corollary:  Sina\v\i's theorem (general case)} %3{8}6L
Let $(\frak A,\bar\mu)$ be an atomless probability algebra, and
$\pi:\frak A\to\frak A$ a measure-preserving automorphism.   Let
$(\frak B,\bar\nu)$ be a probability algebra of countable Maharam type,
and $\phi:\frak B\to\frak B$ a one- or two-sided Bernoulli shift with
$h(\phi)\le h(\pi)$.   Then $(\frak B,\bar\nu,\phi)$ is isomorphic to a
factor of $(\frak A,\bar\mu,\pi)$.

\proof{{\bf (a)} To begin with (down to the end of (b)) suppose that
$\phi$ is two-sided.   Let $\frak B_0$ be a root algebra for $\phi$.
If $\frak B_0$ is purely atomic, then there is a generating
Bernoulli partition $\sequence{i}{b_i}$ for $\phi$ of entropy $h(\phi)$.
By 387E, there is a Bernoulli partition $\sequence{i}{c_i}$ for $\pi$
such that $\bar\mu c_i=\bar\nu b_i$ for every $i$.   Let $\frak C$ be
the closed subalgebra of $\frak A$ generated by
$\{\pi^jc_i:i\in\Bbb N,\,j\in\Bbb Z\}$.   Now
$(\frak C,\bar\mu\restrp\frak C,\pi\restrp\frak C)$
is a factor of $(\frak A,\bar\mu,\pi)$ isomorphic to
$(\frak B,\bar\nu,\phi)$.

\medskip

{\bf (b)} If $\frak B_0$ is not purely atomic, then there is still a
partition of unity $\sequence{i}{b_i}$ in $\frak B_0$ of infinite
entropy.   Again, let $\frak C$ be the closed subalgebra of $\frak A$
generated by $\{\pi^jc_i:i\in\Bbb N,\,j\in\Bbb Z\}$, where
$\sequence{i}{c_i}$ is a Bernoulli partition for $\pi$ such that
$\bar\mu c_i=\bar\nu b_i$ for every $i$.   Now $\pi\restrp\frak C$ is a
Bernoulli
shift of infinite entropy and $\frak C$ has countable Maharam type, so
387K tells us that there is a closed subalgebra
$\frak C_0\subseteq\frak C$ such that $\sequence{k}{\pi^k[\frak C_0]}$
is independent and $(\frak C_0,\bar\mu\restrp\frak C_0)$ is isomorphic to the measure algebra of
Lebesgue measure on $[0,1]$.   But $(\frak B_0,\bar\nu\restrp\frak B_0)$
is a probability algebra of countable Maharam type, so is isomorphic to
a closed subalgebra $\frak C_1$ of $\frak C_0$ (332N).   Of course
$\sequence{k}{\pi^k[\frak C_1]}$ is independent, so if we take
$\frak C_1^*$ to be the closed subalgebra of $\frak A$ generated by
$\bigcup_{k\in\Bbb Z}\pi^k[\frak C_1]$, $\pi\restrp\frak C_1^*$ will be a two-sided Bernoulli shift isomorphic to $\phi$ (385Sf).

\medskip

{\bf (c)} If $\phi$ is a one-sided Bernoulli shift, then 385Sa and 385Sc
show that $(\frak B,\bar\nu,\phi)$ can be represented in terms of a
product measure on a space $X^{\Bbb N}$ and the standard shift operator
on $X^{\Bbb N}$.   Now this extends naturally to the standard two-sided
Bernoulli shift represented by the product measure on $X^{\Bbb Z}$, as
described in 385Sb (cf.\ 385Yf);  so that $(\frak B,\bar\nu,\phi)$ becomes represented as a factor of $(\frak B',\bar\nuprime,\phi')$ where $\phi'$ is a two-sided Bernoulli shift with the same
entropy as $\phi$ (since the entropy is determined by the root algebra, by 385R).   By (a)-(b), $(\frak B',\bar\nuprime,\phi')$ is isomorphic to a factor of
$(\frak A,\bar\mu,\pi)$, so $(\frak B,\bar\nu,\phi)$ also is.
}%end of proof of 387L

\cmmnt{\medskip

\noindent{\bf Remark} Thus $(\frak A,\bar\mu,\pi)$ has factors which are
Bernoulli shifts based on root algebras of all countably-generated types
permitted by the entropy of $\pi$.
}%end of comment

\exercises{
\leader{387X}{Basic exercises (a)}
%\spheader 387Xa %3{8}6Xa
Let $(\frak A,\bar\mu)$ be a probability algebra, and
$\pi:\frak A\to\frak A$ a one- or two-sided Bernoulli shift.   Show that
$\pi^n$ is a Bernoulli shift
for any $n\ge 1$.   \Hint{if $\frak A_0$ is a root algebra for $\pi$,
the closed subalgebra generated by $\bigcup_{j<n}\pi^j[\frak A_0]$ is a
root algebra for $\pi^n$.}
%-

\spheader 387Xb %3{8}6Xb
Let $(\frak A,\bar\mu)$ be a probability algebra,
$\frak B$ a closed subalgebra of $\frak A$ and
$\pi\in\Aut_{\bar\mu}\frak A$ a measure-preserving automorphism such that
$\pi[\frak B]=\frak B$.
Show that if $\pi$ is ergodic or mixing, so is $\pi\restr\frak B$.
%387A

\spheader 387Xc %3{8}6Xc
Let $(\frak A,\bar\mu)$ be a probability algebra of
countable Maharam type, and $\pi:\frak A\to\frak A$ a two-sided
Bernoulli shift.
Show that for any $n\ge 1$ there is a Bernoulli shift
$\phi:\frak A\to\frak A$ such that $\phi^n=\pi$.   \Hint{construct a
Bernoulli shift
$\psi$ such that $h(\psi)=\bover1n h(\pi)$, and use 385Xh and Ornstein's
theorem to show that $\pi$ is isomorphic to $\psi^n$.}
%387K

\spheader 387Xd %3{8}6Xd
Let $\sequence{i}{\alpha_i}$, $\sequence{i}{\beta_i}$ be
non-negative real sequences such that
$\sum_{i=0}^{\infty}\alpha_i=\sum_{i=0}^{\infty}\beta_i=1$ and
$\sum_{i=0}^{\infty}q(\alpha_i)\discretionary{}{}{}
=\sum_{i=0}^{\infty}q(\beta_i)$.   Let
$\mu_0$, $\nu_0$ be the measures on $\Bbb N$ defined by the formulae

\Centerline{$\mu_0E=\sum_{i\in E}\alpha_i$,
\quad$\nu_0E=\sum_{i\in E}\beta_i$}

\noindent for $E\subseteq\Bbb N$.   Set $X=\BbbN^{\Bbb Z}$ and
let $\mu$, $\nu$ be the product measures on $X$ derived from $\mu_0$ and
$\nu_0$.   Show that there is a permutation $f:X\to X$ such that $\nu$ is
precisely the image measure $\mu f^{-1}$ and $f$ is
translation-invariant, that is,
$f(x\theta)=f(x)\theta$ for every $x\in X$, where $\theta(n)=n+1$ for
every $n\in\Bbb Z$.
%387K

\spheader 387Xe Let $(\frak A,\bar\mu,\pi)$ and $(\frak B,\bar\nu,\phi)$ be
probability algebras of countable Maharam type with two-sided Bernoulli shifts.
Suppose that each is isomorphic to a factor of the other.   Show that they are
isomorphic.
%387K

\leader{387Y}{Further exercises (a)} %3{8}6Ya
Suppose that $(\frak A,\bar\mu,\pi)$ and $(\frak B,\bar\nu,\phi)$ are
probability algebras with one-sided Bernoulli shifts, and that they are
isomorphic.   Show that they have isomorphic root algebras.
\Hint{apply the results of \S333 to $(\frak A,\bar\mu,\pi[\frak A])$.}
}%end of exercises

\endnotes{\Notesheader{387} The arguments here are expanded from {\smc
Smorodinsky 71} and {\smc Ornstein 74}.   I have sought the most direct
path to 387I and 387K;  of course there is a great deal more to be said (387Xc is a hint),
and, in particular, extensions of the methods here provide powerful
theorems enabling us to show that automorphisms are Bernoulli shifts.
(See {\smc Ornstein 74}.)
}%end of notes

\discrpage

