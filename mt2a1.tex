\frfilename{mt2a1.tex} 
\versiondate{20.1.13} 
\copyrightdate{1996} 
      
\def\chaptername{Appendix} 
\def\sectionname{Set theory} 
      
\def\Enderton{{\smc Enderton 77}} 
\def\Halmos{{\smc Halmos 60}} 
\def\Henle{{\smc Henle 86}} 
\def\Krivine{{\smc Krivine 71}} 
\def\Lipschutz{{\smc Lipschutz 64}} 
\def\Roitman{{\smc Roitman 90}} 
      
\newsection{2A1} 
      
Especially for the examples in Chapter 21, we need some non-trivial set 
theory, which is best approached through the standard theory of 
cardinals and ordinals;  and elsewhere in this volume I make use of 
Zorn's Lemma. 
Here I give a very brief outline of the results involved, largely 
omitting proofs. 
Most of this material should be in any sound introduction to set theory. 
The references I give are to books which happen to have come my way and 
which I can recommend as reasonably suitable for beginners. 
      
I do not discuss axiom systems or logical foundations.   The set theory 
I employ is `naive' in the sense that I rely on my understanding of 
the collective experience of the last hundred years, rather than on any 
attempt at formal description, to distinguish legitimate from unsafe 
arguments.   There are, however, points in Volume 5 at which such a 
relaxed philosophy becomes inappropriate, and I therefore use arguments 
which can, I believe, be translated into standard 
Zermelo-Fraenkel set theory without new ideas being invoked. 
      
Although in this volume I use the axiom of choice without scruple 
whenever appropriate, I will divide this section into two parts, 
starting with ideas and results not dependent on the axiom of choice 
(2A1A-2A1I) and continuing with the remainder 
(2A1J-2A1P).   I believe that even at this level it helps us to 
understand the nature of the arguments better if we maintain a degree of 
separation. 
      
\leader{2A1A}{Ordered sets (a)} Recall that a {\bf partially ordered 
set} is a set $P$ together with a relation $\le$ on $P$ such that 
      
\inset{if $p\le q$ and $q\le r$ then $p\le r$, 
      
$p\le p$ for every $p\in P$, 
      
if $p\le q$ and $q\le p$ then $p=q$.} 
      
\noindent In this context, I will write $p\ge q$ to mean $q\le p$, and 
$p<q$ or $q>p$ to mean `$p\le q$ and $p\ne q$'.   $\le$ is a 
{\bf partial order} on $P$. 
      
\header{2A1A}{\bf (b)} Let $(P,\le)$ be a partially ordered set, and 
$A\subseteq P$.   A {\bf maximal} element of $A$ 
is a $p\in A$ such that $p\not<a$ for any $a\in A$.   Note that $A$ may 
have more than one maximal element.  An {\bf upper bound} for $A$ is a 
$p\in P$ such that $a\le p$ for every $a\in A$;  a {\bf supremum} or 
{\bf least upper bound} is an upper bound $p$ such that $p\le q$ for 
every upper bound $q$ of $A$.   There can be at most one 
such\cmmnt{, because 
if $p$, $p'$ are both least upper bounds then $p\le p'$ and $p'\le p$}. 
Accordingly we may safely write $p=\sup A$ if $p$ is the least upper 
bound of $A$. 
      
Similarly, a {\bf minimal} element of $A$ is a $p\in A$ such that 
$p\not>a$ for every $a\in A$;  a {\bf lower bound} of $A$ is a $p\in P$ 
such that $p\le a$ for every $a\in A$;  and 
$\inf A=p$ means that 
      
\Centerline{for every $q\in P$, $p\ge q\iff a\ge q\Forall a\in A$.} 
      
A subset $A$ of $P$ is {\bf order-bounded} if it has both an upper bound 
and a lower bound. 
      
A subset $A$ of $P$ is {\bf upwards-directed} if for any $p$, $p'\in A$ 
there is a $q\in A$ such that $p\le q$ and $p'\le q$\cmmnt{;  that is, 
if any non-empty finite subset of $A$ has an upper bound in $A$}. 
Similarly, $A\subseteq P$ is {\bf downwards-directed} if for any $p$, 
$p'\in A$ there is a $q\in A$ such that $q\le p$ and $q\le p'$\cmmnt{; 
that is, if any non-empty finite subset of $A$ has a lower bound in 
$A$}. 
      
\cmmnt{It is sometimes convenient to adapt the notation for closed 
intervals to arbitrary partially ordered sets:}  $[p,q]$ will be 
$\{r:p\le r\le q\}$. 
 
\wheader{2A1A}{0}{0}{0}{24pt} 
      
\header{2A1Ac}{\bf (c)} A {\bf totally ordered set} is a partially 
ordered set $(P,\le)$ such that 
      
\inset{for any $p$, $q\in P$, either $p\le q$ or $q\le p$.} 
      
\noindent $\le$ is then a {\bf total} or {\bf linear} order on $P$.    
 
In any 
totally ordered set we have a {\bf median function}:  for $p$, $q$,  
$r\in P$ set  
 
$$\eqalign{\med(p,q,r) 
&=\max(\min(p,q),\min(p,r),\min(q,r))\cr 
&=\min(\max(p,q),\max(p,r),\max(q,r))\dvro{.}{,}\cr}$$ 
 
\cmmnt{\noindent so that $\med(p,q,r)=q$ if $p\le q\le r$.} 
      
\header{2A1Ad}{\bf (d)} A {\bf lattice} is a partially ordered set 
$(P,\le)$ such that 
      
\inset{for any $p$, $q\in P$, $p\vee q=\sup\{p,q\}$ and 
$p\wedge q=\inf\{p,q\}$ are defined in $P$.} 
      
\header{2A1Ae}{\bf (e)} A {\bf well-ordered set} is a totally ordered 
set $(P,\le)$ such that\cmmnt{ $\inf A$ exists and belongs to $A$ for 
every non-empty set $A\subseteq P$;  that is,} every non-empty subset of 
$P$ has a least element.   In this case 
$\le$ is a {\bf well-ordering} of $P$. 
      
\leader{2A1B}{Transfinite Recursion:  Theorem}  Let $(P,\le)$ be a 
well-ordered set and $X$ any class.   For $p\in P$ write $L_p$ for the 
set $\{q:q\in P,\,q<p\}$ and $X^{L_p}$ for the class of all functions 
from $L_p$ to 
$X$.   Let $F:\bigcup_{p\in P}X^{L_p}\to X$ be any function.   Then 
there is a unique function $f:P\to X$ such that $f(p)=F(f\restr L_p)$ 
for every $p\in P$. 
      
\proof{ There are versions of this result in \Enderton\ (p.\ 175) and 
\Halmos\ (\S18).   Nevertheless I write out a proof, since it seems to 
me that most elementary books on set theory do not give it its proper 
place at the very beginning of the theory of well-ordered sets. 
      
\medskip 
      
{\bf (a)} Let $\Phi$ be the class of all functions $\phi$ such that 
      
\quad ($\alpha$) $\dom\phi$ is a subset of $P$, and 
$L_p\subseteq\dom\phi$ for every $p\in\dom\phi$; 
      
\quad ($\beta$) $\phi(p)\in X$ for every $p\in\dom\phi$, and 
$\phi(p)=F(\phi\restr L_p)$ for every $p\in\dom\phi$. 
      
\medskip 
      
{\bf (b)} If $\phi$, $\psi\in\Phi$ then $\phi$ and $\psi$ agree on 
$\dom\phi\cap\dom\psi$.   \Prf\Quer\ If not, then 
$A=\{q:q\in\dom\phi\cap\dom\psi,\,\phi(q)\ne\psi(q)\}$ is non-empty. 
Because $P$ is well-ordered, $A$ has a least element $p$ say.   Now 
$L_p\subseteq\dom\phi\cap\dom\psi$ and $L_p\cap A=\emptyset$, so 
      
\Centerline{$\phi(p)=F(\phi\restr L_p)=F(\psi\restr L_p)=\psi(p)$,} 
      
\noindent which is impossible.\ \Bang\Qed 
      
\medskip 
      
{\bf (c)} It follows that $\Phi$ is a set, since the function 
$\phi\mapsto\dom\phi$ is an injective function from $\Phi$ to  $\Cal 
PP$, and its inverse is a surjection from a subset of $\Cal PP$ onto 
$\Phi$.    We can therefore, without inhibitions, define a function $f$ 
by writing 
      
\Centerline{$\dom f=\bigcup_{\phi\in\Phi}\dom\phi$, 
\quad$f(p)=\phi(p)$ whenever $\phi\in\Phi$ and $p\in\dom\phi$.} 
      
\noindent (If you think that a function $\phi$ is just the set of 
ordered pairs $\{(p,\phi(p)):p\in\dom\phi\}$, then $f$ becomes 
$\bigcup\Phi$.)   Then $f\in\Phi$.   \Prf\ Of course $f$ is a function 
from a subset of $P$ to $X$.   If $p\in\dom f$, then there is a 
$\phi\in\Phi$ such that $p\in\dom\phi$, in which case 
      
\Centerline{$L_p\subseteq\dom\phi\subseteq\dom f$, 
\quad$f(p)=\phi(p)=F(\phi\restr L_p)=F(f\restr L_p)$. \Qed} 
      
\medskip 
      
{\bf (d)} $f$ is defined everywhere in $P$.   \Prf\Quer\  Otherwise, 
$P\setminus\dom f$ is non-empty and has a least element $r$ say.   Now 
$L_r\subseteq\dom f$.   Define a function $\psi$ by saying that 
$\dom\psi=\{r\}\cup\dom f$, $\psi(p)=f(p)$ for $p\in\dom f$ and 
$\psi(r)=F(f\restr L_r)$.   Then $\psi\in\Phi$, because if 
$p\in\dom\psi$ 
      
\inset{{\it either} $p\in\dom f$ so 
$L_p\subseteq\dom f\subseteq\dom\psi$ and 
      
\Centerline{$\psi(p)=f(p)=F(f\restr L_p)=F(\psi\restr L_p)$}} 
      
\inset{{\it or} $p=r$ so $L_p=L_r\subseteq\dom f\subseteq\dom\psi$ and 
      
\Centerline{$\psi(p)=F(f\restr L_r)=F(\psi\restr L_r)$.}} 
      
\noindent Accordingly $\psi\in \Phi$ and 
$r\in\dom\psi\subseteq\dom f$.\ \Bang\Qed 
      
\medskip 
      
{\bf (e)} Thus $f:P\to X$ is a function such that $f(p)=F(f\restr L_p)$ 
for every $p$.   To see that $f$ is unique, observe that any function of 
this type must belong to $\Phi$, so must agree with $f$ on their common 
domain, which is the whole of $P$. 
}%end of proof of 2A1B 
      
\cmmnt{\medskip 
      
\noindent{\bf Remark} If you have been taught to distinguish between the 
words `set' and `class', you will observe that my naive set theory 
is a relatively tolerant one in that it is willing to allow class 
variables in its theorems. 
} 
      
\vleader{48pt}{2A1C}{Ordinals} An {\bf ordinal} (sometimes called a `von 
Neumann ordinal') is a set $\xi$ such that 
      
\inset{if $\eta\in\xi$ then $\eta$ is a set and $\eta\not\in\eta$, 
      
if $\eta\in\zeta\in\xi$ then $\eta\in\xi$, 
      
writing `$\eta\le\zeta$' to mean `$\eta\in\zeta$ or 
$\eta=\zeta$', $(\xi,\le)$ is well-ordered\dvro{.}{}} 
      
\prooflet{ 
\noindent(\Enderton, p.\ 191;  \Halmos, \S19;  \Henle, p.\ 27; \Krivine, 
p.\ 24;  \Roitman, 3.2.8.\cmmnt{ Of course many set 
theories do not allow sets to belong to themselves, and/or take it for 
granted that every object of discussion is a set, but I prefer not to 
take a view on such points in general.}) 
} 
      
\leader{2A1D}{Basic facts about ordinals (a)} If $\xi$ is an ordinal, 
then every member of $\xi$ is an ordinal. \prooflet{(\Enderton, p.\ 192; 
\Henle, 6.4; 
\Krivine, p.\ 14;  \Roitman, 3.2.10.)} 
      
\header{2A1Db}{\bf (b)} If $\xi$, $\eta$ are ordinals then either 
$\xi\in\eta$ or $\xi=\eta$ or $\eta\in\xi$ (and no two of these can 
occur together).   \prooflet{(\Enderton, 
p.\ 192;  \Henle, 6.4;  \Krivine, p.\ 14;  \Lipschutz, 11.12;  \Roitman, 
3.2.13.)}   It is customary\cmmnt{, in this case,} to 
write $\eta<\xi$ if $\eta\in\xi$ and $\eta\le\xi$ if either $\eta\in\xi$ 
or $\eta=\xi$.   Note that $\eta\le\xi$ iff $\eta\subseteq\xi$. 
      
\header{2A1Dc}{\bf (c)} If $A$ is any non-empty class of ordinals, then 
there is an $\alpha\in A$ such that $\alpha\le\xi$ for every 
$\xi\in A$.   \prooflet{(\Henle, 6.7;  \Krivine, p.\ 15.)} 
      
\header{2A1Dd}{\bf (d)} If $\xi$ is an ordinal, so is $\xi\cup\{\xi\}$; 
call it `$\xi+1$'.   \cmmnt{If $\xi<\eta$ then $\xi+1\le\eta$;}  $\xi+1$ is 
the least ordinal greater than $\xi$.   \prooflet{(\Enderton, p.\ 193; 
\Henle, 6.3; 
\Krivine, p.\ 15.)}   For any ordinal $\xi$, either there is a greatest 
ordinal $\eta<\xi$, in which case $\xi=\eta+1$ and we call $\xi$ a {\bf 
successor ordinal}, or $\xi=\bigcup\xi$, in which case we call $\xi$ a 
{\bf limit ordinal}. 
      
\header{2A1De}{\bf (e)} The first few ordinals are $0=\emptyset$, 
$1=0+1=\{0\}=\{\emptyset\}$, 
$2=1+1=\{0,1\}=\{\emptyset,\{\emptyset\}\}$, 
$3=2+1=\{0,1,2\}$, $\ldots$.   The first infinite ordinal is 
$\omega=\{0,1,2,\ldots\}$, which may be identified with $\Bbb N$. 
      
\header{2A1Df}{\bf (f)} The union of any set of ordinals is an ordinal. 
\prooflet{(\Enderton, p.\ 193; \Henle, 6.8;   \Krivine, p.\ 15; 
\Roitman, 3.2.19.)} 
      
\header{2A1Dg}{\bf (g)} If $(P,\le)$ is any well-ordered set, there is a 
unique ordinal $\xi$ such that $P$ is order-isomorphic to $\xi$, and the 
order-isomorphism is unique.   \prooflet{(\Enderton, pp.\ 187-189; 
\Henle, 6.13;  \Halmos, \S20.)} 
      
\leader{2A1E}{Initial ordinals} An {\bf initial ordinal} is an ordinal 
$\kappa$ such that there is no bijection between $\kappa$ and any member 
of $\kappa$.  \prooflet{(\Enderton, p.\ 197;  \Halmos, \S25;  \Henle, 
p.\ 34;  \Krivine, p.\ 24;  \Roitman, 5.1.10, p.\ 79).} 
      
\leader{2A1F}{Basic facts about initial ordinals (a)} All finite 
ordinals, and the first infinite ordinal $\omega$, are initial ordinals. 
      
\header{2A1Fb}{\bf (b)} For every well-ordered set $P$ there is a unique 
initial ordinal $\kappa$ such that there is a bijection between $P$ and 
$\kappa$. 
      
\header{2A1Fc}{\bf (c)} For every ordinal $\xi$ there is a least initial 
ordinal greater than $\xi$.   \prooflet{(\Enderton, p.\ 195;  \Henle, 
7.2.1.)}   If 
$\kappa$ is an initial ordinal, write $\kappa^+$ for the least initial 
ordinal greater than $\kappa$.   We write $\omega_1$ for $\omega^+$, 
$\omega_2$ for $\omega_1^+$, and so on. 
      
\header{2A1Fd}{\bf (d)} For any initial ordinal $\kappa\ge\omega$ there 
is a bijection between $\kappa\times\kappa$ and $\kappa$;  consequently 
there are bijections between $\kappa$ and $\kappa^r$ for every $r\ge 1$. 
      
\leader{2A1G}{Schr\"oder-Bernstein theorem} I\cmmnt{ remind you of the 
following 
fundamental result:  i}f $X$ and $Y$ are sets and there are injections 
$f:X\to Y$, $g:Y\to X$ then there is a bijection $h:X\to Y$. 
\prooflet{(\Enderton, p.\ 147;  \Halmos, \S22;  \Henle, 7.4; 
\Lipschutz, p.\ 145;  \Roitman, 5.1.2.   \cmmnt{It is also a special 
case of 344D in Volume 3.})} 
      
\leader{2A1H}{Countable subsets of $\Cal P\Bbb N$} \cmmnt{The 
following results will be needed below. 
      
\medskip 
      
}{\bf (a)} There is a bijection between $\Cal P\Bbb N$ and $\Bbb R$. 
\cmmnt{(\Enderton, p.\ 149;  \Lipschutz, p.\ 146.)} 
      
\spheader 2A1Hb Suppose that $X$ is any set such that there is 
an injection from $X$ into $\Cal P\Bbb N$.   Let $\Cal C$ be the set of 
countable subsets of $X$.   Then there is a surjection from 
$\Cal P\Bbb N$ onto $\Cal C$.   \prooflet{\Prf\ Let $f:X\to\Cal P\Bbb N$ be an injection.   Set $f_1(x)=\{0\}\cup\{i+1:i\in f(x)\}$;  then 
$f_1:X\to\Cal P\Bbb N$ is injective and $f_1(x)\ne\emptyset$ for every 
$x\in X$.   Define $g:\Cal P\Bbb N\to\Cal PX$ by setting 
      
\Centerline{$g(A)=\{x:\exists\,n\in\Bbb N,\,f_1(x) 
=\{i:2^n(2i+1)\in A\}\}$} 
      
\noindent for each $A\subseteq\Bbb N$.    Then $g(A)$ is countable, 
since we have an injection 
      
\Centerline{$x\mapsto\min\{n:f_1(x)=\{i:2^n(2i+1)\in A\}\}$} 
      
\noindent from $g(A)$ to $\Bbb N$.   Thus $g$ is a function from $\Cal 
P\Bbb N$ to $\Cal C$.   To see that $g$ is surjective, observe that 
$\emptyset=g(\emptyset)$, while if $C\subseteq X$ is countable and not 
empty there is a surjection $h:\Bbb N\to C$;  now set 
      
\Centerline{$A=\{2^n(2i+1):n\in\Bbb N,\,i\in f_1(h(n))\}$,} 
      
\noindent and see that $g(A)=C$.\ \Qed} 
      
\spheader 2A1Hc Again suppose that $X$ is a set such that there 
is an injection from $X$ to $\Cal P\Bbb N$, and write $H$ for the set of 
functions $h$ such that $\dom h$ is a countable subset of $X$ and $h$ 
takes values in $\{0,1\}$.   Then there is a surjection from $\Cal P\Bbb 
N$ onto $H$.   \prooflet{\Prf\ Let $\Cal C$ be the set of countable 
subsets of $X$ and let $g:\Cal P\Bbb N\to\Cal C$ be a surjection, as in 
(a).   For $A\subseteq\Bbb N$ set 
      
\Centerline{$g_0(A)=g(\{i:2i\in A\})$,\quad$g_1(A)=g(\{i:2i+1\in A\})$,} 
      
\noindent so that $g_0(A)$, $g_1(A)$ are countable subsets of $X$, and 
$A\mapsto(g_0(A),g_1(A))$ is a surjection from $\Cal P\Bbb N$ onto 
$\Cal C\times\Cal C$.   Let $h_A$ be the function with domain $g_0(A)\cup g_1(A)$ such that $h_A(x)=1$ if $x\in g_1(A)$, $0$ if 
$x\in g_0(A)\setminus g_1(A)$.   Then $A\mapsto h_A$ is a surjection from $\Cal P\Bbb N$ onto $H$.\ \Qed} 
      
\leader{2A1I}{Filters} \cmmnt{I pause for a moment to discuss a 
construction which is of great value in investigating topological 
spaces, but has 
other uses, and in its nature belongs to elementary set theory (much 
more elementary, indeed, than the work above). 
      
\header{2A1Ia}}{\bf (a)} Let $X$ be a non-empty set.   A {\bf filter} on 
$X$ is a family $\Cal F$ of subsets of $X$ such that 
      
\inset{$X\in\Cal F$,\quad $\emptyset\notin\Cal F$, 
      
$E\cap F\in\Cal F$ whenever $E$, $F\in\Cal F$, 
      
$E\in\Cal F$ whenever $X\supseteq E\supseteq F\in\Cal F$.} 
      
\noindent The second condition implies (inducing on $n$) that 
$F_0\cap\ldots\cap F_n\in\Cal F$ whenever $F_0,\ldots,F_n\in\Cal F$. 
          
\header{2A1Ib}{\bf (b)} Let $X$, $Y$ be non-empty sets, $\Cal F$ a 
filter on $X$ and $f:D\to Y$ a function, where $D\in\Cal F$.   Then 
      
\Centerline{$\{E:E\subseteq Y,\,f^{-1}[E]\in\Cal F\}$} 
      
\noindent is a filter on $Y$\prooflet{ (because $f^{-1}[Y]=D$, 
$f^{-1}[\emptyset]=\emptyset$, 
$f^{-1}[E\cap F]=f^{-1}[E]\cap f^{-1}[F]$, 
$X\supseteq f^{-1}[E]\supseteq f^{-1}[F]$ whenever  
$Y\supseteq E\supseteq F$)};  I will call it $f[[\Cal F]]$,  
the {\bf image filter} of $\Cal F$ under $f$. 
      
\cmmnt{\medskip 
      
\noindent{\bf Remark} Of course there is a hidden variable in this 
notation.   Ordinarily in this book I regard a function $f$ as being 
defined by its domain $\dom f$ and its values on its domain;  that is, 
it is determined by its graph $\{(x,f(x)):x\in\dom f\}$, and indeed I 
normally do not distinguish between a function and its graph.   This 
means that when I write `$f:D\to Y$ is a function' then the class 
$D=\dom f$ can be recovered from the function, but the class $Y$ cannot; 
all I promise is that $Y$ includes the class $f[D]$ of values of $f$. 
Now in the notation $f[[\Cal F]]$ above we do actually need to know 
which set $Y$ it is to be a filter on, even though this cannot be 
discovered from knowledge of $f$ and $\Cal F$.   So you will always have 
to infer it from the context. 
} 
      
\leader{2A1J}{The Axiom of Choice}\cmmnt{ I come now to the second 
half of this 
section, in which I discuss concepts and theorems dependent on the Axiom 
of Choice.}   Let me remind you of the statement of this axiom: 
      
\inset{\hskip-20pt (AC) `whenever $I$ is a set and $\familyiI{X_i}$ 
is a family of non-empty sets indexed by $I$, there is a function $f$, 
with domain $I$, such that $f(i)\in X_i$ for every $i\in I$'.} 
      
\noindent The function $f$ is a {\bf choice function}\cmmnt{;  it picks  
out one member of each of the given family of non-empty sets  $X_i$}. 
      
\cmmnt{I believe that one's attitude to this principle is a matter for 
individual choice.   It is an indispensable foundation for very large 
parts of twentieth-century pure mathematics, including a substantial 
fraction of the present volume;  but there are also significant areas in 
which principles actually contradictory to it can be employed to 
striking effect, leading -- in my view -- to equally valid mathematics. 
(I will describe one of these in \S567 of Volume 5.) 
At present it is the case that more current mathematical activity, by 
volume, 
depends on asserting the axiom of choice than on all its rivals put 
together;  but it is a matter of judgement and taste where the most 
important, or exciting, ideas are to be found.   For the present volume 
I follow standard practice in twentieth-century abstract analysis, using 
the axiom of choice whenever necessary.} 
      
\vleader{60pt}{2A1K}{Zermelo's Well-Ordering Theorem (a)} The Axiom of  
Choice is equiveridical with each of the statements 
      
\inset{`for every set $X$ there is a well-ordering of $X$',} 
      
\inset{`for every set $X$ there is a bijection between $X$ and some 
ordinal',} 
      
\inset{`for every set $X$ there is a unique initial ordinal $\kappa$ 
such that there is a bijection between $X$ and $\kappa$.'} 

\prooflet{\noindent(\Enderton, p.\ 196 et 
seq.; \Halmos, \S17;  \Henle, 9.1-9.3;  \Krivine, p.\ 20;  \Lipschutz, 
12.1;  \Roitman, 3.6.38.)} 
      
\header{2A1Kb}{\bf (b)} When assuming the axiom of choice, \cmmnt{as I 
do 
nearly everywhere in this treatise,} I write $\#(X)$ for that initial 
ordinal $\kappa$ such that there is a bijection between $\kappa$ and 
$X$;  I call this the {\bf cardinal} of $X$. 
      
\leader{2A1L}{Fundamental consequences of the Axiom of Choice (a)} For 
any two sets $X$ and $Y$, there is a bijection between $X$ and $Y$ iff 
$\#(X)=\#(Y)$.   More generally, there is an injection from $X$ to $Y$ 
iff $\#(X)\le\#(Y)$, and a surjection from $X$ onto $Y$ iff 
either $\#(X)\ge\#(Y)>0$ or $\#(X)=\#(Y)=0$. 
      
\spheader 2A1Lb\cmmnt{ In particular,} 
$\#(\Cal P\Bbb N)=\#(\Bbb R)$; 
write $\frak c$ for this common value, the {\bf cardinal of the 
continuum}.   \cmmnt{Cantor's theorem that $\Cal P\Bbb N$ and $\Bbb R$ are uncountable becomes the result $\omega<\frak c$, that is,} 
$\omega_1\le\frak c$. 
      
\header{2A1Lc}{\bf (c)} 
If $X$ is any infinite set, and $r\ge 1$, then there is a bijection 
between $X^r$ and $X$.   \prooflet{(\Enderton, p.\ 162;  \Halmos, 
\S24.)}   \cmmnt{(I note that we need some form of the axiom of 
choice to prove the result in this generality.   But of course for most 
of the infinite sets arising naturally in mathematics -- sets like $\Bbb 
N$ and $\Cal P\Bbb R$ -- it is easy to prove the result without appeal 
to the axiom of choice.)} 
      
\header{2A1Ld}{\bf (d)} Suppose that $\kappa$ is an infinite cardinal. 
If $I$ is a set with cardinal at most $\kappa$ and  
$\langle A_i\rangle_{i\in I}$ is a family of sets with $\#(A_i)\le\kappa$  
for 
every $i\in I$, then  $\#(\bigcup_{i\in I}A_i)\le\kappa$.   Consequently 
$\#(\bigcup\Cal A)\le\kappa$ whenever $\Cal A$ is a family of sets such 
that $\#(\Cal A)\le\kappa$ and $\#(A)\le\kappa$ for every $A\in\Cal A$. 
\cmmnt{In particular,} $\omega_1$ cannot be expressed as a countable union of 
countable sets, and $\omega_2$ cannot be expressed as a countable union 
of sets with cardinal at most $\omega_1$. 
      
\header{2A1Le}{\bf (e)} Now we can rephrase 2A1Hc as:  if  
$\#(X)\le\frak c$, then $\#(H)\le\frak c$,  
where $H$ is the set of functions from a 
countable subset of $X$ to $\{0,1\}$.   \prooflet{\Prf\ For we have an 
injection from $X$ into $\Cal P\Bbb N$, and therefore a surjection from 
$\Cal P\Bbb N$ onto $H$.\ \Qed} 
      
\spheader 2A1Lf Any non-empty class of cardinals has a least 
member\cmmnt{ (by 2A1Dc)}.   
      
\leader{2A1M}{Zorn's Lemma}\cmmnt{ In 2A1K I described the 
well-ordering 
principle.}   I come now to another proposition which is equiveridical with 
the axiom of choice: 
      
\inset{`Let $(P,\le)$ be a non-empty partially ordered set such that 
every non-empty totally ordered subset of $P$ has an upper bound in $P$. 
Then $P$ has a maximal element.'} 
      
\noindent This is {\bf Zorn's Lemma}.   \prooflet{For the proof that the 
axiom of 
choice implies, and is implied by, Zorn's Lemma, see \Enderton, p.\ 151; 
\Halmos, \S16;  \Henle, 9.1-9.3;  \Roitman, 3.6.38.} 
      
\leader{2A1N}{Ultrafilters} A filter $\Cal F$ on a set $X$ is an {\bf 
ultrafilter} if for 
every $A\subseteq X$ either $A\in\Cal F$ or $X\setminus A\in\Cal F$. 
      
If $\Cal F$ is an ultrafilter on $X$ and $f:D\to Y$ is a function, where 
$D\in\Cal F$, then 
$f[[\Cal F]]$ is an ultrafilter on $Y$\prooflet{ (because 
$f^{-1}[Y\setminus A]=D\setminus f^{-1}[A]$ for every $A\subseteq Y$)}. 
      
One type of ultrafilter can be described easily:  if $x$ is 
any point of a set $X$, then $\Cal F=\{F:x\in F\subseteq X\}$ is an 
ultrafilter on $X$.   (\prooflet{You need only read the 
definitions.  }Ultrafilters of this type are called {\bf principal ultrafilters}.) 
\cmmnt{But it is not obvious that there are any further ultrafilters, 
and 
indeed it is not possible to prove that there are any, without using a 
strong form of the axiom of choice, as follows.} 
      
\leader{2A1O}{The Ultrafilter Theorem}\cmmnt{ As an example of the use 
of Zorn's lemma which will be of great value in studying compact 
topological spaces (2A3N {\it et seq.}, and \S247), I give the following 
result. 
      
\medskip 
      
\noindent{\bf Theorem}} Let $X$ be any non-empty set, and $\Cal F$ a 
filter on $X$.   Then there is an ultrafilter $\Cal H$ on $X$ such that 
$\Cal F\subseteq\Cal H$. 
      
\proof{ (Cf.\ \Henle, 9.4;  \Roitman, 3.6.37.)   Let $\frak P$ be the 
set of all filters on $X$ 
including $\Cal F$, and order $\frak P$ by inclusion, so that, for 
$\Cal G_1$, $\Cal G_2\in\frak P$, $\Cal G_1\le \Cal G_2$ in $\frak P$ iff $\Cal G_1\subseteq\Cal G_2$.  It is easy to see that $\frak P$ is a 
partially ordered set, and it is non-empty because $\Cal F\in \frak P$.   If $\frak Q$ is any non-empty totally ordered 
subset of $\frak P$, then $\Cal H_{\frak Q}=\bigcup\frak Q\in\frak P$. 
\Prf\ Of course $\Cal H_{\frak Q}$ is a family of subsets of $X$. (i) 
Take any $\Cal G_0\in\frak Q$;  then 
$X\in\Cal G_0\subseteq\Cal H_{\frak Q}$.   If $\Cal G\in\frak Q$, then $\Cal G$ is a filter, so 
$\emptyset\notin\Cal G$;  accordingly $\emptyset\notin\Cal H_{\frak Q}$. 
(ii) If $E$, $F\in\Cal H_{\frak Q}$, then there are $\Cal G_1$, $\Cal 
G_2\in\frak Q$ such that $E\in\Cal G_1$ and $F\in\Cal G_2$.   Because 
$\frak Q$ is totally ordered, either $\Cal G_1\subseteq\Cal G_2$ or 
$\Cal G_2\subseteq\Cal G_1$.   In either case, $\Cal G=\Cal G_1\cup\Cal 
G_2\in\frak Q$.   Now $\Cal G$ is a filter containing both $E$ and $F$, 
so it contains $E\cap F$, and $E\cap F\in\Cal H_{\frak Q}$.   (iii) If 
$X\supseteq E\supseteq F\in\Cal H_{\frak Q}$, there is a $\Cal G\in\frak 
Q$ such that $F\in\Cal G$;   and $E\in\Cal G\subseteq\Cal H_{\frak Q}$. 
This shows that $\Cal H_{\frak Q}$ is a filter on $X$.   (iv)  Finally, 
$\Cal H_{\frak Q}\supseteq\Cal G_0\supseteq\Cal F$, so 
$\Cal H_{\frak Q}\in\frak P$.\ \QeD\ Now $\Cal H_{\frak Q}$ is evidently  an upper bound for $\frak Q$ in $\frak P$. 
      
We may therefore apply Zorn's Lemma to find a maximal element $\Cal H$ 
of $\frak P$.   This $\Cal H$ is surely a filter on $X$ including 
$\Cal F$. 
      
Now let $A\subseteq X$ be such that $A\notin \Cal H$.   Consider 
      
\Centerline{$\Cal H_1=\{E:E\subseteq X,\,E\cup A\in\Cal H\}$.} 
      
\noindent This is a filter on $X$.   \Prf\ Of course it is a family of 
subsets of $X$.   (i) $X\cup A=X\in\Cal H$, so $X\in\Cal H_1$. 
$\emptyset\cup A=A\notin\Cal H$ so $\emptyset\notin\Cal H_1$.   (ii) If 
$E$, $F\in\Cal H_1$ then 
      
\Centerline{$(E\cap F)\cup A=(E\cup A)\cap(F\cup A) 
\in\Cal 
H$,} 
      
\noindent so $E\cap F\in\Cal H_1$.   (iii) If 
$X\supseteq E\supseteq F\in\Cal H_1$ then 
$E\cup A\supseteq F\cup A\in\Cal H$, so $E\cup A\in\Cal H$ and 
$E\in\Cal H_1$.\ \QeD\  Also $\Cal H_1\supseteq\Cal H$, so 
$\Cal H_1\in\frak P$.   But $\Cal H$ is a maximal element of $\frak P$, so $\Cal H_1=\Cal H$.   Since $(X\setminus A)\cup A=X\in\Cal H$, 
$X\setminus A\in\Cal H_1$ and $X\setminus A\in\Cal H$. 
      
As $A$ is arbitrary, $\Cal H$ is an ultrafilter, as required. 
}%end of proof of 2A1O 
      
      
      
\leader{2A1P}{}\cmmnt{ I come now to a result from infinitary 
combinatorics for 
which I give a detailed proof, not because it cannot be found in many 
textbooks, but because it is usually given in enormously greater 
generality, to the point indeed that it may be harder to understand why 
the stated theorem covers the present result than to prove the latter 
from first principles. 
      
\medskip 
      
\noindent}{\bf Theorem} (a) Let 
$\langle K_{\alpha}\rangle_{\alpha\in A}$ be 
a family of countable sets, with $\#(A)$ strictly greater than 
$\frak c$, the cardinal of the continuum.   Then there are a set $M$, with cardinal at 
most $\frak c$, and a set $B\subseteq A$, with cardinal strictly greater 
than $\frak c$, such that $K_{\alpha}\cap K_{\beta}\subseteq M$ whenever 
$\alpha$, $\beta$ are distinct members of $B$. 
      
(b) Let $I$ be a set, and 
$\langle f_{\alpha}\rangle_{\alpha\in A}$ a family in $\{0,1\}^I$, the set of functions from $I$ to $\{0,1\}$, with $\#(A)>\frak c$.    If $\langle K_{\alpha}\rangle_{\alpha\in A}$ is any family of countable subsets of 
$I$, then there is a set $B\subseteq A$, with cardinal greater than 
$\frak c$, such that $f_{\alpha}$ and $f_{\beta}$ agree on 
$K_{\alpha}\cap K_{\beta}$ for all $\alpha$, $\beta\in B$. 
      
(c) In particular, under the conditions of (b), there are distinct 
$\alpha$, $\beta\in A$ such that $f_{\alpha}$ and $f_{\beta}$ agree on 
$K_{\alpha}\cap K_{\beta}$. 
      
\proof{{\bf (a)} Choose inductively a family 
$\langle M_{\xi}\rangle_{\xi<\omega_1}$ of sets by the rule 
      
\inset{if there is any set $N$ such that 
      
\inset{($*$) $N$ is disjoint from $\bigcup_{\eta<\xi}M_{\eta}$, 
$\#(N)\le\frak c$ and
%
%\qquad\qquad
$\#(\{\alpha:\alpha\in A,\,K_{\alpha}\cap N=\emptyset\}) 
  \le\frak c$,} 
      
\noindent choose such a set for $M_{\xi}$; 
      
otherwise set $M_{\xi}=\emptyset$.} 
      
\noindent When $M_{\xi}$ has been chosen for every $\xi<\omega_1$, set 
$M=\bigcup_{\xi<\omega_1}M_{\xi}$.   The rule ensures that $\ofamily{\xi}{\omega_1}{M_{\xi}}$ is disjoint and that 
$\#(M_{\xi})\le\frak c$ for every $\xi<\omega_1$, while 
$\omega_1\le\frak c$, so $\#(M)\le\frak c$. 
      
Let $\frak P$ be the family of sets $P\subseteq A$ such that 
$K_{\alpha}\cap K_{\beta}\subseteq M$ for all distinct $\alpha$, $\beta\in P$.   Order 
$\frak P$ by inclusion, so that it is a partially ordered set.   If 
$\frak Q\subseteq\frak P$ is totally ordered, then $\bigcup\frak Q\in\frak P$. 
\Prf\ If $\alpha$, $\beta$ are distinct members of $\bigcup\frak Q$, 
there are $Q_1$, $Q_2\in \frak Q$ such that $\alpha\in Q_1$, 
$\beta\in Q_2$; now 
$P=Q_1\cup Q_2$ is equal to one of $Q_1$, $Q_2$, and in either case 
belongs to $\frak P$ and contains both $\alpha$ and $\beta$, so $K_{\alpha}\cap K_{\beta}\subseteq M$.\ \QeD\  By Zorn's Lemma, 
$\frak P$ has a maximal 
element $B$, and we surely have $K_{\alpha}\cap K_{\beta}\subseteq M$ 
for all distinct $\alpha$, $\beta\in B$. 
      
\Quer\ Suppose, if possible, that $\#(B)\le\frak c$.   Set 
$N=\bigcup_{\alpha\in B}K_{\alpha}\setminus M$.   Then $N$ has cardinal 
at most $\frak c$, being included in a union of at most $\frak c$ 
countable sets.   For every $\gamma\in A\setminus B$, 
$B\cup\{\gamma\}\notin\frak P$, so there must be some $\alpha\in B$ such 
that $K_{\alpha}\cap K_{\gamma}\not\subseteq M$;  that is, 
$K_{\gamma}\cap N\ne\emptyset$.   Thus 
$\{\gamma:K_{\gamma}\cap N=\emptyset\}\subseteq B$ 
has cardinal at most $\frak c$.   But this means that in the rule for 
choosing $M_{\xi}$, there was always an $N$ satisfying the condition 
($*$), and therefore $M_{\xi}$ also did.   Thus 
$C_{\xi}=\{\alpha:K_{\alpha}\cap M_{\xi}=\emptyset\}$ has cardinal at most $\frak c$ for every 
$\xi<\omega_1$.   So $C=\bigcup_{\xi<\omega_1}C_{\xi}$ also has.   But 
the original hypothesis was that $\#(A)>\frak c$, so there is an $\alpha\in A\setminus C$.   In this case, 
$K_{\alpha}\cap M_{\xi}\ne\emptyset$ for 
every $\xi<\omega_1$.   But this means that we have a surjection 
$\phi:K_{\alpha}\cap M\to\omega_1$ given by setting 
      
\Centerline{$\phi(i)=\xi$ if $i\in K_{\alpha}\cap M_{\xi}$.} 
      
\noindent Since $\#(K_{\alpha})\le\omega<\omega_1$, this is 
impossible.\ \Bang 
      
Accordingly $\#(B)>\frak c$ and we have found a suitable pair $M$, $B$. 
      
\medskip 
      
{\bf (b)} By (a), we can find a set $M$, with cardinal at most $\frak c$, 
and a set $B_0\subseteq A$, with cardinal greater than $\frak c$, such that $K_{\alpha}\cap K_{\beta}\subseteq M$ for all distinct $\alpha$, 
$\beta\in B_0$.   Let $H$ be the set of functions from countable subsets of $M$ to $\{0,1\}$;  then 
$f'_{\alpha}=f_{\alpha}\restr(K_{\alpha}\cap M)\in H$ for 
each $\alpha\in B_0$.   Now 
$B_0=\bigcup_{h\in H}\{\alpha:\alpha\in B_0,\,f'_{\alpha}=h\}$ has cardinal greater than $\frak c$, while 
$\#(H)\le\frak c$ (2A1Le), so there must be some $h\in H$ such that 
$B=\{\alpha:\alpha\in B_0,\,f'_{\alpha}=h\}$ has cardinal greater than 
$\frak c$. 
      
If $\alpha$, $\beta$ are distinct members of $B$, then 
$K_{\alpha}\cap K_{\beta}\subseteq M$, because $\alpha$, $\beta\in B_0$;  but this means that 
      
\Centerline{$f_{\alpha}\restr K_{\alpha}\cap K_{\beta} 
=h\restr K_{\alpha}\cap K_{\beta} 
=f_{\beta}\restr K_{\alpha}\cap K_{\beta}$.} 
      
\noindent Thus $B$ has the required property.   (Of course $f_{\alpha}$ 
and $f_{\beta}$ agree on $K_{\alpha}\cap K_{\beta}$ if $\alpha=\beta$.) 
      
\medskip 
      
{\bf (c)} follows at once. 
}%end of proof of 2A1P 
%elementary submodel argument 
      
\cmmnt{\medskip 
      
\noindent{\bf Remark} The result we need in this volume (in 216E) is 
part (c) above.   There are other proofs of this,  
perhaps a little simpler; but the stronger result in part (b) will be 
useful in Volume 3. 
}%end of comment 
      
\discrpage 
      
