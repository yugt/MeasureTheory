\frfilename{mt343.tex}
\versiondate{17.11.10}
\copyrightdate{2001}

\def\chaptername{The lifting theorem}
\def\sectionname{Realization of homomorphisms}

\newsection{343}

We are now in a position to make progress in one of the basic questions
of abstract measure theory.   In \S324 I have already described the
way in which a function between two measure spaces can give rise to a
homomorphism between their measure algebras.   In this section I discuss
some conditions under which we can be sure that a
homomorphism can be represented by a function.

The principal theorem of the section is 343B.  If a measure space
$(X,\Sigma,\mu)$ is locally compact, then many homomorphisms from the
measure algebra of $\mu$ to other measure algebras will be representable
by functions into $X$;  moreover, this characterizes locally compact
spaces.   In general, a homomorphism between measure
algebras can be represented by widely different functions (343I,
343J).  But in some of the most important cases (e.g., Lebesgue
measure) representing functions are `almost' uniquely defined;  I
introduce the concept of `countably separated' measure space to describe these (343D-343H).

\leader{343A}{Preliminary remarks} \cmmnt{It will be helpful to
establish some vocabulary and a couple of elementary facts.

\medskip

}{\bf (a)} If $(X,\Sigma,\mu)$ and $(Y,\Tau,\nu)$ are
measure spaces, with measure algebras $\frak A$ and $\frak B$, I will
say that a function $f:X\to Y$ {\bf represents} a homomorphism
$\pi:\frak B\to\frak A$ if $f^{-1}[F]\in\Sigma$ and
$(f^{-1}[F])^{\ssbullet}=\pi(F^{\ssbullet})$ for every $F\in\Tau$.

\cmmnt{(Perhaps I should emphasize here that some homomorphisms are
representable in this sense, and some are not;  see 343M below for
examples of non-representable homomorphisms.)}

\spheader 343Ab If $(X,\Sigma,\mu)$ and $(Y,\Tau,\nu)$ are
measure spaces, with measure algebras $\frak A$ and $\frak B$,
$f:X\to Y$ is a function, and $\pi:\frak B\to\frak A$ is a
sequentially order-continuous Boolean homomorphism, then

\Centerline{$\{F:F\in\Tau,\,f^{-1}[F]\in\Sigma$ and
$f^{-1}[F]^{\ssbullet}=\pi F^{\ssbullet}\}$}

\noindent is a $\sigma$-subalgebra of $\Tau$.   \cmmnt{(The
verification is elementary.)}

\spheader 343Ac Let $(X,\Sigma,\mu)$ and $(Y,\Tau,\nu)$ be
measure spaces, with measure algebras $\frak A$ and $\frak B$, and
$\pi:\frak B\to\frak A$ a Boolean homomorphism which is represented by
a function $f:X\to Y$.   Let $(X,\hat\Sigma,\hat\mu)$,
$(Y,\hat\Tau,\hat\nu)$ be the completions of $(X,\Sigma,\mu)$,
$(Y,\Tau,\nu)$;  then $\frak A$ and $\frak B$ can be identified with
the measure algebras of $\hat\mu$ and $\hat\nu$\cmmnt{ (322Da)}.   Now
$f$ still represents $\pi$ when regarded as a function from
$(X,\hat\Sigma,\hat\mu)$ to $(Y,\hat\Tau,\hat\nu)$.   \prooflet{\Prf\
If $G$ is $\nu$-negligible, there is a negligible $F\in\Tau$ such
that $G\subseteq F$;  since

\Centerline{$f^{-1}[F]^{\ssbullet}=\pi F^{\ssbullet}=0$,}

\noindent $f^{-1}[F]$ is $\mu$-negligible, so $f^{-1}[G]$ is
negligible, therefore belongs to $\hat\Sigma$.   If $G$ is any element
of $\hat\Tau$, there is an $F\in\Tau$ such that $G\symmdiff F$ is
negligible, so that

\Centerline{$f^{-1}[G]
=f^{-1}[F]\symmdiff f^{-1}[G\symmdiff F]
\in\hat\Sigma$,}

\noindent and

\Centerline{$f^{-1}[G]^{\ssbullet}=f^{-1}[F]^{\ssbullet}
=\pi F^{\ssbullet}=\pi G^{\ssbullet}$.   \Qed}
}%end of prooflet

\leader{343B}{Theorem} Let $(X,\Sigma,\mu)$ be a non-empty
semi-finite measure space, and $(\frak A,\bar\mu)$ its measure algebra.
Let $(Z,\Lambda,\lambda)$ be the Stone space of $(\frak A,\bar\mu)$;
for $E\in\Sigma$ write $E^*$ for the
open-and-closed subset of $Z$ corresponding to the image $E^{\ssbullet}$
of $E$ in $\frak A$.   Then the following are equiveridical.

(i) $(X,\Sigma,\mu)$ is locally compact\cmmnt{ in the sense of 342Ad}.

(ii) There is a family $\Cal K\subseteq\Sigma$ such that ($\alpha$)
whenever $E\in\Sigma$ and $\mu E>0$ there is a $K\in\Cal K$ such that
$K\subseteq E$
and $\mu K>0$ ($\beta$) whenever $\Cal K'\subseteq\Cal K$ is such that
$\mu(\bigcap\Cal K_0)>0$ for every non-empty finite set
$\Cal K_0\subseteq\Cal K'$, then $\bigcap\Cal K'\ne\emptyset$.

(iii) There is a family $\Cal K\subseteq\Sigma$ such that ($\alpha$)$'$
$\mu$ is inner regular with respect to $\Cal K$ ($\beta$) whenever
$\Cal K'\subseteq\Cal K$ is such that $\mu(\bigcap\Cal K_0)>0$ for every
non-empty finite set $\Cal K_0\subseteq\Cal K'$, then
$\bigcap\Cal K'\ne\emptyset$.

(iv) There is a function $f:Z\to X$ such that $f^{-1}[E]\symmdiff E^*$
is negligible for every $E\in\Sigma$.

(v) Whenever $(Y,\Tau,\nu)$ is a complete strictly localizable
measure space, with measure algebra $\frak B$, and
$\pi:\frak A\to\frak B$ is an order-continuous Boolean homomorphism,
then there is a $g:Y\to X$ representing $\pi$.

(vi) Whenever $(Y,\Tau,\nu)$ is a complete strictly localizable
measure space, with measure algebra $\frak B$, and
$\pi:\frak A\to\frak B$ is an order-continuous measure-preserving Boolean
homomorphism, then there is a $g:Y\to X$ representing $\pi$.

\proof{{\bf (a)(i)$\Rightarrow$(ii)} Because $\mu$ is semi-finite, there
is a partition of unity $\langle a_i\rangle_{i\in I}$ in $\frak A$ such
that $\bar\mu a_i<\infty$ for each $i$.   For each $i\in I$, let
$E_i\in\Sigma$
be such that $E_i^{\ssbullet}=a_i$.   Then the subspace measure $\mu_{E_i}$
on $E_i$ is compact;  let $\Cal K_i\subseteq\Cal PE_i$ be a compact class
such that $\mu_{E_i}$ is inner regular with respect to $\Cal K_i$.   Set
$\Cal K=\bigcup_{i\in I}\Cal K_i$.   If $\Cal K'\subseteq\Cal K$ and
$\mu(\bigcap\Cal K_0)>0$ for every non-empty finite
$\Cal K_0\subseteq\Cal K$, then $\Cal K'\subseteq\Cal K_i$ for some $i$,
and surely has the
finite intersection property, so $\bigcap\Cal K'\ne\emptyset$;  thus
$\Cal K'$ satisfies ($\beta$) of condition (ii).   And if $E\in\Sigma$,
$\mu E>0$ then
there must be some $i\in I$ such that $E_i^{\ssbullet}\Bcap a_i\ne 0$,
that is, $\mu(E\cap E_i)>0$, in which case there is a
$K\in\Cal K_i\subseteq\Cal K$
such that $K\subseteq E\cap E_i$ and $\mu K>0$;  so that $\Cal K$
satisfies condition ($\alpha$).

\medskip

{\bf (b)(ii)$\Rightarrow$(iii)} Suppose that $\Cal K\subseteq\Sigma$
witnesses that (ii) is true.   If $\mu X=0$ then $\Cal K$ already
witnesses that (iii) is true, so we need consider only the case
$\mu X>0$.   Set
$\Cal L=\{K_0\cup\ldots\cup K_n:K_0,\ldots,K_n\in\Cal K\}$.   Then
$\Cal L$
witnesses that (iii) is true.   \Prf\ By 342Ba, $\mu$ is inner regular
with respect to $\Cal L$.
Let $\Cal L'\subseteq\Cal L$ be such that $\mu(\bigcap\Cal L_0)>0$ for
every non-empty finite $\Cal L_0\subseteq\Cal L'$.   Then

\Centerline{$\Cal F_0=\{A:A\subseteq X$, there is a finite $\Cal
L_0\subseteq\Cal L'$ such that $X\cap\bigcap\Cal L_0\setminus A$ is
negligible$\}$}

\noindent is a filter on $X$, so there is an ultrafilter $\Cal F$ on $X$
including $\Cal F_0$.   Note that every conegligible set belongs to
$\Cal F_0$, so no negligible set can belong to $\Cal F$.   Set $\Cal
K'=\Cal K\cap\Cal F$;  then $\bigcap\Cal K_0$ belongs to $\Cal F$, so is
not negligible, for every non-empty finite $\Cal K_0\subseteq\Cal K'$.
Accordingly there is some $x\in\bigcap\Cal K'$.   But any member of
$\Cal L'$ is of the form $L=K_0\cup\ldots\cup K_n$ where each
$K_i\in\Cal K$;
because $\Cal F$ is an ultrafilter and $L\in\Cal F$, there must be some
$i\le n$ such that $K_i\in\Cal F$, in which case $x\in K_i\subseteq L$.
Thus $x\in\bigcap\Cal L'$.   As $\Cal L'$ is arbitrary, $\Cal L$
satisfies the condition ($\beta$).\ \Qed

\medskip

{\bf (c)(iii)$\Rightarrow$(iv)}
Let $\Cal K\subseteq\Sigma$ witness that (iii) is true.   For any $z\in
Z$, set $\Cal
K_z=\{K:K\in\Cal K,\,z\in K^*\}$.    If $K_0,\ldots,K_n\in\Cal K_z$,
then $z\in\bigcap_{i\le n}K^*_i=(\bigcap_{i\le n}K_i)^*$, so
$(\bigcap_{i\le n}K_i)^*\ne\emptyset$ and $\mu(\bigcap_{i\le n}K_i)>0$.
By ($\beta$) of condition (iii), $\bigcap\Cal
K_z\ne\emptyset$;  and even if $\Cal K_z=\emptyset$, $X\cap\bigcap\Cal
K_z\ne\emptyset$ because $X$ is non-empty.   So we may choose $f(z)\in
X\cap\bigcap\Cal K_z$.   This defines a function $f:Z\to X$.   Observe
that, for $K\in\Cal K$ and $z\in Z$,

\Centerline{$z\in K^*\Longrightarrow K\in\Cal K_z\Longrightarrow f(z)\in
K\Longrightarrow z\in f^{-1}[K]$,}

\noindent so that $K^*\subseteq f^{-1}[K]$.

Now take any $E\in\Sigma$.   Consider

\Centerline{$U_1=\bigcup\{K^*:K\in\Cal K,\,K\subseteq E\}
\subseteq\bigcup\{E^*\cap f^{-1}[K]:K\in\Cal K,\,K\subseteq E\}
\subseteq E^*\cap f^{-1}[E]$,}

\Centerline{$U_2=\bigcup\{K^*:K\in\Cal K,\,K\subseteq X\setminus E\}
\subseteq (X\setminus E)^*\cap f^{-1}[X\setminus E]
=Z\setminus(f^{-1}[E]\cup E^*)$,}

\noindent so that $f^{-1}[E]\symmdiff E^*\subseteq Z\setminus(U_1\cup
U_2)$.   Now $U_1$ and $U_2$ are open subsets of $Z$, so
$M=Z\setminus(U_1\cup U_2)$ is closed, and in fact $M$ is nowhere dense.
\Prf\Quer\ Otherwise, there is a non-zero $a\in\frak A$ such that the
corresponding open-and-closed set
$\widehat{a}$ is included in $M$, and an $F\in\Sigma$ of non-zero
measure such that
$a=F^{\ssbullet}$.   At least one of $F\cap E$, $F\setminus E$ is
non-negligible and therefore includes a non-negligible member $K$ of
$\Cal K$.   But in this case $K^*$ is a non-empty open subset of $M$
which is included in either $U_1$ or $U_2$, which is
impossible.\ \Bang\Qed

By the definition of $\lambda$ (321J-321K), $M$ is
$\lambda$-negligible, so $f^{-1}[E]\symmdiff E^*\subseteq M$ is
negligible, as required.

\medskip

{\bf (d)(iv)$\Rightarrow$(v)} Now assume that $f:Z\to X$ witnesses
(iv), and let $(Y,\Tau,\nu)$ be a complete strictly localizable measure
space, with measure algebra $\frak B$, and $\pi:\frak A\to\frak B$ an
order-continuous Boolean homomorphism.   If $\nu Y=0$ then any function
from $Y$ to $X$ will represent $\pi$, so we may suppose that $\nu Y>0$.
Write $W$ for the Stone space
of $\frak B$.   Then we have a continuous function $\phi:W\to Z$ such
that $\phi^{-1}[\widehat{a}]=\widehat{\pi a}$ for every $a\in\frak A$
(312Q), and
$\phi^{-1}[M]$ is nowhere dense in $W$ for every nowhere dense
$M\subseteq Z$ (313R).   It follows that $\phi^{-1}[M]$ is meager for
every meager $M\subseteq Z$, that is, $\phi^{-1}[M]$ is negligible in
$W$ for every negligible $M\subseteq Z$.   By 341Q, there is an \imp\
function $h:Y\to W$ such that
$h^{-1}[\vthsp\widehat{b}\vthsp]^{\ssbullet}=b$ for
every $b\in\frak B$.   Consider $g=f\phi h:Y\to X$.

If $E\in\Sigma$, set $a=E^{\ssbullet}\in\frak A$, so that
$E^*=\widehat{a}\subseteq Z$, and $M=f^{-1}[E]\symmdiff E^*$ is
$\lambda$-negligible;  consequently $\phi^{-1}[M]$ is negligible in $W$.
Because $h$ is \imp,

\Centerline{$g^{-1}[E]\symmdiff h^{-1}[\phi^{-1}[E^*]]
=h^{-1}[\phi^{-1}[f^{-1}[E]]]\symmdiff h^{-1}[\phi^{-1}[E^*]]
=h^{-1}[\phi^{-1}[M]]$}

\noindent is negligible.   But $\phi^{-1}[E^*]=\widehat{\pi a}$, so

\Centerline{$g^{-1}[E]^{\ssbullet}
=h^{-1}[\phi^{-1}[E^*]]^{\ssbullet}=\pi a$.}
\noindent As $E$ is arbitrary, $g$ induces the homomorphism $\pi$.

\medskip

{\bf (e)(v)$\Rightarrow$(vi)} is trivial.

\medskip

{\bf (f)(vi)$\Rightarrow$(iv)} Assume (vi).   Let $\nu$ be the
c.l.d.\ version of $\lambda$, $\Tau$ its domain, and $\frak B$ its
measure algebra;  then $\nu$ is strictly localizable (322Rb).   The
embedding $\Lambda\embedsinto\Tau$ corresponds to
an order-continuous measure-preserving Boolean homomorphism from
$\frak A$ to $\frak B$ (322Db).
By (vi), there is a function $f:Z\to X$ such that $f^{-1}[E]\in\Tau$ and
$f^{-1}[E]^{\ssbullet}=(E^*)^{\ssbullet}$ in $\frak B$ for
every $E\in\Sigma$.   But as $\nu$ and $\lambda$ have the same
negligible sets (322Rb), $f^{-1}[E]\symmdiff E^*$ is
$\lambda$-negligible for every $E\in\Sigma$, as required by (iv).

\medskip

{\bf (g)(iv)$\Rightarrow$(i)}\grheada\ To begin with (down to the end of
($\gamma$) below) I suppose that $\mu$ is totally finite.   In this case
we have a function $g:X\to Z$ such that $E\symmdiff g^{-1}[E^*]$ is
negligible
for every $E\in\Sigma$ (341Q again).    We are supposing also that
there is a function $f:Z\to X$ such that $f^{-1}[E]\symmdiff E^*$
is negligible for every $E\in\Sigma$.   Write $\Cal K$ for the family of
sets $K\subseteq X$ such that $K\in\Sigma$ and there is a compact set
$L\subseteq Z$ such that $f[L]\subseteq K\subseteq g^{-1}[L]$.

\medskip

\quad\grheadb\ $\mu$ is inner regular with respect to $\Cal K$.   \Prf\
Take $F\in\Sigma$ and $\gamma<\mu F$.   Choose $\sequencen{V_n}$,
$\sequencen{F_n}$
as follows.  $F_0=F$.   Given that $\mu F_n>\gamma$, then

\Centerline{$\lambda(f^{-1}[F_n]\cap F_n^*)=\lambda F_n^*
=\mu F_n>\gamma$,}

\noindent so there is an open-and-closed set
$V_n\subseteq f^{-1}[F_n]\cap F_n^*$ with $\lambda V_n>\gamma$.
Express $V_n$ as $F_{n+1}^*$ where
$F_{n+1}\in\Sigma$;  since $F_n\symmdiff g^{-1}[F_n^*]$ is negligible,
and $V_n\subseteq F_n^*$, we may take it that
$F_{n+1}\subseteq g^{-1}[F_n^*]$.   Continue.

At the end of the induction, set $K=\bigcap_{n\in\Bbb N}F_n\in\Sigma$
and $L=\bigcap_{n\in\Bbb N}F_n^*$.   Because
$F_{n+1}\setminus F_n\subseteq g^{-1}[F_n^*]\setminus F_n$ is negligible
for each $n$,
$\mu K=\lim_{n\to\infty}\mu F_n\ge\gamma$, while $K\subseteq F$ and $L$
is surely compact.   We have

\Centerline{$L\subseteq\bigcap_{n\in\Bbb N}V_n
\subseteq\bigcap_{n\in\Bbb N}f^{-1}[F_n]=f^{-1}[K]$,}

\noindent so $f[L]\subseteq K$.   Also

\Centerline{$K\subseteq\bigcap_{n\in\Bbb N}F_{n+1}
\subseteq\bigcap_{n\in\Bbb N}g^{-1}[F_n^*]=g^{-1}[L]$.}

\noindent So $K\in\Cal K$.   As $F$ and $\gamma$ are arbitrary, $\mu$ is
inner regular with respect to $\Cal K$.\ \Qed

\medskip

\quad\grheadc\ Next, $\Cal K$ is a compact class.   \Prf\ Suppose that
$\Cal K'\subseteq\Cal K$ has the finite intersection property.   If
$\Cal K'=\emptyset$, of course $\bigcap\Cal K'\ne\emptyset$;  suppose
that $\Cal K'$ is non-empty.   Let $\Cal L$ be the family of closed sets
$L\subseteq Z$
such that $g^{-1}[L]$ includes some member of $\Cal K'$.   Then $\Cal L$
has the finite intersection property, and $Z$ is compact, so there is
some $z\in\bigcap\Cal L$;  also $Z\in\Cal L$, so $z\in Z$.   For any
$K\in\Cal K'$, there is some closed set $L\subseteq Z$ such that
$f[L]\subseteq
K\subseteq g^{-1}[L]$, so that $L\in\Cal L$ and $z\in L$ and $f(z)\in
K$.   Thus $f(z)\in\bigcap\Cal K'$.   As $\Cal K'$ is arbitrary, $\Cal
K$ is a compact class.\ \Qed

So $\Cal K$ witnesses that $\mu$ is a compact measure.

\medskip

\quad\grheadd\ Now consider the general case.
Take any $E\in\Sigma$ of finite measure.   If $E=\emptyset$ then surely
the subspace measure $\mu_E$ is compact.   Otherwise, we can identify
the measure algebra of $\mu_E$ with the principal ideal
$\frak A_{E^{\ssbullet}}$ of
$\frak A$ generated by $E^{\ssbullet}$ (322Ja), and $E^*\subseteq Z$
with the Stone space of $\frak A_{E^{\ssbullet}}$ (312T).   Take any
$x_0\in E$ and define
$\tilde f:E^*\to E$ by setting $\tilde f(z)=f(z)$ if
$z\in E^*\cap f^{-1}[E]$, $x_0$ if $z\in E^*\setminus f^{-1}[E]$.   Then
$f$ and $\tilde f$ agree almost everywhere in $E^*$, so
$\tilde f^{-1}[F]\symmdiff F^*$ is negligible for every $F\in\Sigma_E$,
that is, $\tilde f$ represents the canonical isomorphism between the
measure algebras of $\mu_E$ and the subspace measure $\lambda_{E^*}$ on
$E^*$.   But
this means that condition (iv) is true of $\mu_E$, so $\mu_E$ is
compact, by
($\alpha$)-($\gamma$) above.   As $E$ is arbitrary, $\mu$ is locally
compact.

This completes the proof.
}%end of proof of 343B

\leader{343C}{Examples (a)} Let $I$ be any set.
\dvro{The}{We know that the} usual measure $\nu_I$ on
$\{0,1\}^I$ is
compact\cmmnt{ (342Jd)}.   \dvro{If}{It follows that if}
$(X,\Sigma,\mu)$
is any complete probability space such that the measure algebra
$\frak B_I$ of $\nu_I$ can be embedded as a subalgebra of
the measure algebra $\frak A$ of $\mu$, there is an \imp\ function from
$X$ to $\{0,1\}^I$.   \cmmnt{For infinite $I$, this is so iff every
non-zero principal ideal of $\frak A$ has Maharam type at least
$\kappa$, by 332P.   Of
course this does not depend in any way on the results of the present
chapter.   If
$\frak B_{\kappa}$ can be embedded in $\frak A$, there must be a
stochastically independent family $\langle E_{\xi}\rangle_{\xi<\kappa}$
of sets of measure $\bover12$;  now we get a map
$h:X\to\{0,1\}^{\kappa}$ by
saying that $h(x)(\xi)=1$ iff $x\in E_{\xi}$, which by 254G is \imp.}

\spheader 343Cb In particular, if $\mu$ is atomless, there is an \imp\
function from $X$ to $\{0,1\}^{\Bbb N}$;  since this is
isomorphic\cmmnt{, as measure space,} to $[0,1]$ with Lebesgue
measure\cmmnt{ (254K)}, there is an \imp\ function from $X$ to
$[0,1]$.

\spheader 343Cc More generally, if $(X,\Sigma,\mu)$ is any complete
atomless totally finite measure space, there is an \imp\ function from
$X$ to the interval $[0,\mu X]$ endowed with Lebesgue measure.
\cmmnt{(If $\mu X>0$, apply (b) to the normalized measure
$(\mu X)^{-1}\mu$;  or argue directly from 343B, using the fact that
Lebesgue measure on $[0,\mu X]$ is compact;  or use the idea suggested
in 343Xd.)}

\spheader 343Cd\dvAnew{2008}
In the other direction, if $(X,\Sigma,\mu)$ is a compact
probability space with Maharam type at most $\kappa\ge\omega$,
then there is an
\imp\ function from $\{0,1\}^{\kappa}$ to $X$.   \prooflet{\Prf\
By 332N, there is a measure-preserving homomorphism from the measure
algebra of $\mu$ to the measure algebra of $\nu_{\kappa}$;  by 343B, this
is represented by an \imp\ function from $\{0,1\}^{\kappa}$ to $X$.\ \Qed}

\cmmnt{\spheader 343Ce Throughout the work above -- in \S254 as well
as in 343B -- I have taken the measures involved to be complete.   It
does occasionally happen, in this context, that this restriction is
inconvenient.   Typical results not depending on completeness in the
domain space $X$ are in
343Xc-343Xd.   Of course these depend not only on the very special
nature of the codomain spaces $\{0,1\}^I$ or $[0,1]$, but also on the
measures on
these spaces being taken to be incomplete.
}%end of comment

\leader{343D}{Uniqueness of \dvrocolon{realizations}}\cmmnt{ The
results of 342E-342J, %342E 342F 342G 342H 342I 342J
together with 343B, give a respectable number of contexts in
which homomorphisms
between measure algebras can be represented by functions between measure
spaces.   They say nothing about whether such functions are unique, or
whether we can distinguish, among the possible representations of a
homomorphism, any canonical one.   In fact the proof of 343B, using the
Lifting Theorem as it does, strongly suggests that this is like looking
for a canonical lifting, and I am sure that (outside a handful of very
special cases) any such search is vain.
Nevertheless, we do have a weak kind of uniqueness theorem, valid in a
useful number of spaces, as follows.

\medskip

\noindent}{\bf Definition} A measure space $(X,\Sigma,\mu)$ is
{\bf countably separated} if there is a countable set
$\Cal A\subseteq\Sigma$
separating the points of $X$ in the sense that for any distinct $x$,
$y\in X$ there is an $E\in\Cal A$ containing one but not the other.
\cmmnt{(Of course this is a property of the structure $(X,\Sigma)$
rather than of $(X,\Sigma,\mu)$.)}

\leader{343E}{Lemma} A measure space $(X,\Sigma,\mu)$ is countably
separated iff there is an injective measurable function from $X$ to
$\Bbb R$.

\proof{ If $(X,\Sigma,\mu)$ is countably separated, let $\Cal
A\subseteq\Sigma$ be a countable set separating the points of $X$.   Let
$\sequencen{E_n}$ be a sequence running over $\Cal A\cup\{\emptyset\}$.
Set

\Centerline{$f=\sum_{n=0}^{\infty}3^{-n}\chi E_n:X\to\Bbb R$.}

\noindent Then $f$ is measurable (because every $E_n$ is measurable) and
injective (because if $x\ne y$ in $X$ and
$n=\min\{i:\#(E_i\cap\{x,y\})=1\}$ and $x\in E_n$, then

\Centerline{$f(x)\ge 3^{-n}+\sum_{i<n}3^{-i}\chi E_i(x)
>\sum_{i>n}3^{-i}+\sum_{i<n}3^{-i}\chi E_i(y)\ge f(y)$.)}

On the other hand, if $f:X\to\Bbb R$ is measurable and injective, then
$\Cal A=\{f^{-1}[\,\ocint{-\infty,q}\,]:q\in\Bbb Q\}$ is a countable
subset of $\Sigma$ separating the points of $X$, so $(X,\Sigma,\mu)$ is
countably separated.
}%end of proof of 343E

\cmmnt{\medskip

\noindent{\bf Remark} The construction of the function $f$ from the
sequence $\sequencen{E_n}$ in the proof above is a standard trick;  such
$f$ are sometimes called {\bf Marczewski functionals}.
}%end of comment

\leader{343F}{Proposition} Let $(X,\Sigma,\mu)$ be a countably
separated measure space and $(Y,\Tau,\nu)$ any measure space.  Let
$f$, $g:Y\to X$ be two functions such that $f^{-1}[E]$ and
$g^{-1}[E]$ both belong to $\Tau$, and $f^{-1}[E]\symmdiff g^{-1}[E]$ is
$\nu$-negligible, for every $E\in\Sigma$.   Then
$f=g\,\,\nu$-almost everywhere, and $\{y:y\in Y,\,f(y)\ne g(y)\}$ is
measurable as well as negligible.

\proof{ Let $\Cal A\subseteq\Sigma$ be a countable set separating the
points of  $X$.  Then

\Centerline{$\{y:f(y)\ne g(y)\}=\bigcup_{E\in\Cal A}f^{-1}[E]\symmdiff
g^{-1}[E]$}

\noindent is measurable and negligible.
}%end of proof of 343F

\leader{343G}{Corollary} If, in 343B, $(X,\Sigma,\mu)$ is countably
separated, then the functions $g:Y\to X$ of 343B(v)-(vi) are almost
uniquely defined in the sense that if $f$, $g$ both represent the same
homomorphism from $\frak A$ to $\frak B$ then $f\eae g$.

\leader{343H}{Examples} Leading examples of countably separated measure
spaces are

(i) $\Bbb R$ \cmmnt{(take $\Cal A=\{\ocint{-\infty,q}:q\in\Bbb Q\}$)};

(ii) $\{0,1\}^{\Bbb N}$ \cmmnt{(take $\Cal A=\{E_n:n\in\Bbb N\}$,
where $E_n=\{x:x(n)=1\}$)};

(iii) subspaces (measurable or not) of countably separated spaces;

(iv) finite products of countably separated spaces;

(v) countable products of countably separated probability spaces;

(vi) completions and c.l.d.\ versions of countably separated spaces.

\cmmnt{As soon as we move away from these elementary ideas, however,
some interesting difficulties arise.}

\leader{343I}{Example} Let $\nu_{\frakc}$ be the usual measure on
$X=\{0,1\}^{\frakc}$\cmmnt{, where $\frak c=\#(\Bbb R)$}, and
$\Tau_{\frakc}$ its domain.   Then there is a function $f:X\to X$
such that $f(x)\ne x$ for every $x\in X$, but $E\symmdiff f^{-1}[E]$ is
negligible for every $E\in\Tau_{\frakc}$.   \prooflet{\Prf\ The
set $\frak c\setminus\omega$ is still of cardinal $\frak c$, so there is
an injection $h:\{0,1\}^{\omega}\to\frak c\setminus\omega$.   (As
usual, I am identifying the cardinal number $\frak c$ with the
corresponding initial ordinal.    But if you prefer to argue without the
full axiom of choice, you can express all the same ideas with $\Bbb R$
in the place of $\frak c$ and $\Bbb N$ in the place of $\omega$.)    For
$x\in X$, set

$$\eqalign{f(x)(\xi)&=1-x(\xi)\text{ if }\xi=h(x\restr\omega),\cr
&=x(\xi)\text{ otherwise }.\cr}$$

\noindent Evidently $f(x)\ne x$ for every $x$.   If $E\subseteq X$ is
measurable, then we can find a countable set $J\subseteq\frak c$ and
sets $E'$, $E''$, both determined by coordinates in $J$, such that
$E'\subseteq E\subseteq E''$ and $E''\setminus E'$ is negligible
(254Oc).   Now for any particular $\xi\in\frak c\setminus\omega$,
$\{x:h(x\restr\omega)=\xi\}$ is negligible, being either empty or of the
form $\{x:x(n)=z(n)$ for every $n<\omega\}$ for some
$z\in\{0,1\}^{\omega}$.   So $H=\{x:h(x\restr\omega)\in J\}$ is
negligible.   Now we see that for $x\in X\setminus H$,
$f(x)\restr J=x\restr J$, so for
$x\in X\setminus(H\cup(E''\setminus E'))$,

\Centerline{$x\in E\Longrightarrow x\in E'\Longrightarrow f(x)\in E'\Longrightarrow
f(x)\in E$,}

\Centerline{$x\notin E\Longrightarrow x\notin E''\Longrightarrow f(x)\notin
E''\Longrightarrow f(x)\notin E$.}

\noindent Thus $E\symmdiff f^{-1}[E]\subseteq H\cup(E''\setminus E')$ is
negligible.\ \Qed}

\leader{343J}{The split interval} \cmmnt{I introduce a construction
which here will
seem essentially elementary, but in other contexts is of great
interest, as will appear in Volume 4.

\medskip

}{\bf (a)} Take $I^{\|}$ to consist of two
copies of each point of the unit interval, so that
$I^{\|}=\{t^+:t\in[0,1]\}\cup\{t^-:t\in[0,1]\}$.   For
$A\subseteq I^{\|}$ write
$A_l=\{t:t^-\in A\}$, $A_r=\{t:t^+\in A\}$.   Let $\Sigma$ be the set

\Centerline{$\{E:E\subseteq I^{\|},\,E_l$ and $E_r$ are Lebesgue
measurable and $E_l\symmdiff E_r$ is Lebesgue negligible$\}$.}

\noindent For $E\in\Sigma$, set

\Centerline{$\mu E=\mu_LE_l=\mu_LE_r$}

\noindent where $\mu_L$ is Lebesgue measure on $[0,1]$.
\cmmnt{It is easy to check that} $(I^{\|},\Sigma,\mu)$ is a complete
probability space\cmmnt{ (cf.\ 234F, 234Ye)}.
Also it is compact.   \prooflet{\Prf\ Take $\Cal K$
to be the family of sets $K\subseteq I^{\|}$ such that $K_l=K_r$ is a
compact subset of $[0,1]$, and check that $\Cal K$ is a compact class
and that $\mu$
is inner regular with respect to $\Cal K$;  or use 343Xa below.\ \Qed\ }
The sets $\{t^-:t\in[0,1]\}$
and $\{t^+:t\in[0,1]\}$ are non-measurable subsets of $I^{\|}$;  on both
of them
the subspace measures correspond exactly to $\mu_L$.   We have a
canonical \imp\ function $h:I^{\|}\to[0,1]$ given by
setting $h(t^+)=h(t^-)=t$ for every $t\in[0,1]$;  $h$ induces an
isomorphism between the measure algebras of $\mu$ and $\mu_L$.

$I^{\|}$ is called the {\bf split interval}\cmmnt{ or (especially when
given its
standard topology, as in 343Yc below) the {\bf double arrow space} or
{\bf two arrows space}}.

Now\cmmnt{ the relevance to the present discussion is this:}  we have
a map $f:I^{\|}\to I^{\|}$ given by setting

\Centerline{$f(t^+)=t^-$, $f(t^-)=t^+$ for every $t\in[0,1]$}

\noindent such that $f(x)\ne x$ for every $x$, but
$E\symmdiff f^{-1}[E]$ is negligible for every $E\in\Sigma$, so that $f$
represents the identity homomorphism on the measure algebra of $\mu$.
The\cmmnt{ function $h:I^{\|}\to[0,1]$ is canonical enough, but is
two-to-one, and the} canonical map from the measure algebra of $\mu$ to
the measure
algebra of $\mu_L$ is represented equally by the functions 
$t\mapsto t^-$ and $t\mapsto t^+$, which are nowhere equal.

\spheader 343Jb Consider the direct sum
$(Y,\nu)$ of $(I^{\|},\mu)$ and $([0,1],\mu_L)$;\cmmnt{ for
definiteness,} take $Y$ to be
$(I^{\|}\times\{0\})\cup([0,1]\times\{1\})$.
Setting

\Centerline{$h_1(t^+,0)=h_1(t^-,0)=(t,1)$,\quad $h_1(t,1)=(t^+,0)$,}

\noindent\cmmnt{ we see that }$h_1:Y\to Y$ induces a measure-preserving
involution of the measure algebra $\frak B$ of $\nu$\cmmnt{, corresponding
to its expression as a simple product of the isomorphic measure algebras
of $\mu$ and $\mu_L$}.   But\cmmnt{ $h_1$ is not invertible, and
indeed} there is no invertible function from $Y$ to itself which induces
this involution of $\frak B$.   \prooflet{\Prf\Quer\ Suppose, if
possible,
that $g:Y\to Y$ were such a function.   Looking at the sets

\Centerline{$E_q=[0,q]\times\{1\}$,\quad
$F_q=\{(t^+,0):t\in[0,q]\}\cup\{(t^-,0):t\in[0,q]\}$}

\noindent  for $q\in \Bbb Q$,
we must have $g^{-1}[E_q]\symmdiff F_q$ negligible for every
$q$, so that we must have $g(t^+,0)=g(t^-,0)=(t,1)$ for almost every
$t\in[0,1]$, and $g$ cannot be injective.\ \Bang\Qed}

\spheader 343Jc Thus even with a compact probability space, and an
automorphism $\phi$ of its measure algebra, we cannot be sure of
representing $\phi$ and $\phi^{-1}$ by functions which will be inverses
of each other.

\leader{343K}{}\cmmnt{ 342L has a partial converse.

\medskip

\noindent}{\bf Proposition} If $(X,\Sigma,\mu)$ is a semi-finite
countably separated measure space, it is compact iff it is locally
compact iff it is perfect.

\proof{ We already know that compact measure spaces are locally compact
and locally compact semi-finite measure spaces are perfect (342Ha,
342L).   So suppose that $(X,\Sigma,\mu)$ is a perfect semi-finite
countably
separated measure space.   Let $f:X\to\Bbb R$ be an injective measurable
function (343E).    Consider

\Centerline{$\Cal K
=\{f^{-1}[L]:L\subseteq f[X],\,L$ is compact in $\Bbb R\}$.}

\noindent The definition of `perfect' measure space states exactly that
whenever $E\in\Sigma$ and $\mu E>0$ there is a $K\in\Cal K$ such
that $K\subseteq E$ and $\mu K>0$.   And $\Cal K$ is a compact class.
\Prf\ If $\Cal K'\subseteq\Cal K$ has
the finite intersection property, $\Cal L=\{f[K]:K\in\Cal K'\}$ is
a family of compact sets in $\Bbb R$ with the finite intersection
property, and has non-empty intersection;   so that $\bigcap\Cal K'$ is
also non-empty, because $f$ is injective.\ \QeD\   By 342E,
$(X,\Sigma,\mu)$ is compact.
}%end of proof of 343K

\leader{343L}{}\cmmnt{ The time has come to give examples of spaces
which are {\it not} locally compact, so that we can expect to have
measure-preserving homomorphisms not representable by \imp\ functions.
The most commonly arising ones are covered by the following result.

\medskip

\noindent}{\bf Proposition} Let $(X,\Sigma,\mu)$ be a complete locally
determined countably separated measure space, and $A\subseteq X$ a set
such that the subspace measure $\mu_A$ is perfect.    Then $A$ is
measurable.

\proof{\Quer\ Otherwise, there is a set $E\in\Sigma$ such that
$\mu E<\infty$ and $B=A\cap E\notin\Sigma$.   Let $f:X\to\Bbb R$ be an
injective measurable function (343E again).   Then
$f\restr B$
is $\Sigma_B$-measurable, where $\Sigma_B$ is the domain of the subspace
measure $\mu_B$ on $B$.   Set

\Centerline{$\Cal K=\{f^{-1}[L]:L\subseteq f[B],\,L$ is compact in
$\Bbb R\}$.}

\noindent Just as in the proof of 343K, $\Cal K$ is a compact class and
$\mu_B$ is inner regular with respect to $\Cal K$.
By 342Bb, there is a sequence $\sequencen{K_n}$ in $\Cal K$ such that
$\mu_B(B\setminus\bigcup_{n\in\Bbb N}K_n)=0$.   But of course
$\Cal K\subseteq\Sigma$, because $f$ is $\Sigma$-measurable, so
$\bigcup_{n\in\Bbb N}K_n\in\Sigma$.   Because $\mu$ is complete,
$B\setminus\bigcup_{n\in\Bbb N}K_n\in\Sigma$ and $B\in\Sigma$.\ \Bang
}%end of proof of 343L

\leader{343M}{Example} 343L tells us that any non-measurable set
$X$ of $\BbbR^r$, or of $\{0,1\}^{\Bbb N}$, with their usual measures,
is not perfect, therefore not (locally) compact, when given its subspace
measure.

\cmmnt{To find a non-representable homomorphism, we do not need to go
through the whole apparatus of 343B.   Take $Y$ to be a measurable
envelope of $X$ (132Ee).   Then the identity function from $X$ to $Y$
induces an isomorphism of their measure algebras.   But there is no
function from $Y$ to $X$ inducing the same isomorphism.
\prooflet{\Prf\Quer\ Writing $Z$ for $\BbbR^r$ or $\{0,1\}^{\Bbb N}$
and $\mu$ for its measure, $Z$ is countably separated;  suppose
$\sequencen{E_n}$ is a sequence of measurable sets in $Z$ separating its
points.   For each $n$, $(Y\cap E_n)^{\ssbullet}$ in the measure algebra
of $\mu_Y$ corresponds to $(X\cap E_n)^{\ssbullet}$ in the measure
algebra of $\mu_X$.   So if $f:Y\to X$ were a function representing the
isomorphism of the measure algebras, $(Y\cap E_n)\symmdiff f^{-1}[E_n]$
would have to be negligible for each $n$, and
$A=\bigcup_{n\in\Bbb N}(Y\cap E_n)\symmdiff f^{-1}[E_n]$ would be
negligible.   But for $y\in Y\setminus A$, $f(y)$ belongs to just the
same $E_n$ as $y$ does, so must be equal to $y$.   Accordingly
$X\supseteq Y\setminus A$ and $X$ is measurable.\ \Bang\Qed}
}%end of comment

\exercises{
\leader{343X}{Basic exercises (a)}
%\spheader 343Xa
Let $(X,\Sigma,\mu)$ be a
semi-finite measure space.   (i) Suppose that there is a set
$A\subseteq X$, of full outer measure, such that the
subspace measure on $A$ is compact.   Show that $\mu$ is locally
compact.   \Hint{show that $\mu$ satisfies (ii) or (v) of 343B.}  (ii)
Suppose that for every non-negligible $E\in\Sigma$ there is a
non-negligible set $A\subseteq E$ such that
the subspace measure on $A$ is compact.   Show that $\mu$ is locally
compact.
%343B

\spheader 343Xb Let $\langle X_i\rangle_{i\in I}$ be a
family of non-empty sets, with product $X$;  write $\pi_i:X\to X_i$ for
the coordinate map.   Suppose we are given a $\sigma$-algebra
$\Sigma_i$ of subsets of $X_i$ for each $i$;  let
$\Sigma=\Tensorhat_{i\in I}\Sigma_i$ be the corresponding
$\sigma$-algebra of subsets of $X$
generated by $\{\pi_i^{-1}[E]:i\in I,\,E\in\Sigma_i\}$.
Let $\mu$ be a totally finite measure with domain $\Sigma$, and for
$i\in I$ let $\mu_i$ be the image measure $\mu\pi_i^{-1}$.   Check that
the domain of $\mu_i$ is $\Sigma_i$.   Show that if
every $(X_i,\Sigma_i,\mu_i)$ is compact,
then so is $(X,\Sigma,\mu)$.   \Hint{{\it either} show that $\mu$
satisfies (v) of 343B {\it or} adapt the method of 342Gf.}
%343B

\spheader 343Xc Let  $I$ be any set.   Let $\CalBa$ be the
$\sigma$-algebra of
subsets of $\{0,1\}^I$ generated by the sets $F_i=\{z:z(i)=1\}$ for
$i\in I$, and $\nu$ any probability measure with domain $\CalBa$;  let
$\frak B$ be
the measure algebra of $\nu$.   Let $(X,\Sigma,\mu)$ be a measure space
with measure algebra $\frak A$, and $\phi:\frak B\to\frak A$ an
order-continuous
Boolean homomorphism.   Show that there is an \imp\ function
$f:X\to\{0,1\}^I$ representing $\phi$.   \Hint{for each $i\in I$, take
$E_i\in\Sigma$ such that $E_i^{\ssbullet}=\phi F_i^{\ssbullet}$;  set
$f(x)(i)=1$ if $x\in E_i$, and use 343Ab.}
%343C

\spheader 343Xd Let $(X,\Sigma,\mu)$ be an atomless probability space.
Let $\mu_{\Cal B}$ be the restriction of Lebesgue measure to the
$\sigma$-algebra of Borel subsets of $[0,1]$.   Show that there is a
function $g:X\to[0,1]$ which is \imp\ for $\mu$ and $\mu_{\Cal B}$.
\Hint{find an $f:X\to\{0,1\}^{\Bbb N}$ as in 343Xc, and set $g=hf$ where
$h(z)=\sum_{n=0}^{\infty}2^{-n-1}g(n)$, as in 254K;  or choose
$E_q\in\Sigma$
such that $\mu E_q=q$, $E_q\subseteq E_{q'}$ whenever $q\le q'$ in
$[0,1]\cap\Bbb Q$, and set $f(x)=\inf\{q:x\in E_q\}$ for $x\in E_1$.}
%343D

\spheader 343Xe Let $(X,\Sigma,\mu)$ be a countably separated measure
space, with measure algebra $\frak A$.   (i) Show that $\{x\}\in\Sigma$
for every $x\in X$.   (ii) Show that every atom of $\frak A$ is of the
form $\{x\}^{\ssbullet}$ for some $x\in X$.
%343D

\spheader 343Xf Let $(X,\Sigma,\mu)$ be a semi-finite countably separated
measure space.
(i) Show that $\mu$ is point-supported iff it is complete,
strictly localizable
and purely atomic.   (ii) Show that $\mu$ is atomless iff $\mu\{x\}=0$
for every $x\in X$.
%343D

\spheader 343Xg Let $I^{\|}$ be the split interval, with its usual
measure
$\mu$ described in 343J, and $h:I^{\|}\to[0,1]$ the canonical
surjection.   Show that the canonical isomorphism between the measure
algebras of $\mu$ and Lebesgue measure on $[0,1]$ is given by the
formula `$E^{\ssbullet}\mapsto h[E]^{\ssbullet}$ for every measurable
$E\subseteq I^{\|}$'.
%343J

\spheader 343Xh Let $(X,\Sigma,\mu)$ and $(Y,\Tau,\nu)$ be measure
spaces with measure algebras $(\frak A,\bar\mu)$, $(\frak B,\bar\nu)$.
Suppose that $X\cap Y=\emptyset$ and that we have a measure-preserving
isomorphism $\pi:\frak A\to\frak B$.   Set

\Centerline{$\Lambda=\{W:W\subseteq X\cup Y,\,W\cap X\in\Sigma,\,W\cap
Y\in\Tau,\,\pi(W\cap X)^{\ssbullet}=(W\cap Y)^{\ssbullet}\}$,}

\noindent and for $W\in\Lambda$ set $\lambda W=\mu(W\cap X)=\nu(W\cap
Y)$.   Show that $(X\cup Y,\Lambda,\lambda)$ is a measure space which is
locally compact, or perfect, if $(X,\Sigma,\mu)$ is.
%343J

\sqheader 343Xi Let $(X,\Sigma,\mu)$ be a complete perfect totally
finite measure space, $(Y,\Tau,\nu)$ a complete countably separated
measure space, and $f:X\to Y$ an inverse-measure-preserving function.
Show that $\Tau=\{F:F\subseteq Y,\,f^{-1}[F]\in\Sigma\}$, so that a
function $h:Y\to\Bbb R$ is $\nu$-integrable iff $hf$ is
$\mu$-integrable.
\Hint{if $A\subseteq Y$ and $E=f^{-1}[A]\in\Sigma$, $f\restr E$ is \imp\
for the subspace measures $\mu_E$, $\nu_A$;  by 342Xk, $\nu_A$ is
perfect, so by 343L $A\in\Tau$.   Now use 235J.}
%343L

\spheader 343Xj Let $(X,\Sigma,\mu)$ be a complete compact measure space,
$Y$ a set and $f:Y\to X$ a surjection;  set

\Centerline{$\Tau=\{F:F\subseteq Y$, $f[F]\in\Sigma$,
$\mu(f[F]\cap f[Y\setminus F])=0\}$,
\quad$\nu F=\mu f[F]$ for $F\in\Tau$,}

\noindent so that $\nu$ is a measure on $Y$ and $f$ is \imp\
(234Ye).   Show that $\nu$ is a compact measure.
%343J

\leader{343Y}{Further exercises (a)}
%\spheader 343Ya
Let $(X,\Sigma,\mu)$ be a semi-finite measure space, and suppose that
there is a compact class $\Cal K\subseteq\Cal PX$ such that ($\alpha$)
whenever $E\in\Sigma$ and $\mu E>0$ there is a non-negligible
$K\in\Cal K$ such that
$K\subseteq E$ ($\beta$) whenever $K_0,\ldots,K_n\in\Cal K$ and
$\bigcap_{i\le n}K_i=\emptyset$ then there are measurable sets
$E_0,\ldots,E_n$ such that $E_i\supseteq K_i$ for every $i$ and
$\bigcap_{i\le n}E_i$ is negligible.   Show that $\mu$ is locally
compact.
%343B

\spheader 343Yb(i)\dvArevised{2010}
Show that a countably separated semi-finite measure
space has magnitude and Maharam type at most
$2^{\frak c}$.   (ii) Show that the direct sum of $\frak c$ or fewer
countably separated measure spaces is countably separated.
(iii)\dvAnew{2011} Show that a countably separated perfect measure space
has countable Maharam type.
%343E

\spheader 343Yc Let
$I^{\|}=\{t^+:t\in[0,1]\}\cup\{t^-:t\in[0,1]\}$ be the split interval
(343J).   (i) Show that the rules

\Centerline{$s^-\le t^-\iff s^+\le t^+\iff s\le t$,
\quad $s^+\le t^-\iff s<t$,}

\Centerline{$t^-\le t^+$ for all $t\in[0,1]$}

\noindent define a Dedekind complete total order on $I^{\|}$ with
greatest and least elements.   (ii) Show that the intervals $[0^-,t^-]$,
$[t^+,1^+]$,
interpreted for this ordering, generate a compact Hausdorff topology on
$I^{\|}$ for which the map $h:I^{\|}\to[0,1]$ of 343J is continuous.
(iii) Show
that a subset $E$ of $I^{\|}$ is Borel for this topology iff the sets
$E_r$, $E_l\subseteq[0,1]$, as described in 343Ja, are Borel and
$E_r\symmdiff E_l$ is countable.    (iv) Show that if $f:[0,1]\to\Bbb R$
is of bounded
variation then there is a continuous $g:I^{\|}\to\Bbb R$ such that
$g=fh$ except perhaps at countably many points.
(v) Show that the measure $\mu$ of 343J is inner regular with respect to
the compact subsets of $I^{\|}$.   (vi) Show that we have a lower
density $\phi$ for $\mu$ defined by setting

$$\eqalign{\phi E=\{t^-:\,&0<t\le 1,\,\lim_{\delta\downarrow 0}
  \Bover1{\delta}\mu(E\cap[(t-\delta)^+,t^-])=1\}\cr
&\cup\{t^+:0\le t<1,\,\lim_{\delta\downarrow 0}
  \Bover1{\delta}\mu(E\cap[t^+,(t+\delta)^-])=1\}\cr}$$

\noindent for measurable sets $E\subseteq I^{\|}$.
%343H

\spheader 343Yd Set $X=\{0,1\}^{\frak c}$, with its usual measure
$\nu_{\frakc}$.   Show that there is an inverse-measure-preserving function
$f:X\to X$ such that $f[X]$ is non-measurable but $f$ induces the
identity automorphism of the measure algebra of $\nu_{\frakc}$.
\Hint{use the
idea of 343I.}   Show that under these conditions $f[X]$, with its
subspace measure, must be compact.   \Hint{use 343B(iv).}
%343H

\spheader 343Ye Let $\mu_{Hr}$ be $r$-dimensional Hausdorff measure on
$\BbbR^s$, where $s\ge 1$ is an integer and $r\ge 0$ (\S264).   (i)
Show that $\mu_{Hr}$ is countably separated.   (ii) Show that the
c.l.d.\ version of $\mu_{Hr}$ is compact.   \Hint{264Yi.}
%343K

\spheader 343Yf Give an example of a countably separated probability
space $(X,\Sigma,\mu)$ and a function $f$ from $X$ to a set $Y$ such
that the image measure $\mu f^{-1}$ is not countably separated.
\Hint{use 223B to show that if $E\subseteq\Bbb R$ is Lebesgue measurable
and not negligible, then $E+\Bbb Q$ is conegligible;  or use the
zero-one law to show that if $E\subseteq\Cal P\Bbb N$ is measurable and
not negligible for the usual measure on $\Cal P\Bbb N$, then
$\{a\symmdiff b:a\in E,\,b\in[\Bbb N]^{<\omega}\}$ is conegligible.}
%343M

}%end of exercises

\cmmnt{\Notesheader{343} The points at which the Lifting Theorem
impinges on the work of this section are in the proofs of
(iv)$\Rightarrow$(i) and (iv)$\Rightarrow$(v) in Theorem 343B.
In fact the ideas can be rearranged to give a proof of 343B which does
not rely on the Lifting Theorem;  I give a hint in Volume 4 (413Yc).

I suppose the significant new ideas of this section are in 343B and
343K.   The rest is mostly a matter of being thorough and careful.
But I take this material at a slow pace because there are some
potentially confusing features, and the underlying question is of the
greatest importance:  when, given a Boolean homomorphism from one
measure algebra to another, can we be sure of representing it by a
measurable function between measure spaces?   The concept of `compact'
measure puts the burden firmly on the measure space corresponding to the
{\it domain} of the Boolean homomorphism, which will be the {\it
codomain} of the measurable function.   So the first step is to try to
understand properly which measures are compact, and what other properties
they can be expected to have;  which accounts for much of the length of
\S342.   But having understood that many of our favourite measures
are compact, we have to come to terms with the fact that we still cannot
count on a measure algebra isomorphism corresponding to a measure space
isomorphism.   I introduce the split interval (343J, 343Xg,
343Yc) as a close
approximation to Lebesgue measure on $[0,1]$ which is not isomorphic to
it.   Of course we have already seen a more dramatic example:  the Stone
space of the Lebesgue measure algebra also has the same measure algebra
as Lebesgue measure, while being in almost every other way very much
more complex, as will appear in Volumes 4 and 5.

As 343C suggests, elementary cases in which 343B can be applied are
often amenable to more primitive methods, avoiding not only the concept
of `compact' measure, but also Stone spaces and the Lifting Theorem.
For substantial examples in which we can prove that a measure space
$(X,\mu)$ is
compact, without simultaneously finding direct constructions for \imp\
functions into $X$ (as in 343Xc-343Xd), I think we shall have to wait
until Volume 4.

The concept of `countably separated' measure space does not involve the
measure at all, nor even the null ideal;  it belongs to
the theory of $\sigma$-algebras of sets.   Some simple
permanence properties are in 343H and 343Yb(ii).
Let us note in passing that 343Xi describes some more situations in
which the `image measure catastrophe', described
in 235H, cannot arise.

I include the variants 343B(ii), 343B(iii) and 343Ya of the notion of
`local compactness' because they are not obvious and may illuminate it.
}%end of notes

\discrpage
