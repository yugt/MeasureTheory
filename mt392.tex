\frfilename{mt392.tex}
\versiondate{11.2.08}
\copyrightdate{2008}

\def\chaptername{Measurable algebras}
\def\sectionname{Submeasures}

\newsection{392}

In \S391 I looked at what we can deduce if a Boolean algebra carries a
strictly positive finitely additive functional.   There are important
contexts in which we find ourselves with subadditive, rather than
additive, functionals, and these are what I wish to investigate here.
It turns out that, once we have found the right hypotheses, such
functionals can also provide a criterion for measurability of an algebra
(392G below).   The argument runs through a new idea, using a result in
finite combinatorics (392D).

At the end of the section I include notes on metrics associated with
submeasures (392H) and on products of submeasures (392K).

\leader{392A}{Definition}\dvArevised{2007}
Let $\frak A$ be a Boolean algebra.   A
{\bf submeasure} on $\frak A$ is a functional
$\nu:\frak A\to[0,\infty]$ such that

\inset{$\nu 0 = 0$,

$\nu a\le\nu b$ whenever $a\Bsubseteq b$,

$\nu(a\Bcup b)\le\nu a+\nu b$ for all $a$, $b\in\frak A$.}

\leader{392B}{}\cmmnt{ The following list mostly repeats
ideas we have already used in the context of measures;  but (b) and (c)
are new, and will be the basis of this section.

\medskip

\noindent}{\bf Definitions} Let $\frak A$ be a Boolean algebra and
$\nu:\frak A\to[0,\infty]$ a submeasure.

(a) $\nu$ is {\bf strictly positive} if $\nu a>0$ for every $a\ne 0$.

(b) $\nu$ is {\bf exhaustive} if $\lim_{n\to\infty}\nu a_n=0$ for every
disjoint sequence $\sequencen{a_n}$ in $\frak A$.

(c) $\nu$ is {\bf uniformly exhaustive} if for every $\epsilon>0$ there
is an $n\in\Bbb N$ such that there is no disjoint family
$a_0,\ldots,a_n$ with $\nu a_i\ge\epsilon$ for every $i\le n$.

(d) $\nu$ is {\bf totally finite} if $\nu 1<\infty$.

(e) $\nu$ is {\bf unital} if $\nu 1=1$.

(f) $\nu$ is {\bf atomless} if whenever $a\in\frak A$ and $\nu a>0$ there
is a $b\Bsubseteq a$ such that $\nu b>0$ and $\nu(a\Bsetminus b)>0$.
%G\lowcy\'nski 08 differs

(g)\dvAformerly{393D} If $\nuprime$ is another submeasure on $\frak A$, then $\nuprime$ is
{\bf absolutely continuous} with respect to $\nu$ if for every
$\epsilon>0$ there is a
$\delta>0$ such that $\nuprime a\le\epsilon$ whenever $\nu a\le\delta$.

\leader{392C}{Proposition} Let $\frak A$ be a Boolean algebra.

(a) If there is an exhaustive strictly positive submeasure on $\frak A$,
then $\frak A$ is ccc.

(b) A uniformly exhaustive submeasure on $\frak A$ is exhaustive.

(c) Any non-negative additive functional on $\frak A$ is a uniformly
exhaustive submeasure.

\proof{ These are all elementary.   If $\nu:\frak A\to[0,\infty]$
is an exhaustive strictly positive submeasure, and
$\langle a_i\rangle_{i\in I}$ is a disjoint family in
$\frak A\setminus\{0\}$,
then $\{i:\nu a_i\ge 2^{-n}\}$ must be finite for each $n$, so $I$ is
countable.   (Cf.\ 322G.)   If $\nu:\frak A\to[0,\infty]$ is a
uniformly exhaustive submeasure and $\sequencen{a_n}$ is disjoint in
$\frak A$, then $\{i:\nu a_i\ge 2^{-n}\}$ is finite for each $n$, so
$\lim_{i\to\infty}\nu a_i=0$.   If $\nu:\frak A\to\coint{0,\infty}$ is a
non-negative additive functional, it is a submeasure, by
326Ba and 326Bf.   If $\epsilon>0$, then take
$n\ge\bover1{\epsilon}\nu 1$;  if $a_0,\ldots,a_n$ are disjoint, then
$\sum_{i=0}^n\nu a_i\le\nu 1$, so $\min_{i\le n}\nu a_i<\epsilon$.
}%end of proof of 392C

\leader{392D}{Lemma} Suppose that $k$, $l$, $m\in\Bbb N$ are such that
$3\le k\le l\le m$ and $18mk\le l^2$.   Let $L$, $M$ be sets of sizes
$l$, $m$
respectively.   Then there is a set $R\subseteq M\times L$ such that (i)
each vertical section of $R$ has just three members (ii)
$\#(R[E])\ge\#(E)$ whenever $E\in[M]^{\le k}$;  so that for every
$E\in[M]^{\le k}$ there is an injective function $f:E\to L$ such that
$(x,f(x))\in R$ for every $x\in E$.

\cmmnt{\medskip

\noindent{\bf Recall} that $[M]^{\le k}=\{I:I\subseteq M,\,\#(I)\le k\}$,
$[M]^k=\{I:I\subseteq M,\,\#(I)=k\}$
(3A1J).}%end of comment

\proof{{\bf (a)} We need to know that $n!\ge 3^{-n}n^n$ for every
$n\in\Bbb N$;  this is immediate from the inequality

\Centerline{$\sum_{i=2}^n\ln i\ge\int_1^n\ln x\,dx=n\ln n-n+1$ for every
$n\ge 2$.}

\wheader{392D}{4}{2}{2}{36pt}

{\bf (b)} Let $\Omega$ be the set of those $R\subseteq M\times L$ such
that each vertical section of $R$ has just three members, so that

\Centerline{$\#(\Omega)=\#([L]^3)^m
=\bigl(\Bover{l!}{3!(l-3)!}\bigr)^m$.}

\noindent Let us regard $\Omega$ as a probability space with the
uniform probability.

If $F\in[L]^n$, where $3\le n\le k$, and $x\in M$, then

$$\eqalignno{\Pr(R[\{x\}]\subseteq F)
&=\Bover{\#([F]^3)}{\#([L]^3)}\cr
\noalign{\noindent (because $R[\{x\}]$ is a random member of $[L]^3$)}
&=\Bover{n(n-1)(n-2)}{l(l-1)(l-2)}\le\Bover{n^3}{l^3}\cr}$$

\noindent as $n<l$.   So if $E\in[M]^{n}$ and $F\in[L]^n$, then

$$\eqalignno{\Pr(R[E]\subseteq F)
&=\prod_{x\in E}\Pr(R[\{x\}]\subseteq F)\cr
\noalign{\noindent (because the sets $R[\{x\}]$ are chosen
independently)}
&\le\Bover{n^{3n}}{l^{3n}}.\cr}$$

Accordingly

$$\eqalignno{\Pr(&\text{there is an }E\subseteq M
  \text{ such that }\#(R[E])<\#(E)\le k)\cr
&\le\Pr(\text{there is a non-empty }E\subseteq M
  \text{ such that }\#(R[E])\le\#(E)\le k)\cr
&=\Pr(\text{there is an }E\subseteq M
  \text{ such that }3\le\#(R[E])\le\#(E)\le k)\cr
\noalign{\noindent (because if $E\ne\emptyset$ then $\#(R[E])\ge 3$)}
&\le\sum_{n=3}^{k}\sum_{E\in[M]^{n}}\sum_{F\in[L]^n}
  \Pr(R[E]\subseteq F)
\le\sum_{n=3}^{k}\#([M]^{n})\#([L]^n)\Bover{n^{3n}}{l^{3n}}\cr
&=\sum_{n=3}^{k}\Bover{m!}{n!(m-n)!}\Bover{l!}{n!(l-n)!}
  \Bover{n^{3n}}{l^{3n}}
\le\sum_{n=3}^{k}\Bover{m^{n}l^nn^{3n}}{n!n!l^{3n}}
\le\sum_{n=3}^k\Bover{m^nn^n3^{2n}}{l^{2n}}\cr
\noalign{\noindent (using (a))}
&=\sum_{n=3}^k\bigl(\Bover{9mn}{l^2}\bigr)^n
\le\sum_{n=3}^k\Bover1{2^n}
<1.\cr}$$

\noindent There must therefore be some $R\in\Omega$ such that
$\#(R[E])\ge\#(E)$ whenever $E\subseteq M$ and $\#(E)\le k$.

\medskip

{\bf (c)} If now $E\in [M]^{\le k}$,
the restriction $R_E=R\cap(E\times L)$ has the property that
$\#(R_E[I])\ge\#(I)$ for every $I\subseteq E$.   By Hall's Marriage
Lemma (3A1K) there is an injective function $f:E\to L$ such that
$(x,f(x))\in R_E\subseteq R$ for every $x\in E$.
}%end of proof of 392D

\cmmnt{\medskip

\noindent{\bf Remark} Of course this argument can be widely generalized;
see references in {\smc Kalton \& Roberts 83}.
}%end of comment

\leader{392E}{Lemma} Let $\frak A$ be a Boolean algebra and
$\nu:\frak A\to[0,\infty]$ a uniformly exhaustive submeasure.
Then for any
$\epsilon\in\ocint{0,\nu 1}$ the set $A=\{a:\nu a\ge\epsilon\}$ has
intersection number greater than $0$.

\proof{{\bf (a)} To begin with (down to the end of (d) below), suppose that
$\nu 1=1$.
Because $\nu$ is uniformly exhaustive, there is an $r\ge 1$ such
that whenever $\langle c_i\rangle_{i\in I}$ is a disjoint family in
$\frak A$ then $\#(\{i:\nu c_i>\bover15\epsilon\})\le r$, so that
$\sum_{i\in I}\nu c_i\le r+\bover15\epsilon\#(I)$.   Set
$\delta=\epsilon/5r$, $\eta=\bover1{74}\delta^2$, so that

\Centerline{$\delta-\eta\ge\bover1{18}(\delta-\eta)^2
\ge\bover1{18}(\delta^2-2\eta)=4\eta$.}

\medskip

{\bf (b)} Let $\familyiI{a_i}$ be a non-empty finite family in $A$.
Let $m$ be any multiple of $\#(I)$ greater than or equal to $1/\eta$.
Then there are integers $k$, $l$ such that

\Centerline{$3\eta\le\Bover{k}{m}\le 4\eta
\le\Bover1{18}(\delta-\eta)^2$,
\quad$\delta-\eta\le\Bover{l}{m}\le\delta$,}

\noindent in which case

\Centerline{$3\le k\le l\le m$,
\quad$18mk\le m^2(\delta-\eta)^2\le l^2$.}

\medskip

{\bf (c)} Take a set $M$ of the form $I\times S$ where $\#(S)=m/\#(I)$,
so that $\#(M)=m$.   For $x=(i,s)\in M$ set $d_x=a_i$.   Let $L$ be a
set with $l$ members.   By 392D, there is a set $R\subseteq M\times L$
such that every vertical section of $R$ has just three members and
whenever $E\in[M]^{\le k}$ there is an injective function $f_E:E\to L$
such that $(x,f_E(x))\in R$ for every $x\in E$.

For $E\subseteq M$ set

\Centerline{$b_E
=\inf_{x\in E}d_x\Bsetminus\sup_{x\in M\setminus E}d_x$,}

\noindent so that $\langle b_E\rangle_{E\subseteq M}$ is a partition of
unity in $\frak A$.   For $x\in M$ and $j\in L$ set

\Centerline{$c_{xj}=\sup\{b_E:x\in E\in[M]^{\le k},\,f_E(x)=j\}$.}

\noindent If $x$, $y$ are distinct members of $M$ and $j\in L$ then

\Centerline{$c_{xj}\Bcap c_{yj}
=\sup\{b_E:x$, $y\in E\in[M]^{\le k},\,f_E(x)=f_E(y)=j\}=0$,}

\noindent because every $f_E$ is injective.   Set

\Centerline{$m_j=\#(\{x:x\in M,\,c_{xj}\ne 0\})$}

\noindent for each $j\in L$.   Note that $c_{xj}=0$ if $(x,j)\notin R$,
so $\sum_{j\in L}m_j\le\#(R)=3m$.

We have

\Centerline{$\sum_{x\in M}\nu c_{xj}\le r+\Bover15\epsilon m_j$}

\noindent for each $j$, by the choice of $r$;  so

$$\eqalign{\sum_{x\in M,j\in L}\nu c_{xj}
&\le rl+\Bover15\epsilon\sum_{j\in L}m_j
\le rl+\Bover35m\epsilon\cr
&\le (r\delta+\Bover35\epsilon)m
=\Bover45\epsilon m
<\epsilon m\cr}$$

\noindent by the choice of $l$ and $\delta$.   There must therefore be
some $x\in M$ such that

\Centerline{$\nu(\sup_{j\in L}c_{xj})
\le\sum_{j\in L}\nu c_{xj}
<\epsilon\le\nu d_x$,}

\noindent and $d_x$ cannot be included in

\Centerline{$\sup_{j\in L}c_{xj}=\sup\{b_E:x\in E\in[M]^{\le k}\}$.}

\noindent But as $\sup\{b_E:x\in E\subseteq M\}$ is just $d_x$, there
must be an $E\subseteq M$, of cardinal greater than $k$, such that
$b_E\ne 0$.

Recall now that $M=I\times S$, and that

\Centerline{$k\ge 3\eta m=3\eta\#(I)\#(S)$.}

\noindent The set
$J=\{i:\,\Exists s$, $(i,s)\in E\}$ must therefore have more than
$3\eta\#(I)$ members,
since $E\subseteq J\times S$.   But also $d_{(i,s)}=a_i$ for each
$(i,s)\in E$, so that $\inf_{i\in J}a_i\Bsupseteq b_E\ne 0$.

\medskip

{\bf (d)} As $\familyiI{a_i}$ is arbitrary, the intersection number of
$A$ is at least $3\eta>0$.

\medskip

{\bf (e)} This completes the proof in the case in which $\nu 1=1$.   If
$\nu 1=0$ the result is vacuous.   If $\nu 1>0$, set
$\nu'a=\Bover{\min(\nu a,1)}{\min(\nu 1,1)}$ for each $a$;  then
it is easy to check that $\nu'$ is a uniformly
exhaustive submeasure with $\nu'1=1$, and

\Centerline{$\{a:\nu a\ge\epsilon\}
\subseteq\{a:\nu'a\ge\Bover{\min(\epsilon,1)}{\min(\nu 1,1)}\}$}

\noindent has non-zero intersection number for any
$\epsilon\in\ocint{0,\nu 1}$.
So the result is true in the generality stated.
}%end of proof of 392E

\leader{392F}{Theorem} Let $\frak A$ be a Boolean algebra with a
strictly positive uniformly exhaustive submeasure.   Then $\frak A$
is chargeable\cmmnt{, that is, has
a strictly positive finitely additive functional}.

\proof{ If $\frak A=\{0\}$ this is trivial.   Otherwise, let
$\nu:\frak A\to[0,\infty]$ be a strictly positive uniformly
exhaustive
submeasure.   For each $n$, $A_n=\{a:\nu a\ge\min(2^{-n},\nu 1)\}$ has
intersection number greater than $0$, and
$\bigcup_{n\in\Bbb N}A_n=\frak A\setminus\{0\}$ because $\nu$ is
strictly positive;  so $\frak A$ has a
strictly positive finitely additive functional, by Kelley's theorem
(391J).
}%end of proof of 392F

\leader{392G}{Corollary} Let $\frak A$ be a Boolean algebra.   Then it is
measurable iff it is \wsid\ and Dedekind $\sigma$-complete and has a
strictly positive uniformly exhaustive submeasure.

\proof{ Put 391D and 392F together.
}%end of proof of 392G

\leader{392H}{}\dvAformerly{3{}93B}\cmmnt{ This completes the main work of
this section.
However it will be convenient later to have some more facts available
which belong to the same group of ideas.

\medskip

\noindent}{\bf Metrics from submeasures:  Proposition} Let
$\frak A$ be a Boolean algebra and $\nu$ a
strictly positive totally finite submeasure on $\frak A$.

(a) We have a metric $\rho$ on $\frak A$ defined by the formula

\Centerline{$\rho(a,b)=\nu(a\Bsymmdiff b)$}

\noindent for all $a$, $b\in\frak A$.

(b) The Boolean operations $\Bcup$, $\Bcap$, $\Bsymmdiff$, $\Bsetminus$
and the function $\nu:\frak A\to\Bbb R$ are all uniformly continuous
for $\rho$.

(c) The metric space completion $(\widehat{\frak A},\hat\rho)$ of
$(\frak A,\rho)$ is a Boolean algebra under the natural continuous
extensions of the Boolean operations, and $\nu$ has a unique continuous
extension $\hat\nu$ to $\widehat{\frak A}$ which is again a strictly
positive submeasure.

(d) If $\nu$ is additive, then $(\widehat{\frak A},\hat\nu)$ is a
totally finite measure algebra.

\proof{{\bf (a)-(b)} This is just a generalization of 323A-323B;
essentially the same formulae can be used.   For the triangle inequality
for $\rho$, we have
$a\Bsymmdiff c\Bsubseteq(a\Bsymmdiff b)\Bcup(b\Bsymmdiff c)$, so

\Centerline{$\rho(a,c)
=\nu(a\Bsymmdiff c)
\le\nu(a\Bsymmdiff b)+\nu(b\Bsymmdiff c)
=\rho(a,b)+\rho(b,c)$.}

\noindent For the uniform continuity of the Boolean operations, we have

\Centerline{$(b*c)\Bsymmdiff(b'*c')
\Bsubseteq (b\Bsymmdiff b')\cup(c\Bsymmdiff c')$}

\noindent so that

\Centerline{$\rho(b*c,b'*c')\le\rho(b,b')+\rho(c,c')$}

\noindent for each of the operations $*=\Bcup$, $\Bcap$, $\Bsetminus$
and $\Bsymmdiff$ and all $b$, $c$, $b'$, $c'\in\frak A$.
For the uniform continuity of the function $\nu$ itself, we have

\Centerline{$\nu b\le\nu c+\nu(b\Bsetminus c)\le\nu c+\rho(b,c)$,}

\noindent so that $|\nu b-\nu c|\le\rho(b,c)$.

\medskip

{\bf (c)} $\frak A\times\frak A$ is a dense subset of
$\widehat{\frak A}\times\widehat{\frak A}$, so the Boolean operations on
$\frak A$,
regarded as uniformly continuous functions from $\frak A\times\frak A$
to $\frak A\subseteq\widehat{\frak A}$, have unique extensions to
continuous binary operations on $\widehat{\frak A}$ (3A4G).   If we look at

\Centerline{$A=\{(a,b,c):a\Bsymmdiff(b\Bsymmdiff c)
=(a\Bsymmdiff b)\Bsymmdiff c\}$,}

\noindent this is a closed subset of
$\widehat{\frak A}\times\widehat{\frak A}\times\widehat{\frak A}$,
because the maps
$(a,b,c)\mapsto a\Bsymmdiff(b\Bsymmdiff c)$,
$(a,b,c)\mapsto(a\Bsymmdiff b)\Bsymmdiff c$ are continuous and the topology of $\widehat{\frak A}$
is Hausdorff;  since $A$ includes the dense set
$\frak A\times\frak A\times\frak A$, it is the whole of
$\widehat{\frak A}\times\widehat{\frak A}\times\widehat{\frak A}$, that is,
$a\Bsymmdiff(b\Bsymmdiff c)=(a\Bsymmdiff b)\Bsymmdiff c$ for all $a$,
$b$, $c\in\widehat{\frak A}$.   All the other identities we need to show
that $\widehat{\frak A}$ is a Boolean algebra can be confirmed by the
same method.   Of course $\frak A$ is now a subalgebra of
$\widehat{\frak A}$.

Because $\nu:\frak A\to\coint{0,\infty}$ is uniformly continuous, it has
a unique continuous extension
$\hat\nu:\widehat{\frak A}\to\coint{0,\infty}$.   We have

\Centerline{$\hat\nu 0=0$,
\quad$\hat\nu a\le\hat\nu(a\Bcup b)\le\hat\nu a+\hat\nu b$,
\quad$\hat\nu a=\hat\rho(a,0)$}

\noindent for every $a$, $b\in\frak A$ and therefore for every
$a$, $b\in\widehat{\frak A}$, so $\hat\nu$ is a submeasure on
$\widehat{\frak A}$, and

\Centerline{$\hat\nu a=0\Longrightarrow\hat\rho(a,0)=0
\Longrightarrow a=0$,}

\noindent so $\hat\nu$ is strictly positive.

\medskip

{\bf (d)} We have $\nu(a\Bcup b)+\nu(a\Bcap b)=\nu a+\nu b$ for all $a$,
$b\in\frak A$;  because all the operations are continuous,
$\hat\nu(a\Bcup b)+\hat\nu(a\Bcap b)=\hat\nu a+\hat\nu b$ for all $a$,
$b\in\widehat{\frak A}$.   In particular, since $\hat\nu 0=0$,
$\hat\nu$ is additive.
Next, if $\sequencen{a_n}$ is a non-decreasing sequence in
$\widehat{\frak A}$,
$\hat\rho(a_m\Bsymmdiff a_n)=|\hat\nu a_m-\hat\nu a_n|$ for all $m$, $n\in\Bbb N$,
and $\sequencen{a_n}$ is $\hat\rho$-Cauchy, therefore convergent to some
$a\in\widehat{\frak A}$.
Since

\Centerline{$a\Bcap a_n=\lim_{m\to\infty}a_m\Bcap a_n=a_n$}

\noindent for each $n$,
$a\Bsupseteq a_n$ for every $n$.   If $b\in\widehat{\frak A}$ is any upper
bound for $\{a_n:n\in\Bbb N\}$, then

\Centerline{$b\Bcap a=\lim_{n\to\infty}b\Bcap a_n
=\lim_{n\to\infty}a_n=a$}

\noindent and $b\Bsupseteq a$;  thus $a$ is the least upper bound
of $\{a_n:n\in\Bbb N\}$.

So, first, if $\sequencen{b_n}$ is any sequence in $\widehat{\frak A}$, and
we set $a_n=\sup_{i\le n}b_i$ for each $n$, $\sup_{n\in\Bbb N}a_n$ is
defined and must be equal to $\sup_{n\in\Bbb N}b_n$;  accordingly $\widehat{\frak A}$
is Dedekind $\sigma$-complete.   Next, if $\sequencen{b_n}$ is a disjoint
sequence in $\widehat{\frak A}$, and again we set $a_n=\sup_{i\le n}b_i$ for each $n$,
$a=\sup_{n\in\Bbb N}a_n=\sup_{n\in\Bbb N}b_n$, we shall have

\Centerline{$\hat\nu a
=\lim_{n\to\infty}\hat\nu a_n
=\lim_{n\to\infty}\sum_{i=0}^n\hat\nu b_i
=\sum_{n=0}^{\infty}\hat\nu b_n$;}

\noindent which means that $\hat\nu$ is countably additive, and
$(\widehat{\frak A},\hat\nu)$ is a measure algebra.
}%end of proof of 392H

\leader{392I}{Corollary} Let $\frak A$ be a Boolean algebra and $\nu$ a
non-negative additive functional on $\frak A$.   Then there are a totally
finite measure algebra $(\frak C,\bar\mu)$ and a Boolean homomorphism
$\pi:\frak A\to\frak C$ such that $\nu a=\bar\mu(\pi a)$ for every
$a\in\frak A$.

\proof{ Set $I=\{a:\nu a=0\}$;  then $I\normalsubgroup\frak A$, so we can
form the quotient algebra $\frak B=\frak A/I$ (312L);  let
$\pi:\frak A\to\frak B$ be the canonical map.   As in part (b) of the proof
of 321H, we have an additive functional $\mu:\frak B\to\coint{0,\infty}$
such that $\mu(\pi a)=\nu a$ for every $a\in\frak A$, and (as in 321H)
$\mu$ is strictly positive.   Take
$(\frak C,\bar\mu)$ to be $(\widehat{\frak B},\hat\mu)$ as in 392Hd, so
that $(\frak C,\bar\mu)$ is a totally finite measure algebra.   If we now
think of $\pi$ as a map from $\frak A$ to $\frak C$, it will still be a
Boolean homomorphism, and

\Centerline{$\nu a=\mu(\pi a)=\bar\mu(\pi a)$}

\noindent for every $a\in\frak A$.
}%end of proof of 392I

\leader{392J}{Proposition}\dvAnew{2008}
Let $\frak A$ be a Boolean algebra, $\nu$
an exhaustive submeasure on $\frak A$, and $\sequencen{a_n}$ a sequence in
$\frak A$ such that $\inf_{n\in\Bbb N}\nu a_n>0$.   Then there is an
infinite $I\subseteq\Bbb N$ such that $\nu(\inf_{i\in I\cap n}a_i)>0$ for
every $n\in\Bbb N$.

\cmmnt{\medskip

\noindent{\bf Remark} In the formula $I\cap n$ I am identifying $n$ with
the set of its predecessors, as in 3A1H.
}

\proof{ For finite $J\subseteq\Bbb N$ set $b_J=\inf_{i\in J}a_i$.   Let
$\Cal J$ be the family of those $J\in[\Bbb N]^{<\omega}$ such that
$\limsup_{n\to\infty}\nu(a_n\Bcap b_J)>0$.

\Quer\ Suppose, if possible, that there is no strictly increasing sequence
in $\Cal J$.   Then $\Cal J$ must have a maximal element $J$ say.
Set $a'_n=a_n\Bcap b_J$ for $n\in\Bbb N$ and
$\delta=\limsup_{n\to\infty}\nu a'_n>0$.   For any $n\in\Bbb N\setminus J$,
$J\cup\{n\}\notin\Cal J$ so

\Centerline{$\lim_{m\to\infty}a'_m\Bcap a'_n
=\lim_{m\to\infty}a_m\Bcap b_{J\cup\{n\}}=0$.}

\noindent We can therefore choose inductively a sequence $\sequencen{k_n}$
such that

\Centerline{$k_n>\sup J$,
\quad$\nu a'_{k_n}\ge\Bover34\delta$,
\quad$\nu(a'_{k_n}\Bcap a'_{k_i})\le 2^{-i-2}\delta$
for every $i<n$}

\noindent for each $n\in\Bbb N$.   Now set
$b_n=a_{k_n}\Bsetminus\sup_{i<n}a_{k_i}$ for each $n$.   Then
$\sequencen{b_n}$ is disjoint.   Also

\Centerline{$\Bover34\delta
\le\nu a_{k_n}
\le\nu b_n+\sum_{i=0}^{n-1}\nu(a_{k_n}\Bcap a_{k_i})
\le\nu b_n+\sum_{i=0}^{n-1}2^{-i-2}\delta
\le\nu b_n+\Bover12\delta$}

\noindent and $\nu b_n\ge\bover14\delta$ for every $n$;  which is
impossible.\ \Bang

There must therefore be a strictly increasing sequence $\sequencen{J_n}$ in
$\Cal J$.   Set $I=\bigcup_{n\in\Bbb N}J_n$.   If $n\in\Bbb N$, there is an
$m\in\Bbb N$ such that $I\cap n\subseteq J_m$ and
$\nu(\inf_{i\in I\cap n}a_i)\ge\nu b_{J_m}>0$.   So we have an appropriate
$I$.
}%end of proof of 392J

\leader{*392K}{Products of submeasures}\dvAnew{2008} \cmmnt{There seems
to be no fully satisfying general construction for
products of submeasures.   However the
following method has some interesting features.

\medskip

}{\bf (a)} Let $\frak A$ and $\frak B$ be Boolean algebras with
submeasures $\mu$, $\nu$ respectively.   On the free product
$\frak A\otimes\frak B$\cmmnt{ (\S315)}, we have a
functional $\mu\ltimes\nu$ defined by saying that whenever
$c\in\frak A\otimes\frak B$ is of the form
$\sup_{i\in I}a_i\otimes b_i$ where $\familyiI{a_i}$ is a finite
partition of unity in $\frak A$, then

$$\eqalign{(\mu\ltimes\nu)(c)
&=\min_{J\subseteq I}
  \max(\{\mu(\sup_{i\in J}a_i)\}\cup\{\nu b_i:i\in I\setminus J\})\cr
&=\min\{\epsilon:\epsilon\in[0,\infty],
  \,\mu(\sup\{a_i:i\in I,\,\nu b_i>\epsilon\})
\le\epsilon\}.\cr}$$

\prooflet{\noindent\Prf\ Every $c\in\frak A\otimes\frak B$ 
can be expressed in
this form (315Oa).   Of course this can be done in
many different ways.   But if
$c=\sup_{j\in J}a'_j\otimes b'_j$ is another expression of the same
kind, then $b_i=b'_j$ whenever $a_i\Bcap a'_j\ne 0$.   So

$$\eqalign{\sup\{a_i:i\in I,\,\nu b_i>\epsilon\}
&=\sup\{a_i\Bcap a'_j:i\in I,\,j\in J,\,a_i\Bcap a'_j\ne 0,\,
  \nu b_i>\epsilon\}\cr
&=\sup\{a_i\Bcap a'_j:i\in I,\,j\in J,\,a_i\Bcap a'_j\ne 0,\,
  \nu b'_j>\epsilon\}\cr
&=\sup\{a'_j:j\in J,\,\nu b'_j>\epsilon\}\cr}$$

\noindent for any $\epsilon$, and the two calculations for $\mu\ltimes\nu$
give the same result.\ \Qed}

\cmmnt{Note that }$(\mu\ltimes\nu)(a\otimes b)=\min(\mu a,\nu b)$ for all
$a\in\frak A$ and $b\in\frak B$.

\spheader 392Kb In the context of (a), $\mu\ltimes\nu$ is a submeasure.

\prooflet{\Prf\ By definition, $(\mu\ltimes\nu)c\ge 0$ for every
$c\in\frak A\otimes\frak B$;  and if $c=0$ then $c=1\otimes 0$ and
$(\mu\ltimes\nu)c=0$.

If $c$, $c'$ are two members of $\frak A\otimes\frak B$, express them
in the forms
$c=\sup_{i\in I}a_i\otimes b_i$ and $c'=\sup_{j\in J}a'_j\otimes
b'_j$ where $\familyiI{a_i}$ and
$\family{j}{J}{a'_j}$ are partitions of unity in $\frak A$.   Set
$K=\{(i,j):a_i\Bcap a'_j\ne 0\}\subseteq I\times J$,
$a''_{ij}=a_i\Bcap a'_j$ for $(i,j)\in K$;  then $\langle
a''_{ij}\rangle_{(i,j)\in K}$ is a partition of unity in $\frak A$,
$c=\sup_{(i,j)\in K}a''_{ij}\otimes b_i$ and
$c'=\sup_{(i,j)\in K}a''_{ij}\otimes b'_j$.   Set
$\alpha=(\mu\ltimes\nu)c$, $\beta=(\mu\ltimes\nu)c'$,
$L=\{(i,j):(i,j)\in K$, $\nu b_i>\alpha\}$, $L'=\{(i,j):(i,j)\in K$,
$\nu b'_j>\beta\}$,
$e=\sup\{a_{ij}:(i,j)\in L\}$ and $e'=\sup\{a_{ij}:(i,j)\in L'\}$;
then $\mu e\le\alpha$ and
$\mu e'\le\beta$.   So $\mu(e\Bcup e')\le\alpha+\beta$;  but
$e\Bcup e'=\sup_{(i,j)\in L\cup L'}a''_{ij}$ and

\Centerline{$\nu(b_i\Bcup b'_j)\le\nu b_i+\nu b'_j\le\alpha+\beta$}

\noindent for all $(i,j)\in K\setminus(L\cup L')$.   So
$(\mu\ltimes\nu)(c\Bcup c')\le\alpha+\beta$.

If $c\Bsubseteq c'$, then $b_i\Bsubseteq b'_j$ for every $(i,j)\in
K$.   So $\nu b_i\le\beta$ for every $(i,j)\in K\setminus L'$ and
$(\mu\ltimes\nu)c\le\beta$.

Thus $\mu\ltimes\nu$ is subadditive and order-preserving and is a
submeasure.\ \Qed}

\spheader 392Kc I note that only in exceptional cases will $\mu\ltimes\nu$
be matched with $\nu\ltimes\mu$ by the canonical isomorphism between
$\frak A\otimes\frak B$ and $\frak B\otimes\frak A$;  this product is
not `commutative'.   \cmmnt{(See 392Yc.)}
It is however `associative', in the following sense.
Let $(\frak A_1,\mu_1)$, $(\frak A_2,\mu_2)$, $(\frak A_3,\mu_3)$ be
Boolean algebras endowed with submeasures.   Set

\Centerline{$\lambda_{12}=\mu_1\ltimes\mu_2$,
\quad$\lambda_{(12)3}=\lambda_{12}\ltimes\mu_3$,
\quad$\lambda_{23}=\mu_2\ltimes\mu_3$,
\quad$\lambda_{1(23)}=\mu_1\ltimes\lambda_{23}$.}

\noindent Then the canonical isomorphisms between
$(\frak A_1\otimes\frak A_2)\otimes\frak A_3$,
$\frak A_1\otimes\frak A_2\otimes\frak A_3$
and $\frak A_1\otimes(\frak A_2\otimes\frak A_3)$\cmmnt{ (315L)}
identify $\lambda_{(12)3}$ with $\lambda_{1(23)}$.

\prooflet{\Prf\ Take any $d\in\frak A_1\otimes\frak A_2\otimes\frak A_3$.
Express $d$ as
$\sup_{i\in I}a_i\otimes e_i$ where $\familyiI{a_i}$ is a partition
of unity in $\frak A_1$ and $e_i\in\frak A_2\otimes\frak A_3$ for
each $i$;  express each $e_i$ as
$\sup_{j\in J_i}b_{ij}\otimes c_{ij}$ where $\family{j}{J_i}{b_{ij}}$
is a partition of unity in $\frak A_2$ and $c_{ij}\in\frak A_3$ for
$i\in I$, $j\in J_i$.   In this case,
$\langle a_i\otimes b_{ij}\rangle_{i\in I,j\in J_i}$ is a partition
of unity in $\frak A_1\otimes\frak A_2$ and
$d=\sup_{i\in I,j\in J_i}a_i\otimes b_{ij}\otimes c_{ij}$.

Let $\epsilon>0$.   For $i\in I$, set $J'_i=\{j:j\in J_i$,
$\mu_3c_{ij}>\epsilon\}$,
$e'_i=\sup_{j\in J'_i}b_{ij}$.   Then
$\lambda_{23}(\sup_{j\in J_i}b_{ij}\otimes c_{ij})\le\epsilon$ iff
$\mu_2e'_i\le\epsilon$.
Set $I'=\{i:\mu_2e'_i>\epsilon\}$;  then
$\lambda_{1(23)}d\le\epsilon$ iff
$\mu_1(\sup_{i\in I'}a_i)\le\epsilon$.   From the other direction, set
$f=\sup\{a_i\otimes b_{ij}:i\in I$, $j\in J'_i\}$;  then
$\lambda_{(12)3}d\le\epsilon$ iff $\lambda_{12}f\le\epsilon$.   But
$f=\sup_{i\in I}a_i\otimes e'_i$, so
$\lambda_{12}f\le\epsilon$ iff $\mu_1(\sup_{i\in I'}a_i)\le\epsilon$.

As $\epsilon$ and $d$ are arbitrary,
$\lambda_{(12)3}=\lambda_{1(23)}$, as claimed.\ \Qed}

\spheader 392Kd If $\mu$, $\mu'$ are submeasures on $\frak A$,
$\nu$ and $\nuprime$ are submeasures on $\frak B$,
$\mu$ is absolutely continuous
with respect to $\mu'$ and $\nu$ is
absolutely continuous with respect to $\nuprime$, then
$\mu\ltimes\nu$ is absolutely continuous with respect to
$\mu'\otimes\nuprime$.   \prooflet{\Prf\ For any $\epsilon>0$ there is a
$\delta>0$ such that
$\mu a\le\epsilon$ whenever $\mu'a\le\delta$ and $\nu b\le\epsilon$
whenever $\nuprime b\le\delta$.   If now $c\in\frak A\otimes\frak B$ and
$(\mu'\ltimes\nuprime)(c)\le\delta$, we have an expression
$c=\sup_{i\in I}a_i\otimes b_i$ and a set $J\subseteq I$ such that
$\familyiI{a_i}$ is a partition of unity,
$\mu'(\sup_{i\in J}a_i)\le\delta$ and $\nuprime b_i\le\delta$ for every
$i\in I\setminus J$;  so $\mu(\sup_{i\in J}a_i)\le\epsilon$,
$\nu b_i\le\epsilon$ for every
$i\in I\setminus J$ and $(\mu\ltimes\nu)(c)\le\epsilon$.\ \Qed}

\spheader 392Ke If $\mu$ and $\nu$ are exhaustive, so is
$\mu\ltimes\nu$.   \prooflet{\Prf\ Let $\sequencen{c_n}$ be a sequence in
$\frak A\otimes\frak B$ such that $(\mu\ltimes\nu)c_n>\epsilon>0$ for
every $n$.   For each $n$, express $c_n$ as
$\sup_{i\in I_n}a_{ni}\otimes b_{ni}$ where $\family{i}{I_n}{a_{ni}}$
is a partition of unity;  set $I'_n=\{i:i\in I_n$,
$\nu b_{ni}>\epsilon\}$,
$a_n=\sup_{i\in I'_n}a_{ni}$;  then
$\mu a_n>\epsilon$.   By 392J,
there is an infinite $J\subseteq\Bbb N$ such that
$\inf_{i\in J\cap n}a_i\ne 0$ for every $n\in\Bbb N$.
Let $Z$ be the Stone space of $\frak A$, and write $\widehat{a}\subseteq Z$
for the open-and-closed set corresponding to $a\in\frak A$;  then
there is a $z\in\bigcap_{n\in J}\widehat{a}_n$.   For every $n\in J$
there is an $i_n\in I'_n$ such
that $z\in\widehat{a}_{n,i_n}$.   But now observe that
$\nu b_{n,i_n}>\epsilon$ for every $n\in J$, so there must be distinct
$m$, $n\in J$ such that
$b_{m,i_m}\Bcap b_{n,i_n}\ne 0$;  as $a_{m,i_m}\Bcap a_{n,i_n}$ is
also non-zero, $c_m\Bcap c_n\ne 0$.
As $\sequencen{c_n}$ is arbitrary, $\mu\ltimes\nu$ is exhaustive.\ \Qed}

\spheader 392Kf We can extend the construction to infinite products, as
follows.   Let $I$ be a totally ordered set and
$\familyiI{(\frak A_i,\mu_i)}$ a family of Boolean
algebras endowed with unital submeasures.   For a finite set
$J=\{i_0,\ldots,i_n\}$ where $i_0<\ldots<i_n$ in $I$, let $\lambda_J$
be the product submeasure
$(.(\mu_{i_0}\ltimes\mu_{i_1})\ltimes\ldots)\ltimes\mu_{i_n}$ on
$\frak C_J=\bigotimes_{j\in J}\frak A_j$;  for definiteness, on
$\frak C_{\emptyset}=\{0,1\}$
take $\lambda_{\emptyset}$ to be the unital submeasure, while
$\frak C_{\{i\}}=\frak A_i$ and $\lambda_{\{i\}}=\mu_i$ for each $i\in I$.
Using (c) repeatedly, we see that if $J$, $K\in[I]^{<\omega}$ and
$j<k$ for every $j\in J$,
$k\in K$, then the identification of $\frak C_{J\cup K}$ with
$\frak C_J\otimes\frak C_K$\cmmnt{ (315L)} matches $\lambda_{J\cup K}$
with $\lambda_J\ltimes\lambda_K$.
Moreover, if $K\in[I]^{<\omega}$ and $J$ is any subset of $K$\cmmnt{ (not
necessarily an initial segment)} and
$\varepsilon_{JK}:\frak C_J\to\frak C_K$ is the canonical embedding
corresponding to the identification of $\frak C_K$ with
$\frak C_J\otimes\frak C_{K\setminus J}$, then
$\lambda_J=\lambda_K\varepsilon_{JK}$;  this also is an easy
induction on $\#(K)$.   What this means is that for any subset $M$ of
$I$ we have a submeasure $\lambda_M$ on
$\frak C_M=\bigcup\{\varepsilon_{JM}\frak C_J:J\in[M]^{<\omega}\}$,
being the unique functional such that
$\lambda_M\varepsilon_{JM}=\lambda_J$ for every
$J\in[M]^{<\omega}$.   Finally, if
$L$, $M$ are subsets of $I$ with $l<m$ for every $l\in L$ and
$m\in M$, then $\lambda_{L\cup M}$ can be identified with
$\lambda_L\ltimes\lambda_M$.

\spheader 392Kg I should perhaps have remarked already that if $\mu$ and
$\nu$, in (a), are additive and unital,
then we have an additive function $\lambda'$ on $\frak A\otimes\frak B$
such that
$\lambda'(a\otimes b)=\mu a\cdot\nu b$ for every $a\in\frak A$ and
$b\in\frak B$\cmmnt{ (326E)}.   Now, setting $\lambda=\mu\ltimes\nu$,
each of $\lambda$, $\lambda'$ is absolutely continuous with respect to the
other.   \prooflet{\Prf\ If $c\in\frak A\otimes\frak B$, express $c$ as
$\sup_{i\in I}a_i\otimes b_i$ where
$\familyiI{a_i}$ is a finite partition of unity.
Then $\mu(\sup\{a_i:\nu b_i>\lambda c\})\le\lambda c$, so
$\lambda'c=\sum_{i\in I}\mu a_i\cdot\nu b_i$ is at most $2\lambda c$.
On the other hand,
$\mu(\sup\{a_i:\nu b_i>\sqrt{\lambda'c}\})\le\sqrt{\lambda'c}$, so
$\lambda c\le\sqrt{\lambda'c}$.\ \Qed}

\exercises{
\leader{392X}{Basic exercises (a)}
%\spheader 392Xa
Show that the first two clauses of the definition 392A can be replaced
by `$\nu a\le\nu(a\Bcup b)\le\nu a+\nu b$ whenever $a\Bcap b=0$'.
%392A

\spheader 392Xb Let $\frak A$ be any Boolean algebra and $\nu$ a
finite-valued submeasure on $\frak A$.   (i) Show that
$\nu$ is order-continuous iff whenever
$A\subseteq\frak A$ is non-empty, downwards-directed and has infimum
$0$, then $\inf_{a\in A}\nu a=0$.   (ii) Show that in this case
$\nu$ is exhaustive.   \Hint{if $\sequencen{a_n}$ is disjoint, then
$\bigcup_{n\in\Bbb N}\{b:b\Bsupseteq a_i$ for every $i\ge n\}$ has
infimum $0$.}
%392C

\spheader 392Xc Let $\frak A$ be a Boolean algebra and $\mu$, $\nu$ two
strictly positive submeasures on $\frak A$, each of which is absolutely
continuous with respect to the other.   Show that they induce uniformly
equivalent metrics on $\frak A$ (392H), so that both give the same metric
completion of $\frak A$.
%392H

\spheader 392Xd Let $\frak A$, $\frak B$ be Boolean algebras with
uniformly exhaustive submeasures $\mu$, $\nu$ respectively.
Show that $\mu\ltimes\nu$ is uniformly exhaustive.
%392K

\leader{392Y}{Further exercises (a)}
%\spheader 392Ya
Let $\frak A$ be a Boolean algebra and
$\lambda:\frak A\to[0,1]$ a functional such that $\lambda 0=0$
and $\lambda a\le\lambda(a\Bcup b)\le 2\max(\lambda a,\lambda b)$ for all
$a$, $b\in\frak A$.   Show that there is a submeasure $\nu$ on $\frak A$
such that $\bover12\lambda\le\nu\le\lambda$.
%\Hint{consider first the
%case in which every non-zero value of $\lambda$ is of the form $2^{-k}$.}
%392A

\spheader 392Yb (T.Jech)
Show that a Boolean algebra $\frak A$ is chargeable
iff there are sequences $\sequencen{A_n}$ and
$\sequencen{k_n}$ such that
($\alpha$) $\bigcup_{n\in\Bbb N}A_n=\frak A\setminus\{0\}$
($\beta$) whenever $a$, $b\in\frak A$, $n\in\Bbb N$
and $a\Bcup b\in A_n$ then at least
one of $a$, $b$ belongs to $A_{n+1}$ ($\gamma$) if $n\in\Bbb N$ then
$k_n\in\Bbb N$ and if $a_0,\ldots,a_{k_n}\in\frak A$ are disjoint
then some $a_j$ does not belong to $A_n$.
%\Hint{392Ya.}
%392F 392Ya

\spheader 392Yc I will say that a submeasure $\nu$ on a Boolean algebra
$\frak A$ is {\bf properly atomless} if for every $\epsilon>0$ there is a
finite partition $A$ of unity in $\frak A$ such that $\nu a\le\epsilon$ for
every $a\in A$.   (Compare 326F.)
(i) Show that if $\frak A$ and $\frak B$ are Boolean algebras
with submeasures $\mu$, $\nu$ respectively, we have a functional
$\mu\rtimes\nu:\frak A\otimes\frak B\to[0,\infty]$ defined by saying that

\Centerline{$(\mu\rtimes\nu)(\sup_{i\in I}a_i\otimes b_i)
=\min_{J\subseteq I}
  \max(\{\nu(\sup_{i\in J}b_i)\}\cup\{\mu a_i:i\in I\setminus J\})$}

\noindent whenever $\familyiI{b_i}$ is a finite partition of unity in
$\frak B$ and $a_i\in\frak A$ for each $i\in I$.   (ii) Show that if
$\mu$ is a non-zero properly atomless submeasure,
$\nu$ is a submeasure, and
$\mu\ltimes\nu$ is absolutely continuous with respect to $\mu\rtimes\nu$,
then $\nu$ is uniformly exhaustive.
%392K

\spheader 392Yd (See 328H.)
Let $(I,\le)$ be a non-empty upwards-directed partially
ordered set, and $\familyiI{(\frak A_i,\bar\mu_i)}$
a family of probability algebras;  suppose
that $\pi_{ji}:\frak A_i\to\frak A_j$ is a measure-preserving
Boolean homomorphism whenever $i\le j$, and that
$\pi_{ki}=\pi_{kj}\pi_{ji}$ whenever $i\le j\le k$.
(i) Let $\Cal F$ be the filter

\Centerline{$\{A:A\subseteq I$, there is some $i\in I$ such that $j\in A$
whenever $i\le j\}$,}

\noindent and set
$\nu\familyiI{a_i}=\limsup_{i\to\Cal F}\bar\mu_ia_i$ for
$\familyiI{a_i}\in\prod_{i\in I}\frak A_i$.   Show that $\nu$ is a
submeasure on $\frak A=\prod_{i\in I}\frak A_i$.   (ii) Let $\Cal J$ be the
ideal $\{d:\nu d=0\}$ of $\frak A$, and $\frak D$ the quotient algebra
$\frak A/\Cal J$.   Show that we have a strictly positive
unital submeasure
$\bar\nu$ on $\frak D$ such that $\bar\nu d^{\ssbullet}=\nu d$ for every
$d\in\frak A$, and that $\frak D$ is complete under the metric defined by
$\bar\nu$.   (iii) Show that for each $i\in I$ we have a Boolean
homomorphism $\pi_i:\frak A_i\to\frak D$ defined by setting
$\pi_ia=\family{j}{I}{a_j}^{\ssbullet}$, where $a_j=\pi_{ji}a$ if
$j\ge i$, $0_{\frak A_j}$ otherwise, and that $\bar\nu\pi_i=\bar\mu_i$.
Show that $\pi_i=\pi_j\phi_{ji}$ whenever $i\le j$.
(iv) Show that
$\frak D_0=\bigcup_{i\in I}\pi_i[\frak A_i]$ is a subalgebra of $\frak D$,
and that $\bar\nu\restrp\frak D_0$ is additive.   (v) Let $\frak C$ be the
closure of $\frak D_0$ in $\frak D$, and set
$\bar\lambda=\bar\nu\restrp\frak C$.   Show that $(\frak C,\bar\lambda)$ is
a probability algebra.
(vi) Now suppose that $(\frak B,\bar\nu)$ is a probability algebra, and
that for each $i\in I$ we are given a measure-preserving
Boolean homomorphism $\phi_i:\frak A_i\to\frak B$ such that
$\phi_i=\phi_j\pi_{ji}$ whenever $i\le j$.   Show that
there is a unique measure-preserving Boolean homomorphism
$\phi:\frak C\to\frak B$ such that $\phi\pi_i=\phi_i$ for every $i\in I$.
%328H 392H out of order query

\spheader 392Ye Let $\frak A$ be a Boolean algebra and
$\nu:\frak A\to[0,\infty]$ a submeasure.   Show that
the following are equiveridical:   (i) $\nu$ is uniformly exhaustive;
(ii) whenever
$\sequencen{a_n}$ is a sequence in $\frak A$ such that
$\inf_{n\in\Bbb N}\nu a_n>0$, there is a set
$I\subseteq\Bbb N$, not of zero asymptotic density, such that
$a_i\Bcap a_j\ne 0$ for all $i$, $j\in I$;  (iii)
whenever $\sequencen{a_n}$ is a sequence in $\frak A$ such that
$\inf_{n\in\Bbb N}\nu a_n>0$, there is a set
$I\subseteq\Bbb N$, not of zero asymptotic density, such that
$\nu(\inf_{i\in I,i\le n}a_i)>0$ for every $n\in\Bbb N$.
%392J out of order query
}%end of exercises

\cmmnt{\Notesheader{392} Much of the first part of this section is a
matter of generalizing earlier arguments.   Thus 392C ought by now to be
very easy, while 392Xb recalls the elementary theory of $\tau$-additive
functionals.

The new ideas are in the combinatorics of 392D-392E.
I have cast 392D in the form of an argument in probability theory.   Of
course there is nothing here but simple counting, since the probability
measure simply puts the same mass on each point of $\Omega$, and every
statement of the form `$\Pr(R\,\ldots)\le\ldots$' is just a matter of
counting the elements $R$ of $\Omega$ with the given property.   But I
think many of us find that the probabilistic language makes the
calculations more natural;  in particular, we can use intuitions
associated with the notion of independence of events.   Indeed I
strongly recommend the method.   It has been used to very great effect
in the last sixty years in a wide variety of combinatorial
problems.
392F and 392G together constitute the {\bf Kalton-Roberts theorem}
({\smc Kalton \& Roberts 83}).

}%end of notes

\discrpage

