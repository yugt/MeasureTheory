\frfilename{mt414.tex}
\versiondate{26.1.10}
\copyrightdate{2002}

\def\chaptername{Topologies and measures I}
\def\sectionname{$\tau$-additivity}
\def\IMPLY#1#2{{\bf (#1)$\Rightarrow$(#2)}}
\def\undphi{\underline{\phi}}
\def\undpsi{\underline{\psi}}

\newsection{414}

The second topic I wish to treat is that of `$\tau$-additivity'.   Here
I collect results which do not depend on any strong kind of inner
regularity.   I begin with what I think of as the most characteristic
feature of $\tau$-additivity, its effect on the properties of
semi-continuous functions (414A), with a variety of corollaries, up to
the behaviour of subspace measures
(414K).   A very important property of $\tau$-additive topological
measures is that they are often strictly localizable (414J).

The theory of inner regular
$\tau$-additive measures belongs to the next section, but here I give
two introductory results:  conditions under which a $\tau$-additive
measure will be inner regular with respect to closed sets (414M) and
conditions under which a measure which is inner regular
with respect to closed sets will be $\tau$-additive (414N).   I end the
section with notes on `density' and `lifting' topologies (414P-414R).

\leader{414A}{Theorem} Let $(X,\frak T)$ be a topological space and
$\mu$ an effectively locally finite $\tau$-additive measure on $X$ with
domain $\Sigma$ and measure algebra $\frak A$.

(a) Suppose that $\Cal G$ is a non-empty family in $\Sigma\cap\frak T$
such that $H=\bigcup\Cal G$ also belongs to $\Sigma$.   Then
$\sup_{G\in\Cal G}G^{\ssbullet}=H^{\ssbullet}$ in $\frak A$.

(b) Write $\eusm L$ for the family of
$\Sigma$-measurable lower semi-continuous functions from $X$ to
$\Bbb R$.   Suppose that
$\emptyset\ne A\subseteq\eusm L$ and set $g(x)=\sup_{f\in A}f(x)$ for
every $x\in X$.   If $g$ is $\Sigma$-measurable and finite almost
everywhere, then $\tilde g^{\ssbullet}=\sup_{f\in A}f^{\ssbullet}$ in
$L^0(\mu)$, where $\tilde g(x)=g(x)$ whenever $g(x)$ is finite.

(c) Suppose that $\Cal F$ is a non-empty family of measurable closed
sets such that $\bigcap\Cal F\in\Sigma$.   Then
$\inf_{F\in\Cal F}F^{\ssbullet}=(\bigcap\Cal F)^{\ssbullet}$ in
$\frak A$.

(d) Write $\eusm U$ for the family of
$\Sigma$-measurable upper semi-continuous functions from $X$ to
$\Bbb R$.   Suppose that $A\subseteq\eusm U$ is non-empty and set
$g(x)=\inf_{f\in A}f(x)$ for every $x\in X$.   If $g$ is
$\Sigma$-measurable and finite almost everywhere, then
$\tilde g^{\ssbullet}=\inf_{f\in A}f^{\ssbullet}$ in $L^0(\mu)$, where
$\tilde g(x)=g(x)$ whenever $g(x)$ is finite.

\proof{{\bf (a)} \Quer\ If
$H^{\ssbullet}\ne\sup_{G\in\Cal G}G^{\ssbullet}$, there is a non-zero
$a\in\frak A$ such that $a\Bsubseteq H^{\ssbullet}$ but
$a\Bcap G^{\ssbullet}=0$ for every $G\in\Cal G$.   Express $a$ as
$E^{\ssbullet}$
where $E\in\Sigma$ and $E\subseteq H$.   Because $\mu$ is effectively
locally finite, there is a measurable open set $H_0$ of finite measure
such that $\mu(H_0\cap E)>0$.   Now $\{H_0\cap G:G\in\Cal G\}$ is an
upwards-directed family of measurable open sets with union
$H_0\cap H\supseteq H_0\cap E$;  as $\mu$ is $\tau$-additive, there is a
$G\in\Cal G$
such that $\mu(H_0\cap G)>\mu H_0-\mu(H_0\cap E)$.   But in this case
$\mu(G\cap E)>0$, which is impossible, because
$G^{\ssbullet}\Bcap E^{\ssbullet}=0$.\ \Bang

\medskip

{\bf (b)} For any $\alpha\in\Bbb R$,

\Centerline{$\{x:g(x)>\alpha\}=\bigcup_{f\in A}\{x:f(x)>\alpha\}$,}

\noindent and these are all measurable open sets.   Identifying
$\{x:g(x)>\alpha\}^{\ssbullet}\in\frak A$ with
$\Bvalue{\tilde g^{\ssbullet}>\alpha}$ (364Ib\formerly{3{}64Jb}),
we see from (a) that
$\Bvalue{\tilde g^{\ssbullet}>\alpha}
=\sup_{f\in A}\Bvalue{f^{\ssbullet}>\alpha}$ for every $\alpha$.   But
this means that $\tilde g^{\ssbullet}=\sup_{f\in A}f^{\ssbullet}$, by
364L(a-ii)\formerly{3{}64Mb}.

\medskip

{\bf (c)} Apply (a) to $\Cal G=\{X\setminus F:F\in\Cal F\}$.

\medskip

{\bf (d)} Apply (b) to $\{-f:f\in A\}$.
}%end of proof of 414A

\leader{414B}{Corollary} Let $X$ be a topological space and $\mu$ an
effectively locally finite $\tau$-additive topological measure on $X$.

(a) Suppose that $A$ is a non-empty upwards-directed family of lower
semi-continuous functions from $X$ to $[0,\infty]$.   Set
$g(x)=\sup_{f\in A}f(x)$ in $[0,\infty]$ for every $x\in X$.   Then
$\int g=\sup_{f\in A}\int f$ in $[0,\infty]$.

(b) Suppose that $A$ is a non-empty downwards-directed family of
non-negative continuous real-valued functions on $X$, and that
$g(x)=\inf_{x\in A}f(x)$ for every $x\in X$.   If any member of $A$ is
integrable, then $\int g=\inf_{f\in A}\int f$.

\proof{{\bf (a)} Of course all the $f\in A$, and also $g$, are
measurable functions.   Set $g_n=g\wedge n\chi X$ for every
$n\in\Bbb N$.   Then

\Centerline{$g_n(x)=\sup_{f\in A}(f\wedge n\chi X)(x)$}

\noindent for every $x\in X$, so
$g_n^{\ssbullet}=\sup_{f\in A}(f\wedge n\chi X)^{\ssbullet}$, by 414Ab,
and

\Centerline{$\int g_n=\int g_n^{\ssbullet}
=\sup_{f\in A}\int(f\wedge n\chi X)^{\ssbullet}
=\sup_{f\in A}\int f\wedge n\chi X$}

\noindent by 365Dh.   But now, of course,

\Centerline{$\int g=\sup_{n\in\Bbb N}\int g_n
=\sup_{n\in\Bbb N,f\in A}\int f\wedge n\chi X=\sup_{f\in A}\int f$,}

\noindent as claimed.

\medskip

{\bf (b)} Take an integrable $f_0\in A$, and apply (a) to
$\{(f_0-f)^+:f\in A\}$.
}%end of proof of 414B

\leader{414C}{Corollary} Let $(X,\frak T,\Sigma,\mu)$ be an effectively
locally finite $\tau$-additive
topological measure space and $\Cal F$ a non-empty
downwards-directed family of closed sets.   If $\inf_{F\in\Cal F}\mu F$
is finite, this is the measure of $\bigcap\Cal F$.

\proof{ Setting $F_0=\bigcap\Cal F$, then
$F_0^{\ssbullet}=\inf_{F\in\Cal F}F^{\ssbullet}$, by 414Ac;  now

\Centerline{$\mu F_0=\bar\mu F_0^{\ssbullet}
=\inf_{F\in\Cal F}\bar\mu F^{\ssbullet}=\inf_{F\in\Cal F}\mu F$}

\noindent by 321F.
}%end of 414C

\leader{414D}{Corollary} Let $\mu$ be an effectively locally finite
$\tau$-additive measure on a topological space $X$.   If $\nu$ is a
totally finite measure with the same domain as $\mu$, truly continuous
with respect to $\mu$, then $\nu$ is $\tau$-additive.   In particular,
if $\mu$ is $\sigma$-finite and $\nu$ is absolutely continuous with
respect to $\mu$, then $\nu$ is $\tau$-additive.

\proof{ We have a functional $\bar\nu:\frak A\to\coint{0,\infty}$, where
$\frak A$ is the measure algebra of $\mu$, such that
$\bar\nu E^{\ssbullet}=\nu E$ for every $E$ in the common domain
$\Sigma$ of $\mu$ and $\nu$.   Now $\bar\nu$ is continuous for the
measure-algebra topology of
$\frak A$ (327Cd), therefore completely additive (327Ba), therefore
order-continuous (326Oc\formerly{3{}26Kc}).
So if $\Cal G$ is an upwards-directed family
of open sets belonging to $\Sigma$ with union $G_0\in\Sigma$,

\Centerline{$\sup_{G\in\Cal G}\nu G
=\sup_{G\in\Cal G}\bar\nu G^{\ssbullet}
=\bar\nu G_0^{\ssbullet}
=\nu G_0$}

\noindent because $G_0^{\ssbullet}=\sup_{G\in\Cal G}G^{\ssbullet}$.

The last sentence follows at once, because on a $\sigma$-finite space an
absolutely continuous countably additive functional is truly continuous
(232Bc).
}%end of proof of 414D

\leader{414E}{Corollary} Let $(X,\frak T,\Sigma,\mu)$ be an effectively
locally finite $\tau$-additive topological measure space.   Suppose that
$\Cal G\subseteq\frak T$ is non-empty and upwards-directed, and
$H=\bigcup\Cal G$.   Then

(a) $\mu(E\cap H)=\sup_{G\in\Cal G}\mu(E\cap G)$ for every
$E\in\Sigma$;

(b) if $f$ is a non-negative virtually measurable real-valued function
defined almost everywhere in $X$, then $\int_Hf=\sup_{G\in\Cal G}\int_Gf$
in $[0,\infty]$.

\proof{{\bf (a)} In the measure algebra $(\frak A,\bar\mu)$ of $\mu$,

$$\eqalign{(E\cap H)^{\ssbullet}
&=E^{\ssbullet}\Bcap H^{\ssbullet}
=E^{\ssbullet}\Bcap\sup_{G\in\Cal G}G^{\ssbullet}\cr
&=\sup_{G\in\Cal G}E^{\ssbullet}\Bcap G^{\ssbullet}
=\sup_{G\in\Cal G}(E\cap G)^{\ssbullet},\cr}$$

\noindent using 414Aa and the distributive law 313Ba.   So

\Centerline{$\mu(E\cap H)
=\bar\mu(E\cap H)^{\ssbullet}
=\sup_{G\in\Cal G}\bar\mu(E\cap G)^{\ssbullet}
=\sup_{G\in\Cal G}\mu(E\cap G)$}

\noindent by 321D, because $\Cal G$ and
$\{(E\cap G)^{\ssbullet}:G\in\Cal G\}$ are upwards-directed.

\wheader{414E}{4}{2}{2}{30pt}

{\bf (b)} For each $G\in\Cal G$,

\Centerline{$\int_Gf=\int f\times\chi G
=\int(f\times\chi G)^{\ssbullet}
=\int f^{\ssbullet}\times\chi G^{\ssbullet}$,}

\noindent where $\chi G^{\ssbullet}$ can be interpreted either as
$(\chi G)^{\ssbullet}$ (in $L^0(\mu)$) or as $\chi(G^{\ssbullet})$ (in
$L^0(\frak A)$, where $\frak A$ is the measure algebra of $\mu$);  see
364J\formerly{3{}64K}.
Now $H^{\ssbullet}=\sup_{G\in\Cal G}G^{\ssbullet}$ (414Aa);
since $\chi$ and $\times$ are order-continuous (364Jc,
364N\formerly{3{}64P}),
$f^{\ssbullet}\times\chi H^{\ssbullet}
=\sup_{G\in\Cal G}f^{\ssbullet}\times\chi G^{\ssbullet}$;  so

\Centerline{$\int_Hf
=\int f^{\ssbullet}\times\chi H^{\ssbullet}
=\sup_{G\in\Cal G}\int f^{\ssbullet}\times\chi G^{\ssbullet}
=\sup_{G\in\Cal G}\int_Gf$}

\noindent by 365Dh again.
}%end of proof of 414E

\leader{414F}{Corollary} Let $(X,\frak T,\Sigma,\mu)$ be an effectively
locally finite $\tau$-additive topological measure space.   Then for
every $E\in\Sigma$ there is a unique relatively closed self-supporting
set $F\subseteq E$ such that $\mu(E\setminus F)=0$.

\proof{ Let $\Cal G$ be the set $\{G:G\in\frak T,\,\mu(G\cap E)=0\}$.
Then $\Cal G$ is upwards-directed, so
$\mu(E\cap G^*)=\sup_{G\in\Cal G}\mu(E\cap G)=0$, where
$G^*=\bigcup\Cal G$.   Set $F=E\setminus G^*$.   Then $F\subseteq E$ is
relatively closed, and $\mu(E\setminus F)=0$.   If $H\in\frak T$ and
$H\cap F\ne\emptyset$, then $H\notin\Cal G$ so
$\mu(F\cap H)=\mu(E\cap H)>0$;  thus $F$ is self-supporting.   If
$F'\subseteq E$ is another self-supporting relatively closed set such
that $\mu(E\setminus F')=0$, then
$\mu(F\setminus F')=\mu(F'\setminus F)=0$;  but
as $F\setminus F'$ is relatively open in $F$, and $F'\setminus F$ is
relatively open in $F'$, these must both be empty, and $F=F'$.
}%end of proof of 414F

\leader{414G}{Corollary} If $(X,\frak T,\Sigma,\mu)$ is a Hausdorff
effectively locally finite $\tau$-additive topological measure space and
$E\in\Sigma$ is an atom for $\mu$\cmmnt{ (definition:  211I)}, then
there is an $x\in E$ such that $E\setminus\{x\}$ is negligible.

\proof{ Let $F\subseteq E$ be a self-supporting set such that
$\mu(E\setminus F)=0$.   Since $\mu F=\mu E>0$, $F$ is not empty;  take
$x\in F$.   \Quer\ If $F\ne\{x\}$, let $y\in F\setminus\{x\}$.   Because
$\frak T$ is Hausdorff, there are disjoint open sets $G$, $H$ containing
$x$, $y$ respectively;  and in this case $\mu(E\cap G)=\mu(F\cap G)$ and
$\mu(E\cap H)=\mu(F\cap H)$ are both non-zero, which is impossible,
since $E$ is an atom.\ \Bang

So $F=\{x\}$ and $E\setminus\{x\}$ is negligible.
}%end of proof of 414G

\leader{414H}{Corollary} If $(X,\frak T,\Sigma,\mu)$ is an effectively
locally finite $\tau$-additive topological measure space and $\nu$ is an
indefinite-integral measure over $\mu$\cmmnt{ (definition:
234J\footnote{Formerly 2{}34B.})},
then $\nu$ is a $\tau$-additive topological measure.

\proof{ Because $\nu$ measures every set in $\Sigma$
(234La\footnote{Formerly 2{}34D.}), it is a
topological measure.   To see that it is $\tau$-additive, apply 414Eb to
a Radon-Nikod\'ym derivative of $\nu$.
}%end of proof of 414H

\leader{414I}{Proposition} Let $(X,\frak T,\Sigma,\mu)$ be a complete
locally determined effectively locally finite $\tau$-additive
topological measure space.   If
$E\subseteq X$ and $\Cal G\subseteq\frak T$ are such that
$E\subseteq\bigcup\Cal G$ and $E\cap G\in\Sigma$ for every $G\in\Cal G$,
then $E\in\Sigma$.

\proof{ Set $\Cal K=\{K:K\in\Sigma$, $E\cap K\in\Sigma\}$.   Then
whenever $F\in\Sigma$ and $\mu F>0$ there is a $K\in\Cal K$ included in
$F$ with $\mu K>0$.   \Prf\ Set $K_1=F\setminus\bigcup\Cal G$.   Then
$K_1$ is a member of $\Cal K$ included in $F$.   If $\mu K_1>0$ then we
can stop.   Otherwise,
$\Cal G^*=\{G_0\cup\ldots\cup G_n:G_0,\ldots,G_n\in\Cal G\}$ is an
upwards-directed family of open sets, and

\Centerline{$\sup_{G\in\Cal G^*}\mu(F\cap G)
=\mu(F\cap\bigcup\Cal G^*)
=\mu F>0$,}

\noindent by 414Ea.   So there is a $G\in\Cal G^*$ such that
$\mu(F\cap G)>0$;  but now $E\cap G\in\Sigma$ so $F\cap G\in\Cal K$.\
\Qed

By 412Aa, $\mu$ is inner regular with respect to $\Cal K$;  by 412Ja,
$E\in\Sigma$.
}%end of proof of 414I

\leader{414J}{Theorem} Let $(X,\frak T,\Sigma,\mu)$ be a complete
locally determined effectively locally finite $\tau$-additive
topological measure space.   Then $\mu$ is strictly localizable.

% question: will "locally finite" do in place of "eff loc fin"?
% at least to make $\mu$ localizable

\proof{ Let $\Cal F$ be a maximal disjoint family of self-supporting
measurable sets of
finite measure.   Then whenever $E\in\Sigma$ and $\mu E>0$, there is an
$F\in\Cal F$ such that $\mu(E\cap F)>0$.   \Prf\Quer\ Otherwise, let $G$
be an open set of finite measure such that $\mu(G\cap E)>0$, and set
$\Cal F_0=\{F:F\in\Cal F,\,F\cap G\ne\emptyset\}$.   Then
$\mu(F\cap G)>0$ for every $F\in\Cal F_0$, while $\mu G<\infty$ and
$\Cal F_0$ is disjoint, so
$\Cal F_0$ is countable and $\bigcup\Cal F_0\in\Sigma$.   Set
$E'=E\setminus\bigcup\Cal F_0$;  then
$E\setminus E'=E\cap\bigcup\Cal F_0$ is negligible, so
$\mu(G\cap E')>0$.   By 414F, there is a self-supporting set
$F'\subseteq G\cap E'$ such that $\mu F'>0$.   But in this case
$F'\cap F=\emptyset$ for
every $F\in\Cal F$, so we ought to have added $F'$ to
$\Cal F$.\ \Bang\Qed

This means that $\Cal F$ satisfies the criterion of 213Oa.   Because
$(X,\Sigma,\mu)$ is complete and locally determined, it is strictly
localizable.
}%end of proof of 414J

\leader{414K}{Proposition} Let $(X,\Sigma,\mu)$ be a measure space and
$\frak T$ a topology on $X$, and $Y\subseteq X$ a subset such that the
subspace measure $\mu_Y$ is semi-finite\cmmnt{ (see the remark
following 412O)}.
If $\mu$ is an effectively locally finite $\tau$-additive
topological measure, so is $\mu_Y$.

\proof{ By 412Pe, $\mu_Y$ is an effectively locally finite topological
measure.   Now
suppose that $\Cal H$ is a non-empty upwards-directed family in
$\frak T_Y$ with union $H^*$.   Set

\Centerline{$\Cal G=\{G:G\in\frak T,\,G\cap Y\in\Cal H\}$,
\quad $G^*=\bigcup\Cal G$,}

\noindent so that $\Cal G$ is upwards-directed and $H^*=Y\cap G^*$.
Let $\Cal K$ be the family of sets $K\subseteq X$ such that
$K\cap G^*\setminus G=\emptyset$ for some $G\in\Cal G$.   If
$E\in\Sigma$,

$$\eqalignno{\mu E
&=\mu(E\setminus G^*)+\mu(E\cap G^*)
=\mu(E\setminus G^*)+\sup_{G\in\Cal G}\mu(E\cap G)\cr
\displaycause{414Ea}
&=\sup_{G\in\Cal G}\mu(E\setminus(G^*\setminus G)),\cr}$$

\noindent so $\mu$ is inner regular with respect to $\Cal K$.   By
412Ob, $\mu_Y$ is inner regular with respect to
$\{K\cap Y:K\in\Cal K\}$.   So if $\gamma<\mu_YH^*$, there is a
$K\in\Cal K$ such that $K\cap Y\subseteq H^*$ and
$\mu_Y(K\cap Y)\ge\gamma$.   But now there is a $G\in\Cal G$ such that
$K\cap G^*\setminus G=\emptyset$, so that
$K\cap Y\subseteq G\cap Y\in\Cal H$ and
$\sup_{H\in\Cal H}\mu H\ge\gamma$.   As $\Cal H$ and $\gamma$ are
arbitrary, $\mu$ is $\tau$-additive.
}%end of proof of 414K

\cmmnt{\medskip

\noindent{\bf Remarks} Recall from 214Ic that if $(X,\Sigma,\mu)$ has
locally determined negligible sets (in particular, is either strictly
localizable or complete and locally
determined), then all its subspaces are semi-finite.   In 419C below I
describe a tight locally finite Borel measure with a subset on which the
subspace measure is not
semi-finite, therefore not effectively locally finite or
$\tau$-additive.   In 419A I describe a $\sigma$-finite locally finite
$\tau$-additive topological measure, inner regular with respect to the
closed sets, with a closed subset on which the subspace measure is
totally finite but not $\tau$-additive.
}%end of comment

\leader{414L}{Lemma} Let $(X,\frak T)$ be a topological space, and
$\mu$, $\nu$ two effectively locally finite Borel measures on $X$ which
agree on the open sets.   Then they are equal.

\proof{ Write $\frak T^f$ for the family of open sets of finite measure.
(I do not need to specify which measure I am using here.)   For
$G\in\frak T^f$, set $\mu_GE=\mu(G\cap E)$, $\nu_GE=\nu(G\cap E)$ for
every Borel
set $E$.   Then $\mu_G$ and $\nu_G$ are totally finite Borel measures
which agree on $\frak T$.   By the Monotone Class Theorem (136C),
$\mu_G$ and $\nu_G$ agree on the $\sigma$-algebra generated by
$\frak T$, that is, the Borel $\sigma$-algebra $\Cal B$.   Now, for any
$E\in\Cal B$,

\Centerline{$\mu E=\sup_{G\in\frak T^f}\mu_GE=\sup_{G\in\frak T^f}\nu_GE
=\nu E$,}

\noindent by 412F.   So $\mu=\nu$.
}%end of proof of 414L

\leader{414M}{Proposition} Let $(X,\Sigma,\mu)$ be a measure space with
a regular topology $\frak T$ such that $\mu$ is effectively locally
finite and $\tau$-additive and $\Sigma$ includes a base for $\frak T$.

(a) $\mu G=\sup\{\mu F:F\in\Sigma$ is closed, $F\subseteq G\}$ for every
open set $G\in\Sigma$.

(b) If $\mu$ is inner regular with respect to the $\sigma$-algebra
generated by $\frak T\cap\Sigma$, it is inner regular with respect to
the closed sets.

\proof{{\bf (a)} For $U\in\Sigma\cap\frak T$, the set

\Centerline{$\Cal H_U=\{H:H\in\Sigma\cap\frak T$,
$\overline{H}\subseteq U\}$}

\noindent is an upwards-directed family of open sets, and
$\bigcup\Cal H_U=U$ because $\frak T$ is regular and $\Sigma$ includes a
base for $\frak T$.   Because $\mu$ is $\tau$-additive,
$\mu U=\sup\{\mu H:H\in\Cal H_U\}$.
Now, given $\gamma<\mu G$, we can choose $\sequencen{U_n}$ in
$\Sigma\cap\frak T$ inductively, as follows.   Start by taking
$U_0\subseteq G$ such that $\gamma<\mu U_0<\infty$ (using the hypothesis
that $\mu$ is
effectively locally finite).   Given $U_n\in\Sigma\cap\frak T$ and
$\mu U_n>\gamma$, take $U_{n+1}\in\Sigma\cap\frak T$ such that
$\overline{U}_{n+1}\subseteq U_n$ and $\mu U_{n+1}>\gamma$.   On
completing the induction, set

\Centerline{$F=\bigcap_{n\in\Bbb N}U_n
=\bigcap_{n\in\Bbb N}\overline{U}_n$;}

\noindent then $F$ is a closed set belonging to $\Sigma$, $F\subseteq G$
and $\mu F\ge\gamma$.   As $\gamma$ is arbitrary, we have the result.

\medskip

{\bf (b)} Let $\Sigma_0$ be the $\sigma$-algebra generated by
$\Sigma\cap\frak T$ and set $\mu_0=\mu\restr\Sigma_0$.   Then
$\Sigma_0\cap\frak T=\Sigma\cap\frak T$ is still a base for $\frak T$
and $\mu_0$ is still $\tau$-additive and effectively locally finite, so
by (a) and 412G it is inner regular with respect to the closed sets.
Now we are supposing that $\mu$ is inner regular with respect to
$\Sigma_0$, so $\mu$ is inner regular with respect to the closed sets,
by 412Ab.
}%end of proof of 414M

\leader{414N}{Proposition} Let $(X,\Sigma,\mu)$ be a measure
space and $\frak T$ a topology on $X$.
Suppose that (i) $\mu$ is semi-finite and inner regular with respect
to the closed sets (ii) whenever $\Cal F$ is a
non-empty downwards-directed family of measurable closed sets with empty
intersection and $\inf_{F\in\Cal F}\mu F<\infty$, then $\inf_{F\in\Cal
F}\mu F=0$.   Then $\mu$ is $\tau$-additive.

\proof{ Let $\Cal G$ be a non-empty upwards-directed family of
measurable open sets with measurable union $H$.   Take any
$\gamma<\mu H$.    Because $\mu$ is semi-finite, there is a measurable
set $E\subseteq H$ such that $\gamma<\mu E<\infty$.   Now there is a
measurable closed set $F\subseteq E$ such that $\mu F\ge\gamma$.
Consider $\Cal F=\{F\setminus G:G\in\Cal G\}$.   This is a
downwards-directed family of closed sets of finite measure with empty
intersection.   So $\inf_{G\in\Cal G}\mu(F\setminus G)=0$, that is,

\Centerline{$\gamma\le\mu F=\sup_{G\in\Cal G}\mu(F\cap G)
\le\sup_{G\in\Cal G}\mu G$.}

\noindent As $\gamma$ is arbitrary, $\mu H=\sup_{G\in\Cal G}\mu G$;  as
$\Cal G$ is arbitrary, $\mu$ is $\tau$-additive.
}%end of proof of 414N

\leader{414O}{}\cmmnt{ The following elementary result is worth
noting.

\medskip

\noindent}{\bf Proposition} If $X$ is a hereditarily Lindel\"of
space\cmmnt{ (e.g., if it is separable and metrizable)}
then every measure on $X$ is $\tau$-additive.

\proof{ If $\mu$ is a measure on $X$, with domain $\Sigma$, and
$\Cal G\subseteq\Sigma$ is a non-empty upwards-directed family of
measurable open sets, then there is a sequence $\sequencen{G_n}$ in
$\Cal G$ such that $\bigcup\Cal G=\bigcup_{n\in\Bbb N}G_n$.   Now

\Centerline{$\mu(\bigcup\Cal G)
=\lim_{n\to\infty}\mu(\bigcup_{i\le n}G_i)\le\sup_{G\in\Cal G}\mu G$.}

\noindent As $\Cal G$ is arbitrary, $\mu$ is $\tau$-additive.
}%end of proof of 414O

\leader{414P}{Density \dvrocolon{topologies}}\cmmnt{ Recall that a
lower density for a measure space $(X,\Sigma,\mu)$ is a function
$\undphi:\Sigma\to\Sigma$ such that $\undphi E=\undphi F$ whenever $E$,
$F\in\Sigma$ and $\mu(E\symmdiff F)=0$,  $\mu(E\symmdiff\undphi E)=0$
for every $E\in\Sigma$, $\undphi\emptyset=\emptyset$ and
$\undphi(E\cap F)=\undphi E\Bcap\undphi F$ for all $E$, $F\in\Sigma$
(341C).

\medskip

\noindent}{\bf Proposition} Let $(X,\Sigma,\mu)$ be a complete locally
determined measure space and $\undphi:\Sigma\to\Sigma$ a lower density
such that $\undphi X=X$.   Set

\Centerline{$\frak T=\{E:E\in\Sigma,\,E\subseteq\undphi E\}$.}

\noindent Then $\frak T$ is a topology on $X$, the {\bf density
topology} associated with $\undphi$, and $(X,\frak T,\Sigma,\mu)$ is an
effectively
locally finite $\tau$-additive topological measure space;  $\mu$ is
strictly positive and inner regular with respect to the open sets.

\proof{{\bf (a)(i)} For any $E\in\Sigma$,
$\undphi(E\cap\undphi E)=\undphi E$ because $E\setminus\undphi E$ is
negligible;  consequently $E\cap\undphi E\in\frak T$.   In particular,
$\emptyset=\emptyset\cap\undphi\emptyset$ and
$X=X\cap\undphi X$ belong to $\frak T$.   If $E$, $F\in\frak T$ then

\Centerline{$\undphi(E\cap F)=\undphi E\cap\undphi F\supseteq E\cap F$,}

\noindent so $E\cap F\in\frak T$.

\medskip

\quad{\bf (ii)} Suppose that $\Cal G\subseteq\frak T$ and
$H=\bigcup\Cal G$.   By 341M, $\mu$ is (strictly) localizable, so
$\Cal G$ has an essential supremum $F\in\Sigma$ such that
$F^{\ssbullet}=\sup_{G\in\Cal G}G^{\ssbullet}$ in the measure algebra
$\frak A$ of $\mu$;  that is,
for $E\in\Sigma$, $\mu(G\setminus E)=0$ for every $G\in\Cal G$ iff
$\mu(F\setminus E)=0$.   Now $F\setminus H$ is negligible, by 213K.  On
the other hand,

\Centerline{$G\subseteq\undphi G=\undphi(G\cap F)\subseteq\undphi F$}

\noindent for every $G\in\Cal G$, so $H\subseteq\undphi F$, and
$H\setminus F\subseteq\undphi F\setminus F$ is negligible.   But as
$\mu$ is complete, this means that $H\in\Sigma$.   Also
$\undphi H=\undphi F\supseteq H$, so $H\in\frak T$.
Thus $\frak T$ is closed under arbitrary unions and is a topology.

\medskip

{\bf (b)} By its definition, $\frak T$ is included in $\Sigma$, so $\mu$
is a topological measure.   If $E\in\Sigma$ then $E\cap\undphi E$
belongs to $\frak T$, is
included in $E$ and has the same measure as $E$;  so $\mu$ is inner
regular with respect to the open sets.
As $\mu$ is semi-finite, it is inner regular with respect to the open sets
of finite measure, and is effectively locally finite.
If $E\in\frak T$ is non-empty,
then $\undphi E\supseteq E$ is non-empty, so $\mu E>0$;  thus $\mu$ is
strictly positive.   Finally, if $\Cal G$ is a
non-empty upwards-directed family
in $\frak T$, then the argument of (a-ii) shows that
$(\bigcup\Cal G)^{\ssbullet}=\sup_{G\in\Cal G}G^{\ssbullet}$ in
$\frak A$, so that
$\mu(\bigcup\Cal G)=\sup_{G\in\Cal G}\mu G$.   Thus $\mu$ is
$\tau$-additive.
}%end of proof of 414P

\leader{414Q}{Lifting topologies} Let $(X,\Sigma,\mu)$ be a measure
space and $\phi:\Sigma\to\Sigma$ a lifting\cmmnt{, that is, a Boolean
homomorphism such that $\phi E=\emptyset$ whenever $\mu E=0$ and
$\mu(E\symmdiff\phi E)=0$ for every $E\in\Sigma$ (341A)}.   The {\bf
lifting topology} associated with $\phi$ is the topology generated by
$\{\phi E:E\in\Sigma\}$.   \cmmnt{Note that $\{\phi E:E\in\Sigma\}$ is
a topology base, so is a base for the lifting topology.}

\leader{414R}{Proposition} Let $(X,\Sigma,\mu)$ be a complete locally
determined measure space and $\phi:\Sigma\to\Sigma$ a lifting with
lifting topology $\frak S$ and density topology $\frak T$.   Then
$\frak S\subseteq\frak T\subseteq\Sigma$, and $\mu$ is $\tau$-additive,
effectively locally finite and strictly positive with respect to
$\frak S$.   Moreover, $\frak S$ is zero-dimensional.

\proof{ Of course $\phi$ is a lower density, so we can talk of its
density topology, and since $\phi^2E=\phi E$, $\phi E\in\frak T$ for
every $E\in\Sigma$, so $\frak S\subseteq\frak T$.   Because $\mu$ is
$\tau$-additive and strictly positive with respect to $\frak T$, it must
also be $\tau$-additive and strictly positive with respect to $\frak S$.
If $E\in\Sigma$ and
$\mu E>0$ there is an $F\subseteq E$ such that $0<\mu F<\infty$, and now
$\phi F$ is an $\frak S$-open set of finite measure meeting $E$ in a
non-negligible set;
so $\mu$ is effectively locally finite with respect to $\frak S$.   Of
course $\frak S$ is
zero-dimensional because $\phi[\Sigma]$ is a base for $\frak S$
consisting of open-and-closed sets.
}%end of proof of 414R

\exercises{
\leader{414X}{Basic exercises $\pmb{>}$(a)}
%\spheader 414Xa
Let $(X,\Sigma,\mu)$ and $(Y,\Tau,\nu)$ be measure spaces with
topologies $\frak T$ and $\frak S$, and $f:X\to Y$ a continuous \imp\
function.   Show
that if $\mu$ is $\tau$-additive with respect to $\frak T$ then $\nu$ is
$\tau$-additive with respect to $\frak S$.   Show that if $\nu$ is
locally finite, so is $\mu$.
%414/
%useful in 452(?)

\spheader 414Xb Let $\familyiI{(X_i,\Sigma_i,\mu_i)}$ be a family of
measure spaces, with direct sum $(X,\Sigma,\mu)$;  suppose that we are
given a
topology $\frak T_i$ on $X_i$ for each $i$, and let $\frak T$ be the
disjoint union topology on $X$.   Show that $\mu$ is $\tau$-additive iff
every $\mu_i$ is.
%414/

\sqheader 414Xc Let $(X,\frak T)$ be a topological space and $\mu$ a
totally finite measure on $X$ which is inner regular with respect to the
closed sets.   Suppose that $\mu X=\sup_{G\in\Cal G}\mu G$ whenever
$\Cal G$ is an upwards-directed family of measurable open sets covering
$X$.   Show that $\mu$ is $\tau$-additive.
%414/

\spheader 414Xd Let $\mu$ be an effectively locally finite
$\tau$-additive $\sigma$-finite measure on a topological space $X$, and
$\nu:\dom\mu\to\coint{0,\infty}$ a countably additive functional which is
absolutely continuous with respect to $\mu$.   Show from first
principles that $\nu$ is $\tau$-additive.
%414D

\spheader 414Xe Give an example
of an indefinite-integral measure over Lebesgue measure on $\Bbb R$
which is not effectively locally finite.   \Hint{arrange for every
non-trivial interval to have infinite measure.}
%414D

\spheader 414Xf Let $(X,\frak T)$ be a topological space and $\mu$ a complete locally determined effectively locally finite
$\tau$-additive topological measure on $X$.
Show that if $f$ is a real-valued function, defined on a subset of $X$,
which is locally integrable in the sense of 411Fc, then $f$ is measurable.
%414E

\spheader 414Xg Let $(X,\frak T)$ be a topological space and $\mu$ an
effectively locally finite $\tau$-additive measure on $X$.   Let $\Cal
G$ be a cover of $X$ consisting of measurable open sets, and $\Cal K$
the ideal of subsets of $X$ generated by $\Cal G$.   Show that $\mu$ is
inner regular with respect to $\Cal K$.
%414I

\spheader 414Xh Let $(X,\frak T,\Sigma,\mu)$ be a complete locally
determined effectively locally finite $\tau$-additive topological
measure space, and $A$ a subset of $X$.   Suppose that for every $x\in A$
there is an open set $G$ containing
$x$ such that $A\cap G$ is negligible.   Show that $A$ is negligible.
%414I

\spheader 414Xi Give an alternative proof of 414K based on the fact
that the canonical map from the measure algebra of $\mu$ to the measure
algebra of $\mu_Y$ is order-continuous (322Yd).
%414K

\sqheader 414Xj(i) If $\mu$ is an effectively locally finite
$\tau$-additive Borel measure on a regular topological space, show that
the c.l.d.\ version of $\mu$ is a quasi-Radon measure.
(ii) If $\mu$ is a locally finite, effectively locally finite
$\tau$-additive Borel measure on a locally compact Hausdorff space, show
that $\mu$ is tight, so that
the c.l.d.\ version of $\mu$ is a Radon measure.
%414M

\sqheader 414Xk Let $(X,\Sigma,\mu)$ be a complete locally determined
measure space and $\undphi$ a lower density for $\mu$ such that
$\undphi X=X$;  let $\frak T$ be the corresponding density topology.
(i) Show that a
dense open subset of $X$ must be conegligible.   (ii) Show that a subset
of $X$ is nowhere dense for $\frak T$ iff it is negligible iff it is
meager for $\frak T$.   (iii) Show that a function $f:X\to\Bbb R$ is
$\Sigma$-measurable iff it is $\frak T$-continuous at almost every point
of $X$.   \Hint{if $f$ is measurable, set $E_q=\{x:f(x)>q\}$,
$F_q=\{x:f(x)<q\}$;  show that $f$ is continuous at every point of
$X\setminus\bigcup_{q\in\Bbb Q}((E_q\setminus\undphi E_q)
\cup(F_q\setminus\undphi F_q))$.}  (iv) Show that $\Sigma$ is both the
Borel $\sigma$-algebra of $(X,\frak T)$ and the Baire-property algebra of
$(X,\frak T)$.   (v) Show that $(X,\frak T)$ is a Baire space.
%414P

\spheader 414Xl Let $(X,\Sigma,\mu)$ be a complete locally determined
measure space and $\undphi:\Sigma\to\Sigma$ a lower density such that
$\undphi X=X$, with density topology $\frak T$.   Show that if
$A\subseteq X$ and $E$ is a measurable envelope of $A$ then the
$\frak T$-closure of $A$ is just
$A\cup(X\setminus\undphi(X\setminus E))$.
%414P

\spheader 414Xm Let $\mu$ be Lebesgue measure on $\BbbR^r$, $\Sigma$
its domain, $\intstar:\Sigma\to\Sigma$ lower Lebesgue density (341E) and
$\frak T$ the corresponding density topology.   (i) Show that $\frak T$ is
finer than the usual Euclidean topology of $\BbbR^r$.
(ii) Show that for any
$A\subseteq\Bbb R$, the closure of $A$ for $\frak T$ is just
$A\cup\{x:\limsup_{\delta\downarrow 0}
 \Bover{\mu^*(A\cap B(x,\delta))}{\mu B(x,\delta)}>0\}$, and the
interior is $A\cap\{x:\lim_{\delta\downarrow 0}
 \Bover{\mu_*(A\cap B(x,\delta))}{\mu B(x,\delta)}=1\}$.
%414P

\spheader 414Xn Let $(X,\Sigma,\mu)$ be a complete locally determined
measure space and $\undphi:\Sigma\to\Sigma$ a lower density such that
$\undphi X=X$;  let $\frak T$ be the associated density topology.   Let
$A$ be a subset of $X$ and $E$ a measurable envelope of $A$;  let
$\Sigma_A$ be
the subspace $\sigma$-algebra and $\mu_A$ the subspace measure on $A$.
(i) Show that we have a lower density
$\hbox{$\undphi$}_A:\Sigma_A\to\Sigma_A$ defined by setting
$\hbox{$\undphi$}_A(F\cap A)=A\cap\undphi(E\cap F)$ for every
$F\in\Sigma$.   (ii) Show that $\hbox{$\undphi$}_AA=A$ iff
$A\subseteq\undphi E$, and that in this case the density topology on $A$
derived from $\hbox{$\undphi$}_A$ is just the subspace topology.
%414P

\spheader 414Xo Let $(X,\Sigma,\mu)$ be a complete locally determined
measure space and $\phi:\Sigma\to\Sigma$ a lifting, with density
topology $\frak T$ and lifting topology $\frak S$.   (i) Show that

\Centerline{$\frak T=\{H\cap G:G\in\frak S,\,H$ is conegligible$\}
=\{H\cap\phi E:E\in\Sigma,\,H$ is conegligible$\}$.}

\noindent (ii) Show that if $A\subseteq X$ and $E$ is a measurable
envelope of $A$ then the $\frak T$-closure of $A$ is $A\cup\phi E$.
%414R

\spheader 414Xp Let $(X,\Sigma,\mu)$ be a complete locally determined
measure space and $\phi:\Sigma\to\Sigma$ a lifting;  let $\frak S$ be
its lifting topology.   Let $A$ be a subset of $X$ such that
$A\subseteq\phi E$ for some (therefore any) measurable
envelope $E$ of $A$.  Let $\Sigma_A$ be the
subspace $\sigma$-algebra and $\mu_A$ the subspace measure on $A$.   (i)
Show that we have a lifting $\phi_A:\Sigma_A\to\Sigma_A$ defined by
setting $\phi_A(F\cap A)=A\cap\phi F$ for every $F\in\Sigma$.   (ii)
Show that the lifting topology on $A$ derived from $\phi_A$ is just the
subspace topology.
%414R

\spheader 414Xq Let $(X,\Sigma,\mu)$ and $(Y,\Tau,\nu)$ be
complete locally determined
measure spaces and $f:X\to Y$ an \imp\ function.
Suppose that we have lower densities $\undphi:\Sigma\to\Sigma$ and
$\undpsi:\Tau\to\Tau$ such that $\undphi X=X$, $\undpsi Y=Y$ and
$\undphi f^{-1}[F]=f^{-1}[\undpsi F]$ for every $F\in\Tau$.   (i) Show that
$f$ is continuous for the density topologies of $\undphi$ and $\undpsi$.
(ii) Show that if $\undphi$ and $\undpsi$ are liftings then $f$ is
continuous for the lifting topologies.
%414R

\spheader 414Xr Let $(X,\Sigma,\mu)$ be a complete locally determined
measure space and $\phi:\Sigma\to\Sigma$ a lifting, with associated
lifting topology $\frak S$.   Show that a function $f:X\to\Bbb R$ is
$\Sigma$-measurable iff there is a conegligible set $H$ such that
$f\restr H$ is $\frak S$-continuous.   (Compare 414Xk, 414Xt.)
%414Xk, 414R

\spheader 414Xs Let $(X,\Sigma,\mu)$ be a complete locally determined
measure space and $\phi:\Sigma\to\Sigma$ a lifting.   Let $(Z,\Tau,\nu)$
be the Stone space of the measure algebra of $\mu$, and $f:X\to Z$ the
\imp\ function associated with $\phi$ (341P).   Show that the lifting
topology on $X$ is just $\{f^{-1}[G]:G\subseteq Z$ is open$\}$.
%414R

\spheader 414Xt Let $(X,\Sigma,\mu)$ be a strictly localizable measure
space and $\phi:\Sigma\to\Sigma$ a lifting.   Write $\eusm L^{\infty}$
for the Banach lattice of bounded $\Sigma$-measurable real-valued
functions on $X$, identified with $\eusm L^{\infty}(\Sigma)$ (363H);
let $T:\eusm L^{\infty}\to\eusm L^{\infty}$ be the Riesz homomorphism
associated with $\phi$ (363F).   (i) Show that $T^2=T$.  (ii) Show that
if $X$ is given the lifting topology $\frak S$ defined by $\phi$, then
$T[\eusm L^{\infty}]$ is precisely the space of bounded continuous
real-valued functions on $X$.   (iii) Show that if
$f\in\eusm L^{\infty}$, $x\in X$ and $\epsilon>0$ there is an
$\frak S$-open set $U$ containing $x$ such that
$|(Tf)(x)-\Bover1{\mu V}\int_Vfd\mu|\le\epsilon$ for every
non-negligible measurable set $V$ included in $U$.
%414R

\leader{414Y}{Further exercises (a)}
%\spheader 414Ya
Let $(X,\frak T,\Sigma,\mu)$ be a totally finite topological
measure space.   For $E\in\Sigma$ set

\Centerline{$\mu_{\tau}E
=\inf\{\sup_{G\in\Cal G}\mu(E\cap G):\Cal G\subseteq\frak T$
  is an upwards-directed set with union $X\}$.}

\noindent Suppose {\it either} that $\mu$ is inner regular with respect
to the closed sets {\it or} that $\frak T$ is regular.   Show that
$\mu_{\tau}$ is a $\tau$-additive measure, the
largest $\tau$-additive measure with domain $\Sigma$ which is dominated
by $\mu$.
%/

\spheader 414Yb Let $X$ be a set, $\Sigma$ an algebra of subsets of $X$,
and $\frak T$ a topology on $X$.   Let $M$ be the $L$-space of bounded
finitely additive real-valued functionals on $\Sigma$ (362B).   Let
$N\subseteq M$ be the set of those functionals $\nu$ such that
$\inf_{G\in\Cal G}|\nu|(H\setminus G)=0$ whenever
$\Cal G\subseteq\frak T\cap\Sigma$ is a non-empty upwards-directed
family with union
$H\in\Sigma$.   Show that $N$ is a band in $M$.   (Cf.\ 362Xi.)
%414Ya

\spheader 414Yc Find a probability space $(X,\Sigma,\mu)$ and a topology
$\frak T$ on $X$ such that $\Sigma$ includes a base for $\frak T$ and
$\mu$ is $\tau$-additive, but there is a set $E\in\Sigma$ such that the
subspace measure $\mu_E$ is not $\tau$-additive.
%414K %mt41bits

\spheader 414Yd Let $\intstar$ be lower Lebesgue density on $\BbbR^r$,
and $\frak T$ the associated density topology.   Show that every
$\frak T$-Borel set is an F$_{\sigma}$ set for $\frak T$.
%414P, 414Xm

\spheader 414Ye Let $(X,\rho)$ be a metric space and $\mu$ a strictly
positive locally finite quasi-Radon measure on $X$;  write $\frak T$ for
the topology of $X$ and $\Sigma$ for the domain of $\mu$.   For
$E\in\Sigma$ set $\undphi(E)=\{x:x\in X$, $\lim_{\delta\downarrow 0}
\Bover{\mu(E\cap B(x,\delta))}{\mu B(x,\delta)}=1\}$.   Suppose that
$E\setminus\undphi(E)$ is
negligible for every $E\in\Sigma$ (cf.\ 261D, 472D).   (i) Show that
$\undphi$ is a lower density for $\mu$, with $\undphi(X)=X$.   Let
$\frak T_d$ be the associated density topology.   (ii) Suppose that
$H\in\frak T_d$ and that $K\subseteq H$ is $\frak T$-closed and
$\rho$-totally bounded.   Show that there is a $\frak T$-closed,
$\rho$-totally bounded $K'\subseteq H$ such that $K$ is included in the
$\frak T_d$-interior of $K'$.   (iii) Show that $\frak T_d$ is
completely regular.   \Hint{{\smc Luke\v{s} Mal\'y \& Zaj\'\i\v{c}ek 86}.}
%414P mt41bits

\spheader 414Yf Show that the density topology on $\Bbb R$ associated
with lower Lebesgue density is not normal.
%414P 414Ye mt41bits

\spheader 414Yg Let $\mu$ be Lebesgue measure on $\BbbR^r$, $\Sigma$
its domain, $\intstar:\Sigma\to\Sigma$ lower Lebesgue density and
$\frak T$ the corresponding density topology.   (i) Show that if
$f:\Bbb R^r\to\Bbb R^r$ is a permutation such that $f$ and $f^{-1}$ are
both differentiable
everywhere, with continuous derivatives, then $f$ is a homeomorphism for
$\frak T$.   \Hint{263D.}  (ii) Show that if $\phi:\Sigma\to\Sigma$ is a
lifting and $\frak S$ the corresponding lifting topology, then
$x\mapsto -x$ is not a homeomorphism for $\frak S$.   \Hint{345Xc.}
%414R
}%end of exercises

\cmmnt{\Notesheader{414} I have remarked before that it is one of the
abiding frustrations of measure theory, at least for anyone ambitious to
apply the power of modern general topology to measure-theoretic
problems, that the basic convergence theorems are irredeemably
confined to
sequences.   In Volume 3 I showed that if we move to measure algebras
and function spaces, we can hope that the countable chain condition or
the countable sup property will enable us to replace arbitrary directed
sets with monotonic sequences, thereby giving theorems which apply to
apparently more general types of convergence.   In 414A and its
corollaries we come to a
quite different context in which a measure, or integral, behaves like an
order-continuous functional.   Of course the theorems here depend
directly on the hypothesis of $\tau$-additivity, which rather begs the
question;  but we shall see in the rest of the chapter that this
property does indeed often appear.   For the moment, I remark only that
as Lebesgue measure is $\tau$-additive we certainly have a non-trivial
example to work with.

The hypotheses of the results above move a touch awkwardly between those
with the magic phrase `topological measure' and those without.   The
point is that (as in 412G, for instance) it is sometimes useful to be
able to apply these ideas to Baire measures on completely regular
spaces, which are defined on a base for the topology but may not be
defined on every open set.   The device I have used in the definition of
$\tau$-additivity (411C) makes this possible, at the cost of occasional
paradoxical phenomena like 414Yc.

I hope that no confusion will arise between the two topologies
associated with a lifting on a complete locally determined space.   I
have called
them the `density topology' and the `lifting topology' because the
former can be defined directly from a lower density;  but it would be
equally reasonable to call them the `fine' and `coarse' lifting
topologies.   The
density topology has the apparent advantage of giving us a measure which
is inner regular with respect to the Borel sets, but at the cost of
being rather odd regarded as a topological space (414P, 414Xk, 414Ye,
414Yf).   It has the
important advantage that there are densities (like the Lebesgue lower
density) which have some claim to be called canonical, and others with
useful special properties, as in \S346, while liftings are always
arbitrary and invariance properties for them sometimes unachievable.
So, for instance, the Lebesgue density topology on $\BbbR^r$ is
invariant under diffeomorphisms, which no lifting topology can be
(414Yg).  The lifting topology is well-behaved as a topology, but only
in special circumstances (as in 453Xd) is the measure inner regular with
respect to its Borel sets, and even the closure of a set can be
difficult to determine.

As with inner regularity, $\tau$-additivity can be associated with the
band structure of the space of bounded additive functionals on an
algebra (414Yb);  there will therefore be corresponding decompositions
of measures into $\tau$-additive and `purely non-$\tau$-additive' parts
(cf.\ 414Ya).
}%end of notes

\discrpage

