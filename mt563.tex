\frfilename{mt563.tex}
\versiondate{3.12.13}
\copyrightdate{2008}

\def\chaptername{Choice and determinacy}
\def\sectionname{Borel measures without choice}

\newsection{563}

Having decided that a `Borel set' is to be one obtainable by
a series of operations described by a Borel code, it is a natural step to
say that a `Borel measure' should be one which respects these operations
(563A).   In regular spaces, such measures have strong inner and outer
regularity properties also based on the Borel coding (563D-563F), and we
have effective methods of constructing such measures (563H).   Analytic
sets are universally measurable (563I).   We can use similar ideas to give
a theory of Baire measures on general topological spaces (563J-563K).
In the basic case, of a second-countable space with a codably
$\sigma$-finite measure, we have a measure algebra with many of the same
properties as in the standard theory (563M-563N).

The theory would not be very significant if there were no interesting
Borel-coded measures, so you may wish to glance ahead to \S565 to confirm
that Lebesgue measure can be brought into the framework developed here.

\leader{563A}{Definitions (a)}\cmmnt{ ({\smc Foreman \& Wehrung 91})}
Let $X$ be a second-countable space and
$\Cal B_c(X)$ the algebra of codable Borel subsets of $X$.
I will say that a {\bf Borel-coded measure} on $X$ is a functional
$\mu:\Cal B_c(X)\to[0,\infty]$ such that
$\mu\emptyset=0$ and
$\mu(\bigcup_{n\in\Bbb N}E_n)=\sum_{n=0}^{\infty}\mu E_n$ whenever
$\sequencen{E_n}$ is a disjoint codable family in $\Cal B_c(X)$.

\cmmnt{I will try to remember to say `Borel-coded measure' everywhere
in this section, because these are dangerously different from the `Borel
measures' of \S434.   Their domains are not necessarily $\sigma$-algebras
and while they are finitely additive they need not be countably additive
even in the sense of 326I.}

\spheader 563Ab
\cmmnt{As usual, }I
will say that a subset of $X$ is {\bf negligible} if it is
included in a set of measure $0$\cmmnt{, which here must be a codable
Borel set;
the terms `conegligible', `almost everywhere', `null ideal'
will take their meanings from
this}.   We can now define the {\bf completion} of $\mu$ to be the
natural extension of $\mu$ to the algebra
$\{E\symmdiff A:E\in\Cal B_c(X)$, $A$ is $\mu$-negligible$\}$.

\spheader 563Ac\cmmnt{ Some of the other definitions from the ordinary
theory can be transferred without difficulty (e.g., `totally finite',
`probability'), but we may need to make some
finer distinctions.   For instance,} I will say that a Borel-coded measure
$\mu$ is {\bf semi-finite} if
$\sup\{\mu F:F\subseteq E$, $\mu F<\infty\}=\infty$ whenever
$\mu E=\infty$\cmmnt{;  we no longer have the ordinary
principle of exhaustion (215A), and the definition in 211F, taken
literally, may be too weak.   For `locally finite', however, 411Fa can be
taken just as it is, since all open sets are measurable}.

\spheader 563Ad \dvro{A}{For `$\sigma$-finite' we again have to make a
choice.
The definition in 211C calls only for `a sequence of measurable sets of
finite measure'.   Here the following will be more useful:  a} Borel-coded
measure on $X$ is {\bf codably $\sigma$-finite} if there is a codable
sequence
$\sequencen{E_n}$ in $\Cal B_c(X)$ such that $X=\bigcup_{n\in\Bbb N}E_n$
and $\mu E_n$ is finite for every $n$.

\leader{563B}{Proposition} Let $(X,\frak T)$
be a second-countable space and $\mu$ a Borel-coded measure on $X$.

(a) Let $\sequencen{E_n}$ be a codable sequence in $\Cal B_c(X)$.

\quad(i) $\mu(\bigcup_{n\in\Bbb N}E_n)\le\sum_{n=0}^{\infty}\mu E_n$.

\quad(ii) If $\sequencen{E_n}$ is non-decreasing,
$\mu(\bigcup_{n\in\Bbb N}E_n)=\lim_{n\to\infty}\mu E_n$.

\quad(iii) If $\sequencen{E_n}$ is non-increasing and $\mu E_0$ is finite,
then $\mu(\bigcap_{n\in\Bbb N}E_n)=\lim_{n\to\infty}\mu E_n$.

(b) $\mu$ is $\tau$-additive.

(c)\dvArevised{2013} Suppose that $\frak T$ is T$_1$.
If $\Cal E$ is the algebra of resolvable subsets of $X$\cmmnt{ (562H)}, 
then $\mu\restr\Cal E$ is countably additive in the sense that
$\mu E=\sum_{n=0}^{\infty}\mu E_n$ for any disjoint
family $\sequencen{E_n}$ in $\Cal E$ such that 
$E=\sup_{n\in\Bbb N}E_n$ is defined in $\Cal E$.

\proof{{\bf (a)(i)} Set $F_n=E_n\setminus\bigcup_{i<n}E_i$ for
$n\in\Bbb N$;  then $\sequencen{F_n}$ is codable (562Kc), so

\Centerline{$\mu(\bigcup_{n\in\Bbb N}E_n)
=\mu(\bigcup_{n\in\Bbb N}F_n)
=\sum_{n=0}^{\infty}\mu F_n
\le\sum_{n=0}^{\infty}\mu E_n$.}

\medskip

\quad{\bf (ii)} If $\sequencen{E_n}$ is non-decreasing then, in the
language of (i), $E_n=\bigcup_{i\le n}F_i$ for each $n$, so

\Centerline{$\lim_{n\to\infty}\mu E_n
=\lim_{n\to\infty}\sum_{i=0}^n\mu F_i
=\mu(\bigcup_{n\in\Bbb N}E_n)$.}

\medskip

\quad{\bf (iii)} Apply (ii) to $\sequencen{E_0\setminus E_n}$.
(As remarked in 562J, this will be a codable sequence.)

\medskip

{\bf (b)} Suppose that $\Cal G$ is an upwards-directed family of
open sets with union $H$.   Set $\gamma=\sup_{G\in\Cal G}\mu G$.   Let
$\sequencen{U_n}$ be a sequence running over a base for the topology of
$X$, and for $n\in\Bbb N$ set

\Centerline{$V_n
=\bigcup\{U_i:i\le n$, $U_i\subseteq G$ for some $G\in\Cal G\}$.}

\noindent Then $\sequencen{V_n}$ is a non-decreasing sequence of open sets
with union $H$.   As every $V_n$ is resolvable, $\sequencen{V_n}$ is
codable and

\Centerline{$\mu H=\lim_{n\to\infty}\mu V_n
\le\sup_{G\in\Cal G}\mu G\le\mu H$}

\noindent by (a-ii).

\medskip

{\bf (c)} If $\sequencen{E_n}$ is a disjoint sequence in $\Cal E$ with a
supremum $E$ in $\Cal E$, then $E\supseteq\bigcup_{n\in\Bbb N}E_n$.
If $x\in E$ then $\{x\}$ is closed,
because $\frak T$ is T$_1$, so $\{x\}$ is resolvable (562H) and
$E\setminus\{x\}\in\Cal E$;  as $E\setminus\{x\}$ is not an upper bound of
$\{E_n:n\in\Bbb N\}$, $x\in\bigcup_{n\in\Bbb N}E_n$.   So 
$E=\bigcup_{n\in\Bbb N}E_n$.
Now $\sequencen{E_n}$ is codable, as noted in
562J, so $\mu E=\mu(\bigcup_{n\in\Bbb N}E_n)=\sum_{n=0}^{\infty}\mu E_n$.
}%end of proof of 563B

\leader{563C}{Corollary} Let $X$ be a second-countable space, $\mu$ a
Borel-coded measure on $X$ and
$\sequencen{E_n}$ a sequence of resolvable sets in $X$.

(a)(i) $\bigcup_{n\in\Bbb N}E_n$ is measurable;

\quad(ii) $\mu(\bigcup_{n\in\Bbb N}E_n)\le\sum_{n=0}^{\infty}\mu E_n$;

\quad(iii) if $\sequencen{E_n}$ is disjoint,
$\mu(\bigcup_{n\in\Bbb N}E_n)=\sum_{n=0}^{\infty}\mu E_n$;

\quad(iv) if $\sequencen{E_n}$ is non-decreasing,
$\mu(\bigcup_{n\in\Bbb N}E_n)=\lim_{n\to\infty}\mu E_n$.

(b)(i) $\bigcap_{n\in\Bbb N}E_n$ is measurable;

\quad(ii) if $\sequencen{E_n}$ is non-increasing and
$\inf_{n\in\Bbb N}\mu E_n$ is finite, then
$\mu(\bigcap_{n\in\Bbb N}E_n)=\lim_{n\to\infty}\mu E_n$.

\proof{{\bf (a)} 
Use 562I to find a sequence of codes for $\sequencen{E_n}$,
and apply 563B.

\medskip

{\bf (b)} follows, because $X\setminus E_n$ is resolvable for each $n$.
}%end of proof of 563C

\leader{563D}{}\cmmnt{ The next lemma is primarily intended as a basis for
Theorem 563H, but it will be useful in 563F.

\medskip

\noindent}{\bf Lemma} Let $(X,\frak T)$
be a regular second-countable space and
$\mu:\frak T\to[0,\infty]$ a functional such that

\inset{$\mu\emptyset=0$,

$\mu G\le\mu H$ if $G\subseteq H$,

$\mu G+\mu H=\mu(G\cup H)+\mu(G\cap H)$ for all $G$, $H\in\frak T$,

$\mu(\bigcup_{n\in\Bbb N}G_n)=\lim_{n\to\infty}\mu G_n$ for every
non-decreasing sequence $\sequencen{G_n}$ in $\frak T$,

$\bigcup\{G:G\in\frak T$, $\mu G<\infty\}=X$.}

(a) $\mu(\bigcup_{i\in I}G_i)\le\sum_{i\in I}\mu G_i$ for every countable
family $\familyiI{G_i}$ in $\frak T$.

(b) There is a function $\pi^*:\frak T\times\Bbb N\to\frak T$ such that

\Centerline{$X\setminus G\subseteq\pi^*(G,k)$,
\quad$\mu(G\cap\pi^*(G,k))\le 2^{-k}$}

\noindent whenever $G\in\frak T$ and $k\in\Bbb N$.

(c) Let $\phi:\Cal T\to\Cal B_c(X)$ be an interpretation of Borel codes
defined from a sequence running over $\frak T$\cmmnt{, where
$\Cal T$ is the set of subtrees of $\bigcup_{n\in\Bbb N}\BbbN^n$ without
infinite branches (562A, 562B)}.   Then there are functions $\pi$,
$\pi':\Cal T\times\Bbb N\to\frak T$ such that

\Centerline{$\phi(T)\subseteq\pi(T,n)$,
\quad$X\setminus\phi(T)\subseteq\pi'(T,n)$,
\quad$\mu(\pi(T,n)\cap\pi'(T,n))\le 2^{-n}$}

\noindent for every $T\in\Cal T$ and $n\in\Bbb N$.

\proof{{\bf (a)} This is elementary.   First, $\mu(G\cup H)\le\mu G+\mu H$
for all open sets $G$ and $H$, because $\mu(G\cap H)\ge 0$.   Next,
if $\sequencen{G_n}$ is a sequence of open sets
with union $G$, then

\Centerline{$\mu G=\lim_{n\to\infty}\mu(\bigcup_{i\le n}G_i)
\le\lim_{n\to\infty}\sum_{i=0}^n\mu G_i
=\sum_{n=0}^{\infty}\mu G_n$.}

\noindent Now the step to general countable $I$ is immediate.

\medskip

{\bf (b)} Set $I=\{n:n\in\Bbb N$, $\mu U_n<\infty\}$;  because
$\mu$ is locally finite, $\{U_n:n\in I\}$ is a base for $\frak T$.
Given $G\in\frak T$ and $k\in\Bbb N$,
then for $n\in I$ and $m\in\Bbb N$ set

\Centerline{$W_{nm}=\bigcup\{U_i:i\le m$,
$\overline{U}_i\subseteq U_n\cap G\}$.}

\noindent Because $\frak T$ is regular,
$\bigcup_{m\in\Bbb N}W_{nm}=U_n\cap G$ and
$\mu(U_n\cap G)=\lim_{m\to\infty}\mu W_{nm}$.
Let $m_n$ be the least integer such that
$\mu W_{nm_n}\ge\mu(U_n\cap G)-2^{-k-n-1}$.   Set

\Centerline{$\pi^*(G,k)
=\bigcup_{n\in I}U_n\setminus\overline{W}_{nm_n}\in\frak T$.}

\noindent Because $\overline{W}_{nm}\subseteq U_n\cap G$ for all $m$ and
$n$, while $\bigcup_{n\in I}U_n=X$,
$\pi^*(G,k)\supseteq X\setminus G$.   Now

\Centerline{$\mu(G\cap\pi^*(G,k))
\le\sum_{n\in I}\mu(G\cap U_n\setminus\overline{W}_{nm_n})
\le\sum_{n=0}^{\infty}2^{-k-n-1}
=2^{-k}$,}

\noindent as required.

\medskip

{\bf (c)} Define $\pi(T)$ and $\pi'(T)$ inductively on the rank $r(T)$
of $T$.

\medskip

\quad{\bf (i)} If $r(T)=0$,
set $\pi(T,n)=\emptyset$ and $\pi'(T,n)=X$ for every $n$.
If $r(T)=1$ then $G=\phi(T)$ is open;  set $\pi(T,n)=G$ and
$\pi'(T,n)=\pi^*(G,n)$ for each $n$.

\medskip

\quad{\bf (ii)} For the inductive step to $r(T)>1$,
set $A_T=\{i:\fraction{i}\in T\}$
and 
$T_{\fraction i}=\{\sigma:\fraction{i}^{\smallfrown}\sigma\in T\}$ for
$i\in\Bbb N$, as in 562A.   Set

\Centerline{$\pi(T,n)=\bigcup_{i\in A_T}\pi'(T_{\fraction{i}},n+i+2)$,}

\Centerline{$\pi'(T,n)
=\bigcup_{i\in A_T}
(\pi(T_{\fraction{i}},n+i+2)\cap\pi'(T_{\fraction{i}},n+i+2))
  \cup\pi^*(\pi(T,n),n+1)$.}

\noindent Then

\Centerline{$\phi(T)
=\bigcup_{i\in A_T}X\setminus T_{\fraction{i}}
\subseteq\bigcup_{i\in A_T}\pi'(T_{\fraction{i}},n+i+2)
=\pi(T,n)$,}

$$\eqalign{X\setminus\phi(T)
&=\bigcap_{i\in A_T}\phi(T_{\fraction{i}})
\subseteq(\pi(T,n)\cap\bigcap_{i\in A_T}\phi(T_{\fraction{i}}))
   \cup\pi^*(\pi(T,n),n+1)\cr
&\subseteq\bigl(\bigcup_{i\in A_T}\pi'(T_{\fraction{i}},n+i+2)
     \cap\bigcap_{i\in A_T}\pi(T_{\fraction{i}},n+i+2)\bigr)
   \cup\pi^*(\pi(T,n),n+1)\cr
&\subseteq\bigcup_{i\in A_T}\bigl(\pi'(T_{\fraction{i}},n+i+2)
                                 \cap\pi(T_{\fraction{i}},n+i+2)\bigr)
   \cup\pi^*(\pi(T,n),n+1)\cr
&=\pi'(T,n),\cr}$$

$$\eqalignno{\mu(\pi(T,n)\cap\pi'(T,n))
&\le\sum_{i\in A_T}
  \mu(\pi(T_{\fraction{i}},n+i+2)\cap\pi'(T_{\fraction{i}},n+i+2))\cr
&\mskip200mu +\mu(\pi(T,n)\cap\pi^*(\pi(T,n),n+1))\cr
&\le\sum_{i\in A_T}2^{-n-i-2}+2^{-n-1}
\le 2^{-n}\cr}$$

\noindent for every $n$, so the induction continues.
}%end of proof of 563D

\leader{563E}{Lemma} Let $X$ be a second-countable space and $\Mu$ a
non-empty
upwards-directed set of Borel-coded measures on $X$.   For each codable
Borel set $E\subseteq X$, set $\nu E=\sup_{\mu\in\Mu}\mu E$.   Then
$\nu$ is a Borel-coded measure on $X$.

\proof{ Immediate from the definition in 563Aa.
}%end of proof of 563E

\leader{563F}{Proposition} Let $(X,\frak T)$ be a second-countable
space and $\mu$ a Borel-coded measure on $X$.

(a) For any $F\in\Cal B_c(X)$, we have a Borel-coded measure $\mu_F$ on $X$
defined by saying that $\mu_FE=\mu(E\cap F)$ for every $E\in\Cal B_c(X)$.

(b) We have a semi-finite Borel-coded measure $\mu_{\text{sf}}$ defined by saying
that

\Centerline{$\mu_{\text{sf}}(E)
=\sup\{\mu F:F\in\Cal B_c(X)$, $F\subseteq E$, $\mu F<\infty\}$}

\noindent for every $E\in\Cal B_c(X)$.

(c)(i) If $\mu$ is locally finite it is codably $\sigma$-finite.

\quad(ii) If $\mu$ is codably $\sigma$-finite, it is semi-finite and
there is a totally finite
Borel-coded measure $\nu$ on $X$ with the same null ideal as $\mu$.

\quad(iii) If $\mu$ is codably $\sigma$-finite, there is a non-decreasing
codable sequence of codable Borel sets of finite measure which covers $X$.

(d) If $X$ is regular then the following are equiveridical:

\quad(i) $\mu$ is locally finite;

\quad(ii) $\mu$ is semi-finite,
outer regular with respect to the open sets and inner regular with
respect to the closed sets;

\quad(iii) $\mu$ is semi-finite
and outer regular with respect to the open sets.

(e) If $X$ is regular and $\mu$ is semi-finite, then $\mu$
is inner regular with
respect to the closed sets of finite measure.

(f) If $X$ is Polish and $\mu$ is semi-finite, then $\mu$ is
inner regular with respect to the compact sets.

(g) If $\mu$ is locally finite, and $\nu$ is another Borel-coded measure on
$X$ agreeing with $\mu$ on the open sets, then $\nu=\mu$.

\proof{ Fix a sequence $\sequencen{U_n}$ running over a base for the
topology of $X$.

\medskip

{\bf (a)} The point is just that $\sequencen{E_n\cap F}$ is a
codable family whenever $\sequencen{E_n}$ is a codable family in
$\Cal B_c(X)$ and $F$ is codable.   (562J again.)

\medskip

{\bf (b)} Writing $\mu_F$ for the Borel-coded measure corresponding to a
set $F$ of finite measure, as in (a), we have an upwards-directed family of
measures;  by 563E, its supremum $\mu_{\text{sf}}$ is a Borel-coded measure.
If $E\subseteq X$ is a codable Borel set and $\gamma<\mu_{\text{sf}}E$, then
there is a set $F$ of finite measure such that $\mu(E\cap F)\ge\gamma$;
now

\Centerline{$\gamma\le\mu_{\text{sf}}(E\cap F)=\mu(E\cap F)<\infty$.}

\medskip

{\bf (c)(i)} Set $I=\{i:i\in\Bbb N$, $\mu U_i<\infty\}$;  then
$\familyiI{U_i}$ is a codable family of sets of finite measure covering
$X$.

\medskip

\quad{\bf (ii)} Let $\sequencen{H_n}$ be a codable sequence of sets of
finite measure covering $X$.

\medskip

\qquad\grheada\ If $E\in\Cal B_c(X)$, then
$\sequencen{E\cap\bigcup_{i\le n}H_i}$ is a non-decreasing codable sequence
with union $E$, so

\Centerline{$\mu E
=\sup_{n\in\Bbb N}\mu(E\cap\bigcup_{i\le n}H_i)
\le\sup\{\mu F:F\subseteq E$, $\mu F<\infty\}\le\mu E$.}

\noindent As $E$ is arbitrary, $\mu$ is semi-finite.

\medskip

\qquad\grheadb\ Let $\sequencen{\epsilon_n}$ be a sequence
of strictly positive real numbers such that
$\sum_{n=0}^{\infty}\epsilon_n\mu H_n$ is finite.   Set
$\nu E=\sum_{n=0}^{\infty}\epsilon_n\mu(E\cap H_n)$ for $E\in\Cal B_c(X)$.
Of course $\nu\emptyset=0$ and $\nu X<\infty$.
If $\sequence{k}{E_k}$ is a disjoint codable sequence in $\Cal B_c(X)$,
then $\sequence{k}{E_k\cap H_n}$ is codable for every $n$, so

$$\eqalign{\nu(\bigcup_{k\in\Bbb N}E_k)
&=\sum_{n=0}^{\infty}\epsilon_n\mu(\bigcup_{k\in\Bbb N}E_k\cap H_n)
=\sum_{n=0}^{\infty}\sum_{k=0}^{\infty}\epsilon_n\mu(E_k\cap H_n)\cr
&=\sum_{k=0}^{\infty}\sum_{n=0}^{\infty}\epsilon_n\mu(E_k\cap H_n)
=\sum_{k=0}^{\infty}\nu E_k.\cr}$$

\noindent So $\nu$ is a Borel-coded measure.

If $E\in\Cal B_c(X)$ and $\mu E=0$, then of course
$\nu E=\sum_{n=0}^{\infty}\epsilon_n\mu(E\cap H_n)=0$.   Conversely, if
$\nu E=0$, then $\mu(E\cap H_n)=0$ for every $n$;  but
$\sequencen{E\cap H_n}$, like $\sequencen{H_n}$, is codable, so
$\mu E=\mu(\bigcup_{n\in\Bbb N}E\cap H_n)=0$.   Thus $\mu$ and $\nu$ have
the same sets of zero measure;  it follows at once that they have the
same null ideals.

\medskip

\quad{\bf (iii)} All we have to note is that if $\sequencen{E_n}$ is
a codable sequence of sets of finite measure covering $X$, then
$\sequencen{\bigcup_{i\le n}E_i}$ is codable (562Kb), so gives the
required non-decreasing witness.

\medskip

{\bf (d)}{\bf (i)$\Rightarrow$(ii)}\grheada\  Observe first that
$\mu\restr\frak T$ satisfies the conditions of 563D.
\Prf\ The first three are consequences of the fact that
$\mu:\Cal B_c(X)\to[0,\infty]$ is additive.   If $\sequencen{G_n}$ is a
non-decreasing sequence in $\frak T$, it is a codable sequence of codable
Borel sets, by 562I as usual;  so
$\mu(\bigcup_{n\in\Bbb N}G_n)=\lim_{n\to\infty}\mu G_n$ by 563B(a-ii).
Finally, we are assuming that $\mu$ is locally finite, so the last
condition is satisfied.\ \Qed

Take an interpretation $\phi$ of Borel codes and
functions $\pi$, $\pi':\Cal T\times\Bbb N\to\frak T$ as in 563Dc.

\medskip

\quad\grheadb\ If $E\in\Cal B_c(X)$ and
$\mu E<\gamma$, take $n$ such that
$2^{-n}\le\gamma-\mu E$.   There is a $T\in\Cal T$ such that
$\phi(T)=E$, and now $G=\pi(T,n)$ is
open, $E\subseteq G$ and

\Centerline{$\mu(G\setminus E)\le\mu(G\cap\pi'(T,n))\le 2^{-n}$,}

\noindent so $\mu G\le\gamma$.

\medskip

\quad\grheadc\ If $E\in\Cal B_c(X)$ and $\gamma<\mu E$, take $T\in\Cal T$
such that $\phi(T)=E$ and $n\in\Bbb N$ such that $2^{-n}<\mu E-\gamma$;
now $F=X\setminus\pi'(T,n)$ is closed, $F\subseteq E$ and
$\mu(E\setminus F)\le 2^{-n}$, so $\mu F>\gamma$.   Next, if we set

\Centerline{$F_m
=F\cap\bigcup\{\overline{U_i}:i\le m$, $\mu\overline{U}_i<\infty\}$,}

\noindent $\sequence{m}{F_m}$ will be a non-decreasing sequence of closed
sets of finite measure with union $F$.   The sets $F_m$ are all resolvable,
so $\mu F=\lim_{m\to\infty}\mu F_m$ and there is an $m$ such that
$\mu F_m\ge\gamma$, while $F_m\subseteq E$ is a set of finite
measure.   As $E$ and $\gamma$ are arbitrary, $\mu$ is inner regular with
respect to the closed sets and also semi-finite.

\medskip

\quad{\bf(ii)$\Rightarrow$(iii)} is trivial.

\medskip

\quad{\bf(iii)$\Rightarrow$(i)}\grheada\ If $x\in X$ then $x$ belongs to
some set of finite measure.   \Prf\ Set

\Centerline{$F
=\bigcap\{U_n:n\in\Bbb N$, $x\in U_n\}\setminus\bigcup\{U_n:n\in\Bbb N$,
$x\notin U_n\}$.}

\noindent Then $F$ is a codable Borel set, being the difference of
G$_{\delta}$ sets (562Da), 
and the subspace topology on $F$ is indiscrete.   If $\mu F=0$
we can stop.   Otherwise, there must be an $F'\subseteq F$ such that
$0<\mu F'<\infty$;  but $F'\in\Cal B_c(F)=\{\emptyset,F\}$ (562E),
so $F'=F$ and
again $F$ has finite measure.\ \Qed

\medskip

\qquad\grheadb\ Now as $\mu$ is outer regular with respect to the open
sets, every set of finite measure is included in an open set of finite
measure.   So $\mu$ must be locally finite.

\medskip

{\bf (e)} Suppose that $E\in\Cal B_c(X)$ and
$\gamma<\mu E$.   Then there is an $H\in\Cal B_c(X)$ such that
$H\subseteq E$ and $\gamma<\mu H<\infty$.   Consider the Borel-coded
measure
$\mu_H$ defined from $\mu$ and $H$ as in (a).   This is totally finite,
so (d) tells us that it is outer regular with respect to the open sets and
therefore inner regular with respect to the closed sets,
and there is a closed set $F\subseteq H$ such that $\mu F=\mu_HF\ge\gamma$.
As $E$ and $\gamma$ are arbitrary, $\mu$ is inner regular with respect to
the closed sets of finite measure.

\medskip

{\bf (f)} Let $\rho$ be a complete
metric on $X$ inducing its topology.
If $E\in\Cal B_c(X)$ and $\gamma<\mu E$, let $F\subseteq E$ be a
closed set such that $F\subseteq E$ and $\gamma<\mu F<\infty$.
For each $n\in\Bbb N$ set $J_n=\{i:\diam U_i\le 2^{-n}\}$.
Define $\sequencen{k_n}$, $\sequencen{F_n}$ inductively by saying that
$F_0=F$ and

\Centerline{$k_n=\min\{k:\mu(F_n\cap\bigcup_{i\in J_n\cap k}U_i)>\gamma\}$,
\quad$F_{n+1}=F_n\cap\bigcup_{i\in J_n\cap k_n}\overline{U}_i$}

\noindent for each $n$;  set $K=\bigcap_{n\in\Bbb N}F_n\subseteq E$.   Then
$\mu K=\lim_{n\to\infty}\mu F_n\ge\gamma$.   The point is that $K$ is
compact.   \Prf\ Set $L=\prod_{n\in\Bbb N}J_n\cap k_n\subseteq\NN$.
Then $L$ is compact (561D).   Set 
$L'=\{\alpha:\alpha\in L$, 
$F\cap\bigcap_{i\le n}\overline{U}_{\alpha(i)}\ne\emptyset$ 
for every $n\}$;
then $L'$ is a closed subset of $L$ so is compact.   For $\alpha\in L'$,
$\{F\cap\overline{U}_{\alpha(i)}:i\in\Bbb N\}$ generates a filter 
$\Cal F_{\alpha}$ on $X$ which
is a Cauchy filter because $\diam\overline{U}_{\alpha(i)}
=\diam U_{\alpha(i)}\le 2^{-i}$ for every $i$;  because $X$ is
$\rho$-complete, $f(\alpha)=\lim\Cal F_{\alpha}$ is defined, and belongs to
$F\cap\bigcap_{i\in\Bbb N}\overline{U}_{\alpha(i)}\subseteq K$.   If
$\alpha$, $\beta\in L'$ and $\alpha(i)=\beta(i)$, then
$f(\alpha)$, $f(\beta)$ both belong to $\overline{U}_{\alpha(i)}$ so
$\rho(f(\alpha),f(\beta))\le 2^{-i}$;  thus $f$ is continuous and
$f[L']$ is a compact subset of $K$.
On the other hand, given $x\in K$, we can set
$\alpha(n)=\min\{i:i\in J_n\cap k_n$, $x\in\overline{U}_i\}$ for each
$n$, and now $\alpha\in L'$ and $f(\alpha)=x$.   So $K=f[L']$ is compact.
(See 561Yj.)\ \Qed

As $E$ and $\gamma$ are arbitrary, $\mu$ is inner regular with respect to
the compact sets.

\medskip

{\bf (g)(i)} Consider first the case in which $X$ is regular.   In this
case both $\mu$ and $\nu$ must be outer regular with respect to the open
sets, by (d);  as they agree on the open sets they must be equal.

\medskip

\quad{\bf (ii)} Next, suppose that $\mu X=\nu X$ is finite.
Let $\Cal E$ be the algebra of subsets of $X$
generated by $\{U_n:n\in\Bbb N\}$, and $\frak S$ the topology generated by
$\Cal E$.   As noted in the argument of 562Pb, $\frak S$ is codably Borel
equivalent to the original topology of $X$, so $\mu$ and $\nu$ are still
Borel-coded measures with respect to $\frak S$, and are still locally
finite, because $\frak S$ is finer than $\frak T$;  while $\frak S$ is
regular.   Now any member of $\Cal E$ is expressible in the form
$E=\bigcup_{i\le n}G_i\setminus H_i$ where the $G_i$, $H_i$ are open and
$\langle G_i\setminus H_i\rangle_{i\le n}$ is disjoint.   So

\Centerline{$\mu E=\sum_{i=0}^n\mu G_i-\mu(G_i\cap H_i)=\nu E$.}

\noindent More generally, if $H\in\frak S$, there is a non-decreasing
sequence $\sequencen{E_n}$
in $\Cal E$ with union $H$;  as all the sets in $\Cal E$ are resolvable,
$\sequencen{E_n}$ is codable and

\Centerline{$\mu H=\sup_{n\in\Bbb N}\mu E_n=\nu H$.}

\noindent Thus $\mu$ and $\nu$ agree on $\frak S$;  by (i), they are
equal.

\medskip

\quad{\bf (iii)} Finally, for the general case, set
$V_n=\bigcup\{U_i:i\le n$, $\mu U_i<\infty\}$ for each $n$.   Because $\mu$
is locally finite, $\bigcup_{n\in\Bbb N}V_n=X$.   For each $n\in\Bbb N$ let
$\mu_{V_n}$, $\nu_{V_n}$ be the Borel-coded measures defined from $V_n$ as
in (a).   Then $\mu_{V_n}$ and $\nu_{V_n}$ are totally finite and agree on
the open sets, so are equal.   Now $\sequencen{V_n}$, being a sequence of
open sets, is codable;  so if $E\in\Cal B_c(X)$ the sequence
$\sequencen{E\cap V_n}$ is codable, and

\Centerline{$\mu E=\lim_{n\to\infty}\mu(E\cap V_n)
=\lim_{n\to\infty}\mu_{V_n}E=\nu E$.}

\noindent So in this case also we have $\mu=\nu$.
}%end of proof of 563F

\vleader{72pt}{563G}{Proposition} Let $X$ be a set and
$\theta:\Cal PX\to[0,\infty]$ a submeasure.

(a)

\Centerline{$\Sigma
=\{E:E\subseteq X$, $\theta A=\theta(A\cap E)+\theta(A\setminus E)$
for every $A\subseteq X\}$}

\noindent is an algebra of subsets of $X$, and $\theta\restr\Sigma$ is
additive in the sense that $\theta(E\cup F)=\theta E+\theta F$ in
$[0,\infty]$ whenever $E$, $F\in\Sigma$ are disjoint.

(b) If $E\subseteq X$ and for
every $\epsilon>0$ there is an $F\in\Sigma$ such that $E\subseteq F$ and
$\theta(F\setminus E)\le\epsilon$, then $E\in\Sigma$.

\proof{{\bf (a)} Parts (a)-(c) of the proof of 113C apply unchanged.

\wheader{563G}{6}{2}{2}{36pt}
{\bf (b)} Take any $A\subseteq X$ and $\epsilon>0$.   Let $F\in\Sigma$ be
such that $E\subseteq F$ and $\theta(F\setminus E)\le\epsilon$.   Then

\Centerline{$\theta A\le\theta(A\cap E)+\theta(A\setminus E)
\le\theta(A\cap F)+\theta(A\setminus F)+\theta(F\setminus E)
\le\theta A+\epsilon$.}

\noindent As $\epsilon$ is arbitrary,
$\theta A=\theta(A\cap E)+\theta(A\setminus E)$;  as $A$ is arbitrary,
$E\in\Sigma$.
}%end of proof of 563G

\leader{563H}{Theorem} Let $(X,\frak T)$
be a regular second-countable space and
$\mu:\frak T\to[0,\infty]$ a functional such that

\inset{$\mu\emptyset=0$,

$\mu G\le\mu H$ if $G\subseteq H$,

$\mu G+\mu H=\mu(G\cup H)+\mu(G\cap H)$ for all $G$, $H\in\frak T$,

$\mu(\bigcup_{n\in\Bbb N}G_n)=\lim_{n\to\infty}\mu G_n$ for every
non-decreasing sequence $\sequencen{G_n}$ in $\frak T$,

$\bigcup\{G:G\in\frak T$, $\mu G<\infty\}=X$.}

\noindent Then $\mu$ has a unique extension to a Borel-coded measure on
$X$.

\proof{{\bf (a)} For $A\subseteq X$ set
$\theta A=\inf\{\mu G:A\subseteq G\in\frak T\}$.   Then $\theta$ is a
submeasure on $\Cal PX$ (because $\mu(G\cup H)\le\mu G+\mu H$ for all $G$,
$H\in\frak T$), extending $\mu$ (because $\mu G\le\mu H$ if
$G\subseteq H$).   Set

\Centerline{$\Sigma
=\{E:E\subseteq X$, $\theta A=\theta(A\cap E)+\theta(A\setminus E)$
for every $A\subseteq X\}$}

\noindent and $\nu=\theta\restr\Sigma$, as in 563G.
Let $\phi:\Cal T\to\Cal B_c(X)$ be an interpretation of Borel codes and
$\pi$, $\pi':\Cal T\times\Bbb N\to\frak T$ corresponding
functions as in 563Dc.
Now $\Cal B_c(X)\subseteq\Sigma$.   \Prf\
Given $T\in\Cal T$, $A\subseteq X$ and $n\in\Bbb N$,
let $G\in\frak T$ be such that $A\subseteq G$ and
$\mu G\le\theta A+2^{-n}$.   Then

$$\eqalign{\theta A
&\le\theta(A\cap\phi(T))+\theta(A\setminus\phi(T))
\le\theta(A\cap\pi(T,n))+\theta(A\cap\pi'(T,n))\cr
&\le\mu(G\cap\pi(T,n))+\mu(G\cap\pi'(T,n))\cr
&=\mu(G\cap(\pi(T,n)\cup\pi'(T,n))+\mu(G\cap\pi(T,n)\cap\pi'(T,n))\cr
&\le\mu G+\mu(\pi(T,n)\cap\pi'(T,n))
\le\theta A+2^{-n+1}.\cr}$$

\noindent As $A$ and $n$ are arbitrary, $\phi(T)\in\Sigma$.\ \Qed

\medskip

{\bf (b)} Let $\sequencen{T_n}$ be a sequence in $\Cal T$ such that
$\sequencen{E_n}$ is disjoint, where $E_n=\phi(T_n)$ for each $n$;  set
$E=\bigcup_{n\in\Bbb N}E_n$.
Then, for any $k\in\Bbb N$,
$E\subseteq\bigcup_{n\in\Bbb N}\pi(T_n,k+n)$, so

$$\eqalignno{\sum_{n=0}^{\infty}\nu E_n
&=\lim_{n\to\infty}\nu(\bigcup_{i\le n}E_i)
\le\nu E
\le\nu(\bigcup_{n\in\Bbb N}\pi(T_n,k+n))\cr
\displaycause{563Da}
&=\sum_{n=0}^{\infty}\nu E_n+\nu(\pi(T_n,k+n)\setminus E_n)\cr
\displaycause{563Ga}
&\le\sum_{n=0}^{\infty}\nu E_n+\mu(\pi(T_n,k+n)\cap\pi'(T_n,k+n))\cr
&\le\sum_{n=0}^{\infty}\nu E_n+2^{-k-n}
=2^{-k+1}+\sum_{n=0}^{\infty}\nu E_n;\cr}$$

\noindent as $k$ is arbitrary, $\sum_{n=0}^{\infty}\nu E_n=\nu E$;  as
$\sequencen{T_n}$ is arbitrary, $\nu\restr\Cal B_c(X)$ is a Borel-coded
measure extending $\mu$.

\medskip

{\bf (c)} At the same time we see that if $\lambda$ is any other
Borel-coded
measure extending $\mu$, we must have $\lambda E\le\theta E=\nu E$ for
every $E\in\Cal B_c(X)$.   In the other direction,

$$\eqalign{\lambda(\phi(T))
&\ge\lambda(\pi(T,n))-\lambda(\pi(T,n)\cap\pi'(T,n))\cr
&=\mu(\pi(T,n))-\mu(\pi(T,n)\cap\pi'(T,n))
\ge\nu(\phi(T))-2^{-n}\cr}$$

\noindent for every $T\in\Cal T$ and $n\in\Bbb N$, so $\lambda E\ge\nu E$
for every $E\in\Cal B_c(X)$.   Thus $\nu\restr\Cal B_c(X)$ is the only
Borel-coded measure extending $\mu$.
}%end of proof of 563H

\leader{563I}{Theorem} Let $X$ be a Hausdorff second-countable space,
$\mu$ a codably $\sigma$-finite Borel-coded measure on $X$, and
$A\subseteq X$ an analytic set.   Then there are a codable Borel set
$E\supseteq A$ and a sequence $\sequencen{K_n}$ of compact subsets
of $A$ such that $E\setminus\bigcup_{n\in\Bbb N}K_n$ is negligible.
Consequently $A$ is measured by the completion of $\mu$.

\proof{{\bf (a)} By 563F(c-ii),
there is a totally finite Borel-coded measure on
$X$ with the same negligible sets as $\mu$;  so it will be enough to
consider the case in which $\mu$ itself is totally finite.

If $A$ is empty, the result is trivial.   So we may suppose that
there is a continuous surjection $f:\NN\to A$.   For
$\sigma\in S=\bigcup_{n\in\Bbb N}\BbbN^n$ set
$I_{\sigma}=\{\alpha:\sigma\subseteq\alpha\in\NN\}$.
Fix on a sequence running over a base for the topology of $X$ and the
corresponding interpretation
$\phi:\Cal T\to\Cal B_c(X)$ of Borel codes.

\medskip

{\bf (b)} For $\sigma\in S$ and $\xi<\omega_1$ define $E_{\sigma\xi}$ by
saying that

\Centerline{$E_{\sigma 0}=\overline{f[I_{\sigma}]}$,}

\Centerline{$E_{\sigma,\xi+1}
=\bigcup_{i\in\Bbb N}E_{\sigma^{\smallfrown}\fraction{i},\xi}$,}

\Centerline{$E_{\sigma\xi}=\bigcap_{\eta<\xi}E_{\sigma\eta}$ if $\xi>0$ is
a countable limit ordinal.}

\noindent Then $\ofamily{\xi}{\omega_1}{E_{\sigma\xi}}$
is a non-increasing family of sets including $f[I_{\sigma}]$.

\medskip

{\bf (c)} For every $\xi<\omega_1$,
$\langle E_{\sigma\eta}\rangle_{\sigma\in S,\eta\le\xi}$ is a codable
family of codable Borel sets.   \Prf\ It is enough to consider the case
$\xi\ge\omega$.   Because $\xi$ is countable, we have a function
$\tilde\Theta_1:\bigcup_{J\subseteq\xi}\Cal T^J\to\Cal T$
such that $\phi(\tilde\Theta_1(\family{\eta}{J}{T_{\eta}}))
=\bigcup_{\eta\in J}\phi(T_{\eta})$ for every $J\subseteq\xi$ (562Cb).
Also, of course, we have a function $\Theta_0:\Cal T\to\Cal T$
such that $\phi(\Theta_0(T))=X\setminus\phi(T)$ for every $T\in\Cal T$.
Next, all the sets $E_{\sigma 0}$ are closed, therefore
resolvable.   So we have a family $\family{\sigma}{S}{T_{\sigma 0}}$ in
$\Cal T$ such that $\phi(T_{\sigma 0})=E_{\sigma 0}$ for every $\sigma$.
Now we can set

\Centerline{$T_{\sigma,\eta+1}
=\tilde\Theta_1(\sequence{i}{T_{\sigma^{\smallfrown}\fraction{i},\eta}})$}

\noindent if $\eta<\xi$,

\Centerline{$T_{\sigma\eta}
=\Theta_0(\tilde\Theta_1(\ofamily{\zeta}{\eta}
   {\Theta_0(T_{\sigma\zeta})}))$}

\noindent if $\eta\le\xi$ is a non-zero limit ordinal, and
$\phi(T_{\sigma\eta})$ will be equal to $E_{\sigma\eta}$ as required.\ \Qed

\medskip

{\bf (d)} Let
$\langle\epsilon_{\sigma}\rangle_{\sigma\in S}$ be a
summable family of strictly positive real numbers, and for $\xi<\omega_1$
set

\Centerline{$\gamma(\xi)
=\sum_{\sigma\in S}
  \epsilon_{\sigma}\mu(E_{\sigma\xi})$.}

\noindent Then $\gamma:\omega_1\to\Bbb R$ is non-increasing.   There is
therefore a $\xi<\omega_1$ such that $\gamma(\xi+1)=\gamma(\xi)$ (561A),
that is,
$\mu(E_{\sigma,\xi+1})=\mu(E_{\sigma\xi})$ for every $\sigma\in S$.

\medskip

{\bf (e)(i)} Set $E=E_{\emptyset\xi}$.
Of course $A=f[I_{\emptyset}]\subseteq E$.
Now define $\alpha_n\in\NN$, for $n\in\Bbb N$, as follows.   Given
$\ofamily{i}{m}{\alpha_n(i)}$, set

\Centerline{$G_{nm}
=\bigcup\{E_{\sigma\xi}:\sigma\in\BbbN^m$, $\sigma(i)\le\alpha_n(i)$
for every $i<m\}$,}

\Centerline{$G_{nmk}
=\bigcup\{E_{\sigma^{\smallfrown}\fraction{j},\xi}:
\sigma\in\BbbN^m$, $j\le k$, $\sigma(i)\le\alpha_n(i)$
for every $i<m\}$,}

\noindent Then $\langle G_{nm}\rangle_{n,m\in\Bbb N}$ is codable,
and $\lim_{k\to\infty}\mu G_{nmk}=\mu G_{nm}$ for all $m$,
$n\in\Bbb N$.   \Prf\ By (c), there is a family
$\langle T_{\sigma\eta}\rangle_{\sigma\in S,\eta\le\xi+1}$
in $\Cal T$ such
that $\phi(T_{\sigma\eta})=E_{\sigma\eta}$ whenever $\sigma\in S$ and
$\eta\le\xi+1$.
This time, we need a function
$\tilde\Theta_1:\bigcup_{J\subseteq S}\Cal T^J\to\Cal T$
such that $\phi(\tilde\Theta_1(\family{\sigma}{J}{T_{\sigma}}))
=\bigcup_{\sigma\in J}\phi(T_{\sigma})$ whenever
$J\subseteq S$ and $\family{\sigma}{J}{T_{\sigma}}$ is a family
in $\Cal T$, and a function $\Theta_3:\Cal T\times\Cal T\to\Cal T$ such
that $\phi(\Theta_3(T,T'))=\phi(T)\setminus\phi(T')$ for all $T$,
$T'\in\Cal T$.   Setting

\Centerline{$T'_{nm}
=\tilde\Theta_3(\langle T_{\sigma\xi}
  \rangle_{\sigma\in\BbbN^m,\,\sigma(i)\le\alpha_n(i)\Forall i<m})$,}

\Centerline{$T'_{nmk}
=\tilde\Theta_3(\langle T_{\sigma^{\smallfrown}\fraction{j},\xi}
  \rangle_{\sigma\in\BbbN^m,\,j\le k,\,\sigma(i)\le\alpha_n(i)
  \Forall i<m})$,}

\noindent we have $\phi(T'_{nm})=G_{nm}$ and $\phi(T'_{nmk})=G_{nmk}$ for
all $m$, $n$, $k\in\Bbb N$.   In particular, all the $G_{nm}$ and $G_{nmk}$
are codable Borel sets, and $\langle G_{nm}\rangle_{n,m\in\Bbb N}$ is
codable.   Moreover, for any particular pair $m$ and $n$,
$\sequence{k}{G_{nmk}}$ is a codable sequence;
we therefore have $\lim_{k\to\infty}\mu G_{nmk}=\mu G$, where
$G=\bigcup_{k\in\Bbb N}G_{nmk}$.   Next,

\Centerline{$G=\bigcup\{E_{\sigma,\xi+1}:\sigma\in\BbbN^m$,
$\sigma(i)\le\alpha_n(i)$ for every $i<m\}$,}

\noindent so

\Centerline{$G\symmdiff G_{nm}
\subseteq\bigcup\{E_{\sigma\xi}\setminus E_{\sigma,\xi+1}:
\sigma\in S\}$.}

\noindent Since
$\langle E_{\sigma\xi}\setminus E_{\sigma,\xi+1}
\rangle_{\sigma\in S}$ is a countable family of negligible sets coded by
$\langle\Theta_3(T_{\sigma\xi},T_{\sigma,\xi+1})\rangle_{\sigma\in S}$,
$G\symmdiff G_{nm}$ also is negligible and

\Centerline{$\mu G_{nm}=\mu G=\lim_{k\to\infty}\mu G_{nmk}$.\ \Qed}

\noindent Take the least $\alpha_n(m)\in\Bbb N$ such that
$\mu G_{n,m,\alpha_n(m)}\ge\mu G_{nm}-2^{-n-m}$, and continue.

\medskip

\quad{\bf (ii)} Set

\Centerline{$L_n
=\{\alpha:\alpha\in\NN$, $\alpha(i)\le\alpha_n(i)$ for every
$i\in\Bbb N\}$.}

\noindent Then $L_n$ is compact (561D), and $f[L_n]\subseteq A$.
Also $f[L_n]\supseteq\bigcap_{m\in\Bbb N}G_{nm}$.   \Prf\ If
$x\in\bigcap_{m\in\Bbb N}G_{nm}$, then for each $m\in\Bbb N$ let $\sigma_m$
be the lexicographically first member of
$\{\sigma:\sigma\in\Bbb N^m$, $\sigma(i)\le\alpha_n(i)$ for every $i<m\}$
such that $x\in E_{\sigma_m,\xi}$, and let $\beta_m\in\NN$ be such
that $\sigma_m\subseteq\beta_m$ and $\beta_m(i)=0$ for $i\ge m$.   Then
$\beta_m\in L_n$ for every $m$, so $\sequence{m}{\beta_m}$ has a cluster
point $\alpha\in L_n$.   \Quer\ If $f(\alpha)\ne x$, we have an open
neighbourhood $U$ of $f(\alpha)$ such that $x\notin\overline{U}$.
Let $m\in\Bbb N$ be such that $I_{\alpha\restr m}\subseteq f^{-1}[U]$;
then there is a $k\ge m$ such that
$\alpha\restr m=\beta_k\restr m=\sigma_k\restr m$.   Now

\Centerline{$x\in E_{\sigma_k,\xi}
\subseteq E_{\sigma_k,0}
\subseteq\overline{f[I_{\sigma_k}]}
\subseteq\overline{f[I_{\alpha\restr m}]}\subseteq\overline{U}$.
\Bang}

\noindent So $x=f(\alpha)\in f[L_n]$.\ \Qed

But $\sequence{m}{G_{nm}}$ is codable, and $G_{n0}=E$,
so we must have

$$\eqalign{\mu(E\setminus f[L_n])
&\le\mu(E\setminus G_{n0})
  +\sum_{m=0}^{\infty}\mu(G_{nm}\setminus G_{n,m+1})\cr
&=\sum_{m=0}^{\infty}\mu(G_{nm}\setminus G_{n,m,\alpha_n(m)})
\le\sum_{m=0}^{\infty}2^{-n-m}
=2^{-n+1}.\cr}$$

\noindent(Of course $f[L_n]$ is compact, therefore closed, therefore
measurable.)

\medskip

{\bf (f)} Set $K_n=f[L_n]$ for each $n$.   Then $\sequencen{K_n}$ is a
sequence of compact subsets of $A$;  because the $K_n$ are resolvable,
$F=\bigcup_{n\in\Bbb N}K_n$ is a codable Borel set.   For each $n$,

\Centerline{$\mu(E\setminus F)\le\mu(E\setminus K_n)
\le 2^{-n+1}$;}

\noindent so $E\setminus F$ is negligible.
Thus $E$ and $\sequencen{K_n}$ have the required properties.

Of course it now follows that $E\setminus A\subseteq E\setminus F$ is
negligible, so that the completion of $\mu$ measures $A$.
}%end of proof of 563I

\leader{563J}{Baire-coded measures} \cmmnt{Working from 562T,
we can develop a
theory of Baire measures on general topological spaces, as follows.}
If $X$ is a topological space, and $\CalBa_c(X)$ its algebra of
codable Baire sets,
a {\bf Baire-coded measure} on $X$ will be a
function $\mu:\CalBa_c(X)\to[0,\infty]$ such that $\mu\emptyset=0$ and
$\mu(\bigcup_{n\in\Bbb N}E_n)=\sum_{n=0}^{\infty}\mu E_n$ for every
disjoint codable sequence $\sequencen{E_n}$ in $\CalBa_c(X)$.

\vleader{72pt}{563K}{Proposition} (a)
If $X$ and $Y$ are topological spaces, $f:X\to Y$ is a
continuous function and $\mu$ is a Baire-coded measure on $X$, then
$F\mapsto\mu f^{-1}[F]:\CalBa_c(Y)\to[0,\infty]$ is a Baire-coded measure
on $Y$.

(b) Suppose that $\mu$ is a Baire-coded measure on a
topological space $X$, and $\sequencen{E_n}$ is a codable family in
$\CalBa_c(X)$.   Then

\quad(i) $\mu(\bigcup_{n\in\Bbb N}E_n)\le\sum_{n=0}^{\infty}\mu E_n$;

\quad(ii) If $\sequencen{E_n}$ is non-decreasing,
$\mu(\bigcup_{n\in\Bbb N}E_n)=\lim_{n\to\infty}\mu E_n$;

\quad(iii) If $\sequencen{E_n}$ is non-increasing and $\mu E_0$ is finite,
then $\mu(\bigcap_{n\in\Bbb N}E_n)=\lim_{n\to\infty}\mu E_n$.

(c) Let $X$ be a topological space and $\Mu$ a non-empty
upwards-directed family of Baire-coded measures on $X$.   Set
$\nu E=\sup_{\mu\in\Mu}\mu E$ for every codable Baire set $E\subseteq X$.
Then $\nu$ is a Baire-coded measure on $X$.

\proof{{\bf (a)} Use 562T(b-iv).

\medskip

{\bf (b)} Recall that, by 562T(b-i),
there must be a continuous function $f:X\to\BbbR^{\Bbb N}$ and a
codable sequence $\sequencen{F_n}$ in $\Cal B_c(\BbbR^{\Bbb N})$ such that
$E_n=f^{-1}[F_n]$ for every $n$.
By (a), $F\mapsto\mu f^{-1}[F]:\Cal B_c(\BbbR^{\Bbb N})\to[0,\infty]$ is a
Borel-coded measure on $\BbbR^{\Bbb N}$.   Applying 563Ba to
$\sequencen{F_n}$, we get the result here.

\medskip

{\bf (c)} As 563E.
}%end of proof of 563K

\leader{563L}{Proposition}
Suppose that $X$ is a topological space;  write $\Cal G$
for the lattice of cozero subsets of $X$.   Let
$\mu:\Cal G\to[0,\infty]$ be such that

\inset{$\mu\emptyset=0$,

$\mu G\le\mu H$ if $G\subseteq H$,

$\mu G+\mu H=\mu(G\cup H)+\mu(G\cap H)$ for all $G$, $H\in\Cal G$,

$\mu(\bigcup_{n\in\Bbb N}G_n)=\lim_{n\to\infty}\mu G_n$ whenever
$\sequencen{G_n}$ is a
non-decreasing sequence in $\Cal G$ and there is a sequence
$\sequencen{f_n}$ of continuous functions from $X$ to $\Bbb R$ such that
$G_n=\{x:f_n(x)\ne 0\}$ for every
$n$,\cmmnt{\footnote{\smallerfonts Observe that
$\bigcup_{n\in\Bbb N}G_n$ is a cozero set, defined by
$f:X\to\eightBbb{R}$ where
$f(x)=\sup_{n\in\Bbb N}\min(2^{-n},|f_n(x)|)$ for each $x$.}}

$\mu G=\sup\{\mu H:H\in\Cal G$, $H\subseteq G$, $\mu H<\infty\}$ for every
$G\in\Cal G$.
}

\noindent Then there is a Baire-coded measure on $X$ extending
$\mu$;  if $\mu X$ is finite, then the extension is unique.

\proof{{\bf (a)} Suppose to begin with that $\mu X$ is finite.

\medskip

\quad{\bf (i)} For
each continuous $f:X\to\BbbR^{\Bbb N}$, consider the
functional $G\mapsto\mu f^{-1}[G]$ for open $G\subseteq\BbbR^{\Bbb N}$.
This satisfies the conditions of 563H.   \Prf\ Only the fourth requires
attention.    Fix a metric $\rho$ defining the topology of
$\BbbR^{\Bbb N}$.   If $\sequencen{H_n}$ is a non-decreasing sequence
of open sets in $\BbbR^{\Bbb N}$ with union $H$, set

\Centerline{$h_n(z)=\min(1,\rho(z,\BbbR^{\Bbb N}\setminus H_n))$}

\noindent for $n\in\Bbb N$ and $z\in\BbbR^{\Bbb N}$, counting
$\rho(z,\emptyset)$ as $\infty$ if necessary.   In this case,
setting $G_n=f^{-1}[H_n]$, $G_n=\{x:h_nf(x)>0\}$ for each $n$;  so

\Centerline{$\mu f^{-1}[H]=\mu(\bigcup_{n\in\Bbb N}G_n)
=\lim_{n\to\infty}\mu G_n=\lim_{n\to\infty}\mu f^{-1}[H_n]$.  \Qed}

\noindent There is therefore a unique
Borel-coded measure $\nu_f$ on $\BbbR^{\Bbb N}$ such that
$\nu_fH=\mu f^{-1}[H]$ for every open set $G\subseteq\BbbR^{\Bbb N}$.

\medskip

\quad{\bf (ii)} If $f:X\to\BbbR^{\Bbb N}$ is continuous and
$F\in\Cal B_c(\BbbR^{\Bbb N})$, then
$\nu_fF=\inf\{\mu G:f^{-1}[F]\subseteq G\in\Cal G\}$.   \Prf\
By 563Fd, $\nu_f$ is outer regular with respect to the open sets, so

$$\eqalign{\nu_fF
&=\inf\{\nu_fH:H\subseteq\BbbR^{\Bbb N}\text{ is open}, F\subseteq H\}\cr
&=\inf\{\mu f^{-1}[H]:H\subseteq\BbbR^{\Bbb N}\text{ is open},
   F\subseteq H\}
\ge\inf\{\mu G:f^{-1}[F]\subseteq G\in\Cal G\}.\cr}$$

\noindent In the other direction, if $G\in\Cal G$ and
$f^{-1}[F]\subseteq G$, take any $\epsilon>0$.
There is an open set $H\subseteq\BbbR^{\Bbb N}$ such that
$\BbbR^{\Bbb N}\setminus H\subseteq F$ and
$\nu_f(F\cap H)\le\epsilon$.
But this means $G\cup f^{-1}[H]=X$ and

\Centerline{$\mu G\ge\mu X-\mu f^{-1}[H]
=\nu_f\BbbR^{\Bbb N}-\nu_fH\ge\nu_fF-\epsilon$.}

\noindent As $\epsilon$ is arbitrary, $\nu_fF\le G$;  as $G$ is arbitrary,
$\nu_fF\le\inf\{\mu G:f^{-1}[F]\subseteq G\in\Cal G\}$.\ \Qed

\medskip

\quad{\bf (iii)}
This means that if we set $\nu E=\inf\{\mu G:E\subseteq G\in\Cal G\}$ for
$E\in\CalBa_c(X)$, we shall have $\nu f^{-1}[F]=\nu_fF$ whenever
$f:X\to\BbbR^{\Bbb N}$ is continuous and $F\in\Cal B_c(X)$.
It follows that $\nu$ is a Baire-coded measure on $X$.   \Prf\ Of course
$\nu\emptyset=0$.
If $\sequencen{E_n}$ is a disjoint
codable sequence in $\CalBa_c(X)$, there are a continuous
$f:X\to\BbbR^{\Bbb N}$ and a codable sequence $\sequencen{F_n}$ of coded
Borel sets in $\BbbR^{\Bbb N}$ such that $E_n=f^{-1}[F_n]$ for every $n$,
by 562T(b-i).   Set $F'_n=F_n\setminus\bigcup_{i<n}F_i$ for $n\in\Bbb N$;
then $\sequencen{F'_n}$ is a codable sequence (562Kc), so

\Centerline{$\nu(\bigcup_{n\in\Bbb N}E_n)
=\nu_f(\bigcup_{n\in\Bbb N}F'_n)
=\sum_{n=0}^{\infty}\nu_fF'_n
=\sum_{n=0}^{\infty}\nu E_n$.  \Qed}

\noindent Of course $\nu$ extends $\mu$.

\medskip

\quad{\bf (iv)}
As for uniqueness, if $\nu'$ is any other Baire-coded measure on $X$
extending $\mu$, and $f:X\to\BbbR^{\Bbb N}$ is a continuous function, then
$F\mapsto\nu'f^{-1}[F]$ is a Borel-coded measure on
$\BbbR^{\Bbb N}$ which agrees
with $\nu_f$ on open sets and is therefore equal to $\nu_f$ (563Fg);  
it follows at once that $\nu'=\nu$.

\medskip

{\bf (b)} For the general case, let $\Cal G^f$ be
$\{H:H\in\Cal G$, $\mu H<\infty\}$, and for $H\in\Cal G^f$ define
$\mu_H:\Cal G\to\coint{0,\infty}$ by setting $\mu_HG=\mu(G\cap H)$ for
every $G\in\Cal G$.   Then $\mu_H$ satisfies all the conditions of the
proposition.   \Prf\ Everything is elementary;  for the hypothesis on
non-decreasing sequences in $\Cal G$, note that there is a continuous
function $f:X\to\Bbb R$ such that $H=\{x:f(x)\ne 0\}$, so that
if $\sequencen{f_n}$ is a
sequence of real-valued continuous function defining a sequence
$\sequencen{G_n}$ in $\Cal G$, then $\sequencen{f_n\times f}$ defines
$\sequencen{G_n\cap H}$.\ \Qed

There is therefore a unique Baire-coded measure $\nu_H$ on $X$ extending
$\mu_H$.   Now if $H$, $H'\in\Cal G^f$ and $H\subseteq H'$,
$\nu_HE=\nu_{H'}(E\cap H)$ for every $E\in\CalBa_c(X)$.   \Prf\ The
functional $E\mapsto\nu_{H'}(E\cap H)$ is a Baire-coded measure on $X$
extending $\mu_H$, so must be equal to $\nu_H$.\ \QeD\   In particular,
$\nu_HE\le\nu_{H'}E$ for every codable Baire set $E\subseteq X$.

Now set $\nu E=\sup\{\nu_HE:H\in\Cal G^f\}$ for $E\in\CalBa_c(X)$.
By 563Kc, $\nu$ is a Baire-coded measure on $X$;  and by the final
hypothesis of this proposition, $\nu$ extends $\mu$.
}%end of proof of 563L

\leader{563M}{Measure algebras} If $\mu$ is either a Borel-coded measure or
a Baire-coded measure, we can form the quotient Boolean algebra
$\frak A=\dom\mu/\{E:\mu E=0\}$ and the functional
$\bar\mu:\frak A\to[0,\infty]$ defined by setting
$\bar\mu E^{\ssbullet}=\mu E$ for every $E\in\dom\mu$;\cmmnt{ as in 321H,}
$\bar\mu$ is a strictly positive additive functional from $\frak A$ to
$[0,\infty]$.   As in \S323, we have a topology and
uniformity on $\frak A$ defined by the pseudometrics
$(a,b)\mapsto\bar\mu(c\Bcap(a\Bsymmdiff b))$ for $c\in\frak A$ of
finite measure;  if $\mu$ is semi-finite, the topology is Hausdorff.

\leader{563N}{Theorem} Let $X$ be a second-countable space, and $\mu$ a
codably $\sigma$-finite Borel-coded measure on $X$.   Let $\frak A$ and
$\bar\mu$ be as in 563M.   Then $\frak A$ is complete for its
measure-algebra uniformity, therefore Dedekind complete.

\proof{{\bf (a)} There is a codable sequence of sets of finite measure
covering $X$.   By 562Pb, we can find a codably Borel equivalent
second-countable topology $\frak S$
on $X$, generated by a countable algebra $\Cal E$ of subsets of $X$,
for which all these sets are open, so that $\mu$ becomes locally finite,
while $\Cal S$ is regular and second-countable.
Let $\sequencen{H_n}$ be a sequence running over $\Cal E$;  note that
$\sequencen{H_n}$ is codable.

\medskip

{\bf (b)} $\{H^{\ssbullet}:H\in\Cal E\}$ is dense in $\frak A$ for the
measure-algebra topology.   \Prf\ Suppose that $a$, $c\in\frak A$,
$\epsilon>0$ and $\bar\mu c<\infty$.   Express $a$ as $E^{\ssbullet}$
and $c$ as $F^{\ssbullet}$
where $E$, $F\in\Cal B_c(X)$.   By 563Fd, there is a $G\in\frak S$ such
that $E\cap F\subseteq G$ and $\mu(G\setminus(E\cap F))\le\epsilon$.   
Setting
$G_n=\bigcup\{H_i:i\le n$, $H_i\subseteq G\}$,  $\sequence{n}{G_n\cap F}$
is a non-decreasing codable sequence with union $G\cap F$, so there is an
$n\in\Bbb N$ such that $\mu((G\setminus G_n)\cap F)\le\epsilon$.   In this
case

\Centerline{$\bar\mu(c\Bcap(a\Bsymmdiff G_n^{\ssbullet}))
=\mu(F\cap(E\symmdiff G_n))
\le\mu(F\cap(G\setminus G_n))+\mu(G\setminus(E\cap F))
\le 2\epsilon$,}

\noindent while $G_n\in\Cal E$.   As $a$, $c$ and $\epsilon$ are arbitrary,
we have the result.\ \Qed

\medskip

{\bf (c)} $\frak A$ is complete for the measure-algebra uniformity.   \Prf\
Set $\tilde H_n=\bigcup\{H_i:i\le n$, $\mu H_i<\infty\}$,
$c_n=\tilde H_n^{\ssbullet}$ for each $n$.
Let $\Cal F$ be a Cauchy filter on $\frak A$ for the measure-algebra
uniformity.   For each $n\in\Bbb N$, there is an $A\in\Cal F$ such that
$\bar\mu(c_n\Bcap(a\Bsymmdiff b))\le 2^{-n}$ for all
$a$, $b\in A$;  there is a $b_0\in A$;  and there is an $m\in\Bbb N$ such
that $\bar\mu(c_n\Bcap(b_0\Bsymmdiff H_m^{\ssbullet}))\le 2^{-n}$, so
that

\Centerline{\hfill
$\{a:\bar\mu(c_n\Bcap(a\Bsymmdiff H_m^{\ssbullet}))\le 2^{-n+1}\}
\in\Cal F$.\hfill(*)}

\noindent Let $m_n$ be the first $m$ for which (*) is true, and set $d_n=H_{m_n}^{\ssbullet}$.   Note that

\Centerline{$\bar\mu(c_i\Bcap(d_{i+1}\Bsymmdiff d_i))\le 3\cdot 2^{-i}$}

\noindent for each $i$, because there must be an $a\in\frak A$ such that
$\bar\mu(c_i\Bcap(a\Bsymmdiff d_i))\le 2^{-i+1}$
and $\bar\mu(c_{i+1}\Bcap(a\Bsymmdiff d_{i+1}))\le 2^{-i}$.

Set $E=\bigcap_{n\in\Bbb N}\bigcup_{i\ge n}H_{m_i}$;  because
$\sequencen{\bigcup_{i\ge n}H_{m_i}}$ is codable, $E\in\Cal B_c(X)$.   Set
$d=E^{\ssbullet}$.   If $n\in\Bbb N$, then

\Centerline{$E\symmdiff H_{m_n}
\subseteq\bigcup_{i\ge n}H_{m_{i+1}}\symmdiff H_{m_i}$}

\noindent and $\sequence{i}{\tilde H_n\cap(H_{m_{i+1}}\symmdiff H_{m_i})}$
is codable, so

$$\eqalign{\bar\mu(c_n\Bcap(d\Bsymmdiff d_n))
&=\mu(\tilde H_n\cap(E\symmdiff H_{m_n}))
\le\sum_{i=n}^{\infty}\mu(\tilde H_n\cap(H_{m_{i+1}}\symmdiff H_{m_i}))\cr
&\le\sum_{i=n}^{\infty}3\cdot 2^{-i}
=6\cdot 2^{-n}.\cr}$$

Take any $c\in\frak A$ such that $\bar\mu c$ is finite, and $\epsilon>0$.
Express $c$ as $F^{\ssbullet}$, where $\mu F<\infty$.   Then
$\sequencen{F\cap\tilde H_n}$ is a non-decreasing codable sequence with
union $F$, so there is an $n\in\Bbb N$ such that
$\mu(F\setminus\tilde H_n)\le\epsilon$ and $2^{-n}\le\epsilon$.   Now

$$\eqalign{
\{a:\bar\mu(c\Bcap(a\Bsymmdiff d))\le 9\epsilon\}
&\supseteq\{a:\bar\mu(c_n\Bcap(a\Bsymmdiff d))\le 8\epsilon\}\cr
&\supseteq\{a:\bar\mu(c_n\Bcap(a\Bsymmdiff d_n))\le 2\epsilon\}
\in\Cal F.\cr}$$

\noindent As $c$ and $\epsilon$ are arbitrary, $\Cal F\to d$
for the measure-algebra topology;  as $\Cal F$ is arbitrary, $\frak A$ is
complete.\ \Qed

\medskip

{\bf (d)} Now suppose that $A\subseteq\frak A$ is a non-empty set,
and $B$ the family of its upper bounds, so that $B$ is downwards-directed.
As in 323D, the filter $\Cal F(B\closedownarrow)$ generated by
$\{B\cap[0,b]:b\in B\}$ is Cauchy for the measure-algebra uniformity,
so has a limit, which is
$\inf B=\sup A$.   As $A$ is arbitrary, $\frak A$ is Dedekind complete.
}%end of proof of 563N

\exercises{\leader{563X}{Basic exercises (a)}
%\spheader 563Xa
Let $X$, $Y$ be second-countable spaces, $\mu$ a
Borel-coded measure on $X$, and $f:X\to Y$ a codable Borel function.
Show that $F\mapsto\mu f^{-1}[F]:\Cal B_c(Y)\to[0,\infty]$ is a Borel-coded
measure on $Y$.
%563A

\spheader 563Xb Let $X$ be a regular second-countable space and $\mu$ a
locally finite Borel-coded measure on $X$.   Show that for every
$E\in\Cal B_c(X)$ there are an F$_{\sigma}$ set $F\subseteq E$ and a
G$_{\delta}$ set $H\supseteq E$ such that $\mu(H\setminus F)=0$.
%563F

\spheader 563Xc Let $X$ be a regular second-countable space.   Show that a
function $\mu$ is a codable Borel measure on $X$ iff it is a codable Baire
measure on $X$.   \Hint{562Xk, 562Xl.}
%563J

\spheader 563Xd Let $X$ be a topological space.   Show that any semi-finite
Baire-coded measure on $X$ is inner regular with respect to the zero
sets.
%563K

\spheader 563Xe\dvAnew{2014} 
Let $X$ be a zero-dimensional compact Hausdorff space, $\Cal E$ the 
algebra of open-and-closed subsets of $X$ and 
$\mu_0:\Cal E\to\coint{0,\infty}$ an additive functional.   
Show that there is a unique Baire-coded measure on $X$ extending $\mu_0$.
%563L

%\leader{563Y}{Further exercises (a)}
}%end of exercises

\leader{563Z}{Problem} Suppose we define `probability space' in the
conventional way, following literally the formulations in 111A, 112A and
211B.   Is it relatively consistent with ZF to suppose that every
probability space is purely atomic in the sense of 211K?

\endnotes{
\Notesheader{563} The arguments above are generally
drawn from those used earlier in this treatise;  the new discipline
required
is just to systematically respect the self-denying ordinance renouncing the
axiom of choice, as in part (f) of the proof of 563F.   
This does involve us in deeper analyses at a number of
points.   In 563Dc, for instance, we need functions $\pi$,
$\pi'$ defined on $\Cal T\times\Bbb N$, not $\Cal B_c(X)\times\Bbb N$,
because the rank function of $\Cal T$ gives us a foundation for induction.
(In 563Db we can use a function $\pi^*$ defined on $\frak T\times\Bbb N$,
but this is because we have canonical codes for open sets.)
In 563I we can no longer assume the existence of measurable envelopes, let
alone a whole family of them as used in the standard proof in 431A, and
have to find another construction, watching carefully to make sure that we
get not only a countable ordinal $\xi$ but a codable family of sets
$E_{\sigma\eta}$ leading to the measurable envelopes $E_{\sigma\xi}$;
back in 561A, there was a moment when we needed to resist the temptation to
suppose that a sequence in $\omega_1$ must have a supremum in $\omega_1$.

Note that we have to distinguish between `negligible' and `outer measure
zero'.   The natural meaning of the latter is
`for every $\epsilon>0$ there is a measurable set
$E\supseteq A$ with $\mu E\le\epsilon$'.   Even for outer regular
measures, when a set of outer measure zero must be included in open sets of
small measure, we cannot be sure that there is a sequence of such sets from
which we can define a set of measure zero including $A$ (565Xb).

In 563K I have kept the proofs short by quoting results from earlier in the
section.   But you may find it illuminating to look for a list of
properties of codable families of codable Baire sets which would support
formally independent proofs.

In 563M-563N I am taking care to avoid the phrase `measure algebra' in
the formal exposition.   The reason is that the definition in \S321
demands a Dedekind $\sigma$-complete algebra, and in the generality of
563M there is no reason to suppose that this will be satisfied.   In the
special context of 563N, of course, there is no difficulty.

There is something I ought to point out here.   The problem is not that the
principal arguments of \S\S111-113 and \S\S121-123 depend on the axiom of
choice.   If you wish, you can continue to define `$\sigma$-algebra',
`measure', `outer measure', `measurable function' and `integral' with the
same forms of words as used in Volume 1, and the basic theorems, up to and
including the convergence theorems, will still be true.   The problem is
that on these definitions the formulae of \S\S114-115 may not give an outer
measure, and we may have nothing corresponding to Lebesgue measure.   It
does not quite follow that every probability space is purely atomic
(there is a question here:  see
563Z), but clearly we are not going to get a theory which can respond to
any of the basic challenges dealt with in Volume 2
(Fundamental Theorem of Calculus, geometric measure
theory, probability distributions, Fourier series), and I think it more
useful to develop a new structure which can carry an effective
version of the Lebesgue theory (see \S565).
}%end of notes

\discrpage

