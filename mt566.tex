\frfilename{mt566.tex}
\versiondate{22.8.14}
\copyrightdate{2006}

\def\chaptername{Choice and determinacy}
\def\sectionname{Countable choice}

\def\spwidehat{\widehat{\phantom{m}}}

\newsection{566}

With AC($\omega$) measure theory becomes recognisable.   The
definition of Lebesgue measure used in Volume 1 gives us a true
countably additive Radon measure;  the most important divergence from the
standard theory is the possibility that every subset of $\Bbb R$ is
Lebesgue measurable\cmmnt{ (see 567G below)}.
With occasional exceptions (most
notably, in the theory of infinite products) we can use the work of Volume
2.   In Volume 3, we lose the two best theorems in the abstract
theory of measure algebras, Maharam's theorem and the Lifting Theorem;
but function spaces and ergodic theory are relatively unaffected.
Even in Volume 4, a good proportion of the ideas can be applied in some
form.

\cmmnt{\leader{566A}{} Nearly all
mathematicians working on the topics of this treatise
spend most of their time thinking in the framework of ZFC.   When we move
to weaker theories, we have a number of alternative strategies available.

\spheader 566Aa Some of the time, all we have to do is to check that our
previous arguments remain valid.   In the present context, moving from full
ZF + AC to ZF + AC($\omega$), this is true of
most of Volumes 1 and 2 and useful fragments thereafter.   In particular,
for most of the basic theory of the Lebesgue integral countable choice is
adequate.
Sometimes, of course, we have to trim our theorems back a bit\cmmnt{,
as in 566E, 566I, 566M, 566N, 566R and 566Xc}.

\spheader 566Ab Some results have to be dropped altogether.   For
instance, we no longer have a construction of a non-Lebesgue-measurable
subset of $\Bbb R$, and the Lifting Theorem disappears.

\spheader 566Ac Some results become so much weaker that they change their
character entirely.   For instance, the Hahn-Banach theorem, Baire's
theorem, Stone's theorem and Maharam's theorem survive only in sharply
restricted forms\cmmnt{ (561Xh, 561E, 561F, 566Nb)}.

\spheader 566Ad Sometimes we find that while proofs rely on the axiom of
choice, the results can be proved without it, or with something much
weaker.   Of course this is often a reason to regard the original
proof as inappropriate.
\cmmnt{Some of the ultrafilters in Volume 4 are there
just to save a couple of lines of argument, and renouncing them actually
brings ideas into clearer focus.}   But there are occasions when
the less scrupulous approach makes it a good deal easier for us to develop
appropriate intuitions.
\cmmnt{There is an example in the theory of the spaces
$S(\frak A)$ and $L^{\infty}(\frak A)$ in Chapter 36.   If we think of
$S(\frak A)$ as a quotient of a free linear space\cmmnt{ (361Ya)} and of
$L^{\infty}(\frak A)$ as the $\|\,\|_{\infty}$-completion of $S(\frak A)$,
we can prove all the basic results which come from their identification
with spaces of functions on the Stone space of $\frak A$;  but for most of
us such an approach would seriously complicate the process of understanding
the nature of the objects being constructed.   I used the representation
theorems in the theory of free products\cmmnt{ (\S315, \S325)} for
the same reason.}

On other occasions, we may need new ideas\cmmnt{,
as in 566F-566H, %566F 566G 566H
566L and 566P-566Q.   A deeper example is
in \cmmnt{562V/}566O\cmmnt{, where I set out
alternative routes to the results of 364F and 434T}.
Here we have quite a lot of extra distance to travel,
but at the same time we see some new territory}.

\spheader 566Ae More subtly, it may be useful to re-consider some
definitions\cmmnt{;  e.g., the distinction between `ccc' and `countable sup
property' for Boolean algebras\cmmnt{ (566Xd)}}.
I have made an effort in this book to use definitions which
will be appropriate in the absence of the axiom of choice, but in a number
of places this would lead to a division of a concept in potentially
confusing ways.

The ordinary theory of cardinals
depends so essentially on the existence of well-orderings that
it is often unclear what we can do without them.
However some theorems, which appear to involve the
theory of infinite cardinals, can be rescued if we re-interpret the
statements.
\cmmnt{Sometimes the cardinal $\frak c$ can be simply replaced by
$\Bbb R$ or $\Cal P\Bbb N$\cmmnt{ (343I, 491G)}.   Sometimes a statement
`$\#(X)\ge\frak c$' can be replaced by `there is
an injection from $\Cal P\Bbb N$ into $X$' or
`there is a surjection from $X$ onto
$\Cal P\Bbb N$'\cmmnt{ (344H, 4A2G(j-ii))};  %mt56bits
similarly, `$\#(X)\le\frak c$' might mean
`there is an injection from $X$ into $\Cal P\Bbb N$'
or `there is a surjection from $\Cal P\Bbb N$ onto
$X\cup\{\emptyset\}$'\cmmnt{ (4A1O, 4A3Fa)}.
Of course `$\#(X)=\frak c$' usually
becomes `there is a bijection between $X$ and
$\Cal P\Bbb N$'\cmmnt{ (423K)};  but
it might mean `there are an injection from $\Cal P\Bbb N$ into $X$ and a
surjection from $\Cal P\Bbb N$ onto $X$'\cmmnt{ (4A3Fb)},
or the other way round, or just two surjections.}

When dealing with a property which is invariant under
equipollence, it may be right to drop the concept of `cardinal' altogether,
and re-phrase a definition in more primitive terms\cmmnt{, as in 566Xl}.

Elsewhere\cmmnt{, as in 2A1Fd and 4A1E,}
we have results which refer to initial ordinals and hence to
well-orderable sets.   But the theory of cardinal functions is so bound up
with the idea that cardinal numbers form a well-ordered class that much
greater adjustments are necessary.
\cmmnt{I offer the following idea for
consideration.   For a metric space $(X,\rho)$ and a dense set
$D\subseteq X$, set

\Centerline{$\Cal U(X,\rho,D)=\{\{y:y\in X$, $\rho(x,y)<2^{-n}\}:
x\in D$, $n\in\Bbb N\}$,}

\noindent so that $\Cal U(X,\rho,D)$ is a base for the topology of $X$.
The existence (in ZF) of this function $\Cal U$
corresponds to the ZFC result that
`$w(X)$ is at most the cardinal product $\omega\times d(X)$ for every
metrizable space $X$'.}

\medskip

\spheader 566Af Another way to preserve the ideas of a theorem in the new
environment is to make some small variation in its hypotheses.
\cmmnt{For
instance, Urysohn's Lemma, in its usual form, demands DC.   So if we are
working with AC($\omega$) alone, we cannot be sure that compact Hausdorff
spaces are completely regular;  similarly, there may be uniformities
not definable from pseudometrics.   For a general topologist, this is
important.   But a measure theorist may be happy to simply add `completely
regular' to the hypotheses of a theorem\cmmnt{, as in 561G and 566Xk}.
In \S\S412-413 I repeatedly mention families $\Cal K$ which are closed
under disjoint finite unions.   Results starting from this hypothesis tend
to depend on DC;  but if we take $\Cal K$ to be closed under $\cup$,
AC($\omega$) may well be enough\cmmnt{ (566D)}.
A more dramatic change, but one which still leads to interesting results,
is in 566I.}
}%end of comment
%so assume all uniform sps, in particular all top gps, are c reg

\cmmnt{\leader{566B}{Volume 1} With countable choice,
Lebesgue outer measure becomes an outer measure in the usual sense, so we
can use \Caratheodory's method to define a measure space in the sense of
112A.   No further
difficulties arise in the work of Chapters 11 and 12, and we
can proceed exactly as before to the convergence theorems.
Indeed all the theorems of Volume 1 are available, with a single
exceptional feature:
the construction of non-measurable sets\cmmnt{ in 134B and 134D,} and a
non-measurable function\cmmnt{ in 134Ib}.   \cmmnt{(I will return
to this point in \S567.)}   In particular, the union of countably many
countable sets is countable.
}%end of comment

\cmmnt{\leader{566C}{Volume 2}
In Volume 2 also we find that arguments using more than countable choice
are the exception rather than the rule.   \cmmnt{Naturally,
they appear oftener in the more abstract topics of Chapter 21.}
One is in 211L;  we can no longer
be sure that a strictly localizable space is localizable, though a
$\sigma$-finite measure space does have to be localizable\cmmnt{,
since the choice demanded in the proof of 211Ld can then be performed
over a countable index set}.   There is a similar problem in 213J;
a strictly localizable space might fail to have locally determined
negligible sets, and might have a subset without a measurable
envelope.   Again\cmmnt{, in 214Ia}, it is not clear that a subspace of a
strictly localizable space must be strictly localizable.
\cmmnt{In 211P I ask for a non-Borel subset of
$\Bbb R$, and give an answer involving a non-measurable set;  but with
AC($\omega$) we have a non-Borel analytic set\cmmnt{ as in 423L}.
\cmmnt{(See also 566Xb.)}}

A more important gap arises in the theory of infinite products of
probability spaces.   The first problem is that if we have an
uncountable family $\familyiI{(X_i,\Sigma_i,\mu_i)}$ of probability spaces,
there is no assurance that $\prod_{i\in I}X_i$ is non-empty.
\cmmnt{In concrete
cases, this is not usually a serious worry.}   But there is another one.
\cmmnt{The proof of 254F makes an appeal to DC.}
I do not think that there can be
a construction of a product measure on even a sequence of arbitrary
probability spaces which does not use some form of dependent choice.
However a partial version, adequate for many purposes\cmmnt{ (including
the essential needs of Chapter 27)}, can be done with countable choice
alone\cmmnt{ (566I)}.
We can now continue through \S254 with the proviso that
every infinite family of probability spaces for which we consider a
product measure
should be a family of perfect probability spaces with non-empty
product.   There will be a difficulty in 254L, concerning the product of
subspaces of full outer measure\cmmnt{, where the modification
essentially confines it to non-empty products of conegligible sets}.
%so no 464C I think
%but maybe under AD, all semi-finite measure spaces perfect?
%495D might be fun to investigate, see 495F
\cmmnt{For 254N, it will be helpful to know that (under the conditions
of 566I)
the product of perfect spaces is again perfect.   \prooflet{The
proof of this fact
(451Jc) is scattered through Volume 4, but (given that we have a product
probability measure) needs only countably many choices at each step.}}

When we come to products of probability spaces in Chapter 27, we shall
again have to restrict the applications of the results, but at each point
only sufficiently to ensure that we have the product probability measures
discussed.
}%end of comment

\leader{566D}{\dvrocolon{Exhaustion}}\cmmnt{ The
versions of the principle of exhaustion in 215A all seem to require
DC rather than AC($\omega$).   For many applications, however, we can make
do with a weaker result, as follows.   I
include some corollaries showing that
in many familiar cases we can continue to use the intuitions developed in
the main text.

\medskip

\noindent}{\bf Proposition} [AC($\omega$)]
(a) Let $P$ be a partially ordered set such that
$p\vee q=\sup\{p,q\}$ is defined for all $p$, $q\in P$, and
$f:P\to\Bbb R$ an order-preserving function.    Then there is a
non-decreasing sequence $\sequencen{p_n}$ in $P$ such that
$\lim_{n\to\infty}f(p_n)=\sup_{p\in P}f(p)$.

(b) Let $(X,\Sigma,\mu)$ be a measure space and
$\Cal E\subseteq\Sigma$ a non-empty set such that
$\sup_{E\in\Cal E}\mu E$ is
finite and $E\cup F\in\Cal E$ for every $E$, $F\in\Cal E$.
Then there is a non-decreasing
sequence $\sequencen{F_n}$ in $\Cal E$ such that, setting
$F=\bigcup_{n\in\Bbb N}F_n$, $\mu F=\sup_{E\in\Cal E}\mu E$ and
$E\setminus F$ is negligible for every $E\in\Cal E$.

(c) Let $(X,\Sigma,\mu)$ be a measure space and
$\Cal K$ a family of sets such that

\inset{($\alpha$) $K\cup K'\in\Cal K$ for all $K$, $K'\in\Cal K$,

($\beta$) whenever $E\in\Sigma$ is non-negligible there is a
non-negligible $K\in\Cal K\cap\Sigma$ such that $K\subseteq E$.}

\noindent Then $\mu$ is inner regular with respect to $\Cal K$.

(d)(i) Let $(X,\Sigma,\mu)$ be a semi-finite measure space.
Then $\mu$ is inner regular with respect to the family of sets of finite
measure.

\quad(ii) Let $(X,\Sigma,\mu)$ be a perfect measure space.   Then whenever
$E\in\Sigma$, $f:X\to\Bbb R$ is measurable and $\gamma<\mu E$, there is a
compact set $K\subseteq f[E]$ such that $\mu f^{-1}[K]\ge\gamma$.

\proof{{\bf (a)} For each $n\in\Bbb N$, set
$\gamma_n=\sup_{p\in P}\min(n,f(p)-2^{-n})$.   Then there is a sequence
$\sequencen{q_n}$ in $P$ such that $f(q_n)\ge\gamma_n$ for each $n$;  set
$p_n=\sup_{i\le n}q_i$ for each $n$.

\medskip

{\bf (b)} By (a) there is a non-decreasing
sequence $\sequencen{F_n}$ in $\Cal E$ such that
$\sup_{n\in\Bbb N}\mu F_n=\sup_{E\in\Cal E}\mu E$;  set
$F=\bigcup_{n\in\Bbb N}F_n$.

\medskip

{\bf (c)} Because $\mu$ is inner regular with respect to $\Cal K$ iff it is
inner regular with respect to $\Cal K\cup\{\emptyset\}$, we may suppose
that $\emptyset\in\Cal K$.   Take $F\in\Sigma$, and consider
$\Cal E=\{K:K\in\Cal K\cap\Sigma$, $K\subseteq F\}$.   \Quer\ If
$\sup_{E\in\Cal E}\mu E<\mu F$, let $\sequencen{E_n}$ be a non-decreasing
sequence in $\Cal E$ such that $\mu(E\setminus\bigcup_{n\in\Bbb N}E_n)=0$
for every $E\in\Cal E$ ((b) above).
Set $G=\bigcup_{n\in\Bbb N}E_n$;  then
$\mu G=\sup_{n\in\Bbb N}\mu E_n<\mu F$, so $\mu(F\setminus G)>0$.
But now there ought to be a non-negligible $K\in\Cal K\cap\Sigma$ such that
$K\subseteq F\setminus G$, in which case $K\in\Cal E$ and
$\mu(K\setminus G)>0$.\ \Bang

\medskip

{\bf (d)(i)} Apply (c) with $\Cal K$ the family of sets of finite measure.

\medskip

\quad{\bf (ii)} Apply (c) to the subspace measure $\mu_E$ and
$\Cal K=\{f^{-1}[K]:K\subseteq f[E]$ is compact$\}$.
}%end of proof of 566D

\leader{566E}{}\cmmnt{ The problem recurs in parts of 215B, where
I list characterizations of $\sigma$-finiteness, and in 215C.   It seems
equally
that a ccc semi-finite measure algebra may fail to be $\sigma$-finite,
though a $\sigma$-finite measure algebra has to be ccc.
%322G
We have a stripped-down version of 215B, with one of its fragments used
in \S235, as follows:

\medskip

\noindent}{\bf Proposition} [AC($\omega$)]
Let $(X,\Sigma,\mu)$ be a semi-finite
measure space.   Write $\Cal N$ for the $\sigma$-ideal
of $\mu$-negligible sets.

(a) The following are equiveridical:

\quad(i) $\mu$ is $\sigma$-finite;

\quad(ii) either $\mu X=0$ or there is a probability measure
$\nu$ on $X$ with the same domain and the same negligible sets as $\mu$;

\quad(iii) there is a measurable integrable function $f:X\to\ocint{0,1}$;

\quad(iv) either $\mu X=0$ or there is a measurable function
$f:X\to\ooint{0,\infty}$ such that $\int fd\mu=1$.

(b) If $\mu$ is $\sigma$-finite, then

\quad(i) every disjoint family in $\Sigma\setminus\Cal N$ is countable;

\quad(ii) for every $\Cal E\subseteq\Sigma$ there is a countable
$\Cal E_0\subseteq\Cal E$ such that $E\setminus\bigcup\Cal E_0$ is
negligible for every $E\in\Cal E$.

(c) Suppose that $\mu$ is $\sigma$-finite, $(Y,\Tau,\nu)$ is a
semi-finite measure space, and $\phi:X\to Y$ is a
$(\Sigma,\Tau)$-measurable function such that $\mu\phi^{-1}[F]>0$ whenever
$\nu F>0$.   Then $\nu$ is $\sigma$-finite.

\proof{{\bf (a)} Use the methods of 215B.

\medskip

{\bf (b)} By (a-ii), we may suppose that $\mu$ is totally finite.

\medskip

\quad{\bf (i)} If $\Cal E\subseteq\Sigma\setminus\Cal N$ is disjoint, then
$\Cal E_n=\{E:E\in\Cal E$, $\mu E\ge 2^{-n}\}$ is finite for every $n$, so
$\Cal E=\bigcup_{n\in\Bbb N}\Cal E_n$ is countable.

\medskip

\quad{\bf (ii)} Let $\Cal H$ be the set of finite unions of members of
$\Cal E$.   By 566Db, there is a sequence $\sequencen{H_n}$ in $\Cal H$
such that $\mu(H\setminus\bigcup_{n\in\Bbb N}H_n)=0$ for every
$H\in\Cal H$.   For each $n\in\Bbb N$, choose a finite set
$\Cal H_n\subseteq\Cal E$ such that $H_n=\bigcup\Cal H_n$;  then
$\Cal E_0=\bigcup_{n\in\Bbb N}\Cal H_n$ has the required properties.

\medskip

{\bf (c)} Again, we may suppose that $\mu$ is totally finite.
For each $m\in\Bbb N$ let $\Cal H_m$ be the set of those $F\in\Tau$ such
that $\nu F<\infty$ and $\mu\phi^{-1}[F]\ge 2^{-m}$.   Then any disjoint
family in $\Cal H_m$ has at most $\lfloor 2^m\mu X\rfloor$ members,
so each $\Cal H_m$ has a maximal disjoint subset;  choose a sequence
$\sequence{m}{\Cal E_m}$ such that $\Cal E_m$ is a maximal disjoint subset
of $\Cal H_m$ for each $m$.   Then $\Cal E=\bigcup_{m\in\Bbb N}\Cal E_m$ is
a countable family of sets of finite measure in $Y$.   Now
$Z=Y\setminus\bigcup\Cal E$ is negligible.   \Prf\Quer\ Otherwise, there is a
non-negligible set $F$ of finite measure disjoint from $\bigcup\Cal E$;
now there is an $m$ such that $F\in\Cal H_m$, so that $\Cal E_m$ was not
maximal.\ \Bang\QeD\   So $\Cal E\cup\{Z\}$ witnesses that $\nu$ is
$\sigma$-finite.
}%end of proof of 566E

\leader{566F}{Atomless \dvrocolon{algebras}}\cmmnt{ To
make atomless measure spaces and
measure algebras recognisable, we need a more penetrating
argument than that previously used in 215D.

\medskip

\noindent}{\bf Lemma}\dvArevised{2014} [AC($\omega$)]
Let $\frak A$ be a Dedekind $\sigma$-complete Boolean algebra, and
$\mu$ a positive countably additive functional on $\frak A$ such that
$\mu 1=1$.   Suppose that whenever $a\in\frak A$ and $\mu a>0$ there is a
$b\Bsubseteq a$ such that $0<\mu b<\mu a$.
Then there is a function $f:\frak A\times[0,1]\to\frak A$ such that
$f(a,\alpha)\Bsubseteq a$ and $\bar\mu f(a,\alpha)=\min(\alpha,\bar\mu a)$
for $a\in\frak A$ and $\alpha\in[0,1]$, and
$\alpha\mapsto f(a,\alpha)$ is non-decreasing for every $a\in\frak A$.

\proof{{\bf (a)} Just as in part (a) of the proof of
215D, we see by induction on $n$
that for every $b\in\frak A$ such that $\mu b>0$ and every
$n\in\Bbb N$, there is a
$c\Bsubseteq b$ such that $0<\mu c\le 2^{-n}\mu b$.

\medskip

{\bf (b)} If $b\in\frak A$ and $\mu b>0$,
there is a $c\subseteq b$ such that
$\bover13\mu b<\mu c\le\bover23\mu b$.  \Prf\Quer\ Otherwise,
set $\gamma=\sup\{\mu c:c\subseteq b$, $\mu c\le\bover23\mu b\}$
and let
$\sequencen{c_n}$ be a sequence in $\frak A$ such that
$c_n\subseteq b$ and $\gamma-2^{-n}\le\mu c_n\le\gamma$ for every $n$.
Set $d_n=\sup_{i\le n}c_i$ for each $n$, and
$d=\sup_{n\in\Bbb N}d_n$.
Inducing on $n$, we see that $\mu d_n\le\bover23\mu b$ so
$\mu d_n\le\bover13\mu b$ for each $n$, and $\mu d\le\bover13\mu b$.
Now by (a) there is an $e\subseteq b\setminus d$ such that
$0<\mu e\le\bover13\mu b$.   In this case,
$\mu(d\cup e)\le\bover23\mu b$, so

\Centerline{$\gamma
\ge\mu(d\cup e)\ge\mu e+\sup_{n\in\Bbb N}\mu c_n>\gamma$.
\Bang\Qed}

\medskip

{\bf (c)} For each $n\in\Bbb N$
there is a finite partition of unity into
elements of measure at most $(\bover23)^n$.
\Prf\ Induce on $n$, using (b) for the inductive step.\ \Qed

\medskip

{\bf (d)} Choose a sequence $\sequence{k}{C_k}$ of finite partitions of
unity such that $\mu c\le 2^{-k}$ for every
$k\in\Bbb N$ and $c\in C_n$.   Set $C=\bigcup_{k\in\Bbb N}C_k$;  then
$C$ is countable.   Moreover, whenever $a\in\frak A$ and
$\beta>0$, there must be a $c\in C$ such that
$a\Bcap c\ne 0$ and $\mu c\le\beta$.
Let $\sequencen{c_n}$ be a sequence running over $C$.

\medskip

{\bf (e)} Define
$\sequencen{f_n}$, $\sequencen{g_n}$ inductively by saying that,
for $a\in\frak A$ and $\alpha\in[0,1]$,

\Centerline{$f_0(a,\alpha)=0$, \quad$g_0(a,\alpha)=a$}

$$\eqalign{f_{n+1}(a,\alpha)
&=f_n(a,\alpha)\Bcup(c_n\Bcap g_n(a,\alpha))
   \text{ if }\mu(f_n(a,\alpha)\Bcup(c_n\Bcap g_n(a,\alpha)))\le\alpha,\cr
&=f_n(a,\alpha)\text{ otherwise},\cr
g_{n+1}(a,\alpha)
&=g_n(a,\alpha)
   \text{ if }\mu(f_n(a,\alpha)\Bcup(c_n\Bcap g_n(a,\alpha)))\le\alpha,\cr
&=f_n(a,\alpha)\Bcup(c_n\Bcap g_n(a,\alpha))
   \text{ otherwise}.\cr}$$

\noindent Then

\Centerline{$f_n(a,\alpha)\Bsubseteq f_{n+1}(a,\alpha)
\Bsubseteq g_{n+1}(a,\alpha)\Bsubseteq g_n(a,\alpha)\Bsubseteq a$,}

\Centerline{$\mu f_n(a,\alpha)\le\alpha$,
\quad$\mu g_n(a,\alpha)\ge\min(\mu a,\alpha)$}

\noindent for every $n\in\Bbb N$.   Set
$f(a,\alpha)=\sup_{n\in\Bbb N}f_n(a,\alpha)$.   Then
$f(a,\alpha)\Bsubseteq a$ and $\mu f(a,\alpha)\le\alpha$ whenever
$a\in\frak A$ and $\alpha\in[0,1]$.

\medskip

{\bf (f)(i)} \Quer\ If $a\in\frak A$ and $\alpha\in[0,1]$ are such that
$\mu f(a,\alpha)<\min(\mu a,\alpha)$, set
$b=\inf_{n\in\Bbb N}g_n(a,\alpha)$.   Then $f(a,\alpha)\Bsubseteq b$ and
$\mu b\ge\min(\mu a,\alpha)$.   By (d), there is an $n\in\Bbb N$ such that

\Centerline{$c_n\Bcap b\Bsetminus f(a,\alpha)\ne 0$,
\quad$\mu c_n\le\min(\mu a,\alpha)-\mu f(a,\alpha)$.}

\noindent In this case,
$\mu(f_n(a,\alpha)\Bcup(c_n\Bcap g_n(a,\alpha)))\le\min(\mu a,\alpha)$
so

\Centerline{$f(a,\alpha)\Bsupseteq f_{n+1}(a,\alpha)
\Bsupseteq c_n\Bcap g_n(a,\alpha))
\Bsupseteq c_n\Bcap b$,}

\noindent which is impossible.\ \Bang

So $\mu f(a,\alpha)=\min(\mu a,\alpha)$ for all $a\in\frak A$ and
$\alpha\in[0,1]$.

\medskip

\quad{\bf (ii)} If $a\in\frak A$ and $0\le\alpha\le\beta\le 1$, then for
every $n\in\Bbb N$

\inset{{\it either} $f_n(a,\alpha)=f_n(a,\beta)$ and
$g_n(a,\alpha)=g_n(a,\beta)$

{\it or} $g_n(a,\alpha)\Bsubseteq f_n(a,\beta)$.}

\noindent(Induce on $n$.)   So $f(a,\alpha)\Bsubseteq f(a,\beta)$.

Thus we have a suitable function $f$.
}%end of proof of 566F

%216E looks problematic

\cmmnt{
\leader{566G}{Vitali's theorem}\cmmnt{ The arguments I presented
for Vitali's theorem
in 221A/261B and 471N-471O, and for the similar result in 472B, involve
the inductive construction of a sequence, which
ordinarily is a signal that DC is being used.   In 565F I suggested a
weaker form of Vitali's theorem which is adequate for its most important
applications in measure theory.}   With AC($\omega$)\cmmnt{, however,} we
can get most of the results as previously stated\cmmnt{, if
we refine our methods slightly}.

\cmmnt{
\spheader 566Ga In 261B, we have a family $\Cal I$ of closed balls
in $\BbbR^r$ and
we wish to choose inductively a disjoint sequence $\sequencen{I_n}$ in
$\Cal I$ such that

\Centerline{$\diam I_n
\ge\Bover12\sup\{\diam I:I\in\Cal I$, $I\cap\bigcup_{i<n}I_i=\emptyset\}$}

\noindent for every $n$.   We have already reduced the problem to the case
in which
$\sup_{I\in\Cal I}\diam I$ is finite and for any finite disjoint
subset of $\Cal I$ there is a member of $\Cal I$ disjoint from all of them.
Let $\sequence{m}{G_m}$ run over the family of all open balls with
centres in $\BbbQ^r$ and rational radii.   For $m\in\Bbb N$ set
$\Cal K_m=\Cal I\cap\Cal PG_m$, and let $\Cal I'\subseteq\Cal I$ be a
countable set such that
$\sup_{I\in\Cal I'\cap\Cal K_m}\diam I=\sup_{I\in\Cal K_m}\diam I$ for
every $m\in\Bbb N$ such that $\Cal K_m$ is non-empty;  this can be found
with countably many choices.

Now, when we come to choose $I_n$, we can always pick a member of
$\Cal I'$.   \Prf\ If
$\Cal I_n=\{I:I\in\Cal I$, $I\cap\bigcup_{i<n}I_i=\emptyset\}$,
$\gamma_n=\sup_{I\in\Cal I_n}\diam I$
and $I\in\Cal I_n$ is such that $\diam I>\bover12\gamma_n$, there is an
$m\in\Bbb N$ such that
$I\subseteq G_m\subseteq\BbbR^r\setminus\bigcup_{i<n}I_i$, in which case
there is an $I'\in\Cal I'\cap\Cal K_m$ such that
$\diam I'\ge\bover12\gamma_n$, and $I'$ is eligible to be $I_n$.\
\QeD\  Because $\Cal I'$ is well-orderable, we can set out a rule
for making these choices, and the argument can proceed as written, without
recourse to the devices of \S565.

\spheader 566Gb A similar trick can be used in 472B.   Here, given a
family $\Cal I$ of closed balls, we wish to choose a sequence
$\sequencen{B_n}$ in $\Cal I$ such that the centre of $B_n$ does not belong
to $\bigcup_{i<n}B_i$ and, subject to this, the diameter of $B_n$ is
nearly as large as it
could be.   This time, take $\Cal K_m$ to be the set of members of $\Cal I$
with centres in $G_m$, use countably many choices to find a countable set
$\Cal I'\subseteq\Cal I$ with adequately large intersections with every
$\Cal K_m$, and choose $\sequencen{B_n}$ from $\Cal I'$.

At the next step, in 472C, we have to do this repeatedly, but the same
method works;  in fact, we can work inside a fixed family $\Cal I'$
chosen as above.   (See 472Yd.)

\spheader 566Gc The version in 471N-471O is not manageable in quite the
same way.   If, however, we assume that the metric spaces there are
locally compact and separable, we can use the same idea as in (a) above
to limit our search to countable subfamilies of the given family $\Cal F$.
}%end of comment
}%end of comment

\leader{566H}{Bounded additive \dvrocolon{functionals}}\cmmnt{ We
come to another obstacle in the proof of 231E.   The
argument given there
relies on DC to show that a countably additive functional
is bounded.   But we can avoid this, at the cost of an
extra manoeuvre, as follows.

\medskip

\noindent}{\bf Lemma} [AC($\omega$)]
Let $\frak A$ be a Boolean algebra and $\nu:\frak A\to\Bbb R$ an additive
functional such that $\{\nu a_n:n\in\Bbb N\}$ is bounded
for every disjoint
sequence $\sequencen{a_n}$ in $\frak A$.   Then $\nu$ is bounded.

\proof{ \Quer\ Suppose, if possible, otherwise.   Then there is a
sequence $\sequencen{b_n}$ in $\Sigma$ such that $|\nu b_n|\ge 2^nn$ for
every $n\in\Bbb N$.   For each $n\in\Bbb N$, let $\frak B_n$ be the
subalgebra of $\frak A$
generated by $\{b_i:i<n\}$;  then $\frak B_n$ has
at most $2^n$ atoms, so there must be an atom $a$ of $\frak B_n$ such that
$|\nu(a\Bcap b_n)|\ge n$.   Choose a sequence $\sequencen{c_n}$ such that
$c_n$ is an atom of $\frak B_n$ and $|\nu d_n|\ge n$ for every $n$, where
$d_n=c_n\Bcap b_n$;  note that $d_n$ is an atom of $\frak B_{n+1}$,
so that if $n<m$ then
either $d_m\Bsubseteq d_n$ or $d_m\Bcap d_n=0$.   By Ramsey's
theorem (4A1G), there is an infinite $I\subseteq\Bbb N$ such that

\inset{{\it either} $\family{n}{I}{d_n}$ is disjoint

{\it or} $d_m\subseteq d_n$ whenever $m$, $n\in I$ and
$n<m$.}

\noindent Now the first alternative is certainly impossible, because
$\{\nu d_n:n\in I\}$ is unbounded.
So we have the second.   But in this case we can define a strictly
increasing sequence $\sequence{k}{n_k}$ in $I$ such that
$n_{k+1}\ge k+|\nu d_{n_k}|$ for each $k$.   Set
$a_k=d_{n_k}\Bsetminus d_{n_{k+1}}$ for each $k$;  then $\sequence{k}{a_k}$
is disjoint and $|\nu a_k|\ge k$ for each $k$, so again we have a
contradiction.\ \Bang
}%end of proof of 566H

\leader{566I}{Infinite products:  Theorem} [AC($\omega$)] Let
$\familyiI{(X_i,\Sigma_i,\mu_i)}$ be a family of perfect probability
spaces such that $X=\prod_{i\in I}X_i$ is non-empty.
Then there is a complete probability measure $\lambda$ on $X$ such that

(i) if $E_i\in\Sigma_i$ for every $i\in I$, and $\{i:E_i\ne X_i\}$ is
countable, then $\lambda(\prod_{i\in I}E_i)$ is defined and equal to
$\prod_{i\in I}\mu_iE_i$;

(ii) $\lambda$ is inner regular with respect to
$\Tensorhat_{i\in I}\Sigma_i$.

\proof{ The only point at which the construction in 254A-254F needs
re-examination is in the proof that the standard outer measure on $X$ gives
it outer measure $1$.

\medskip

{\bf (a)} I recall the definitions.   For a cylinder
$C=\prod_{i\in I}C_i$, set $\theta_0C=\prod_{i\in I}\mu_iC_i$;  for
$A\subseteq X$, set

\Centerline{$\theta A
=\inf\{\sum_{n=0}^{\infty}\theta_0C_n:C_n\in\Cal C
  \text{ for every }n\in\Bbb N$, $A\subseteq\bigcup_{n\in\Bbb N}C_n\}$;}

\noindent $\lambda$ will be the measure defined from $\theta$ by
\Caratheodory's method.

\Quer\ Suppose, if possible, that $\theta X<1$.
Then we have a sequence
$\sequencen{C_n}$ of cylinder sets, covering $X$, with
$\sum_{n=0}^{\infty}\theta C_n=1-2\epsilon$ where $\epsilon>0$.
Express each $C_n$ as
$\prod_{i\in I}E_{ni}$ where
$J_n=\{i:E_{ni}\ne X_i\}$ is finite;  let $J$ be the countable set
$\bigcup_{n\in\Bbb N}J_n$;
take $K=\#(J)$ (identifying $\Bbb N$ with $\omega$),
and a bijection $k\mapsto i_k:K\to J$.

For each $k\in K$ and $n\in\Bbb N$, set $L_k=\{i_j:j<k\}\subseteq J$ and
$\alpha_{nk}=\prod_{i\in I\setminus L_k}\mu_iE_{ni}$.
If $J$ is finite, $L_{\#(J)}=J$ and $\alpha_{n,\#(J)}=1$ for every
$n$.   We have $\alpha_{n0}=\theta_0C_n$ for each $n$, so
$\sum_{n=0}^{\infty}\alpha_{n0}=1-2\epsilon$.   For $n\in\Bbb N$,
$k\in K$ and $t\in X_{i_k}$ set

$$\eqalign{f_{nk}(t)&=\alpha_{n,k+1}\text{ if }t\in E_{n,i_k},\cr
&=0\text{ otherwise}.\cr}$$

\noindent Then

\Centerline{$\int f_{nk}d\mu_{i_k}
=\alpha_{n,k+1}\mu_{i_k}E_{ni_k}=\alpha_{nk}$.}

\medskip

{\bf (b)} For each $k\in K$, let $h_k:X_{i_k}\to\Bbb R$ be the Marczewski
functional defined by setting

\Centerline{$h_k(t)=\sum_{n=0}^{\infty}3^{-n}\chi E_{ni_k}(t)$}

\allowmorestretch{468}{
\noindent for $t\in X_k$.   Because $\mu_k$ is perfect, there is
for each $k\in K$ a compact set $Q\subseteq h_k[X_{i_k}]$
such that $\mu_{i_k}h_k^{-1}[Q]\ge 1-2^{-k}\epsilon$.
Choose $\family{k}{K}{Q_k}$ such that $Q_k\subseteq h_k[X_{i_k}]$ is
compact and
$\mu_{i_k}h_k^{-1}[Q_k]\ge 1-2^{-k}\epsilon$ for every $k\in K$.
Observe that if $k\in K$ and $n\in\Bbb N$ then
$f_{nk}=\alpha_{nk}\chi E_{ni_k}$ is of the form $\alpha_{nk}g_nh_k$ where
$g_n:\Bbb R\to[0,1]$ is continuous.
}

\medskip

{\bf (c)} Define non-empty sets
$F_k\subseteq H_k\subseteq X_{i_k}$ inductively, for $k\in K$,
as follows.   The inductive hypothesis will be that
$\sum_{n\in M_k}\alpha_{nk}\le 1-2^{-k+1}\epsilon$, where
$M_k=\{n:n\in\Bbb N$, $F_j\subseteq E_{ni_j}$ whenever $j<k\}$;
of course $M_0=\Bbb N$, so the induction starts.
Given that $k\in K$ and that

\Centerline{$1-2^{-k+1}\epsilon
\ge\sum_{n\in M_k}\alpha_{nk}
=\sum_{n\in M_k}\int f_{nk}d\mu_{i_k}
=\int(\sum_{n\in M_k}f_{nk})d\mu_{i_k}$,}

\noindent the set

\Centerline{$H_k=\{t:t\in X_{i_k}$,
  $\sum_{n\in M_k}f_{nk}(t)\le 1-2^{-k}\epsilon\}$}

\allowmorestretch{468}{
\noindent must have measure greater than $2^{-k}\epsilon$ and meets
$h_k^{-1}[Q_k]$.   But observe
that $\sum_{n\in M_k}f_{nk}=g'_kh_k$ where
$g'_k=\penalty-100\sum_{n\in M_k}\alpha_{nk}g_n$ is lower semi-continuous, so that
$H_k=h_k^{-1}[G_k]$ where
$G_k=\{\alpha:g'_k(\alpha)\le 1-2^{-k}\epsilon\}$ is closed.
Since $H_k$ meets $h_k^{-1}[Q_k]$, $Q_k\cap G_k$ is non-empty and has a
least member $\beta_k$;  set $F_k=h_k^{-1}[\{\beta_k\}]$.
Because $Q_k\subseteq h_k[X_{i_k}]$, $F_k$ is non-empty.
}

Examine

\Centerline{$M_{k+1}=\{n:n\in M_k$, $F_k\subseteq E_{ni_k}\}$.}

\noindent There certainly is some $t^*\in F_k$, and because $h_k\restr F_k$
is constant, $M_{k+1}=\{n:n\in M_k$, $t^*\in E_{ni_k}\}$.   In this case

\Centerline{$\sum_{n\in M_{k+1}}\alpha_{n,k+1}
=\sum_{n\in M_k}f_{nk}(t^*)\le 1-2^{-k}\epsilon$}

\noindent and the induction proceeds.

\medskip

{\bf (d)} At the end of the induction, either finite or infinite,
choose $t_k\in F_k$ for $k\in K$.   We are supposing that $X$ has a member
$x^*$;  define $x\in X$ by setting $x(i_k)=t_k$ for $k\in K$ and
$x(i)=x^*(i)$ for $i\in I\setminus J$.   Then there is supposed to be an
$m\in\Bbb N$ such that $x\in C_m$, so that $m\in M_k$ for every $k$.
But at some stage we shall have
$J_m\subseteq L_k$ (allowing $k=\#(K)$ if $K$ is finite) and
$\alpha_{mk}=1$, which is impossible.\ \Bang
}%end of proof of 566I

\leader{566J}{}\cmmnt{ In
particular, 566I applies to all products $\{0,1\}^I$ and
$[0,1]^I$ with their usual measures.   For these we have Kakutani's theorem
that the usual measures are topological measures (415E),
which turns out to be valid with countable choice alone.

\medskip

\noindent}{\bf Theorem} [AC($\omega$)]
(a) Let $\familyiI{(X_i,\frak T_i,\Sigma_i,\mu_i)}$ be a
family of metrizable Radon probability spaces such that every $\mu_i$ is
strictly positive and $X=\prod_{i\in I}X_i$ is non-empty.   
Then the product measure on $X$ is a quasi-Radon measure.

(b) If $I$ is well-orderable then the product measure on $\{0,1\}^I$ is a
completion regular Radon measure.

\proof{{\bf (a)(i)} We had better check immediately that every $X_i$ is
separable.   The point is that because $\mu_i$ is a
totally finite measure inner regular with respect to the compact sets,
there is a conegligible K$_{\sigma}$ set;  because $\mu_i$ is strictly
positive, this is dense;  and countable choice is enough to ensure that a
compact metrizable space is second-countable, therefore separable.
It follows that $\prod_{i\in J}X_i$ is separable, therefore
second-countable, for every countable $J\subseteq I$.

\medskip

\quad{\bf (ii)} Because every $\mu_i$ is a Radon measure it is perfect,
so we
have a product probability measure on $X$.   Now we can repeat the argument
of 416Ua.

\medskip

{\bf (b)} Put (a), 561D and 416G together.
}%end of proof of 566J

\cmmnt{\leader{566K}{Volume 3} Turning to the concerns of Volume 3,
the elementary theory
of measure algebras is not radically changed.   But Lemma 311D is
hopelessly lost;  we no longer have Stone
spaces, and need to re-examine any proof which appears to rely on them.
Another result which changes is 313K;  order-dense sets\cmmnt{, as
defined in
313J,} need no longer give rise to partitions of unity.   So a localizable
measure algebra does not need to be isomorphic to a simple product of
totally finite measure algebras.   Similarly,
condition (ii) of 316H is no longer sufficient to prove weak
$(\sigma,\infty)$-distributivity.   However some of the
constructions which I described in terms of Stone spaces, in particular,
the Loomis-Sikorski theorem, the
Dedekind completion of a Boolean algebra, the localization of a
semi-finite measure algebra, free products and measure-algebra free
products, can be done by other methods which remain effective with
AC($\omega$) at most\cmmnt{;  see 566L, 561Yg, 323Xh and 325Yc}.

The theory of ccc algebras is rather different\cmmnt{ (566M, 566Xd)}.
Maharam's theorem\cmmnt{ (331I, 332B)} is surely unprovable
without something like the full axiom of choice;  and the Lifting
Theorem\cmmnt{ (341K)} is
equally inaccessible under the rules of this section.    We do
however have useful special cases of results in Chapters 33 and
34\cmmnt{ (566N)}.
%any hope for 332Q?

A good start can be made on the elementary
theory of Riesz spaces without any form of the axiom of
choice\cmmnt{ (see 561H)}, and with
AC($\omega$) we can go a long way\cmmnt{, as in 566Q}.   What is
missing is the Hahn-Banach theorem (for non-separable spaces)
and many representation theorems.   Similarly, the function spaces of
Chapter 36 are recognisable, provided that (for general Boolean algebras
$\frak A$)
we think of $S(\frak A)$ as a quotient space of the free linear space
generated by $\frak A$, and of
$L^{\infty}(\frak A)$ as the $\|\,\|_{\infty}$-completion of
$S(\frak A)$.   \cmmnt{Of
course we have to take care at every point to avoid the
use of Stone spaces.}   \cmmnt{One
place at which this involves us in a new
argument is in 566O.}   Most of the arguments of Chapter 24 remain valid,
so the basic theory of $L^p$ spaces in \S\S365-366 survives.
What is perhaps
surprising is that if we take the trouble we can still reach the most
important results on weak compactness (566P, 566Q).

In the ergodic theory of Chapter 38, a good proportion of the classical
results survive.   There are difficulties with some of the extensions
of the classical theory in \S\S381-382.   For instance, the definition of
`full subgroup' of the group of automorphisms of a Boolean
algebra\cmmnt{ in 381Be} assumes that order-dense sets include
partitions of unity.   \cmmnt{If not,
this definition may fail to be equivalent to
the formulation in 381Ia.   The latter would seem to be the more natural
one to use.}   However, the definition as given seems to work for the
principal needs of Chapter 38\cmmnt{ (see 381I)}.

Frol\'\i k's theorem\cmmnt{ in the generality 382D-382E} needs
something approaching AC, and with AC($\omega$) alone
there seems no hope of getting results for
general Dedekind complete algebras along the lines of the main theorems of
\S382.   For measurable algebras, however, we do have a version of
382Eb\cmmnt{ (566R)}.

Many of the later results of Chapter 38 are equally robust,
at least in their leading applications to measure algebras.   We
have to remember that we do not know that measurable algebras have many
involutions\cmmnt{, and
even among those which do there is no assurance that 382Q
will be true}.   So in \S\S383-384 we find ourselves restricted rather
further, to those measurable algebras in which every
non-zero element is the support of an involution;  but these include the
standard examples (566N).
}%end of comment

\leader{566L}{The Loomis-Sikorski theorem} [AC($\omega$)] (a) Let $\frak A$
be a Dedekind $\sigma$-complete Boolean algebra.   Then there are a set
$X$, a $\sigma$-algebra $\Sigma$ of subsets of $X$ and a $\sigma$-ideal
$\Cal I$ of $\Sigma$ such that $\frak A\cong\Sigma/\Cal I$.

(b) Let $(\frak A,\bar\mu)$ be a measure algebra.   Then it is isomorphic
to the measure algebra of a measure space.

\proof{{\bf (a)(i)} Set $X=\{0,1\}^{\frak A}$, and
$\Sigma=\Tensorhat_{\frak A}\Cal P(\{0,1\})$.   For $a\in\frak A$ set
$\widehat{a}=\{x:x\in X$, $x(a)=1\}\in\Sigma$.   Let $\Cal I$ be the
$\sigma$-ideal of $\Sigma$ generated by sets of the form

\Centerline{$\widehat{a\Bsymmdiff b}\symmdiff\widehat{a}
  \symmdiff\widehat{b}$,
\quad$(\inf_{n\in\Bbb N}a_n)\spwidehat
\symmdiff\bigcap_{n\in\Bbb N}\widehat{a_n}$}

\noindent for $a$, $b\in\frak A$ and sequences $\sequencen{a_n}$ in
$\frak A$, together with the set $\{x:x(1)=0\}$.

\medskip

\quad{\bf (ii)} (The key.) $\widehat{a}\notin\Cal I$ for any
$a\in\frak A\setminus\{0\}$.   \Prf\ If $E\in\Cal I$ then (using
AC($\omega$)) we can find sequences $\sequencen{a_n}$, $\sequencen{b_n}$ in
$\frak A$, together with a double sequence
$\langle c_{ni}\rangle_{n,i\in\Bbb N}$, such that, setting
$c_n=\inf_{i\in\Bbb N}c_{ni}$ for each $n$,

\Centerline{$F
=\{x:x(1)=0\}
 \cup\bigcup_{n\in\Bbb N}\bigl(\widehat{a_n\Bsymmdiff b_n}
   \symmdiff\widehat{a_n}\symmdiff\widehat{b_n}\bigr)
 \cup\bigcup_{n\in\Bbb N}\bigl(\widehat{c_n}
   \symmdiff\bigcap_{i\in\Bbb N}\widehat{c_{ni}}\bigr)$}

\noindent includes $E$.   Let $\frak B$ be the subalgebra of $\frak A$
generated by

\Centerline{$\{a\}\cup\{a_n:n\in\Bbb N\}\cup\{b_n:n\in\Bbb N\}
  \cup\{c_n:n\in\Bbb N\}\cup\{c_{ni}:n,i\in\Bbb N\}$.}

\noindent Then $\frak B$ is countable, so we can choose inductively a
sequence $\sequencen{d_n}$ in $\frak B\setminus\{0\}$ such that $d_0=a$
and, for each $n\in\Bbb N$,

\inset{$d_{n+1}\Bsubseteq d_n$,

either $d_{n+1}\Bsubseteq a_n$ or $d_{n+1}\Bcap a_n=0$,

either $d_{n+1}\Bsubseteq b_n$ or $d_{n+1}\Bcap b_n=0$,

either $d_{n+1}\Bsubseteq c_n$ or there is an $i\in\Bbb N$ such that
$d_{n+1}\Bcap c_{ni}=0$.}

\noindent Define $x\in X$ by saying that

$$\eqalign{x(d)
&=1\text{ if }d\Bsupseteq d_n\text{ for some }n\in\Bbb N,\cr
&=0\text{ otherwise.}\cr}$$

\noindent Then $x\in\widehat{a}\setminus F$ and
$\widehat{a}\not\subseteq E$;  as $E$ is arbitrary,
$\widehat{a}\notin\Cal I$.\ \Qed

\medskip

\quad{\bf (iii)} Set

\Centerline{$\Sigma_0=\{E:E\in\Sigma$, $E\symmdiff\widehat{a}\in\Cal I$ for
some $a\in\frak A\}$.}

\noindent Then $\Sigma_0$ is closed under symmetric difference and
countable intersections and contains $X$ (because
$X\symmdiff\widehat{1}\in\Cal I$).   So $\Sigma_0$ is a $\sigma$-algebra of
sets;  as it contains $\widehat{a}$ for every $a\in\frak A$, it is equal to
$\Sigma$.

\medskip

\quad{\bf (iv)} From (ii) we see that
$\widehat{a\Bsymmdiff b}$, and therefore $\widehat{a}\symmdiff\widehat{b}$,
do not belong to $\Cal I$ for any distinct $a$, $b\in\frak A$.   With
(iii), this tells us that we have a function $\pi:\Sigma\to\frak A$ defined
by setting $\pi E=a$ whenever $E\symmdiff\widehat{a}\in\Cal I$.   Now
$\pi X=1$ and $\pi$ preserves symmetric difference and countable infima, so
is a sequentially order-continuous Boolean homomorphism;  its kernel is
$\Cal I$, so $\frak A\cong\Sigma/\Cal I$, as required.

\medskip

{\bf (b)} This is now easy;  we can use the familiar argument of 321J.
}%end of proof of 566L

\leader{566M}{Measure algebras:  Proposition} [AC($\omega$)]
(a) Let $\frak A$ be a measurable algebra.

\quad(i) For any $A\subseteq\frak A$ there is a countable $B\subseteq A$
with the same upper bounds as $A$.

\quad(ii) $\frak A$ is Dedekind complete.

\quad(iii) If $D\subseteq\frak A$ is order-dense and $c\in D$ whenever
$c\Bsubseteq d\in D$, there is a partition of unity included in $D$.

(b) Let $(\frak A,\bar\mu)$ be a $\sigma$-finite measure algebra and
$\frak B$ a subalgebra of $\frak A$ such that
$(\frak B,\bar\mu\restrp\frak B)$ is a semi-finite measure algebra.
Then $(\frak B,\bar\mu\restrp\frak B)$ is a $\sigma$-finite
measure algebra.

\proof{{\bf (a)} (Cf.\ 322G, 316E, 322Cc.)
Let $\bar\mu$ be such that $(\frak A,\bar\mu)$
is a totally finite measure algebra.

\medskip

\quad{\bf (i)}
Let $A^*$ be the set of suprema of finite
subsets of $A$, and set $\gamma=\sup_{a\in A^*}\bar\mu a$.
There is a sequence $\sequencen{a_n}$ in $A^*$ such that
$\sup_{n\in\Bbb N}\bar\nu a_n=\gamma$;  let
$B\subseteq A$ be a countable set such that every $a_n$ is the supremum of
a finite subset of $B$.   Then any upper bound $c$ of $B$ is an upper bound
of $A$.   \Prf\ Take $d\in A$.   Then $a\Bcup a_n\in A^*$, so

\Centerline{$\bar\mu(a\Bsetminus c)\le\bar\mu(a\Bsetminus a_n)
=\bar\mu(a\Bcup a_n)-\bar\mu a_n\le\gamma-\bar\mu a_n$}

\noindent for every $n$, and $\bar\mu(a\Bsetminus c)=0$, that is,
$a\Bsubseteq c$.\ \Qed

\medskip

\quad{\bf (ii)} follows at once from (i), since $\frak A$ is Dedekind
$\sigma$-complete.

\medskip

\quad{\bf (iii)} By (i), there is a sequence $\sequencen{d_n}$ in $D$ with
supremum $1$;  now $\sequencen{d_n\Bsetminus\sup_{i<n}d_i}$ is a partition
of unity included in $D$.

\medskip

{\bf (b)} (Cf.\ 322Nc.)   Write $\frak B^f$ for the ring
$\{b:b\in\frak B$, $\bar\mu b<\infty\}$.   Let $\sequencen{a_n}$ be a
non-decreasing sequence in $\frak A$, with supremum $1$, such that
$\bar\mu a_n<\infty$ for every $n$.    For each $n\in\Bbb N$, set
$\alpha_n=\sup\{\bar\mu(b\Bcap a_n):b\in\frak B^f\}$;  choose a sequence
$\sequencen{b_n}$ in $\frak B^f$ such that
$\bar\mu(b_n\Bcap a_n)\ge\alpha_n-2^{-n}$ for every $n$.   \Quer\ If
$1$ is not the supremum of $\{b_n:n\in\Bbb N\}$ in $\frak B$, let
$b\in\frak B\setminus\{0\}$ be such that $b\Bcap b_n=0$ for every $n$.
Because $\bar\mu\restrp\frak B$ is semi-finite, there is a non-zero
$b'\in\frak B^f$ included in $b$.   But now
$0<\bar\mu b'=\sup_{n\in\Bbb N}\bar\mu(b'\Bcap a_n)$, so there is an
$n\in\Bbb N$ such that $\bar\mu(b'\Bcap a_n)>2^{-n}$;  in which case
$b'\Bcup b_n\in\frak B^f$ and $\bar\mu((b'\Bcup b_n)\Bcap a_n)>\alpha_n$,
which is impossible.\ \Bang
}%end of proof of 566M

\leader{566N}{Characterizing the usual measure on $\{0,1\}^{\Bbb N}$:
Theorem} [AC($\omega$)]
(a) Let $(X,\Sigma,\mu)$ be an atomless, perfect,
complete, countably separated probability space.   Then it is isomorphic to
$\{0,1\}^{\Bbb N}$ with its usual measure.

(b) Let $(\frak A,\bar\mu)$ be an atomless
probability algebra of countable Maharam type.   Then it
is isomorphic to the measure algebra of the usual measure on
$\{0,1\}^{\Bbb N}$.

(c) An atomless measurable algebra of countable Maharam type is
homogeneous.

(d) For any infinite set $I$, the measure algebra of the usual measure on
$\{0,1\}^I$ is homogeneous.

\proof{{\bf (a)} (Cf.\ 344I.)   Write $\nu$ for the usual measure on
$Y=\{0,1\}^{\Bbb N}$, and $\Tau$ for its domain.

\medskip

\quad{\bf (i)} Let $\Cal H$ be a countable subset of $\Sigma$ separating
the points of $X$, and $\sequencen{E_n}$ a sequence running over
$\Cal H$ with cofinal repetitions.   Let $f:\Sigma\times[0,1]\to\Sigma$ and
be a function as in 566F.
Define $g:\Sigma\times\Bbb N\to\frak A$ by setting

$$\eqalign{g(E,n)&=f(E\Bcap E_n,\Bover12\bar\mu E)
  \text{ if }\bar\mu(E\Bcap E_n)\ge\Bover12\bar\mu E,\cr
&=f(E\setminus E_n,\Bover12\bar\mu E)\text{ otherwise }.\cr}$$

\noindent
Define $\sequencen{\Cal E_n}$ inductively by saying that
$\Cal E_0=\{X\}$ and

\Centerline{$\Cal E_{n+1}
=\{g(E,n):E\in\Cal E_n\}\cup\{E\setminus g(E,n):E\in\Cal E_n\}$}

\noindent for each $n$.   Then each $\Cal E_n$ is a partition of unity
consisting of $2^n$ elements of measure $2^{-n}$.   Set
$G_n=\bigcup_{E\in\Cal E_n}g(E,n)$ for each $n$, and let
$\Sigma_0$ be the $\sigma$-subalgebra of $\Sigma$ generated by
$\{G_n:n\in\Bbb N\}$.

For $H\in\Sigma$ and $n\in\Bbb N$, set

\Centerline{$\gamma_n(H)
=2^{-n}\#(\{E:E\in\Cal E_n$, $E\cap H\ne\emptyset$ and
$E\not\subseteq H\}$.}

\noindent Then $\gamma_{n+1}(H)\le\gamma_n(H)$ for every $n$, and
$\gamma_{n+1}(E_n)\le\bover12\gamma_n(E_n)$.   Since every member of
$\Cal H$
appears infinitely often as an $E_n$, $\lim_{n\to\infty}\gamma_n(H)=0$ for
every $H\in\Cal H$.   But this means that if $H\in\Cal H$ and we set
$H'=\bigcup\{E:E\in\bigcup_{n\in\Bbb N}\Cal E_n$, $E\subseteq H\}$ and
$H''=X\setminus\bigcup\{E:E\in\bigcup_{n\in\Bbb N}\Cal E_n$,
$H\cap E=\emptyset\}$, then $H'$ and $H''$ both belong to $\Sigma_0$,
$H'\subseteq H\subseteq H''$ and $H''\setminus H'$ is negligible.

\medskip

\quad{\bf (ii)} Define $\phi_0:X\to Y$ by setting
$\phi_0(x)=\sequencen{\chi G_n(x)}$ for $x\in Y$.   Then $\phi_0$ is
$\Sigma_0$-measurable.   Consider the image measure $\mu\phi_0^{-1}$.
This is a topological measure, and because $\mu$ is perfect and complete
(and $Y$ is homeomorphic to a subset of $\Bbb R$)
$\mu\phi_0^{-1}$ is a Radon measure.   If $n\in\Bbb N$ and $z\in\{0,1\}^n$
then $\phi_0^{-1}\{y:z\subseteq y\in Y\}$ belongs to
$\Cal E_n$ and has measure $2^{-n}$, so $\mu\phi_0^{-1}$ and
$\nu$ agree on such sets;  both being Radon measures, they must be equal.

\medskip

\quad{\bf (iii)} Now observe that
$\Sigma_0=\{\phi_0^{-1}[F]:F\subseteq Y$ is Borel$\}$.   We have seen that
if $H\in\Cal H$ there are $H'$, $H''\in\Sigma_0$ such that
$H'\subseteq H\subseteq H''$ and $H''\setminus H'$ is negligible.
Set $X_1=X\setminus\bigcup_{H\in\Cal H}H\setminus H'$, so that
$X_1\subseteq X$ is $\mu$-conegligible.   Now $\phi_0\restr X_1$ is
injective.
\Prf\ If $x$, $x'$ are distinct members of $X_0$, there is an
$H\in\Cal H$ containing one and not the other;  as neither belongs to
$H\setminus H'$, $H'$ contains one and not the other;  as
$H'=\phi_0^{-1}[F]$ for some $F\subseteq Y$, $\phi_0(x)\ne\phi_0(x')$.\
\QeD\   We also find that $\phi_0[X_1]$ is $\nu$-conegligible.   \Prf\
Because $\nu=\mu\phi_0^{-1}$, $\phi_0[X]$ is $\nu$-conegligible.   For each
$H\in\Cal H$,

$$\eqalignno{\nu\phi_0[H\setminus H']
&=\mu\phi_0^{-1}[\phi_0[H\setminus H']]
\le\mu\phi_0^{-1}[\phi_0[H''\setminus H']]
=\mu(H''\setminus H')
\Displaycause{because $H''\setminus H'\in\Sigma_0$ so is the inverse image
of a subset of $Y$}
=0.
\cr}$$

\noindent So $\phi_0[\bigcup_{H\in\Cal H}H\setminus H']$ is
$\nu$-negligible and $\phi_0[X_1]$ is $\nu$-conegligible.\ \Qed

\medskip

\quad{\bf (iv)}
It follows that if we set $\phi_1=\phi_0\restr X_1$ then the subspace
measure $\nu_{Y_1}$ is just the image measure $\mu_{X_1}\phi_1^{-1}$.
\Prf\ If $F\subseteq Y_1$ then

\Centerline{$\mu_{X_1}\phi_1^{-1}[F]=\mu(X_1\cap\phi_0^{-1}[F])
=\mu\phi_0^{-1}[F]=\nu F=\nu_{Y_1}F$}

\noindent if any of these is defined.\ \QeD\  But as $\phi_1$ is a
bijection, this means that it is an isomorphism between $(X_1,\mu_{X_1})$
and $(Y_1,\nu_{Y_1})$.

\medskip

\quad{\bf (v)} There is no reason to suppose that $X\setminus X_1$ and
$Y\setminus Y_1$ are equipollent, so $\phi_1$ may not be directly
extendable to an isomorphism between $X$ and $Y$.   However, there is a
negligible subset $D$ of $Y_1$ which is equipollent with
$\Bbb R$.
\Prf\ Let $K\subseteq Y_1$ be a non-negligible compact set.   Set
$S_2=\bigcup_{n\in\Bbb N}\{0,1\}^n$ and define $\family{z}{S_2}{K_z}$
inductively, as follows.   $K_{\emptyset}=K$.   Given that $z\in\{0,1\}^n$
and that $K_z$ is a non-negligible compact set, take the first $m\ge 2n+2$
such that $J=\{w:w\in\{0,1\}^m$, $\nu\{y:w\subseteq y\in K_z\}>0\}$ has
more than one member, let $w$, $w'$ be the lexicographically two first
members of $J$, and set

\Centerline{$K_{z^{\smallfrown}0}=\{y:w\subseteq y\in K_z\}$,
\quad$K_{z^{\smallfrown}1}=\{y:w'\subseteq y\in K_z\}$;}

\noindent continue.   This will ensure that
$0<\nu K_z\le 4^{-n}$ for every
$z\in\{0,1\}^n$.   Set $D=\bigcap_{n\in\Bbb N}\bigcup_{z\in\{0,1\}^n}K_z$;
then $D$ is negligible and
equipollent with $\{0,1\}^{\Bbb N}$ and $\Bbb R$.\ \Qed

Now set $X_2=X_1\setminus\phi_1^{-1}[D]$ and $Y_2=Y_1\setminus D$.
$\phi_2=\phi_1\restr X_2$ is an isomorphism between the conegligible sets
$X_2$ and $Y_2$ with their subspace measures.   Since $\Cal H$
separates the points of $X$, we surely have an injective function from
$X\setminus X_2$ to $\Bbb R$, while we also have an injective function from
$\Bbb R$ to $\phi_1^{-1}[D]\subseteq X\setminus X_2$.   So $X\setminus X_2$
is equipollent with $\Bbb R$.   Similarly, $Y\setminus Y_2$ is equipollent
with $\Bbb R$.   So $\phi_2:X_2\to Y_2$ can be extended to a bijection
$\phi:X\to Y$, which will be the required isomorphism between
$(X,\Sigma,\mu)$ and $(Y,\Tau,\nu)$.

\medskip

{\bf (b)} (Cf.\ 331I.)   We can use the same idea as in (a).
Let $\sequencen{a_n}$ be a sequence running over a
$\tau$-generating set $A\subseteq\frak A$ with cofinal repetitions.
Let $f:\frak A\times[0,1]\to\frak A$ be a function as in 566F.
Define $g:\frak A\times\Bbb N\to\frak A$ by setting

$$\eqalign{g(a,n)&=f(a\Bcap a_n,\Bover12\bar\mu a)
  \text{ if }\bar\mu(a\Bcap a_n)\ge\Bover12\bar\mu a,\cr
&=f(a\Bsetminus a_n,\Bover12\bar\mu a)\text{ otherwise }.\cr}$$

\noindent
Define $\sequencen{B_n}$ inductively by saying that
$B_0=\{1\}$ and

\Centerline{$B_{n+1}
=\{g(b,n):b\in B_n\}\cup\{b\Bsetminus g(b,n):b\in B_n\}$}

\noindent for each $n$.   Then each $B_n$ is a partition of unity
consisting of $2^n$ elements of measure $2^{-n}$.   Let $\frak B$ be the
closed subalgebra of $\frak A$ generated by $\bigcup_{n\in\Bbb N}B_n$;
then $\frak B$ is isomorphic to the measure algebra of the usual measure on
$\{0,1\}^{\Bbb N}$.

For $a\in\frak A$ and $n\in\Bbb N$, set

\Centerline{$\gamma_n(a)
=2^{-n}\#(\{b:b\in B_n$, $a\cap b\notin\{0,b\}\})$.}

\noindent Then $\gamma_{n+1}(a)\le\gamma_n(a)$ for every $n$, and
$\gamma_{n+1}(a_n)\le\bover12\gamma_n(a_n)$.   Since every member of $A$
appears infinitely often as an $a_n$, $\lim_{n\to\infty}\gamma_n(a)=0$ for
every $a\in A$.   But this means that $A\subseteq\frak B$ and
$\frak B=\frak A$.   So we have the required isomorphism.

\medskip

{\bf (c)} (Cf.\ 331N.)   If $\frak A$ is such an algebra,
any non-zero principal ideal of $\frak A$ is atomless and of
countable Maharam type and supports a probability measure,
so must be isomorphic to the measure algebra of the usual measure on
$\{0,1\}^{\Bbb N}$ and to $\frak A$.

\medskip

{\bf (d)} For $J\subseteq I$,
write $\nu_J$ for the usual measure on $\{0,1\}^J$, $\Tau_J$ for its
domain and $(\frak B_J,\bar\nu_J)$
for its measure algebra.   If $a\in\frak B_I$ is
non-zero, then it is of the form $E^{\ssbullet}$ for some $E\in\Tau_I$
determined by coordinates in a countable subset $J$ of $I$.   Identifying
$\{0,1\}^I$ with $\{0,1\}^J\times\{0,1\}^{I\setminus J}$, we have an
$F\in\Tau_J$ such that $E=F\times\{0,1\}^{I\setminus J}$.   Let
$b\in\frak B_J$ be the equivalence class of $F$.   Now we can
think of the probability algebra free product
$\frak B_J\tensorhat\frak B_{I\setminus J}$ as the metric completion of the
algebraic free product $\frak B_J\otimes\frak B_{I\setminus J}$, and as
such isomorphic to $\frak B_I$ under an isomorphism which identifies the
principal ideal $(\frak B_I)_a$ with
$(\frak B_J)_b\tensorhat\frak B_{I\setminus J}$.   By (b),
$((\frak B_J)_b,\bar\nu_J\restr(\frak B_J)_b)$ is isomorphic, up to a
scalar multiple of the measure, to $(\frak B_J,\bar\nu_J)$;  so we have

\Centerline{$(\frak B_I)_a
\cong(\frak B_J)_b\tensorhat\frak B_{I\setminus J}
\cong\frak B_J\tensorhat\frak B_{I\setminus J}
\cong\frak B_I$.}

\noindent As $a$ is arbitrary, $\frak B_I$ is homogeneous.
}%end of proof of 566N

\leader{566O}{Boolean values:  Proposition} [AC($\omega$)]
(a) Let $\frak B$ be the algebra
of open-and-closed subsets of $\{0,1\}^{\Bbb N}$, and
$\Cal B(\{0,1\}^{\Bbb N})$ the Borel $\sigma$-algebra.   If $\frak A$ is a
Dedekind $\sigma$-complete Boolean algebra and $\pi:\frak B\to\frak A$ is
a Boolean homomorphism, $\pi$ has a unique extension to a sequentially
order-continuous Boolean homomorphism from $\Cal B(\{0,1\}^{\Bbb N})$ to
$\frak A$.

(b) Let $\frak A$ be a Dedekind
$\sigma$-complete Boolean algebra.   Then there is a bijection between
$L^0=L^0(\frak A)$ and the set $\Phi$ of sequentially order-continuous
Boolean homomorphisms from the algebra $\Cal B(\Bbb R)$ of Borel subsets of
$\Bbb R$ to $\frak A$, defined by saying that $u\in L^0$ corresponds to
$\phi\in\Phi$ iff $\Bvalue{u>\alpha}=\phi(\ooint{\alpha,\infty})$ for
every $\alpha\in\Bbb R$.

(c) Let $(\frak A,\bar\mu)$ be a localizable
measure algebra.   Write $\Sigma_{\text{um}}$ for the algebra of
universally measurable subsets of $\Bbb R$.
Then for any $u\in\cmmnt{ L^0=\mskip5mu}L^0(\frak A)$, we have a sequentially
order-continuous Boolean homomorphism
$E\mapsto\Bvalue{u\in E}:\Sigma_{\text{um}}\to\frak A$ such that

$$\eqalign{\Bvalue{u\in E}
&=\sup\{\Bvalue{u\in F}:F\subseteq E\text{ is Borel}\}
=\sup\{\Bvalue{u\in K}:K\subseteq E\text{ is compact}\}\cr
&=\inf\{\Bvalue{u\in F}:F\supseteq E\text{ is Borel}\}
=\inf\{\Bvalue{u\in G}:G\supseteq E\text{ is open}\}\cr}$$

\noindent for every $E\in\Sigma_{\text{um}}$, while

\Centerline{$\Bvalue{u\in\ooint{\alpha,\infty}\,}=\Bvalue{u>\alpha}$}

\noindent for every $\alpha\in\Bbb R$.

\proof{{\bf (a)}
As in \S562, let $\Cal T$ be the set of trees without
infinite branches in $S^*=\bigcup_{n\ge 1}\BbbN^n$.   For $n\in\Bbb N$
set $E_n=\{x:x\in\{0,1\}^{\Bbb N}$, $x(n)=1\}\in\frak B$ and
$a_n=\pi E_n\in\frak A$.   Let $\phi:\Cal T\to\frak A$ and
$\psi:\Cal T\to\Cal B(\{0,1\}^{\Bbb N})$ be the corresponding
interpretations of Borel codes, as in 562V.
Then $\phi(T)=\phi(T')$ whenever $\psi(T)=\psi(T')$ (562V), and (using
AC($\omega$)) it is easy to check that
$\psi[\Cal T]=\Cal B(\{0,1\}^{\Bbb N})$ (cf.\ 562Db), so
we have a function $\tilde\pi:\Cal B(\{0,1\}^{\Bbb N})\to\frak A$ defined
by saying that $\tilde\pi(\psi(T))=\phi(T)$ for every $T\in\Cal T$.
Now if $\sequencen{F_n}$ is any sequence of Borel subsets of
$\{0,1\}^{\Bbb N}$, we have a
$T\in\Cal T$ such that $F_n=\psi(T_{\fraction{n}})$ for every $n$
and no $T_{\fraction{n}}$ is empty (see 562Bb).   In this case

$$\eqalign{\tilde\pi(\bigcup_{n\in\Bbb N}\{0,1\}^{\Bbb N}\setminus F_n)
&=\tilde\pi(\psi(T))
=\phi(T)\cr
&=\sup_{n\in\Bbb N}1\Bsetminus\phi(T_{\fraction{n}})
=\sup_{n\in\Bbb N}1\Bsetminus\tilde\pi F_n.\cr}$$

\noindent So $\tilde\pi$ is a sequentially order-continuous Boolean
homomorphism.   Since it agrees with $\pi$ on $\{E_n:n\in\Bbb N\}$ it must
agree with $\pi$ on $\frak B$.

Of course the extension is unique because if
$\tilde\pi':\Cal B(\{0,1\}^{\Bbb N}\to\frak A$ is any
sequentially order-continuous Boolean homomorphism extending $\pi$ then
$\{E:\tilde\pi'E=\tilde\pi E\}$ is a $\sigma$-algebra of sets including
$\frak B$ and therefore containing every open set.

\medskip

{\bf (b)} (Cf.\ 364F.)  Let $\Cal E$ be the algebra of subsets of $\Bbb R$
generated by sets of the form $\ooint{q,\infty}$ for $q\in\Bbb Q$.   Then
$\Cal E$ is an atomless countable Boolean algebra, so is isomorphic to the
algebra $\frak B$;  let $\theta:\frak B\to\Cal E$ be an isomorphism.
Define $f:\{0,1\}^{\Bbb N}\to[-\infty,\infty]$ by setting
$f(x)=\sup\{q:q\in\Bbb Q$, $x\in\theta^{-1}\ooint{q,\infty}\}$.   Then $f$
is Borel measurable.

Take any $u$ in $L^0$.   It is easy to check that we
have a Boolean homomorphism
$\pi:\Cal E\to\frak A$ defined by saying that
$\pi\ooint{q,\infty}=\Bvalue{u>q}$ for every $q\in\Bbb Q$.
By (a), there is a sequentially order-continuous Boolean homomorphism
$\tilde\pi:\Cal B(\{0,1\}^{\Bbb N})\to\frak A$ extending
$\pi\theta:\frak B\to\frak A$.
Set $\phi E=\tilde\pi(f^{-1}[E])$ for
$E\in\Cal B(\Bbb R)$.

If $\alpha\in\Bbb R$ then

$$\eqalign{\phi(\ooint{\alpha,\infty})
&=\tilde\pi\{x:f(x)>\alpha\}
=\tilde\pi(\bigcup\{\theta^{-1}\ooint{q,\infty}:q\in\Bbb Q,\,q>\alpha\})
  \cr
&=\sup\{\tilde\pi(\theta^{-1}\ooint{q,\infty}):q\in\Bbb Q,\,q>\alpha\}\cr
&=\sup\{\pi\ooint{q,\infty}:q\in\Bbb Q,\,q>\alpha\}\cr
&=\sup\{\Bvalue{u>q}:q\in\Bbb Q,\,q>\alpha\}
=\Bvalue{u>\alpha}.\cr}$$

\noindent It follows that

$$\eqalign{\phi\Bbb R
&=\sup_{n\in\Bbb N}\tilde\pi(f^{-1}\ooint{-n,\infty})
  \Bsetminus\inf_{n\in\Bbb N}\tilde\pi(f^{-1}\ooint{n,\infty})\cr
&=\sup_{n\in\Bbb N}\Bvalue{u>-n}\Bsetminus\inf_{n\in\Bbb N}\Bvalue{u>n}
=1,\cr}$$

\noindent and therefore that $\phi\in\Phi$.   For the rest of the argument
we can follow the method of 364F.

\medskip

{\bf (c)} (Cf.\ 434T.)

\medskip

\quad{\bf (i)}
To begin with, consider the case in which $\bar\mu$ is
totally finite.   In this case, we have a non-decreasing function
$g:\Bbb R\to\coint{0,\infty}$ defined by saying that
$g(\alpha)=\bar\mu 1-\bar\mu\Bvalue{u>\alpha})$ for
$\alpha\in\Bbb R$.   Let $\nu_g$
be the corresponding Lebesgue-Stieltjes measure (114Xa), and
$(\frak C,\bar\nu_g)$ its measure algebra.   Note that $g$ is continuous on
the right, so that
$\nu_g\ocint{\alpha,\beta}
=\bar\mu\Bvalue{u>\alpha}-\bar\mu\Bvalue{u>\beta}$
whenever $\alpha\le\beta$ in $\Bbb R$.   Let $\frak D$ be the
subalgebra of $\frak C$ generated by
$\{\ocint{-\infty,\alpha}^{\ssbullet}:\alpha\in\Bbb R\}$.   Then we have a
measure-preserving Boolean
homomorphism $\pi:\frak D\to\frak A$ defined uniquely by saying
that $\pi(\ooint{\alpha,\infty}^{\ssbullet})=\Bvalue{u>\alpha}$ for
$\alpha\in\Bbb R$.   Because $\frak D$ is dense in $\frak C$ for the
measure-algebra topology, $\pi$ has a unique extension to a
measure-preserving Boolean homomorphism $\tilde\pi:\frak C\to\frak A$.

Because $\Sigma_{\text{um}}\subseteq\dom\nu_g$, we can define
$\Bvalue{u\in E}$ to be $\tilde\pi E^{\ssbullet}$ for
$E\in\Sigma_{\text{um}}$, and this will give us a sequentially
order-continuous Boolean homomorphism from $\Sigma_{\text{um}}$ to
$\frak A$ such that
$\Bvalue{u\in\ooint{\alpha,\infty}\,}=\Bvalue{u>\alpha}$
for every $\alpha$.   As for the other formulae, they are immediate from
the facts that $\nu_g$ is inner regular with respect to the compact sets
and outer regular with respect to the open sets.

\medskip

\quad{\bf (ii)} We need to observe that these properties uniquely define
$\Bvalue{u\in E}$.   \Prf\ Let $\Cal E$ be the algebra of subsets of
$\Bbb R$ generated by $\{\ooint{\alpha,\infty}:\alpha\in\Bbb R\}$.
The requirement
$\Bvalue{u\in\ooint{\alpha,\infty}\,}=\Bvalue{u>\alpha}$
determines the values of $\Bvalue{u\in E}$ for $E\in\Cal E$.
Next, if $G\subseteq\Bbb R$ is open and
$K\subseteq G$ is compact there is an $E\in\Cal E$ such that
$K\subseteq E\subseteq G$.   Consequently
$\Bvalue{u\in K}=\inf\{\Bvalue{u\in E}:E\in\Cal E$, $E\supseteq K\}$
is fixed for every compact $K\subseteq\Bbb R$.   Finally, the inner
regularity condition
$\Bvalue{u\in E}=\sup\{\Bvalue{u\in K}:K\subseteq E$ is compact$\}$
determines $\Bvalue{u\in E}$ for other $E\in\Sigma_{\text{um}}$.\ \Qed

\medskip

\quad{\bf (iii)} Now turn to the general case of a localizable measure
algebra $(\frak A,\bar\mu)$ and $u\in L^0(\frak A)$.   Let $\frak A^f$ be
the ideal of elements of finite measure.   Then for each $a\in\frak A^f$ we
have a corresponding homomorphism
$E\mapsto\Bvalue{u\in E}_a$ from $\Sigma_{\text{um}}$ to the principal
ideal $\frak A_a$.   If $a\Bsubseteq b\in\frak A^f$, we can use the
uniqueness described in (ii) to see that
$\Bvalue{u\in E}_a=a\Bcap\Bvalue{u\in E}_b$ for every $E$.   So if we set
$\Bvalue{u\in E}=\sup_{a\in\frak A^f}\Bvalue{u\in E}_a$, we shall have
$\Bvalue{u\in E}_a=a\Bcap\Bvalue{u\in E}$ whenever $a\in\frak A^f$ and
$E\in\Sigma_{\text{um}}$.   It is now easy to check that
$E\mapsto\Bvalue{u\in E}$ has the required properties.
}%end of proof of 566O

\leader{566P}{Weak \dvrocolon{compactness}}\cmmnt{ In the
absence of
Tychonoff's theorem, the theory of weak compactness in normed spaces
becomes uncertain.   However AC($\omega$) is enough to give a couple of the
principal results involving classical Banach spaces, starting with
Hilbert space.

\medskip

\noindent}{\bf Theorem} [AC($\omega$)] Let $U$ be a Hilbert space.   Then
bounded sets in $U$ are relatively weakly compact.

\proof{ If $U$ is finite-dimensional, this is trivial;  so let us suppose
that $U$ is infinite-dimensional.   Let $A\subseteq U$ be a bounded set,
and $\overline{A}$ its closure for the weak topology;  let $\Cal F_0$ be a
family of weakly closed subsets of $\overline{A}$
with the finite intersection
property, and $\Cal F$ the filter on $U$ generated by $\Cal F_0$.

\medskip

{\bf (a)} For closed subspaces $V$ of $U$, let
$P_V:U\to V$ be the orthogonal projection from $U$ onto $V$ (561Ib), and
set $\gamma_V=\liminf_{u\to\Cal F}\|P_Vu\|^2$.
Because $\Cal F$ contains a
bounded set, $\gamma_V\le\gamma_U<\infty$ for every $V$.
If $V_0$, $V_1$ are
orthogonal subspaces of $U$, then
$\|P_{V_0+V_1}u\|^2=\|P_{V_0}u\|^2+\|P_{V_1}u\|^2$ for every $u\in U$, so
$\gamma_{V_0+V_1}\ge\gamma_{V_0}+\gamma_{V_1}$.

\medskip

{\bf (b)} Set
$\gamma=\sup\{\gamma_V:V$ is a finite-dimensional linear subspace of $U\}$,
and choose a sequence $\sequencen{V_n}$ of finite-dimensional subspaces of
$U$ such that $\gamma=\sup_{n\in\Bbb N}\gamma_{V_n}$;  because $U$ is
infinite-dimensional, we can do this in such a way that $\dim V_n\ge n$ for
each $n$.  Let $W$ be the
closed linear span of $\bigcup_{n\in\Bbb N}V_n$.   If $V$ is a
finite-dimensional linear subspace of $W^{\perp}$, then

\Centerline{$\gamma\ge\gamma_{V+V_n}\ge\gamma_V+\gamma_{V_n}$}

\noindent for every $n$, so $\gamma_V=0$.

\medskip

{\bf (c)} If $F\in\Cal F$, $V\subseteq W^{\perp}$ is a finite-dimensional
linear subspace, and $\epsilon>0$, then $F\cap\{u:\|P_Vu\|\le\epsilon\}$ is
non-empty.   We can therefore extend $\Cal F$ to the filter $\Cal G$
generated by sets of this type, and $\lim_{u\to\Cal G}\innerprod{u}{w}=0$
for every $w\in W^{\perp}$.

\medskip

{\bf (d)} Let $\sequencen{e_n}$ be an orthonormal basis for $W$.   Define
$\sequencen{\Cal G_n}$ as follows.   $\Cal G_0=\Cal G$.   Given that
$\Cal G_n$ is a filter on $U$ containing a bounded set, set
$\alpha_n=\liminf_{u\to\Cal G_n}\innerprod{u}{e_n}$, and let $\Cal G_{n+1}$
be the filter generated by
$\Cal G_n\cup\{\{u:\innerprod{u}{e_n}<\alpha\}:\alpha>\alpha_n\}$;  then
$\alpha_n=\lim_{u\to\Cal G_{n+1}}\innerprod{u}{e_n}$.   Set
$\Cal H=\bigcup_{n\in\Bbb N}\Cal G_n$;  then
$\alpha_n=\lim_{u\to\Cal H}\innerprod{u}{e_n}$ for each $n$.

For any $n\in\Bbb N$,

\Centerline{$\sum_{i=0}^n\alpha_i^2
=\sum_{i=0}^n\lim_{u\to\Cal H}\innerprod{u}{e_i}^2
=\lim_{u\to\Cal H}\sum_{i=0}^n\innerprod{u}{e_i}^2
\le\limsup_{u\to\Cal H}\|u\|^2
<\infty$}

\noindent because $\Cal H$ contains a bounded set.   So
$\sum_{n=0}^{\infty}\alpha_n^2$ is finite and
$v=\sum_{n=0}^{\infty}\alpha_ne_n$ is defined in $U$.

\medskip

{\bf (e)} Now

\Centerline{$\innerprod{v}{e_n}
=\alpha_n=\lim_{u\to\Cal H}\innerprod{u}{e_n}$}

\noindent for every $n$;  again because $\Cal H$ contains a bounded set,
$\innerprod{v}{w}=\lim_{u\to\Cal H}\innerprod{u}{w}$ for every $w\in W$.
On the other hand, if $w\in W^{\perp}$,

\Centerline{$\lim_{u\to\Cal H}\innerprod{u}{w}
=\lim_{u\to\Cal G}\innerprod{u}{w}=0=\innerprod{v}{w}$.}

\noindent Since $W+W^{\perp}=U$,
$\lim_{u\to\Cal H}\innerprod{u}{w}=\innerprod{v}{w}$ for every $w\in U$.
By 561Ic, $v$ is the limit of $\Cal H$ for the weak topology on $U$,
and must belong to every member of $\Cal F_0$.

As $\Cal F_0$ is arbitrary, $\overline{A}$ is weakly
compact and $A$ is relatively weakly compact.
}%end of proof of 566P

\leader{566Q}{\bf Theorem} [AC($\omega$)]
Let $U$ be an $L$-space.   Then a subset of $U$ is weakly relatively
compact iff it is uniformly integrable.

\proof{{\bf (a)(i)}\grheada\ Recall that $U$ is a Banach lattice with an
order-continuous norm (354N), so is Dedekind complete (354Ee) and all its
bands are complemented (353I);  for a band $V$ in $U$, let $P_V:U\to V$ be
the band projection onto $V$.

\medskip

\qquad\grheadb\ If $u\in U$ there is an $f\in U^*$ such that
$\|f\|\le 1$ and
$f(u)=\|u\|$.   \Prf\ Let $V$ be the band generated by $u^+$ and
$W=V^{\perp}$ its band complement.
Set $f(v)=\int P_Vv-\int P_Wv$ for $v\in U$.
Since $\|v\|=\|P_Vv\|+\|P_Wv\|$ for
every $v\in U$, $\|f\|\le 1$.   Also $P_Vu=u^+$ and $P_Wu=-u^-$
so $f(u)=\int|u|=\|u\|$.\ \Qed

\medskip

\qquad\grheadc\ If $A\subseteq U$ is weakly bounded it is norm-bounded.
\Prf\Quer\ Otherwise, choose for each $n\in\Bbb N$ a $u_n\in A$ and
$f_n\in U^*$ such that
$\|u_n\|\ge n$, $\|f_n\|=1$ and $f_n(u_n)=\|u_n\|\ge n$.
For $f\in U^*$ set $\rho_A(f)=\sup_{u\in A}|f(u)|$.
Define $\sequence{k}{n_k}$ by setting
$n_k=\lceil 2\cdot 3^k(k+\sum_{i=0}^{k-1}3^{k-i}\rho_A(f_{n_i}))\rceil$
for each $k$.
Set $f=\sum_{i=0}^{\infty}3^{-i}f_{n_i}$.   Then for any $k\in\Bbb N$ we
have

$$\eqalign{\rho_A(f)
&\ge f(u_{n_k})
=\sum_{i=0}^{\infty}3^{-i}f_{n_i}(u_{n_k})\cr
&\ge 3^{-k}f_{n_k}(u_{n_k})-\sum_{i=0}^{k-1}3^{-i}\rho_A(f_{n_i})
     -\sum_{i=k+1}^{\infty}3^{-i}\|u_{n_k}\|\cr
&=\Bover1{2\cdot 3^k}\|u_{n_k}\|-\sum_{i=0}^{k-1}3^{-i}\rho_A(f_{n_i})
\ge k.  \text{ \Bang\Qed}\cr}$$

\medskip

\quad{\bf (ii)}
Now let $K\subseteq U$ be a weakly relatively countably compact
set.   Let $\frak A$ be the band algebra of $U$.   For $V\in\frak A$
set $\nu V=\sup_{u\in K}\|P_Vu\|$ (counting $\sup\emptyset$ as $0$).
Then $\nu$ is a submeasure on $\frak A$.   By (i-$\gamma$),
$K$ is norm-bounded and $\nu$ is
finite-valued;  set $\alpha=\nu U=\sup_{u\in K}\|u\|$.

$\nu$ is exhaustive.
\Prf\Quer\ Otherwise, let $\sequencen{V_n}$ be a disjoint sequence in
$\frak A$ such that
$\epsilon=\bover16\inf_{n\in\Bbb N}\nu V_n$ is greater than $0$.
For each $n\in\Bbb N$ choose $u_n\in K$ and $f_n\in U^*$
such that $\|f_n\|\le 1$ and
$f_n(P_nu_n)=\|P_nu_n\|\ge 5\epsilon$, where here I write $P_n$
for $P_{V_n}$.   Let $v_0$
be a cluster point of $\sequencen{u_n}$ in $U$ for the weak topology of
$U$.   Note that $\sum_{n=0}^{\infty}\|P_nu\|\le\|u\|$ for any $u\in U$;
let $m\in\Bbb N$ be such that $\sum_{n=m}^{\infty}\|P_nv_0\|\le\epsilon$.
For $n\in\Bbb N$, set $g_n(u)=f_n(P_nu)$ for $u\in U$.

We can now build a strictly increasing sequence $\sequence{k}{n_k}$ such
that

\Centerline{$n_0\ge m$,}

\Centerline{$\sum_{i=0}^{k-1}|g_{n_i}(u_{n_k})|
\le\epsilon+\sum_{i=0}^{k-1}|g_{n_i}(v_0)|$,}

\Centerline{$\|P_{n_k}u_{n_i}\|\le 2^{-k}\epsilon$ whenever $i<k$}

\noindent for every $k\in\Bbb N$.   Let $v_1$ be a weak cluster point of
$\sequence{k}{u_{n_k}}$, and $l\in\Bbb N$ such that
$\sum_{k=l}^{\infty}\|P_{n_k}v_1\|\le\epsilon$.   Set
$g=\sum_{k=l}^{\infty}g_{n_k}$;  this is defined in $U^*$ because

\Centerline{$\sum_{k=l}^{\infty}|g_{n_k}(u)|
\le\sum_{k=l}^{\infty}\|P_{n_k}u\|\le\|u\|$}

\noindent for every $u\in U$.   Of course $|g(v_1)|\le\epsilon$.   On the
other hand, for any $k\ge l$,

$$\eqalign{g(u_{n_k})
&=g_{n_k}(u_{n_k})+\sum_{i=l}^{k-1}g_{n_i}(u_{n_k})
   +\sum_{i=k+1}^{\infty}g_{n_i}(u_{n_k})\cr
&\ge 5\epsilon-\sum_{i=0}^{k-1}|g_{n_i}(u_{n_k})|
   -\sum_{i=k+1}^{\infty}\|P_{n_i}u_{n_k}\|   \cr
&\ge 5\epsilon-\sum_{i=0}^{k-1}|g_{n_i}(v_0)|-\epsilon
   -\sum_{i=k+1}^{\infty}2^{-i}\epsilon\cr
&\ge 4\epsilon-\sum_{n=m}^{\infty}\|P_nv_0\|
   -2^{-k}\epsilon
\ge 2\epsilon\cr}$$

\noindent and $v_1$ cannot be a weak cluster point of
$\sequence{k}{u_{n_k}}$.\ \Bang\Qed

\medskip

\quad{\bf (iii)} In fact $\nu$ is uniformly exhaustive.
\Prf\Quer\ Otherwise, let $\epsilon>0$ be such that there are arbitrarily
long disjoint strings in $\frak A$ of elements of submeasure greater than
$2\epsilon$.   Set
$q(n)=\lceil\Bover{2^nn\alpha}{\epsilon}\rceil$ for each $n$,
and choose a family
$\langle V_{ni}\rangle_{n\in\Bbb N,i\le q(n)}$ such that
$\langle V_{ni}\rangle_{i\le q(n)}$ is a disjoint family in $\frak A$
for each $n$ and $\nu V_{ni}>2\epsilon$ for all $n$ and $i$;
adapting the temporary notation of (ii), I set $P_{ni}=P_{V_{ni}}$ for
$i\le q(n)$.   Now choose
$u_{ni}\in K$ such that $\|P_{ni}u_{ni}\|\ge 2\epsilon$ for all $i$ and
$n$.   Because $\sum_{i=0}^{q(n)}\|P_{ni}u\|\le\|u\|$ for every $u\in U$
and $n\in\Bbb N$, we can define inductively a sequence $\sequencen{i_n}$
such that $i_n\le q(n)$ and
$\|P_{ni_n}u_{mi_m}\|\le 2^{-n}\epsilon$ whenever $m<n$.

Now set

\Centerline{$W_{mn}=V_{mi_m}\cap\bigcap_{m<k\le n}V_{ki_k}^{\perp}$,
\quad$Q_{mn}=P_{W_{mn}}$}

\noindent for $m\le n$,

\Centerline{$W_m=\bigcap_{n\ge m}W_{mn}$,
\quad$Q_m=P_{W_m}$}

\noindent for $m\in\Bbb N$.
For any $u\in U$ and $m\le n$,

\Centerline{$|P_{mi_m}u|
=P_{mi_m}|u|\le Q_{mn}|u|+\sum_{k=m+1}^nP_{ki_k}|u|$,}

\Centerline{$\|P_{mi_m}u\|
\le\|Q_{mn}u\|+\sum_{k=m+1}^n\|P_{ki_k}u\|$,}

\noindent so

$$\eqalign{\|Q_{mn}u_{mi_m}\|
&\ge\|P_{mi_m}u_{mi_m}\|-\sum_{k=m+1}^n\|P_{ki_k}u_{mi_m}\|\cr
&\ge 2\epsilon-\sum_{k=m+1}^{\infty}2^{-k}\epsilon
\ge\epsilon.\cr}$$

\noindent Next, if $u\ge 0$, $\langle Q_{mn}u\rangle_{n\ge m}$ is a
non-increasing sequence, and its infimum belongs to
$\bigcap_{n\ge m}W_{mn}$, so must be equal to $Q_mu$;  accordingly
$Q_mu$ is the norm-limit of $\langle Q_{mn}u\rangle_{n\ge m}$.   The same
is therefore true for every $u\in U$, and in particular

\Centerline{$\|Q_mu_{mi_m}\|=\lim_{n\to\infty}\|Q_{mn}u_{mi_m}\|
\ge\epsilon$.}

\noindent Consequently $\nu W_m\ge\epsilon$.   But
$W_m\cap W_n\subseteq V_{ni_n}^{\perp}\cap V_{ni_n}=\{0\}$ whenever $n>m$,
so this contradicts (ii).\ \Bang\Qed

\medskip

\quad{\bf (iv)} Now take any $\epsilon>0$.
Then there is a $u^*\in U^+$ such
that $\int(|u|-u^*)^+\le\epsilon$ for every $u\in K$.   \Prf\
If $\alpha\le\epsilon$ we can take $u^*=0$ and stop.   Otherwise, there
is a largest $n\in\Bbb N$ such that there are disjoint
$V_0,\ldots,V_n\in\frak A$ such that $\nu V_i>\epsilon$ for every $n$.
Take $u_0,\ldots,u_n\in K$ such that $\|P_{V_i}u_i\|>\epsilon$ for each
$i$.   Let $\gamma>0$ be such that
$\|P_{V_i}u_i\|-\Bover{\alpha}{\gamma}>\epsilon$ for every $i\le n$, and
set $u^*=\gamma\sum_{i=0}^n|u_i|$.
\Quer\ Suppose that $u\in K$ is such that $\int(|u|-u^*)^+>\epsilon$.
Let $W$ be the band generated by $(|u|-u^*)^+$, so that
$\nu W\ge\|P_Wu\|>\epsilon$.   For each $i\le n$, set
$W_i=V_i\cap W^{\perp}$;  then

\Centerline{$|P_Wu_i|\le\Bover1{\gamma}P_Wu^*\le\Bover1{\gamma}|u|$,
\quad$|P_{W_i}u_i|\ge|P_{V_i}u_i|-\Bover1{\gamma}|u|$,}

\Centerline{$\nu W_i\ge\|P_{W_i}u_i\|
\ge\|P_{V_i}u_i\|-\Bover{\alpha}{\gamma}>\epsilon$.}

\noindent
But now $W_0,\ldots,W_n,W$ witnesses that $n$ was not maximal.\ \Bang\
So $\sup_{u\in K}\int(|u|-u^*)^+\le\epsilon$, as required.\ \Qed

As $\epsilon$ is arbitrary, $K$ is uniformly integrable.   Thus every
relatively weakly compact subset of $U$ is uniformly integrable.

\medskip

{\bf (b)(i)} In the reverse direction, suppose to begin with
that $(\frak A,\bar\mu)$ is a totally finite measure algebra, and
that $A\subseteq L^1=L^1(\frak A,\bar\mu)$ is uniformly integrable;  let
$\Cal F$ be a filter on $L^1$ containing $A$.    Write $\Cal V$ for the set
of neighbourhoods of $0$ for the weak topology
$\frak T_s(L^1,(L^1)^*)$.\footnote{\smallerfonts
Of course $(L^1)^*$ can be identified
with $L^{\infty}(\frak A)$, but if
you don't wish to trace through the arguments for this, and confirm that
they can be carried out without appealing to anything more than
AC($\omega$), you can defer the exercise for the time being.}

\medskip

\qquad\grheada\ For each $n\in\Bbb N$ let $M_n\ge 0$ be such that
$\|(|u|-M_n\chi 1)^+\|_1\le 2^{-n}$ for every $u\in A$, and
define sets $K_n\subseteq[-M_n\chi 1,M_n\chi 1]$ and filters $\Cal F_n$
as follows.   $\Cal F_0=\Cal F$.   Given that $\Cal F_n$ contains $A$,
define $\phi_n:L^1\to L^2=L^2(\frak A,\bar\mu)$ by setting
$\phi_n(u)=\med(-M_n\chi 1,u,M_n\chi 1)$ for each $u\in L^1$, and consider
the filter $\phi_n[[\Cal F_n]]$.   This is a filter on the Hilbert space
$L^2$ containing the $\|\,\|_2$-bounded set $[-M_n\chi 1,M_n\chi 1]$, so
the set $K_n^*$ of its $\frak T_s(L^2,L^2)$-cluster points is non-empty, by
566P;  as $K_n^*$ is $\frak T_s(L^2,L^2)$-closed, it is
$\frak T_s(L^2,L^2)$-compact.
As $[-M_n\chi 1,M_n\chi 1]$ is $\|\,\|_2$-closed and convex, it is
$\frak T_s(L^2,L^2)$-closed (561Ie) and includes $K_n^*$.
Set $\gamma_n=\inf\{\|u\|_2:u\in K_n^*\}$.   As all the sets
$\{u:\|u\|_2\le\alpha\}$, for $\alpha>\gamma_n$, are
$\frak T_s(L^2,L^2)$-closed and meet $K_n^*$,
$K_n=\{u:u\in K_n^*$, $\|u\|_2\le\gamma_n\}$ is non-empty.

Suppose that $G\in\Cal V$.
Because the embedding $L^2\embedsinto L^1$ is norm-continuous, it is weakly
continuous, and $G\cap L^2$ is a $\frak T_s(L^2,L^2)$-neighbourhood of $0$.
It follows that
$x+G$ meets every member of $\phi_n[[\Cal F_n]]$ for every $x\in K^*_n$;
so $K_n+G$ meets every member of $\phi_n[[\Cal F_n]]$.   We can therefore
extend $\Cal F_n$ to the filter $\Cal F_{n+1}$ generated by

\Centerline{$\Cal F_n\cup\{\phi_n^{-1}[K_n+G]:G\in\Cal V\}$}

\noindent and continue.

\medskip

\qquad\grheadb\
Set $\Cal G=\bigcup_{n\in\Bbb N}\Cal F_n$ and
$B=\{u:u\in L^1$, $\|u\|_1\le 1\}$.   Then for each $n\in\Bbb N$
there is a finite set $J\subseteq L^1$ such that
$J+G+2^{-n+1}B\in\Cal G$ for every $G\in\Cal V$.  \Prf\ $K_n$ is a
$\frak T_s(L^2,L^2)$-closed subset of $K_n^*$, so is
$\frak T_s(L^2,L^2)$-compact;  also it is included in the sphere
$S=\{u:\|u\|_2=\gamma_n\}$.   Because $\|\,\|_2$ is locally uniformly
rotund, it is a Kadec norm (467B)
and the norm and weak topologies on $S$ coincide;  consequently $K_n$ is
$\|\,\|_2$-compact.   Since $\|\,\|_1$ and $\|\,\|_2$ give rise to the same
topology on any $\|\,\|_{\infty}$-bounded set, $K_n$ is $\|\,\|_1$-compact.
There is therefore a finite set $J\subseteq K_n$ such that
$K_n\subseteq J+2^{-n}B$.

Take any $G\in\Cal V$.   Then
$\|u-\phi_n(u)\|=\|(|u|-M_n\chi 1)^+\|\le 2^{-n}$ for
every $u\in A$, so

\Centerline{$J+G+2^{-n+1}B\supseteq(K_n+G)+2^{-n}B
\supseteq A\cap\phi_n^{-1}[K_n+G]\in\Cal F_{n+1}\subseteq\Cal G$.\ \Qed}

\medskip

\qquad\grheadc\ For each $n\in\Bbb N$ choose a minimal
finite set $J_n\subseteq L^1$ such that $J_n+G+2^{-n+1}B\in\Cal G$
for every
$G\in\Cal V$.   Note that $(x+G+2^{-n+1}B)\cap D$ must be non-empty
whenever $n\in\Bbb N$, $x\in J_n$, $G\in\Cal V$ and $D\in\Cal G$.
\Prf\Quer\ Otherwise,

\Centerline{$(J_n\setminus\{x\})+G'+2^{-n+1}B
\supseteq(J_n+(G\cap G')+2^{-n+1}B)\cap D$}

\noindent belongs to $\Cal G$ for every $G'\in\Cal V$, and $J_n$ was not
minimal.\ \Bang\Qed

\medskip

\qquad\grheadd\ For any $n\in\Bbb N$ and $u\in J_n$ there is a
$v\in J_{n+1}$ such that $\|u-v\|_1\le 2^{-n+1}+2^{-n}$.   \Prf\Quer\
Otherwise, by (a-i-$\beta$) above, we can
choose for each $v\in J_{n+1}$ an $f_v\in(L^1)^*$ such that
$\|f_v\|=1$ and
$f_v(v-u)=\|v-u\|=2^{-n+1}+2^{-n}+\delta_v$ where $\delta_v>0$;  set

\Centerline{$G
=\{w:|f_v(w)|<\bover12\delta_v$ for every $v\in J_{n+1}\}\in\Cal V$.}

\noindent Then $u+G+2^{-n+1}B$ does not meet $J_{n+1}+G+2^{-n}B$,
contradicting ($\gamma$) here.\ \Bang\Qed

\medskip

\qquad\grheade\ Because $\bigcup_{n\in\Bbb N}J_n$ is countable, therefore
well-orderable, we can define inductively a sequence $\sequencen{u_n}$ such
that $u_n\in J_n$ and $\|u_n-u_{n+1}\|\le 2^{-n+1}+2^{-n}$ for every $n$.
Now $\sequencen{u_n}$ is Cauchy, so has a limit $u$ in $L^1$.
If $G\in\Cal V$ there is an $n\in\Bbb N$ such that
$u+G\supseteq u_n+\bover12G+2^{-n}B$, so $u+G$ meets every member of
$\Cal G$;  thus $u$ is a weak cluster point of $\Cal G$ and of $\Cal F$.
As $\Cal F$ is arbitrary, $A$ is relatively weakly compact.

\medskip

\quad{\bf (ii)} Now suppose that $U$ is an arbitrary $L$-space and
$A\subseteq U$ is a uniformly integrable set.   Then we can choose a
sequence $\sequencen{e_n}$ in $U^+$ such that
$\|(|u|-e_n)^+\|\le 2^{-n}$ for every $n\in\Bbb N$ and $u\in A$.   Set
$e=\sum_{n=0}^{\infty}\Bover1{1+2^n\|e_n\|}e_n$ in $U$, and let $V$ be the
band in $U$ generated by $e$.   Then $A\subseteq V$, and of course $A$ is
uniformly integrable in $V$.   By 561Hb, we have a
totally finite measure algebra $(\frak A,\bar\mu)$ and a normed Riesz space
isomorphism $T:V\to L^1(\frak A,\bar\mu)$;   now $T[A]$ is uniformly
integrable in $L^1(\frak A,\bar\mu)$, therefore relatively weakly
compact, by (i).   But this means that $A$ is relatively weakly compact in
$V$;  as the embedding $V\embedsinto U$ is weakly continuous, $A$ is
relatively weakly compact in $U$.

This completes the proof.
}%end of proof of 566Q

\vleader{48pt}{566R}{Automorphisms of measurable algebras:
Theorem} [AC($\omega$)] Let $\frak A$ be a measurable algebra.

(a) Every automorphism of $\frak A$ has a separator.

(b) Every $\pi\in\Aut\frak A$ is a product of at most three exchanging
involutions belonging to the full subgroup of $\Aut\frak A$ generated by
$\pi$.

\proof{{\bf (a)}  (Cf.\ 382Eb.)   Take $\pi\in\Aut\frak A$.
Let $\bar\mu$ be such that $(\frak A,\bar\mu)$ is a totally finite
measure algebra.   For $a\in\Bbb N$ set $\psi(a)=\sup_{n\in\Bbb Z}\pi^na$,
so that $\pi(\psi(a))=\psi(a)$.   Note that if $a\Bcap\psi(b)=0$ then
$\psi(a)\Bcap\psi(b)=0$.   Set
$A=\{a:a\in\frak A$, $a\Bcap\pi a=0\}$ and choose a sequence
$\sequencen{a_n}$ in $A$ such that

\Centerline{$\sup_{n\in\Bbb N}\bar\mu(\psi(a_n))
=\sup_{a\in A}\bar\mu(\psi(a))$.}

\noindent Define $\sequencen{b_n}$, $\sequencen{c_n}$ by saying that

\Centerline{$c_0=0$,
\quad$b_n=a_n\setminus\psi(c_n)$,
\quad$c_{n+1}=b_n\Bcup c_n$}

\noindent for each $n$.   Inducing on $n$ we see that $b_n$ and $c_n$
belong to $A$ and that $\psi(c_{n+1})\Bsupseteq\psi(a_n)$ for every $n$.
Set $c=\sup_{n\in\Bbb N}c_n$;  then $c\in A$ and
$\psi(c)\Bsupseteq\psi(a_n)$ for every $n$.

Now $c$ is a separator for $\pi$.   \Prf\Quer\ Otherwise, there is a
non-zero $d\Bsubseteq 1\Bsetminus\psi(c)$ such that $d\Bcap\pi d=0$
(381Ei).   In this case $d\Bcup c\in A$ and

\Centerline{$\bar\mu(\psi(d\Bcup c))>\bar\mu(\psi(c))
=\sup_{n\in\Bbb N}\bar\mu(\psi(a_n))=\sup_{a\in A}\bar\mu(\psi(a))$,}

\noindent which is impossible.\ \Bang\Qed

\medskip

{\bf (b)} We can now work through the proofs of 382A-382M to confirm that
there is no essential use of anything beyond countable choice there, so
long as we suppose that we are working with measurable algebras.
(There is an inductive construction in the proof of 382J.   To do this with
AC($\omega$) rather than DC, we need to check that every element of the
construction can be made determinate following an initial countable set of
choices;  in the case there, we need to
check that the existence assertions of 382D
and 382I can be represented as functions, as in
566Xh and 566Xj.) %566Xh 566Xi 566Xj
Since the proof of 382K speaks of the Stone representation
theorem, there seems to be a difficulty here, unless we take the
alternative route suggested in 382Yb.
But note that while the general Stone theorem
has a strength little short of full AC, the representation of a
{\it countable} Boolean algebra $\frak B$
as the algebra of open-and-closed subsets
of a compact Hausdorff Baire space can be done in ZF alone (561F).
In part (f) of the proof of 382K, therefore,
take $\frak B$ to be a countable subalgebra of $\frak A$ such that

\inset{$e_n$, $u'_n$, $u''_n$, $v'_l$, $v''_l$, $d_{lj}$, $d'_{lj}$,
$\supp(\pi\phi)^k$, $\supp(\pi\phi_1)^k\in\frak B$
whenever $n\in\Bbb N$ and $j$, $k$, $l\ge 1$,

$c_0$, $c_1$, $\supp\phi_2\in\frak B$,

$\frak B$ is closed under the functions $\pi$, $\phi_1$, $\phi_2$, $\phi$
and $\tilde\pi_n$ for $n\in\Bbb N$,}

\noindent and let $Z$ be the Stone space of $\frak B$.   Now we can perform
the arguments of the rest of the proof in $Z$ to show that
$c_0=\inf_{n\ge 1}\supp(\pi\phi)^n$ is zero, as required.
}%end of proof of 566R

\cmmnt{\leader{566S}{Volume 4} In Volume 4, naturally,
a rather larger proportion of the
ideas become inaccessible without strong forms of the axiom of choice.
Since we are missing the most useful representation theorems, many
results have to be abandoned altogether.   More subtly, we seem to lose the
result that Radon measures are localizable\cmmnt{ (416B)}.
Nevertheless, a good deal can still be done\cmmnt{, if
we follow the principles
set out in 566Ae-566Af}.   Most notably, we have a workable theory of Haar
measure on completely regular locally compact
topological groups\cmmnt{, because the Riesz representation theorems of
\S436 are still available, and we can use 561G instead of 441C}.
I should remark, however, that in the absence of Tychonoff's theorem we may
have fewer compact groups than we expect.
And the theory of dual groups in \S445 depends heavily on AC.

The descriptive set theory of Chapter 42 is hardly touched, and
enough of the rest of the volume survives to make it worth
checking any point of particular interest.   Most of Chapter 46 depends
heavily on the Hahn-Banach theorem and therefore becomes limited to
cases in which we have a good grasp of dual spaces\cmmnt{, as in 561Xh}.
There are some difficulties in the geometric measure theory of arbitrary
metric spaces in \S471, but the rest of the chapter seems to stand up.
The abstract theory of
gauge integrals in \S482 is expressed in forms which need DC at least, but
I think that the basic facts about the Henstock
integral\cmmnt{ (\S483)} are
unaffected.   There are some interesting challenges in Chapter 49, but
there the eclectic nature of the arguments
means that we cannot expect much of the theory to keep its shape.
}%end of comment

\leader{566T}{}\cmmnt{ I give one result which may not be obvious and
helps to keep things in order.

\medskip

\noindent}{\bf Proposition} [AC($\omega$)] Let $I$ be any set, and $X$ a
separable metrizable space.
Then the Baire $\sigma$-algebra $\CalBa(X^I)$ of
$X^I$ is equal to the $\sigma$-algebra $\Tensorhat_I\Cal B(X)$ generated by
sets of the form $\{x:x(i)\in E\}$ for $i\in I$ and Borel sets
$E\subseteq X$.

\proof{{\bf (a)} Every open set in $X$ is a cozero set, so $\Cal B(X)=\CalBa(X)$
and $\{x:x(i)\in E\}\in\CalBa(X^I)$ whenever $i\in I$ and $E\in\Cal B(X)$;
accordingly $\Tensorhat_I\Cal B(X)\subseteq\CalBa(X^I)$.

\medskip

{\bf (b)} Fix a sequence $\sequencen{U_n}$ running
over a base for the topology of $X$.   For
$\sigma\in S=\bigcup_{J\in[I]^{<\omega}}\Bbb N^J$ set

\Centerline{$C_{\sigma}
=\{x:x\in X$, $x(i)\in U_{\sigma(i)}$ for every $i\in\dom\sigma\}
\in\Tensorhat_I\Cal B(X)$.}

\noindent Then $\{C_{\sigma}:\sigma\in S\}$ is a base for the topology of
$X^I$.   If $W\subseteq X^I$ is a regular open set, there is a countable
set $R\subseteq S$ such that $W=\bigcup_{\sigma\in R}C_{\sigma}$.
\Prf\ Let $R^*$ be the set of those $\sigma\in S$ such that
$C_{\sigma}\subseteq W$, and $R$ the set of minimal members of $R^*$
(ordering $S$ by extension of functions).   Then every member of $R^*$
extends some member of $R$, so

\Centerline{$\bigcup_{\sigma\in R}C_{\sigma}
=\bigcup_{\sigma\in R^*}C_{\sigma}=W$.}

\noindent For $n\in\Bbb N$ set
$R_n=\{\sigma:\sigma\in R$, $\#(\sigma)=n$, $\sigma(i)<n$ for every
$i\in\dom\sigma\}$.

\Quer\ Suppose, if possible, that $n\in\Bbb N$ and $R_n$ is infinite.
Then there is a sequence $\sequence{k}{\sigma_k}$ of distinct elements of
$R_n$;  set $J_k=\dom\sigma_k$ for each $k$.   Let $M\subseteq\Bbb N$ be an
infinite set such that $\family{k}{M}{J_k}$ is a $\Delta$-system with root
$J$ say.   Then there is a $\sigma\in n^J$ such that
$M'=\{k:k\in M$, $\sigma_k\restr J=\sigma\}$ is infinite.

In this case, however,

\Centerline{$C_{\sigma}
\subseteq\interior\overline{\bigcupop_{k\in M'}C_{\sigma_k}}
\subseteq\interior\overline{W}=W$}

\noindent and $\sigma\in R^*$, so that $\sigma_k\notin R$ for $k\in M'$;
which is impossible.\ \Bang

Thus every $R_n$ is countable and $R=\bigcup_{n\in\Bbb N}R_n$ is
countable.\  \Qed

\medskip

{\bf (c)} This shows that every regular open subset of $X^I$ is a countable
union of open cylinder sets and belongs to
$\Tensorhat_I\Cal B(X)$.   Consequently every cozero set belongs to
$\Tensorhat_I\Cal B(X)$.   \Prf\ If $f:X^I\to\Bbb R$ is continuous, then
for each rational $q>0$ set $W_q=\interior\{x:|f(x)|\ge q\}$.   Then
$W_q$ is a regular open set so belongs to
$\Tensorhat_I\Cal B(X)$.   But now
$\{x:f(x)\ne 0\}=\bigcup_{q\in\Bbb Q,q>0}W_q$ is the union of countably
many sets in $\Tensorhat_I\Cal B(X)$ and itself belongs to
$\Tensorhat_I\Cal B(X)$.\ \Qed

So $\Tensorhat_I\Cal B(X)\supseteq\CalBa(X^I)$ and the two are equal.
}%end of proof of 566T

\cmmnt{
\leader{566U}{Dependent choice} If we allow ourselves to use the stronger
principle DC rather than AC($\omega$) alone, we get some useful
simplifications.   The difficulties with the principle of exhaustion in
\S215 and 566D above
disappear, and there is no longer any obstacle to the
construction of product measures in 254F, provided only that we know we
have a non-empty product space.   So a typical theorem on product measures
will now begin `let $\familyiI{(X_i,\Sigma_i,\mu_i)}$ be a family of
probability spaces such that $X=\prod_{i\in I}X_i$ is non-empty'.   Later,
we now have Baire's theorem (both for complete metric spaces and for
locally compact Hausdorff spaces) and Urysohn's Lemma (so we can drop the
formulation `completely regular locally compact topological group').
The most substantial gap in Volume 4
which is now filled seems to be in the abstract
theory of gauge integrals in \S482.
But I cannot point to a result which is essential to the structure of
this treatise and can be proved in ZF + DC but not in ZF + AC($\omega$).
}%end of comment

\exercises{\leader{566X}{Basic exercises (a)}
%\spheader 566Xa
[AC($\omega$)] Let $(X,\rho)$ be a metric space.
(i) Show that $X$
is compact iff it is sequentially compact iff it is countably compact
iff it is complete and totally bounded.  (ii) Show that if $X$ is separable
then every subspace of $X$ is separable.

\spheader 566Xb [AC($\omega$)] Show that there is a surjection from
$\Bbb R$ onto its Borel $\sigma$-algebra,
so that there must be a non-Borel subset of $\Bbb R$.
%566C

\spheader 566Xc\dvAnew{2014}(i) (Cf.\ 313K)
Let $\frak A$ be a Boolean algebra, and
$D\subseteq\frak A$ an order-dense set.   Show that
$a=\sup\{d:d\in D$, $d\Bsubseteq a\}$ for every $a\in\frak A$.
(ii) (Cf.\ 322Eb)
Let $(\frak A,\bar\mu)$ be a semi-finite measure algebra.   Show that
$a=\sup\{b:b\Bsubseteq a$, $\bar\mu b<\infty\}$ for every $a\in\frak A$.
%566M

\spheader 566Xd
Let us say that a Boolean algebra $\frak A$ has the
{\bf countable sup property} if for every $A\subseteq\frak A$ there is a
countable $B\subseteq A$ with the same upper bounds as $A$.
(i) Show that a Dedekind $\sigma$-complete Boolean
algebra with the countable sup property is Dedekind complete.
(ii) Show that a countably additive functional on a
Boolean algebra with the countable sup property is completely additive.
%566M

\spheader 566Xe [AC($\omega$)] Show that if there is a
translation-invariant lifting for Lebesgue measure then there is a subset
of $\Bbb R$ which is not Lebesgue measurable.   \Hint{345F.}
%345F

\spheader 566Xf [AC($\omega$)] Show that if $1<p<\infty$ and
$(\frak A,\bar\mu)$ is a
measure algebra, the unit ball of $L^p(\frak A,\bar\mu)$ (\S366) is weakly
compact.   \Hint{part (b) of the proof of 566Q.}
%566Q

\spheader 566Xg [AC($\omega$)] (i) Let $\frak A$ be a measurable algebra.
Show that the
unit ball of $L^{\infty}=L^{\infty}(\frak A)$ is compact for
$\frak T_s(L^{\infty},(L^{\infty})^{\times})$ (definition: 3A5Ea).
(ii) Let $U$ be an $L$-space with a weak order unit.   Show
that the unit ball of $U^*$ is weak*-compact.   \Hint{561Hb.}
%566Q

\spheader 566Xh Let $\frak A$ be a Dedekind $\sigma$-complete Boolean
algebra.   Show that there is a function
$f:\Aut\frak A\times\frak A\to\frak A$ such that if $\pi\in\Aut\frak A$ and
$a$ is a separator for $\pi$ then $a\Bcap f(\pi,a)=0$ and
$f(\pi,a)\Bcup\pi f(\pi,a)\Bcup\pi^2f(\pi,a)$ is the support of $\pi$.
\Hint{382D.}
%566R

\spheader 566Xi Let $\frak A$ be a Dedekind complete Boolean
algebra and $G$ a well-orderable subgroup of $\Aut\frak A$.
Let $G^*$ be the full subgroup of $\Aut\frak A$ generated by $G$.
Show that there is a function
$f:G^*\times G\to\frak A$ such that $\family{\phi}{G}{f(\pi,\phi)}$ is a
partition of unity for each $\pi\in G^*$ and $\pi a=\phi a$ whenever
$\pi\in G^*$, $\phi\in G$ and $a\Bsubseteq f(\pi,\phi)$.   \Hint{381I.}
%566R

\spheader 566Xj [AC($\omega$)]
Let $\frak A$ be a Dedekind complete Boolean algebra and
$G$ a countable subgroup of $\Aut\frak A$ such that every
member of $G$ has a separator.   Let $G^*$ be the full subgroup
of $\Aut\frak A$ generated by $G$.    Show that there is a
function
$g:G^*\to\frak A$ such that $g(\pi)$ is a separator for $\pi$ for every
$\pi\in G^*$.   \Hint{566Xi, 382Id.}
%566R

\spheader 566Xk [AC($\omega$)] Let $(X,\frak T)$ be a completely regular
locally compact Hausdorff space, and $f:C_k(X)\to\Bbb R$ a positive
linear functional.   Show that
there is a unique Radon measure $\mu$ on $X$ such that
$f(u)=\int u\,d\mu$ for every $u\in C_k(X)$.
%436J

\spheader 566Xl [AC($\omega$)]
Say that a set $X$ is {\bf measure-free} if whenever $\mu$
is a probability measure with domain $\Cal PX$ there is an $x\in X$ such
that $\mu\{x\}>0$.   (i) Show that
the following are equiveridical:
($\alpha$) $\Bbb R$ is not measure-free;
($\beta$) there is a semi-finite measure space $(X,\Cal PX,\mu)$ which
is not purely atomic;
($\gamma$) there is a measure $\mu$ on $[0,1]$ extending Lebesgue
measure and measuring every subset of $[0,1]$.
(ii) Prove 438B for point-finite families $\familyiI{E_i}$ such that
the index set $I$ is measure-free.
%438B

\leader{566Y}{Further exercises (a)}
%\spheader 566Ya
[AC($\omega$)] Show that if $U$ is an $L$-space, and
$\sequencen{u_n}$ is a bounded sequence in $U$, then there are a
subsequence $\sequencen{v_n}$ of $\sequencen{u_n}$ and a $w\in U$ such that
$\sequencen{\bover1{n+1}\sum_{i=0}^nw_i}$ is order*-convergent to $w$ for
every subsequence $\sequencen{w_n}$ of $\sequencen{v_n}$.   \Hint{in the
proof of 276H, show that we can find a countably-generated filter to
replace the ultrafilter $\Cal F$.}
%276H

\spheader 566Yb [AC($\omega$)] Let
$X$ be a completely regular compact Hausdorff topological group and $\mu$ a
left Haar measure on $X$.   Show that if $w\in L^2(\mu)$ then
$u\mapsto u*w:L^2(\mu)\to C(X)$ is a
compact linear operator.   \Hint{444V.}
%444

\spheader 566Yc [AC($\omega$)]
Let $\familyiI{(X_i,\sequencen{U_{in}})}$ be a family such
that $X_i$ is a separable metrizable space and $\sequencen{U_{in}}$ is a
base for the topology of $X_i$ for each $i\in I$.   Show that
$\CalBa(\prod_{i\in I}X_i)=\Tensorhat_{i\in I}\Cal B(X_i)$.
%566S

\spheader 566Yd
[DC] Let $U$ be an inner product space and $K\subseteq U$
a convex weakly compact set.   Show that $K$ has an extreme point.
}%end of exercises

\leader{566Z}{Problem} Is it relatively consistent with ZF + AC($\omega$)
to suppose that there is a non-zero atomless rigid measurable algebra?

\endnotes{
\Notesheader{566} In this section I have taken a lightning tour through the
material of Volumes 1 to 4, pausing over a rather odd selection of results,
mostly chosen to exhibit the alternative arguments which are available.
In the first place, I am trying to suggest something of the quality of the
world of measure theory, and of analysis in general, under this particular
set of rules.   Perhaps I should say that my real objective is the next
section, with DC rather than AC($\omega$), because DC is believed to be
compatible with the axiom of determinacy, and ZF + DC + AD is not a poor
relation of ZFC, as ZF + AC($\omega$) sometimes seems to be, but a
potential rival.

I have a second reason for taking all this trouble,
which is a variation on one of the reasons
for `generalization' as found in
twentieth-century pure mathematics.   When we `generalize' an argument,
moving (for example) from metric spaces to topological spaces, or from
Lebesgue measure to abstract measures, we are usually stimulated by some
particular question which demands the new framework.   But the process
frequently has a lasting value which is quite independent of its
motivation.
It forces us to re-examine the nature of the proofs we are using,
discarding or adapting those steps which depend on the original context,
and isolating those which belong in some other class of ideas.   In the
same way, renouncing the use of AC forces us to look more closely at
critical points, and decide which of them correspond to some deeper
principle.

Something I have not attempted to do is to look for models in which my
favourite theorems are actually false.
An interesting class of problems is concerned with
`exact engineering', that is, finding combinatorial propositions which will
be equivalent, in ZF, to given results which are not provable in ZF.   For
instance, Baire's theorem for complete metric spaces is actually equivalent
to DC ({\smc Blair 77}), while Baire's theorem for compact Hausdorff spaces
may be weaker ({\smc Fossy \& Morillon 98}).   I am not presenting any such
results here.   However, if we take Maharam's theorem as an example of a
central result of measure theory with ZFC which is surely unprovable
without a strong form of AC, we can ask just how false it can be;  and I
offer 566Z as a sample target.
}%end of notes

\discrpage

