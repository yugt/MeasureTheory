\frfilename{mt235.tex}
\versiondate{30.3.03/20.8.08}
\copyrightdate{1994}

\def\chaptername{The Radon-Nikod\'ym theorem}
\def\sectionname{Measurable transformations}
\def\tildeTau{\tilde{\text{T}}}

\newsection{235}

I turn now to a topic which is separate from the
Radon-Nikod\'ym theorem, but which seems to fit better here than in
either of the next two chapters.   I seek to give results which will
generalize the basic formula of calculus

\Centerline{$\int g(y)dy=\int g(\phi(x))\phi'(x)dx$}

\noindent in the context of a general transformation $\phi$ between
measure spaces.   The principal results are I suppose 235A/235E, which
are very similar expressions of the basic idea, and 235J, which gives a
general criterion for a stronger result.   A formulation from a
different direction is in 235R.

\leader{235A}{}\cmmnt{ I start with the basic result, which is already
sufficient for a large proportion of the applications I have in mind.

\medskip

\noindent}{\bf Theorem} Let $(X,\Sigma,\mu)$ and $(Y,\Tau,\nu)$ be
measure spaces, and $\phi:D_{\phi}\to Y$, $J:D_J\to\coint{0,\infty}$
functions defined on conegligible subsets $D_{\phi}$, $D_J$ of $X$
such that

\Centerline{$\int J\times\chi(\phi^{-1}[F])d\mu$ exists $=\nu F$}

\noindent whenever $F\in\Tau$ and $\nu F<\infty$.   Then

\Centerline{$\int_{\phi^{-1}[H]}J\times g\phi\,d\mu$ exists =
$\int_Hg\,d\nu$}

\noindent for every $\nu$-integrable function $g$ taking values in
$[-\infty,\infty]$ and every $H\in\Tau$, provided that
we interpret $(J\times g\phi)(x)$ as $0$ when $J(x)=0$ and $g(\phi(x))$
is undefined.   Consequently, interpreting $J\times f\phi$ in the same way,

\Centerline{$\underline{\intop}fd\nu
\le\underline{\intop}J\times f\phi\,d\mu
\le\overline{\intop}J\times f\phi\,d\mu\le\overline{\intop}fd\nu$}

\noindent for every $[-\infty,\infty]$-valued function $f$ defined almost
everywhere in $Y$.

\proof{{\bf (a)} If $g$ is a simple function, say
$g=\sum_{i=0}^na_i\chi F_i$ where $\nu F_i<\infty$ for each $i$, then

\Centerline{$\int J\times g\phi\,d\mu
=\sum_{i=0}^na_i\int J\times\chi(\phi^{-1}[F_i])\,d\mu
=\sum_{i=0}^na_i\nu F_i=\int g\,d\nu$.}

\medskip

{\bf (b)} If $\nu F=0$ then $\int J\times\chi(\phi^{-1}[F])=0$ so $J=0$
a.e.\ on
$\phi^{-1}[F]$.   So if $g$ is defined $\nu$-a.e., $J=0\,\,\mu$-a.e.\ on
$X\setminus\dom(g\phi)=(X\setminus D_{\phi})\cup\phi^{-1}[Y\setminus\dom
g]$, and, on the convention proposed, $J\times g\phi$ is defined
$\mu$-a.e.   Moreover, if $\lim_{n\to\infty}g_n
=g\,\,\nu$-a.e., then $\lim_{n\to\infty}J\times g_n\phi=J\times
g\phi\,\,\mu$-a.e.   So if $\sequencen{g_n}$ is a non-decreasing
sequence of simple functions converging almost everywhere to $g$,
$\sequencen{J\times g_n\phi}$ will be a non-decreasing sequence of
integrable functions converging almost everywhere to
$J\times g\phi$;  by B.Levi's theorem,

\Centerline{$\int J\times g\phi\,d\mu$ exists $=\lim_{n\to\infty}
\int J\times g_n\phi\,d\mu=\lim_{n\to\infty}\int g_nd\nu
=\int g\,d\nu$.}

\medskip

{\bf (c)} If $g=g^+-g^-$, where $g^+$ and $g^-$ are $\nu$-integrable
functions, then

\Centerline{$\int J\times g\phi\,d\mu
=\int J\times g^+\phi\,d\mu-\int J\times g^-\phi\,d\mu
=\int g^+d\nu-\int g^-d\nu
=\int g\,d\nu$.}

\medskip

{\bf (d)} This deals with the case $H=Y$.   For the general case, we
have

$$\eqalignno{\int_Hg\,d\nu
&=\int(g\times\chi H)d\nu\cr
\displaycause{131Fa}
&=\int J\times(g\times\chi H)\phi\,d\mu
=\int J\times g\phi\times\chi(\phi^{-1}[H])d\mu
=\int_{\phi^{-1}[H]}J\times g\phi\,d\mu\cr}$$

\noindent by 214F.

\medskip

{\bf (e)} For the upper and lower integrals, I note first that if $F$ is
$\nu$-negligible then $\int J\times\chi(\phi^{-1}[F])d\mu=0$, so that
$J=0\,\,\mu$-a.e.\ on $\phi^{-1}[F]$.   It follows that if $f$ and $g$ are
$[-\infty,\infty]$-valued functions on subsets of $Y$ and $f\leae g$, then
$J\times f\phi\leae J\times g\phi$.   Now if $\overline{\int}fd\nu=\infty$,
we surely have 
$\overline{\int}J\times f\phi\,d\mu\le\overline{\int}fd\nu$.   Otherwise, 

$$\eqalign{\overline{\int}fd\nu
&=\inf\{\int g\,d\nu:g\text{ is }\nu-\text{integrable and }f\leae g\}\cr
&=\inf\{\int J\times g\phi\,d\mu:
  g\text{ is }\nu\text{-integrable and }f\leae g\}\cr
&\le\inf\{\int h\,d\mu:
  h\text{ is }\mu\text{-integrable and }J\times f\phi\leae h\}
=\overline{\intop}J\times f\phi\,d\mu.\cr}$$

\noindent Similarly, or applying this argument to $-f$, we have
$\underline{\int}J\times f\phi\,d\mu\le\underline{\int}f\,d\nu$.
}%end of proof of 235A

\cmmnt{
\leader{235B}{Remarks (a)} Note the particular convention

\Centerline{$0\times\text{undefined}=0$}

\noindent which I am applying to the interpretation of $J\times g\phi$.
This is the first of a number
of technical points which will concern us in this section.   The point
is that if $g$ is defined $\nu$-almost everywhere, then for any
extension of $g$ to a function $g_1:Y\to\Bbb R$ we shall, on this
convention, have $J\times g\phi=J\times g_1\phi$ except on
$\{x:J(x)>0,\,\phi(x)\in Y\setminus\dom g\}$, which is negligible;  so
that

\Centerline{$\int J\times g\phi\,d\mu
=\int J\times g_1\phi\,d\mu
=\int g_1d\nu
=\int g\,d\nu$}

\noindent if $g$ and $g_1$ are integrable.   Thus the convention is
appropriate here, and while it adds a phrase to the statements of many
of the results of this section, it makes their application smoother.
(But I ought to insist that I am using this as a local convention only,
and the ordinary rule $0\times\text{undefined}=\text{undefined}$ will
stand elsewhere in this treatise unless explicitly overruled.)

\header{235Bb}{\bf (b)} I have had
to take care in the formulation of this theorem to distinguish between
the hypothesis

\Centerline{$\int J(x)\chi(\phi^{-1}[F])(x)\mu(dx)$ exists $=\nu F$
whenever $\nu F<\infty$}

\noindent and the perhaps more elegant alternative

\Centerline{$\int_{\phi^{-1}[F]}J(x)\mu(dx)$ exists $=\nu F$
whenever
$\nu F<\infty$,}

\noindent which is not quite adequate for the theorem.   (See 235Q
below.)   Recall that by $\int_Af$ I mean $\int(f\restr A)d\mu_A$, where
$\mu_A$ is the subspace measure on $A$ (214D).   It is possible for
$\int_A(f\restr A)d\mu_A$ to be defined even
when $\int f\times\chi A\,d\mu$ is not;  for instance, take $\mu$ to be
Lebesgue measure on $[0,1]$, $A$ any non-measurable subset of $[0,1]$,
and $f$ the constant function with value $1$;  then $\int_Af=\mu^*A$,
but $f\times\chi A=\chi A$ is not $\mu$-integrable.   It is however the
case that if $\int f\times\chi A\,d\mu$ is defined, then so is
$\int_Af$, and the two are equal;  this is a consequence of 214F.
While 235P shows that in most of the cases relevant to the present
volume the distinction can be passed over, it is important to avoid
assuming that $\phi^{-1}[F]$ is measurable for every $F\in\Tau$.   A
simple example
is the following.   Set $X=Y=[0,1]$.   Let $\mu$ be Lebesgue measure on
$[0,1]$, and define $\nu$ by setting

\Centerline{$\Tau=\{F:F\subseteq[0,1],\,F\cap[0,\bover12]$ is Lebesgue
measurable$\}$,}

\Centerline{$\nu F=2\mu(F\cap[0,\bover12])$ for every $F\in\Tau$.}

\noindent Set $\phi(x)=x$ for every $x\in[0,1]$.   Then we have

\Centerline{$\nu F=\int_FJ\,d\mu
=\int J\times\chi(\phi^{-1}[F])d\mu$}

\noindent for every $F\in\Tau$, where $J(x)=2$ for $x\in[0,\bover12]$
and $J(x)=0$ for $x\in\ocint{\bover12,1}$.   But of course there are
subsets $F$ of $[\bover12,1]$ which are not  Lebesgue measurable (see
134D), and such an $F$ necessarily belongs to $\Tau$, even though
$\phi^{-1}[F]$ does not belong to the domain $\Sigma$ of $\mu$.

The point here is that if $\nu F_0=0$  then we expect to have $J=0$ on
$\phi^{-1}[F_0]$, and it is of no importance whether $\phi^{-1}[F]$ is
measurable for $F\subseteq F_0$.
}%end of comment

\leader{235C}{}\cmmnt{ Theorem 235A is concerned with integration, and
accordingly the hypothesis $\int J\times\chi(\phi^{-1}[F])d\mu=\nu F$
looks only at sets $F$ of finite measure.   If we wish to consider
measurability of non-integrable functions, we need a slightly stronger
hypothesis.   I approach this version more gently, with a couple of
lemmas.

\medskip

\noindent}{\bf Lemma} Let $\Sigma$, $\Tau$ be
$\sigma$-algebras of subsets of $X$ and $Y$ respectively.   Suppose that
$D\subseteq X$ and that $\phi:D\to Y$ is a function such that
$\phi^{-1}[F]\in\Sigma_D$, the subspace $\sigma$-algebra, for every
$F\in\Tau$.   Then $g\phi$ is $\Sigma$-measurable for every
$[-\infty,\infty]$-valued
$\Tau$-measurable function $g$ defined on a subset of $Y$.

\proof{ Set $C=\dom g$ and $B=\dom g\phi=\phi^{-1}[C]$.   If
$a\in\Bbb R$,
then there is an $F\in\Tau$ such that $\{y:g(y)\le a\}=F\cap C$.   Now
there is an $E\in\Sigma$ such that $\phi^{-1}[F]=E\cap D$.   So

\Centerline{$\{x:g\phi(x)\le a\}=B\cap E\in\Sigma_B$.}

\noindent As $a$ is arbitrary, $g\phi$ is $\Sigma$-measurable.
}%end of proof of 235C

\leader{235D}{}\cmmnt{ Some of the results below are easier when we
can move freely between measure spaces and their completions (212C).
The next lemma is what we need.

\medskip

\noindent}{\bf Lemma} Let $(X,\Sigma,\mu)$ and $(Y,\Tau,\nu)$ be
measure spaces, with completions $(X,\hat\Sigma,\hat\mu)$ and
$(Y,\hat\Tau,\hat\nu)$.   Let $\phi:D_{\phi}\to Y$, $J:D_J\to
\coint{0,\infty}$ be functions defined on conegligible subsets of $X$.

(a) If $\int J\times\chi(\phi^{-1}[F])d\mu=\nu F$ whenever $F\in\Tau$
and $\nu F<\infty$, then $\int J\times\chi(\phi^{-1}[F])d\hat\mu=\hat\nu
F$ whenever $F\in\hat\Tau$ and $\hat\nu F<\infty$.

(b) If $\int J\times\chi(\phi^{-1}[F])d\mu=\nu F$ whenever $F\in\Tau$,
then $\int J\times\chi(\phi^{-1}[F])d\hat\mu=\hat\nu F$ whenever
$F\in\hat\Tau$.

\proof{ Both rely on the fact that either hypothesis is enough to ensure
that $\int J\times\chi(\phi^{-1}[F])d\mu=0$ whenever $\nu F=0$.
Accordingly, if $F$ is $\nu$-negligible, so that there is an $F'\in\Tau$
such that $F\subseteq F'$ and $\nu F'=0$, we shall have

\Centerline{$\int J\times\chi(\phi^{-1}[F])d\mu
=\int J\times\chi(\phi^{-1}[F'])d\mu=0$.}

\noindent But now, given any $F\in\hat\Tau$, there is an $F_0\in\Tau$
such that $F_0\subseteq F$ and $\hat\nu(F\setminus F_0)=0$, so that

$$\eqalign{\int J\times\chi(\phi^{-1}[F])d\hat\mu
&=\int J\times\chi(\phi^{-1}[F])d\mu\cr
&=\int J\times\chi(\phi^{-1}[F_0])d\mu
   +\int J\times\chi(\phi^{-1}[F\setminus F_0])d\mu\cr
&=\nu F_0=\hat\nu F,\cr}$$

\noindent provided (for part (a)) that $\hat\nu F<\infty$.
}%end of proof of 235D

\cmmnt{\medskip

\noindent{\bf Remark} Thus if we have the hypotheses of any of the
principal results of this section valid for a pair of non-complete
measure spaces, we can expect to be able to draw some conclusion by
applying the result to the completions of the given spaces.
}%end of comment

\leader{235E}{}\cmmnt{ Now I come to the alternative version of 235A.

\medskip

\noindent}{\bf Theorem} Let $(X,\Sigma,\mu)$ and $(Y,\Tau,\nu)$ be
measure spaces, and $\phi:D_{\phi}\to Y$,
$J:D_J\to\coint{0,\infty}$ two functions defined on conegligible subsets
of $X$ such that

\Centerline{$\int J\times\chi(\phi^{-1}[F])d\mu=\nu F$}

\noindent for every $F\in\Tau$, allowing $\infty$ as a value of the
integral.

(a) $J\times g\phi$ is $\mu$-virtually measurable for every
$\nu$-virtually measurable function $g$ defined on a subset of $Y$.

(b) Let $g$ be a $\nu$-virtually measurable $[-\infty,\infty]$-valued
function defined on a conegligible subset of $Y$.   Then
$\int J\times g\phi\,d\mu=\int g\,d\nu$ whenever either integral is
defined in
$[-\infty,\infty]$, if we interpret $(J\times g\phi)(x)$ as $0$ when
$J(x)=0$ and $g(\phi(x))$ is undefined.

\proof{ Let $(X,\hat\Sigma,\hat\mu)$ and $(Y,\hat\Tau,\hat\nu)$ be the
completions of $(X,\Sigma,\mu)$ and $(Y,\Tau,\nu)$.   By 235D,

\Centerline{$\int J\times\chi(\phi^{-1}[F])d\hat\mu=\hat\nu F$}

\noindent for every $F\in\hat\Tau$.   Recalling that a real-valued
function is $\mu$-virtually measurable iff it is
$\hat\Sigma$-measurable (212Fa), and that $\int fd\mu=\int fd\hat\mu$ if
either is defined in $[-\infty,\infty]$ (212Fb), the conclusions we are
seeking are

\inset{(a)$'\,\,J\times g\phi$ is $\hat\Sigma$-measurable for every
$\hat\Tau$-measurable function $g$ defined on a subset of $Y$;}

\inset{(b)$'\,\,\int J\times g\phi\,d\hat\mu=\int g\,d\hat\nu$ whenever
$g$ is a $\hat\Tau$-measurable function defined
almost everywhere in $Y$ and either integral is defined in
$[-\infty,\infty]$.}

\medskip

{\bf (a)} When I write

\Centerline{$\int J\times\chi D_{\phi}d\mu
=\int J\times\chi(\phi^{-1}[Y])d\mu=\nu Y$,}

\noindent which is part of the hypothesis of this theorem, I mean to
imply that $J\times\chi D_{\phi}$ is $\mu$-virtually measurable, that
is, is $\hat\Sigma$-measurable.   Because $D_{\phi}$ is conegligible, it
follows that $J$ is
$\hat\Sigma$-measurable, and its domain $D_J$, being conegligible, also
belongs to $\hat\Sigma$.   Set $G=\{x:x\in D_J,\,J(x)>0\}\in\hat\Sigma$.
Then for any set $A\subseteq X$, $J\times\chi A$ is
$\hat\Sigma$-measurable iff
$A\cap G\in\hat\Sigma$.   So the hypothesis is just that
$G\cap\phi^{-1}[F]\in\hat\Sigma$ for every $F\in\hat\Tau$.

Now let $g$ be a $[-\infty,\infty]$-valued function, defined on a subset
$C$ of $Y$,
which is $\hat\Tau$-measurable.   Applying 235C to $\phi\restr G$, we
see that $g\phi\restr G$ is $\hat\Sigma$-measurable, so
$(J\times g\phi)\restr G$ is $\hat\Sigma$-measurable.   On the other
hand, $J\times g\phi$ is zero almost everywhere in $X\setminus G$, so
(because $G\in\hat\Sigma$) $J\times g\phi$ is $\hat\Sigma$-measurable,
as required.

\medskip

{\bf (b)(i)} Suppose first that $g\ge 0$.   Then $J\times g\phi\ge 0$,
so (a) tells us that $\int J\times g\phi$ is defined in $[0,\infty]$.

\medskip

\qquad\grheada\ If $\int g\,d\hat\nu<\infty$ then
$\int J\times g\phi\,d\hat\mu=\int g \,d\hat\nu$ by 235A.

\medskip

\qquad\grheadb\ If there is some $\epsilon>0$ such that
$\hat\nu H=\infty$, where $H=\{y:g(y)\ge\epsilon\}$, then

\Centerline{$\int J\times g\phi\,d\hat\mu
\ge\epsilon\int J\times\chi(\phi^{-1}[H])d\hat\mu
=\epsilon\hat\nu H=\infty$,}

\noindent so

\Centerline{$\int J\times g\phi\,d\hat\mu=\infty=\int
g\,d\hat\nu$.}

\medskip

\qquad\grheadc\ Otherwise,

$$\eqalign{\int J\times g\phi\,d\hat\mu
&\ge\sup\{\int J\times h\phi\,d\hat\mu:
   h\text{ is }\hat\nu\text{-integrable},
   \,0\le h\le g\}\cr
&=\sup\{\int h\,d\hat\nu:h\text{ is }\hat\nu\text{-integrable},
   \,0\le h\le g\}
=\int g\,d\hat\nu
=\infty,\cr}$$

\noindent so once again $\int J\times \phi\,d\hat\mu=\int g\,d\hat\nu$.

\medskip

\quad{\bf (ii)} For general real-valued $g$, apply (i) to $g^+$ and
$g^-$ where $g^+=\bover12(|g|+g)$, $g^-=\bover12(|g|-g)$;   the point is
that $(J\times g\phi)^+=J\times g^+\phi$ and
$(J\times g\phi)^-=J\times g^-\phi$, so that

\Centerline{$\int J\times g\phi
=\int J\times g^+\phi-\int J\times g^-\phi
=\int g^+-\int g^- %
=\int g$}

\noindent if either side is defined in $[-\infty,\infty]$.
}%end of proof of 235E

\cmmnt{
\leader{235F}{Remarks (a)} Of course there are two special
cases of this
theorem which between them carry all its content:  the case $J=1$ a.e.\
and the case in which $X=Y$ and $\phi$ is the identity function.   If
$J=\chi X$ we are very close to 235G below, and if $\phi$ is the
identity function we are close to the indefinite-integral measures of
\S234.

\header{235Fb}{\bf (b)} As in 235A, we can strengthen the
conclusion of (b) in 235E to

\Centerline{$\int_{\phi^{-1}[F]}J\times
g\phi\,d\mu=\int_Fg\,d\nu$}

\noindent whenever $F\in\Tau$ and $\int_Fg\,d\nu$ is defined in
$[-\infty,\infty]$.
}%end of comment

\leader{235G}{Theorem} Let $(X,\Sigma,\mu)$  and $(Y,\Tau,\nu)$ be
measure spaces and $\phi:X\to Y$ an \imp\ function.   Then

(a) if $g$ is a $\nu$-virtually measurable $[-\infty,\infty]$-valued
function
defined on a subset of $Y$, $g\phi$ is $\mu$-virtually measurable;

(b) if $g$ is a $\nu$-virtually measurable $[-\infty,\infty]$-valued
function
defined on a conegligible subset of $Y$, $\int g\phi\,d\mu=\int g\,d\nu$
if either integral is defined in $[-\infty,\infty]$;

(c) if $g$ is a $\nu$-virtually measurable $[-\infty,\infty]$-valued
function defined on a conegligible subset of $Y$, and $F\in\Tau$, then
$\int_{\phi^{-1}[F]}g\phi\,d\mu=\int_Fg\,d\nu$ if either integral is
defined in $[-\infty,\infty]$.

\proof{{\bf (a)} This follows immediately from 234Ba and 235C;   taking
$\hat\Sigma$, $\hat\Tau$ to be the domains of the completions of $\mu$,
$\nu$ respectively, $\phi^{-1}[F]\in\hat\Sigma$ for every $F\in\hat\Tau$,
so if $g$ is $\hat\Tau$-measurable then $g\phi$ will be
$\hat\Sigma$-measurable.

\medskip

{\bf (b)} Apply 235E with $J=\chi X$;  we have

\Centerline{$\int J\times\chi(\phi^{-1}[F])d\mu
=\mu\phi^{-1}[F]=\nu F$}

\noindent for every $F\in\Tau$, so

\Centerline{$\int g\phi=\int J\times g\phi=\int g$}

\noindent if either integral is defined in $[-\infty,\infty]$.

\medskip

{\bf (c)} Apply (b) to $g\times\chi F$.
}%end of proof of 235G

\cmmnt{
\leader{235H}{The image measure catastrophe}
Applications of 235A would run much more smoothly if we could say

\inset{`$\int g\,d\nu$ exists and is equal to 
$\int J\times g\phi\,d\mu$
for every $g:Y\to\Bbb R$ such that $J\times g\phi$ is
$\mu$-integrable'.}

\noindent Unhappily there is no hope of a universally applicable result
in this direction.   Suppose, for instance, that $\nu$ is Lebesgue
measure on $Y=[0,1]$, that $X\subseteq[0,1]$ is a
non-Lebesgue-measurable set of outer measure $1$ (134D), that $\mu$ is
the subspace measure $\nu_X$ on $X$, and that $\phi(x)=x$ for $x\in X$.
Then

\Centerline{$\mu\phi^{-1}F=\nu^*(X\cap F)=\nu F$}

\noindent for every  Lebesgue measurable set $F\subseteq Y$, so we can
take $J=\chi X$ and the hypotheses of 235A and 235E will be satisfied.
But if we write $g=\chi X:[0,1]\to\{0,1\}$, then $\int g\phi\,d\mu$ is
defined even though $\int g\,d\nu$ is not.

The point here is that there is a set $A\subseteq Y$ such that (in the
language of 235A/235E) $\phi^{-1}[A]\in\Sigma$ but $A\notin\hat\Tau$.
This is the {\bf image measure catastrophe}.   
The search for contexts in which we can be sure that it does not occur 
will be one of the motive
themes of Volume 4.   For the moment, I will offer some general remarks
(235I-235J), and describe one of the important cases in which the
problem does not arise (235K).
}%end of comment


\leader{235I}{Lemma}
Let $\Sigma$, $\Tau$ be $\sigma$-algebras of
subsets of $X$, $Y$ respectively, and $\phi$ a function from a subset
$D$ of $X$ to $Y$.   Suppose that $G\subseteq X$ and that

\Centerline{$\Tau=\{F:F\subseteq Y,\,G\cap\phi^{-1}[F]\in\Sigma\}$.}

\noindent Then a real-valued function $g$, defined on a member of
$\Tau$, is $\Tau$-measurable iff $\chi G\times g\phi$ is
$\Sigma$-measurable.

\proof{ Because surely $Y\in\Tau$, the hypothesis implies that $G\cap
D=G\cap\phi^{-1}[Y]$ belongs to $\Sigma$.

Let $g:C\to\Bbb R$ be a function, where $C\in\Tau$.   Set
$B=\dom(g\phi)=\phi^{-1}[C]$, and for $a\in\Bbb R$ set
$F_a=\{y:g(y)\ge a\}$,

\Centerline{$E_a=G\cap\phi^{-1}[F_a]
=\{x:x\in G\cap B,\,g\phi(x)\ge a\}$.}

\noindent Note that $G\cap B\in\Sigma$ because $C\in\Tau$.

\medskip

\quad{\bf (i)} If $g$ is $\Tau$-measurable, then
$F_a\in\Tau$ and $E_a\in\Sigma$ for every $a$.   Now

\Centerline{$G\cap\{x:x\in B,\,g\phi(x)\ge a\}=G\cap\phi^{-1}[F_a]
=E_a$,}

\noindent so $\{x:x\in B,\,(\chi G\times g\phi)(x)\ge a\}$ is either
$E_a$ or
$E_a\cup(B\setminus G)$, and in either case is relatively
$\Sigma$-measurable in $B$.   As $a$ is arbitrary, $\chi G\times g\phi$
is $\Sigma$-measurable.

\medskip

\quad{\bf (ii)} If $\chi G\times g\phi$ is $\Sigma$-measurable, then,
for any $a\in\Bbb R$,

\Centerline{$E_a=\{x:x\in G\cap B,\,(\chi G\times g\phi)(x)\ge
a\}\in\Sigma$}

\noindent because $G\cap B\in\Sigma$ and $\chi G\times g\phi$ is
$\Sigma$-measurable.
So $F_a\in\Tau$.   As $a$ is arbitrary, $g$ is
$\Tau$-measurable.
}%end of proof of 235I

\leader{235J}{Theorem}
Let $(X,\Sigma,\mu)$ and $(Y,\Tau,\nu)$ be
complete measure spaces.   Let $\phi:D_{\phi}\to Y$,
$J:D_J\to\coint{0,\infty}$ be functions defined on conegligible subsets
of $X$, and set $G=\{x:x\in D_J,\,J(x)>0\}$.   Suppose that

\Centerline{$\Tau
=\{F:F\subseteq Y,\,G\cap\phi^{-1}[F]\in\Sigma\}$,}

\Centerline{$\nu F=\int J\times\chi(\phi^{-1}[F])d\mu$ for every
$F\in\Tau$.}

\noindent Then, for any real-valued function $g$ defined on a subset
of $Y$,
$\int J\times g\phi\,d\mu=\int g\,d\nu$ whenever either integral is
defined in $[-\infty,\infty]$, provided that we interpret
$(J\times g\phi)(x)$ as $0$ when $J(x)=0$ and $g(\phi(x))$ is undefined.

\proof{ If $g$ is $\Tau$-measurable and defined almost everywhere, this
is a consequence of 235E.   So I have to show that if $J\times g\phi$ is
measurable and defined almost everywhere, so is $g$.
Set $W=Y\setminus\dom g$.   Then $J\times g\phi$ is undefined on
$G\cap\phi^{-1}[W]$, because $g\phi$ is undefined there and we cannot
take advantage of the escape clause available when $J=0$;  so
$G\cap\phi^{-1}[W]$ must be negligible, therefore measurable, and
$W\in\Tau$.   Next,

\Centerline{$\nu W=\int J\times\chi(\phi^{-1}[W])=0$}

\noindent because $J\times\chi(\phi^{-1}[W])$ can be non-zero only on
the negligible set $G\cap\phi^{-1}[W]$.   So $g$ is defined almost
everywhere.

Note that the hypothesis surely implies that $J\times\chi
D_{\phi}=J\times\chi(\phi^{-1}[Y])$ is measurable, so that $J$ is
measurable (because $D_{\phi}$ is conegligible) and $G\in\Sigma$.
Writing $K(x)=1/J(x)$ for $x\in G$, $0$ for $x\in X\setminus G$, the
function $K:X\to\Bbb R$ is measurable, and

\Centerline{$\chi G\times g\phi=K\times J\times g\phi$}

\noindent is measurable.   So 235I tells
us that $g$ must be measurable, and we're done.
}%end of proof of 235J

\cmmnt{\medskip

\noindent{\bf Remark} When $J=\chi X$, the hypothesis of this theorem
becomes

\Centerline{$\Tau=\{F:F\subseteq Y,\,\phi^{-1}[F]\in\Sigma\}$,
\quad $\nu F=\mu\phi^{-1}[F]$ for every $F\in\Tau$;}

\noindent that is, $\nu$ is the image measure $\mu\phi^{-1}$ as defined
in 234D.
}%end of comment

\leader{235K}{Corollary}
Let $(X,\Sigma,\mu)$ be a complete measure
space, and $J$ a non-negative measurable function defined on a
conegligible subset of $X$.   Let $\nu$ be the associated
indefinite-integral measure, and $\Tau$ its domain.   Then for any
real-valued function $g$ defined on a subset of $X$,
$g$ is $\Tau$-measurable iff $J\times g$ is $\Sigma$-measurable, and
$\int g\,d\nu=\int J\times g\,d\mu$ if either integral is defined in
$[-\infty,\infty]$, provided that we interpret $(J\times g)(x)$ as $0$
when $J(x)=0$ and $g(x)$ is undefined.

\proof{ Put 235J, taking $Y=X$ and $\phi$ the identity function,
together with 234Ld.
}%end of proof of 235K

\cmmnt{
\leader{235L}{Applying the Radon-Nikod\'ym theorem}
In order to
use 235A-235J effectively, we need to be able to find suitable functions
$J$.   This can be difficult -- some very special examples will take up
most of Chapter 26 below.   But there are many circumstances in
which we can be sure that such $J$ exist, even if we do not know what
they are.    A minimal requirement is that if $\nu F<\infty$ and
$\mu^*\phi^{-1}[F]=0$ then $\nu F=0$, because
$\int J\times\chi(\phi^{-1}[F])d\mu$ will be zero for any $J$.   A
sufficient condition, in the special case of indefinite-integral
measures, is in 234O.   Another is the following.
}%end of comment

\leader{235M}{Theorem}
Let $(X,\Sigma,\mu)$ be a
$\sigma$-finite measure space, $(Y,\Tau,\nu)$ a semi-finite measure
space, and $\phi:D\to Y$ a function such that

(i) $D$ is a conegligible subset of $X$,

(ii) $\phi^{-1}[F]\in\Sigma$ for every $F\in\Tau$;

(iii) $\mu\phi^{-1}[F]>0$ whenever $F\in\Tau$ and $\nu F>0$.

\noindent Then there is a $\Sigma$-measurable function
$J:X\to\coint{0,\infty}$
such that $\int J\times\chi\phi^{-1}[F]\,d\mu=\nu F$ for every
$F\in\Tau$.

\proof{{\bf (a)} To begin with (down to the end of (c) below) let us
suppose that $D=X$ and that $\nu$ is totally finite.

Set $\tildeTau=\{\phi^{-1}[F]:F\in\Tau\}\subseteq\Sigma$.   Then
$\tildeTau$ is a $\sigma$-algebra of subsets of $X$.   \Prf\ (i)

\Centerline{$\emptyset=\phi^{-1}[\emptyset]\in\tildeTau$.}

\noindent (ii) If $E\in\tildeTau$, take $F\in\Tau$ such that
$E=\phi^{-1}[F]$, so that

\Centerline{$X\setminus E=\phi^{-1}[Y\setminus F]\in\tildeTau$.}

\noindent (iii) If $\sequencen{E_n}$ is any sequence in $\tildeTau$,
then for each $n\in\Bbb N$ choose $F_n\in\Tau$ such that
$E_n=\phi^{-1}[F_n]$;  then

\Centerline{$\bigcup_{n\in\Bbb N}E_n=\phi^{-1}[\bigcup_{n\in\Bbb
N}F_n]\in\tildeTau$.  \Qed}

\noindent Next, we have a totally finite measure
$\tilde\nu:\tildeTau\to[0,\nu Y]$ given by setting

\Centerline{$\tilde\nu(\phi^{-1}[F])=\nu F$ for every $F\in\Tau$.}

\noindent\Prf\ (i)  If $F$, $F'\in\Tau$ and
$\phi^{-1}[F]=\phi^{-1}[F']$, then $\phi^{-1}[F\symmdiff F']=\emptyset$,
so $\mu(\phi^{-1}[F\symmdiff F'])=0$ and $\nu(F\symmdiff F')=0$;
consequently $\nu F=\nu F'$.   This shows that $\tilde\nu$ is
well-defined.   (ii) Now

\Centerline{$\tilde\nu\emptyset
=\tilde\nu(\phi^{-1}[\emptyset])
=\nu\emptyset=0$.}

\noindent (iii) If $\sequencen{E_n}$ is a disjoint sequence in
$\tildeTau$, let $\sequencen{F_n}$ be a sequence in $\Tau$ such that
$E_n=\phi^{-1}[F_n]$ for each $n$;  set
$F'_n=F_n\setminus\bigcup_{m<n}F_m$ for each $n$;  then
$E_n=\phi^{-1}[F'_n]$ for each $n$, so

\Centerline{$\tilde\nu(\bigcup_{n\in\Bbb N}E_n)
=\tilde\nu(\phi^{-1}[\bigcup_{n\in\Bbb N}F'_n])
=\nu(\bigcup_{n\in\Bbb N}F'_n)
=\sum_{n=0}^{\infty}\nu F'_n
=\sum_{n=0}^{\infty}\tilde\nu E_n$.   \Qed}

Finally, observe that if $\tilde\nu E>0$ then $\mu E>0$, because
$E=\phi^{-1}[F]$ where $\nu F>0$.

\medskip

{\bf (b)} By 215B(ix) there is a $\Sigma$-measurable function
$h:X\to\ooint{0,\infty}$ such that $\int h\,d\mu$ is finite.   Define
$\tilde\mu:\tildeTau\to\coint{0,\infty}$ by setting
$\tilde\mu E=\int_Eh\,d\mu$ for every $E\in\tildeTau$;  then $\tilde\mu$
is a totally
finite measure.   If $E\in\tildeTau$ and $\tilde\mu E=0$, then (because
$h$ is strictly positive) $\mu E=0$ and $\tilde\nu E=0$.
Accordingly we may apply the Radon-Nikod\'ym theorem to $\tilde\mu$ and
$\tilde\nu$ to see that there is a $\tildeTau$-measurable function
$g:X\to\Bbb R$
such that $\int_Eg\,d\tilde\mu=\tilde\nu E$ for every $E\in\tildeTau$.
Because $\tilde\nu$ is non-negative, we may suppose that $g\ge 0$.

\medskip

{\bf (c)} Applying 235A to $\mu$, $\tilde\mu$, $h$ and the identity
function from $X$ to itself, we see that

\Centerline{$\int_Eg\times h\,d\mu=\int_Eg\,d\tilde\mu
=\tilde\nu E$}

\noindent for every $E\in\tildeTau$, that is, that

\Centerline{$\int J\times\chi(\phi^{-1}[F])d\mu = \nu F$}

\noindent for every $F\in\Tau$, writing $J=g\times h$.

\medskip

{\bf (d)} This completes the proof when $\nu$ is totally finite and
$D=X$.   For the general case, if $Y=\emptyset$ then $\mu X=0$ and the
result is trivial.   Otherwise, let $\hat\phi$ be any extension of
$\phi$ to a function from $X$ to $Y$ which is constant on
$X\setminus D$;  then
$\hat\phi^{-1}[F]\in\Sigma$ for every $F\in\Tau$, because
$D=\phi^{-1}[Y]\in\Sigma$ and $\hat\phi^{-1}[F]$ is always either
$\phi^{-1}[F]$ or $(X\setminus D)\cup\phi^{-1}[F]$.   Now $\nu$ must be
$\sigma$-finite.
\Prf\ Use the criterion of 215B(ii).   If $\Cal F$ is a disjoint family
in $\{F:F\in\Tau,\,0<\nu F<\infty\}$, then
$\Cal E=\{\hat\phi^{-1}[F]:F\in\Cal F\}$ is a disjoint family in
$\{E:\mu E>0\}$, so $\Cal E$ and $\Cal F$ are countable.   \Qed

Let $\sequencen{Y_n}$ be a partition of $Y$ into sets of finite
$\nu$-measure, and for each $n\in\Bbb N$ set $\nu_nF=\nu(F\cap Y_n)$ for
every $F\in\Tau$.   Then $\nu_n$ is a totally finite measure on $Y$, and
if $\nu_nF>0$ then $\nu F>0$ so

\Centerline{$\mu\hat\phi^{-1}[F]=\mu\phi^{-1}[F]>0$.}

\noindent Accordingly $\mu$, $\hat\phi$ and $\nu_n$ satisfy the
assumptions of the theorem together with those of (a) above, and there
is a $\Sigma$-measurable function $J_n:X\to\coint{0,\infty}$ such that

\Centerline{$\nu_nF=\int J_n\times\chi(\phi^{-1}[F])d\mu$}

\noindent for every $F\in\Tau$.   Now set
$J=\sum_{n=0}^{\infty}J_n\times\chi(\phi^{-1}[Y_n])$, so that
$J:X\to\coint{0,\infty}$ is $\Sigma$-measurable.   If $F\in\Tau$, then

$$\eqalign{\int J\times\chi(\phi^{-1}[F])d\mu
&=\sum_{n=0}^{\infty}
  \int J_n\times\chi(\phi^{-1}[Y_n])\times\chi(\phi^{-1}[F])d\mu\cr
&=\sum_{n=0}^{\infty}\int J_n\times\chi(\phi^{-1}[F\cap Y_n])d\mu
=\sum_{n=0}^{\infty}\nu(F\cap Y_n)
=\nu F,\cr}$$

\noindent as required.
}%end of proof of 235M

\leader{235N}{Remark}
Theorem 235M can fail if $\mu$ is only strictly
localizable rather than $\sigma$-finite.   \prooflet{\Prf\ Let $X=Y$ be
an uncountable set, $\Sigma=\Cal PX$,  $\mu$ counting measure on $X$
(112Bd), $\Tau$ the countable-cocountable $\sigma$-algebra of $Y$, $\nu$
the countable-cocountable measure on $Y$ (211R), $\phi:X\to Y$ the
identity map.   Then $\phi^{-1}[F]\in\Sigma$ and $\mu\phi^{-1}[F]>0$
whenever $\nu F>0$.   But if $J$ is any
$\mu$-integrable function on $X$, then $F=\{x:J(x)\ne 0\}$ is countable
and

\Centerline{$\nu(Y\setminus F)=1\ne 0
=\int_{\phi^{-1}[Y\setminus F]}J\,d\mu$.  \Qed}
}

\leader{*235O}{}\cmmnt{ There are some simplifications 
in the
case of $\sigma$-finite spaces;  in particular, 235A and 235E become
conflated.   I will give an adaptation of the hypotheses of 235A which
may be used in the $\sigma$-finite case.   First a lemma.

\medskip

\noindent}{\bf Lemma} Let $(X,\Sigma,\mu)$ be a measure space and $f$ a
non-negative integrable function on $X$.   If $A\subseteq X$ is such
that $\int_Af+\int_{X\setminus A}f=\int f$, then $f\times\chi A$ is
integrable.

\proof{ By 214Eb, there are $\mu$-integrable functions
$f_1$, $f_2$ such that $f_1$ extends $f\restr A$, $f_2$ extends
$f\restr X\setminus A$, and

\Centerline{$\int_Ef_1=\int_{E\cap A}f$,
\quad $\int_Ef_2=\int_{E\setminus A}f$}

\noindent for every $E\in\Sigma$.   Because $f$ is non-negative,
$\int_Ef_1$ and $\int_Ef_2$ are non-negative for every $E\in\Sigma$, and
$f_1$, $f_2$ are non-negative a.e.   Accordingly we have
$f\times\chi A\leae f_1$ and $f\times\chi(X\setminus A)\leae f_2$, so
that $f\leae f_1+f_2$.   But also

\Centerline{$\int f_1+f_2
=\int_Xf_1+\int_Xf_2=\int_Af+\int_{X\setminus A}f=\int f$,}

\noindent so $f\eae f_1+f_2$.   Accordingly

\Centerline{$f_1\eae f-f_2\leae f-f\times\chi(X\setminus A)
=f\times\chi A\leae f_1$}

\noindent and $f\times\chi A\eae f_1$ is integrable.
}%end of proof of 235O

\leader{*235P}{Proposition}
Let $(X,\Sigma,\mu)$ be a complete
measure space and $(Y,\Tau,\nu)$ a complete $\sigma$-finite measure
space.   Suppose that $\phi:D_{\phi}\to Y$,
$J:D_J\to\coint{0,\infty}$ are functions defined on conegligible subsets
$D_{\phi}$, $D_J$ of $X$ such that
$\int_{\phi^{-1}[F]}J\,d\mu$ exists and is equal to $\nu F$ whenever
$F\in\Tau$ and $\nu F<\infty$.

(a) $J\times g\phi$ is $\Sigma$-measurable for every
$\Tau$-measurable real-valued function $g$ defined on a subset of $Y$.

(b) If $g$ is a $\Tau$-measurable real-valued function defined almost
everywhere in $Y$, then $\int J\times g\phi\,d\mu=\int g\,d\nu$ whenever
either integral is defined in $[-\infty,\infty]$,  interpreting
$(J\times g\phi)(x)$ as $0$ when $J(x)=0$, $g(\phi(x))$ is undefined.

\proof{ The point is that the hypotheses of 235E are
satisfied.   To see this, let us write $\Sigma_C=\{E\cap C:E\in\Sigma\}$
and $\mu_C=\mu^*\restr \Sigma_C$ for the subspace measure on $C$, for
each $C\subseteq X$.   Let $\sequencen{Y_n}$ be a non-decreasing
sequence of sets with union $Y$ and with $\nu Y_n<\infty$ for every
$n\in\Bbb N$, starting from $Y_0=\emptyset$.

\medskip

{\bf (i)} Take any $F\in\Tau$ with $\nu F<\infty$, and set
$F_n=F\cup Y_n$ for each $n\in\Bbb N$;  write $C_n=\phi^{-1}[F_n]$.

Fix $n$ for the moment.   Then our hypothesis implies that

\Centerline{$\int_{C_0} J\,d\mu+\int_{C_n\setminus C_0} J\,d\mu
=\nu F+\nu(F_n\setminus F)=\nu F_n=\int_{C_n}J\,d\mu$.}

\noindent If we regard the subspace measures on $C_0$ and
$C_n\setminus C_0$ as derived from the measure $\mu_{C_n}$ of $C_n$
(214Ce), then 235O
tells us that $J\times\chi C_0$ is $\mu_{C_n}$-integrable, and there is
a $\mu$-integrable function $h_n$ such that  $h_n$ extends
$(J\times\chi C_0)\restr C_n$.

Let $E$ be a $\mu$-conegligible set, included in the domain $D_{\phi}$
of $\phi$, such that $h_n\restr E$ is $\Sigma$-measurable for every $n$.
Because $\sequencen{C_n}$ is a non-decreasing sequence with union
$\phi^{-1}[\bigcup_{n\in\Bbb N}F_n]
\ifdim\pagewidth=390pt\penalty-100\fi
=D_{\phi}$,

\Centerline{$(J\times\chi C_0)(x)=\lim_{n\to\infty}h_n(x)$}

\noindent for every $x\in E$, and $(J\times\chi C_0)\restr E$ is
measurable.   At the same time, we know that there is a $\mu$-integrable
$h$ extending $J\restr C_0$, and $0\leae J\times\chi C_0\leae|h|$.
Accordingly $J\times \chi C_0$ is integrable, and (using 214F)

\Centerline{$\int J\times\chi\phi^{-1}[F]\,d\mu
=\int J\times\chi C_0\,d\mu
=\int_{C_0}J\restr C_0\,d\mu_{C_0}
=\nu F$.}

\medskip

{\bf (ii)} This deals with $F$ of finite measure.   For general
$F\in\Tau$,

\Centerline{$\int J\times\chi(\phi^{-1}[F])\,d\mu
=\lim_{n\to\infty}\int J\times\chi(\phi^{-1}[F\cap Y_n])\,d\mu
=\lim_{n\to\infty}\nu(F\cap Y_n)
=\nu F$.}

\noindent So the hypotheses of 235E are satisfied, and the result
follows at once.
}%end of proof of 235P

\leader{*235Q}{}\cmmnt{ I remarked in 235Bb that a 
difficulty can
arise in 235A, for general measure spaces, if we speak of
$\int_{\phi^{-1}[F]}J\,d\mu$ in the hypothesis, in
place of $\int J\times\chi(\phi^{-1}[F])d\mu$.   Here is an example.

\medskip

\noindent}{\bf Example} Set $X=Y=[0,2]$.   Write $\Sigma_L$ for the
algebra of Lebesgue measurable subsets of $\Bbb R$, and $\mu_L$ for
Lebesgue measure;  write $\mu_c$ for counting measure on $\Bbb R$.   Set

\Centerline{$\Sigma=\Tau
=\{E:E\subseteq[0,2],\,E\cap\coint{0,1}\in\Sigma_L\}$\dvro{.}{;}}

\noindent\cmmnt{of course this is a $\sigma$-algebra of subsets of
$[0,2]$.  }For $E\in\Sigma=\Tau$, set

\Centerline{$\mu E=\nu
E=\mu_L(E\cap\coint{0,1})+\mu_c(E\cap[1,2])$\dvro{.}{;}}

\noindent\cmmnt{then $\mu$ is a complete measure -- in effect, it is
the direct sum of Lebesgue measure on $\coint{0,1}$ and counting measure
on $[1,2]$ (see 214L).   It is easy to see that

\Centerline{$\mu^*B=\mu_L^*(B\cap\coint{0,1})+\mu_c(B\cap[1,2])$}

\noindent for every $B\subseteq[0,2]$.   }Let $A\subseteq\coint{0,1}$ be
a non-Lebesgue-measurable set such that $\mu_L^*(E\setminus A)=\mu_LE$
for every Lebesgue measurable
$E\subseteq\coint{0,1}$\cmmnt{ (see 134D)}.   Define
$\phi:[0,2]\to[0,2]$ by
setting $\phi(x)=x+1$ if $x\in A$, $\phi(x)=x$ if $x\in[0,2]\setminus
A$.

If $F\in\Sigma$, then $\mu^*(\phi^{-1}[F])=\mu F$.   \prooflet{\Prf\
{\bf (i)} If
$F\cap[1,2]$ is finite, then $\mu F=\mu_L(F\cap[0,1])+\#(F\cap[1,2])$.
Now

\Centerline{$\phi^{-1}[F]=(F\cap\coint{0,1}\setminus
A)\cup(F\cap[1,2])\cup\{x:x\in A,\,x+1\in F\}$;}

\noindent as the last set is finite, therefore $\mu$-negligible,

\Centerline{$\mu^*(\phi^{-1}[F])
=\mu_L^*(F\cap\coint{0,1}\setminus A)+\#(F\cap[1,2])
=\mu_L(F\cap\coint{0,1})+\#(F\cap[1,2])
=\mu F$.}

\noindent {\bf (ii)} If $F\cap[1,2]$ is infinite, so is
$\phi^{-1}[F]\cap[1,2]$, so

\Centerline{$\mu^*(\phi^{-1}[F])=\infty=\mu F$.   \Qed}
}

This means that if we set $J(x)=1$ for every $x\in[0,2]$,
\cmmnt{

\Centerline{$\int_{\phi^{-1}[F]}J\,d\mu
=\mu_{\phi^{-1}[F]}(\phi^{-1}[F])
=\mu^*(\phi^{-1}[F])
=\mu F$}

\noindent for every $F\in\Sigma$, and} $\phi$, $J$ satisfy the amended
hypotheses for 235A.   But if we set $g=\chi\coint{0,1}$, then $g$ is
$\mu$-integrable, with $\int g\,d\mu=1$, while
\cmmnt{

\Centerline{$J(x)g(\phi(x))=1$ if $x\in[0,1]\setminus A$, $0$
otherwise,}

\noindent so, because $A\notin\Sigma$,} $J\times g\phi$ is
not\cmmnt{ measurable, and therefore (since $\mu$ is complete)
not} $\mu$-integrable.

\leader{235R}{Reversing the \dvrocolon{burden}}\cmmnt{ 
Throughout the work above, I have been using the formula

\Centerline{$\int J\times g\phi=\int g$,}

\noindent as being the natural extension of the formula

\Centerline{$\int g=\int g\phi\times \phi'$}

\noindent of ordinary advanced calculus.   But we can also move the
`derivative' $J$ to the other side of the equation, as follows.

\medskip

\noindent}{\bf Theorem} Let $(X,\Sigma,\mu)$, $(Y,\Tau,\nu)$ be measure
spaces and $\phi:X\to Y$, $J:Y\to\coint{0,\infty}$ functions such that
$\int_FJ\,d\nu$ and $\mu\phi^{-1}[F]$ are defined in $[0,\infty]$ and
equal for every $F\in\Tau$.   Then
$\int g\phi\,d\mu=\int J\times g\,d\nu$
whenever $g$ is $\nu$-virtually measurable and defined $\nu$-almost
everywhere and either integral is defined in $[-\infty,\infty]$.

\proof{ Let $\nu_1$ be the indefinite-integral measure over $\nu$
defined by $J$, and $\hat\mu$ the completion of $\mu$.   Then $\phi$ is
\imp\ for $\hat\mu$ and $\nu_1$.   \Prf\ If $F\in\Tau$, then
$\nu_1F=\int_FJ\,d\nu=\mu\phi^{-1}[F]$;  that is, $\phi$ is \imp\ for
$\mu$ and $\nu_1\restr\Tau$.   Since $\nu_1$ is the completion of
$\nu_1\restr\Tau$ (234Lb), $\phi$ is \imp\ for $\mu$ and $\nu_1$
(234Ba).\ \Qed

Of course we can also regard $\nu_1$ as being an indefinite-integral
measure over the completion $\hat\nu$ of $\nu$ (212Fb).   So if $g$ is
$\nu$-virtually measurable and defined $\nu$-almost everywhere,

\Centerline{$\int J\times g\,d\nu
=\int J\times g\,d\hat\nu
=\int g\,d\nu_1
=\int g\phi\,d\hat\mu
=\int g\phi\,d\mu$}

\noindent if any of the five integrals is defined in $[-\infty,\infty]$,
by 235K, 235Gb and 212Fb again.
}%end of proof of 235R

\exercises{
\leader{235X}{Basic exercises (a)}
%\spheader 235Xa
Explain what 235A tells us when $X=Y$,
$\Tau=\Sigma$, $\phi$ is the identity function and $\nu E=\alpha\mu E$
for every $E\in\Sigma$.
%235A

\spheader 235Xb Let $(X,\Sigma,\mu)$ be a measure space, $J$ an
integrable non-negative real-valued function on $X$, and
$\phi:D_{\phi}\to\Bbb R$ a measurable function, where $D_{\phi}$ is a
conegligible subset of $X$.   Set

\Centerline{$g(a)=\int_{\{x:\phi(x)\le a\}}J$}

\noindent for $a\in\Bbb R$, and let $\mu_g$ be the Lebesgue-Stieltjes
measure associated with $g$.   Show that
$\int J\times f\phi\,d\mu=\int fd\mu_g$ for every $\mu_g$-integrable
real function $f$.
%235A

\spheader 235Xc Let $\Sigma$, $\Tau$ and $\Lambda$ be
$\sigma$-algebras of subsets of $X$, $Y$ and $Z$ respectively.   Let us
say that a function $\phi:A\to Y$, where $A\subseteq X$, is
$(\Sigma,\Tau)$-measurable if $\phi^{-1}[F]\in\Sigma_A$, the subspace
$\sigma$-algebra of $A$, for every $F\in\Tau$.   Suppose that
$A\subseteq X$, $B\subseteq Y$, $\phi:A\to Y$ is
$(\Sigma,\Tau)$-measurable and $\psi:B\to Z$ is
$(\Tau,\Lambda)$-measurable.   Show that $\psi\phi$ is
$(\Sigma,\Lambda)$-measurable.   Deduce 235C.
%235C

\spheader 235Xd Let $(X,\Sigma,\mu)$ be a measure
space and $(Y,\Tau,\nu)$ a semi-finite measure space.   Let
$\phi:D_{\phi}\to Y$ and $J:D_J\to\coint{0,\infty}$ be
functions defined on conegligible subsets $D_{\phi}$, $D_J$ of $X$
such that $\int J\times\chi(\phi^{-1}[F])d\mu$ exists $=\nu F$ whenever
$F\in\Tau$ and $\nu F<\infty$.   Let $g$ be a
$\Tau$-measurable
real-valued function, defined on a conegligible subset of $Y$.   Show
that $J\times g\phi$ is $\mu$-integrable iff $g$ is $\nu$-integrable,
and the integrals are then equal, provided we interpret
$(J\times g\phi)(x)$ as
$0$ when $J(x)=0$ and $g(\phi(x))$ is undefined.
%235E

\spheader 235Xe Let $(X,\Sigma,\mu)$ be a measure space and
$E\in\Sigma$.   Define a measure $\mu\LLcorner E$ on $X$ by setting
$(\mu\LLcorner E)(F)=\mu(E\cap F)$ whenever $F\subseteq X$ is such that
$F\cap E\in\Sigma$.   Show that, for any function $f$ from a subset of
$X$ to $[-\infty,\infty]$, $\int fd(\mu\LLcorner E)=\int_Efd\mu$ if
either is defined in $[-\infty,\infty]$.
%235G

\sqheader 235Xf Let $g:\Bbb R\to\Bbb R$ be a
non-decreasing function which is absolutely continuous on
every closed bounded interval, and $\mu_g$ the associated
Lebesgue-Stieltjes measure (114Xa, 225Xd).   Write $\mu$ for Lebesgue
measure on $\Bbb R$, and let $f:\Bbb R\to\Bbb R$ be a function.   Show
that $\int f\times g'\,d\mu=\int f\,d\mu_g$ in the sense that if one of
the integrals exists, finite or infinite, so does the other, and they
are then equal.
%235J

\spheader 235Xg Let $g:\Bbb R\to\Bbb R$ be a
non-decreasing function and $J$ a non-negative real-valued
$\mu_g$-integrable function, where $\mu_g$ is the Lebesgue-Stieltjes
measure defined from $g$.   Set
$h(x)=\int_{\ocint{-\infty,x}}J\,d\mu_g$ for each $x\in\Bbb R$, and let
$\mu_h$ be the Lebesgue-Stieltjes measure associated with $h$.   Show
that, for any $f:\Bbb R\to\Bbb R$,
$\int f\times J\,d\mu_g=\int fd\mu_h$, in the sense that if one of the
integrals is defined in $[-\infty,\infty]$ so is the other, and they are
then equal.
%235J, 235Xf

\sqheader 235Xh Let $X$ be a set and $\lambda$, $\mu$, $\nu$ three
measures on $X$ such that $\mu$ is an indefinite-integral measure over
$\lambda$, with Radon-Nikod\'ym derivative $f$, and $\nu$ is an
indefinite-integral measure over $\mu$, with Radon-Nikod\'ym derivative
$g$.   Show that $\nu$ is an indefinite-integral measure over $\lambda$,
and that $f\times g$ is a Radon-Nikod\'ym derivative of $\nu$ with
respect to $\lambda$, provided we interpret $(f\times g)(x)$ as $0$ when
$f(x)=0$ and $g(x)$ is undefined.
%235K

\spheader 235Xi In 235M, if $\nu$ is not semi-finite, show that
we can still find a $J$ such that $\int_{\phi^{-1}[F]}J\,d\mu=\nu F$ for
every set $F$ of finite measure.   \Hint{use the `semi-finite version'
of $\nu$, as described in 213Xc.}
%235M

\spheader 235Xj Let $(X,\Sigma,\mu)$ be a $\sigma$-finite
measure space, and $\Tau$ a $\sigma$-subalgebra of $\Sigma$.   Let
$\nu:\Tau\to\Bbb R$ be a countably additive functional such that
$\nu F=0$ whenever $F\in\Tau$ and $\mu F=0$.   Show that there is a
$\mu$-integrable function $f$ such that $\int_Ffd\mu=\nu F$ for every
$F\in\Tau$.   \Hint{use the method of 235M, applied to the positive and
negative parts of $\nu$.}
%235M

\spheader 235Xk Let $(X,\Sigma,\mu)$ and $(Y,\Tau,\nu)$ be
measure spaces, with completions $(X,\hat\Sigma,\hat\mu)$ and
$(Y,\hat\Tau,\hat\nu)$.   Let $\phi:D_{\phi}\to Y$,
$J:D_J\to \coint{0,\infty}$ be functions defined on conegligible subsets
of $X$. Show that if $\int_{\phi^{-1}[F]}J\,d\mu=\nu F$ whenever
$F\in\Tau$ and
$\nu F<\infty$, then $\int_{\phi^{-1}[F]}J\,d\mu=\nu F$ whenever
$F\in\hat\Tau$ and $\hat\nu F<\infty$.   Hence, or otherwise, show that
235Pb is valid for non-complete spaces $(X,\Sigma,\mu)$ and
$(Y,\Tau,\nu)$.
%235P

\spheader 235Xl Let $(X,\Sigma,\mu)$ be a complete measure space, 
$Y$ a set, $\phi:X\to Y$ a function and $\nu=\mu\phi^{-1}$ the 
corresponding image measure on $Y$.   Let $\nu_1$ be an
indefinite-integral measure over $\nu$.   Show that there is an
indefinite-integral measure $\mu_1$ over $\mu$ such that $\nu_1$ is the 
image measure $\mu_1\phi^{-1}$.
%??

\spheader 235Xm
 Let $(X,\Sigma,\mu)$ and $(Y,\Tau,\nu)$ be measure spaces,  
and $\phi:X\to Y$ an \imp\ function.   Let $\nu_1$ be an 
indefinite-integral measure over $\nu$.   Show that there is an 
indefinite-integral measure $\mu_1$ over $\mu$ such that $\phi$ is \imp\  
for $\mu_1$ and $\nu_1$. 
%235G out of order query 

\spheader 235Xn\dvAnew{2013}  
Let $(X,\Sigma,\mu)$ and $(Y,\Tau,\nu)$ be 
measure spaces, and $\phi:X\to Y$ an \imp\ function.   Show that
$\overline{\int}h\phi\,d\mu\le\overline{\int}h\,d\nu$ for every
real-valued function $h$ defined almost everywhere in $Y$.
(Compare 234Bf.)
%235G 

\leader{235Y}{Further exercises (a)} 
%\spheader 235Ya
Write $\Tau$ for the algebra of Borel subsets of
$Y=[0,1]$, and $\nu$ for the restriction of Lebesgue measure to $\Tau$.
Let $A\subseteq[0,1]$ be a set such that both $A$ and $[0,1]\setminus A$
have Lebesgue outer measure $1$, and set $X=A\cup[1,2]$.   Let $\Sigma$
be the algebra of relatively Borel subsets of $X$, and set 
$\mu E=\mu_A(A\cap E)$ for $E\in\Sigma$, where $\mu_A$ is the subspace
measure induced on $A$ by Lebesgue measure.   Define $\phi:X\to Y$ by
setting $\phi(x)=x$ if $x\in A$, $x-1$ if $x\in X\setminus A$.   Show
that $\nu$ is the image measure $\mu\phi^{-1}$, but that,
setting $g=\chi([0,1]\setminus A)$, $g\phi$ is $\mu$-integrable while
$g$ is not $\nu$-integrable.

\spheader 235Yb Let $(X,\Sigma,\mu)$ be a probability space and $\Tau$ a
$\sigma$-subalgebra of $\Sigma$.   Let $f$ be a non-negative
$\mu$-integrable function with $\int fd\mu=1$, so that its
indefinite-integral measure $\nu$ is a probability measure.   Let $g$ be
a $\nu$-integrable real-valued function and set $h=f\times g$,
intepreting
$h(x)$ as $0$ if $f(x)=0$ and $g(x)$ is undefined.   Let $f_1$, $h_1$ be
conditional expectations of $f$, $h$ on $\Tau$ with respect to the
measure $\mu$, and set $g_1=h_1/f_1$, interpreting $g_1(x)$ as $0$ if
$h_1(x)=0$ and $f_1(x)$ is either $0$ or undefined.   Show that $g_1$ is
a conditional expectation of $g$ on $\Tau$ with respect to the measure
$\nu$.
}%end of exercises

\endnotes{
\Notesheader{235} I see that I have taken up a great deal of space in
this section with technicalities;  the hypotheses of the theorems vary
erratically, with completeness, in particular, being invoked at
apparently arbitrary intervals, and ideas repeat themselves in a
haphazard pattern.   There
is nothing deep, and most of the work consists in laboriously verifying
details.   The trouble with this topic is that it is useful.   The
results here are abstract expressions of integration-by-substitution;
they have applications all over measure theory.
I cannot therefore content myself with theorems which will elegantly
express the underlying ideas, but must seek formulations which I can
quote in later arguments.

I hope that the examples in 235Bb, 235H, 235N, 235Q, 234Ya and
235Ya will go some way to persuade you that there are real traps for the
unwary, and that the careful verifications written out at such length
are necessary.   On the other hand, it is happily the case that in
simple contexts, in which the measures $\mu$, $\nu$ are $\sigma$-finite
and the transformations $\phi$ are Borel isomorphisms, no insuperable
difficulties arise, and in particular the image measure catastrophe does
not trouble us.   But for further work in this direction I refer you to
the applications in \S263, \S265 and \S271, and to Volume 4.
}%end of notes

\frnewpage

