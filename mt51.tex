\frfilename{mt51.tex} 
\versiondate{3.1.15} 
\copyrightdate{2003} 
      
\def\chaptername{Cardinal functions} 
\newchapter{51} 
      
The primary object of this volume is to explore those topics in measure 
theory in which questions arise which are undecided by the ordinary 
axioms of set theory.   We immediately face a new kind of interaction 
between the propositions we consider.   If two statements are 
undecidable, we can ask whether either implies the other.   Almost at 
once we find ourselves trying to make sense of a bewildering tangle of 
uncoordinated patterns.   The most successful method so far found of 
listing the multiple connexions present is to reduce as many arguments 
as possible to investigations of the relationships between specially 
defined cardinal numbers.   In any particular model of set theory (so 
long as we are using the axiom of choice) these numbers must be in a 
linear order, so we can at least estimate the number of potential 
configurations, and focus our attention on the possibilities which seem 
most accessible or most interesting.   At the very beginning of the 
theory, for instance, we can ask whether $\frak c=2^{\omega}$ is equal 
to $\omega_1$, or $\omega_2$, or $\omega_{\omega_1}$, or $2^{\omega_1}$. 
For Lebesgue measure, perhaps the first question to ask is:  if 
$\ofamily{\xi}{\omega_1}{E_{\xi}}$ is a family of measurable sets, is 
$\bigcup_{\xi<\omega_1}E_{\xi}$ necessarily measurable?   If the 
continuum hypothesis is true, certainly not;  but if $\frak c>\omega_1$, 
either `yes' or `no' becomes possible.   The way in which it is now 
customary to express this is to say that 
`$\omega_1\le\add\Cal N\le\frak c$, and $\omega_1\le\add\Cal N<\frak c$, 
$\omega_1<\add\Cal N\le\frak c$ and $\omega_1<\add\Cal N<\frak c$ are 
all possible', where $\add\Cal N$ is defined as the least cardinal of 
any family $\Cal E$ of Lebesgue measurable sets such that 
$\bigcup\Cal E$ is not measurable.   (Actually it is not usually defined 
in quite this way, but that is what it comes to.) 
      
At this point I suggest that you turn to 522B, where you will find a 
classic picture (`Cicho\'n's diagram') of the relationships between ten 
cardinals intermediate between $\omega_1$ and $\frak c$, with 
$\add\Cal N$ immediately above $\omega_1$.   As this diagram already 
makes clear, one can define rather a lot of cardinal numbers. 
Furthermore, the relationships between them are not entirely expressible 
in terms of the partial order in which we say that 
$\kappa_{\frak a}\preceq\kappa_{\frak b}$ if we can prove in ZFC that 
$\kappa_{\frak a}\le\kappa_{\frak b}$.   Even in Cicho\'n's diagram we 
have results of the type $\add\Cal M=\min(\frak b,\cov\Cal M)$ in which 
three cardinals are involved.   It is clear that the framework which has 
been developed over the last thirty-five years is only a beginning. 
Nevertheless, I am confident that it will maintain a leading role as the 
theory evolves.   The point is that at least some of the cardinals 
($\add\Cal N$, $\frak b$ and $\cov\Cal M=\frakmctbl$, for instance) 
describe such 
important features of such important structures that they appear 
repeatedly in arguments relating to diverse topics, and give us a chance 
to notice unexpected connexions. 
      
The first step is to list and classify the relevant cardinals.   This is 
the purpose of the present chapter.   In fact the definitions here are 
mostly of a general type.   Associated with any ideal of sets, for 
instance, we have four cardinals (`additivity', `cofinality', 
`unformity' and `covering number';  see 511F).   Most of the cardinals 
examined in this volume can be defined by one of a limited number of 
processes from some more or less naturally arising structure;  thus 
$\add\Cal N$, already mentioned, is normally defined as the additivity 
of the ideal of Lebesgue negligible subsets of $\Bbb R$, and 
$\cov\Cal M$ is the covering number of the ideal of meager subsets of  
$\Bbb R$.   Another 
important type of definition is in terms of whole classes of structure: 
thus Martin's cardinal $\frak m$ can be regarded as the least Martin 
number (definition:  511Dg) of any ccc Boolean algebra. 
      
\S511 lists some of the cardinals associated with partially ordered 
sets, Boolean algebras, topological spaces and ideals of sets.   Which 
structures count as `naturally arising' is a matter of taste and 
experience, but it turns out that many important ideas can be expressed 
in terms of cardinals associated with relations, and some of these are 
investigated in \S512.   The core ideas of the chapter are most clearly  
manifest in their application to partially ordered sets,  
which I look at in \S513.   In \S514 I run through the elementary results 
connecting the cardinal functions of topological spaces and associated 
Boolean algebras and partially ordered sets.   \S515 is a brief 
excursion into abstract Boolean algebra. 
\S516 is a discussion of `precalibers'.   \S517 is an introduction to 
the theory of `Martin numbers', which (following the principles I have 
just tried to explain) I will use as vehicles for the arguments which 
have been used to make deductions from Martin's axiom.    
\S518 gives results on Freese-Nation numbers and tight filtrations of  
Boolean algebras which can be expressed in general terms  
and are relevant to questions in measure theory. 
      
\discrpage 
      
