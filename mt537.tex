\frfilename{mt537.tex}
\versiondate{12.8.13}
\copyrightdate{2005}

\def\shrplusN{\shr^+\mskip-4mu\Cal N}

\def\chaptername{Topologies and measures III}
\def\sectionname{Sierpi\'nski sets, shrinking numbers
and strong Fubini theorems}

\newsection{537}

\leaveitout{
A defn Sierp set
B elem facts
C defn entangled
D elem facts
E lemma
F Sierp > entangled
G entangled > nonproductive ccc
H scal mable fns
I,J  \iint f_y(u_x)dxdy = \iint f_y(u_x)dydx
537K repeated integration:  Shipman
537L Sierp
537M upper integrals
537N,537O,537P,537Q repeated upper/lower integration;  non/shr v cov
537R,537S  repeated integration:  shr/cov
}

W.Sierpi\'nski observed that if the continuum hypothesis is true then there
are uncountable subsets of $\Bbb R$ which have no uncountable negligible
subsets, and that such sets lead to curious phenomena;  he also observed
that, again assuming the continuum hypothesis, there would be a
(non-measurable) function $f:[0,1]^2\to\{0,1\}$ for which Fubini's theorem
failed radically, in the sense that

\Centerline{$\iint f(x,y)dxdy=0$,\quad$\iint f(x,y)dydx=1$.}

\noindent In this section I set out to explore these two insights in the
light of the concepts introduced in Chapter 52.   I start with
definitions of `Sierpi\'nski' and `strongly Sierpi\'nski' set (537A), with
elementary facts and an excursion into the theory of `entangled' sets
(537C-537G).  %537C 537D 537E 537F 537G
Turning to repeated integration, I look at three interesting cases in
which,
for different reasons, some form of separate measurability is enough
to ensure equality of repeated integrals (537I, 537L, 537S).
Working a bit harder, we find
that there can be valid non-trivial inequalities of the form
$\overline{\int}\underline{\int}dxdy
\le\overline{\int}\,\overline{\int}dydx$
(537N-537Q).  %537N 537O 537P 537Q

As elsewhere, I will write $\Cal N(\mu)$ for the null ideal of a measure
$\mu$.

\leader{537A}{Definitions (a)}
If $(X,\Sigma,\mu)$ is a measure space, a subset of $X$ is a
{\bf Sierpi\'nski set} if it is uncountable but meets every negligible set
in a countable set.

\spheader 537Ab If $(X,\Sigma,\mu)$ is a measure space, a subset $A$ of $X$
is a
{\bf strongly Sierpi\'nski set} if it is uncountable and for every $n\ge 1$
and for every set $W\subseteq X^n$ which is negligible for
the\cmmnt{ (c.l.d.)} product measure on $X^n$, the set
$\{u:u\in A^n\cap W$, $u(i)\ne u(j)$ for $i<j<n\}$ is countable.

\leader{537B}{Proposition} (a) Let $(X,\Sigma,\mu)$
be a measure space and $A\subseteq X$ a Sierpi\'nski set.

\quad(i) $\add\Cal N(\mu)=\non\Cal N(\mu)=\omega_1$
and $\cov\Cal N(\mu)\ge\#(A)$.

\quad(ii) If $\{x\}$ is negligible for every $x\in A$, then
$\cf\Cal N(\mu)\ge\cff([\#(A)]^{\le\omega})$.

(b) Suppose that $(X,\Sigma,\mu)$ and $(Y,\Tau,\nu)$ are measure spaces
such that singleton subsets of $Y$ are negligible.   Let
$f:X\to Y$ be an \imp\ function.

\quad(i) If $A\subseteq X$ is a Sierpi\'nski set, then $f[A]$ is a
Sierpi\'nski set in $Y$ and $\#(f[A])=\#(A)$.

\quad(ii) Now suppose that $\nu$ is $\sigma$-finite.
If $A\subseteq X$ is a strongly Sierpi\'nski set, then $f[A]$ is a
strongly Sierpi\'nski set in $Y$.

(c) Suppose that $\lambda$ and $\kappa$ are infinite cardinals and that
$(X,\Sigma,\mu)$ is a locally compact semi-finite measure space of Maharam
type at most $\lambda$ in which singletons are negligible and $\mu X>0$.
Give $\{0,1\}^{\lambda}$ its usual measure.

\quad(i) If $\{0,1\}^{\lambda}$ has a
Sierpi\'nski subset of size $\kappa$, then $X$ has a
Sierpi\'nski subset of size $\kappa$.

\quad(ii) If $\{0,1\}^{\lambda}$ has a
strongly Sierpi\'nski subset of size $\kappa$, then $X$ has a
strongly Sierpi\'nski subset of size $\kappa$.

\proof{{\bf (a)(i)} We are told that $A$ is uncountable;  now any subset of
$A$ with $\omega_1$ members witnesses that $\non\Cal N(\nu)\le\omega_1$.
On the other hand, if $\Cal E$ is a family of negligible sets covering $X$,
then $\#(A)\le\max(\omega,\#(\Cal E))$, so $\#(\Cal E)\ge\#(A)$;  as
$\Cal E$ is arbitrary, $\cov\Cal N(\mu)\ge\#(A)$.

\medskip

\quad{\bf (ii)} If $\{x\}$ is negligible for every $x\in A$, then $[A]^{\le\omega}\subseteq\Cal N(\mu)$,
and the identity function is a Tukey function
from $[A]^{\le\omega}$ to $\Cal N(\mu)$;
so $\cff[A]^{\le\omega}\le\cf\Cal N(\mu)$.

\medskip

{\bf (b)(i)} If $y\in Y$, then $\{y\}$ and
$f^{-1}[\{y\}]$ are negligible, so $A\cap f^{-1}[\{y\}]$ is countable;
consequently $\#(A)\le\max(\omega,\#(f[A]))$ and $\#(f[A])=\#(A)$.
If $F\subseteq Y$ is negligible, then $f^{-1}[F]$ is negligible so
$A\cap f^{-1}[F]$ and $f[A]\cap F$ are countable.   So $f[A]$ is a
Sierpi\'nski set.

\medskip

\quad{\bf (ii)} Let $W\subseteq Y^n$ be a negligible set
for the product measure $\lambda'$ on $Y^n$, where $n\ge 1$.   Define
$\pmb{f}:X^n\to Y^n$ by saying that
$\pmb{f}(x_0,\ldots,x_{n-1})=(f(x_0),\ldots,f(x_{n-1}))$ for
$x_0,\ldots,x_{n-1}\in X$.   Because $\nu$ is $\sigma$-finite,
$\pmb{f}$ is \imp\ for $\lambda$ and $\lambda'$ (251Wp).   
If $W$ is $\lambda'$-negligible,
then $\pmb{f}^{-1}[W]$ is $\lambda$-negligible, and
$B=\{u:u\in A^n\cap\pmb{f}^{-1}[W],\,u(i)\ne u(j)$ for $i<j<n\}$ is
countable.   Consequently

\Centerline{$\{v:v\in f[A]^n\cap W,\,v(i)\ne v(j)\text{ for }i<j<n\}
\subseteq\pmb{f}[B]$}

\noindent is countable.

\medskip

{\bf (c)} Take any set $E\subseteq X$ of non-zero finite measure, and give
$E$ its normalized subspace measure $\mu'_E=(\mu E)^{-1}\mu_E$.
Then there is an
$f:\{0,1\}^{\lambda}\to E$ which is \imp\ for $\nu_{\lambda}$ and
$\mu'_E$ (343Cd).  
So (b) tells us that $E$ has a subset $A$ of size
$\kappa$ which is Sierpi\'nski or strongly Sierpi\'nski for $\mu'_E$.
But now $A$ is still Sierpi\'nski or strongly Sierpi\'nski for $\mu$.
}%end of proof of 537B

\leader{537C}{Entangled sets (a) Definition}
If $X$ is a totally ordered set, then $X$ is {\bf $\omega_1$-entangled}
if whenever $n\ge 1$, $I\subseteq n$ and
$\langle x_{\xi i}\rangle_{\xi<\omega_1,i<n}$
is a family of distinct elements of $X$, then
there are distinct $\xi$, $\eta<\omega_1$ such that
$I=\{i:i<n$, $x_{\xi i}\le x_{\eta i}\}$.

\medskip

{\bf (b)} Give $\{0,1\}^{\Bbb N}$ its lexicographic ordering, that is,

\Centerline{$x\le y$ iff either $x=y$ or there is an $n\in\Bbb N$ such that
$x\restr n=y\restr n$ and $x(n)<y(n)$.}

\noindent Then\cmmnt{ the map
$x\mapsto\bover23\sum_{n=0}^{\infty}3^{-n}x(n):\{0,1\}^{\Bbb N}\to\Bbb R$
is an order-isomorphism between $\{0,1\}^{\Bbb N}$ and the Cantor set, so}
any
$\omega_1$-entangled subset of $\{0,1\}^{\Bbb N}$ can be transferred to an
$\omega_1$-entangled subset of $\Bbb R$.

\leader{537D}{Lemma} Let $X$ be an $\omega_1$-entangled totally ordered set.

(a) There is a countable set $D\subseteq X$ which meets $[x,y]$ whenever
$x<y$ in $X$.

(b) Whenever $n\ge 1$, $I\subseteq n$ and
$\langle x_{\xi i}\rangle_{\xi<\omega_1,i<n}$
is a family of distinct elements of $X$,
there are $\xi<\eta<\omega_1$ such that
$I=\{i:i<n$, $x_{\xi i}\le x_{\eta i}\}$.

\proof{{\bf (a)(i)} There is a countable set $D_0\subseteq X$ which meets
$[x,z]$ whenever $x<y<z$ in $X$.
\Prf\Quer\ Otherwise, choose
$\langle x_{\xi i}\rangle_{\xi<\omega_1,i<3}$ inductively so that
$x_{\xi 0}<x_{\xi 1}<x_{\xi 2}$ and $[x_{\xi 0},x_{\xi 2}]$ does not meet
$\{x_{\eta i}:\eta<\xi$, $i<3\}$.    Now, if $\xi$, $\eta<\omega_1$ are
different, we cannot have

\Centerline{$x_{\xi 0}\le x_{\eta 0}$,
\quad$x_{\xi 1}>x_{\eta 1}$,
\quad$x_{\xi 2}\le x_{\eta 2}$.}

\noindent So $\langle x_{\xi i}\rangle_{\xi<\omega_1,i<3}$ witnesses that
$X$ is not $\omega_1$-entangled.\ \Bang\Qed

\medskip

\quad{\bf (ii)}
Set $A=\{(x,y):x<y$, $[x,y]\cap D_0=\emptyset\}$.   Note that if
$(x,y)$, $(x',y')\in A$ are distinct, then $[x,y]\cap[x',y']=\emptyset$,
since otherwise $[\min(x,x'),\max(y,y')]$ would be an interval
disjoint from $D_0$ with at least three elements.    It follows that $A$ is
countable.
\Prf\Quer\ Otherwise, let $\ofamily{\xi}{\omega_1}{(x_{\xi 0},x_{\xi 1})}$
be a family of
distinct elements of $A$.   Then all the $x_{\xi i}$ are
distinct.   But if $\xi$, $\eta<\omega_1$ are different, we cannot have

\Centerline{$x_{\xi 0}\le x_{\eta 0}$,\quad$x_{\xi 1}>x_{\eta 1}$.}

\noindent So $\langle x_{\xi i}\rangle_{\xi<\omega_1,i<2}$ witnesses that
$X$ is not entangled.\ \Bang\Qed

\medskip

\quad{\bf (iii)} So if we set $D=D_0\cup\{x:(x,y)\in A\}$ we shall have a suitable
countable set.

\medskip

{\bf (b)} For $i<n$ write $\le_i\,=\,\le$ if $i\in I$, $\le_i\,=\,\ge$ if
$i\in n\setminus I$;  we are seeking $\xi<\eta$ such that
$x_{\xi i}\le_i x_{\eta i}$ for every $i<n$.   For each family
$\pmb{d}=\ofamily{i}{n}{d_i}$ in $D$, set
$A_{\pmb{d}}=\{\xi:x_{\xi i}\le_id_i$ for each $i<n\}$.   Let
$\zeta<\omega_1$ be
such that $A_{\pmb{d}}\cap\zeta\ne\emptyset$ whenever $\pmb{d}\in D^n$ and
$A_{\pmb{d}}\ne\emptyset$.   Now there are distinct $\xi'$,
$\eta\in\omega_1\setminus\zeta$ such that $x_{\xi'i}\le_i x_{\eta i}$ for
every $i<n$.   For each $i<n$, there is a $d_i\in D$ such that
$x_{\xi'i}\le_i d_i\le_i x_{\eta i}$.   Set $\pmb{d}=\ofamily{i}{n}{d_i}$;
then $\xi'\in A_{\pmb{d}}$ so there is a $\xi\in\zeta\cap A_{\pmb{d}}$.
Now $\xi<\eta$ and $x_{\xi i}\le_i x_{\eta i}$ for every $i$, as
required.
}%end of proof of 537D

\leader{537E}{Lemma} Suppose that $n\ge 1$, $I\subseteq n$ and that
$A\subseteq(\{0,1\}^{\Bbb N})^n$ is non-negligible for the usual product
measure $\nu_{\Bbb N}^n$ on $(\{0,1\}^{\Bbb N})^n$.
Let $\le$ be the lexicographic
ordering of $\{0,1\}^{\Bbb N}$.   Then there are $v$, $w\in A$
such that $v(i)\ne w(i)$ for every $i<n$ and $\{i:i<n$, $v(i)\le w(i)\}=I$.

\proof{ For each $k\in\Bbb N$ let $\Sigma_k$ be the algebra of subsets of
$X=(\{0,1\}^{\Bbb N})^n$ generated by sets of the form
$\{v:v\in X$, $v(i)(j)=1\}$ for $i<n$ and $j<k$.   Then
$\sequence{k}{\Sigma_k}$ is a non-decreasing sequence of
finite algebras
and the $\sigma$-algebra generated by $\bigcup_{k\in\Bbb N}\Sigma_k$ is the
Borel $\sigma$-algebra $\Cal B(X)$ of $X$.   Let $E\in\Cal B(X)$ be a
measurable envelope of $A$ for $\nu_{\Bbb N}^n$.   For each $k\in\Bbb N$,
let $f_k$ be the conditional expectation of $\chi E$ on $\Sigma_k$,
that is,

\Centerline{$f_k(u)
=2^{kn}\nu_{\Bbb N}^n\{v:v\in E$, $v(i)\restr k=u(i)\restr k$
for every $i<n\}$}

\noindent for $u\in X$.   By L\'evy's martingale theorem (275I),
$\chi E\eae\lim_{k\to\infty}f_k$.   In particular, there are a $u\in A$
and a $k\in\Bbb N$ such that $f_k(u)>1-2^{-n}$.   But this means that

\Centerline{$F=\{v:v\in E$, $v(i)\restr k=u(i)\restr k$ for every $i<n\}$}

\noindent has measure greater than $2^{-kn}(1-2^{-n})$, and both
the sets

\Centerline{$F'=\{v:v\in F$, $v(i)(k)=0$ for $i\in I$, $v(i)(k)=1$ for
$i\in n\setminus I\}$,}

\Centerline{$F''=\{w:w\in F$, $w(i)(k)=1$ for $i\in I$, $w(i)(k)=0$ for
$i\in n\setminus I\}$,}

\noindent must have positive measure.   Accordingly we can find
$v\in A\cap F'$ and $w\in A\cap F''$, and these will serve.
}%end of proof of 537E

\leader{537F}{Corollary} Suppose that $A\subseteq\{0,1\}^{\Bbb N}$ is
strongly Sierpi\'nski for the usual measure on $\{0,1\}^{\Bbb N}$.
Then $A$ is $\omega_1$-entangled for the
lexicographic ordering of $\{0,1\}^{\Bbb N}$.

\proof{ Let $\langle x_{\xi i}\rangle_{\xi<\omega_1,i<n}$ be a family of
distinct points in $A$, where $n\ge 1$, and $I$ a subset of $n$.
Then $x_{\xi}=\ofamily{i}{n}{x_{\xi i}}$ belongs to $A^n$, and has no two
coordinates the same, for every $\xi<\omega_1$.   So
$D=\{x_{\xi}:\xi<\omega_1\}$ cannot be negligible.   By 537E, there are
distinct $\xi$, $\eta<\omega_1$ such that
$I=\{i:x_{\xi i}\le x_{\eta i}\}$.
}%end of proof of 537F

\leader{537G}{Theorem}\cmmnt{ ({\smc Todor\v{c}evi\'c 85})}
Suppose that there is an $\omega_1$-entangled totally ordered set $X$ of
size $\kappa\ge\omega_1$.
Then there are two upwards-ccc partially ordered
sets $P$, $Q$ such that $c^{\uparrow}(P\times Q)\ge\kappa$.

\proof{{\bf (a)} Let $Y\subseteq X$ be a set such that
$\#(Y)=\#(X\setminus Y)=\kappa$, and $f:Y\to X\setminus Y$ an injective
function.   Set

\Centerline{$P
=\{I:I\in[Y]^{<\omega}$, $f\restr I$ is order-preserving$\}$,}

\Centerline{$Q
=\{I:I\in[Y]^{<\omega}$, $f\restr I$ is order-reversing$\}$,}

\noindent both ordered by $\subseteq$.   Then
$\{(\{y\},\{y\}):y\in Y\}$ is an up-antichain in $P\times Q$, so
$c^{\uparrow}(P\times Q)\ge\kappa$.

\medskip

{\bf (b)} $P$ is upwards-ccc.   \Prf\ Let
$\ofamily{\alpha}{\omega_1}{I_{\alpha}}$ be a family in $P$.
By the $\Delta$-system Lemma (4A1Db), there is an uncountable set
$A\subseteq\omega_1$ such that $\family{\alpha}{A}{I_{\alpha}}$ is a
$\Delta$-system with root $I$ say;  now there is an $n\in\Bbb N$ such that
$B=\{\alpha:\alpha\in A$, $\#(I_{\alpha}\setminus I)=n\}$ is uncountable.
If $n=0$ then $I_{\alpha}=I_{\beta}$ are upwards-compatible for any
$\alpha$, $\beta\in B$ and we can stop.

If $n\ge 1$, enumerate $I_{\alpha}\setminus I$ in increasing order as
$\ofamily{i}{n}{x_{\alpha i}}$, for each $\alpha\in B$.
Let $D\subseteq X$ be a countable set such that $D$ meets every interval in
$X$ with more than one member (537Da).   For $i<j<n$ and
$\alpha\in B$ let $d_{\alpha i j}$, $d'_{\alpha i j}\in D$ be such that
$x_{\alpha i}\le d_{\alpha i j}\le x_{\alpha j}$ and
$f(x_{\alpha i})\le d'_{\alpha i j}\le f(x_{\alpha j})$.   (Because
$I_{\alpha}\in P$, $f\restr I_{\alpha}$ is order-preserving so
$f(x_{\alpha i})<f(x_{\alpha j})$.)   Let
$\langle d_{ij}\rangle_{i<j<n}$, $\langle d'_{ij}\rangle_{i<j<n}$
be such that

\Centerline{$C=\{\alpha:\alpha\in B$,
$d_{\alpha ij}=d_{ij}$ and $d'_{\alpha ij}=d'_{ij}$ whenever $i<j<n\}$}

\noindent is uncountable.

Consider the family $\langle y_{\alpha i}\rangle_{\alpha\in C,i<2n}$ where
$y_{\alpha i}=x_{\alpha i}$ and $y_{\alpha,i+n}=f(x_{\alpha i})$ if $i<n$.
Because $X$ is entangled, there must be distinct $\alpha$, $\beta\in C$
such that $y_{\alpha i}\le y_{\beta i}$ for every $i<2n$, that is,
$x_{\alpha i}\le x_{\beta i}$ and $f(x_{\alpha i})\le f(x_{\beta i})$ for
every $i<n$.   But now examine $I=I_{\alpha}\cup I_{\beta}$.   If $x$,
$x'\in I$ and $x\le x'$,

\inset{{\it either} both $x$ and $x'$ belong to $I_{\alpha}$ and
$f(x)\le f(x')$ because $I_{\alpha}\in P$,

{\it or} both $x$ and $x'$ belong to $I_{\beta}$ and
$f(x)\le f(x')$,

{\it or} $x=x_{\alpha i}$ and $x'=x_{\beta j}$ where $i<j<n$, so that

\Centerline{$f(x)=f(x_{\alpha i})\le d'_{ij}\le f(x_{\beta_j})=f(x')$,}

{\it or} $x=x_{\beta i}$ and $x'=x_{\alpha j}$ where $i<j<n$, so that
$f(x)\le f(x')$,

{\it or} $x=x_{\alpha i}$ and $x'=x_{\beta i}$ where $i<n$, so that
$f(x)=f(x_{\alpha i})\le f(x_{\beta i})=f(x')$.}

\noindent (Note that we cannot have $x=x_{\alpha i}$ and $x'=x_{\beta j}$
with $j<i$, because in this case $x_{\beta j}\le d_{ji}\le x_{\alpha i}$
while $x_{\beta j}\ne x_{\alpha i}$;  nor can we have
$x=x_{\beta i}<x'=x_{\alpha i}$ with $i<n$.)
So $f\restr I$ is order-preserving and
$I\in P$ witnesses that $I_{\alpha}$ and $I_{\beta}$ are upwards-compatible
in $P$.   As $\ofamily{\alpha}{\omega_1}{I_{\alpha}}$ is arbitrary,
$P$ is upwards-ccc.\ \Qed

\woddheader{537G}{4}{2}{2}{48pt}

{\bf (c)} Similarly, $Q$ is upwards-ccc.   \Prf\ The principal
changes needed in the argument above are

\inset{----- in the choice of the $d'_{\alpha ij}$, we need to write
`$f(x_{\alpha i})\ge d'_{\alpha i j}\ge f(x_{\alpha j})$';

----- in the choice of particular $\alpha$ and $\beta$ in the set $C$,
we need to write
`$y_{\alpha i}\le y_{\beta i}$ for $i<n$ and
$y_{\alpha i}\ge y_{\beta i}$ for $n\le i<2n$'.  \Qed}

\noindent So $P$ and $Q$ satisfy our requirements.
}%end of proof of 537G

\leader{537H}{Scalarly measurable functions (a) Definition}
Let $X$ be a set, $\Sigma$ a $\sigma$-algebra of
subsets of $X$ and $U$ a linear topological space.
A function $\phi:X\to U$ is {\bf scalarly \hbox{($\Sigma$-)}measurable} if
$f\phi:X\to\Bbb R$ is ($\Sigma$-)measurable for every $f\in U^*$.

\spheader 537Hb If $\phi:X\to U$ is scalarly measurable, $V$ is another
linear topological space and $T:U\to V$ is a continuous linear operator,
then
$T\phi:X\to V$ is scalarly measurable\prooflet{, because $hT\in U^*$ for
every $h\in V^*$}.

\spheader 537Hc If $U$ is a separable metrizable locally convex space
and $\phi:X\to U$ is scalarly measurable, then it is measurable.
\prooflet{\Prf\
$\Tau=\{F:F\subseteq U$, $\phi^{-1}[F]\in\Sigma\}$ includes the
cylindrical $\sigma$-algebra of $U$ (4A3T), which is the Borel
$\sigma$-algebra (4A3V).\ \Qed}

\leader{537I}{Proposition} Let $(X,\Sigma,\mu)$ and $(Y,\Tau,\nu)$ be
probability spaces  and $U$ a reflexive Banach space.   Suppose that
$x\mapsto u_x:X\rightarrow  U$ and $y\mapsto f_y:Y\rightarrow U^*$ are
bounded scalarly measurable functions.   Then
$\iint f_y(u_x)\mu(dx)\nu(dy)$ and 
$\iint f_y(u_x)\nu(dy)\mu(dx)$ are defined and equal.

\proof{{\bf (a)(i)} Recall from 467Hc that if
$V\subseteq U$ and $W\subseteq U^*$ are closed linear subspaces, I call
them a
`projection pair' if $U=V\oplus W^{\smallcirc}$ and $\|v+v'\|\ge\|v\|$ for
all $v\in V$ and $v'\in W^{\smallcirc}$.   We need to know that this
is symmetric;  that is, that in this case

\Centerline{$U^*=W\oplus V^{\smallcirc}$,
\quad$\|g+g'\|\ge\|g\|$ for all $g\in W$, $g'\in V^{\smallcirc}$.}

\noindent\Prf\ Note first that if $g\in W\cap V^{\smallcirc}$, then
$g(u)=0$ for every $u\in W^{\smallcirc}+V$, that is, $g=0$.
Now take any $f\in U^*$.   Define $g:U\to\Bbb R$ by saying that
$g(v+v')=f(v)$ for $v\in V$, $v'\in W^{\smallcirc}$.   Then $g$ is linear
and continuous and $\|g\|\le\|f\|$.   Now $g(v')=0$ for every
$v'\in W^{\smallcirc}$, that is, $g\in W^{\smallcirc\smallcirc}$, which is
the weak*-closure of $W$ (4A4Eg);  but as $U$ and $U^*$ are reflexive, this
is just the norm-closure of $W$, which is equal to $W$.
Set $g'=f-g$.   Then $g'\in V^{\smallcirc}$.   This shows that
$f\in W+V^{\smallcirc}$;  as $f$ is arbitrary,
$U^*=W\oplus V^{\smallcirc}$.   Finally, I remarked in the course of the
argument that $\|g\|\le\|f\|$, which is what we need to know to check that
$\|g\|\le\|g+g'\|$ whenever $g\in W$ and $g'\in V^{\smallcirc}$.\ \Qed

\medskip

\quad{\bf (ii)} Because $U$ is reflexive, its unit ball is weakly compact, so $U$
is surely weakly compactly generated, therefore weakly K-countably
determined (467M).   Now turn to Lemma 467J.   This tells us that
there is a family $\Cal M$ of subsets of $U\cup U^*$ such that

\inset{for every $B\subseteq X\cup X^*$ there is an $M\in\Cal M$ such that
$B\subseteq M$ and $\#(M)\le\max(\omega,\#(B))$;

whenever $\Cal M'\subseteq\Cal M$ is upwards-directed, then
$\bigcup\Cal M'\in\Cal M$;

whenever $M\in\Cal M$ then $(V_M,W_M)$
is a projection pair of subspaces of $U$ and $U^*$,}

\noindent where I write $V_M=\overline{M\cap U}$ and
$W_M=\overline{M\cap U^*}$.   For $M\in\Cal M$,

\Centerline{$U=V_M\oplus W_M^{\smallcirc}$,
\quad$U^*=W_M\oplus V_M^{\smallcirc}$;}

\noindent let $P_M:U\to V_M$ and $Q_M:U^*\to W_M$ be the corresponding
projections.   Since $\|v\|\le\|v+v'\|$ whenever $v\in V_M$ and
$v'\in W_M^{\smallcirc}$, $\|P_M\|\le 1$;  similarly, $\|Q_M\|\le 1$.

If $u\in U$, $f\in U^*$ and $M\in\Cal M$, then

\Centerline{$f(P_Mu)=(Q_Mf)(u)=(Q_Mf)(P_Mu)$.}

\noindent\Prf\ Express $u$ as $v+v'$ and $f$ as $g+g'$, where $v\in V_M$,
$v'\in W_M^{\smallcirc}$, $g\in W_M$ and $g'\in V_M^{\smallcirc}$.   Then

\Centerline{$f(v)=g(v)=g(u)$,}

\noindent that is,

\Centerline{$f(P_Mu)=(Q_Mf)(P_Mu)=(Q_Mf)(u)$.  \Qed}

\medskip

\quad{\bf (iii)} If $M_0$, $M_1\in\Cal M$ and $M_0\subseteq M_1$ then
$P_{M_0}=P_{M_0}P_{M_1}=P_{M_1}P_{M_0}$.   \Prf\ If $u\in U$, express it as
$v_0+v'_0$ where $v_0\in V_{M_0}$ and $v'_0\in W_{M_0}^{\smallcirc}$;
now express $v'_0$ as $v_1+v'_1$ where $v_1\in V_{M_1}$ and
$v'_1\in W_{M_1}^{\smallcirc}$.   Then

\Centerline{$P_{M_0}u=v_0\in V_{M_1}$,}

\noindent so $P_{M_1}P_{M_0}u=P_{M_0}u$.   On the other hand,
$u=v_0+v_1+v'_1$ where $v_0+v_1\in V_{M_1}$ and
$v'_1\in W_{M_1}^{\smallcirc}$,  so $P_{M_1}u=v_0+v_1$;  and as
$v_1=v'_0-v'_1$ belongs to $W_{M_0}^{\smallcirc}$,
$P_{M_0}P_{M_1}u=v_0=P_{M_0}u$.\ \Qed

\medskip

\quad{\bf (iv)} If $\ofamily{\xi}{\zeta}{M_{\xi}}$ is a non-decreasing
family in $\Cal M$,
where $\zeta$ is a non-zero limit ordinal, then we know
that $M=\bigcup_{\xi<\zeta}M_{\xi}$ belongs to $\Cal M$.   Now

\Centerline{$P_Mu=\lim_{\xi\uparrow\zeta}P_{M_{\xi}}u$}

\noindent for every $u\in U$, the limit being for the norm topology on $U$.
\Prf\ Let $\epsilon>0$.   We know that $P_Mu\in V_M=\overline{M\cap U}$,
so there is a $u'\in M\cap U$ such that $\|u'-P_Mu\|\le\bover12\epsilon$.
Let $\xi<\zeta$ be such that $u'\in M_{\xi}$.   If $\xi\le\eta<\zeta$, then

$$\eqalignno{\|P_{M_{\eta}}u-P_Mu\|
&=\|P_{M_{\eta}}(P_Mu-u')+P_M(u'-P_Mu)\|\cr
\displaycause{because $u'\in V_{M_{\eta}}$, so $P_Mu'=P_{M_{\eta}}u'=u'$}
&\le 2\|P_Mu-u'\|
\le\epsilon.  \text{ \Qed}\cr}$$

\medskip

\quad{\bf (v)} Similarly,

\Centerline{$Q_{M_0}=Q_{M_0}Q_{M_1}=Q_{M_1}Q_{M_0}$}

\noindent whenever $M_0$, $M_1\in\Cal M$ and $M_0\subseteq M_1$, and

\Centerline{$Q_Mf=\lim_{\xi\uparrow\zeta}Q_{M_{\xi}}f$}

\noindent whenever $f\in U^*$ and $\zeta$ is a non-zero limit ordinal
and $\ofamily{\xi}{\zeta}{M_{\xi}}$ is a non-decreasing family in
$\Cal M$ with union $M$.

\medskip

{\bf (b)} Now let $\Cal M_0$ be $\{M:M\in\Cal M$, $\#(M)\le\omega\}$.
Then there is an $M_0\in\Cal M_0$ such that

\Centerline{$P_{M_0}(u_x)=P_{M}(u_x)\,\,\mu$-a.e.($x$)}

\noindent whenever $M_0\subseteq M\in\Cal M_0$.

\Prf\Quer\ Suppose, if possible,
otherwise.   Then we can choose inductively an increasing
family $\langle M_{\xi}\rangle_{\xi<\omega_1}$ in $\Cal M_0$
such that

\Centerline{$\mu\{x:P_{M_{\xi+1}}(u_x)\ne P_{M_{\xi}}(u_x)\}
>0$ for every $\xi<\omega_1$,}

\Centerline{$M_{\xi}=\bigcup_{\eta<\xi}M_{\eta}$ whenever
$\xi<\omega_1$ is a non-zero countable limit ordinal.}

\noindent (The set of $x$ for which
$P_{M_{\xi+1}}(u_x)\ne P_{M_{\xi}}(u_x)$ is necessarily measurable
because $x\mapsto P_{M_{\xi+1}}u_x-P_{M_{\xi}}u_x$ is scalarly measurable,
by 537Hb, therefore measurable for the norm topology, by 537Hc, since
$V_{M_{\xi+1}}$ is separable.)   Now there must be a $\delta>0$ such that

\Centerline{$A=\{\xi:\xi<\omega_1$, $\mu E_{\xi}\ge\delta\}$}

\noindent is infinite, where

\Centerline{$E_{\xi}
=\{x:\|P_{M_{\xi+1}}(u_x)-P_{M_{\xi}}(u_x)\|\ge\delta\}$}

\noindent for each $\xi<\omega_1$.   But in this case
there must be an $x\in X$ such that

\Centerline{$A'=\{\xi:\xi\in A$, $x\in E_{\xi}\}$}

\noindent is infinite.   (Take a sequence $\sequencen{\xi_n}$ of distinct
points in $A$, and $x\in\bigcap_{n\in\Bbb N}\bigcup_{m\ge n}E_{\xi_m}$.)
Let $\zeta$ be any cluster point of $A'$ in $\omega_1$.   Then

\Centerline{$P_{M_{\zeta}}(u_x)
=\lim_{\xi\uparrow\zeta}P_{M_{\xi}}(u_x)$}

\noindent ((a-iv) above), which is impossible.\ \Bang\Qed

\medskip


{\bf (c)} Similarly, there is an $M_1\in\Cal M_0$ such that
$M_1\supseteq M_0$ and

\Centerline{$P_{M_1}(f_y)=P_M(f_y)\,\,\nu$-a.e.($y$)}

\noindent whenever $M_1\subseteq M\in\Cal M_0$.
Because $x\mapsto P_{M_1}(u_x)$ and $y\mapsto Q_{M_1}(f_y)$
are scalarly measurable maps to norm-separable spaces, they are
norm-measurable;  again because $V_{M_1}$ and $W_{M_1}$ are separable,
$(x,y)\to(P_{M_1}u_x,Q_{M_1}f_y):X\times Y\to V_{M_1}\times W_{M_1}$ is
$\Sigma\tensorhat\Tau$-measurable (418Bd).   Because
$(f,x)\mapsto f(x):U^*\times U\to\Bbb R$ is norm-continuous,
$(x,y)\mapsto(Q_{M_1}f_y)(P_{M_1}u_x)$ is
$\Sigma\tensorhat\Tau$-measurable, and

\Centerline{$\iint(Q_{M_1}f_y)(P_{M_1}u_x)\mu(dx)\nu(dy)
=\iint(Q_{M_1}f_y)(P_{M_1}u_x)\nu(dy)\mu(dx)$}

\noindent by Fubini's theorem (252C).

Now observe that if $y\in Y$ there is an $M\in\Cal M_0$ such that
$M_1\subseteq M$ and $f_y\in M$.   So

$$\eqalign{\int f_y(u_x)\mu(dx)
&=\int (Q_Mf_y)(u_x)\mu(dx)
=\int f_y(P_Mu_x)\mu(dx)\cr
&=\int f_y(P_{M_1}u_x)\mu(dx)
=\int(Q_{M_1}f_y)(P_{M_1}u_x)\mu(dx).\cr}$$

\noindent This is true for every $y$.   So
$\iint f_y(u_x)\mu(dx)\nu(dy)$ is defined and equal to
$\iint(Q_{M_1}f_y)(P_{M_1}u_x)\mu(dx)\nu(dy)$.
Similarly,

\Centerline{$\iint f_y(u_x)\nu(dy)\mu(dx)
= \iint(Q_{M_1}f_y)(P_{M_1}u_x)\nu(dy)\mu(dx)$.}

\noindent Putting these together, we have the result.
}%end of proof of 537I

\leader{537J}{Corollary} Let $(X,\Sigma,\mu)$, $(Y,\Tau,\nu)$ and
$(Z,\Lambda,\sigma)$ be
probability spaces.   Let $x\mapsto U_x:X\rightarrow\Lambda$
and $y\mapsto V_y:Y\rightarrow\Lambda$ be functions such that

\Centerline{$x\mapsto\sigma(U_x\cap W)$,
\quad$y\mapsto\sigma(V_y\cap W)$}

\noindent are measurable for every $W\in\Lambda$.   Then
$\iint\sigma(U_x\cap V_y)\mu(dx)\nu(dy)$ and
$\iint\sigma(U_x\cap V_y)\nu(dy)\mu(dx)$ are defined and equal.

\proof{{\bf (a)} For $x\in X$ set $u_x=(\chi U_x)^{\ssbullet}$ in
$L^2(\sigma)$.   Then $x\mapsto u_x$ is scalarly measurable.   \Prf\
If $f\in U^*$, there is a $v\in L^2(\sigma)$ such that
$f(u)=\int u\times v$ for every $u\in L^2(\sigma)$ (244K).
Let $\epsilon>0$.   Then there are $W_0,\ldots,W_n\in\Lambda$ and
$\alpha_0,\ldots,\alpha_n\in\Bbb R$ such that
$\|v-\sum_{i=0}^n\alpha_i(\chi W_i)^{\ssbullet}\|_2\le\epsilon$ (244Ha),
so that

\Centerline{$|f(u_x)-\sum_{i=0}^n\alpha_i\sigma(U_x\cap W_i)|
=|\int u_x\times v-\int u_x\times\sum_{i=0}^n\alpha_i(\chi W_i)^{\ssbullet}|
\le\epsilon\|u_x\|_2\le\epsilon$}

\noindent for every $x\in X$.   Now the function
$x\mapsto\sum_{i=0}^n\alpha_i\sigma(U_x\cap W_i)$ is $\Sigma$-measurable.
So we see that the function $x\mapsto f(u_x)$ is uniformly approximated by
$\Sigma$-measurable functions and is itself $\Sigma$-measurable.   As $f$
is arbitrary, $x\mapsto u_x$ is scalarly measurable.\ \Qed

\medskip

{\bf (b)} Similarly, setting $v_y=(\chi V_y)^{\ssbullet}$ for $y\in Y$,
$y\mapsto v_y:Y\to L^2(\sigma)$ is scalarly measurable.   Identifying
$L^2(\sigma)$ with its dual, 537I tells us that


\Centerline{$\iint (u_x|v_y)\mu(dx)\nu(dy)
= \iint (u_x|v_y)\nu(dy)\mu(dx)$,}

\noindent that is, that

\Centerline{$\iint\sigma(U_x\cap V_y)\mu(dx)\nu(dy)
= \iint\sigma(U_x\cap V_y)\nu(dy)\mu(dx)$.}
}%end of proof of 537J

\leader{537K}{}\cmmnt{ The next few paragraphs
will be concerned with upper and lower integrals.   For the basic theory of
these, see \S133 and 214J.

\medskip

\noindent}{\bf Theorem}\cmmnt{ ({\smc Freiling 86}, {\smc Shipman 90})}
Let $\langle(X_j,\Sigma_j,\mu_j)\rangle_{j\le m}$ be a
finite sequence of probability spaces and $\langle\kappa_j\rangle_{j\le m}$
a sequence of cardinals such that $X_j^{\Bbb N}$, with its
product measure $\mu_j^{\Bbb N}$, has a subset with cardinal $\kappa_j$
which is not covered by $\kappa_{j-1}$ negligible sets (if $j\ge 1$) and is
not negligible (if $j=0$).    Let $f:\prod_{j\le m}X_j\to\Bbb R$ be a
bounded function, and suppose that $\sigma:m+1\to m+1$ and
$\tau:m+1\to m+1$ are permutations.   Set

\Centerline{$I=\underlineint\ldots\underlineint f(x_0,\ldots,x_m)
   dx_{\sigma(m)}\ldots dx_{\sigma(0)}$,}

\Centerline{$I'=\overlineint\ldots\overlineint f(x_0,\ldots,x_m)
   dx_{\tau(m)}\ldots dx_{\tau(0)}$.}

\noindent Then $I\le I'$.

\proof{ Let $M\ge 0$ be such that $|f(x_0,\ldots,x_m)|\le M$ for
all $x_0,\ldots,x_m$.

\medskip

{\bf (a)} Set $Z=\prod_{j\le m}X_j^{\Bbb N}$.
The key fact is that we can find negligible sets
$W(\pmb{u})\subseteq X_k^{\Bbb N}$, for $k\le m$ and
$\pmb{u}\in\prod_{j\le m,j\ne k}X_j^{\Bbb N}$, such that

\Centerline{$I
\le\liminf_{n\to\infty}\Bover1{n+1}\sum_{i=0}^nf(t_{0i},\ldots,t_{mi})$}

\noindent whenever
$\langle t_j\rangle_{j\le m}=\langle\sequence{i}{t_{ji}}\rangle_{j\le m}$
is such that
$t_k\notin W(t_0,\ldots,t_{k-1},t_{k+1},\ldots,t_m)$ for every $k$.
\Prf\ Because the formula

\Centerline{$\liminf_{n\to\infty}\Bover1{n+1}\sum_{i=0}^n
  f(t_{0i},\ldots,t_{mi})$}

\noindent is tolerant of permutations of the coordinates $0,\ldots,m$, it
is enough to consider the case $\sigma(j)=j$ for $j\le m$, so that

\Centerline{$I=\underlineint\ldots\underlineint f(x_0,\ldots,x_m)
   dx_m\ldots dx_0$.}

\medskip

\quad{\bf (i)} Define $D_0,\ldots,D_{m+1}$ as follows.
$D_0=\{\emptyset\}=\prod_{j<0}X_j^{\Bbb N}$.   For $0<k\le m$ let
$D_k$ be the set of those $(t_0,\ldots,t_{k-1})\in\prod_{j<k}X_j^{\Bbb N}$
such that

\Centerline{$I\le\liminf_{n\to\infty}\Bover1{n+1}\sum_{i=0}^n
  \underlineint\ldots\underlineint f(t_{0i},\ldots,t_{k-1,i},x_k,\ldots,x_m)
     dx_m\ldots dx_k$,}

\noindent where $t_j=\sequence{i}{t_{ji}}$ for
$j<k$.   For $k<m$ and
$\pmb{u}=(u_0,\ldots,u_{k-1},u_{k+1},\ldots,u_m)$ in
$\prod_{j\le m,j\ne k}X_j^{\Bbb N}$, set

$$\eqalign{W(\pmb{u})&=\emptyset\text{ if }
  (u_0,\ldots,u_{k-1})\notin D_k,\cr
&=\{t:t\in X_k^{\Bbb N},\,(u_0,\ldots,u_{k-1},t)\notin D_{k+1}\}
  \text{ otherwise}.\cr}$$

\medskip

\quad{\bf (ii)} $W(\pmb{u})\subseteq X_k^{\Bbb N}$ is negligible.
To see this, we need consider only
the case in which $(u_0,\ldots,u_{k-1})$ belongs to $D_k$.
Express $u_j$ as
$\sequence{i}{u_{ji}}$ for $j<k$, and for $i\in\Bbb N$ define
$h_i:X_k\to\Bbb R$ by setting

\Centerline{$h_i(x)=\underlineint\ldots\underlineint
f(u_{0i},\ldots,u_{k-1,i},x,x_{k+1},\ldots,x_m)
  dx_m\ldots dx_{k+1}$}

\noindent for $x\in X_k$.    Now the definition of $D_k$ tells us just that

\Centerline{$I\le\liminf_{n\to\infty}\Bover1{n+1}\sum_{i=0}^n
  \underlineint\ldots\underlineint
     f(u_{0i},\ldots,u_{k-1,i},x_k,\ldots,x_m)dx_m\ldots dx_k$,}

\noindent that is, that

\Centerline{$I\le\liminf_{n\to\infty}\Bover1{n+1}\sum_{i=0}^n
   \underlineint h_i(x)dx$.}

\noindent For each $i\in\Bbb N$ let $g_i:X_k\to[-M,M]$ be a measurable
function such that $g_i(x)\le h_i(x)$ for every $x$ and
$\int g_id\mu_k=\underline{\int}h_id\mu_k$.
Now consider the functions $\tilde g_i:X_k^{\Bbb N}\to\Bbb R$ defined by
setting $\tilde g_i(t)=g_i(t_i)$ for $t=\sequence{i}{t_i}\in X_k^{\Bbb N}$.
We have $\int\tilde g_id\mu_k^{\Bbb N}=\underline{\int}h_id\mu_k$
for each $i$, while $\sequence{i}{\tilde g_i}$ is a uniformly bounded
independent sequence of random variables.   By the strong law of large
numbers in the form 273H,

\Centerline{$\lim_{n\to\infty}\Bover1{n+1}
  \sum_{i=0}^n(\tilde g_i(t)-\int\tilde g_id\mu_k^{\Bbb N})
=0$}

\noindent for almost every $t\in X_k^{\Bbb N}$.   Since

\Centerline{$\liminf_{n\to\infty}\Bover1{n+1}
  \sum_{i=0}^n\int\tilde g_id\mu_k^{\Bbb N}
=\liminf_{n\to\infty}\Bover1{n+1}\sum_{i=0}^n\underlineint h_id\mu_k
\ge I,$}

\noindent we have

$$\eqalign{I
&\le\liminf_{n\to\infty}\Bover1{n+1}\sum_{i=0}^n\tilde g_i(t)
\le\liminf_{n\to\infty}\Bover1{n+1}\sum_{i=0}^nh_i(t_i)\cr
&=\liminf_{n\to\infty}\Bover1{n+1}\sum_{i=0}^n
  \underline{\int}\ldots\underline{\int}
  f(u_{0i},\ldots,u_{k-1,i},t_i,x_{k+1},\ldots,x_m)
    dx_m\ldots dx_{k+1}\cr}$$

\noindent for almost every $t=\sequence{i}{t_i}\in X_k^{\Bbb N}$, that is,
$(u_0,\ldots,u_{k-1},t)\in D_{k+1}$ for almost every $t\in X_k^{\Bbb N}$,
that is, $W(\pmb{u})$ is negligible, as required.

\medskip

\quad{\bf (iii)} Suppose that
$\pmb{t}=(t_0,\ldots,t_m)\in Z$ is such that
$t_k\notin W(t_0,\ldots,t_{k-1},t_{k+1},\ldots,t_m)$ for every $k<m$.
Then $(t_0,\ldots,t_k)\in D_{k+1}$ for every $k$;  in particular,
$\pmb{t}\in D_{m+1}$ and, writing $t_j=\sequence{i}{t_{ji}}$ for each $j$,

\Centerline{$I
\le\liminf_{n\to\infty}\Bover1{n+1}\sum_{i=0}^nf(t_{0i},\ldots,t_{mi})$.
\Qed}

\medskip

{\bf (b)} Similarly, or applying the argument above to $-f$,
we have negligible sets
$W'(\pmb{u})\subseteq X_k^{\Bbb N}$, for $k\le m$ and
$\pmb{u}\in\prod_{j\le m,j\ne k}X_j^{\Bbb N}$, such that

\Centerline{$I'
\ge\limsup_{n\to\infty}\Bover1{n+1}\sum_{i=0}^nf(t_{0i},\ldots,t_{mi})$}

\noindent whenever
$\langle t_j\rangle_{j\le m}=\langle\sequence{i}{t_{ji}}\rangle_{j\le m}$
is such that
$t_k\notin W'(t_0,\ldots,t_{k-1},t_{k+1},\ldots,t_m)$ for every $k$.
Enlarging the
$W'(\pmb{u})$ if necessary, we may suppose that
$W'(\pmb{u})\supseteq W(\pmb{u})$ for every $\pmb{u}$.

\medskip

{\bf (c)} Now the point of the construction is that
we can find a $\pmb{t}=(t_0,\ldots,t_m)\in Z$ such that
$t_k\notin W'(t_0,\ldots,t_{k-1},t_{k+1},\penalty-100\ldots,t_m)$ for
every $k$.   \Prf\ For each $k\le m$ let $A_k\subseteq X_k^{\Bbb N}$
be a non-negligible set
of size $\kappa_k$ which (if $k\ge 1$)
cannot be covered by $\kappa_{k-1}$ negligible sets.
Choose $t_m$, $t_{m-1},\ldots,t_0$ in such a way that

\Centerline{$t_k\in A_k$,
\quad$t_k\notin W(\pmb{u})$ whenever
$\pmb{u}\in\prod_{j<k}A_j\times\prod_{k<j\le m}\{t_j\}$;}

\noindent this is always possible because
$\#(A_0\times\ldots\times A_{k-1})=\kappa_{k-1}$ if $k\ge 1$.\ \Qed

So we get

$$\eqalign{I
&\le\liminf_{n\to\infty}\Bover1{n+1}\sum_{i=0}^nf(t_{0i},\ldots,t_{mi})\cr
&\le\limsup_{n\to\infty}\Bover1{n+1}\sum_{i=0}^nf(t_{0i},\ldots,t_{mi})
\le I',\cr}$$

\noindent as claimed.
}%end of proof of 537K

\leader{537L}{Corollary}
Let $\langle(X_j,\Sigma_j,\mu_j)\rangle_{j\le m}$ be a
finite sequence of probability spaces such that $X_j^{\Bbb N}$, with its
product measure $\mu_j^{\Bbb N}$, has a Sierpi\'nski set with cardinal
$\omega_{j+1}$ for each $j\le m$.    Let $f:\prod_{j\le m}X_j\to\Bbb R$ be a
bounded function, and suppose that $\sigma:m+1\to m+1$ and
$\tau:m+1\to m+1$ are permutations such that the two repeated integrals

\Centerline{$I=\int\ldots\int f(x_0,\ldots,x_m)
   dx_{\sigma(m)}\ldots dx_{\sigma(0)}$,}

\Centerline{$I'=\int\ldots\int f(x_0,\ldots,x_m)
   dx_{\tau(m)}\ldots dx_{\tau(0)}$,}

\noindent are both defined.   Then $I=I'$.

\proof{ Apply 537K in both directions.
}%end of proof of 537L

\leader{537M}{}\cmmnt{ A pair of simple facts which I never got round to
spelling out will be useful below.

\medskip

\noindent}{\bf Lemma} Suppose that $(X,\Sigma,\mu)$ is a totally
finite measure space and $f$ is a $[0,\infty]$-valued function defined
almost everywhere in $X$.

(a) If
$\gamma<\overline{\int}f$, then there is a measurable integrable function
$g:X\to\coint{0,\infty}$ such that $\int g\ge\gamma$ and
$\{x:x\in\dom f$, $g(x)\le f(x)\}$ has full outer measure in $X$.

(b) If $\underline{\int}f<\gamma$, then there is a measurable integrable
function $g:X\to\coint{0,\infty}$ such that $\int g\le\gamma$ and
$\{x:x\in\dom f$, $f(x)\le g(x)\}$ has full outer measure in $X$.

\proof{{\bf (a)} By 135H(b-i),

\Centerline{$\overlineint f
=\sup_{k\in\Bbb N}\overlineint\min(f(x),k)\mu(dx)$;}

\noindent let $k\in\Bbb N$ be such that
$\overline{\int}f_k>\gamma$, where $f_k(x)=\min(f(x),k)$ for
$x\in\dom f$.   Because $\mu X<\infty$, $\overline{\int}f_k$ is finite.
By 133J(a-i), there is an integrable $h$ such that
$\int h=\overline{\int}f_k$ and $f_k\leae h$;  adjusting $h$ on a
negligible set if necessary, we can arrange that $h$ is defined (and
finite) everywhere
on $X$ and is measurable.   Set $\epsilon=(\int h-\gamma)/(1+\mu X)$, and
$g=h-\epsilon\chi X$;  then by the last part of 133J(a-i), 

\Centerline{$\{x:x\in\dom f$, $g(x)\le f(x)\}
=\{x:x\in\dom f$, $h(x)\le f(x)+\epsilon\}$}

\noindent has full outer measure in $X$, while $\int g\ge\gamma$.

\medskip

{\bf (b)} By 135Ha,
there is a measurable $h:X\to[0,\infty]$
such that $h\leae f$ and $\int h=\underline{\int}f$;  as $\int h$ is
finite, $h$ is finite a.e.\ and can be adjusted to be finite
everywhere.   Set
$\epsilon=(\gamma-\int h)/(1+\mu X)$, and
$g=h+\epsilon\chi X$;  then $\int g\le\gamma$ and $\{x:f(x)\le g(x)\}$ has
full outer measure.
}%end of proof of 537M

\leader{537N}{}\cmmnt{ For ordinary two-variable repeated integrals we
can squeeze a little bit more out than is given by 537K.

\medskip

\noindent}{\bf Proposition}
Let $(X,\Sigma,\mu)$ be a semi-finite measure space,
$(Y,\Tau,\nu)$ a probability space, and
$\nu^{\Bbb N}$ the product measure on $Y^{\Bbb N}$.
If $\non(E,\Cal N(\mu))<\cov\Cal N(\nu^{\Bbb N})$ for every
$E\in\Sigma\setminus\Cal N(\mu)$, then

\Centerline{$\underlineint\,\underlineint f(x,y)\nu(dy)\mu(dx)
\le\overlineint\,\overlineint f(x,y)\mu(dx)\nu(dy)$}

\noindent for every function $f:X\times Y\to[0,\infty]$.

\proof{{\bf (a)} To begin with, suppose that $\mu X<\infty$ and
$\#(X)<\cov\Cal N(\nu^{\Bbb N})$.    For each $y\in Y$, let
$h_y:X\to[0,\infty]$ be a measurable function
such that $f(x,y)\le h_y(x)$ for every
$x\in X$ and $\int h_yd\mu=\overline{\int}f(x,y)\mu(dx)$;
let $v:Y\to[0,\infty]$
be a measurable function such that $\int h_yd\mu\le v(y)$ for every
$y\in Y$ and
$\int v\,d\nu=\overline{\int}\,\overline{\int}f(x,y)\mu(dx)\nu(dy)$.
If this is infinite, we can stop.   Otherwise, for each $x\in X$ let
$g_x:Y\to[0,\infty]$ be a measurable function such that
$g_x(y)\le f(x,y)$ for every $y\in Y$ and
$\int g_xd\nu=\underline{\int}f(x,y)\nu(dy)$, and let
$u:X\to[0,\infty]$ be a measurable function such that
$u(x)\le\int g_x\,d\nu$ for every $x$ and
$\int u\,d\mu=\underline{\int}\,\underline{\int}f(x,y)\nu(dy)\mu(dx)$.

As $\#(X)<\cov\Cal N(\nu^{\Bbb N})$,
we can find a sequence $\sequence{i}{y_i}$ in $Y$ such that

\Centerline{$\int v\,d\nu
=\lim_{n\to\infty}\Bover1{n+1}\sum_{i=0}^nv(y_i)$}

\noindent and

\Centerline{$\int g_xd\nu
=\lim_{n\to\infty}\Bover1{n+1}\sum_{i=0}^ng_x(y_i)$}

\noindent for every $x\in X$.   (For by 273J, the set of such sequences 
is the
intersection of fewer than $\cov\Cal N(\nu^{\Bbb N})$ conegligible sets
in $Y^{\Bbb N}$, and cannot be empty.)   If $x\in X$, then

\Centerline{$u(x)
\le\int g_xd\nu
=\liminf_{n\to\infty}\Bover1{n+1}\sum_{i=0}^ng_x(y_i)
\le\liminf_{n\to\infty}\Bover1{n+1}\sum_{i=0}^nh_{y_i}(x)$.}

\noindent So

$$\eqalignno{\underline{\int}\,\underline{\int}f(x,y)\nu(dy)\mu(dx)
&=\int u\,d\mu
\le\liminf_{n\to\infty}\Bover1{n+1}\sum_{i=0}^n\int h_{y_i}d\mu\cr
\displaycause{by Fatou's Lemma}
&\le\liminf_{n\to\infty}\Bover1{n+1}\sum_{i=0}^nv(y_i)
=\int v\,d\nu\cr
&=\overline{\int}\,\overline{\int}f(x,y)\mu(dx)\nu(dy),\cr}$$

\noindent as required.

\medskip

{\bf (b)} Now suppose that $\mu$ is totally finite and that
$X$ has a subset $A$ of full outer measure with
$\#(A)<\cov\Cal N(\nu^{\Bbb N})$.   Let $\mu_A$ be the subspace measure on
$A$.   Then for any $q:X\to[0,\infty]$ we have

\Centerline{$\underlineint q\,d\mu
\le\underlineint(q\restr A)d\mu_A
\le\overlineint(q\restr A)d\mu_A
\le\overlineint q\,d\mu$}

\noindent (214J).   So, writing
$f_A$ for the restriction of $f$ to $A\times Y$,

$$\eqalignno{\underline{\int}\,\underline{\int}f(x,y)\nu(dy)\mu(dx)
&\le\underline{\int}\,\underline{\int}f_A(x,y)\nu(dy)\mu_A(dx)\cr
&\le\overline{\int}\,\overline{\int}f_A(x,y)\mu_A(dx)\nu(dy)\cr
\displaycause{by (a)}
&\le\overline{\int}\,\overline{\int}f(x,y)\mu(dx)\nu(dy).\cr}$$

\medskip

{\bf (c)} For the general case, let $u:X\to[0,\infty]$ be a
measurable function such that $u(x)\le\underline{\int}f(x,y)\nu(dy)$ for
every $x\in X$ and
$\int u\,d\mu=\underline{\int}\,\underline{\int}f(x,y)\nu(dy)\mu(dx)$.
Take any $\gamma<\int u\,d\mu$.   Because $\mu$ is semi-finite, there is
a non-empty set $F\in\Sigma$ of finite measure such that
$\int_Fu\,d\mu>\gamma$.
Now let $\Cal E$ be the family of measurable
sets $E\subseteq F$ of finite measure for which there is a non-empty
set $A\subseteq E$, with cardinal
less than $\cov\Cal N(\nu^{\Bbb N})$, such that $\mu^*A=\mu E$, that is,
$A$ has full outer measure for the subspace measure $\mu_E$, that is,
$E$ is a measurable envelope of $A$.   Then $\Cal E$ is closed under finite
unions and every non-empty member of $\Sigma$ includes a member of
$\Cal E$.   So there
is a non-decreasing sequence $\sequence{k}{E_k}$ in $\Cal E$ such that
$\bigcup_{k\in\Bbb N}E_k\subseteq F$ and
$F\setminus\bigcup_{k\in\Bbb N}E_k$ is negligible.   In this case,
$\gamma<\int_Fu\,d\mu=\lim_{k\to\infty}\int_{E_k}u\,d\mu$, so there is a
$k\in\Bbb N$ such that $\gamma\le\int_{E_k}u\,d\mu$.   Set $E=E_k$.

Consider the restriction $f_E$ of $f$ to $E\times Y$ and the subspace
measure $\mu_E$ on $E$.   We have

$$\eqalignno{\gamma
&\le\int_Eu\,d\mu
=\int(u\restr E)d\mu_E
\le\underline{\int}\,\underline{\int}f_E(x,y)\nu(dy)\mu_E(dx)\cr
&\le\overline{\int}\,\overline{\int}f_E(x,y)\mu_E(dx)\nu(dy)\cr
\displaycause{because $E\in\Cal E$, so we can use (b)}
&\le\overline{\int}\,\overline{\int}f(x,y)\mu(dx)\nu(dy)\cr}$$

\noindent because $\overline{\int}f_E(x,y)\mu_E(dx)
\le\overline{\int}f(x,y)\mu(dx)$ for every $y$, by 214Ja or otherwise.
Since $\gamma$ is arbitrary,

\Centerline{$\underlineint\,\underlineint f(x,y)\nu(dy)\mu(dx)
\le\overlineint\,\overlineint f(x,y)\mu(dx)\nu(dy)$}

\noindent in this case also.
}%end of proof of 537N

\leader{537O}{Corollary}
Let $(X,\Sigma,\mu)$ and $(Y,\Tau,\nu)$ be probability spaces, and
$\nu^{\Bbb N}$ the product measure on $Y^{\Bbb N}$.
If $\shrplusN(\mu)\le\cov\Cal N(\nu^{\Bbb N})$ then

\Centerline{$\overlineint\underlineint f(x,y)\nu(dy)\mu(dx)
\le\overlineint\,\overlineint f(x,y)\mu(dx)\nu(dy)$}

\noindent for every
function $f:X\times Y\to\coint{0,\infty}$.

\proof{ Take any
$\gamma<\overline{\int}\underline{\int}f(x,y)\nu(dy)\mu(dx)$.
By 537Ma, there are a measurable function $u:X\to\coint{0,\infty}$ and a
set $A$ of full outer measure in $X$ such that $\int u\,d\mu\ge\gamma$ and
$u(x)\le\underline{\int}f(x,y)\nu(dy)\mu(dx)$ for every $x\in A$.
Let $\mu_A$ be the subspace measure on $A$, and $f_A$ the restriction
of $f$ to $A\times Y$.   If $B\subseteq A$ is any non-negligible relatively
measurable set, there is a non-negligible $D\subseteq B$ such that
$\#(D)<\shrplusN(\mu)$, so

\Centerline{$\non(B,\Cal N(\mu_A))
=\non(B,\Cal N(\mu))\le\#(D)<\cov\Cal N(\nu^{\Bbb N})$.}

\noindent So

$$\eqalignno{\gamma
&\le\int u\,d\mu
=\int(u\restr A)d\mu_A
\le\underline{\int}\,\underline{\int}f_A(x,y)\nu(dy)\mu_A(dx)\cr
\displaycause{because $u\restr A$ is measurable and
$(u\restr A)(x)\le\underline{\int}f_A(x,y)\nu(dy)$ for every $x\in A$}
&\le\overline{\int}\,\overline{\int}f_A(x,y)\mu_A(dx)\nu(dy)\cr
\displaycause{by 537N}
&\le\overline{\int}\,\overline{\int}f(x,y)\mu(dx)\nu(dy)\cr}$$

\noindent because $\overline{\int}f_A(x,y)\mu_A(dx)
\le\overline{\int}f(x,y)\mu(dx)$ for every $y$, by 214J again.
As $\gamma$ is arbitrary, we have the result.
}%end of proof of 537O

\cmmnt{\medskip

\noindent{\bf Remark} There is a similar inequality, under different
hypotheses, in 543C below.
}

\leader{537P}{Corollary}
Let $(X,\Sigma,\mu)$ and $(Y,\Tau,\nu)$ be probability spaces, and
$\nu^{\Bbb N}$ the product measure on $Y^{\Bbb N}$;  suppose that
$\shrplusN(\mu)\le\cov\Cal N(\nu^{\Bbb N})$, and that
$f:X\times Y\to\Bbb R$ is bounded.

(a)

\Centerline{$\overlineint\underlineint f(x,y)\nu(dy)\mu(dx)
\le\overlineint\,\overlineint f(x,y)\mu(dx)\nu(dy)$,}

\Centerline{$\underlineint \,\underlineint f(x,y)\mu(dx)\nu(dy)
\le\underlineint\overlineint f(x,y)\nu(dy)\mu(dx)$.}

(b) If $\iint f(x,y)\mu(dx)\nu(dy)$ is defined, and $\int f(x,y)\nu(dy)$
is defined for almost every $x$, then the other repeated integral
$\iint f(x,y)\nu(dy)\mu(dx)$ is
defined and equal to $\iint f(x,y)\mu(dx)\nu(dy)$.

\proof{{\bf (a)} Apply 537O to the functions $(x,y)\mapsto M+f(x,y)$,
$(x,y)\mapsto M-f(x,y)$ for suitable $M$.

\wheader{537P}{6}{2}{2}{60pt}

{\bf (b)} By (a),

$$\eqalign{\iint f(x,y)\mu(dx)\nu(dy)
&\le\underline{\int}\int f(x,y)\nu(dy)\mu(dx)\cr
&\le\overline{\int}\int f(x,y)\nu(dy)\mu(dx)
\le\iint f(x,y)\mu(dx)\nu(dy).\cr}$$
}%end of proof of 537P

\leader{537Q}{}\cmmnt{ We can extend the second part of 537Pa, as well as
the first, to unbounded functions, if we strengthen the set-theoretic
hypothesis.

\medskip

\noindent}{\bf Proposition}\cmmnt{ ({\smc Humke \& Laczkovich 05})}
Let $(X,\Sigma,\nu)$ and $(Y,\Tau,\mu)$ be probability spaces, and
$\mu^{\Bbb N}$, $\nu^{\Bbb N}$ the product measures on $X^{\Bbb N}$,
$Y^{\Bbb N}$ respectively.
If $\shrplusN(\mu^{\Bbb N})\le\cov\Cal N(\nu^{\Bbb N})$ then
$\underline{\int}\,\underline{\int}f(x,y)\mu(dx)\nu(dy)
\le\underline{\int}\overline{\int}f(x,y)\nu(dy)\mu(dx)$
for every function $f:X\times Y\to\coint{0,\infty}$.

\proof{ \Quer\ Suppose, if possible, otherwise.

\medskip

{\bf (a)} There is a measurable function $u:Y\to\coint{0,\infty}$ such that

\Centerline{$u(y)\le\underlineint f(x,y)\mu(dx)$ for every $y$,
\quad$\underlineint\overlineint f(x,y)\nu(dy)\mu(dx)
<\int u\,d\nu$.}

\noindent
Since $\int u\,d\nu$ is the supremum of the integrals of the non-negative
simple functions dominated by $u$, we may suppose that $u$ itself is a
simple function;  express it as $\sum_{j=0}^m\alpha_j\chi F_j$ where
$\alpha_j\ge 0$ for each $i$ and $(F_0,\ldots,F_m)$ is a partition of $Y$
into measurable sets.   Now

$$\eqalignno{\sum_{j=0}^m\underline{\int}\overline{\int}
  f(x,y)\chi F_j(y)\nu(dy)\mu(dx)
&\le\underline{\int}\sum_{j=0}^m
  \overline{\int}f(x,y)\chi F_j(y)\nu(dy)\mu(dx)\cr
\displaycause{133J(b-v)}
&\le\underline{\int}\overline{\int}f(x,y)\nu(dy)\mu(dx)\cr
\displaycause{because if $x\in X$ and $q:Y\to[0,\infty]$ is measurable and
$f(x,y)\le q(y)$ for every $y$, then
\ifnum\stylenumber=12\else the sum \fi
$\sum_{j=0}^m\overline{\int}f(x,y)\chi F_j(y)\nu(dy)$ is at most
$\sum_{j=0}^m\int q\times\chi F_jd\nu=\int q\,d\nu$}
&<\int u\,d\nu
=\sum_{j=0}^m\alpha_j\nu F_j.\cr}$$

\noindent There are therefore a $j\le m$ and a $\gamma<1$ such that

\Centerline{$\underlineint\overlineint
  f(x,y)\chi F_j(y)\nu(dy)\mu(dx)
<\gamma\alpha_j\nu F_j$.}

\noindent Now there is a measurable function
$v:X\to\coint{0,\infty}$ such that
$\int v\,d\mu\le\gamma\alpha_j\nu F_j$ and

\Centerline{$D
=\{x:x\in X$, $\overlineint f(x,y)\chi F_j(y)\nu(dy)\le v(x)\}$}

\noindent has full outer measure in $X$, by 537Mb.

\medskip

{\bf (b)} For $y\in Y$ and $\pmb{x}=\sequence{i}{x_i}\in X^{\Bbb N}$, set
$h(\pmb{x},y)=\liminf_{n\to\infty}\Bover1{n+1}\sum_{i=0}^nf(x_i,y)$.
If $y\in Y$, then $\underline{\int}f(x,y)\mu(dx)\le h(\pmb{x},y)$ for
$\mu^{\Bbb N}$-almost every $\pmb{x}$.   \Prf\ We have a measurable
function $q:X\to\coint{0,\infty}$ such that $q(x)\le f(x,y)$ for every $x$
and

$$\eqalign{\underline{\int}f(x,y)\mu(dx)
&=\int q\,d\mu
\le\liminf_{n\to\infty}\Bover1{n+1}\sum_{i=0}^nq(x_i)\cr
&\le\liminf_{n\to\infty}\Bover1{n+1}\sum_{i=0}^nf(x_i,y)
=h(\pmb{x},y)\cr}$$

\noindent for almost every $\pmb{x}=\sequence{i}{x_i}$.\ \QeD\   At the
same time,

\Centerline{$V
=\{\sequence{i}{x_i}:\liminf_{n\to\infty}\Bover1{n+1}\sum_{i=0}^nv(x_i)
   \le\gamma\alpha_j\nu F_j\}$}

\noindent is conegligible in $X^{\Bbb N}$, because
$\int v\,d\mu\le\gamma\alpha_j\nu F_j$.

\wheader{537Q}{6}{2}{2}{36pt}

{\bf (c)} Set

\Centerline{$W=\{(\pmb{x},y):\pmb{x}\in X^{\Bbb N}$, $y\in F_j$,
$h(\pmb{x},y)\ge\alpha_j\}$}

\noindent and consider the function $\chi W:X^{\Bbb N}\times Y\to\{0,1\}$.
If $y\in F_j$ then $\underline{\int}f(x,y)\mu(dx)\ge\alpha_j$
so $W^{-1}[\{y\}]$ is conegligible in $X^{\Bbb N}$.
On the other hand, if $\pmb{x}=\sequence{i}{x_i}$ belongs to
$V\cap D^{\Bbb N}$,

$$\eqalignno{\overline{\int}\alpha_j\chi W(\pmb{x},y)\nu(dy)
&\le\overline{\int}h(\pmb{x},y)\chi F_j(y)\nu(dy)\cr
&=\overline{\int}\liminf_{n\to\infty}\Bover1{n+1}
  \sum_{i=0}^nf(x_i,y)\chi F_j(y)\nu(dy)\cr
&\le\liminf_{n\to\infty}\overline{\int}\Bover1{n+1}
  \sum_{i=0}^nf(x_i,y)\chi F_j(y)\nu(dy)\cr
\displaycause{133Kb}
&\le\liminf_{n\to\infty}\Bover1{n+1}
  \sum_{i=0}^n\overline{\int}f(x_i,y)\chi F_j(y)\nu(dy)\cr
\displaycause{133J(b-ii)}
&\le\liminf_{n\to\infty}\Bover1{n+1}\sum_{i=0}^nv(x_i)
\le\gamma\alpha_j\nu F_j.\cr}$$

\medskip

{\bf (d)}
As $V$ is conegligible and $D^{\Bbb N}$ has full outer measure (254Lb),

$$\eqalign{\underline{\int}\overline{\int}
  \chi W(\pmb{x},y)\nu(dy)\mu^{\Bbb N}(d\pmb{x})
&\le\gamma\nu F_j
<\nu F_j
=\iint\chi W(\pmb{x},y)\mu^{\Bbb N}(d\pmb{x})\nu(dy)\cr
&=\underline{\int}\,\underline{\int}
  \chi W(\pmb{x},y)\mu^{\Bbb N}(d\pmb{x})\nu(dy).\cr}$$

\noindent But we are supposing that
$\shrplusN(\mu^{\Bbb N})\le\cov\Cal N(\nu^{\Bbb N})$, so this contradicts
537P.\ \Bang

So we have the result.
}%end of proof of 537Q

\leader{537R}{Lemma} Let $(X,\Sigma,\mu)$ be a complete probability space
and $(Y,\Tau,\nu)$ a
probability space such that 
$\shrplusN(\mu)\penalty-50\le\cov\Cal N(\nu^{\Bbb N})$,
where $\nu^{\Bbb N}$ is the product measure on $Y^{\Bbb N}$.   Let
$f:X\times Y\to\Bbb R$ be a bounded function which is measurable in
each variable separately, and set
$u(x)=\int f(x,y)\nu(dy)$ for $x\in X$.  Then $u:X\to\Bbb R$ is measurable.

\proof{ \Quer\ Otherwise, there are a non-negligible measurable set
$E\subseteq X$ and $\alpha$, $\beta\in\Bbb R$ such that
$\alpha<\beta$ and

\Centerline{$\mu^*\{x:x\in E$, $u(x)\le\alpha\}
=\mu^*\{x:x\in E$, $u(x)\ge\beta\}=\mu E$}

\noindent (413G).   Let $A\subseteq\{x:x\in E$, $u(x)\le\alpha\}$ and
$B\subseteq\{x:x\in E$, $u(x)\ge\beta\}$ be sets with cardinal less than
$\shrplusN(\mu)$ and outer measure greater than $\bover12\mu E$
(521Ca).   Let $\sequence{i}{y_i}$ be a sequence in $Y$ such that

\Centerline{$u(x)=\lim_{n\to\infty}\Bover1{n+1}\sum_{i=0}^nf(x,y_i)$}

\noindent for every $x\in A\cup B$.   Because $x\mapsto f(x,y_i)$ is
measurable for each $i$, $u\restr A\cup B$ is measurable;   but this means
that $A$ and $B$ can be separated by measurable sets, which is impossible,
because $\mu^*A+\mu^*B>\mu E$.\ \Bang
}%end of proof of 537R

\leader{537S}{Proposition} Let $(X,\Sigma,\mu)$ and $(Y,\Tau,\nu)$ be
probability spaces such that

\Centerline{$\shrplusN(\mu)\le\cov\Cal N(\nu^{\Bbb N})$,}

\noindent where $\nu^{\Bbb N}$ is the product measure on $Y^{\Bbb N}$,
and

\Centerline{$\cff([\tau(\nu)]^{\le\omega})<\cov(E,\Cal N(\mu))$ for every
$E\in\Sigma\setminus\Cal N(\mu)$,}

\noindent where $\tau(\nu)$ is the Maharam type of $\nu$.
Let $f:X\times Y\to\coint{0,\infty}$ be a
function which is measurable in each variable separately.  Then
$\iint f(x,y)\mu(dx)\nu(dy)$ and $\iint f(x,y)\nu(dy)\mu(dx)$ exist and
are equal.

\proof{{\bf (a)}
Let $\tilde\Lambda\supseteq\Sigma\tensorhat\Tau$ be the
$\sigma$-algebra of sets
$W\subseteq X\times Y$ such that all the vertical and horizontal
sections of $W$ are measurable.   If $W\in\tilde\Lambda$, then
$x\mapsto\nu W[\{x\}]:X\to[0,1]$ is measurable,
by 537R. If $W\in\tilde\Lambda$ and almost every horizontal section of
$W$ is negligible, then

$$\eqalign{\overline{\int}\nu W[\{x\}]\mu(dx)
&=\overline{\int}\underline{\int}\chi W(x,y)\nu(dy)\mu(dx)\cr
&\le\overline{\int}\,\overline{\int}\chi W(x,y)\mu(dx)\nu(dy)
=0\cr}$$

\noindent by 537Pa, so almost every vertical section of $W$ is
negligible.

\medskip

{\bf (b)} Let $(\frak B,\bar\nu)$ be the measure algebra of $(Y,\Tau,\nu)$.
If $W\in\tilde\Lambda$ and there is a metrically separable subalgebra
$\frak C$ of
$\frak B$ containing $W[\{x\}]^{\ssbullet}$ for every $x\in X$, then
there is a $W'\in\Sigma\tensorhat\Tau$ such that
$W[\{x\}]\symmdiff W'[\{x\}]$ is negligible for almost every $x$.
\Prf\ Note first that for every $F\in\Tau$ the map

\Centerline{$x\mapsto\nu(W[\{x\}]\symmdiff F)
=\nu((W\symmdiff(X\times F))[\{x\}]$}

\noindent is measurable, by (a).   So
$x\mapsto W[\{x\}]^{\ssbullet}:X\to\frak C$ is measurable, by 418Bc.   By
418T(b-ii), there is a $W'\in\Sigma\tensorhat\Tau$ such that
$W[\{x\}]^{\ssbullet}=W'[\{x\}]^{\ssbullet}$ for almost every $x$.\ \Qed

\medskip

{\bf (c)} In fact we find that for any $W\in\tilde\Lambda$ there is a
$W'\in\Sigma\tensorhat\Tau$ such that
$W[\{x\}]\symmdiff W'[\{x\}]$ is negligible for almost every $x$.
\Prf\ Set $\kappa=\tau(\nu)=\tau(\frak B)$, and let
$\ofamily{\xi}{\kappa}{e_{\xi}}$ generate $\frak B$.   Let
$\Cal K\subseteq[\kappa]^{\le\omega}$ be a cofinal set of size
$\cff[\kappa]^{\le\omega}$.   For $K\in\Cal K$, let $\frak B_K$
be the closed subalgebra of $\frak B$ generated by
$\{e_{\xi}:\xi\in K\}$ and $A_K$ the set
$\{x:x\in X$, $W[\{x\}]^{\ssbullet}\in\frak B_K\}$.
Note that $K\mapsto A_K$ is non-decreasing and that the union of any
sequence in $\Cal K$ is included in a member of $\Cal K$.   
So there is a $K_0\in\Cal K$
such that $\mu^*A_{K_0}=\sup_{K\in\Cal K}\mu^*A_K$.

If $E$ is a measurable envelope of $A_{K_0}$, then
$\{A_K\setminus E:K\in\Cal K\}$ is a cover of $X\setminus E$ by negligible
sets.
So $\cov(X\setminus E,\penalty-100\Cal N(\mu))\le\cff[\kappa]^{\le\omega}$ and
$X\setminus E$ must be negligible, that is, $A_{K_0}$ has full outer
measure.

Taking a sequence
$\sequencen{F_n}$ in $\Tau$ such that $\{F_n^{\ssbullet}:n\in\Bbb N\}$
is dense in $\frak B_{K_0}$, we see from (a) that
\ifnum\stylenumber=12 the function \else\fi
$x\mapsto\inf_{n\in\Bbb N}\nu(W[\{x\}]\symmdiff F_n)$ is measurable, while
it is zero on $A_{K_0}$.   So
$W[\{x\}]^{\ssbullet}\in\frak B_{K_0}$ for almost every $x\in X$,
that is, $A_{K_0}$ is actually conegligible.   Taking a measurable
conegligible set $E'\subseteq A_{K_0}$ and applying (b) to
$W\cap(E'\times Y)$, we see that there is
a $W'\in\Sigma\tensorhat\Tau$ such that $W[\{x\}]\symmdiff W'[\{x\}]$ is
negligible for almost every $x\in X$.\ \Qed

\medskip

{\bf (d)} Now turn to the function $f$ under consideration.   
For $q\in\Bbb Q$ set
$W_q=\{(x,y):f(x,y)\ge q\}\in\tilde\Lambda$.   By (c), we have
$V_q\in\Sigma\tensorhat\Tau$ such that
$V_q[\{x\}]\symmdiff W_q[\{x\}]$ is $\nu$-negligible for $\mu$-almost
every $x$, and therefore $W_q^{-1}[\{y\}]\symmdiff V_q^{-1}[\{y\}]$ is
$\mu$-negligible for $\nu$-almost every $y$, by (a).   If $q\le q'$ then
$W_{q'}\setminus W_q$ is empty, so $V_{q'}[\{x\}]\setminus V_q[\{x\}]$ is
$\nu$-negligible for $\mu$-almost every $x$, and
$V_{q'}\setminus V_q$ is $(\mu\times\nu)$-negligible, where $\mu\times\nu$
is the product measure on
$X\times Y$.   Similarly, $\bigcap_{q'<q}V_{q'}\setminus V_q$ is
negligible for every $q$.   Moreover, writing $V_{\infty}$ for
$\bigcap_{q\in\Bbb Q}V_q$, $V_{\infty}[\{x\}]$ is $\nu$-negligible for
$\mu$-almost every $x$, so $(\mu\times\nu)V_{\infty}=0$;  similarly,
$(\mu\times\nu)V_0=1$.   There is therefore a
$\Sigma\tensorhat\Tau$-measurable $g:X\times Y\to\coint{0,\infty}$ 
such that
$V_q\symmdiff\{(x,y):g(x,y)\ge q\}$ is $(\mu\times\nu)$-negligible for
every $q\in\Bbb Q$.   In this case,

\Centerline{$\{x:f(x,y)\ne g(x,y)\}$ is $\mu$-negligible for
$\nu$-almost every $y$,}

\Centerline{$\{y:f(x,y)\ne g(x,y)\}$ is $\nu$-negligible for
$\mu$-almost every $x$,}

\noindent and

$$\eqalign{\iint f(x,y)\mu(dx)\nu(dy)
&=\iint g(x,y)\mu(dx)\nu(dy)\cr
&=\iint g(x,y)\nu(dy)\mu(dx)
=\iint f(x,y)\nu(dy)\mu(dx)\cr}$$

\noindent by 252H.

\medskip

{\bf (e)} Finally, if $f$ is unbounded, set $f_k(x,y)=\min(f(x,y),k)$ for
each $k\in\Bbb N$.   Then

$$\eqalign{\iint f(x,y)\mu(dx)\nu(dy)
&=\lim_{k\to\infty}\iint f_k(x,y)\mu(dx)\nu(dy)\cr
&=\lim_{k\to\infty}\iint f_k(x,y)\nu(dy)\mu(dx)
=\iint f(x,y)\nu(dy)\mu(dx).\cr}$$
}%end of proof of 537S

\exercises{\leader{537X}{Basic exercises (a)}(i)
%\sqheader 537Xa
Let $(X,\Sigma,\mu)$ be a measure space such that
singletons are negligible and $\cf\Cal N(\mu)=\omega_1$.   Show that there
is a Sierpi\'nski subset of $X$.   (ii) Show that if $\mu$ is Lebesgue
measure on $\Bbb R$ and
$\cf\Cal N(\mu)=\omega_1$, then there is a strongly Sierpi\'nski subset of
$\Bbb R$.
%537A

\spheader 537Xb Show that for any uncountable cardinal $\kappa$
there is a purely atomic probability space with a
strongly Sierpi\'nski set of size $\kappa$.
%537A

\spheader 537Xc Let $(X,\Sigma,\mu)$ be a measure space.   Show that the
union of any sequence of Sierpi\'nski sets in $X$ is again a Sierpi\'nski
set in $X$.
%537A

\spheader 537Xd Let $(X,\Sigma,\mu)$ be a measure space and $Y$ any
subspace of $X$.   Show that a subset of $Y$ is a Sierpi\'nski set for
the subspace measure on $Y$ iff it is a Sierpi\'nski set for $\mu$.
%537B

\spheader 537Xe Suppose that  $\lambda$ is an infinite cardinal
and the usual measure $\nu_{\lambda}$ on $\{0,1\}^{\lambda}$ has a
Sierpi\'nski set of size $\kappa$.
Show that $\nu_{\lambda}$ has a Sierpi\'nski set $A$ such that
$\#(A\cap E)=\kappa$ whenever $\nu_{\lambda}E>0$.
%537B

\spheader 537Xf Let $(X,\rho)$ be a non-separable metric space with
$r$-dimensional Hausdorff measure, where $r>0$.   Show that $X$ has a
Sierpi\'nski subset of size equal to the topological density of
$X$.
%537B

\sqheader 537Xg Suppose that $\non\Cal N<\cov\Cal N$, where $\Cal N$ is the
null ideal of Lebesgue measure on $\Bbb R$.   Let $(X,\frak T,\Sigma,\mu)$
and $(Y,\frak S,\Tau,\nu)$ be Radon probability spaces of countable Maharam
type, and $f:X\times Y\to\coint{0,\infty}$ a function such that
$I=\iint f(x,y)\mu(dx)\nu(dy)$ and $I'=\iint f(x,y)\nu(dy)\mu(dx)$ are both
defined.   Show that $I=I'$.
%537N

\sqheader 537Xh Let $(X,\Sigma,\mu)$ be a probability space in which
there is a well-ordered family in $\Cal N(\mu)$ with union $X$;  e.g.,
because
$\non\Cal N(\mu)=\#(X)$ or $\add\Cal N(\mu)=\cov\Cal N(\mu)$.
Show that there is a function $f:X\times X\to[0,1]$ such that
$\int f(x,y)\mu(dx)=0$ for every $y\in X$ and $\int f(x,y)\mu(dy)=1$
for every $x\in X$.

\sqheader 537Xi (In this exercise, all integrals are to be taken with
respect to one-dimensional Lebesgue measure $\mu$.)
(i) Find a function $f:[0,1]^2\to\{0,1\}$
such that $\int\overline{\int}f(x,y)dxdy=1$ but $\iint f(x,y)dydx=0$.
\Hint{there is a disjoint family $\family{y}{[0,1]}{A_y}$ of sets of full
outer measure.}  (ii) Find a function $f:[0,1]^2\to\{0,1\}$
such that $\iint f(x,y)dxdy=1$ but $\int\underline{\int}f(x,y)dydx=0$.
(iii) Find a function $f:[0,1]^2\to\{0,1\}$ such that
$\overline{\int}\int f(x,y)dxdy=1$ but
$\underline{\int}\int f(x,y)dydx=0$.   \Hint{enumerate $[0,1]$ as
$\ofamily{\xi}{\frak c}{x_{\xi}}$ in such a way that
$\{x_{\xi}:\xi<\non\Cal N(\mu)\}$ has full outer measure;  set
$f(x_{\xi},x_{\eta})=1$ if $\eta<\xi$.}

%\leader{537Y}{Further exercises (a)}

}%end of exercises

\leader{537Z}{Problems (a)} Is it relatively consistent with ZFC to suppose
that $\Bbb R$, with Lebesgue measure, has a Sierpi\'nski subset but no
strongly Sierpi\'nski subset?

\spheader 537Zb Is it relatively consistent with ZFC to suppose that there
is a probability space $(X,\mu)$ such that $(X,\mu)$ has a Sierpi\'nski set
but its power $(X^{\Bbb N},\mu^{\Bbb N})$ does not?

\endnotes{
\Notesheader{537}
It is easy to see that if $\frak c=\omega_1$ then there is a
strongly Sierpi\'nski
set of size $\omega_1$ for Lebesgue measure (537Xa).
Countable-cocountable measures have strongly Sierpi\'nski sets for trivial
reasons.   To eliminate all Sierpi\'nski sets (on the definition of 537A)
from atomless complete locally determined measure spaces, it is enough to
ensure that the uniformity of Lebesgue measure is greater than $\omega_1$
(537Bb).   For the simplest models with
non-trivial Sierpi\'nski sets of size greater than $\omega_1$, see 552E
below.

The `entangled sets' of 537C-537G %537C 537D 537E 537F 537G
belong rather to combinatorics than to measure theory;  I go as far as I do
into this theory because
it is interesting in view of 552E.   But it includes a
proof that if the continuum hypothesis is true then there are two ccc
partially ordered sets whose product is not ccc, which in its own context
is of great importance.

Fubini's theorem is so important in measure theory that exploration of its
boundaries has been a perennial challenge.   I gave elementary examples in
252Xf-252Xg to show that as soon as we abandon the requirement that
$\iint|f(x,y)|dxdy<\infty$ our repeated integrals can be expected to be
unreliable.   But for non-negative functions $f$ on $\sigma$-finite spaces,
measurability is enough to ensure that repeated integrals are equal
(252H).   In this section I look for results which will be valid for
non-measurable functions.   In 537I-537J we have a rather esoteric example
-- or, some would say, an example from a topic which I have neglected in
this book -- which is unusual in that it is a theorem of ZFC;  for a note
on its ancestry see {\smc Fremlin 93}, 5L.   In 537K-537L we see that,
in the presence of a sufficient supply of Sierpi\'nski sets,  for
instance, we
must have $\iint f(x,y)dxdy=\iint f(x,y)dydx$  for ordinary
bounded real-valued functions on the product of probability spaces,
as long as both repeated integrals are defined.   The argument here depends
on using the strong law of large numbers to replace an integral
$\int f(x,y)dx$ by the limit of a sequence of averages of values
$f(x_i,y)$.   This is why the Sierpi\'nski sets must be available not in
the original probability spaces $X_0,\ldots,X_m$ but in their powers
$X_j^{\Bbb N}$.   Of course for our favourite spaces, starting with
$[0,1]$, $(X^{\Bbb N},\mu^{\Bbb N})$ is isomorphic to $(X,\mu)$, so this
does not seem too large a step;  but it begs an obvious question (537Zb).
For any result of this kind we certainly need some special axiom (537Xh).

In 537L the hypothesis includes strong `separate measurability' conditions;
we need not only separate measurability, but
measurability of the functions $x\mapsto\int f(x,y)dy$ and
$y\mapsto\int f(y,x)dx$.   With a different set-theoretic hypothesis we can
relax these (537S).   I approach this form through ideas
from {\smc Humke \& Laczkovich 05}, where there is a careful analysis of
repeated integrals of the form $\underline{\int}\overline{\int}$, etc.
My own version is in 537N-537Q.   %537N 537O 537P 537Q
At every step there are ZFC examples to show that we cannot change the
formulae involving
$\underline{\int}$, $\overline{\int}$ without disaster (537Xi);
but it is not so clear that the set-theoretic hypotheses offered are
unimprovable.

}%end of notes

\discrpage

