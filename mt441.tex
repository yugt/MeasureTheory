\frfilename{mt441.tex}
\versiondate{3.1.06}
\copyrightdate{1997}

\def\high#1{\lceil#1\rceil}
\def\rti{right-{\vthsp}translation-{\vthsp}invariant}
\def\tr{\mathop{\text{tr}}\nolimits}
\def\trs{^{\top}}

\def\chaptername{Topological groups}
\def\sectionname{Invariant measures on locally compact spaces}

\newsection{441}

I begin this chapter with the most important theorem on
the existence of invariant measures:  every locally compact Hausdorff
group has left and right Haar measures (441E).   I derive this as a
corollary of a general result concerning invariant measures on locally
compact spaces (441C), which has other interesting consequences (441H).

\leader{441A}{Group actions} I recall a fundamental definition from
group theory.

\medskip

{\bf (a)} If $G$ is a group and $X$ is a set, an {\bf action} of $G$ on
$X$ is a function $(a,x)\mapsto a\action x:G\times X\to X$ such that

\Centerline{$(ab)\action x=a\action(b\action x)$ for all $a$, $b\in G$,
$x\in X$,}

\Centerline{$e\action x=x$ for every $x\in X$}

\noindent where $e$ is the identity of $G$\cmmnt{ (4A5B)}.   In this
context I write

\Centerline{$a\action A=\{a\action x:x\in A\}$}

\noindent for $a\in G$, $A\subseteq X$.   If $f$ is a function defined on a
subset of $X$, then $(a\action f)(x)=f(a^{-1}\action x)$ whenever
$a\in G$ and $x\in X$ and $a^{-1}\action x\in\dom f$\cmmnt{ (4A5C(c-i))}.

\spheader 441Ab If a group $G$ acts on a set $X$, a measure $\mu$ on $X$
is {\bf $G$-invariant} if $\mu(a^{-1}\action E)$ is defined and equal to
$\mu E$ whenever $a\in G$ and $\mu$ measures $E$.
%former 4{}41Ac-4{}41Ad now in 4A5C

\cmmnt{(Of course this is the same thing as saying that
$\mu(a\action E)=\mu E$ for every $a\in G$ and measurable set $E$;  I
use the formula
with $a^{-1}$ so as to match my standard practice when $a$ is actually a
function from $X$ to $X$.)}

\leader{441B}{}\cmmnt{ It will be useful later to be able to quote the
following elementary results.

\medskip

\noindent}{\bf Lemma} Let $X$ be a topological space, $G$ a group, and
$\action$ an action of $G$ on $X$ such that $x\mapsto a\action x$ is
continuous for every $a\in G$.

(a) If $\mu$ is a quasi-Radon measure on $X$ such that
$\mu(a\action U)\le\mu U$ for every open set $U\subseteq X$ and every
$a\in G$, then $\mu$ is $G$-invariant.

(b) If $\mu$ is a Radon measure on $X$ such that
$\mu(a\action K)\le\mu K$ for every compact set $K\subseteq X$ and every
$a\in G$, then $\mu$ is $G$-invariant.

\proof{ Note first that the maps $x\mapsto a\action x$ are actually
homeomorphisms (4A5Bd), so that $a\action U$ and $a\action K$ will be
open, or compact, as $U$ and $K$ are.   Next, the inequality $\le$ in
the hypotheses is an insignificant refinement;  since we must also have

\Centerline{$\mu U=\mu(a^{-1}\action a\action U)\le\mu(a\action U)$}

\noindent in (a),

\Centerline{$\mu K=\mu(a^{-1}\action a\action K)\le\mu(a\action K)$}

\noindent in (b), we always have equality here.

Now fix $a\in G$, and set $T_a(x)=a\action x$ for $x\in X$.   Then
$T_a$ is a homeomorphism, so the image measure $\mu T_a^{-1}$ will be
quasi-Radon, or Radon, if $\mu$ is.   In (a), we are told that
$\mu T_a^{-1}$ agrees with $\mu$ on the open sets, while in (b) we are
told that they agree on the compact sets;  so in both cases we have
$\mu=\mu T_a^{-1}$, by 415H(iii) or 416E(b-ii).   Consequently we have
$\mu T_a^{-1}[E]=\mu E$ whenever $\mu$ measures $E$.   As $a$ is
arbitrary, $\mu$ is $G$-invariant.
}%end of proof of 441B

\leader{441C}{Theorem}\cmmnt{ ({\smc Steinlage 75})} Let $X$ be a
non-empty locally compact Hausdorff
space and $G$ a group acting on $X$.   Suppose that

\inset{(i) $x\mapsto a\action x$ is continuous for every $a\in G$;

(ii) every orbit $\{a\action x:a\in G\}$ is dense;

(iii) whenever $K$ and $L$ are disjoint compact subsets of $X$ there is
a non-empty open subset $U$ of $X$ such that, for every $a\in G$, at
most one of $K$, $L$ meets $a\action U$.}

\noindent Then there is a non-zero $G$-invariant Radon measure $\mu$ on
$X$.

\proof{{\bf (a)} $\bigcup_{a\in G}a\action U=X$
for every non-empty open $U\subseteq X$.   \Prf\ If $x\in X$, then the
orbit of $x$ must meet $U$, so there is a $a\in G$ such that
$a\action x\in U$;  but this means that $x\in a^{-1}\action U$.\ \QeD

Fix some point $z_0$ of $X$ and write $\Cal V$ for the set of open sets
containing $z_0$.   Then if $K$, $L$ are disjoint compact subsets of $X$
there is a $U\in\Cal V$ such that, for every $a\in G$, at most one of
$K$, $L$ meets $a\action U$.   \Prf\ By hypothesis, there is a non-empty
open set $V$ such that, for every $a\in G$, at most one of $K$, $L$
meets $a\action V$.   Now there is an $b\in G$ such that
$b\action z_0\in V$;  set
$U=b^{-1}\action V$.   Because $b^{-1}$ acts on $X$ as a homeomorphism,
$U\in\Cal V$;  and if $a\in G$, then
$a\action U=(ab^{-1})\action V$ can meet at most one of $K$ and
$L$.\ \Qed

\medskip

{\bf (b)} If $U\in\Cal V$ and $A\subseteq X$ is any relatively compact
set, then $\{a\action U:a\in G\}$ is an open cover of $X$, so there is a
finite set $I\subseteq G$ such that
$\overline{A}\subseteq\bigcup_{a\in I}a\action U$.   Write $\high{A:U}$
for $\min\{\#(I):I\subseteq G,\,A\subseteq\bigcup_{a\in I}a\action U\}$.

\medskip

{\bf (c)} The following facts are now elementary.

\medskip

\quad{\bf (i)} If $U\in\Cal V$ and $A$, $B\subseteq X$ are relatively
compact, then

\Centerline{$0\le\high{A:U}\le\high{A\cup B:U}
\le\high{A:U}+\high{B:U}$,}

\noindent and $\high{A:U}=0$ iff $A=\emptyset$.

\medskip

\quad{\bf (ii)} If $U$, $V\in\Cal V$ and $V$ is relatively compact, and
$A\subseteq X$ also is relatively compact, then

\Centerline{$\high{A:U}\le\high{A:V}\,\high{V:U}$.}

\noindent\Prf\ If $A\subseteq\bigcup_{a\in I}a\action V$ and
$V\subseteq\bigcup_{b\in J}b\action U$, then
$A\subseteq\bigcup_{a\in I,b\in J}(ab)\action U$.\ \Qed

\medskip

\quad{\bf (iii)} If $U\in\Cal V$, $A\subseteq X$ is relatively compact
and $b\in G$, then $\high{b\action A:U}=\high{A:U}$.   \Prf\ If
$I\subseteq G$ and $A\subseteq\bigcup_{a\in I}a\action U$, then
$b\action A\subseteq\bigcup_{a\in I}(ba)\action U$, so
$\high{b\action A:U}\le\#(I)$;  as $I$ is arbitrary,
$\high{b\action A:U}\le\high{A:U}$.
On the other hand, the same argument shows that

\Centerline{$\high{A:U}\le\high{b^{-1}\action b\action A:U}
=\high{A:U}$,}

\noindent so we must have equality.\ \Qed

\medskip

{\bf (d)} Fix a relatively compact $V_0\in\Cal V$.   (This is the first
place where we use the hypothesis that $X$ is locally compact.)   For
every $U\in\Cal V$ and every relatively compact set $A\subseteq X$ write

$$\lambda_UA=\bover{\high{A:U}}{\high{V_0:U}}.$$

\noindent Then (c) tells us immediately that

\quad(i) if $A$, $B\subseteq X$ are relatively compact,

\Centerline{$0\le\lambda_UA\le\lambda_U(A\cup B)
\le\lambda_UA+\lambda_UB$;}

\quad(ii) $\lambda_UA\le\high{A:V_0}$ for every relatively compact
$A\subseteq X$;

\quad(iii) $\lambda_U(b\action A)=\lambda_UA$ for every relatively
compact $A\subseteq X$ and every $b\in G$;

\quad(iv) $\lambda_UV_0=1$.

\medskip

{\bf (e)} Now for the point of the hypothesis (iii) of the theorem.   If
$K$, $L$ are disjoint compact subsets of $X$, there is a $V\in\Cal V$
such that $\lambda_U(K\cup L)=\lambda_UK+\lambda_UL$ whenever
$U\in\Cal V$ and $U\subseteq V$.   \Prf\ By (a), there is a $V\in\Cal V$
such that
any translate $a\action V$ can meet at most one of $K$ and $L$.   Take
any $U\in\Cal V$ included in $V$.   Let $I\subseteq G$ be such that
$\bigcup_{a\in I}a\action U\supseteq K\cup L$ and
$\#(I)=\high{K\cup L:U}$.   Then

\Centerline{$I'=\{a:a\in I,\,K\cap a\action U\ne\emptyset\}$,
\quad$I''=\{a:a\in I,\,L\cap a\action U\ne\emptyset\}$}

\noindent are disjoint.  $K\subseteq\bigcup_{a\in I'}a\action U$, so
$\high{K:U}\le\#(I')$, and similarly $\high{L:U}\le\#(I'')$.   But this
means that

\Centerline{$\high{K:U}+\high{L:U}\le\#(I')+\#(I'')\le\#(I)
=\high{K\cup L:U}$,}

\Centerline{$\lambda_UK+\lambda_UL\le\lambda_U(K\cup L)$.}

\noindent As we already know that
$\lambda_U(K\cup L)\le\lambda_UK+\lambda_UL$, we must have equality, as
claimed.\ \Qed

\medskip

{\bf (f)} Now let $\Cal F$ be an ultrafilter on $\Cal V$ containing all
sets of the form $\{U:U\in\Cal V,\,U\subseteq V\}$ for $V\in\Cal V$.
If $A\subseteq X$ is relatively compact, $0\le\lambda_UA\le\high{A:V_0}$
for every $U\in\Cal V$, so $\lambda A=\lim_{U\to\Cal F}\lambda_UA$ is
defined in $[0,\high{A:V_0}]$.   From (d-i) and (d-iii) we see that

\Centerline{$0\le\lambda(b\action A)=\lambda A
\le\lambda(A\cup B)\le\lambda A+\lambda B$}

\noindent for all relatively compact $A$, $B\subseteq X$ and $b\in G$.
From (d-iv) we see that $\lambda V_0=1$.   Moreover, from (e) we see
that if $K$, $L\subseteq X$ are disjoint compact sets,

\Centerline{$\{U:U\in\Cal V,\,\lambda_U(K\cup L)
=\lambda_UK+\lambda_UL\}\in\Cal F$,}

\noindent so $\lambda(K\cup L)=\lambda K+\lambda L$.

\medskip

{\bf (g)} By 416M, there is a Radon measure $\mu$ on $X$ such that

\Centerline{$\mu K=\inf\{\lambda L:L\subseteq X$ is compact,
$K\subseteq\interior L\}$}

\noindent for every compact set $K\subseteq X$.   Now $\mu$ is
$G$-invariant.   \Prf\ Take $b\in G$.   If $K$, $L\subseteq X$ are
compact and $K\subseteq\interior L$, then
$b\action K\subseteq\interior b\action L$, because $x\mapsto b\action x$
is a homeomorphism;  so

\Centerline{$\mu(b\action K)\le\lambda(b\action L)
=\lambda L$.}

\noindent As $L$ is arbitrary, $\mu(b\action K)\le\mu K$.   As $b$ and
$K$ are arbitrary, $\mu$ is $G$-invariant, by 441Bb.\ \Qed

\medskip

{\bf (h)} Finally, $\mu V_0\ge\lambda V_0\ge 1$, so $\mu$ is non-zero.
}%end of proof of 441C

\leader{441D}{}\cmmnt{ The hypotheses of 441C are deliberately drawn
as widely as possible.   The principal application is the one for which
the chapter is named.

\medskip

\noindent}{\bf Definition} If $G$ is a topological group, a {\bf left
Haar measure} on $G$ is a non-zero quasi-Radon measure $\mu$ on $G$
which is invariant for the left action of $G$ on itself\cmmnt{, that
is, $\mu(aE)=\mu E$ whenever $\mu$ measures $E$ and $a\in G$}.

Similarly, a {\bf right Haar measure} is a non-zero quasi-Radon measure
$\mu$ such that $\mu(Ea)=\mu E$ for every $E\in\dom\mu$, $a\in G$.

\cmmnt{(My reasons for requiring `quasi-Radon' here will appear in
\S\S442 and 443.)}

\leader{441E}{Theorem} A locally compact Hausdorff topological group has
left and right Haar measures, which are both Radon measures.

\proof{ Both the left and right actions of $G$ on itself satisfy the
conditions of 441C.   \Prf\ In both cases, condition (i) is just the
(separate) continuity of multiplication, and (ii) is trivial, as every
orbit is the whole of $G$.   As for (iii), let us take the left action
first.  Given disjoint compact subsets $K$, $L$ of $G$, then
$M=\{y^{-1}z:y\in K,\,z\in L\}$ is a compact subset of $G$ not
containing the identity $e$.   Because the topology is Hausdorff, $M$ is
closed and $X\setminus M$ is a neighbourhood of $e$.   Because
multiplication and inversion are continuous, there are open
neighbourhoods $V$, $V'$ of $e$ such that $uv^{-1}\in G\setminus M$
whenever $u\in V$ and $v\in V'$.   Set $U=V\cap V'$;  then $U$ is a
non-empty open set in $G$.

\Quer\ Suppose, if possible, that there is a $a\in G$ such that $aU$
meets both $K$ and $L$.   Take $y\in K\cap aU$ and $z\in L\cap aU$.
Then $a^{-1}y\in U\subseteq V$ and $a^{-1}z\in U\subseteq V'$, so

\Centerline{$y^{-1}z=(a^{-1}y)^{-1}a^{-1}z\in G\setminus M$;}

\noindent but also $y^{-1}z\in M$.\ \Bang

Thus $aU$ meets at most one of $K$, $L$ for any $a\in G$.   As $K$ and
$L$ are arbitrary, condition (iii) of 441C is satisfied.

For the right action, we use the same ideas, but vary the formulae.
Set $M=\{yz^{-1}:y\in K,\,z\in L\}$, and choose $V$ and $V'$ such that
$uv^{-1}\in X\setminus M$ for $u\in V$, $v\in V'$.   Then if $a\in G$,
$y\in K$ and $z\in L$, $za(ya)^{-1}\in M$ and one of $za$, $ya$ does not
belong to $U=V\cap V'$, that is, one of $z$, $y$ does not belong to
$Ua^{-1}=a\action U$.\ \Qed

Then 441C provides us with non-zero left and right Haar measures on $G$,
and also tells us that they are Radon measures.
}%end of proof of 441E

\leader{441F}{}\cmmnt{ A different type of example is provided by
locally compact metric spaces.

\medskip

\noindent}{\bf Definition} If $(X,\rho)$ is any metric space, its {\bf
isometry group} is the set of permutations $g:X\to X$ which are {\bf
isometries}, that is, $\rho(g(x),g(y))=\rho(x,y)$ for all $x$, $y\in X$.

\leader{441G}{The topology of an isometry group} Let $(X,\rho)$ be a
metric space and $G$ the isometry group of $X$.

\spheader 441Ga Give $G$ the topology
of pointwise convergence inherited from the product topology of $X^X$.
Then $G$ is a Hausdorff topological group and the action of $G$ on $X$
is continuous.    \prooflet{\Prf\ If
$x\in X$, $g_0$, $h_0\in G$ and $\epsilon>0$, then
$V=\{g:\rho(gh_0(x),g_0h_0(x))\le\bover12\epsilon\}$ is a neighbourhood
of $g_0$ and $V'=\{h:\rho(h(x),h_0(x))\le\bover12\epsilon\}$ is a
neighbourhood of $h_0$.   If $g\in V$ and $h\in V'$ then

\Centerline{$\rho(gh(x),g_0h_0(x))
\le\rho(gh(x),gh_0(x))+\rho(gh_0(x),g_0h_0(x))
\le\rho(h(x),h_0(x))+\Bover12\epsilon
\le\epsilon$.}

\noindent As $g_0$, $h_0$ and $\epsilon$ are arbitrary, the function
$(g,h)\mapsto gh(x)$ is continuous;  as $x$ is arbitrary, multiplication
on $G$ is continuous.    As for inversion, suppose that $g_0\in G$,
$\epsilon>0$ and $x\in X$.   Then
$V=\{g:\rho(gg_0^{-1}(x),x)\le\epsilon\}$ is a neighbourhood of $g_0$,
and if $g\in V$ then

\Centerline{$\rho(g^{-1}(x),g_0^{-1}(x))
=\rho(x,gg_0^{-1}(x))\le\epsilon$.}

\noindent Because $g_0$, $\epsilon$ and $x$ are arbitrary, inversion on
$G$ is continuous, and $G$ is a topological group.   Because $X$ is
Hausdorff, so is $G$.

To see that the action is continuous, take $g_0\in G$, $x_0\in X$ and
$\epsilon>0$.   Then $V=\{g:g\in G,\,\rho(g(x_0),g_0(x_0))\penalty-100
\le\bover12\epsilon\}$ is a neighbourhood of
$g_0$.   If $g\in V$ and $x\in U(x_0,\bover12\epsilon)$, then

\Centerline{$\rho(g(x),g_0(x_0))
\le\rho(g(x),g(x_0))+\rho(g(x_0),g_0(x_0))
\le\rho(x,x_0)+\Bover12\epsilon
\le\epsilon$.}

\noindent As $g_0$, $x_0$ and $\epsilon$ are arbitrary,
$(g,x)\mapsto g(x):G\times X\to X$ is continuous.\ \Qed}

\spheader 441Gb If $X$ is compact, so is $G$.   \prooflet{\Prf\
By Tychonoff's theorem (3A3J), $X^X$ is compact.   Suppose that
$g\in X^X$ belongs to the closure of $G$ in $X^X$.   For any $x$,
$y\in X$, the set $\{f:f\in X^X$, $\rho(f(x),f(y))=\rho(x,y)\}$ is closed
and includes $G$, so contains $g$;  thus $g$ is an isometry.   \Quer\ If
$g[X]\ne X$, take $x\in X\setminus g[X]$ and set $x_n=g^n(x)$ for every
$n\in\Bbb N$.   Because $g$ is continuous and $X$ is compact, $g[X]$ is
closed and there is some $\delta>0$ such that
$U(x,\delta)\cap g[X]=\emptyset$.   But this means that

\Centerline{$\rho(x_m,x_n)=\rho(g^m(x),g^m(x_{n-m}))
=\rho(x,x_{n-m})\ge\delta$}

\noindent whenever $m<n$, so that $\sequencen{x_n}$ can have no cluster
point in $X$;  which is impossible, because $X$ is supposed to be
compact.\ \BanG\   This shows that $g$ is surjective and belongs to $G$.
As $g$ is arbitrary, $G$ is closed in $X^X$, therefore compact.\ \Qed}

\leader{441H}{Theorem} If $(X,\rho)$ is a non-empty locally compact
metric space with isometry group $G$, then there is a non-zero
$G$-invariant Radon measure on $X$.

\proof{{\bf (a)} Fix any $x_0\in X$, and set
$Z=\overline{\{g(x_0):g\in G\}}$;  then $Z$ is a closed subset of $X$,
so is in itself locally
compact.   Let $H$ be the isometry group of $Z$.

\medskip

{\bf (b)} We need to know that $g\restr Z\in H$ for every $g\in G$.
\Prf\ Because $g:X\to X$ is a homeomorphism,

\Centerline{$g[Z]=\overline{\{gg'(x_0):g'\in G\}}=Z$,}

\noindent so $g\restr Z$ is a permutation of $Z$, and of
course it is an isometry, that is, belongs to $H$.\ \Qed

\medskip

{\bf (c)} Now $Z$ and $H$ satisfy the conditions of 441C.

\medskip

\Prf{\bf (i)} is true just because all isometries are continuous.

\medskip

\quad{\bf (ii)} Take $z\in Z$ and let $U$ be a
non-empty relatively open subset of $Z$.   Then $U=Z\cap V$ for some
open set $V\subseteq X$;  as $Z\cap V\ne\emptyset$, there must be a
$g_0\in G$ such that $g_0(x_0)\in V$.   At the same time, there is a
sequence $\sequencen{h_n}$ in $G$ such that
$z=\lim_{n\to\infty}h_n(x_0)$.   Now

\Centerline{$\rho(g_0(x_0),g_0h_n^{-1}(z))
=\rho(h_n(x_0),z)\to 0$}

\noindent as $n\to\infty$,
so there is some $n$ such that $g_0h_n^{-1}(z)\in V$;  of
course $g_0h_n^{-1}(z)$ also belongs to $U$, while $g_0h_n^{-1}\restr Z$
belongs to $H$, by (b) above.   As $U$ is arbitrary, the $H$-orbit of
$z$ is dense in $Z$;  as $z$ is arbitrary, $H$ satisfies condition (ii)
of 441C.

\medskip

\quad{\bf (iii)} Given that $K$ and $L$ are disjoint compact subsets of
$Z$, there must be a $\delta>0$ such that $\rho(y,z)\ge\delta$ for every
$y\in K$, $z\in L$.   Let $U$ be the relatively open ball
$\{z:z\in Z,\,\rho(z,x_0)<\bover12\delta\}$.   Then for any $h\in H$,
$\rho(y,z)<\delta$ for any $y$, $z\in h[U]$, so $h[U]$ cannot
meet both $K$ and $L$.\ \Qed

\medskip

{\bf (d)} 441C therefore provides us with a non-zero $H$-invariant Radon
measure $\nu$ on $Z$.   Setting $\mu E=\nu(E\cap Z)$ whenever
$E\subseteq X$ and $E\cap Z\in\dom\nu$, it is easy to check that $\mu$
is $G$-invariant (using (b) again) and is a Radon measure on $Z$.
}%end of proof of 441H

\cmmnt{
\leader{441I}{Remarks (a)} Evidently there is a degree of overlap
between the cases above.   In an abelian group, for instance, the left
and right group actions necessarily give rise to the same invariant
measures.   If we take $X=\BbbR^2$, it has a group structure (addition)
for which we have invariant measures (e.g., Lebesgue measure);  these
are just the translation-invariant measures.   But 441H tells us that we
also have measures which are invariant under all isometries (rotations
and reflections as well as translations);  from where we now stand,
there is no surprise remaining in the fact that Lebesgue measure is
invariant under this much larger group.   (Though if you look back at
Chapter 26, you will see that a bare-handed proof of this takes a
certain amount of effort.)   If we turn next to the unit sphere
$\{x:\|x\|=1\}$ in $\BbbR^3$, we find that there is no useful group
structure, but it is a compact metric space, so carries invariant
measures, e.g., two-dimensional Hausdorff measure.

\spheader 441Ib
The arguments of 441C leave open the question of how far the invariant
measures constructed there are unique.   Of course any scalar multiple
of an invariant measure will again be invariant.   It is natural to give
a special place to invariant probability measures, and call them
`normalized';  whenever we have a non-zero totally finite invariant
measure we
shall have an invariant probability measure.   Counting measure on any
set will be invariant under any action of any group, and it is natural
to say that these measures also are `normalized';  when faced with a
finite set with two or more elements, we have to choose which
normalization seems most reasonable in the context.

\spheader 441Ic We shall see in 442B that Haar measures (with a given
handedness) are necessarily scalar multiples of each other.   In
442Ya, 443Ud and 443Xy we have further situations
in which invariant measures are essentially unique.   If, in
441C, there are non-trivial $G$-invariant subsets of $X$, we do not
expect such a result.   But there are interesting cases in which the
question seems to be open.
}%end of comment

\leader{441J}{}\cmmnt{ Of course we shall be much concerned with
integration with respect to invariant measures.   The results we need
are elementary corollaries of theorems already dealt with at length, but
it will be useful to have them spelt out.

\medskip

\noindent}{\bf Proposition} Let $X$ be a set, $G$ a group acting on $X$,
and $\mu$ a $G$-invariant measure on $X$.   If $f$ is a real-valued
function defined on a subset of $X$, and $a\in G$, then
$\int f(x)\mu(dx)=\int f(a\action x)\mu(dx)$ if either integral is
defined in $[-\infty,\infty]$.

\proof{ Apply 235G\formerly{2{}35I}
to the \imp\ functions $x\mapsto a\action x$ and
$x\mapsto a^{-1}\action x$.
}%end of proof of 441J

\vleader{72pt}{441K}{Theorem} Let $X$ be a set, $G$ a group acting on $X$, and
$\mu$ a $G$-invariant measure on $X$ with measure algebra $\frak A$.

(a) We have an action of $G$ on $\frak A$ defined by setting
$a\action E^{\ssbullet}=(a\action E)^{\ssbullet}$ whenever $a\in G$ and
$\mu$ measures $E$.

(b) We have an action of $G$ on $L^0=L^0(\mu)$ defined by setting
$a\action f^{\ssbullet}=(a\action f)^{\ssbullet}$ for every $a\in G$,
$f\in\eusm L^0(\mu)$.  

(c) For $1\le p\le\infty$ the formula of (b) defines actions of $G$ on
$L^p=L^p(\mu)$, and $\|a\action u\|_p=\|u\|_p$ for every $u\in L^p$,
$a\in G$.

\proof{{\bf (a)} If $E$, $F\in\dom\mu$ and
$E^{\ssbullet}=F^{\ssbullet}$, then (because $x\mapsto a^{-1}\action x$
is \imp) $(a\action E)^{\ssbullet}=(a\action F)^{\ssbullet}$.    So the
given formula does define a function from $G\times\frak A$ to $\frak A$.
It is now easy to check that it is an action.

\medskip

{\bf (b)} Let $f\in\eusm L^0=\eusm L^0(\mu)$, $a\in G$.   Set
$\phi_a(x)=a^{-1}\action x$ for $x\in X$, so that $\phi_a:X\to X$ is
\imp.   Then $a\action f=f\phi_a$ belongs to $\eusm L^0$.   If $f$,
$g\in\eusm L^0$ and $f\eae g$, then $f\phi_a\eae g\phi_a$, so
$(a\action f)^{\ssbullet}=(a\action g)^{\ssbullet}$.   This shows that
the given formula defines a function from $G\times L^0$ to $L^0$, and
again it is easy to see that it is an action.

\medskip

{\bf (c)} If $f\in\eusm L^p(\mu)$ then

\Centerline{$\int|a\action f|^pd\mu=\int|f(a^{-1}\action x)|^pd\mu
=\int|f|^pd\mu$}

\noindent by 441J.   So $a\action f\in\eusm L^p(\mu)$ and
$\|a\action f^{\ssbullet}\|_p=\|f^{\ssbullet}\|_p$.   Thus we have a
function from $G\times L_p$ to $L^p$, and once more it must be an
action.
}%end of proof of 441K

\leader{441L}{Proposition} Let $X$ be a locally compact Hausdorff space
and $G$ a group acting on $X$ in such a way that $x\mapsto a\action x$
is continuous for every $a\in G$.   If $\mu$ is a Radon measure on $X$,
then $\mu$ is $G$-invariant iff
$\int f(x)\mu(dx)=\int f(a\action x)\mu(dx)$ for every $a\in G$ and
every continuous function $f:X\to\Bbb R$ with compact support.

\proof{ For $a\in G$, set $T_a(x)=a\action x$ for every $x\in X$.
Then $\nu_a=\mu T_a^{-1}$ is a Radon measure on $X$.   If $f\in C_k(X)$,
then

\Centerline{$\int fd\nu_a=\int f T_ad\mu$}

\noindent by 235G.   Now

$$\eqalignno{\mu\text{ is }G\text{-invariant}
&\iff\nu_a=\mu\text{ for every }a\in G\cr
&\iff\int fd\nu_a=\int fd\mu\text{ for every }a\in G,\,f\in C_k(X)\cr
\displaycause{416E(b-v)}
&\iff\int fT_ad\mu=\int fd\mu\text{ for every }a\in G,\,f\in
C_k(X)\cr}$$

\noindent as claimed.
}%end of proof of 441L

\exercises{\leader{441X}{Basic exercises $\pmb{>}$(a)}
%\spheader 441Xa
Let $X$ be a set.   (i) Show that there is a one-to-one correspondence
between actions $\action$ of the group $\Bbb Z$ on $X$ and permutations
$f:X\to X$ defined by the formula
$n\action x=f^n(x)$.   (ii) Show that if $f:X\to X$ is a permutation, a
measure $\mu$ on $X$ is $\Bbb Z$-invariant for the corresponding action
iff $f$ and $f^{-1}$ are both
\imp.   (iii) Show that if $X$ is a compact Hausdorff space and
$\action$ is a continuous action of $\Bbb Z$ on $X$, then there is a
$\Bbb Z$-invariant Radon probability
measure on $X$.   \Hint{437T.}
%441A

\spheader 441Xb
Let $(X,\Tau,\nu)$ be a measure space and $G$ a group acting on $X$.
Set $\Sigma=\{E:E\subseteq X,\,g\action E\in\Tau$ for every $g\in G\}$,
and for $E\in\Sigma$ set

$$\eqalign{\mu E=\sup\{\sum_{i=0}^n\nu(g_i\action F_i):n\in\Bbb N,
  &\,F_0,\ldots,F_n\text{ are disjoint subsets of }E\cr
&\text{belonging to }\Sigma,\,
  g_i\in G\text{ for each }i\le n\}\cr}$$

\noindent  (cf.\ 112Yd).  Show that $\mu$ is a $G$-invariant measure on
$X$.
%441A

\spheader 441Xc
Let $X$ be a topological space and $G$ a group acting on
$X$ such that ($\alpha$) all the maps $x\mapsto a\action x$ are
continuous ($\beta$) all the orbits of $G$ are dense.   Show that any
non-zero $G$-invariant quasi-Radon measure on $X$ is strictly positive.
%441C

\sqheader 441Xd Let $G$ be a compact Hausdorff topological group.  (i)
Show that its conjugacy classes are closed.   (ii) Show that if $K$,
$L\subseteq G$ are disjoint compact sets then
$\{ac^{-1}da^{-1}:a\in G,\,c\in K,\,d\in L\}$ is a compact set not
containing $e$, so that
there is a neighbourhood $U$ of $e$ such that whenever $c^{-1}d\in U$
and $a\in G$ then either $aca^{-1}\notin K$ or $ada^{-1}\notin L$.
(iii) Show that every conjugacy class of $G$ carries a Radon probability
measure which is invariant under the conjugacy action of $G$.
%441C

\spheader 441Xe Let $(G,\cdot)$ be a topological group.   (i) On $G$
define a binary operation
$\diamond$ by saying that $x\diamond y=y\cdot x$ for all $x$,
$y\in G$.   Show that $(G,\diamond)$ is a topological group isomorphic
to $(G,\cdot)$, and that any element of $G$ has the same inverse for
either group operation.   (ii) Suppose that $\mu$
is a left Haar measure on $(G,\cdot)$.   Show that $\mu$ is a right Haar
measure on $(G,\diamond)$.   (iii) Set $\phi(a)=a^{-1}$ for
$a\in G$.   Show that if $\mu$ is a left Haar measure on $(G,\cdot)$ then
the image measure
$\mu\phi^{-1}$ is a right Haar measure on $(G,\cdot)$.   (iv)
Show that $(G,\cdot)$ has a left Haar measure iff it has a right Haar
measure.
(v) Show that $(G,\cdot)$ has a left Haar probability measure iff it has
a totally finite left Haar measure iff it has a right Haar probability
measure.   (iv) Show that $(G,\cdot)$ has a $\sigma$-finite left Haar
measure iff it has a $\sigma$-finite right Haar measure.
%441D

\sqheader 441Xf(i) For Lebesgue measurable
$E\subseteq\Bbb R\setminus\{0\}$, set $\nu E=\int_E\bover1{|x|}dx$.
Show that $\nu$ is
a (two-sided) Haar measure if $\Bbb R\setminus\{0\}$ is given the group
operation of multiplication.   (ii) For Lebesgue measurable
$E\subseteq\Bbb C\setminus\{0\}$, identified with
$\Bbb R^2\setminus\{0\}$, set $\nu E=\int_E\bover1{|z|^2}\mu(dz)$, where
$\mu$ is two-dimensional Lebesgue measure.   Show that $\nu$ is a
(two-sided)
Haar measure on $\Bbb C\setminus\{0\}$ if we take complex multiplication
for the group operation.   \Hint{263D.}
%441D

\sqheader 441Xg(i) Show that Lebesgue measure is a (two-sided) Haar
measure on $\BbbR^r$, for any $r\ge 1$, if we take addition for the
group operation.   (ii) Show that the usual measure on $\{0,1\}^I$ is a
two-sided Haar measure on $\{0,1\}^I$, for any set $I$, if we give
$\{0,1\}^I$ the group operation corresponding to its identification with
$\Bbb Z_2^I$.   (iii) Describe the corresponding Haar measure on
$\Cal PI$ when $\Cal PI$ is given the group operation $\symmdiff$.
%441D

\spheader 441Xh Let $G$ be a locally compact Hausdorff topological
group.   (i) Show that any (left) Haar measure on $G$ must be strictly
positive.   (ii) Show that $G$ has a totally finite (left) Haar measure
iff it is compact.   (iii) Show that $G$ has a $\sigma$-finite (left)
Haar measure iff it is $\sigma$-compact.
%441D

\sqheader 441Xi(i) Let $G$ and $H$ be topological groups with left Haar
measures $\mu$ and $\nu$.   Show that the quasi-Radon product measure on
$G\times H$ (417N) is a left Haar measure on $G\times H$.   (ii) Let
$\familyiI{G_i}$ be a family of topological groups, and suppose that
each $G_i$ has a left Haar probability measure (as happens, for
instance, if each $G_i$ is compact).   Show that the quasi-Radon product
measure on $\prod_{i\in I}G_i$ (417O) is a left Haar measure on
$\prod_{i\in I}G_i$.
%441D

\spheader 441Xj(i) Show that any (left) Haar measure on a topological
group, as defined in 441D, must be locally finite.   (ii) Show that any
(left) Haar measure on a locally compact Hausdorff group must be a Radon
measure.
%441D

\spheader 441Xk Let $r\ge 1$ be an integer, and set
$X=\{x:x\in\Bbb R^r,\,\|x\|=1\}$.   Let $G$ be the group of orthogonal
$r\times r$ real
matrices, so that $G$ acts transitively on $X$.   Show that (when given
its natural topology as a subset of $\BbbR^{r^2}$) $G$ is a compact
Hausdorff topological group.   Let $\mu$ be a left Haar measure on $G$,
and $x$ any point of $X$;  set $\phi_x(T)=Tx$ for $T\in G$.   Show that
the image measure
$\mu\phi_x^{-1}$ is a $G$-invariant measure on $X$, independent of the
choice of $x$.
%441E

\spheader 441Xl Let $X$ be a non-abelian Hausdorff
topological group with a left Haar probability measure $\mu$.   Let
$\lambda$ be the quasi-Radon product measure on $X^2$.   Show that
$\lambda\{(x,y):xy=yx\}\le\bover58$.   \Hint{if $Z$ is the centre of $X$,
$X/Z$ is not cyclic, so $\mu Z\le\bover14$.}
%441E

\spheader 441Xm Let $X$ be a compact metric space, and $g:X\to X$ any
isometry.   Show that $g$ is surjective.   \Hint{if $x\in X$, then
$\rho(g^mx,g^nx)\ge\rho(x,g[X])$ for any $m<n$.}
%441F

\spheader 441Xn Let $(X,\rho)$ be a locally compact metric space,
and $\Cal C$ the set of closed subsets of $X$ with its Fell
topology (4A2T).   Show that if $G$ is the isometry group of $X$ with
its topology of pointwise convergence, then $(g,F)\mapsto g[F]$
is a continuous action of $G$ on $\Cal C$.
%441F

\spheader 441Xo Let $(X,\rho)$ be a metric space and $\Cal K$ the family
of compact subsets of $X$ with the topology induced by
the Vietoris topology on the space of closed subsets of $X$ (4A2T).
Show that if $G$ is the isometry group of $X$ with its topology of
pointwise convergence, then $(g,K)\mapsto g[K]$ is a continuous
action of $G$ on $\Cal K$.
%441F

\sqheader 441Xp Let $(X,\rho)$ be a metric space, and $G$ its isometry
group with the topology $\frak T$ of pointwise convergence.   (i) Show
that if $X$ is compact, $\frak T$ can
be defined by the metric $(g,h)\mapsto\max_{x\in X}\rho(g(x),h(x))$.
(ii) Show that if
$\{y:\rho(y,x)\le\gamma\}$ is compact for every $x\in X$ and $\gamma>0$,
then $G$ is locally compact.   (iii) Show that if $X$ is separable then
$G$ is metrizable.   (iv) Show that if $(X,\rho)$ is complete then
$G$ is complete under its bilateral uniformity.
(v) Show that if $X$ is separable and $(X,\rho)$
is complete then $G$ is Polish.
%441G

\sqheader 441Xq Give $\Bbb N$ the zero-one metric $\rho$.
Let $G$ be the isometry group of $\Bbb N$ (that
is, the group of all permutations of $\Bbb N$) with its
topology of pointwise convergence.
(i) Show that $G$ is a G$_{\delta}$ subset of $\NN$, so is a Polish
group.   (ii) Show that if we set $\Delta(g,h)=\min\{n:n\in\Bbb N$,
$g(n)\ne h(n)\}$
and $\sigma(g,h)=1/(1+\Delta(g^{-1},h^{-1}))$ for distinct $g$,
$h\in G$, then $\sigma$ is a \rti\ metric on $G$ inducing its topology.
(iii) Show that there
is no complete \rti\ metric on $G$ inducing its topology.   \Hint{any
such metric must have the same Cauchy sequences as $\sigma$.}
(iv) Show that $G$ is not locally compact.
%441G

\sqheader 441Xr Let $r\ge 1$ be an integer, and $S_{r-1}$ the sphere
$\{x:x\in\BbbR^r,\,\|x\|=1\}$.   (i) Show that every isometry $\phi$
from $S_{r-1}$ to itself corresponds to an orthogonal $r\times r$ matrix
$T$.   \Hint{$T=\langle\phi(e_i)\dotproduct e_j\rangle_{i,j<r}$.}
(ii) Show that the topology of pointwise convergence on the isometry
group of $S_{r-1}$ corresponds to the topology on the set of $r\times r$
matrices regarded as a subset of $\BbbR^{r^2}$.
%441G

\spheader 441Xs Let $X$ be a locally compact metric space and $G$ a
subgroup of the isometry group of $X$.   Show that for every $x\in X$
there is a non-zero $G$-invariant Radon measure on
$\overline{\{g(x):g\in G\}}$.
%441H

\spheader 441Xt\dvAnew{2009} Let $r\ge 1$ be an integer, and
$X=\coint{0,1}^r$.
Let $G$ be the set of $r\times r$ matrices with integer
coefficients and determinant $\pm 1$, and for $A\in G$, $x\in X$ say that
$A\action x=\Matrix{\fraction{\eta_1}\\ \ldots\\ \fraction{\eta_r}}$ where
$\Matrix{\eta_1\\ \ldots\\ \eta_r}=Ax$ and $\fraction{\alpha}$ is the
fractional part of $\alpha$ for each $\alpha\in\Bbb R$.
(i) Show that $\action$ is an action of $G$ on $X$, and that
Lebesgue measure on $X$ is $G$-invariant.
(ii) Show that if
$X$ is given the compact Hausdorff topology corresponding to the bijection
$\alpha\mapsto(\cos 2\pi\alpha,\sin 2\pi\alpha)$ from $X$ to the unit
circle in $\Bbb R^2$, and $G$ is given its discrete topology, the
action is continuous.

\leader{441Y}{Further exercises (a)}
%\spheader 441Ya
Let $(X,\rho)$ be a metric space, and $\Cal C$ the family of
non-empty closed
subsets of $X$, with its Hausdorff metric $\tilde\rho$ (4A2T).   Show
that if $G$ is the isometry group of $X$, $(g,F)\mapsto g[F]$ is an
action of $G$ on $\Cal C$.
%query ?continuous

\spheader 441Yb Take $1\le s\le r\in\Bbb N$.   Let $\Cal C$ be the set of
closed subsets of $\BbbR^r$ with its Fell topology.   Let
$\Cal C_s\subseteq\Cal C$ be the set of $s$-dimensional linear subspaces
of $\BbbR^r$.    Show that $\Cal C_s$ is a closed subset of
$\Cal C$, therefore
a compact metrizable space in its own right, and that the group $G$ of
orthogonal $r\times r$ matrices acts transitively on $\Cal C_s$,
so that there is a $G$-invariant
Radon probability measure on $\Cal C_s$.
%441Ya, 441C

\spheader 441Yc For $w$, $z\in\Bbb C\setminus\{0\}$ set
$\rho(w,z)=|\Ln(\bover{w}{z})|$, where $\Ln v=\ln|v|+i\arg v$ for
non-zero complex numbers $v$.   (i) Show that $\rho$ is a metric on
$\Bbb C\setminus\{0\}$.   (ii) Show that the 2-dimensional Hausdorff
measure $\mu^{(\rho)}_{H2}$ derived from $\rho$ (471A) is a Haar measure
for the multiplicative group $\Bbb C\setminus\{0\}$.   (iii) Show that
$\mu^{(\rho)}_{H2}=\bover4{\pi}\nu$, where $\nu$ is the measure of
441Xf(ii).   \Hint{264I.}
%441Xf  441D

\spheader 441Yd Let $X$ be the group of real $r\times r$ orthogonal
matrices, where $r\ge 2$ is an integer.   Give $X$ the Euclidean metric,
regarding it as a subset of $\BbbR^{r^2}$.   (i) Show that the left and
right actions of $X$ on itself are distance-preserving.   (ii) Show that
$\bover{r(r-1)}2$-dimensional Hausdorff measure on $X$ is a two-sided
Haar measure.
%441D

\spheader 441Ye Let $X=\text{SO}(3)$ be the set of real $3\times 3$
orthogonal matrices with determinant $1$.   Give $X$ the metric
corresponding to its embedding in $9$-dimensional Euclidean space.
(i) Show that $X$ can be parametrized as the set of matrices

$$\phi\Matrix{z\\ \alpha\\ \theta}
=\Matrix{z&-\cos\theta\sqrt{1-z^2}&\sin\theta\sqrt{1-z^2}\\
\cos\alpha\sqrt{1-z^2}&z\cos\alpha\cos\theta-\sin\alpha\sin\theta
  &-z\cos\alpha\sin\theta-\sin\alpha\cos\theta\\
\sin\alpha\sqrt{1-z^2}&z\sin\alpha\cos\theta+\cos\alpha\sin\theta
  &-z\sin\alpha\sin\theta+\cos\alpha\cos\theta}$$

\noindent with $z\in[-1,1]$, $\alpha\in[-\pi,\pi]$ and
$\theta\in[-\pi,\pi]$.   (ii) Show that if $T$ is the $9\times 3$ matrix
which is the derivative of $\phi$ at $\Matrix{z\\ \alpha\\ \theta}$,
\leaveitout{$T$ is
$$\Matrix{1&0&0\\
\Bover{z\cos\theta}{\sqrt{1-z^2}}&0&\sin\theta\sqrt{1-z^2}\\
-\Bover{z\sin\theta}{\sqrt{1-z^2}}&0&\cos\theta\sqrt{1-z^2}\\
-\Bover{z\cos\alpha}{\sqrt{1-z^2}}&-\sin\alpha\sqrt{1-z^2}&0\\
\cos\alpha\cos\theta&-z\sin\alpha\cos\theta-\cos\alpha\sin\theta
  &-z\cos\alpha\sin\theta-\sin\alpha\cos\theta\\
-\cos\alpha\sin\theta&z\sin\alpha\sin\theta-\cos\alpha\cos\theta
  &-z\cos\alpha\cos\theta+\sin\alpha\sin\theta\\
-\Bover{z\sin\alpha}{\sqrt{1-z^2}}&\cos\alpha\sqrt{1-z^2}&0\\
\sin\alpha\cos\theta&z\cos\alpha\cos\theta-\sin\alpha\sin\theta
  &-z\sin\alpha\sin\theta+\cos\alpha\cos\theta\\
-\sin\alpha\sin\theta&-z\cos\alpha\sin\theta-\sin\alpha\cos\theta
  &-z\sin\alpha\cos\theta-\cos\alpha\sin\theta}$$
}
then $T\trs T=\Matrix{\Bover2{\sqrt{1-z^2}}&0&0\\0&2&2z\\0&2z&2}$ has
constant determinant,
so that if $\mu$ is Lebesgue measure on $[-1,1]\times[-\pi,\pi]^2$
then the image measure $\mu\phi^{-1}$ is a Haar measure on $X$.
\Hint{441Yd, 265E.}   (iii)
Show that if $A\in X$ corresponds to a rotation through an angle
$\gamma(A)\in[0,\pi]$
then its trace $\tr(A)$ (that is, the sum of its diagonal entries) is
$1+2\cos\gamma(A)$.   \Hint{$\tr(AB)=\tr(BA)$ for any square matrices $A$ and
$B$ of the same size.}   (iv) Show that if $X$ is given its Haar
probability measure then $\cos\gamma(A)$ has expectation $-\bover12$.
\leaveitout{$\Pr(\cos\gamma(A)\le q)
=1-\Bover{\arccos q}{\pi}+\Bover1{\pi}\sqrt{1-q^2}$;  the distribution
of $\gamma(A)$ has density function $x\mapsto\bover1{\pi}(1-\cos x)$ for
$0\le x\le\pi$ I think}
%441Yd 441D

\spheader 441Yf Let $(X,\Cal W)$ be a locally compact Hausdorff uniform
space, and suppose that $G$ is a group acting on $X$ `uniformly
equicontinuously';  that is, for every $W\in\Cal W$ there is a
$V\in\Cal W$ such that $(a\action x,a\action y)\in W$ whenever
$(x,y)\in V$ and
$a\in G$.   Show that there is a non-zero $G$-invariant Radon measure on
$X$.
%441H

\spheader 441Yg In 441Xk, show that $\mu\phi_x^{-1}$ is a scalar
multiple of Hausdorff $(r-1)$-dimensional measure on $X$.
%441Xk, 441E

\spheader 441Yh For any topological spaces $X$ and $Y$, and any set $G$
of functions from $X$ to $Y$, the {\bf compact-open} topology on $G$ is
the topology generated by sets of the form
$\{g:g\in G,\,g[K]\subseteq H\}$, where $K\subseteq X$ is compact and
$H\subseteq Y$ is open.   Show that if $(X,\rho)$ is a metric space and
$G$ is the isometry group of $X$, then the topology of pointwise
convergence on $G$ is its compact-open topology.
%441G

\leaveitout{\spheader 441Y? Let $U$ be a Hilbert space, and write $S$
for the unit sphere $\{u:\|u\|=1\}$.   Show that any isometry of $S$
corresponds to a bounded linear operator from $U$ to itself.   Show that
the topology of pointwise convergence on the isometry group of $S$
corresponds to the strong operator topology on $\eurm B(U;U)$.
%441Xr
}%end of leaveitout

\spheader 441Yi Let $X$ be a compact Hausdorff space and $G$ the group
of all homeomorphisms from $X$ to itself.   (i) Let $\Rho$ be the family
of
all continuous pseudometrics on $X$ (see 4A2Jg).   For $\rho\in\Rho$ and
$g$, $h\in G$, set $\tau_{\rho}(g,h)=\max_{x\in X}\rho(g(x),h(x))$.
Show that every $\tau_{\rho}$ is a \rti\ pseudometric on $G$, and that
$G$ with the topology generated by $\{\tau_{\rho}:\rho\in\Rho\}$ is a
topological group.
(ii) Show that this is the compact-open topology as defined in 441Yh.
(iii) Show that if $X$ is metrizable then $G$ is Polish.
%441G

\spheader 441Yj Let $X$ be a compact metric space and $G$ the isometry
group of $X$.   Show that every $G$-orbit in $X$ is closed.
%441H

\spheader 441Yk Let $T$ be any set, and $\rho$ the $\{0,1\}$-valued
metric on $T$.   Let $X$ be the set of partial orders on $T$, regarded
as a subset of $\Cal P(T\times T)$.   Show that $X$ is compact (cf.\
418Xv).   Let $G$ be the group of isometries of $T$ with its topology of
pointwise convergence.   Set
$\pi\action x=\{(t,u):(\pi^{-1}(t),\pi^{-1}(u))\in x\}$ for
$\pi\in G$ and $x\in X$.   Show that $\action$ is a continuous action of
$G$ on $X$.   Show that there is a strictly positive $G$-invariant Radon
probability measure $\mu$ on $X$.
%441I  what are the extreme points of $X$?

\spheader 441Yl Let $X$ be a set, $G$ a group acting on $X$, and $\mu$ a
$G$-invariant measure on $X$ with measure algebra $\frak A$.   Show that
if $\tau$ is any rearrangement-invariant extended Fatou norm on
$L^0(\frak A)$ then the formula of 441Kb defines a norm-preserving
action of $G$ on the Banach space $L^{\tau}$.
%441K

\spheader 441Ym (M.Elekes \& T.Keleti) Let $X$ be a set, $G$ a group
acting on $X$,  $\Sigma$ a
$\sigma$-algebra of subsets of $X$ such that $g\action E\in\Sigma$
whenever $E\in\Sigma$ and $g\in G$, and $H$ an element of $\Sigma$.
Suppose that
$\mu$ is a measure, with domain the subspace $\sigma$-algebra
$\Sigma_H$, such that $\mu(g\action E)=\mu E$ whenever $E\in\Sigma_H$
and $g\in G$ are such that $g\action E\subseteq H$.   (i) Show that
$\sum_{n=0}^{\infty}\mu E_n=\sum_{n=0}^{\infty}\mu E'_n$ whenever
$\sequencen{E_n}$ and $\sequencen{E'_n}$ are sequences in $\Sigma_H$ for
which there are sequences $\sequencen{g_n}$, $\sequencen{g'_n}$ in $G$
such that $\sequencen{g_n\action E_n}$ and
$\sequencen{g'_n\action E'_n}$ are partitions of the same subset of $X$.
(ii) Show that there is a $G$-invariant measure with domain $\Sigma$
which extends $\mu$.
%??

\spheader 441Yn Let $\action_X$ be an action of a group $G$ on a set $X$,
$\mu$ a $G$-invariant measure on $X$, $(\frak A,\bar\mu)$ its measure
algebra and $\action_{\frak A}$ the induced action on $\frak A$.
Set $Z=X^G$;  define $\phi:X\to Z$ by
setting $\phi(x)=\family{g}{G}{g^{-1}\action x}$ for $x\in X$;
let $\nu$ be the image
measure $\mu\phi^{-1}$, and $(\frak B,\bar\nu)$ its measure algebra.
Let $\action_Z$ be the left shift action of $G$ on $Z$;  show
that $\nu$ is $\action_Z$-invariant, so that there is an induced action
$\action_{\frak B}$ on $\frak B$.   Show that
$(\frak A,\bar\mu,\action_{\frak A})$ and
$(\frak B,\bar\nu,\action_{\frak B})$ are isomorphic.
%441K

\spheader 441Yo\dvAnew{2010}
Let $X$ be a topological space, $G$ a topological group and $\action$ a
continuous action of $G$ on $X$.   Let $M_{\text{qR}}^+$ be
the set of totally
finite quasi-Radon measures on $X$.   As in 4A5B-4A5C, write
$a\action E=\{a\action x:x\in E\}$ for $a\in G$ and $E\subseteq X$,
and $(a\action f)(x)=f(a^{-1}\action x)$ for $a\in G$, $x\in X$ and a
real-valued function $f$ defined at $a^{-1}\action x$.
(i) Show that we have
an action $\action$ of $G$ on $M_{\text{qR}}^+$ defined by saying that
$(a\action\nu)(E)=\nu(a^{-1}\action E)$ whenever $a\in G$,
$\nu\in M_{\text{qR}}^+$ and $E\subseteq X$ are such that $\nu$ measures
$a^{-1}\action E$.
(ii) Show that this action is continuous if we give $M_{\text{qR}}^+$
its narrow topology.   (iii) Show that if $\nu\in M_{\text{qR}}^+$,
$f\in\eusm L^1(\nu)$ is non-negative
and $f\nu$ is the corresponding indefinite-integral measure, then
$a\action(f\nu)$ is the
indefinite-integral measure $(a\action f)(a\action\nu)$ for every $a\in G$.

\spheader 441Yp\dvAnew{2012}
Let $X$ be a set, $G$ a group acting on $X$
and $\mu$ a totally finite $G$-invariant measure on
$X$ with domain $\Sigma$.   Suppose there is a probability measure $\nu$ on
$G$, with domain $\Tau$, such that
$(a,x)\mapsto a^{-1}\action x:G\times X\to X$ is
$(\Tau\tensorhat\Sigma,\Sigma)$-measurable and $\nu$ is invariant under the
left action of $G$ on itself.   Let $u\in L^0(\mu)$ be such that
$a\action u=u$ for every $a\in G$.  Show that there is an
$f\in\eusm L^0(\mu)$ such that $f^{\ssbullet}=u$ and $a\action f=f$ for
every $a\in G$.   \Hint{if $u=g^{\ssbullet}$ where $g:X\to\Bbb R$ is
$\Sigma$-measurable and $\mu$-integrable, try
$f(x)=\int g(a^{-1}\action x)\nu(da)$ when this is defined.}
%441K

\spheader 441Yq\dvAnew{2012}
Let $X$ be a topological space, $G$ a compact Hausdorff
group, $\action$ a continuous action of $G$ on $X$,
and $\mu$ a $G$-invariant quasi-Radon measure on $X$.
Let $u\in L^0(\mu)$ be such that
$a\action u=u$ for every $a\in G$.  Show that there is an
$f\in\eusm L^0(\mu)$ such that $f^{\ssbullet}=u$ and $a\action f=f$ for
every $a\in G$.
%441K

%could do quaternions to match 441Xf
}%end of exercises

\endnotes{
\Notesheader{441}
The proof I give of 441C is essentially the same as the proof of 441E in
{\smc Halmos 50}, \S58.

In part (f) of the proof of 441C I use an ultrafilter, relying on a
fairly strong consequence of the Axiom of Choice.   In this volume, as
in the last, I generally employ the axiom of choice without stopping to
consider whether it is really needed.   But Haar measure, at least, is
so important that I point out now that it can be built with much weaker
principles.   For a construction of Haar measure not dependent on
choice, see 561G in Volume 5.   I
ought to remark that the argument there leads us to a translation-invariant
linear functional rather than a measure, and that while there is still a
version of the Riesz representation theorem (564I), we
may get something less
than a proper countably additive measure if we do not have countable
choice.   Moreover, in the absence of
the full axiom of choice, we may find that we have fewer locally compact
topological groups than we expect.

While Haar measure is surely the pre-eminent application of the theory
here, I think that some of the other consequences of 441C (441H, 441Xd,
441Xk, 441Yb, 441Yf) are sufficiently striking to justify the trouble
involved in the extra generality.   I ought to remark that there are
important examples of invariant measures which have nothing to do with
441C.   Some of these
will appear in \S449;  for the moment I note only 441Xa.

{\smc Federer 69}, \S2.7, develops a general theory of `covariant'
measures $\mu$ (`relatively invariant' in {\smc Halmos 50}) such that
$\mu(a\action E)=\psi(a)\mu E$ for appropriate
sets $E\subseteq X$ and $a\in G$, where $\psi:G\to\ooint{0,\infty}$ is a
homomorphism;  for instance, taking $\mu$ to be Lebesgue measure on
$\BbbR^r$, we have $\mu T[E]=|\det T|\mu E$ for every linear space
isomorphism $T:\BbbR^r\to\BbbR^r$ and every measurable set $E$ (263A).
The theory I have described here can deal only with the subgroup of
isometric linear isomorphisms (that is, the orthogonal group).
Covariant measures arise naturally in many other contexts, such as 443T
below.

Hausdorff measures, being defined in terms of metrics, are necessarily
invariant under isometries, so appear naturally in this context,
starting with 264I.   There are interesting challenges both in finding
suitable metrics and in establishing exact constants, as in
441Yc-441Yd.

It is worth pausing over the topology of an isometry group, as described
in 441G.   It is quite surprising that such an elementary idea should
give us a topological group at all.   I offer some exercises
(441Xp-441Xr, %441Xp 441Xq 441Xr
441Yh) to help you relate the construction to material which may
be more familiar.
It is of course a `weak' topology, except when the underlying space is
compact (441Xp(i)).   See also 441Yi.   These groups are rarely locally
compact, and you may find them pushed out of your mind by the
extraordinary theory which you
will see developed in the next hundred pages; %104 or so
but in the last two sections of this chapter, and in \S493, they will
become leading examples.
}%end of notes

\discrpage

