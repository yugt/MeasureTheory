\frfilename{mt121.tex}
\versiondate{21.12.03}
\copyrightdate{1994}
     
\def\chaptername{Integration}
\def\sectionname{Measurable functions}
     
\newsection{121}
     
In this section, I take a step back to develop ideas relating to
$\sigma$-algebras of sets, following \S111;  there will be no mention
of `measures' here, except in the exercises.   The aim is to
establish the concept of `measurable function' (121C) and a variety of 
associated techniques.
The best single example of a $\sigma$-algebra to bear in mind when
reading this chapter is probably the $\sigma$-algebra of Borel subsets of
$\Bbb R$\cmmnt{ (111G)};  the 
$\sigma$-algebra of Lebesgue measurable subsets of 
$\Bbb R$\cmmnt{ (114E)} is a good second.
     
Throughout the exposition here (starting with 121A) I seek to deal with
functions which are not defined on the whole of the space $X$ under
consideration.   I believe that there are compelling reasons for facing
up to such functions at an early stage\cmmnt{ (see 121G)};  but undeniably
they add to the technical difficulties, and it would be fair to read
through the chapter once with the mental reservation that all functions
are taken to be defined everywhere, before returning to deal with the
general case.
     
     
\leader{121A}{Lemma} Let $X$ be a set and $\Sigma$ a $\sigma$-algebra of
subsets of $X$.   Let $D$ be any subset of $X$ and write
     
\Centerline{$\Sigma_D=\{E\cap D:E\in\Sigma\}$.}
     
\noindent Then $\Sigma_D$ is a $\sigma$-algebra of subsets of $D$.
     
\proof{{\bf (i)} $\emptyset=\emptyset\cap D\in\Sigma_D$ because
$\emptyset\in\Sigma$.
     
\medskip
     
{\bf (ii)} If $F\in\Sigma_D$, there is an $E\in\Sigma$ such that
$F=E\cap D$;  now $D\setminus F=(X\setminus E)\cap D\in\Sigma_D$ because
$X\setminus E\in\Sigma$.
     
\medskip
     
{\bf (iii)} If $\langle F_n\rangle_{n\in\Bbb N}$ is any sequence in
$\Sigma_D$, then for each $n\in\Bbb N$ we may choose an $E_n\in\Sigma$
such that $F_n=E_n\cap D$;  now
$\bigcup_{n\in\Bbb N}F_n=(\bigcup_{n\in\Bbb N}E_n)\cap D\in\Sigma_D$ because
$\bigcup_{n\in\Bbb N}E_n\in\Sigma$.
}%end of proof of 121A
     
\medskip
     
\noindent{\bf Notation} I will call $\Sigma_D$ the {\bf subspace
$\sigma$-algebra} of subsets of $D$, and I will say that its members are
{\bf relatively measurable} in $D$.   
$\Sigma_D$ is also sometimes called the {\bf trace} of $\Sigma$ on $D$.
     
\leader{121B}{Proposition} Let $X$ be a set, $\Sigma$ a $\sigma$-algebra
of subsets of $X$, and $D$ a subset of $X$.   Write $\Sigma_D$ for the
subspace $\sigma$-algebra of subsets of $D$.   Then for any function
$f:D\to\Bbb R$ the following assertions are equiveridical\cmmnt{, that is, if one of them is true so are all the others}:
     
\quad(i) $\{x:f(x)<a\}\in\Sigma_D$ for every $a\in\Bbb R$;
     
\quad(ii) $\{x:f(x)\le a\}\in\Sigma_D$ for every $a\in\Bbb R$;
     
\quad(iii) $\{x:f(x)>a\}\in\Sigma_D$ for every $a\in\Bbb R$;
     
\quad(iv) $\{x:f(x)\ge a\}\in\Sigma_D$ for every $a\in\Bbb R$.
     
\proof{{\bf (i)$\Rightarrow$(ii)} Assume (i), and let
$a\in\Bbb R$.   Then
     
\Centerline{$\{x:f(x)\le a\}
=\bigcap_{n\in\Bbb N}\{x:f(x)<a+2^{-n}\}\in\Sigma_D$}
     
\noindent because $\{x:f(x)<a+2^{-n}\}\in\Sigma_D$ for every $n$ and
$\Sigma_D$ is closed under countable intersections (111Dd).
Because $a$ is arbitrary, (ii) is true.
     
\wheader{121B}{4}{2}{2}{72pt}
     
{\bf (ii)$\Rightarrow$(iii)} Assume (ii), and let $a\in\Bbb R$.   Then
     
\Centerline{$\{x:f(x)>a\}=D\setminus\{x:f(x)\le a\}\in\Sigma_D$}
     
\noindent because $\{x:f(x)\le a\}\in\Sigma_D$ and $\Sigma_D$ is closed
under complementation.   Because $a$ is arbitrary, (iii) is
true.
     
\medskip
     
{\bf (iii)$\Rightarrow$(iv)} Assume (iii), and let $a\in\Bbb R$.   Then
     
\Centerline{$\{x:f(x)\ge a\}
=\bigcap_{n\in\Bbb N}\{x:f(x)>a-2^{-n}\}\in\Sigma_D$}
     
\noindent because $\{x:f(x)>a-2^{-n}\}\in\Sigma_D$ for every $n$ and
$\Sigma_D$ is closed under countable intersections.
Because $a$ is arbitrary, (iv) is true.
     
\medskip
     
{\bf (iv)$\Rightarrow$(i)} Assume (iv), and let $a\in\Bbb R$.   Then
     
\Centerline{$\{x:f(x)<a\}=D\setminus\{x:f(x)\ge a\}\in\Sigma_D$}
     
\noindent because $\{x:f(x)\ge a\}\in\Sigma_D$ and $\Sigma_D$ is closed
under complementation.   Because $a$ is arbitrary, (i) is
true.
}%end of proof of 121B
     
\leader{121C}{Definition} Let $X$ be a set, $\Sigma$ a $\sigma$-algebra
of subsets of $X$, and $D$ a subset of $X$.   A function $f:D\to\Bbb R$
is called {\bf measurable} (or {\bf $\Sigma$-measurable}) if it
satisfies any, or equivalently all, of the conditions (i)-(iv) of 121B.
     
If $X$ is $\Bbb R$ or $\BbbR^r$, and $\Sigma$ is its Borel
$\sigma$-algebra\cmmnt{ (111G)}, a $\Sigma$-measurable function is
called {\bf Borel measurable}.   If $X$ is $\Bbb R$ or $\BbbR^r$, and 
$\Sigma$ is
the $\sigma$-algebra of Lebesgue measurable sets\cmmnt{ (114E, 115E)},
a $\Sigma$-measurable function is called {\bf Lebesgue measurable}.
     
\cmmnt{\medskip
     
\noindent{\bf Remark} Naturally the principal case here is when $D=X$.
However, partially-defined functions are so common, and so important, in
analysis (consider, for instance, the real function $\ln\sin$) that it
seems worth while, from the beginning, to establish techniques for
handling them efficiently.
     
Many authors develop a theory of `extended real numbers' at this
point, working with $[-\infty,\infty]=\Bbb R\cup\{-\infty,\infty\}$, and
defining measurability for functions taking values in this set.   I
outline such a theory in \S135 below.
}%end of comment
     
\leader{121D}{Proposition} Let $X$ be $\BbbR^r$ for some $r\ge 1$, $D$
a subset of $X$, and $g:D\to\Bbb R$ a function.
     
(a) If $g$ is Borel measurable it is Lebesgue measurable.
     
(b) If $g$ is continuous it is Borel measurable.
     
(c) If $r=1$ and $g$ is monotonic it is Borel measurable.
     
\proof{{\bf (a)} This is immediate from the definitions in 121C,
if we recall that the Borel $\sigma$-algebra is included in the Lebesgue
$\sigma$-algebra (114G, 115G).
     
\medskip
     
{\bf (b)} Take $a\in\Bbb R$.   Set
     
\Centerline{$\Cal G=\{G:G\subseteq\BbbR^r\text{ is
open},\,g(x)<a\Forall x\in G\cap D\}$,}
     
\Centerline{$G_0=\bigcup\Cal G
=\{x:\exists\enskip G\in\Cal G,\,x\in G\}$.}
     
\noindent Then $G_0$ is a union of open sets, therefore open (1A2Bd).
Next,
     
\Centerline{$\{x:g(x)<a\}=G_0\cap D$.}
     
\noindent\Prf\ {\bf (i)} If $g(x)<a$, then (because $g$ is
continuous) there is a $\delta>0$ such that $|g(y)-g(x)|<a-g(x)$
whenever $y\in D$ and $\|y-x\|<\delta$.   But
$\{y:\|y-x\|<\delta\}$ is open (1A2D),  so belongs to $\Cal G$ and is
included in
$G_0$, and $x\in G_0\cap D$.   {\bf (ii)} If $x\in
G_0\cap D$, then there is a $G\in\Cal G$ such that $x\in G$;  now
$g(y)<a$ for every $y\in G\cap D$, so, in particular, $g(x)<a$.\ \Qed
     
Finally, $G_0$, being open, is a Borel set.   As $a$ is
arbitrary, $g$ is Borel measurable.
     
\medskip
     
{\bf (c)} Suppose first that $g$ is non-decreasing.   Let $a\in\Bbb R$
and write $E=\{x:g(x)<a\}$.  If $E=D$ or $E=\emptyset$ then of course it
is the intersection of $D$ with a Borel set.   Otherwise, $E$ is
non-empty and bounded above in $\Bbb R$, so has a supremum $c\in\Bbb R$.
Now $E$ must be either $D\cap\ooint{-\infty,c}$ or
$D\cap\ocint{-\infty,c}$, according to whether $c\in E$ or not, and in
either case is the intersection of $D$ with a Borel set (see 114G).
     
Similarly, if $g$ is non-increasing, $\{x:g(x)>a\}$ will again be the
intersection of $D$ with either $\emptyset$ or $\Bbb R$ or
$\ocint{-\infty,c}$ or $\ooint{-\infty,c}$ for some $c$.   So in this
case 121B(iii) will be satisfied.
     
\medskip
     
\noindent{\bf Remark} I see that in part (b) of the above proof I use
some basic facts about open sets in $\BbbR^r$.   These are covered in
detail in \S1A2.   If they are new to you it would probably be sensible
to rehearse the arguments with $r=1$, so that $D\subseteq\Bbb R$, before
embracing the general case.
}%end of proof of 121D
     
     
\vleader{108pt}{121E}{Theorem} 
Let $X$ be a set and $\Sigma$ a $\sigma$-algebra
of subsets of $X$.   Let $f$ and $g$ be real-valued functions defined on
domains $\dom f$, $\dom g\subseteq X$.
     
(a) If $f$ is constant it is
measurable.
     
(b) If $f$ and $g$ are measurable, so is $f+g$, where
$(f+g)(x)=f(x)+g(x)$ for $x\in
\dom f\cap\dom g$.
     
(c) If $f$ is measurable and $c\in\Bbb R$, then $cf$ is measurable,
where
$(cf)(x)=c\cdot f(x)$ for $x\in \dom f$.
     
(d) If $f$ and $g$ are measurable, so is $f\times g$, where $(f\times
g)(x)=f(x)\times g(x)$ for $x\in \dom f\cap\dom g$.
     
(e) If $f$ and $g$ are measurable, so is $f/g$, where
$(f/g)(x)=f(x)/g(x)$ when
$x\in \dom f\cap\dom g$ and $g(x)\ne 0$.
     
(f) If $f$ is measurable and $E\subseteq\Bbb R$ is a Borel set, then
there is an
$F\in\Sigma$ such that $f^{-1}[E]=\{x:f(x)\in E\}$ is equal to $F\cap
\dom f$.
     
(g) If $f$ is measurable and $h$ is  a Borel measurable function from a
subset $\dom h$
of $\Bbb R$ to $\Bbb R$, then $hf$ is measurable, where
$(hf)(x)=h(f(x))$ for $x\in\dom(hf)=\{y:y\in\dom f,\,f(y)\in\dom h\}$.
     
(h) If $f$ is measurable and $A$ is any set, then $f\restr A$ is
measurable, where $\dom(f\restr A)=A\cap\dom f$ and
$(f\restr A)(x)=f(x)$ for $x\in A\cap\dom f$.
     
\proof{ For any $D\subseteq X$ write $\Sigma_D$ for the subspace
$\sigma$-algebra of subsets of $D$.
     
\medskip
     
{\bf (a)} If $f(x)=c$ for every $x\in \dom f$, then
$\{x:f(x)<a\}=\dom f$ if $c<a$, $\emptyset$ otherwise, and in either
case
belongs to $\Sigma_{\dom f}$.
     
\medskip
     
{\bf (b)} Write $D=\dom(f+g)=\dom f\cap\dom g$.   If $a\in\Bbb R$ then
set $K=\{(q,q'):q,\,q'\in\Bbb Q,\,q+q'\le
a\}$.   Then $K$ is a subset of $\Bbb Q\times\Bbb Q$, so is countable
(111Fb, 1A1E).      For $q\in\Bbb Q$ choose sets $F_q$,
$G_q\in\Sigma$ such that
     
\Centerline{$\{x:f(x)<q\}=F_q\cap\dom f$,
\quad$\{x:g(x)<q\}=G_q\cap\dom g$.}
     
\noindent For each $(q,q')\in K$, the set
     
\Centerline{$E_{qq'}
=\{x:f(x)<q,\,g(x)<q'\}=F_q\cap G_{q'}\cap D$}
     
\noindent belongs to $\Sigma_D$.
Finally, if $(f+g)(x)<a$, then we can find
$q\in\ooint{f(x),a-g(x)}$, $q'\in\ocint{g(x),a-q}$, so that
$(q,q')\in K$ and
$x\in E_{qq'}$;  while if $(q,q')\in K$ and $x\in E_{qq'}$, then
$(f+g)(x)<q+q'\le a$.   Thus
     
\Centerline{$\{x:(f+g)(x)<a\}=\bigcup_{(q,q')\in K}E_{qq'}\in\Sigma_D$}
     
\noindent by 111Fa.   As $a$ is arbitrary, $f+g$ is measurable.
     
\medskip
     
{\bf (c)} Write $D=\dom f$.   Let $a\in\Bbb R$.   If $c>0$, then
     
\Centerline{$\{x:cf(x)<a\}=\{x:f(x)<\Bover{a}{c}\}\in\Sigma_D$.}
     
\noindent If $c<0$, then
     
\Centerline{$\{x:cf(x)<a\}=\{x:f(x)>\Bover{a}{c}\}\in\Sigma_D$.}
     
\noindent While if $c=0$, then $\{x:cf(x)<a\}$ is either $D$ or
$\emptyset$, as in (a) above, so belongs to $\Sigma_D$.   As $a$ is
arbitrary, $cf$ is measurable.
     
\medskip
     
{\bf (d)} Write $D=\dom(f\times g)=\dom f\cap\dom g$.   Let
$a\in\Bbb R$.   Let $K$ be
     
\Centerline{$\{(q_1,q_2,q_3,q_4):q_1,\ldots,q_4\in\Bbb Q,\,uv<a
\text{ whenever }u\in\ooint{q_1,q_2},\,v\in\ooint{q_3,q_4}\}$.}
     
\noindent Then $K$ is countable.   For $q\in\Bbb Q$ choose sets $F_q$,
$F'_q$, $G_q$, $G'_q\in\Sigma$ such that
     
\Centerline{$\{x:f(x)<q\}=F_q\cap\dom f$,
\quad$\{x:f(x)>q\}=F'_q\cap\dom f$,}
     
\Centerline{$\{x:g(x)<q\}=G_q\cap\dom g$,
\quad$\{x:g(x)>q\}=G'_q\cap\dom g$.}
     
\noindent  For $(q_1,q_2,q_3,q_4)\in K$ set
     
$$\eqalign{E_{q_1q_2q_3q_4}
&=\{x:f(x)\in\ooint{q_1,q_2},\,g(x)\in\ooint{q_3,q_4}\}\cr
&=D\cap F'_{q_1}\cap F_{q_2}\cap G'_{q_3}\cap G_{q_4}
\in\Sigma_D;\cr}$$
     
\noindent then $E=\bigcup_{(q_1,q_2,q_3,q_4)\in
K}E_{q_1q_2q_3q_4}\in\Sigma_D$.
     
Now $E=\{x:(f\times g)(x)<a\}$.   \Prf\ {\bf (i)} If $(f\times g)(x)<a$,
set
$u=f(x)$, $v=g(x)$.   Set
     
\Centerline{$\eta=\min(1,\Bover{a-uv}{1+|u|+|v|})>0$.}
     
\noindent   Take
$q_1,\ldots,q_4\in\Bbb Q$ such that
     
\Centerline{$u-\eta\le q_1<u<q_2\le u+\eta$,
\quad$v-\eta\le q_3<v<q_4\le v+\eta$.}
     
\noindent If $u'\in\ooint{q_1,q_2}$,
$v'\in\ooint{q_3,q_4}$, then $|u'-u|<\eta$ and $|v'-v|<\eta$, so
     
$$\eqalign{u'v'-uv
&=(u'-u)(v'-v)+(u'-u)v+u(v'-v)\cr
&<\eta^2+\eta|v|+|u|\eta
\le\eta(1+|u|+|v|)
\le a-uv,\cr}$$
     
\noindent and $u'v'<a$.   Accordingly $(q_1,q_2,q_3,q_4)\in K$.
Also $x\in E_{q_1q_2q_3q_4}$, so $x\in E$.   Thus $\{x:(f\times
g)(x)<a\}\subseteq E$.   {\bf (ii)} On the other hand, if $x\in E$,
there are
$q_1,\ldots,q_4$ such that $(q_1,q_2,q_3,q_4)\in K$ and $x\in
E_{q_1q_2q_3q_4}$, so that $f(x)\in\ooint{q_1,q_2}$ and
$g(x)\in\ooint{q_3,q_4}$ and $f(x)g(x)<a$.   So
$E\subseteq\{x:(f\times g)(x)<a\}$.\ \Qed
     
Thus $\{x:(f\times g)(x)<a\}\in\Sigma_D$.   As $a$ is arbitrary,
$f\times g$ is measurable.
     
\medskip
     
{\bf (e)} In view of (d), it will be enough to show that $1/g$ is
measurable.   Now if $a>0$,
$\{x:1/g(x)<a\}=\{x:g(x)>1/a\}\cup\{x:g(x)<0\}$;  if $a<0$, then
$\{x:1/g(x)<a\}=\{x:1/a<g(x)<0\}$;  and if $a=0$, then
$\{x:1/g(x)<a\}=\{x:g(x)<0\}$.   And all of these belong to
$\Sigma_{\dom 1/g}$.
     
\medskip
     
{\bf (f)} Write $D=\dom f$ and consider the set
     
\Centerline{$\Tau=\{E:E\subseteq\Bbb R,\,f^{-1}[E]\in\Sigma_D\}$.}
     
\noindent Then $\Tau$ is a $\sigma$-algebra of subsets of $\Bbb R$.
\Prf\ {(i)} $f^{-1}[\emptyset]=\emptyset\in\Sigma_D$, so
$\emptyset\in\Tau$.   {(ii)} If $E\in\Tau$, then
$f^{-1}[\Bbb R\setminus E]=D\setminus f^{-1}[E]\in\Sigma_D$ so
$\Bbb R\setminus E\in\Tau$.
{(iii)} If $\langle E_n\rangle_{n\in\Bbb N}$ is a sequence in
$\Tau$, then
$f^{-1}[\bigcup_{n\in\Bbb N}E_n]
=\bigcup_{n\in\Bbb N}f^{-1}[E_n]\in\Sigma_D$ because $\Sigma_D$ is a
$\sigma$-algebra, so $\bigcup_{n\in\Bbb N}E_n\in\Tau$.\ \Qed
     
Next, $\Tau$ contains all sets of the form $H_a=\ooint{-\infty,a}$ for
$a\in\Bbb R$, by the definition of measurability of $f$.   The result
follows by arguments already used in 114G above.
First, all open subsets of $\Bbb R$
belong to $\Tau$.   \Prf\ Let $G\subseteq\Bbb R$ be open.   Let
$K\subseteq\Bbb Q\times\Bbb Q$ be the set of pairs $(q,q')$ of rational
numbers such that $\coint{q,q'}\subseteq G$.    $K$
is countable.   Also, every $\coint{q,q'}$ belongs to $\Tau$, being
$H_{q'}\setminus H_q$.   So
$G'=\bigcup_{(q,q')\in K}\coint{q,q'}\in\Tau$.
     
By the definition of $K$, $G'\subseteq G$.   On the other hand, if $x\in
G$, there is a $\delta>0$
such that $\ooint{x-\delta,x+\delta}\subseteq G$.
Now there are rational numbers $q\in\ocint{x-\delta,x}$ and
$q'\in\ocint{x,x+\delta}$, so that $(q,q')\in K$ and $x\in
\coint{q,q'}\subseteq G'$.
As $x$ is arbitrary, $G=G'$ and $G\in\Tau$.\ \Qed
     
Finally, $\Tau$ is a
$\sigma$-algebra of subsets of $\Bbb R$ including the family of open
sets, so must contain every Borel set, by the definition of Borel set
(111G).
     
\medskip
     
{\bf (g)} If $a\in\Bbb R$, then $\{y:h(y)<a\}$ is of the form 
$E\cap\dom h$, where $E$ is a Borel subset of $\Bbb R$.   
Next, $f^{-1}[E]$ is of
the form $F\cap\dom f$, where $F\in\Sigma$, by (f) above.   So
     
\Centerline{$\{x:(hf)(x)<a\}=F\cap\dom hf\in\Sigma_{\dom hf}$.}
     
\noindent   As $a$ is
arbitrary, $hf$ is measurable.
     
\medskip
     
{\bf (h)} The point is that $\Sigma_{A\cap\dom f}=\{E\cap
A:E\in\Sigma_{\dom f}\}$.   So if $a\in\Bbb R$,
     
\Centerline{$\{x:(f\restr
A)(x)<a\}=A\cap\{x:f(x)<a\}\in\Sigma_{\dom(f\restr A)}$.}
}%end of proof of 121E
     
\cmmnt{\medskip
     
\noindent{\bf Remarks} Of course part (c) of this theorem is just a
matter of putting (a) and (d) together, while (e) is a consequence of
(d), (g) and the fact that continuous functions are Borel measurable
(121Db).
     
I hope you will recognise the technique in the proof of part (d) as a
version of arguments which may be used to prove that the limit of a
product is the product of the limits, or that the product of continuous
functions is continuous.   In fact (b) and (d) here, as well as the
theorems on sums and products of limits, are consequences of the fact
that addition and multiplication are continuous functions.   In
121K I give a general result which may be used to exploit such
facts.
     
Really, part (f) here is the essence of the concept of `measurable'
real-valued function.   The point of the definition in 121B-121C is that
the Borel $\sigma$-algebra of $\Bbb R$ can be generated by any of the
families $\{\ooint{-\infty,a}:a\in\Bbb R\}$,
$\{\ocint{-\infty,a}:a\in\Bbb R\},\ldots$.   (See 121Yc(ii).)
There are many routes covering this territory in rather
fewer words than I have used, at the cost of slightly greater
abstraction.
}%end of comment
     
     
\leader{121F}{Theorem} Let $X$ be a set and $\Sigma$ a $\sigma$-algebra
of
subsets of $X$.   Let $\langle f_n\rangle_{n\in\Bbb N}$ be a sequence of
$\Sigma$-measurable real-valued functions with domains included in $X$.
     
(a) Define a function $\lim_{n\to\infty}f_n$ by writing
     
\Centerline{$(\lim_{n\to\infty}f_n)(x)=\lim_{n\to\infty}f_n(x)$}
     
\noindent for all those
$x\in\bigcup_{n\in\Bbb N}\bigcap_{m\ge n}\dom f_m$ for which the limit exists in $\Bbb R$.   Then $\lim_{n\to\infty}f_n$ is
$\Sigma$-measurable.
     
     
(b) Define a function $\sup_{n\in\Bbb N}f_n$ by writing
     
\Centerline{$(\sup_{n\in\Bbb N}f_n)(x)=\sup_{n\in\Bbb N}f_n(x)$}
     
\noindent for all those $x\in\bigcap_{n\in\Bbb N}\dom f_n$ for which the
supremum exists in $\Bbb R$.   Then $\sup_{n\in\Bbb N}f_n$ is
$\Sigma$-measurable.
     
(c) Define a function $\inf_{n\in\Bbb N}f_n$ by writing
     
\Centerline{$(\inf_{n\in\Bbb N}f_n)(x)=\inf_{n\in\Bbb N}f_n(x)$}
     
\noindent for all those $x\in\bigcap_{n\in\Bbb N}\dom f_n$ for which the
infimum exists in $\Bbb R$.   Then $\inf_{n\in\Bbb N}f_n$ is
$\Sigma$-measurable.
     
(d) Define a function $\limsup_{n\to\infty}f_n$ by writing
     
\Centerline{$(\limsup_{n\to\infty}f_n)(x)=\limsup_{n\to\infty}f_n(x)$}
     
\noindent for all those 
$x\in\bigcup_{n\in\Bbb N}\bigcap_{m\ge n}\dom f_m$ for which the 
$\limsup$ exists in $\Bbb R$.   Then $\limsup_{n\to\infty}f_n$ is
$\Sigma$-measurable.
     
(e) Define a function $\liminf_{n\to\infty}f_n$ by writing
     
\Centerline{$(\liminf_{n\to\infty}f_n)(x)=\liminf_{n\to\infty}f_n(x)$}
     
\noindent for all those $x\in\bigcup_{n\in\Bbb N}\bigcap_{m\ge n}\dom
f_m$ for which the $\liminf$ exists in $\Bbb R$.   Then
$\liminf_{n\in\Bbb N}f_n$ is
$\Sigma$-measurable.
     
\proof{ For $n\in\Bbb N$, $a\in\Bbb R$ choose
$H_n(a)\in\Sigma$ such that $\{x:f_n(x)\le a\}=H_n(a)\cap \dom f_n$.
The proofs are now a matter of observing the following facts:
     
\medskip
     
{\bf (a)}
$\{x:(\lim_{n\to\infty}f_n)(x)\le a\}
=\dom(\lim_{n\to\infty}f_n)
  \cap\bigcap_{k\in\Bbb N}\bigcup_{n\in\Bbb N}
    \bigcap_{m\ge n}H_m(a+2^{-k})$;
     
\medskip
     
     
{\bf (b)} $\{x:(\sup_{n\in\Bbb N}f_n)(x)\le a\}=\dom(\sup_{n\in\Bbb
N}f_n)\cap\bigcap_{n\in\Bbb N}H_n(a)$;
     
\medskip
     
{\bf (c)} $\inf_{n\in\Bbb N}f_n=-\sup_{n\in\Bbb N}(-f_n)$;
     
\medskip
     
{\bf (d)} $\limsup_{n\to\infty}f_n
=\lim_{n\to\infty}\sup_{m\in\Bbb N}f_{m+n}$;
     
\medskip
     
{\bf (e)} $\liminf_{n\to\infty}f_n=-\limsup_{n\to\infty}(-f_n)$.
}%end of proof of 121F
     
     
\cmmnt{
\leader{121G}{Remarks} It is at this point that we first
encounter clearly the problem of functions which are not defined
everywhere.  (The quotient $f/g$ of 121Ee also need not be defined
everywhere on the
common domain of $f$ and $g$, but it is less important and more
easily dealt with.)   The whole point of the theory of measure and
integration, since Lebesgue, is that we can deal with limits of
sequences of functions;  and the set on which $\lim_{n\to\infty}f_n(x)$
exists can be decidedly irregular, even for apparently well-behaved
functions $f_n$.   (If you have encountered the theory of Fourier
series, then an appropriate example to think of is the sequence of
partial sums $f_n(x)={1\over 2}a_0+\sum_{k=1}^n(a_k\cos kx+b_k\sin kx)$
of a Fourier series in which $\sum_{k=1}^{\infty}|a_k|+|b_k|=\infty$, so
that the series is not uniformly absolutely summable, but may be
conditionally summable at certain points.)
     
I have tried to make it clear what domains I mean
to attach to the functions $\sup_{n\in\Bbb N}f_n$,
$\lim_{n\to\infty}f_n$, etc.   The guiding principle is that they should
be the set of all $x\in X$ for which the defining formulae
$\sup_{n\in\Bbb N}f_n(x)$, $\lim_{n\to\infty}f_n(x)$ can be interpreted
as real numbers.   (As I noted in 121C, I am for the time being avoiding
`$\infty$' as a value of a function, though it gives little
difficulty, and some formulae are more naturally interpreted by allowing
it.)   But in the case of $\lim$, $\limsup$, $\liminf$ it should be
noted that I am using the restrictive definition, that
$\lim_{n\to\infty}a_n$ can be regarded as existing only when there is
some $n\in\Bbb N$ such that $a_m$ is defined for every $m\ge n$.   There
are occasions when it would be more natural to admit the limit when we
know only that $a_m$ is defined for infinitely many $m$;  but such a
convention could make 121Fa false, unless care was taken.
     
As in 111E-111F, we can use the ideas of parts (b), (c) here to discuss
functions of the form $\sup_{k\in K}f_k$, $\inf_{k\in K}f_k$ for any
family $\langle f_k\rangle_{k\in K}$ of measurable functions indexed by
a non-empty countable set $K$.
     
In this theorem and the last, the functions $f$, $g$, $f_n$ have been
permitted to have arbitrary domains, and consequently there is nothing
that can be said about the domains of the constructed functions.
However, it is of course the case that if the original functions have
measurable domains, so do the functions constructed from them by the
rules I propose.   I spell out the details in the next proposition.
}%end of comment
     
\leader{121H}{Proposition}  Let $X$ be a set and $\Sigma$ a
$\sigma$-algebra
of subsets of $X$;  let $f$, $g$ and $f_n$, for $n\in\Bbb N$, be
$\Sigma$-measurable real-valued functions whose domains belong to
$\Sigma$.   Then all the functions
     
\Centerline{$f+g$, \quad$f\times g$, \quad$f/g$,}
     
\Centerline{$\sup_{n\in\Bbb N}f_n$, \quad$\inf_{n\in\Bbb N}f_n$,
\quad$\lim_{n\to\infty}f_n$, \quad$\limsup_{n\to\infty}f_n$,
\quad$\liminf_{n\to\infty}f_n$}
     
\noindent have domains belonging to $\Sigma$.   Moreover, if $h$ is a
Borel measurable real-valued function defined on a Borel subset of
$\Bbb R$, then $\dom hf\in\Sigma$.
     
\proof{ For the first two, we have $\dom(f+g)=\dom(f\times
g)=\dom f\cap \dom g$.   Next, if $E$ is a Borel subset of $\Bbb R$,
there is an $H\in\Sigma$ such that
$f^{-1}[E]=H\cap\dom f$;  so
$f^{-1}[E]\in\Sigma$.   Thus
     
\Centerline{$\dom hf=f^{-1}[\dom h]\in\Sigma$.}
     
\noindent Setting $h(a)=1/a$ for $a\in\Bbb R\setminus\{0\}$, we see that
$\dom(1/f)\in\Sigma$.   ($\dom h=\Bbb R\setminus\{0\}$ is a Borel set
because it is open.)   Similarly, $\dom(1/g)$ and $\dom(f/g)=\dom
f\cap\dom(1/g)$ belong to $\Sigma$.
     
Now for the infinite combinations.   Set
$H_n(a)=\{x:x\in\dom f_n,\,f_n(x)<a\}$ for
$n\in\Bbb N$, $a\in\Bbb R$;  as just explained, every $H_n(a)$
belongs to $\Sigma$.   Now
     
\Centerline{$\dom(\sup_{n\in\Bbb N}f_n)
=\bigcup_{m\in\Bbb N}\bigcap_{n\in\Bbb N}H_n(m)\in\Sigma$.}
     
\noindent Next, $|f_m-f_n|$ is
measurable, with domain in $\Sigma$, for all $m$, $n\in\Bbb N$ (applying
the results above to $-f_n=-1\cdot f_n$, $f_m-f_n=f_m+(-f_n)$ and
$|f_m-f_n|=|\,\,|\circ(f_m-f_n)$), so
     
\Centerline{$G_{mnk}
=\{x:x\in\dom f_m\cap\dom f_n,\,|f_m(x)-f_n(x)|\le 2^{-k}\}\in\Sigma$}
     
\noindent for all $m$, $n$, $k\in\Bbb N$.   Accordingly
     
\Centerline{$\dom(\lim_{n\to\infty}f_n)
=\{x:\Exists n,\,\langle f_m(x)\rangle_{m\ge n}$ is Cauchy$\}
=\bigcap_{k\in\Bbb N}\bigcup_{n\in\Bbb N}\bigcap_{m\ge n}G_{mnk}
\in\Sigma$.}
     
Manipulating the above results as in (c), (d) and (e) of the proof of
121F, we easily complete the proof.
}%end of proof of 121H
     
\cmmnt{\medskip
     
\noindent{\bf Remark} Note the use of the General Principle of
Convergence in the proof above.   I am not sure whether this will strike
you as `natural', and there are alternative methods;  but the formula
     
\Centerline{$\{x:\lim_{n\to\infty}f_n(x)$ exists in $\Bbb R\}
=\{x:\sequencen{f_n(x)}$ is Cauchy$\}$}
     
\noindent is one worth storing in your long-term memory.
}%end of comment
     
\leader{*121I}{}\cmmnt{ I end this section with two results which can be safely passed by on first reading, but which you will need at some point
to master if you wish to go farther into measure theory than the present
chapter, as both are essential parts of the concept of `measurable
function'.
     
\medskip
     
\noindent}{\bf Proposition} Let $X$ be a set and $\Sigma$ a
$\sigma$-algebra of subsets of $X$.   Let $D$ be a subset of $X$ and
$f:D\to\Bbb R$ a function.   Then $f$ is measurable iff there is a
measurable function $h:X\to\Bbb R$ extending $f$.
     
\proof{{\bf (a)} If $h:X\to\Bbb R$ is measurable and $f=h\restr D$, then
$f$ is measurable by 121Eh.
     
\medskip
     
{\bf (b)} Now suppose that $f$ is measurable.
     
\medskip
     
\quad{\bf (i)} For each
$q\in\Bbb Q$, the set $D_q=\{x:x\in D,\,f(x)\le q\}$ belongs to the subspace $\sigma$-algebra $\Sigma_D$, that is, there is an $E_q\in\Sigma$ such that $D_q=E_q\cap D$.   Set
     
\Centerline{$F=X\setminus\bigcup_{q\in\Bbb Q}E_q$,}
     
\Centerline{$G=\bigcap_{n\in\Bbb N}\bigcup_{q\in\Bbb Q,q\le -n}E_q$;}
     
\noindent then both $F$ and $G$ belong to $\Sigma$, and are disjoint
from $D$.   \Prf\ If $x\in D$, there is a $q\in\Bbb Q$ such that
$f(x)\le q$, so that $x\in E_q$ and $x\notin F$.   Also there is an
$n\in\Bbb N$ such that $f(x)>-n$, so that $x\notin E_{q'}$ for $q'\le-n$
and $x\notin G$.\ \Qed
     
Set $H=X\setminus(F\cup G)\in\Sigma$.   For $x\in H$,
     
\Centerline{$\{q:q\in\Bbb Q,\,x\in E_q\}$}
     
\noindent is non-empty and bounded below, so we may set
     
\Centerline{$h(x)=\inf\{q:x\in E_q\}$;}
     
\noindent for $x\in F\cup G$, set $h(x)=0$.   This defines
$h:X\to\Bbb R$.
     
\medskip
     
\quad{\bf (ii)} $h(x)=f(x)$ for $x\in D$.   \Prf\ As remarked above, $x\in H$.   If $q\in\Bbb Q$ and $x\in E_q$, then $f(x)\le q$;  consequently $h(x)\ge f(x)$.   On the other
hand, given $\epsilon>0$, there is a
$q\in\Bbb Q\cap[f(x),f(x)+\epsilon]$, and now $x\in E_q$, so
$h(x)\le q\le f(x)+\epsilon$;  as $\epsilon$ is arbitrary,
$h(x)\le f(x)$.\ \Qed
     
\medskip
     
\quad{\bf (iii)} $h$ is measurable.   \Prf\ If $a>0$ then
     
\Centerline{$\{x:h(x)<a\}=(H\cap\bigcup_{q<a}E_q)\cup(F\cup
G)\in\Sigma$,}
     
\noindent while if $a\le 0$
     
\Centerline{$\{x:h(x)<a\}=H\cap\bigcup_{q<a}E_q\in\Sigma$.   \Qed}
     
\noindent This completes the proof.
}%end of proof of 121I
     
\leader{*121J}{}\cmmnt{ The next proposition may illuminate 121E, as well as
being indispensable for the work of Volume 2.   I start with a useful
description of the Borel sets of $\BbbR^r$.
     
\medskip
     
\noindent}{\bf Lemma} Let $r\ge 1$ be an integer, and write $\Cal J$ for
the family of subsets of $\BbbR^r$ of the form
$\{x:\xi_i\le\alpha\}$ where $i\le r$, $\alpha\in\Bbb R$, writing
$x=(\xi_1,\ldots,\xi_r)$, as in \S115.   Then the
$\sigma$-algebra of subsets of $\BbbR^r$ generated by $\Cal J$ is
precisely the $\sigma$-algebra $\Cal B$ of Borel subsets of $\BbbR^r$.
     
\proof{{\bf (a)}   All the sets in $\Cal J$ are closed, so must
belong to $\Cal B$;  writing $\Sigma$ for the $\sigma$-algebra
generated by $\Cal J$, we must have $\Sigma\subseteq\Cal B$.
     
\medskip
     
     
{\bf (b)} The next step is to observe that all half-open
intervals of the form
     
\Centerline{$\ocint{a,b}
=\{x:\alpha_i<\xi_i\le\beta_i\Forall i\le r\}$}
     
\noindent belong to $\Sigma$;  this is because
     
\Centerline{$\ocint{a,b}=\bigcap_{i\le
r}(\{x:\xi_i\le\beta_i\}\setminus\{x:\xi_i\le\alpha_i\})$.}
     
\noindent     It follows that all open sets belong to $\Sigma$.
\Prf\ (Compare the proof of 121Ef.)  Let $G\subseteq\BbbR^r$ be an open
set.   Let $\Cal I$ be the set
of all intervals of the form
$\ocint{q,q'}$ which are included in $G$, where $q$, $q'\in\BbbQ^r$.
Then $\Cal
I$ is a countable subset of $\Sigma$, so (because $\Sigma$ is a
$\sigma$-algebra) $\bigcup\Cal I\in\Sigma$.   By the definition of
$\Cal I$, $\bigcup\Cal I\subseteq G$.   But also, if $x\in G$, there is a $\delta>0$ such that the open ball $U(x,\delta)$ with centre $x$ and radius $\delta$ is included in $G$ (1A2A).
Now, for each $i\le r$, we can find rational numbers $\alpha_i$,
$\beta_i$ such that
     
\Centerline{$\xi_i-\Bover{\delta}r
\le\alpha_i<\xi_i\le\beta_i<\xi_i+\Bover{\delta}r$,}
     
\noindent so that
     
\Centerline{$x\in\ocint{a,b}\subseteq U(x,\delta)\subseteq G$}
     
\noindent and $x\in\ocint{a,b}\in\Cal I$.   Thus $x\in\bigcup\Cal I$.
As $x$ is arbitrary, $G\subseteq\bigcup\Cal I$ and $G=\bigcup\Cal
I\in\Sigma$.\ \Qed
     
\medskip
     
{\bf (c)} Thus $\Sigma$ is a $\sigma$-algebra of sets containing
every open set, and must include $\Cal B$, the smallest such
$\sigma$-algebra.
}%end of proof of 121J
     
\cmmnt{\medskip
     
\noindent{\bf Remark} Compare the proof of 115G.
}%end of comment
     
\leader{*121K}{Proposition} Let $X$ be a set and $\Sigma$ a
$\sigma$-algebra of subsets of $X$.   Let $r\ge 1$ be an integer, and
$f_1,\ldots,f_r$ measurable functions defined on subsets of $X$.
Set $D=\bigcap_{i\le r}\dom f_i$ and for $x\in D$ set
$f(x)=(f_1(x),\ldots,f_r(x))\in\BbbR^r$.   Then
     
(a) for any Borel set $E\subseteq\BbbR^r$, $f^{-1}[E]$ belongs to the subspace $\sigma$-algebra $\Sigma_D$;
     
(b) if $h$ is a Borel measurable function from a subset $\dom h$ of
$\BbbR^r$ to $\Bbb R$, then the composition $hf$ is measurable.
     
\proof{{\bf (a)(i)} Consider the set
     
\Centerline{$\Tau=\{E:E\subseteq\BbbR^r,\,f^{-1}[E]\in\Sigma_D\}$.}
     
\noindent Then $\Tau$ is a $\sigma$-algebra of subsets of $\BbbR^r$.
\Prf\ (Compare 121Ef.) {\bf ($\pmb{\alpha}$)}
$f^{-1}[\emptyset]=\emptyset\in\Sigma_D$, so
$\emptyset\in\Tau$.   {\bf ($\pmb{\beta}$)} If $E\in\Tau$,
then $f^{-1}[\BbbR^r\setminus
E]=D\setminus f^{-1}[E]\in\Sigma_D$ so $\Bbb R\setminus E\in\Tau$.
{\bf ($\pmb{\gamma}$)} If $\langle E_n\rangle_{n\in\Bbb N}$ is a
sequence in $\Tau$, then
$f^{-1}[\bigcup_{n\in\Bbb N}E_n]
=\bigcup_{n\in\Bbb N}f^{-1}[E_n]\in\Sigma_D$ because $\Sigma_D$ is a
$\sigma$-algebra, so $\bigcup_{n\in\Bbb N}E_n\in\Tau$.\ \Qed
     
\medskip
     
\quad{\bf (ii)}
Next, for any $i\le r$ and $\alpha\in\Bbb R$, $J=\{x:\xi_i\le\alpha\}$
belongs to $\Tau$, because
     
\Centerline{$f^{-1}[J]
=\{x:x\in D,\, f_i(x)\le \alpha\}\in\Sigma_D$.}
     
\noindent So $\Tau$ includes the family $\Cal J$ of 121J and therefore
includes the $\sigma$-algebra $\Cal B$ generated by $\Cal J$, that is,
contains every Borel subset of $\BbbR^r$.
     
\medskip
     
{\bf (b)} Now the rest follows by the argument of 121Eg.   If
$a\in\Bbb R$, then $\{y:y\in\dom h,\,h(y)<a\}$ is of the form $E\cap\dom h$, where
$E$ is a
Borel subset of $\BbbR^r$, so $\{x:x\in\dom(hf),\,(hf)(x)<a\}
=f^{-1}[E]\cap\dom(hf)$ belongs to
$\Sigma_{\dom hf}$.
}%end of proof of 121K
     
\exercises{
\leader{121X}{Basic exercises $\pmb{>}$(a)}
%\spheader 121Xa
Let $X$ be a set, $\Sigma$ a $\sigma$-algebra of
subsets of $X$, and $D\subseteq X$.   Let $\sequencen{D_n}$ be a
partition of $D$ into relatively measurable sets and $\sequencen{f_n}$
a sequence of measurable real-valued functions such that
$D_n\subseteq\dom f_n$ for each $n$.   Define $f:D\to\Bbb R$ by setting
$f(x)=f_n(x)$ whenever $n\in\Bbb N$, $x\in D_n$.   Show that $f$ is
measurable.
%121C
     
\spheader 121Xb Let $X$ be a set and $\Sigma$ a
$\sigma$-algebra of subsets of $X$.   If $f$ and $g$ are measurable
real-valued functions defined on subsets of $X$, show that $f^+$, $f^-$,
$f\wedge g$ and $f\vee g$ are measurable, where
     
\Centerline{$f^+(x)=\max(f(x),0)$ for $x\in\dom f$,}
     
\Centerline{$f^-(x)=\max(-f(x),0)$ for $x\in\dom f$,}
     
\Centerline{$(f\vee g)(x)=\max(f(x),g(x))$ for $x\in\dom f\cap\dom g$,}
     
\Centerline{$(f\wedge g)(x)=\min(f(x),g(x))$ for
$x\in\dom f\cap\dom g$.}
%121E
     
\sqheader 121Xc Let $(X,\Sigma,\mu)$ be a measure space.   Write
$\eusm L^0$ for the set of real-valued functions $f$ such that
($\alpha$) $\dom f$ is a conegligible subset of $X$ ($\beta$) there is a
conegligible set $E\subseteq X$ such that $f\restr E$ is measurable.
(i) Show that the set $E$ of clause ($\beta$) in the last sentence may
be taken to belong to $\Sigma$ and be included in $\dom f$.   (ii) Show
that if $f$, $g\in\eusm L^0$ and $c\in\Bbb R$, then $f+g$, $cf$,
$f\times g$, $|f|$, $f^+$, $f^-$, $f\wedge g$, $f\vee g$
all belong to $\eusm L^0$.   (iii) Show that if $f$, $g\in\eusm L^0$ and
$g\ne 0$ a.e.\ then $f/g\in\eusm L^0$.   (iv) Show that if
$\sequencen{f_n}$ is a sequence in $\eusm L^0$ then the functions
     
\Centerline{$\lim_{n\to\infty}f_n$,\quad$\sup_{n\in\Bbb N}f_n$,
\quad$\inf_{n\in\Bbb N}f_n$,\quad $\limsup_{n\to\infty}f_n$,
\quad$\liminf_{n\to\infty}f_n$}
     
\noindent belong to $\eusm L^0$ whenever they are defined almost
everywhere as real-valued functions.   (v) Show that if $f\in\eusm L^0$
and $h:\Bbb R\to\Bbb R$ is Borel measurable then $hf\in\eusm L^0$.
%121F
     
\sqheader 121Xd Consider the following four families of subsets
of $\Bbb R$:
     
\Centerline{$\Cal A_1=\{\ooint{-\infty,a}:a\in\Bbb R\}$,
\quad$\Cal A_2=\{\ocint{-\infty,a}:a\in\Bbb R\}$,}
     
\Centerline{$\Cal A_3=\{\ooint{a,\infty}:a\in\Bbb R\}$,
\quad$\Cal A_4=\{\coint{a,\infty}:a\in\Bbb R\}$.}
     
\noindent Show that for each $j$ the $\sigma$-algebra of subsets of
$\Bbb R$ generated by $\Cal A_j$ is the $\sigma$-algebra of Borel sets.
%121J
     
\spheader 121Xe Let $D$ be any subset of $\BbbR^r$, where $r\ge 1$.
Write $\frak T_D$ for the set $\{G\cap D:G\subseteq\BbbR^r$ is
open$\}$.  (i) Show that $\frak T_D$ satisfies the properties of open
sets listed in 1A2B.   (ii) Let $\Cal B$ be the 
$\sigma$-algebra of Borel sets
in $\BbbR^r$, and $\Cal B(D)$ the subspace $\sigma$-algebra on $D$.
Show that $\Cal B(D)$ is just the $\sigma$-algebra of subsets of $D$
generated by $\frak T_D$.   \Hint{($\alpha$) observe that
$\frak T_D\subseteq\Cal B(D)$ ($\beta$) consider 
$\{E:E\subseteq\BbbR^r,\,E\cap D$ 
belongs to the $\sigma$-algebra generated by $\frak T_D\}$.}
%121K
     
\spheader 121Xf Let $(X,\Sigma,\mu)$ be a measure space and
define $\eusm L^0$ as in 121Xc.   Show that if $f_1,\ldots,f_r$ belong
to $\eusm L^0$ and $h:\BbbR^r\to\Bbb R$ is Borel measurable then
$h(f_1,\ldots,f_r)$ belongs to $\eusm L^0$.
%121Xc 121K
     
\leader{121Y}{Further exercises (a)}
%\spheader 121Ya
Let $X$ and $Y$ be sets, $\Sigma$ a $\sigma$-algebra of subsets of
$X$, $\phi:X\to Y$ a function and $g$ a real-valued function defined on a
subset of $Y$.   Set
$\Tau=\{F:F\subseteq Y,\,\phi^{-1}[F]\in\Sigma\}$;   then $\Tau$ is a
$\sigma$-algebra of subsets of $Y$ (see 111Xc).   (i) Show that if $g$ is
$\Tau$-measurable then $g\phi$ is $\Sigma$-measurable.   
(ii) Give an example in which $g\phi$ is
$\Sigma$-measurable but $g$ is not $\Tau$-measurable.
(iii) Show that if
$g\phi$ is $\Sigma$-measurable and {\it either} $\phi$ is injective 
{\it or} $\dom(g\phi)\in\Sigma$ {\it or} $\phi[X]\subseteq\dom g$, 
then $g$ is $\Tau$-measurable.\footnote{I am grateful to P.Wallace Thompson
for pointing out the error in the original version of this exercise.}   
%121C
     
\spheader 121Yb Let $X$ and $Y$ be sets, $\Tau$ a
$\sigma$-algebra of subsets of $Y$ and $\phi:X\to Y$ a function.   Set
$\Sigma=\{\phi^{-1}[F]:F\in\Tau\}$, as in 111Xd.   Show that a
function $f:X\to\Bbb R$ is $\Sigma$-measurable iff there is a
$\Tau$-measurable function $g:Y\to\Bbb R$ such that $f=g\phi$.
%121I
     
\spheader 121Yc Let $X$ and $Y$ be sets and $\Sigma$, $\Tau$
$\sigma$-algebras of subsets of $X$, $Y$ respectively.   I say that a
function $\phi:X\to Y$ is $(\Sigma,\Tau)${\bf-measurable} if
$\phi^{-1}[F]\in\Sigma$ for every $F\in\Tau$.   (i) Show that if
$\Sigma$, $\Tau$, $\Upsilon$ are $\sigma$-algebras of subsets of $X$,
$Y$, $Z$ respectively, and $\phi:X\to Y$ is $(\Sigma,\Tau)$-measurable,
$\psi:Y\to Z$ is $(\Tau,\Upsilon)$-measurable, then $\psi\phi:X\to Z$ is
$(\Sigma,\Upsilon)$-measurable.   
(ii) Suppose that $\Cal A\subseteq\Tau$ is such that $\Tau$ is the
$\sigma$-algebra of subsets of $Y$ generated by $\Cal A$ (111Gb).   Show
that $\phi:X\to Y$ is $(\Sigma,\Tau)$-measurable iff
$\phi^{-1}[A]\in\Sigma$ for every $A\in\Cal A$.
(iii) For $r\ge 1$, write
$\Cal B_r$ for the
$\sigma$-algebra of Borel subsets of $\BbbR^r$.   Show that if $X$ is
any set and $\Sigma$ is a $\sigma$-algebra of subsets of $X$, then a
function $f:X\to\BbbR^r$ is $(\Sigma,\Cal B_r)$-measurable iff
$\pi_if:X\to\Bbb R$ is $(\Sigma,\Cal B_1)$-measurable for every
$i\le r$, writing $\pi_i(x)=\xi_i$ for $i\le r$,
$x=(\xi_1,\ldots,\xi_r)\in\BbbR^r$.    (iv) Rewrite these ideas
for partially-defined functions.
%121+
     
\spheader 121Yd Let $X$ be a set and $\Sigma$ a $\sigma$-algebra
of subsets of $X$.   For $r\ge 1$, $D\subseteq X$ say that a function
$\phi:D\to\BbbR^r$ is {\bf measurable} if $\phi^{-1}[G]$ is relatively
measurable in $D$ for every open set $G\subseteq\BbbR^r$.   If
$X=\BbbR^s$ and $\Sigma$ is the 
$\sigma$-algebra $\Cal B_s$ of Borel subsets of $\BbbR^s$, 
say that $\phi$ is {\bf Borel measurable}.
(i) Show that $\phi$ is measurable in this sense iff all its
coordinate functions $\phi_i:D\to\Bbb R$ are measurable in the sense of
121C, taking $\phi(x)=(\phi_i(x),\ldots,\phi_r(x))$ for $x\in D$.   (In
particular, this definition agrees with 121C when $r=1$.)
(ii) Show that $\phi:D\to\BbbR^r$ is measurable iff it is
$(\Sigma,\Cal B_r)$-measurable in the sense of 121Yc.
(iii) Show that if $\phi:D\to\BbbR^r$ is measurable and
$\psi:E\to\BbbR^s$ is Borel measurable, where $E\subseteq\BbbR^r$,
then $\psi\phi:\phi^{-1}[E]\to\BbbR^s$ is measurable.
(iv) Show that any continuous function from a subset of
$\BbbR^s$ to $\BbbR^r$ is Borel measurable.
%121Yc
     
\spheader 121Ye Let $X$ be a set and $\theta$ an outer measure
on $X$;  let $\mu$ be the measure defined from $\theta$ by
\Caratheodory's method, and $\Sigma$ its domain.   Suppose that
$f:X\to\Bbb R$ is a function such that
     
\Centerline{$\theta\{x:x\in A,\,f(x)\le a\}
  +\theta\{x:x\in A,\,f(x)\ge b\}\le\theta A$}
     
\noindent whenever $A\subseteq X$ and $a<b$ in $\Bbb R$.   Show that $f$
is $\Sigma$-measurable.   ({\it Hint\/}: suppose that $a\in\Bbb R$ and
$\theta A<\infty$.   Set
     
\Centerline{$B_k=\{x:x\in A,\,
 a+\Bover1{2k+2}\le f(x)\le a+\Bover1{2k+1}\}$,}
     
\Centerline{$B'_k=\{x:x\in A,\,
  a+\Bover1{2k+3}\le f(x)\le a+\Bover1{2k+2}\}$}
     
\noindent for $k\in\Bbb N$.   Show that 
$\sum_{k=0}^{\infty}\theta B_k\le\theta A$, 
and check a similar result for $B'_k$.   Hence show that
     
\Centerline{$\theta\{x:x\in A,\,f(x)>a\}
=\lim_{k\to\infty}\theta\{x:x\in A,\,f(x)\ge a+\Bover1k\}$.)}
%121+
}%end of exercises
     
\endnotes{
\Notesheader{121} I find myself offering no fewer than three definitions
of `measurable function', in 121C, 121Yc and 121Yd.   It is in fact the
last which is probably the most important and the best guide to further
ideas.   Nevertheless, the overwhelming majority of applications refer
to real-valued functions, and the four equivalent conditions of 121B are
the most natural and most convenient to use.   The fact that they all
coincide with the condition of 121Yd corresponds to the fact that they
are all of the form
     
\Centerline{$f^{-1}[E]\in\Sigma_D$ for every $E\in\Cal A$}
     
\noindent where $\Cal A$ is a family of subsets of $\Bbb R$ generating
the Borel $\sigma$-algebra (121Xd, 121Yc(ii)).
     
The class of measurable functions may well be the widest you have yet
seen, not counting the family of all real-valued functions;  all easily
describable functions between subsets of $\Bbb R$ are measurable.   Not
only is the space of measurable functions closed under addition and
multiplication and composition with continuous functions (121E), but any
natural operation acting on a sequence of measurable functions will
produce a measurable function (121F, 121Xb, 121Xa).   It is {\it not}
however the case that the composition of two Lebesgue measurable
functions from $\Bbb R$ to itself is always Lebesgue measurable;  I
offer a counter-example in 134Ib.   The point here is that a function is
called `measurable' if it is $(\Sigma,\Cal B)$-measurable, in the
language of 121Yc, where $\Cal B$ is the 
$\sigma$-algebra of Borel sets.   Such a
function can well fail to be $(\Sigma,\Sigma)$-measurable, if $\Sigma$
properly includes $\Cal B$, so the natural argument for compositions
(121Yc(i)) fails.   Nevertheless, for reasons which I will hint at in
\S134, non-Lebesgue-measurable functions are hard to come by, and only
in the most rarefied kinds of real analysis do they appear in any
natural way.   You may therefore approach the question of whether a
particular function is Lebesgue measurable with reasonable confidence
that it is, and that the proof is merely a challenge to your technique.
     
You will observe that the results of 121E are mostly covered by
121I-121K, which also include large parts of 114G and 115G;  and that
121Kb is covered by 121Yd(iii).   You can count yourself as having
mastered this part of the subject when you find my exposition tediously
repetitive.   Of course, in order to deduce 121Ed from 121K, for
instance, you have to know that multiplication, regarded as a function
from $\BbbR^2$ to $\Bbb R$, is continuous, therefore Borel measurable;
the proof of this is embedded in the proof I give of 121Ed (look at the
formula for $\eta$ half way through).
}%end of notes
     
\discrpage
     
     
