\frfilename{mt364.tex}
\versiondate{16.7.11}
\copyrightdate{1996}

\def\chaptername{Function spaces}
\def\sectionname{$L^0$}

\def\dottimes{\mskip5mu\dot\times\mskip5mu}
\def\frdotplus{\mskip5mu\dot+\mskip5mu}
\def\smalldashint{\vrule height2.6pt width4.3pt depth-2.4pt
  \mskip-8.5mu\smallint}

\newsection{364}

My next objective is to develop an abstract construction corresponding
to the $L^0(\mu)$ spaces of \S241.   These generalized $L^0$ spaces
will form the basis of the work of the rest of this chapter and also the
next;  partly because their own properties are remarkable, but even
more because they form a framework for the study of Archimedean Riesz
spaces in general (see \S368).   There seem to be significant new
difficulties, and I take the space to describe an approach which
can be made essentially independent of the route through Stone spaces
used in the last three sections\cmmnt{ (364Ya)}.   
I embark directly on a definition in
the new language (364A), and relate it to the constructions of \S241
(364B-364D, 364I) and \S\S361-363 (364J).   The ideas of Chapter 27 can
also be expressed in this language;  I make a start on developing the
machinery for this in 364F, with the formula `$\Bvalue{u\in E}$',
`the region in which $u$ belongs to $E$', and some exercises
(364Xd-364Xf).    Following through
the questions addressed in \S363, I discuss Dedekind completeness in
$L^0$ (364L-364M), properties of its multiplication (364N), the
expression of the original algebra in terms of $L^0$ (364O), the action
of Boolean homomorphisms on $L^0$ (364P) and product spaces (364R).   In
364S-364V I describe representations of the $L^0$ space of a regular
open algebra.

\leader{364A}{The set $L^0(\frak A)$ }{\bf (a) Definition}
Let $\frak A$ be a Dedekind
$\sigma$-complete Boolean algebra.   I will write $L^0(\frak A)$ for the
set of all functions $\alpha\mapsto\Bvalue{u>\alpha}:\Bbb R\to\frak A$
such that

\inset{($\alpha$) $\Bvalue{u>\alpha}=\sup_{\beta>\alpha}\Bvalue{u>\beta}$
in $\frak A$ for every $\alpha\in\Bbb R$,

($\beta$) $\inf_{\alpha\in\Bbb R}\Bvalue{u>\alpha}=0$,

($\gamma$) $\sup_{\alpha\in\Bbb R}\Bvalue{u>\alpha}=1$.}

\cmmnt{
\spheader 364Ab\dvAformerly{3{}64B} My reasons for using the notation
`$\Bvalue{u>\alpha}$' rather than `$u(\alpha)$' will I hope become
clear in the next few paragraphs.   For the moment, if you think of
$\frak A$ as a $\sigma$-algebra of sets and of $L^0(\frak A)$ as the
family of $\frak A$-measurable real-valued functions, then
$\Bvalue{u>\alpha}$ corresponds to the set $\{x:u(x)>\alpha\}$ (364Ia).

\spheader 364Ac Some readers will recognise the formula
`$\Bvalue{\ldots}$' as belonging to the language of forcing, so that
$\Bvalue{u>\alpha}$ could be read as `the Boolean value of the
proposition ``$u>\alpha${''}'.   But a vocalisation closer to my
intention might be `the region where $u>\alpha$'.

\spheader 364Ad Note that condition ($\alpha$) of (a) automatically
ensures that $\Bvalue{u>\alpha}\Bsubseteq\Bvalue{u>\alpha'}$ whenever
$\alpha'\le\alpha$ in $\Bbb R$.
}

\spheader 364Ae In fact it will sometimes be convenient to note that the
conditions of (a) can be replaced by

\inset{($\alpha'$) $\Bvalue{u>\alpha}
=\sup_{q\in\Bbb Q,q>\alpha}\Bvalue{u>q}$ for every
$\alpha\in\Bbb R$,

($\beta'$) $\inf_{n\in\Bbb N}\Bvalue{u>n}=0$,

($\gamma'$) $\sup_{n\in\Bbb N}\Bvalue{u>-n}=1$;}

\noindent the point being that we need look only at suprema and infima
of countable subsets of $\frak A$.

%\def\spheader#1#2#3#4#5{\header{#1#2#3#4#5}{\bf (#5)}}
\header{364Af}{\bf *(f)}\dvAnew{2008}
Indeed, \cmmnt{because the function
$\alpha\mapsto\Bvalue{u>\alpha}$ is determined by its values on $\Bbb Q$,}
we have the option of declaring $L^0(\frak A)$ to be the set of functions
$\alpha\mapsto\Bvalue{u>\alpha}:\Bbb Q\to\frak A$ such that

\inset{($\alpha''$) $\Bvalue{u>q}
=\sup_{q'\in\Bbb Q,q'>q}\Bvalue{u>q'}$ for every
$q\in\Bbb Q$,

($\beta'$) $\inf_{n\in\Bbb N}\Bvalue{u>n}=0$,

($\gamma'$) $\sup_{n\in\Bbb N}\Bvalue{u>-n}=1$.}

\cmmnt{\noindent However I shall hold this in reserve until I come to
forcing constructions in Chapter 55 of Volume 5.}

\cmmnt{
\spheader 364Ag In order to integrate this construction into the framework
of the rest of this book, I
match it with an alternative route to the same object, based on
$\sigma$-algebras and $\sigma$-ideals of sets, as follows.
}%end of comment

\leader{364B}{Proposition}\dvAformerly{3{}64C} 
Let $X$ be a set, $\Sigma$ a
$\sigma$-algebra of subsets of $X$, and $\Cal I$ a $\sigma$-ideal of
$\Sigma$.

(a) Write $\eusm L^0=\eusm L^0_{\Sigma}$ for the space of all
$\Sigma$-measurable functions from $X$ to $\Bbb R$.   Then
$\eusm L^0$, with its linear structure, ordering and multiplication
inherited from $\Bbb R^X$, is a Dedekind
$\sigma$-complete $f$-algebra with multiplicative identity.

(b) Set

\Centerline{$\eusm W=\eusm W_{\Cal I}
=\{f:f\in\eusm L^0,\,\{x:f(x)\ne 0\}\in\Cal I\}$.}

\noindent Then

\quad (i) $\eusm W$ is a sequentially order-closed solid linear subspace
and ideal of $\eusm L^0$;

\quad (ii) the quotient space $\eusm L^0/\eusm W$, with its inherited
linear, order and multiplicative structures, is a Dedekind
$\sigma$-complete Riesz space and an $f$-algebra with a multiplicative
identity;

\quad (iii) for $f$, $g\in\eusm L^0$,
$f^{\ssbullet}\le g^{\ssbullet}$ in $\eusm L^0/\eusm W$ iff
$\{x:f(x)>g(x)\}\in\Cal I$,
and $f^{\ssbullet}=g^{\ssbullet}$ in $\eusm L^0/\eusm W$ iff
$\{x:f(x)\ne g(x)\}\in\Cal I$.

\proof{ (Compare 241A-241H.)  %241A 241B 241C 241D 241E 241H

\medskip

{\bf (a)} The point is just that $\eusm L^0$ is a sequentially
order-closed Riesz subspace and subalgebra of $\Bbb R^X$.   The facts we
need to know -- that constant functions belong to $\eusm L^0$, that
$f+g$, $\alpha f$, $f\times g$,
$\sup_{n\in\Bbb N}f_n$ belong to $\eusm L^0$ whenever $f$, $g$, $f_n$ do
and $\{f_n:n\in\Bbb N\}$ is bounded above -- are all covered by
121E-121F.
Its multiplicative identity is of course the constant function $\chi X$.

\medskip

{\bf (b)(i)} The necessary verifications are all elementary.

\medskip

\quad{\bf (ii)} Because $\eusm W$ is a solid linear subspace of the
Riesz space $\eusm L^0$, the quotient inherits a Riesz space structure
(351J, 352Jb);  because $\eusm W$ is an ideal of the ring
$(\eusm L^0,+,\times)$, $\eusm L^0/\eusm W$
inherits a multiplication;  it is a commutative algebra because
$\eusm L^0$ is;  and has a multiplicative identity
$e=\chi X^{\ssbullet}$ because $\chi X$ is the identity of $\eusm L^0$.

To check that $\eusm L^0/\eusm W$ is an $f$-algebra it is enough to
observe that, for any non-negative $f$, $g$ , $h\in\eusm L^0$,

\Centerline{$f^{\ssbullet}\times g^{\ssbullet}
=(f\times g)^{\ssbullet}\ge 0$,}

\noindent and if $f^{\ssbullet}\wedge g^{\ssbullet}=0$ then
$\{x:f(x)>0\}\cap\{x:g(x)>0\}\in\Cal I$, so that
$\{x:f(x)h(x)>0\}\cap\{x:g(x)>0\}\in\Cal I$ and

\Centerline{$(f^{\ssbullet}\times h^{\ssbullet})\wedge g^{\ssbullet}
=(h^{\ssbullet}\times f^{\ssbullet})\wedge g^{\ssbullet}=0$.}

\noindent Finally, $\eusm L^0/\eusm W$ is
Dedekind $\sigma$-complete, by 353J(a-iii).

\medskip

\quad{\bf (iii)}
For $f$, $g\in\eusm L^0$,

\Centerline{$f^{\ssbullet}\le g^{\ssbullet}
\iff (f-g)^+\in\eusm W
\iff\{x:f(x)>g(x)\}=\{x:(f-g)^+(x)\ne 0\}\in\Cal I$}

\noindent (using the fact that the canonical map from $\eusm L^0$ to
$\eusm L^0/\eusm W$ is a Riesz homomorphism, so that
$((f-g)^+)^{\ssbullet}=(f^{\ssbullet}-g^{\ssbullet})^+$).   Similarly

\Centerline{$f^{\ssbullet}=g^{\ssbullet}
\iff f-g\in\eusm W
\iff\{x:f(x)\ne g(x)\}=\{x:(f-g)(x)\ne 0\}\in\Cal I$.}
}%end of proof of 364B

\vleader{108pt}{364C}{\bf Theorem}\dvAformerly{3{}64D}
Let $X$ be a set and $\Sigma$ a
$\sigma$-algebra of subsets of $X$.   Let $\frak A$ be a Dedekind
$\sigma$-complete Boolean algebra and $\pi:\Sigma\to\frak A$ a
surjective Boolean homomorphism, with kernel a $\sigma$-ideal $\Cal I$;
define $\eusm L^0=\eusm L^0_{\Sigma}$ and $\eusm W=\eusm W_{\Cal I}$
as in 364B, so that
$U=\eusm L^0/\eusm W$ is a Dedekind $\sigma$-complete $f$-algebra with multiplicative
identity.

(a) We have a canonical bijection  $T:U\to L^0=L^0(\frak A)$ defined by
the formula

\Centerline{$\Bvalue{Tf^{\ssbullet}>\alpha}=\pi\{x:f(x)>\alpha\}$}

\noindent for every $f\in\eusm L^0$ and $\alpha\in\Bbb R$.

(b)(i) For any $u$, $v\in U$,

\Centerline{$\Bvalue{T(u+v)>\alpha}
=\sup_{q\in\Bbb Q}\Bvalue{Tu>q}\Bcap\Bvalue{Tv>\alpha-q}$}

\noindent for every $\alpha\in\Bbb R$.

\quad(ii) For any $u\in U$ and $\gamma>0$,

\Centerline{$\Bvalue{T(\gamma
u)>\alpha}=\Bvalue{Tu>\bover{\alpha}{\gamma}}$}

\noindent for every $\alpha\in\Bbb R$.

\quad(iii) For any $u$, $v\in U$,

\Centerline{$u\le v\iff\Bvalue{Tu>\alpha}\Bsubseteq\Bvalue{Tv>\alpha}$
for
every $\alpha\in\Bbb R$.}

\quad(iv) For any $u$, $v\in U^+$,

\Centerline{$\Bvalue{T(u\times v)>\alpha}
=\sup_{q\in\Bbb Q,q>0}\Bvalue{Tu>q}\Bcap\Bvalue{Tv>\bover{\alpha}q}$}

\noindent for every $\alpha\ge 0$.

\quad(v) Writing $e=(\chi X)^{\ssbullet}$ for the multiplicative
identity of $U$, we have

\Centerline{$\Bvalue{Te>\alpha}=1$ if $\alpha<1$, $0$ if $\alpha\ge 1$.}

\proof{{\bf (a)(i)} Given $f\in \eusm L^0$, set
$\zeta_f(\alpha)=\pi\{x:f(x)>\alpha\}$ for $\alpha\in\Bbb R$.   Then it
is easy to see that $\zeta_f$ satisfies the conditions
($\alpha$)$'$-($\gamma$)$'$ of 364Ae, because $\pi$ is sequentially
order-continuous (313Qb).    Moreover, if $f^{\ssbullet}=g^{\ssbullet}$
in $U$, then

\Centerline{$\zeta_f(\alpha)\Bsymmdiff \zeta_g(\alpha)
=\pi(\{x:f(x)>\alpha\}\symmdiff\{x:g(x)>\alpha\})
=0$}

\noindent for every $\alpha\in\Bbb R$, because

\Centerline{$\{x:f(x)>\alpha\}\symmdiff\{x:g(x)>\alpha\}
\subseteq\{x:f(x)\ne g(x)\}\in\Cal I$,}

\noindent and $\zeta_f=\zeta_g$.   So we have a well-defined member $Tu$
of $L^0$ defined by the given formula, for any $u\in U$.

\medskip

\quad{\bf (ii)} Next, given $w\in L^0$, there is a $u\in \eusm L^0/\eusm
W$ such that $Tu=w$.   \Prf\
For each $q\in\Bbb Q$, choose $F_q\in\Sigma$ such
that $\pi F_q=\Bvalue{w>q}$ in $\frak A$.    Note that if $q'\ge q$ then

\Centerline{$\pi(F_{q'}\setminus
F_q)=\Bvalue{u>q'}\Bsetminus\Bvalue{u>q}=0$,}


\noindent so $F_{q'}\setminus F_q\in\Cal I$.   Set

\Centerline{$H=\bigcup_{q\in\Bbb Q}F_q\setminus\bigcap_{n\in\Bbb
N}\bigcup_{q\in\Bbb Q,q\ge n}F_q\in\Sigma$,}

\noindent and for $x\in X$ set

$$\eqalign{f(x)&=\sup\{q:q\in\Bbb Q,\,x\in F_q\}\text{ if }x\in H,\cr
&=0\text{ otherwise.}\cr}$$

\noindent ($H$ is chosen just to make the formula here give a finite
value for every $x$.)   We have

$$\eqalignno{\pi H
&=\sup_{q\in\Bbb Q}\Bvalue{w>q}\Bsetminus\inf_{n\in\Bbb N}\sup_{q\in\Bbb
Q,q\ge n}\Bvalue{w>q}\cr
&=1_{\frak A}\Bsetminus\inf_{n\in\Bbb
N}\Bvalue{w>n}
=1_{\frak A}\Bsetminus 0_{\frak A}
=1_{\frak A},\cr}$$

\noindent so $X\setminus H\in\Cal I$.
Now, for any $\alpha\in\Bbb R$,

$$\eqalign{\{x:f(x)>\alpha\}
&=\bigcup_{q\in\Bbb Q,q>\alpha}F_q\cup(X\setminus H)
  \text{ if }\alpha<0,\cr
&=\bigcup_{q\in\Bbb Q,q>\alpha}F_q\setminus(X\setminus H)
\text{ if }\alpha\ge 0,\cr}$$

\noindent and in either case belongs to $\Sigma$;  so that $f\in\eusm
L^0$ and $f^{\ssbullet}$ is defined in $L^0$.
Next, for any $\alpha\in\Bbb R$,

$$\eqalign{\Bvalue{Tf^{\ssbullet}>\alpha}
&=\pi\{x:f(x)>\alpha\}
=\pi(\bigcup_{q\in\Bbb Q,q>\alpha}F_q)\cr
&=\sup_{q\in\Bbb Q,q>\alpha}\Bvalue{w>q}
=\Bvalue{w>\alpha},\cr}$$

\noindent and $Tf^{\ssbullet}=w$.\ \Qed

\medskip

\quad{\bf (iii)} Thus $T$ is surjective.   To see that it is injective,
observe that if $f$, $g\in\eusm L^0$, then

$$\eqalign{Tf^{\ssbullet}=Tg^{\ssbullet}
&\Longrightarrow\Bvalue{Tf^{\ssbullet}>\alpha}
  =\Bvalue{Tg^{\ssbullet}>\alpha}
   \text{ for every }\alpha\in\Bbb R\cr
&\Longrightarrow\pi\{x:f(x)>\alpha\}=\pi\{x:g(x)>\alpha\}
   \text{ for every }\alpha\in\Bbb R\cr
&\Longrightarrow\{x:f(x)>\alpha\}\symmdiff\{x:g(x)>\alpha\}\in\Cal I
   \text{ for every }\alpha\in\Bbb R\cr
&\Longrightarrow\{x:f(x)\ne g(x)\}
   =\bigcup_{q\in\Bbb Q}(\{x:f(x)>q\}\symmdiff\{x:g(x)>q\})\in\Cal I\cr
&\Longrightarrow f^{\ssbullet}=g^{\ssbullet}.\cr}$$

\noindent So we have the claimed bijection.

\medskip

{\bf (b)(i)} Let $f$, $g\in\eusm L^0$ be such that $u=f^{\ssbullet}$ and
$v=g^{\ssbullet}$, so that $u+v=(f+g)^{\ssbullet}$.
For any $\alpha\in\Bbb R$,

$$\eqalignno{\Bvalue{T(u+v)>\alpha}
&=\pi\{x:f(x)+g(x)>\alpha\}\cr
&=\pi(\bigcup_{q\in\Bbb Q}\{x:f(x)>q\}\cap\{x:g(x)>\alpha-q\})\cr
&=\sup_{q\in\Bbb Q}\pi\{x:f(x)>q\}\Bcap\pi\{x:g(x)>\alpha-q\}\cr
\noalign{\noindent (because $\pi$ is a sequentially order-continuous
Boolean homomorphism)}
&=\sup_{q\in\Bbb Q}\Bvalue{Tu>q}\Bcap\Bvalue{Tv>\alpha-q}.\cr}$$

\medskip

\quad{\bf (ii)} Let $f\in\eusm L^0$ be such that $f^{\ssbullet}=u$, so
that $(\gamma f)^{\ssbullet}=\gamma u$.   For any $\alpha\in\Bbb R$,

\Centerline{$\Bvalue{T(\gamma u)>\alpha}
=\pi\{x:\gamma f(x)>\alpha\}
=\pi\{x:f(x)>\Bover{\alpha}{\gamma}\}
=\Bvalue{Tu>\bover{\alpha}{\gamma}}$.}

\medskip

\quad{\bf (iii)} Let $f$, $g\in\eusm L^0$ be such that $f^{\ssbullet}=u$
and $g^{\ssbullet}=v$.   Then

$$\eqalignno{u\le v
&\iff\{x:f(x)>g(x)\}\in\Cal I\cr
\noalign{\noindent (see 364B(b-iii))}
&\iff\bigcup_{q\in\Bbb Q}\{x:f(x)>q\ge g(x)\}\in\Cal I\cr
&\iff \{x:f(x)>\alpha\}\setminus\{x:g(x)>\alpha\}\in\Cal I
   \text{ for every }\alpha\in\Bbb R\cr
&\iff \pi\{x:f(x)>\alpha\}\Bsetminus\pi\{x:g(x)>\alpha\}=0
   \text{ for every }\alpha\cr
&\iff \Bvalue{Tu>\alpha}\Bsubseteq\Bvalue{Tv>\alpha}
   \text{ for every }\alpha.\cr}$$

\medskip

\quad {\bf (iv)} Now suppose that $u$, $v\ge 0$, so that they can be
expressed as $f^{\ssbullet}$, $g^{\ssbullet}$ where $f$, $g\ge 0$ in
$\eusm L^0$ (351J), and $u\times v=(f\times g)^{\ssbullet}$.   If
$\alpha\ge0$, then

$$\eqalign{\Bvalue{T(u\times v)>\alpha}
&=\pi(\bigcup_{q\in\Bbb Q,q>0}
   \{x:f(x)>q\}\cap\{x:g(x)>\Bover{\alpha}q\})\cr
&=\sup_{q\in\Bbb Q,q>0}
   \pi\{x:f(x)>q\}\Bcap\pi\{x:g(x)>\Bover{\alpha}q\}\cr
&=\sup_{q\in\Bbb Q,q>0}
   \Bvalue{Tu>q}\Bcap\Bvalue{Tv>\bover{\alpha}q}.\cr}$$

\medskip

\quad{\bf (v)} This is trivial, because

$$\eqalign{\Bvalue{T(\chi X)^{\ssbullet}>\alpha}
&=\pi\{x:(\chi X)(x)>\alpha\}\cr
&=\pi X=1\text{ if }\alpha<1,\cr
&=\pi\emptyset=0\text{ if }\alpha\ge 1.\cr}$$
}%end of proof of 364C

\leader{364D}{Theorem}\dvAformerly{3{}64E}
Let $\frak A$ be a Dedekind $\sigma$-complete
Boolean algebra.   Then $L^0=L^0(\frak A)$ has the structure of a
Dedekind $\sigma$-complete $f$-algebra with multiplicative identity $e$,
defined by saying

\Centerline{$\Bvalue{u+v>\alpha}
=\sup_{q\in\Bbb Q}\Bvalue{u>q}\Bcap\Bvalue{v>\alpha-q}$,}

\noindent whenever $u$, $v\in L^0$ and $\alpha\in\Bbb R$,

\Centerline{$\Bvalue{\gamma
u>\alpha}=\Bvalue{u>\bover{\alpha}{\gamma}}$}

\noindent whenever $u\in L^0$, $\gamma\in\ooint{0,\infty}$ and
$\alpha\in\Bbb R$,

\Centerline{$u\le v\iff\Bvalue{u>\alpha}\Bsubseteq\Bvalue{v>\alpha}$ for
every $\alpha\in\Bbb R$,}

\Centerline{$\Bvalue{u\times v>\alpha}
=\sup_{q\in\Bbb Q,q>0}\Bvalue{u>q}\Bcap\Bvalue{v>\bover{\alpha}q}$}

\noindent whenever $u$, $v\ge 0$ in $L^0$ and $\alpha\ge 0$,

\Centerline{$\Bvalue{e>\alpha}=1$ if $\alpha<1$, $0$ if $\alpha\ge 1$.}

\proof{{\bf (a)} By the Loomis-Sikorski theorem (314M), we can find a
set $Z$ (the Stone space of $\frak A$), a
$\sigma$-algebra $\Sigma$ of subsets of $Z$ (the algebra generated by
the open-and-closed sets and the ideal $\Cal M$ of meager sets) and a
surjective sequentially order-continuous Boolean homomorphism
$\pi:\Sigma\to\frak A$ (corresponding to the identification between
$\frak A$ and the quotient $\Sigma/\Cal M$).   Consequently, defining
$\eusm L^0=\eusm L^0_{\Sigma}$ and $\eusm W=\eusm W_{\Cal M}$
as in 364B, we have a bijection between the
Dedekind $\sigma$-complete $f$-algebra $\eusm L^0/\eusm W$ and $L^0$
(364Ca).   Of course this endows $L^0$ itself with the structure of a
Dedekind $\sigma$-complete $f$-algebra;  and 364Cb tells us that the
description of the algebraic operations above is consistent with this
structure.

\medskip

{\bf (b)} In fact the $f$-algebra structure is completely defined by
the description offered.   For while scalar multiplication is not
described for $\gamma\le 0$, the assertion that $L^0$ is a Riesz space
implies that $0u=0$ and that $\gamma u=(-\gamma)(-u)$ for $\gamma<0$;
so if we have formulae to describe $u+v$ and $\gamma u$ for $\gamma>0$,
this suffices to define the linear structure of $L^0$.   Note that we
have an element $\underline{0}$ in $L^0$ defined by setting

\Centerline{$\Bvalue{\underline{0}>\alpha}=0$ if $\alpha\ge 0$, $1$ if
$\alpha<0$,}

\noindent and the formula for $u+v$ shows us that

\Centerline{$\Bvalue{\underline{0}+u>\alpha}
=\sup_{q\in\Bbb Q}\Bvalue{\underline{0}>q}\Bcap\Bvalue{u>\alpha-q}
=\sup_{q\in\Bbb Q,q<0}\Bvalue{u>\alpha-q}
=\Bvalue{u>\alpha}$}

\noindent for every $\alpha$, so that $\underline{0}$ is the zero of
$L^0$.   As for multiplication, if $L^0$ is to be an $f$-algebra we
must have

\Centerline{$\Bvalue{u\times v>\alpha}
\Bsupseteq\Bvalue{\underline{0}>\alpha}=1$}

\noindent whenever $u$, $v\in(L^0)^+$ and $\alpha<0$, because
$u\times v\ge\underline{0}$.   So the formula offered is sufficient to
determine $u\times v$ for non-negative $u$ and $v$;  and for others we
know that

\Centerline{$u\times v=(u^+\times v^+)-(u^+\times v^-)
-(u^-\times v^+)+(u^-\times v^-)$,}

\noindent so the whole of the multiplication of $L^0$ is defined.
}%end of proof of 364D

\leader{364E}{}\cmmnt{ The rest of this section will be devoted to
understanding the structure just established.   I start with a pair of
elementary facts.

\medskip

\noindent}{\bf Lemma}\dvAformerly{3{}64F} 
 Let $\frak A$ be a Dedekind $\sigma$-complete
Boolean algebra.

(a) If $u$, $v\in L^0=L^0(\frak A)$ and $\alpha$, $\beta\in\Bbb R$,

\Centerline{$\Bvalue{u+v>\alpha+\beta}
\Bsubseteq\Bvalue{u>\alpha}\Bcup\Bvalue{v>\beta}$.}

(b) If $u$, $v\ge 0$ in $L^0$ and $\alpha$, $\beta\ge 0$ in $\Bbb R$,

\Centerline{$\Bvalue{u\times v>\alpha\beta}
\Bsubseteq\Bvalue{u>\alpha}\Bcup\Bvalue{v>\beta}$.}

\proof{{\bf (a)} For any $q\in\Bbb Q$, either $q\ge\alpha$ and
$\Bvalue{u>q}\Bsubseteq\Bvalue{u>\alpha}$, or $q\le\alpha$ and
$\Bvalue{v>\alpha+\beta-q}$\vthsp$\Bsubseteq\Bvalue{v>\beta}$;  thus in
all cases

\Centerline{$\Bvalue{u>q}\Bcap\Bvalue{v>\alpha+\beta-q}
\Bsubseteq\Bvalue{u>\alpha}\Bcup\Bvalue{v>\beta}$;}

\noindent taking the supremum over $q$, we have the result.

\medskip

{\bf (b)} The same idea works, replacing $\alpha+\beta-q$ by
$\alpha\beta/q$ for $q>0$.
}%end of proof of 364E

\leader{364F}{}\cmmnt{ Yet another description of $L^0$ is sometimes
appropriate, and leads naturally to an important construction (364H).

\medskip

\noindent}{\bf Proposition}\dvAformerly{3{}64G}
Let $\frak A$ be a Dedekind
$\sigma$-complete Boolean algebra.   Then there is a bijection between
$L^0=L^0(\frak A)$ and the set $\Phi$ of sequentially order-continuous
Boolean homomorphisms from the algebra $\Cal B$ of Borel subsets of
$\Bbb R$ to $\frak A$, defined by saying that $u\in L^0$ corresponds to
$\phi\in\Phi$ iff $\Bvalue{u>\alpha}=\phi(\ooint{\alpha,\infty})$ for
every $\alpha\in\Bbb R$.

\proof{{\bf (a)} If $\phi\in\Phi$, then the map
$\alpha\mapsto\phi(\ooint{\alpha,\infty})$ satisfies the conditions of
364Ae, so corresponds to an element $u_{\phi}$ of $L^0$.

\medskip

{\bf (b)} If $\phi$, $\psi\in \Phi$ and $u_{\phi}=u_{\psi}$, then
$\phi=\psi$.   \Prf\ Set $\Cal A=\{E:E\in\Cal B,\,\phi(E)=\psi(E)\}$.
Then $\Cal A$ is a
$\sigma$-subalgebra of $\Cal B$, because $\phi$ and $\psi$ are both
sequentially order-continuous Boolean homomorphisms, and contains
$\ooint{\alpha,\infty}$ for every $\alpha\in\Bbb R$.   Now $\Cal A$
contains $\ocint{-\infty,\alpha}$ for every $\alpha$, and therefore
includes $\Cal B$ (121J).   But this means that $\phi=\psi$.\ \Qed

\medskip

{\bf (c)} Thus $\phi\mapsto u_{\phi}$ is injective.   But it is also
surjective.   \Prf\ As in 364D, take a set $Z$, a $\sigma$-algebra
$\Sigma$ of subsets of $Z$ and a surjective sequentially
order-continuous Boolean homomorphism $\pi:\Sigma\to\frak A$;  let
$T:\eusm L^0_{\Sigma}/\eusm W_{\pi^{-1}[\{0\}]}\to L^0$
be the bijection described in 364C.   If
$u\in L^0$, there is an $f\in\eusm L^0_{\Sigma}$ such that
$Tf^{\ssbullet}=u$.
Now consider $\phi E=\pi f^{-1}[E]$ for $E\in\Cal B$.   $f^{-1}[E]$
always belongs to $\Sigma$ (121Ef), so $\phi E$ is always
well-defined;   $E\mapsto f^{-1}[E]$ and $\pi$ are sequentially
order-continuous, so $\phi$ also is;  and

\Centerline{$\phi(\ooint{\alpha,\infty})=\pi\{z:f(z)>\alpha\}
=\Bvalue{u>\alpha}$}

\noindent for every $\alpha$, so $u=u_{\phi}$.\ \Qed

Thus we have the declared bijection.
}%end of proof of 364F

\leader{364G}{Definition}\dvAformerly{3{}64H} 
In the context of 364F, I will write
$\Bvalue{u\in E}$, `the region where $u$ takes values in $E$', for
$\phi(E)$, where $\phi:\Cal B\to\frak A$ is the homomorphism
corresponding to $u\in L^0$.   Thus
$\Bvalue{u>\alpha}=\Bvalue{u\in\ooint{\alpha,\infty}\,}$.
\cmmnt{In the same spirit} I write $\Bvalue{u\ge\alpha}$ for
$\Bvalue{u\in\coint{\alpha,\infty}\,}
=\inf_{\beta<\alpha}\Bvalue{u>\beta}$,
$\Bvalue{u\ne 0}=\Bvalue{|u|>0}=\Bvalue{u>0}\Bcup\Bvalue{u<0}$
and so on, so that\cmmnt{ (for instance)}
$\Bvalue{u=\alpha}=\Bvalue{u\in\{\alpha\}}
=\Bvalue{u\ge\alpha}\Bsetminus\Bvalue{u>\alpha}$ for $u\in L^0$ and
$\alpha\in\Bbb R$.

\leader{364H}{Proposition}\dvAformerly{3{}64I} 
Let $\frak A$ be a Dedekind
$\sigma$-complete Boolean algebra, $E\subseteq\Bbb R$ a Borel set, and
$h:E\to\Bbb R$ a Borel measurable function.   Then whenever
$u\in L^0=L^0(\frak A)$ is such that $\Bvalue{u\in E}=1$, there is an
element $\bar h(u)$ of $L^0$ defined by saying that
$\Bvalue{\bar h(u)\in F}=\Bvalue{u\in h^{-1}[F]}$ for every Borel set
$F\subseteq\Bbb R$.

\proof{ All we have to observe is that
$F\mapsto \Bvalue{u\in h^{-1}[F]}$ is a sequentially order-continuous
Boolean homomorphism.   (The condition `$\Bvalue{u\in E}=1$' ensures
that $\Bvalue{u\in h^{-1}[\Bbb R]}=1$.)
}%end of proof of 364H

\leader{364I}{Examples}\dvAformerly{3{}64J} \cmmnt{Perhaps I should spell
out the most
important contexts in which we apply these ideas, even though they have
in effect already been mentioned.

\medskip

}{\bf (a)} Let $X$ be a set and $\Sigma$ a $\sigma$-algebra of
subsets of $X$.   Then we may identify $L^0(\Sigma)$ with the space
$\eusm L^0=\eusm L^0_{\Sigma}$ of
$\Sigma$-measurable real-valued functions on $X$.
\cmmnt{(This is the case $\frak A=\Sigma$ of 364C.)}   For
$f\in\eusm L^0$,
$\Bvalue{f\in E}$\cmmnt{ (364G)} is just $f^{-1}[E]$, for any
Borel set $E\subseteq\Bbb R$;  and if $h$ is a Borel measurable
function, $\bar h(f)$\cmmnt{ (364H)} is just the composition $hf$, for
any $f\in\eusm L^0$.

\spheader 364Ib Now suppose that $\Cal I$ is a $\sigma$-ideal of
$\Sigma$ and that $\frak A=\Sigma/\Cal I$.   Then, as in 364C, we
identify $L^0(\frak A)$ with a quotient $\eusm L^0/\eusm W_{\Cal I}$.
For
$f\in\eusm L^0$, $\Bvalue{f^{\ssbullet}\in E}=f^{-1}[E]^{\ssbullet}$,
and $\bar h(f^{\ssbullet})=(hf)^{\ssbullet}$, for any Borel set $E$ and
any Borel measurable function $h:\Bbb R\to\Bbb R$.

\spheader 364Ic In particular, if $(X,\Sigma,\mu)$ is a measure space
with measure algebra $\frak A$, then $L^0(\frak A)$ becomes identified
with $L^0(\mu)$ as defined in \S241.

\cmmnt{The same remarks as in 363I apply here;  the space
$\eusm L^0(\mu)$ of 241A is larger than the space
$\eusm L^0=\eusm L^0_{\Sigma}$ considered here.
But for every $f\in\eusm L^0(\mu)$ there is a
$g\in\eusm L^0_{\Sigma}$ such that $g\eae f$ (241Bk), so that
$L^0(\mu)\mskip2mu=\mskip2mu\eusm L^0(\mu)/\mskip-4mu\eae$
can be identified with $\eusm L^0_{\Sigma}/\eusm N$, where $\eusm N$ is
the set of functions in $\eusm L^0$ which are zero almost
everywhere (241Yc).
}

\leader{364J}{Embedding $S$ and $L^{\infty}$ in $L^0$:
Proposition}\dvAformerly{3{}64K} Let
$\frak A$ be a Dedekind $\sigma$-complete Boolean algebra.

(a) We have a canonical embedding of $L^{\infty}=L^{\infty}(\frak A)$ as
an order-dense solid linear subspace of $L^0=L^0(\frak A)$;  it is the
solid linear subspace generated by the multiplicative identity $e$ of
$L^0$.   Consequently $S=S(\frak A)$ also is embedded as an order-dense
Riesz subspace and subalgebra of $L^0$.

(b) This embedding respects the linear, lattice and multiplicative
structures of $L^{\infty}$ and $S$, and the definition of
$\Bvalue{u>\delta}$, for $u\in S^+$ and $\delta\ge 0$, given in 361Eg.

(c) For $a\in\frak A$, $\chi a$, when regarded as a member of $L^0$, can
be described by the formula

$$\eqalign{\Bvalue{\chi a>\alpha}
&=1\text{ if }\alpha<0,\cr
&=a\text{ if }0\le\alpha<1,\cr
&=0\text{ if }1\le\alpha.\cr}$$

\noindent The function $\chi:\frak A\to L^0$ is additive, injective,
order-continuous and a lattice homomorphism.

(d) For every $u\in (L^0)^+$ there is a non-decreasing sequence
$\sequencen{u_n}$ in $S$ such that $u_0\ge 0$ and
$\sup_{n\in\Bbb N}u_n=u$.

\proof{ Let $Z$, $\Sigma$, $\Cal M$,
$\eusm L^0=\eusm L^0_{\Sigma}$, $\eusm W=\eusm W_{\Cal M}$
and $\pi$ be as in the
proof of 364D.   I defined $L^{\infty}$
to be the space $C(Z)$ of continuous real-valued functions on $Z$
(363A);  but because $\frak A$ is Dedekind
$\sigma$-complete, there is an alternative representation as
$\eusm L^{\infty}/\eusm W\cap\eusm L^{\infty}$, where $\eusm L^{\infty}$
is the space of bounded $\Sigma$-measurable functions from $Z$ to
$\Bbb R$ (363Hb).   Put like this, we clearly have an embedding of
$L^{\infty}\cong\eusm L^{\infty}/\eusm W\cap\eusm L^{\infty}$ in
$L^0\cong\eusm L^0/\eusm W$;  and this embedding represents
$L^{\infty}$ as a Riesz subspace and subalgebra of $L^0$, because
$\eusm L^{\infty}$ is a Riesz subspace and subalgebra of $\eusm L^0$.
$L^{\infty}$ becomes the solid linear subspace of $L^0$ generated by
$(\chi Z)^{\ssbullet}=e$, because $\eusm L^{\infty}$ is the solid linear
subspace of $\eusm L^0$ generated by $\chi Z$.   To see that
$L^{\infty}$ is order-dense in $L^0$, we have only to note that
$f=\sup_{n\in\Bbb N}f\wedge n\chi Z$ in $\eusm L^0$ for every
$f\in\eusm L^0$, and therefore (because the
map $f\mapsto f^{\ssbullet}$ is sequentially order-continuous)
$u=\sup_{n\in\Bbb N}u\wedge ne$ in $L^0$ for every $u\in L^0$.

To identify $\chi a$, we have the formula
$\chi(\pi E)=(\chi E)^{\ssbullet}$, as in 363H(b-iii);  
but this means that, if $a=\pi E$,

$$\eqalign{\Bvalue{\chi a>\alpha}=\pi\{z:\chi E(z)>\alpha\}
&=\pi Z=1\text{ if }\alpha<0,\cr
&=\pi E=a\text{ if }0\le\alpha<1,\cr
&=\pi\emptyset=0\text{ if }\alpha\ge 1,\cr}$$

\noindent using the formula in 364Ca.   Evidently $\chi$ is injective.

Because $S$ is an order-dense Riesz subspace and subalgebra of
$L^{\infty}$ (363C), the same embedding represents it as an
order-dense
Riesz subspace and subalgebra of $L^0$.   (For `order-dense', use
352N(c-iii).)   Concerning the formula $\Bvalue{u>\delta}$, suppose that
$u\in S^+$ and $\delta\ge 0$;  express $u$ as
$\sum_{j=0}^m\beta_j\chi b_j$, where
$b_0,\ldots,b_m\in\frak A$ are disjoint and $\beta_j\ge 0$ for every $j$.
Then we have disjoint sets $F_0,\ldots,F_m\in\Sigma$ such that
$\pi F_j=b_j$ for every $j$, and $u$ is identified with
$(\sum_{j=0}^m\beta_j\chi F_j)^{\ssbullet}$.   Using 364Ca, we have

\Centerline{$\Bvalue{u>\delta}
=\pi\{z:\sum_{j=0}^m\beta_j\chi F_j(z)>\delta\}
=\pi(\bigcup\{F_j:\beta_j>\delta\})
=\sup\{b_j:\beta_j>\delta\}$,}

\noindent matching the expression in the proof of 361Eg.   So the new
interpretation of $\Bvalue{\ldots}$ matches the former definition in the
special case envisaged in 361E.

Because $\chi:\frak A\to L^{\infty}$ is additive, order-continuous and a
lattice homomorphism (363D), and the embedding map
$L^{\infty}\embedsinto L^0$ also is, $\chi:\frak A\to L^0$ has the same
properties.

Finally, if $u\ge 0$ in $L^0$, we can represent it as $f^{\ssbullet}$
where $f\ge 0$ in $\eusm L^0$.   For $n\in\Bbb N$ set

$$\eqalign{f_n(z)
&=2^{-n}k\text{ if }2^{-n}k\le f(z)<2^{-n}(k+1)
   \text{ where }0\le k<4^n,\cr
&=0\text{ if }f(z)\ge 2^n;\cr}$$

\noindent then $\sequencen{f_n^{\ssbullet}}$ is a non-decreasing
sequence in $S^+$ with supremum $u$.
}%end of proof of 364J


\leader{364K}{Corollary} Let $(\frak A,\bar\mu)$ be a measure algebra.
Then $S(\frak A^f)$ can be embedded as a Riesz subspace of
$L^0(\frak A)$, which is order-dense iff $(\frak A,\bar\mu)$ is
semi-finite.

\proof{ (Recall that $\frak A^f$ is the ring $\{a:\bar\mu a<\infty\}$.)
The embedding $\frak A^f\embedsinto\frak A$ is an injective ring
homomorphism, so induces an embedding of $S(\frak A^f)$ as a Riesz
subspace of $S(\frak A)$, by 361J.   Now $S(\frak A^f)$ is order-dense
in $S(\frak A)$ iff $(\frak A,\bar\mu)$ is semi-finite.   \Prf\ (i) If
$(\frak A,\bar\mu)$ is semi-finite and $v>0$ in $S(\frak A)$, then $v$
is expressible as $\sum_{j=0}^n\beta_j\chi b_j$ where $\beta_j\ge 0$ for
each $j$ and some $\beta_j\chi b_j$ is non-zero;  now there is a
non-zero $a\in\frak A^f$ such that $a\Bsubseteq b_j$, so that
$0<\beta_j\chi a\in S(\frak A^f)$ and $\beta_j\chi a\le v$.   As $v$ is
arbitrary, $S(\frak A^f)$ is quasi-order-dense, therefor order-dense
(353A).  (ii) If $S(\frak A^f)$ is order-dense in $S(\frak A)$ and
$b\in\frak A\setminus\{0\}$, there is a $u>0$ in $S(\frak A^f)$ such
that $u\le\chi b$;  now there are $\alpha>0$,
$a\in\frak A^f\setminus\{0\}$ such that
$\alpha\chi a\le u$, in which case $a\Bsubseteq b$.\ \Qed

Now because $S(\frak A^f)\subseteq S(\frak A)$ and $S(\frak A)$ is
order-dense in $L^0(\frak A)$, we must have

$$\eqalign{S(\frak A^f)\text{ is order-dense in }L^0(\frak A)
&\iff S(\frak A^f)\text{ is order-dense in }S(\frak A)\cr
&\iff (\frak A,\bar\mu)\text{ is semi-finite}.\cr}$$
}%end of proof of 364K

\leader{364L}{Suprema and infima in \dvrocolon{$L^0$}}\cmmnt{ We know
that any $L^0(\frak A)$ is a Dedekind $\sigma$-complete partially
ordered set.   There is a useful description of
suprema for this ordering in (a) of the next result.
We do not have such a simple formula for
general infima (though see 364Xm), but facts in (b) are useful.

\medskip

\noindent}{\bf Proposition}\dvAformerly{3{}64M-3{}64N}
Let $\frak A$ be a Dedekind $\sigma$-complete Boolean algebra, and
$L^0=L^0(\frak A)$.

(a) Let $A$ be a subset of $L^0$.

\quad(i) $A$ is bounded above in $L^0$ iff there is a sequence
$\sequencen{c_n}$ in $\frak A$, with infimum $0$, such that
$\Bvalue{u>n}\Bsubseteq c_n$ for every $u\in A$.

\quad(ii) If $A$ is non-empty, then $A$ has a supremum in $L^0$ iff
$c_{\alpha}=\sup_{u\in A}\Bvalue{u>\alpha}$ is defined in $\frak A$ for
every $\alpha\in\Bbb R$ and $\inf_{n\in\Bbb N}c_n=0$;  and in this case
$c_{\alpha}=\Bvalue{\sup A>\alpha}$ for every $\alpha$.

\quad(iii) If $A$ is non-empty and bounded above, then $A$ has a supremum in
$L^0$ iff $\sup_{u\in A}\Bvalue{u>\alpha}$ is defined in $\frak A$ for
every $\alpha\in\Bbb R$.

(b)(i) If $u$, $v\in L^0$, then
$\Bvalue{u\wedge v>\alpha}=\Bvalue{u>\alpha}\Bcap\Bvalue{v>\alpha}$ for
every $\alpha\in\Bbb R$.

\quad(ii) If $A$ is a non-empty subset of $(L^0)^+$,
then $\inf A=0$ in $L^0$ iff $\inf_{u\in A}\Bvalue{u>\alpha}=0$ in
$\frak A$ for every $\alpha>0$.

\proof{{\bf (a)(i)}\grheada\ If $A$ has an upper bound $u_0$, set
$c_n=\Bvalue{u_0>n}$ for each $n$;  then $\sequencen{c_n}$ satisfies the
conditions.

\medskip

\quad\grheadb\ If $\sequencen{c_n}$ satisfies the conditions, set

$$\eqalign{\phi(\alpha)&=1\text{ if }\alpha<0,\cr
&=\inf_{i\le n}c_i\text{ if }n\in\Bbb N,\,\alpha\in\coint{n,n+1}.\cr}$$

\noindent    Then it is easy to check that $\phi$ satisfies the
conditions of 364Aa, since $\inf_{n\in\Bbb N}c_n=0$.   So there is a
$u_0\in L^0$ such that $\phi(\alpha)=\Bvalue{u_0>\alpha}$ for each
$\alpha$.   Now, given $u\in
A$ and $\alpha\in\Bbb R$,

$$\eqalign{\Bvalue{u>\alpha}&\Bsubseteq 1=\Bvalue{u_0>\alpha}
\text{ if }\alpha<0,\cr
&\Bsubseteq\inf_{i\le n}\Bvalue{u>i}\Bsubseteq\inf_{i\le n}c_i
=\Bvalue{u_0>\alpha}\text{ if }n\in\Bbb N,\,
\alpha\in\coint{n,n+1}.\cr}$$

\noindent Thus $u_0$ is an upper bound for $A$ in $L^0$.

\medskip

\quad{\bf (ii)}\grheada\ Suppose that
$c_{\alpha}=\sup_{u\in A}\Bvalue{u>\alpha}$ is
defined in $\frak A$ for every $\alpha$, and
that $\inf_{n\in\Bbb N}c_n=0$.   Then, for any $\alpha$,

\Centerline{$\sup_{q\in\Bbb Q,q>\alpha}c_q
=\sup_{u\in A,q\in\Bbb Q,q>\alpha}\Bvalue{u>q}
=\sup_{u\in A}\Bvalue{u>\alpha}
=c_{\alpha}$.}

\noindent Also, we are supposing that $A$ contains some $u_0$, so that

\Centerline{$\sup_{n\in\Bbb N}c_{-n}
\Bsupseteq\sup_{n\in\Bbb N}\Bvalue{u_0>-n}
=1$.}

\noindent Accordingly there is a $u^*\in L^0$ such that
$\Bvalue{u^*>\alpha}=c_{\alpha}$ for every $\alpha\in\Bbb R$.
But now, for $v\in L^0$,

$$\eqalign{v\text{ is an upper bound for }A
&\iff\Bvalue{u>\alpha}\Bsubseteq\Bvalue{v>\alpha}
\text{ for every }u\in A,\,\alpha\in\Bbb R\cr
&\iff\Bvalue{u^*>\alpha}\Bsubseteq\Bvalue{v>\alpha}
\text{ for every }\alpha\cr
&\iff u^*\le v,\cr}$$

\noindent so that $u^*=\sup A$ in $L^0$.

\medskip

\qquad\grheadb Now suppose that $u^*=\sup A$ is defined in $L^0$.   Of
course $\Bvalue{u^*>\alpha}$ must be an upper bound for
$\{\Bvalue{u>\alpha}:u\in A\}$ for every $\alpha$.   \Quer\ Suppose we
have an $\alpha$ for which it is not the least upper bound, that is,
there is a $c\Bsubset\Bvalue{u^*>\alpha}$ which is an upper bound for
$\{\Bvalue{u>\alpha}:u\in A\}$.   Define $\phi:\Bbb R\to\frak A$ by
setting

$$\eqalign{\phi(\beta)
&=c\Bcap\Bvalue{u^*>\beta}\text{ if }\beta\ge\alpha,\cr
&=\Bvalue{u^*>\beta}\text{ if }\beta<\alpha.\cr}$$

\noindent It is easy to see that $\phi$ satisfies the conditions of 364Aa
(we need the distributive law 313Ba to check that
$\phi(\beta)=\sup_{\gamma>\beta}\phi(\gamma)$ if $\beta\ge\alpha$), so
corresponds to a member $v$ of $L^0$.    But we now find that $v$ is an
upper bound for $A$ (because if $u\in A$ and $\beta\ge\alpha$ then

\Centerline{$\Bvalue{u>\beta}
\Bsubseteq\Bvalue{u>\alpha}\Bcap\Bvalue{u^*>\beta}
\Bsubseteq c\Bcap\Bvalue{u^*>\beta}=\Bvalue{v>\beta}$,)}

\noindent that $v\le u^*$ and that $v\ne u^*$ (because
$\Bvalue{v>\alpha}=c\ne\Bvalue{u^*>\alpha}$);  but this is impossible,
because $u^*$ is supposed to be the supremum of $A$.\ \BanG\  Thus if
$u^*=\sup A$ is defined in $L^0$, then
$\sup_{u\in A}\Bvalue{u>\alpha}=\Bvalue{u^*>\alpha}$
is defined in $\frak A$ for every $\alpha\in\Bbb R$.   Also, of course,

\Centerline{$\inf_{n\in\Bbb N}\sup_{u\in A}\Bvalue{u>n}
=\inf_{n\in\Bbb N}\Bvalue{u^*>n}=0$.}

\medskip

\quad{\bf (iii)} This is now easy.   If $A$ has a supremum, then surely it
satisfies the condition, by (b).   If $A$ satisfies the condition, then
we have a family $\langle c_{\alpha}\rangle_{\alpha\in\Bbb R}$ as
required in (b);  but also, by (a) or otherwise, there is a sequence
$\sequencen{c'_n}$ such that $c_n\Bsubseteq c'_n$ for every $n$ and
$\inf_{n\in\Bbb N}c'_n=0$, so $\inf_{n\in\Bbb N}c_n$ also is $0$, and
both conditions in (b) are satisfied, so $A$ has a supremum.

\medskip

{\bf (b)(i)} Take $Z$, $\eusm L^0$ and $\pi$ as in the proof of
364D.   Express $u$ as $f^{\ssbullet}$, $v$ as $g^{\ssbullet}$ where
$f$, $g\in\eusm L^0$, so that $u\wedge v=(f\wedge g)^{\ssbullet}$,
because the canonical map from $\eusm L^0$ to $L^0$ is a Riesz
homomorphism (351J).   Then

$$\eqalign{\Bvalue{u\wedge v>\alpha}
&=\pi\{z:\min(f(z),g(z))>\alpha\}
=\pi(\{z:f(z)>\alpha\}\cap\{z:g(z)>\alpha\})\cr
&=\pi\{z:f(z)>\alpha\}\Bcap\pi\{z:g(z)>\alpha\}
=\Bvalue{u>\alpha}\Bcap\Bvalue{v>\alpha}\cr}$$

\noindent for every $\alpha$.

\medskip

\quad{\bf (ii)}\grheada\
If $\inf_{u\in A}\Bvalue{u>\alpha}=0$ for every $\alpha>0$,
and $v$ is any lower bound
for $A$, then $\Bvalue{v>\alpha}$ must be $0$ for every $\alpha>0$, so
that $\Bvalue{v>0}=0$;  now since $\Bvalue{0>\alpha}=0$ for
$\alpha\ge 0$, $1$ for $\alpha<0$, $v\le 0$.   As $v$ is arbitrary,
$\inf A=0$.

\medskip

\qquad\grheadb If $\alpha>0$ is such that $\inf_{u\in A}\Bvalue{u>\alpha}$
is undefined, or not equal to $0$, let $c\in\frak A$ be such that
$0\ne c\Bsubseteq\Bvalue{u>\alpha}$ for every $u\in A$, and consider
$v=\alpha\chi c$.   Then $\Bvalue{v>\beta}=\Bvalue{\chi
c>\bover{\beta}{\alpha}}$ is $1$ if $\beta<0$, $c$ if $0\le\beta<\alpha$
and $0$ if $\beta\ge\alpha$.   If $u\in A$ then $\Bvalue{u>\beta}$ is
$1$ if $\beta<0$ (since $u\ge 0$), at least $\Bvalue{u>\alpha}\Bsupseteq
c$ if $0\le\beta<\alpha$, and always includes $0$;   so that
$v\le u$.   As $u$ is arbitrary, $\inf A$ is either undefined in $L^0$
or not $0$.
}%end of proof of 364L

\leader{364M}{}\cmmnt{ Now we have a reward for our labour, in that
the following basic theorem is easy.

\medskip

\noindent}{\bf Theorem}\dvAformerly{3{}64O}
For a Dedekind $\sigma$-complete Boolean algebra
$\frak A$, $L^0=L^0(\frak A)$ is Dedekind complete iff $\frak A$ is.

\proof{ The description of suprema in 364L(a-iii) makes it obvious that if
$\frak A$ is Dedekind complete, so that $\sup_{u\in A}\Bvalue{u>\alpha}$
is always defined, then $L^0$ must be Dedekind complete.
On the other hand, if $L^0$ is Dedekind complete, then so is
$L^{\infty}(\frak A)$ (by 364J and 353J(b-i)), so that $\frak A$ also is
Dedekind complete, by 363Mb.
}%end of proof of 364M

\leader{364N}{The multiplication of \dvrocolon{$L^0$}}\cmmnt{ I have
already observed that $L^0$ is always an $f$-algebra with identity;  in
particular
(because $L^0$ is surely Archimedean) the map $u\mapsto u\times v$ is
order-continuous for every $v\ge 0$ (353Oa), and multiplication is
commutative (353Ob, or otherwise).   The multiplicative identity is
$\chi 1$ (364D, 364Jc).  By 353Pb, or otherwise, $u\times v=0$ iff
$|u|\wedge|v|=0$.   There is one special feature of multiplication in
$L^0$ which I can mention here.

\medskip

\noindent}{\bf Proposition}\dvAformerly{3{}64P} Let $\frak A$
be a Dedekind $\sigma$-complete Boolean algebra.
Then an element $u$ of $L^0=L^0(\frak A)$ has a multiplicative inverse in 
$L^0$ iff $|u|$ is a weak order unit in $L^0$ iff $\Bvalue{|u|>0}=1$.

\proof{ If $u$ is invertible, then $|u|$ is a weak order unit, by
353Pc or otherwise.   In this case, setting
$c=1\Bsetminus\Bvalue{|u|>0}$, we see that

\Centerline{$\Bvalue{|u|\wedge\chi c>0}=\Bvalue{|u|>0}\Bcap c=0$}

\noindent (364L(b-i)), so that $|u|\wedge\chi c\le 0$ and $\chi c=0$, that
is, $c=0$;  so $\Bvalue{|u|>0}$ must be $1$.   To complete the circuit,
suppose that $\Bvalue{|u|>0}=1$.   Let $Z$, $\Sigma$,
$\eusm L^0=\eusm L^0_{\Sigma}$,
$\pi$, $\Cal M$ be as in the proof of
364D, and $S:\eusm L^0\to L^0$ the canonical map, so that
$\Bvalue{Sh>\alpha}=\pi\{z:h(z)>\alpha\}$ for every $h\in\eusm L^0$,
$\alpha\in\Bbb R$.   Express $u$ as $Sf$ where $f\in\eusm L^0$.   Then
$\pi\{z:|f(z)|>0\}=\Bvalue{S|f|>0}=1$, so $\{z:f(z)=0\}\in\Cal M$.   Set

\Centerline{$g(z)=\Bover1{f(z)}$ if $f(z)\ne 0$,
\quad$g(z)=0$ if $f(z)=0$.}

\noindent Then $\{z:f(z)g(z)\ne 1\}\in\Cal M$ so

\Centerline{$u\times Sg=S(f\times g)=S(\chi Z)=\chi 1$}

\noindent and $u$ is invertible.
}%end of proof of 364N

\cmmnt{\medskip

\noindent{\bf Remark} The repeated phrase `by 353x or otherwise'
reflects the fact that the abstract methods there can all be replaced in
this case by simple direct arguments based on the construction in
364B-364D.
}%end of comment

\leader{364O}{Recovering the algebra:  Proposition} Let
$\frak A$ be a Dedekind $\sigma$-complete Boolean algebra.   For
$a\in\frak A$
write $V_a$ for the band in $L^0=L^0(\frak A)$ generated by $\chi a$.
Then $a\mapsto V_a$ is a Boolean isomorphism between $\frak A$ and the
algebra of projection bands in $L^0$.

\proof{ I copy from the argument for 363J, itself based on 361K.   If
$a\in\frak A$ and $w\in L^0$ then $w\times\chi a\in V_a$.
\Prf\ If $v\in V_a^{\perp}$ then $|\chi a|\wedge|v|=0$, so
$\chi a\times v=0$, so $(w\times\chi a)\times v=0$, so
$|w\times\chi a|\wedge|v|=0$;
thus $w\times\chi a\in V_a^{\perp\perp}$, which is equal to $V_a$
because $L^0$ is Archimedean (353Ba).\ \QeD\  Now, if
$a\in\frak A$, $u\in V_a$ and $v\in V_{1\Bsetminus a}$, then
$|u|\wedge|v|=0$
because $\chi a\wedge\chi(1\Bsetminus a)=0$;  and if $w\in
L^{0}(\frak A)$ then

\Centerline{$w=(w\times\chi a)+(w\times\chi(1\Bsetminus a))\in
V_a+V_{1\Bsetminus a}$.}

\noindent So
$V_a$ and $V_{1\Bsetminus a}$ are complementary projection bands in
$L^0$.   Next, if $U\subseteq L^0$ is
a projection band, then $\chi 1$ is expressible as $u+v=u\vee v$ where
$u\in U$, $v\in U^{\perp}$.   Setting $a=\Bvalue{u>\bover12}$,
$a'=\Bvalue{v>\bover12}$ we must have  $a\Bcup a'=1$ and $a\Bcap a'=0$
(using 364L), so that $a'=1\Bsetminus a$;  also
$\bover12\chi a\le u$, so that $\chi a\in U$, and
similarly $\chi(1\Bsetminus a)\in U^{\perp}$.
In this case $V_a\subseteq U$ and
$V_{1\Bsetminus a}\subseteq U^{\perp}$, so $U$ must be $V_a$ precisely.
Thus $a\mapsto V_a$ is surjective.   Finally, just as in 361K,
$a\Bsubseteq b\iff V_a\subseteq V_b$, so we have a Boolean isomorphism.
}%end of proof of 364O

\leader{364P}{}\cmmnt{ I come at last to the result corresponding to
361J and 363F.

\medskip

\noindent}{\bf Theorem}\dvAformerly{3{}64R}
Let $\frak A$ and $\frak B$ be Dedekind $\sigma$-complete Boolean algebras,
and $\pi:\frak A\to\frak B$ a sequentially order-continuous Boolean
homomorphism.

(a) We have a multiplicative sequentially order-continuous Riesz
homomorphism $T_{\pi}:L^0(\frak A)\to L^0(\frak B)$ defined by the
formula

\Centerline{$\Bvalue{T_{\pi}u>\alpha}=\pi\Bvalue{u>\alpha}$}

\noindent whenever $\alpha\in\Bbb R$ and $u\in L^0(\frak A)$.

(b) Defining $\chi a\in L^0(\frak A)$ as in 364J,
$T_{\pi}(\chi a)=\chi(\pi a)$ in $L^0(\frak B)$ for every $a\in\frak A$.
If we regard $L^{\infty}(\frak A)$ and $L^{\infty}(\frak B)$ as embedded
in $L^0(\frak A)$ and $L^0(\frak B)$ respectively, then $T_{\pi}$, as
defined here, agrees on $L^{\infty}(\frak A)$ with $T_{\pi}$ as defined
in 363F.

(c) $T_{\pi}$ is order-continuous iff $\pi$ is order-continuous,
injective iff $\pi$ is injective, surjective iff $\pi$ is surjective.

(d) $\Bvalue{T_{\pi}u\in E}=\pi\Bvalue{u\in E}$ for every $u\in
L^0(\frak A)$ and every Borel set $E\Bsubseteq\Bbb R$;  consequently
$\bar hT_{\pi}=T_{\pi}\bar h$ for every Borel measurable
$h:\Bbb R\to\Bbb R$, writing $\bar h$ indifferently for the associated
maps from $L^0(\frak A)$ to itself and from $L^0(\frak B)$ to
itself\cmmnt{ (364H)}.

(e) If $\frak C$ is another Dedekind $\sigma$-complete Boolean algebra
and $\theta:\frak B\to\frak C$ another sequentially order-continuous
Boolean homomorphism then $T_{\theta\pi}=T_{\theta}T_{\pi}:L^0(\frak
A)\to L^0(\frak C)$.

\proof{ I write $T$ for $T_{\pi}$.

\medskip

{\bf (a)(i)} To see that $Tu$ is well-defined in $L^0(\frak B)$ for
every $u\in L^0(\frak A)$, all we need to do is to check that the map
$\alpha\mapsto\pi\Bvalue{u>\alpha}:\Bbb R\to\frak B$ satisfies the
conditions of 364Ae, and this is easy, because $\pi$ preserves all
countable suprema and infima.

\medskip

\quad{\bf (ii)} To see that $T$ is linear and order-preserving and
multiplicative, we can use the formulae of 364D.  For instance, if $u$,
$v\in L^0(\frak A)$, then

$$\eqalign{\Bvalue{Tu+Tv>\alpha}
&=\sup_{q\in\Bbb Q}\Bvalue{Tu>q}\Bcap\Bvalue{Tv>\alpha-q}
=\sup_{q\in\Bbb Q}\pi\Bvalue{u>q}\Bcap\pi\Bvalue{v>\alpha-q}\cr
&=\pi(\sup_{q\in\Bbb Q}\Bvalue{u>q}\Bcap\Bvalue{v>\alpha-q})
=\pi\Bvalue{u+v>\alpha}
=\Bvalue{T(u+v)>\alpha}\cr}$$

\noindent for every $\alpha\in\Bbb R$, so that $Tu+Tv=T(u+v)$.   In the
same way,

\Centerline{$T(\gamma u)=\gamma Tu$ whenever $\gamma>0$,}

\Centerline{$Tu\le Tv$ whenever $u\le v$,}

\Centerline{$Tu\times Tv=T(u\times v)$ whenever $u$, $v\ge 0$,}

\noindent so that, using the distributive laws, $T$ is linear and
multiplicative.

To see that $T$ is a sequentially order-continuous Riesz homomorphism,
suppose that $A\subseteq L^0(\frak A)$ is a countable non-empty set with
a supremum $u^*\in L^0(\frak A)$;  then
$T[A]$ is a non-empty subset of $L^0(\frak B)$ with an upper bound
$Tu^*$, and

$$\eqalignno{\sup_{u\in A}\Bvalue{Tu>\alpha}
&=\sup_{u\in A}\pi\Bvalue{u>\alpha}
=\pi(\sup_{u\in A}\Bvalue{u>\alpha})
=\pi\Bvalue{u^*>\alpha}\cr
\noalign{\noindent (using 364La)}
&=\Bvalue{Tu^*>\alpha}\cr}$$

\noindent for every $\alpha\in\Bbb R$.   So using 364La again,
$Tu^*=\sup_{u\in A}Tu$.   Now this is true, in particular, for doubleton
sets $A$, so that $T$ is a Riesz homomorphism;  and also for
non-decreasing sequences, so that $T$ is sequentially
order-continuous.

\medskip

{\bf (b)} The identification of $T(\chi a)$ with $\chi(\pi a)$ is
another almost trivial verification.   It follows that $T$ agrees with
the map of 363F on $S(\frak A)$, if we think of $S(\frak A)$ as a
subspace of $L^0(\frak A)$.   Next, if
$u\in L^{\infty}(\frak A)\subseteq L^0(\frak A)$, and
$\gamma=\|u\|_{\infty}$, then $|u|\le \gamma\chi 1_{\frak A}$, so that
$|Tu|\le \gamma\chi 1_{\frak B}$, and
$Tu\in L^{\infty}(\frak B)$, with $\|Tu\|_{\infty}\le\gamma$.   Thus
$T\restr L^{\infty}(\frak A)$ has norm at most $1$.   As it agrees with
the map of 363F on $S(\frak A)$, which is $\|\,\|_{\infty}$-dense in
$L^{\infty}(\frak A)$ (363C), and both are continuous, they must agree
on the whole of $L^{\infty}(\frak A)$.

\medskip

{\bf (c)(i)}\grheada\ Suppose that $\pi$ is order-continuous, and that
$A\subseteq L^0(\frak A)$ is a non-empty set with a supremum
$u^*\in L^0(\frak A)$.   Then for any $\alpha\in\Bbb R$,

$$\eqalignno{\Bvalue{Tu^*>\alpha}
&=\pi\Bvalue{u^*>\alpha}
=\pi(\sup_{u\in A}\Bvalue{u>\alpha})\cr
\noalign{\noindent (by 364La)}
&=\sup_{u\in A}\pi\Bvalue{u>\alpha}\cr
\noalign{\noindent (because $\pi$ is order-continuous)}
&=\sup_{u\in A}\Bvalue{Tu>\alpha}.\cr}$$

\noindent As $\alpha$ is arbitrary, $Tu^*=\sup T[A]$, by 364La again.
As $A$ is arbitrary, $T$ is order-continuous (351Ga).

\medskip

\qquad\grheadb\ Now suppose that $T$ is order-continuous and that
$A\subseteq\frak A$ is a non-empty set with supremum $c$ in $\frak A$.
Then $\chi c=\sup_{a\in A}\chi a$ (364Jc) so

\Centerline{$\chi(\pi c)=T(\chi c)=\sup_{a\in A}T(\chi a)
=\sup_{a\in A}\chi(\pi a)$.}

\noindent But now

\Centerline{$\pi c
=\Bvalue{\chi(\pi c)>0}
=\sup_{a\in A}\Bvalue{\chi(\pi a)>0}
=\sup_{a\in A}\pi a$.}

\noindent As $A$ is arbitrary, $\pi$ is order-continuous.

\medskip

\quad{\bf (ii)}\grheada\ If $\pi$ is injective and $u$, $v$ are distinct
elements of $L^0(\frak A)$, then there must be some $\alpha$ such that
$\Bvalue{u>\alpha}\ne\Bvalue{v>\alpha}$, in which case
$\Bvalue{Tu>\alpha}\ne\Bvalue{Tv>\alpha}$ and $Tu\ne Tv$.

\medskip

\qquad\grheadb\ Now suppose that $T$ is injective.   It is easy to see
that $\chi:\frak A\to L^0(\frak A)$ is injective, so that $T\chi:\frak
A\to L^0(\frak B)$ is injective;  but this is the same as $\chi\pi$ (by
(b)), so $\pi$ must also be injective.

\medskip

\quad{\bf (iii)}\grheada\ Suppose that $\pi$ is surjective.   Let
$\Sigma$ be a $\sigma$-algebra of sets such that there is a sequentially
order-continuous Boolean surjection $\phi:\Sigma\to\frak A$.   Then
$\pi\phi:\Sigma\to\frak B$ is surjective.   So given
$w\in L^0(\frak B)$, there is an $f\in\eusm L^0_{\Sigma}$ such that
$\Bvalue{w>\alpha}=\pi\phi\{x:f(x)>\alpha\}$ for every $\alpha\in\Bbb R$
(364C).   But, also by 364C, there is a $u\in L^0(\frak A)$ such that
$\Bvalue{u>\alpha}=\phi\{x:f(x)>\alpha\}$ for every $\alpha$.   And now
of course $Tu=w$.   As $w$ is arbitrary, $T$ is surjective.

\medskip

\qquad\grheadb\ If $T$ is surjective, and $b\in\frak B$, there must be
some $u\in L^0(\frak A)$ such that $Tu=\chi b$.   Now set
$a=\Bvalue{u>0}$ and see that $\pi a=\Bvalue{\chi b>0}=b$.   As $b$ is
arbitrary, $\pi$ is surjective.

\medskip

{\bf (d)} The map $E\mapsto\pi\Bvalue{u\in E}$ is a sequentially
order-continuous Boolean homomorphism, equal to $\Bvalue{Tu\in E}$ when
$E$ is of the form $\ooint{\alpha,\infty}$;  so by 364F the two are
equal for all Borel sets $E$.

If $h:\Bbb R\to\Bbb R$ is a Borel measurable function, $u\in L^0(\frak
A)$ and $E\subseteq\Bbb R$ is a Borel set, then

$$\eqalign{\Bvalue{\bar h(Tu)\in E}
&=\Bvalue{Tu\in h^{-1}[E]}
=\pi\Bvalue{u\in h^{-1}[E]}\cr
&=\pi\Bvalue{\bar h(u)\in E}
=\Bvalue{T(\bar h(u))\in E}.\cr}$$

\noindent As $E$ and $u$ are arbitrary, $T\bar h=\bar hT$.

\medskip

{\bf (e)} This is immediate from (a).
}%end of proof of 364P

\leader{364Q}{Proposition}\dvAformerly{3{}64Xr, 3{}72H}
Let $X$ and $Y$ be sets, $\Sigma$, $\Tau$
$\sigma$-algebras of subsets of $X$, $Y$ respectively, and $\Cal I$,
$\Cal J\,\,\sigma$-ideals of $\Sigma$, $\Tau$.   Set
$\frak A=\Sigma/\Cal I$ and $\frak B=\Tau/\Cal J$.   Suppose that
$\phi:X\to Y$ is
a function such that $\phi^{-1}[F]\in\Sigma$ for every $F\in\Tau$ and
$\phi^{-1}[F]\in\Cal I$ for every $F\in\Cal J$.

(a) There is a
sequentially order-continuous Boolean homomorphism
$\pi:\frak B\to\frak A$ defined by saying that
$\pi F^{\ssbullet}=\phi^{-1}[F]^{\ssbullet}$
for every $F\in\Tau$.

(b) Let $T_{\pi}:L^0(\frak B)\to L^0(\frak A)$ be the
Riesz homomorphism corresponding to $\pi$, as defined in 364P.
If we identify
$L^0(\frak B)$ with $\eusm L^0_{\Tau}/\eusm W_{\Cal J}$ and
$L^0(\frak A)$ with $\eusm L^0_{\Sigma}/\eusm W_{\Cal I}$ in the manner
of 364B-364C, then $T_{\pi}(g^{\ssbullet})=(g\phi)^{\ssbullet}$ for every
$g\in\eusm L^0_{\Tau}$.

(c) Let $Z$ be a third set, $\Upsilon$ a $\sigma$-algebra of subsets of
$Z$, $\Cal K$ a $\sigma$-ideal of $\Upsilon$, and $\psi:Y\to Z$ a function
such that $\psi^{-1}[G]\in\Tau$ for every $G\in\Upsilon$ and
$\psi^{-1}[G]\in\Cal J$ for every $G\in\Cal K$.   Let
$\theta:\frak C\to\frak B$ and $T_{\theta}:L^0(\frak C)\to L^0(\frak B)$ be
the homomorphisms corresponding to $\psi$ as in (a)-(b).   Then
$\pi\theta:\frak C\to\frak A$ and
$T_{\pi}T_{\theta}:L^0(\frak C)\to L^0(\frak A)$ correspond to
$\psi\phi:X\to Y$ in the same way.

(d) Now suppose that $\mu$ and $\nu$ are measures with domains $\Sigma$,
$\Tau$ and null ideals $\Cal N(\mu)$, $\Cal N(\nu)$ respectively, and that
$\Cal I=\Sigma\cap\Cal N(\mu)$ and $\Cal J=\Tau\cap\Cal N(\nu)$.   In this
case, identifying $L^0(\frak A)$, $L^0(\frak B)$ with $L^0(\mu)$ and
$L^0(\nu)$ as in 364Ic, we have $g\phi\in\eusm L^0(\mu)$ and
$T_{\pi}(g^{\ssbullet})=(g\phi)^{\ssbullet}$ for every
$g\in\eusm L^0(\nu)$.

\proof{{\bf (a)} The argument is essentially that of 324A-324B, somewhat
simplified.   Explicitly:  if $F_1$, $F_2\in\Tau$ and
$F_1^{\ssbullet}=F_2^{\ssbullet}$, then $F_1\symmdiff F_2\in\Cal J$ so
$\phi^{-1}[F_1]\symmdiff\phi^{-1}[F_2]=\phi^{-1}[F_1\symmdiff F_2]$ belongs
to $\Cal I$ and $\phi^{-1}[F_1]^{\ssbullet}=\phi^{-1}[F_2]^{\ssbullet}$.
So the formula offered defines a map $\pi:\frak B\to\frak A$.
It is a Boolean homomorphism, because if $F_1$, $F_2\in\Tau$ then

$$\eqalign{\pi F_1^{\ssbullet}\Bsymmdiff\pi F_2^{\ssbullet}
&=\phi^{-1}[F_1]^{\ssbullet}\Bsymmdiff\phi^{-1}[F_2]^{\ssbullet}
=(\phi^{-1}[F_1]\symmdiff\phi^{-1}[F_2])^{\ssbullet}\cr
&=\phi^{-1}[F_1\symmdiff F_2]^{\ssbullet}
=\pi(F_1\symmdiff F_2)^{\ssbullet}
=\pi(F_1^{\ssbullet}\Bsymmdiff F_2^{\ssbullet}),\cr}$$

\noindent so $\pi(b_1\Bsymmdiff b_2)=\pi b_1\Bsymmdiff b_2$ for all $b_1$,
$b_2\in\frak B$.
Similarly $\pi(b_1\Bcap b_2)=\pi b_1\Bcap b_2$ for all $b_1$,
$b_2\in\frak B$, and of course

\Centerline{$\pi 1_{\frak B}=\pi Y^{\ssbullet}
=\phi^{-1}[Y]^{\ssbullet}=X^{\ssbullet}=1_{\frak A}$.}

To see that $\pi$ is sequentially order-continuous, let
$\sequencen{b_n}$ be a sequence in $\frak B$.  For each $n$ we may
choose an $F_n\in\Tau$ such that $F_n^{\ssbullet}=b_n$, and set
$F=\bigcup_{n\in\Bbb N}F_n$.   As the map
$H\mapsto H^{\ssbullet}:\Tau\to\frak B$ is sequentially order-continuous
(313Qb), $F^{\ssbullet}=\sup_{n\in\Bbb N}b_n$ in $\frak B$.   Now

$$\eqalign{\pi(\sup_{n\in\Bbb N}b_n)
&=\pi F^{\ssbullet}
=\phi^{-1}[F]^{\ssbullet}
=(\bigcup_{n\in\Bbb N}\phi^{-1}[F_n])^{\ssbullet}\cr
&=\sup_{n\in\Bbb N}\phi^{-1}[F_n]^{\ssbullet}
=\sup_{n\in\Bbb N}\pi F_n^{\ssbullet}
=\sup_{n\in\Bbb N}\pi b_n.\cr}$$

\noindent So $\pi$ is sequentially order-continuous, by 313Lc.

\medskip

{\bf (b)} Now suppose that $g:Y\to\Bbb R$ is $\Tau$-measurable;  write
$v$ for $g^{\ssbullet}$ in
$\eusm L^0_{\Tau}/\eusm W_{\Cal J}\cong L^0(\frak B)$.
Set $f=g\phi$;  then

\Centerline{$\{x:f(x)>\alpha\}=\phi^{-1}[\{y:g(y)>\alpha\}]$}

\noindent belongs to $\Sigma$ for every $\alpha\in\Bbb R$, so
$f$ is $\Sigma$-measurable and we can speak of
$u=f^{\ssbullet}$ in
$\eusm L^0_{\Sigma}/\eusm W_{\Cal I}\cong L^0(\frak A)$.   Now, by 364Ca,

$$\eqalign{\Bvalue{u>\alpha}
&=\{x:f(x)>\alpha\}^{\ssbullet}
=\phi^{-1}[\{y:g(y)>\alpha\}]^{\ssbullet}\cr
&=\pi\{y:g(y)>\alpha\}^{\ssbullet}
=\pi\Bvalue{v>\alpha}
=\Bvalue{T_{\pi}v>\alpha}\cr}$$

\noindent for every $\alpha\in\Bbb R$, and

\Centerline{$(g\phi)^{\ssbullet}=f^{\ssbullet}=u=T_{\pi}v=T_{\pi}g^{\ssbullet}$,}

\noindent as claimed.

\medskip

{\bf (c)} Starting from the
facts that $(\psi\phi)^{-1}[G]=\phi^{-1}[\psi^{-1}[G]]$ for every
$G\in\Upsilon$ and $h(\psi\phi)=(h\psi)\phi$ for every
$h\in\eusm L^0_{\Upsilon}$, we just have to run through the formulae.

\medskip

{\bf (d)} If $g\in\eusm L^0(\nu)$, there are a
$g_0\in\eusm L^0_{\Tau}$ and an $F\in\Cal J$ such that $g(y)$ is defined
and equal to $g_0(y)$ for every $y\in Y\setminus F$.   In this case,
$\phi^{-1}[F]\in\Cal I$ and
$g\phi(x)$ is defined and equal to $g_0\phi(x)$ for every
$x\in X\setminus\phi^{-1}[F]$, so $g\phi\in\eusm L^0(\mu)$ and

\Centerline{$(g\phi)^{\ssbullet}=(g_0\phi)^{\ssbullet}
=T_{\pi}(g_0^{\ssbullet})=T_{\pi}(g^{\ssbullet})$}

\noindent by (b).
}%end of proof of 364Q

\leader{364R}{Products:  Proposition} Let
$\langle\frak A_i\rangle_{i\in I}$ be a family of Dedekind
$\sigma$-complete Boolean algebras, with simple product $\frak A$.   If
$\pi_i:\frak A\to\frak A_i$ is the coordinate map for each $i$, and
$T_i:L^0(\frak A)\to L^0(\frak A_i)$ the corresponding homomorphism,
then $u\mapsto Tu=\langle T_iu\rangle_{i\in I}:
L^0(\frak A)\to\prod_{i\in I}L^0(\frak A_i)$ is a multiplicative
Riesz space isomorphism,
so $L^0(\frak A)$ may be identified with the $f$-algebra product
$\prod_{i\in I}L^0(\frak A_i)$\cmmnt{ (352Wc)}.

\proof{ Because each $\pi_i$ is a surjective order-continuous Boolean
homomorphism, 364P assures us that there are corresponding surjective
multiplicative Riesz homomorphisms $T_i$.   So all we need to check is
that the multiplicative Riesz homomorphism
$T:L^0(\frak A)\to\prod_{i\in I}L^0(\frak A_i)$ is a bijection.

If $u$, $v\in L^0(\frak A)$ are distinct, there must be some
$\alpha\in\Bbb R$ such that $\Bvalue{u>\alpha}\ne\Bvalue{v>\alpha}$.
In this case there must be an $i\in I$ such that
$\pi_i\Bvalue{u>\alpha}\ne\pi_i\Bvalue{v>\alpha}$, that is,
$\Bvalue{T_iu>\alpha}\ne\Bvalue{T_iv>\alpha}$.   So $T_iu\ne T_iv$ and
$Tu\ne Tv$.   As $u$, $v$ are arbitrary, $T$ is injective.

If $w=\langle w_i\rangle_{i\in I}$ is any member of
$\prod_{i\in I}L^0(\frak A_i)$, then for $\alpha\in\Bbb R$ set

\Centerline{$\phi(\alpha)=\langle\Bvalue{w_i>\alpha}\rangle_{i\in
I}\in\frak A$.}

\noindent It is easy to check that $\phi$ satisfies the conditions of
364Aa, because, for instance,

\Centerline{$\sup_{\beta>\alpha}\pi_i\phi(\beta)
=\sup_{\beta>\alpha}\Bvalue{w_i>\beta}=\Bvalue{w_i>\alpha}
=\pi_i\phi(\alpha)$}

\noindent for every $i$, so that
$\sup_{\beta>\alpha}\phi(\beta)=\phi(\alpha)$, for every $\alpha\in\Bbb
R$;  and the other two conditions are also satisfied because they are
satisfied coordinate-by-coordinate.   So there is a $u\in L^0(\frak A)$
such that $\phi(\alpha)=\Bvalue{u>\alpha}$ for every $\alpha$, that is,
$\pi_i\Bvalue{u>\alpha}=\Bvalue{w_i>\alpha}$ for all $\alpha$, $i$, that
is, $T_iu=w_i$ for every $i$, that is, $Tu=w$.   As $w$ is arbitrary,
$T$ is surjective and we are done.
}%end of proof of 364R

\leader{*364S}{Regular open \dvrocolon{algebras}}\cmmnt{ I noted in
314P that for every topological space $X$ there is a corresponding
Dedekind complete Boolean algebra $\RO(X)$ of regular open sets.   We
have an identification of $L^0(\RO(X))$ as a space of equivalence
classes of
functions, different in kind from the representations above, as follows.
This is hard work (especially if we do it in full generality), but
instructive.   I start with a temporary definition.

\wheader{*364S}{4}{2}{2}{36pt}

\noindent}{\bf Definition} Let $(X,\frak T)$ be a topological space and
$f:X\to\Bbb R$ a function.   For $x\in X$ write

\Centerline{$\omega(f,x)
=\inf_{G\in\frak T,x\in G}\sup_{y,z\in G}|f(y)-f(z)|$}

\noindent (allowing $\infty$).

\leader{*364T}{Theorem} Let $X$ be any topological space, and
$\RO(X)$ its regular open algebra.   Let $U$ be the set of functions
$f:X\to\Bbb R$ such that $\{x:\omega(f,x)<\epsilon\}$ is dense in $X$
for every $\epsilon>0$.  Then $U$ is a Riesz subspace of $\Bbb R^X$,
closed under multiplication, and we have a surjective multiplicative
Riesz homomorphism $T:U\to L^0(\RO(X))$ defined by writing

\Centerline{$\Bvalue{Tf>\alpha}
=\sup_{\beta>\alpha}\interior\overline{\{x:f(x)>\beta\}}$,}

\noindent the supremum being taken in $\RO(X)$, for every
$\alpha\in\Bbb R$ and $f\in U$.   The kernel of $T$ is
the set $W$ of functions $f:X\to\Bbb R$ such that
$\interior\{x:|f(x)|\le\epsilon\}$ is dense for every $\epsilon>0$, so
$L^0(\RO(X))$ can be identified, as $f$-algebra, with the
quotient space $U/W$.

\proof{{\bf (a)(i)}\grheada\ The first thing to observe is that for any
$f\in\Bbb R^X$ and $\epsilon>0$ the set

$$\eqalign{\{x:\omega(f,x)<\epsilon\}
&=\bigcup\{G:G\subseteq X\text{ is open and non-empty}\cr
&\qquad\qquad\qquad\qquad
  \text{ and }\sup_{y,z\in G}|f(y)-f(z)|<\epsilon\}\cr}$$

\noindent is open.

\medskip

\qquad\grheadb\ Next, it is easy to see that

\Centerline{$\omega(f+g,x)\le\omega(f,x)+\omega(g,x)$,}

\Centerline{$\omega(\gamma f,x)=|\gamma|\omega(f,x)$,}

\Centerline{$\omega(|f|,x)\le\omega(f,x)$,}

\noindent for all $f$, $g\in\Bbb R^X$ and $\gamma\in\Bbb R$.

\medskip

\qquad\grheadc\ Thirdly, it is useful to know that if $f\in U$ and
$G\subseteq X$ is a non-empty open set, then there is a non-empty open
set $G'\subseteq G$ on which $f$ is bounded.   \Prf\ Take any $x_0\in G$
such that $\omega(f,x_0)<1$;  then there is a non-empty open set $G'$
containing $x_0$ such that $|f(y)-f(z)|<1$ for all $y$, $z\in G'$, and
we may suppose that $G'\subseteq G$.   But now $|f(x)|\le 1+|f(x_0)|$
for every $x\in G'$.\ \Qed

\medskip

\quad{\bf (ii)} So if $f$, $g\in U$ and $\gamma\in\Bbb R$ then

\Centerline{$\{x:\omega(f+g,x)<\epsilon\}
\supseteq\{x:\omega(f,x)<\Bover12\epsilon\}
   \cap\{x:\omega(g,x)<\Bover12\epsilon\}$}

\noindent is the intersection of two dense open sets and is therefore
dense, while

\Centerline{$\{x:\omega(\gamma f,x)<\epsilon\}
\supseteq\{x:\omega(f,x)<\Bover{\epsilon}{1+|\gamma|}\}$,}

\Centerline{$\{x:\omega(|f|,x)<\epsilon\}
\supseteq\{x:\omega(f,x)<\epsilon\}$}

\noindent are also dense.   As $\epsilon$ is arbitrary, $f+g$,
$\gamma f$ and $|f|$ all belong to $U$;  as $f$, $g$ and $\gamma$ are arbitrary, $U$ is a Riesz subspace of $\Bbb R^X$.

\medskip

\quad{\bf (iii)} If $f$, $g\in U$ then $f\times g\in U$.   \Prf\ Take
$\epsilon>0$ and let $G_0$ be a non-empty open subset of $X$.   By the
last remark in (i) above, there is a non-empty open set
$G_1\subseteq G_0$ such that $|f|\vee|g|$ is bounded on $G_1$;  say
$\max(|f(x)|,|g(x)|)\le\gamma$ for every $x\in G_1$.

Set $\delta=\Bover{\epsilon}{2\gamma+1}>0$.   Then there is an
$x\in G_1$ such that $\omega(f,x)<\delta$ and $\omega(g,x)<\delta$.
Let $H$, $H'$ be open sets containing $x$ such that $|f(y)-f(z)|\le\delta$
for all $y$, $z\in H$ and $|g(y)-g(z)|\le\delta$ for all $y$, $z\in H'$.
Consider $G=G_1\cap H\cap H'$.   This is an open set containing $x$, and
if $y$, $z\in G$ then

$$\eqalign{|f(y)g(y)-f(z)g(z)|
&\le|f(y)-f(z)||g(z)|+|f(z)||g(y)-g(z)|\cr
&\le\delta\gamma+\gamma\delta.\cr}$$

\noindent Accordingly

\Centerline{$\omega(f\times g,x)\le 2\delta\gamma<\epsilon$,}

\noindent while $x\in G_0$.   As $G_0$ is arbitrary,
$\{x:\omega(f\times g,x)<\epsilon\}$ is dense;  as $\epsilon$ is
arbitrary, $f\times g\in U$.\ \Qed

Thus $U$ is a subalgebra of $\Bbb R^X$.

\woddheader{*364T}{6}{2}{2}{30pt}

{\bf (b)} Now, for $f\in U$, consider the map $\phi_f:\Bbb R\to\RO(X)$
defined by setting

\Centerline{$\phi_f(\alpha)
=\sup_{\beta>\alpha}\interior\overline{\{x:f(x)>\beta\}}$}

\noindent for every $\alpha\in\Bbb R$.   Then $\phi_f$ satisfies the
conditions of 364Aa.   \Prf\ (See 314P for the calculation of suprema and
infima in $\RO(X)$.)  (i) If $\alpha\in\Bbb R$ then

$$\eqalign{\phi_f(\alpha)
&=\sup_{\beta>\alpha}\interior\overline{\{x:f(x)>\beta\}}
=\sup_{\gamma>\beta>\alpha}\interior\overline{\{x:f(x)>\gamma\}}\cr
&=\sup_{\beta>\alpha}\sup_{\gamma>\beta}
   \interior\overline{\{x:f(x)>\gamma\}}
=\sup_{\beta>\alpha}\phi_f(\beta).\cr}$$

\noindent (ii) If $G_0\subseteq X$ is a non-empty open set, then there
is a non-empty open set $G_1\subseteq G_0$ such that $f$ is bounded on
$G_1$;  say $|f(x)|<\gamma$ for every $x\in G_1$.   If $\beta>\gamma$
then $G_1$ does not meet $\{x:f(x)>\beta\}$, so
$G_1\cap\interior\overline{\{x:f(x)>\gamma\}}=\emptyset$;  as $\beta$ is
arbitrary, $G_1\cap\phi_f(\gamma)=\emptyset$ and
$G_0\not\subseteq\inf_{\alpha\in\Bbb R}\phi_f(\alpha)$.   On the other
hand, $G_1\subseteq\{x:f(x)>-\gamma\}$, so

\Centerline{$G_1\subseteq\interior\overline{\{x:f(x)>-\gamma\}}
\subseteq\phi_f(-\gamma)$}

\noindent and $G_0\cap\sup_{\alpha\in\Bbb R}\phi_f(\alpha)\ne\emptyset$.
As $G_0$ is arbitrary, $\inf_{\alpha\in\Bbb R}\phi_f(\alpha)=\emptyset$
and $\sup_{\alpha\in\Bbb R}\phi_f(\alpha)=X$.\ \Qed

\medskip

{\bf (c)} Thus we have a map $T:U\to L^0=L^0(\RO(X))$ defined by
setting $\Bvalue{Tf>\alpha}=\phi_f(\alpha)$ whenever $\alpha\in\Bbb R$ and
$f\in U$.

It is worth noting that

\Centerline{$\{x:f(x)>\alpha+\omega(f,x)\}
\subseteq\Bvalue{Tf>\alpha}
\subseteq\{x:f(x)+\omega(f,x)\ge\alpha\}$}

\noindent for every $f\in U$ and $\alpha\in\Bbb R$.   \Prf\ (i) If
$f(x)>\alpha+\omega(f,x)$, set
$\delta=\bover12(f(x)-\alpha-\omega(f,x))>0$.   Then there is an open
set $G$ containing $x$ such that $|f(y)-f(z)|<\omega(f,x)+\delta$ for
every $y$, $z\in G$, so that $f(y)>\alpha+\delta$ for every $y\in G$,
and

\Centerline{$x\in\interior\{y:f(y)>\alpha+\delta\}
\subseteq\Bvalue{Tf>\alpha}$.}

\noindent (ii) If $f(x)+\omega(f,x)<\alpha$, set
$\delta=\bover12(\alpha-f(x)-\omega(f,x))>0$;  then there is an open
neighbourhood $G$ of $x$ such that $|f(y)-f(z)|<\omega(f,x)+\delta$ for
every $y$, $z\in G$, so that $f(y)<\alpha$ for every $y\in G$.
Accordingly $G$ does not meet $\{y:f(y)>\beta\}$ nor
$\overline{\{y:f(y)>\beta\}}$ for any $\beta>\alpha$,
$G\cap\Bvalue{Tf>\alpha}=\emptyset$ and
$x\notin\Bvalue{Tf>\alpha}$.\ \Qed

\medskip

{\bf (d)} $T$ is additive.   \Prf\ Let $f$, $g\in U$ and
$\alpha<\beta\in\Bbb R$.
Set $\delta=\bover15(\beta-\alpha)>0$,
$H=\{x:\omega(f,x)<\delta,\,\omega(g,x)<\delta\}$;
then $H$ is the intersection of two dense open
sets, so is itself dense and open.

\medskip

\quad{\bf (i)} If $x\in H\cap\Bvalue{T(f+g)>\beta}$, then
$(f+g)(x)+\omega(f+g,x)\ge\beta$;  but $\omega(f+g,x)\le 2\delta$ (see
(a-i-$\beta$) above), so $f(x)+g(x)\ge\beta-2\delta>\alpha+2\delta$ and
there is a $q\in\Bbb Q$ such that

\Centerline{$f(x)>q+\delta\ge q+\omega(f,x)$,
\quad$g(x)>\alpha-q+\delta\ge\alpha-q+\omega(g,x)$.}

\noindent Accordingly

\Centerline{$x\in\Bvalue{Tf>q}\cap\Bvalue{Tg>\alpha-q}
\subseteq\Bvalue{Tf+Tg>\alpha}$.}

\noindent Thus
$H\cap\Bvalue{T(f+g)>\beta}\subseteq\Bvalue{Tf+Tg>\alpha}$.   Because
$H$ is dense, $\Bvalue{T(f+g)>\beta}\subseteq\Bvalue{Tf+Tg>\alpha}$.

\medskip

\quad{\bf (ii)} If $x\in H$, then

$$\eqalignno{x\in\bigcup_{q\in\Bbb Q}
&(\Bvalue{Tf>q}\cap\Bvalue{Tg>\beta-q})\cr
&\Longrightarrow\exists\,q\in\Bbb Q,\,f(x)+\omega(f,x)\ge q,\,
   g(x)+\omega(g,x)\ge\beta-q\cr
&\Longrightarrow f(x)+g(x)+2\delta\ge\beta\cr
&\Longrightarrow (f+g)(x)\ge\alpha+3\delta>\alpha+\omega(f+g,x)\cr
&\Longrightarrow x\in\Bvalue{T(f+g)>\alpha}.\cr}$$

\noindent Thus

\Centerline{$H\cap\bigcup_{q\in\Bbb
Q}(\Bvalue{Tf>q}\cap\Bvalue{Tg>\beta-q})
\subseteq\Bvalue{T(f+g)>\alpha}$.}

\noindent Because $H$ is dense and
$\bigcup_{q\in\Bbb Q}(\Bvalue{Tf>q}\cap\Bvalue{Tg>\beta-q})$ is open,

$$\eqalign{\Bvalue{Tf+Tg>\beta}
&=\interior\overline{\bigcup_{q\in\Bbb Q}\Bvalue{Tf>q}
   \cap\Bvalue{Tg>\beta-q}}\cr
&\subseteq\interior\overline{\Bvalue{T(f+g)>\alpha}}
=\Bvalue{T(f+g)>\alpha}.\cr}$$

\medskip

\quad{\bf (iii)} Now let $\beta\downarrow\alpha$;  we have

$$\eqalign{\Bvalue{T(f+g)>\alpha}
&=\sup_{\beta>\alpha}\Bvalue{T(f+g)>\beta}
\subseteq\Bvalue{Tf+Tg>\alpha}\cr
&=\sup_{\beta>\alpha}\Bvalue{Tf+Tg>\beta}
\subseteq\Bvalue{T(f+g)>\alpha},\cr}$$

\noindent so $\Bvalue{T(f+g)>\alpha}=\Bvalue{Tf+Tg>\alpha}$.   As
$\alpha$ is arbitrary, $T(f+g)=Tf+Tg$;  as $f$ and $g$ are arbitrary,
$T$ is additive.\ \Qed

\medskip

{\bf (e)} It is now easy to see that $T$ is linear.   \Prf\ If
$\gamma>0$, $f\in U$ and $\alpha\in\Bbb R$ then

$$\eqalign{\Bvalue{T(\gamma f)>\alpha}
&=\sup_{\beta>\alpha}\interior\overline{\{x:\gamma f(x)>\beta\}}
=\sup_{\beta>\alpha}
  \interior\overline{\{x:f(x)>\Bover{\beta}{\gamma}\}}\cr
&=\sup_{\beta>\alpha/\gamma}\interior\overline{\{x:f(x)>\beta\}}
=\Bvalue{Tf>\Bover{\alpha}{\gamma}}
=\Bvalue{\gamma Tf>\alpha}.\cr}$$

\noindent As $\alpha$ is arbitrary, $T(\gamma f)=\gamma Tf$;  because we
already know that $T$ is additive, this is enough to show that $T$ is
linear.\ \Qed

\medskip

{\bf (f)} In fact $T$ is a Riesz homomorphism.   \Prf\ If $f\in U$ and
$\alpha\ge 0$ then

$$\eqalign{\Bvalue{T(f^+)>\alpha}
&=\sup_{\beta>\alpha}\interior\overline{\{x:f^+(x)>\beta\}}
=\sup_{\beta>\alpha}\interior\overline{\{x:f(x)>\beta\}}\cr
&=\Bvalue{Tf>\alpha}
=\Bvalue{(Tf)^+>\alpha}.\cr}$$

\noindent If $\alpha<0$ then

\Centerline{$\Bvalue{T(f^+)>\alpha}
=\sup_{\beta>\alpha}\interior\overline{\{x:f^+(x)>\beta\}}
=X=\Bvalue{(Tf)^+>\alpha}$.   \Qed}

\medskip

{\bf (g)} Of course the constant function $\chi X$ belongs to $U$, and
is its multiplicative identity;  and $T(\chi X)$ is the multiplicative
identity of $L^0$, because

$$\eqalign{\Bvalue{T(\chi X)>\alpha}
&=\sup_{\beta>\alpha}\interior\overline{\{x:(\chi X)(x)>\beta\}}\cr
&=X\text{ if }\alpha<1,\,\emptyset\text{ if }\alpha\ge 1.\cr}$$

\noindent By 353Pd, or otherwise, $T$ is multiplicative.

\medskip

{\bf (h)}  The kernel of $T$ is $W$.   \Prf\ (i) For $f\in U$,

$$\eqalignno{Tf=0
&\Longrightarrow\Bvalue{T|f|>0}=\Bvalue{|Tf|>0}=\emptyset\cr
&\Longrightarrow\{x:|f(x)|>\omega(|f|,x)\}=\emptyset\cr
&\Longrightarrow\interior\{x:|f(x)|\le\epsilon\}
  \supseteq\{x:\omega(|f|,x)<\epsilon\}
  \text{ is dense for every }\epsilon>0\cr
&\Longrightarrow f\in W.\cr}$$

\noindent (ii) If $f\in W$, then, first,

\Centerline{$\{x:\omega(f,x)<\epsilon\}
\supseteq\interior\{x:|f(x)|\le\Bover13\epsilon\}$}

\noindent is dense for every $\epsilon>0$, so $f\in U$;  and next, for
any $\beta>0$, $\overline{\{x:|f(x)|>\beta\}}$ does not meet the dense
open set $\interior\{x:|f(x)|\le\beta\}$, so

\Centerline{$\Bvalue{|Tf|>0}=\Bvalue{T|f|>0}
=\sup_{\beta>0}\interior\overline{\{x:|f(x)|>\beta\}}=\emptyset$}

\noindent and $Tf=0$.\ \Qed

\medskip

{\bf (i)} Finally, $T$ is surjective.   \Prf\ Take any $u\in L^0$.
Define $\tilde f:X\to[-\infty,\infty]$ by setting
$\tilde f(x)=\sup\{\alpha:x\in\Bvalue{u>\alpha}\}$ for each $x$,
counting $\inf\emptyset$ as $-\infty$.   Then

\Centerline{$\{x:\tilde
f(x)>\alpha\}=\bigcup_{\beta>\alpha}\Bvalue{u>\beta}$}

\noindent is open, for every $\alpha\in\Bbb R$.   The set

\Centerline{$\{x:\tilde f(x)=\infty\}
=\bigcap_{\alpha\in\Bbb R}\Bvalue{u>\alpha}$}

\noindent is nowhere dense, because
$\inf_{\alpha\in\Bbb R}\Bvalue{u>\alpha}=\emptyset$ in $\RO(X)$;  while

\Centerline{$\{x:\tilde f(x)=-\infty\}
=X\setminus\bigcup_{\alpha\in\Bbb R}\Bvalue{u>\alpha}$}

\noindent also is nowhere dense, because
$\sup_{\alpha\in\Bbb R}\Bvalue{u>\alpha}=X$ in $\RO(X)$.   Accordingly
$E=\interior\{x:\tilde f(x)\in\Bbb R\}$ is dense.    Set
$f(x)=\tilde f(x)$ for $x\in E$, $0$ for $x\in X\setminus E$.

Let $\epsilon>0$.   If $G\subseteq X$ is a non-empty open set, there is
an $\alpha\in\Bbb R$ such that $G\not\subseteq\Bvalue{u>\alpha}$, so
$G_1=G\setminus\overline{\Bvalue{u>\alpha}}\ne\emptyset$, and
$\tilde f(x)\le\alpha$ for every $x\in G_1$.   Set

\Centerline{$\alpha'=\sup_{x\in G_1}\tilde f(x)\le\alpha<\infty$.}

\noindent Because $E$ meets $G_1$, $\alpha'>-\infty$.   Then
$G_2=G_1\cap\Bvalue{u>\alpha'-\bover12\epsilon}$ is a
non-empty open subset of $G$ and
$\alpha'-\bover12\epsilon\le\tilde f(x)\le\alpha'$ for
every $x\in G_2$.   Accordingly $|f(y)-f(z)|\le\bover12\epsilon$ for all
$y$, $z\in G_2$, and $\omega(f,x)<\epsilon$ for all $x\in G_2$.
As $G$ is arbitrary, $\{x:\omega(f,x)<\epsilon\}$ is dense;  as
$\epsilon$ is arbitrary, $f\in U$.

Take $\alpha<\beta$ in $\Bbb R$, and set
$\delta=\bover12(\beta-\alpha)$.   Then
$H=E\cap\{x:\omega(f,x)<\delta\}$ is a dense open set, and

$$\eqalign{H\cap\Bvalue{Tf>\beta}
&\subseteq H\cap\{x:f(x)+\omega(f,x)\ge\beta\}
\subseteq E\cap\{x:f(x)>\alpha\}\cr
&\subseteq\{x:\tilde f(x)>\alpha\}
\subseteq\Bvalue{u>\alpha}.\cr}$$

\noindent As $H$ is dense,
$\Bvalue{Tf>\beta}\subseteq\Bvalue{u>\alpha}$.   In the other direction

$$\eqalign{H\cap\Bvalue{u>\beta}
&\subseteq H\cap\{x:\tilde f(x)\ge\beta\}
=H\cap\{x:f(x)\ge\beta\}\cr
&\subseteq\{x:f(x)>\alpha+\omega(f,x)\}
\subseteq\Bvalue{Tf>\alpha},\cr}$$

\noindent so $\Bvalue{u>\beta}\subseteq\Bvalue{Tf>\alpha}$.   Just as in
(d) above, this is enough to show that $Tf=u$.   As $u$ is arbitrary,
$T$ is surjective.\ \Qed

This completes the proof.
}%end of proof of 364T

\leader{*364U}{Compact spaces} Suppose now that $X$ is a compact
Hausdorff topological space.
In this case the space $U$ of 364T is just the space of functions
$f:X\to\Bbb R$ such that $\{x:f$ is continuous at $x\}$ is dense in $X$.
\prooflet{\Prf\ It is easy to see that

\Centerline{$\{x:f$ is continuous at
$x\}=\{x:\omega(f,x)=0\}=\bigcap_{n\in\Bbb N}H_n$}

\noindent where $H_n=\{x:\omega(f,x)<2^{-n}\}$ for
each $n$.   Each $H_n$ is an open set (see part (a-i-$\alpha$) of the
proof of 364T), so by Baire's theorem (3A3G) $\bigcap_{n\in\Bbb N}H_n$
is dense iff every $H_n$ is dense, that is, iff $f\in U$.\ \Qed}

Now $W$, as defined in 364T,
becomes $\{f:f\in U,\,\{x:f(x)=0\}$ is dense$\}$.   \prooflet{\Prf\ (i) If
$f\in W$, then $T|f|=0$, so (by the formula in (c) of the proof of 364T)
$|f(x)|\le\omega(|f|,x)$ for every $x$.   But $\{x:\omega(f,x)=0\}$ is
dense, because $f\in U$, so $\{x:f(x)=0\}$ also is dense.   (ii) If
$f\in U$ and $\{x:f(x)=0\}$ is dense, then

\Centerline{$\omega(f,x)
\ge\inf_{x\in G\text{ is open}}\sup_{y\in G}|f(y)-f(x)|
\ge|f(x)|$}

\noindent for every $x\in X$.   So for any $\epsilon>0$,
$\interior\{x:|f(x)|\le\epsilon\}\supseteq\{x:\omega(f,x)<\epsilon\}$ is
dense, and $f\in W$.\ \Qed}


\cmmnt{In the case of extremally disconnected spaces, we can go
farther.}

\leader{*364V}{Theorem} Let $X$ be a compact Hausdorff extremally
disconnected space, and $\RO(X)$ its regular open algebra.   Write
$C^{\infty}=C^{\infty}(X)$ for the space of continuous functions
$g:X\to[-\infty,\infty]$ such that $\{x:g(x)=\pm\infty\}$ is nowhere
dense.   Then we have a bijection $S:C^{\infty}\to L^0=L^0(\RO(X))$
defined by saying that

\Centerline{$\Bvalue{Sg>\alpha}=\overline{\{x:g(x)>\alpha\}}$}

\noindent for every $\alpha\in\Bbb R$.   Addition and multiplication in
$L^0$ correspond to the operations $\frdotplus$, $\dottimes$ on
$C^{\infty}$ defined by saying that $g\frdotplus h$, $g\dottimes h$ are
the unique elements of $C^{\infty}$ agreeing with $g+h$, $g\times h$ on
$\{x:g(x),\,h(x)$ are both finite$\}$.   Scalar multiplication in $L^0$
corresponds to the operation

\Centerline{$(\gamma g)(x)=\gamma g(x)$ for $x\in X$, $g\in C^{\infty}$,
$\gamma\in\Bbb R$}

\noindent on $C^{\infty}$ (counting $0\cdot\infty$ as $0$), while the
ordering of $L^0$ corresponds to the relation

\Centerline{$g\le h\iff g(x)\le h(x)$ for every $x\in X$.}

\proof{{\bf (a)} For $g\in C^{\infty}$, set $H_g=\{x:g(x)\in\Bbb R\}$,
so that $H_g$ is a dense open set, and define $Rg:X\to\Bbb R$ by
setting $(Rg)(x)=g(x)$ if $x\in H_g$, $0$ if $x\in X\setminus H_g$.
Then $Rg$ is continuous at every point of $H_g$, so belongs to the space
$U$ of 364T-364U.   Set $Sg=T(Rg)$, where $T:U\to L^0$ is the map
of 364T.   Then

\Centerline{$\Bvalue{Sg>\alpha}=\overline{\{x:g(x)>\alpha\}}$}

\noindent for every $\alpha\in\Bbb R$.   \Prf\ (i) $\omega(g,x)=0$ for
every $x\in H_g$, so, if $\beta>\alpha$,

\Centerline{$H_g\cap\Bvalue{Sg>\beta}
\subseteq\{x:x\in H_g,\,(Rg)(x)\ge\beta\}
\subseteq\{x:g(x)\ge\beta\}$}

\noindent by the formula in part (c) of the proof of 364T.   As
$\Bvalue{Sg>\beta}$ is open and $H_g$ is dense,

\Centerline{$\Bvalue{Sg>\beta}
\subseteq\overline{H_g\cap\Bvalue{Sg>\beta}}\subseteq\{x:g(x)\ge\beta\}
\subseteq\{x:g(x)>\alpha\}$.}

\noindent Now

\Centerline{$\Bvalue{Sg>\alpha}=\sup_{\beta>\alpha}\Bvalue{Sg>\beta}
=\interior\overline{\bigcupop_{\beta>\alpha}\Bvalue{Sg>\beta}}
\subseteq\overline{\{x:g(x)>\alpha\}}$.}

\noindent (ii) In the other direction,
$H_g\cap\{x:g(x)>\alpha\}\subseteq\Bvalue{Sg>\alpha}$, by the other half
of the formula in the proof of 364T.   Again because $\{x:g(x)>\alpha\}$
is open and $H_g$ is dense,

\Centerline{$\overline{\{x:g(x)>\alpha\}}
\subseteq\overline{\Bvalue{Sg>\alpha}}=\Bvalue{Sg>\alpha}$}

\noindent because $X$ is extremally disconnected (see 314S).\ \Qed

\medskip

{\bf (b)} Thus $S=TR$ defined by the formula offered.
Now if $g$, $h\in C^{\infty}$ and $g\le h$, we surely have
$\{x:g(x)>\alpha\}
\subseteq\{x:h(x)>\alpha\}$ for every $\alpha$, so
$\Bvalue{Sg>\alpha}\subseteq\Bvalue{Sh>\alpha}$ for every $\alpha$ and
$Sg\le Sh$.   On the other hand, if $g\not\le h$ then $Sg\not\le Sh$.
\Prf\ Take $x_0$ such that $g(x_0)>h(x_0)$, and $\alpha\in\Bbb R$ such
that $g(x_0)>\alpha>h(x_0)$;  set $H=\{x:g(x)>\alpha>h(x)\}$;  this is a
non-empty open set and $H\subseteq\Bvalue{Sg>\alpha}$.   On the other
hand, $H\cap\{x:h(x)>\alpha\}=\emptyset$ so
$H\cap\Bvalue{Sh>\alpha}=\emptyset$.   Thus
$\Bvalue{Sg>\alpha}\not\subseteq\Bvalue{Sh>\alpha}$ and
$Sg\not\le Sh$.\ \QeD\   In particular, $S$ is injective.

\medskip

{\bf (c)} $S$ is surjective.   \Prf\ If $u\in L^0$, set

\Centerline{$g(x)
=\sup\{\alpha:x\in\Bvalue{u>\alpha}\}\in[-\infty,\infty]$}

\noindent for every $x\in X$, taking $\sup\emptyset=-\infty$.   Then,
for any $\alpha\in\Bbb R$,
$\{x:g(x)>\alpha\}=\bigcup_{\beta>\alpha}\Bvalue{u>\alpha}$ is open.
On the other hand,

\Centerline{$\{x:g(x)<\alpha\}
=\bigcup_{\beta<\alpha}\{x:x\notin\Bvalue{u>\beta}\}$}

\noindent also is open, because all the sets $\Bvalue{u>\beta}$ are
open-and-closed.   So $g:X\to[-\infty,\infty]$ is continuous.
Also

\Centerline{$\{x:g(x)>-\infty\}
=\bigcup_{\alpha\in\Bbb R}\Bvalue{u>\alpha}$,}

\Centerline{$\{x:g(x)<\infty\}
=\bigcup_{\alpha\in\Bbb R}X\setminus\Bvalue{u>\alpha}$}

\noindent are dense, so $g\in C^{\infty}$.    Now, for
any $\alpha\in\Bbb R$,

$$\eqalign{\Bvalue{Sg>\alpha}
&=\overline{\{x:g(x)>\alpha\}}
=\overline{\bigcup_{\beta>\alpha}\Bvalue{u>\beta}}\cr
&=\interior\overline{\bigcup_{\beta>\alpha}\Bvalue{u>\beta}}
=\sup_{\beta>\alpha}\Bvalue{u>\beta}
=\Bvalue{u>\alpha}.\cr}$$

\noindent So $Sg=u$.   As $u$ is arbitrary, $S$ is surjective.\ \Qed

\medskip

{\bf (d)} Accordingly $S$ is a bijection.   I have already checked (in
part (b)) that it is an isomorphism of the order structures.   For the
algebraic operations, observe that if $g$, $h\in C^{\infty}$ then there
are $f_1$, $f_2\in C^{\infty}$ such that $Sg+Sh=Sf_1$ and $Sg\times
Sh=Sf_2$, that is,

\Centerline{$T(Rg+Rh)=TRg+TRh=TRf_1$,
\quad$T(Rg\times Rh)=TRg\times TRh=TRf_2$.}

\noindent   But this means that

\Centerline{$T(Rg+Rh-Rf_1)=T((Rg\times Rh)-Rf_2)=0$,}

\noindent so that $Rg+Rh-Rf_1$, $(Rg\times Rh)-Rf_2$ belong to $W$, as
defined in 364T-364U, and
are zero on dense sets (364U).   Since we know also that the set
$G=\{x:g(x)$, $h(x)$ are both finite$\}$ is a dense open set, while $g$,
$h$, $f_1$ and $f_2$ are all continuous, we must have
$f_1(x)=g(x)+h(x)$, $f_2(x)=g(x)h(x)$ for every $x\in G$.   And of
course this uniquely specifies $f_1$ and $f_2$ as members of
$C^{\infty}$.

Thus we do have operations $\frdotplus$, $\dottimes$ as
described, rendering $S$ additive and multiplicative.   As for scalar
multiplication, it is easy to check that $R(\gamma g)=\gamma Rg$ (at
least, unless $\gamma=0$, which is trivial), so that
$S(\gamma g)=\gamma Sg$ for every $g\in C^{\infty}$.
}%end of proof of 364V

\exercises{\leader{364X}{Basic exercises $\pmb{>}$(a)}
%\sqheader 364Xa
Let $\frak A$ be a Dedekind $\sigma$-complete Boolean
algebra.   For $u$, $v\in L^0=L^0(\frak A)$ set
$\Bvalue{u<v}=\Bvalue{v>u}=\Bvalue{v-u>0}$,
$\Bvalue{u\le v}=\Bvalue{v\ge u}=1\Bsetminus\Bvalue{v<u}$,
$\Bvalue{u=v}=\Bvalue{u\le v}\Bcap\Bvalue{v\le u}$.   (i) Show that
$(\Bvalue{u<v},\Bvalue{u=v},\Bvalue{u>v})$ is always a partition of
unity in $\frak A$.   (ii) Show that for any $u$, $u'$, $v$,
$v'\in L^0$, $\Bvalue{u\le u'}\Bcap\Bvalue{v\le v'}
\discretionary{}{}{}\Bsubseteq\Bvalue{u+v\le u'+v'}$ and
$\Bvalue{u=u'}\Bcap\Bvalue{v=v'}\Bsubseteq\Bvalue{u\times v=u'\times v'}$.
%364F

\spheader 364Xb Let $\frak A$ be a Dedekind $\sigma$-complete Boolean
algebra.   (i) Show that if $u$, $v\in L^0=L^0(\frak A)$ and $\alpha$,
$\beta\in\Bbb R$ then $\Bvalue{u+v\ge\alpha+\beta}
\Bsubseteq\Bvalue{u\ge\alpha}\Bcup\Bvalue{v\ge\beta}$.   (ii) Show that
if $u$, $v\in(L^0)^+$ and $\alpha$, $\beta\ge 0$ then
$\Bvalue{u\times v\ge\alpha\beta}
\Bsubseteq\Bvalue{u\ge\alpha}\Bcup\Bvalue{v\ge\beta}$.
%364F, 364E

\spheader 364Xc Let $\frak A$ be a Dedekind $\sigma$-complete Boolean
algebra and $u\in L^0(\frak A)$.   Show that
$\{\Bvalue{u\in E}:E\subseteq\Bbb R$ is Borel$\}$ is
the $\sigma$-subalgebra of $\frak A$ generated by
$\{\Bvalue{u>\alpha}:\alpha\in\Bbb R\}$.
%364F

\sqheader 364Xd Let $(\frak A,\bar\mu)$ be a probability algebra.   Show
that for any $u\in L^0(\frak A)$ there is a unique Radon probability
measure $\nu$ on $\Bbb R$ (the {\bf distribution} of $u$) such that
$\nu E=\bar\mu\Bvalue{u\in E}$ for every Borel set $E\subseteq\Bbb R$.
\Hint{271B.}
%364F

\spheader 364Xe Let $(\frak A,\bar\mu)$ be a probability algebra, and
$\langle u_i\rangle_{i\in I}$ any family in $L^0(\frak A)$;  for each
$i\in I$ let $\frak B_i$ be the closed subalgebra of $\frak A$ generated
by $\{\Bvalue{u_i>\alpha}:\alpha\in\Bbb R\}$.   Show that the following
are equiveridical:  (i) $\bar\mu(\inf_{i\in J}\Bvalue{u_i>\alpha_i})
=\prod_{i\in J}\bar\mu\Bvalue{u_i>\alpha_i}$
whenever $J\subseteq I$ is finite and $\alpha_i\in\Bbb R$ for each
$i\in J$ (ii) $\langle\frak B_i\rangle_{i\in I}$ is
stochastically independent in the
sense of 325L.   (In this case we may call $\langle u_i\rangle_{i\in I}$
{\bf $\bar\mu$-(stochastically )independent}.)
%364F

\spheader 364Xf Let $(\frak A,\bar\mu)$ be a probability algebra and
$u$, $v$ two $\bar\mu$-independent members of $L^0(\frak A)$.
Show that the distribution of their sum is the convolution of their
distributions.    \Hint{272T}.
%364F

\sqheader 364Xg Let $\frak A$ be a Dedekind $\sigma$-complete Boolean
algebra and $g$, $h:\Bbb R\to\Bbb R$  Borel measurable functions.   (i)
Show that $\bar g\bar h=\bar{gh}$, where $\bar g$, $\bar h:L^0\to L^0$
are defined as in 364H.   (ii) Show that
$\overline{g+h}(u)=\bar g(u)+\bar h(u)$,
$\overline{g\times h}(u)=\bar g(u)\times\bar h(u)$ for
every $u\in L^0=L^0(\frak A)$.   (iii) Show that if $\sequencen{h_n}$ is
a sequence of Borel measurable functions on $\Bbb R$ and
$\sup_{n\in\Bbb N}h_n=h$, then
$\sup_{n\in\Bbb N}\bar{h}_n(u)=\bar h(u)$ for
every $u\in L^0$.   (iv) Show that if $h$ is non-decreasing and
continuous on the left, then $\bar h(\sup A)=\sup\bar h[A]$ whenever
$A\subseteq L^0$ is a non-empty set with a supremum in $L^0$.
%364H

\spheader 364Xh Let $\frak A$ be a Dedekind $\sigma$-complete Boolean
algebra.   (i) Show that $S(\frak A)$ can be identified ($\alpha$) with
the set of
those $u\in L^0=L^0(\frak A)$ such that
$\{\Bvalue{u>\alpha}:\alpha\in\Bbb R\}$ is finite ($\beta$) with the set
of those $u\in L^0$ such that $\Bvalue{u\in I}=1$ for some finite
$I\subseteq\Bbb R$.   (ii) Show that
$L^{\infty}(\frak A)$ can be identified with the set of those $u\in L^0$
such that $\Bvalue{u\in[-\alpha,\alpha]}=1$ for some $\alpha\ge 0$, and
that $\|u\|_{\infty}$ is the smallest such $\alpha$.
%364J

\spheader 364Xi Show that if $\frak A$ is a Dedekind $\sigma$-complete
Boolean algebra, and $u\in L^0(\frak A)$, then for any $\alpha\in\Bbb R$

\Centerline{$\Bvalue{u>\alpha}
=\inf_{\beta>\alpha}
  \sup\{a:a\in\frak A,\,u\times\chi a\ge\beta\chi a\}$}

\noindent (compare 363Xh).
%364J

\sqheader 364Xj\dvAnew{2008;  originally 3{}64Xw}
Let $\frak A$ be a Dedekind $\sigma$-complete Boolean
algebra and $\nu:\frak A\to\Bbb R$ a non-negative
finitely additive functional.
Let $\smalldashint:L^{\infty}(\frak A)\to\Bbb R$
be the corresponding linear
functional, as in 363L.   Write $U$ for the set of those
$u\in L^0(\frak A)$ such that
$\sup\{\smalldashint v:v\in L^{\infty}(\frak A)$, $v\le|u|\}$ is finite.
Show that $\smalldashint$ has an
extension to a non-negative linear functional on $U$.
%364J

\spheader 364Xk Let $\frak A$ be a Dedekind $\sigma$-complete Boolean
algebra and $u\ge 0$ in $L^0=L^0(\frak A)$.   Show that
$u=\sup_{q\in\Bbb Q}q\chi\Bvalue{u>q}$ in $L^0$.
%364L

\spheader 364Xl(i) Let $\frak A$ be a Dedekind $\sigma$-complete
Boolean algebra and $A\subseteq L^0(\frak A)$ a non-empty countable set
with supremum $w$.   Show that
$\Bvalue{w\in G}\Bsubseteq\sup_{u\in A}\Bvalue{u\in G}$ for every open
set $G\subseteq\Bbb R$.
(ii) Let $(\frak A,\bar\mu)$ be a localizable measure algebra and
$A\subseteq L^0(\frak A)$ a non-empty set with supremum $w$.   Show that
$\Bvalue{w\in G}\Bsubseteq\sup_{u\in A}\Bvalue{u\in G}$ for every open
set $G\subseteq\Bbb R$.
%364L

\spheader 364Xm
Let $\frak A$ be a Dedekind $\sigma$-complete Boolean algebra and
$A\subseteq L^0=L^0(\frak A)$ a non-empty set which is bounded below in
$L^0$.   Suppose that $\phi_0(\alpha)=\inf_{u\in A}\Bvalue{u>\alpha}$ is
defined in $\frak A$ for every $\alpha\in\Bbb R$.   Show that $v=\inf A$
is defined in $L^0$, and that
$\Bvalue{v>\alpha}=\sup_{\beta>\alpha}\phi_0(\beta)$ for every
$\alpha\in\Bbb R$.
%364L

\spheader 364Xn Let $(X,\Sigma,\mu)$ be a measure space and
$f:X\to\coint{0,\infty}$ a function;  set
$A=\{g^{\ssbullet}:g\in\eusm L^0(\mu),\,g\leae f\}$.   (i) Show that if
$(X,\Sigma,\mu)$
either is localizable or has the measurable envelope property
(213Xl), then $\sup A$ is defined in $L^0(\mu)$.   (ii) Show that if
$(X,\Sigma,\mu)$ is complete and locally determined and $w=\sup A$ is
defined in $L^0(\mu)$, then $w\in A$.

\spheader 364Xo Let $\frak A$ be a Dedekind $\sigma$-complete Boolean
algebra.   Show that if $u$, $v\in L^0=L^0(\frak A)$ then the
following are equiveridical:  ($\alpha$)
$\Bvalue{|v|>0}\Bsubseteq\Bvalue{|u|>0}$ ($\beta$) $v$ belongs to the
band in $L^0$ generated by $u$ ($\gamma$) there is a $w\in L^0$ such
that $u\times w=v$.
%364N

\sqheader 364Xp Let $\frak A$ be a Dedekind $\sigma$-complete Boolean
algebra and $a\in\frak A$;  let $\frak A_a$ be the principal ideal of
$\frak A$ generated by $a$.   Show that $L^0(\frak A_a)$ can be
identified, as $f$-algebra, with the band in $L^0(\frak A)$ generated
by $\chi a$.
%364O

\spheader 364Xq Let $\frak A$ and $\frak B$ be Dedekind
$\sigma$-complete Boolean algebras, and $\pi:\frak A\to\frak B$ a
sequentially order-continuous Boolean homomorphism.   Let
$T:L^0(\frak A)\to L^0(\frak B)$ be the corresponding Riesz homomorphism
(364P).   Show that (i) the kernel of $T$ is the sequentially
order-closed solid linear subspace of $L^0(\frak A)$ generated by
$\{\chi a:a\in\frak A,\,\pi a=0\}$   (ii) the set of values of $T$
is the sequentially
order-closed linear subspace of $L^0(\frak B)$ generated by
$\{\chi(\pi a):a\in\frak A\}$.
%364P

\spheader 364Xr Let $\frak A$ and $\frak B$ be Dedekind
$\sigma$-complete Boolean algebras, and $\pi:\frak A\to\frak B$ a
sequentially order-continuous Boolean homomorphism, with
$T:L^0(\frak A)\to L^0(\frak B)$ the associated operator.   Suppose that
$h$ is a Borel measurable real-valued function defined on a Borel subset
of $\Bbb R$.   Show that $\bar h(Tu)=T\bar h(u)$ whenever
$u\in L^0(\frak A)$ and $\bar h(u)$ is defined in the sense of 364H.
%364P

\spheader 364Xs Let $(\frak A,\bar\mu)$ and $(\frak B,\bar\nu)$ be
probability algebras, and $\pi:\frak A\to\frak B$ a measure-preserving
Boolean homomorphism;  let $T:L^0(\frak A)\to L^0(\frak B)$ be the
corresponding Riesz homomorphism.   Show that if $\familyiI{u_i}$ is a
family in $L^0(\frak A)$, it is $\bar\mu$-independent iff
$\familyiI{Tu_i}$ is $\bar\nu$-independent.
%364P 364Xe

\sqheader 364Xt Let $\frak A$ be a Dedekind $\sigma$-complete Boolean
algebra and $\frak B$ a $\sigma$-subalgebra of $\frak A$.   Show that
$L^0(\frak B)$ can be identified with the sequentially order-closed
Riesz subspace of $L^0(\frak A)$ generated by $\{\chi b:b\in\frak B\}$.
%364P

\spheader 364Xu Let $\frak A$ be a Dedekind $\sigma$-complete Boolean
algebra and $\pi:\frak A\to\frak A$ a sequentially order-continuous
Boolean homomorphism;  let $T_{\pi}:L^0(\frak A)\to L^0(\frak A)$ be the
corresponding Riesz homomorphism.   Let $\frak C$ be the fixed-point
subalgebra of $\pi$.   Show that $\{u:u\in L^0(\frak A)$, $T_{\pi}u=u\}$
can be identified with $L^0(\frak C)$.
%364P 364Xt

\spheader 364Xv Use the ideas of part (d) of the proof of 364T to show
that the operator $T$ there is multiplicative, without appealing to
353P.
%364T

\spheader 364Xw\dvAnew{2012} Let $\frak A$ be a Dedekind $\sigma$-complete
Boolean algebra and $\frak B$ an order-closed subalgebra of $\frak A$.
Show that $L^0(\frak B)$, regarded as a subset of $L^0(\frak A)$,
is order-closed in $L^0(\frak A)$.
%364Xt 364P out of order query

\leader{364Y}{Further exercises $\pmb{>}$(a)}%
%\spheader 364Ya
(i) Show directly, without using the Loomis-Sikorski theorem or the
Stone representation,
that if $\frak A$ is any Dedekind $\sigma$-complete Boolean algebra then
the formulae of 364D define a group operation $+$ on $L^0(\frak A)$, and
generally an $f$-algebra structure.   (ii) Defining
$\chi:\frak A\to L^0(\frak A)$ by the formula in 364Jc, show that
$S(\frak A)$ and $L^{\infty}(\frak A)$ can be identified with the linear
span of $\{\chi a:a\in\frak A\}$ and the solid linear subspace of
$L^0(\frak A)$ generated by $e=\chi 1$.   (iii) Still without using the
Loomis-Sikorski theorem, explain how to define
$\bar h:L^0(\frak A)\to L^0(\frak A)$ for continuous functions
$h:\Bbb R\to\Bbb R$.   (iv) Check that these ideas are sufficient to yield
364L-364R, %364L 364M 364N 364O 364P 364R
except that in 364Pd we may have difficulty with arbitrary
Borel functions $h$.
%364D, 364J

\spheader 364Yb Let $\frak A$ be a Dedekind $\sigma$-complete Boolean
algebra and $\pmb{u}=(u_1,\ldots,u_n)$ a member of $L^0(\frak A)^n$.
Write $\Cal B_n$ for the algebra of Borel sets in $\BbbR^n$.
(i) Show that there
is a unique sequentially order-continuous Boolean homomorphism
$E\mapsto\Bvalue{\pmb{u}\in E}:\Cal B_n\to\frak A$ such that
$\Bvalue{\pmb{u}\in E}=\inf_{i\le n}\Bvalue{u_i>\alpha_i}$ when
$E=\prod_{i\le n}\ooint{\alpha_i,\infty}$.
(ii) Show that for every sequentially order-continuous Boolean
homomorphism $\phi:\Cal B_n\to\frak A$ there is a unique
$\pmb{u}\in L^0(\frak A)^n$ such that 
$\phi E=\Bvalue{\pmb{u}\in E}$ for every $E\in\Cal B_n$.
%364F

\spheader 364Yc Let $\frak A$ be a Dedekind $\sigma$-complete Boolean
algebra, $n\ge 1$ and $h:\BbbR^n\to\Bbb R$ a Borel measurable function.
Show that we have a corresponding function 
$\bar h:L^0(\frak A)^n\to L^0(\frak A)$ defined by saying that 
$\Bvalue{\bar h(\pmb{u})\in E}
=\Bvalue{\pmb{u}\in h^{-1}[E]}$ for every Borel set
$E\subseteq\Bbb R$ and $\pmb{u}\in L^0(\frak A)^n$.
%364H

\spheader 364Yd Suppose that $h_1(x,y)=x+y$, $h_2(x,y)=xy$,
$h_3(x,y)=\max(x,y)$ for all $x$, $y\in\Bbb R$.   Show that, in the
language of 364Yc, $\bar h_1(u,v)=u+v$, $\bar h_2(u,v)=u\times v$, $\bar
h_3(u,v)=u\vee v$ for all $u$, $v\in L^0$.
%364Yc, 364H

\spheader 364Ye Let $\frak A$ be a Dedekind $\sigma$-complete Boolean
algebra.   Show that $\frak A$ is ccc iff $L^0(\frak A)$ has the
countable sup property.
%364J 363Yb

\spheader 364Yf Let $\frak A$ and $\frak B$ be Dedekind
$\sigma$-complete Boolean algebras, and $T:L^0(\frak A)\to L^0(\frak B)$
a Riesz homomorphism such that $Te=e'$, where $e$, $e'$
are the multiplicative identities of $L^0(\frak A)$, $L^0(\frak B)$
respectively.   Show that there is a unique sequentially
order-continuous Boolean homomorphism $\pi:\frak A\to\frak B$ such that
$T=T_{\pi}$ in the sense of 364P.   \Hint{use 353Pd.   Compare
375A below.}
%364P

\spheader 364Yg Let $\frak A$ and $\frak B$ be
Dedekind $\sigma$-complete Boolean algebras
and $\pi:\frak A\to\frak B$ a sequentially order-continuous ring
homomorphism.
(i) Show that we have a multiplicative sequentially order-continuous Riesz
homomorphism $T_{\pi}:L^0(\frak A)\to L^0(\frak B)$ defined by the
formula

\Centerline{$\Bvalue{T_{\pi}u>\alpha}=\pi\Bvalue{u>\alpha}$}

\noindent whenever $u\in L^0(\frak A)$ and $\alpha>0$.
(ii) Show that $T_{\pi}$ is order-continuous iff $\pi$ is order-continuous,
injective iff $\pi$ is injective, and surjective iff $\pi$ is surjective.
(iii) Show that if
$\frak C$ is another Dedekind $\sigma$-complete Boolean algebra
and $\theta:\frak B\to\frak C$ another sequentially order-continuous
ring homomorphism then
$T_{\theta\pi}=T_{\theta}T_{\pi}:L^0(\frak A)\to L^0(\frak C)$.
%364P out of order

\spheader 364Yh Suppose, in 364T, that $X=\Bbb Q$.   (i) Show that there
is an $f\in W$ such that $f(q)>0$ for every $q\in\Bbb Q$.   (ii) Show
that there is a $u\in L^0$ such that no $f\in U$ representing $u$ can be
continuous at any point of $\Bbb Q$.
%364T

\spheader 364Yi Let $X$ and $Y$ be topological spaces and $\phi:X\to Y$
a continuous function such that $\phi^{-1}[M]$ is nowhere dense in $X$
for every nowhere dense subset $M$ of $Y$.   (Cf.\ 313R.)   (i) Show
that we have an order-continuous Boolean homomorphism $\pi$ from the
regular open algebra $\RO(Y)$ of $Y$ to the regular open algebra
$\RO(X)$ of $X$ defined by the formula
$\pi G=\interior\overline{\phi^{-1}[G]}$ for every $G\in\RO(Y)$.
(ii) Show that if $U_X$, $U_Y$
are the function spaces of 364T  then $g\phi\in U_X$ for every
$g\in U_Y$.   (iii) Show that if $T_X:U_X\to L^0(\RO(X))$,
$T_Y:U_Y\to L^0(\RO(Y))$ are the canonical surjections, and $T:L^0(\RO(Y))\to L^0(\RO(X))$ is the homomorphism corresponding to $\pi$, then
$T(T_Yg)=T_X(g\phi)$ for every $g\in U_Y$.   (iv) Rewrite these ideas
for the special case in which $X$ is a dense subset of $Y$ and $\phi$ is
the identity map, showing that in this case $\pi$ and $T$ are
isomorphisms.
%364T

\spheader 364Yj Let $X$ be a Baire space, $\RO(X)$ its algebra of
regular open sets, $\Cal M$ its ideal of meager sets, and
$\widehat{\Cal B}$ the Baire-property
$\sigma$-algebra $\{G\symmdiff A:G\subseteq X$ is open, $A\in\Cal M\}$,
so that $\RO(X)$ can be identified with $\widehat{\Cal B}/\Cal M$
(314Yd).   (i)
Repeat the arguments of 364U in this context.  (ii) Show that the space
$U$ of 364T-364U is a subspace of
$\eusm L^0=\eusm L^0_{\widehat{\Cal B}}$, and
that $W=U\cap\eusm W$ where
$\eusm W=\{f:f\in\Bbb R^X,\,\{x:f(x)\ne 0\}\in\Cal M\}$, so that the
representations of $L^0(\RO(X))$ as $U/W$,
$\eusm L^0/\eusm W$ are consistent.
%364U

\spheader 364Yk Work through the arguments of 364T and 364Yj for the
case of compact Hausdorff $X$, seeking simplifications based on 364U.
%364U, 364Yj

\spheader 364Yl Let $X$ be an extremally disconnected compact Hausdorff
space with regular open algebra $\RO(X)$.   Let $U_0$ be the space of
real-valued functions $f:X\to\Bbb R$ such that $\interior\{x:f$ is
continuous at $x\}$ is dense.   Show that $U_0$ is a Riesz subspace of
the space $U$ of 364T, and that every member of $L^0(\RO(X))$ is
represented by a member of $U_0$.
%364V

\spheader 364Ym Let $X$ be a Baire space.   Let $Q$ be the set of all
continuous real-valued functions defined on subsets of $X$, and $Q^*$
the set of all members of $Q$ which are maximal in the sense that there
is no member of $Q$ properly extending them.   (i) Show that the domain
of any member of $Q^*$ is a dense G$_{\delta}$ set.   (ii) Show that we
can define addition and multiplication and scalar multiplication on
$Q^*$ by saying that $f\frdotplus g$, $f\dottimes g$, $\gamma.f$ are to
be the unique members of $Q^*$ extending the partially-defined functions
$f+g$, $f\times g$, $\gamma f$, and that these definitions render $Q^*$
an $f$-algebra if we say that $f\le g$ iff $f(x)\le g(x)$ for every
$x\in\dom f\cap\dom g$.   (iii) Show that every member of $Q^*$ has an
extension to a member of $U$, as defined in 364T, and that these
extensions define an isomorphism between $Q^*$ and $L^0(\RO(X))$, where
$\RO(X)$ is the regular open algebra of $X$.   (iv) Show that if $X$ is
compact, Hausdorff and extremally disconnected, then every member of
$Q^*$ has a unique extension to a member of $C^{\infty}(X)$, as defined
in 364V.
%364V

\spheader 364Yn Let $X$ be an extremally disconnected Hausdorff space,
and $Z$ any compact Hausdorff space.   Show that if $D\subseteq X$ is
dense and $f:D\to Z$ is continuous, there is a continuous $g:X\to Z$
extending $f$.
%364V

\spheader 364Yo\dvAnew{2012}
Let $(\frak A,\bar\mu)$ be a probability algebra.
(i) Show that for any $\pmb{u}=(u_1,\ldots,u_n)\in L^0(\frak A)^n$ 
there is a unique Radon probability measure $\nu$ on
$\Bbb R^n$ such that 
$\nu(\prod_{1\le i\le n}\ooint{\alpha,\infty})
=\bar\mu(\inf_{1\le i\le n}\Bvalue{u_i>\alpha_i})$ for all
$\alpha_1,\ldots,\alpha_n\in\Bbb R$, and that now 
$\nu E=\bar\mu\Bvalue{\pmb{u}\in E}$ for every Borel set 
$E\subseteq\BbbR^n$.   
I will call $\nu$ the {\bf distribution} of $\pmb{u}$.
(ii) Show that $(u_1,\ldots,u_n)$ is stochastically
independent iff $\nu$ is expressible as $\prod_{1\le i\le n}\nu_i$ where
$\nu_i$ is a Radon probability measure on $\Bbb R$ for each $i$.
(iii) Write $\frak A_{\pmb{u}}$ for the closed subalgebra
$\{\Bvalue{\pmb{u}\in E}:E\subseteq\BbbR^n$ is a Borel set$\}$;  check that
$u_i\in L^0(\frak A_{\pmb{u}})$ for every $i$.
Suppose that $(\frak B,\bar\nu)$ is another probability algebra and
that $\pmb{v}=(v_1,\ldots,v_n)\in(L^0(\frak B))^n$.
Show that the following are equiveridical:  ($\alpha$) there
is a measure-preserving
isomorphism $\pi:\frak A_{\pmb{u}}\to\frak B_{\pmb{v}}$ such that
$T_{\pi}u_i=v_i$ for every $i$ ($\beta$) $\pmb{u}$ and $\pmb{v}$ have the
same distribution.
%364Xe 364Yb 364F 364P out of order query

}%end of exercises

\endnotes{
\Notesheader{364} This has been a long section, and so far all we have
is a supposedly thorough grasp of the construction of $L^0$ spaces;
discussion of their properties still lies ahead.   The difficulties seem
to stem from a variety of causes.   First, $L^0$ spaces have a rich
structure, being linear ordered spaces with multiplications;
consequently all the main theorems have to check rather a lot of
different aspects.   Second, unlike $L^{\infty}$ spaces, they are not
accessible by means of the theory of normed spaces, so I must expect to
do more of the work here rather than in an appendix.   But this is in
fact a crucial difference, because it affects the proof of the central
theorem 364D.   The point is that a given algebra $\frak A$ will be
expressible in the form $\Sigma/\Cal I$ for a variety of algebras
$\Sigma$ of sets.   Consequently any definition of $L^0(\frak A)$ as a
quotient $\eusm L^0_{\Sigma}/\eusm W_{\Cal I}$
must include a check that the
structure produced is independent of the particular pair $\Sigma$,
$\Cal I$ chosen.

The same question arises with $S(\frak A)$ and $L^{\infty}(\frak A)$.
But in the case of $S$, I was able to use a general theory of additive
functions on $\frak A$ (see the proof of 361L), while in the case of
$L^{\infty}$ I could quote the result for $S$ and a little theory of
normed spaces (see the proof of 363H).
The theorems of \S368 will show, among other things, that a
similar approach (describing $L^0$ as a special kind of extension of $S$
or $L^{\infty}$) can be made to work in the present situation.
I have chosen, however, an alternative route using a novel technique.
The price is the time required to develop skill in the technique, and to
relate it to the earlier approach (364C, 364D, 364J).   The reward is a
construction which is based directly on the algebra $\frak A$,
independent of any representation (364A), and methods of dealing with it
which are complementary to those of the previous three sections.
In particular, they can be used in the absence of the full axiom of
choice (364Ya).

I have deliberately chosen the notation $\Bvalue{u>\alpha}$ from the
theory of forcing.   I do not propose to try to explain myself here, but
I remark that much of the labour of this section is a necessary basis
for understanding real analysis in Boolean-valued models of set theory.
The idea is that just as a function $f:X\to\Bbb R$ can be described in
terms of the sets $\{x:f(x)>\alpha\}$, so can an element $u$ of
$L^0(\frak A)$ be described in terms of the regions $\Bvalue{u>\alpha}$
of $\frak A$ where in some sense $u$ is greater than $\alpha$.   This
description is well adapted to discussion of the order struction of
$L^0(\frak A)$ (see 364L-364M), but rather ill-adapted to discussion of
its
linear and multiplicative structures, which leads to a large part of the
length of the work above.   Once we have succeeded in describing the
algebraic operations on $L^0$ in terms of the values of
$\Bvalue{u>\alpha}$, however, as in 364D, the fundamental result on the
action of Boolean homomorphisms (364P) is elegant and reasonably
straightforward.

The concept `$\Bvalue{u>\alpha}$' can be dramatically generalized to
the concept `$\Bvalue{(u_1,\ldots,u_n)\in E}$', where $E$ is a Borel
subset of $\BbbR^n$ and $u_1,\ldots,u_n\in L^0(\frak A)$ (364G,
364Yb).   This is supposed to recall the notation $\Pr(X\in E)$,
already used in Chapter 27.   If, as sometimes seems reasonable, we wish
to regard a random variable as a member of $L^0(\mu)$ rather than of
$\eusm L^0(\mu)$, then `$\Bvalue{u\in E}$' is the appropriate
translation of `$X^{-1}[E]$'.   The reasons why we can reach all Borel
sets $E$ here,
but then have to stop, seem to lie fairly deep;  I will return to this
question in 566O in Volume 5.
We see that we have
here another potential definition of $L^0(\frak A)$, as the set of
sequentially order-continuous Boolean homomorphisms from the Borel
$\sigma$-algebra of $\Bbb R$ to $\frak A$.   This is suitably
independent of realizations of $\frak A$, but makes the $f$-algebra
structure of $L^0$ difficult to elucidate, unless we move to a further
level of abstraction in the definitions, as in 364Yd.

I take the space to describe the $L^0$ spaces of general regular open
algebras in detail (364T) partly to offer a demonstration of an
appropriate technique, and partly to show that we are not limited to
$\sigma$-algebras of sets and their quotients.   This really is a new
representation;  for instance, it does not meld in any straightforward
way with the constructions of 364F-364H.   Of course the most
important examples are compact Hausdorff spaces, for which alternative
methods are available (364U-364V, 364Yj, 364Yl, 364Ym);  from the point
of view of applications, indeed, it is worth working through the details of
the theory for compact
Hausdorff spaces (364Yk).   The version in 364V is derived
from {\smc Vulikh 67}.   But I have starred everything from 364S on,
because I shall not rely on this work later for anything essential.
%368G uses 364V
}%end of comment

\discrpage

