\frfilename{mt263.tex}
\versiondate{4.4.13}
\copyrightdate{2000}

\def\diam{\mathop{\text{diam}}}
\def\dist{\mathop{\text{dist}}}

\def\chaptername{Change of variable in the integral}
\def\sectionname{Differentiable transformations in $\BbbR^r$}

\newsection{263}

\def\headlinesectionname{Differentiable transformations in
$\eightBbb R^r$}

This section is devoted to the proof of a single major theorem
(263D) concerning differentiable transformations between subsets of
$\BbbR^r$.   There will be a generalization of this result in \S265,
and those with some familiarity with the topic, or sufficient hardihood,
may wish to read \S264 before taking this section and \S265
together.   I end with a few simple corollaries and an extension of the
main result which can be made in the one-dimensional case (263J).

Throughout this section, as in the rest of the chapter, $\mu$ will
denote Lebesgue measure on $\BbbR^r$.

\leader{263A}{Linear \dvrocolon{transformations}}\cmmnt{ I begin with
the special
case of linear operators, which is not only the basis of the proof of
263D, but is also one of its most important
applications, and is indeed sufficient for many very striking results.

\medskip

\noindent}{\bf Theorem} Let $T$ be a real $r\times r$ matrix;  regard
$T$ as a
linear operator from $\BbbR^r$ to itself.   Let $J=|\det T|$ be the
modulus of its determinant.   Then

\Centerline{$\mu T[E]=J\mu E$}

\noindent for every measurable set $E\subseteq\BbbR^r$.   If
$T$ is a permutation (that is, if $J\ne 0$), then

\Centerline{$\mu F=J\mu(T^{-1}[F])$}

\noindent for every measurable $F\subseteq\BbbR^r$, and

\Centerline{$\int_F g\,d\mu=J\int_{T^{-1}[F]} gT\,d\mu$}

\noindent for every integrable function $g$ and measurable set $F$.

\proof{{\bf (a)} The first step is to show that $T[I]$ is
measurable for every half-open interval $I\subseteq\BbbR^r$.   \Prf\
Any non-empty half-open interval
$I=\coint{a,b}$ is a countable union of closed
intervals $I_n =[a,b-2^{-n}\tbf{1}]$, and each $I_n$ is compact (2A2F),
so that $T[I_n]$ is compact (2A2Eb), therefore closed (2A2Ec),
therefore measurable (115G), and $T[I]=\bigcup_{n\in\Bbb N}T[I_n]$
is measurable.\ \Qed

\medskip

{\bf (b)} Set $J^*=\mu T[\,\coint{\tbf{0},\tbf{1}}\,]$, where
$\tbf{0}=(0,\dots,0)$ and $\tbf{1}=(1,\ldots,1)$;  because
$T[\,\coint{\tbf{0},\tbf{1}}\,]$ is bounded, $J^*<\infty$.   (I will
eventually show that $J^*=J$.)   It is convenient to deal with the case
of singular $T$ first.   Recall that $T$, regarded as a linear
transformation from $\BbbR^r$ to itself, is either bijective or onto a
proper linear subspace.   In the latter case, take any $e\in\Bbb
R^r\setminus T[\BbbR^r]$;  then the sets

\Centerline{$T[\,\coint{\tbf{0},\tbf{1}}\,]+\gamma e$,}

\noindent as $\gamma$ runs over $[0,1]$, are disjoint and all of the
same measure $J^*$, because $\mu$ is translation-invariant (134A);
moreover, their union is bounded, so has finite outer measure.   As
there are infinitely many such $\gamma$, the common measure $J^*$ must
be zero.   Now observe that

\Centerline{$T[\BbbR^r]=\bigcup_{z\in \Bbb
Z^r}T[\,\coint{\tbf{0},\tbf{1}}\,]+Tz$,}

\noindent and

\Centerline{$\mu(T[\,\coint{\tbf{0},\tbf{1}}\,]+Tz)=J^*=0$}

\noindent for every $z\in\Bbb Z^r$, while $\Bbb Z^r$ is countable, so
$\mu T[\Bbb R^r]=0$.   At the same time, because
$T$ is singular, it has zero determinant, and $J=0$.
Accordingly

\Centerline{$\mu T[E]=0=J\mu E$}

\noindent for every measurable $E\subseteq\BbbR^r$, and we're done.

\medskip

{\bf (c)} Henceforth, therefore, let us assume that $T$ is
non-singular.
Note that it and its inverse are continuous, so that $T$ is a
homeomorphism, and $T[G]$ is open iff $G$ is open.

If $a\in\BbbR^r$ and $k\in\Bbb N$, then

\Centerline{$\mu T[\,\coint{a,a+2^{-k}\tbf{1}}\,]=2^{-kr}J^*$.}

\noindent   \Prf\ Set
$J^*_k=\mu T[\,\coint{\tbf{0},2^{-k}\tbf{1}}\,]$.   Now
$T[\,\coint{a,a+2^{-k}\tbf{1}}\,]
=T[\,\coint{\tbf{0},2^{-k}\tbf{1}}\,]+Ta$;  because $\mu$ is
translation-invariant, its measure also is $J^*_k$.   Next,
$\coint{\tbf{0},\tbf{1}}$ is expressible as a disjoint union of $2^{kr}$
sets of the form $\coint{a,a+2^{-k}\tbf{1}}$;  consequently,
$T[\,\coint{\tbf{0},\tbf{1}}\,]$ is expressible as a disjoint union of
$2^{kr}$ sets of the form $T[\,\coint{a,a+2^{-k}\tbf{1}}\,]$, and


\Centerline{$J^*=\mu T[\,\coint{\tbf{0},\tbf{1}}\,]=2^{kr}J^*_k$,}

\noindent that is,
$J^*_k=2^{-kr}J^*$, as claimed.\ \Qed

\medskip

{\bf (d)} Consequently $\mu T[G]=J^*\mu G$ for every open set
$G\subseteq\BbbR^r$.   \Prf\ For each $k\in\Bbb N$, set

\Centerline{$Q_k
=\{z:z\in\Bbb Z^r,\,\coint{2^{-k}z,2^{-k}z+2^{-k}\tbf{1}}\subseteq G\}$,}

\Centerline{$G_k=\bigcup_{z\in Q_k}
\coint{2^{-k}z,2^{-k}z+2^{-k}\tbf{1}}$.}

\noindent Then $G_k$ is a disjoint union of $\#(Q_k)$ sets of the form
$\coint{2^{-k}z,2^{-k}z+2^{-k}\tbf{1}}$, so $\mu G_k=2^{-kr}\#(Q_k)$;
also, $T[G_k]$ is a disjoint union of $\#(Q_k)$ sets of the form
$T[\,\coint{2^{-k}z,2^{-k}z+2^{-k}\tbf{1}}\,]$, so has measure
$2^{-kr}J^*\#(Q_k)=J^*\mu G_k$, using (c).

Observe next that $\sequence{k}{G_k}$ is a non-decreasing sequence with
union $G$, so that

\Centerline{$\mu T[G]=\lim_{k\to\infty}\mu T[G_k]
=\lim_{k\to\infty}J^*\mu G_k
=J^*\mu G$.   \Qed}

\medskip

{\bf (e)} It follows that $\mu^*T[A]=J^*\mu^* A$ for every
$A\subseteq\Bbb R^r$.   \Prf\ Given $A\subseteq\BbbR^r$ and
$\epsilon>0$, there are
open sets $G$, $H$ such that $G\supseteq A$, $H\supseteq T[A]$,
$\mu G\le\mu^* A+\epsilon$ and $\mu H\le\mu^*T[A]+\epsilon$ (134Fa).
Set $G_1=G\cap T^{-1}[H]$;  then $G_1$ is open because $T^{-1}[H]$
is.   Now $\mu T[G_1]=J^*\mu G_1$, so

$$\eqalign{\mu^* T[A]
&\le\mu T[G_1]
=J^*\mu G_1
\le J^*\mu^* A+J^*\epsilon\cr
&\le J^*\mu G_1+J^*\epsilon
=\mu T[G_1]+J^*\epsilon
\le\mu H+J^*\epsilon\cr
&\le\mu^*T[A]+\epsilon+J^*\epsilon.\cr}$$

\noindent As $\epsilon$ is arbitrary, $\mu^*T[A]=J^*\mu^* A$.\ \Qed

\medskip

{\bf (f)} Consequently $\mu T[E]$ exists and is equal to $J^*\mu E$
for every measurable $E\subseteq\BbbR^r$.   \Prf\ Let
$E\subseteq\BbbR^r$ be measurable, and take any
$A\subseteq\BbbR^r$.   Set $A'=T^{-1}[A]$.   Then

$$\eqalign{\mu^*(A\cap T[E])+\mu^*(A\setminus T[E])
&=\mu^*(T[A'\cap E])+\mu^*(T[A'\setminus E])\cr
&=J^*(\mu^*(A'\cap E)+\mu^*(A'\setminus E))\cr
&=J^*\mu^* A'
=\mu^*T[A']
=\mu^* A.\cr}$$

\noindent As $A$ is arbitrary, $T[E]$ is measurable, and now


\Centerline{$\mu T[E]=\mu^*T[E]=J^*\mu^* E=J^*\mu E$.  \Qed}

\medskip

{\bf (g)} We are at last ready for the calculation of $J^*$.
Recall that the matrix $T$ must be expressible as $PDQ$, where $P$ and
$Q$ are orthogonal matrices and $D$ is diagonal, with non-negative
diagonal entries (2A6C).  Now we must have

\Centerline{$T[\,\coint{\tbf{0},\tbf{1}}\,]
=P[D[Q[\,\coint{\tbf{0},\tbf{1}}\,]]]$,}

\noindent so, using (f),

\Centerline{$J^*=J^*_PJ^*_DJ^*_Q$,}

\noindent where $J^*_P=\mu P[\,\coint{\tbf{0},\tbf{1}}\,]$, etc.
Now
we find that $J^*_P=J^*_Q=1$.   \Prf\ Let $B=B(\tbf{0},1)$ be the unit
ball of $\BbbR^r$.    Because $B$ is closed, it is measurable;  because
it is
bounded, $\mu B<\infty$;  and because $B$ includes the non-empty
half-open interval $\coint{\tbf{0},r^{-1/2}\tbf{1}}$, $\mu B>0$.   Now
$P[B]=Q[B]=B$, because $P$ and $Q$ are orthogonal matrices;
so we have

\Centerline{$\mu B=\mu P[B]=J^*_P\mu B$,}

\noindent and $J^*_P$ must be $1$;  similarly, $J^*_Q=1$.\ \Qed

\medskip

{\bf (h)} So we have only to calculate $J^*_D$.   Suppose the
coefficients of $D$ are $\delta_1,\ldots,\delta_r\ge 0$, so that
$Dx=(\delta_1\xi_1,\ldots,\delta_r\xi_r)\penalty-100=d\times x$.   We have
been assuming since the beginning of (c) that $T$ is non-singular, so
no $\delta_i$ can be $0$.   Accordingly

\Centerline{$D[\,\coint{\tbf{0},\tbf{1}}\,]
=\coint{\tbf{0},d}$,}

\noindent and

\Centerline{$J^*_D=\mu\coint{\tbf{0},d}=\prod_{i=1}^r\delta_i
=\det D$.}

\noindent Now because $P$ and $Q$ are orthogonal, both have determinant
$\pm 1$, so $\det T=\pm\det D$ and $J^*=\pm\det T$;  because $J^*$ is
surely non-negative, $J^*=|\det T|=J$.

\medskip

{\bf (i)} Thus $\mu T[E]=J\mu E$ for every Lebesgue measurable
$E\subseteq\BbbR^r$.   As $T$ is non-singular, we may use the
above argument to show
that $T^{-1}[F]$ is measurable for every measurable $F$, and

\Centerline{$\mu F=\mu T[T^{-1}[F]]=J\mu T^{-1}[F]
=\int J\times\chi(T^{-1}[F])\,d\mu$,}

\noindent identifying $J$ with the constant function with value $J$.
By 235A,

\Centerline{$\int_F g\,d\mu=\int_{T^{-1}[F]}JgT\,d\mu
=J\int_{T^{-1}[F]}gT\,d\mu$}

\noindent for every integrable function $g$ and measurable set $F$.
}%end of proof of 263A

\cmmnt{
\leader{263B}{Remark} Perhaps I should have warned you that I
should be calling on the results of \S235.   But if they were fresh in
your mind the formulae of the statement of the theorem will have
recalled them, and if not then it is perhaps better to turn back to them
now rather than before reading the theorem, since they are used only in
the last sentence of the proof.

I have taken the argument above at a leisurely,
not to say pedestrian, pace.   The point is that while the
translation-invariance of Lebesgue measure, and its behaviour under
simple magnification of a single coordinate, are more or less built into
the definition, its behaviour under general rotations is not, since a
rotation takes half-open intervals into skew cuboids.   Of
course the calculation of the measure of such an object is not really
anything to do with the Lebesgue theory, and it will be clear that much
of the argument would apply equally to any geometrically reasonable
notion of $r$-dimensional volume.

We come now to the central result of the chapter.  We have already done
some of the detail work in 262M.   The next basic element is the
following lemma.
}%end of comment

\leader{263C}{Lemma} Let $T$ be a real $r\times r$ matrix;  set
$J=|\det T|$.
Then for any $\epsilon>0$ there is a $\zeta=\zeta(T,\epsilon)>0$ such
that

(i) $|\det S-\det T|\le\epsilon$ whenever $S$ is an $r\times r$ matrix
and $\|S-T\|\le\zeta$;

(ii) whenever $D\subseteq\BbbR^r$ is a bounded set and
$\phi:D\to\BbbR^r$ is a function such that
$\|\phi(x)-\phi(y)-T(x-y)\|\le\zeta\|x-y\|$ for all $x$, $y\in D$, then
$|\mu^*\phi[D]-J\mu^*D|\le\epsilon\mu^*D$.

\proof{{\bf (a)} Of course (i) is the easy part.   Because
$\det S$ is a continuous function of the coefficients of $S$, and the
coefficients of $S$ must be close to those of $T$ if $\|S-T\|$ is small
(262Hb), there is surely a $\zeta_0>0$ such that
$|\det S-\det T|\le\epsilon$ whenever $\|S-T\|\le\zeta_0$.

\medskip

{\bf (b)(i)} Write $B=B(\tbf{0},1)$ for the unit ball of $\BbbR^r$, and
consider $T[B]$.      We know that $\mu T[B]=J\mu B$ (263A).    Let
$G\supseteq T[B]$ be an open set such that $\mu G\le (J+\epsilon)\mu B$
(134Fa again).   Because $B$ is compact (2A2F again) so is $T[B]$, so
there is a $\zeta_1>0$ such that $T[B]+\zeta_1B\subseteq G$ (2A2Ed).
This means that $\mu^*(T[B]+\zeta_1B)\le(J+\epsilon)\mu B$.

\medskip

\quad{\bf (ii)}
Now suppose that $D\subseteq\BbbR^r$ is a bounded set, and that
$\phi:D\to\BbbR^r$ is a function such that
$\|\phi(x)-\phi(y)-T(x-y)\|\le\zeta_1\|x-y\|$ for all $x$, $y\in D$.
Then if $x\in D$ and $\delta>0$,

\Centerline{$\phi[D\cap B(x,\delta)]\subseteq \phi(x) + \delta T[B] +
\delta\zeta_1B$,}

\noindent because if $y\in D\cap B(x,\delta)$ then $T(y-x)\in\delta
T[B]$ and

$$\eqalign{\phi(y)
&= \phi(x) + T(y-x) + (\phi(y)-\phi(x)-T(y-x))\cr
&\in\phi(x) + \delta T[B]+\zeta_1\|y-x\|B
\subseteq\phi(x)+\delta T[B]+\zeta_1\delta B.\cr}$$

\noindent Accordingly

$$\eqalign{\mu^*\phi[D\cap B(x,\delta)]
&\le\mu^*(\delta T[B]+\delta\zeta_1B)
=\delta^r\mu^*(T[B]+\zeta_1B)\cr
&\le\delta^r(J+\epsilon)\mu B
=(J+\epsilon)\mu B(x,\delta).\cr}$$

Let $\eta>0$.   Then there is a sequence $\sequencen{B_n}$ of balls in
$\BbbR^r$ such that $D\subseteq\bigcup_{n\in\Bbb N}B_n$,
$\sum_{n=0}^{\infty}\mu B_n\le\mu^*D+\eta$ and the sum of the measures
of those $B_n$ whose centres do not lie in $D$ is at most $\eta$ (261F).
Let $K$ be the set of those $n$ such that the centre of $B_n$ lies in
$D$.   Then $\mu^*\phi[D\cap B_n]\le(J+\epsilon)\mu B_n$ for every
$n\in K$.    Also, of course, $\phi$ is $(\|T\|+\zeta_1)$-Lipschitz, so
$\mu^*\phi[D\cap B_n]\le(\|T\|+\zeta_1)^r\mu B_n$
for $n\in\Bbb N\setminus K$ (262D).   Now

$$\eqalign{\mu^*\phi[D]
&\le\sum_{n=0}^{\infty}\mu^*\phi[D\cap B_n]\cr
&\le\sum_{n\in K}(J+\epsilon)\mu B_n
   +\sum_{n\in\Bbb N\setminus K}(\|T\|+\zeta_1)^r\mu B_n\cr
&\le(J+\epsilon)(\mu^*D+\eta)+\eta(\|T\|+\zeta_1)^r.\cr}$$

\noindent As $\eta$ is arbitrary,

\Centerline{$\mu^*\phi[D]\le (J+\epsilon)\mu^*D$.}

\medskip

{\bf (c)} If $J=0$, we can stop here, setting
$\zeta=\min(\zeta_0,\zeta_1)$;  for then we surely have $|\det S-\det
T|\le\epsilon$ whenever $\|S-T\|\le\zeta$, while if $\phi:D\to\Bbb
R^r$ is such that $\|\phi(x)-\phi(y)-T(x-y)\|\le\zeta\|x-y\|$ for all
$x$, $y\in D$, then

\Centerline{$|\mu^*\phi[D]-J\mu^*D|=\mu^*\phi[D]\le\epsilon\mu^*D$.}

\noindent If $J\ne 0$, we have more to do.   Because $T$ has non-zero
determinant, it has an inverse $T^{-1}$, and $|\det T^{-1}|=J^{-1}$.
As in (b-i) above, there is a $\zeta_2>0$ such that
$\mu^*(T^{-1}[B]+\zeta_2B)\le(J^{-1}+\epsilon')\mu B$, where
$\epsilon'=\epsilon/J(J+\epsilon)$.   Repeating (b), we see that if
$C\subseteq\Bbb R^r$ is bounded and $\psi:C\to\BbbR^r$ is such that
$\|\psi(u)-\psi(v)-T^{-1}(u-v)\|\le\zeta_2\|u-v\|$ for all $u$,
$v\in C$, then $\mu^*\psi[C]\le(J^{-1}+\epsilon')\mu^*C$.

Now suppose that
$D\subseteq\BbbR^r$ is bounded and $\phi:D\to\BbbR^r$ is such that
$\|\phi(x)-\phi(y)-T(x-y)\|\le\zeta_2' \|x-y\|$ for all $x$, $y\in D$,
where $\zeta_2'=\min(\zeta_2,\|T^{-1}\|)/2\|T^{-1}\|^2>0$.   Then

\Centerline{$\|T^{-1}(\phi(x)-\phi(y))-(x-y)\|
\le\|T^{-1}\|\zeta_2'\|x-y\|\le\Bover12\|x-y\|$}

\noindent for all $x$, $y\in D$, so $\phi$ must be injective;  set
$C=\phi[D]$ and $\psi=\phi^{-1}:C\to D$.   Note that $C$ is bounded,
because

\Centerline{$\|\phi(x)-\phi(y)\|\le(\|T\|+\zeta_2')\|x-y\|$}

\noindent whenever $x$, $y\in D$.   Also

\Centerline{$\|T^{-1}(u-v)-(\psi(u)-\psi(v))\|
\le\|T^{-1}\|\zeta_2'\|\psi(u)-\psi(v)\|\le\Bover12\|\psi(u)-\psi(v)\|$}

\noindent for all $u$, $v\in C$.   But this means that

\Centerline{$\|\psi(u)-\psi(v)\|-\|T^{-1}\|\|u-v\|
\le\Bover12\|\psi(u)-\psi(v)\|$}

\noindent and $\|\psi(u)-\psi(v)\|\le 2\|T^{-1}\|\|u-v\|$ for all $u$,
$v\in C$, so that

\Centerline{$\|\psi(u)-\psi(v)-T^{-1}(u-v)\|
\le 2\zeta_2'\|T^{-1}\|^2\|u-v\|\le\zeta_2\|u-v\|$}

\noindent for all $u$, $v\in C$.

It follows that

\Centerline{$\mu^*D=\mu^*\psi[C]\le(J^{-1}+\epsilon')\mu^*C
=(J^{-1}+\epsilon')\mu^*\phi[D]$,}

\noindent and

\Centerline{$J\mu^*D\le (1+J\epsilon')\mu^*\phi[D]$.}

\medskip

{\bf (d)} So if we set $\zeta=\min(\zeta_0,\zeta_1,\zeta_2')>0$, and if
$D\subseteq\BbbR^r$, $\phi:D\to\BbbR^r$ are such that  $D$ is bounded
and
$\|\phi(x)-\phi(y)-T(x-y)\|\le\zeta\|x-y\|$ for all $x$, $y\in D$, we
shall have

\Centerline{$\mu^*\phi[D]\le(J+\epsilon)\mu^*D$,}

\Centerline{$\mu^*\phi[D]\ge J\mu^*D-J\epsilon'\mu^*\phi[D]
\ge J\mu^*D-J\epsilon'(J+\epsilon)\mu^*D
= J\mu^*D-\epsilon\mu^*D$,}

\noindent so we get the required formula

\Centerline{$|\mu^*\phi[D]-J\mu^*D|\le\epsilon\mu^*D$.}
}%end of proof of 263C

\leader{263D}{}\cmmnt{ We are ready for the theorem.

\medskip

\noindent}{\bf Theorem} Let $D\subseteq\BbbR^r$ be any set, and
$\phi:D\to\BbbR^r$ a function differentiable relative to its domain at
each point of $D$.   For each $x\in D$ let $T(x)$ be a derivative of
$\phi$ relative to $D$ at $x$, and set $J(x)=|\det T(x)|$.   Then

\quad (i) $J:D\to\coint{0,\infty}$ is a measurable function,

\quad (ii) $\mu^*\phi[D]\le\int_DJ\,d\mu$,

\noindent allowing $\infty$ as the value of the integral.  If $D$ is
measurable, then

\quad (iii) $\phi[D]$ is measurable.

\noindent If $D$ is measurable and $\phi$ is injective, then

\quad (iv) $\mu\phi[D]=\int_DJ\,d\mu$,

\quad (v) for every real-valued function $g$ defined on a subset of
$\phi[D]$,

\Centerline{$\int_{\phi[D]}g\,d\mu=\int_{D}J\times g\phi\,d\mu$}

\noindent if either integral is defined in $[-\infty,\infty]$, provided
we interpret $J(x)g(\phi(x))$ as zero when $J(x)=0$ and $g(\phi(x))$ is
undefined.

\proof{{\bf (a)} To see that $J$ is measurable, use 262P;  the function
$T\mapsto|\det T|$ is a continuous function of the coefficients of $T$,
and the coefficients of $T(x)$ are measurable functions of $x$, by 262P,
so $x\mapsto|\det T(x)|$ is measurable (121K).   We also know that if
$D$ is measurable, $\phi[D]$ will be measurable, by 262Ob.   Thus (i)
and (iii) are done.

\medskip

{\bf (b)} For the moment, assume
that $D$ is bounded, and fix $\epsilon>0$.    For $r\times r$ matrices
$T$, take $\zeta(T,\epsilon)>0$ as in 263C.   Take $\sequencen{D_n}$,
$\sequencen{T_n}$ as in 262M, so that $\sequencen{D_n}$ is a disjoint
cover of $D$ by sets which are relatively measurable in $D$, and each
$T_n$ is an $r\times r$ matrix such that

\Centerline{$\|T(x)-T_n\|\le\zeta(T_n,\epsilon)$ whenever $x\in D_n$,}

\Centerline{$\|\phi(x)-\phi(y)-T_n(x-y)\|\le\zeta(T_n,\epsilon)\|x-y\|$
for all $x$, $y\in D_n$.}

\noindent Then, setting $J_n=|\det T_n|$, we have

\Centerline{$|J(x)-J_n|\le\epsilon$ for every $x\in D_n$,}

\Centerline{$|\mu^*\phi[D_n]-J_n\mu^*D_n|\le\epsilon\mu^*D_n$,}

\noindent by the choice of $\zeta(T_n,\epsilon)$.   So we have

\Centerline{$\int_DJ\,d\mu
\le\sum_{n=0}^{\infty}J_n\mu^*D_n+\epsilon\mu^*D
\le\int_DJ\,d\mu+2\epsilon\mu^*D$;}

\noindent I am using here the fact that all the $D_n$ are relatively
measurable in $D$, so that, in particular,
$\mu^*D=\sum_{n=0}^{\infty}\mu^*D_n$.   Next,

\Centerline{$\mu^*\phi[D]\le\sum_{n=0}^{\infty}\mu^*\phi[D_n]
\le\sum_{n=0}^{\infty}J_n\mu^*D_n+\epsilon\mu^*D$.}

\noindent Putting these together,

\Centerline{$\mu^*\phi[D]\le\int_DJ\,d\mu+2\epsilon\mu^*D$.}

If $D$ is measurable and $\phi$ is injective, then all the $D_n$ are
measurable subsets of $\BbbR^r$, so all the $\phi[D_n]$ are measurable,
and they are also disjoint.   Accordingly

\Centerline{$\int_DJ\,d\mu
\le\sum_{n=0}^{\infty}J_n\mu D_n+\epsilon\mu D
\le\sum_{n=0}^{\infty}(\mu\phi[D_n]+\epsilon\mu D_n)+\epsilon\mu D
=\mu\phi[D]+2\epsilon\mu D$.}

Since $\epsilon$ is arbitrary, we get

\Centerline{$\mu^*\phi[D]\le\int_DJ\,d\mu$,}

\noindent and if $D$ is measurable and $\phi$ is injective,

\Centerline{$\int_DJ\,d\mu\le\mu\phi[D]$;}

\noindent thus we have (ii) and (iv), on the assumption that $D$ is
bounded.

\medskip

{\bf (c)} For a general set $D$, set $B_k=B(\tbf{0},k)$;  then

\Centerline{$\mu^*\phi[D]=\lim_{k\to\infty}\mu^*\phi[D\cap B_k]
\le\lim_{k\to\infty}\int_{D\cap B_k}J\,d\mu
=\int_DJ\,d\mu$,}

\noindent with equality if $\phi$ is injective and $D$ is measurable.

\medskip

{\bf (d)} For part (v), I seek to show that the
hypotheses of 235J are satisfied, taking $X=D$ and $Y=\phi[D]$.   \Prf\
Set $G=\{x:x\in D,\,J(x)>0\}$.

\medskip

\quad\grheada\ If $F\subseteq\phi[D]$ is measurable, then there are
Borel sets $F_1$, $F_2$ such
that $F_1\subseteq F\subseteq F_2$ and $\mu(F_2\setminus F_1)=0$.   Set
$E_j=\phi^{-1}[F_j]$ for each $j$, so that
$E_1\subseteq \phi^{-1}[F]\subseteq E_2$, and both the sets $E_j$ are
measurable,
because $\phi$ and $\dom\phi$ are measurable.   Now, applying (iv) to
$\phi\restr E_j$,

\Centerline{$\int_{E_j}J\,d\mu=\mu\phi[E_j]=\mu(F_j\cap\phi[D])=\mu F$}

\noindent for both $j$, so $\int_{E_2\setminus E_1}J\,d\mu=0$ and $J=0$
a.e.\ on $E_2\setminus E_1$.   Accordingly
$J\times\chi(\phi^{-1}[F])\eae J\times\chi E_1$, and
$\int J\times\chi(\phi^{-1}[F])d\mu$ exists and is equal to
$\int_{E_1}J\,d\mu=\mu F$.   At the same time,
$(\phi^{-1}[F]\cap G)\symmdiff(E_1\cap G)$ is negligible, so
$\phi^{-1}[F]\cap G$ is measurable.

\medskip

\quad\grheadb\ If $F\subseteq\phi[D]$ and $G\cap\phi^{-1}[F]$ is
measurable, then we know that $\mu\phi[D\setminus G]=\int_{D\setminus
G}J=0$ (by (iv) applied to $\phi\restr D\setminus G$),
so $F\setminus\phi[G]$ must be negligible;  while
$F\cap\phi[G]=\phi[G\cap\phi^{-1}[F]]$ also is measurable, by (iii).
Accordingly $F$ is measurable whenever $G\cap\phi^{-1}[F]$ is
measurable.

\medskip

Thus all the hypotheses of 235J are satisfied.\ \QeD\   Now (v) can be
read off from the conclusion of 235J.
}%end of proof of 263D

\cmmnt{
\leader{263E}{Remarks (a)} This is a version of the classical result on
change of
variable in a many-dimensional integral.   What I here call $J(x)$ is
the {\bf Jacobian} of $\phi$ at $x$;  it describes the change in volumes
of objects near $x$, following the rule already established in 263A for
functions with constant derivative.   The idea of the
proof is also the classical one:  to break the set $D$ up into small
enough pieces $D_m$ for us to be able to approximate $\phi$ by affine
operators $y\mapsto \phi(x) + T_{m}(y-x)$ on each.   The potential
irregularity of the set $D$, which in this theorem may be any set, is
compensated for by a corresponding freedom in
choosing the sets $D_m$.   In fact there is a further decomposition of
the sets $D_m$ hidden in part (b-ii) of the proof of 263C;  each $D_m$
is essentially covered by a disjoint family of balls, the measures of
whose images we can estimate with an adequate accuracy.   There is
always a danger of a negligible exceptional set, and we need the crude
inequalities of the proof of 262D to deal with it.

\header{263Eb}{\bf (b)} Throughout the work of this chapter, from 261B
to 263D, I have chosen
balls $B(x,\delta)$ as the basic shapes to work with.   I think it
should be clear that in fact any reasonable shapes would do just as
well.    In particular, the `balls'

\Centerline{$B_1(x,\delta)
=\{y:\sum_{i=1}^r|\eta_i-\xi_i|\le\delta\}$,\quad
$B_{\infty}(x,\delta)=\{y:|\eta_i-\xi_i|\le\delta\Forall
i\}$}

\noindent would serve perfectly.   There are many alternatives.   We
could use sets of the form $C(x,k)$, for $x\in\BbbR^r$ and
$k\in\Bbb N$, defined to be the half-open cube of the form
$\coint{2^{-k}z,2^{-k}(z+\tbf{1})}$ with $z\in\Bbb Z^r$ containing $x$,
instead;  or even $C'(x,\delta)=\coint{x,x+\delta\tbf{1}}$.   In all
such cases we have versions of the density theorems (261Yb-261Yc) which
support the remaining theory.


\header{263Ec}{\bf (c)} I have presented 263D as a theorem about
differentiable functions, because that is the normal form in which one
uses it in
elementary applications.   However, the proof depends essentially on the
fact that a differentiable function is a countable union of Lipschitz
functions, and 263D would follow at once from the same theorem
proved for Lipschitz functions only.   Now the fact is that the theorem
applies to {\it any} countable union of Lipschitz functions,
because a Lipschitz function is differentiable almost everywhere.   For
more advanced work (see {\smc Federer 69} or {\smc Evans \& Gariepy 92},
or Chapter 47 in Volume 4) it seems clear that Lipschitz
functions are the vital ones, so I spell out the result.
}%end of comment

\leader{*263F}{Corollary} Let $D\subseteq\BbbR^r$ be any set and
$\phi:D\to\BbbR^r$ a Lipschitz function.   Let $D_1$ be the set of
points at which $\phi$ has a derivative relative to $D$, and for each
$x\in D_1$ let $T(x)$ be such a derivative, with $J(x)=|\det T(x)|$.
Then

\quad (i) $D\setminus D_1$ is negligible;

\quad (ii) $J:D_1\to\coint{0,\infty}$ is measurable;

\quad (iii) $\mu^*\phi[D]\le\int_DJ(x)dx$.

\noindent If $D$ is measurable, then

\quad (iv) $\phi[D]$ is measurable.

\noindent If $D$ is measurable and $\phi$ is injective, then

\quad (v) $\mu\phi[D]=\int_DJ\,d\mu$,

\quad (vi) for every real-valued function $g$ defined on a subset of
$\phi[D]$,

\Centerline{$\int_{\phi[D]}g\,d\mu=\int_{D}J\times g\phi\,d\mu$}

\noindent if either integral is defined in $[-\infty,\infty]$, provided
we interpret $J(x)g(\phi(x))$ as zero when $J(x)=0$ and $g(\phi(x))$ is
undefined.

\proof{ This is now just a matter of putting 262Q and 263D
together, with a little help from 262D.   Use 262Q to show that
$D\setminus D_1$ is negligible, 262D to show that $\phi[D\setminus D_1]$
is negligible, and apply 263D to $\phi\restr D_1$.
}%end of proof of 263F

\dvro{\leader{263G}{Proposition}
$\int_{-\infty}^{\infty}e^{-t^2/2}dt=\sqrt{2\pi}$.}
{\leader{263G}{Polar coordinates in the plane} I offer an elementary
example with a useful consequence.   Define $\phi:\BbbR^2\to\BbbR^2$
by setting $\phi(\rho,\theta)=(\rho\cos\theta,\rho\sin\theta)$ for
$\rho$, $\theta\in\BbbR^2$.   Then $\phi'(\rho,\theta)
=\Matrix{\cos\theta&-\rho\sin\theta\\ \sin\theta&\rho\cos\theta}$, so
$J(\rho,\theta)=|\rho|$ for all $\rho$, $\theta$.   Of course $\phi$ is
not injective, but if we restrict it to the domain
$D=\{(0,0)\}\cup\{(\rho,\theta):\rho>0,\,-\pi<\theta\le\pi\}$ then
$\phi\restr D$ is a bijection between $D$ and $\BbbR^2$, and

\Centerline{$\int g\,d\xi_1d\xi_2=\int_Dg(\phi(\rho,\theta))\rho\,d\rho
d\theta$}

\noindent for every real-valued function $g$ which is integrable over
$\BbbR^2$.

Suppose, in particular, that we set

\Centerline{$g(x)=e^{-\|x\|^2/2}
=e^{-\xi_1^2/2}e^{-\xi_2^2/2}$}

\noindent for $x=(\xi_1,\xi_2)\in\Bbb R$.   Then

\Centerline{$\int g(x)dx=\int e^{-\xi_1^2/2}d\xi_1
\int e^{-\xi_2^2/2}d\xi_2$,}

\noindent as in 253D.   Setting $I=\int e^{-t^2/2}dt$, we have
$\int g=I^2$.   (To see that $I$ is well-defined in $\Bbb R$, note that
the integrand is continuous, therefore measurable, and that

\Centerline{$\int_{-1}^1e^{-t^2/2}dt\le 2$,}

\Centerline{$\int_{-\infty}^{-1}e^{-t^2/2}dt
=\int_1^{\infty}e^{-t^2/2}dt
\le\int_1^{\infty}e^{-t/2}dt
=\lim_{a\to\infty}\int_1^ae^{-t/2}dt
=\Bover12e^{-1/2}$}

\noindent are both finite.)   Now looking at the alternative expression
we have

$$\eqalignno{I^2
&=\int g(x)dx
=\int_Dg(\rho\cos\theta,\rho\sin\theta)\rho\,d(\rho,\theta)\cr
&=\int_De^{-\rho^2/2}\rho\,d(\rho,\theta)
=\int_0^{\infty}\int_{-\pi}^{\pi}\rho e^{-\rho^2/2}d\theta d\rho\cr
\noalign{\noindent (ignoring the point $(0,0)$, which has zero measure)}
&=\int_0^{\infty}2\pi\rho e^{-\rho^2/2}d\rho
=2\pi\lim_{a\to\infty}\int_0^a\rho e^{-\rho^2/2}d\rho\cr
&=2\pi\lim_{a\to\infty}(-e^{-a^2/2}+1)
=2\pi.\cr}$$

\noindent Consequently

\Centerline{$\int_{-\infty}^{\infty}e^{-t^2/2}dt=I=\sqrt{2\pi}$,}

\noindent which is one of the many facts every mathematician should
know, and in particular is vital for Chapter 27 below.
}%end of dvro

\leader{263H}{Corollary} If $k\in\Bbb N$ is odd,

\Centerline{$\int_{-\infty}^{\infty}x^ke^{-x^2/2}dx=0$;}

\noindent if $k=2l\in\Bbb N$ is even, then

\Centerline{$\int_{-\infty}^{\infty}x^ke^{-x^2/2}dx
=\Bover{(2l)!}{2^ll!}\sqrt{2\pi}$.}

\proof{{\bf (a)} To see that all the integrals are well-defined and
finite, observe that
$\lim_{x\to\pm\infty}x^ke^{-x^2/4}=0$, so that
$M_k=\sup_{x\in\Bbb R}|x^ke^{-x^2/4}|$ is finite, and

\Centerline{$\int_{-\infty}^{\infty}|x^ke^{-x^2/2}|dx
\le M_k\int_{-\infty}^{\infty}e^{-x^2/4}dx<\infty$.}

\medskip

{\bf (b)} If $k$ is odd, then substituting $y=-x$ we get

\Centerline{$\int_{-\infty}^{\infty}x^ke^{-x^2/2}dx
=-\int_{-\infty}^{\infty}y^ke^{-y^2/2}dy$,}

\noindent so that both integrals must be zero.

\medskip

{\bf (c)} For even $k$, proceed by induction.   Set
$I_l=\int_{-\infty}^{\infty}x^{2l}e^{-x^2/2}dx$.
$I_0=\sqrt{2\pi}=\bover{0!}{2^00!}\sqrt{2\pi}$ by 263G.   For the
inductive step to $l+1\ge 1$, integrate by parts to see that

\Centerline{$\int_{-a}^ax^{2l+1}\cdot xe^{-x^2/2}dx
=-a^{2l+1}e^{-a^2/2}+(-a)^{2l+1}e^{-a^2/2}
  +\int_{-a}^a(2l+1)x^{2l}e^{-x^2/2}dx$}

\noindent for every $a\ge 0$.   Letting $a\to\infty$,

\Centerline{$I_{l+1}=(2l+1)I_l$.}

\noindent Because

\Centerline{$\Bover{(2(l+1))!}{2^{l+1}(l+1)!}\sqrt{2\pi}
=(2l+1)\Bover{(2l)!}{2^ll!}\sqrt{2\pi}$,}

\noindent the induction proceeds.
}%end of proof of 263H

\leader{263I}{}\cmmnt{ The following is a version of 263D for non-injective
transformations.

\medskip

\noindent}{\bf Theorem}\dvAnew{2013}
Let $D\subseteq\BbbR^r$ be a measurable set, and
$\phi:D\to\BbbR^r$ a function differentiable relative to its domain at
each point of $D$.   For each $x\in D$ let $T(x)$ be a derivative of
$\phi$ relative to $D$ at $x$, and set $J(x)=|\det T(x)|$.

(a) Let $\nu$ be counting measure on $\BbbR^r$.   Then
$\int_{\BbbR^r}\nu(\phi^{-1}[\{y\}])dy$ and
$\int_DJ\,d\mu$ are defined in $[0,\infty]$ and equal.

(b) Let $g$ be a real-valued function defined on a subset of $\phi[D]$
such that $\int_Dg(\phi(x))\det T(x)dx$ is defined in $\Bbb R$,
interpreting $g(\phi(x))\det T(x)$ as zero when $\det T(x)=0$ and
$g(\phi(x))$ is undefined.   Set

\Centerline{$C=\{y:y\in\phi[D]$, $\phi^{-1}[\{y\}]$ is finite$\}$,
\quad$R(y)=\sum_{x\in\phi^{-1}[\{y\}]}\sgn\det T(x)$}

\noindent for $y\in C$, where $\sgn(0)=0$ and
$\sgn(\alpha)=\Bover{\alpha}{|\alpha|}$ for non-zero $\alpha$.
If we interpret $g(y)R(y)$ as zero when $g(y)=0$ and $R(y)$ is undefined,
then $\int_{\phi[D]}g\times R\,d\mu$ is defined and equal to
$\int_Dg(\phi(x))\det T(x)dx$.

\proof{{\bf (a)} By 263D(i), $J$ is measurable, so
$\int_DJ\,d\mu$ is defined in $[0,\infty]$ and
$D_0=\{x:x\in D$, $J(x)=0\}$ is measurable.   Applying 263D(ii) to
$\phi\restr D_0$, we see that $\phi[D_0]$ is negligible.

Applying 262M to $\phi\restr D\setminus D_0$,
the set $A$ of non-singular
$r\times r$-matrices and $\zeta(S)=\bover1{2\|S^{-1}\|}$ for $S\in A$,
we have a partition $\sequencen{E_n}$ of $D\setminus D_0$ into measurable
sets and a sequence $\sequencen{T_n}$ in $A$ such that

\Centerline{$\|\phi(x)-\phi(y)-T_n(x-y)\|\le\Bover1{2\|T_n^{-1}\|}\|x-y\|$,
\quad$\|T(x)-T_n\|\le\Bover1{2\|T_n^{-1}\|}$}

\noindent whenever
$n\in\Bbb N$ and $x$, $y\in E_n$.   In this case, for $x$, $y\in E_n$,

\Centerline{$\phi(x)=\phi(y)
\,\Longrightarrow\,\|x-y\|\le\|T_n^{-1}\|\|T_n(x-y)\|
\le\Bover12\|x-y\|
\,\Longrightarrow\, x=y$,}

\noindent so $\phi\restr E_n$ is injective, for each $n$.
Consequently
$\#(\phi^{-1}[\{y\}])=\#(\{n:y\in\phi[E_n]\})$ for
$y\in\phi[D]\setminus\phi[D_0]$, and

$$\eqalignno{\int_{\phi[D]}\nu(\phi^{-1}[\{y\}])dy
&=\int_{\phi[D]\setminus\phi[D_0]}\nu(\phi^{-1}[\{y\}])dy
=\sum_{n=0}^{\infty}\mu(\phi[E_n]\setminus\phi[D_0])\cr
\displaycause{applying 263D(iii) to $\phi\restr E_n$, we know that
$\phi[E_n]$ is measurable for each $n$}
&=\sum_{n=0}^{\infty}\mu\phi[E_n]
=\sum_{n=0}^{\infty}\int_{E_n}J\cr
\displaycause{applying 263D(iv) to $\phi\restr E_n$}
&=\int_{D\setminus D_0}J
=\int_DJ,\cr}$$

\noindent each sum or integral being defined in $[0,\infty]$ because the
next one is.

\medskip

{\bf (b)(i)} Setting $D'=\phi^{-1}[\dom g]$, we see that
$D\setminus(D_0\cup D')$ is negligible, so (using 263D(ii) again)

\Centerline{$\phi[D]\setminus\dom g=\phi[D]\setminus\phi[D']
\subseteq\phi[D_0]\cup\phi[D\setminus(D_0\cup D')]$}

\noindent is negligible.   Next, if we set
$C'=C\cup g^{-1}[\{0\}]$, $\phi[D]\setminus C'$ is negligible.
\Prf\ For each $m\in\Bbb N$, set
$F_m=\{y:y\in\dom g$, $|g(y)|\ge 2^{-m}\}$.   Then

\Centerline{$\phi^{-1}[F_m]\setminus D_0=\{x:x\in D$, $J(x)\ne 0$,
   $g(\phi(x))$ is defined and $|g(\phi(x))|\ge 2^{-m}\}$}

\noindent is measurable (because $J\times g\phi$ is measurable) and
(applying (a) to $\phi\restr\phi^{-1}[F_m]$)

\Centerline{$\int_{F_m}\nu(\phi^{-1}[\{y\}])dy
=\int_{\phi^{-1}[F_m]}J\,d\mu
\le 2^m\int_{\phi^{-1}[F_m]}|J\times g(\phi)|\,d\mu$}

\noindent is finite.   But this means that $\nu(\phi^{-1}[\{y\}])$ must be
finite for almost every $y\in F_m$, that is, that $F_m\setminus C$ is
negligible.   As $m$ is arbitrary, $\dom g\setminus C'$ and
$\phi[D]\setminus C'$ are negligible.\ \Qed

\medskip

\quad{\bf (ii)} Taking $\sequencen{E_n}$ and $\sequencen{T_n}$ as in (a),
set $\epsilon_n=\sgn\det T_n\in\{-1,1\}$ for each $n$.   Then
$\sgn\det T(x)=\epsilon_n$ whenever $n\in\Bbb N$ and $x\in E_n$.   \Prf\
For any $\alpha\in[0,1]$,

\Centerline{$\|(\alpha T(x)+(1-\alpha)T_n)-T_n\|
\le\|T(x)-T_n\|\le\Bover1{2\|T_n^{-1}\|}$,}

\noindent so (using 2A4Fd)
$\|(\alpha T(x)+(1-\alpha)T_n)-I_r\|\le\Bover12$,
where $I_r$ is the $r\times r$ identity matrix,
and $\alpha T(x)+(1-\alpha)T_n$ is non-singular
(since if $(\alpha T(x)+(1-\alpha)T_n)z=0$, then
$\|z\|\le\bover12\|z\|$).   Thus $\det(\alpha T(x)+(1-\alpha)T_n)$ is
non-zero for $0\le\alpha\le 1$.   But as
$\alpha\mapsto\det(\alpha T(x)+(1-\alpha)T_n)$ is continuous,
$\epsilon_n=\sgn\det T_n=\sgn\det T(x)$.\ \Qed

\medskip

\quad{\bf (iii)} Now

$$\eqalignno{\int_Dg(\phi(x))\det T(x)\,dx
&=\int_{D\setminus D_0}g(\phi(x))\det T(x)\,dx\cr
&=\sum_{n=0}^{\infty}\int_{E_n}g(\phi(x))\det T(x)\,dx\cr
\displaycause{in this series of formulae, each sum and integral is
well-defined because the preceding ones are}
&=\sum_{n=0}^{\infty}\epsilon_n\int_{E_n}g(\phi(x))J(x)\,dx
=\sum_{n=0}^{\infty}\epsilon_n\int_{\phi[E_n]}g\,dx\cr
&=\sum_{n=0}^{\infty}\epsilon_n\int_{\phi[E_n]\cap C'}g\,dx
=\sum_{n=0}^{\infty}\epsilon_n\int_{C'}g\times\chi(\phi[E_n])\,dx\cr}$$

\noindent(using 131Fa, if you like, for the last step).
Since we also have

\Centerline{$\infty
>\int_D|g(\phi(x))\det T(x)|\,dx
=\sum_{n=0}^{\infty}\int_{C'}|g|\times\chi(\phi[E_n])\,dx$}

\noindent (going through the same stages with the absolute values of
integrands), we have

$$\eqalignno{\int_Dg(\phi(x))\det T(x)\,dx
&=\sum_{n=0}^{\infty}\epsilon_n\int_{C'}g\times\chi(\phi[E_n])\,dx\cr
&=\int_{C'}g\times\sum_{n=0}^{\infty}\epsilon_n\chi(\phi[E_n])\,dx\cr
&=\int_{C'\setminus\phi[D_0]}
   g\times\sum_{n=0}^{\infty}\epsilon_n\chi(\phi[E_n])\,dx
=\int_{C'\setminus\phi[D_0]}g\times R\,d\mu\cr
\displaycause{because if $y\in C'\setminus\phi[D_0]$, either $g(y)=0$ or
$R(y)$ is defined and equal to
$\sum_{n=0}^{\infty}\epsilon_n\chi(\phi[E_n])(y)$}
&=\int_{\phi[D]}g\times R\,d\mu,\cr}$$

\noindent as claimed.
}%end of proof of 263I

\leader{263J}{The one-dimensional \dvrocolon{case}}\cmmnt{ The
restriction to injective functions $\phi$ in 263D(v) is unavoidable in
the context of the result there.   But in the substitutions of
elementary calculus it is not always essential.   In the hope of
clarifying the position I give a result here which covers many of the
standard tricks.

\medskip

\noindent}{\bf Proposition}
Let $I\subseteq\Bbb R$ be an interval with more
than one point, and $\phi:I\to\Bbb R$ a function which is absolutely
continuous on any closed bounded subinterval of $I$.   Write $u=\inf I$,
$u'=\sup I$ in
$[-\infty,\infty]$, and suppose that $v=\lim_{x\downarrow u}\phi(x)$ and
$v'=\lim_{x\uparrow u'}\phi(x)$ are defined in $[-\infty,\infty]$.   Let
$g$ be a real function such that
$\int_Ig(\phi(x))\phi'(x)dx$ is defined,
on the understanding that we interpret $g(\phi(x))\phi'(x)$ as $0$ when
$\phi'(x)=0$ and $g(\phi(x))$ is undefined.   Then
$\int_v^{v'}g$ is defined and equal to $\int_Ig(\phi(x))\phi'(x)dx$, where
here we interpret $\int_v^{v'}g$ as $-\int_{v'}^vg$ if
$v'<v$.

\proof{{\bf (a)} $\phi$ is differentiable almost everywhere
on $I$ and $\phi[A]$ is negligible for every negligible
$A\subseteq I$.   \Prf\ We can express $I$ as the union of a
sequence $\sequencen{I_n}$ of closed bounded intervals such that
$\phi\restr I_n$ is absolutely continuous for every $n$.   By
225Cb and 225G, $\phi\restr I_n$ is differentiable almost everywhere
on $I_n$ and $\phi[A]$ is negligible for every negligible
$A\subseteq I_n$, for each $n$.   So
$\phi$ is differentiable almost everywhere
on $\bigcup_{n\in\Bbb N}I_n=I$ and
$\phi[A]=\bigcup_{n\in\Bbb N}\phi[A\cap I_n]$ is negligible for every
negligible $A\subseteq I$.\ \Qed

Because $\phi\restr J$ is continuous for every closed interval
$J\subseteq I$, $\phi$ is continuous.   By the Intermediate Value Theorem,
$\phi[I]$ is an interval including $\ooint{\min(v,v'),\max(v,v')}$.

\medskip

{\bf (b)} Let $D\subseteq I$ be the domain of $\phi'$.   For $x\in D$,
we can think of $\phi'(x)$ as a $1\times 1$ matrix with determinant
$\phi'(x)$.   As $I\setminus D$ is negligible,
$\phi[I]\setminus\phi[D]\subseteq\phi[I\setminus D]$ is negligible.
Now $\int_Dg(\phi(x))\phi'(x)dx=\int_Ig(\phi(x))\phi'(x)dx$.
Applying 263I to $\phi\restr D$ and $g\restr\phi[D]$, we see that
$\int_{\phi[D]}g\times R$ is defined and equal to
$\int_Dg\phi\times\phi'$, where
$R(y)=\sum_{x\in D\cap\phi^{-1}[\{y\}]}\sgn\phi'(x)$ whenever
$y\in\phi[D]$ and $D\cap\phi^{-1}[\{y\}]$ is finite.

\medskip

{\bf (c)} (The key.) Set $D_0=\{x:x\in D$, $\phi'(x)=0\}$.   By 263D(ii),
applied to $\phi\restr D_0$, $\phi[D_0]$ is negligible.
Set

\Centerline{$C=\{y:y\in\phi[D]\cap\dom g\setminus(\phi[D_0]\cup\{v,v'\})$,
$\phi^{-1}[\{y\}]$ is finite, $g(y)\ne 0\}$.}

\noindent If $y\in C$ and $K=\phi^{-1}[\{y\}]$, then

$$\eqalign{R(y)
=\sum_{x\in K}\sgn\phi'(x)
&=1\text{ if }v<y<v',\cr
&=-1\text{ if }v'<y<v,\cr
&=0\text{ otherwise}.\cr}$$

\noindent\Prf\
If $J\subseteq I\setminus K$ is an interval, $\phi(z)\ne y$ for
$z\in J$;  since $\phi$ is continuous, the Intermediate Value Theorem
tells us that $\sgn(\phi(z)-y)$ is constant on $J$.
Also $\phi'(x)\ne 0$ for every $x\in K$, because $y\notin\phi[D_0]$.
A simple induction
on $\#(K\cap\ooint{-\infty,z})$ shows that
$\sgn(\phi(z)-y)=\sgn(v-y)+2\sum_{x\in K,x<z}\sgn\phi'(x)$ for every
$z\in I\setminus K$;  taking the limit as $z\uparrow u'$,
$\sum_{x\in K}\sgn\phi'(x)=\bover12(\sgn(v'-y)-\sgn(v-y))$.
(Here we may have to interpret $\sgn(\pm\infty)$ as $\pm 1$ in the
obvious way.)   This turns out to be just what we need to know.\ \Qed

\medskip

{\bf (d)} So now we have

$$\eqalign{\int_Ig\phi\times\phi'
&=\int_Dg\phi\times\phi'
=\int_{\phi[D]}g\times R\cr
&=\int_{\phi[D]}g\times R\times\chi C
=\int_v^{v'}g\times\chi C
=\int_v^{v'}g\cr}$$

\noindent because $\phi[D]\setminus(C\cup g^{-1}[\{0\}])$ is negligible.
}%end of proof of 263J

\exercises{
\leader{263X}{Basic exercises (a)}
%\spheader 263Xa
Let $(X,\Sigma,\mu)$ be any measure space, $f\in\eusm L^0(\mu)$ and
$p\in\coint{1,\infty}$.   Show that $f\in\eusm L^p(\mu)$ iff

\Centerline{$\gamma=p\int_0^{\infty}\alpha^{p-1}
\mu^*\{x:x\in\dom f,\,|f(x)|>\alpha\}d\alpha$}

\noindent is finite, and in this case $\|f\|_p=\gamma^{1/p}$.
\Hint{$\int|f|^p=\int_0^{\infty}\mu^*\{x:|f(x)|^p>\beta\}d\beta$, by
252O;  now substitute $\beta=\alpha^p$.}
%263D

\spheader 263Xb Let $f$ be an integrable function defined almost
everywhere in $\BbbR^r$.   Show that if $\alpha<r-1$ then
$\sum_{n=1}^{\infty}n^{\alpha}|f(nx)|$ is finite for almost every
$x\in\BbbR^r$.   \Hint{estimate
$\sum_{n=0}^{\infty}n^{\alpha}\int_B|f(nx)|dx$ for balls $B$
centered at the origin.}
%263D

\spheader 263Xc Let $A\subseteq\ooint{0,1}$ be a set such that
$\mu^*A=\mu^*([0,1]\setminus A)=1$, where $\mu$ is Lebesgue measure on
$\Bbb R$.   Set $D=A\cup\{-x:x\in\ooint{0,1}\setminus A\}
\subseteq[-1,1]$, and set $\phi(x)=|x|$ for $x\in D$.   Show that $\phi$
is injective, that $\phi$ is differentiable relative to its domain
everywhere in $D$, and that $\mu^*\phi[D]<\int_D|\phi'(x)|dx$.
%263D

\spheader 263Xd Let $\phi:D\to\BbbR^r$ be a function differentiable
relative to $D$ at each point of $D\subseteq\BbbR^r$, and suppose that
for each
$x\in D$ there is a non-singular derivative $T(x)$ of $\phi$ at $x$.
Show that $D$ is expressible as
$\bigcup_{k\in\Bbb N}D_k$ where $D_k=D\cap\overline{D}_k$ and
$\phi\restr D_k$ is injective for each $k$.
%263D

\sqheader 263Xe(i) Show that for any Lebesgue measurable
$E\subseteq\Bbb R$ and $t\in\Bbb R\setminus\{0\}$,
$\int_{tE}\bover1{|u|}du=\int_E\bover1{|u|}du$.   (ii) For $t\in\Bbb R$,
$u\in\Bbb R\setminus\{0\}$ set $\phi(t,u)=(\bover{t}{u},u)$.   Show that
$\int_{\phi[E]}\bover1{|tu|}d(t,u)=\int_E\bover1{|tu|}d(t,u)$ for any
Lebesgue measurable $E\subseteq\BbbR^2$.
%263D

\sqheader 263Xf Define $\phi:\BbbR^3\to\BbbR^3$ by
setting

\Centerline{$\phi(\rho,\theta,\alpha)
=(\rho\sin\theta\sin\alpha,\rho\cos\theta\sin\alpha,\rho\cos\alpha)$.}

\noindent Show that $\det\phi'(\rho,\theta,\alpha)=\rho^2\sin\alpha$.
%263G

\spheader 263Xg Show that if $k=2l+1$ is odd, then
$\int_0^{\infty}x^ke^{-x^2/2}dx=2^ll!$.   (Compare 252Xi.)
%263H

\leader{263Y}{Further exercises (a)}
%\spheader 263Ya
Define a measure $\nu$ on $\Bbb R$ by setting
$\nu E=\int_E\bover1{|x|}dx$ for Lebesgue measurable sets
$E\subseteq\Bbb R$.   For $f$, $g\in\eusm L^1(\nu)$ set
$(f*g)(x)=\int f(\bover{x}{t})g(t)\nu(dt)$ whenever this is defined in
$\Bbb R$.   (i) Show that $f*g=g*f\in\eusm L^1(\nu)$.   (ii) Show that
$\int h(x)(f*g)(x)\nu(dx)=\int h(xy)f(x)f(y)\nu(dx)\nu(dy)$ for every
$h\in\eusm L^{\infty}(\nu)$.   (iii) Show that $f*(g*h)=(f*g)*h$ for
every $h\in\eusm L^1(\nu)$.   \Hint{263Xe.}
%263Xe 263D

\spheader 263Yb Let $E\subseteq\BbbR^2$ be a measurable set such that
$\limsup_{\alpha\to\infty}
\bover1{\alpha^2}\mu_2(E\cap B(\tbf{0},\alpha))>0$, writing $\mu_2$ for
Lebesgue measure on
$\BbbR^2$.   Show that there is some $\theta\in\ocint{-\pi,\pi}$ such
that $\mu_1E_{\theta}=\infty$, where
$E_{\theta}=\{\rho:\rho\ge 0,\,(\rho\cos\theta,\rho\sin\theta)\in E\}$.
\Hint{show that $\mu_2(E\cap B(\tbf{0},\alpha))
\le\alpha\int_{-\pi}^{\pi}\mu_1E_{\theta})d\theta$.}
Generalize to higher dimensions and to functions other than $\chi E$.
%263G

\spheader 263Yc Let $E\subseteq\BbbR^r$ be a measurable set,
and $\phi:E\to\BbbR^r$ a function differentiable relative to its domain,
with a derivative $T(x)$, at each point $x$ of $E$;  set
$J(x)=|\det T(x)|$.
Show that for any integrable function $g$ defined on $\phi[E]$,

\Centerline{$\int g(y)\#(\phi^{-1}[\{y\}])dy
=\int_EJ(x)g(\phi(x))dx$.}
%263I

\spheader 263Yd Find a proof of 263J based on the ideas of \S225.
\Hint{225Xe.}
%263J

\spheader 263Ye Let $f:[a,b]\to\Bbb R$ be a function of bounded
variation, where $a<b$ in $\Bbb R$, with Lebesgue decomposition
$f=f_p+f_{cs}+f_{ac}$ as in 226Cd;  let $\mu$ be Lebesgue measure on
$\Bbb R$.   Show that the following are equiveridical:  (i) $f_{cs}$ is
constant;  (ii) $\mu f[\,[c,d]\,]\le\int_c^d|f'|d\mu$ whenever
$a\le c\le d\le b$;   (iii)
$\mu^*f[A]\le\int_A|f'|d\mu$ for every $A\subseteq[a,b]$;   (iv) $f[A]$
is negligible for every negligible set $A\subseteq[a,b]$.   \Hint{for
(iv)$\Rightarrow$(i) put 226Yd and 263D(ii) together to show that
$|f(d)-f(c)|\le\int_c^d|f'|d\mu+\Var_{[c,d]}f_p$ whenever
$a\le c\le d\le b$, and therefore that
$\Var_{[a,b]}f\le\Var_{[a,b]}f_p+\Var_{[a,b]}f_{ac}$.}
%263D

\spheader 263Yf\dvAnew{2013} Suppose that $r=2$ and that
$\phi:\BbbR^2\to\BbbR^2$ is continuously differentiable with non-singular
derivative $T$ at $\tbf{0}$.   (i) Show that there is an $\epsilon>0$
such that whenever
$\Gamma$ is a small circle with centre $\tbf{0}$ and radius at most
$\epsilon$ then
$\phi\restr\Gamma$ is a homeomorphism between $\Gamma$ and a simple closed
curve around $\tbf{0}$.   (ii) Show that if $\det T>0$, then
for such circles $\phi(x)$ runs anticlockwise around $\phi[\Gamma]$ as $x$
runs anticlockwise around $\Gamma$.   (iii) What happens if
$\det T<0$?
}%end of exercises

\endnotes{
\Notesheader{263} Yet again, approaching 263D, I find myself having to
choose between giving an accessible, relatively weak result and making
the extra effort to set out a theorem which is somewhere near the
natural boundary of what is achievable within the concepts being
developed in this volume;  and, as usual, I go for the more powerful
form.   There are three basic sources of difficulty:  (i) the fact that
we are dealing with more than one dimension;  (ii) the fact that we are
dealing with irregular domains;   (iii) the fact that we are dealing
with arbitrary integrable functions.   I do not think I need to
apologise for (iii) in a book on measure theory.   Concerning (ii), it
is quite true that the principal applications of these results are to
cases in which the transformation $\phi$ is differentiable everywhere,
with continuous derivative, and the set $D$ has negligible boundary;
and in these cases there are substantial simplifications available --
mostly because the sets $D_m$ of the proof of 263D can be taken to be
cubes.   Nevertheless, I think any form of the result which makes such
assumptions is deeply unsatisfactory at this level, being an awkward
compromise between ideas natural to the Riemann integral and those
natural to the Lebesgue integral.   Concerning (i), it might even have
been right to lay out the whole argument for the case $r=1$  before
proceeding to the general case, as I did in \S\S114-115, because the
one-dimensional case is already important and interesting;  and if you
find the work above difficult -- which it is -- and your immediate
interests are in one-dimensional integration by substitution, then I
think you might find it worth your time to reproduce the $r=1$ argument
yourself, up to a proof of 263J.   In fact the biggest difference is
in 263A, which becomes nearly trivial;  the work of 262M and 263C
becomes more readable, because all the matrices turn into scalars and we
can drop the word `determinant', but I do not think we can dispense
with any of the ideas, at least if we wish to obtain 263D as stated.
(But see 263Yd.)

I found myself insisting, in the last paragraph, that a distinction can
be made between `ideas natural to the Riemann integral and those
natural to the Lebesgue integral'.   We are approaching deep questions
here, like `what are books on measure theory for?', which I do not
think can be answered without some -- possibly unconscious --
reference to the question `what is mathematics for?'.   I do of
course want to present here some of the wonderful general theorems which
arise in the Lebesgue theory.   But more important than any specific
theorem is a general idea of what can be proved by these methods.
It is the essence of modern measure theory that continuity does not
matter, or, if you prefer, that measurable functions are in some sense
so nearly continuous that we do not have to add hypotheses of continuity
in our theorems.   Now this is in a sense a great liberation, and the
Lebesgue integral is now the standard one.   But you must not regard the
Riemann integral as outdated.   The intuitions on which it is founded --
for instance, that the surface of a solid body has zero volume -- remain
of great value in their proper context, which certainly includes the
study of differentiable functions with continuous derivatives.   What I
am saying here is that I believe we can use these intuitions best if we
maintain a division, a flexible and permeable one, of course, between
the ideas of the two theories;  and that when transferring a theorem
from one side of the boundary to the other we should do so
whole-heartedly, seeking to express the full power of the methods we are
using.

I have already said that the essential difference between the
one-dimensional and multi-dimensional cases lies in 263A, where the
Jacobian $J=|\det T|$ enters the argument.   Shorn of the technical
devices necessary to deal with arbitrary Lebesgue measurable sets, this
amounts to a calculation of the volume of the parallelepiped $T[I]$
where $I$ is the interval $\coint{\tbf{0},\tbf{1}}$.   I have dealt with
this by a little bit of algebra, saying that the result is essentially
obvious if $T$ is diagonal, whereas if $T$ is an isometry it follows
from the fact that the unit ball is left invariant;  and the algebra
comes in to express an arbitrary matrix as a product of diagonal and
orthogonal matrices.    It is also plain from 261F that Lebesgue
measure must be rotation-invariant as well as translation-invariant;
that is to
say, it is invariant under all isometries.   Another way of looking at
this will appear in the next section.

I feel myself that the centre of the argument for 263D is in the lemma
263C.   This is where we turn the exact result for linear operators into
an approximate result for almost-linear functions;  and the whole point
of differentiability is that a differentiable function is well
approximated, near any point of its domain, by a linear
operator.   The lemma involves two rather different ideas.   To show
that $\mu^*\phi[D]\le(J+\epsilon)\mu^*D$, we look first at balls and
then use Vitali's theorem to see that $D$ is economically covered by
balls, so that an upper bound for $\mu^*\phi[D]$ in terms of a sum
$\sum_{B\in\Cal I_0}\mu^*\phi[D\cap B]$ is adequate.    To obtain a
lower bound, we need to reverse the argument by looking at
$\psi=\phi^{-1}$, which involves checking first that $\phi$ is
invertible, and then that $\psi$ is appropriately linked to $T^{-1}$.
I have written out exact formulae for $\epsilon'$, $\zeta_2'$ and so on,
but this is only in case you do not trust your intuition;  the fact that
$\|\phi^{-1}(u)-\phi^{-1}(v)-T^{-1}(u-v)\|$ is small compared with
$\|u-v\|$ is pretty clearly a consequence of the hypothesis that
$\|\phi(x)-\phi(y)-T(x-y)\|$ is small compared with $\|x-y\|$.

The argument of 263D itself is now a matter of breaking the set $D$ up
into appropriate pieces on each of which $\phi$ is sufficiently nearly
linear for 263C to apply, so that

\Centerline{$\mu^*\phi[D]\le\sum_{m=0}^{\infty}\mu^*\phi[D_m]
\le\sum_{m=0}^{\infty}(J_m+\epsilon)\mu^*D_m$.}

\noindent With a little care (taken in 263C, with its condition (i)), we
can also ensure that the Jacobian $J$ is
well approximated by $J_m$ almost everywhere in $D_m$, so that
$\sum_{m=0}^{\infty}J_m\mu^*D_m\bumpeq\int_DJ(x)dx$.

These ideas, joined with the results of \S262, bring us to the point

\Centerline{$\int_EJ\,d\mu=\mu\phi[E]$}

\noindent when $\phi$ is injective and $E\subseteq D$ is measurable.
We need a final trick, involving Borel sets, to translate this into

\Centerline{$\int_{\phi^{-1}[F]}J\,d\mu=\mu F$}

\noindent whenever $F\subseteq\phi[D]$ is measurable, which is what is
needed for the application of 235J.

I hope that you long ago saw, and were delighted by, the device in 263G.
Once again, this is not really Lebesgue integration;  but I include it
just to show that the machinery of this chapter can be turned to deal
with the classical results, and that indeed we have a tiny profit from
our labour, in that no apology need be made for the boundary of the set
$D$ into which the polar coordinate system maps the plane.   I have
already given the actual
result as an exercise in 252Xi.   That involved (if you chase through
the references) a one-dimensional substitution (performed in 225Xh),
Fubini's theorem and an application of the formulae of \S235;  that is
to say, very much the same elements as those used above, though in a
different order.   I could present this with no mention of
differentiation in higher dimensions because the first change of
variable was in one dimension, and the second (involving the function
$x\mapsto\|x\|$, in 252Xi(i)) was of a particularly simple type, so that
a different method could be used to find the function $J$.

The function $R(y)=\sum_{x\in\phi^{-1}[\{y\}]}\sgn\det T(x)$ of 263Ib
belongs to rather deeper notions in differential geometry than I wish to
enlarge on here.   In the one-dimensional context it simply counts up- and
down-crossings of $y$ (see part (c) of the proof of 263J), because we can
think of each $T(x)$ as a scalar which is either positive or negative.
(It is relevant that in this case we can assume that $T(x)$ is non-singular
whenever $\phi(x)=y$.)   In higher dimensions, I suppose the first thing to
look for is a geometric interpretation of $\sgn\det T(x)$.   (See 263Ye.)
A geometric interpretation of the sum is something else again.
But it is worth noting that (subject to a natural interpretation of
`0$\times$undefined') we can
relate $\int_Dg\phi\times\det\phi'$ to $\int_{\phi[D]}g\times R$ where
$R$ is definable from $\phi$ without reference to $g$;  it counts folds
in the graph of $\phi$.

The abstract ideas to which this treatise is devoted do not, indeed,
lead us to many particular examples on which to practise the ideas of
this section.   The ones which do arise tend to be very straightforward,
as in 263G, 263Xa-263Xb and 263Xe.   I mention the last because it
provides a formula needed to discuss a new type of convolution (263Ya).
In effect, this depends on the multiplicative group
$\Bbb R\setminus\{0\}$ in place of the additive group $\Bbb R$ treated
in \S255.   The formula $\bover1x$ in the definition of $\nu$ is of
course the derivative of $\ln x$, and $\ln$ is an isomorphism between
$(\ooint{0,\infty},\cdot,\nu)$ and $(\Bbb R,+,\text{Lebesgue measure})$.
}%end of notes

\discrpage
