\frfilename{mt281.tex} 
\versiondate{4.12.12} 
\copyrightdate{1994} 
      
\def\chaptername{Fourier analysis} 
\def\sectionname{The Stone-Weierstrass theorem} 
      
\newsection{281} 
      
Before we begin work on the real subject of this chapter, it will be 
helpful to have a reasonably general statement of a fundamental theorem 
on the approximation of continuous functions.   In fact I give a variety 
of forms (281A, 281E, 281F and 281G, together with 281Ya, 281Yd and 
281Yg), all of which are sometimes useful.   I end the section with a 
version of Weyl's Equidistribution Theorem (281M-281N). 
      
\leader{281A}{Stone-Weierstrass theorem:  first form} 
Let $X$ be a topological space and $K$ a compact subset of $X$.   Write 
$C_b(X)$ for the space of all bounded continuous real-valued functions 
on $X$, so that $C_b(X)$ is a linear space over $\Bbb R$. 
Let $A\subseteq C_b(X)$ be such that 
      
\inset{$A$ is a linear subspace of $C_b(X)$;} 
      
\inset{$|f|\in A$ for every $f\in A$;} 
      
\inset{$\chi X\in A$;} 
      
\inset{whenever $x$, $y$ are distinct points of $K$ there is an $f\in A$ 
such that $f(x)\ne f(y)$.} 
      
\noindent Then for every continuous $h:K\to\Bbb R$ and $\epsilon>0$ 
there is an $f\in A$ such that 
      
\inset{$|f(x)-h(x)|\le\epsilon$ for every $x\in K$,} 
      
\inset{if $K\ne\emptyset$, $\inf_{x\in X}f(x)\ge\inf_{x\in K}h(x)$ and 
$\sup_{x\in X}f(x)\le\sup_{x\in K}h(x)$.} 
      
\cmmnt{\medskip 
      
\noindent{\bf Remark} I have stated this theorem in its natural context, 
that of general topological spaces.   But if these are unfamiliar to 
you, you do not in fact need to know what they are.   If you read 
`let $X$ be a topological space' as `let $X$ be a subset of $\Bbb R^r$'  
and `$K$ is a compact subset of $X$' as `$K$ is a subset of 
$X$ which is closed and bounded in $\BbbR^r$', you will have enough for 
all the applications in this chapter.   In order to follow the proof, of 
course, you will need to know a little about compactness in $\BbbR^r$; 
I have written out the necessary facts in \S2A2. 
} 
      
\proof{{\bf (a)} If $K$ is empty, then we can take $f=\tbf{0}$ to be the 
constant function with value $0$.   So henceforth let us suppose that 
$K\ne\emptyset$. 
      
\medskip 
      
{\bf (b)} The first point to note is that if $f$, $g\in A$ then  
$f\wedge g$ and $f\vee g$ belong to $A$, where 
      
\Centerline{$(f\wedge g)(x)=\min(f(x),g(x))$,\quad 
$(f\vee g)(x)=\max(f(x),g(x))$} 
      
\noindent for every $x\in X$;  this is because 
      
\Centerline{$f\wedge g=\Bover12(f+g-|f-g|)$,\quad 
$f\vee g=\Bover12(f+g+|f-g|)$.} 
      
\noindent It follows by induction on $n$ that $f_0\wedge\ldots\wedge f_n$  
and $f_0\vee\ldots\vee f_n$ belong to $A$ for all $f_0,\ldots,f_n\in A$. 
      
\medskip 
      
{\bf (c)} If $x$, $y$ are distinct points of $K$, and $a$, $b\in\Bbb R$, 
there is an $f\in A$ such that $f(x)=a$ and $f(y)=b$.   \Prf\ Start from 
$g\in A$ such that $g(x)\ne g(y)$;  this is the point at which we use 
the last of the list of four hypotheses on $A$.   Set 
      
\Centerline{$\alpha=\Bover{a-b}{g(x)-g(y)}$,\quad 
$\beta=\Bover{bg(x)-ag(y)}{g(x)-g(y)}$,\quad 
$f=\alpha g+\beta\chi X\in A$.   \Qed} 
      
\medskip 
      
{\bf (d)} (The heart of the proof lies in the next two paragraphs.)  Let 
$h:K\to\coint{0,\infty}$ be a continuous function and $x$ 
any point of $K$.   For any $\epsilon>0$, there is an $f\in A$ such that 
$f(x)=h(x)$ and $f(y)\le h(y)+\epsilon$ for every $y\in K$.   \Prf\ 
Let $\Cal G_x$ be the family of those open sets $G\subseteq X$ for which 
there is some $f\in A$ such that $f(x)=h(x)$ and $f(w)\le h(w)+\epsilon$ 
for every $w\in K\cap G$.   I claim that $K\subseteq\bigcup\Cal G_x$. 
To see this, take any $y\in K$.   By (c), there is an $f\in A$ such that 
$f(x)=h(x)$ and $f(y)=h(y)$.   Now $h-f\restr K:K\to\Bbb R$ is a 
continuous function, taking the value $0$ at $y$, so there is an open 
subset $G$ of $X$, containing $y$, such that 
$(h-f\restr K)(w)\ge-\epsilon$ for every $w\in G\cap K$, that is, 
$f(w)\le h(w)+\epsilon$ for every $w\in G\cap K$.   Thus $G\in\Cal G_x$ 
and $y\in\bigcup\Cal G_x$, as required. 
      
Because $K$ is compact, $\Cal G_x$ has a finite subcover 
$G_0,\ldots,G_n$ 
say.   For each $i\le n$, take $f_i\in A$ such that $f_i(x)=h(x)$ and 
$f_i(w)\le h(w)+\epsilon$ for every $w\in G_i\cap K$.   Then 
      
\Centerline{$f=f_0\wedge f_1\wedge\ldots\wedge f_n\in A$,} 
      
\noindent by (b), and evidently $f(x)=h(x)$, while if $y\in K$ there is 
some $i\le n$ such that $y\in G_i$, so that 
      
\Centerline{$f(y)\le f_i(y)\le h(y)+\epsilon$. \Qed} 
      
      
\medskip 
      
{\bf (e)} If $h:K\to\Bbb R$ is any continuous function and $\epsilon>0$, 
there is an $f\in A$ such that $|f(y)-h(y)|\le\epsilon$ for every $y\in 
K$.   \Prf\ This time, let $\Cal G$ be the set of those open subsets 
$G$ of $X$ for which there is some $f\in A$ such that $f(y)\le 
h(y)+\epsilon$ for every $y\in K$ and $f(x)\ge h(x)-\epsilon$ for every 
$x\in G\cap K$.   Once again, $\Cal G$ is an open cover of $K$.   To 
see this, take any $x\in K$.   By (d), there is an $f\in A$ such that 
$f(x)=h(x)$ and $f(y)\le h(y)+\epsilon$ for every $y\in K$.   Now 
$h-f\restr K:K\to\Bbb R$ is a continuous function which is zero at $x$, 
so there is an open subset $G$ of $X$, containing $x$, such that 
$(h-f\restr K)(w)\le\epsilon$ for every $w\in G\cap K$, that is, 
$f(w)\ge 
h(w)-\epsilon$ for every $w\in G\cap K$.   Thus $G\in\Cal G$ and 
$x\in\bigcup\Cal G$, as required. 
      
Because $K$ is compact, $\Cal G$ has a finite subcover $G_0,\ldots,G_m$ 
say.   For each $j\le m$, take $f_j\in A$ such that $f_j(y)\le 
h(y)+\epsilon$ for every $y\in K$ and $f_j(w)\ge h(w)-\epsilon$ for 
every $w\in G_j\cap K$.   Then 
      
\Centerline{$f=f_0\vee f_1\vee\ldots\vee f_m\in A$,} 
      
\noindent by (b), and evidently $f(y)\le h(y)+\epsilon$ for every $y\in 
K$,  while if $x\in K$ there is some $j\le m$ such that $x\in G_j$, so 
that 
      
\Centerline{$f(x)\ge f_j(x)\ge h(x)-\epsilon$.} 
      
\noindent Thus $|f(x)-h(x)|\le\epsilon$ for every $x\in K$, as required. 
\Qed 
      
\medskip 
      
{\bf (f)} Thus we have an $f$ satisfying the first of the two 
requirements of the theorem.   But for the second, set 
$M_0=\inf_{x\in K}h(x)$ and $M_1=\sup_{x\in K}h(x)$, and 
      
\Centerline{$f_1=\med(M_0\chi X,f,M_1\chi X)
=(M_0\chi X)\vee(f\wedge M_1\chi X)$;} 
      
\noindent $f_1$ satisfies the second condition as well as the first. 
(I am tacitly assuming here what is 
in fact the case, that $M_0$ and $M_1$ are finite;  this is because $K$ 
is compact -- see 2A2G or 2A3N.) 
}%end of proof of 281A 
      
\leader{281B}{}\cmmnt{ We need some simple 
tools, belonging to the basic theory of normed spaces;  but I hope they 
will be accessible even if you have not encountered `normed spaces' 
before, if you keep a finger at the beginning of \S2A4 as you 
read the next lemma. 
      
\medskip 
      
\noindent}{\bf Lemma} Let $X$ be any set.   Write $\ell^{\infty}(X)$ 
for the set of bounded functions from $X$ to $\Bbb R$.   For 
$f\in\ell^{\infty}(X)$, set 
      
\Centerline{$\|f\|_{\infty}=\sup_{x\in X}|f(x)|$,} 
      
\noindent counting the supremum as $0$ if $X$ is empty.   Then 
      
(a) $\ell^{\infty}(X)$ is a normed space. 
      
(b) Let $A\subseteq\ell^{\infty}(X)$ be a subset and 
$\overline{A}$ its closure\cmmnt{ (2A3D)}. 
      
\quad (i) If $A$ is a linear subspace of $\ell^{\infty}(X)$, so is 
$\overline{A}$. 
      
\quad (ii) If $f\times g\in A$ whenever $f$, $g\in A$, then  
$f\times g\in\overline{A}$ whenever $f$, $g\in\overline{A}$. 
      
\quad (iii) If $|f|\in A$ whenever $f\in A$, then $|f|\in\overline{A}$ 
whenever $f\in\overline{A}$. 
      
\proof{{\bf (a)} This is a routine verification.   To confirm 
that $\ell^{\infty}(X)$ is a linear space over $\Bbb R$, we have to 
check that $f+g$, $cf$ belong to $\ell^{\infty}(X)$ whenever $f$, 
$g\in\ell^{\infty}(X)$ and $c\in\Bbb R$;   simultaneously we can 
confirm that $\|\,\|_{\infty}$ is a norm on $\ell^{\infty}(X)$ by 
observing that 
      
\Centerline{$|(f+g)(x)|\le|f(x)|+|g(x)| 
\le\|f\|_{\infty}+\|g\|_{\infty}$,} 
      
\Centerline{$|cf(x)|=|c||f(x)|\le|c|\|f\|_{\infty}$} 
      
\noindent whenever $f$, $g\in\ell^{\infty}(X)$ and $c\in\Bbb R$. 
It is worth noting at the same time that if $f$, $g\in\ell^{\infty}(X)$, 
then 
      
\Centerline{$|(f\times g)(x)| 
=|f(x)||g(x)|\le\|f\|_{\infty}\|g\|_{\infty}$} 
      
\noindent for every $x\in X$, so that 
$\|f\times g\|_{\infty}\le\|f\|_{\infty}\|g\|_{\infty}$. 
      
(Of course all these remarks are very elementary special cases of parts 
of \S243;  see 243Xl.) 
      
\medskip 
      
{\bf (b)} Recall that 
      
\Centerline{$\overline{A}=\{f:f\in\ell^{\infty}(X),\Forall 
\epsilon>0\Exists f_1\in A, 
\,\|f-f_1\|_{\infty}\le\epsilon\}$} 
      
\noindent (2A3Kb).   Take $f$, $g\in\overline{A}$ and $c\in\Bbb R$, 
and let $\epsilon>0$.   Set 
      
\Centerline{$\eta 
=\min(1,\Bover{\epsilon}{2+|c|+\|f\|_{\infty}+\|g\|_{\infty}})>0$.} 
      
\noindent Then there are $f_1$, $g_1\in A$ such that 
$\|f-f_1\|_{\infty}\le\eta$ and $\|g-g_1\|_{\infty}\le\eta$. 
      
Now 
      
$$\eqalignno{\|(f+g)-(f_1+g_1)\|_{\infty} 
&\le\|f-f_1\|_{\infty}+\|g-g_1\|_{\infty} 
\le 2\eta 
\le\epsilon,\cr 
\cr 
\|cf-cf_1\|_{\infty} 
&=|c|\|f-f_1\|_{\infty} 
\le|c|\eta 
\le\epsilon,\cr 
\cr 
\|(f\times g)-(f_1\times g_1)\|_{\infty} 
&=\|(f-f_1)\times g+f\times(g-g_1)-(f-f_1)\times(g-g_1)\|_{\infty}\cr 
&\le\|(f-f_1)\times g\|_{\infty}+\|f\times(g-g_1)\|_{\infty} 
  +\|(f-f_1)\times(g-g_1)\|_{\infty}\cr 
&\le\|f-f_1\|_{\infty}\|g\|_{\infty}+\|f\|_{\infty}\|g-g_1)\|_{\infty} 
  +\|f-f_1\|_{\infty}\|g-g_1\|_{\infty}\cr 
&\le\eta(\|g\|_{\infty}+\|f\|_{\infty}+\eta) 
\le\eta(\|g\|_{\infty}+\|f\|_{\infty}+1) 
\le\epsilon,\cr 
\cr 
\||f|-|f_1|\|_{\infty} 
&\le\|f-f_1\|_{\infty} 
\le\eta 
\le\epsilon.\cr}$$ 
      
\quad{\bf (i)} If $A$ is a linear subspace, then $f_1+g_1$ and $cf_1$ 
belong to $A$.   As $\epsilon$ is arbitrary, $f+g$ and $cf$ belong to 
$\overline{A}$.   As $f$, $g$ and $c$ are arbitrary, $\overline{A}$ is a 
linear subspace of $\ell^{\infty}(X)$. 
      
\quad{\bf (ii)} If $A$ is closed under multiplication, then $f_1\times 
g_1\in A$.   As $\epsilon$ is arbitrary, $f\times g\in \overline{A}$. 
      
\quad{\bf (iii)} If the absolute values of functions in $A$ belong to 
$A$, then $|f_1|\in A$.   As $\epsilon$ is arbitrary, 
$|f|\in\overline{A}$. 
}%end of proof of 281B 
      
\leader{281C}{Lemma} There is a sequence $\sequencen{p_n}$ of real 
polynomials such that $\lim_{n\to\infty}p_n(x) 
\ifdim\pagewidth=390pt\break\fi 
=|x|$ uniformly for 
$x\in[-1,1]$. 
      
\proof{{\bf (a)} By the Binomial Theorem we have 
      
\Centerline{$(1-x)^{1/2} 
=1-\Bover12x-\Bover1{4{\cdot}2!}x^2 
  -\Bover{1{\cdot}3}{2^3{\cdot}3!}x^3-\ldots 
=-\sum_{n=0}^{\infty}\Bover{(2n)!}{(2n-1)(2^nn!)^2}x^n$} 
      
\noindent whenever $|x|<1$, with the convergence being uniform on any 
interval $[-a,a]$ with $0\le a<1$.   (For a proof of this, see almost 
any book on real or complex analysis.   If you have no favourite text to 
hand, you can try to construct a proof from the following facts:  (i) 
the radius of convergence of the series is $1$, so on any interval 
$[-a,a]$, with $0\le a<1$, it is uniformly absolutely summable (ii) 
writing $f(x)$ for the sum of the series for $|x|<1$, use Lebesgue's 
Dominated Convergence Theorem to find expressions for the indefinite 
integrals $\int_0^xf$, $-\int_{-x}^0f$ and show that these are 
$\bover23(1-(1-x)f(x))$, $\bover23(1-(1+x)f(-x))$ for $0\le x<1$ (iii) 
use the Fundamental Theorem of Calculus to show that 
$f(x)+2(1-x)f'(x)=0$ 
(iv) show that $\bover{d}{dx}\bigl(\bover{f(x)^2}{1-x}\bigr)=0$ and 
hence (v) that $f(x)^2=1-x$ whenever $|x|<1$.   Finally, show that 
because $f$ is continuous and non-zero in $\ooint{-1,1}$, $f(x)$ must be 
the positive square root of $1-x$ throughout.) 
      
We have a further fragment of information.   If we set 
      
\Centerline{$q_0(x)=1$, \quad$q_1(x)=1-\Bover12x$, 
\quad$q_n(x)=-\sum_{k=0}^n\Bover{(2k)!}{(2k-1)(2^kk!)^2}x^k$} 
      
\noindent for $n\ge 2$ and $x\in[0,1]$, so that $q_n$ is the $n$th partial sum of the binomial series for $(1-x)^{1/2}$, then we have 
$\lim_{n\to\infty}q_n(x)=(1-x)^{1/2}$ for every $x\in\coint{0,1}$.   But 
also every $q_n$ is non-increasing on $[0,1]$, and 
$\sequencen{q_n(x)}$ is a non-increasing sequence for 
each $x\in[0,1]$.   So we must have 
      
\Centerline{$\sqrt{1-x}\le q_n(x) 
\Forall n\in\Bbb N,\,x\in\coint{0,1}$,} 
      
\noindent and therefore, because all the $q_n$ are continuous, 
      
\Centerline{$\sqrt{1-x}\le q_n(x) 
\Forall n\in\Bbb N,\,x\in[0,1]$.} 
      
\noindent Moreover, given $\epsilon>0$, set $a=1-{1\over 4}\epsilon^2$, 
so that $\sqrt{1-a}=\bover{\epsilon}2$.   Then there is an 
$n_0\in\Bbb N$ such that $q_n(x)-\sqrt{1-x}\le\bover{\epsilon}2$ for 
every $x\in[0,a]$ and $n\ge n_0$.   In particular, $q_n(a)\le\epsilon$, 
so $q_n(x)\le\epsilon$ and $q_n(x)-\sqrt{1-x}\le\epsilon$ whenever
$x\in[a,1]$ and $n\ge n_0$.   This means that 
      
\Centerline{$0\le q_n(x)-\sqrt{1-x}\le\epsilon 
\Forall n\ge n_0,\,x\in[0,1]$;} 
      
\noindent as $\epsilon$ is arbitrary, 
$\sequencen{q_n(x)}\to\sqrt{1-x}$ uniformly on $[0,1]$. 
      
\medskip 
      
{\bf (b)} Now set $p_n(x)=q_n(1-x^2)$ for $x\in\Bbb R$.   Because each 
$q_n$ is a real polynomial of degree $n$, each $p_n$ is a real 
polynomial of degree $2n$.   Next, 
      
$$\eqalign{\sup_{|x|\le 1}|p_n(x)-|x|| 
&=\sup_{|x|\le 1}|q_n(1-x^2)-\sqrt{1-(1-x^2)}|\cr 
&=\sup_{y\in[0,1]}|q_n(y)-\sqrt{1-y}| 
\to 0\cr}$$ 
      
\noindent as $n\to\infty$, so $\lim_{n\to\infty}p_n(x)=|x|$ uniformly 
for $|x|\le 1$, as required. 
}%end of proof of 281C 
      
\leader{281D}{Corollary} Let $X$ be a set, and $A$ a norm-closed linear 
subspace of $\ell^{\infty}(X)$ containing $\chi X$ and such that 
$f\times g\in A$ whenever $f$, $g\in A$.   Then $|f|\in A$ for every 
$f\in A$. 
      
\proof{ Set 
      
\Centerline{$f_1=\Bover1{1+\|f\|_{\infty}}f$,} 
      
\noindent so that $f_1\in A$ and $\|f_1\|_{\infty}\le 1$.   Because $A$ 
contains $\chi X$ and is closed under multiplication, 
$p\frsmallcirc f_1\in A$ 
for every polynomial $p$ with real coefficients.   In particular, 
$g_n=p_n\frsmallcirc f_1\in A$ for every $n$, where $\sequencen{p_n}$ is 
the 
sequence of 281C.   Now, because $|f_1(x)|\le 1$ for every $x\in X$, 
      
\Centerline{$\|g_n-|f_1|\|_{\infty} 
=\sup_{x\in X}|p_n(f_1(x))-|f_1(x)|| 
\le\sup_{|y|\le 1}|p_n(y)-|y|| 
\to 0$} 
      
\noindent as $n\to\infty$.   Because $A$ is $\|\,\|_{\infty}$-closed, 
$|f_1|\in A$;  consequently $|f|\in A$, as claimed. 
}%end of proof of 281D 
      
\leader{281E}{Stone-Weierstrass theorem:  second form} Let $X$ be a 
topological space and $K$ a compact subset of $X$.   Write $C_b(X)$ for 
the space of all bounded continuous real-valued functions on $X$.   Let 
$A\subseteq C_b(X)$ be such that 
      
\inset{$A$ is a linear subspace of $C_b(X)$;} 
      
\inset{$f\times g\in A$ for every $f$, $g\in A$;} 
      
\inset{$\chi X\in A$;} 
      
\inset{whenever $x$, $y$ are distinct points of $K$ there is an $f\in A$ 
such that $f(x)\ne f(y)$.} 
      
\noindent Then for every continuous $h:K\to\Bbb R$ and $\epsilon>0$ 
there is an $f\in A$ such that 
      
\inset{$|f(x)-h(x)|\le\epsilon$ for every $x\in K$,} 
      
\inset{if $K\ne\emptyset$, $\inf_{x\in X}f(x)\ge\inf_{x\in K}h(x)$ and 
$\sup_{x\in X}f(x)\le\sup_{x\in K}h(x)$.} 
      
\proof{ Let $\overline{A}$ be the 
$\|\,\|_{\infty}$-closure of $A$ in $\ell^{\infty}(X)$.   It is helpful 
to know that $\overline{A}\subseteq C_b(X)$;  this is because the 
uniform limit of continuous functions is continuous.   (But if this is 
new to you, or your memory has faded, don't take time to look it up 
now;  just read `$\overline{A}\cap C_b(X)$' in place of 
`$\overline{A}$' in the rest of this argument.)   By 281B-281D, 
$\overline{A}$ is a linear subspace of $C_b(X)$ and $|f|\in\overline{A}$ 
for every $f\in\overline{A}$, so the conditions of 281A apply to 
$\overline{A}$. 
      
Take a continuous $h:K\to\Bbb R$ and an $\epsilon>0$.   The cases in 
which $K=\emptyset$ or $h$ is constant are trivial, because all constant 
functions belong to $A$;  so I suppose that $M_0=\inf_{x\in K}h(x)$ and 
$M_1=\sup_{x\in K}h(x)$ are defined and distinct.   As observed at the 
end of the proof of 281A, $M_0$ and $M_1$ are finite.   Set 
      
\Centerline{$\eta=\min(\bover13\epsilon,\bover12(M_1-M_2))>0$,\quad 
$\tilde h(x)=\med(M_0+\eta,h(x),M_1-\eta)$ for $x\in K$} 
      
\noindent (definition:  2A1Ac), 
so that $\tilde h:K\to\Bbb R$ is continuous and  
$M_0+\eta\le\tilde h(x)\le M_1-\eta$ for every $x\in K$.      By 281A,  there is an 
$f_0\in\overline{A}$ such that 
$|f_0(x)-\tilde h(x)|\le\eta$ for every $x\in K$ and $M_0+\eta\le 
f_0(x)\le M_1-\eta$ for every $x\in X$.   Now there is an $f\in A$ such 
that 
$\|f-f_0\|_{\infty}\le \eta$, so that 
      
\Centerline{$|f(x)-h(x)|\le|f(x)-f_0(x)|+|f_0(x)-\tilde h(x)| 
+|\tilde h(x)-h(x)|\le 3\eta\le\epsilon$} 
      
\noindent for every $x\in K$, while 
      
\Centerline{$M_0\le f_0(x)-\eta\le f(x)\le f_0(x)+\eta\le M_1$} 
      
\noindent for every $x\in X$. 
}%end of proof of 281E 
      
\leader{281F}{Corollary:  Weierstrass' theorem} Let $K$ be any closed 
bounded subset of 
$\Bbb R$.   Then every continuous $h:K\to\Bbb R$ can be uniformly 
approximated on $K$ by polynomials. 
      
\proof{ Apply 281E with $X=K$ (noting that $K$, being closed and 
bounded, is compact), and $A$ the set of polynomials with real 
coefficients, regarded as functions from $K$ to $\Bbb R$. 
}%end of proof of 281F 
      
\leader{281G}{Stone-Weierstrass theorem:  third form} Let $X$ be a 
topological space and $K$ a compact subset of $X$.   Write 
$C_b(X;\Bbb C)$ 
for the space of all bounded continuous complex-valued functions on 
$X$, so that $C_b(X;\Bbb C)$ is a linear space over $\Bbb C$.   Let 
$A\subseteq C_b(X;\Bbb C)$ be such that 
      
\inset{$A$ is a linear subspace of $C_b(X;\Bbb C)$;} 
      
\inset{$f\times g\in A$ for every $f$, $g\in A$;} 
      
\inset{$\chi X\in A$;} 
      
\inset{the complex conjugate $\bar f$ of $f$ belongs to $A$ for every 
$f\in A$;} 
      
\inset{whenever $x$, $y$ are distinct points of $K$ there is an $f\in A$ 
such that $f(x)\ne f(y)$.} 
      
\noindent Then for every continuous $h:K\to\Bbb C$ and $\epsilon>0$ 
there is an $f\in A$ such that 
      
\inset{$|f(x)-h(x)|\le\epsilon$ for every $x\in K$,} 
      
\inset{if $K\ne\emptyset$, $\sup_{x\in X}|f(x)|\le\sup_{x\in K}|h(x)|$.} 
      
\proof{ If $K=\emptyset$, or $h$ is identically zero, we 
can take $f=\tbf{0}$.   So let us suppose that 
$M=\sup_{x\in K}|h(x)|>0$. 
      
\medskip 
      
{\bf (a)} Set 
      
\Centerline{$A_{\Bbb R}=\{f:f\in A,\,f(x)$ is real for every $x\in 
X\}$.} 
      
\noindent Then $A_{\Bbb R}$ satisfies the conditions of 281E.   \Prf\ 
(i) Evidently $A_{\Bbb R}$ is a subset of $C_b(X)=C_b(X;\Bbb R)$, is 
closed under addition, multiplication by real scalars and pointwise 
multiplication of functions, and contains $\chi X$.  If $x$, $y$ are 
distinct points of $K$, there is an $f\in A$ such that $f(x)\ne f(y)$. 
Now 
      
\Centerline{$\Real f=\Bover12(f+\bar f)$, 
\quad$\Imag f=\Bover1{2i}(f-\bar 
f)$} 
      
\noindent both belong to $A$ and are real-valued, so belong to $A_{\Bbb 
R}$, and at least one of them takes different values at $x$ and $y$. 
\Qed 
      
\medskip 
      
{\bf (b)} Consequently, given a continuous function $h:K\to\Bbb C$ and 
$\epsilon>0$, we may apply 281E twice to find $f_1$, $f_2\in A_{\Bbb R}$ 
such that 
      
\Centerline{$|f_1(x)-\Real(h(x))|\le\eta$,\quad 
$|f_2(x)-\Imag(h(x))|\le\eta$} 
      
\noindent for every $x\in K$, where 
$\eta=\min(\bover12,M,\bover16\epsilon)>0$.   
Setting $g=f_1+if_2$, we have $g\in A$ 
and 
$|g(x)-h(x)|\le2\eta$ for every $x\in K$. 
      
\medskip 
      
{\bf (c)} Set $M_1=\|g\|_{\infty}$.   If $M_1\le M$ we can take $f=g$ and 
stop.   Otherwise, consider the function 
      
\Centerline{$\phi(t)=\Bover{M-\eta}{\max(M,\sqrt{t})}$} 
%\eta\le M so \phi\ge 0
      
\noindent for $t\in [0,M_1^2]$.   By Weierstrass' theorem 
(281F), there is a real polynomial $p$ such that 
$|\phi(t)-p(t)|\le\Bover{\eta}{M_1}$ whenever $0\le t\le M_1^2$.   Note 
that $|g|^2=g\times\bar g\in A$, so that 
      
\Centerline{$f=g\times p(|g|^2)\in A$.} 
      
\noindent Now 
      
\Centerline{$|p(t)|\le\phi(t)+\Bover{\eta}{M_1} 
\le\phi(t)+\Bover{\eta}{\max(M,\sqrt{t})} 
=\Bover{M}{\max(M,\sqrt{t})}$} 
      
\noindent whenever $0\le t\le M_1^2$, so 
      
      
\Centerline{$|f(x)|\le|g(x)|\Bover{M}{\max(M,|g(x)|)}\le M$} 
      
\noindent for every $x\in X$.    Next, if $0\le t\le\min(M_1,M+2\eta)^2$, 
      
\Centerline{$|1-p(t)|\le\Bover{\eta}{M_1}+1-\phi(t) 
\le\Bover{\eta}M+1-\Bover{M-\eta}{M+2\eta} 
\le\Bover{4\eta}M$.} 
      
\noindent    Consequently, if $x\in K$, so that 
      
\Centerline{$|g(x)|\le\min(M_1,|h(x)|+2\eta)\le 
\min(M_1,M+2\eta)$,} 
      
\noindent we shall have 
      
\Centerline{$|1-p(|g(x)|^2)|\le\Bover{4\eta}M$,} 
      
\noindent and 
      
$$\eqalign{|f(x)-h(x)| 
&\le|g(x)-h(x)|+|g(x)||1-p(|g(x)|^2)|\cr 
&\le 2\eta+\Bover{4\eta}M(M+2\eta) 
\le 2\eta+\Bover{4\eta}M(M+1) 
%\eta\le\bover12
\le\epsilon,\cr}$$ 
%\eta\le\bover16\epsilon
      
\noindent as required. 
}% end of proof of 281G 
      
\cmmnt{ 
\medskip 
      
\noindent{\bf Remark} Of course we could have saved ourselves effort by 
settling for 
      
\Centerline{$\sup_{x\in X}|f(x)|\le 2\sup_{x\in K}|h(x)|$,} 
      
\noindent which would be quite good enough for the applications below. 
} 
      
\leader{281H}{Corollary} Let $[a,b]\subseteq\Bbb R$ be a 
non-empty bounded closed interval and $h:[a,b]\to\Bbb C$ a continuous 
function.   Then for any $\epsilon>0$ there are $y_0,\ldots,y_n\in\Bbb 
R$ and $c_0,\ldots,c_n\in\Bbb C$ such that 
      
\Centerline{$|h(x)-\sum_{k=0}^nc_ke^{iy_kx}|\le\epsilon$ for every 
$x\in[a,b]$,} 
      
\Centerline{$\sup_{x\in\Bbb 
R}|\sum_{k=0}^nc_ke^{iy_kx}|\le\sup_{x\in[a,b]}|h(x)|$.} 
      
\proof{ Apply 281G with $X=\Bbb R$, $K=[a,b]$ and $A$ the 
linear span of the functions $x\mapsto e^{iyx}$ as $y$ runs over $\Bbb 
R$.} 
      
\leader{281I}{Corollary} Let $S^1$ be the unit circle 
$\{z:|z|=1\}\subseteq\Bbb C$.   Then for any continuous function 
$h:S^1\to\Bbb C$ and $\epsilon>0$, there are $n\in\Bbb N$ and 
$c_{-n},c_{-n+1},\ldots,c_0,\ldots,c_n\in\Bbb C$ such that 
$|h(z)-\sum_{k=-n}^nc_kz^k|\le\epsilon$ for every $z\in S^1$. 
      
\proof{ Apply 281G with $X=K=S^1$ and $A$ the linear span 
of the functions $z\mapsto z^k$ for $k\in\Bbb Z$. 
} 
      
\leader{281J}{Corollary} Let $h:[-\pi,\pi]\to\Bbb C$ be a continuous 
function such that $h(\pi)=h(-\pi)$.   Then for any $\epsilon>0$ there 
are $n\in\Bbb N$, $c_{-n},\ldots,c_n\in\Bbb C$ such that 
$|h(x)-\sum_{k=-n}^nc_ke^{ikx}|\le\epsilon$ for every $x\in[-\pi,\pi]$. 
      
\proof{ The point is that $\tilde h:S^1\to\Bbb C$ is 
continuous on $S^1$, where $\tilde h(z)=h(\arg z)$;  this is because 
$\arg$ is 
continuous everywhere except at $-1$, and 
      
\Centerline{$\lim_{x\downarrow-\pi}h(x) 
=h(-\pi)=h(\pi)=\lim_{x\uparrow\pi}h(x)$,} 
      
\noindent so 
      
\Centerline{$\lim_{z\in S^1,z\to-1}\tilde h(z)=h(\pi)=\tilde h(-1)$.} 
      
\noindent Now by 281I there are $c_{-n},\ldots,c_n\in\Bbb C$ such that 
$|\tilde h(z)-\sum_{k=-n}^nc_kz^k|\le\epsilon$ for every $z\in S^1$, and 
these coefficients 
serve equally for $h$. 
} 
      
\vleader{60pt}{281K}{Corollary} Suppose that $r\ge 1$ and that 
$K\subseteq\BbbR^r$ is a non-empty closed bounded set.   Let 
$h:K\to\Bbb C$ be a continuous 
function, and $\epsilon>0$.   Then there are $y_0,\ldots,y_n\in\Bbb Q^r$ 
and $c_0,\ldots,c_n\in\Bbb C$ such that 
      
\Centerline{$|h(x)-\sum_{k=0}^nc_ke^{iy_k\dotproduct x}|\le\epsilon$ for 
every $x\in K$,} 
      
\Centerline{$\sup_{x\in\Bbb R^r}|\sum_{k=0}^nc_ke^{iy_k\dotproduct x}| 
\le\sup_{x\in K}|h(x)|$,} 
      
\noindent writing $y\dotproduct x=\sum_{j=1}^r\eta_j\xi_j$ when 
$y=(\eta_1,\ldots,\eta_r)$ and $x=(\xi_1,\ldots,\xi_r)$ belong to 
$\BbbR^r$. 
      
      
\proof{ Apply 281G with $X=\BbbR^r$ and $A$ the linear 
span of the functions $x\mapsto e^{iy\dotproduct x}$ as $y$ runs over 
$\BbbQ^r$.} 
      
\leader{281L}{Corollary}  Suppose that $r\ge 1$ and that 
$K\subseteq\BbbR^r$ is a non-empty closed bounded set.   Let 
$h:K\to\Bbb R$ be a continuous 
function, and $\epsilon>0$.   Then there are $y_0,\ldots,y_n\in\BbbR^r$ 
and $c_0,\ldots,c_n\in\Bbb C$ such that, writing 
$g(x)=\sum_{k=0}^nc_ke^{iy_k\dotproduct x}$, $g$ is real-valued and 
      
\Centerline{$|h(x)-g(x)|\le\epsilon$ for every $x\in K$,} 
      
\Centerline{$\inf_{y\in K}h(y)\le g(x)\le\sup_{y\in K}h(y)$ for every 
$x\in\BbbR^r$.} 
      
\proof{ Apply 281E with $X=\BbbR^r$ and $A$ the set of 
{\it real-}valued functions on $\BbbR^r$ which are {\it complex} linear 
combinations of the functions $x\mapsto e^{iy\dotproduct x}$;  as 
remarked 
in 
part (a) of the proof of 281G, $A$ satisfies the conditions of 281E. 
} 
      
\leader{281M}{Weyl's Equidistribution Theorem}\cmmnt{ We are now ready 
for one of the basic results of number theory.   I shall actually apply 
it to provide an example in \S285 below, but (at least in the 
one-variable case) it is 
surely on the (rather long) list of things which every pure 
mathematician should know.    For the sake of the application I have in 
mind, I give the full $r$-dimensional version, but you may wish to take 
it in the first place with $r=1$. 
      
It will be helpful to have a notation for `fractional part'.}    For 
any real number $x$, write $\fraction{x}$ for that number in 
$\coint{0,1}$ such that $x-\fraction{x}$ is an integer.   \cmmnt{Now 
for the 
theorem.} 
      
\leader{281N}{Theorem} Let $\eta_1,\ldots,\eta_r$ be real numbers such 
that $1,\eta_1,\ldots,\eta_r$ are 
linearly independent over $\Bbb Q$.   Then whenever 
$0\le \alpha_j\le\beta_j\le 1$ for each $j\le r$, 
      
\Centerline{$\lim_{n\to\infty}\Bover1{n+1}\#(\{m:m\le 
n,\,\fraction{m\eta_j}\in[\alpha_j,\beta_j]$ for every $j\le r\}) 
=\prod_{j=1}^r(\beta_j-\alpha_j)$.} 
      
\cmmnt{\medskip 
      
\noindent{\bf Remark} Thus the theorem says that the long-term 
proportion of the $r$-tuples $(\fraction{m\eta_1}, 
\ifdim\pagewidth=390pt\break\fi 
\ldots,\fraction{m\eta_r})$ 
which belong to the interval $[a,b]\subseteq[\tbf{0},\tbf{1}]$ is just 
the Lebesgue measure $\mu[a,b]$ of the interval.   Of course the 
condition 
`$1,\eta_1,\ldots,\eta_r$ are linearly independent over $\Bbb Q$' is 
necessary as well as sufficient (281Xg).} 
      
\proof{{\bf (a)} Write $y=(\eta_1,\ldots,\eta_r)\in\BbbR^r$, 
      
\Centerline{$\fraction{my} 
=(\fraction{m\eta_1},\ldots,\fraction{m\eta_r}) 
\in\coint{\tbf{0},\tbf{1}}=\coint{0,1}^r$} 
      
\noindent for each $m\in\Bbb N$.   Set 
$I=[\tbf{0},\tbf{1}]=[0,1]^r$, and for any function $f:I\to\Bbb R$ write 
      
\Centerline{$\overline{L}(f) 
=\limsup_{n\to\infty}\Bover1{n+1}\sum_{m=0}^{n}f(\fraction{my})$,} 
      
\Centerline{$\underline{L}(f) 
=\liminf_{n\to\infty}\Bover1{n+1}\sum_{m=0}^{n}f(\fraction{my})$;} 
      
\noindent and for $f:I\to\Bbb C$ write 
      
\Centerline{$L(f) 
=\lim_{n\to\infty}\Bover1{n+1}\sum_{m=0}^nf(\fraction{my})$} 
      
\noindent if the limit exists.    It will be worth noting that for 
non-negative functions $f$, $g$, $h:I\to\Bbb R$ such that $h\le f+g$, 
      
\Centerline{$\overline{L}(h)\le\overline{L}(f)+\overline{L}(g)$,} 
      
\noindent and that $L(cf+g)=cL(f)+L(g)$ for any two functions $f$, 
$g:I\to\Bbb C$ such that $L(f)$ and $L(g)$ exist, and any $c\in\Bbb C$. 
      
\medskip 
      
{\bf (b)} I mean to show that $L(f)$ exists and is equal to $\int_{I}f$ 
for (many) continuous functions $f$.   The key step is to consider 
functions of the form 
      
\Centerline{$f(x)=e^{2\pi ik\dotproduct x}$,} 
      
\noindent where $k=(\kappa_1,\ldots,\kappa_r)\in\Bbb Z^r$.   In this 
case, if $k\ne\tbf{0}$, 
      
\Centerline{$k\dotproduct y=\sum_{j=1}^r\kappa_j\eta_j\notin\Bbb Z$} 
      
\noindent because $1,\eta_1,\ldots,\eta_r$ are linearly independent over 
$\Bbb Q$.   So 
      
$$\eqalignno{L(f) 
&=\lim_{n\to\infty}\Bover1{n+1} 
  \sum_{m=0}^ne^{2\pi ik\dotproduct \fraction{my}} 
=\lim_{n\to\infty}\Bover1{n+1}\sum_{m=0}^ne^{2\pi imk\dotproduct y}\cr 
\noalign{\noindent (because 
$mk\dotproduct y-k\dotproduct \fraction{my} 
=\sum_{j=1}^r\kappa_j(m\eta_j-\fraction{m\eta_j})$ is an integer)} 
&=\lim_{n\to\infty} 
\bover{1-e^{2\pi i(n+1)k\dotproduct y}}{(n+1)(1-e^{2\pi ik\dotproduct 
y})}\cr 
\noalign{\noindent (because $e^{2\pi ik\dotproduct y}\ne 1$)} 
&=0,\cr}$$ 
      
\noindent because $|1-e^{2\pi i(n+1)k\dotproduct y}|\le 2$ for every 
$n$.   Of course we can also calculate the integral of $f$ over $I$, 
which is 
      
$$\eqalignno{\int_If(x)dx 
&=\int_Ie^{2\pi ik\dotproduct x}dx 
=\int_I\prod_{j=1}^re^{2\pi i\kappa_j\xi_j}dx\cr 
\noalign{\noindent (writing $x=(\xi_1,\ldots,\xi_r)$)} 
&=\int_{0}^1\ldots\int_0^1\prod_{j=1}^r 
  e^{2\pi i\kappa_j\xi_j}d\xi_r\ldots d\xi_1\cr 
&=\int_{0}^1e^{2\pi i\kappa_r\xi_r}d\xi_r 
  \ldots\int_0^1e^{2\pi i\kappa_1\xi_1}d\xi_1 
=0\cr}$$ 
      
\noindent because at least one $\kappa_j$ is non-zero, and for this $j$ 
we must have 
      
\Centerline{$\int_0^1e^{2\pi i\kappa_j\xi_j}d\xi_j 
=\Bover{1}{2\pi i\kappa_j}(e^{2\pi i\kappa_j}-1)=0$.} 
      
\noindent So we have $L(f)=\int_If=0$ when $k\ne\tbf{0}$.   On the other 
hand, if $k=\tbf{0}$, then $f$ is constant with value $1$, so 
      
\Centerline{$L(f) 
=\lim_{n\to\infty}\Bover1{n+1}\sum_{m=0}^nf(\fraction{my}) 
=\lim_{n\to\infty}1=1=\int_If(x)dx$.} 
      
\medskip 
      
{\bf (c)} Now write 
$\partial I=[\tbf{0},\tbf{1}]\setminus\ooint{\tbf{0},\tbf{1}}$, the 
boundary of 
$I$.   If $f:I\to\Bbb C$ is continuous and $f(x)=0$ for 
$x\in\partial I$, then $L(f)=\int_If$.   \Prf\ As in 281I, let $S^1$ be 
the unit circle $\{z:z\in\Bbb C,\,|z|=1\}$, and set 
$K=(S^1)^r\subseteq\Bbb C^r$. 
If we think of $K$ as a subset of $\BbbR^{2r}$, it is closed and 
bounded.   Let $\phi:K\to I$ be given by 
      
\Centerline{$\phi(\zeta_1,\ldots,\zeta_r) 
=(\Bover12+\Bover{\arg\zeta_1}{2\pi},\ldots, 
\Bover12+\Bover{\arg\zeta_r}{2\pi})$} 
      
\noindent for $\zeta_1,\ldots,\zeta_r\in S^1$.   Then $h=f\phi:K\to\Bbb 
C$ is continuous, because $\phi$ is continuous on 
$(S^1\setminus\{-1\})^r$ and 
      
\Centerline{$\lim_{w\to z}f\phi(w)=f\phi(z)=0$} 
      
\noindent for any $z\in K\setminus(S^1\setminus\{-1\})^r$.   (Compare 
281J.)   Now apply 281G with $X=K$ and $A$ the set of polynomials in 
$\zeta_1,\ldots,\zeta_r,\zeta_1^{-1},\ldots,\zeta_r^{-1}$ to see that, 
given $\epsilon>0$, there is a function of the form 
      
\Centerline{$g(z)=\sum_{k\in 
J}c_k\zeta_1^{\kappa_1}\ldots\zeta_r^{\kappa_r}$,} 
      
\noindent for some finite set $J\subseteq\Bbb Z^r$ and constants 
$c_k\in\Bbb C$ for $k\in J$, such that 
      
\Centerline{$|g(z)-h(z)|\le\epsilon$ for every $z\in K$.} 
      
\noindent Set 
      
\Centerline{$\tilde g(x) 
=g(e^{\pi i(2\xi_1-1)},\ldots,e^{\pi i(2\xi_r-1)}) 
=\sum_{k\in J}c_ke^{\pi ik\dotproduct (2x-\tbf{1})} 
=\sum_{k\in J}(-1)^{k\tbf{.1}}c_ke^{2\pi ik\dotproduct x}$,} 
      
\noindent so that $\tilde g\phi = g$, and see that 
      
\Centerline{$\sup_{x\in I}|\tilde g(x)-f(x)| 
=\sup_{z\in K}|g(z)-h(z)|\le\epsilon$.} 
      
Now $\tilde g$ is of the form dealt with in (a), so we must have 
$L(\tilde g)=\int_I\tilde g$.    Let $n_0$ be such that 
      
\Centerline{$\bigl|\int_I\tilde g 
  -\Bover1{n+1}\sum_{m=0}^n\tilde g(\fraction{my})\bigr| 
\le\epsilon$} 
      
\noindent for every $n\ge n_0$.   Then 
      
\Centerline{$|\int_If-\int_I\tilde g|\le\int_I|f-\tilde g|\le\epsilon$} 
      
\noindent and 
      
$$\eqalign{|\Bover1{n+1}\sum_{m=0}^{n}\tilde g(\fraction{my}) 
-\Bover1{n+1}\sum_{m=0}^{n}f(\fraction{my})| 
&\le\Bover1{n+1}\sum_{m=0}^{n} 
 |\tilde g(\fraction{my})-f(\fraction{my})|\cr 
&\le\Bover1{n+1}(n+1)\epsilon 
=\epsilon\cr}$$ 
      
\noindent for every $n\in\Bbb N$.   So for $n\ge n_0$ we must have 
      
\Centerline{$|\Bover1{n+1}\sum_{m=0}^{n}f(\fraction{my}) 
-\int_If|\le3\epsilon$.} 
      
\noindent As $\epsilon$ is arbitrary, $L(f)=\int_If$, as required. 
\Qed 
      
\medskip 
      
{\bf (d)} Observe next that if $a$, 
$b\in\ooint{\tbf{0},\tbf{1}}=\ooint{0,1}^r$, and $\epsilon>0$, there are 
continuous functions $f_1$, $f_2$ such that 
      
\Centerline{$f_1\le\chi[a,b]\le f_2\le\chi\ooint{\tbf{0},\tbf{1}}$, 
\quad $\int_If_2-\int_If_1\le\epsilon$.} 
      
\noindent\Prf\ This is elementary.   For $n\in\Bbb N$, define 
$h_n:\Bbb R\to[0,1]$ by setting $h_n(\xi)=0$ if $\xi\le 0$, $2^n\xi$ if 
$0\le\xi\le 2^{-n}$ and $1$ if $\xi\ge 2^{-n}$.   Set 
      
\Centerline{$f_{1n}(x) 
=\prod_{j=1}^rh_n(\xi_j-\alpha_j)h_n(\beta_j-\xi_j)$,} 
      
\Centerline{$f_{2n}(x) 
=\prod_{j=1}^r(1-h_n(\alpha_j-\xi_j))(1-h_n(\xi_j-\beta_j))$} 
      
\noindent for $x=(\xi_1,\ldots,\xi_r)\in\BbbR^r$.   (Compare the proof 
of 242Oa.)   Then 
$f_{1n}\le\chi[a,b]\le f_{2n}$ for each $n$, 
$f_{2n}\le\chi\ooint{\tbf{0},\tbf{1}}$ for all $n$ so large that 
      
\Centerline{$2^{-n} 
\le\min(\min_{j\le r}\alpha_j,\min_{j\le r}(1-\beta_j))$,} 
      
\noindent and $\lim_{n\to\infty}f_{2n}(x)-f_{1n}(x)=0$ for every $x$, so 
      
\Centerline{$\lim_{n\to\infty}\int_If_{2n}-\int_If_{1n}=0$.} 
      
\noindent Thus we can take $f_1=f_{1n}$, $f_2=f_{2n}$ for any $n$ large 
enough.   \Qed 
      
\medskip 
      
{\bf (e)} It follows that if $a$, $b\in\ooint{\tbf{0},\tbf{1}}$ and 
$a\le b$, $L(\chi[a,b])=\mu[a,b]$.   \Prf\ Let $\epsilon>0$.   Take 
$f_1$, $f_2$ as in (d).   Then, using (c), 
      
\Centerline{$\overline{L}(\chi[a,b])\le\overline{L}(f_2) 
=L(f_2)=\int_If_2\le\int_If_1+\epsilon\le\mu[a,b]+\epsilon$,} 
      
\Centerline{$\underline{L}(\chi[a,b])\ge\underline{L}(f_1) 
=L(f_1)=\int_If_1\ge\int_If_2-\epsilon\ge\mu[a,b]-\epsilon$,} 
      
\noindent so 
      
\Centerline{$\mu[a,b]-\epsilon\le\underline{L}(\chi[a,b]) 
\le\overline{L}(\chi[a,b])\le\mu[a,b]+\epsilon$.} 
      
\noindent   As $\epsilon$ is arbitrary, 
      
\Centerline{$\mu[a,b]=\overline{L}(\chi[a,b])=\underline{L}(\chi[a,b]) 
=L(\chi[a,b])$,} 
      
\noindent as required.\ \Qed 
      
\medskip 
      
{\bf (f)} To complete the proof, take any $a$, $b\in I$ with $a\le b$. 
For $0<\epsilon\le\bover12$, set 
$I_{\epsilon}=[\epsilon\tbf{1},(1-\epsilon)\tbf{1}]$, so that 
$I_{\epsilon}$ is a closed interval included in 
$\ooint{\tbf{0},\tbf{1}}$ and $\mu I_{\epsilon}=(1-2\epsilon)^r$.   Of 
course $L(\chi I)=\mu I=1$, so 
      
\Centerline{$L(\chi(I\setminus I_{\epsilon})) 
=L(\chi I)-L(\chi I_{\epsilon}) 
=1-\mu I_{\epsilon}$,} 
      
\noindent and 
      
$$\eqalign{\mu[a,b]-1+\mu I_{\epsilon} 
&\le\mu[a,b]+\mu I_{\epsilon}-\mu([a,b]\cup I_{\epsilon}) 
=\mu([a,b]\cap I_{\epsilon})\cr 
&=L(\chi([a,b]\cap I_{\epsilon})) 
\le\underline{L}(\chi([a,b]))\cr 
&\le\overline{L}(\chi([a,b])) 
\le\overline{L}(\chi([a,b]\cap I_{\epsilon})) 
  +\overline{L}(\chi(I\setminus I_{\epsilon}))\cr 
&=L(\chi([a,b]\cap I_{\epsilon}))+1-\mu I_{\epsilon}\cr 
&=\mu([a,b]\cap I_{\epsilon})+1-\mu I_{\epsilon} 
\le\mu[a,b]+1-\mu I_{\epsilon}.\cr}$$ 
      
\noindent As $\epsilon$ is arbitrary, 
      
\Centerline{$\mu[a,b]=\overline{L}(\chi[a,b])=\underline{L}(\chi[a,b]) 
=L(\chi[a,b])$,} 
      
\noindent as stated. 
}%end of proof of 281N 
      
\exercises{ 
\leader{281X}{Basic exercises (a)} 
%\spheader 281Xa 
Let $A$ be the set of those bounded continuous 
functions  $f:\BbbR^r\times\BbbR^r\to\Bbb R$ which are expressible in 
the form $f(x,y)=\sum_{k=0}^ng_k(x)g'_k(y)$, where all the $g_k$, $g'_k$ 
are continuous functions from $\BbbR^r$ to $\Bbb R$.   Show that for 
any 
bounded continuous function $h:\BbbR^r\times\BbbR^r\to\Bbb R$ and any 
bounded set $K\subseteq\BbbR^r\times\BbbR^r$ and any $\epsilon>0$, 
there is an $f\in A$ such that 
$|f(x,y)-h(x,y)|\le\epsilon$ for every $(x,y)\in K$ and 
$\sup_{x,y\in\BbbR^r}|f(x,y)|\le\sup_{x,y\in \BbbR^r}|h(x,y)|$. 
%281E 
      
\spheader 281Xb Let $K$ be a closed bounded set in $\BbbR^r$, 
where $r\ge 1$, and $h:K\to\Bbb R$ a continuous function.   Show that 
for any $\epsilon>0$ there is a polynomial $p$ in $r$ variables such 
that $|h(x)-p(x)|\le\epsilon$ for every $x\in K$. 
%281F 
      
\sqheader 281Xc Let $[a,b]$ be a non-empty closed 
interval of $\Bbb R$ and $h:[a,b]\to\Bbb R$ a continuous function. 
Show that for any $\epsilon>0$ there are 
$y_0,\ldots,y_n,a_0,\ldots,a_n,b_0,\ldots,b_n\in\Bbb R$ such that 
      
\Centerline{$|h(x)-\sum_{k=0}^n(a_k\cos y_kx+b_k\sin y_kx)|\le\epsilon$ 
for every $x\in [a,b]$,} 
      
\Centerline{$\sup_{x\in\Bbb R}|\sum_{k=0}^n(a_k\cos y_kx+b_k\sin 
y_kx)|\le\sup_{x\in [a,b]}|h(x)|$.} 
%281H 
      
\spheader 281Xd Let $h$ be a complex-valued function on 
$\ocint{-\pi,\pi}$ such that $|h|^p$ is integrable, where 
$1\le p<\infty$.   Show that for every $\epsilon>0$ there is a function 
of the form $x\mapsto f(x)=\sum_{k=-n}^nc_ke^{ikx}$, where 
$c_{-k},\ldots,c_k\in\Bbb C$, such that 
$\int_{-\pi}^{\pi}|h-f|^p\le\epsilon$.   (Compare 244H.) 
%281H 
      
\sqheader 281Xe Let $h:[-\pi,\pi]\to\Bbb R$ be a continuous 
function such that $h(\pi)=h(-\pi)$, and $\epsilon>0$.   Show that there 
are $a_0,\ldots,a_n,b_1,\ldots,b_n\in\Bbb R$ such that 
      
\Centerline{$|h(x)-\Bover12a_0-\sum_{k=1}^n(a_k\cos kx+b_k\sin kx)| 
\le\epsilon$} 
      
\noindent for every $x\in[-\pi,\pi]$. 
%281J 
      
\spheader 281Xf Let $K$ be a non-empty closed bounded set in $\BbbR^r$, 
where $r\ge 1$, and $h:K\to\Bbb R$ a continuous function.   Show that 
for any $\epsilon>0$ there are $y_0,\ldots,y_n\in\BbbR^r$, 
$a_0,\ldots,a_n,b_0,\ldots,b_n\in\Bbb R$ such that 
      
\Centerline{$|h(x) 
  -\sum_{k=0}^n(a_k\cos(y_k\dotproduct x)+b_k\sin(y_k\dotproduct x))| 
\le\epsilon$ for every $x\in K$,} 
      
\Centerline{$\sup_{x\in\Bbb R} 
 |\sum_{k=0}^n(a_k\cos(y_k\dotproduct x)+b_k\sin(y_k\dotproduct x))| 
\le\sup_{x\in K}|h(x)|$,} 
      
\noindent interpreting $y\dotproduct x$ as in 281K. 
%281K 
      
\spheader 281Xg Let $y_1,\ldots,y_r$ be real numbers such that
$1,y_1,\ldots,y_r$ are 
not linearly independent over $\Bbb Q$.   Show that there is a 
non-trivial interval $[a,b]\subseteq[\tbf{0},\tbf{1}]\subseteq\BbbR^r$ 
such that $(\fraction{my_1},\ldots,\fraction{my_r})\notin[a,b]$ for 
every $m\in\Bbb Z$. 
%281M 
      
\spheader 281Xh Let $\eta_1,\ldots,\eta_r$ be real numbers such 
that $1,\eta_1,\ldots,\eta_r$ are 
linearly independent over $\Bbb Q$.   Suppose that 
$0\le\alpha_j\le\beta_j\le 1$ for each $j\le r$.   
Show that for every $\epsilon>0$ there is an $n_0\in\Bbb N$ such that 
      
\Centerline{$|\prod_{j=1}^r(\beta_j-\alpha_j)
 -\Bover1{n+1}\#(\{m:k\le m\le k+n,\,\fraction{m\eta_j}
 \in[\alpha_j,\beta_j]$ for every $j\le r\})|\le\epsilon$} 
 
\noindent whenever $n\ge n_0$ and $k\in\Bbb N$.   ({\it Hint\/}: 
in the proof of 281N, set  
 
\Centerline{$\overline{L}(f) 
=\limsup_{n\to\infty}\sup_{k\in\Bbb N}\Bover1{n+1}\sum_{m=k}^{k+n}f(\fraction{my})$.)} 
%281N 
      
\leader{281Y}{Further exercises (a)} Show that under the hypotheses of 
281A, there is 
an $f\in\overline{A}$, the $\|\,\|_{\infty}$-closure of $A$ in $C_b(X)$, 
such that $f\restr K=h$.   ({\it Hint\/}:  take $f=\lim_{n\to\infty}f_n$ 
where 
      
\Centerline{$\|f_{n+1}-f_n\|_{\infty}\le\sup_{x\in K}|f_n(x)-h(x)| 
\le 2^{-n}$} 
      
\noindent for every $n\in\Bbb N$.) 
%281A 
      
      
\spheader 281Yb Let $X$ be a topological space and 
$K\subseteq X$ a compact subset.   Suppose that for any distinct points 
$x$, $y$ of $K$ there is a continuous function $f:X\to\Bbb R$ such that 
$f(x)\ne f(y)$.   Show that for any $r\in\Bbb N$ and any continuous 
$h:K\to\BbbR^r$ there is a continuous $f:X\to\BbbR^r$ extending $h$. 
\Hint{consider $r=1$ first.} 
%281Ya, 281A 
      
\spheader 281Yc Let $\langle X_i\rangle_{i\in I}$ be any family 
of compact Hausdorff spaces, and $X$ their product as topological 
spaces.   For each $i$, write $C(X_i)$ for the set of continuous 
functions from $X_i$ to $\Bbb R$, and $\pi_i:X\to X_i$ for the 
coordinate map.   Show that the subalgebra of $C(X)$ generated by  
$\{f\pi_i:i\in I,\,f\in C(X_i)\}$ is 
$\|\,\|_{\infty}$-dense in $C(X)$.   ({\it Note\/}:  you will need to 
know that $X$ is compact, and that if $Z$ is any compact Hausdorff space 
then for any distinct $z$, $w\in Z$ there is an $f\in C(Z)$ such that 
$f(z)\ne f(w)$.   For references see 3A3J and 3A3Bf in the next volume.) 
%281A 
      
\spheader 281Yd Let $X$ be a topological space and $K$ a compact 
subset of $X$.   Let $A$ be a linear subspace of the space $C_b(X)$ of 
bounded
real-valued continuous functions on $X$ such that $|f|\in A$ for every 
$f\in A$.   Let $h:K\to\Bbb R$ be a continuous function such that 
whenever $x$, $y\in K$ there is an $f\in A$ such that $f(x)=h(x)$ and 
$f(y)=h(y)$.   Show that for every $\epsilon>0$ there is an $f\in A$ 
such that $|f(x)-h(x)|\le\epsilon$ for every $x\in K$. 
%281A 
      
\spheader 281Ye Let $X$ be a compact topological space and write 
$C(X)$ for the set of continuous functions from $X$ to $\Bbb R$. Suppose 
that $h\in C(X)$, and let $A\subseteq C(X)$ be such that 
      
\inset{$A$ is a linear subspace of $C(X)$;} 
      
\inset{{\it either} $|f|\in A$ for every $f\in A$ {\it or} 
$f\times g\in A$ for every $f$, $g\in A$ 
{\it or} $f\times f\in A$ for every $f\in A$;} 
      
\inset{whenever $x$, $y\in X$ and $\delta>0$ there is an $f\in A$ such 
that $|f(x)-h(x)|\le\delta$ and $|f(y)-h(y)|\le\delta$.} 
      
\noindent Show that for every $\epsilon>0$ there is an $f\in A$ such 
that $|h(x)-f(x)|\le\epsilon$ for every $x\in X$. 
%281E 
      
\spheader 281Yf Let $X$ be a compact topological space and $A$ a 
$\|\,\|_{\infty}$-closed linear subspace of the space $C(X)$ of 
continuous functions from $X$ to $\Bbb R$.   Show that the following are 
equiveridical: 
      
\quad (i) $|f|\in A$ for every $f\in A$; 
      
\quad (ii) $f\times f\in A$ for every $f\in A$; 
      
\quad (iii) $f\times g\in A$ for all $f$, $g\in A$, 
      
\noindent and that in this case $A$ is closed in $C(X)$ for the topology 
defined by the pseudometrics 
      
\Centerline{$(f,g)\mapsto|f(x)-g(x)|: 
  C(X)\times C(X)\to\coint{0,\infty}$} 
      
\noindent as $x$ runs over $X$ (the `topology of pointwise convergence' 
on $C(X)$). 
%281E 
      
\spheader 281Yg Show that under the hypotheses of 281G there is 
an $f\in\overline{A}$, the $\|\,\|_{\infty}$-closure of $A$ in 
$C_b(X;\Bbb C)$, such that $f\restr K=h$ and (if $K\ne\emptyset$) 
$\|f\|_{\infty}=\sup_{x\in K}|h(x)|$. 
%281G 
      
\spheader 281Yh Let $y\in\Bbb R$ be irrational.   Show that for 
any Riemann integrable function $f:[0,1]\to\Bbb R$, 
      
\Centerline{$\int_0^1f(x)dx 
=\lim_{n\to\infty}\Bover1{n+1}\sum_{m=0}^nf(\fraction{my})$,} 
      
\noindent writing $\fraction{my}$ for the fractional part of $my$. 
({\it Hint\/}:  recall {\it Riemann's criterion:} for any $\epsilon>0$, 
there are $a_0,\ldots,a_n$ with $0=a_0\le a_1\le \ldots\le a_n=1$ and 
      
\Centerline{$\sum\{a_{j}-a_{j-1}:j\le n,\,\sup_{x\in[a_{j-1},a_j]}f(x) 
-\inf_{x\in[a_{j-1},a_j]}f(x)\ge\epsilon\}\le\epsilon$.)} 
%281N 
      
\ifdim\pagewidth>467pt\fontdimen3\tenrm=1.84pt 
  \fontdimen4\tenrm=1.22pt\fi 
\spheader 281Yi Let $\sequencen{t_n}$ be a sequence in $[0,1]$.   Show 
that the following are equiveridical: 
(i) $\lim_{n\to\infty}\bover1{n+1}\sum_{k=0}^nf(t_k) 
\ifdim\pagewidth>467pt\penalty-100\fi 
=\int_0^1f$ for every continuous function $f:[0,1]\to\Bbb R$; 
(ii) 
\ifdim\pagewidth=390pt\break\fi 
$\lim_{n\to\infty}\bover1{n+1}\sum_{k=0}^nf(t_k)=\int_0^1f$ for 
every Riemann integrable function $f:[0,1]\to\Bbb R$; 
(iii) $\liminf_{n\to\infty}\bover1{n+1}\#(\{k:k\le n,\,t_k\in G\}) 
\ge\mu G$ for every open set $G\subseteq[0,1]$, where $\mu$ is Lebesgue
measure on $\Bbb R$; 
(iv) $\lim_{n\to\infty}\bover1{n+1}\#(\{k:k\le n,\,t_k\le\alpha\}) 
=\alpha$ for every $\alpha\in[0,1]$; 
(v) $\lim_{n\to\infty}\bover1{n+1}\#(\{k:k\le n,\,t_k\in E\})=\mu E$ for 
every $E\subseteq[0,1]$ such that $\mu(\interior E)=\mu\overline{E}$ 
(vi) $\lim_{n\to\infty}\bover1{n+1}\sum_{k=0}^ne^{2\pi imt_k}=0$ for 
every $m\ge 1$. 
(Cf. 273J.   Such sequences $\sequencen{t_n}$ are called {\bf 
equidistributed} or {\bf uniformly distributed}.) 
%281N 
\fontdimen3\tenrm=1.67pt\fontdimen4\tenrm=1.11pt 
      
\spheader 281Yj Show that the sequence $\sequencen{\fraction{\ln(n+1)}}$ 
is not equidistributed. 
%281Yi, 281N 
      
\spheader 281Yk Give $[0,1]^{\Bbb N}$ its product measure $\lambda$. 
Show that $\lambda$-almost every sequence 
$\sequencen{t_n}\in[0,1]^{\Bbb N}$ is equidistributed in the sense of 
281Yi.   \Hint{273J.} 
%281Yi, 281N 
      
\spheader 281Yl Let $f:[0,1]^2\to\Bbb C$ be a continuous function. 
Show that if $\gamma\in\Bbb R$ is irrational then 
$\lim_{a\to\infty}\bover1a\int_0^af(\fraction{t},\fraction{\gamma t})dt 
\penalty-100=\int_{[0,1]^2}f$. 
\Hint{first consider functions of the form 
$x\mapsto e^{2\pi ik\dotproduct x}$.} 
%281N 
 
\spheader 281Ym A sequence $\sequencen{t_n}$ in $[0,1]$ is  
{\bf well-distributed} (with respect to Lebesgue measure $\mu$) if  

\Centerline{$\liminf_{n\to\infty}\inf_{l\in\Bbb N}
  \Bover1{n+1}\#(\{k:l\le k\le l+n$,  
$t_k\in G\})\ge\mu G$}

\noindent for every open set $G\subseteq[0,1]$    
(i) Show that $\sequencen{t_n}$ is well-distributed iff  
$\lim_{n\to\infty}\sup_{l\in\Bbb N} 
|\int_0^1f-\bover1{n+1}\sum_{k=l}^{l+n}f(t_k)|=0$ for every continuous  
$f:[0,1]\to\Bbb R$.    
(ii) Show that $\sequencen{\fraction{n\alpha}}$ is well-distributed for  
every irrational $\alpha$. 
%281Xh 281Yi 281N 
}%end of exercises 
      
\endnotes{\Notesheader{281} I have given three statements (281A, 
281E and 281G) of the Stone-Weierstrass theorem, with an acknowledgment 
(281F) of Weierstrass' own version, and three further forms (281Ya, 
281Yd, 281Yg) in the exercises.   Yet another will appear in \S4A6 in 
Volume 4.   Faced with such a multiplicity, you may wish to try 
your own hand at writing out theorems which will cover some or all of 
these versions.   I myself see no way of doing it without setting up a 
confusing list of alternative hypotheses and conclusions.   At which 
point, I ask `what is a theorem, anyway?', and answer, it is a 
stopping-place on our journey;  it is a place where we can rest, and 
congratulate ourselves on our achievement;  it is a place which we can 
learn to recognise, and use as a starting point for new adventures;  it 
is a place we can describe, and share with others.   For some theorems, 
like Fermat's last theorem, there is a canonical statement, an exactly 
locatable 
point.   For others, like the Stone-Weierstrass theorem here, we reach a 
mass of closely related results, all depending on some arrangement of 
the arguments laid out in 281A-281G and 281Ya (which introduces a new 
idea), and all useful in different ways.   I suppose, indeed, that most 
authors would prefer the versions 281Ya and 281Yg, which eliminate the 
variable 
$\epsilon$ which appears in 281A, 281E and 281G, at the expense of 
taking a closed subspace $A$.   But I find that the corollaries which 
will be useful later (281H-281L) are more naturally expressed in terms 
of linear subspaces which are not closed. 
      
The applications of the theorem, or the theorems, 
or the method -- choose your own expression -- are legion; 
only a few of them are here. 
An apparently innocent one is in 281Xa and, in a different variant, in 
281Yc;  these are enormously important in their own domains.   In this 
volume the principal application will be to 285L below, depending on 
281K, and it is perhaps right to note that there is an alternative 
approach to this particular result, based on ideas in 282G.   But I 
offer Weyl's equidistribution theorem 
(281M-281N) as evidence that we can expect to find good use for these 
ideas in almost any branch of mathematics. 
} 
      
\discrpage 
      
