\frfilename{mt366.tex}
\versiondate{10.11.08}
\copyrightdate{1995}

\def\chaptername{Function spaces}
\def\sectionname{$L^p$}

\newsection{366}

In this section I apply the methods of this chapter to $L^p$ spaces,
where $1<p<\infty$.   The constructions proceed without surprises up to
366E, translating the ideas of \S244 by the methods used in \S365.
Turning to the action of Boolean homomorphisms on $L^p$ spaces, I
introduce a
space $M^0$, which can be regarded as the part of $L^0$ that can be
determined from the ring $\frak A^f$ of elements of $\frak A$ of finite
measure (366F), and which includes $L^p$ whenever $1\le p<\infty$.   Now
a measure-preserving ring homomorphism from $\frak A^f$ to $\frak B^f$
acts on the $M^0$ spaces in a way which includes injective Riesz
homomorphisms from $L^p(\frak A,\bar\mu)$ to $L^p(\frak B,\bar\nu)$ and
surjective positive linear operators from $L^p(\frak B,\bar\nu)$ to
$L^p(\frak A,\bar\mu)$ (366H).   The latter may be regarded as
conditional expectation operators (366J).   The case $p=2$ (366K-366L)
is of course by far the most important.   As with the familiar spaces
$L^p(\mu)$ of Chapter 24, we have complex versions
$L^p_{\Bbb C}(\frak A,\bar\mu)$ with the expected properties (366M).

\leader{366A}{Definition}  Let $(\frak A,\bar\mu)$ be a measure algebra
and suppose that $1<p<\infty$.   For $u\in L^0(\frak A)$, define
$|u|^p\in L^0(\frak A)$ by setting

$$\eqalign{\Bvalue{|u|^p>\alpha}
&=\Bvalue{|u|>\alpha^{1/p}}\text{ if }\alpha\ge 0,\cr
&=1\text{ if }\alpha<0.\cr}$$

\noindent\cmmnt{(In the language of 364H, $|u|^p=\bar h(u)$,
where $h(t)=|t|^p$ for $t\in\Bbb R$.)   }Set

\Centerline{$L^p_{\bar\mu}=L^p(\frak A,\bar\mu)
=\{u:u\in L^0(\frak A),\,|u|^p\in L^1(\frak A,\bar\mu)\}$,}

\noindent and for $u\in L^0(\frak A)$ set

\Centerline{$\|u\|_p=(\int|u|^p)^{1/p}=\||u|^p\|_1^{1/p}$,}

\noindent counting $\infty^{1/p}$ as $\infty$\cmmnt{, so that
$L^p_{\bar\mu}=\{u:u\in L^0(\frak A)$, $\|u\|_p<\infty\}$.}

\leader{366B}{Theorem} Let $(X,\Sigma,\mu)$ be a measure space, and
$(\frak A,\bar\mu)$ its measure algebra.   Then the canonical
isomorphism between $L^0(\mu)$ and $L^0(\frak A)$\cmmnt{ (364Ic)}
makes $L^p(\mu)$\cmmnt{, as defined in \S244,} correspond to
$L^p(\frak A,\bar\mu)$.

\proof{ What we really have to check is that if $w\in L^0(\mu)$
corresponds to $u\in L^0(\frak A)$, then $|w|^p$, as defined in 244A,
corresponds to $|u|^p$ as defined in 366A.   But this was noted in
364Ib.

Now, because the isomorphism between $L^0(\mu)$ and $L^0(\frak A)$
matches $L^1(\mu)$ with $L^1_{\bar\mu}$ (365B), we can be sure
that $|w|^p\in L^1(\mu)$ iff $|u|^p\in L^1_{\bar\mu}$, and that
in this case

\Centerline{$\|w\|_p=\bigl(\int|w|^p\bigr)^{1/p}
=\bigl(\int|u|^p\bigr)^{1/p}=\|u\|_p$,}

\noindent as required.
}%end of proof of 366B

\leader{366C}{Corollary} For any measure algebra $(\frak A,\bar\mu)$ and
$p\in\ooint{1,\infty}$, $L^p=L^p(\frak A,\bar\mu)$ is a solid linear
subspace of $L^0(\frak A)$.   It is a Dedekind complete Banach lattice
under its uniformly convex
norm $\|\,\|_p$.   Setting $q=p/(p-1)$, $(L^p)^*$ is
identified with $L^q(\frak A,\bar\mu)$ by the duality
$(u,v)\mapsto\int u\times v$.   Writing $\frak A^f$ for the ring
$\{a:a\in\frak A,\,\bar\mu a<\infty\}$, $S(\frak A^f)$ is norm-dense in
$L^p$.

\proof{ Because we can find a measure space $(X,\Sigma,\mu)$ such that
$(\frak A,\bar\mu)$ is isomorphic to the measure algebra of $\mu$
(321J), this is just a digest of the results in 244B,
244E-244H, %244E 244F 244G 244H
244K, 244L and 244O\Latereditions.
(Of course $S(\frak A^f)$ corresponds to
the space $S$ of equivalence classes of simple functions in 244Ha,
just as in 365F.)
}%end of proof of 366C

\leader{366D}{}\cmmnt{ I can add a little more, corresponding to
365C and 365M.

\medskip

\noindent}{\bf Theorem} Let $(\frak A,\bar\mu)$ be a measure algebra,
and $p\in\ooint{1,\infty}$.

(a) The norm $\|\,\|_p$ on $L^p=L^p(\frak A,\bar\mu)$ is
order-continuous.

(b) $L^p$ has the Levi property.

(c) Setting $q=p/(p-1)$, the canonical identification of
$L^q=L^q(\frak A,\bar\mu)$ with $(L^p)^*$ is a Riesz space isomorphism
between $L^q$ and $(L^p)^{\sim}=(L^p)^{\times}$.

(d) $L^p$ is a perfect Riesz space.

\proof{{\bf (a)} Suppose that $A\subseteq L^p$ is non-empty,
downwards-directed and has infimum $0$.   For $u$, $v\ge 0$ in $L^p$,
$u\le v\Rightarrow u^p\le v^p$ (by the definition in 366A, or
otherwise), so $B=\{u^p:u\in A\}$ is downwards-directed.   If
$v_0=\inf B$ in $L^1=L^1(\frak A,\bar\mu)$, then $v_0^{1/p}$ (defined by
the formula in 366A, or otherwise) is less than or equal to every member
of $A$, so must be $0$, and $v_0=0$.   Accordingly $\inf B=0$ in $L^1$.
Because $\|\,\|_1$ is order-continuous (365C),

\Centerline{$\inf_{u\in A}\|u\|_p=\inf_{u\in A}\|u^p\|_1^{1/p}
=(\inf_{v\in B}\|v\|_1)^{1/p}=0$.}

\noindent As $A$ is arbitrary, $\|\,\|_p$ is order-continuous.

\medskip

{\bf (b)} Now suppose that $A\subseteq (L^p)^+$ is non-empty,
upwards-directed and norm-bounded.   Then $B=\{u^p:u\in A\}$ is
non-empty, upwards-directed and norm-bounded in $L^1$.   So $v_0=\sup B$
is defined in $L^1$, and $v_0^{1/p}$ is an upper bound for $A$ in $L^p$.

\medskip

{\bf (c)} By 356Dd, $(L^p)^*=(L^p)^{\sim}=(L^p)^{\times}$.   The
extra information we need is that the identification of $L^q$ with
$(L^p)^*$ is an order-isomorphism.   \Prf\ ($\alpha$) If $w\in (L^q)^+$
and $u\in (L^p)^+$ then $u\times w\ge 0$ in $L^1$, so
$(Tw)(u)=\int u\times w\ge 0$, writing $T:L^q\to(L^p)^*$ for the
canonical bijection.   As $u$
is arbitrary, $Tw\ge 0$.  As $w$ is arbitrary, $T$ is a positive linear
operator.   ($\beta$) If $w\in L^q$ and $Tw\ge 0$, consider
$u=(w^-)^{q/p}$.   Then $u\ge 0$ in $L^p$ and $w^+\times u=0$
(because $\Bvalue{w^+>0}\Bcap\Bvalue{u>0}
=\Bvalue{w^+>0}\Bcap\Bvalue{w^->0}=0$), so

\Centerline{$0\le (Tw)(u)=\int w\times u=-\int w^-\times u
=-\int(w^-)^q\le 0$,}

\noindent and $\int(w^-)^q=0$.   But as $(w^-)^q\ge 0$ in $L^1$, this
means that $(w^-)^q$ and $w^-$ must be $0$, that is, $w\ge 0$.   As $w$
is arbitrary, $T^{-1}$ is positive and $T$ is an order-isomorphism.\
\Qed

\medskip

{\bf (d)} This is an immediate consequence of (c), since $p=q/(q-1)$, so
that $L^p$ can be identified with $(L^q)^*=(L^q)^{\times}$.   From
356M we see that it is also a consequence of (a) and (b).
}%end of proof of 366D

\leader{366E}{Proposition} Let $(\frak A,\bar\mu)$ be a semi-finite
measure algebra, and $p\in[1,\infty]$.   Set $q=p/(p-1)$ if
$1<p<\infty$, $q=\infty$ if $p=1$ and $q=1$ if $p=\infty$.   Then

\Centerline{$L^q(\frak A,\bar\mu)
=\{u:u\in L^0(\frak A),\,u\times v\in L^1(\frak A,\bar\mu)$ for every
$v\in L^p(\frak A,\bar\mu)\}$.}

\proof{{\bf (a)} We already know that if $u\in L^p=L^p(\frak A,\bar\mu)$
and $v\in L^q=L^q(\frak A,\bar\mu)$ then
$u\times v\in L^1=L^1(\frak A,\bar\mu)$;  this is elementary if
$p\in\{1,\infty\}$ and otherwise is covered by 366C.

\medskip

{\bf (b)} So suppose that $u\in L^0\setminus L^p$.   If $p=1$ then of
course $\chi 1\in L^{\infty}=L^q$ and $u\times\chi 1\notin L^1$.   If
$p>1$ set

\Centerline{$A=\{w:w\in S(\frak A^f),\,0\le w\le|u|\}$.}

\noindent Because $\bar\mu$ is semi-finite, $S(\frak A^f)$ is
order-dense in $L^0$ (364K), and $|u|=\sup A$.   Because the
norm on $L^p$ has the Levi property (365C, 366Db, 363Ba) and $A$ is not
bounded above in $L^p$, $\sup_{w\in A}\|w\|_p=\infty$.

For each $n\in\Bbb N$ choose $w_n\in A$ with $\|w_n\|_p>4^n$.   Then
there is a $v_n\in L^q$ such that $\|v_n\|_q=1$ and $\int w_n\times
v_n\ge 4^n$.   \Prf\ ($\alpha$) If $p<\infty$ this is covered by 366C,
since $\|w_n\|_p=\sup\{\int w_n\times v:\|v\|_q\le 1\}$.   ($\beta$) If
$p=\infty$ then $\Bvalue{w_n>4^n}\ne 0$;  because $\bar\mu$ is
semi-finite, there is a $b\Bsubseteq\Bvalue{w_n>4^n}$ such that
$0<\bar\mu b<\infty$, and $\|\bover1{\bar\mu b}\chi b\|_1=1$, while
$\int w_n\times\bover1{\bar\mu b}\chi b\ge 4^n$.\ \Qed

Because $L^q$ is complete (363Ba, 366C),
$v=\sum_{n=0}^{\infty}2^{-n}|v_n|$ is defined in $L^q$.   But now

\Centerline{$\int|u|\times v\ge2^{-n}\int w_n\times v_n\ge 2^n$}

\noindent for every $n$, so $u\times v\notin L^1$.
}%end of proof of 366E

\cmmnt{\medskip

\noindent{\bf Remark} This result is characteristic of perfect subspaces
of $L^0$;  see 369C and 369J.}

\leader{366F}{}\cmmnt{ The next step is to look at the action of
Boolean homomorphisms, as in 365O.   It will be convenient to be able
to deal with all $L^p$ spaces at once by introducing names for a pair of
spaces which include all of them.

\medskip

\noindent}{\bf Definition} Let $(\frak A,\bar\mu)$ be a measure algebra.
Write

\Centerline{$M^0_{\bar\mu}
=M^0(\frak A,\bar\mu)
=\{u:u\in L^0(\frak A),\,
  \bar\mu\Bvalue{|u|>\alpha}<\infty$ for every $\alpha>0\}$,}

\Centerline{$M^{1,0}_{\bar\mu}
=M^{1,0}(\frak A,\bar\mu)
=\{u:u\in M^0_{\bar\mu},\,u\times\chi a\in L^1(\frak A,\bar\mu)$
  whenever $\bar\mu a<\infty\}$.}

\leader{366G}{Lemma} Let $(\frak A,\bar\mu)$ be any measure algebra.
Write $M^0=M^0(\frak A,\bar\mu)$, etc.

(a) $M^0$ and $M^{1,0}$ are Dedekind complete solid linear subspaces of
$L^0$ which include $L^p$ for every $p\in\coint{1,\infty}$;  moreover,
$M^0$ is closed under multiplication.

(b) If $u\in M^0$ and $u\ge 0$, there is a
non-decreasing sequence $\sequencen{u_n}$ in $S(\frak A^f)$ such that
$u=\sup_{n\in\Bbb N}u_n$.

(c) $M^{1,0}
=\{u:u\in L^0,\,(|u|-\epsilon\chi 1)^+\in L^1$ for every $\epsilon>0\}
=L^1+(L^{\infty}\cap M^0)$.

(d) If $u$, $v\in M^{1,0}$ and $\int_au\le\int_av$ whenever
$\bar\mu a<\infty$, then $u\le v$;  so if $\int_au=\int_av$ whenever
$\bar\mu a<\infty$, $u=v$.

\proof{{\bf (a)} If $u$, $v\in M^0$ and $\gamma\in\Bbb R$, then for any
$\alpha>0$

\Centerline{$\Bvalue{|u+v|>\alpha}
\Bsubseteq\Bvalue{|u|>\bover12\alpha}
\Bcup\Bvalue{|v|>\bover12\alpha}$,}

\Centerline{$\Bvalue{|\gamma u|>\alpha}
\Bsubseteq\Bvalue{|u|>\bover{\alpha}{1+|\gamma|}}$,}

\Centerline{$\Bvalue{|u\times v|>\alpha}
\Bsubseteq\Bvalue{|u|>\sqrt{\alpha}\thinspace}
\Bcup\Bvalue{|v|>\sqrt{\alpha}\thinspace}$}

\noindent (364E) are of finite measure.   So $u+v$, $\gamma u$ and
$u\times v$ belong to $M^0$.   Thus $M^0$ is a linear subspace of $L^0$
closed under
multiplication.   If $u\in M^0$, $|v|\le |u|$ and $\alpha>0$, then
$\Bvalue{|v|>\alpha}\Bsubseteq\Bvalue{|u|>\alpha}$ has finite measure;
thus $v\in M^0$ and $M^0$ is a solid linear subspace of $L^0$.   It
follows that $M^{1,0}$ also is.   If $u\in L^p=L^p(\frak A,\bar\mu)$,
where $p<\infty$, and $\alpha>0$, then
$\Bvalue{|u|>\alpha}=\Bvalue{|u|^p>\alpha^p}$ has finite measure, so
$u\in M^0$;  moreover, if $\bar\mu a<\infty$, then $\chi a\in L^q$,
where $q=p/(p-1)$, so $u\times\chi a\in L^1$;  thus $u\in M^{1,0}$.

To see that $M^0$ is Dedekind complete, observe that if
$A\subseteq(M^0)^+$ is non-empty and bounded above by $u_0\in M^0$,
and $\alpha>0$, then
$\{\Bvalue{u>\alpha}:u\in A\}$ is bounded above by
$\Bvalue{u_0>\alpha}\in\frak A^f$, so has a
supremum in $\frak A$ (321C).   Accordingly $\sup A$ is defined in $L^0$
(364L(a-iii)) and belongs to $M^0$.
Finally, $M^{1,0}$, being a solid linear
subspace of $M^0$, must also be Dedekind complete.

\medskip

{\bf (b)} If $u\ge 0$ in $M^0$, then there is a non-decreasing sequence
$\sequencen{u_n}$ in $S=S(\frak A)$ such that $u=\sup_{n\in\Bbb N}u_n$
and $u_0\ge 0$ (364Jd).   But now every $u_n$ belongs to
$S\cap M^0=S(\frak A^f)$, just as in 365F.

\medskip

{\bf (c)(i)} If $u\in M^{1,0}$ and $\epsilon>0$, then
$a=\Bvalue{|u|>\epsilon}\in\frak A^f$, so $u\times\chi a\in
L^1=L^1(\frak A,\bar\mu)$;  but $(|u|-\epsilon\chi 1)^+\le |u|\times\chi
a$, so $(|u|-\epsilon\chi 1)^+\in L^1$.

\medskip

\quad{\bf (ii)} Suppose that $u\in L^0$ and $(|u|-\epsilon\chi 1)^+\in
L^1$ for every $\epsilon>0$.   Then, given $\epsilon>0$,
$v=(|u|-\bover12\epsilon\chi 1)^+\in L^1$, and
$\bar\mu\Bvalue{v>\bover12\epsilon}<\infty$;  but
$\Bvalue{|u|>\epsilon}\Bsubseteq\Bvalue{v>\bover12\epsilon}$, so also
has finite measure.   Thus $u\in M^0$.   Next, if $a\in\frak A^f$, then
$|u\times\chi a|\le\chi a+(|u|-\chi 1)^+\in L^1$, so $u\in M^{1,0}$.

\medskip

\quad{\bf (iii)} Of course $L^1$ and $L^{\infty}\cap M^0$ are included
in $M^{1,0}$, so their linear sum also is.   On the other hand, if $u\in
M^{1,0}$, then

\Centerline{$u=(u^+-\chi 1)^+-(u^--\chi 1)^++(u^+\wedge\chi 1)
-(u^-\wedge\chi 1)\in L^1+(L^{\infty}\cap M^0)$.}

\medskip

{\bf (d)} Take $\alpha>0$ and set $a=\Bvalue{u-v>\alpha}$.   Because
both $u$ and
$v$ belong to $M^{1,0}_{\bar\mu}$, $\bar\mu a<\infty$ and
$\int_au\le\int_av$, that is, $\int_au-v\le 0$;  so $a$ must be $0$
(365Dc).   As $\alpha$ is arbitrary, $u-v\le 0$ and $u\le v$.
If $\int_au=\int_av$ for every $a\in\frak A^f$, then $v\le u$ so
$u=v$.
}%end of proof of 366G

\leader{366H}{Theorem} Let $(\frak A,\bar\mu)$ and $(\frak B,\bar\nu)$
be measure algebras.   Let $\pi:\frak A^f\to\frak B^f$ be a
measure-preserving ring homomorphism.

(a)(i) We have a unique order-continuous Riesz homomorphism
$T=T_{\pi}:M^0(\frak A,\bar\mu)\to M^0(\frak B,\bar\nu)$ such that
$T(\chi a)=\chi(\pi a)$ for every $a\in\frak A^f$.

\quad(ii) $\Bvalue{Tu>\alpha}=\pi\Bvalue{u>\alpha}$ for every
$u\in M^0(\frak A,\bar\mu)$ and $\alpha>0$.

\quad(iii) $T$ is injective and multiplicative.

\quad(iv) For $p\in[1,\infty]$ and $u\in M^0(\frak A,\bar\mu)$,
$\|Tu\|_p=\|u\|_p$;  in particular, $Tu\in L^p(\frak B,\bar\nu)$ iff
$u\in L^p(\frak A,\bar\mu)$.   \cmmnt{Consequently }$\int Tu=\int u$
whenever $u\in L^1(\frak A,\bar\mu)$.

\quad(v) For $u\in M^0(\frak A,\bar\mu)$,
$Tu\in M^{1,0}(\frak B,\bar\nu)$ iff $u\in M^{1,0}(\frak A,\bar\mu)$.

(b)(i) We have a unique order-continuous positive linear operator
$P=P_{\pi}:M^{1,0}(\frak B,\bar\nu)\to M^{1,0}(\frak A,\bar\mu)$ such
that
$\int_aPv=\int_{\pi a}v$ whenever $v\in M^{1,0}(\frak B,\bar\nu)$ and
$a\in\frak A^f$.

\quad(ii) If $u\in M^0(\frak A,\bar\mu)$,
$v\in M^{1,0}(\frak B,\bar\nu)$ and
$v\times Tu\in M^{1,0}(\frak B,\bar\nu)$, then
$P(v\times Tu)=u\times Pv$.

\quad(iii) If $q\in\coint{1,\infty}$ and $v\in L^q(\frak B,\bar\nu)$,
then
$Pv\in L^q(\frak A,\bar\mu)$ and $\|Pv\|_q\le\|v\|_q$;  if
$v\in L^{\infty}(\frak B)\cap M^0(\frak B,\bar\nu)$, then
$Pv\in L^{\infty}(\frak A)$ and $\|Pv\|_{\infty}\le\|v\|_{\infty}$.

\quad(iv) $PTu=u$ for every $u\in M^{1,0}(\frak A,\bar\mu)$;  in
particular, $P[L^p(\frak B,\bar\nu)]=L^p(\frak A,\bar\mu)$ for every
$p\in\coint{1,\infty}$.

(c) If $(\frak C,\lambda)$ is another measure algebra and
$\theta:\frak B^f\to\frak C^f$ another measure-preserving ring
homomorphism, then $T_{\theta\pi}
=T_{\theta}T_{\pi}:M^0(\frak A,\bar\mu)\to M^0(\frak C,\bar\lambda)$ and
$P_{\theta\pi}=P_{\pi}P_{\theta}:M^{1,0}(\frak C,\bar\lambda)\to
M^{1,0}(\frak A,\bar\mu)$.

(d) Now suppose that $\pi[\frak A^f]=\frak B^f$, so that $\pi$ is a
measure-preserving isomorphism between the rings $\frak A^f$ and
$\frak B^f$.

\quad(i) $T$ is a Riesz space isomorphism between $M^0(\frak A,\bar\mu)$
and $M^0(\frak B,\bar\nu)$, and its inverse is $T_{\pi^{-1}}$.

\quad(ii) $P$ is a Riesz space isomorphism between
$M^{1,0}(\frak B,\bar\nu)$ and $M^{1,0}(\frak A,\bar\mu)$, and its
inverse is $P_{\pi^{-1}}$.

\quad(iii) The restriction of $T$ to $M^{1,0}(\frak A,\bar\mu)$ is
$P^{-1}=P_{\pi^{-1}}$;  the restriction of
$T^{-1}=T_{\pi^{-1}}$ to $M^{1,0}(\frak B,\bar\nu)$ is $P$.

\quad(iv) For any $p\in\coint{1,\infty}$,
$T\restr L^p(\frak A,\bar\mu)=P_{\pi^{-1}}\restr L^p(\frak A,\bar\mu)$
and $P\restr L^p(\frak B,\bar\nu)
=T_{\pi^{-1}}\restr L^p(\frak B,\bar\nu)$ are
the two halves of a Banach lattice isomorphism between
$L^p(\frak A,\bar\mu)$ and $L^p(\frak B,\bar\nu)$.

\proof{{\bf (a)(i)} By 361J, $\pi$ induces a multiplicative Riesz
homomorphism $T_0:S(\frak A^f)\to S(\frak B^f)$ which is
order-continuous because $\pi$ is (361Ad, 361Je).   If
$u\in S(\frak A^f)$ and $\alpha>0$, then
$\Bvalue{T_0u>\alpha}=\pi\Bvalue{u>\alpha}$.
\Prf\ Express $u$ as $\sum_{i=0}^n\alpha_i\chi a_i$ where
$a_0,\ldots,a_n$ are disjoint in $\frak A^f$;  then
$T_0u=\sum_{i=0}^n\alpha_i\chi(\pi a_i)$, so

\Centerline{$\Bvalue{T_0u>\alpha}
=\sup\{\pi a_i:i\le n,\,\alpha_i>\alpha\}
=\pi(\sup\{a_i:i\le n,\,\alpha_i>\alpha\})
=\pi\Bvalue{u>\alpha}$.   \Qed}

Now if $u_0\ge 0$ in $M^0_{\bar\mu}$,
$\sup\{T_0u:u\in S(\frak A^f),\,0\le u\le u_0\}$ is defined in
$M^0_{\bar\nu}$.   \Prf\ Set
$A=\{u:u\in S(\frak A^f),\,0\le u\le u_0\}$.   Because $u_0=\sup A$
(366Gb),

\Centerline{$\sup_{u\in A}\Bvalue{Tu>\alpha}
=\sup_{u\in A}\pi\Bvalue{u>\alpha}
=\pi(\sup_{u\in A}\Bvalue{u>\alpha})
=\pi\Bvalue{u_0>\alpha}$}

\noindent is defined and belongs to $\frak B^f$ for any $\alpha>0$.
Also

\Centerline{$\inf_{n\ge 1}\sup_{u\in A}\Bvalue{Tu>n}
=\pi(\inf_{n\ge 1}\Bvalue{u_0>n})=0$.}

\noindent By 364L(a-ii), $v_0=\sup T_0[A]$ is defined in $L^0(\frak B)$,
and $\Bvalue{v_0>\alpha}=\pi\Bvalue{u_0>\alpha}\in\frak B^f$ for every
$\alpha>0$, so $v_0\in M^0_{\bar\nu}$, as required.\ \Qed

Consequently $T_0$ has a unique extension to an order-continuous Riesz
homomorphism $T:M^0_{\bar\mu}\to M^0_{\bar\nu}$ (355F).

\medskip

\quad{\bf (ii)} If $u_0\in M^0_{\bar\mu}$ and $\alpha>0$, then

$$\eqalignno{\Bvalue{Tu_0>\alpha}
&=\Bvalue{Tu_0^+>\alpha}\cr
\noalign{\noindent (because $T$ is a Riesz homomorphism)}
&=\sup_{u\in S(\frak A^f),0\le u\le u_0^+}\Bvalue{Tu>\alpha}\cr
\noalign{\noindent (because $T$ is order-continuous and $S(\frak A^f)$
is order-dense in $M^0_{\bar\mu}$)}
&=\pi\Bvalue{u_0>\alpha}\cr}$$

\noindent by the argument used in (i).

\medskip

\quad{\bf (iii)} I have already remarked, at the beginning of the proof
of (i), that $T(u\times u')=Tu\times Tu'$ for $u$, $u'\in S(\frak A^f)$.
Because both $T$ and $\times$ are order-continuous and $S(\frak A^f)$ is
order-dense in $M^0_{\bar\mu}$,

$$\eqalign{T(u_0\times u_1)
&=\sup\{T(u\times u'):u,\,u'\in S(\frak A^f),\,0\le u\le u_0,\,
   0\le u'\le u_1\}\cr
&=\sup_{u,u'}Tu\times Tu'
=Tu_0\times Tu_1\cr}$$

\noindent whenever $u_0$, $u_1\ge 0$ in $M^0_{\bar\mu}$.   Because $T$
is linear and $\times$ is bilinear, it follows that $T$ is
multiplicative on $M^0_{\bar\mu}$.

To see that it is injective, observe that if $u\ne 0$ in $M^0_{\bar\mu}$
then there is some $\alpha>0$ such that $a=\Bvalue{|u|>\alpha}\ne 0$, so
that $0<\alpha\chi\pi a\le T|u|=|Tu|$ and $Tu\ne 0$.

\medskip

\quad{\bf (iv)}\grheada\ For any $\alpha>0$,

\Centerline{$\Bvalue{|Tu|^p>\alpha}
=\Bvalue{T|u|>\alpha^{1/p}}
=\pi\Bvalue{|u|>\alpha^{1/p}}
=\pi\Bvalue{|u|^p>\alpha}$.}

\noindent So

\Centerline{$\||Tu|^p\|_1
=\int_0^{\infty}\bar\nu\Bvalue{|Tu|^p>\alpha}\,d\alpha
=\int_0^{\infty}\bar\mu\Bvalue{|u|^p>\alpha}\,d\alpha
=\||u|^p\|_1$.}

\noindent If $p<\infty$ then, taking $p$th roots, $\|Tu\|_p=\|u\|_p$.

\medskip

\qquad\grheadb\ As for the case $p=\infty$, if
$u\in L^{\infty}(\frak A)$ and $\gamma=\|u\|_{\infty}>0$ then
$\Bvalue{|u|>\gamma}=0$, so
$\Bvalue{|Tu|>\gamma}=\pi\Bvalue{|u|>\gamma}=0$.   This shows that
$\|Tu\|_{\infty}\le\gamma$.   On the other hand, if $0<\alpha<\gamma$
then $a=\Bvalue{|u|>\alpha}\ne 0$, and
$\alpha\chi a\le|u|$ so $\alpha\chi(\pi a)\le|Tu|$;  as
$\pi a\ne 0$ (because $\bar\nu(\pi a)=\bar\mu a>0$),
$\|Tu\|_{\infty}>\alpha$.   This shows that
$\|Tu\|_{\infty}=\|u\|_{\infty}$, at least when $u\ne 0$;  but the case
$u=0$ is trivial.

\medskip

\qquad\grheadc\ If $u\in L^1_{\bar\mu}$, then

\Centerline{$\int Tu=\|(Tu)^+\|_1-\|(Tu)^-\|_1=\|Tu^+\|_1
-\|Tu^-\|_1=\|u^+\|_1-\|u^-\|_1=\int u$.}

\medskip

\quad{\bf (v)} If $u\in M^{1,0}_{\bar\mu}$ and $\epsilon>0$, then
$T(|u|\wedge\epsilon\chi 1_{\frak A})=|Tu|\wedge\epsilon\chi 1_{\frak
B}$.   \Prf\ Set $a=\Bvalue{|u|>\epsilon}\in\frak A^f$.   Then
$|u|\wedge\epsilon\chi 1_{\frak A}=\epsilon\chi a+|u|-|u|\times\chi a$
and
$\Bvalue{|Tu|>\epsilon}=\pi a$.   So

$$\eqalign{T(|u|\wedge\epsilon\chi 1_{\frak A})
&=T(\epsilon\chi a)+T|u|-T(|u|\times\chi a)\cr
&=\epsilon\chi(\pi a)+|Tu|-|Tu|\times\chi(\pi a)
=|Tu|\wedge\epsilon\chi 1_{\frak B}.  \text{ \Qed}}$$

Consequently

\Centerline{$T(|u|-\epsilon\chi 1_{\frak A})^+
=T(|u|-|u|\wedge\epsilon\chi 1_{\frak A})
=(|Tu|-\epsilon\chi 1_{\frak B})^+$.}

\noindent But this means that $(|u|-\epsilon\chi 1_{\frak A})^+\in
L^1_{\bar\mu}$ iff $(|Tu|-\epsilon\chi 1_{\frak B})^+\in L^1_{\bar\nu}$.
Since this is true for every $\epsilon>0$, 366Gc tells us that $u\in
M^{1,0}_{\bar\mu}$ iff $Tu\in M^{1,0}_{\bar\nu}$.

\medskip


{\bf (b)(i)}\grheada\ By 365Pa, we have an order-continuous positive
linear operator $P_0:L^1_{\bar\nu}\to L^1_{\bar\mu}$ such that
$\int_aP_0v=\int_{\pi a}v$ for every $v\in L^1_{\bar\nu}$ and
$a\in\frak A^f$.

\medskip

\qquad\grheadb\ We now find that if $v_0\ge 0$ in $M^{1,0}_{\bar\nu}$
and $B=\{v:v\in L^1_{\bar\nu},\,0\le v\le v_0\}$, then $P_0[B]$ has a
supremum in $L^0(\frak A)$ which belongs to $M^{1,0}_{\bar\mu}$.   \Prf\
Because $B$ is upwards-directed and $P_0$ is order-preserving, $P_0[B]$
is upwards-directed.   If $\alpha>0$ and $v\in B$ and
$a=\Bvalue{P_0v>\alpha}$, then

\Centerline{$v
\le(v_0-\bover{\alpha}2\chi 1_{\frak B})^+
  +\Bover{\alpha}2\chi 1_{\frak B}$,}

\noindent so

$$\eqalign{\alpha\bar\mu a
&\le\int_aP_0v
=\int_{\pi a}v
\le\int(v_0-\Bover{\alpha}2\chi 1_{\frak B})^+
 +\Bover{\alpha}2\bar\nu(\pi a)\cr
&=\int(v_0-\Bover{\alpha}2\chi 1_{\frak B})^+
 +\Bover{\alpha}2\bar\mu a\cr}$$

\noindent and

\Centerline{$\bar\mu\Bvalue{P_0v>\alpha}
\le\Bover2{\alpha}\int(v_0-\bover{\alpha}2\chi 1_{\frak B})^+$.}

\noindent Thus $\{\Bvalue{P_0v>\alpha}:v\in B\}$ is an upwards-directed
set in $\frak A^f$ with measures bounded above in $\Bbb R$, and

\Centerline{$c_{\alpha}=\sup_{v\in B}\Bvalue{P_0v>\alpha}$}
%\Centerline needed for Lulu version

\noindent is defined in $\frak A^f$.  Also

\Centerline{$\inf_{n\ge 1}\bar\mu c_n
\le\inf_{n\ge 1}\Bover2n\int(v_0-\bover{n}2\chi 1_{\frak B})^+
=0$.}

\noindent So $\inf_{n\in\Bbb N}c_n=0$ and $P_0[B]$ has a supremum
$u_0\in L^0(\frak A)$ (364L(a-ii)).   As
$\Bvalue{u_0>\alpha}=c_{\alpha}\in\frak A^f$ for every $\alpha>0$,
$u_0\in M^0_{\bar\mu}$.   If $c\in\frak A^f$, then

\Centerline{$\int_cu_0=\sup_{v\in B}\int_cP_0v
=\sup_{v\in B}\int_{\pi c}v\le\int_{\pi c}v_0<\infty$,}

\noindent so $u_0\in M^{1,0}_{\bar\mu}$.\ \Qed

\medskip

\qquad\grheadc\ Now 355F tells us that $P_0$ has a unique extension
to an order-continuous positive linear operator
$P:M^{1,0}_{\bar\nu}\to M^{1,0}_{\bar\mu}$.   If $v_0\ge 0$ in
$M^{1,0}_{\bar\nu}$ and $a\in\frak A^f$, then, as remarked above,

$$\eqalign{\int_aPv_0
&=\sup\{\int_aP_0v:v\in L^1_{\bar\nu},\,0\le v\le v_0\}\cr
&=\sup\{\int_{\pi a}v:v\in L^1_{\bar\nu},\,0\le v\le v_0\}
=\int_{\pi a}v_0;\cr}$$

\noindent because $P$ is linear, $\int_aPv=\int_{\pi a}v$ for every
$v\in M^{1,0}_{\bar\nu}$, $a\in\frak A^f$.

\medskip

\qquad\grheadd\ By 366Gd, $P$ is uniquely defined by the formula

\Centerline{$\int_aPv=\int_{\pi a}v$ whenever
$v\in M^{1,0}_{\bar\nu}$ and $a\in\frak A^f$.}

\medskip

\quad{\bf (ii)} Because $M^0_{\bar\mu}$ is closed under multiplication,
$u\times Pv$ certainly belongs to $M^0_{\bar\mu}$.

\medskip

\qquad\grheada\ Suppose that $u$, $v\ge 0$.   Fix $c\in\frak A^f$ for
the moment.
Suppose that $u'\in S(\frak A^f)$.    Then
we can express $u'$ as $\sum_{i=0}^n\alpha_i\chi a_i$ where $a_i\in\frak
A^f$ for every $i\le n$.   Accordingly

\Centerline{$\int_c u'\times Pv
=\sum_{i=0}^n\alpha_i\int_{c\Bcap a_i}Pv
=\sum_{i=0}^n\alpha_i\int v\times\chi(\pi a_i)\times\chi(\pi c)
=\int_{\pi c} v\times Tu'$.}

\noindent Next, we
can find a non-decreasing sequence $\sequencen{u_n}$ in $S(\frak A^f)^+$
with supremum $u$, and

$$\eqalign{\sup_{n\in\Bbb N}\int_c u_n\times Pv
&=\sup_{n\in\Bbb N}\int_{\pi c} v\times Tu_n
=\int_{\pi c}\sup_{n\in\Bbb N}v\times Tu_n\cr
&=\int_{\pi c} v\times\sup_{n\in\Bbb N}Tu_n
=\int_{\pi c} v\times Tu,\cr}$$

\noindent using the order-continuity of $T$, $\int$ and $\times$.
But this means that $u\times Pv=\sup_{n\in\Bbb N}u_n\times Pv$ is
integrable over $c$ and that $\int_c u\times Pv=\int_{\pi c} v\times
Tu$.   As $c$ is arbitrary, $u\times Pv=P(v\times Tu)\in
M^{1,0}_{\bar\mu}$.

\medskip

\qquad\grheadb\ For general $u$, $v$,

\Centerline{$v^+\times Tu^+\,+\,v^+\times Tu^-\,+\,v^-\times
Tu^+\,+\,v^-\times Tu^-=|v|\times T|u|=|v\times Tu|\in
M^{1,0}_{\bar\nu}$}

\noindent (because $T$ is a Riesz
homomorphism), so we may apply ($\alpha$) to each of the four products;
combining them, we get $P(v\times Tu)=u\times Pv$, as required.

\medskip

\quad{\bf (iii)} Because $P$ is a positive operator, we surely have
$|Pv|\le P|v|$, so it will be enough to show that $\|Pv\|_q\le\|v\|_q$
for $v\ge 0$ in $L^q_{\bar\nu}$.

\medskip

\qquad \grheada\ I take the case $q=1$ first.   In this case, for any
$a\in\frak A^f$, we have $\int_aPv=\int_{\pi a}v\le\|v\|_1$.
In particular, setting $a_n=\Bvalue{Pv>2^{-n}}$,
$\int_{a_n}Pv\le\|v\|_1$.   But $Pv=\sup_{n\in\Bbb N}Pv\times\chi a_n$,
so

\Centerline{$\|Pv\|_1=\sup_{n\in\Bbb N}\int_{a_n}Pv\le\|v\|_1$.}

\medskip

\qquad\grheadb\ Next, suppose that $q=\infty$, so that
$v\in L^{\infty}(\frak B)^+$;  say $\|v\|_{\infty}=\gamma$.   \Quer\ If
$\gamma>0$ and $a=\Bvalue{Pv>\gamma}\ne 0$, then

\Centerline{$\gamma\bar\mu a<\int_aPv
=\int_{\pi a}v\le\gamma\bar\nu(\pi a)=\gamma\bar\mu a$.   \Bang}

\noindent So $\Bvalue{Pv>\gamma}=0$ and $Pv\in L^{\infty}(\frak A)$,
with $\|Pv\|_{\infty}\le\|v\|_{\infty}$, at least when
$\|v\|_{\infty}>0$;
but the case $\|v\|_{\infty}=0$ is trivial.

\medskip

\qquad\grheadc\ I come at last to the `general' case
$q\in\ooint{1,\infty}$, $v\in L^q_{\bar\nu}$.   In this case set
$p=q/(q-1)$.   If $u\in L^p_{\bar\mu}$ then $Tu\in L^p_{\bar\nu}$ so
$Tu\times v\in L^1_{\bar\nu}$ and

$$\eqalignno{|\int u\times Pv|
&\le\|u\times Pv\|_1
=\|P(Tu\times v)\|_1\cr
\noalign{\noindent (by (ii))}
&\le\|Tu\times v\|_1\cr
\noalign{\noindent (by ($\alpha$) just above)}
&=\int|Tu|\times|v|
\le\|Tu\|_p\|v\|_q
=\|u\|_p\|v\|_q\cr
}$$

\noindent by (a-iii) of this theorem.   But this means that $u\mapsto
\int u\times Pv$ is a bounded linear functional on $L^p_{\bar\mu}$, and
is therefore represented by some $w\in L^q_{\bar\mu}$ with
$\|w\|_q\le\|v\|_q$.   If $a\in\frak A^f$ then
$\chi a\in L^p_{\bar\mu}$, so
$\int_aw=\int_aPv$;  accordingly $Pv$ is actually equal to $w$ (by
366Gd) and
$\|Pv\|_q=\|w\|_q\le\|v\|_q$, as claimed.

\medskip

\quad{\bf (iv)} If $u\in M^{1,0}_{\bar\mu}$ and $a\in\frak A^f$, we must
have

\Centerline{$\int_aPTu=\int_{\pi a}Tu=\int T(\chi a)\times Tu=\int
T(\chi a\times u)=\int \chi a\times u=\int_au$,}

\noindent using (a-iv) to see that $\int\chi a\times u$ is defined and
equal to $\int T(\chi a\times u)$.   As $a$ is arbitrary, $u\in
M^{1,0}_{\bar\mu}$ and $PTu=u$.

\medskip

{\bf (c)} As usual, in view of the uniqueness of $T_{\theta\pi}$ and
$P_{\theta\pi}$, all we have to check is that

\Centerline{$T_{\theta}T(\chi a)=T_{\theta}\chi(\pi a)
=\chi(\theta\pi a)=T_{\theta\pi}(\chi a)$,}

\Centerline{$\int_aPP_{\theta}w=\int_{\pi a}P_{\theta}w
=\int_{\theta\pi a}w=\int_aP_{\theta\pi}w$}

\noindent whenever $a\in\frak A^f$ and $w\in M^{1,0}_{\bar\lambda}$.

\medskip

{\bf (d)(i)} By (c), $T_{\pi^{-1}}T=T_{\pi^{-1}\pi}$ must be the
identity operator on $M^0_{\bar\mu}$;  similarly, $TT_{\pi^{-1}}$ is the
identity operator on $M^0_{\bar\nu}$.   Because $T$ and $T_{\pi^{-1}}$
are Riesz homomorphisms, they must be the two halves of a Riesz space
isomorphism.

\medskip

\quad{\bf (ii)} In the same way, $P$ and $P_{\pi^{-1}}$ must be the two
halves of an ordered linear space isomorphism between
$M^{1,0}_{\bar\mu}$ and $M^{1,0}_{\bar\nu}$, and are therefore both
Riesz homomorphisms.

\medskip

\quad{\bf (iii)} By (b-iv), $PTu=u$ for every $u\in M^{1,0}_{\bar\mu}$,
so $T\restr M^{1,0}_{\bar\mu}$ must be $P^{-1}$.
Similarly $P=P_{\pi^{-1}}^{-1}$ is the restriction of
$T^{-1}=T_{\pi^{-1}}$ to $M^{1,0}_{\bar\nu}$.

\medskip

\quad{\bf (iv)} Because $T^{-1}[L^p_{\bar\nu}]=L^p_{\bar\mu}$
(by (a-iv)), and $T$ is a bijection between $M^0_{\bar\mu}$ and
$M^0_{\bar\nu}$, $T\restr L^p_{\bar\mu}$ must be a Riesz space
isomorphism between $L^p_{\bar\mu}$ and $L^p_{\bar\nu}$;  (a-iv) also
tells us that it is norm-preserving.   Now its inverse is
$P\restr L^p_{\bar\nu}$, by (iii) here.
}%end of proof of 366H

\leader{366I}{Corollary} Let $(\frak A,\bar\mu)$ be a measure algebra,
and $\frak B$ a $\sigma$-subalgebra of $\frak A$.
Then, for any $p\in\coint{1,\infty}$,
$L^p(\frak B,\bar\mu\restrp\frak B)$
can be identified, as Banach lattice, with the closed linear subspace of
$L^p(\frak A,\bar\mu)$ generated by
$\{\chi b:b\in\frak B,\,\bar\mu b<\infty\}$.

\proof{ The identity map $b\mapsto b:\frak B\to\frak A$ induces an
injective Riesz homomorphism $T:L^0(\frak B)\to L^0(\frak A)$ (364P)
such that $Tu\in L^p_{\frak A}=L^p(\frak A,\bar\mu)$ and
$\|Tu\|_p=\|u\|_p$ whenever $p\in\coint{1,\infty}$ and
$u\in L^p_{\frak B}=L^p(\frak B,\bar\mu\restrp\frak B)$ (366H(a-iv)).
Because $S(\frak B^f)$, the linear span of
$\{\chi b:b\in\frak B,\,\bar\mu b<\infty\}$,
is dense in $L^p_{\frak B}$ (366C), the image of $L^p_{\frak B}$ in
$L^p_{\frak A}$ must be the closure of the image of
$S(\frak B^f)$ in $L^p_{\frak A}$, that is, the closed linear span of
$\{\chi b:b\in\frak B^f\}$ interpreted as a subset of $L^p_{\frak A}$.
}%end of proof of 366I

\leader{366J}{Corollary} If $(\frak A,\bar\mu)$ is a probability
algebra, $\frak B$ is a closed subalgebra of $\frak A$, and
$P:L^1(\frak A,\bar\mu)\to L^1(\frak B,\bar\mu\restrp\frak B)$ is the
conditional
expectation operator\cmmnt{ (365R)}, then $\|Pu\|_p\le\|u\|_p$
whenever $p\in[1,\infty]$ and $u\in L^p(\frak A,\bar\mu)$.

\proof{ Because $(\frak A,\bar\mu)$ is totally finite,
$M^{1,0}(\frak A,\bar\mu)=L^1_{\bar\mu}$, so that the operator $P$ of
366Hb can be identified with the conditional expectation operator of
365R.   Now 366H(b-iii) gives the result.
}%end of proof of 366J

\cmmnt{\medskip

\noindent{\bf Remark} Of course this is also covered by 244M.
}

\vleader{72pt}{366K}{Corollary} Let $(\frak A,\bar\mu)$ and
$(\frak B,\bar\nu)$ be measure algebras, and $\pi:\frak A^f\to\frak B^f$ a
measure-preserving ring homomorphism.   Let
$T:L^2(\frak A,\bar\mu)\to L^2(\frak B,\bar\nu)$ and
$P:L^2(\frak B,\bar\nu)\to L^2(\frak A,\bar\mu)$ be the
corresponding operators\cmmnt{, as in 366H}.   Then
$TP:L^2(\frak B,\bar\nu)\to L^2(\frak B,\bar\nu)$ is an orthogonal
projection, its range $TP[L^2(\frak B,\bar\nu)]$
being isomorphic, as Banach lattice, to $L^2(\frak A,\bar\mu)$.   The
kernel of $TP$ is just

\Centerline{$\{v:v\in L^2(\frak B,\bar\nu),\,\int_{\pi a}v=0$ for
every $a\in\frak A^f\}$.}

\proof{ Most of this is simply because $T$ is a
norm-preserving Riesz homomorphism (so that $T[L^2_{\bar\mu}]$ is
isomorphic to $L^2_{\bar\mu}$), $PT$ is the identity on $L^2_{\bar\mu}$
(so that $(TP)^2=TP$) and $\|P\|\le 1$ (so that $\|TP\|\le 1$).   These
are enough to ensure that $TP$ is a projection of norm at most $1$, that
is, an orthogonal projection.   Also

$$\eqalign{TPv=0
&\iff Pv=0\iff\int_aPv=0\text{ for every }a\in\frak A^f\cr
&\iff \int_{\pi a}v=0\text{ for every }a\in\frak A^f.\cr}$$
}%end of proof of 366K

\leader{366L}{Corollary} Let $(\frak A,\bar\mu)$ be a measure algebra,
and $\pi:\frak A^f\to\frak A^f$ a measure-preserving ring automorphism.
Then there is a corresponding Banach lattice isomorphism $T$ of
$L^2=L^2(\frak A,\bar\mu)$ defined by writing $T(\chi a)=\chi(\pi a)$
for every $a\in\frak A^f$.   Its inverse is defined by the formula

\Centerline{$\int_aT^{-1}u=\int_{\pi a}u$ for every $u\in L^2$,
$a\in\frak A^f$.}

\proof{ In the language of 366H, $T=T_{\pi}$ and $T^{-1}=P_{\pi}$.}

%could add corollaries corresponding to 242L, 244Nb

\leader{*366M}{Complex $L^p$ spaces}\dvAnew{2011}
{\bf (a)} Just as in \S\S241-244,
%241J, 242P, 243K and 244P\formerly{2{}44O},
we have `complex' versions of all the spaces
considered in this chapter.\cmmnt{   Using the representation theorems for
Boolean algebras, we can get effective descriptions of
these matching the ones in Chapter 24.}   Thus for any Boolean algebra
$\frak A$ with Stone space $Z$, we can identify
$L^{\infty}_{\Bbb C}(\frak A)$ with the space $C(Z;\Bbb C)$ of continuous
functions from $Z$ to $\Bbb C$;  inside this, we have a
$\|\,\|_{\infty}$-dense subspace $S_{\frak C}(\frak A)$ consisting of
complex
linear combinations of indicator functions of open-and-closed sets.
If $\frak A$ is a Dedekind
$\sigma$-complete Boolean algebra, identified with a quotient
$\Sigma/\Cal M$ where $\Sigma$ is a $\sigma$-algebra of subsets of a set $Z$
and $\Cal M$ is a $\sigma$-ideal of $\Sigma$, then we can write
$\eusm L^0_{\Bbb C}$ for the set of functions from $Z$ to $\Bbb C$ such
that their real and imaginary parts are both $\Sigma$-measurable,
$\Cal W_{\Bbb C}$ for the set of those $f\in\eusm L^0_{\Bbb C}$ such that
$\{z:f(z)\ne 0\}$ belongs to $\Cal M$, and
$L^0_{\Bbb C}=L^0_{\Bbb C}(\frak A)$ for
the linear space quotient $\eusm L^0_{\Bbb C}/\Cal W_{\Bbb C}$.
As in 241J, we\cmmnt{ find that we} have
a natural embedding of $L^0=L^0(\frak A)$ in
$L^0_{\Bbb C}$ and functions

\Centerline{$\Real:L^0_{\Bbb C}\to L^0$,
\quad$\Imag:L^0_{\Bbb C}\to L^0$,
\quad$|\mskip10mu|:L^0_{\Bbb C}\to L^0$,
\quad$\bar{\phantom{u}}:L^0_{\Bbb C}\to L^0_{\Bbb C}$}

\noindent such that

\Centerline{$u=\Real(u)+i\Imag(u)$,
\quad$\Real(u+v)=\Real(u)+\Real(v)$,
\quad$\Imag(u+v)=\Imag(u)+\Imag(v)$,}

\Centerline{$\Real(\alpha u)=\Real(\alpha)\Real(u)-\Imag(\alpha)\Imag(u)$,
\quad$\Imag(\alpha u)=\Real(\alpha)\Imag(u)+\Imag(\alpha)\Real(u)$,}

\Centerline{$|\alpha u|=|\alpha||u|$,
\quad$|u+v|\le|u|+|v|$,
\quad$|u|=\sup_{|\gamma|=1}\Real(\gamma u)$,}

\Centerline{$\bar u=\Real(u)-i\Imag(u)$,
\quad$\overline{u+v}=\bar u+\bar v$,
\quad$\overline{\alpha u}=\bar\alpha\bar u$}

\noindent for all $u$, $v\in L^0_{\Bbb C}$ and $\alpha\in\Bbb C$.

I seem to have omitted to mention it in 241J, but of course we also have a
multiplication

\Centerline{$u\times v=(\Real(u)\times\Real(v)-\Imag(u)\times\Imag(v))
  +i(\Real(u)\times\Imag(v)+\Imag(u)\times\Real(v))$,}

\noindent for which we have\cmmnt{ the expected formulae}

\Centerline{$u\times v=v\times u$,
\quad$u\times(v\times w)=(u\times v)\times w$,
\quad$u\times(v+w)=(u\times v)+(u\times w)$,}

\Centerline{$(\alpha u)\times v=u\times(\alpha v)=\alpha(u\times v)$,}

\Centerline{$\overline{u\times v}=\bar u\times\bar v$,
\quad$|u\times v|=|u|\times|v|$,
\quad$u\times\bar u=|u|^2=(\Real(u))^2+(\Imag(u))^2$}

\noindent for $u$, $v\in L^0_{\Bbb C}$ and $\alpha\in\Bbb C$.

\spheader 366Mb
If $(\frak A,\bar\mu)$ is a measure algebra and $1\le p<\infty$, we can
think of $L^p_{\Bbb C}(\frak A,\bar\mu)$ as the set of those
$u\in L^0_{\Bbb C}$ such that $|u|\in L^p(\frak A,\bar\mu)$, with
its norm defined by the formula $\|u\|_p=\||u|\|_p$;  this will make
$L^p_{\Bbb C}(\frak A,\bar\mu)$ a Banach space\cmmnt{ (cf.\ 242Pb,
244Pb\formerly{2{}44O})}, with dual $L^q(\frak A,\bar\mu)$ where
$\bover1p+\bover1q=1$ if $p>1$\cmmnt{ (244Pb again)}.   (Similarly, if
$(\frak A,\bar\mu)$ is localizable, the dual of
$L^1_{\Bbb C}(\frak A,\bar\mu)$ can be identified with
$L^{\infty}_{\Bbb C}$\cmmnt{, as in 365Mc}.)

Writing $S_{\Bbb C}(\frak A^f)$ for the space of linear combinations of
indicator functions of elements of $\frak A$ of finite measure,
$S_{\Bbb C}(\frak A^f)$ is dense in $L^p_{\Bbb C}(\frak A,\bar\mu)$
whenever $1\le p<\infty$\cmmnt{, as in 366C}.

\spheader 366Mc
\cmmnt{Of course }$L^1$- and $L^2$-spaces
have\cmmnt{ special additional features,
their} integrals and inner products.   Here we\cmmnt{ can} set

\Centerline{$\int u=\int\Real(u)+i\int\Imag(u)$}

\noindent for $u\in L^1_{\Bbb C}(\frak A,\bar\mu)$, and
$\int:L^1_{\Bbb C}(\frak A,\bar\mu)\to\Bbb C$ becomes a
$\Bbb C$-linear functional.
As for $L^2$,\cmmnt{ we see at once from the formulae above that}

\Centerline{$|u\times v|=|u|\times|v|\in L^1(\frak A,\bar\mu)$,
\quad$u\times v\in L^1_{\Bbb C}(\frak A,\bar\mu)$,
\quad$\int u\times\bar u=\|u\|_2^2$}

\noindent for $u$, $v\in L^2_{\Bbb C}(\frak A,\bar\mu)$.   So if we set

\Centerline{$\innerprod{u}{v}=\int u\times\bar v$}

\noindent for $u$, $v\in L^2_{\Bbb C}(\frak A,\bar\mu)$,
$L^2_{\Bbb C}(\frak A,\bar\mu)$ becomes a complex Hilbert space.

\spheader 366Md\cmmnt{ In
the language of the present chapter we have something else to look at.}
If $\frak A$, $\frak B$ are Dedekind $\sigma$-complete Boolean algebras
and $\pi:\frak A\to\frak B$ is
a sequentially order-continuous Boolean homomorphism, then we have a
linear operator $T_{\pi}:L^0_{\Bbb C}(\frak A)\to L^0_{\Bbb C}(\frak B)$
defined by setting
$T_{\pi}u
=T_{\pi}^{\text{real}}(\Real(u))+iT^{\text{real}}_{\pi}(\Imag(u))$,
where $T^{\text{real}}_{\pi}:L^0(\frak A)\to L^0(\frak B)$
is the Riesz homomorphism
described in 364P.   \cmmnt{Of course }$T_{\pi}$,
like $T^{\text{real}}_{\pi}$, will be
multiplicative;  \cmmnt{hence, or otherwise, }$T_{\pi}|u|=|T_{\pi}u|$
for every
$u\in L^0_{\Bbb C}(\frak A)$.   \cmmnt{Observe that}
$T_{\pi}\bar u=\overline{T_{\pi}u}$ for every $u\in L^0_{\Bbb C}(\frak A)$.
Also,\cmmnt{ as in 364Pe,} if $\frak C$ is another
Dedekind $\sigma$-complete
Boolean algebra and $\pi:\frak A\to\frak B$ and $\phi:\frak B\to\frak C$
are sequentially order-continuous Boolean homomorphisms,
$T_{\phi\pi}=T_{\phi}T_{\pi}$.   So if $\pi:\frak A\to\frak A$ is a
Boolean automorphism, $T_{\pi}$ will be a bijection with inverse
$T_{\pi^{-1}}$.

\spheader 366Me
Similarly, if $(\frak A,\bar\mu)$ is a measure algebra and
$\pi:\frak A\to\frak A$ is a measure-preserving Boolean homomorphism,
$\int T_{\pi}u=\int u$ for every $u\in L^1_{\Bbb C}(\frak A,\bar\mu)$.
If $u$, $v\in L^2_{\Bbb C}(\frak A,\bar\mu)$, then

\Centerline{$\innerprod{T_{\pi}u}{T_{\pi}v}
=\int T_{\pi}u\times\overline{T_{\pi}v}
=\int T_{\pi}u\times T_{\pi}\bar v
=\int T_{\pi}(u\times\bar v)
=\int u\times\bar v
=\innerprod{u}{v}$.}

\noindent If $\pi$ is\cmmnt{ actually} a measure-preserving Boolean
automorphism, we shall have

\Centerline{$\innerprod{T_{\pi}u}{v}
=\innerprod{T_{\pi^{-1}}T_{\pi}u}{T_{\pi^{-1}}v}
=\innerprod{u}{T_{\pi}^{-1}v}$}

\noindent for all $u$, $v\in L^2_{\Bbb C}(\frak A,\bar\mu)$.

\exercises{\leader{366X}{Basic exercises (a)}
%\spheader 366Xa
Let $(\frak A,\bar\mu)$ be a measure algebra and $p\in\ooint{1,\infty}$.
Show that $\|u\|_p^p
=p\int_0^{\infty}\alpha^{p-1}\bar\mu\Bvalue{|u|>\alpha}\,d\alpha$
for every $u\in L^0(\frak A)$.  (Cf.\ 263Xa.)
%366A

\sqheader 366Xb Let $(\frak A,\bar\mu)$ be a localizable measure algebra
and $p\in[1,\infty]$.   Show that the band algebra of
$L^p_{\bar\mu}$ is isomorphic to $\frak A$.   (Cf.\ 365S.)

\spheader 366Xc Let $(\frak A,\bar\mu)$ be a measure algebra and
$p\in\ooint{1,\infty}$.   Show that $L^p_{\bar\mu}$ is separable
iff $L^1_{\bar\mu}$ is.
%366C

\spheader 366Xd
Let $(\frak A,\bar\mu)$ be a measure algebra.   (i) Show that
$L^{\infty}(\frak A)\cap M^0_{\bar\mu}$ and
$L^{\infty}(\frak A)\cap M^{1,0}_{\bar\mu}$, as defined in 366F,
are equal.   (ii)
Call this intersection $M^{\infty,0}_{\bar\mu}$.   Show that it
is a norm-closed solid linear subspace of $L^{\infty}(\frak A)$,
therefore a Banach lattice in its own right.
%366G

\spheader 366Xe Let $(\frak A,\bar\mu)$ be a semi-finite measure algebra
and $(\widehat{\frak A},\hat\mu)$ its localization (322Q).   Show that
the natural embedding of $\frak A$ in $\widehat{\frak A}$ induces a
Banach lattice isomorphism between $L^p_{\bar\mu}$ and
$L^p_{\hat\mu}$ for every $p\in\coint{1,\infty}$, so
that the band algebra of $L^p_{\bar\mu}$ can be identified
with $\widehat{\frak A}$.
%366H

\spheader 366Xf Let $(\frak A,\bar\mu)$ be a semi-finite measure algebra
which is not localizable (cf.\ 211Ye, 216D), and
$(\widehat{\frak A},\hat\mu)$ its localization.   Let
$\pi:\frak A\to\widehat{\frak A}$ be the identity embedding, so that
$\pi$ is an order-continuous
measure-preserving Boolean homomorphism.   Show that if we set
$v=\chi b$ where $b\in\widehat{\frak A}\setminus\frak A$, then there is
no $u\in L^{\infty}(\frak A)$ such that $\int_au=\int_{\pi a}v$ whenever
$\bar\mu a<\infty$.
%366H

\spheader 366Xg In 366H, show that $\Bvalue{Tu\in E}=\pi\Bvalue{u\in E}$
(notation:  364G)
whenever $u\in M^0_{\bar\mu}$ and $E\subseteq\Bbb R$ is a Borel set such
that $0\notin\overline{E}$.
%366H

\sqheader 366Xh Let $(\frak A,\bar\mu)$ be a measure algebra and let $G$
be the group of all measure-preserving ring automorphisms of
$\frak A^f$.   Let $H$ be the group of all Banach lattice automorphisms
of $L^2_{\bar\mu}$.   Show that the map $\pi\mapsto T$ of 366L is
an injective group homomorphism from $G$ to $H$, so that $G$ is
represented as a subgroup of $H$.
%366L

\spheader 366Xi Let $\langle(\frak A_i,\bar\mu_i)\rangle_{i\in I}$ be
any family of measure algebras, with simple product $(\frak A,\bar\mu)$
(322L).   Show that for any $p\in\coint{1,\infty}$,
$L^p_{\bar\mu}$ can be
identified, as normed Riesz space, with the solid linear subspace

\Centerline{$\{u:\|u\|
=\bigl(\sum_{i\in I}\|u(i)\|_p^p\bigr)^{1/p}<\infty\}$}

\noindent of $\prod_{i\in I}L^p_{\bar\mu_i}$.

\spheader 366Xj Let $\frak A$ be a Dedekind $\sigma$-complete Boolean
algebra and $\bar\mu$, $\bar\nu$ two functionals rendering $\frak A$ a
semi-finite measure algebra.   Show that for any $p\in\coint{1,\infty}$,
$L^p_{\bar\mu}$ and $L^p_{\bar\nu}$ are isomorphic as
normed Riesz spaces.   \Hint{use 366Xe to reduce to the case in which
$\frak A$ is Dedekind complete.   Take $w\in L^0(\frak A)$ such that
$\int_aw\,d\bar\mu=\bar\nu a$ for every $a\in\frak A$ (365T).   Set
$Tu=w^{1/p}\times u$ for $u\in L^p_{\bar\mu}$.}

\spheader 366Xk Let $(\frak A,\bar\mu)$ and $(\frak B,\bar\nu)$ be
semi-finite measure
algebras, and $p\in\coint{1,\infty}$.   Show that the following are
equiveridical:  (i) $L^p_{\bar\mu}$ and $L^p_{\bar\nu}$ are isomorphic
as
Banach lattices;  (ii) $L^p_{\bar\mu}$ and $L^p_{\bar\nu}$ are
isomorphic as
Riesz spaces;  (iii) $\frak A$ and $\frak B$ have isomorphic Dedekind
completions.
%366Xj

\spheader 366Xl\dvAformerly{3{}63Xj}
For a Boolean algebra $\frak A$, state and prove
results corresponding to 363C, 363Ea and 363F-363I %363F 363G 363H 363I
for $L^{\infty}_{\Bbb C}(\frak A)$ as defined in 366Ma.
%366M

\leader{366Y}{Further exercises (a)}
Let $(\frak A,\bar\mu)$ be a measure algebra and suppose that $0<p<1$.
Write $L^p=L^p_{\bar\mu}=L^p(\frak A,\bar\mu)$ for
$\{u:u\in L^0(\frak A),\,|u|^p\in L^1_{\bar\mu}\}$, and for
$u\in L^p$ set $\tau(u)=\int|u|^p$.
(i) Show that $\tau$ defines a Hausdorff linear space topology on $L^p$
(see 2A5B).
(ii) Show that if $A\subseteq L^p$ is non-empty, downwards-directed and
has infimum $0$ then $\inf_{u\in A}\tau(u)=0$.   (iii) Show that if
$A\subseteq L^p$ is non-empty, upwards-directed and bounded in the
linear topological space sense then $A$ is bounded above.
(iv) Show that $(L^p)^{\sim}=(L^p)^{\times}$ is just the set of
continuous linear functionals from $L^p$ to $\Bbb R$, and is $\{0\}$ iff
$\frak A$ has no atom of finite measure.
%366A

\spheader 366Yb Let $(\frak A,\bar\mu)$ be a measure algebra.   Show
that $M^0(\frak A,\bar\mu)$ has the countable sup property.
%366F

\spheader 366Yc Let $(\frak A,\bar\mu)$ be a measure algebra and define
$M^{\infty,0}_{\bar\mu}$ as in 366Xd.   Show that
$(M^{\infty,0}_{\bar\mu})^{\times}$ can be identified with
$L^1_{\bar\mu}$.
%366G

\spheader 366Yd In 366H, show that if $\tilde T:M^0(\frak A,\bar\mu)\to
M^0(\frak B,\bar\nu)$ is any positive linear operator such that
$\tilde T(\chi a)=\chi(\pi a)$ for every $a\in\frak A^f$, then
$\tilde T$ is order-continuous, so is equal to $T_{\pi}$.
%366H

\spheader 366Ye Let $(\frak A,\bar\mu)$ be a measure algebra.   (i)
Show that there is a natural one-to-one correspondence between
$M^{1,0}(\frak A,\bar\mu)$ and the set of additive functionals
$\nu:\frak A^f\to\Bbb R$ such that $\nu<\backstep{4}<\mu$
in the double sense that
for every $\epsilon>0$ there are $\delta$, $M>0$ such that $|\nu
a|\le\epsilon$ whenever $\mu a\le\delta$ and $|\nu a|\le\epsilon\mu a$
whenever $\mu a\ge M$.   (ii) Use this description of $M^{1,0}$ to prove
366H(b-i).
%366H

\spheader 366Yf In 366H, show that the following are equiveridical:
($\alpha$) $\pi[\frak A^f]=\frak B^f$;  ($\beta$) $T=T_{\pi}$ is
surjective;  ($\gamma$) $P=P_{\pi}$ is injective;  ($\delta$) $P$ is
a Riesz homomorphism;  ($\epsilon$) there is some $q\in[1,\infty]$ such
that $\|Pv\|_q=\|v\|_q$ for every $v\in L^q_{\bar\nu}$;  ($\zeta$)
$TPv=v$ for every $v\in M^{1,0}_{\bar\nu}$.
%366H

\spheader 366Yg Let $(\frak A,\bar\mu)$ and $(\frak B,\bar\nu)$ be
measure algebras, and suppose that $\pi:\frak A^f\to\frak B^f$ is a
measure-preserving ring homomorphism, as in 366H;  let
$T:M^0_{\bar\mu}\to M^0_{\bar\nu}$ be the associated
linear operator.   Show that if $0<p<1$ (as in 366Ya) then
$L^p_{\bar\mu}\subseteq M^0_{\bar\mu}$ and
$T^{-1}[L^p_{\bar\nu}]=L^p_{\bar\mu}$.
%366H, 366Ya

\spheader 366Yh Let $(\frak A,\bar\mu)$ be a totally finite measure
algebra.   (i) For each Boolean automorphism $\pi:\frak A\to\frak A$,
let $T_{\pi}:L^0(\frak A)\to L^0(\frak A)$ be the associated Riesz space
isomorphism, and let $w_{\pi}\in(L^1_{\bar\mu})^+$ be such that
$\int_aw_{\pi}=\mu(\pi^{-1}a)$ for every $a\in\frak A$ (365Ea).   Set
$Q_{\pi}u=T_{\pi}u\times\sqrt{w_{\pi}}$ for $u\in L^0(\frak A)$.   Show
that $\|Q_{\pi}u\|_2=\|u\|_2$ for every $u\in L^2_{\bar\mu}$.   (ii)
Show that if $\pi$, $\phi:\frak A\to\frak A$ are Boolean automorphisms
then $Q_{\pi\phi}=Q_{\pi}Q_{\phi}$.

\spheader 366Yi Let $(\frak A,\bar\mu)$ be a measure algebra, and
$\pi:\frak A^f\to\frak A^f$ a measure-preserving Boolean homomorphism,
with associated linear operator
$T_{\pi}:M^0_{\bar\mu}\to M^0_{\bar\mu}$.   Show that the following are
equiveridical:  (i) there is some $p\in\coint{1,\infty}$ such that
$\{T_{\pi}^n\restr L^p_{\bar\mu}:n\in\Bbb N\}$ is relatively compact in
$\eurm B(L^p_{\bar\mu};L^p_{\bar\mu})$ for the strong operator topology;
(ii) for every $p\in\coint{1,\infty}$,
$\{T_{\pi}^n\restr L^p_{\bar\mu}:n\in\Bbb N\}$ is relatively compact in
$\eurm B(L^p_{\bar\mu};L^p_{\bar\mu})$ for the strong operator topology;
(iii) $\{\pi^na:n\in\Bbb N\}$ is relatively compact in $\frak A^f$, for
the strong measure-algebra topology, for every $a\in\frak A^f$.

\spheader 366Yj\dvAformerly{3{}64Yn}
Let $\frak A$ be a Dedekind $\sigma$-complete Boolean
algebra.   Show that $L^0_{\Bbb C}(\frak A)$ can be identified with the
complexification of $L^0(\frak A)$ as defined in 354Yl.
%366M

\spheader 366Yk\dvAformerly{3{}64Yn}
Write $\Cal B(\Bbb C)$ for the Borel $\sigma$-algebra of
$\Bbb C\cong\BbbR^2$ as defined in 111Gd.   Show that if $\frak A$ is a
Dedekind $\sigma$-complete Boolean algebra, we have a unique function
$(u,E)\mapsto\Bvalue{u\in E}:
L^0_{\Bbb C}(\frak A)\times\Cal B(\Bbb C)\to\frak A$ such that
(i) for any $u\in L^0_{\Bbb C}(\frak A)$, the function
$E\mapsto\Bvalue{u\in E}$ is a sequentially order-continuous Boolean
homomorphism from $\Cal B(\Bbb C)$ to $\frak A$ (ii) if $E_0$,
$E_1\subseteq\Bbb R$ are Borel sets, then
$\Bvalue{u\in E_0\times E_1}
=\Bvalue{\Real(u)\in E_0}\Bcap\Bvalue{\Imag(u)\in E_1}$ for every
$u\in L^0_{\Bbb C}(\frak A)$ (iii) if $\phi:\Cal B(\Bbb C)\to\frak A$ is a
sequentially order-continuous Boolean homomorphism, then there is a unique
$u\in L^0_{\Bbb C}(\frak A)$ such that $\phi(E)=\Bvalue{u\in E}$ for every
$E\in\Cal B(\Bbb C)$.
%366M 364G

\spheader 366Yl\dvAformerly{3{}64Yn}
A function $h:\Bbb C\to\Bbb C$ is called {\bf Borel
measurable} if its real and imaginary parts are
$\Cal B(\Bbb C)$-measurable, where
$\Cal B(\Bbb C)$ is the Borel $\sigma$-algebra of $\Bbb C$.
Let $\frak A$ be a Dedekind $\sigma$-complete Boolean algebra.
(i) Show that for every Borel measurable $h:\Bbb C\to\Bbb C$ and
$u\in L^0_{\Bbb C}(\frak A)$ we have an element
$\bar h(u)\in L^0_{\Bbb C}(\frak A)$ such that
$\Bvalue{\bar h(u)\in E}=\Bvalue{u\in h^{-1}[E]}$ for every
$E\in\Cal B(\Bbb C)$.   (ii) Show that if $\pi:\frak A\to\frak A$ is a
sequentially order-continuous Boolean homomorphism and
$T:L^0_{\Bbb C}(\frak A)\to L^0_{\Bbb C}(\frak A)$ the corresponding linear
operator (366Mc), then $T\bar h=\bar hT$ for every Borel measurable
$h:\Bbb C\to\Bbb C$.
%366M 366Yk 364G 364Pd

\spheader 366Ym\dvAformerly{3{}63Yi}
Show that a normed space over $\Bbb C$ has the
Hahn-Banach property of 363R for complex spaces iff it is isomorphic to
$L^{\infty}_{\Bbb C}(\frak A)$ for some Dedekind
complete Boolean algebra $\frak A$.
%363R
}%end of exercises


\cmmnt{
\Notesheader{366}
The $L^p$ spaces, for $1\le p\le\infty$, constitute the most important
family of leading examples for the theory of Banach lattices, and it is
not to be wondered at that their properties reflect a wide variety of
general results.   Thus 366Dd and 366E can both be regarded as special
cases of theorems about perfect Riesz spaces (356M and 369D).   In
a different direction, the concept of `Orlicz space' (369Xd below)
generalizes the $L^p$ spaces if they are regarded as normed subspaces of
$L^0$ invariant
under measure-preserving automorphisms of the underlying algebra.   Yet
another generalization looks at the (non-locally-convex) spaces $L^p$
for $0<p<1$ (366Ya).

In 366H and its associated results I try to emphasize the way in which
measure-preserving homomorphisms of the underlying algebras induce both
`direct' and `dual' operators on $L^p$ spaces.   We have already
seen the phenomenon in 365P.   I express this in a slightly different
form in 366H, noting that we really do need the
homomorphisms to be measure-preserving, for the dual operators as well
as the direct operators, so we no longer have the shift in the
hypotheses which
appears between 365O and 365P.    Of course all these refinements in the
hypotheses are irrelevant to the principal applications of the results,
and they make substantial demands on the reader;  but I believe that the
demands are actually demands to expand one's imagination, to encompass
the different
ways in which the spaces depend on the underlying measure algebras.

In the context of 366H, $L^{\infty}$ is set apart from the other $L^p$
spaces, because $L^{\infty}(\frak A)$ is not in general determined by
the ideal $\frak A^f$, and the hypotheses of 366H do not look outside
$\frak A^f$.   366H(a-iv) and 366H(b-iii) reach only the space
$M^{\infty,0}$ as defined in 366Xd.    To deal with $L^{\infty}$ we need
slightly stronger hypotheses.   If we are given a measure-preserving
Boolean homomorphism from $\frak A$ to $\frak B$, rather than from
$\frak A^f$ to
$\frak B^f$, then of course the direct operator $T$ has a natural
version acting on $L^{\infty}(\frak A)$ and indeed on
$M^{1,\infty}_{\bar\mu}$, as in 363F and 369Xm.    If we know
that $(\frak A,\bar\mu)$ is
localizable, then $\frak A$ can be recovered from $\frak A^f$, and the
dual operator $P$ acts on $L^{\infty}(\frak B)$, as in 369Xm.   But in
general we can't expect this to work (366Xf).

Of course 366H can be applied to many other spaces;  for reasons which
will appear in \S\S371 and 374, the archetypes are not really $L^p$
spaces at all, but the spaces $M^{1,0}$ (366F) and  $M^{1,\infty}$.

I include 366L and 366Yh as pointers to one of the important
applications of these
ideas:  the investigation of properties of a measure-preserving
homomorphism in terms of its action on $L^p$ spaces.   The case $p=2$ is
the most useful because the group of unitary operators (that is, the
normed space automorphisms) of $L^2$ has been studied intensively.

}%end of comment

\discrpage

