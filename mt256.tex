\frfilename{mt256.tex}
\versiondate{6.8.15}
\copyrightdate{2000}

\def\chaptername{Product measures}
\def\sectionname{Radon measures on $\BbbR^r$}

\newsection{256}

\def\headlinesectionname{Radon measures on {$\eightBbb R^r$}}

In the next section, and again in Chapters 27 and 28,
we need to consider the principal class of measures on Euclidean spaces.
For a proper discussion of this class, and the interrelationships
between the measures and the topologies involved, we must wait until
Volume 4.   For the moment, therefore, I present definitions adapted to
the case in hand, warning you that the correct generalizations are not
quite obvious.   I give the definition (256A) and a characterization
(256C) of Radon measures on Euclidean spaces, and theorems on the
construction of
Radon measures as indefinite integrals (256E, 256J), as image measures
(256G) and as product measures (256K).   In passing I give a version of
Lusin's theorem concerning measurable functions on Radon measure spaces
(256F).

Throughout this section, $r$ and $s$ will be integers greater than or equal
to $1$.

\vleader{48pt}{256A}{Definitions} Let $\nu$ be a measure on $\BbbR^r$ 
and $\Sigma$ its domain.

\spheader 256Aa $\nu$ is a {\bf topological measure} if every open set
belongs to $\Sigma$.   \cmmnt{Note that in this case every Borel set,
and in particular every closed set, belongs to $\Sigma$.}

\spheader 256Ab $\nu$ is {\bf locally finite} if every bounded set has
finite outer measure.

\spheader 256Ac If $\nu$ is a topological measure, it is {\bf inner
regular with respect to the compact sets} if

\Centerline{$\nu E=\sup\{\nu K:K\subseteq E$ is compact$\}$}

\noindent for every $E\in\Sigma$.   \cmmnt{(Because $\nu$ is a
topological measure, and compact sets are closed (2A2Ec), $\nu K$ is
defined for every compact set $K$.)}

\spheader 256Ad $\nu$ is a {\bf Radon measure} if it is a complete
locally finite topological measure which is inner regular with respect
to the compact sets.


\leader{256B}{}\cmmnt{ It will be convenient to be able to call on the
following elementary facts.

\medskip

\noindent}{\bf Lemma} Let $\nu$ be a Radon measure on $\BbbR^r$, and
$\Sigma$ its domain.

(a) $\nu$ is $\sigma$-finite.

(b) For any $E\in\Sigma$ and any
$\epsilon>0$ there are a closed
set $F\subseteq E$ and an open set $G\supseteq E$ such that
$\nu(G\setminus F)\le\epsilon$.

(c) For every $E\in\Sigma$ there is a set
$H\subseteq E$, expressible as the union of a sequence of compact sets,
such that $\nu(E\setminus H)=0$.

(d) Every continuous real-valued function on $\BbbR^r$ is
$\Sigma$-measurable.

(e) If $h:\BbbR^r\to\Bbb R$ is continuous and has bounded support,
then $h$ is $\nu$-integrable.

\proof{{\bf (a)} For each $n\in\Bbb N$, $B(\tbf{0},n)=\{x:\|x\|\le n\}$
is a closed bounded set,
therefore Borel.   So if $\nu$ is a Radon measure on $\BbbR^r$,
$\sequencen{B(\tbf{0},n)}$ is a sequence of
sets of finite measurea covering $\BbbR^r$.

\medskip

{\bf (b)} Set $E_n=\{x:x\in E,\,n\le\|x\|<n+1\}$ for each $n$.   Then
$\nu E_n<\infty$, so there is a compact set $K_n\subseteq E_n$ such that
$\nu K_n\ge\nu E_n-2^{-n-2}\epsilon$.   Set $F=\bigcup_{n\in\Bbb N}K_n$;
then

\Centerline{$\nu(E\setminus F)
=\sum_{n=0}^{\infty}\nu(E_n\setminus K_n)\le\Bover12\epsilon$.}

\noindent Also $F\subseteq E$ and $F$ is closed because

\Centerline{$F\cap B(\tbf{0},n)=\bigcup_{i\le n}K_i\cap B(\tbf{0},n)$}

\noindent is closed for each $n$.

In the same way, there is a closed set $F'\subseteq\BbbR^r\setminus E$
such that $\nu((\BbbR^r\setminus E)\setminus F')\le\bover12\epsilon$.
Setting $G=\BbbR^r\setminus F'$, we see that $G$ is open, that
$G\supseteq E$ and that $\nu(G\setminus E)\le\bover12\epsilon$, so that
$\nu(G\setminus F)\le\epsilon$, as required.

\medskip

{\bf (c)} By (b), we can choose for each $n\in\Bbb N$ a closed set
$F_n\subseteq E$ such that $\nu(E\setminus F_n)\le 2^{-n}$.   Set
$H=\bigcup_{n\in\Bbb N}F_n$;  then $H\subseteq E$ and $\nu(E\setminus
H)=0$, and also $H=\bigcup_{m,n\in\Bbb N}B(\tbf{0},m)\cap F_n$ is a
countable union of compact sets.

\medskip

{\bf (d)} If $h:\BbbR^r\to\Bbb R$ is continuous, all the sets
$\{x:h(x)>a\}$ are open, so belong to $\Sigma$.

\medskip

{\bf (e)} By (d), $h$ is measurable.   Now we are supposing that there
is some $n\in\Bbb N$ such that $h(x)=0$ whenever $x\notin B(\tbf{0},n)$.
Since $B(\tbf{0},n)$ is compact (2A2F), $h$ is bounded on $B(\tbf{0},n)$
(2A2G), and we have $|h|\le \gamma\chi B(\tbf{0},n)$ for some $\gamma$;
since $\nu B(\tbf{0},n)$ is finite, $h$ is $\nu$-integrable.
}%end of proof of 256B

\leader{256C}{Theorem} A measure $\nu$ on $\BbbR^r$ is a Radon measure
iff it is the completion of a locally finite measure defined on the
$\sigma$-algebra\cmmnt{ $\Cal B$} of Borel subsets of $\BbbR^r$.

\proof{{\bf (a)} Suppose first that $\nu$ is a Radon measure.   Write
$\Sigma$ for its domain.

\medskip

\quad{\bf (i)} Set $\nu_0=\nu\restr\Cal B$.   Then $\nu_0$ is a measure
with domain $\Cal B$, and it is locally finite because
$\nu_0B(\tbf{0},n)=\nu B(\tbf{0},n)$ is finite for every $n$.   Let
$\hat\nu_0$ be the completion of $\nu_0$ (212C).

\medskip

\quad{\bf (ii)} If $\hat\nu_0$ measures $E$, there are $E_1$,
$E_2\in\Cal B$ such that $E_1\subseteq E\subseteq E_2$ and
$\nu_0(E_2\setminus E_1)=0$.   Now $E\setminus E_1\subseteq E_2\setminus
E_1$ must be
$\nu$-negligible;  as $\nu$ is complete, $E\in\Sigma$ and

\Centerline{$\nu E=\nu E_1=\nu_0 E_1=\hat\nu_0E$.}

\medskip

\quad{\bf (iii)} If $E\in\Sigma$, then by 256Bc there is a
Borel set $H\subseteq E$ such that $\nu(E\setminus H)=0$.   Equally,
there is a Borel set $H'\subseteq\BbbR^r\setminus E$ such that
$\nu((\BbbR^r\setminus E)\setminus H')=0$, so that we have $H\subseteq
E\subseteq\BbbR^r\setminus H'$ and

\Centerline{$\nu_0((\BbbR^r\setminus H')\setminus H)=\nu((\Bbb
R^r\setminus H')\setminus H)=0$.}

\noindent So $\hat\nu_0E$ is defined and equal to $\nu_0E_1=\nu E$.

This shows that $\nu=\hat\nu_0$ is the completion of the locally finite
Borel measure $\nu\restr\Cal B$.   And this is true for any Radon
measure $\nu$ on $\BbbR^r$.

\medskip

{\bf (b)} For the rest of the proof, I suppose that $\nu_0$ is a locally
finite measure with domain $\Cal B$ and $\nu$ is its completion.    Write
$\Sigma$ for the domain of $\nu$.   We say that a subset of $\BbbR^r$
is a {\bf K$_{\sigma}$ set} if it is expressible as the union of a
sequence of compact sets.   Note that every K$_{\sigma}$ set is a Borel
set, so belongs to $\Sigma$.   Set

\Centerline{$\Cal A=\{E:E\in\Sigma$, there is a K$_{\sigma}$ set
$H\subseteq E$ such that $\nu(E\setminus H)=0\}$,}

\Centerline{$\Sigma=\{E:E\in\Cal A,\,\BbbR^r\setminus E\in\Cal A\}$.}

\medskip

{\bf (c)(i)} Every open set is itself a K$_{\sigma}$ set, so belongs to
$\Cal A$.
\Prf\ Let $G\subseteq\BbbR^r$ be open.   If $G=\emptyset$ then $G$ is
compact and the result is trivial.   Otherwise, let $\Cal I$ be the set
of closed intervals of the form $[q,q']$, where $q$, $q'\in\Bbb Q^r$,
which are included in $G$.   Then all the members of $\Cal I$ are closed
and bounded, therefore compact.   If $x\in G$, there is a $\delta>0$
such that $B(x,\delta)=\{y:\|y-x\|\le\delta\}\subseteq G$;  now there is
an $I\in\Cal I$ such that $x\in I\subseteq B(x,\delta)$.   Thus
$G=\bigcup\Cal I$.   But $\Cal I$ is countable, so $G$ is K$_{\sigma}$.\
\Qed

\medskip

\quad{\bf (ii)} Every closed subset of $\Bbb R$ is K$_{\sigma}$, so
belongs to $\Cal A$.
\Prf\ If $F\subseteq\Bbb R$ is closed, then
$F=\bigcup_{n\in\Bbb N}F\cap B(\tbf{0},n)$;  but every $F\cap
B(\tbf{0},n)$ is closed and bounded, therefore compact.\ \Qed

\medskip

\quad{\bf (iii)} If $\sequencen{E_n}$ is any sequence in $\Cal A$, then
$E=\bigcup_{n\in\Bbb N}E_n$ belongs to
$\Cal A$.   \Prf\ For each $n\in\Bbb N$ we have a countable family
$\Cal K_n$ of compact subsets of $E_n$ such that $\nu(E_n\setminus\bigcup\Cal K_n)=0$;  now
$\Cal K=\bigcup_{n\in\Bbb N}\Cal K_n$ is a countable
family of compact subsets of $E$, and
$E\setminus\bigcup\Cal K
\subseteq\bigcup_{n\in\Bbb N}(E_n\setminus\bigcup\Cal K_n)$ is
$\nu$-negligible.\ \Qed

\medskip

\quad{\bf (iv)} If $\sequencen{E_n}$ is any sequence in $\Cal A$, then
$F=\bigcap_{n\in\Bbb N}E_n\in\Cal A$.   \Prf\ For each $n\in\Bbb N$, let
$\sequence{i}{K_{ni}}$ be a sequence of compact subsets of $E_n$ such
that $\nu(E_n\setminus\bigcup_{i\in\Bbb N}K_{ni})=0$.   Set
$K'_{nj}=\bigcup_{i\le j}K_{ni}$ for each $j$, so that

\Centerline{$\nu(E_n\cap H)=\lim_{j\to\infty}\nu(K'_{nj}\cap H)$}

\noindent for every $H\in\Sigma$.   Now, for each $m$, $n\in\Bbb N$,
choose $j(m,n)$ such that

\Centerline{$\nu(E_n\cap B(\tbf{0},m)\cap K'_{n,j(m,n)})
\ge\nu(E_n\cap B(\tbf{0},m))-2^{-(m+n)}$.}

\noindent Set $K_m=\bigcap_{n\in\Bbb N}K'_{n,j(m,n)}$;  then $K_m$ is
closed (being an intersection of closed sets) and bounded (being a
subset of $K'_{0,j(m,0)}$), therefore compact.   Also $K_m\subseteq F$,
because $K'_{n,j(m,n)}\subseteq E_n$ for each $n$, and

\Centerline{$\nu(F\cap B(\tbf{0},m)\setminus K_m)
\le\sum_{n=0}^{\infty}\nu(E_n\cap B(\tbf{0},m)\setminus K'_{n,j(m,n)})
\le\sum_{n=0}^{\infty}2^{-(m+n)}
=2^{-m+1}$.}

\noindent Consequently $H=\bigcup_{m\in\Bbb N}K_m$ is a K$_{\sigma}$
subset of $F$ and

\Centerline{$\nu(F\cap B(\tbf{0},m)\setminus H)
\le\inf_{k\ge m}\nu(F\cap B(\tbf{0},k)\setminus H_k)
=0$}

\noindent for every $m$, so $\nu(F\setminus H)=0$ and $F\in\Cal A$.\ \Qed

\medskip

{\bf (d)} $\Sigma$ is a $\sigma$-algebra of subsets of $\Bbb R$.
\Prf\ {(i)} $\emptyset$ and its complement are open, so belong to $\Cal
A$ and therefore to $\Sigma$.
{(ii)} If $E\in\Sigma$ then both $\BbbR^r\setminus E$ and $\Bbb
R^r\setminus(\BbbR^r\setminus E)=E$ belong to $\Cal A$, so $\Bbb
R^r\setminus E\in\Sigma$.   {(iii)} Let $\sequencen{E_n}$ be a
sequence in $\Sigma$ with union $E$.   By (a-iii) and (a-iv),

\Centerline{$E\in\Cal A$,
\quad$\BbbR^r\setminus E
=\bigcap_{n\in\Bbb N}(\BbbR^r\setminus E_n)\in\Cal A$,}

\noindent so $E\in\Sigma$.\  \Qed

\medskip

{\bf (e)} By (c-i) and (c-ii), every open set belongs to $\Sigma$;
consequently every Borel set
belongs to $\Sigma$ and therefore to $\Cal A$.   Now if $E$ is any
member
of $\Sigma$, there is a Borel set $E_1\subseteq E$ such that
$\nu(E\setminus E_1)=0$ and a K$_{\sigma}$ set $H\subseteq E_1$ such
that $\nu(E_1\setminus H)=0$.   Express $H$ as $\bigcup_{n\in\Bbb N}K_n$
where every $K_n$ is compact;  then

\Centerline{$\nu E
=\nu H
=\lim_{n\to\infty}\nu(\bigcup_{i\le n}K_i)
\le\sup_{K\subseteq E\text{ is compact}}\nu K
\le\nu E$}

\noindent because $\bigcup_{i\in n}K_i$ is a compact subset of $E$ for
every $n$.

\medskip

{\bf (f)} Thus $\nu$ is inner regular with respect to the compact sets.
But of course it is complete (being the completion of $\nu_0$) and  a
locally
finite topological measure (because $\nu_0$ is);  so it is a Radon
measure.   This completes the proof.
}%end of proof of 256C

\leader{256D}{Proposition} If $\nu$ and $\nuprime$ are two Radon
measures on $\BbbR^r$, the following are equiveridical:

(i) $\nu=\nuprime$;

(ii) $\nu K=\nuprime K$ for every compact set $K\subseteq\BbbR^r$;

(iii) $\nu G=\nuprime G$ for every open set $G\subseteq\BbbR^r$;

(iv) $\int h\,d\nu=\int h\,d\nuprime$ for every continuous function
$h:\BbbR^r\to\Bbb R$ with bounded support.

\proof{{\bf (a)(i)$\Rightarrow$(iv)} is trivial.

\medskip

{\bf (b)(iv)$\Rightarrow$(iii)} If (iv) is true, and $G\subseteq\BbbR^r$
is an open set, then for each $n\in\Bbb N$ set

\Centerline{$h_n(x)
=\min(1,2^n\inf_{y\in\BbbR^r\setminus(G\cap B(\tbf{0},n))}\|y-x\|)$}

\noindent for $x\in\BbbR^r$.   Then $h_n$ is continuous (in fact
$|h_n(x)-h_n(x')|\le 2^n\|x-x'\|$ for all $x$, $x'\in\BbbR^r$) and zero
outside $B(\tbf{0},n)$, so $\int h_nd\nu=\int h_nd\nuprime$.   Next,
$\sequencen{h_n(x)}$ is a non-decreasing sequence converging to
$\chi G(x)$ for every $x\in\BbbR^r$.   So

\Centerline{$\nu G=\lim_{n\to\infty}\int h_nd\nu
=\lim_{n\to\infty}\int h_nd\nuprime=\nuprime G$,}

\noindent by 135Ga.   As $G$ is arbitrary, (iii) is true.

\medskip

{\bf (c)(iii)$\Rightarrow$(ii)} If (iii) is true, and
$K\subseteq\BbbR^r$ is compact, let $n$ be so large that $\|x\|<n$ for
every $x\in K$.   Set $G=\{x:\|x\|<n\}$, $H=G\setminus K$.   Then $G$
and $H$ are open and $G$ is bounded, so $\nu G=\nuprime G$ is finite,
and

\Centerline{$\nu K=\nu G-\nu H=\nuprime G-\nuprime H=\nuprime K$.}

\noindent As $K$ is arbitrary, (ii) is true.

\medskip

{\bf (d)(ii)$\Rightarrow$(i)} If $\nu$, $\nuprime$ agree on the compact
sets, then

\Centerline{$\nu E=\sup_{K\subseteq E\text{ is compact}}\nu K
=\sup_{K\subseteq E\text{ is compact}}\nuprime K
=\nuprime E$}

\noindent for every Borel set $E$.   So
$\nu\restr\Cal B=\nuprime\restr\Cal B$, where $\Cal B$ is the algebra of Borel sets.
But since $\nu$ and $\nuprime$ are both the completions of their
restrictions to $\Cal B$, they are identical.
}%end of proof of 256D

\leader{256E}{}\cmmnt{ It is I suppose time I gave some examples of
Radon measures.   However it will save a few lines if I first establish
some basic constructions.   You may wish to glance ahead to 256H at this
point.

\medskip

\noindent}{\bf Theorem} Let $\nu$ be a Radon measure on $\BbbR^r$, with
domain $\Sigma$, and
$f$ a non-negative $\Sigma$-measurable function defined on a
$\nu$-conegligible subset of $\BbbR^r$.   Suppose that $f$ is
{\bf locally integrable} in the sense that $\int_Efd\nu<\infty$ for every
bounded set $E$.
Then the indefinite-integral measure $\nuprime$ on $\BbbR^r$ defined by
saying that

\Centerline{$\nuprime E=\int_Efd\nu$ whenever
$E\cap\{x:x\in\dom f,\,f(x)>0\}\in\Sigma$}

\noindent is a Radon measure on $\BbbR^r$.

\proof{ For the construction of $\nuprime$, see 234I-234L.
Indefinite-integral measures, as I define them, are always complete (234I).
$\nuprime$ is locally finite because $f$ is
locally integrable.   $\nuprime$ is a
topological measure because every open set belongs to $\Sigma$ and
therefore to the domain $\Sigma'$ of $\nuprime$.
To see that $\nuprime$ is inner regular with respect to the compact
sets, take any set $E\in\Sigma'$, and set
$E'=\{x:x\in E\cap\dom f,\,f(x)>0\}$.
Then $E'\in\Sigma$, so there is a set $H\subseteq E'$,
expressible as the union of a sequence of compact sets,
such that $\nu(E'\setminus H)=0$.   In this case

\Centerline{$\nuprime(E\setminus H)=\int_{E\setminus H}fd\nu=0$.}

\noindent Let $\sequencen{K_n}$ be a sequence of compact sets with union
$H$;  then

\Centerline{$\nuprime E=\nuprime H
=\lim_{n\to\infty}\nuprime(\bigcup_{i\le n}K_i)
\le\sup_{K\subseteq E\text{ is compact}}\nuprime K\le\nuprime E$.}

\noindent As $E$ is arbitrary, $\nuprime$ is inner regular with respect
to the compact sets.
}%end of proof of 256E

\leader{256F}{Theorem} Let $\nu$ be a Radon measure on $\BbbR^r$, and
$\Sigma$ its domain.   Let $f:D\to\Bbb R$ be a $\Sigma$-measurable
function, where $D\subseteq\BbbR^r$.   Then for every $\epsilon>0$
there is a closed set $F\subseteq\BbbR^r$ such that
$\nu(\BbbR^r\setminus F)\le\epsilon$ and $f\restr F$ is continuous.

\proof{ By 121I, there is a $\Sigma$-measurable function
$h:\BbbR^r\to\Bbb R$ extending $f$.
Enumerate $\Bbb Q$ as $\sequencen{q_n}$.
For each $n\in\Bbb N$ set $E_n=\{x:h(x)\le q_n\}$, $E'_n=\{x:h(x)>q_n\}$
and use 256Bb to choose closed sets  $F_n\subseteq E_n$,
$F'_n\subseteq E'_n$ such that
$\nu(E_n\setminus F_n)\le 2^{-n-2}\epsilon$ and
$\nu(E'_n\setminus F'_n)\le 2^{-n-2}\epsilon$.   Set
$F=\bigcap_{n\in\Bbb N}(F_n\cup F'_n)$;  then $F$ is closed and

\Centerline{$\nu(\BbbR^r\setminus F)
\le\sum_{n=0}^{\infty}\nu(\BbbR^r\setminus(F_n\cup F'_n))
\le\sum_{n=0}^{\infty}\nu(E_n\setminus F_n)+\nu(E'_n\setminus F'_n)
\le\epsilon$.}

I claim that $h\restr F$ is continuous.   \Prf\ Suppose that $x\in F$
and $\delta>0$.   Then there are $m$, $n\in\Bbb N$ such that

\Centerline{$h(x)-\delta\le q_m<h(x)\le q_n\le h(x)+\delta$.}

\noindent This means that $x\in E'_m\cap E_n$;  consequently
$x\notin F_m\cup F'_n$.
Because $F_m\cup F'_n$ is closed, there is an $\eta>0$
such that $y\notin F_m\cup F'_n$ whenever $\|y-x\|\le\eta$.
Now suppose that $y\in F$ and $\|y-x\|\le\eta$.   Then
$y\in(F_m\cup F'_m)\cap(F_n\cup F'_n)$ and $y\notin F_m\cup F'_n$,
so $y\in F'_m\cap F_n\subseteq E'_m\cap E_n$ and $q_m<h(y)\le q_n$.
Consequently
$|h(y)-h(x)|\le\delta$.   As $x$ and $\delta$ are arbitrary, $h\restr F$
is continuous.\ \QeD\
It follows that $f\restr F=(h\restr F)\restr D$ is continuous, as required.
}%end of proof of 256F

\leader{256G}{Theorem} Let $\nu$ be a Radon measure on $\BbbR^r$, with
domain $\Sigma$, and
suppose that $\phi:\BbbR^r\to\BbbR^s$ is measurable in the sense that
all its coordinates are $\Sigma$-measurable.   If the image measure
$\nuprime=\nu\phi^{-1}$\cmmnt{ (234D)} is locally finite, it is a
Radon measure.

\proof{ Write $\Sigma'$ for the
domain of $\nuprime$.   If $\phi=(\phi_1,\ldots,\phi_s)$, then

\Centerline{$\phi^{-1}[\{y:\eta_j\le\alpha\}]
=\{x:\phi_j(x)\le\alpha\}\in\Sigma$,}

\noindent so $\{y:\eta_j\le\alpha\}\in\Sigma'$
for every $j\le s$, $\alpha\in\Bbb R$, where I write
$y=(\eta_1,\ldots,\eta_s)$ for $y\in\BbbR^s$.   Consequently every
Borel subset of $\BbbR^s$ belongs to $\Sigma'$ (121J), and $\nuprime$
is a topological measure.   It is complete by 234Eb.

The point is of course that $\nuprime$ is inner regular with respect to
the compact
sets.   \Prf\ Suppose that $F\in\Sigma'$ and that $\gamma<\nuprime F$.
For each $j\le s$, there is a closed set $H_j\subseteq\BbbR^r$ such
that $\phi_j\restr H_j$ is continuous and
$\nu(\BbbR^r\setminus H_j)<\bover1{s}(\nuprime F-\gamma)$, by 256F.   Set $H=\bigcap_{j\le s}H_j$;  then $H$ is closed and $\phi\restr H$ is continuous and

\Centerline{$\nu(\BbbR^r\setminus H)<\nuprime F-\gamma
=\nu\phi^{-1}[F]-\gamma$,}

\noindent so that
$\nu(\phi^{-1}[F]\cap H)>\gamma$.   Let $K\subseteq\phi^{-1}[F]\cap H$
be a compact set such that $\nu K\ge\gamma$, and set $L=\phi[K]$.
Because $K\subseteq H$ and $\phi\restr H$ is continuous, $L$ is compact
(2A2Eb).   Of course $L\subseteq F$, and

\Centerline{$\nuprime L=\nu\phi^{-1}[L]\ge\nu K\ge\gamma$.}

\noindent As $F$ and $\gamma$ are arbitrary, $\nuprime$ is inner regular
with respect to the compact sets.\ \Qed

Since $\nuprime$ is locally finite by the hypothesis of the theorem, it
is a Radon measure.
}%end of proof of 256G

\leader{256H}{Examples}\cmmnt{ I come at last to the promised
examples.

\medskip

} {\bf (a)} Lebesgue measure on $\BbbR^r$ is a Radon measure.
\prooflet{(It is a topological measure by 115G, and inner regular with
respect to the compact sets by 134Fb.)}

\spheader 256Hb A point-supported measure on $\BbbR^r$ is a Radon measure
iff it is locally finite.   \prooflet{\Prf\ Let $\mu$ be a point-supported
measure on $\BbbR^r$.   If it is a Radon measure, then of course it is
locally finite.   If it is locally finite, then surely it is a 
complete topological
measure, since it measures every subset of $\BbbR^r$.   Let 
$h:\BbbR^r\to[0,\infty]$ be such that $\mu E=\sum_{x\in E}h(x)$ for every
$E\subseteq\BbbR^r$.   Take any $E\subseteq\BbbR^r$.   Then

$$\eqalign{\mu E
&=\sum_{x\in E}h(x)
=\sup_{I\subseteq E\text{ is finite}}\sum_{x\in I}h(x)\cr&
=\sup_{I\subseteq E\text{ is finite}}\mu I
\le\sup_{K\subseteq E\text{ is compact}}\mu K
\le\mu E
\cr}$$

\noindent so $\mu E=\sup_{K\subseteq E\text{ is compact}}\mu K$;
thus $\mu$ is inner regular with respect to the compact sets and is
a Radon measure.\ \Qed}%end of prooflet

\spheader 256Hc \cmmnt{Now we come to a new idea.}  Recall that the
Cantor set $C$\cmmnt{ (134G)} is a closed Lebesgue negligible subset of
$[0,1]$, and that the Cantor function\cmmnt{ (134H)} is a
non-decreasing
continuous function $f:[0,1]\to[0,1]$ such that $f(0)=0$, $f(1)=1$ and
$f$ is constant on each of the intervals composing $[0,1]\setminus C$.
It follows that if we set $g(x)=\bover12(x+f(x))$ for $x\in[0,1]$, then
$g:[0,1]\to[0,1]$ is a continuous permutation\cmmnt{ such that the
Lebesgue measure of $g[C]$ is $\bover12$ (134I);
consequently $g^{-1}:[0,1]\to[0,1]$ is
continuous}.   Now extend $g$ to a permutation $h:\Bbb R\to\Bbb R$ by
setting $h(x)=x$ for $x\in\Bbb R\setminus[0,1]$.
\cmmnt{Then $h$ and $h^{-1}$ are continuous.   Note that $h[C]=g[C]$
has Lebesgue measure $\bover12$.}

Let $\nu_1$ be the indefinite-integral measure defined from  Lebesgue
measure $\mu$ on $\Bbb R$ and
the function $2\chi(h[C])$\cmmnt{;  that is, $\nu_1E=2\mu(E\cap h[C])$ whenever
this is defined}.   \cmmnt{By 256E,} $\nu_1$ is a Radon measure, and
$\nu_1h[C]=\nu_1\Bbb R=1$.   Let $\nu$ be the measure $\nu_1(h^{-1})^{-1}$,
that is,
$\nu E=\nu_1h[E]$ for just those $E\subseteq\Bbb R$ such that
$h[E]\in\dom\nu_1$.   Then $\nu$ is a Radon probability measure on
$\Bbb R$\cmmnt{, by 256G}, and $\nu C=1$,
$\nu(\Bbb R\setminus C)=\mu C=0$.

\cmmnt{
\leader{256I}{Remarks (a)} The measure $\nu$ of 256Hc, sometimes called
{\bf Cantor measure}, is a classic example, and as such has many
constructions, some rather more natural than the one I use here (see
256Xk, and also 264Ym below).   But I choose the method above
because it
yields directly, without further investigation or any appeal to more
advanced general theory, the fact that $\nu$ is a Radon measure.

\spheader 256Ib The examples above are chosen to represent
the extremes under the `Lebesgue decomposition' described in
232I.   If $\nu$ is a (totally finite) Radon measure on $\BbbR^r$, we
can use 232Ib to express its restriction $\nu\restr\Cal B$ to the
Borel $\sigma$-algebra as $\nu_p+\nu_{ac}+\nu_{cs}$, where $\nu_p$ is
the `point-mass' or `atomic' part of $\nu\restr\Cal B$, $\nu_{ac}$ is
the `absolutely
continuous' part (with respect to Lebesgue measure), and $\nu_{cs}$ is
the `atomless singular part'.   In the example of 256Hb, we have
$\nu\restr\Cal B=\nu_p$;  in 256E, if we start from Lebesgue measure, we
have $\nu\restr\Cal B=\nu_{ac}$;
and in 256Hc we have $\nu\restr\Cal B=\nu_{cs}$.
}%end of comment

\leader{256J}{Absolutely continuous Radon \dvrocolon{measures}}\cmmnt{
It is worth pausing a moment over the indefinite-integral measures
described in 256E.

\medskip

\noindent}{\bf Proposition} Let $\nu$ be a Radon measure
on $\BbbR^r$, and write $\mu$ for Lebesgue measure on
$\BbbR^r$.   Then the following are equiveridical:

\quad(i) $\nu$ is an indefinite-integral measure over $\mu$;

\quad(ii) $\nu E=0$ whenever $E$ is a Borel subset of $\BbbR^r$ and
$\mu E=0$.

\noindent In this case, if $g\in\eusm L^0(\mu)$ and
$\int_Eg\,d\mu=\nu E$ for
every Borel set $E\subseteq\BbbR^r$, then $g$ is a Radon-Nikod\'ym
derivative of $\nu$ with respect to $\mu$\cmmnt{ in the sense of
232Hf}.

\proof{{\bf (a)(i)$\Rightarrow$(ii)} If $f$ is a Radon-Nikod\'ym
derivative of $\nu$ with respect to $\mu$, then of course

\Centerline{$\nu E=\int_Efd\mu=0$}

\noindent whenever $\mu E=0$.

\medskip

\quad{\bf (ii)$\Rightarrow$(i)} If $\nu E=0$ for every $\mu$-negligible
Borel set $E$, then $\nu E$ is defined and equal to $0$ for
every $\mu$-negligible set $E$, because $\nu$ is complete and any
$\mu$-negligible set is included in a $\mu$-negligible Borel set.
Consequently $\dom\nu$ includes the
domain $\Sigma$ of $\mu$, since every Lebesgue measurable set is
expressible as the union of a Borel set and a negligible set.

For each $n\in\Bbb N$ set $E_n=\{x:n\le\|x\|<n+1\}$, so that
$\sequencen{E_n}$ is a partition of $\BbbR^r$ into bounded Borel sets.
Set $\nu_nE=\nu(E\cap E_n)$ for every Lebesgue measurable set $E$ and
every $n\in\Bbb N$.   Now $\nu_n$ is absolutely continuous with respect
to $\mu$ (232Ba), so by the Radon-Nikod\'ym theorem (232F)
there is a $\mu$-integrable function $f_n$ such that
$\int_Ef_nd\mu=\nu_n E$
for every Lebesgue measurable set $E$.   Because $\nu_n E\ge 0$ for
every $E\in\Sigma$,
$f_n\ge 0$ a.e.;  because $\nu_n(\BbbR^r\setminus E_n)=0$, $f_n=0$
a.e.\ on $\BbbR^r\setminus E_n$.   Now if we set

\Centerline{$f=\max(0,\sum_{n=0}^{\infty}f_n)$,}

\noindent $f$ will be defined $\mu$-a.e.\ and we shall have

\Centerline{$\int_Efd\mu
=\sum_{n=0}^{\infty}\int_Ef_nd\mu
=\sum_{n=0}^{\infty}\nu(E\cap E_n)=\nu E$}

\noindent for every Borel set $E$, so that the indefinite-integral
measure $\nuprime$ defined by $f$ and $\mu$ agrees with $\nu$ on the
Borel sets.   Since this ensures that $\nuprime$ is locally finite,
$\nuprime$ is a Radon measure, by 256E, and is equal to $\nu$, by 256D.
Accordingly $\nu$ is an indefinite-integral measure over $\mu$.

\medskip

{\bf (b)} As in (a-ii) above, $g$ must be locally integrable and the
indefinite-integral measure defined by $g$ agrees with $\nu$ on the
Borel sets, so is identical with $\nu$.
}%end of proof of 256J

\leader{256K}{\dvrocolon{Products}}\cmmnt{ The class of Radon measures
on Euclidean spaces is stable under a wide variety of operations, as we
have already seen;  in particular, we have the following.

\medskip

\noindent}{\bf Theorem} Let $\nu_1$, $\nu_2$ be Radon
measures on $\BbbR^r$ and $\BbbR^s$ respectively.
Let $\lambda$ be their c.l.d.\ product measure on
$\BbbR^r\times\BbbR^s$.   Then $\lambda$ is a Radon measure.

\cmmnt{\medskip

\noindent{\bf Remark} When I say that $\lambda$ is `Radon' according
to the definition in 256A, I am of course identifying
$\BbbR^r\times\BbbR^s$ with $\BbbR^{r+s}$, as in 251M-251N.
}

\proof{ I hope the following notation will seem natural.
Write $\Sigma_1$, $\Sigma_2$
for the domains of $\nu_1$, $\nu_2$;  $\Cal B_r$, $\Cal B_s$ for the
Borel $\sigma$-algebras of $\BbbR^r$, $\BbbR^s$;  $\Lambda$ for the
domain of $\lambda$;  and $\Cal B$ for the Borel $\sigma$-algebra of
$\BbbR^{r+s}$.

Because each $\nu_i$ is the completion of its restriction to the Borel
sets (256C), $\lambda$ is the product of $\nu_1\restr\Cal B_r$ and
$\nu_2\restr\Cal B_s$ (251T).   Because $\nu_1\restr\Cal B_r$ and
$\nu_2\restr\Cal B_s$ are $\sigma$-finite (256Ba, 212Ga), $\lambda$ must
be the completion of its restriction to $\Cal B_r\tensorhat\Cal B_s$,
which by 251M is identified with $\Cal B$.   Setting
$Q_n=\{(x,y):\|x\|\le n$, $\|y\|\le n\}$ we have

\Centerline{$\lambda Q_n=\nu_1\{x:\|x\|\le n\}\cdot\nu_2\{y:\|y\|\le n\}
<\infty$}

\noindent for every $n$, while every bounded subset of $\BbbR^{r+s}$ is
included in some $Q_n$.   So $\lambda\restr\Cal B$ is locally finite,
and its completion $\lambda$ is a Radon measure, by 256C.
}%end of proof of 256K

\cmmnt{
\leader{256L}{Remark} We see from 253I that if $\nu_1$ and
$\nu_2$ are Radon measures on $\BbbR^r$ and $\BbbR^s$ respectively, and
both are indefinite-integral measures over Lebesgue measure, then their
product measure on $\BbbR^{r+s}$ is also an indefinite-integral measure
over Lebesgue measure.
}%end of comment

\leader{*256M}{}\cmmnt{ For the sake of applications in \S286 below, I
include another result, which is in fact one of the fundamental
properties of Radon measures, as will appear in \S414.

\medskip

\noindent}{\bf Proposition} Let $\nu$ be a Radon measure on $\Bbb R^r$,
and $D$ any subset of $\Bbb R^r$.   Let $\Phi$ be a non-empty
upwards-directed family of non-negative continuous functions from $D$ to
$\Bbb R$.   For $x\in D$ set $g(x)=\sup_{f\in\Phi}f(x)$ in $[0,\infty]$.
Then

(a) $g:D\to[0,\infty]$ is lower semi-continuous,
therefore Borel measurable;

(b) $\int_Dg\,d\nu=\sup_{f\in\Phi}\int_Dfd\nu$.

\proof{{\bf (a)} For any $u\in[-\infty,\infty]$,

\Centerline{$\{x:x\in D,\,g(x)>u\}
=\bigcup_{f\in\Phi}\{x:x\in D,\,f(x)>u\}$}

\noindent is an open set for the subspace topology on $D$ (2A3C), so is
the intersection of $D$ with a Borel subset of $\Bbb R^r$.   This is
enough to show that $g$ is Borel measurable (121B-121C).

\medskip

{\bf (b)} Accordingly $\int_Dg\,d\nu$ will be defined in $[0,\infty]$,
and of course $\int_Dg\,d\nu\ge\sup_{f\in\Phi}\int_Dfd\nu$.

For the reverse inequality, observe that there is a countable set
$\Psi\subseteq\Phi$ such that $g(x)=\sup_{f\in\Psi}f(x)$ for every
$x\in D$.   \Prf\ For $a\in\Bbb Q$, $q$, $q'\in\BbbQ^r$ set

\Centerline{$\Phi_{aqq'}
=\{f:f\in\Phi,\,f(y)>a\text{ whenever }y\in D\cap[q,q']\}$,}

\noindent interpreting $[q,q']$ as in 115G.   Choose
$f_{aqq'}\in\Phi_{aqq'}$ if $\Phi_{aqq'}$ is not empty, and arbitrarily
in $\Phi$ otherwise;  and set
$\Psi=\{f_{aqq'}:a\in\Bbb Q,\,q,\,q'\in\BbbQ^r\}$, so that $\Psi$ is a
countable subset of $\Phi$.   If $x\in D$ and $b<g(x)$, there is an
$a\in\Bbb Q$ such that $b\le a<g(x)$;  there is an
$\hat f\in\Phi$ such that $\hat f(x)>a$;  because $\hat f$ is
continuous, there are $q$, $q'\in\Bbb Q^r$ such that $q\le x\le q'$ and
$\hat f(y)\ge a$ whenever $y\in D\cap[q,q']$;  so that
$\hat f\in\Phi_{aqq'}$, $\Phi_{aqq'}\ne\emptyset$,
$f_{aqq'}\in\Phi_{aqq'}$ and $\sup_{f\in\Psi}f(x)\ge f_{aqq'}(x)\ge b$.
As $b$ is
arbitrary, $g(x)=\sup_{f\in\Psi}f(x)$.\ \Qed

Let $\sequencen{f_n}$ be a sequence running over $\Psi$.   Because $\Phi$ is
upwards-directed, we can choose $\sequencen{f'_n}$ in $\Phi$ inductively
in such a way that $f'_{n+1}\ge\max(f'_n,f_n)$ for every $n\in\Bbb N$.
So $\sequencen{f'_n}$ is a non-decreasing sequence in $\Phi$ and
$\sup_{n\in\Bbb N}f'_n(x)\ge\sup_{f\in\Psi}f(x)=g(x)$ for every
$x\in D$.   By B.Levi's theorem,

\Centerline{$\int_Dg\,d\nu
\le\sup_{n\in\Bbb N}\int_Df'_nd\nu\le\sup_{f\in\Phi}\int_Dfd\nu$,}

\noindent and we have the required inequality.
}%end of proof of 256M

\exercises{
\leader{256X}{Basic exercises $\pmb{>}$(a)}
%\spheader 256Xa
Let $\nu$ be a measure on $\BbbR^r$.   (i) Show that it is locally
finite, in the sense of 256Ab, iff for every $x\in\BbbR^r$ there is a
$\delta>0$ such that $\nu^*B(x,\delta)<\infty$.   \Hint{the sets
$B(\tbf{0},n)$ are compact.}   (ii) Show that in this case $\nu$ is
$\sigma$-finite.
%256A

\sqheader 256Xb Let $\nu$ be a Radon measure on $\BbbR^r$ and $\Cal G$
a non-empty
upwards-directed family of open sets in $\BbbR^r$.   (i) Show that
$\nu(\bigcup\Cal G)=\sup_{G\in\Cal G}\nu G$.   \Hint{observe that if
$K\subseteq\bigcup\Cal G$ is compact, then $K\subseteq G$ for some
$G\in\Cal G$.}   (ii) Show that
$\nu(E\cap\bigcup\Cal G)=\sup_{G\in\Cal G}\nu(E\cap G)$ for every set $E$ which is measured by $\nu$.
%256A

\sqheader 256Xc Let $\nu$ be a Radon measure on $\BbbR^r$ and $\Cal F$
a non-empty downwards-directed family of closed sets in $\BbbR^r$ such
that $\inf_{F\in\Cal F}\nu F<\infty$.   (i) Show that
$\nu(\bigcap\Cal F)=\inf_{F\in\Cal F}\nu F$.   \Hint{apply 256Xb(ii) to $\Cal G=\{\BbbR^r\setminus F:F\in\Cal F\}$.}   (ii) Show that
$\nu(E\cap\bigcap\Cal F)=\inf_{F\in\Cal F}\nu(E\cap F)$ for every
$E$ in the domain of $\nu$.
%256A

\sqheader 256Xd Show that a Radon measure $\nu$ on $\BbbR^r$ is
atomless iff $\nu\{x\}=0$ for every $x\in\BbbR^r$.   \Hint{apply 256Xc
with $\Cal F=\{F:F\subseteq E$ is closed, not negligible$\}$.}
%256A

\spheader 256Xe Let $\nu_1$, $\nu_2$ be Radon measures on
$\BbbR^r$, and $\alpha_1$, $\alpha_2\in\ooint{0,\infty}$.   Set
$\Sigma=\dom\nu_1\cap\dom\nu_2$, and for $E\in\Sigma$ set
$\nu E=\alpha_1\nu_1E+\alpha_2\nu_2E$.   Show that $\nu$ is a Radon measure
on $\BbbR^r$.   Show that $\nu$ is an indefinite-integral measure over
Lebesgue measure iff $\nu_1$, $\nu_2$ are, and that in this case a linear
combination of
of Radon-Nikod\'ym derivatives of $\nu_1$ and $\nu_2$ is a
Radon-Nikod\'ym derivative of $\nu$.
%256A

\sqheader 256Xf Let $\nu$ be a Radon measure on $\BbbR^r$.   (i) Show
that there is a unique closed set $F\subseteq\BbbR^r$ such that, for
open sets $G\subseteq\BbbR^r$, $\nu G>0$ iff $G\cap F\ne\emptyset$.
($F$ is called the {\bf support} of $\nu$.)   (ii) Generally, a
set $A\subseteq\BbbR^r$ is called {\bf self-supporting} if $\nu^*(A\cap
G)>0$ whenever $G\subseteq\BbbR^r$ is an open set meeting $A$.   Show
that for every closed set $F\subseteq\BbbR^r$ there is a unique
self-supporting closed set $F'\subseteq F$ such that $\nu(F\setminus
F')=0$.
%256A

\sqheader 256Xg Show that a measure $\nu$ on $\Bbb R$ is a
Radon measure iff it is a Lebesgue-Stieltjes measure as
described in 114Xa.   Show that in this case $\nu$ is an
indefinite-integral measure over Lebesgue measure iff the function
$x\mapsto\nu[a,x]:[a,b]\to\Bbb R$ is
absolutely continuous whenever $a\le b$ in $\Bbb R$.
%256C

\spheader 256Xh Let $\nu$ be a Radon measure on $\BbbR^r$.   Let
$C_k$ be the space of continuous real-valued functions on
$\BbbR^r$ with bounded supports.   Show that for every
$\nu$-integrable function $f$ and every $\epsilon>0$ there is a
$g\in C_k$ such that $\int|f-g|d\nu\le\epsilon$.   \Hint{use
arguments from 242O, but in (a-i) of the proof there start with {\it
closed} intervals $I$.}
%256C

\spheader 256Xi Let $\nu$ be a Radon measure on $\BbbR^r$, and $\nu^*$ the
corresponding outer measure.   Show that
$\nu A=\inf\{\nu G:G\supseteq A$ is open$\}$ for every set
$A\subseteq\BbbR^r$.
%256D

\spheader 256Xj Let $\nu$, $\nuprime$ be two Radon measures on
$\BbbR^r$, and
suppose that $\nu I=\nuprime I$ for every half-open interval
$I\subseteq\BbbR^r$ (definition:  115Ab).   Show that $\nu=\nuprime$.
%256D

\spheader 256Xk Let $\nu$ be Cantor measure (256Hc).
(i) Show that if $C_n$ is the $n$th set used in the construction of the
Cantor set, so that $C_n$ consists of $2^n$ intervals of length
$3^{-n}$, then $\nu I=2^{-n}$ for each of the intervals $I$ composing
$C_n$.   (ii) Let $\lambda$ be the usual measure on $\{0,1\}^{\Bbb N}$
(254J).   Define $\phi:\{0,1\}^{\Bbb N}\to \Bbb R$ by setting
$\phi(x)=\bover23\sum_{n=0}^{\infty}3^{-n}x(n)$ for each
$x\in\{0,1\}^{\Bbb N}$.   Show that $\phi$ is a bijection between
$\{0,1\}^{\Bbb N}$ and $C$.   (iii) Show that if $\Cal B$ is the Borel
$\sigma$-algebra of $\Bbb R$, then $\{\phi^{-1}[E]:E\in\Cal B\}$ is
precisely the $\sigma$-algebra of subsets of $\{0,1\}^{\Bbb N}$
generated by the sets $\{x:x(n)=i\}$ for
$n\in\Bbb N$, $i\in\{0,1\}$.   (iv) Show that $\phi$ is an isomorphism
between $(\{0,1\}^{\Bbb N},\lambda)$ and $(C,\nu_C)$, where $\nu_C$ is
the subspace measure on $C$ induced by $\nu$.
%256H

\spheader 256Xl Let $\nu$ and $\nuprime$ be two Radon measures on
$\BbbR^r$.   Show that $\nuprime$ is an indefinite-integral measure over
$\nu$ iff $\nuprime E=0$ whenever $\nu E=0$, and in this case a function
$f$ is a Radon-Nikod\'ym derivative of $\nuprime$ with respect to $\nu$
iff $\int_Efd\nu=\nuprime E$ for every Borel set $E$.
%256J

\leader{256Y}{Further exercises (a)}
%\spheader 256Ya
Let $\nu$ be a Radon measure on $\BbbR^r$, and $X$ any subset of $\Bbb
R^r$;  let $\nu_X$ be the subspace measure on $X$ and $\Sigma_X$ its
domain, and give $X$ its
subspace topology.   Show that $\nu_X$ has the following
properties: (i) $\nu_X$ is complete and locally determined;   (ii) every
open subset of $X$ belongs to $\Sigma_X$;  (iii)
$\nu_XE=\sup\{\nu_XF:F\subseteq E$ is closed in $X\}$ for every
$E\in\Sigma_X$;   (iv) whenever $\Cal G$ is a non-empty
upwards-directed family of open subsets of $X$,
$\nu_X(\bigcup\Cal G)=\sup_{G\in\Cal G}\nu_XG$;
(v) every point of $X$ belongs to an open set of finite measure.
%256A

\spheader 256Yb Let $\nu$ be a Radon measure on $\BbbR^r$, with domain
$\Sigma$, and $f:\Bbb R^r\to\Bbb R$ a function.   Show that the following
are equiveridical:  (i) $f$ is $\Sigma$-measurable;
(ii) for every non-negligible set
$E\in\Sigma$ there is a non-negligible $F\in\Sigma$  such
that $F\subseteq E$ and $f\restr F$ is continuous;  (iii) for every set
$E\in\Sigma$, $\nu E=\sup_{K\in\Cal K_f,K\subseteq E}\nu K$, where
$\Cal K_f=\{K:K\subseteq\BbbR^r$ is compact, $f\restr K$ is continuous$\}$.
\Hint{for (ii)$\Rightarrow$(i), apply 215B(iv) to $\Cal K_f$.}
%256F

\spheader 256Yc Take $\nu$, $X$, $\nu_X$ and $\Sigma_X$ as in 256Ya.
Suppose that
$f:X\to\Bbb R$ is a function.   Show that $f$ is $\Sigma_X$-measurable
iff for every non-negligible measurable set $E\subseteq X$ there is a
non-negligible measurable $F\subseteq E$ such that $f\restr F$ is
continuous.
%256F

\spheader 256Yd\dvAnew{2011}(i) Let $\lambda$ be the usual measure on
$\{0,1\}^{\Bbb N}$.   Define $\psi:\{0,1\}^{\Bbb N}\to\{0,1\}^{\Bbb N}$ by
setting $\psi(x)(i)=x(i+1)$ for $x\in\{0,1\}^{\Bbb N}$ and $j\in\Bbb N$.
Show that $\psi$ is \imp.   (ii) Define $\theta:\Bbb R\to\Bbb R$ by setting
$\theta(t)=\fraction{3t}=3t-\lfloor 3t\rfloor$ for $t\in\Bbb R$.
Show that $\theta$ is \imp\ for Cantor measure as defined in 256Hc.
%256H 256Xk 

\spheader 256Ye Let $\sequencen{\nu_n}$ be a sequence of Radon measures
on $\BbbR^r$.   Show that there is a Radon measure $\nu$ on $\BbbR^r$
such that every $\nu_n$ is an indefinite-integral measure over $\nu$.
\Hint{find a sequence $\sequencen{\alpha_n}$ of
strictly positive numbers such that $\sum_{n=0}^{\infty}\alpha_n\nu_n
B(\tbf{0},k)<\infty$ for every $k$, and set
$\nu=\sum_{n=0}^{\infty}\alpha_n\nu_n$, using the idea of 256Xe.}
%256J, 256Xe, 256Xl

\spheader 256Yf A set
$G\subseteq\BbbR^{\Bbb N}$ is {\bf open} if for every $x\in G$ there are
$n\in\Bbb N$, $\delta>0$ such that

\Centerline{$\{y:y\in\BbbR^{\Bbb N},\,|y(i)-x(i)|<\delta$ for every
$i\le n\}\subseteq G$.}

\noindent The {\bf Borel $\sigma$-algebra} of $\BbbR^{\Bbb N}$ is the
$\sigma$-algebra $\Cal B$ of subsets of
$\BbbR^{\Bbb N}$ generated, in the sense of 111Gb, by the family
$\frak T$ of open sets.   (i) Show that $\frak T$ is a topology (2A3A).
(ii) Show that a filter $\Cal F$ on $\BbbR^{\Bbb N}$ converges to
$x\in\BbbR^{\Bbb N}$ iff $\pi_i[[\Cal F]]\to x(i)$ for every
$i\in\Bbb N$, where
$\pi_i(y)=y(i)$ for $i\in\Bbb N$, $y\in\BbbR^{\Bbb N}$.   (iii)
Show that $\Cal B$ is the $\sigma$-algebra generated by sets of the form
$\{x:x\in \BbbR^{\Bbb N},\,x(i)\le a\}$,
where $i$ runs over $\Bbb N$ and $a$ runs over $\Bbb R$.
(iv) Show that if $\alpha_i\ge 0$ for every $i\in\Bbb N$, then
$\{x:|x(i)|\le\alpha_i\Forall i\in\Bbb N\}$ is compact.
%2A3R
(v) Show that any open set in $\BbbR^{\Bbb N}$ is the union of a
sequence of closed sets.   \Hint{look at sets of the form
$\{x:q_i\le x(i)\le q'_i\Forall i\le n\}$, where $q_i$, $q'_i\in\Bbb Q$ for $i\le n$.}   (vi) Show that if $\nu_0$ is any probability  measure with domain
$\Cal B$, then its completion $\nu$ is inner regular with respect to the
compact sets, and therefore may be called a `Radon measure on
$\BbbR^{\Bbb N}$'.   \Hint{show that there are compact sets of measure
arbitrarily
close to $1$, and therefore that every open set, and every closed set,
includes a K$_{\sigma}$ set of the same measure.}
%256K  256Cprf
}%end of exercises


\endnotes{
\Notesheader{256} Radon measures on Euclidean spaces are very special,
and the results of this section do not give clear pointers to the
direction the theory takes when applied to other kinds of topological
space.   With the material here you could make
a stab at developing a theory of Radon measures on
complete separable metric spaces, provided you use 256Xa as the basis
for your definition of `locally finite'.   These are the spaces for
which a version of 256C is true.   (See 256Yf.)   But for
generalizations to other
types of topological space, and for the more
interesting parts of the theory on $\BbbR^r$, I must ask you to wait
for Volume 4.
My purpose in introducing Radon measures here is strictly
limited;  I wish only to give a basis for \S257 and \S271 sufficiently
solid not
to need later revision.   In fact I think that all we really need are
the Radon probability measures.

The chief technical difficulty in the definition of
`Radon measure' here lies
in the insistence on completeness.   It may well be that for
everything studied in this volume, it would be simpler to look at
locally finite measures with domain the algebra of Borel sets.
This would involve us in a number of circumlocutions when dealing with
Lebesgue measure itself and its derivates, since Lebesgue measure
is defined on a larger $\sigma$-algebra;  but the serious objection
arises in the more advanced theory, when non-Borel sets of
various kinds become central.   Since my aim in this book is to provide
secure foundations for the study of all aspects of measure theory, I ask
you to take a little extra trouble now in order to avoid the possibility
of having to re-work all your ideas later.
The extra trouble arises, for instance, in 256D, 256Xe and 256Xj;  since
different Radon measures are defined on different $\sigma$-algebras, we
have to check that two Radon measures which agree on the compact sets,
or on the open sets, have the same domains.   On the credit side, some
of the power of 256G arises from the fact that the Radon image measure
$\nu\phi^{-1}$ is defined on the whole $\sigma$-algebra
$\{F:\phi^{-1}[F]\in\dom(\nu)\}$, not just on the Borel sets.

The further technical point that Radon measures are expected to be
locally finite gives less difficulty;  its effect is that from most
points of view there is little difference between a general Radon
measure and a totally finite Radon measure.   The extra condition which
obviously has to be put into the hypotheses of such results as 256E and
256G is no burden on either intuition or memory.

In effect, we have two definitions of Radon measures on Euclidean
spaces:  they are the inner regular locally finite topological measures,
and they are also the completions of the locally finite Borel measures.
The equivalence of these definitions is Theorem 256C.   The latter
definition is the better adapted to 256K, and the former to 256G.    The
`inner regularity' of the basic definition refers to compact sets;
we also have forms of inner regularity with respect to closed sets
(256Bb) and K$_{\sigma}$ sets (256Bc), and a complementary notion of `outer
regularity' with respect to open sets (256Xi).

}%end of notes

\discrpage

