\frfilename{mt472.tex}
\versiondate{22.3.11}
\copyrightdate{2000}

\def\chaptername{Geometric measure theory}
\def\sectionname{Besicovitch's Density Theorem}

\newsection{472}

The first step in the program of the next few sections
is to set out some very remarkable properties of
Euclidean space.   We find that in $\BbbR^r$, for geometric reasons
(472A), we have versions of Vitali's theorem (472B-472C) and Lebesgue's
Density Theorem (472D) for arbitrary Radon measures.   I add a version
of the Hardy-Littlewood Maximal Theorem (472F).

Throughout the section, $r\ge 1$ will be a fixed integer.   As usual, I
write $B(x,\delta)$ for the closed ball with centre $x$ and radius
$\delta$.   $\|\,\|$ will represent the Euclidean norm, and
$x\dotproduct y$ the scalar product of $x$ and $y$, so that
$x\dotproduct y=\sum_{i=1}^r\xi_i\eta_i$ if $x=(\xi_1,\ldots,\xi_r)$ and
$y=(\eta_1,\ldots,\eta_r)$.

\leader{472A}{Besicovitch's Covering Lemma} Suppose that $\epsilon>0$ is
such that $(5^r+1)(1-\epsilon-\epsilon^2)^r>(5+\epsilon)^r$.   Let
$x_0,\ldots,x_n\in\BbbR^r$, $\delta_0,\ldots,\delta_n>0$ be such that

\Centerline{$\|x_i-x_j\|>\delta_i$,
\quad$\delta_j\le(1+\epsilon)\delta_i$}

\noindent whenever $i<j\le n$.   Then

\Centerline{$\#(\{i:i\le n,\,\|x_i-x_n\|\le\delta_i+\delta_n\})
\le 5^r$.}

\proof{ Set $I=\{i:i\le n,\,\|x_i-x_n\|\le\delta_n+\delta_i\}$.

\medskip

{\bf (a)} It will simplify the formulae of the main argument if we
suppose for the time being that $\delta_n=1$;  in this case
$1\le(1+\epsilon)\delta_i$, so that $\delta_i\ge\Bover1{1+\epsilon}$
for every $i\le n$, while we still have $\delta_i<\|x_i-x_n\|$ for every
$i<n$, and $\|x_i-x_n\|\le 1+\delta_i$ for every $i\in I$.

For $i\in I$, define $x'_i$ by saying that

\inset{-- if $\|x_i-x_n\|\le 2+\epsilon$, $x_i'=x_i$;}

\inset{-- if $\|x_i-x_n\|>2+\epsilon$, $x'_i$ is to be that point of the
closed line segment from $x_n$ to $x_i$ which is at distance
$2+\epsilon$ from $x_n$.}

\medskip

{\bf (b)} The point is that $\|x'_i-x'_j\|>1-\epsilon-\epsilon^2$
whenever $i$, $j$ are distinct members of $I$.   \Prf\ We may suppose
that $i<j$.

\medskip

\quad{\bf case 1} Suppose that $\|x_i-x_n\|\le 2+\epsilon$ and
$\|x_j-x_n\|\le 2+\epsilon$.   In this case

\Centerline{$\|x'_i-x'_j\|
=\|x_i-x_j\|\ge\delta_i\ge\Bover1{1+\epsilon}\ge 1-\epsilon$.}

\medskip

\quad{\bf case 2} Suppose that $\|x_i-x_n\|\ge 2+\epsilon
\ge\|x_j-x_n\|$.   In this case

$$\eqalign{\|x'_i-x'_j\|
&=\|x'_i-x_j\|
\ge\|x_i-x_j\|-\|x_i-x'_i\|\cr
&\ge\delta_i-\|x_i-x_n\|+2+\epsilon
\ge\delta_i-\delta_i-1+2+\epsilon
=1+\epsilon.\cr}$$

\medskip

\quad{\bf case 3} Suppose that $\|x_i-x_n\|\le 2+\epsilon
\le\|x_j-x_n\|$.   Then

$$\eqalignno{\|x'_i-x'_j\|
&=\|x_i-x'_j\|
\ge\|x_i-x_j\|-\|x_j-x'_j\|
>\delta_i-\|x_j-x_n\|+2+\epsilon\cr
&\ge\delta_i-\delta_j-1+2+\epsilon
\ge\delta_i-\delta_i(1+\epsilon)+1+\epsilon
\ge 1+\epsilon-\epsilon(2+\epsilon)\cr
\noalign{\noindent (because $\delta_i<\|x_i-x_n\|\le 2+\epsilon$)}
&=1-\epsilon-\epsilon^2.\cr}$$

\medskip

\quad{\bf case 4} Suppose that $2+\epsilon\le\|x_j-x_n\|
\le\|x_i-x_n\|$. Let $y$ be the point on the line segment between $x_i$
and $x_n$ which is the same distance from $x_n$ as $x_j$.   In this case

\Centerline{$\|y-x_j\|
\ge\|x_i-x_j\|-\|x_i-y\|
\ge\delta_i-\|x_i-x_n\|+\|x_j-x_n\|
\ge\|x_j-x_n\|-1$.}

\noindent Because the triangles $(x_n,y,x_j)$ and $(x_n,x'_i,x'_j)$ are
similar,

\Centerline{$\|x'_i-x'_j\|=\Bover{2+\epsilon}{\|x_j-x_n\|}\|y-x_j\|
\ge(2+\epsilon)\Bover{\|x_j-x_n\|-1}{\|x_j-x_n\|}\ge 1+\epsilon$}

\noindent because $\|x_j-x_n\|\ge 2+\epsilon$.

\medskip

\quad{\bf case 5} Suppose that $2+\epsilon\le\|x_i-x_n\|\le\|x_j-x_n\|$.
This time, let $y$ be the point on the line segment from $x_n$ to $x_j$
which is the same distance from $x_n$ as $x_i$ is.   We now have

$$\eqalign{\|y-x_i\|
&\ge\|x_i-x_j\|-\|x_j-y\|
>\delta_i-\|x_j-x_n\|+\|x_i-x_n\|\cr
&\ge\delta_j-\epsilon\delta_i-(\delta_j+1)+\|x_i-x_n\|\cr
&=\|x_i-x_n\|-1-\epsilon\delta_i
\ge\|x_i-x_n\|(1-\epsilon)-1,\cr}$$

\noindent so that

$$\eqalign{\|x'_i-x'_j\|
&=\Bover{2+\epsilon}{\|x_i-x_n\|}\|y-x_i\|
>(2+\epsilon)\Bover{\|x_i-x_n\|(1-\epsilon)-1}{\|x_i-x_n\|}\cr
&\ge(2+\epsilon)\Bover{(2+\epsilon)(1-\epsilon)-1}{2+\epsilon}
=1-\epsilon-\epsilon^2.\cr}$$

So we have the required inequality in all cases.\ \Qed

\medskip

{\bf (c)} Now consider the balls
$B(x'_i,\bover{1-\epsilon-\epsilon^2}2)$ for $i\in I$.   These are
disjoint, all have Lebesgue measure
$2^{-r}\beta_r(1-\epsilon-\epsilon^2)^r$ where $\beta_r$ is the measure
of the unit ball $B(\tbf{0},1)$, and are all included in the ball
$B(x_n,2+\epsilon+\bover{1-\epsilon}2)$, which has measure
$2^{-r}\beta_r(5+\epsilon)^r$.   So we must have

\Centerline{$2^{-r}\beta_r(1-\epsilon-\epsilon^2)^r\#(I)
\le 2^{-r}\beta_r(5+\epsilon)^r$.}

\noindent But $\epsilon$ was declared to be so small that this implies
that $\#(I)\le 5^r$, as claimed.

\medskip

{\bf (d)} This proves the lemma in the case $\delta_n=1$.   For the
general case, replace each $x_i$ by $\delta_n^{-1}x_i$ and each
$\delta_i$ by $\delta_i/\delta_n$;  the change of scale does not affect
the hypotheses or the set $I$.
}%end of proof of 472A

\leader{472B}{Theorem} Let $A\subseteq\BbbR^r$ be a bounded set,
and $\Cal I$ a family of non-trivial closed balls in $\BbbR^r$ such
that every point of $A$ is the centre of a member of $\Cal I$.   Then
there is a family $\langle\Cal I_k\rangle_{k<5^r}$ of countable
subsets of $\Cal I$ such that each $\Cal I_k$ is disjoint and
$\bigcup_{k<5^r}\Cal I_k$ covers $A$.

\proof{{\bf (a)} For each $x\in A$ let $\delta_x>0$ be such that
$B(x,\delta_x)\in\Cal I$.   If either $A$ is empty or $\sup_{x\in
A}\delta_x=\infty$, the result is trivial.   (In the latter case, take
$x\in A$ such that $\delta_x\ge\diam A$ and set
$\Cal I_0=\{B(x,\delta_x)\}$, $\Cal I_k=\emptyset$ for $k>0$.)   So let us
suppose henceforth that $\{\delta_x:x\in A\}$ is bounded in $\Bbb R$.
In this case, $C=\bigcup_{x\in A}B(x,\delta_x)$ is bounded in $\Bbb R^r$.

Fix $\epsilon>0$ such that
$(5^r+1)(1-\epsilon-\epsilon^2)^r>(5+\epsilon)^r$.

\medskip

{\bf (b)} Choose inductively a sequence $\sequencen{B_n}$ in
$\Cal I\cup\{\emptyset\}$ as follows.
Given $\langle B_i\rangle_{i<n}$, then
if $A\subseteq\bigcup_{i<n}B_i$ set $B_n=\emptyset$.   Otherwise, set
$\alpha_n=\sup\{\delta_x:x\in A\setminus\bigcup_{i<n}B_i\}$, choose
$x_n\in A\setminus\bigcup_{i<n}B_i$ such that
$(1+\epsilon)\delta_{x_n}\ge\alpha_n$, set $B_n=B(x_n,\delta_{x_n})$ and
continue.

Now whenever $n\in\Bbb N$, $I_n=\{i:i<n,\,B_i\cap B_n\ne\emptyset\}$ has
fewer than $5^r$ members.   \Prf\ We may suppose that $B_n\ne\emptyset$,
in which case $B_i=B(x_i,\delta_{x_i})$ for every $i\le n$, and the
$x_i$, $\delta_{x_i}$ are such that, whenever $i<j\le n$,

\Centerline{$x_j\notin B_i$, i.e., $\|x_i-x_j\|>\delta_{x_i}$,}

\Centerline{$\delta_{x_j}\le\alpha_i\le(1+\epsilon)\delta_{x_i}$.}

\noindent But now 472A gives the result at once.
\Qed

\medskip

{\bf (c)} We may therefore define a function
$f:\Bbb N\to\{0,1,\ldots,5^r-1\}$ by setting

\Centerline{$f(n)=\min\{k:0\le k<5^r,\,f(i)\ne k$ for every $i\in
I_n\}$}

\noindent for every $n\in\Bbb N$.   Set $\Cal I_k=\{B_i:i\in\Bbb
N,\,f(i)=k,\,B_i\ne\emptyset\}$  for each $k<5^r$.   By the choice of
$f$, $i\notin I_j$, so that $B_i\cap B_j=\emptyset$, whenever $i<j$ and
$f(i)=f(j)$;  thus every $\Cal I_k$ is disjoint.   Since $B_i\subseteq
C$ for every $i$, $\sum\{\mu B_i:f(i)=k\}\le\mu^*C$ for every $k<5^r$,
and $\sum_{i=0}^{\infty}\mu B_i\le 5^r\mu^*C$ is finite.

\medskip

{\bf (d)} \Quer\ Suppose, if possible, that

\Centerline{$A\not\subseteq\bigcup_{k<5^r}\bigcup\Cal
I_k=\bigcup_{n\in\Bbb N}B_n$.}

\noindent Take $x\in A\setminus\bigcup_{n\in\Bbb N}B_n$.   Then, first,
$A\not\subseteq\bigcup_{i<n}B_i$ for every $n$, so that $\alpha_n$ is
defined;  next, $\alpha_n\ge\delta_x$, so that
$(1+\epsilon)\delta_{x_n}\ge\delta_x$ for every $n$.   But this means
that $\mu B_n\ge\beta_r\bigl(\Bover{\delta_x}{1+\epsilon}\bigr)^r$
for every $n$, and $\sum_{n=0}^{\infty}\mu B_n=\infty$;  which is
impossible.\ \Bang

\medskip

{\bf (e)} Thus $A\subseteq\bigcup_{k<5^r}\bigcup\Cal I_k$, as required.
}%end of proof of 472B

\leader{472C}{Theorem} Let $\lambda$ be a Radon measure on
$\BbbR^r$, $A$ a subset of $\BbbR^r$ and $\Cal I$ a family of
non-trivial closed
balls in $\BbbR^r$ such that every point of $A$ is the centre of
arbitrarily small members of $\Cal I$.   Then

(a) there is a countable disjoint $\Cal I_0\subseteq\Cal I$ such that
$\lambda(A\setminus\bigcup\Cal I_0)=0$;

(b) for every $\epsilon>0$ there is a countable
$\Cal I_1\subseteq\Cal I$ such that $A\subseteq\bigcup\Cal I_1$ and
$\sum_{B\in\Cal I_1}\lambda B\le\lambda^*A+\epsilon$.

\proof{{\bf (a)(i)} The first step is to show that if $A'\subseteq A$ is
bounded then there is a finite disjoint set $\Cal J\subseteq\Cal I$ such
that $\lambda^*(A'\cap\bigcup\Cal J)\ge 6^{-r}\lambda^*A'$.   \Prf\ If
$\lambda^*A'=0$ take $\Cal J=\emptyset$.   Otherwise, by 472B, there is
a
family $\langle\Cal J_k\rangle_{k<5^r}$ of disjoint countable subsets of
$\Cal I$ such that $\bigcup_{k<5^r}\Cal J_k$ covers $A'$.   Accordingly

\Centerline{$\lambda^*A'
\le\sum_{k=0}^{5^r-1}\lambda^*(A'\cap\bigcup\Cal J_k)$}

\noindent and there is some $k<5^r$ such that
$\lambda^*(A'\cap\bigcup\Cal J_k)\ge 5^{-r}\lambda^*A'$.   Let
$\sequence{i}{B_i}$ be a sequence running
over $\Cal J_k$;  then

\Centerline{$\lim_{n\to\infty}\lambda^*(A'\cap\bigcup_{i\le n}B_i)
=\lambda^*(A'\cap\bigcup\Cal J_k)\ge 5^{-r}\lambda^*A'$,}

\noindent so there is some $n\in\Bbb N$ such that
$\lambda^*(A'\cap\bigcup_{i\le n}B_i)\ge 6^{-r}\lambda^*A'$, and we can
take
$\Cal J=\{B_i:i\le n\}$.\ \Qed

\medskip

\quad{\bf (ii)} Now choose $\sequencen{\Cal K_n}$ inductively, as
follows.   Start by fixing on a sequence $\sequencen{m_n}$ running over
$\Bbb N$ with cofinal repetitions.
Take $\Cal K_0=\emptyset$.   Given that $\Cal K_n$ is a finite disjoint
subset of $\Cal I$, set
$\Cal I'=\{B:B\in\Cal I,\,B\cap\bigcup\Cal K_n=\emptyset\}$,
$A_n=A\cap B(\tbf{0},m_n)\setminus\bigcup\Cal K_n$.
Because every point of $A$ is the centre of arbitrarily small members of
$\Cal I$, and $\bigcup\Cal K_n$ is closed, every member of $A_n$ is the
centre of (arbitrarily small) members of $\Cal I'$, and (i) tells us
that there is a finite disjoint set $\Cal J_n\subseteq\Cal I'$ such that
$\lambda^*(A_n\cap\bigcup\Cal J_n)\ge 6^{-r}\lambda^*A_n$.   Set
$\Cal K_{n+1}=\Cal K\cup\Cal J_n$, and continue.   At the end of the
induction, set $\Cal I_0=\bigcup_{n\in\Bbb N}\Cal K_n$;  because
$\sequencen{\Cal K_n}$ is non-decreasing and every $\Cal K_n$ is
disjoint, $\Cal I_0$ is disjoint, and of course
$\Cal I_0\subseteq\Cal I$.

The effect of this construction is to ensure that

$$\eqalignno{\lambda^*(A\cap B(\tbf{0},m_n)\setminus\bigcup\Cal K_{n+1})
&=\lambda^*(A_n\setminus\bigcup\Cal J_n)
=\lambda^*A_n-\lambda^*(A_n\cap\bigcup\Cal J_n)\cr
\noalign{\noindent (because $\bigcup\Cal J_n$ is a closed set, therefore
measured by $\lambda$)}
&\le(1-6^{-r})\lambda^*A_n\cr
&=(1-6^{-r})
  \lambda^*(A\cap B(\tbf{0},m_n)\setminus\bigcup\Cal K_n)\cr}$$

\noindent for every $n$.   So, for any $m\in\Bbb N$,

\Centerline{$\lambda^*(A\cap B(\tbf{0},m)\setminus\bigcup\Cal K_n)
\le\lambda^*(A\cap B(\tbf{0},m))
  (1-6^{-r})^{\#(\{j:j<n,m_j=m\})}
\to 0$}

\noindent as $n\to\infty$, and
$\lambda^*(A\cap B(\tbf{0},m)\setminus\bigcup\Cal I_0)=0$.   As $m$ is
arbitrary, $\lambda^*(A\setminus\bigcup\Cal I_0)=0$, as required.

\medskip

{\bf (b)(i)} Let $E\supseteq A$ be such that $\lambda E=\lambda^*A$, and
$H\supseteq E$ an open set such that
$\lambda H\le\lambda E+\bover12\epsilon$
(256Bb).   Set $\Cal I'=\{B:B\in\Cal I,\,B\subseteq H\}$.   Then every
point of $A$ is the centre of arbitrarily small members of $\Cal I'$, so
by (a) there is a disjoint family $\Cal I_0\subseteq\Cal I'$ such that
$\lambda(A\setminus\bigcup\Cal I_0)=0$.   Of course

\Centerline{$\sum_{B\in\Cal I_0}\lambda B
=\lambda(\bigcup\Cal I_0)\le\lambda H\le\lambda^*A+\bover12\epsilon$.}

\medskip

\quad{\bf (ii)} For $m\in\Bbb N$ set
$A_m=A\cap B(\tbf{0},m)\setminus\bigcup\Cal I_0$.   Then there is a
$\Cal J_m\subseteq\Cal I$, covering $A_m$, such that
$\sum_{B\in\Cal J_m}\lambda B\le 2^{-m-2}\epsilon$.   \Prf\  There is an
open set $G\supseteq A_m$
such that $\lambda G\le 5^{-r}2^{-m-2}\epsilon$.   Now
$\Cal I''=\{B:B\in\Cal I,\,B\subseteq G\}$ covers $A_m$, so there is a
family $\langle\Cal J_{mk}\rangle_{k<5^r}$ of disjoint countable
subfamilies of
$\Cal I''$ such that $\Cal J_m=\bigcup_{k<5^r}\Cal J_{mk}$ covers $A_m$.
For each $k$,

\Centerline{$\sum_{B\in\Cal J_{mk}}\lambda B=\lambda(\bigcup\Cal J_{mk})
\le\lambda G$,}

\noindent so

\Centerline{$\sum_{B\in\Cal J_m}\lambda B\le 5^r\lambda G
\le 2^{-m-2}\epsilon$.
\Qed}

\medskip

\quad{\bf (iii)} Setting
$\Cal I_1=\Cal I_0\cup\bigcup_{m\in\Bbb N}\Cal J_m$ we have a cover of
$A$ by members of $\Cal I$, and

$$\eqalign{\sum_{B\in\Cal I_1}\lambda B
&\le\sum_{B\in\Cal I_0}\lambda B
  +\sum_{m=0}^{\infty}\sum_{B\in\Cal J_m}\lambda B\cr
&\le\lambda^*A+\Bover12\epsilon+\sum_{m=0}^{\infty}2^{-m-2}\epsilon
=\lambda^*A+\epsilon.\cr}$$
}%end of proof of 472C

\leader{472D}{Besicovitch's Density Theorem} Let $\lambda$ be any Radon
measure on $\BbbR^r$.   Then, for any locally $\lambda$-integrable
real-valued function $f$,

(a) $f(y)
=\lim_{\delta\downarrow 0}\Bover1{\lambda B(y,\delta)}
  \int_{B(y,\delta)}fd\lambda$,

(b) $\lim_{\delta\downarrow 0}\Bover1{\lambda B(y,\delta)}
  \int_{B(y,\delta)}|f(x)-f(y)|\lambda(dx)=0$

\noindent for $\lambda$-almost every $y\in\BbbR^r$.

\cmmnt{\medskip

\noindent{\bf Remark} The theorem asserts that, for $\lambda$-almost
every $y$, limits of the form
$\lim_{\delta\downarrow 0}\Bover1{\lambda B(y,\delta)}\ldots$ are
defined;  \cmmnt{in my usage,} this includes the assertion that
$\lambda B(y,\delta)\ne 0$ for all sufficiently small $\delta>0$.
}

\proof{ (Compare 261C and 261E.)

\medskip

{\bf (a)} Let $Z$ be the support of $\lambda$ (411Nd);  then $Z$ is
$\lambda$-conegligible and $\lambda B(y,\delta)>0$ whenever $y\in Z$ and
$\delta>0$.   For $q<q'$ in $\Bbb Q$ and $n\in\Bbb N$ set

\Centerline{$A_{nqq'}=\{y:y\in Z\cap\dom f,\,\|y\|<n,\,f(y)\le q,\,
  \limsup_{\delta\downarrow 0}\Bover1{\lambda B(y,\delta)}
  \biggerint_{B(y,\delta)}fd\lambda>q'\}$.}

\noindent Then $\lambda A_{nqq'}=0$.   \Prf\ Let $\epsilon>0$.   Then
there is an
$\eta\in\ocint{0,\epsilon}$ such that $\int_F|f|d\lambda\le\epsilon$
whenever $F\subseteq B(\tbf{0},n)$ and $\lambda F\le\eta$ (225A).   Let
$E$ be a measurable envelope of $A_{nqq'}$ included in
$\{y:y\in Z\cap\dom f,\,f(y)\le q,\,\|y\|<n\}$, and take an open set
$G\supseteq E$ such that $G\subseteq B(\tbf{0},n)$ and
$\lambda(G\setminus E)\le\eta$ (256Bb again).
Let $\Cal I$ be the family of
non-singleton closed balls $B\subseteq G$ such that
$\int_Bf\ge q'\lambda B$.   Then every point of $A_{nqq'}$ is the centre of arbitrarily small members of
$\Cal I$, so there is a disjoint family $\Cal I_0\subseteq\Cal I$ such
that $\lambda(A_{nqq'}\setminus\bigcup\Cal I_0)=0$ (472C).   Now
$\lambda(E\setminus\bigcup\Cal I_0)=0$ and
$\lambda((\bigcup\Cal I_0)\setminus E)\le\eta\le\epsilon$, so

$$\eqalign{q'\lambda E
&\le q'\lambda(\bigcup\Cal I_0)+\epsilon|q'|
=\sum_{B\in\Cal I_0}q'\lambda B+\epsilon|q'|\cr
&\le\sum_{B\in\Cal I_0}\int_Bfd\lambda+\epsilon|q'|
=\int_{\bigcup\Cal I_0}fd\lambda+\epsilon|q'|\cr
&\le\int_Efd\lambda+\epsilon(1+|q'|)
\le q\lambda E+\epsilon(1+|q'|),\cr}$$

\noindent and

\Centerline{$(q'-q)\lambda^*A_{nqq'}=(q'-q)\lambda E
\le(1+|q'|)\epsilon$.}

\noindent As $\epsilon$ is arbitrary, $\lambda^*A_{nqq'}=0$.\ \Qed

As $n$, $q$ and $q'$ are arbitrary,

\Centerline{$\limsup_{\delta\downarrow 0}\Bover1{\lambda B(y,\delta)}
 \biggerint_{B(y,\delta)}f\le f(y)$}

\noindent for $\lambda$-almost every $y\in Z$, therefore for
$\lambda$-almost every $y\in\BbbR^r$.   Similarly, or applying the
same argument to $-f$,

\Centerline{$\liminf_{\delta\downarrow 0}\Bover1{\lambda B(y,\delta)}
 \biggerint_{B(y,\delta)}f\ge f(y)$}

\noindent for $\lambda$-almost every $y$, and

\Centerline{$\lim_{\delta\downarrow 0}\Bover1{\lambda B(y,\delta)}
 \biggerint_{B(y,\delta)}f$ exists $=f(y)$}

\noindent for $\lambda$-almost every $y$.

\medskip

{\bf (b)} Now, for each $q\in\Bbb Q$ set
$g_q(x)=|f(x)-q|$ for $x\in\dom f$.   By (a), we have a
$\lambda$-conegligible set $D$ such that

\Centerline{$\lim_{\delta\downarrow 0}\Bover1{\lambda B(y,\delta)}
  \biggerint_{B(y,\delta)}g_qd\lambda=g_q(y)$}

\noindent for every $y\in D$ and $q\in\Bbb Q$.   Now, if
$y\in D$ and $\epsilon>0$, there is a $q\in\Bbb Q$ such that
$|f(y)-q|\le\epsilon$, and a $\delta_0>0$ such that

\Centerline{$|\Bover1{\lambda B(y,\delta)}
  \biggerint_{B(y,\delta)}g_qd\lambda-g_q(y)|
\le\epsilon$}

\noindent whenever $0<\delta\le\delta_0$.   But in this case

$$\eqalign{\Bover1{\lambda B(y,\delta)}
&\int_{B(y,\delta)}|f(x)-f(y)|\lambda(dx)\cr
&\le\Bover1{\lambda B(y,\delta)}\int_{B(y,\delta)}|f(x)-q|\lambda(dx)
  +\Bover1{\lambda B(y,\delta)}\int_{B(y,\delta)}|q-f(y)|\lambda(dx)\cr
&\le 3\epsilon.\cr}$$

\noindent As $\epsilon$ is arbitrary,

\Centerline{$\lim_{\delta\downarrow 0}\Bover1{\lambda B(y,\delta)}
  \biggerint_{B(y,\delta)}|f(x)-f(y)|\lambda(dx)
=0$;}

\noindent as this is true for every $y\in D$, the theorem is proved.
}%end of proof of 472D

\leader{*472E}{Proposition} Let $\lambda$, $\lambda'$ be Radon
measures on $\BbbR^r$, and $G\subseteq\BbbR^r$ an open set.   Let $Z$
be the support of $\lambda$, and for $x\in Z\cap G$ set

\Centerline{$M(x)=\sup\{\Bover{\lambda'B}{\lambda B}:B\subseteq G$ is a
non-trivial ball with centre $x\}$.}

\noindent Then

\Centerline{$\lambda\{x:x\in Z,\,M(x)\ge t\}\le\Bover{5^r}t\lambda'G$}

\noindent for every $t>0$.

\proof{ The function
$M:Z\to[0,\infty]$ is lower semi-continuous.   \Prf\ If $M(x)>t\ge 0$,
there is a $\delta>0$ such that $B(x,\delta)\subseteq G$ and
$\lambda'B(x,\delta)>t\lambda B(x,\delta)$.   Because $\lambda$ is a
Radon measure, there is an open set $V\supseteq B(x,\delta)$ such that
$V\subseteq G$ and
$\lambda'B(x,\delta)>t\lambda V$;  because $B(x,\delta)$ is compact,
there is an $\eta>0$ such that $B(x,\delta+2\eta)\subseteq V$.   Now if
$y\in Z$ and $\|y-x\|\le\eta$,

\Centerline{$B(x,\delta)\subseteq B(y,\delta+\eta)\subseteq V$,}

\noindent so $\lambda'B(y,\delta+\eta)>t\lambda B(y,\delta+\eta)$ and
$M(y)>t$.\ \Qed

In particular, $H_t=\{x:x\in Z\cap G,\,M(x)>t\}$ is always measured by
$\lambda$.   Now, given $t>0$, let $\Cal I$ be the set of non-trivial
closed balls $B\subseteq G$ such that $\lambda'B>t\lambda B$.   By 472B,
there is a family $\ofamily{k}{5^r}{\Cal I_k}$ of countable disjoint
subsets of $\Cal I$ such that $\bigcup_{k<5^r}\Cal I_k$ covers $H_t$.
So

\Centerline{$\lambda H_t
\le\sum_{k=0}^{5^r-1}\sum_{B\in\Cal I_k}\lambda B
\le\Bover1t\sum_{k=0}^{5^r-1}\sum_{B\in\Cal I_k}\lambda'B
\le\Bover{5^r}t\lambda'G$,}

\noindent as claimed.
}%end of proof of 472E

\leader{*472F}{Theorem} Let $\lambda$ be a Radon measure on
$\BbbR^r$, and $f\in\eusm L^p(\lambda)$ any function, where
$1<p<\infty$.   Let $Z$ be the support of $\lambda$, and for $x\in Z$
set $f^*(x)=\sup_{\delta>0}\Bover1{\lambda B(x,\delta)}
  \int_{B(x,\delta)}|f|d\lambda$.   Then $f^*$ is lower semi-continuous,
and $\|f^*\|_p\le 2\bigl(\Bover{5^rp}{p-1}\bigr)^{1/p}\|f\|_p$.

\proof{{\bf (a)} Replacing $f$ by $|f|$ if necessary, we may suppose
that $f\ge 0$.   $Z$ is $\lambda$-conegligible, so that $f^*$
is defined $\lambda$-almost everywhere.   Next, $f^*$ is lower
semi-continuous.   \Prf\ I repeat an idea from the proof of 472E.
If $f^*(x)>t\ge 0$, there is a $\delta>0$ such that
$\int_{B(x,\delta)}|f|d\lambda>t\lambda B(x,\delta)$.   Because
$\lambda$ is a Radon measure, there is an open set
$V\supseteq B(x,\delta)$ such that
$\int_{B(x,\delta)}|f|d\lambda>t\lambda V$;  because $B(x,\delta)$ is
compact, there is an $\eta>0$ such that $B(x,\delta+2\eta)\subseteq V$;
and now $f^*(y)>t$ for every $y\in Z\cap B(x,\eta)$.\ \Qed

\medskip

{\bf (b)} For $t>0$, set
$H_t=\{x:x\in Z,\,f^*(x)>t\}$ and $F_t=\{x:x\in\dom f,\,f(x)\ge t\}$.
Then

\Centerline{$\lambda H_t
\le\Bover{2\cdot 5^r}{t}\biggerint_{F_{t/2}}fd\lambda$.}

\noindent\Prf\ Set $g=f\times\chi F_{t/2}$.   Because
$(\bover{t}2)^p\lambda F_{t/2}\le\|f\|_p^p$ is finite, $\lambda F_{t/2}$
is finite,  $\chi F_{t/2}\in\eusm L^q(\lambda)$ (where
$\bover1p+\bover1q=1$) and $g\in\eusm L^1(\lambda)$ (244Eb).
Let $\lambda'$ be the indefinite-integral measure defined by $g$ over
$\lambda$ (234J);
then $\lambda'$ is totally finite, and is a Radon measure (416S).   Set

\Centerline{$M(x)=\sup\{\Bover{\lambda'B}{\lambda B}:B\subseteq\BbbR^r$
is a non-trivial ball with centre $x\}$}

\noindent for $x\in Z$.   Then $f^*(x)\le M(x)+\bover{t}2$ for every
$x\in Z$, just because

\Centerline{$\biggerint_Bfd\lambda
\le\Bover{t}2\lambda B+\int_Bg\,d\lambda
=\Bover{t}2\lambda B+\lambda'B$}

\noindent for every closed ball $B$.   Accordingly

$$\eqalignno{\lambda H_t
&\le\lambda\{x:M(x)>\Bover{t}2\}
\le\Bover{2\cdot 5^r}{t}\lambda'\BbbR^r\cr
\displaycause{by 472E}
&=\Bover{2\cdot 5^r}t\biggerint_{F_{t/2}}fd\lambda. \text{ \Qed}\cr}$$

\medskip

{\bf (c)} As in part (c) of the proof of 286A, we now have

$$\eqalign{\int(f^*)^pd\lambda
&=\int_0^{\infty}\lambda\{x:f^*(x)^p>t\}dt
=p\int_0^{\infty}t^{p-1}\lambda\{x:f^*(x)>t\}dt\cr
&\le 2\cdot 5^rp\int_0^{\infty}t^{p-2}\int_{F_{t/2}}fd\lambda dt
=2\cdot 5^rp\int_{\Bbb R^r}f(x)\int_0^{2f(x)}t^{p-2}dt\lambda(dx)\cr
&=2\cdot 5^rp\int_{\Bbb R^r}\Bover{2^{p-1}}{p-1}f(x)^p\lambda(dx)
=\Bover{2^p5^rp}{p-1}\int f^pd\lambda.\cr}$$

\noindent Taking $p$th roots, we have the result.
}%end of proof of 472F

\exercises{\leader{472X}{Basic exercises (a)}
%\spheader 472Xa
Show that if $\lambda$, $\lambda'$ are Radon measures on $\Bbb R^r$
which agree on closed balls, they are equal.   (Cf.\ 466Xj.)
%472C

\spheader 472Xb Let $\lambda$ be a Radon measure on $\BbbR^r$.   Let
$A\subseteq\BbbR^r$ be a non-empty set, and
$\epsilon>0$.   Show that there is a sequence $\sequencen{B_n}$ of
closed balls in $\BbbR^r$, all of radius at most $\epsilon$ and with
centres in $A$, such that $A\subseteq\bigcup_{n\in\Bbb N}B_n$ and
$\sum_{n=0}^{\infty}\lambda B_n\le\lambda^*A+\epsilon$.
%472C

\spheader 472Xc Let $\lambda$ be a non-zero Radon measure on $\BbbR^r$
and $Z$ its support.
Show that we have a lower density $\phi$ (definition: 341C) for the
subspace measure $\lambda_Z$ defined by
setting $\phi E=\{x:x\in Z$, $\lim_{\delta\downarrow 0}
\Bover{\lambda(E\cap B(x,\delta))}{\lambda B(x,\delta)}=1\}$ whenever
$\lambda_Z$ measures $E$.
%472D

\spheader 472Xd\dvAnew{2012}
Let $\lambda$ be a Radon measure on $\BbbR^r$, and
$f$ a locally $\lambda$-integrable function.   Show that
$E=\{y:g(y)=\lim_{\delta\downarrow 0}\Bover1{\lambda B(y,\delta)}
\int_{B(y,\delta)}f\,d\lambda$ is defined in $\Bbb R\}$ is a Borel set, and
that $g:E\to\Bbb R$ is Borel measurable.
%472D

\leader{472Y}{Further exercises (a)}%
%\spheader 472Ya
(i) Let $\Cal I$ be a finite family of intervals (open,
closed or half-open) in $\Bbb R$.   Show that there are subfamilies
$\Cal I_0$, $\Cal I_1\subseteq\Cal I$, both disjoint, such that
$\Cal I_0\cup\Cal I_1$ covers $\bigcup\Cal I$.   \Hint{induce on
$\#(\Cal I)$.}    Show that this remains true if any totally ordered set
is put in place of $\Bbb R$.   (ii) Show that if $\Cal I$ is any family
of non-empty intervals in $\Bbb R$ such that none contains the centre
of any other, then $\Cal I$ is expressible as $\Cal I_0\cup\Cal I_1$
where both $\Cal I_0$ and $\Cal I_1$ are disjoint.
%472B

\spheader 472Yb Let $m=m(r)$ be the largest number such that there are
$u_1,\ldots,u_m\in\BbbR^r$ such that $\|u_i\|=1$ for every $i$ and
$\|u_i-u_j\|\ge 1$ for all $i\ne j$.   Let $A\subseteq\BbbR^r$ be a
bounded set and $x\mapsto\delta_x:A\to\ooint{0,\infty}$ a bounded
function;  set
$B_x=B(x,\delta_x)$ for $x\in A$.   (i) Show that $m<3^r$.
(ii) Show that there is an $\epsilon\in\ocint{0,\bover1{10}}$ such that
whenever $\|u_0\|=\ldots=\|u_m\|=1$ there are distinct $i$,
$j\le m$ such that $u_i\dotproduct u_j>\bover12(1+\epsilon)$.
(iii) Suppose that $u$, $v\in\BbbR^r$ are such that
$\bover13\le\|u\|\le 1$, $\|v\|\le 1+\epsilon$ and $\|u-v\|>1$.   Show
that the angle $u\widehat{\tbf{0}}v$ has cosine at most
$\bover12(1+\epsilon)$.   \Hint{maximise $\Bover{a^2+b^2-c^2}{2ab}$
subject to $\bover13\le a\le 1$, $b\le 1+\epsilon$ and $c\ge 1$.}   (iv)
Suppose that $\sequencen{x_n}$ is a
sequence in $A$ such that $x_n\notin B_{x_i}$ for $i<n$ and
$(1+\epsilon)\delta_{x_n}
\ge\sup\{\delta_x:x\in A\setminus\bigcup_{i<n}B_{x_i}\}$ for every $n$.
Show that $A\subseteq\bigcup_{n\in\Bbb N}B_{x_n}$.
(v) Take $y\in\BbbR^r$.   Show that there is at most one $n$ such that
$\|y-x_n\|\le\bover13\delta_{x_n}$.
(vi) Show that if $i<j$,
$\bover13\delta_{x_i}\le\|y-x_i\|\le\delta_{x_i}$ and
$\|y-x_j\|\le\delta_j$ then the cosine of the angle
$x_i\widehat{y}x_j$ is at most $\bover12(1+\epsilon)$.
(vii) Show that $\#(\{i:y\in B_{x_i}\})\le m+1$.

Hence show that if $\Cal I$ is any family of non-trivial closed balls
such that every point of $A$ is the centre of some member of $\Cal I$,
then there is a countable $\Cal I_0\subseteq\Cal I$, covering $A$, such
that no point of $\BbbR^r$ belongs to more than $3^r$ members of
$\Cal I_0$.
%472B

\spheader 472Yc Use 472Yb to prove an alternative version of 472B, but
with the constant $9^r+1$ in place of $5^r$.
%472Yb 472B

\spheader 472Yd\dvAnew{2010}
Let $A\subseteq\BbbR^r$ be a bounded set, and $\Cal I$ a
family of non-trivial closed balls in $\BbbR^r$ such that whenever $x\in A$
and $\epsilon>0$ there is a ball $B(y,\delta)\in\Cal I$ such that
$\|x-y\|\le\epsilon\delta$.   Show that there is a family
$\langle\Cal I_k\rangle_{k<5^r}$ of
subsets of $\Cal I$ such that each $\Cal I_k$ is disjoint and
$\bigcup_{k<5^r}\Cal I_k$ covers $A$.
%472B  query out of order

\spheader 472Ye\dvAnew{2010} Give an example of a strictly positive
Radon probability measure $\mu$ on a
compact metric space $(X,\rho)$ for which there is a Borel set
$E\subseteq X$ such that

\Centerline{$\liminf_{\delta\downarrow 0}
  \Bover{\mu(E\cap B(x,\delta))}{\mu B(x,\delta)}=0$,
\quad$\limsup_{\delta\downarrow 0}
  \Bover{\mu(E\cap B(x,\delta))}{\mu B(x,\delta)}=1$}

\noindent for every $x\in X$.
%472D mt47bits

\spheader 472Yf Let $\lambda$ be a Radon measure on $\BbbR^r$, and $f$ a
$\lambda$-integrable real-valued function.   Show that
$\sup_{\delta>0}\Bover1{\lambda B(x,\delta)}
  \int_{B(x,\delta)}|f|d\lambda$  is defined and finite for
$\lambda$-almost every $x\in\BbbR^r$.
%472F

\spheader 472Yg Let $\lambda$, $\lambda'$ be Radon measures on
$\Bbb R^r$.   (i) Show that $g(x)=\lim_{\delta\downarrow 0}
\Bover{\lambda'B(x,\delta)}{\lambda B(x,\delta)}$ is defined in
$\Bbb R$ for $\lambda$-almost every $x$.   (ii) Setting
$\lambda_0=\sup_{n\in\Bbb N}\lambda'\wedge n\lambda$ in the cone of Radon
measures on $\Bbb R^r$ (437Yi),
show that $g$ is a Radon-Nikod\'ym derivative of $\lambda_0$ with respect
to $\lambda$.
\Hint{show that if $\lambda$ and $\lambda'$ are mutually singular then
$g=0\,\,\lambda$-a.e.}
%472E
}%end of exercises

\endnotes{
\Notesheader{472}
I gave primacy to the `weak' Vitali's theorem in 261B because I
think it is easier than the `strong' form in 472C, it uses the same
ideas as the original one-dimensional theorem in 221A, and it is
adequate for the needs of Volume 2.   Any proper study of general
measures on $\BbbR^r$, however, will depend on the ideas in 472A-472C.
You will see that in 472B, as in other forms of Vitali's theorem, there
is a key step in which a sequence is chosen greedily.   This time we
must look much more carefully at the geometry of $\BbbR^r$ because we
can no longer rely on a measure to tell us what is happening.   (Though
you will observe that I still use the elementary properties of Euclidean
volume in the argument of 472A.)   Once we have reached 472C, however,
we are in a position to repeat all the arguments of
261C-261E %261C 261D 261E
in much greater generality (472D), and, as a bonus, can refine
261F (472Xb).   For more in this direction see {\smc Mattila 95} and
{\smc Federer 69}, \S2.8.

It is natural to ask whether the constant `$5^r$' in 472B is best
possible.   The argument of 472A is derived from {\smc Sullivan 94},
where a more thorough analysis is given.   It seems that even for $r=2$
the best constant is unknown.   (For $r=1$, the best constant is $2$;
see 472Ya.)   Note that even for finite families $\Cal I$ we should have
to find the colouring number of a graph (counting two balls as linked if
they intersect), so it may well be a truly difficult problem.   The
method in 472B amounts to using the greedy colouring algorithm after
ordering the balls by size, and one does not expect such approaches to
give exact colouring numbers.   Of
course the questions addressed here depend only on the existence of {\it
some} function of $r$ to do the job.

An alternative argument runs through a kind of pointwise version of 472A
(472Yb-472Yc).   It gives a worse constant but is attractive in other
ways.   For many of the applications of 472C, the result of 472Yb is
already sufficient.

The constant $2\bigl(\Bover{5^rp}{p-1}\bigr)^{1/p}$ in 472F makes no
pretence to be `best', or even `good'.   The only reason for giving a
formula at all is to emphasize the remarkable fact that it does not
depend on the measure $\lambda$.   The theorems of this section are
based on the metric geometry of Euclidean space, not on any special
properties of Lebesgue measure.   The constants {\it do} depend on the
dimension, so that even in Hilbert space (for instance) we cannot expect
any corresponding results.
}%end of notes

\discrpage


