\frfilename{mt312.tex}
\versiondate{29.5.07}
\copyrightdate{1994}

\def\chaptername{Boolean algebras}
\def\sectionname{Homomorphisms}
\def\IMPLY#1#2{{\bf (#1)$\Rightarrow$(#2)}}

\newsection{312}

I continue the theory of Boolean algebras with a section on subalgebras,
ideals and homomorphisms.   From now on, I will relegate Boolean rings
which are not algebras to the exercises;  I think there is no need to
set out descriptions of the (mostly trifling)
modifications necessary to deal
with the extra generality.   The first part of the section
(312A-312L) concerns the translation of the basic concepts of ring
theory into the language which I propose to use for Boolean algebras.
312M shows that the order relation on a Boolean algebra defines the
algebraic structure, and in 312N-312O I give a fundamental result on the
extension of homomorphisms.
I end the section with results relating the previous ideas to the Stone
representation of a Boolean algebra (312P-312T).

\leader{312A}{Subalgebras} Let $\frak A$ be a Boolean algebra.   I will
use the phrase {\bf subalgebra of} $\frak A$ to mean a subring of
$\frak A$ containing its multiplicative
identity\cmmnt{ $1=1_{\frak A}$}.

\leader{312B}{Proposition} Let $\frak A$ be a Boolean algebra, and
$\frak B$ a subset of $\frak A$.   Then the following are equiveridical, that is, if one is true so are the others:

(i) $\frak B$ is a subalgebra of $\frak A$;

(ii) $0\in\frak B$, $a\Bcup b\in\frak B$ for all $a$, $b\in\frak B$, and
$1\Bsetminus a\in\frak B$ for all $a\in\frak B$;

(iii) $\frak B\ne\emptyset$, $a\Bcap b\in\frak B$ for all $a$,
$b\in\frak B$, and $1\Bsetminus a\in\frak B$ for all $a\in\frak B$.

\proof{{\bf (i)$\Rightarrow$(iii)} If $\frak B$ is a subalgebra of
$\frak A$, and $a$,
$b\in\frak B$, then of course we shall have

\Centerline{$0$, $1\in\frak B$, so $\frak B\ne\emptyset$,}

\Centerline{$a\Bcap b\in\frak B$,
\quad $1\Bsetminus a=1\Bsymmdiff a\in\frak B$.}

\medskip

{\bf (iii)$\Rightarrow$(ii)} If (iii) is true, then there is some
$b_0\in\frak B$;  now $1\Bsetminus b_0\in\frak B$, so

\Centerline{$0=b_0\cap(1\Bsetminus b_0)\in\frak B$.}

\noindent If $a$, $b\in\frak B$, then

\Centerline{$a\Bcup b=1\Bsetminus((1\Bsetminus a)\Bcap(1\Bsetminus b))
\in\frak B$.}

\noindent So (ii) is true.

\medskip

{\bf (ii)$\Rightarrow$(i)} If (ii) is true, then for any $a$,
$b\in\frak B$,

\Centerline{$a\Bcap b=1\Bsetminus((1\Bsetminus a)\Bcup(1\Bsetminus
b))\in\frak B$,}

\Centerline{$a\Bsymmdiff b=(a\Bcap(1\Bsetminus
b))\Bcup(b\Bcap(1\Bsetminus a))\in\frak B$,}

\noindent so (because also $0\in\frak B$) $\frak B$ is a subring of
$\frak A$, and

\Centerline{$1=1\Bsetminus 0\in\frak B$,}

\noindent so $\frak B$ is a subalgebra.
}%end of proof of 312B

\cmmnt{\medskip

\noindent{\bf Remark} Thus an algebra of subsets of a set $X$, as
defined in 136E or 311Bb, is just a subalgebra of the Boolean algebra
$\Cal PX$.
}%end of comment

\leader{312C}{Ideals in Boolean algebras:  Proposition} If $\frak A$ is
a Boolean algebra, a set $I\subseteq\frak A$ is an ideal of $\frak A$
iff $0\in I$, $a\Bcup b\in I$ for all $a$, $b\in I$, and $a\in I$ whenever $b\in I$ and $a\Bsubseteq b$.

\proof{{\bf (a)} Suppose that $I$ is an ideal.   Then of course $0\in
I$.   If $a$, $b\in I$ then $a\Bcap b\in I$ so $a\Bcup b=(a\Bsymmdiff
b)\Bsymmdiff(a\Bcap b)\in I$.   If
$b\in I$ and $a\Bsubseteq b$ then $a=a\Bcap b\in I$.

\medskip

{\bf (b)} Now suppose that $I$ satisfies the conditions proposed.   If
$a$, $b\in I$ then

\Centerline{$a\Bsymmdiff b\Bsubseteq a\Bcup b\in I$}

\noindent so $a\Bsymmdiff b\in I$, while of course $-a=a\in I$,  and
also $0\in
I$, by hypothesis;  thus $I$ is a subgroup of
$(\frak A,\hskip-.2em\Bsymmdiff\hskip-.2em)$.
Finally,
if $a\in I$ and $b\in\frak A$ then

\Centerline{$a\Bcap b\Bsubseteq a\in I$,}

\noindent so $b\Bcap a=a\Bcap b\in I$;  thus $I$ is an ideal.
}%end of proof of 312C

\cmmnt{
\leader{312D}{Principal ideals} Of course, while an ideal $I$
in a Boolean algebra $\frak A$ is necessarily a subring, it is not as a
rule a subalgebra, except in the special case $I=\frak A$.   But if we say that a {\bf principal ideal} of $\frak A$ is the ideal $\frak A_a$
generated by a single element $a$ of $\frak A$, we have a special
phenomenon.
}

\leader{312E}{Proposition} Let $\frak A$ be a Boolean algebra, and $a$
any element of $\frak A$.   Then the principal ideal $\frak A_a$ of
$\frak A$ generated by $a$ is just $\{b:b\in\frak A,\,b\Bsubseteq a\}$,
and (with the inherited operations
$\Bcap\restrp\frak A_a\times\frak A_a$,
$\Bsymmdiff\restrp\frak A_a\times\frak A_a$) is a Boolean algebra
in its own right, with multiplicative identity $a$.

\proof{$b\Bsubseteq a$ iff $b\Bcap a=b$, so that

\Centerline{$\frak A_a=\{b:b\Bsubseteq a\}=\{b\Bcap a:b\in\frak A\}$}

\noindent is an ideal of $\frak A$, and of course it is the smallest ideal of $\frak A$ containing $a$.   Being an ideal, it is a subring;  the idempotent
relation $b\Bcap b=b$ is inherited from $\frak A$, so it is a Boolean
ring;  and $a$ is plainly its multiplicative identity.
}%end of proof of 312E

\leader{312F}{Boolean homomorphisms} Now suppose that $\frak A$ and
$\frak B$ are two Boolean algebras.   I will use the phrase {\bf Boolean
homomorphism} to mean a function $\pi:\frak A\to\frak B$ which is a
ring homomorphism\cmmnt{ (that is,
$\pi(a\Bsymmdiff b)=\pi a\Bsymmdiff\pi b$,
$\pi(a\Bcap b)=\pi a\Bcap\pi b$ for all $a$, $b\in\frak A$)} and is
uniferent, that is, $\pi(1_{\frak A})=1_{\frak B}$.

\leader{312G}{Proposition} Let $\frak A$, $\frak B$ and $\frak C$ be
Boolean algebras.

(a) If $\pi:\frak A\to\frak B$ is a Boolean homomorphism, then
$\pi[\frak A]$ is a subalgebra of $\frak B$.

(b) If $\pi:\frak A\to\frak B$ and $\theta:\frak B\to\frak C$ are
Boolean homomorphisms, then $\theta\pi:\frak A\to\frak C$ is a Boolean
homomorphism.

(c) If $\pi:\frak A\to\frak B$ is a bijective Boolean homomorphism, then
$\pi^{-1}:\frak B\to\frak A$ is a Boolean homomorphism.

\proof{ These are all immediate consequences of the corresponding
results for ring homomorphisms (3A2D).
}%end of proof of 312G

\leader{312H}{Proposition} Let $\frak A$ and $\frak B$ be Boolean
algebras, and $\pi:\frak A\to\frak B$ a function.   Then the following
are equiveridical:

(i) $\pi$ is a Boolean homomorphism;

(ii) $\pi(a\Bcap b)=\pi a\Bcap\pi b$ and 
$\pi(1_{\frak A}\Bsetminus a)=1_{\frak B}\Bsetminus\pi a$ for all 
$a$, $b\in\frak A$;

(iii) $\pi(a\Bcup b)=\pi a\Bcup\pi b$ and 
$\pi(1_{\frak A}\Bsetminus a)=1_{\frak B}\Bsetminus\pi a$ 
for all $a$, $b\in\frak A$;

(iv) $\pi(a\Bcup b)=\pi a\Bcup\pi b$ and $\pi a\Bcap\pi b=0_{\frak B}$
whenever $a$, $b\in\frak A$ and $a\Bcap b=0_{\frak A}$, and
$\pi(1_{\frak A})=1_{\frak B}$.

\proof{\IMPLY{i}{iv} If $\pi$ is a Boolean homomorphism then of course
$\pi(1_{\frak A})=1_{\frak B}$;  also, given that $a\Bcap b=0$ in $\frak
A$,

\Centerline{$\pi a\Bcap\pi b=\pi(a\Bcap b)=\pi(0_{\frak A})=0_{\frak
B}$,}

\Centerline{$\pi(a\Bcup b)=\pi(a\Bsymmdiff b)=\pi a\Bsymmdiff\pi b=\pi
a\Bcup\pi b$.}

\medskip

\IMPLY{iv}{iii} Assume (iv), and take $a$, $b\in\frak A$.   Then

\Centerline{$\pi a=\pi(a\Bcap b)\Bcup\pi(a\Bsetminus b)$,
\quad$\pi b=\pi(a\Bcap b)\Bcup\pi(b\Bsetminus a)$,}

\noindent so

\Centerline{$\pi(a\Bcup b)
=\pi a\Bcup\pi(b\Bsetminus a)
=\pi(a\Bcap b)\Bcup\pi(a\Bsetminus b)\Bcup\pi(b\Bsetminus a)
=\pi a\Bcup\pi b$.}

\noindent Taking $b=1\Bsetminus a$, we must have

\Centerline{$1_{\frak B}=\pi(1_{\frak A})=\pi a\Bcup\pi(1_{\frak
A}\Bsetminus a)$,
\quad$0_{\frak B}=\pi a\Bcap\pi(1_{\frak A}\Bsetminus a)$,}

\noindent so $\pi(1_{\frak A}\setminus a)=1_{\frak B}\Bsetminus\pi a$.
Thus (iii) is true.

\medskip

\IMPLY{iii}{ii} If (iii) is true and $a$, $b\in\frak A$, then

$$\eqalign{\pi(a\Bcap b)
&=\pi(1_{\frak A}\Bsetminus((1_{\frak A}\Bsetminus a)
  \Bcup(1_{\frak A}\Bsetminus b)))\cr
&=1_{\frak B}\Bsetminus((1_{\frak B}\Bsetminus\pi a)
  \Bcup(1_{\frak B}\Bsetminus\pi b)))
=\pi a\Bcap\pi b.\cr}$$

\noindent So (ii) is true.

\medskip

\IMPLY{ii}{i} If (ii) is true, then

$$\eqalign{\pi(a\Bsymmdiff b)
&=\pi((1_{\frak A}
    \Bsetminus((1_{\frak A}\Bsetminus a)\Bcap(1_{\frak A}\Bsetminus b)))
  \Bcap(1_{\frak A}\Bsetminus(a\Bcap b)))\cr
&=(1_{\frak B}\Bsetminus((1_{\frak B}\Bsetminus\pi a)\Bcap(1_{\frak
B}\Bsetminus\pi b))\Bcap(1_{\frak B}\Bsetminus(\pi a\Bcap\pi b)))
=\pi a\Bsymmdiff\pi b\cr}$$

\noindent for all $a$, $b\in\frak A$, so $\pi$ is a ring homomorphism;
and now

\Centerline{$\pi(1_{\frak A})=\pi(1_{\frak A}\Bsetminus 0_{\frak A})
=1_{\frak B}\Bsetminus\pi(0_{\frak A})
=1_{\frak B}\Bsetminus 0_{\frak B}
=1_{\frak B}$,}

\noindent so that $\pi$ is a Boolean homomorphism.
}%end of proof of 312H

\leader{312I}{Proposition} If $\frak A$, $\frak B$ are Boolean algebras
and $\pi:\frak A\to\frak B$ is a Boolean homomorphism, then
$\pi a\Bsubseteq\pi b$ whenever $a\Bsubseteq b$ in $\frak A$.

\proof{

\Centerline{$a\Bsubseteq b\Longrightarrow a\Bcap b=a
\Longrightarrow \pi a\Bcap\pi b=\pi a
\Longrightarrow \pi a\Bsubseteq\pi b$.}
}%end of proof of 312I

\leader{312J}{Proposition} Let $\frak A$ be a Boolean algebra, and $a$
any member of $\frak A$.   Then the map $b\mapsto a\cap b$ is a
surjective Boolean homomorphism from
$\frak A$ onto the principal ideal $\frak A_a$ generated by $a$.

\proof{ This is an elementary verification.
}%end of proof of 312J

\leader{*312K}{Fixed-point subalgebras}\dvAnew{2009}\cmmnt{ For 
future reference
I introduce the following idea.}   If $\frak A$ is a Boolean algebra and
$\pi:\frak A\to\frak A$ is a Boolean homomorphism, then
$\{a:a\in\frak A$, $\pi a=a\}$ is a subalgebra of
$\frak A$\prooflet{ (put the definitions 312A and 312F together)};
I will call it the {\bf fixed-point subalgebra} of $\pi$.

\leader{312L}{Quotient algebras:  Proposition} Let $\frak A$ be a
Boolean algebra and $I$ an ideal of $\frak A$.   Then the quotient ring
$\frak A/I$\cmmnt{ (3A2F)} is a Boolean algebra, and the canonical map
$a\mapsto a^{\ssbullet}:\frak A\to\frak A/I$ is a Boolean homomorphism,
so that

\Centerline{$(a\Bsymmdiff b)^{\ssbullet}=a^{\ssbullet}\Bsymmdiff
b^{\ssbullet}$,
\quad$(a\Bcup b)^{\ssbullet}=a^{\ssbullet}\Bcup b^{\ssbullet}$,
\quad$(a\Bcap b)^{\ssbullet}=a^{\ssbullet}\Bcap b^{\ssbullet}$,
\quad$(a\Bsetminus b)^{\ssbullet}=a^{\ssbullet}\Bsetminus
b^{\ssbullet}$}

\noindent for all $a$, $b\in\frak A$.

(b) The order relation on $\frak A/I$ is defined by the formula

\Centerline{$a^{\ssbullet}\Bsubseteq b^{\ssbullet}\iff a\Bsetminus b\in
I$.}

\noindent For any $a\in\frak A$,

\Centerline{$\{u:u\Bsubseteq a^{\ssbullet}\}
=\{b^{\ssbullet}:b\Bsubseteq a\}$.}

\proof{{\bf (a)} Of course the map
$a\mapsto a^{\ssbullet}=\{a\Bsymmdiff b:b\in I\}$
is a ring homomorphism (3A2Fd).   Because

\Centerline{$(a^{\ssbullet})^2=(a^2)^{\ssbullet}=a^{\ssbullet}$}

\noindent for every $a\in\frak A$, $\frak A/I$ is a Boolean ring;
because $1^{\ssbullet}$ is a multiplicative identity, it is a Boolean
algebra, and $a\mapsto a^{\ssbullet}$ is a Boolean homomorphism.   The
formulae given are now elementary.

\medskip

{\bf (b)} We have

\Centerline{$a^{\ssbullet}\Bsubseteq b^{\ssbullet}
\iff a^{\ssbullet}\Bsetminus b^{\ssbullet}=0
\iff a\Bsetminus b\in I$.}

\noindent Now

\Centerline{$\{u:u\Bsubseteq a^{\ssbullet}\}
=\{u\Bcap a^{\ssbullet}:u\in\frak A/I\}
=\{(b\Bcap a)^{\ssbullet}:b\in\frak A\}
=\{b^{\ssbullet}:b\Bsubseteq a\}$.}
}%end of proof of 312L

\leader{312M}{}\cmmnt{ The above results are both repetitive and
nearly trivial.   Now I come to something with a little more meat to it.

\medskip

\noindent}{\bf Proposition} If $\frak A$ and $\frak B$ are Boolean
algebras and
$\pi:\frak A\to\frak B$ is a bijection such that $\pi a\Bsubseteq\pi b$
whenever $a\Bsubseteq b$, then $\pi$ is a Boolean algebra isomorphism.

\proof{{\bf (a)} Because $\pi$ is surjective, there must be $c_0$,
$c_1\in\frak A$ such that $\pi c_0=0_{\frak B}$, $\pi c_1=1_{\frak B}$;
now $\pi(0_{\frak A})\Bsubseteq\pi c_0$ and
$\pi c_1\Bsubseteq \pi(1_{\frak  A})$, so we must have
$\pi(0_{\frak A})=0_{\frak B}$ and $\pi(1_{\frak A})=1_{\frak B}$.

\medskip

{\bf (b)} If $a\in\frak A$, then $\pi a\Bcup\pi(1_{\frak A}\Bsetminus
a)=1_{\frak B}$.   \Prf\  There is a $c\in\frak A$ such that $\pi
c=1_{\frak B}\Bsetminus(\pi a\Bcup\pi(1_{\frak A}\Bsetminus a))$.   Now

\Centerline{$\pi(c\Bcap a)\Bsubseteq\pi c\Bcap\pi a=0_{\frak B}$,
\quad $\pi(c\Bsetminus a)\Bsubseteq\pi c\Bcap\pi(1_{\frak A}\Bsetminus
a)=0_{\frak B}$;}

\noindent as $\pi$ is injective, $c\Bcap a=c\Bsetminus a=0_{\frak A}$
and $c=0_{\frak A}$, $\pi c=0_{\frak B}$, $\pi a\Bcup\pi(1_{\frak
A}\Bsetminus a)=1_{\frak B}$.   \Qed

\medskip

{\bf (c)} If $a\in\frak A$, then $\pi a\Bcap\pi(1_{\frak A}\Bsetminus
a)=0_{\frak B}$.   \Prf\  It may be clear to you that this is just a
dual form of (b).   If not, I repeat the argument in the form now
appropriate.   There is a $c\in\frak A$ such that $\pi c=1_{\frak
B}\Bsetminus(\pi a\Bcap\pi(1_{\frak A}\Bsetminus a))$.   Now

\Centerline{$\pi(c\Bcup a)\Bsupseteq\pi c\Bcup\pi a=1_{\frak B}$,
\quad $\pi(c\Bcup(1_{\frak A}\Bsetminus a))
\Bsupseteq\pi c\Bcup\pi(1_{\frak A}\Bsetminus a)=1_{\frak B}$;}

\noindent as $\pi$ is injective, $c\Bcup a=c\Bcup(1_{\frak A}\Bsetminus
a)=1_{\frak A}$ and $c=1_{\frak A}$, $\pi c=1_{\frak B}$, $\pi
a\Bcap\pi(1_{\frak A}\Bsetminus a)=0_{\frak B}$.   \Qed

\medskip

{\bf (d)} Putting (b) and (c) together, we have $\pi(1_{\frak
A}\Bsetminus a)=1_{\frak B}\Bsetminus \pi a$ for every $a\in\frak A$.
Now $\pi(a\Bcup b)=\pi a\Bcup\pi b$ for every $a$, $b\in\frak A$.
\Prf\ Surely $\pi a\Bcup\pi b\Bsubseteq\pi(a\Bcup b)$.   Let $c\in\frak
A$ be such that $\pi c=\pi(a\Bcup b)\Bsetminus(\pi a\Bcup\pi b)$.   Then

\Centerline{$\pi(c\Bcap a)\Bsubseteq\pi c\Bcap\pi a=0_{\frak B}$,
\quad $\pi(c\Bcap b)\Bsubseteq\pi c\Bcap\pi b=0_{\frak B}$,}

\noindent so $c\Bcap a=c\Bcap b=0$ and $c\Bsubseteq 1_{\frak
A}\Bsetminus(a\Bcup b)$;  accordingly

\Centerline{$\pi c\Bsubseteq \pi(1_{\frak A}\Bsetminus(a\Bcup
b))=1_{\frak B}\Bsetminus\pi(a\Bcup b)$;}

\noindent as also $\pi c\Bsubseteq\pi(a\Bcup b)$, $\pi c=0_{\frak B}$
and $\pi(a\Bcup b)=\pi a\Bcup\pi b$.   \Qed

\medskip

{\bf (e)} So the conditions of 312H(iii) are satisfied and $\pi$ is a
Boolean homomorphism;  being bijective, it is an isomorphism.
}%end of proof of 312M

\leader{312N}{}\cmmnt{ I turn next to a fundamental lemma on the
construction of
homomorphisms.   We need to start with a proper description of a certain
type of subalgebra.

\medskip

\noindent}{\bf Lemma} Let $\frak A$ be a Boolean algebra, and $\frak
A_0$ a subalgebra of $\frak A$;  let $c$ be any member of $\frak A$.
Then

\Centerline{$\frak A_1
=\{(a\Bcap c)\Bcup(b\Bsetminus c):a,\,b\in\frak A_0\}$}

\noindent is a subalgebra of $\frak A$;  it is the subalgebra of
$\frak A$ generated by $\frak A_0\cup\{c\}$.

\proof{ We have to check the following:

\Centerline{$a=(a\Bcap c)\Bcup(a\Bsetminus c)\in\frak A_1$}

\noindent for every $a\in\frak A_0$, so $\frak A_0\subseteq\frak A_1$;
in particular, $0\in\frak A_1$.

\Centerline{$1\Bsetminus((a\Bcap c)\Bcup(b\Bsetminus c))
=((1\Bsetminus a)\Bcap c)\Bcup((1\Bsetminus b)\Bsetminus c)
\in\frak A_1$}

\noindent for all $a$, $b\in\frak A_0$, so $1\Bsetminus d\in\frak A_1$
for every $d\in\frak A_1$.

\Centerline{$(a\Bcap c)\Bcup(b\Bsetminus c)
\Bcup(a'\Bcap c)\Bcup(b'\Bsetminus c)
=((a\Bcup a')\Bcap c)\Bcup((b\Bcup b')\Bsetminus c)\in\frak A_1$}

\noindent for all $a$, $b$, $a'$, $b'\in\frak A_0$, so
$d\cup d'\in\frak A_1$ for all $d$, $d'\in\frak A_1$.   Thus $\frak A_1$
is a subalgebra of $\frak A$ (using 312B).

\Centerline{$c=(1\Bcap c)\Bcup(0\Bsetminus c)\in\frak A_1$,}

\noindent so $\frak A_1$ includes $\frak A_0\cup\{c\}$;  and finally
it is clear that any subalgebra of $\frak A$ including
$\frak A_0\cup\{c\}$, being closed under $\Bcap$, $\Bcup$ and
complementation,
must include $\frak A_1$, so that $\frak A_1$ is the subalgebra of
$\frak A$ generated by $\frak A_0\cup\{c\}$.
}%end of proof of 312N

\leader{312O}{Lemma}\dvAformerly{3{}12N} 
Let $\frak A$ and $\frak B$ be Boolean algebras,
$\frak A_0$ a subalgebra of $\frak A$, $\pi:\frak A_0\to\frak B$ a
Boolean homomorphism, and $c\in\frak A$.   If $v\in\frak B$ is such that
$\pi a\Bsubseteq v\Bsubseteq\pi b$ whenever $a$, $b\in\frak A_0$ and
$a\Bsubseteq c\Bsubseteq b$, then there is a unique Boolean homomorphism
$\pi_1$ from the subalgebra $\frak A_1$ of $\frak A$ generated by
$\frak A_0\cup\{c\}$ such that $\pi_1$ extends $\pi$ and $\pi_1c=v$.

\proof{{\bf (a)} The basic fact we need to know is that if $a$, $a'$,
$b$, $b'\in\frak A_0$ and

\Centerline{$(a\Bcap c)\Bcup(b\Bsetminus c)=d=(a'\Bcap
c)\Bcup(b'\Bsetminus c)$,}

\noindent  then

\Centerline{$(\pi a\Bcap v)\Bcup(\pi b\Bsetminus v)=(\pi a'\Bcap
v)\Bcup(\pi b'\Bsetminus v)$.}

\noindent   \Prf\ We have

\Centerline{$a\Bcap c=d\Bcap c=a'\Bcap c$.}

\noindent Accordingly $(a\Bsymmdiff a')\Bcap c=0$ and $c\Bsubseteq
1\Bsetminus(a\Bsymmdiff a')$.   Consequently (since $a\Bsymmdiff a'$
surely belongs to $\frak A_0$)

\Centerline{$v\Bsubseteq\pi(1\Bsetminus(a\Bsymmdiff a'))=1\Bsetminus(\pi
a\Bsymmdiff\pi a')$,}

\noindent and

\Centerline{$\pi a\Bcap v=\pi a'\Bcap v$.}

\noindent Similarly,

\Centerline{$b\Bsetminus c=d\Bsetminus c=b'\Bsetminus c$,}

\noindent so

\Centerline{$(b\Bsymmdiff b')\Bsetminus c=0$,
\quad $b\Bsymmdiff b'\Bsubseteq c$,
\quad$\pi b\Bsymmdiff\pi b'=\pi(b\Bsymmdiff b')\Bsubseteq v$}

\noindent and

\Centerline{$\pi b\Bsetminus v=\pi b'\Bsetminus v$.}

\noindent Putting these together, we have the result.   \Qed

\medskip

{\bf (b)} Consequently, we have a function  $\pi_1$ defined by writing

\Centerline{$\pi_1((a\Bcap c)\Bcup(b\Bsetminus c))
=(\pi a\Bcap v)\Bcup(\pi b\Bsetminus v)$}

\noindent for all $a$, $b\in\frak A_0$;  and 312N tells us that the
domain of $\pi_1$ is just $\frak A_1$.   Now $\pi_1$ is a Boolean
homomorphism.  \Prf\ This amounts to running through the proof of 312N
again.

\medskip

\quad{\bf (i)} If $a$, $b\in\frak A_0$, then

$$\eqalign{\pi_1(1\Bsetminus((a\Bcap c)\Bcup(b\Bsetminus c)))
&=\pi_1(((1\Bsetminus a)\Bcap c)\Bcup((1\Bsetminus b)\Bsetminus c))\cr
&=(\pi(1\Bsetminus a)\Bcap v)\Bcup(\pi(1\Bsetminus b)\Bsetminus v)\cr
&=((1\Bsetminus\pi a)\Bcap v)\Bcup((1\Bsetminus\pi b)\Bsetminus v)\cr
&=1\Bsetminus((\pi a\Bcap v)\Bcup(\pi b\Bsetminus v))
=1\Bsetminus\pi_1((a\Bcap c)\Bcup(b\Bsetminus c)).\cr}$$

\noindent So $\pi_1(1\Bsetminus d)=1\Bsetminus\pi_1d$ for every
$d\in\frak A_1$.

\medskip

\quad{\bf (ii)} If $a$, $b$, $a'$, $b'\in\frak A_0$, then

$$\eqalign{\pi_1((a\Bcap c)\Bcup(b\Bsetminus c)
    \Bcup(a'\Bcap c)\Bcup(b'\Bsetminus c))
&=\pi_1(((a\Bcup a')\Bcap c)\Bcup((b\Bcup b')\Bsetminus c))\cr
&=(\pi(a\Bcup a')\Bcap v)\Bcup(\pi(b\Bcup b')\Bsetminus v)\cr
&=((\pi a\Bcup \pi a')\Bcap v)\Bcup((\pi b\Bcup\pi b')\Bsetminus v)\cr
&=(\pi a\Bcap v)\Bcup(\pi b\Bsetminus v)
    \Bcup(\pi a'\Bcap v)\Bcup(\pi b'\Bsetminus v)\cr
&=\pi_1((a\Bcap c)\Bcup(b\Bsetminus c))
    \Bcup\pi_1((a'\Bcap c)\Bcup(b'\Bsetminus c)).\cr}$$

\noindent So $\pi_1(d\Bcup d')=\pi_1d\Bcup\pi_1d'$ for all $d$,
$d'\in\frak A_1$.

By 312H(iii), $\pi_1$ is a Boolean homomorphism.   \Qed

\medskip

{\bf (c)} If $a\in\frak A_0$, then

\Centerline{$\pi_1a=\pi_1((a\Bcap c)\Bcup(a\Bsetminus c))
=(\pi a\Bcap v)\Bcup(\pi a\Bsetminus v)
=\pi a$,}

\noindent so $\pi_1$ extends $\pi$.   As for the action of $\pi_1$ on
$c$,

\Centerline{$\pi_1c=\pi_1((1\Bcap c)\Bcup(0\Bsetminus c))
=(\pi 1\Bcap v)\Bcup(\pi 0\Bsetminus v)
=(1\Bcap v)\Bcup(0\Bsetminus v)=v$,}

\noindent as required.

\medskip

{\bf (d)} Finally, the formula of (b) is the only possible definition
for any Boolean homomorphism from $\frak A_1$ to $\frak B$ which will
extend $\pi$ and take $c$ to $v$.   So $\pi_1$ is unique.
}%end of proof of 312O

\leader{312P}{Homomorphisms and
Stone \dvrocolon{spaces}}\cmmnt{ Because the Stone space $Z$
of a Boolean algebra $\frak A$ (311E) can be constructed explicitly from
the algebraic structure of $\frak A$, it must in principle be possible
to describe any feature of the Boolean structure of $\frak A$ in terms
of $Z$.   In the next few paragraphs I work through the most important
identifications.

\woddheader{312P}{6}{2}{2}{42pt}

\noindent}{\bf Proposition} Let $\frak A$ be a Boolean algebra, and $Z$
its Stone space;  write $\widehat a\subseteq Z$ for the open-and-closed
set corresponding to $a\in\frak A$.   Then there is a one-to-one
correspondence between ideals $I$ of $\frak A$ and open sets
$G\subseteq Z$, given by the formulae

\Centerline{$G=\bigcup_{a\in I}\widehat a$,
\quad$I=\{a:\widehat a\subseteq G\}$.}

\proof{{\bf (a)} For any ideal $I\normalsubgroup\frak A$, set
$H(I)=\bigcup_{a\in I}\widehat a$;  then $H(I)$ is a union of open
subsets of $Z$, so is open.   For any open set $G\subseteq Z$, set
$J(G)=\{a:a\in\frak A,\,\widehat a\subseteq G\}$;  then $J(G)$ satisfies
the conditions of 312C, so is an ideal of $\frak A$.

\medskip

{\bf (b)} If $I\normalsubgroup \frak A$, then $J(H(I))=I$.   \Prf\ (i)
If $a\in I$, then $\widehat a\subseteq H(I)$ so $a\in J(H(I))$.   (ii)
If $a\in J(H(I))$, then
$\widehat a\subseteq H(I)=\bigcup_{b\in I}\widehat b$.   Because
$\widehat a$ is compact and all the $\widehat b$ are open,
there must be finitely many $b_0,\ldots,b_n\in I$ such that
$\widehat a\subseteq \widehat b_0\cup\ldots\cup\widehat b_n$.   But now
$a\Bsubseteq b_0\Bcup\ldots\Bcup b_n\in I$, so $a\in I$.   \Qed

\medskip

{\bf (c)} If $G\subseteq Z$ is open, then $H(J(G))=G$.   \Prf\ (i) If
$z\in G$, then (because $\{\widehat a:a\in\frak A\}$ is a base for the
topology of $Z$) there is an $a\in\frak A$ such that
$z\in\widehat a\subseteq G$;  now $a\in J(G)$ and $z\in H(J(G))$.   (ii)
If $z\in H(J(G))$, there is an $a\in J(G)$ such that $z\in\widehat a$;
now $\widehat a\subseteq G$, so $z\in G$.   \Qed

This shows that the maps $G\mapsto J(G)$, $I\mapsto H(I)$ are two halves
of a one-to-one correspondence, as required.
}%end of proof of 312P

\leader{312Q}{Theorem}\dvAformerly{3{}12P}
Let $\frak A$, $\frak B$ be Boolean algebras,
with Stone spaces $Z$, $W$;  write $\widehat a\subseteq Z$,
$\widehat b\subseteq W$ for the open-and-closed sets corresponding to
$a\in\frak A$, $b\in\frak B$.   Then we have a
one-to-one correspondence between Boolean homomorphisms
$\pi:\frak A\to\frak B$ and continuous functions $\phi:W\to Z$, given by
the formula

\Centerline{$\pi a=b\iff\phi^{-1}[\widehat a]=\widehat b$,}

\noindent that is, $\phi^{-1}[\widehat{a}]=\widehat{\pi a}$.

\proof{{\bf (a)} Recall that I have constructed $Z$ and $W$ as the sets of
Boolean homomorphisms from $\frak A$ and $\frak B$ to $\Bbb Z_2$ (311F).
So if $\pi:\frak A\to\frak B$ is any Boolean homomorphism, and $w\in W$,
$\psi_{\pi}(w)=w\pi$ is a Boolean homomorphism from $\frak A$ to
$\Bbb Z_2$ (312Gb), and belongs to $Z$.   Now
$\psi_{\pi}^{-1}[\widehat a]=\widehat{\pi a}$ for every $a\in\frak A$.
\Prf\

\Centerline{$\psi_{\pi}^{-1}[\widehat a]
=\{w:\psi_{\pi}(w)\in\widehat a\}
=\{w:w\pi\in\widehat a\}
=\{w:w\pi(a)=1\}
=\{w:w\in\widehat{\pi a}\}$. \Qed}

\noindent Consequently $\psi_{\pi}$ is continuous.   \Prf\ Let $G$ be
any open subset of $Z$.   Then
$G=\bigcup\{\widehat a:\widehat a\subseteq G\}$, so

\Centerline{$\psi_{\pi}^{-1}[G]
=\bigcup\{\psi_{\pi}^{-1}[\widehat a]:\widehat a\subseteq G\}
=\bigcup\{\widehat{\pi a}:\widehat a\subseteq G\}$}

\noindent is open.   As $G$ is arbitrary, $\psi_{\pi}$ is continuous.
\Qed

\medskip

{\bf (b)} If $\phi:W\to Z$ is continuous, then for any $a\in\frak A$ the
set $\phi^{-1}[\widehat a]$ must be an open-and-closed set in $W$;
consequently there is a unique member of $\frak B$, call it
$\theta_{\phi}a$, such that
$\phi^{-1}[\widehat a]=\widehat{\theta_{\phi}a}$.   Observe that, for
any $w\in W$ and $a\in\frak A$,

\Centerline{$w(\theta_{\phi}a)=1
\iff w\in\widehat{\theta_{\phi}a}
\iff\phi(w)\in\widehat{a}
\iff(\phi(w))(a)=1$,}

\noindent so $\phi(w)=w\theta_{\phi}$.

Now $\theta_{\phi}$ is a Boolean homomorphism.   \Prf\ (i) If $a$,
$b\in\frak A$ then

\Centerline{$\theta_{\phi}(a\Bcup b)\sphat\hskip0.2em
=\phi^{-1}[(a\Bcup b)\sphat\hskip0.2em]
=\phi^{-1}[\widehat a\cup\widehat b]
=\phi^{-1}[\widehat a]\cup\phi^{-1}[\widehat b]
=\widehat{\theta_{\phi}a}\cup\widehat{\theta_{\phi}b}
=(\theta_{\phi}a\Bcup\theta_{\phi}b)\sphat$,}

\noindent so $\theta_{\phi}(a\Bcup b)
=\theta_{\phi}a\Bcup\theta_{\phi}b$.
(ii) If $a\in\frak A$, then

\Centerline{$\theta_{\phi}(1\Bsetminus a)\sphat\hskip0.2em
=\phi^{-1}[(1\Bsetminus a)\sphat\hskip0.2em]
=\phi^{-1}[Z\setminus\widehat a]
=W\setminus\phi^{-1}[\widehat a]
=W\setminus\widehat{\theta_{\phi}a}
=(1\Bsetminus\theta_{\phi}a)\sphat$,}

\noindent so $\theta_{\phi}(1\Bsetminus a)=1\Bsetminus\theta_{\phi}a$.
(iii) By 312H, $\theta_{\phi}$ is a Boolean homomorphism.   \Qed

\medskip

{\bf (c)} For any Boolean homomorphism $\pi:\frak A\to\frak B$,
$\pi=\theta_{\psi_{\pi}}$.   \Prf\ For $a\in\frak A$,

\Centerline{$(\theta_{\psi_{\pi}}a)\sphat\hskip0.2em
=\psi_{\pi}^{-1}[\widehat a]
=\widehat{\pi a}$,}

\noindent so $\theta_{\psi_{\pi}}a=a$.   \Qed

\medskip

{\bf (d)} For any continuous function $\phi:W\to Z$,
$\phi=\psi_{\theta_{\phi}}$.   \Prf\ For any $w\in W$,

\Centerline{$\psi_{\theta_{\phi}}(w)
=w\theta_{\phi}
=\phi(w)$.  \Qed}

\medskip

{\bf (e)} Thus $\pi\mapsto\psi_{\pi}$, $\phi\mapsto\theta_{\phi}$ are
the two halves of a one-to-one correspondence, as required.
}%end of proof of 312Q

\leader{312R}{Theorem}\dvAformerly{3{}12Q}
Let $\frak A$, $\frak B$, $\frak C$ be Boolean
algebras, with Stone spaces $Z$, $W$ and $V$.   Let
$\pi:\frak A\to\frak B$ and $\theta:\frak B\to\frak C$ be Boolean
homomorphisms, with
corresponding continuous functions $\phi:W\to Z$ and $\psi:V\to W$.
Then the Boolean homomorphism $\theta\pi:\frak A\to\frak C$ corresponds
to the continuous function $\phi\psi:V\to Z$.

\proof{ For any $a\in\frak A$,

\Centerline{$\widehat{\theta\pi a}
=(\theta(\pi a))\sphat\hskip0.2em
=\psi^{-1}[\widehat{\pi a}]
=\psi^{-1}[\phi^{-1}[\widehat a]]
=(\phi\psi)^{-1}[\widehat a]$.}
}%end of proof of 312R

\leader{312S}{Proposition} Let $\frak A$ and $\frak B$ be Boolean
algebras, with Stone spaces $Z$ and $W$, and $\pi:\frak A\to\frak B$ a
Boolean homomorphism, with associated continuous function $\phi:W\to Z$.
Then

(a) $\pi$ is injective iff $\phi$ is surjective;

(b) $\pi$ is surjective iff $\phi$ is injective.

\proof{{\bf (a)} If $a\in\frak A$, then

\Centerline{$\widehat a\cap\phi[W]=\emptyset
\iff\phi^{-1}[\widehat a]=\emptyset
\iff\widehat{\pi a}=\emptyset
\iff\pi a=0$.}

\noindent Now $W$ is compact, so $\phi[W]$ also is compact, therefore
closed, and

$$\eqalign{\phi\text{ is not surjective}
&\iff Z\setminus\phi[W]\ne\emptyset\cr
&\iff \text{ there is a non-zero }a\in\frak A\text{ such that }
   \widehat a\subseteq Z\setminus\phi[W]\cr
&\iff \text{ there is a non-zero }a\in\frak A\text{ such that }
   \pi a=0\cr
&\iff\pi\text{ is not injective}\cr}$$

\noindent (3A2Db).

\medskip

{\bf (b)(i)} If $\pi$ is surjective and $w$, $w'$ are distinct members
of $W$, then there is a $b\in\frak B$ such that $w\in\widehat b$ and
$w'\notin\widehat b$.   Now $b=\pi a$ for some $a\in\frak A$, so
$\phi(w)\in\widehat a$ and $\phi(w')\notin\widehat a$, and
$\phi(w)\ne\phi(w')$.   As $w$ and $w'$ are arbitrary, $\phi$ is
injective.

\medskip

\quad{\bf (ii)} If $\phi$ is injective and $b\in\frak B$, then
$K=\phi[\widehat b]$, $L=\phi[W\setminus\widehat b]$ are disjoint
compact subsets of $Z$.   Consider
$I=\{a:a\in\frak A,\,L\cap\widehat a=\emptyset\}$.   Then
$\bigcup_{a\in I}\widehat a=Z\setminus L\supseteq K$.   Because $K$ is
compact and every $\widehat a$ is open, there is a
finite family $a_0,\ldots,a_n\in I$ such that
$K\subseteq\widehat a_0\cup\ldots\cup\widehat a_n$.   Set
$a=a_0\Bcup\ldots\Bcup a_n$.   Then
$\widehat a=\widehat a_0\cup\ldots\cup\widehat a_n$ includes $K$ and is
disjoint from $L$.   So $\widehat{\pi a}=\phi^{-1}[\widehat a]$ includes
$\widehat b$ and is disjoint from $W\setminus\widehat b$;  that is,
$\widehat{\pi a}=\widehat b$ and $\pi a=b$.   As $b$ is arbitrary, $\pi$
is surjective.
}%end of proof of 312S

\leader{312T}{Principal ideals} If $\frak A$ is a Boolean algebra and
$a\in\frak A$, we have a natural surjective Boolean homomorphism
$b\mapsto b\Bcap a:\frak A\to\frak A_a$, the principal ideal generated
by $a$\cmmnt{ (312J)}.   Writing $Z$ for the Stone space of $\frak A$
and $Z_a$ for the
Stone space of $\frak A_a$, this homomorphism must correspond to an
injective continuous function $\phi:Z_a\to Z$\cmmnt{ (312Sb)}.
\cmmnt{Because $Z_a$ is
compact and $Z$ is Hausdorff,} $\phi$ must be a homeomorphism between
$Z_a$ and its image $\phi[Z_a]\subseteq Z$\cmmnt{ (3A3Dd)}.
\cmmnt{To identify $\phi[Z_a]$, note that it is compact, therefore
closed, and that

$$\eqalign{Z\setminus\phi[Z_a]
&=\bigcup\{\widehat b:b\in\frak A,\,\widehat b\cap\phi[Z_a]
  =\emptyset\}\cr
&=\bigcup\{\widehat b:\phi^{-1}[\widehat b]=\emptyset\}
=\bigcup\{\widehat b:b\Bcap a=0\}
=Z\setminus\widehat a,\cr}$$

\noindent so that} $\phi[Z_a]=\widehat a$.   \cmmnt{It
is therefore natural to
identify $Z_a$ with the open-and-closed set $\widehat a\subseteq Z$.}

\exercises{
\leader{312X}{Basic exercises (a)} Let $\frak A$ be a Boolean ring, and
$\frak B$ a subset of $\frak A$.   Show that $\frak B$ is a subring of
$\frak A$ iff $0\in\frak B$ and $a\Bcup b$, $a\Bsetminus b\in\frak B$
for all $a$, $b\in\frak B$.
%312B

\spheader 312Xb Let $\frak A$ be a Boolean algebra and $\frak B$ a
subset of $\frak A$.   Show that $\frak B$ is a subalgebra of $\frak A$
iff $1\in\frak B$ and $a\Bsetminus b\in\frak B$ for all $a$,
$b\in\frak B$.
%312B

\spheader 312Xc Let $\frak A$ be a Boolean algebra.   Suppose that
$I\subseteq A\subseteq\frak A$ are such that $1\in A$, $a\Bcap b\in I$
for all $a$, $b\in I$ and $a\Bsetminus b\in A$ whenever $a$, $b\in A$
and $b\subseteq a$.   Show that $A$ includes the subalgebra of $\frak A$
generated by $I$.   ({\it Hint\/}:  136Xf.)
%312B

\spheader 312Xd Show that if $\frak A$ is
a Boolean ring, a set $I\subseteq\frak A$ is an ideal of $\frak A$ iff
$0\in I$, $a\Bcup b\in I$ for all $a$, $b\in I$, and $a\in I$ whenever
$b\in I$ and $a\Bsubseteq b$.
%312C

\spheader 312Xe Let $\frak A$ and $\frak B$ be Boolean algebras, and
$\pi:\frak A\to\frak B$ a function such that
(i) $\pi(a)\Bsubseteq\pi(b)$ whenever $a\Bsubseteq b$
(ii) $\pi(a)\Bcap\pi(b)=0_{\frak B}$ whenever $a\Bcap b=0_{\frak A}$
(iii) $\pi(a)\Bcup\pi(b)\Bcup\pi(c)=1_{\frak B}$ whenever
$a\Bcup b\Bcup c=1_{\frak A}$.   Show that $\pi$ is a Boolean
homomorphism.
%312H

\spheader 312Xf Let $\frak A$ be a Boolean ring, and $a$ any member of
$\frak A$.   Show that the map $b\mapsto a\cap b$ is a ring homomorphism
from $\frak A$ onto the principal ideal $\frak A_a$ generated by $a$.
%312G

\spheader 312Xg Let $\frak A_1$ and $\frak A_2$ be Boolean rings, and
let $\frak B_1$, $\frak B_2$ be the Boolean algebras constructed from
them by the method of 311Xc.   Show that any ring homomorphism from
$\frak A_1$ to $\frak A_2$ has a unique extension to a Boolean
homomorphism from $\frak B_1$ to $\frak B_2$.
%312G

\spheader 312Xh Let $\frak A$ and $\frak B$ be Boolean rings,
$\frak A_0$ a subalgebra of $\frak A$, $\pi:\frak A_0\to\frak B$ a
ring homomorphism, and $c\in\frak A$.   Show that if $v\in\frak B$ is
such that $\pi a\Bsetminus v=\pi b\Bcap v=0$ whenever $a$, $b\in\frak
A_0$ and $a\Bsetminus c=b\Bcap c=0$, then there is a unique ring
homomorphism $\pi_1$
from the subring $\frak A_1$ of $\frak A$ generated by $\frak
A_0\cup\{c\}$ such that $\pi_1$ extends $\pi_0$ and $\pi_1c=v$.
%312M

\spheader 312Xi Let $\frak A$ be a Boolean ring, and $Z$ its Stone
space.   Show that there is a one-to-one correspondence between ideals
$I$ of $\frak A$ and open sets $G\subseteq Z$, given by the formulae
$G=\bigcup_{a\in I}\widehat a$, $I=\{a:\widehat a\subseteq G\}$.
%312P

\spheader 312Xj Let $\frak A$ be a Boolean algebra, and suppose that
$\frak A$ is the subalgebra of itself generated by $\frak A_0\cup\{c\}$,
where $\frak A_0$ is a subalgebra of $\frak A$ and $c\in\frak A$.   Let
$Z$ be the Stone space of $\frak A$ and $Z_0$ the Stone space of
$\frak A_0$.   Let $\psi:Z\to Z_0$ be the continuous surjection
corresponding to the embedding of $\frak A_0$ in $\frak A$.   Show that
$\psi\restrp\widehat c$ and $\psi\restr Z\setminus\widehat c$
are injective.

Now let $\frak B$ be another Boolean algebra, with Stone space $W$, and
$\pi:\frak A_0\to\frak B$ a Boolean homomorphism, with corresponding
function $\phi:W\to Z_0$.   Show that there is a continuous function
$\phi_1:W\to Z$ such that $\psi\phi_1=\phi$ iff there is an
open-and-closed set $V\subseteq W$ such that
$\phi[V]\subseteq\psi[\widehat c]$ and
$\phi[W\setminus V]\subseteq\psi[Z\setminus\widehat c]$.
%312Q

\spheader 312Xk Let $\frak A$ be a Boolean algebra, with Stone space
$Z$, and $I$ an ideal of $\frak A$, corresponding to an open set
$G\subseteq Z$.   Show that the Stone space of the quotient algebra
$\frak A/I$ may be identified with $Z\setminus G$.
%312Q

\leader{312Y}{Further exercises (a)}
%\spheader 312Ya
Find a function $\phi:\Cal P\{0,1,2\}\to\Bbb Z_2$ such that
$\phi(1\Bsetminus a)=1\Bsetminus\phi a$
for every $a\in\Cal P\{0,1,2\}$ and $\phi(a)\Bsubseteq\phi(b)$ whenever
$a\Bsubseteq b$, but $\phi$ is not a Boolean homomorphism.

\header{312Yb}{\bf (b)} Let $\frak A$ be the Boolean ring of finite
subsets of $\Bbb N$.   Show that there is a permutation
$\pi:\frak A\to\frak A$ such that $\pi a\subseteq \pi b$ whenever
$a\subseteq b$ but $\pi$ is not a ring homomorphism.
%mt32bits

\header{312Yc}{\bf (c)} Let $\frak A$, $\frak B$ be Boolean rings, with
Stone spaces $Z$, $W$.   Show that we have a
one-to-one correspondence between ring homomorphisms
$\pi:\frak A\to\frak B$ and continuous functions $\phi:H\to Z$, where
$H\subseteq W$ is an open set, such that
$\phi^{-1}[K]$ is compact for every compact set $K\subseteq Z$, given by
the formula $\pi a=b\iff\phi^{-1}[\widehat a]=\widehat b$.

\header{312Yd}{\bf (d)} Let $\frak A$, $\frak B$, $\frak C$ be Boolean
rings, with Stone spaces $Z$, $W$ and $V$.   Let $\pi:\frak A\to\frak B$
and $\theta:\frak B\to\frak C$ be ring homomorphisms, with corresponding
continuous functions $\phi:H\to Z$ and $\psi:G\to W$.   Show that the
ring homomorphism $\theta\pi:\frak A\to\frak C$ corresponds to the
continuous function $\phi\psi:\psi^{-1}[H]\to Z$.

\header{312Ye}{\bf (e)} Let $\frak A$ and $\frak B$ be Boolean rings,
with Stone spaces $Z$ and $W$, and $\pi:\frak A\to\frak B$ a ring
homomorphism, with associated continuous function $\phi:H\to Z$.   Show
that $\pi$ is injective iff $\phi[H]$ is dense in $Z$, and that $\pi$ is
surjective iff $\phi$ is injective and $H=W$.

\header{312Yf}{\bf (f)} Let $\frak A$ be a Boolean ring and
$a\in\frak A$. Show that the Stone space of the principal ideal
$\frak A_a$ of $\frak A$ generated by $a$ can be identified with the
compact open set $\widehat a$
in the Stone space of $\frak A$.   Show that the identity map is a ring
homomorphism from $\frak A_a$ to $\frak A$, and corresponds to the
identity function on $\widehat a$.
}%end of exercises

\endnotes{
\Notesheader{312} The definitions of `subalgebra' and `Boolean
homomorphism' (312A, 312F), like that of `Boolean algebra',
are a trifle arbitrary, but will be a convenient way of mandating
appropriate treatment of multiplicative identities.   I run through the
work of 312A-312J essentially for completeness;  once you are familiar
with Boolean algebras, they should all seem obvious.   312M has a little
bit more to it.   It
shows that the order structure of a Boolean algebra defines the ring
structure, in a fairly strong sense.

I call 312O a `lemma', but actually it is the most important result in
this section;  it is the basic tool we have for extending a homomorphism
from a subalgebra to a slightly larger one, and with Zorn's Lemma
(another \lq lemma' which deserves a capital L) will
provide us with general methods of constructing homomorphisms.

In 312P-312T I describe the basic relationships between the Boolean
homomorphisms and continuous functions on Stone spaces.   312Q-312R show
that, in the language of category theory, the Stone representation
provides a `contravariant functor' from the category of Boolean algebras
with Boolean homomorphisms to the category of topological spaces with
continuous functions.   Using 311I-311J, we know exactly which
topological spaces appear, the zero-dimensional compact Hausdorff
spaces;  and we know
also that the functor is faithful, that is, that we can recover Boolean
algebras and homomorphisms from the corresponding topological spaces and
continuous functions.   There is an agreeable duality in 312S.   All of
this can be done for Boolean rings, but there are some extra
complications
(312Yc-312Yf).

To my mind, the very essence of the theory of Boolean algebras is the
fact that they are abstract rings, but at the same time can be thought
of `locally' as algebras of sets.   Consequently we can bring two quite
separate kinds of intuition to bear.   312O gives an example of a
ring-theoretic problem, concerning the extension of homomorphisms, which
has a resolution in terms of the order relation, a concept most
naturally described in terms of algebras-of-sets.   It is very much a
matter of taste and habit, but I myself find that a Boolean homomorphism
is easiest
to think of in terms of its action on finite subalgebras, which are
directly representable as $\Cal PX$ for some finite $X$ (311Xe);  the
corresponding continuous map between Stone spaces is less helpful.   I
offer 312Xj, the Stone-space version of 312O, for you to test your own
intuitions on.
}%end of notes

\discrpage

