\frfilename{mt4a7.tex}
\versiondate{5.10.13}
\copyrightdate{2013}
     
\def\chaptername{Appendix}
\def\sectionname{`Later editions only'}
     
\newsection{4A7}

In this edition of Volume 4 I refer to a handful of fragments which I have
interpolated into earlier
volumes and which have not yet appeared in a printed version.
For the time being they may be found, in context, in the online
drafts listed in
{\tt http://www.essex.ac.uk{\bsp}maths{\bsp}people{\bsp}fremlin{\bsp}mtcont.htm}.

\leader{235Xn}{Exercise} 
Let $(X,\Sigma,\mu)$ and $(Y,\Tau,\nu)$ be 
measure spaces, and $\phi:X\to Y$ an \imp\ function.   Show that
$\overline{\int}h\phi\,d\mu\le\overline{\int}h\,d\nu$ for every
real-valued function $h$ defined almost everywhere in $Y$.
(Compare 234Bf.)

\leader{254Yh}{Exercise}
Let $f:[0,1]\to[0,1]^2$ be a function
which is \imp\ for Lebesgue planar measure on $[0,1]^2$ and
Lebesgue linear measure on $[0,1]$, as in 134Yl;  let $f_1$, $f_2$ be the
coordinates of $f$.   Define $g:[0,1]\to[0,1]^{\Bbb N}$ by setting
$g(t)=\sequencen{f_1f_2^n(t)}$ for $0\le t\le 1$.   Show that $g$ is
\imp.   \Hint{show that $g_n:[0,1]\to[0,1]^{n+1}$ is \imp\ for every 
$n\ge 1$, where 
$g_n(t)=(f_1(t),f_1f_2(t),\ldots,\penalty-100f_1f_2^{n-1}(t),f_2^n(t))$ 
for $t\in[0,1]$.}

\leader{2A5B}{Definition} If $U$ is a linear space over $\RoverC$, a
functional $\tau:U\to\coint{0,\infty}$ is an {\bf F-seminorm} if

\inset{(i) $\tau(u+v)\le\tau(u)+\tau(v)$ for all $u$, $v\in U$,
$\tau\in\Tau$;

(ii) $\tau(\alpha u)\le\tau(u)$ if $u\in U$, $|\alpha|\le 1$,
$\tau\in\Tau$;

(iii) $\lim_{\alpha\to 0}\tau(\alpha u)=0$ for every $u\in U$,
$\tau\in \Tau$.}

\leader{535Yd}{Exercise}
Let $(X,\Sigma,\mu)$ be a countably separated
perfect complete strictly localizable measure space,
$\frak A$ its measure algebra and $G$ a subgroup of $\Aut\frak A$ of
cardinal at most $\min(\add\Cal N,\frak p)$, where $\Cal N$ is the
null ideal of Lebesgue measure on $\Bbb R$.   Show that there is an action
$\action$ of $G$ on $X$ such that
$\pi\action E=\{\pi\action x:x\in E\}$ belongs to $\Sigma$ and
$(\pi\action E)^{\ssbullet}=\pi(E^{\ssbullet})$ whenever
$\pi\in G$ and $E\in\Sigma$.   \Hint{344C, 425Ya.}

\leader{536C}{Proposition} Let $(X,\Sigma,\mu)$ be a probability space
such that the $\pi$-weight $\pi(\mu)$ of $\mu$ is at most $\frak p$.   If
$K\subseteq\eusm L^0$ is $\frak T_p$-compact then it is
$\frak T_m$-compact.

\discrpage   
          
     

