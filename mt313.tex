\frfilename{mt313.tex}
\versiondate{8.6.11}
\copyrightdate{1995}

\def\chaptername{Boolean algebras}
\def\sectionname{Order-continuous homomorphisms}

\newsection{313}

Because a Boolean algebra has a natural partial order\cmmnt{ (311H)},
we have corresponding notions of upper bounds, lower bounds, suprema and
infima.   These are particularly important in the Boolean algebras
arising in
measure theory, and the infinitary operations `$\sup$' and `$\inf$'
require rather more care than the basic binary operations `$\Bcup$',
`$\Bcap$', because intuitions from elementary set theory are sometimes
misleading.   I therefore take a section to work through the most
important properties of these operations, together with the
homomorphisms which preserve them.

\leader{313A}{Relative complementation:  Proposition} Let $\frak A$ be a
Boolean algebra, $e$ a member of $\frak A$, and $A$ a non-empty subset
of $\frak A$.

(a) If $\sup A$ is defined in $\frak A$, then $\inf\{e\Bsetminus a:a\in
A\}$ is defined and equal to $e\setminus\sup A$.

(b) If $\inf A$ is defined in $\frak A$, then $\sup\{e\Bsetminus a:a\in
A\}$ is defined and equal to $e\setminus\inf A$.

\proof{{\bf (a)} Writing $a_0$ for $\sup A$, we have $e\Bsetminus
a_0\Bsubseteq e\Bsetminus a$ for every $a\in A$, so $e\Bsetminus a_0$ is
a lower bound for $C=\{e\Bsetminus a:a\in A\}$.   Now suppose that $c$
is any lower
bound for $C$.   Then (because $A$ is not empty) $c\Bsubseteq e$, and

\Centerline{$a=(a\Bsetminus e)\Bcup(e\Bsetminus(e\Bsetminus a))
\Bsubseteq (a_0\Bsetminus e)\Bcup(e\Bsetminus c)$}

\noindent for every $a\in A$.   Consequently
$a_0\Bsubseteq(a_0\Bsetminus e)\Bcup(e\Bsetminus c)$ is disjoint from
$c$
and

\Centerline{$c=c\Bcap e\Bsubseteq e\Bsetminus a_0$.}

\noindent Accordingly $e\Bsetminus a_0$ is the greatest lower bound of
$C$,
as claimed.

\medskip

{\bf (b)} This time set $a_0=\inf A$, $C=\{e\Bsetminus a:a\in A\}$.   As
before, $e\Bsetminus a_0$ is surely an upper bound for $C$.   If $c$ is
any upper bound for $C$, then

\Centerline{$e\Bsetminus c\Bsubseteq e\Bsetminus(e\Bsetminus a)
=e\Bcap a\Bsubseteq a$}

\noindent for every $a\in A$, so $e\Bsetminus c\Bsubseteq a_0$ and
$e\Bsetminus a_0\Bsubseteq c$.   As $c$ is arbitrary, $e\Bsetminus a_0$
is indeed the least upper bound of $C$.
}%end of proof of 313A

\cmmnt{\medskip

\noindent{\bf Remark} In the arguments above I repeatedly encourage
you to treat $\Bcap$, $\Bcup$, $\Bsetminus$, $\Bsubseteq$ as if they
were the corresponding operations and relation of basic set theory.
This is perfectly safe so long as we take care that every manipulation
so justified has only finitely many elements of the Boolean algebra in
hand at once.
}

\leader{313B}{General distributive laws:  Proposition} Let $\frak A$ be
a Boolean algebra.

(a) If $e\in\frak A$ and $A\subseteq\frak A$ is a non-empty set such
that $\sup A$ is defined in $\frak A$, then $\sup\{e\Bcap a:a\in A\}$ is
defined and equal to $e\Bcap\sup A$.

(b) If $e\in\frak A$ and $A\subseteq\frak A$ is a non-empty set such
that $\inf A$ is defined in $\frak A$, then $\inf\{e\Bcup a:a\in A\}$ is
defined and equal to $e\Bcup\inf A$.

(c) Suppose that $A$, $B\subseteq\frak A$ are non-empty and $\sup A$,
$\sup B$ are
defined in $\frak A$.   Then $\sup\{a\Bcap b:a\in A,\,b\in B\}$ is
defined and is equal to $\sup A\Bcap\sup B$.

(d) Suppose that $A$, $B\subseteq\frak A$ are non-empty and $\inf A$,
$\inf B$ are
defined in $\frak A$.   Then $\inf\{a\Bcup b:a\in A,\,b\in B\}$ is
defined and is equal to $\inf A\Bcup\inf B$.

\proof{{\bf (a)} Set

\Centerline{$B=\{e\Bsetminus a:a\in A\}$,\quad $C=\{e\Bsetminus b:b\in
B\}=\{e\Bcap a:a\in A\}$.    }

\noindent Using 313A, we have

\Centerline{$\inf B=e\Bsetminus\sup A$,
\quad$\sup C=e\Bsetminus\inf B=e\Bcap\sup A$,}

\noindent as required.

\medskip

{\bf (b)} Set $a_0=\inf A$, $B=\{e\Bcup a:a\in A\}$.   Then $e\Bcup
a_0\Bsubseteq e\Bcup a$ for every $a\in A$, so $e\Bcup a_0$ is a lower
bound for $B$.   If $c$ is any lower bound for $B$, then $c\Bsetminus
e\Bsubseteq a$ for every $a\in A$, so $c\Bsetminus e\Bsubseteq a_0$ and
$c\Bsubseteq e\Bcup a_0$;  thus $e\Bcup a_0$ is the greatest lower bound
for $B$, as claimed.

\wheader{313B}{6}{2}{2}{30pt}

{\bf (c)} By (a), we have

\Centerline{$a\Bcap\sup B=\sup_{b\in B}a\Bcap b$}

\noindent for every $a\in A$, so

\Centerline{$\sup_{a\in A,b\in B}a\Bcap b
=\sup_{a\in A}(a\Bcap\sup B)=\sup A\Bcap\sup B$,}

\noindent using (a) again.

\medskip

{\bf (d)} Similarly, using (b) twice,

\Centerline{$\inf_{a\in A,b\in B}a\Bcup b
=\inf_{a\in A}(a\Bcup\inf B)=\inf A\Bcup\inf B$.}
}%end of proof of 313B

\leader{313C}{}\cmmnt{ As always, it is worth developing a
representation of the concepts of $\sup$ and $\inf$ in terms of Stone
spaces.

\medskip

\noindent}{\bf Proposition} Let $\frak A$ be a Boolean algebra, and $Z$
its Stone space;  for $a\in\frak A$ write $\widehat a$ for the
corresponding open-and-closed subset of $Z$.

(a) If $A\subseteq\frak A$ and $a_0\in\frak A$ then $a_0=\sup A$ in
$\frak A$ iff $\widehat a_0=\overline{\bigcup_{a\in A}\widehat a}$.

(b) If $A\subseteq\frak A$ is non-empty and $a_0\in\frak A$ then
$a_0=\inf A$ in $\frak A$ iff
$\widehat a_0=\interior\bigcap_{a\in A}\widehat a$.

(c) If $A\subseteq\frak A$ is non-empty then $\inf A=0$ in $\frak A$ iff
$\bigcap_{a\in A}\widehat a$ is nowhere dense in $Z$.

\proof{{\bf (a)} For any $b\in\frak A$,

$$\eqalignno{b\text{ is an upper bound for }A
&\iff\widehat a\subseteq\widehat b\text{ for every }a\in A\cr
&\iff\bigcup_{a\in A}\widehat a\subseteq\widehat b
\iff\overline{\bigcup_{a\in A}\widehat a}\subseteq\widehat b\cr}$$

\noindent because $\widehat b$ is certainly closed in $Z$.   It follows
at once that if $\widehat a_0$ is actually equal to
$\overline{\bigcup_{a\in A}\widehat a}$ then $a_0$ must be the least
upper bound of $A$ in $\frak A$.   On the other hand, if $a_0=\sup A$,
then $\overline{\bigcup_{a\in A}\widehat a}\subseteq\widehat a_0$.
\Quer\ If $\widehat a_0\ne\overline{\bigcup_{a\in A}\widehat a}$,
then $\widehat a_0\setminus\overline{\bigcup_{a\in A}\widehat a}$ is a
non-empty open set in $Z$, so includes $\widehat b$ for some non-zero
$b\in\frak A$;  now
$\widehat a\subseteq\widehat a_0\setminus\widehat b$, so
$a\Bsubseteq a_0\Bsetminus b$ for
every $a\in A$, and $a_0\Bsetminus b$ is an upper bound for $A$ strictly
less than $a_0$.\ \BanG\  Thus $\widehat a_0$ must be exactly
$\overline{\bigcup_{a\in A}\widehat a}$.

\medskip

{\bf (b)} Take complements:  setting $a_1=1\Bsetminus a_0$, we have

$$\eqalignno{a_0=\inf A
&\iff a_1=\sup_{a\in A}1\Bsetminus a\cr
\noalign{\noindent (by 313A)}
&\iff\widehat a_1=\overline{\bigcup_{a\in A}Z\setminus\widehat a}\cr
&\iff\widehat a_0
  =Z\setminus\overline{\bigcup_{a\in A}Z\setminus\widehat a}
=\interior\bigcap_{a\in A}\widehat a.\cr}$$

\medskip

{\bf (c)} Since $\bigcap_{a\in A}\widehat a$ is surely a closed set, it
is nowhere dense iff it has empty interior, that is, iff $0=\inf A$.
}%end of proof of 313C

\leader{313D}{}\cmmnt{ I started the section with the results above
because they are easily stated and of great importance.   But I must now
turn to some new definitions, and I think it may help to clarify the
ideas involved if I give them in their own natural context, even though
this is far more general than we have any immediate need for here.

\medskip

\noindent}{\bf Definitions} Let $P$ be a partially ordered set and $C$ a
subset of $P$.

\spheader 313Da $C$ is {\bf order-closed} if $\sup A\in C$ whenever $A$ is a
non-empty upwards-directed subset of $C$ such that $\sup A$ is defined
in $P$, and
$\inf A\in C$ whenever $A$ is a non-empty downwards-directed subset of
$C$ such that $\inf A$ is defined in $P$.

\spheader 313Db $C$ is {\bf sequentially order-closed} if
$\sup_{n\in\Bbb N}p_n\in C$ whenever $\sequencen{p_n}$ is a
non-decreasing sequence in $C$ such
that $\sup_{n\in\Bbb N}p_n$ is defined in $P$, and
$\inf_{n\in\Bbb N}p_n\in C$ whenever $\sequencen{p_n}$ is a
non-increasing sequence in
$C$ such that $\inf_{n\in\Bbb N}p_n$ is defined in $P$.

\cmmnt{
\medskip

\noindent{\bf Remark} I hope it is obvious that an order-closed set is
sequentially order-closed.
}

\leader{313E}{Order-closed subalgebras and ideals}\cmmnt{ Of course,
in the very special cases of a subalgebra or ideal
of a Boolean algebra, the concepts `order-closed' and
`sequentially order-closed' have expressions simpler than those in 313D.
I spell them out.

\medskip

}{\bf (a)} Let $\frak B$ be a subalgebra of a Boolean algebra $\frak A$.

\medskip

\quad{\bf (i)} The following are equiveridical:

\qquad($\alpha$) $\frak B$ is order-closed in $\frak A$;

\qquad ($\beta$) $\sup B\in\frak B$ whenever $B\subseteq\frak B$ and
$\sup B$ is defined in $\frak A$;

\qquad ($\beta'$) $\inf B\in\frak B$ whenever $B\subseteq\frak B$ and
$\inf B$ is defined in $\frak A$;

\qquad($\gamma$) $\sup B\in\frak B$ whenever
$B\subseteq\frak B$ is non-empty and upwards-directed and $\sup B$ is
defined in $\frak A$;

\qquad($\gamma'$) $\inf B\in\frak B$ whenever $B\subseteq\frak B$ is
non-empty
and downwards-directed and $\inf B$ is defined in $\frak A$.

\prooflet{\Prf\ Of course ($\beta)\Rightarrow(\gamma$).   If ($\gamma$)
is true and $B\subseteq\frak B$ is any set with a supremum in
$\frak A$, then $B'=\{0\}\cup\{b_0\Bcup\ldots\Bcup b_n:b_0,\ldots,b_n\in
B\}$ is a non-empty upwards-directed set with the same upper bounds as
$B$, so $\sup B=\sup B'\in \frak B$.   Thus ($\gamma)\Rightarrow(\beta$)
and ($\beta$), ($\gamma$) are equiveridical.   Next, if ($\beta$) is
true
and $B\subseteq\frak B$ is a set with an infimum in $\frak A$, then
$B'=\{1\Bsetminus b:b\in\frak B\}\subseteq\frak B$ and $\sup
B'=1\Bsetminus\inf B$ is defined, so $\sup B'$ and $\inf B$ belong to
$\frak B$ .   Thus ($\beta)\Rightarrow(\beta'$).   In the same way,
($\gamma')\iff(\beta')\Rightarrow(\beta$) and ($\beta$), ($\beta'$),
($\gamma$), $(\gamma'$) are all equiveridical.   But since we also have
($\alpha)\iff(\gamma)\&(\gamma')$, $(\alpha)$ is equiveridical with the
others.\ \Qed}

\cmmnt{Replacing the sets $B$ above by sequences, the same arguments
provide conditions for $\frak B$ to be sequentially order-closed, as
follows.}

\medskip

\quad{\bf (ii)} The following are equiveridical:

\qquad($\alpha$) $\frak B$ is sequentially order-closed in $\frak A$;

\qquad($\beta$) $\sup_{n\in\Bbb N}b_n\in\frak B$ whenever
$\sequencen{b_n}$ is a sequence in $\frak B$ and $\sup_{n\in\Bbb N}b_n$
is defined in $\frak A$;

\qquad($\beta'$) $\inf_{n\in\Bbb N}b_n\in\frak B$ whenever
$\sequencen{b_n}$
is a sequence in $\frak B$ and $\inf_{n\in\Bbb N}b_n$ is defined in
$\frak A$;

\qquad ($\gamma$) $\sup_{n\in\Bbb N}b_n\in\frak B$ whenever
$\sequencen{b_n}$ is a non-decreasing sequence in $\frak B$ and
$\sup_{n\in\Bbb N}b_n$ is defined in $\frak A$;

\qquad ($\gamma'$)
$\inf_{n\in\Bbb N}b_n\in\frak B$ whenever $\sequencen{b_n}$ is a
non-increasing sequence in $\frak B$ and $\inf_{n\in\Bbb N}b_n$ is
defined in $\frak A$.

\spheader 313Eb Now suppose that $I$ is an ideal of $\frak A$.
Then\cmmnt{ if $A\subseteq I$ is non-empty all lower bounds of $A$
necessarily belong to $I$;  so that}

\inset{$I$ is order-closed iff $\sup A\in I$ whenever $A\subseteq I$ is
non-empty, upwards-directed and has a supremum in $\frak A$;}

\inset{$I$ is sequentially order-closed iff $\sup_{n\in\Bbb N}a_n\in I$
whenever $\sequencen{a_n}$ is a non-decreasing sequence in $I$ with a
supremum in $\frak A$.}

\cmmnt{\noindent Moreover, because $I$ is closed under $\Bcup$,}

\inset{$I$ is order-closed iff $\sup A\in I$ whenever $A\subseteq I$ has
a supremum in $\frak A$;}

\inset{$I$ is sequentially order-closed iff $\sup_{n\in\Bbb N}a_n\in I$
whenever $\sequencen{a_n}$ is a sequence in $I$ with a
supremum in $\frak A$.}

\medskip

{\bf (c)}\cmmnt{ If $\frak A=\Cal PX$ is a power set, then a
sequentially order-closed subalgebra of $\frak A$ is just a
$\sigma$-algebra of sets,
while a sequentially order-closed ideal of $\frak A$ is a what I have
called a $\sigma$-ideal of sets (112Db).   If $\frak A$ is itself a
$\sigma$-algebra of sets, then a sequentially order-closed subalgebra of
$\frak A$ is a `$\sigma$-subalgebra' in the sense of 233A.

Accordingly} I will normally use the phrases {\bf $\sigma$-subalgebra},
{\bf $\sigma$-ideal} for sequentially order-closed subalgebras and
ideals of Boolean algebras.

\leader{313F}{Order-closures and generated sets (a)}\cmmnt{ It is an
immediate consequence of the definitions that

}(i) \dvro{If}{if} $\Cal S$ is any non-empty family of subalgebras of a
Boolean algebra $\frak A$, then $\bigcap\Cal S$ is a subalgebra of
$\frak A$;

(ii) if $\Cal F$ is any non-empty family of order-closed subsets of a
partially ordered set $P$, then $\bigcap\Cal F$ is an order-closed
subset of $P$;

(iii) if $\Cal F$ is any non-empty family of sequentially order-closed
subsets of a partially ordered set $P$, then $\bigcap\Cal F$ is a
sequentially order-closed subset of $P$.

\spheader 313Fb Consequently, given any Boolean algebra $\frak A$ and a
subset $B$ of $\frak A$, we have a smallest subalgebra $\frak B$ of
$\frak A$ including $B$, being the intersection of all the subalgebras
of $\frak A$ which include $B$;
a smallest $\sigma$-subalgebra $\frak B_{\sigma}$ of $\frak A$ including
$B$, being the intersection of all the $\sigma$-subalgebras of $\frak A$
which include $B$;
and a smallest order-closed subalgebra $\frak B_{\tau}$ of $\frak A$
including $B$, being the intersection of all the order-closed
subalgebras of $\frak A$ which include $B$.
We call $\frak B$, $\frak B_{\sigma}$ and $\frak B_{\tau}$ the
subalgebra, $\sigma$-subalgebra and order-closed subalgebra {\bf
generated} by $B$.   \cmmnt{(I will return to this in 331E.)}

\spheader 313Fc If $\frak A$ is a Boolean algebra and $\frak B$ any
subalgebra
of $\frak A$, then the smallest order-closed subset $\overline{\frak B}$
of $\frak A$ which includes $\frak B$ is again a subalgebra of $\frak A$
(so is the order-closed subalgebra of $\frak A$ generated by $\frak B$).
\prooflet{\Prf\ (i) $0\in\frak B\subseteq\overline{\frak B}$.
(ii) The set $\{b:1\Bsetminus b\in\overline{\frak B}\}$
is order-closed (use 313A) and includes $\frak B$, so includes
$\overline{\frak B}$;  thus $1\Bsetminus b\in\overline{\frak B}$ for
every $b\in\overline{\frak B}$.   (iii) If $c\in\frak B$, the set
$\{b:b\Bcup c\in\overline{\frak B}\}$ is order-closed (use 313Bb) and
includes $\frak B$, so includes $\overline{\frak B}$;  thus
$b\Bcup c\in\overline{\frak B}$ whenever $b\in\overline{\frak B}$ and
$c\in\frak B$.  (iv) If $c\in\overline{\frak B}$, the set
$\{b:b\Bcup c\in\overline{\frak B}\}$ is order-closed and includes
$\frak B$ (by (iii)), so includes $\overline{\frak B}$;  thus
$b\Bcup c\in\overline{\frak B}$ whenever $b$, $c\in\overline{\frak B}$.
(v) By 312B(ii), $\overline{\frak B}$ is a subalgebra of
$\frak A$.\ \Qed}%end of prooflet

\leader{313G}{}\cmmnt{ This is a convenient moment at which to spell
out an abstract version of the Monotone Class Theorem (136B).

\medskip

\noindent}{\bf Lemma} Let $\frak A$ be a Boolean algebra.

(a) Suppose that $1\in I\subseteq A\subseteq\frak A$ and that

\Centerline{$a\Bcap b\in I$ for all $a$, $b\in I$,}

\Centerline{$b\Bsetminus a\in A$ whenever $a$, $b\in A$ and
$a\Bsubseteq b$.}

\noindent Then $A$ includes the subalgebra of $\frak A$ generated by
$I$.

(b) If moreover $\sup_{n\in\Bbb N}a_n\in A$ for every non-decreasing
sequence $\sequencen{a_n}$ in $A$ with a supremum in $\frak A$, then $A$
includes the $\sigma$-subalgebra of $\frak A$ generated by $I$.

(c) And if $\sup C\in A$ whenever $C\subseteq A$ is an upwards-directed
set with a supremum in $\frak A$, then $A$ includes the order-closed
subalgebra of $\frak A$ generated by $I$.

\proof{{\bf (a)(i)} Let $\frak P$ be the family of all sets $J$ such
that $I\subseteq J\subseteq A$ and $a\Bcap b\in J$ for all $a$, $b\in J$.
Then $I\in\frak P$ and if $\frak Q\subseteq\frak P$ is upwards-directed
and not empty, $\bigcup\frak Q\in\frak P$.   By Zorn's Lemma, $\frak P$
has a maximal element $\frak B$.

\medskip

\quad{\bf (ii)} Now

\Centerline{$\frak B=\{c:c\in\frak A,\,c\Bcap b\in A$ for every
$b\in\frak B\}$.}

\noindent\Prf\ If $c\in\frak B$, then of course $c\Bcap b\in\frak
B\subseteq A$ for every $b\in\frak B$, because $\frak B\in\frak P$.   If
$c\in\frak A\setminus\frak B$, consider

\Centerline{$J=\frak B\cup\{c\Bcap b:b\in\frak B\}$.}

\noindent Then $c=c\Bcap 1\in J$ so $J$ properly includes $\frak B$ and
cannot belong to $\frak P$.   On the other hand, if $b_1$, $b_2\in\frak
B$,

\Centerline{$b_1\Bcap b_2\in\frak B\subseteq J$,
\quad $(c\Bcap b_1)\Bcap b_2=b_1\Bcap(c\Bcap b_2)
=(c\Bcap b_1)\Bcap(c\Bcap b_2)=c\Bcap(b_1\Bcap b_2)\in J$,}

\noindent so $c_1\Bcap c_2\in J$ for all $c_1$, $c_2\in J$;  and of
course $I\subseteq\frak B\subseteq J$.   So $J$ cannot be a subset of
$A$, and there must be a $b\in\frak B$ such that
$c\Bcap b\notin A$.\ \Qed

\medskip

\quad{\bf (iii)} Consequently $c\Bsetminus b\in\frak B$ whenever $b$,
$c\in\frak B$ and $b\Bsubseteq c$.   \Prf\ If $a\in\frak B$, then
$b\Bcap a$, $c\Bcap a\in\frak B\subseteq A$ and
$b\Bcap a\Bsubseteq c\Bcap a$, so

\Centerline{$(c\Bsetminus b)\Bcap a
=(c\Bcap a)\Bsetminus(b\Bcap a)\in A$}

\noindent by the hypothesis on $A$.   By (ii),
$c\setminus b\in\frak B$.\ \Qed

\medskip

\quad{\bf (iv)} It follows that $\frak B$ is a subalgebra of $\frak A$.
\Prf\ If $b\in\frak B$, then

\Centerline{$b\Bsubseteq 1\in I\subseteq\frak B$,}

\noindent so $1\Bsetminus b\in\frak B$.   If $a$, $b\in\frak B$, then

\Centerline{$a\Bcup b=1\Bsetminus((1\Bsetminus a)\Bcap(1\Bsetminus
b))\in\frak B$.}

\noindent $0=1\Bsetminus 1\in\frak B$, so that the conditions of
312B(ii) are satisfied.\ \Qed

Now the subalgebra of $\frak A$ generated by $I$ is included in
$\frak B$ and therefore in $A$, as required.

\medskip

{\bf (b)} Now suppose that $\sup_{n\in\Bbb N}a_n$ belongs to $A$
whenever $\sequencen{a_n}$ is a non-decreasing sequence in $A$ with a
supremum in $\frak A$.   Then $\frak B$, as defined in part (a) of the
proof, is a $\sigma$-subalgebra of $\frak A$.   \Prf\ Let
$\sequencen{b_n}$ be a non-decreasing sequence in $\frak B$ with a
supremum $c$ in $\frak A$.   Then for any $b\in\frak B$,
$\sequencen{b_n\Bcap b}$ is a non-decreasing sequence in $A$ with a
supremum $c\Bcap b$ in $\frak A$ (313Ba).   So $c\Bcap b\in A$.   As $b$
is arbitrary, $c\in\frak B$, by the criterion in (a-ii) above.   As
$\sequencen{b_n}$ is arbitrary, $\frak B$ is a $\sigma$-subalgebra, by
313Ea.\ \Qed

Accordingly the $\sigma$-subalgebra of $\frak A$ generated by $I$ is
included in $\frak B$ and therefore in $\frak A$.

\medskip

{\bf (c)} Finally, if $\sup C\in A$ whenever $C$ is a non-empty
upwards-directed subset of $A$ with a least upper bound in $\frak A$,
$\frak B$ is order-closed.
\Prf\ Let $C\subseteq\frak B$ be a non-empty upwards-directed set with a
supremum $c$ in $\frak A$.   Then for any $b\in\frak B$,
$\{c\Bcap b:c\in C\}$ is a non-empty upwards-directed set in $A$ with
supremum
$c\Bcap b$ in $\frak A$.   So $c\Bcap b\in A$.   As $b$ is arbitrary,
$c\in\frak B$.   As $C$ is arbitrary, $\frak B$ is order-closed in
$\frak A$ (313Ea(i-$\alpha$)).\ \Qed

Accordingly the order-closed subalgebra of $\frak A$ generated by $I$ is
included in $\frak B$ and therefore in $\frak A$.

}%end of proof of 313G


\leader{313H}{Definitions}\cmmnt{ It is worth distinguishing various
types of
supremum- and infimum-preserving function.   Once again, I do this in
almost the widest possible context.}   Let $P$ and $Q$ be two
partially ordered sets, and $\phi:P\to Q$ an {\bf order-preserving}
function,
that is, a function such that $\phi(p)\le\phi(q)$ in $Q$ whenever
$p\le q$ in $P$.

\spheader 313Ha I say that $\phi$ is {\bf order-continuous} if (i)
$\phi(\sup A)=\sup_{p\in A}\phi(p)$ whenever $A$ is a
non-empty upwards-directed subset of $P$ and $\sup A$ is defined in $P$
(ii) $\phi(\inf A)=\inf_{p\in A}\phi(p)$ whenever $A$ is a
non-empty downwards-directed subset of $P$ and $\inf A$ is defined in
$P$.

\spheader 313Hb I say that $\phi$ is {\bf sequentially order-continuous} or
{\bf $\sigma$-order-continuous} if (i)
$\phi(p)=\sup_{n\in\Bbb N}\phi(p_n)$ whenever $\sequencen{p_n}$ is a
non-decreasing sequence in $P$ and $p=\sup_{n\in\Bbb N}p_n$ in $P$ (ii)
$\phi(p)=\inf_{n\in\Bbb N}\phi(p_n)$ whenever $\sequencen{p_n}$ is a
non-increasing sequence in $P$ and $p=\inf_{n\in\Bbb N}p_n$ in $P$.

\cmmnt{\medskip

\noindent{\bf Remark} You may feel that one of the equivalent
formulations in Proposition 313Lb gives a clearer idea of what is really
being demanded of $\phi$ in the ordinary cases we shall be looking at.
}

\leader{313I}{Proposition} Let $P$, $Q$ and $R$ be partially ordered
sets, and $\phi:P\to Q$, $\psi:Q\to R$ order-preserving functions.

(a) $\psi\phi:P\to R$ is order-preserving.

(b) If $\phi$ and $\psi$ are order-continuous, so is $\psi\phi$.

(c) If $\phi$ and $\psi$ are sequentially order-continuous, so is
$\psi\phi$.

(d) $\phi$ is order-continuous iff $\phi^{-1}[B]$ is order-closed for
every order-closed $B\subseteq Q$.

\proof{{\bf (a)-(c)} I think the only point that needs remarking is that
if $A\subseteq P$ is upwards-directed, then $\phi[A]\subseteq Q$ is
upwards-directed, because $\phi$ is
order-preserving.   So if $\sup A$
is defined in $P$ and $\phi$, $\psi$ are order-continuous, we shall have

\Centerline{$\psi(\phi(\sup A))=\psi(\sup\phi[A])=\sup\psi[\phi[A]]$.}

\medskip

{\bf (d)(i)} Suppose that $\phi$ is order-continuous and that
$B\subseteq Q$ is order-closed.   Let $A\subseteq\phi^{-1}[B]$ be a
non-empty upwards-directed set with supremum $p\in P$.   Then
$\phi[A]\subseteq B$ is non-empty and upwards-directed, because $\phi$
is order-preserving, and $\phi(p)=\sup\phi[A]$ because $\phi$ is
order-continuous.   Because $B$ is order-closed, $\phi(p)\in B$ and
$p\in\phi^{-1}[B]$.   Similarly, if $A\subseteq\phi^{-1}[B]$ is
non-empty and downwards-directed, and $\inf A$ is defined in $P$, then
$\phi(\inf A)=\inf\phi[A]\in B$ and $\inf A\in\phi^{-1}[B]$.   Thus
$\phi^{-1}[B]$ is order-closed;  as $B$ is arbitrary, $\phi$ satisfies
the condition.

\medskip

\quad{\bf (ii)} Now suppose that $\phi^{-1}[B]$ is order-closed in $P$
whenever $B\subseteq Q$ is order-closed in $Q$.   Let $A\subseteq P$ be
a non-empty upwards-directed subset of $P$ with a supremum $p\in P$.
Then $\phi(p)$ is an upper bound of $\phi[A]$.   Let $q$ be any upper
bound of $\phi[A]$ in $Q$.   Consider $B=\{r:r\le q\}$;  then
$B\subseteq Q$ is upwards-directed and
order-closed, so $\phi^{-1}[B]$ is order-closed.   Also
$A\subseteq\phi^{-1}[B]$ is non-empty and upwards-directed and has
supremum $p$, so $p\in\phi^{-1}[B]$ and $\phi(p)\in B$, that is,
$\phi(p)\le q$.   As $q$ is arbitrary, $\phi(p)=\sup\phi[A]$.
Similarly, $\phi(\inf A)=\inf\phi[A]$ whenever $A\subseteq P$ is
non-empty, downwards-directed and has an infimum in $P$;  so $\phi$ is
order-continuous.
}%end of proof of 313I

\leader{313J}{}\cmmnt{ It is useful to introduce here the
following notion.

\medskip

\noindent}{\bf Definition} Let $\frak A$ be a Boolean algebra.   A set
$D\subseteq\frak A$ is {\bf order-dense} if for every non-zero
$a\in\frak A$ there is a non-zero $d\in D$ such that $d\Bsubseteq a$.

\cmmnt{\medskip

\noindent{\bf Remark} Many authors use the simple word `dense' where
I have insisted on the phrase `order-dense'.   In the work of this
treatise it will be important to distinguish clearly between this
concept of `dense' set and the topological concept (2A3U).
}

\leader{313K}{Lemma} If $\frak A$ is a Boolean algebra and
$D\subseteq\frak A$ is order-dense, then for any $a\in\frak A$ there is
a disjoint $C\subseteq D$ such that $\sup C=a$;  in particular,
$a=\sup\{d:d\in D,\,d\Bsubseteq a\}$ and there is a
partition of unity $C\subseteq D$.

\proof{ Set $D_a=\{d:d\in D,\,d\Bsubseteq a\}$.  Applying Zorn's lemma
to the family $\Cal C$ of disjoint sets
$C\subseteq D_a$, we have a maximal $C\in\Cal C$.   Now if $b\in\frak A$
and $b\notBsupseteq a$, there is a $d\in D$ such that $0\ne d\Bsubseteq
a\Bsetminus b$.
Because $C$ is maximal, there must be a $c\in C$ such that
$c\Bcap d\ne 0$, so that $c\notBsubseteq b$.   Turning this round, any
upper bound of $C$ must include $a$, so that $a=\sup C$.   It follows at
once that $a=\sup D_a$.

Taking $a=1$ we obtain a partition of unity included in $D$.
}%end of proof of 313K

\vleader{72pt}{313L}{Proposition} Let $\frak A$ and $\frak B$ be Boolean
algebras and $\pi:\frak A\to\frak B$ a Boolean homomorphism.

(a) $\pi$ is order-preserving.

(b) The following are equiveridical:

\quad (i) $\pi$ is order-continuous;

\quad (ii) whenever $A\subseteq\frak A$ is non-empty and
downwards-directed and $\inf A=0$ in $\frak A$, then $\inf\pi[A]=0$ in
$\frak B$;

\quad (iii) whenever $A\subseteq\frak A$ is non-empty and
upwards-directed and $\sup A=1$ in $\frak A$, then $\sup\pi[A]=1$ in
$\frak B$;

\quad (iv) whenever $A\subseteq\frak A$ and $\sup A$ is defined in
$\frak A$, then $\pi(\sup A)=\sup\pi[A]$ in $\frak B$;

\quad (v) whenever $A\subseteq\frak A$ and $\inf A$ is defined in
$\frak A$, then $\pi(\inf A)=\inf\pi[A]$ in $\frak B$;

\quad (vi) whenever $C\subseteq\frak A$ is a partition of unity, then
$\pi[C]$ is a partition of unity in $\frak B$.

(c) The following are equiveridical:

\quad (i) $\pi$ is sequentially order-continuous;

\quad (ii) whenever $\sequencen{a_n}$ is a non-increasing sequence in
$\frak A$ and $\inf_{n\in\Bbb N}a_n=0$ in $\frak A$, then
$\inf_{n\in\Bbb N}\pi a_n=0$ in $\frak B$;

\quad (iii) whenever $A\subseteq\frak A$ is countable and $\sup A$ is
defined in $\frak A$, then $\pi(\sup A)=\sup\pi[A]$ in $\frak B$;

\quad (iv) whenever $A\subseteq\frak A$ is countable and $\inf A$ is
defined in $\frak A$, then $\pi(\inf A)=\inf\pi[A]$ in $\frak B$;

\quad (v) whenever $C\subseteq\frak A$ is a countable partition of
unity, then $\pi[C]$ is a partition of unity in $\frak B$.

(d)\dvAnew{2010} If $\pi$ is bijective, it is order-continuous.

\proof{{\bf (a)} This is 312I.

\medskip

{\bf (b)(i)$\Rightarrow$(ii)} is trivial, as $\pi 0=0$.

\medskip

\quad{\bf (ii)$\Rightarrow$(iv)} Assume (ii), and let $A$ be any subset
of $\frak A$ such that $c=\sup A$ is defined in $\frak A$.   If
$A=\emptyset$, then $c=0$ and $\sup\pi[A]=0=\pi c$.   Otherwise, set

\Centerline{$A'=\{a_0\Bcup\ldots\Bcup a_n:a_0,\ldots,a_n\in A\}$,
\quad$C=\{c\setminus a:a\in A'\}$.}

\noindent Then $A'$ is upwards-directed and has the same upper bounds as
$A$, so $c=\sup A'$ and $0=\inf C$, by 313Aa.   Also $C$ is
downwards-directed, so $\inf\pi[C]=0$ in $\frak B$.   But now

\Centerline{$\pi[C]=\{\pi c\Bsetminus\pi a:a\in A'\}
=\{\pi c\Bsetminus b:b\in\pi[A']\}$,}

\Centerline{$\pi[A']
=\{\pi a_0\Bcup\ldots\Bcup\pi a_n:a_0,\ldots,a_n\in A\}
=\{b_0\Bcup\ldots\Bcup b_n:b_0,\ldots,b_n\in \pi[A]\}$,}

\noindent because $\pi$ is a Boolean homomorphism.   Again using 313Aa
and the fact that $b\Bsubseteq\pi c$ for every $b\in\pi[A']$, we get

\Centerline{$\pi c=\sup\pi[A']=\sup\pi[A]$.}

\noindent As $A$ is arbitrary, (iv) is satisfied.

\medskip

\quad{\bf (iv)$\Rightarrow$(v)} If $A\subseteq\frak A$ and $c=\inf A$
is defined in $\frak A$, then $1\Bsetminus c=\sup_{a\in A}1\Bsetminus
a$, so

\Centerline{$\pi c=1\Bsetminus\pi(1\Bsetminus c)
=1\Bsetminus\sup_{a\in A}\pi(1\Bsetminus a)
=\inf_{a\in A}1\Bsetminus\pi(1\Bsetminus a)
=\inf_{a\in A}\pi a$.}

\medskip

\quad{\bf (v)$\Rightarrow$(ii)} is trivial, because $\pi 0=0$.

\medskip

\quad{\bf (iv)$\Rightarrow$(iii)} is similarly trivial.

\medskip

\quad{\bf (iii)$\Rightarrow$(vi)} Assume (iii), and let $C$ be a
partition of unity in $\frak A$.   Then
$C'=\{c_0\Bcup\ldots\Bcup c_n:c_0,\ldots,c_n\in C\}$ is upwards-directed
and has supremum $1$, so $\sup\pi[C']=1$.   But (because $\pi$ is a
Boolean homomorphism) $\pi[C]$ and $\pi[C']$ have the same upper bounds,
so $\sup\pi[C]=1$, as required.

\medskip

\quad{\bf (vi)$\Rightarrow$(ii)} Assume (vi), and let
$A\subseteq\frak A$ be a set with infimum $0$.   Set

\Centerline{$D=\{d:d\in\frak A,\,\exists\,a\in A,\,d\Bcap a=0\}$.}

\noindent Then $D$ is order-dense in $\frak A$.   \Prf\ If $e\in\frak
A\Bsetminus\{0\}$, then there is an $a\in A$ such that $e\notBsubseteq
a$, so that $e\Bsetminus a$ is a non-zero member of $D$ included in
$e$.\ \QeD\  Consequently there is a partition of unity $C\subseteq D$,
by 313K.   But now if $b$ is any lower bound for
$\pi[A]$ in $\frak B$, we must have $b\Bcap\pi d=0$ for every $d\in D$,
so $\pi c\Bsubseteq 1\Bsetminus b$ for every $c\in C$, and
$1\Bsetminus b=1$, $b=0$.   Thus $\inf\pi[A]=0$.   As $A$ is arbitrary,
(ii) is satisfied.

\medskip

\quad{\bf (v)\&(iv)$\Rightarrow$(i)} is trivial.

\medskip

{\bf (c)} We can use nearly identical arguments, remembering only to
interpolate the word `countable' from time to time.   I spell out
the new version of (ii)$\Rightarrow$(iv), even though it requires no
more than an adaptation of the language.  Assume (ii), and let $A$ be a
countable subset of $\frak A$ with a supremum $c\in\frak A$.   If
$A=\emptyset$, then $c=0$ so $\pi c=0=\sup\pi[A]$.   Otherwise, let
$\sequencen{a_n}$ be a sequence running over $A$;  set
$a'_n=a_0\Bcup\ldots\Bcup a_n$ and $c_n=c\Bsetminus a'_n$ for each $n$.
Then $\sequencen{a'_n}$ is non-decreasing, with supremum $c$, and
$\sequencen{c_n}$ is non-increasing, with infimum $0$;  so
$\inf_{n\in\Bbb N}\pi c_n=0$ and

\Centerline{$\sup_{n\in\Bbb N}\pi a_n
=\sup_{n\in\Bbb N}\pi a'_n=\pi c$.}

For (v)$\Rightarrow$(ii), however, a different idea is involved.
Assume (v), and suppose that $\sequencen{a_n}$ is a non-increasing
sequence in $\frak A$ with infimum $0$.   Set $c_0=1\Bsetminus a_0$,
$c_n=a_{n-1}\Bsetminus a_{n}$ for $n\ge 1$;  then $C=\{c_n:n\in\Bbb N\}$
is a partition of unity in $\frak A$ (because if $c\Bcap c_n=0$ for
every $n$, then $c\Bsubseteq a_n$ for every $n$), so $\pi[C]$ is a
partition of unity in $\frak B$.
Now if $b\Bsubseteq\pi a_n$ for every $n$,
$b\Bcap\pi c_n$ for every $n$, so $b=0$;  thus $\inf_{n\in\Bbb
N}\pi a_n=0$.   As $\sequencen{a_n}$ is arbitrary, (ii) is satisfied.

\medskip

{\bf (d)} Suppose that $A\subseteq\frak A$ is non-empty and
$\inf A=0$ in $\frak A$.   Let $b\in\frak B$ be a
lower bound for $\pi[A]$.   Because $\pi$ is surjective,
there is a $c\in\frak A$ such that
$\pi c=b$.   If $a\in A$, then

\Centerline{$\pi(a\Bcap c)
=\pi a\Bcap\pi c=\pi a\Bcap b=b=\pi c$;}

\noindent because $\pi$ is injective, $a\Bcap c=c$ and
$c\Bsubseteq a$.   As $a$ is arbitrary, $c$ is a lower bound of $A$
and must be $0$;  so $b_0=\pi 0=0$.   As $b$ is arbitrary, $\inf\pi[A]=0$;
as $A$ is arbitrary, $\pi$ is order-continuous, by (b)(ii)$\Rightarrow$(i).
}%end of proof of 313L

\leader{313M}{}\cmmnt{ The following result is perfectly elementary,
but it will save a moment later on to have it spelt out.

\medskip

\noindent}{\bf Lemma} Let $\frak A$ and $\frak B$ be Boolean algebras
and $\pi:\frak A\to\frak B$ an order-continuous Boolean homomorphism.

(a) If $\frak D$ is an order-closed subalgebra of $\frak B$, then
$\pi^{-1}[\frak D]$ is an order-closed subalgebra of $\frak A$.

(b) If $\frak C$ is the order-closed subalgebra of $\frak A$ generated
by $C\subseteq\frak A$, then the order-closed subalgebra $\frak D$ of
$\frak B$ generated by $\pi[C]$ includes $\pi[\frak C]$.

(c) Now suppose that $\pi$ is surjective and that $C\subseteq\frak A$ is
such that the order-closed subalgebra of $\frak A$ generated by $C$ is
$\frak A$ itself.   Then the order-closed subalgebra of $\frak B$
generated by $\pi[C]$ is $\frak B$.

\proof{{\bf (a)} Setting $\frak C=\pi^{-1}[\frak D]$:  if $a$,
$a'\in\frak C$ then $\pi(a\Bcap b)=\pi a\Bcap\pi b$,
$\pi(a\Bsymmdiff a')=\pi a\Bsymmdiff\pi a'\in\frak D$, so $a\Bcap a'$,
$a\Bsymmdiff a'\in\frak C$;  $\pi 1=1\in\frak D$ so $1\in\frak C$;
thus $\frak C$ is a subalgebra of
$\frak A$.  By 313Id, $\frak C$ is order-closed.

\medskip

{\bf (b)} By (a), $\pi^{-1}[\frak D]$ is an order-closed subalgebra of
$\frak A$.   It includes $C$ so includes $\frak C$, and $\pi[\frak
C]\subseteq \frak D$.

\medskip

{\bf (c)} In the language of (b), we have $\frak C=\frak A$, so $\frak
D$
must be $\frak B$.
}%end of proof of 313M.

\leader{313N}{Definition} The phrase {\bf regular embedding} is
sometimes
used to mean an injective order-continuous Boolean homomorphism;  a
subalgebra $\frak B$ of a Boolean algebra $\frak A$ is said to be
{\bf regularly embedded} in $\frak A$ if the identity map from $\frak B$
to $\frak A$ is order-continuous\cmmnt{, that is, if whenever
$b\in\frak B$ is the supremum (in
$\frak B$) of $B\subseteq\frak B$, then $b$ is also the supremum in
$\frak A$ of $B$;  and similarly for infima}.   \cmmnt{One important
case is when $\frak B$ is order-dense (313O);  another is in 314Ga
below}.

It will be useful to be able to say `$\frak B$ can be regularly embedded
in $\frak A$' to mean that there is an injective order-continuous
Boolean homomorphism from $\frak B$ to $\frak A$;  that is, that
$\frak B$ is isomorphic to a regularly embedded subalgebra of $\frak A$.
In this form it is obvious\cmmnt{ from 313Ib} that if $\frak C$ can be
regularly embedded in $\frak B$, and $\frak B$ can be regularly embedded
in $\frak A$, then $\frak C$ can be regularly embedded in $\frak A$.

\leader{313O}{Proposition} Let $\frak A$ be a Boolean algebra and
$\frak B$ an order-dense subalgebra of $\frak A$.   Then $\frak B$ is
regularly
embedded in $\frak A$.   In particular, if $B\subseteq\frak B$ and
$c\in\frak B$ then $c=\sup B$ in $\frak B$ iff $c=\sup B$ in $\frak A$.

\proof{ I have to show that the identity homomorphism
$\iota:\frak B\to\frak A$ is order-continuous.   \Quer\ Suppose, if
possible,
otherwise.   By 313L(b-ii), there is a non-empty set $B\subseteq\frak B$
such that $\inf B=0$ in $\frak B$ but $B=\iota[B]$ has a non-zero lower
bound $a\in\frak A$.   In this case, however (because $\frak B$ is
order-dense) there is a non-zero $d\in\frak B$ with $d\Bsubseteq a$, in
which case $d$ is a non-zero lower bound for $B$ in $\frak B$.\ \Bang
}%end of proof of 313O

\leader{313P}{}\cmmnt{ The most important use of these ideas to us
concerns quotient algebras (313Q);  I approach by means of a
superficially more general result.

\medskip

\noindent}{\bf Theorem} Let $\frak A$ and $\frak B$ be Boolean algebras
and $\pi:\frak A\to\frak B$ a Boolean homomorphism
with kernel $I$.

(a)(i) If $\pi$ is order-continuous then $I$ is order-closed.

\quad(ii) If $\pi[\frak A]$ is regularly embedded in $\frak B$ and $I$
is order-closed then $\pi$ is order-continuous.

(b)(i) If $\pi$ is sequentially order-continuous then $I$ is
a $\sigma$-ideal.

\quad(ii) If $\pi[\frak A]$ is regularly embedded in $\frak B$ and $I$
is a $\sigma$-ideal then $\pi$ is sequentially order-continuous.

\proof{{\bf (a)(i)} If $A\subseteq I$ is upwards-directed and
has a supremum $c\in \frak A$, then $\pi c=\sup\pi[A]=0$, so $c\in I$.
As remarked in 313Eb, this shows that $I$ is order-closed.

\medskip

\quad{\bf (ii)} We are supposing that the identity map from
$\pi[\frak A]$
to $\frak B$ is order-continuous, so it will be enough to show that
$\pi$ is order-continuous when regarded as a map from $\frak A$ to
$\pi[\frak A]$.   Suppose that $A\subseteq\frak A$ is non-empty and
downwards-directed and that $\inf A=0$.   \Quer\ Suppose, if possible,
that $0$ is not the greatest lower bound of $\pi[A]$ in $\pi[\frak A]$.
Then there is a
$c\in\frak A$ such that $0\ne\pi c\Bsubseteq\pi a$ for every $a\in A$.
Now

\Centerline{$\pi(c\Bsetminus a)=\pi c\Bsetminus\pi a=0$}

\noindent for every $a\in A$, so $c\Bsetminus a\in I$ for every
$a\in A$.   The set $C=\{c\Bsetminus a:a\in A\}$ is upwards-directed and
has supremum $c$;  because $I$ is order-closed, $c=\sup C\in I$, and
$\pi c=0$, contradicting the specification of $c$.\ \BanG\  Thus
$\inf\pi[A]=0$ in either $\pi[\frak A]$ or $\frak B$.   As $A$ is
arbitrary, $\pi$ is order-continuous, by the criterion (ii) of 313Lb.

\medskip

{\bf (b)} Argue in the same way, replacing each set $A$ by a sequence.
}%end of proof of 313P

\leader{313Q}{Corollary} Let $\frak A$ be a Boolean algebra and $I$ an
ideal of $\frak A$;  write $\pi$ for the canonical map from $\frak A$ to
$\frak A/I$.

(a) $\pi$ is order-continuous iff $I$ is order-closed.

(b) $\pi$ is sequentially order-continuous iff $I$ is a $\sigma$-ideal.

\proof{ $\pi[\frak A]=\frak A/I$ is surely regularly embedded in
$\frak A/I$.
}%end of proof of 313Q

\leader{313R}{}\cmmnt{ For order-continuous homomorphisms, at least,
there is an elegant characterization in terms of Stone spaces.

\medskip

\noindent}{\bf Proposition} Let $\frak A$ and $\frak B$ be Boolean
algebras, and $\pi:\frak A\to\frak B$ a Boolean homomorphism.   Let $Z$
and $W$ be their Stone spaces, and $\phi:W\to Z$ the corresponding
continuous function\cmmnt{ (312Q)}.   Then the following are
equiveridical:

(i) $\pi$ is order-continuous;

(ii) $\phi^{-1}[M]$ is nowhere dense in $W$ for every nowhere dense set
$M\subseteq Z$;

(iii) $\interior\phi[H]\ne\emptyset$ for every non-empty open set
$H\subseteq W$.

\proof{{\bf (a)(i)$\Rightarrow$(iii)} Suppose that $\pi$ is
order-continuous.   \Quer\ Suppose, if possible, that $H\subseteq W$ is
a non-empty open set and $\interior\phi[H]=\emptyset$.   Let
$b\in\frak B\setminus\{0\}$ be such that $\widehat{b}\subseteq H$.
Then $\phi[\widehat{b}]$ has empty interior;  but also it is a closed
set, so its complement is dense.   Set
$A=\{a:a\in\frak A,\,\widehat{a}\cap\phi[\widehat{b}]=\emptyset\}$.
Then $\bigcup_{a\in A}\widehat{a}=Z\setminus\phi[\widehat{b}]$ is a
dense open set, so $\sup A=1$ in $\frak A$ (313Ca).   Because $\pi$ is
order-continuous, $\sup\pi[A]=1$ in $\frak B$ (313L(b-iii)), and there
is an $a\in A$ such that $\pi a\Bcap b\ne 0$.   But this means that
$\widehat{b}\cap\phi^{-1}[\widehat{a}]\ne\emptyset$ and
$\phi[\widehat{b}]\cap\widehat{a}\ne\emptyset$, contrary to the
definition of $A$.\ \Bang

Thus there is no such set $H$, and (iii) is true.

\medskip

{\bf (b)(iii)$\Rightarrow$(ii)} Now assume (iii).   If $M\subseteq Z$ is
nowhere dense, set $N=\phi^{-1}[\overline{M}]$, so that $N\subseteq W$
is a closed set.   If $H=\interior N$, then
$\interior\phi[H]\subseteq\interior\overline{M}=\emptyset$, so (iii)
tells us that $H$ is empty;  thus $N$ and $\phi^{-1}[M]$ are nowhere
dense, as required by (ii).

\medskip

{\bf (c)(ii)$\Rightarrow$(i)} Assume (ii), and let $A\subseteq\frak A$
be a non-empty set such that $\inf A=0$ in $\frak A$.   Then
$M=\bigcap_{a\in A}\widehat{a}$ has empty interior in $Z$ (313Cb), so
(being closed) is nowhere dense, and $\phi^{-1}[M]$ also is nowhere
dense.   If $b\in\frak B\setminus\{0\}$, then

\Centerline{$\widehat{b}\not\subseteq\phi^{-1}[M]
=\bigcap_{a\in A}\phi^{-1}[\widehat{a}]
=\bigcap_{a\in A}\widehat{\pi a}$,}

\noindent so $b$ is not a lower bound for $\pi[A]$.   This shows that
$\inf\pi[A]=0$ in $\frak B$.   As $A$ is arbitrary, $\pi$ is
order-continuous (313L(b-ii)).
}%end of proof of 313R

\leader{313S}{Upper envelopes}\dvAformerly{3{}14V}
{\bf (a)} Let $\frak A$ be a Boolean algebra,
and $\frak C$ a subalgebra of $\frak A$.   For $a\in\frak A$, the
{\bf upper envelope} of $a$ in $\frak C$\cmmnt{,
or {\bf projection} of $a$ on $\frak C$,} is

\Centerline{$\upr(a,\frak C)=\inf\{c:c\in\frak C,\,a\Bsubseteq c\}$}

\noindent if the infimum is defined in $\frak C$.

\cmmnt{\medskip

\noindent{\bf Remark} Note that the infima here are to be taken in
the subalgebra, so that $\upr(a,\frak C)$ will always belong to $\frak C$.
In the great majority of elementary applications, $\frak C$ will be
order-closed in $\frak A$, so that we do not need to distinguish between
infima in $\frak C$ and infima in $\frak A$.   But see 313Yh.}

\spheader 313Sb If $A\subseteq\frak A$ is such that $\upr(a,\frak C)$ is
defined for every $a\in A$, $a_0=\sup A$ is defined in $\frak A$ and
$c_0=\sup_{a\in A}\upr(a,\frak C)$ is defined in $\frak C$, then
$c_0=\upr(a_0,\frak C)$.   \prooflet{\Prf\ If $c\in\frak C$ then

$$\eqalign{c_0\Bsubseteq c
&\iff\upr(a,\frak C)\Bsubseteq c\text{ for every }a\in A\cr
&\iff a\Bsubseteq c\text{ for every }a\in A
\iff a_0\Bsubseteq c.\text{ \Qed}\cr}$$

\noindent}\cmmnt{In particular, }
$\upr(a\Bcup a',\frak C)=\upr(a,\frak C)\Bcup\upr(a',\frak C)$ whenever
the right-hand side is defined.

\spheader 313Sc If $a\in\frak A$ is such that $\upr(a,\frak C)$ is
defined, then $\upr(a\Bcap c,\frak C)=c\Bcap\upr(a,\frak C)$ for every
$c\in\frak C$.   \prooflet{\Prf\ For $c'\in\frak C$,

$$\eqalign{a\Bcap c\Bsubseteq c'
&\iff a\Bsubseteq c'\Bcup(1\Bsetminus c)\cr
&\iff \upr(a,\frak C)\Bsubseteq c'\Bcup(1\Bsetminus c)
\iff c\Bcap\upr(a,\frak C)\Bsubseteq c'.\text{ \Qed}\cr}$$
}%end of prooflet

\exercises{
\leader{313X}{Basic exercises (a)} Use 313C to give alternative proofs
of 313A and 313B.
%313C

\spheader 313Xb Let $P$ be a partially ordered set.   Show
that there is a topology on $P$ for which the closed sets are just the
order-closed sets.
%313D

\spheader 313Xc  Let $P$ be a partially ordered set, $Q\subseteq P$ an
order-closed set, and $R$ a subset of $Q$ which is order-closed in $Q$
when $Q$ is given the partial ordering induced by that of $P$.
Show that $R$ is order-closed in $P$.
%313D

\sqheader 313Xd  Let $\frak A$ be a Boolean algebra.   Suppose that
$1\in I\subseteq\frak A$ and that $a\Bcap b\in I$ for all $a$, $b\in I$.
(i) Let $\frak B$ be the intersection of all those subsets $A$ of
$\frak A$ such that $I\subseteq A$ and $b\Bsetminus a\in A$ whenever
$a$, $b\in A$ and $a\Bsubseteq b$.   Show that $\frak B$ is a subalgebra
of $\frak A$. (ii) Let $\frak B_{\sigma}$ be the intersection of all
those subsets
$A$ of $\frak A$ such that $I\subseteq A$, $b\Bsetminus a\in A$ whenever
$a$, $b\in A$ and $a\Bsubseteq b$ and $\sup_{n\in\Bbb N}b_n\in A$
whenever $\sequencen{b_n}$ is a non-decreasing sequence in $A$ with a
supremum in $\frak A$.   Show that $\frak B_{\sigma}$ is a
$\sigma$-subalgebra of $\frak A$.
(iii) Let $\frak B_{\tau}$ be the intersection of all those subsets $A$
of $\frak A$ such that $I\subseteq A$, $b\Bsetminus a\in A$ whenever
$a$, $b\in A$ and $a\Bsubseteq b$ and $\sup B\in A$ whenever $B$ is a
non-empty upwards-directed subset of $A$ with a supremum in $\frak A$.
Show that $\frak B_{\tau}$ is an order-closed subalgebra of $\frak A$.
(iv) Hence give a proof of 313G not relying on Zorn's Lemma or any other
use of the axiom of choice.
%313G

\spheader 313Xe Let $\frak A$ be a Boolean algebra, and
$\frak B$ a subalgebra of $\frak A$.   Let $\frak B_{\sigma}$ be the
smallest sequentially order-closed subset of $\frak A$ including
$\frak B$.   Show that $\frak B_{\sigma}$ is a subalgebra of $\frak A$.
%313E

\sqheader 313Xf Let $X$ be a set, and $\Cal A$ a subset of $\Cal PX$.
Show that $\Cal A$ is an order-closed subalgebra of $\Cal PX$ iff it is
of the form $\{f^{-1}[F]:F\subseteq Y\}$ for some set $Y$ and function
$f:X\to Y$.
%313G

\spheader 313Xg Let $P$ and $Q$ be partially ordered sets, and
$\phi:P\to Q$ an order-preserving function.   Show that $\phi$ is
sequentially order-continuous iff $\phi^{-1}[C]$ is sequentially
order-closed in $\frak A$ for every sequentially order-closed
$C\subseteq\frak B$.
%313I

\spheader 313Xh For partially ordered sets $P$ and $Q$, let us
call a function $\phi:P\to Q$ {\bf monotonic} if it is {\it either}
order-preserving {\it or} order-reversing.   State and prove definitions
and results corresponding to 313H, 313I and 313Xg for general monotonic
functions.
%313I, 313Xg

\sqheader 313Xi Let $\frak A$ be a Boolean algebra.   Show that
the operations $(a,b)\mapsto a\Bcup b$ and $(a,b)\mapsto a\Bcap b$ are
order-continuous operations from $\frak A\times\frak A$ to $\frak A$,
if we give $\frak A\times\frak A$ the product partial order, saying that
$(a,b)\le (a',b')$ iff $a\Bsubseteq a'$ and $b\Bsubseteq b'$.
%313H

\spheader 313Xj Let $\frak A$ be a Boolean algebra.   Show that if a
subalgebra of $\frak A$ is order-dense then it is dense in the topology
of 313Xb.
%313J

\sqheader 313Xk Let $\frak A$ be a Boolean algebra and
$A\subseteq\frak A$ any disjoint set.   Show that there is a partition
of unity in $\frak A$ including $A$.
%313K

\sqheader 313Xl Let $\frak A$, $\frak B$ be Boolean algebras and
$\pi_1$, $\pi_2:\frak A\to\frak B$ two order-continuous Boolean
homomorphisms.   Show that $\{a:\pi_1a=\pi_2a\}$ is an order-closed
subalgebra of $\frak A$.
%313L

\spheader 313Xm Let $\frak A$ and $\frak B$ be Boolean algebras and
$\pi_1$, $\pi_2:\frak A\to\frak B$ two Boolean homomorphisms.   Suppose
that $\pi_1$ and $\pi_2$ agree on some order-dense subset of $\frak A$,
and that one of them is order-continuous.   Show that they are equal.
\Hint{if $\pi_1$ is order-continuous, $\pi_2a\Bsupseteq\pi_1a$ for every
$a$.}
%313L

\spheader 313Xn Let $\frak A$ and $\frak B$ be Boolean algebras,
$\frak A_0$ an order-dense subalgebra of $\frak A$, and
$\pi:\frak A\to\frak B$ a Boolean homomorphism.   Show that $\pi$ is
order-continuous iff $\pi\restrp\frak A_0:\frak A_0\to\frak B$ is
order-continuous.
%313L

\spheader 313Xo Let $\frak A$ be a Boolean algebra and
$\pi:\frak A\to\frak A$ a Boolean homomorphism with fixed-point
subalgebra $\frak C$ (312K).   (i) Show that if $\pi$ is sequentially
order-continuous then $\frak C$ is a $\sigma$-subalgebra of $\frak A$.
(ii) Show that if $\pi$ is order-continuous then $\frak C$ is
order-closed.
%313M

\sqheader 313Xp Let $\frak A$ be a Boolean algebra.   For
$A\subseteq\frak A$ set $A^{\perp}=\{b:a\Bcap b=0\Forall a\in A\}$.
(i) Show that
$A^{\perp}$ is an order-closed ideal of $\frak A$.   (ii) Show that a
set $A\subseteq\frak A$ is
an order-closed ideal of $\frak A$ iff $A=A^{\perp\perp}$.
(iii) Show that if $I\subseteq\frak A$ is an order-closed ideal then
$\{a^{\ssbullet}:a\in I^{\perp}\}$ is an order-dense ideal in the
quotient algebra $\frak A/I$.
%313Q

\spheader 313Xq Let $\frak A$ and $\frak B$ be Boolean algebras,
with Stone spaces $Z$ and $W$;  let $\pi:\frak A\to\frak B$ be a Boolean
homomorphism, and $\phi:W\to Z$ the corresponding continuous function.
Show that the following are equiveridical:  (i)
$\pi$ is order-continuous;
(ii) $\interior\phi^{-1}[F]=\phi^{-1}[\interior F]$ for every closed
$F\subseteq Z$ (iii) $\overline{\phi^{-1}[G]}=\phi^{-1}[\overline{G}]$
for every open $G\subseteq Z$.
%313R

\spheader 313Xr Let $\frak A$ and $\frak B$ be Boolean algebras,
$\pi:\frak A\to\frak B$ an injective
Boolean homomorphism and $\frak C$ a Boolean
subalgebra of $\frak A$.   Suppose that $a\in\frak A$ is such that
$c=\upr(a,\frak C)$ is defined.   Show that
$\upr(\pi a,\pi[\frak C])$ is defined and equal to $\pi c$.
%313S  \Latereditions

\leader{313Y}{Further exercises (a)} Prove 313A-313C for general Boolean
rings.
%313C

\spheader 313Yb Let $P$ be any partially ordered set, and let
$\frak T$ be the topology of 313Xb.  (i) Show that a sequence
$\sequencen{p_n}$ in $P$ is $\frak T$-convergent to $p\in P$ iff every
subsequence of $\sequencen{p_n}$ has a monotonic sub-subsequence with
supremum or infimum equal to $p$.   (ii) Show that a subset $A$ of $P$
is sequentially order-closed, in the sense of 313Db, iff the
$\frak T$-limit of any $\frak T$-convergent sequence in $A$ belongs to
$A$.   (iii) Suppose that $A$ is an upwards-directed subset of $P$ with
supremum $p_0\in P$.   For $a\in A$ set $F_a=\{p:a\le p\in A\}$, and let
$\Cal F$ be the filter on $P$ generated by $\{F_a:a\in A\}$.   Show that
$\Cal F\to p_0$ for $\frak T$.   (iv) Show that if $Q$ is another
partially ordered set, endowed with a topology $\frak S$ in the same
way, then a monotonic function $\phi:P\to Q$ is order-continuous iff it
is continuous for the topologies $\frak T$ and $\frak S$, and is
sequentially order-continuous iff it is sequentially continuous for
these topologies.
%313Xb, 313D

\spheader 313Yc Let $U$ be a Banach lattice (242G, 354Ab).   Show that
its norm is order-continuous in the sense of 242Yg and 354Dc iff its
restriction to
$\{u:u\ge 0\}$ is order-continuous in the sense of 313Ha.
%313H

\spheader 313Yd\dvAnew{2010} Let $P$ and $Q$ be lattices, and $f:P\to Q$ a
bijective lattice homomorphism.   Show that $f$ is order-continuous.
%313L

\spheader 313Ye Let $\frak A$ and $\frak B$ be Boolean algebras,
with Stone spaces $Z$ and $W$, and $\pi:\frak A\to\frak B$ a Boolean
homomorphism, with associated continuous function $\phi:W\to Z$.   Show
that $\pi$ is sequentially order-continuous iff $\phi^{-1}[M]$ is
nowhere dense for every nowhere dense zero set $M\subseteq Z$.
%313R

\spheader 313Yf Let $\frak A$ and $\frak B$ be Boolean algebras with
Stone spaces $Z$ and $W$ respectively, $\pi:\frak A\to\frak B$ a Boolean
homomorphism and $\phi:W\to Z$ the corresponding continuous function.
Show that $\pi[\frak A]$ is order-dense in $\frak B$ iff $\phi$ is {\bf
irreducible}, that is, $\phi[F]\ne\phi[W]$ for any proper closed subset
$F$ of $W$.
%313R

\spheader 313Yg Let $\frak A$ and $\frak B$ be Boolean algebras with
Stone spaces $Z$ and $W$ respectively, $\pi:\frak A\to\frak B$ a Boolean
homomorphism and $\phi:W\to Z$ the corresponding continuous function.
Show that the following are equiveridical: (i) $\pi$ is injective and
order-continuous;  (ii) for $M\subseteq Z$, $M$ is nowhere dense iff
$\phi^{-1}[M]$ is nowhere dense.
%313R

\spheader 313Yh Let $\frak A$ be a Boolean algebra and $\frak C$ a Boolean
subalgebra of $\frak A$.
Let $\Cal I$ be the set of those $a\in\frak A$ such that the
upper envelope $\upr(a,\frak C)$ is zero.
(i) Show that $\Cal I$ is an ideal in $\frak A$.
(ii) Show that $\frak C$ is regularly embedded in $\frak A$ iff
$\Cal I=\{0\}$.
(iii) Let $\pi:\frak A\to\frak A/\Cal I$ be the canonical map.
Show that $\pi\restrp\frak C$ is injective and order-continuous.
%313S \Latereditions
}%end of exercises

\endnotes{
\Notesheader{313} I give `elementary' proofs of 313A-313B because I
believe that they help to exhibit the relevant aspects of the structure
of Boolean algebras;  but various abbreviations are possible, notably if
we allow ourselves to use the Stone representation (313Xa).   313A and
313Ba-b can be expressed by saying that the Boolean operations $\Bcup$,
$\Bcap$ and $\Bsetminus$ are (separately) order-continuous.   Of course,
$\Bsetminus$ is order-reversing, rather than order-preserving, in the
second variable;  but the natural symmetry in the concept of partial
order means that the ideas behind 313H-313I can be applied equally well
to order-reversing functions (313Xh).  In fact, $\Bcup$ and $\Bcap$ can
be regarded as order-continuous functions on the product space (313Bc-d,
313Xi).   Clearly 313Bc-d can be extended into forms valid for any
finite sequence $A_0,\ldots,A_n$ of subsets of $\frak A$ in place of
$A$, $B$.   But if we seek to go to infinitely many subsets of $\frak A$
we find ourselves saying something new;  see 316G-316J below.

Proposition 313C, and its companions 313R, 313Xq and 313Ye, are worth
studying not only as a useful technique, but also in order to understand
the difference between $\sup A$, where $A$ is a set in a Boolean
algebra, and $\bigcup\Cal A$, where $\Cal A$ is a family of sets.
Somehow $\sup A$ can be larger, and $\inf A$ smaller, than one's first
intuition might suggest, corresponding to the fact that not every subset
of the Stone space corresponds to an element of the Boolean algebra.

I should like to use the words `order-closed' and
`sequentially order-closed' to mean closed, or sequentially closed, for
some more or less canonical topology.   The difficulty is that while a
great many topologies can be defined from a partial order (one is
described in 313Xb and 313Yb, and another in 367Yb and 393L),
none of them has such pre-eminence that
it can be called `the' order-topology.   Accordingly there is a degree
of arbitrariness in the language I use here.   Nevertheless
(sequentially)
order-closed subalgebras and ideals are of such importance that they
seem to deserve a concise denotation.   The same remarks apply to
(sequential) order-continuity.   Concerning the term `order-dense' in
313J, this has little to do with density in any topological
sense, but the word `dense', at least, is established in this context.

With all these definitions, there is a good deal of scope for possible
interrelations.   The most important to us is 313Q, which will be used
repeatedly (typically, with $\frak A$ an algebra of sets), but I think
it is worth having the expanded version in 313P available.

I take the opportunity to present an abstract form of an important lemma
on $\sigma$-algebras generated by families closed under $\cap$ (136B,
313Gb).   This time round I use the Zorn's Lemma argument in the text
and suggest the alternative, `elementary' method in the exercises
(313Xd).   The two methods are opposing extremes in the sense that the
Zorn's Lemma argument looks for maximal subalgebras included in $A$
(which are not unique, and have to be picked out using the axiom of
choice) and the other approach seeks minimal subalgebras including $I$
(which are uniquely defined, and can be described without the axiom of
choice).

Note that the concept of `order-closed algebra of sets' is not
particularly useful;  there are too few order-closed subalgebras of
$\Cal PX$ and they are of too simple a form (313Xf).   It is in abstract
Boolean algebras that the idea becomes important.   In
many of the most
important partially ordered sets of measure theory, the sequentially
order-closed sets are the same as the order-closed sets (see, for
instance, 316Fb below), and most of the important order-closed
subalgebras dealt with in this chapter can be thought of as
$\sigma$-subalgebras which are order-closed because they happen to lie
in the right kind of algebra.   

}%end of notes

\discrpage

