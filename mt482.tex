\frfilename{mt482.tex}
\versiondate{11.5.10/14.3.11}
\copyrightdate{2001}

\def\closure{\mathop{\text{cl}}}
\def\Eta{\text{H}}

\def\chaptername{Gauge integrals}
\def\sectionname{General theory}

\newsection{482}

I turn now to results which can be applied to a wide variety of
tagged-partition structures.   The first step is a `Saks-Henstock'
lemma
(482B), a fundamental property of tagged-partition structures allowing
subdivisions.   In order to relate gauge integrals to the ordinary
integrals treated elsewhere in this treatise, we need to know when
gauge-integrable functions are measurable (482E) and when integrable
functions are gauge-integrable (482F).   There are significant
difficulties when we come to interpret gauge integrals over subspaces,
but I give a partial result in 482G.   482I, 482K and 482M
are gauge-integral versions of the Fundamental Theorem of Calculus,
B.Levi's theorem and Fubini's theorem, while 482H is a limit theorem of a
new kind, corresponding to classical improper integrals.

Henstock's integral (481J-481K) remains the most important example and
the natural test case for the ideas here;  I will give the details in
the next section, and you may wish to take the two sections in
parallel.

\leader{482A}{Lemma} Suppose that $(X,T,\Delta,\frak R)$ is a
tagged-partition structure allowing subdivisions\cmmnt{ (481G)},
witnessed by $\Cal C\subseteq\Cal PX$.

(a) Whenever $\delta\in\Delta$, $\Cal R\in\frak R$ and $E$ belongs to
the subalgebra of $\Cal PX$ generated by $\Cal C$, there is a
$\delta$-fine $\pmb{s}\in T$ such that $W_{\pmb{s}}\subseteq E$ and
$E\setminus W_{\pmb{s}}\in\Cal R$.

(b) Whenever $\delta\in\Delta$, $\Cal R\in\frak R$ and $\pmb{t}\in T$
is $\delta$-fine, there is a $\delta$-fine $\Cal R$-filling 
$\pmb{t}'\in T$ including $\pmb{t}$.

(c) Suppose that $f:X\to\Bbb R$, $\nu:\Cal C\to\Bbb R$,
$\delta\in\Delta$, $\Cal R\in\frak R$ and $\epsilon\ge 0$ are such
that
$|S_{\pmb{t}}(f,\nu)-S_{\pmb{t}'}(f,\nu)|\le\epsilon$ whenever
$\pmb{t}$, $\pmb{t}'\in T$ are $\delta$-fine and $\Cal R$-filling.
Then

\quad(i) $|S_{\pmb{t}}(f,\nu)-S_{\pmb{t}'}(f,\nu)|\le\epsilon$
whenever
$\pmb{t}$, $\pmb{t}'\in T$ are $\delta$-fine and
$W_{\pmb{t}}=W_{\pmb{t}'}$;

\quad(ii) whenever $\pmb{t}\in T$ is $\delta$-fine, and
$\delta'\in\Delta$, there is a $\delta'$-fine $\pmb{s}\in T$ such that
$W_{\pmb{s}}\subseteq W_{\pmb{t}}$ and
$|S_{\pmb{s}}(f,\nu)-S_{\pmb{t}}(f,\nu)|\le\epsilon$.

(d) Suppose that $f:X\to\Bbb R$ and
$\nu:\Cal C\to\Bbb R$ are such that
$I_{\nu}(f)
=\lim_{\pmb{t}\to\Cal F(T,\Delta,\frak R)}S_{\pmb{t}}(f,\nu)$
is defined\cmmnt{, where $\Cal F(T,\Delta,\frak R)$ is the filter
described in 481F}.   Then for any $\epsilon>0$ there is a
$\delta\in\Delta$
such that $S_{\pmb{t}}(f,\nu)\le I_{\nu}(f)+\epsilon$ for every
$\delta$-fine $\pmb{t}\in T$.

(e)\dvAnew{2011} Suppose that $f:X\to\Bbb R$ and
$\nu:\Cal C\to\Bbb R$ are such that $I_{\nu}(f)
=\lim_{\pmb{t}\to\Cal F(T,\Delta,\frak R)}S_{\pmb{t}}(f,\nu)$
is defined.   Then for any $\epsilon>0$ there is a $\delta\in\Delta$ 
such that $|S_{\pmb{t}}(f,\nu)|\le\epsilon$ whenever $\pmb{t}\in T$ is
$\delta$-fine and $W_{\pmb{t}}=\emptyset$.

\proof{{\bf (a)} By 481He, there is a non-increasing sequence
$\sequence{k}{\Cal R_k}$ in $\frak R$ such
that $\bigcup_{i\le k}A_i\in\Cal R$ whenever $A_i\in\Cal R_i$ for
every $i\le k$ and $\langle A_i\rangle_{i\le k}$ is disjoint.   Let
$\Cal C_0$ be a finite subset of $\Cal C$ such that $E$
belongs to the subalgebra of $\Cal PX$ generated by $\Cal C_0$, and
let $\Cal C_1\supseteq\Cal C_0$ be a finite subset of $\Cal C$ such that
$X\setminus W\in\Cal R_0$, where $W=\bigcup\Cal C_1$ (481G(v)).   Then
either $E\subseteq W$ or $E\supseteq X\setminus W$.   In either case,
$E\cap W$ belongs to the ring generated by $\Cal C$, so is expressible
as $\bigcup_{i<n}C_i$ where $\ofamily{i}{n}{C_i}$ is a disjoint family
in $\Cal C$ (481Hd).

For each $i<n$, let $\pmb{s}_i$ be a $\delta$-fine member of $T$ such
that $W_{\pmb{s}_i}\subseteq C_i$ and
$C_i\setminus W_{\pmb{s}_i}\in\Cal R_{i+1}$ (481G(vii)).   Set
$\pmb{s}=\bigcup_{i<n}\pmb{s}_i$.   Because
$\ofamily{i}{n}{W_{\pmb{s}_i}}$ is disjoint and $T$ is a
straightforward
set of tagged partitions, $\pmb{s}\in T$;  $\pmb{s}$ is $\delta$-fine
because every $\pmb{s}_i$ is;
$W_{\pmb{s}}=\bigcup_{i<n}W_{\pmb{s}_i}$
is included in $E$;  and $E\setminus W_{\pmb{s}}$ is either
$\bigcup_{i<n}C_i\setminus W_{\pmb{s}_i}$ or
$(X\setminus W)\cup\bigcup_{i<n}(C_i\setminus W_{\pmb{s}_i})$, and in
either case belongs to $\Cal R$, by the choice of
$\sequencen{\Cal R_n}$.

\medskip

{\bf (b)} Set $E=X\setminus W_{\pmb{t}}$.   By (a), there is a
$\delta$-fine $\pmb{s}\in T$ such that $W_{\pmb{s}}\subseteq E$ and
$E\setminus W_{\pmb{s}}\in\Cal R$.   Set
$\pmb{t}'=\pmb{t}\cup\pmb{s}$;
this works.

\medskip

{\bf (c)(i)} As in (b), there is a $\delta$-fine $\pmb{s}\in T$ such
that $W_{\pmb{s}}\cap W_{\pmb{t}}=\emptyset$ and
$\pmb{t}\cup\pmb{s}$ is $\Cal R$-filling.   Now
$W_{\pmb{t}\cup\pmb{s}}=W_{{\pmb{t}'}\cup\pmb{s}}$,
so ${\pmb{t}'}\cup\pmb{s}$ also is $\Cal R$-filling, and

\Centerline{$|S_{\pmb{t}}(f,\nu)-S_{\pmb{t}'}(f,\nu)|
=|S_{\pmb{t}\cup\pmb{s}}(f,\nu)
  -S_{\pmb{t}'\cup\pmb{s}}(f,\nu)|
\le\epsilon$.}

\medskip

\quad{\bf (ii)} Replacing $\delta'$ by a lower bound of
$\{\delta,\delta'\}$ in $\Delta$ if necessary, we may suppose that
$\delta'\subseteq\delta$.   Enumerate $\pmb{t}$ as
$\ofamily{i}{n}{(x_i,C_i)}$.   Let $\sequence{k}{\Cal R_k}$ be a
sequence in $\frak R$ such that $\bigcup_{i\le k}A_i\in\Cal R$
whenever
$\langle A_i\rangle_{i\le k}$ is disjoint and $A_i\in\Cal R_i$ for
every
$i\le k$.   For each $i<n$,
let $\pmb{s}_i$ be a $\delta'$-fine member of $T$ such that
$W_{\pmb{s}_i}\subseteq C_i$ and
$C_i\setminus W_{\pmb{s}_i}\in\Cal R_{i+1}$, and set
$\pmb{s}=\bigcup_{i<n}\pmb{s}_i$, so that $\pmb{s}\in T$ is
$\delta'$-fine.   By (a), there is a $\delta$-fine $\pmb{u}\in T$ such
that $W_{\pmb{u}}\cap W_{\pmb{t}}=\emptyset$ and
$X\setminus(W_{\pmb{t}}\cup W_{\pmb{u}})\in\Cal R_0$.   Set
$\pmb{t}'=\pmb{t}\cup\pmb{u}$, $\pmb{s}'=\pmb{s}\cup\pmb{u}$;
then $\pmb{t}'$ and $\pmb{s}'$ are $\delta$-fine and
$\Cal R$-filling, because

\Centerline{$X\setminus W_{\pmb{s}'}
=(X\setminus(W_{\pmb{t}}\cup W_{\pmb{u}}))
  \cup\bigcup_{i<n}(C_i\setminus W_{\pmb{s}_i})
\in\Cal R$,}

\noindent by the choice of $\sequence{k}{\Cal R_k}$.   So

\Centerline{$|S_{\pmb{t}}(f,\nu)-S_{\pmb{s}}(f,\nu)|
=|S_{\pmb{t}'}(f,\nu)-S_{\pmb{s}'}(f,\nu)|\le\epsilon$,}

\noindent as required.

\medskip

{\bf (d)} There are $\delta\in\Delta$ and $\Cal R\in\frak R$ such that
$|S_{\pmb{t}}(f,\nu)-I_{\nu}(f)|\le\epsilon$ whenever $\pmb{t}\in T$
is
$\delta$-fine and $\Cal R$-filling.   If $\pmb{t}\in T$ is an
arbitrary
$\delta$-fine tagged partition, there is a $\delta$-fine
$\Cal R$-filling $\pmb{t}'\supseteq\pmb{t}$, by (b), so

\Centerline{$S_{\pmb{t}}(f,\nu)\le S_{\pmb{t}'}(f,\nu)
\le I_{\nu}(f)+\epsilon$,}

\noindent as claimed.

\medskip

{\bf (e)} Let $\delta\in\Delta$, $\Cal R\in\frak R$ be such that
$|S_{\pmb{s}}(f,\nu)-I_{\nu}(f)|\le\bover12\epsilon$ whenever 
$\pmb{s}\in T$ is $\delta$-fine and $\Cal R$-filling.   If $\pmb{t}\in T$
is $\delta$-fine and $W_{\pmb{t}}=\emptyset$, take any
$\delta$-fine $\Cal R$-filling $\pmb{s}\in T$, and consider
$\pmb{s}'=\pmb{s}\setminus\pmb{t}$, $\pmb{s}''=\pmb{s}\cup\pmb{t}$.
Because $W_{\pmb{s}}\cap W_{\pmb{t}}=\emptyset$, both $\pmb{s}'$ and
$\pmb{s}''$ belong to $T$;  both are $\delta$-fine;  and because
$W_{\pmb{s}'}=W_{\pmb{s}''}=W_{\pmb{s}}$, both are $\Cal R$-filling.
So

\Centerline{$|S_{\pmb{t}}(f,\nu)|
=|S_{\pmb{s}''}(f,\nu)-S_{\pmb{s}'}(f,\nu)|
\le|S_{\pmb{s}''}(f,\nu)-I_{\nu}(f)|
+|S_{\pmb{s}'}(f,\nu)-I_{\nu}(f)|
\le\epsilon$,}

\noindent as required.
}%end of proof of 482A

\leader{482B}{Saks-Henstock Lemma} Let $(X,T,\Delta,\frak R)$ be a
tagged-partition structure allowing subdivisions, witnessed by $\Cal
C$,
and $f:X\to\Bbb R$, $\nu:\Cal C\to\Bbb R$ functions such that
$I_{\nu}(f)=\lim_{\pmb{t}\to\Cal F(T,\Delta,\frak
R)}S_{\pmb{t}}(f,\nu)$
is defined.   Let $\Cal E$ be the algebra of subsets of $X$ generated
by
$\Cal C$.   Then there is a unique additive functional
$F:\Cal E\to\Bbb R$ such that for every $\epsilon>0$ there are
$\delta\in\Delta$ and $\Cal R\in\frak R$ such that

\quad($\alpha$) $\sum_{(x,C)\in\pmb{t}}|F(C)-f(x)\nu C|\le\epsilon$
for
every $\delta$-fine $\pmb{t}\in T$,

\quad($\beta$) $|F(E)|\le\epsilon$ whenever $E\in\Cal E\cap\Cal R$.

\noindent Moreover, $F(X)=I_{\nu}(f)$.

\proof{{\bf (a)} For $E\in\Cal E$, write $T_E$ for the set of those
$\pmb{t}\in T$ such that, for every $(x,C)\in\pmb{t}$, either
$C\subseteq E$ or $C\cap E=\emptyset$.   For any $\delta\in\Delta$,
$\Cal R\in\frak R$ and finite $\Cal D\subseteq\Cal E$ there is a
$\delta$-fine $\pmb{t}\in\bigcap_{E\in\Cal D}T_E$ such that
$E\setminus W_{\pmb{t}}\in\Cal R$ for every $E\in\Cal D$.   \Prf\ Let
$\sequencen{\Cal R_n}$ be a sequence in
$\frak R$ such that whenever $A_i\in\Cal R_i$ for $i\le n$ and
$\langle A_i\rangle_{i\le n}$ is disjoint then
$\bigcup_{i\le n}A_i\in\Cal R$ (481He again).   Let $\Cal E_0$ be the
subalgebra of $\Cal E$ generated by $\Cal D$, and enumerate the atoms
of $\Cal E_0$ as $\ofamily{i}{n}{E_i}$.   By 482Aa, there is for each
$i<n$ a $\delta$-fine $\pmb{s}_i\in T$ such that
$W_{\pmb{s}_i}\subseteq E_i$
and $E_i\setminus W_{\pmb{s}_i}\in\Cal R_i$.   Set
$\pmb{t}=\bigcup_{i<n}\pmb{s}_i$.   If $E\in\Cal D$ then
$E=\bigcup_{i\in J}E_i$ for some $J\subseteq n$.    For any
$(x,C)\in\pmb{t}$, there is some $i<n$ such that $C\subseteq E_i$, so
that $C\subseteq E$ if $i\in J$, $C\cap E=\emptyset$ otherwise;  thus
$\pmb{t}\in T_E$.   Moreover,
$E\setminus W_{\pmb{t}}=\bigcup_{i\in J}(E_i\setminus W_{\pmb{s}_i})$
belongs to $\Cal R$.\ \Qed

\medskip

{\bf (b)} We therefore have a filter $\Cal F^*$ on $T$ generated by
sets
of the form

\Centerline{$T_{E\delta\Cal R}
=\{\pmb{t}:\pmb{t}\in T_E$ is $\delta$-fine,
  $E\setminus W_{\pmb{t}}\in\Cal R\}$}

\noindent as $\delta$ runs over $\Delta$, $\Cal R$ runs over $\frak R$
and $E$ runs over $\Cal E$.   For $\pmb{t}\in T$, $E\subseteq X$ set
$\pmb{t}_E=\{(x,C):(x,C)\in\pmb{t},\,C\subseteq E\}$.   Now
$F(E)=\lim_{\pmb{t}\to\Cal F^*}S_{\pmb{t}_E}(f,\nu)$ is defined for
every $E\in\Cal E$.
\Prf\ For any $\epsilon>0$, there are $\delta\in\Delta$,
$\Cal R\in\frak R$ such that
$|I_{\nu}(f)-S_{\pmb{t}}(f,\nu)|\le\epsilon$ for every $\delta$-fine
$\Cal R$-filling $\pmb{t}\in T$.   Let $\Cal R'\in\frak R$ be such
that
$A\cup B\in\Cal R$ for all disjoint $A$, $B\in\Cal R'$.   If
$\pmb{t}$,
$\pmb{t}'$ belong to
$T_{E,\delta,\Cal R'}=T_{X\setminus E,\delta,\Cal R'}$, then set

\Centerline{$\pmb{s}=\{(x,C):(x,C)\in\pmb{t}',\,C\subseteq E\}
  \cup\{(x,C):(x,C)\in\pmb{t},\,C\cap E=\emptyset\}$.}

\noindent Then $\pmb{s}\in T_E$ is $\delta$-fine, and also
$E\setminus W_{\pmb{s}}=E\setminus W_{\pmb{t}'}$,
$(X\setminus E)\setminus W_{\pmb{s}}
=(X\setminus E)\setminus W_{\pmb{t}}$ both belong to $\Cal R'$;  so
their union $X\setminus W_{\pmb{s}}$ belongs to $\Cal R$, and
$\pmb{s}$
is $\Cal R$-filling.   Accordingly

$$\eqalign{|S_{\pmb{t}_E}(f,\nu)-S_{\pmb{t}'_E}(f,\nu)|
&=|S_{\pmb{t}}(f,\nu)-S_{\pmb{s}}(f,\nu)|\cr
&\le|S_{\pmb{t}}(f,\nu)-I_{\nu}(f)|+|I_{\nu}(f)-S_{\pmb{s}}(f,\nu)|
\le 2\epsilon.\cr}$$

\allowmorestretch{468}{
\noindent As $\epsilon$ is arbitrary, this is enough to show that
$\liminf_{\pmb{t}\to\Cal F^*}S_{\pmb{t}_E}(f,\nu)
=\limsup_{\pmb{t}\to\Cal F^*}S_{\pmb{t}_E}(f,\nu)$, so that the limit
$\lim_{\pmb{t}\to\Cal F^*}S_{\pmb{t}_E}(f,\nu)$ is defined (2A3Sf).\
\Qed
}%end of allowmorestretch

\medskip

{\bf (c)} $F(\emptyset)=0$.   \Prf\ Let $\epsilon>0$.   
By 482Ae, there is a
$\delta\in\Delta$ such that $|S_{\pmb{t}}(f,\nu)|\le\epsilon$ whenever
$\pmb{t}\in T$ is $\delta$-fine and $W_{\pmb{t}}=\emptyset$.   Since
$\{\pmb{t}:\pmb{t}$ is $\delta$-fine$\}$ belongs to $\Cal F^*$,

\Centerline{$|F(\emptyset)|
=|\lim_{\pmb{t}\to\Cal F^*}S_{\pmb{t}_{\emptyset}}(f,\nu)|\le\epsilon$;}

\noindent as $\epsilon$ is arbitrary, $F(\emptyset)=0$.\ \Qed

If $E$, $E'\in\Cal E$, then

\Centerline{$S_{\pmb{t}_{E\cup E'}}(f,\nu)+S_{\pmb{t}_{E\cap E'}}(f,\nu)
=S_{\pmb{t}_E}(f,\nu)+S_{\pmb{t}_{E'}}(f,\nu)$}

\noindent for every $\pmb{t}\in T_E\cap T_{E'}$;  as $T_E\cap T_{E'}$
belongs to $\Cal F^*$,

\Centerline{$F(E\cup E')+F(E\cap E')=F(E)+F(E')$.}

\noindent Since $F(\emptyset)=0$, $F(E\cup E')=F(E)+F(E')$ whenever
$E\cap E'=\emptyset$, and $F$ is additive.

\medskip

{\bf (d)} Now suppose that $\epsilon>0$.   Let $\delta\in\Delta$,
$\Cal R^*\in\frak R$ be such that
$|I_{\nu}(f)-S_{\pmb{t}}(f,\nu)|\le\bover14\epsilon$ for every
$\delta$-fine, $\Cal R^*$-filling $\pmb{t}\in T$.   Let
$\Cal R\in\frak R$ be such that $A\cup B\in\Cal R^*$ for all disjoint
$A$, $B\in\Cal R$.

\medskip

\quad{\bf (i)} If $\pmb{t}\in T$ is $\delta$-fine, then
$|F(W_{\pmb{t}})-S_{\pmb{t}}(f,\nu)|\le\bover12\epsilon$.   \Prf\ For
any $\eta>0$, there is a $\delta$-fine $\pmb{s}\in T$ such that

\inset{$|I_{\nu}(f)-S_{\pmb{s}}(f,\nu)|\le\eta$,

for every $(x,C)\in\pmb{s}$, either $C\subseteq W_{\pmb{t}}$ or
$C\cap W_{\pmb{t}}=\emptyset$,

$(X\setminus W_{\pmb{t}})\setminus W_{\pmb{s}}\in\Cal R$,
$W_{\pmb{t}}\setminus W_{\pmb{s}}\in\Cal R$,

$|F(W_{\pmb{t}})-\sum_{(x,C)\in\pmb{s},C\subseteq W_{\pmb{t}}}f(x)\nu
C|
\le\eta$}

\noindent because the set of $\pmb{s}$ with these properties belongs
to
$\Cal F^*$.   Now, setting
$\pmb{s}_1=\{(x,C):(x,C)\in\pmb{s},\,C\subseteq W_{\pmb{t}}\}$ and
$\pmb{t}'=\pmb{t}\cup(\pmb{s}\setminus\pmb{s}_1)$,
$\pmb{t}'$ is $\delta$-fine and $\Cal R^*$-filling, like $\pmb{s}$, so

$$\eqalign{|F(W_{\pmb{t}})-S_{\pmb{t}}(f,\nu)|
&\le|F(W_{\pmb{t}})-S_{\pmb{s}_1}(f,\nu)|
  +|S_{\pmb{s}_1}(f,\nu)-S_{\pmb{t}}(f,\nu)|\cr
&\le\eta+|S_{\pmb{s}}(f,\nu)-S_{\pmb{t}'}(f,\nu)|\cr
&\le\eta+|S_{\pmb{s}}(f,\nu)-I_{\nu}(f)|
  +|I_{\nu}(f)-S_{\pmb{t}'}(f,\nu)|
\le\eta+\Bover12\epsilon.\cr}$$

\noindent As $\eta$ is arbitrary we have the result.\ \Qed

\medskip

\quad{\bf (ii)} So if $\pmb{t}\in T$ is $\delta$-fine,
$\sum_{(x,C)\in\pmb{t}}|F(C)-f(x)\nu C|\le\epsilon$.   \Prf\ Set
$\pmb{t}'=\{(x,C):(x,C)\in\pmb{t},\,F(C)\le f(x)\nu C\}$,
$\pmb{t}''=\pmb{t}\setminus\pmb{t}''$.   Then both $\pmb{t}'$ and
$\pmb{t}''$ are $\delta$-fine, so

$$\eqalign{\sum_{(x,C)\in\pmb{t}}&|F(C)-f(x)\nu C|\cr
&=|\sum_{(x,C)\in\pmb{t}'}F(C)-f(x)\nu C|
  +|\sum_{(x,C)\in\pmb{t}''}F(C)-f(x)\nu C|
\le\epsilon.  \text{ \Qed}\cr}$$

\medskip

\quad{\bf (iii)} If $E\in\Cal E\cap\Cal R$, then
$|F(E)|\le\epsilon$.   \Prf\ Let $\Cal R'\in\frak R$ be such that
$A\cup B\in\Cal R$ whenever $A$, $B\in\Cal R'$ are disjoint.   Let
$\pmb{t}$ be such that

\inset{$\pmb{t}\in T_E$ is $\delta$-fine,

$E\setminus W_{\pmb{t}}$ and
$(X\setminus E)\setminus W_{\pmb{t}}$ both belong to $\Cal R'$,

$|F(E)-S_{\pmb{t}_E}(f,\nu)|\le\bover12\epsilon$;}

\noindent once again, the set of candidates belongs to $\Cal F^*$, so
is
not empty.   Then $\pmb{t}$ and
$\pmb{t}_{X\setminus E}$ are both
$\Cal R^*$-filling and $\delta$-fine, so

\Centerline{$|F(E)|\le\Bover12\epsilon+|S_{\pmb{t}_E}(f,\nu)|
=\Bover12\epsilon
  +|S_{\pmb{t}}(f,\nu)-S_{\pmb{t}_{X\setminus E}}(f,\nu)|
\le\epsilon$. \Qed}

\noindent As $\epsilon$ is arbitrary, this shows that $F$ has all the
required properties.

\medskip

{\bf (e)} I have still to show that $F$ is unique.   Suppose that
$F':\Cal E\to\Bbb R$ is another functional with the same properties,
and
take $E\in\Cal E$ and $\epsilon>0$.   Then there are $\delta$,
$\delta'\in\Delta$ and $\Cal R$, $\Cal R'\in\frak R$ such that

\inset{$\sum_{(x,C)\in\pmb{t}}|F(C)-f(x)\nu C|\le\epsilon$ for every
$\delta$-fine $\pmb{t}\in T$,

$\sum_{(x,C)\in\pmb{t}}|F'(C)-f(x)\nu C|\le\epsilon$ for every
$\delta'$-fine $\pmb{t}\in T$,

$|F(R)|\le\epsilon$ whenever $R\in\Cal E\cap\Cal R$,

$|F'(R)|\le\epsilon$ whenever $R\in\Cal E\cap\Cal R'$.}

\noindent Now taking $\delta''\in\Delta$ such that
$\delta''\subseteq\delta\cap\delta'$, and $\Cal R''\in\frak R$ such
that
$\Cal R''\subseteq\Cal R\cap\Cal R'$, there is a $\delta''$-fine
$\pmb{t}\in T$ such that $E'=W_{\pmb{t}}$ is included in $E$ and
$E\setminus E'\in\Cal R''$.   In this case

$$\eqalignno{|F(E)-F'(E)|
&\le|F(E\setminus E')|+\sum_{(x,C)\in\pmb{t}}|F(C)-F'(C)|
  +|F'(E\setminus E')|\cr
\displaycause{because $F$ and $F'$ are both additive}
&\le 2\epsilon+\sum_{(x,C)\in\pmb{t}}|F(C)-f(x)\nu C|
   +\sum_{(x,C)\in\pmb{t}}|F'(C)-f(x)\nu C|
\le 4\epsilon.\cr}$$

\noindent As $\epsilon$ and $E$ are arbitrary, $F=F'$, as required.

\medskip

{\bf (f)} Finally, to calculate $F(X)$, take any $\epsilon>0$.   Let
$\delta\in\Delta$ and $\Cal R\in\frak R$ be such that
$\sum_{(x,C)\in\pmb{t}}|F(C)-f(x)\nu C|\le\epsilon$ for every
$\delta$-fine $\pmb{t}\in T$
and $|F(E)|\le\epsilon$ whenever $E\in\Cal E\cap\Cal R$.   
Let $\pmb{t}$ be any
$\delta$-fine $\Cal R$-filling member of $T$ such that
$|S_{\pmb{t}}(f,\nu)-I_{\nu}(f)|\le\epsilon$.   Then, because $F$ is
additive,

$$\eqalign{|F(X)-I_{\nu}(f)|
&\le|F(X)-F(W_{\pmb{t}})|
  +|\sum_{(x,C)\in\pmb{t}}F(C)-f(x)\nu C|
+|S_{\pmb{t}}(f,\nu)-I_{\nu}(f)|\cr
&\le 3\epsilon.\cr}$$

\noindent As $\epsilon$ is arbitrary, $F(X)=I_{\nu}(f)$.
}%end of proof of 482B

\leader{482C}{Definition} In the context of 482B, I will call the
function $F$ the {\bf Saks-Henstock indefinite integral} of
$f$\cmmnt{;  of course it depends on the whole structure
$(X,T,\Delta,\frak R,\Cal C,f,\nu)$ and not just on $(X,f,\nu)$}.
\cmmnt{You should {\it not} take it for granted that
$F(E)=I_{\nu}(f\times\chi E)$ (482Ya); but see 482G.
}%end of comment

\leader{482D}{}\cmmnt{ The Saks-Henstock lemma characterizes the
gauge integral, as follows.

\medskip

\noindent}{\bf Theorem} Let $(X,T,\Delta,\frak R)$ be a
tagged-partition structure allowing subdivisions, witnessed by $\Cal C$,
and $\nu:\Cal C\to\Bbb R$ any function.   Let $\Cal E$ be the algebra of
subsets of $X$ generated by $\Cal C$.   If $f:X\to\Bbb R$ is any
function, then the following are equiveridical:

\inset{(i)
$I_{\nu}(f)
=\lim_{\pmb{t}\to\Cal F(T,\Delta,\frak R)}S_{\pmb{t}}(f,\nu)$
is defined in $\Bbb R$;

(ii) there is an additive functional $F:\Cal E\to\Bbb R$ such that

\quad($\alpha$) for every $\epsilon>0$ there is a $\delta\in\Delta$
such that $\sum_{(x,C)\in\pmb{t}}|F(C)-f(x)\mu C|\le\epsilon$ for
every $\delta$-fine $\pmb{t}\in T$,

\quad($\beta$) for every $\epsilon>0$ there is an $\Cal R\in\frak R$
such that $|F(E)|\le\epsilon$ for every $E\in\Cal E\cap\Cal R$;

(iii)\dvAnew{2011} 
there is an additive functional $F:\Cal E\to\Bbb R$ such that

\quad($\alpha$) for every $\epsilon>0$ there is a $\delta\in\Delta$
such that $|F(W_{\pmb{t}})-\sum_{(x,C)\in\pmb{t}}f(x)\mu C|\le\epsilon$ 
for every $\delta$-fine $\pmb{t}\in T$,

\quad($\beta$) for every $\epsilon>0$ there is an $\Cal R\in\frak R$
such that $|F(E)|\le\epsilon$ for every $E\in\Cal E\cap\Cal R$.}

\noindent In this case, $F(X)=I_{\nu}(f)$.

\proof{ (i)$\Rightarrow$(ii) is just 482B above, and (ii)$\Rightarrow$(iii)
is elementary, because $F(W_{\pmb{t}})=\sum_{(x,C)\in\pmb{t}}F(C)$ whenever
$F:\Cal E\to\Bbb R$ is additive and $\pmb{t}\in T$;  so let us assume
(iii) and seek to prove (i).    Given $\epsilon>0$, take $\delta\in\Delta$
and $\Cal R\in\frak R$ such that ($\alpha$) and ($\beta$) of (iii)
are satisfied.   Let $\pmb{t}\in T$ be $\delta$-fine and $\Cal R$-filling.
Then

\Centerline{$|F(X)-S_{\pmb{t}}(f,\mu)|
\le|F(X\setminus W_{\pmb{t}})|
  +|F(W_{\pmb{t}})-\sum_{(x,C)\in\pmb{t}}f(x)\mu C|
\le 2\epsilon$.}

\noindent As $\epsilon$ is arbitrary, $I_{\nu}(f)$ is defined and
equal to $F(X)$.
}%end of proof of 482D

\leader{482E}{Theorem} Let $(X,\rho)$ be a metric space and $\mu$ a
complete locally determined measure on $X$ with domain $\Sigma$.   Let
$\Cal C$, $Q$, $T$, $\Delta$ and $\frak R$ be such that

\inset{(i) $\Cal C\subseteq\Sigma$ and $\mu C$ is finite for every
$C\in\Cal C$;

(ii) $Q\subseteq X\times\Cal C$, and for each $C\in\Cal C$, $(x,C)\in Q$
for almost every $x\in C$;

(iii) $T$ is the straightforward set of tagged partitions generated by
$Q$;

(iv) $\Delta$ is a downwards-directed family of gauges
on $X$ containing all the uniform metric gauges;

(v) if $\delta\in\Delta$, there are a negligible set $F\subseteq X$ and
a neighbourhood gauge $\delta_0$ on $X$ such that
$\delta\supseteq\delta_0\setminus(F\times\Cal PX)$;

(vi) $\frak R$ is a downwards-directed collection of families of
subsets
of $X$ such that whenever $E\in\Sigma$, $\mu E<\infty$ and
$\epsilon>0$,
there is an $\Cal R\in\frak R$ such that $\mu^*(E\cap R)\le\epsilon$
for every $R\in\Cal R$;

(vii) $T$ is compatible with $\Delta$ and $\frak R$.}

\noindent Let $f:X\to\Bbb R$ be any function such that
$I_{\mu}(f)=\lim_{\pmb{t}\to\Cal F(T,\Delta,\frak
R)}S_{\pmb{t}}(f,\mu)$
is defined.   Then $f$ is $\Sigma$-measurable.

\proof{ \Quer\ Suppose, if possible, otherwise.

Because $\mu$ is complete and locally determined, there are a
measurable
set $E$ of non-zero finite measure and $\alpha<\beta$ in $\Bbb R$ such
that

\Centerline{$\mu^*\{x:x\in E,\,f(x)\le\alpha\}
=\mu^*\{x:x\in E,\,f(x)\ge\beta\}=\mu E$}

\noindent (413G).   Let $\epsilon>0$ be such that
$(\beta-\alpha)(\mu E-3\epsilon)>2\epsilon$.
Let $\delta\in\Delta$, $\Cal R\in\frak R$ be such that
$|S_{\pmb{t}}(f,\mu)-I_{\nu}(f)|\le\epsilon$ whenever $\pmb{t}\in T$ is
$\delta$-fine and $\Cal R$-filling.   By (v), there are a negligible set
$F\subseteq X$ and a family $\family{x}{X}{G_x}$
of open sets such that $x\in G_x$ for every $x\in X$ and
$\delta\supseteq\{(x,C):x\in X\setminus F$, $C\subseteq G_x\}$.
For $m\ge 1$, set

\Centerline{$A_m
=\{x:x\in E\setminus F,\,f(x)\le\alpha,\,U_{1/m}(x)\subseteq G_x\}$,}

\noindent writing $U_{1/m}(x)$ for $\{y:\rho(y,x)<\bover1m\}$,

\Centerline{$B_m
=\{x:x\in E\setminus F,\,f(x)\ge\beta,\,U_{1/m}(x)\subseteq G_x\}$.}

\noindent Then there is some $m\ge 1$ such that
$\mu^*A_m\ge\mu E-\epsilon$ and $\mu^*B_m\ge\mu E-\epsilon$.   By (iv),
there is a $\delta'\in\Delta$ such that

\Centerline{$\delta'
\subseteq\delta\cap\{(x,C):x\in C\subseteq U_{1/3m}(x)\}$.}

\noindent By (vi), there is an $\Cal R'\in\frak R$
such that $\Cal R'\subseteq\Cal R$
and $\mu^*(R\cap E)\le\epsilon$ for every $R\in\Cal R'$.

Let $\pmb{t}$ be any $\delta'$-fine
$\Cal R'$-filling member of $T$.   Enumerate $\pmb{t}$ as
$\ofamily{i}{n}{(x_i,C_i)}$.   Set

\Centerline{$J=\{i:i<n,\,C_i\cap A_m$ is negligible$\}$,
\quad$J'=\{i:i<n,\,C_i\cap B_m$ is negligible$\}$.}

\noindent Then

\Centerline{$\mu(E\cap\bigcup_{i\in J}C_i)\le\mu_*(E\setminus A_m)
=\mu E-\mu^*(E\cap A_m)\le\epsilon$,}

\noindent and similarly $\mu(E\cap\bigcup_{i\in J'}C_i)\le\epsilon$.
Also, because $X\setminus\bigcup_{i<n}C_i=X\setminus W_{\pmb{t}}$
belongs to $\Cal R'$,
$\mu(E\setminus\bigcup_{i<n}C_i)\le\epsilon$.   So, setting
$K=n\setminus(J\cup J')$, $\sum_{i\in K}\mu C_i\ge\mu E-3\epsilon$.

For $i\in K$, $\mu^*(C_i\cap A_m)>0$, while
$\{x:x\in C_i,\,(x,C_i)\notin Q\}$ is negligible, by (ii), so we can find
$x'_i\in C_i\cap A_m$ such that $(x'_i,C_i)\in Q$;  similarly, there
is an $x_i''\in C_i\cap B_m$ such that $(x_i'',C_i)\in Q$.   For other
$i<n$, set $x'_i=x_i''=x_i$.  Now $\pmb{s}=\{(x_i',C_i):i<n\}$ and
$\pmb{s}'=\{(x_i'',C_i):i<n\}$ belong to $T$.   Of course they are
$\Cal R'$-filling, therefore $\Cal R$-filling, because $\pmb{t}$ is.
We also see that, because
$(x_i,C_i)\in\delta'$, the diameter of $C_i$ is at
most $\bover{2}{3m}$ for each $i<n$, so that $C_i\subseteq G_{x'_i}$;  as
also $x'_i\in A_m\subseteq X\setminus F$, $(x'_i,C_i)\in\delta$,
for each $i\in K$.
But since surely $(x_i,C_i)\in\delta'\subseteq\delta$ for
$i\in n\setminus K$, this means that $\pmb{s}$ is $\delta$-fine.
Similarly, $\pmb{s}'$ is $\delta$-fine.

We must therefore have

\Centerline{$|S_{\pmb{s}'}(f,\mu)-S_{\pmb{s}}(f,\mu)|
\le|S_{\pmb{s}'}(f,\mu)-I_{\mu}(f)|+|S_{\pmb{s}}(f,\mu)-I_{\mu}(f)|
\le 2\epsilon$.}

\noindent But

$$\eqalign{S_{\pmb{s}'}(f,\mu)-S_{\pmb{s}}(f,\mu)
&=\sum_{i\in K}(f(x_i'')-f(x_i'))\mu C_i\cr
&\ge(\beta-\alpha)\sum_{i\in K}\mu C_i
\ge(\beta-\alpha)(\mu E-3\epsilon)
>2\epsilon\cr}$$

\noindent by the choice of $\epsilon$.\ \Bang

So we have the result.
}%end of proof of 482E

\leader{482F}{Proposition} Let $X$, $\Sigma$, $\mu$, $\frak T$, $T$,
$\Delta$ and $\frak R$ be such that

\inset{(i) $(X,\Sigma,\mu)$ is a measure space;

(ii) $\frak T$ is a topology on $X$ such that $\mu$ is inner regular
with respect to the closed sets and outer regular with respect to the
open sets;

(iii) $T\subseteq[X\times\Sigma]^{<\omega}$ is a set of
tagged partitions such that $C\cap C'$ is empty whenever $(x,C)$,
$(x',C')$ are distinct members of any $\pmb{t}\in T$;

(iv) $\Delta$ is a set of gauges on $X$ containing every neighbourhood
gauge on $X$;

(v) $\frak R$ is a collection of families of subsets of $X$ such that
whenever $\mu E<\infty$ and $\epsilon>0$ there is an $\Cal R\in\frak
R$
such that $\mu^*(E\cap R)\le\epsilon$ for every $R\in\Cal R$;

(vi) $T$ is compatible with $\Delta$ and $\frak R$.}

\noindent Then
$I_{\mu}(f)=\lim_{\pmb{t}\to\Cal F(T,\Delta,\frak
R)}S_{\pmb{t}}(f,\mu)$
is defined and equal to $\int fd\mu$ for every $\mu$-integrable
function
$f:X\to\Bbb R$.

\proof{{\bf (a)} It is worth noting straight away that we can replace
$(X,\Sigma,\mu)$ by its completion $(X,\hat\Sigma,\hat\mu)$.   \Prf\
We need to check that $\hat\mu$ is inner and outer regular.   But inner
regularity is 412Ha, and outer regularity is equally elementary:  if
$\hat\mu E<\gamma$, there is an $E'\in\Sigma$ such that
$E\subseteq E'$
and $\mu E'=\hat\mu E$ (212C), and now there is an open set
$G\in\Sigma$
such that $E'\subseteq G$ and $\mu G\le\gamma$, so that $E\subseteq G$
and $\hat\mu G\le\gamma$.   Since we are not changing $T$ or $\Delta$
or
$\frak R$, $I_{\hat\mu}(f)=I_{\mu}(f)$ if either is defined;  while
also
$\int fd\mu=\int fd\hat\mu$ if either is defined, by 212Fb.\ \Qed

So let us suppose that $\mu$ is actually complete.

\medskip

{\bf (b)} In this case, $f$ is measurable.   Suppose to begin with
that
it is non-negative.   Let $\epsilon>0$.   For $m\in\Bbb Z$, set
$E_m=\{x:x\in X,\,(1+\epsilon)^m\le f(x)<(1+\epsilon)^{m+1}\}$.   Then
$E_m$ is measurable and has finite measure, so there is a measurable
open set $G_m\supseteq E_m$ such that
$(1+\epsilon)^{m+1}\mu(G_m\setminus E_m)\le 2^{-|m|}\epsilon$.

Take a set $H_0$ of finite measure and $\eta_0>0$ such that
$\int_Efd\mu\le\epsilon$ whenever $E\in\Sigma$ and
$\mu(E\cap H_0)\le 2\eta_0$ (225A);  replacing $H_0$ by
$\{x:x\in H_0,\,f(x)>0\}$ if necessary, we may suppose that
$H_0\subseteq\bigcup_{m\in\Bbb Z}E_m$.   Let $F\subseteq H_0$ be a
closed set such that $\mu(H_0\setminus F)\le\eta_0$.

Define $\family{x}{X}{V_x}$ by setting $V_x=G_m$ if $m\in\Bbb Z$ and
$x\in E_m$, $V_x=X\setminus F$ if $f(x)=0$.   Let $\delta\in\Delta$ be
the corresponding neighbourhood gauge
$\{(x,C):x\in X,\,C\subseteq V_x\}$.
Let $\Cal R\in\frak R$ be such that $\mu^*(R\cap H_0)\le\eta_0$ for
every $R\in\Cal R$.

Suppose that $\pmb{t}$ is any $\delta$-fine $\Cal R$-filling member of
$T$.   Enumerate $\pmb{t}$ as $\ofamily{i}{n}{(x_i,C_i)}$.   For each
$m\in\Bbb Z$, set $J_m=\{i:i<n,\,x_i\in E_m\}$.   Then
$C_i\subseteq V_{x_i}\subseteq G_m$ for every $i\in J_m$, so

$$\eqalign{S_{\pmb{t}}(f,\mu)
&=\sum_{i<n}f(x_i)\mu C_i
=\sum_{m\in\Bbb Z}\sum_{i\in J_m}f(x_i)\mu C_i
\le\sum_{m\in\Bbb Z}(1+\epsilon)^{m+1}\mu G_m\cr
&\le(1+\epsilon)\sum_{m\in\Bbb Z}(1+\epsilon)^m\mu E_m
   +\sum_{m\in\Bbb Z}(1+\epsilon)^{m+1}\mu(G_m\setminus E_m)\cr
&\le(1+\epsilon)\int fd\mu+\sum_{m\in\Bbb Z}2^{-|m|}\epsilon
=(1+\epsilon)\int fd\mu+3\epsilon.\cr}$$

\noindent On the other hand, set $F'=F\cap\bigcup_{i<n}C_i$.   Because
$X\setminus\bigcup_{i<n}C_i\in\Cal R$,
$\mu(H_0\setminus F')\le 2\eta_0$, and

$$\eqalign{S_{\pmb{t}}(f,\mu)
&\ge\sum_{m\in\Bbb Z}\sum_{i\in J_m}f(x_i)\mu(C_i\cap F)\cr
&\ge\Bover1{1+\epsilon}\sum_{m\in\Bbb Z}\sum_{i\in J_m}
  (1+\epsilon)^{m+1}\mu(C_i\cap F)\cr
&\ge\Bover1{1+\epsilon}\int_{F'}fd\mu
\ge\Bover1{1+\epsilon}(\int fd\mu-\epsilon).\cr}$$

What this means is that

\Centerline{$\{\pmb{t}:\Bover1{1+\epsilon}(\biggerint fd\mu-\epsilon)
\le S_{\pmb{t}}(f,\nu)
\le(1+\epsilon)\int fd\mu+3\epsilon\}$}

\noindent belongs to $\Cal F(T,\Delta,\frak R)$, for any $\epsilon>0$.
So $\lim_{\pmb{t}\to\Cal F(T,\Delta,\frak R)}S_{\pmb{t}}(f,\mu)$ is
defined and equal to $\int fd\mu$.

\medskip

{\bf (c)} In general, $f$ is expressible as $f^+-f^-$ where $f^+$ and
$f^-$ are non-negative integrable functions, so

\Centerline{$I_{\nu}(f)=I_{\nu}(f^+)-I_{\nu}(f^-)=\int fd\mu$}

\noindent by 481Ca.
}%end of proof of 482F

\leader{482G}{Proposition} Let $(X,T,\Delta,\frak R)$ be a
tagged-partition structure allowing subdivisions, witnessed by $\Cal C$.
Suppose that

\inset{(i) $\frak T$ is a topology on $X$, and $\Delta$ is the set of
neighbourhood gauges on $X$;

(ii) $\nu:\Cal C\to\Bbb R$ is a function which is additive in the sense that if 
$C_0,\ldots,C_n\in\Cal C$ are disjoint and have union $C\in\Cal C$, then 
$\nu C=\sum_{i=0}^n\nu C_i$;

(iii) whenever $E\in\Cal C$ and $\epsilon>0$, there are
closed sets $F\subseteq E$, $F'\subseteq X\setminus E$ such that
$\sum_{(x,C)\in\pmb{t}}|\nu C|\le\epsilon$ whenever $\pmb{t}\in T$ and
$W_{\pmb{t}}\cap(F\cup F')=\emptyset$;

(iv) for every $E\in\Cal C$ and $x\in X$ there is a neighbourhood $G$ of
$x$ such that if $C\in\Cal C$, $C\subseteq G$ and $\{(x,C)\}\in T$, there
is a finite partition $\Cal D$ of $C$ into members of $\Cal C$, each either
included in $E$ or disjoint from $E$, such that
$\{(x,D)\}\in T$ for every $D\in\Cal D$;

(v) for every $C\in\Cal C$ and $\Cal R\in\frak R$, there is an 
$\Cal R'\in\frak R$ such that $C\cap A\in\Cal R$ whenever $A\in\Cal R'$.}

\noindent Let $f:X\to\Bbb R$ be a function such that
$I_{\nu}(f)=\lim_{\pmb{t}\to\Cal F(T,\Delta,\frak R)}S_{\pmb{t}}(f,\nu)$
is defined.   Let $\Cal E$ be the algebra of subsets of $X$ generated
by $\Cal C$, and $F:\Cal E\to\Bbb R$ the Saks-Henstock indefinite
integral of $f$.   Then
$I_{\nu}(f\times\chi E)$ is defined and equal to $F(E)$ for every
$E\in\Cal E$.

\proof{{\bf (a)} Because both $F$ and $I_{\nu}$ are additive, and
$F(X)=I_{\nu}(f)$, and either $E$ or its complement is a finite
disjoint union of members of $\Cal C$ (see 481Hd), 
it is enough to consider the case in which
$E\in\Cal C$.  

\medskip

{\bf (b)} Let $\epsilon>0$.   For each $x\in X$ let $G_x$ be an
open set containing $x$ such that whenever
$C\in\Cal C$, $C\subseteq G$ and $\{(x,C)\}\in T$, there
is a finite partition $\Cal D$ of $C$ into members of $\Cal C$ such that
$\{(x,D)\}\in T$ for every $D\in\Cal D$ and every member of $\Cal D$ is
either included in $E$ or disjoint from $E$.
For each $n\in\Bbb N$, let $F_n\subseteq E$, $F'_n\subseteq X\setminus E$
be closed sets such that 
$\sum_{(x,C)\in\pmb{t}}|\nu C|\le\Bover{2^{-n}\epsilon}{n+1}$ 
whenever $\pmb{t}\in T$ and
$W_{\pmb{t}}\cap(F_n\cup F_n')=\emptyset$;  now define $G'_x$, for 
$x\in X$, by saying that

$$\eqalign{G'_x
&=G_x\setminus F'_n\text{ if }x\in E\text{ and }n\le|f(x)|<n+1,\cr
&=G_x\setminus F_n\text{ if }x\in X\setminus E
   \text{ and }n\le|f(x)|<n+1.\cr}$$

\noindent Let $\delta_0\in\Delta$ be the neighbourhood gauge defined by
the family $\family{x}{X}{G'_x}$.   Let $\delta\in\Delta$ and
$\Cal R_1\in\frak R$ be such that $\delta\subseteq\delta_0$,
$\sum_{(x,C)\in\pmb{t}}|F(C)-f(x)\nu C|\le\epsilon$ for every
$\delta$-fine $\pmb{t}\in T$, and $|F(E)|\le\epsilon$ for every
$E\in\Cal E\cap\Cal R_1$.   Let
$\Cal R\in\frak R$ be such that $R\cap E\in\Cal R_1$ whenever
$R\in\Cal R$.

\medskip

{\bf (c)} As in the proof of 482B,
let $T_E$ be the set of those $\pmb{t}\in T$ such that, for
each $(x,C)\in\pmb{t}$, either $C\subseteq E$ or $C\cap E=\emptyset$.
The key to the proof is the following fact:  if $\pmb{t}\in T$ is
$\delta$-fine, then there is a $\delta$-fine
$\pmb{s}\in T_E$ such that $W_{\pmb{s}}=W_{\pmb{t}}$ and
$S_{\pmb{s}}(g,\nu)=S_{\pmb{t}}(g,\nu)$ for every $g:X\to\Bbb R$.
\Prf\ For each $(x,C)\in\pmb{t}$, we know that 
$C\subseteq G'_x\subseteq G_x$, because
$\delta\subseteq\delta_0$.   Let $\Cal D_{(x,C)}$ be a 
finite partition of
$C$ into members of $\Cal C$, each either included in $E$ or disjoint from
$E$, such that $\{(x,D)\}\in T$ for every $D\in\Cal D_{(x,C)}$.   
Then $\pmb{s}=\{(x,D):(x,C)\in\pmb{t}$, $D\in\Cal D_{(x,C)}\}$ belongs to
$T_E$.   Because
$\delta$ is a neighbourhood gauge, $(x,D)\in\delta$ whenever
$(x,C)\in\pmb{t}$ and $D\in\Cal D_{(x,C)}$, so $\pmb{s}$ is $\delta$-fine.

If $g:X\to\Bbb R$ is any function,

$$\eqalignno{S_{\pmb{s}}(g,\nu)
&=\sum_{(x,C)\in\pmb{t}}\sum_{D\in\Cal D_{(x,C)}}g(x)\nu D\cr
&=\sum_{(x,C)\in\pmb{t}}g(x)\sum_{D\in\Cal D_{(x,C)}}\nu D
=\sum_{(x,C)\in\pmb{t}}g(x)\nu C\cr
\displaycause{because $\nu$ is additive}
&=S_{\pmb{t}}(g,\nu).  \text{ \Qed}\cr}$$

\medskip

{\bf (d)} Now suppose that $\pmb{t}\in T$ is $\delta$-fine and 
$\Cal R$-filling.   Let $\pmb{s}\in T_E$ be as in (c), and set

\Centerline{$\pmb{s}^*=\{(x,D):(x,D)\in\pmb{s}$, $x\in E$, 
  $D\subseteq E\}$,}
  
\Centerline{$\pmb{s}'=\{(x,D):(x,D)\in\pmb{s}$, $x\notin E$, 
  $D\subseteq E\}$,}

\Centerline{$\pmb{s}''=\{(x,D):(x,D)\in\pmb{s}$, $x\in E$, 
  $D\cap E=\emptyset\}$.}

\noindent Because $\pmb{s}\in T_E$, 

\Centerline{$W_{\pmb{s}^*\cup\pmb{s}'}
=E\cap W_{\pmb{s}}=E\cap W_{\pmb{t}}$}

\noindent and
$E\setminus W_{\pmb{s}^*\cup\pmb{s}'}=E\setminus W_{\pmb{t}}$ belongs to 
$\Cal R_1$, by the choice of $\Cal R$.   Accordingly

\Centerline{$|F(E)-S_{\pmb{s}^*\cup\pmb{s}'}(f,\nu)|
\le|F(E)-F(W_{\pmb{s}^*\cup\pmb{s}'})|
   +|F(W_{\pmb{s}^*\cup\pmb{s}'})-S_{\pmb{s}^*\cup\pmb{s}'}(f,\nu)|
\le 2\epsilon$}

\noindent because $\pmb{s}^*\cup\pmb{s}'\subseteq\pmb{s}$ is $\delta$-fine.

For $n\in\Bbb N$ set

\Centerline{$\pmb{s}'_n=\{(x,D):(x,D)\in\pmb{s}'$, $n\le|f(x)|<n+1\}$,}

\Centerline{$\pmb{s}''_n=\{(x,D):(x,D)\in\pmb{s}''$, $n\le|f(x)|<n+1\}$.}

\noindent Then $W_{\pmb{s}'_n}\subseteq E\setminus F_n$.
\Prf\ If $(x,D)\in\pmb{s}'_n$, there is a $C\in\Cal C$ such that 
$D\subseteq E\cap C$ and $(x,C)\in\pmb{t}$, while $x\notin E$,
so that $C\subseteq G'_x$
and $C\cap F_n=\emptyset$.\ \QeD\  Similarly, 
$W_{\pmb{s}''_n}\subseteq(X\setminus E)\setminus F'_n$.   Thus
$W_{\pmb{s}'_n\cup\pmb{s}''_n}$ is disjoint from $F_n\cup F_n'$ and

$$\eqalign{|S_{\pmb{s}'_n}(f,\nu)-S_{\pmb{s}''_n}(f,\nu)|
&=|\sum_{(x,D)\in\pmb{s}'_n}f(x_i)\nu D
  -\sum_{(x,D)\in\pmb{s}''_n}f(x_i)\nu D|\cr
&\le\sum_{(x,D)\in\pmb{s}'_n\cup\pmb{s}''_n}|f(x_i)||\nu D|\cr
&\le(n+1)\sum_{(x,D)\in\pmb{s}'_n\cup\pmb{s}''_n}|\nu D|
\le 2^{-n}\epsilon\cr}$$

\noindent by the choice of $F_n$ and $F'_n$.   	

Consequently, 

$$\eqalignno{|F(E)-S_{\pmb{t}}(f\times\chi E,\nu)|
&=|F(E)-S_{\pmb{s}}(f\times\chi E,\nu)|
=|F(E)-S_{\pmb{s}^*\cup\pmb{s}''}(f,\nu)|\cr
\displaycause{because 
$\pmb{s}^*\cup\pmb{s}''=\{(x,D):(x,D)\in\pmb{s}$, $x\in E\}$}
&\le|F(E)-S_{\pmb{s}^*\cup\pmb{s}'}(f,\nu)|
   +|S_{\pmb{s}'}(f,\nu)-S_{\pmb{s}''}(f,\nu)|\cr
\displaycause{because $\pmb{s}^*$, $\pmb{s}'$ and $\pmb{s}''$ are disjoint
subsets of $\pmb{s}$}   
&\le 2\epsilon
   +|\sum_{n=0}^{\infty}S_{\pmb{s}'_n}(f,\nu)
       -\sum_{n=0}^{\infty}S_{\pmb{s}''_n}(f,\nu)|\cr
\displaycause{the infinite
sums are well-defined because $\pmb{s}$ is finite, so that
all but finitely many terms are zero}
&\le 2\epsilon
   +\sum_{n=0}^{\infty}|S_{\pmb{s}'_n}(f,\nu)
       -S_{\pmb{s}''_n}(f,\nu)|\cr
&\le 2\epsilon
   +\sum_{n=0}^{\infty}2^{-n}\epsilon
=4\epsilon.\cr}$$

\noindent As $\epsilon$ is arbitrary, $I_{\nu}(f\times\chi E)$ is defined
and equal to $F(E)$, as required.
}%end of proof of 482G

\leader{482H}{Proposition} Suppose that $X$, $\frak T$, $\Cal C$, $\nu$,
$T$, $\Delta$ and $\frak R$ satisfy the conditions (i)-(v) of 482G, and
that $f:X\to\Bbb R$, $\sequencen{H_n}$, $H$ and $\gamma$ are such that

\inset{(vi) $\sequencen{H_n}$ is a sequence of open subsets of $X$
with union $H$,

(vii) $I_{\nu}(f\times\chi H_n)$ is defined for every $n\in\Bbb N$,

(viii) $\lim_{\pmb{t}\to\Cal F(T,\Delta,\frak R)}
I_{\nu}(f\times\chi W_{\pmb{t}\restr H})$ is defined and
equal to $\gamma$,}

\noindent where $\pmb{t}\restr H=\{(x,C):(x,C)\in\pmb{t}$, $x\in H\}$
for $\pmb{t}\in T$.   Then $I_{\nu}(f\times\chi H)$ is defined and
equal to $\gamma$.

\proof{ Let $\epsilon>0$.   For each $n\in\Bbb N$, let $F_n$ be the
Saks-Henstock indefinite integral of $f\times\chi H_n$.   Let
$\delta_n\in\Delta$ be such that

$$\eqalign{2^{-n}\epsilon
&\ge\sum_{(x,C)\in\pmb{s}}|F_n(C)-(f\times\chi H_n)(x)\nu C|\cr
&\ge|F_n(W_{\pmb{s}})-S_{\pmb{s}}(f\times\chi H_n,\nu)|\cr}$$

\noindent whenever $\pmb{s}\in T$ is $\delta_n$-fine.   Set

$$\eqalign{\tilde\delta&=\{(x,A):x\in X\setminus H,\,A\subseteq X\}\cr
&\mskip100mu
  \cup\bigcup_{n\in\Bbb N}\{(x,A):x\in H_n\setminus\bigcup_{i<n}H_i,\,
  A\subseteq H_n,\,(x,A)\in\delta_n\},\cr}$$

\noindent so that $\tilde\delta\in\Delta$.   Note that if $x\in H$ and
$C\in\Cal C$ and $(x,C)\in\tilde\delta$, then there is some $n\in\Bbb
N$
such that $x\in H_n$ and $C\subseteq H_n$, so that

\Centerline{$I_{\nu}(f\times\chi C)
=I_{\nu}((f\times\chi H_n)\times\chi C)=F_n(C)$}

\noindent is defined, by 482G;  this means that
$I_{\nu}(f\times\chi W_{\pmb{t}\restr H})$ will be defined for every
$\tilde\delta$-fine $\pmb{t}\in T$.
Let $\delta\in\Delta$, $\Cal R\in\frak R$ be
such that $|\gamma-I_{\nu}(f\times\chi W_{\pmb{t}\restr
H})|\le\epsilon$
whenever $\pmb{t}\in T$ is $\delta$-fine and $\Cal R$-filling.

Let $\pmb{t}\in T$ be
$(\delta\cap\tilde\delta)$-fine and $\Cal R$-filling.   For
$n\in\Bbb N$, set
$\pmb{t}_n=\{(x,C):(x,C)\in\pmb{t},\,x\in H_n\setminus\bigcup_{i<n}H_i\}$.
Then $\pmb{t}\restr H=\bigcup_{n\in\Bbb N}\pmb{t}_n$, and $\pmb{t}_n$ is
$\delta_n$-fine and $W_{\pmb{t}_n}\subseteq H_n$ for every $n$.   So

$$\eqalignno{|\gamma-S_{\pmb{t}}(f\times\chi H,\nu)|
&=|\gamma-\sum_{n=0}^{\infty}S_{\pmb{t}_n}(f\times\chi H_n,\nu)|\cr
&\le|\gamma-I_{\nu}(f\times\chi W_{\pmb{t}\restr H})|
  +\sum_{n=0}^{\infty}|I_{\nu}(f\times\chi W_{\pmb{t}_n})
       -S_{\pmb{t}_n}(f\times\chi H_n,\nu)|\cr
\displaycause{note that $\pmb{t}_n=\emptyset$ for all but finitely
many $n$, so that $I_{\nu}(f\times\chi W_{\pmb{t}\restr H})
=\sum_{n=0}^{\infty}I_{\nu}(f\times\chi W_{\pmb{t}_n})$}
&\le\epsilon
  +\sum_{n=0}^{\infty}|I_{\nu}(f\times\chi H_n\times\chi W_{\pmb{t}_n})
  -S_{\pmb{t}_n}(f\times\chi H_n,\nu)|\cr
\displaycause{because $\pmb{t}$ is $\delta$-fine and $\Cal R$-filling,
while $W_{\pmb{t}_n}\subseteq H_n$ for each $n$}
&=\epsilon+\sum_{n=0}^{\infty}|F_n(W_{\pmb{t}_n})
  -S_{\pmb{t}_n}(f\times\chi H_n,\nu)|\cr
\displaycause{by 482G}
&\le\epsilon+\sum_{n=0}^{\infty}2^{-n}\epsilon\cr
\displaycause{because every $\pmb{t}_n$ is $\delta_n$-fine}
&=3\epsilon.\cr}$$

\noindent As $\epsilon$ is arbitrary, $\gamma=I_{\nu}(f\times\chi H)$,
as claimed.
}%end of proof of 482H

\cmmnt{\medskip

\noindent{\bf Remark} For applications of this result see 483Bd and
483N.
}%end of comment

\leader{482I}{Integrating a \dvrocolon{derivative}}\cmmnt{ As will
appear in the next two sections, the real strength of gauge integrals is
in their power to integrate derivatives.   I give an elementary general
expression of this fact.   In the formulae below, we can think of $f$ as
a `derivative' of $F$ if $\nu=\theta$ is strictly positive and additive
and we rephrase condition (iii) as

\Centerline{`$\lim_{C\to\Cal G_x}\bover{F(C)}{\nu C}=f(x)$',}

\noindent where $\Cal G_x$ is the filter on $\Cal C$ generated by the
sets $\{C:(x,C)\in\delta\}$ as $\delta$ runs over $\Delta$.

\medskip

\noindent}{\bf Theorem} Let $X$, $\Cal C\subseteq\Cal PX$,
$\Delta\subseteq\Cal P(X\times\Cal PX)$,
$\frak R\subseteq\Cal P\Cal PX$, $T\subseteq[X\times\Cal
C]^{<\omega}$,
$f:X\to\Bbb R$, $\nu:\Cal C\to\Bbb R$, $F:\Cal C\to\Bbb R$,
$\theta:\Cal C\to[0,1]$ and $\gamma\in\Bbb R$ be such that

\inset{(i) $T$ is a straightforward set of tagged partitions which is
compatible with $\Delta$ and $\frak R$,

(ii) $\Delta$ is a full set of gauges on $X$,

(iii) for every $x\in X$ and $\epsilon>0$ there is a $\delta\in\Delta$
such that $|f(x)\nu C-F(C)|\le\epsilon\theta C$ whenever
$(x,C)\in\delta$,

(iv) $\sum_{i=0}^n\theta C_i\le 1$ whenever $C_0,\ldots,C_n\in\Cal C$
are disjoint,

(v) for every $\epsilon>0$ there is an $\Cal R\in\frak R$ such that
$|\gamma-\sum_{C\in\Cal C_0}F(C)|\le\epsilon$ whenever
$\Cal C_0\subseteq\Cal C$ is a finite disjoint set and
$X\setminus\bigcup\Cal C_0\in\Cal R$.}

\noindent Then
$I_{\nu}(f)=\lim_{\pmb{t}\to\Cal F(T,\Delta,\frak
R)}S_{\pmb{t}}(f,\nu)$
is defined and equal to $\gamma$.

\proof{ Let $\epsilon>0$.   For each $x\in X$ let $\delta_x\in\Delta$
be
such that
$|f(x)\nu C-F(C)|\le\epsilon\theta C$ whenever $(x,C)\in\delta_x$.
Because $\Delta$ is full, there is a $\delta\in\Delta$ such that
$(x,C)\in\delta_x$ whenever $(x,C)\in\delta$.   Let $\Cal R\in\frak R$
be as in (v).   If $\pmb{t}\in T$ is $\delta$-fine and
$\Cal R$-filling, then

$$\eqalignno{|S_{\pmb{t}}(f,\nu)-\gamma|
&\le|\gamma-\sum_{(x,C)\in\pmb{t}}F(C)|
  +\sum_{(x,C)\in\pmb{t}}|f(x)\nu C-F(C)|\cr
&\le\epsilon+\sum_{(x,C)\in\pmb{t}}\epsilon\theta C\cr
\displaycause{because $X\setminus\bigcup_{(x,C)\in\pmb{t}}C\in\Cal R$,
while $(x,C)\in\delta_x$ whenever $(x,C)\in\pmb{t}$}
&\le 2\epsilon\cr}$$

\noindent by condition (iv).   As $\epsilon$ is arbitrary, we have the
result.
}%end of proof of 482I

\leader{482J}{Definition}
Let $X$ be a set, $\Cal C$ a family of subsets of $X$,
$T\subseteq[X\times\Cal C]^{<\omega}$ a family of tagged partitions,
$\nu:\Cal C\to\coint{0,\infty}$ a function, and $\Delta$ a family of
gauges on $X$.   I will say that $\nu$ is {\bf moderated} (with respect
to $T$ and $\Delta$) if there are a $\delta\in\Delta$ and a function
$h:X\to\ooint{0,\infty}$ such that $S_{\pmb{t}}(h,\nu)\le 1$ for every
$\delta$-fine $\pmb{t}\in T$.

\leader{482K}{B.Levi's theorem}\dvArevised{2010} Let $(X,T,\Delta,\frak R)$
be a
tagged-partition structure allowing subdivisions, witnessed by $\Cal C$,
such that $\Delta$ is countably full, and
$\nu:\Cal C\to\coint{0,\infty}$ a function
which is moderated with respect to $T$ and $\Delta$.

Let $\sequencen{f_n}$ be a non-decreasing sequence of functions from
$X$ to $\Bbb R$ with supremum $f:X\to\Bbb R$.   If
$\gamma=\lim_{n\to\infty}I_{\nu}(f_n)$ is defined in
$\Bbb R$, then $I_{\nu}(f)$ is defined and equal to $\gamma$.

\proof{ As in the proof of 123A, we may, replacing $\sequencen{f_n}$ by
$\sequencen{f_n-f_0}$ if necessary, suppose that $f_n(x)\ge 0$ for every
$n\in\Bbb N$ and $x\in X$.

\medskip

{\bf (a)} Take $\epsilon>0$.   Then there is a $\delta\in\Delta$ such
that $S_{\pmb{t}}(f,\nu)\le\gamma+4\epsilon$ for every $\delta$-fine
$\pmb{t}\in T$.

\noindent\Prf\ Fix a strictly positive function
$h:X\to\ooint{0,\infty}$
and a $\tilde\delta\in\Delta$ such that $S_{\pmb{t}}(h,\nu)\le 1$ for
every $\tilde\delta$-fine $\pmb{t}\in T$.
For each $n\in\Bbb N$ choose $\delta_n\in\Delta$ and
$\Cal R_n\in\frak R$ such that
$|I_{\nu}(f_n)-S_{\pmb{t}}(f,\nu)|\le 2^{-n-1}\epsilon$ for every
$\delta_n$-fine $\Cal R_n$-filling $\pmb{t}\in T$.   For each
$x\in X$, take $r_x\in\Bbb N$ such that
$f(x)\le f_{r_x}(x)+\epsilon h(x)$.   Let $\delta\in\Delta$ be such
that $(x,C)\in\tilde\delta\cap\delta_{r_x}$ for every $(x,C)\in\delta$.

Suppose that $\pmb{t}\in T$ is $\delta$-fine.   Enumerate $\pmb{t}$ as
$\ofamily{i}{n}{(x_i,C_i)}$.   Let $k\in\Bbb N$ be so large that
$S_{\pmb{t}}(f,\nu)\le S_{\pmb{t}}(f_k,\nu)+\epsilon$ and $r_{x_i}\le
k$
for every $i<n$.   For $m\le k$, set $J_m=\{i:i<n,\,r_{x_i}=m\}$.
For
each $i\in J_m$, $(x_i,C_i)\in\delta_m$.   By 482A(c-ii), there is a
$\delta_k$-fine $\pmb{s}_m\in T$ such that
$W_{\pmb{s}_m}\subseteq\bigcup_{i\in J_m}C_i$ and
$|S_{\pmb{s}_m}(f_m,\nu)-\sum_{i\in J_m}f_m(x_i)\nu C_i|
\le 2^{-m}\epsilon$.

Set $\pmb{s}=\bigcup_{m\le k}\pmb{s}_m$, so that $\pmb{s}$ is a
$\delta_k$-fine member of $T$ and

$$\eqalignno{\sum_{m=0}^k\sum_{i\in J_m}f_m(x_i)\nu C_i
&\le\sum_{m=0}^kS_{\pmb{s}_m}(f_m,\nu)+2^{-m}\epsilon
\le\sum_{m=0}^kS_{\pmb{s}_m}(f_k,\nu)+2^{-m}\epsilon\cr
&\le S_{\pmb{s}}(f_k,\nu)+2\epsilon
\le I_{\nu}(f_k)+3\epsilon\cr
\displaycause{because $\pmb{s}$ extends to a $\delta_k$-fine
$\Cal R_k$-filling member of $T$, by 482Ab}
&\le\gamma+3\epsilon.\cr}$$

Now $\pmb{t}$ is $\tilde\delta$-fine, so $S_{\pmb{t}}(h,\nu)\le 1$.
Accordingly

$$\eqalign{S_{\pmb{t}}(f,\nu)
&=\sum_{i<n}f(x_i)\nu C_i
=\sum_{m=0}^k\sum_{i\in J_m}f(x_i)\nu C_i\cr
&\le\sum_{m=0}^k\sum_{i\in J_m}(f_m(x_i)+\epsilon h(x_i))\nu C_i
\le\gamma+3\epsilon+\epsilon S_{\pmb{t}}(h,\nu)
\le\gamma+4\epsilon,\cr}$$

\noindent as required.\ \Qed

As $\epsilon$ is arbitrary,
$\limsup_{\pmb{t}\to\Cal F(T,\Delta,\frak R)}S_{\pmb{t}}(f,\nu)$ is at
most $\gamma$.

\medskip

{\bf (b)} On the other hand, given $\epsilon>0$, there is an
$n\in\Bbb N$ such that $I_{\nu}(f_n)\ge\gamma-\epsilon$.   So taking
$\delta_n\in\Delta$, $\Cal R_n\in\frak R$ as in (a) above,

\Centerline{$S_{\pmb{t}}(f,\nu)\ge S_{\pmb{t}}(f_n,\nu)
\ge I_{\nu}(f_n)-\epsilon\ge\gamma-2\epsilon$}

\noindent for every $\delta_n$-fine $\Cal R_n$-filling $\pmb{t}\in T$.
So $\liminf_{\pmb{t}\to\Cal F(T,\Delta,\frak R)}S_{\pmb{t}}(f,\nu)$ is
at
least $\gamma$, and $I_{\nu}(f)=\gamma$.
}%end of proof of 482K

\leader{482L}{Lemma} Let $X$ be a set, $\Cal C$ a family of subsets of
$X$, $\Delta$ a countably full downwards-directed set of gauges on
$X$, $\frak R\subseteq\Cal P\Cal PX$ a downwards-directed collection of
residual families, and $T\subseteq[X\times\Cal C]^{<\omega}$ a
straightforward set of
tagged partitions of $X$ compatible with $\Delta$ and $\frak R$.
Suppose further that whenever $\delta\in\Delta$,
$\Cal R\in\frak R$ and $\pmb{t}\in T$ is $\delta$-fine, there is a
$\delta$-fine $\Cal R$-filling $\pmb{t}'\in T$ including $\pmb{t}$.
\cmmnt{(For instance, $(X,T,\Delta,\frak R)$ might be a tagged-partition
structure allowing subdivisions, as in 482Ab.)}   If
$\nu:\Cal C\to\coint{0,\infty}$ and $f:X\to\coint{0,\infty}$ are such
that $I_{\nu}(f)=0$, and $g:X\to\Bbb R$ is such that $g(x)=0$ whenever
$f(x)=0$, then $I_{\nu}(g)=0$.

\proof{ Let $\epsilon>0$.   For each $n\in\Bbb N$, let
$\delta_n\in\Delta$, $\Cal R_n\in\frak R$ be such that
$S_{\pmb{t}}(f,\nu)\le 2^{-n}\epsilon$ for every $\delta_n$-fine
$\Cal R_n$-filling $\pmb{t}\in T$.   For $x\in X$, set
$\phi(x)=\min\{n:|g(x)|\le nf(x)\}$;  let $\delta\in\Delta$ be such
that
$(x,C)\in\delta_{\phi(x)}$ whenever $(x,C)\in\delta$.

Let $\pmb{t}$ be any $\delta$-fine member of $T$.   Then
$|S_{\pmb{t}}(g,\nu)|\le 2\epsilon$.   \Prf\ Enumerate $\pmb{t}$ as
$\ofamily{i}{n}{(x_i,C_i)}$.   For each $m\in\Bbb N$, set
$K_m=\{i:i<n,\,\phi(x_i)=m\}$;  then $\{(x_i,C_i):i\in K_m\}$ is a
$\delta_m$-fine member of $T$, so extends to a $\delta_m$-fine
$\Cal R_m$-filling member $\pmb{t}_m$ of $T$, and

\Centerline{$\sum_{i\in K_m}|g(x_i)\nu C_i|
\le m\sum_{i\in K_m}f(x_i)\nu C_i
\le mS_{\pmb{t}_m}(f,\nu)
\le 2^{-m}m\epsilon$.}

\noindent Summing over $m$,

\Centerline{$|S_{\pmb{t}}(g,\nu)|
\le\sum_{m=0}^{\infty}2^{-m}m\epsilon=2\epsilon$.  \Qed}

\noindent Because $T$ is compatible with $\Delta$ and $\Cal R$, this
is
enough to show that $I_{\nu}(g)$ is defined and equal to $0$.
}%end of proof of 482L

\leader{482M}{Fubini's theorem} Suppose that, for $i=1$ and $i=2$, we
have $X_i$, $\frak T_i$, $T_i$, $\Delta_i$, $\Cal C_i$ and $\nu_i$
such
that

\inset{(i) $(X_i,\frak T_i)$ is a topological space;

(ii) $\Delta_i$ is the set of neighbourhood gauges on $X_i$;

(iii) $T_i\subseteq[X_i\times\Cal C_i]^{<\omega}$ is a straightforward
set of tagged partitions, compatible with $\Delta_i$ and
$\{\{\emptyset\}\}$;

(iv) $\nu_i:\Cal C_i\to\coint{0,\infty}$ is a function;

(v) $\nu_1$ is moderated with respect to $T_1$ and $\Delta_1$;

(vi) whenever $\delta\in\Delta_1$ and $\pmb{s}\in T_1$ is
$\delta$-fine, there is a $\delta$-fine $\pmb{s}'\in T_1$, including
$\pmb{s}$, such that $W_{\pmb{s}'}=X_1$.}

\noindent Write $X$ for $X_1\times X_2$;  $\Delta$ for the set of
neighbourhood gauges on $X$;  $\Cal C$ for
$\{C\times D:C\in\Cal C_1,\,D\in\Cal C_2\}$;  $Q$ for
$\{((x,y),C\times D):\{(x,C)\}\in T_1,\,\{(y,D)\}\in T_2\}$;
$T$ for the straightforward set of tagged partitions generated by $Q$;
and set
$\nu(C\times D)=\nu_1C\cdot\nu_2D$ for $C\in\Cal C_1$, $D\in\Cal C_2$.

(a) $T$ is compatible with $\Delta$ and $\{\{\emptyset\}\}$.

(b) Let $I_{\nu_1}$, $I_{\nu_2}$ and $I_{\nu}$ be the gauge integrals
defined by these structures\cmmnt{ as in 481C-481F}.   Suppose that
$f:X\to\Bbb R$ is such that $I_{\nu}(f)$ is defined.   Set
$f_x(y)=f(x,y)$ for $x\in X_1$, $y\in X_2$.   Let $g:X_1\to\Bbb R$ be
any function such that $g(x)=I_{\nu_2}(f_x)$ whenever this is defined.
Then $I_{\nu_1}(g)$ is defined and equal to $I_{\nu}(f)$.

\proof{{\bf (a)} Let $\delta\in\Delta$;  we seek a tagged partition
$\pmb{u}\in T$ such that $W_{\pmb{u}}=X$.   Let
$\family{(x,y)}{X}{V_{xy}}$ be the family of open sets in $X$ defining
$\delta$;  choose open sets $G_{xy}\subseteq X_1$,
$H_{xy}\subseteq X_2$
such that $(x,y)\in G_{xy}\times H_{xy}\subseteq V_{xy}$ for all
$x\in X_1$, $y\in X_2$.   For each $x\in X_1$, let $\delta_x$ be the
neighbourhood gauge on $X_2$ defined from the family
$\family{y}{X_2}{H_{xy}}$.   Then there is a $\delta_x$-fine tagged
partition $\pmb{t}_x\in T_2$ such that $W_{\pmb{t}_x}=X_2$.   Set
$G_x=X_1\cap\bigcap_{(y,D)\in\pmb{t}_x}G_{xy}$.

The family $\family{x}{X_1}{G_x}$ defines a neighbourhood gauge
$\delta^*$ on $X_1$, and there is a $\delta^*$-fine $\pmb{s}\in T_1$
such that $W_{\pmb{s}}=X_1$. Now consider

\Centerline{$\pmb{u}
=\{((x,y),C\times D):(x,C)\in\pmb{s},\,(y,D)\in\pmb{t}_x\}$.}

\noindent Then it is easy to check (just as in part (b) of the proof
of 481O) that $\pmb{u}$ is a $\delta$-fine member of $T$ with
$W_{\pmb{u}}=X$.

\medskip

{\bf (b)(i)} Set $A=\{x:x\in X_1,\,I_{\nu_2}(f_x)$ is defined$\}$.
Let $h:X_1\to\Bbb R$ be any function such that $h(x)=0$ for every
$x\in A$.   For $x\in X_1$, set

\Centerline{$h_0(x)
=\inf(\{1\}\cup\{\sup_{\pmb{t},\pmb{t}'\in F}S_{\pmb{t}}(f_x,\nu_2)
  -S_{\pmb{t}'}(f_x,\nu_2):
F\in\Cal F(T_2,\Delta_2,\{\{\emptyset\}\})\})$.}

\noindent (Thus
$I_{\nu_2}(f_x)$ is defined iff $h_0(x)=0$.)   Then
$I_{\nu_1}(h_0)=0$.
\Prf\ Let $\epsilon>0$.   Then there is a $\delta\in\Delta$ such that
$S_{\pmb{u}}(f,\nu)-S_{\pmb{u}'}(f,\nu)\le\epsilon$ whenever
$\pmb{u}$,
$\pmb{u}'\in T$ are $\delta$-fine and $W_{\pmb{u}}=W_{\pmb{u}'}=X$.
Define $\family{(x,y)}{X}{V_{xy}}$, $\family{(x,y)}{X}{G_{xy}}$,
$\family{(x,y)}{X}{H_{xy}}$ and $\family{x}{X_1}{\delta_x}$ from
$\delta$ as in (a) above.   For each $x\in X_1$, we can find
$\delta_x$-fine partitions $\pmb{t}_x$, $\pmb{t}'_x\in T_2$ such that
$W_{\pmb{t}_x}=W_{\pmb{t}'_x}=X_2$ and
$S_{\pmb{t}_x}(f_x,\nu_2)-S_{\pmb{t}'_x}(f_x,\nu_2)\ge\bover12h_0(x)$.
Set $G_x=X_1\cap\bigcap_{(y,D)\in\pmb{t}_x\cup\pmb{t}'_x}G_{xy}$.

Let $\delta^*$ be the neighbourhood gauge on $X_1$ defined from
$\family{x}{X_1}{G_x}$.   Let $\pmb{s}$ be any $\delta^*$-fine member
of $T_1$ with $W_{\pmb{s}}=X_1$.   Set

\Centerline{$\pmb{u}
=\{((x,y),C\times D):(x,C)\in\pmb{s},\,(y,D)\in\pmb{t}_x\}$,}

\Centerline{$\pmb{u}'
=\{((x,y),C\times D):(x,C)\in\pmb{s},\,(y,D)\in\pmb{t}'_x\}$,}

\noindent Then $\pmb{u}$ and $\pmb{u}'$ are $\delta$-fine members of
$T$ with $W_{\pmb{u}}=W_{\pmb{u}'}=X$, so

$$\eqalign{S_{\pmb{s}}(h_0,\nu_1)
&=\sum_{(x,C)\in\pmb{s}}h_0(x)\nu_1(C)
\le 2\sum_{(x,C)\in\pmb{s}}
  (S_{\pmb{t}_x}(f_x,\nu_2)-S_{\pmb{t}'_x}(f_x,\nu_2))\nu_1(C)\cr
&=2\bigl(\sum_{\Atop{(x,C)\in\pmb{s}}{(y,D)\in\pmb{t}_x}}
      f(x,y)\nu_1(C)\nu_2(D)
  -\sum_{\Atop{(x,C)\in\pmb{s}}{(y,D)\in\pmb{t}'_x}}
      f(x,y)\nu_1(C)\nu_2(D)\bigr)\cr
&=2\bigl(S_{\pmb{u}}(f,\nu)-S_{\pmb{u}'}(f,\nu)\bigr)
\le 2\epsilon.\cr}$$

\noindent As $\epsilon$ is arbitrary, $I_{\nu_1}(h_0)=0$.\ \Qed

By 482L, $I_{\nu_1}(h)$ also is zero.

\medskip

\quad{\bf (ii)} Again take any $\epsilon>0$, and let $\delta\in\Delta$
be such that $|S_{\pmb{u}}(f,\nu)-I_{\nu}(f)|\le\epsilon$ for every
$\delta$-fine $\pmb{u}\in T$ such that $W_{\pmb{u}}=X$.
Define $\family{(x,y)}{X}{V_{xy}}$, $\family{(x,y)}{X}{G_{xy}}$,
$\family{(x,y)}{X}{H_{xy}}$ and $\family{x}{X_1}{\delta_x}$ from
$\delta$ as in (a) and (i) above.   Let $\tilde\delta\in\Delta_1$,
$\tilde h:X_1\to\ooint{0,\infty}$ be such that
$S_{\pmb{s}}(\tilde h,\nu_1)\le 1$ for every $\tilde\delta$-fine
$\pmb{s}\in T_1$.

For $x\in A$, let $\pmb{t}_x\in T_2$ be $\delta_x$-fine and
such that $W_{\pmb{t}_x}=X_2$ and
$|S_{\pmb{t}_x}(f_x,\nu_2)-I_{\nu_2}(f_x)|\le\epsilon\tilde h(x)$; for
$x\in X_1\setminus A$, let $\pmb{t}_x$ be any $\delta_x$-fine member
of
$T_2$ such that $W_{\pmb{t}_x}=X_2$.   Set
$G_x=X_1\cap\bigcap_{(y,D)\in\pmb{t}_x}G_{xy}$ for every $x\in X_1$.
Let $\delta^*$ be the neighbourhood gauge on $X_1$ defined by
$\family{x}{X_1}{G_x}$.

Set $h_1(x)=g(x)-S_{\pmb{t}_x}(f_x,\nu_2)$ for $x\in X_1\setminus A$,
$0$ for $x\in A$.   Then $I_{\nu_1}(|h_1|)=0$, by (a).   Let
$\delta'\in\Delta_1$ be such that
$\delta'\subseteq\delta^*\cap\tilde\delta$ and
$S_{\pmb{s}}(|h_1|,\nu_1)\le\epsilon$ for every $\delta'$-fine
$\pmb{s}\in T_1$ such that $W_{\pmb{s}}=X_1$.

Now suppose that $\pmb{s}\in T_1$ is $\delta'$-fine and that
$W_{\pmb{s}}=X_1$.   Set

\Centerline{$\pmb{u}=\{((x,y),C\times
D):(x,C)\in\pmb{s},\,(y,D)\in\pmb{t}_x\}$,}

\noindent so that $\pmb{u}\in T$ is $\delta$-fine and $W_{\pmb{u}}=X$.
Then

$$\eqalign{|S_{\pmb{s}}(g,\nu_1)-I_{\nu}(f)|
&\le|S_{\pmb{u}}(f,\nu)-I_{\nu}(f)|
  +|S_{\pmb{s}}(g,\nu_1)-S_{\pmb{u}}(f,\nu)|\cr
&\le\epsilon+\bigl|\sum_{(x,C)\in\pmb{s}}
  \bigl(g(x)-\sum_{(y,D)\in\pmb{t}_x}f(x,y)\nu_2D\bigr)\nu_1C\bigr|\cr
&=\epsilon+\bigl|\sum_{(x,C)\in\pmb{s}}
  \bigl(g(x)-S_{\pmb{t}_x}(f_x,\nu_2)\bigr)\nu_1C\bigr|\cr
&\le\epsilon+\sum_{(x,C)\in\pmb{s}}|h_1(x)|\nu_1C\cr
&\qquad\qquad+\sum_{\Atop{(x,C)\in\pmb{s}}{x\in A}}
     |g(x)-\sum_{(y,D)\in\pmb{t}_x}f(x,y)\nu_2D|\nu_1C\cr
&\le 2\epsilon
  +\sum_{\Atop{(x,C)\in\pmb{s}}{x\in A}}\epsilon\tilde h(x)\nu_1(C)
\le 2\epsilon
  +\epsilon S_{\pmb{s}}(\tilde h,\nu_1)
\le 3\epsilon.\cr}$$

\noindent As $\epsilon$ is arbitrary, $I_{\nu_1}(g)$ is defined and
equal to $I_{\nu}(f)$, as claimed.
}%end of proof of 482M

\exercises{\leader{482X}{Basic exercises (a)}
%\spheader 482Xa
Let $(X,T,\Delta,\frak R)$ be a tagged-partition structure allowing
subdivisions, witnessed by $\Cal C$, and $\Cal E$ the algebra of
subsets
of $X$ generated by $\Cal C$.   Write $\eusm I$ for the set of pairs
$(f,\nu)$ such that $f:X\to\Bbb R$ and $\nu:\Cal C\to\Bbb R$ are
functions and
$I_{\nu}(f)=\lim_{\pmb{t}\to\Cal F(T,\Delta,\frak R)}S_{\pmb{t}}(f,\nu)$
is defined;  for $(f,\nu)\in\eusm I$, let $F_{f\nu}:\Cal E\to\Bbb R$ be
the corresponding Saks-Henstock indefinite integral.   Show that
$(f,\nu)\mapsto F_{f\nu}$ is bilinear in the sense that

\Centerline{$F_{f+g,\nu}=F_{f\nu}+F_{g\nu}$,
\quad$F_{\alpha f,\nu}=F_{f,\alpha\nu}=\alpha F_{f\nu}$,
\quad$F_{f,\mu+\nu}=F_{f\mu}+F_{f\nu}$}

\noindent whenever $(f,\nu)$, $(g,\nu)$ and $(f,\mu)$ belong to
$\eusm I$ and $\alpha\in\Bbb R$.
%482B

\spheader 482Xb Let $(X,T,\Delta,\frak R)$ be a tagged-partition
structure allowing subdivisions, witnessed by $\Cal C$, and $\Cal E$
the
algebra of subsets of $X$ generated by $\Cal C$.   Let $f:X\to\Bbb R$
and $\nu:\Cal C\to\coint{0,\infty}$ be functions such that
$I_{\nu}(f)=\lim_{\pmb{t}\to\Cal F(T,\Delta,\frak
R)}S_{\pmb{t}}(f,\nu)$
is defined, and let $F:\Cal E\to\Bbb R$ be the corresponding
Saks-Henstock indefinite integral.   Show that $F$ is non-negative iff
$I_{\nu}(f^-)=0$, where $f^-(x)=\max(0,-f(x))$ for every $x\in X$.
%482B

\sqheader 482Xc
Let $X$ be a zero-dimensional compact Hausdorff space, $\Cal E$ the
algebra of open-and-closed subsets of $X$,
$Q=\{(x,E):x\in E\in\Cal E\}$, $T$ the straightforward set of tagged
partitions generated by $Q$, $\Delta$ the set of neighbourhood gauges
on $X$ and $\nu:\Cal E\to\Bbb R$ a
non-negative additive functional.   Let $\mu$ be the corresponding Radon
measure on $X$ (416Qa) and $I_{\nu}$ the gauge integral defined by
$(X,T,\Delta,\{\{\emptyset\}\})$ (cf.\ 481Xh).   Show that, for
$f:X\to\Bbb R$, $I_{\nu}(f)=\int fd\mu$ if either is defined in
$\Bbb R$.   \Hint{if $f$ is measurable but not $\mu$-integrable, take
$x_0$ such that $f$ is not integrable on any neighbourhood of $x_0$.
Given $\delta\in\Delta$, fix a $\delta$-fine partition containing
$(x_0,V_0)$ for some $V_0$;  now replace $(x_0,V_0)$ by refinements
$\{(x_0,V'_0),(x_1,V_1),\ldots,(x_n,V_n)\}$, where
$\sum_{i=1}^nf(x_i)\nu V_i$ is large, to show that
$S_{\pmb{t}}(f,\nu)$
cannot be controlled by $\delta$.}
%481Xh, 482E, 482F

\sqheader 482Xd Let $(X,\frak T,\Sigma,\mu)$ be an effectively locally
finite $\tau$-additive topological measure space in which $\mu$ is
inner
regular with respect to the closed sets and outer regular with respect
to the open sets (see 412W).   Let $T$ be the straightforward set of
tagged partitions generated by $X\times\{E:\mu E<\infty\}$, $\Delta$
the set of neighbourhood gauges on $X$, and
$\frak R=\{\Cal R_{E\eta}:\mu E<\infty,\,\eta>0\}$, where
$\Cal R_{E\eta}=\{F:\mu(F\cap E)\le\eta\}$, as in N.   Show that if
$I_{\mu}$ is the associated gauge integral, and $f:X\to\Bbb R$ is a
function, then $I_{\mu}(f)=\int fd\mu$ if either is defined in $\Bbb R$.
%482E, 482F, 482Xc

\spheader 482Xe Let $\Cal C$ be the set of non-empty subintervals of
$X=[0,1]$, $T$ the straightforward tagged-partition structure
generated
by $[0,1]\times\Cal C$, and $\Delta$ the set of neighbourhood gauges
on $[0,1]$, as in 481M.   Let $\mu$ be any Radon measure on $[0,1]$, and
$I_{\mu}$ the gauge integral defined from $\mu$ and the
tagged-partition
structure $([0,1],T,\Delta,\{\{\emptyset\}\})$.   Show that, for any
$f:[0,1]\to\Bbb R$, $I_{\mu}(f)=\int fd\mu$ if either is defined in
$\Bbb R$.
%482E, 482F, 482Xc

\spheader 482Xf Let $\Cal C$ be the set of non-empty subintervals of
$[0,1]$, $T$ the straightforward tagged-partition structure generated
by $\{(x,C):C\in\Cal C,\,x\in\overline{C}\}$, and $\Delta$ the set of
neighbourhood gauges on $X$, as in 481J.   Let $\Cal E$ be the ring
of subsets of $[0,1]$ generated by $\Cal C$,
$\nu:\Cal E\to\Bbb R$ a bounded additive functional, and
$I_{\nu}$ the gauge integral defined from $\nu$ and
$(X,T,\Delta,\{\{\emptyset\}\})$.   Show that
$I_{\nu}(\chi\coint{a,b})=\lim_{x\uparrow a,y\uparrow b}\nu([x,y])$
whenever $0<a<b\le 1$.
%482G

\spheader 482Xg Let $\Cal C$ be the set of non-empty subintervals of
$X=[0,1]$, $T$ the straightforward tagged-partition structure
generated
by $\{(x,C):C\in\Cal C,\,x\in\overline{C}\}$, and $\Delta$ the set of
uniform metric gauges on $[0,1]$, as in 481I.   Let $\mu$ be the Dirac
measure on $[0,1]$ concentrated at $\bover12$, and let
$I_{\mu}
=\lim_{\pmb{t}\to\Cal F(T,\Delta,\{\{\emptyset\}\})}S_{\pmb{t}}(\,,\mu)$
be the corresponding gauge integral.   Show that $I_{\mu}(\chi[0,1])$
is
defined but that $I_{\mu}(\chi[0,\bover12])$ is not.
%482G

\sqheader 482Xh(i) Show that the McShane integral on an interval
$[a,b]$ as described in 481M coincides with the Lebesgue integral on
$[a,b]$.   (ii) Show that if $(X,\frak T,\Sigma,\mu)$ is a quasi-Radon
measure space and $\mu$ is outer regular with respect to the open sets
then the McShane integral as described in 481N coincides with the
usual integral.
%482G

\spheader 482Xi Explain how the results in 481Xb-481Xc can be regarded
as special cases of 482H.

\spheader 482Xj Let $(X,\frak T)$ be a topological space, $\Cal C$ a
ring of subsets of $X$, $T\subseteq[X\times\Cal C]^{<\omega}$ a
straightforward set of tagged partitions,
$\nu:\Cal C\to\coint{0,\infty}$ an additive function, and $\Delta$ the
family of neighbourhood gauges on $X$.   Suppose that there is a
sequence $\sequencen{G_n}$ of open sets, covering $X$, such that
$\sup\{\nu C:C\in\Cal C,\,C\subseteq G_n\}$ is finite for every
$n\in\Bbb N$.   Show that $\nu$ is moderated with respect to $T$ and
$\Delta$.
%482J

\spheader 482Xk Let $(X,\frak T,\Sigma,\mu)$ be a quasi-Radon measure
space.   Let $T$ be the straightforward tagged-partition structure
generated by
$\{(x,E):\mu E<\infty,\,x\in E\}$ and $\Delta$ the set of all
neighbourhood gauges on $X$.   Show that $\mu$ is moderated with
respect
to $T$ and $\Delta$ iff there is a sequence of open sets of finite
measure covering $X$.
%482J

\spheader 482Xl Let $r\ge 1$ be an integer, and $\mu$ a Radon measure
on $\BbbR^r$.   Let $Q$ be the set of pairs $(x,C)$ where
$x\in\BbbR^r$ and $C$ is a closed ball with centre $x$, and $T$ the
straightforward set of tagged partitions generated by $Q$.   Let
$\Delta$ be the set of neighbourhood gauges on $\BbbR^r$, and
$\frak R=\{\Cal R_{E\eta}:\mu E<\infty,\,\eta>0\}$, where
$\Cal R_{E\eta}=\{F:\mu(F\cap E)\le\eta\}$, as in 482Xd.   (i) Show
that $T$ is compatible with $\Delta$ and $\frak R$.   \Hint{472C.}
(ii) Show that if
$I_{\mu}$ is the associated gauge integral, and $f:\BbbR^r\to\Bbb R$
is a function, then $I_{\mu}(f)=\int fd\mu$ if either is defined in
$\Bbb R$.
%482?

\spheader 482Xm\dvAnew{2012}
Let $(X,T,\Delta,\frak R)$, $\Sigma$ and $\nu$ be as in 481Xj, so that
$(X,T,\Delta,\frak R)$ is a tagged-partition structure allowing
subdivisions, witnessed by an algebra $\Sigma$ of subsets of $X$, and
$\nu:\Sigma\to\coint{0,\infty}$ is additive.   Let $I_{\nu}$ be the
corresponding gauge integral, and $V\subseteq\BbbR^X$ its domain.   
(i) Show that $\chi E\in V$ and $I_{\nu}(\chi E)=\nu E$
for every $E\in\Sigma$.   (ii) Show that if $f\in\BbbR^X$ then $f\in V$ iff
for every $\epsilon>0$ there is a disjoint family $\Cal E\subseteq\Sigma$
such that $\sum_{E\in\Cal E}\nu E=\nu X$ and
$\sum_{E\in\Cal E}\nu E\cdot\sup_{x,y\in E}|f(x)-f(y)|\le\epsilon$.
(iii) Show that if
$\sequencen{f_n}$ is a non-decreasing sequence in $V$ with supremum 
$f\in\BbbR^X$, and $\gamma=\sup_{n\in\Bbb N}I_{\nu}(f_n)$ is finite, then
$I_{\nu}(f)$ is defined and equal to $\gamma$.
(iv) Show that $I_{\nu}$ extends $\dashint d\nu$ as described in
363L, if we identify $L^{\infty}(\Sigma)$ with a space $\eusm L^{\infty}$
of functions as in 363H.   (v) Show that if $\Sigma$ is a $\sigma$-algebra
of sets then $I_{\nu}$ extends $\dashint d\nu$ as described in
364Xj.


\leader{482Y}{Further exercises (a)}\dvAnew{2010}
%\spheader 482Ya
Let $X$ be the interval $[0,1]$,
$\Cal C$ the family of subintervals of $X$,
$Q$ the set $\{(x,C):C\in\Cal C$, $x\in\overline{\interior C}\}$, $T$ the
straightforward set of tagged partitions generated by $Q$, $\Delta$ the set
of neighbourhood gauges on $X$, and $\frak R$ the singleton
$\{[X]^{<\omega}\}$.   Show that $(X,T,\Delta,\frak R)$ is a
tagged-partition structure allowing subdivisions, witnessed by $\Cal C$.
For $C\in\Cal C$ set $\nu C=1$ if $0\in\overline{\interior C}$, $0$
otherwise, and let $f$ be $\chi\{0\}$.   Show that
$I_{\nu}(f)=\lim_{\pmb{t}\to\Cal F(T,\Delta,\frak R)}S_{\pmb{t}}(f,\nu)$
is defined and equal
to $1$.   Let $F$ be the Saks-Henstock indefinite integral of $f$.
Show that $F(\ocint{0,1})=1$.
%482B

\spheader 482Yb
Set $X=\Bbb R$ and let $\Cal C$ be the family of
non-empty bounded intervals in $X$;  set
$Q=\{(x,C):C\in\Cal C,\,x=\inf C\}$, and let $T$ be the
straightforward
set of tagged partitions generated by $Q$.   Let $\frak T$ be the
Sorgenfrey right-facing topology on $X$, and $\Delta$ the set of
neighbourhood gauges for $\frak T$.   Set
$\Cal R_n=\{E:E\in\Sigma,\,\mu([-n,n]\cap E)\le 2^{-n}\}$ for
$n\in\Bbb N$, where $\mu$ is Lebesgue measure on $\Bbb R$ and $\Sigma$
its domain, and write $\frak R=\{\Cal R_n:n\in\Bbb N\}$.   Show that
$(X,T,\Delta,\frak R)$ is a tagged-partition structure allowing
subdivisions.   Show that if $f:X\to\Bbb R$ is such that $I_{\mu}(f)$
is defined, then $f$ is Lebesgue measurable.
%482E

\spheader 482Yc Give an example of $X$, $\frak T$, $\Sigma$, $\mu$,
$T$, $\Delta$, $\Cal C$, $f$ and $C$ such that
$(X,\frak T,\Sigma,\mu)$ is a compact metrizable Radon probability
space, $\Delta$ is the set of all neighbourhood gauges on $X$,
$(X,T,\Delta,\{\{\emptyset\}\})$ is a tagged-partition structure
allowing subdivisions, witnessed by
$\Cal C$, $f:X\to\Bbb R$ is a function such that $I_{\mu}(f)
=\lim_{\pmb{t}\to\Cal F(T,\Delta,\{\{\emptyset\}\})}S_{\pmb{t}}(f,\mu)$
is defined, $C\in\Cal C$ is a closed set with negligible boundary, and
$I_{\mu}(f\times\chi C)$ is not defined.
%482G

\spheader 482Yd Suppose that, for $i=1$ and $i=2$, we have a
tagged-partition structure $(X_i,T_i,\Delta_i,\frak R_i)$ allowing
subdivisions, witnessed by a ring $\Cal C_i\subseteq\Cal PX_i$, where
$X_i$ is a topological space, $\Delta_i$ is the set of all neighbourhood
gauges on $X_i$, and $\frak R_i$ is the simple residual structure
complementary to $\Cal C_i$, as in 481Yb.   Set
$X=X_1\times X_2$ and let $\Delta$ be the set of neighbourhood gauges
on $X$;  set $\Cal C=\{C\times D:C\in\Cal C_1,\,D\in\Cal C_2\}$;  let
$\frak R$ be the simple residual structure on $X$ complementary to
$\Cal C$;  and let $T$ be the straightforward tagged-partition
structure generated by
$\{((x,y),C\times D):\{(x,C)\}\in T_1,\,\{(y,D)\}\in T_2\}$.   For each
$i$, let $\nu_i:\Cal C_i\to\coint{0,\infty}$ be a function moderated
with respect to $T_i$ and $\Delta_i$, and define
$\nu:\Cal C\to\coint{0,\infty}$ by setting
$\nu(C\times D)=\nu_1C\cdot\nu_2D$ for $C\in\Cal C_1$, $D\in\Cal C_2$.
Show that $T$ is compatible with $\Delta$ and $\frak R$.   Let
$I_{\nu_1}$, $I_{\nu_2}$, $I_{\nu}$ be the gauge integrals defined by
these structures.   Suppose that $f:X\to\Bbb R$ is such that
$I_{\nu}(f)$ is defined.   Set $f_x(y)=f(x,y)$ for $x\in X_1$,
$y\in X_2$.   Let $g:X_1\to\Bbb R$ be any function such that
$g(x)=I_{\nu_2}(f_x)$ whenever this is defined.   Show that
$I_{\nu_1}(g)$ is defined and equal to $I_{\nu}(f)$.
%482M
}%end of exercises

\endnotes{
\Notesheader{482}
In 482E, 482F and 482G the long lists of conditions reflect the variety
of possible applications of these arguments.   The price to be paid for
the versatility of the constructions here is a theory which is rather
weak in the absence of special hypotheses.
As everywhere in this book, I try to set ideas out in maximal convenient
generality;   you may feel that in this section the generality is
becoming inconvenient;  but the theory of gauge integrals has not, to my
eye, matured to the point that we can classify the systems here even as
provisionally as I set out to classify topological measure spaces in
Chapters 41 and 43.

Enthusiasts for gauge integrals offer two substantial arguments for
taking them seriously, apart from the universal argument in pure
mathematics, that these structures offer new patterns for our delight
and new challenges to our ingenuity.   First, they say, gauge integrals
integrate more functions than Lebesgue-type integrals, and it is the
business of a theory of integration to integrate as many functions as
possible;  and secondly, gauge integrals offer an easier path to the
principal theorems.   I have to say that I think the first argument is
sounder than the second.   It is quite true that the Henstock integral
on $\Bbb R$ (481K) can be rigorously defined in fewer words, and with
fewer concepts, than the Lebesgue integral.   The style of Chapters 11
and 12 is supposed to be better adapted to the novice than the style
of this chapter, but you will have no difficulty in putting the ideas of
481A, 481C, 481J and 481K together into an elementary definition of an
integral for real functions in which the only non-trivial argument is
that establishing the existence of enough tagged partitions (481J),
corresponding I suppose to Proposition 114D.   But the path I took in
defining the integral in \S122, though arduous at that point, made (I
hope) the convergence theorems of \S123 reasonably natural;  the proof
of 482K, on the other hand, makes significant demands on our
technique.
Furthermore, the particular clarity of the one-dimensional Henstock
integral is not repeated in higher dimensions.   Fubini's theorem,
with
exact statement and full proof, even for products of Lebesgue measures
on Euclidean spaces, is a lot to expect of an undergraduate;  but
Lebesgue measure on $\BbbR^r$ makes sense in a way that it is quite
hard
to repeat with gauge integrals.   (For instance, Lebesgue measure is
invariant under isometries;  this is not particularly easy to
prove -- see 263A -- but at least it is true;  if we want a gauge
integral which
is invariant under isometries, then we have to use a construction such
as 481O, which does not directly match any natural general definition
of `product gauge integral' along the lines of 481P, 482M or 482Yd.)

In my view, a stronger argument for taking gauge integrals seriously
is their `power', that is, their ability to provide us with integrals of
many functions in consistent ways.   482E, 482F and 482I give us an
idea
of what to expect.   If we start from a measure space $(X,\Sigma,\mu)$
and build a gauge integral $I_{\mu}$ from a set
$T\subseteq[X\times\Sigma]^{<\omega}$ of tagged partitions, then we
can hope that integrable functions will be gauge-integrable, with the
right
integrals (482F);  while gauge-integrable functions will be measurable
(482E).   What this means is that for {\it non-negative} functions,
the integrals will coincide.   Any `new' gauge-integrable functions $f$
will be such that $\int f^+=\int f^-=\infty$;  the gauge integral will
offer
a process for cancelling the divergent parts of these integrals.   On
the other hand, we can hope for a large class of gauge-integrable
derivatives.   In the next two sections, I will explain how this works
in the Henstock and Pfeffer
integrals.   For simple examples calling for such procedures, see the
formulae of \S\S282 and 283;  for radical applications of the idea, see
{\smc Muldowney 87}.

Against this, gauge integrals are not effective in `general' measure
spaces, and cannot be, because there is no structure in an abstract
measure space $(X,\Sigma,\mu)$ which allows us to cancel an infinite
integral $\int f^+=\int_Ff$ against
$\int f^-=\int_{X\setminus F}f$.   Put another way, if a
tagged-partition structure is invariant under all automorphisms of the
structure $(X,\Sigma,\mu)$, as in 481Xf-481Xg, we cannot expect
anything
better than the standard integral.   In order to get something new, the
most important step seems to be the specification of a family
$\Cal C$ of `regular' sets, preliminary to describing a set $T$ of
tagged partitions.   To get a `powerful' gauge integral, we want a fine
filter on $T$, corresponding to a small set $\Cal C$ and a large set of
gauges.   The residual
families of 481F are generally introduced just to ensure `compatibility'
in the sense described there;  as a rule, we try to keep them simple.
But even if we take the set of all neighbourhood gauges, as in the
Henstock integral, this is not enough unless we also sharply restrict
both the family $\Cal C$ and the permissible tags
(482Xc-482Xe).  %482Xc 482Xd 482Xe
The most successful restrictions, so far, have been `geometric', as in
481J and 481O, and 484F below.   Further limitations on admissible
pairs $(x,C)$,
as in 481L and 481Q, in which $\Cal C$ remains the set of intervals, but
fewer tags are permitted, also lead to very interesting results.

Another limitation in the scope of gauge integrals is the difficulty
they have in dealing with spaces of infinite measure.   Of course we
expect to have to specify a limiting procedure if we are to calculate
$I_{\mu}(f)$ from sums $S_{\pmb{t}}(f,\mu)$ which necessarily consider
only sets of finite measure, and this is one of the functions of the
collections $\frak R$ of residual families.   But this is not yet
enough.   In B.Levi's theorem (482K) we already need to suppose that our
set-function $\nu$ is `moderated' in order to determine how closely
$f_n(x)$ needs to approximate each $f(x)$.   The condition

\Centerline{$S_{\pmb{t}}(h,\nu)\le 1$ for every $\delta$-fine
$\pmb{t}$}

\noindent of 482J is very close to saying that $I_{\mu}(h)\le 1$.   But
it is {\it not} the same as saying that $\mu$ is $\sigma$-finite;  it
suggests rather that $X$ should be covered by a sequence of open sets
of finite measure (482Xk).

Because gauge integrals are not absolute -- that is, we can have
$I_{\nu}(f)$ defined and finite while $I_{\nu}(|f|)$ is not -- we are
bound to have difficulties with integrals $\int_Hf$, even if we
interpret these in the simplest way, as $I_{\nu}(f\times\chi H)$, so
that we do not need a theory of subspaces as developed in \S214.
482G(iii)-(v) are an
attempt to find a reasonably versatile sufficient set of conditions.
The `multiplier problem', for a given gauge integral $I_{\nu}$, is the
problem of characterizing the functions $g$ such that
$I_{\nu}(f\times g)$ is defined whenever $I_{\nu}(f)$ is defined, and
even for some intensively studied integrals remains challenging.   In
484L I will give an important case which is not covered by 482G.

One of the striking features of gauge integrals is that there is no
need
to assume that the set-function $\nu$ is countably additive.   We can
achieve countable additivity of the integral -- in the form of
B.Levi's
theorem, for instance -- by requiring only that the set of gauges
should
be `countably full' (482K, 482L;  contrast 482Xg).   If we watch our
definitions carefully, we can make this match the rules for Stieltjes
integrals (114Xa, 482Xf).   In 481Db I have already remarked on the
potential use of gauge integrals in vector integration.

It is important to recognise that a value $F(E)$ of a Saks-Henstock
indefinite integral (482B-482C) cannot be
identified with either $I_{\nu}(f\times\chi E)$ or with
$I_{\nu\restrp\Cal PE}(f\restr E)$, because in the formula
$F(E)=\lim_{\pmb{t}\to\Cal F^*}S_{\pmb{t}_E}(f,\nu)$ used in the proof of
482B the tags of the
partitions $\pmb{t}_E$ need not lie in $E$.   (See 482Ya.)
The idea of 482G is
to impose sufficient conditions to ensure that the contributions of
`straddling' elements $(x,C)$, where either $x\in E$ and
$C\not\subseteq E$ or $x\notin E$ and $C\cap E\ne\emptyset$, are
insignificant.   To achieve this we seem to need both a regularity
condition on the functional $\nu$ (condition (iii) of 482G)
and a geometric condition on the set $\Cal C$ underlying $T$ (482G(iv)).
As usual,
the regularity condition required is closer to {\it outer} than to
{\it inner} regularity, in contexts in which there is a distinction.

I am not sure that I have the `right' version of Proposition 482E.
The hypothesis there is that we have a metric space.   But in the
principal
non-metrizable cases the result is still valid (482Xc-482Xd, 482Yb),
and the same happens in 482Xl, where condition 482E(ii) is not
satisfied.
Proposition 482H is a `new' limit theorem;  it shows that certain improper
integrals from the classical theory can be represented as gauge integrals.
The hypotheses seem, from where we
are standing at the moment, to be exceedingly restrictive.   In the
leading examples in \S483, however, the central requirement 482H(viii) is
satisfied for straightforward geometric reasons.

Gauge integrals insist on finite functions defined everywhere.   But
since we have an effective theory of negligible sets (482L), we can
easily get a consistent theory of integration for functions which are
defined and real-valued almost everywhere if we say that

\Centerline{$I_{\nu}(f)=I_{\nu}(g)$ whenever $g:X\to\Bbb R$ extends
$f\restr f^{-1}[\Bbb R]$}

\noindent whenever $I_{\nu}(g)=I_{\nu}(g')$ for all such extensions.
}%end of notes

\discrpage

