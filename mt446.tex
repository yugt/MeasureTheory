\frfilename{mt446.tex}
\versiondate{8.10.13}
\copyrightdate{1998}

\def\chaptername{Topological groups}
\def\sectionname{The structure of locally compact groups}

\def\ker{\mathop{\text{ker}}}

\newsection{446}

I develop those fragments of the structure theory of locally compact
Hausdorff topological groups which are needed for the main theorem of
the next section.   Theorem 446B here is of independent interest, being
both itself important and with a proof which uses the measure theory
of this chapter in an interesting way;  but the rest of the section,
from 446D on, is starred.   Note that in this section, unlike the last,
groups are not expected to be abelian.

\leader{446A}{Finite-dimensional representations (a) Definitions (i)}
For any $r\in\Bbb N$, write $M_r=M_r(\Bbb R)$ for the space of
$r\times r$ real matrices.   If we identify it with the space
$\eurm B(\BbbR^r;\BbbR^r)$, where $\BbbR^r$ is given its Euclidean norm,
then $M_r$ becomes a unital Banach algebra\cmmnt{ (4A6C)}, with
identity $I$, the $r\times r$ identity matrix.   Write $GL(r,\Bbb R)$
for the group of invertible elements of $M_r$.

\medskip

\quad{\bf (ii)} Let $X$ be a topological group.   A
{\bf finite-dimensional representation} of $X$ is a continuous
homomorphism from $X$ to a group of the form $GL(r,\Bbb R)$ for some
$r\in\Bbb N$.   If the homomorphism is injective the representation is
called {\bf faithful}\cmmnt{ (cf.\ 4A5Be)}.

\medskip

{\bf (b)} Observe that if $X$ is any topological group and $\phi$ is a
finite-dimensional representation with kernel $Y$, then $X/Y$ has a
faithful finite-dimensional representation $\psi$ defined by writing
$\psi(x^{\ssbullet})=\phi(x)$ for every $x\in X$\cmmnt{ (4A5La)}.

\leader{446B}{Theorem} Let $X$ be a compact Hausdorff topological group.
Then for any $a\in X$, other than the identity, there is a
finite-dimensional representation $\phi$ of $X$ such that
$\phi(a)\ne I$;  and we can arrange that $\phi(x)$ is an orthogonal matrix
for every $x\in X$.

\proof{{\bf (a)} Let $U$ be a symmetric neighbourhood of the identity
$e$ in $X$ such that $a\notin UU$.   Because $X$ is completely regular
(3A3Bb), there is a non-zero continuous function
$h:X\to\coint{0,\infty}$ such that $h(x)=0$ for every
$x\in X\setminus U$;  replacing $h$ by $x\mapsto h(x)+h(x^{-1})$ if
necessary, we may suppose that
$h(x)=h(x^{-1})$ for every $x$.   Let $\mu$ be a (left) Haar measure on
$X$ (441E), and set $w=h^{\ssbullet}$ in $L^0(\mu)$.

\medskip

{\bf (b)} Define an operator $T$ from $L^2=L^2(\mu)$ to itself by
setting

\Centerline{$Tu=u*w$ for every $u\in L^2$,}

\noindent where $*$ is convolution;  that is,
$Tf^{\ssbullet}=(f*h)^{\ssbullet}$ for every
$f\in\eusm L^2=\eusm L^2(\mu)$.  Then $T$ is a compact self-adjoint
operator on the real Hilbert space $L^2$ (444V).

\medskip

{\bf (c)} For any $z\in X$, define $S_z:L^2\to L^2$ by setting
$S_zu=z\action_lu$ for $u\in L^2$,
where $\action_l$ is the left shift action, so that $S_z$ is a
norm-preserving linear operator (443Ge).
Also $S_z$ commutes with $T$.   \Prf\  (443Ge).  ThenIf $f\in\eusm L^2$, then

$$\eqalignno{S_zTf^{\ssbullet}
&=(z\action_l(f*h))^{\ssbullet}
=((z\action_lf)*h)^{\ssbullet}\cr
\noalign{\noindent (444Of)}
&=TS_zf^{\ssbullet}.\text{ \Qed}\cr}$$

\medskip

{\bf (d)} Now $S_aTw\ne Tw$.   \Prf\ Set $g=h*h$, so that
$Tw=g^{\ssbullet}$ and $g$, $a\action_lg$ are both continuous functions
(444Rc).   Then

\Centerline{$(a\action_lg)(e)=g(a^{-1})
=\biggerint h(y)h(y^{-1}a^{-1})dy=0$}

\noindent because if $y\in U$ then $y^{-1}a^{-1}\notin U$, while

\Centerline{$g(e)=\biggerint h(y)h(y^{-1})dy=\int h(y)^2dy>0$.}

\noindent So the open set $\{x:g(x)\ne (a\action_lg)(x)\}$ is non-empty;
because $\mu$ is strictly positive (442Aa), it is non-negligible, and

\Centerline{$S_aTw=(a\action_lg)^{\ssbullet}\ne g^{\ssbullet}=Tw$.  \Qed}

\medskip

{\bf (e)} The closed linear subspace $\{u:S_au=u\}$ therefore does not
include $T[L^2]$.   But the linear span of
$\{Tv:v$ is an eigenvector of $T\}$ is dense in $T[L^2]$ (4A4M),
so there is an eigenvector
$v^*$ of $T$ such that $S_aTv^*\ne Tv^*$.   Let $\gamma\in\Bbb R$ be
such that $Tv^*=\gamma v^*$;  since $Tv^*\ne 0$, $\gamma\ne 0$, and
$V=\{u:u\in L^2$, $Tu=\gamma u\}$ is finite-dimensional (4A4Lb).

\medskip

{\bf (f)} $S_z[V]\subseteq V$ for every $z\in X$.   \Prf\

\Centerline{$TS_zu=S_zTu=S_z(\gamma u)=\gamma S_zu$}

\noindent for every $u\in V$.\ \Qed

We therefore have a map $z\mapsto S_z\restrp V:X\to\eurm B(V;V)$.
As observed in 443Gc, this is actually a semigroup
homomorphism, and of course $S_e\restrp V$ is the identity of
$\eurm B(V;V)$, so $S_z\restrp V$ is always invertible, and we have a
group homomorphism from $X$ to the group of invertible elements of
$\eurm B(V;V)$.   Taking an orthonormal basis $(v_1,\ldots,v_r)$ of $V$,
we have a homomorphism $\phi$ from $X$ to $GL(r,\Bbb R)$, defined by
setting $\phi(z)=\langle\innerprod{S_zv_i}{v_j}\rangle_{1\le i,j\le r}$ for
every $z\in X$.   Moreover, $\phi$ is continuous.   \Prf\ For any
$u\in L^2$,
$z\mapsto S_zu:X\to L^2$ is continuous, by 443Gf.   But this means that
all the maps $z\mapsto\innerprod{S_zv_i}{v_j}$ are continuous;
since the topology
of $GL(r,\Bbb R)$ can be defined in terms of these functionals (see the
formulae in 262H), $\phi$ is continuous.\ \Qed

Thus $\phi$ is a finite-dimensional representation of $X$.   But $V$ was
chosen to contain $v^*$;  of course $Tv^*\in V$, while
$S_aTv^*\ne Tv^*$;  so that $\phi(a)$ is not the identity.

\medskip

{\bf (g)} Finally, $\phi(z)$ is an orthogonal matrix for every $z\in X$.
\Prf\ As observed in (c), $S_z$ is norm-preserving, so $S_z\restrp V$ is
again norm-preserving.   By 4A4Jb,
$\innerprod{S_zv_i}{S_zv_j}=\innerprod{v_i}{v_j}$ for $1\le i,j\le r$, that
is, $\phi(z)$ is orthogonal.\ \Qed
}%end of proof of 446B

\leader{446C}{Corollary} Let $X$ be a compact Hausdorff topological
group.   Then for any neighbourhood $U$ of the identity of $X$ there is
a finite-dimensional representation of $X$ with kernel included in $U$.

\proof{ Let $\Phi$ be the set of finite-dimensional representations of
$X$.   By 446B, $\bigcap_{\phi\in\Phi}\ker(\phi)=\{e\}$, where $e$ is
the identity of $X$.   Because $X\setminus\interior U$ is compact and
disjoint from $\bigcap_{\phi\in\Phi}\ker(\phi)$ (and $\ker(\phi)$ is
closed for every $\phi$), there must be $\phi_0,\ldots,\phi_n\in\Phi$
such that $\bigcap_{i\le n}\ker(\phi_i)\subseteq U$.   For each $i\le
n$, let $r_i\in\Bbb N$ be the integer such that $\phi_i$ is a continuous
homomorphism from $X$ to $GL(r_i,\Bbb R)$.   Set $r=\sum_{i=0}^nr_i$.
Then we have a map $\phi:X\to GL(r,\Bbb R)$
given by the formula

$$\phi(x)=\Matrix{\phi_0(x)&\tbf{0}&\ldots&\tbf{0}\\
   \tbf{0}&\phi_1(x)&\ldots&\tbf{0}\\
   \ldots&\ldots&\ldots&\ldots\\
   \tbf{0}&\tbf{0}&\ldots&\phi_n(x)}$$

\noindent for every $x\in X$.   It is easy to check that $\phi$ is a
continuous homomorphism, and
$\ker(\phi)=\bigcap_{i\le n}\ker(\phi_i)\subseteq U$.
So we have an appropriate representation of $X$.
}%end of proof of 446C


\leader{*446D}{Notation (a)}\cmmnt{ It will help to be clear on an
elementary point of notation.}   If $X$ is a group and $A$ is a subset
of $X$ I will write
$A^0=\{e\}$\cmmnt{, where $e$ is the identity of $X$,}
%observe that  \emptyset^0=\{e\}  cf  0^0=1
and $A^{n+1}=AA^n$ for $n\in\Bbb N$\cmmnt{,
so that $A^3=\{x_1x_2x_3:x_1,\,x_2,\,x_3\in A\}$, etc}.
\cmmnt{Now we find that $A^{m+n}=A^mA^n$ and $A^{mn}=(A^m)^n$ for all
$m$, $n\in\Bbb N$.   Writing $A^{-1}=\{x^{-1}:x\in A\}$ as usual, we
have $(A^r)^{-1}=(A^{-1})^r$.   But note that if we also continue to
write $A^{-1}=\{x^{-1}:x\in A\}$, then $AA^{-1}$ is not in general equal
to $A^0$;  and that there is no simple relation between $A^r$, $B^r$ and
$(AB)^r$, unless $X$ is abelian.}

\wheader{446D}{0}{0}{0}{72pt}

\spheader 446Db\cmmnt{ In the rest of this section, I shall make
extensive use
of the following device.}   If $X$ is a group with identity $e$, $e\in
A\subseteq X$ and
$n\in\Bbb N$, write $D_n(A)=\{x:x\in X,\,x^i\in A$ for every $i\le n\}$.

\inset{(i) $D_0(A)=X$.}

\inset{(ii) $D_1(A)=A$.}

\inset{(iii) $D_n(A)\subseteq D_m(A)$ whenever $m\le n$.}

\inset{(iv) $D_{mn}(A)\subseteq D_m(D_n(A))$ for all $m$, $n\in\Bbb N$.}

\cmmnt{\noindent \Prf\ If $x\in D_{mn}(A)$ then $(x^i)^j\in A$
whenever $j\le n$ and $i\le m$.\ \Qed}

\inset{(v) If $r\in\Bbb N$ and $A^r\subseteq B$, then $D_n(A)\subseteq
D_{nr}(B)$ for every $n\in\Bbb N$;  in particular, $A\subseteq D_r(B)$.}

\prooflet{\noindent\Prf\ For $r=0$ this is trivial.   Otherwise, take
$x\in D_n(A)$ and $i\le nr$.   Then
we can express $i$ as $i_1+\ldots+i_r$ where $i_j\le n$ for each $j$, so
that

\Centerline{$x^i=\prod_{j=1}^rx^{i_j}\in A^r\subseteq B$.  \Qed}
}%end of prooflet

\inset{(vi) If $A=A^{-1}$ then $D_n(A)=D_n(A)^{-1}$ for every $n\in\Bbb
N$.}

\inset{(vii) If $D_m(A)\subseteq B$ where $m\in\Bbb N$, then
$D_{mn}(A)\subseteq D_n(B)$ for every $n\in\Bbb N$.}

\prooflet{\noindent\Prf\ If $x\in D_{mn}(A)$ and $i\le n$, then
$x^{ij}\in A$ for
every $j\le m$, so $x^i\in D_m(A)\subseteq B$.\ \Qed}

\medskip

\spheader 446Dc In (b), if $X$ is a topological group and $A$ is closed,
then every $D_n(A)$ is closed;  if moreover $A$ is compact, then
$D_n(A)$ is compact for every $n\ge 1$.   If $A$ is a neighbourhood of
$e$, then so is every $D_n(A)$\prooflet{, because the map $x\mapsto x^i$
is continuous for every $i\le n$}.

\leader{*446E}{Lemma} Let $X$ be a group, and $U\subseteq X$.   Let
$f:X\to\coint{0,\infty}$ be a bounded function such that $f(x)=0$ for
$x\in X\setminus U$;  set $\alpha=\sup_{x\in X}f(x)$.   Let
$A\subseteq X$ be a symmetric set containing $e$, and $K$ a set
including $A^k$, where $k\ge 1$.   Define $g:X\to\coint{0,\infty}$ by
setting

\Centerline{$g(x)=\Bover1k\sum_{i=0}^{k-1}\sup\{f(yx):y\in A^i\}$}

\noindent for $x\in X$.   Then

(a) $f(x)\le g(x)\le\alpha$ for every $x\in X$, and $g(x)=0$ if
$x\notin K^{-1}U$.

(b) $|g(ax)-g(x)|\le\Bover{j\alpha}k$ if $j\in\Bbb N$, $a\in A^j$ and
$x\in X$.

(c) For any $x$, $z\in X$,
$|g(x)-g(z)|\le\sup_{y\in K}|f(yx)-f(yz)|$.

\proof{{\bf (a)} Of course

\Centerline{$f(x)=\Bover1k\sum_{i=0}^{k-1}f(ex)
\le g(x)\le\Bover1k\sum_{i=0}^{k-1}\alpha=\alpha$}

\noindent for every $x$.   Suppose that $g(x)\ne 0$.   Then there must
be an $i<k$ and a
$y\in A^i$ such that $f(yx)\ne 0$, so that $yx\in U$.   But also,
because $e\in A$, $y\in A^k\subseteq K$, so
$x\in y^{-1}U\subseteq K^{-1}U$.

\medskip

{\bf (b)} Suppose first that $j=1$, so that $a\in A$.   If $\epsilon>0$
there are $y_i\in A^i$, for $i<k$, such that
$g(ax)\le\Bover1k\sum_{i=0}^{k-1}f(y_iax)+\epsilon$.   Now $y_ia\in
A^{i+1}$ for each $i$, so

\Centerline{$g(x)\ge\Bover1k\sum_{i=0}^{k-2}f(y_iax)
\ge g(ax)-\epsilon-\Bover{\alpha}k$.}

\noindent As $\epsilon$ is arbitrary, $g(x)\ge g(ax)-\bover{\alpha}k$.
Similarly, as $a^{-1}\in A$ (because $A$ is symmetric), $g(ax)\ge
g(x)-\bover{\alpha}k$ and
$|g(ax)-g(x)|\le\bover{\alpha}k$.

For the general case, induce on $j$.   (If $j=0$, then $a=e$ and the
result is trivial.)

\medskip

{\bf (c)} Set $\gamma=\sup_{y\in K}|f(yx)-f(yz)|$.   If $\epsilon>0$,
there are $y_i\in A^i$, for $i<k$, such that
$g(x)\le\Bover1k\sum_{i=0}^{k-1}f(y_ix)+\epsilon$.   Now every $y_i$
belongs to $K$, so

\Centerline{$g(z)\ge\Bover1k\sum_{i=0}^{k-1}f(y_iz)
\ge\Bover1k\sum_{i=0}^{k-1}(f(y_ix)-\gamma)
\ge g(x)-\epsilon-\gamma$.}

\noindent As $\epsilon$ is arbitrary, $g(z)\ge g(x)-\gamma$;
similarly, $g(x)\ge g(z)-\gamma$.
}%end of proof of 446E

\leader{*446F}{Lemma} Let $X$ be a locally compact Hausdorff
topological group and $\sequencen{A_n}$ a sequence of
closed symmetric subsets
of $X$ all containing the identity $e$ of $X$.   Suppose that for every
neighbourhood $W$ of $e$ there is
an $n_0\in\Bbb N$ such that $A_n\subseteq W$ for every $n\ge n_0$.   Let
$U$ be a compact neighbourhood of $e$ and suppose that for each
$n\in\Bbb N$ we have $k(n)\in\Bbb N$ such that $A_n^{k(n)}\subseteq U$
and $A_n^{k(n)+1}\not\subseteq U$.   Let $\Cal F$ be a non-principal
ultrafilter on $\Bbb N$ and write $Q$ for the limit
$\lim_{n\to\Cal F}A_n^{k(n)}$ in the space $\Cal C$
of closed subsets of $X$ with the Fell topology.

(i) If $Q^2=Q$ then $Q$ is a compact subgroup of $X$ included in $U$ and
meeting the boundary of $U$.

(ii) If $Q^2\ne Q$ then there are a neighbourhood $W$ of $e$ and an
infinite set $I\subseteq\Bbb N$ such that for every $n\in I$ there are
an $x\in A_n$ and an $i\le k(n)$ such that $x^i\notin W$.

\proof{{\bf (a)} I ought to begin by explaining why the limit
$\lim_{n\to\Cal F}A_n^{k(n)}$ is defined;  this is just because the
Fell topology is always compact (4A2T(b-iii)) and when based on
a locally compact Hausdorff space is Hausdorff (4A2T(e-ii)).

Because $U$ is closed, $\{F:F\in\Cal C$, $F\subseteq U\}$ is closed,
by the definition of the Fell topology (4A2T(a-ii));
because every $A_n^{k(n)}$ is included in
$U$, so is $Q$, and $Q$ is compact.   Because
$x\mapsto x^{-1}$ is a homeomorphism of $X$,
$F\mapsto F^{-1}$ is a homeomorphism of $\Cal C$, and

\Centerline{$Q^{-1}=\lim_{n\to\Cal F}(A_n^{k(n)})^{-1}
=\lim_{n\to\Cal F}(A_n^{-1})^{k(n)}
=\lim_{n\to\Cal F}A_n^{k(n)}=Q$.}

\noindent And of course $e\in Q$ because $e\in A_n^{k(n)}$ for every
$n$ and $\{(x,F):x\in F\in\Cal C\}$ is closed in $X\times\Cal C$
(4A2T(e-i)).

For each $n\in\Bbb N$, we have an $a_n\in A_n^{k(n)}$ and an
$x_n\in A_n$ such that $a_nx_n\notin U$.   Now $a=\lim_{n\to\Cal F}a_n$
is defined (because $U$ is compact), and belongs to $Q$.   Also
$\lim_{n\to\infty}x_n=e$, because every neighbourhood of $e$ includes
all but finitely many of the $A_n$, so
$a=\lim_{n\to\Cal F}a_nx_n\notin\interior U$, and $a$ belongs to the
boundary of $U$.   Thus $Q$ meets the boundary of $U$.

\medskip

{\bf (b)} From (a) we see that if $Q^2=Q$ then $Q$ is a compact subgroup
of $X$, included in $U$ and meeting the boundary of $U$.   So henceforth
let us suppose that $Q^2\ne Q$ and seek to prove (ii).

Let $w\in Q^2\setminus Q$.   Let $W_0\subseteq U$ be an open
neighbourhood of $e$ such that $W_0wW_0^2\cap QW_0^2=\emptyset$ (4A5Ee).

\medskip

{\bf (c)} Fix a left Haar measure $\mu$ on $X$.   Let
$f:X\to\coint{0,\infty}$ be a continuous function such that
$\{x:f(x)>0\}\subseteq W_0$ and $\int f(x)dx=1$.   Set
$\alpha=\sup_{x\in X}f(x)$ and $\beta=\int f(x)^2dx$, so that $\alpha$ is
finite and $\beta>0$.   $W_0U^2W_0\subseteq U^4$ is open and
relatively compact, so has
finite measure, and there is an $\eta>0$ such that

\Centerline{$2\eta(1+\alpha\mu(W_0U^2W_0))<\beta$.}

\noindent Let $W$ be a neighbourhood of $e$ such that $W\subseteq W_0$
and $|f(yax)-f(ybx)|\le\eta$ whenever $y\in(U^{-1})^2\cup U$, $x\in X$
and $ab^{-1}\in W$ (4A5Pa).

\medskip

{\bf (d)} Express $w$ as $w'w''$ where $w'$, $w''\in Q$.   Then

\Centerline{$\{n:A_n^{k(n)}\cap W_0w'\ne\emptyset\}$,
\quad$\{n:A_n^{k(n)}\cap w''W_0\ne\emptyset\}$,}

\Centerline{$\{n:A_n^{k(n)}\subseteq QW_0\}
=\{n:A_n^{k(n)}\cap(U\setminus QW_0)=\emptyset\}$}

\noindent all belong to $\Cal F$, by the definition of the Fell topology.
Also

\Centerline{$\{n:k(n)\ge 1\}$}

\noindent is cofinite in $\Bbb N$, because $A_n\subseteq U$ for all $n$
large enough.   Let $I$ be the intersection of these four sets, so that
$I$ belongs to $\Cal F$ and must be infinite.

\medskip

{\bf (e)} Let $n\in I$.   \Quer\ Suppose, if possible, that $x^i\in W$
for every $x\in A_n$ and $i\le k(n)$.   (The rest of the proof will be a
search for a contradiction.)   Note that $k(n)\ge 1$.

Choose $x_j\in A_n$, for $j<2k(n)$, such that the products
$x_{2k(n)-1}x_{2k(n)-2}\ldots x_{k(n)}$,
$x_{k(n)-1}\ldots x_{0}$ belong to $W_0w'$, $w''W_0$ respectively;  set
$\tilde w=x_{2k(n)-1}\ldots x_0$, so that

\Centerline{$\tilde w\in A_n^{2k(n)}\cap W_0w'w''W_0\subseteq W_0wW_0$.}

\noindent Since $A_n^{k(n)}\subseteq QW_0$ and $W_0wW_0^2\cap QW_0^2$ is
empty, $\tilde wW_0$ does not meet $A_n^{k(n)}W_0$.

\medskip

{\bf (f)} Define $g:X\to\coint{0,\infty}$ by setting

\Centerline{$g(x)
=\Bover1{k(n)}\sum_{i=0}^{k(n)-1}\sup\{f(yx):y\in A_n^i\}$.}

\noindent Then $g(\tilde wx)f(x)=0$ for every $x\in X$.   \Prf\ If
$f(x)\ne 0$, then $x\in W_0$, so $\tilde wx\notin A_n^{k(n)}W_0$, and
$g(\tilde wx)=0$, by 446Ea.\ \QeD\  Accordingly

\Centerline{$\biggerint(g(x)-g(\tilde wx))f(x)dx
=\int g(x)f(x)dx\ge\beta$}

\noindent since $g\ge f$ (446Ea).

Set $y_0=e$ and $y_{i+1}=x_{i}y_i$ for $i<2k(n)$, so that
$y_{2k(n)}=\tilde w$ and

\Centerline{$y_i\in A_n^i\subseteq A_n^{2k(n)}
=(A_n^{k(n)})^2\subseteq U^2$}

\noindent for every $i\le 2k(n)$.   Then

\Centerline{$g(x)-g(\tilde wx)
=\sum_{i=0}^{2k(n)-1}g(y_ix)-g(y_{i+1}x)$}

\noindent for every $x\in X$.   Let $i<2k(n)$ be such that

\Centerline{$\biggerint(g(y_ix)-g(y_{i+1}x))f(x)dx
\ge\Bover1{2k(n)}\int(g(x)-g(\tilde wx))f(x)dx
\ge\Bover{\beta}{2k(n)}$.}

\noindent Set $u=x_i$ and $v=y_i$, so that $u\in A_n$, $v\in U^2$ and

$$\eqalign{\int(g(x)-g(ux))f(v^{-1}x)dx
&=\int(g(vx)-g(uvx))f(x)dx\cr
&=\int(g(y_ix)-g(y_{i+1}x))f(x)dx
\ge\Bover{\beta}{2k(n)}.\cr}$$

\medskip

{\bf (g)} We have

$$\eqalign{\int(g(x)-g&(u^{k(n)}x))f(v^{-1}x)dx\cr
&=\sum_{j=0}^{k(n)-1}\int(g(u^jx)-g(u^{j+1}x))f(v^{-1}x)dx\cr
&=\sum_{j=0}^{k(n)-1}\int(g(x)-g(ux))f(v^{-1}u^{-j}x)dx\cr
&=k(n)\int(g(x)-g(ux))f(v^{-1}x)dx\cr
  &\qquad\qquad +\sum_{j=0}^{k(n)-1}\int(g(x)-g(ux))
                    (f(v^{-1}u^{-j}x)-f(v^{-1}x))dx,\cr}$$

\noindent that is,

$$\eqalign{k(n)\int&(g(x)-g(ux))f(v^{-1}x)dx\cr
&=\int(g(x)-g(u^{k(n)}x))f(v^{-1}x)dx\cr
 &\qquad\qquad -\sum_{j=0}^{k(n)-1}\int(g(x)-g(ux))
        (f(v^{-1}u^{-j}x)-f(v^{-1}x))dx.\cr}$$

\noindent Set

\Centerline{$\beta_1=\sum_{j=0}^{k(n)-1}$$\biggerint(g(x)-g(ux))
        (f(v^{-1}u^{-j}x)-f(v^{-1}x))dx$,}

\Centerline{$\beta_2=\biggerint(g(x)-g(u^{k(n)}x))f(v^{-1}x)dx$;}

\noindent then

\Centerline{$\beta_2-\beta_1
=k(n)\biggerint(g(x)-g(ux))f(v^{-1}x)dx
\ge\bover12\beta$.}

\medskip

{\bf (h)}\grheada\ Examine $\beta_1$.   We know that, because $u\in A_n$,
$|g(x)-g(ux)|\le\Bover{\alpha}{k(n)}$ for every $x$ (see 446Eb).   On
the other hand, we are supposing that $x^j\in W$ for every $j\le k(n)$
and every $x\in A_n$, so, in particular, $u^j\in W\subseteq W_0$ for
every $j\le k(n)$.   Also, as noted in (f), $v\in U^2$.  So for any
$j<k(n)$ we must have
$|f(v^{-1}u^{-j}x)-f(v^{-1}x)|\le\eta$ for every $x\in X$, by the choice
of $W$, while $f(v^{-1}u^{-j}x)-f(v^{-1}x)=0$ unless $x\in W_0U^2W_0$.
So

$$\eqalign{|\beta_1|
&\le\sum_{j=0}^{k(n)-1}\int|g(x)-g(ux)|
        |f(v^{-1}u^{-j}x)-f(v^{-1}x)|dx\cr
&\le\sum_{j=0}^{k(n)-1}\Bover{\alpha}{k(n)}\eta\mu(W_0U^2W_0)
=\alpha\eta\mu(W_0U^2W_0).\cr}$$

\medskip

\quad\grheadb\ Now consider $\beta_2$.   As $u^{k(n)}\in W$,
$|f(zu^{k(n)}x)-f(zx)|\le\eta$ for every $z\in U$ and $x\in X$,
by the choice of $W$, so (because $A_n^{k(n)}\subseteq U$)
$|g(u^{k(n)}x)-g(x)|\le\eta$ for every $x$ (446Ec).   Accordingly

\Centerline{$|\beta_2|\le\eta\biggerint f(v^{-1}x)dx
=\eta\int f(x)dx=\eta$.}

\medskip

{\bf (i)} But this means that

\Centerline{$\beta\le 2(|\beta_1|+|\beta_2|)
\le 2\eta(1+\alpha\mu(W_0U^2W_0))<\beta$,}

\noindent which is absurd.\ \Bang

\medskip

{\bf (j)} Thus for every $n\in I$ there are an $x\in A_n$ and an $i\le
k(n)$ such that $x^i\notin W$, and (ii) is true.   This completes the
proof.
}%end of proof of 446F

\leader{*446G}{`Groups with no small subgroups' (a) Definition} Let
$X$ be a topological group.   We say that $X$ {\bf has no small
subgroups} if there is a neighbourhood $U$ of the identity $e$ of $X$
such that the only subgroup of $X$ included in $U$ is $\{e\}$.

\spheader 446Gb If $X$ is a Hausdorff topological group and $U$ is a
compact symmetric neighbourhood of the identity $e$ such that the only
subgroup of $X$ included in $U$ is $\{e\}$, then $\{D_n(U):n\in\Bbb N\}$
is a base of neighbourhoods of
$e$\cmmnt{, where $D_n(U)=\{x:x\in X,\,x^i\in U$
for every $i\le n\}$}.   \prooflet{\Prf\ By 446Dc,
$\langle D_n(U)\rangle_{n\ge 1}$ is a non-increasing
sequence of compact neighbourhoods of $e$, and if
$x\in\bigcap_{n\in\Bbb N}D_n(U)$ then $x^i\in U$ for every $i\in\Bbb N$;
as $U^{-1}=U$, $U$
includes the subgroup $\{x^i:i\in\Bbb Z\}$, so $x=e$.   Thus
$\bigcap_{n\in\Bbb N}D_n(U)=\{e\}$ and $\{D_n(U):n\in\Bbb N\}$ is a base
of neighbourhoods of $e$ (4A2Gd).\ \Qed}

\spheader 446Gc In particular, a locally compact Hausdorff topological
group with no small subgroups is metrizable\cmmnt{ (4A5Q)}.

\leader{*446H}{Lemma} Let $X$ be a locally compact Hausdorff
topological group.   \cmmnt{For $A\subseteq X$, $n\in\Bbb N$ set
$D_n(A)=\{x:x^i\in A$ for every $i\le n\}$.}
If $U\subseteq X$ is a compact symmetric
neighbourhood of the identity which does not include any subgroup of $X$
other than $\{e\}$, then
there is an $r\ge 1$ such that $D_{rn}(U)^n\subseteq U$ for every
$n\in\Bbb N$.

% and $D_n(U)^n\subseteq U^r$ for every $n\ge 1$, not needed I think

\proof{\Quer\ Suppose, if possible, that for every $r\ge 1$
there is an $n_r\in\Bbb N$ such that $D_{rn_r}(U)^{n_r}\not\subseteq U$.
Of course $n_r\ge 1$.   Set $A_0=U$ and
$A_r=D_{rn_r}(U)$ for $r\ge 1$.   Note that $D_{n_1}(U)\subseteq A_0$ but
$D_{n_1}(U)^{n_1}\not\subseteq U$, so $A_0^{n_1}\not\subseteq U$.
We therefore have, for every $r\in\Bbb N$, a $k_r$ such that
$A_r^{k(r)}\subseteq U$ but $A_r^{k(r)+1}\not\subseteq U$.   Also, by
446Gb, every neighbourhood of $e$ includes all but finitely many of the
$A_r$.   We can therefore apply 446F to the sequence
$\sequence{r}{A_r}$.   Of course $k(r)\ge 1$ for every $r$, while
$k(r)<n_r$ for $r\ge 1$.   Since $U$ includes no
non-trivial subgroup, (i) of 446F is impossible, and we are left with
(ii).   Let $W$, $I$ be as declared there, so that
$A_r\not\subseteq D_{k(r)}(W)$ for every $r\in I$.   There must be some
$m\ge 1$ such that $D_m(U)\subseteq W$ (446Gb).
Take $r\in I$ such that $r\ge m$;  then $rn_r\ge mk(r)$, so

$$\eqalignno{A_r
&=D_{rn_r}(U)
\subseteq D_{mk(r)}(U)
\subseteq D_{k(r)}(D_m(U))\cr
\noalign{\noindent (446D(b-iv))}
&\subseteq D_{k(r)}(W),\cr}$$

\noindent which is impossible.\ \Bang
}%end of proof of 446H

\leaveitout{(proof of other bit)
{\bf (b)} Thus we have some $l\ge 1$ such that
$D_{ln}(U)^n\subseteq U$ for every $n\in\Bbb N$.   Set $r=2l$;  we surely have
$D_{rn}(U)^n\subseteq U$ for every $n\ge 1$.   If $1\le n\le l$, then
$D_n(U)^n\subseteq U^n\subseteq U^r$.   If $n\ge l$, then $n$ is
expressible as $il+j$ where
$i\ge 1$, $j<l$.   So

\Centerline{$D_n(U)^n
\subseteq D_{il}(U)^{(i+1)l}
\subseteq D_{il}(U)^{2il}
=(D_{il}(U)^i)^{r}
\subseteq U^r$.}

\noindent So we have the result.}

\leader{*446I}{Lemma} Let $X$ be a locally compact Hausdorff
topological
group and $U$ a compact symmetric neighbourhood of the identity in $X$
such that $U$ does not include any subgroup of $X$ other than $\{e\}$.
\dvro{Let}{For $n\in\Bbb N$,
set $D_n(U)=\{x:x^i\in U$ for every $i\le n\}$, and
let} $\Cal F$ be any non-principal ultrafilter on $\Bbb N$.
Suppose that $\sequencen{x_n}$ is a sequence in $X$ such that
$x_n\in D_n(U)$ for every $n\in\Bbb N$.   Then we have a continuous
homomorphism
$q:\Bbb R\to X$ defined by setting $q(t)=\lim_{n\to\Cal F}x_n^{i(n)}$
whenever $\sequencen{i(n)}$ is a sequence in $\Bbb Z$ such that
$\lim_{n\to\Cal F}\Bover{i(n)}n=t$ in $\Bbb R$.

\proof{{\bf (a)} If $\sequencen{i(n)}$ is any sequence in $\Bbb N$ such
that $\lim_{n\to\Cal F}\Bover{i(n)}n$ is defined in $\Bbb R$, then
$\lim_{n\to\Cal F}x_n^{i(n)}$ is defined in $X$.   \Prf\ There is some
$m\in\Bbb N$ such that $m>\lim_{n\to\Cal F}\Bover{i(n)}n$, so that
$J=\{n:i(n)\le mn\}\in\Cal F$;  but if $n\in J$, then

\Centerline{$x_n\in D_n(U)\subseteq D_{mn}(U^m)\subseteq
D_{i(n)}(U^m)$,}

\noindent by 446D(b-v), and $x_n^{i(n)}\in U^m$.   But this means that
$\Cal F$ contains $\{n:x_n^{i(n)}\in U^m\}$;  as $U^m$ is compact,
$\lim_{n\to\Cal F}x_n^{i(n)}$ is defined in $X$.\ \Qed

More generally, if $\sequencen{i(n)}$ is any sequence in $\Bbb Z$ such
that $\lim_{n\to\Cal F}\Bover{i(n)}n$ is defined in $\Bbb R$, then
$\lim_{n\to\Cal F}x_n^{i(n)}$ is defined in $X$.   \Prf\ At least one of
$\{n:i(n)\ge 0\}$, $\{n:i(n)\le 0\}$ belongs to $\Cal F$.   In the
former case,
$\lim_{n\to\Cal F}x_n^{i(n)}=\lim_{n\to\Cal F}x_n^{\max(0,i(n))}$ is
defined;  in the latter case,

\Centerline{$\lim_{n\to\Cal F}x_n^{i(n)}
=\lim_{n\to\Cal F}(x_n^{\max(0,-i(n))})^{-1}
=(\lim_{n\to\Cal F}x_n^{\max(0,-i(n))})^{-1}$}

\noindent is defined.\ \Qed

\medskip

{\bf (b)} If $V$ is any neighbourhood of $e$, there is a $\delta>0$ such
that $\lim_{n\to\Cal F}x_n^{i(n)}\in V$ whenever $\sequencen{i(n)}$ is a
sequence in $\Bbb Z$ such that $\lim_{n\to\Cal F}|\Bover{i(n)}n|\le\delta$.
\Prf\ By 446Gb, there is an $m\ge 1$ such that $D_m(U)\subseteq V$.
Take $\delta<\bover1m$.   If $\lim_{n\to\Cal F}|\Bover{i(n)}n|\le\delta$, then
$J=\{n:m|i(n)|\le n\}\in\Cal F$.   But if $n\in J$, then

\Centerline{$x_n\in D_n(U)\subseteq D_{m|i(n)|}(U)
\subseteq D_{|i(n)|}(D_m(U))$}

\noindent and $x_n^{|i(n)|}\in D_m(U)$;  since $D_m(U)$, like $U$, is
symmetric (446D(b-vi)), $x_n^{i(n)}\in D_m(U)$.   This is true for every
$n\in J$, so $\lim_{n\to\Cal F}x_n^{i(n)}\in D_m(U)\subseteq V$, as
required.\ \Qed

\medskip

{\bf (c)} It follows at once that if $\sequencen{i(n)}$ is a
sequence in $\Bbb Z$ such that $\lim_{n\to\Cal F}\Bover{i(n)}n=0$, then
$\lim_{n\to\Cal F}x_n^{i(n)}=e$.   Consequently,
if $\sequencen{i(n)}$, $\sequencen{j(n)}$ are sequences
in $\Bbb Z$ such that $\lim_{n\to\Cal F}\Bover{i(n)}n$ and
$\lim_{n\to\Cal F}\Bover{j(n)}n$ both exist in $\Bbb R$ and are equal, then
$\lim_{n\to\Cal F}x_n^{i(n)}=\lim_{n\to\Cal F}x_n^{j(n)}$.   \Prf\ Set
$k(n)=i(n)-j(n)$.   Then $\lim_{n\to\Cal F}x_n^{k(n)}=e$ because
$\lim_{n\to\Cal F}\Bover{k(n)}n=0$.   But now

\Centerline{$\lim_{n\to\Cal F}x_n^{i(n)}
=\lim_{n\to\Cal F}x_n^{j(n)}x_n^{k(n)}
=\lim_{n\to\Cal F}x_n^{j(n)}$.\ \Qed}

\medskip

{\bf (d)} We do therefore have a function $q:\Bbb R\to X$ defined by the
given formula.   Now $q(s+t)=q(s)q(t)$ for all $s$, $t\in\Bbb R$.
\Prf\ Take sequences $\sequencen{i(n)}$, $\sequencen{j(n)}$ such that
$s=\lim_{n\to\infty}\Bover{i(n)}n$, $t=\lim_{n\to\infty}\Bover{j(n)}n$;  then
$s+t=\lim_{n\to\infty}\Bover{i(n)+j(n)}n$, so

\Centerline{$q(s+t)
=\lim_{n\to\Cal F}x_n^{i(n)+j(n)}
=\lim_{n\to\Cal F}x_n^{i(n)}x_n^{j(n)}
=q(s)q(t)$.  \Qed}

\medskip

{\bf (e)} Thus $q$ is a homomorphism.   Finally, (b) shows that it is
continuous at $0$, so it must be continuous (4A5Fa).
}%end of proof of 446I

\leader{*446J}{Lemma} Let $X$ be a locally compact Hausdorff
topological
group with no small subgroups.   Then there is a neighbourhood $V$ of
the identity $e$ such that $x=y$ whenever $x$, $y\in V$ and $x^2=y^2$.

\proof{ Let $U$ be a symmetric compact neighbourhood of $e$ not
including any subgroup of $X$ except $\{e\}$.   Let $\sequence{n}{U_n}$
be a non-increasing sequence of neighbourhoods of $e$ comprising a base
of neighbourhoods of $e$ (446Gc), and with $U_0=U$.

\Quer\ Suppose, if possible, that for each $n\in\Bbb N$ there are
distinct $x_n$, $y_n\in U_n$ such that $x_n^2=y_n^2$.   Set
$a_n=x_n^{-1}y_n$;  then

\Centerline{$x_n^{-1}a_nx_n=x_n^{-2}y_nx_n=y_n^{-1}x_n=a_n^{-1}$.}

\noindent Accordingly

\Centerline{$x_n^{-1}a_n^mx_n=(x_n^{-1}a_nx_n)^m=a_n^{-m}$}

\noindent for every $m\in\Bbb N$.

Since $U$ includes no non-trivial subgroup, and $a_n\ne e$, there is a
$k(n)\in\Bbb N$ such that $a_n^i\in U$ for $i\le k(n)$ and
$a_n^{k(n)+1}\notin U$.   Let $\Cal F$ be any non-principal ultrafilter
on $\Bbb N$;  then $a=\lim_{n\to\Cal F}a_n^{k(n)}$ is defined and
belongs to $U$.   Also, because $\sequencen{x_n}$ and $\sequencen{y_n}$
both converge to $e$, so does $\sequencen{a_n}$, and

\Centerline{$a=\lim_{n\to\Cal F}a_n^{k(n)+1}\in X\setminus\interior U$;}

\noindent thus $a$ cannot be $e$.

However, $x_n^{-1}a_n^{k(n)}x_n=a_n^{-k(n)}$ for each $n$, so

\Centerline{$e^{-1}ae=\lim_{n\to\Cal F}x_n^{-1}a_n^{k(n)}x_n
=\lim_{n\to\Cal F}(a_n^{k(n)})^{-1}
=a^{-1}$.}

\noindent So $a=a^{-1}$ and $\{e,a\}$ is a non-trivial subgroup of $X$
included in $U$, which is supposed to be impossible.\ \Bang

Thus some $U_n$ serves for $V$.
}%end of proof of 446J

\leader{*446K}{Lemma} Let $X$ be a locally compact Hausdorff
topological
group with no small subgroups.   \cmmnt{For $A\subseteq X$ set
$D_n(A)=\{x:x^i\in A$ for every $i\le n\}$.}   Then there is a compact
symmetric neighbourhood $U$ of the identity $e$ such that whenever $V$
is a neighbourhood of $e$ there are an $n_0\in\Bbb N$ and a
neighbourhood $W$ of $e$ such that whenever $n\ge n_0$, $x\in D_n(U)$,
$y\in D_n(U)$ and $x^ny^n\in W$, then $xy\in D_n(V)$.

\proof{{\bf (a)} Let $U_0$ be a compact symmetric neighbourhood of $e$
such that ($\alpha$) $U_0^3$ includes no subgroup of $X$ other than
$\{e\}$ ($\beta$) whenever $x$, $y\in U_0$ and $x^2=y^2$, then $x=y$;
such a neighbourhood exists by 446J.
Let $r\ge 1$ be such that $D_{rn}(U_0)^n\subseteq U_0$ for every
$n\in\Bbb N$ (446H).   Let $U$ be a compact symmetric neighbourhood of
$e$ such that $U^r\subseteq U_0$.   In this case
$D_n(U)\subseteq D_{rn}(U_0)$ for
every $n$, by 446D(b-v).   So $D_n(U)^n\subseteq U_0$ for every $n$.

\medskip

{\bf (b)} Fix a left Haar measure $\mu$ on $X$.   Let
$f:X\to\coint{0,\infty}$ be a continuous function such that
$\int f(x)dx=1$ and $f(x)=0$ for $x\in X\setminus U_0$.   Set
$\alpha=\sup_{x\in X}|f(x)|$,
$\beta=\int f(x)^2dx$, so that $\alpha$ is finite and $\beta>0$.   For
$n\ge 1$, set

\Centerline{$f_n(x)
=\Bover1n\sum_{i=0}^{n-1}\sup\{f(yx):y\in D_n(U)^i\}$}

\noindent for $x\in X$.   Because $D_n(U)^n\subseteq U_0$, we can
apply 446E to see that, for each $n$,

\inset{(i) $f_n\ge f$,}

\inset{(ii) $f_n(x)=0$ if $x\notin U_0^2$,}

\inset{(iii) $|f_n(ax)-f_n(x)|\le\Bover{j\alpha}n$ if $j\in\Bbb N$,
$a\in D_n(U)^j$ and $x\in X$,}

\inset{(iv) for any $x$, $z\in X$,
$|f_n(x)-f_n(z)|\le\sup_{y\in U_0}|f(yx)-f(yz)|$.}

\noindent It follows that

\inset{(v) for any $\epsilon>0$ there is a
neighbourhood $W$ of $e$ such that $|f_n(ax)-f_n(bx)|\le\epsilon$
whenever $a$, $b$, $x\in X$, $n\in\Bbb N$ and $ab^{-1}\in W$}

\noindent (4A5Pa again).

\medskip

{\bf (c)} It will help to have the following fact available.   Suppose
we are given sequences $\sequencen{x_n}$, $\sequencen{y_n}$ such that
$x_n$ and $y_n$ belong to $D_n(U)$ for every $n\in\Bbb N$ and
$\lim_{n\to\infty}x_n^ny_n^n=e$.   Write

\Centerline{$\gamma_n
=\sup\{|f_n(y_n^jx)-f_n(x_n^{-j}x)|:j\le n,\,x\in X\}$}

\noindent for each $n\in\Bbb N$.   Then $\lim_{n\to\infty}\gamma_n=0$.
\Prf\Quer\ Otherwise, there is an $\eta>0$ such that
$J=\{n:\gamma_n>\eta\}$ is infinite.   Let $W$ be a neighbourhood of $e$
such that $|f_n(ax)-f_n(bx)|\le\eta$ whenever $n\in\Bbb N$, $x$, $a$,
$b\in X$ and
$ab^{-1}\in W$ ((b-v) above);  let $W'$ be a neighbourhood of $e$ such
that $ab\in W$ whenever $a$, $b\in U$ and $ba\in W'$ (4A5Ej).
Then for each $n\in J$ there must be a
$j(n)\le n$ such that $y_n^{j(n)}x_n^{j(n)}\notin W$, while $x_n^{j(n)}$
and $y_n^{j(n)}$ both belong to $U$, so that
$x^{j(n)}y^{j(n)}\notin W'$.   Let $\Cal F$ be
a non-principal ultrafilter on $\Bbb N$ containing $J$.   By 446I, there
are continuous homomorphisms $q$, $\tilde q$ from $\Bbb R$ to $X$ such
that $q(t)=\lim_{n\to\Cal F}x_n^{-i(n)}$,
$\tilde q(t)=\lim_{n\to\Cal F}y_n^{i(n)}$ whenever $\sequencen{i(n)}$
is a sequence in $\Bbb Z$ such
that $\lim_{n\to\Cal F}\Bover{i(n)}n=t$ in $\Bbb R$.   (Of course $x_n^{-1}\in
D_n(U)$ for every $n$, by 446D(b-vi).)   Setting
$t_0=\lim_{n\to\Cal F}\Bover{j(n)}n\in[0,1]$,

\Centerline{$q(-t_0)\tilde q(t_0)
=\lim_{n\to\Cal F}x_n^{j(n)}y_n^{j(n)}
\notin\interior W'$,}

\noindent so $q(t_0)\ne \tilde q(t_0)$ and $q\ne\tilde q$.   But

\Centerline{$q(-1)\tilde q(1)=\lim_{n\to\Cal F}x_n^ny_n^n=e$,}

\noindent so $q(1)=\tilde q(1)$.   Now if $0\le i(n)\le n$, then
$x_n^{-i(n)}\in D_n(U)^n\subseteq U_0$;  so if $0\le t\le 1$, $q(t)\in
U_0$.   Similarly, $\tilde q(t)\in U_0$ whenever $t\in[0,1]$.   But
recall that $U_0$ was chosen so that if $x$, $y\in U_0$ and $x^2=y^2$
then $x=y$.   In particular, since $q(\bover12)$ and $\tilde
q(\bover12)$ both belong to $U_0$, and their squares $q(1)$, $\tilde
q(1)$ are equal, $q(\bover12)=\tilde q(\bover12)$.   Repeating this
argument, we see that $q(2^{-k})=\tilde q(2^{-k})$ for every $k\in\Bbb
N$, so that $q(2^{-k}i)=\tilde q(2^{-k}i)$ for every $k\in\Bbb N$,
$i\in\Bbb Z$;  since $q$ and $\tilde q$ are supposed to be continuous,
they must be equal;  but $q(t_0)\ne\tilde q(t_0)$.\ \Bang\Qed

\medskip

{\bf (d)} Now let $V$ be any neighbourhood of $e$.

\Quer\ Suppose, if possible, that for every neighbourhood $W$ of $e$ and
$n_0\in\Bbb N$ there are $n\ge n_0$ and $x$, $y\in D_n(U)$ such that
$x^ny^n\in W$ but $xy\notin D_n(V)$.   For $k\in\Bbb N$ choose
$n_k\in\Bbb N$ and $\tilde x_k$, $\tilde y_k\in D_{n_k}(U)$ such that
$\tilde x_k^{n_k}\tilde y_k^{n_k}\in D_k(U)$ but
$\tilde x_k\tilde y_k\notin D_{n_k}(V)$, and $n_k>n_{k-1}$ if $k\ge 1$.
Now we know that
$\sequence{k}{D_k(U)}$ is a non-increasing sequence constituting a base
of neighbourhoods of $e$ (446Gb), so
$\lim_{k\to\infty}\tilde x_k^{n_k}\tilde y_k^{n_k}=e$.    Set
$J=\{n_k:k\in\Bbb N\}$.   For
$n=n_k$, set $x_n=\tilde x_k$, $y_n=\tilde y_k$;  for
$n\in\Bbb N\setminus J$, set $x_n=y_n=e$.
Then $x_n$, $y_n\in D_n(U)$ for every
$n\in\Bbb N$ and $\sequencen{x_n^ny_n^n}\to e$ as $n\to\infty$, while
$x_ny_n\notin D_n(V)$ for $n\in J$.

We know from (c) that

\Centerline{$\gamma_n
=\sup\{|f_n(y_n^jx)-f_n(x_n^{-j}x)|:j\le n,\,x\in X\}\to 0$}

\noindent as $n\to\infty$;  for future reference, take a sequence
$\langle j(n)\rangle_{n\ge 1}$ such that $1\le j(n)\le n$ for every
$n\ge 1$, $\lim_{n\to\infty}\Bover{j(n)}n=0$ and
$\lim_{n\to\infty}\Bover{n\gamma_n}{j(n)}=0$.

\medskip

{\bf (e)} Let $l\ge 1$ be such that $D_l(U_0^3)\subseteq V$ (446Gb
again), and set
$K=U_0^{2l+1}$.   Then $D_{nl}(U_0^3)\subseteq D_n(V)$
for every $n$ (446D(b-vii)).   So $x_ny_n\notin D_{nl}(U_0^3)$ for
$n\in J$.   For each $n\in J$ choose $m(n)\le nl$ such that
$(x_ny_n)^{m(n)}\notin U_0^3$;  for $n\in\Bbb N\setminus J$, set
$m(n)=1$.
Then $(x_ny_n)^{-m(n)}x\notin U_0^2$ for any $x\in U_0$ and $n\in J$, and
$f_n((x_ny_n)^{-m(n)}x)f(x)=0$ for any $x\in X$ and $n\in J$.   So

\Centerline{$|\biggerint(f_n((x_ny_n)^{-m(n)}x)-f_n(x))f(x)dx|
=\int f_n(x)f(x)dx\ge\beta$}

\noindent for $n\in J$, because $f_n\ge f$ ((b-i) above).

\medskip

{\bf (f)} Of course $m(n)>0$ for every $n\in J$, therefore for every
$n$.   So we can set

\Centerline{$g_n(x)=\Bover1{m(n)}\sum_{i=0}^{m(n)-1}f((x_ny_n)^ix)$}

\noindent for $x\in X$ and $n\in\Bbb N$.   Note that $g_n$, like $f$, is
non-negative, and also that $\int g_n(x)dx=\int f(x)dx=1$.   We need to
know that $g_n(x)=0$ if $x\notin K$;  this is because
$(x_ny_n)^i\in D_n(U)^{2nl}\subseteq U_0^{2l}$ for every $i\le m(n)$,
while $f(x)=0$ if
$x\notin U_0$, and $U_0$ is symmetric.

We also have

$$\eqalign{|g_n(ax)-g_n(x)|
&\le\sup_{i<m(n)}|f((x_ny_n)^iax)-f((x_ny_n)^ix)|\cr
&\le\sup\{|f(wax)-f(wx)|:w\in U_0^{2l}\}\cr}$$

\noindent for every $n\in\Bbb N$ and $a$, $x\in X$ (cf.\ 446Ec), so for
every $\eta>0$ there must be a neighbourhood $W$ of $e$ such that
$|g_n(ax)-g(x)|\le\eta$ whenever $n\in\Bbb N$, $a\in W$ and $x\in X$, by
4A5Pa once more.   Since also $g_n(ax)=g_n(x)=0$ if
$a\in U_0$ and $x\notin
U_0^{-1}K$, and $U_0^{-1}K$ has finite measure, we see that for every
$\eta>0$ there is a neighbourhood $W$ of $e$ such that
$\int|g_n(ax)-g(x)|dx\le\eta$ for every $a\in W$, $n\in\Bbb N$.

\medskip

{\bf (g)} Returning to the formula in (e), we see that, for any
$n\in J$,

$$\eqalignno{\beta
&\le\bigl|\int\bigl(f_n((x_ny_n)^{-m(n)}x)-f_n(x)\bigr)
                                                  f(x)dx\bigr|\cr
&=\bigl|\sum_{i=0}^{m(n)-1}
  \int\bigl(f_n((x_ny_n)^{-i-1}x)-f_n((x_ny_n)^{-i}x)\bigr)
                                                 f(x)dx\bigr|\cr
&=\bigl|\sum_{i=0}^{m(n)-1}
  \int\bigl(f_n((x_ny_n)^{-1}x)-f_n(x)\bigr)f((x_ny_n)^ix)dx\bigr|\cr
&=m(n)\bigl|\int\bigl(f_n((x_ny_n)^{-1}x)-f_n(x)\bigr)g_n(x)dx\bigr|\cr
&\le ln\bigl|\int\bigl(f_n(y_n^{-1}x_n^{-1}x)-f_n(x)\bigr)
                                               g_n(x)dx\bigr|\cr
&\le ln\bigl|\int\bigl(f_n(y_n^{-1}x)-f_n(x)-f_n(y_n^{-1}x_n^{-1}x)
   +f_n(x_n^{-1}x)\bigr)g_n(x)dx\bigr|\cr
&\qquad\qquad\qquad
   +ln\bigl|\int\bigl(2f_n(x)-f_n(x_n^{-1}x)-f_n(y_n^{-1}x)\bigr)
                                                g_n(x)dx\bigr|.\cr}$$

\noindent Next,

$$\eqalign{j(n)\bigl(2&f_n(x)-f_n(x_n^{-1}x)-f_n(y_n^{-1}x)\bigr)\cr
&=j(n)\bigl(f_n(x)-f_n(x_n^{-1}x)\bigr)
  -f_n(x)+f_n(x_n^{-j(n)}x)\cr
&\qquad\qquad +j(n)\bigl(f_n(x)-f_n(y_n^{-1}x)\bigr)
  -f_n(x)+f_n(y_n^{-j(n)}x)\cr
&\qquad\qquad +2f_n(x)-f_n(x_n^{-j(n)}x)
  -f_n(y_n^{-j(n)}x)\cr}$$

\noindent for every $x$, and finally

$$\eqalign{\int\bigl(2f_n(x)&-f_n(x_n^{-j(n)}x)-f_n(y_n^{-j(n)}x)\bigr)
                                                 g_n(x)dx\cr
&=\int\bigl(f_n(x)-f_n(y_n^{-j(n)}x)\bigr)g_n(x)dx
  -\int\bigl(f_n(y_n^{j(n)}x)-f_n(x)\bigr)g_n(x)dx\cr
&\qquad\qquad\qquad
  +\int\bigl(f_n(y_n^{j(n)}x)-f_n(x_n^{-j(n)}x)\bigr)g_n(x)dx\cr
&=\int\bigl(f_n(y_n^{j(n)}x)-f_n(x)\bigr)g_n(y_n^{j(n)}x)dx
   -\int\bigl(f_n(y_n^{j(n)}x)-f_n(x)\bigr)g_n(x)dx\cr
&\qquad\qquad\qquad
  +\int\bigl(f_n(y_n^{j(n)}x)-f_n(x_n^{-j(n)}x)\bigr)g_n(x)dx\cr
&=\int\bigl(f_n(y_n^{j(n)}x)-f_n(x)\bigr)
                    \bigl(g_n(y_n^{j(n)}x)-g_n(x)\bigr)dx\cr
&\qquad\qquad\qquad
  +\int\bigl(f_n(y_n^{j(n)}x)-f_n(x_n^{-j(n)}x)\bigr)g_n(x)dx.\cr}$$

So if we write

\Centerline{$\beta_{1n}
  =ln\biggerint\bigl(f_n(y_n^{-1}x)-f_n(x)-f_n(y_n^{-1}x_n^{-1}x)
     +f_n(x_n^{-1}x)\bigr)g_n(x)dx$,}

\Centerline{$\beta_{2n}
  =\Bover{ln}{j(n)}\biggerint\bigl(j(n)(f_n(x)-f_n(x_n^{-1}x))
    -f_n(x)+f_n(x_n^{-j(n)}x)\bigr)g_n(x)dx$,}

\Centerline{$\beta_{3n}
  =\Bover{ln}{j(n)}\biggerint\bigl(j(n)(f_n(x)-f_n(y_n^{-1}x))
    -f_n(x)+f_n(y_n^{-j(n)}x)\bigr)g_n(x)dx$,}

\Centerline{$\beta_{4n}
  =\Bover{ln}{j(n)}\biggerint\bigl(f_n(y_n^{j(n)}x)-f_n(x)\bigr)
                        \bigl(g_n(y_n^{j(n)}x)-g_n(x)\bigr)dx$,}

\Centerline{$\beta_{5n}=\Bover{ln}{j(n)}
  \biggerint\bigl(f_n(y_n^{j(n)}x)-f_n(x_n^{-j(n)}x)\bigr)g_n(x)dx$}

\noindent for $n\ge 1$, we have

\Centerline{$|\beta_{1n}|+|\beta_{2n}|+|\beta_{3n}|
+|\beta_{4n}|+|\beta_{5n}|\ge\beta$}

\noindent for every $n\in J$.

\medskip

{\bf (h)} Now $\beta_{1n}\to 0$ as $n\to\infty$.   \Prf\

$$\eqalign{\bigl|\int\bigl(f_n&(y_n^{-1}x)-f_n(x)-f_n(y_n^{-1}x_n^{-1}x)
     +f_n(x_n^{-1}x)\bigr)g_n(x)dx\bigr|\cr
&=\bigl|\int\bigl(f_n(y_n^{-1}x)-f_n(x)\bigr)g_n(x)dx
   -\int\bigl(f_n(y_n^{-1}x_n^{-1}x)-f_n(x_n^{-1}x)\bigr)
                                                  g_n(x)dx\bigr|\cr
&=\bigl|\int\bigl(f_n(y_n^{-1}x)-f_n(x)\bigr)g_n(x)dx
   -\int\bigl(f_n(y_n^{-1}x)-f_n(x)\bigr)g_n(x_nx)dx\bigr|\cr
&=\bigl|\int\bigl(f_n(y_n^{-1}x)-f_n(x)\bigr)
                        \bigl(g_n(x)-g_n(x_nx)\bigr)dx\bigr|\cr
&\le\Bover{\alpha}n\int|g_n(x)-g_n(x_nx)|dx\cr}$$

\noindent by (b-iii).   So

\Centerline{$|\beta_{1n}|
\le\alpha l\biggerint|g_n(x)-g_n(x_nx)|dx\to 0$}

\noindent as $n\to\infty$, by (f), since surely
$\sequencen{x_n}\to e$.\ \Qed

\medskip

{\bf (i)} Now look at $\beta_{2n}$.   If we set

\Centerline{$\gamma'_n
=\sup\{\biggerint|g_n(ax)-g(x)|dx:a\in D_n(U)^{j(n)}\}$,}

\noindent then $\gamma'_n\to 0$ as $n\to\infty$.   \Prf\ Given
$\epsilon>0$, there is a neighbourhood $W$ of $e$ such that
$\int|g_n(ax)-g_n(x)|dx\le\epsilon$ whenever $n\ge 1$ and $a\in W$,
as noted at the end of (f).   Let $p$ be such that $D_p(U_0)\subseteq
W$.   Then, for all $n$ large enough, $pj(n)\le n$, so that
$D_n(U)^{pj(n)}\subseteq U_0$ and $D_n(U)^{j(n)}\subseteq
D_p(U_0)\subseteq W$ (446D(b-v)) and
$\int|g_n(ax)-g(x)|dx\le\epsilon$
for every $a\in D_n(U)^{j(n)}$.\ \Qed

We have

$$\eqalignno{\bigl|\int\bigl(&j(n)(f_n(x)-f_n(x_n^{-1}x))-f_n(x)
  +f_n(x_n^{-j(n)}x)\bigr)g_n(x)dx\bigr|\cr
&=\bigl|\sum_{i=0}^{j(n)-1}\int\bigl(f_n(x)-f_n(x_n^{-1}x)
  -f_n(x_n^{-i}x)+f_n(x_n^{-i-1}x)\bigr)g_n(x)dx\bigr|\cr
&=\bigl|\sum_{i=0}^{j(n)-1}\int\bigl(f_n(x)-f_n(x_n^{-1}x)\bigr)g_n(x)dx
  -\int\bigl(f_n(x_n^{-i}x)-f_n(x_n^{-i-1}x)\bigr)g_n(x)dx\bigr|\cr
&=\bigl|\sum_{i=0}^{j(n)-1}\int\bigl(f_n(x)-f_n(x_n^{-1}x)\bigr)g_n(x)dx
  -\int\bigl(f_n(x)-f_n(x_n^{-1}x)\bigr)g_n(x_n^ix)dx\bigr|\cr
&=\bigl|\sum_{i=0}^{j(n)-1}\int\bigl(f_n(x)-f_n(x_n^{-1}x)\bigr)
                                 \bigl(g_n(x)-g_n(x_n^ix)\bigr)dx|\cr
&\le\sum_{i=0}^{j(n)-1}\int|f_n(x)-f_n(x_n^{-1}x)|
  |g_n(x)-g_n(x_n^ix)|dx
\le j(n)\Bover{\alpha}n\gamma'_n.\cr}$$

\noindent So

\Centerline{$|\beta_{2n}|\le l\alpha\gamma'_n\to 0$}

\noindent as $n\to\infty$.   Similarly, $\langle\beta_{3n}\rangle_{n\ge
1}\to 0$.

\woddheader{446K}{4}{2}{2}{30pt}

{\bf (j)} As for $\beta_{4n}$, we have

\Centerline{$\biggerint|f_n(y_n^{j(n)}x)-f_n(x)|
  |g_n(y_n^{j(n)}x)-g_n(x)|dx
\le\Bover{\alpha j(n)}n\gamma'_n$,}

\noindent putting (b-iii) and the definition of
$\gamma'_n$ together.   So

\Centerline{$|\beta_{4n}|\le l\alpha\gamma'_n\to 0$}

\noindent as $n\to\infty$.

\medskip

{\bf (k)} We come at last to $\beta_{5n}$.   Here, for every $n\ge 1$,

$$\eqalign{\bigl|\int\bigl(f_n(y_n^{j(n)}x)-f_n(x_n^{-j(n)}x)\bigr)
                                             g_n(x)dx\bigr|
&\le\int\bigl|f_n(y_n^{j(n)}x)-f_n(x_n^{-j(n)}x)\bigr|g_n(x)dx\cr
&\le\gamma_n\int g_n(x)dx
=\gamma_n\cr}$$

\noindent by the definition of $\gamma_n$ in (c) above.   So

\Centerline{$|\beta_{5n}|\le l\Bover{n}{j(n)}\gamma_n\to 0$}

\noindent by the choice of the $j(n)$.

\medskip

{\bf (l)} Thus $\beta_{in}\to 0$ as $n\to\infty$ for every $i$.   But
this is impossible, because $0<\beta\le\sum_{i=1}^5|\beta_{in}|$ for
every $n\in J$.\ \Bang

This contradiction shows that we must be able to find a neighbourhood
$W$ of $e$ and an $n_0\in\Bbb N$ such that $xy\in D_n(V)$ whenever $n\ge
n_0$, $x$, $y\in D_n(U)$ and $x^ny^n\in W$;  as $V$ is arbitrary, $U$
has the property required.
}%end of proof of 446K

\leader{*446L}{Definition} Let $X$ be a topological group.   A
{\bf $B$-sequence} in $X$ is a non-increasing sequence $\sequencen{V_n}$
of closed neighbourhoods of the identity, constituting a base of
neighbourhoods of the identity, such that there is some $M$ such that
for every $n\in\Bbb N$ the set $V_nV_n^{-1}$ can be covered by at most
$M$ left translates of $V_n$.

\leader{*446M}{Proposition} Let $X$ be a locally compact Hausdorff
topological group with no small subgroups.   Then it has a $B$-sequence.

\proof{{\bf (a)} For $A\subseteq X$ set $D_n(A)=\{x:x^i\in A$ for every
$i\le n\}$.   We know from 446K that there is a compact symmetric
neighbourhood $U$ of the identity $e$ such that whenever $V$ is a
neighbourhood of $e$ there are an $n_0\in\Bbb N$ and a neighbourhood $W$
of $e$ such that whenever $n\ge n_0$, $x\in D_n(U)$, $y\in D_n(U)$ and
$x^ny^n\in W$, then $xy\in D_n(V)$.   Shrinking $U$ if necessary, we may
suppose that $U$ includes no subgroup of $X$ other than $\{e\}$, so that
there is an $r\ge 1$ such that $D_{rn}(U)^n\subseteq U$ for every
$n\in\Bbb N$ (446H).

Let $V$ be a closed symmetric neighbourhood of $e$ such that
$V^{2r}\subseteq U$.   Then $D_n(V)^2\subseteq D_n(U)$ for every
$n\in\Bbb N$.   \Prf\ $D_n(V)\subseteq D_{2rn}(U)$, by 446D(b-v), so

\Centerline{$(D_n(V)^2)^n\subseteq D_{2rn}(U)^{2n}\subseteq U$,}

\noindent and $D_n(V)^2\subseteq D_n(U)$ (446D(b-v) again).\ \QeD\
Take $n_0\in\Bbb N$ and a neighbourhood $W$ of $e$ such that
whenever $n\ge n_0$, $x$, $y\in D_n(U)$ and $x^ny^n\in W^{-1}W$,
then $xy\in D_n(V)$.

\medskip

{\bf (b)} Let $M$ be so large that
$U$ can be covered by $M$ left translates of $W$.   Then for any
$n\ge n_0$, $D_n(V)D_n(V)^{-1}=D_n(V)^2$ can be covered by $M$ left
translates of $D_n(V)$.

\Prf\ Let $z_0,\ldots,z_{M-1}$ be such that
$U\subseteq\bigcup_{i\le M}z_iW$.   For each $i<M$, set
$A_i=\{x:x\in D_n(U)$, $x^n\in z_iW\}$;  if $A_i\ne\emptyset$ choose
$x_i\in A_i$;  otherwise, set $x_i=e$.

For any $y\in D_n(V)^2$, $y\in D_n(U)$, so $y^n\in U$ and there is some
$i<M$ such that $y\in A_i$.   In this
case $x_i$ also belongs to $A_i$.   Now $z_i^{-1}y^n$ and
$z_i^{-1}x_i^n$ both belong to $W$, so $x_i^{-n}y^n$ belongs to
$W^{-1}W$, and $x_i^{-1}y\in D_n(V)$, by the choice of $W$ and $n_0$.
But this means that $y\in x_iD_n(V)$.   As $y$ is arbitrary,
$D_n(V)^2\subseteq\bigcup_{i<M}x_iD_n(V)$ is covered by $M$ left
translates of $D_n(V)$.\ \Qed

\medskip

{\bf (c)} But this means that $\sequencen{D_{n+n_0}(V)}$ is a
$B$-sequence in $X$.   (It constitutes a base of neighbourhoods of $e$
by 446Gb, as usual.)
}%end of proof of 446M

\leader{*446N}{Proposition} Let $X$ be a locally compact Hausdorff
topological group with a faithful finite-dimensional representation.
Then it has a $B$-sequence.

\proof{{\bf (a)} Let $\phi:X\to GL(r,\Bbb R)$ be a faithful
finite-dimensional representation.   Identifying $M_r$ with the Banach
algebra $\eurm B=\eurm B(\BbbR^r;\BbbR^r)$, where
$\BbbR^r$ is given the Euclidean norm, we see that $GL(r,\Bbb R)$
is an open subset of $\eurm B$
(4A6H).   Note also that the operator norm $\|\,\|$
of $\eurm B$ is equivalent
%4{}A4Ia
to its `Euclidean' norm corresponding to an
identification with $\BbbR^{r^2}$, that is, writing
$\|T\|_{HS}=\sqrt{\sum_{i=1}^r\sum_{j=1}^r\tau_{ij}^2}$ if
$T=\langle\tau_{ij}\rangle_{1\le i,j\le r}$, $\|\,\,\|_{HS}$ is
equivalent to $\|\,\,\|$.    (See the inequalities in 262H.)   In
particular, all the balls $B(T,\delta)=\{S:\|S-T\|\le\delta\}$ are
closed for the Euclidean norm (4A2Lj).
If we write $\mu_L$ for Lebesgue measure on
$\eurm B$, identified with $\BbbR^{r^2}$, and set
$\gamma=\mu_LB(\tbf{0},1)$, then $0<\gamma<\infty$ (because
$B(\tbf{0},1)$ includes, and is included in, non-trivial Euclidean
balls) and $\mu_LB(T,\delta)=\delta^{r^2}\gamma$ for every
$T\in\eurm B$ and $\delta\ge 0$ (using 263A, or otherwise).

\medskip

{\bf (b)} We need to recall a basic inequality concerning
inversion in Banach algebras.   If $T\in\eurm B$ and
$\|T-I\|\le\bover12$, then $T$ is invertible and

\Centerline{$\|T^{-1}-I\|\le\Bover{\|T-I\|}{1-\|T-I\|}\le 1$}

\noindent (4A6H), so $\|T^{-1}\|\le 2$.

\medskip

{\bf (c)} Now let $V$ be a compact neighbourhood of the identity
$e$ of $X$.   Let $V_1$ be a neighbourhood of $e$ such that
$(V_1V_1^{-1})^{-1}V_1V_1^{-1}\subseteq V$.   For $\delta>0$, set
$U_{\delta}=\{x:x\in V$, $\|\phi(x)-I\|\le\delta\}$.   Then each
$U_{\delta}$ is a compact neighbourhood of $e$, because $\phi$ is
continuous.   Also

\Centerline{$\bigcap_{\delta>0}U_{\delta}
=\{x:x\in V,\,\phi(x)=I\}=\{e\}$.}

\noindent So
$\{U_{\delta}:\delta>0\}$ is a base of neighbourhoods of $e$ (4A2Gd),
and there is a $\delta_1>0$ such that $U_{\delta_1}\subseteq V_1$;  of
course we may suppose that $\delta_1\le\bover18$.

\medskip

{\bf (d)} If $\delta\le\bover12$ and
$x\in U_{\delta}U_{\delta}^{-1}$, then $\|\phi(x)-I\|\le 4\delta$.
\Prf\ Express $x$ as $yz^{-1}$ where $y$,
$z\in U_{\delta}$.   Then $\|\phi(z)-I\|\le\bover12$, so
$\|\phi(z^{-1})\|\le 2$ and

\Centerline{$\|\phi(x)-I\|
=\|(\phi(y)-\phi(z))\phi(z^{-1})\|
\le 2\|\phi(y)-\phi(z)\|\le 4\delta$.  \Qed}

\medskip

{\bf (e)} Now if $\delta\le\delta_1$,
$U_{\delta}U_{\delta}^{-1}$ can be covered by at most $m=17^{r^2}$ left
translates of $U_{\delta}$.   \Prf\Quer\ Suppose, if possible,
otherwise.   Then we can choose
$x_0,\ldots,x_m\in U_{\delta}U_{\delta}^{-1}$ such that
$x_j\notin x_iU_{\delta}$ whenever
$i<j\le m$.   If $i<j\le m$, then

\Centerline{$x_i^{-1}x_j\in
  (U_{\delta}U_{\delta}^{-1})^{-1}U_{\delta}U_{\delta}^{-1}
\subseteq (V_1V_1^{-1})^{-1}V_1V_1^{-1}
\subseteq V$,}

\noindent and $x_i^{-1}x_j\notin U_{\delta}$, so
$\|\phi(x_i^{-1}x_j)-I\|>\delta$.
Set $T_i=\phi(x_i)$ for each $i\le m$;  then

\Centerline{$\|T_i-I\|\le 4\delta\le\bover12$}

\noindent for each $i$, by (d), while

\Centerline{$\delta<\|\phi(x_i^{-1}x_j)-I\|=\|T_i^{-1}T_j-I\|
\le\|T_i^{-1}\|\|T_j-T_i\|\le 2\|T_j-T_i\|$}

\noindent whenever $i<j\le m$.   Write
$B_i=B(T_i,\bover{\delta}4)$ for each $i$;  then all the $B_i$ are
disjoint.   But also they are all included in $B(I,\bover{17\delta}4)$,
so we have

\Centerline{$(17^{r^2}+1)\bigl(\Bover{\delta}4\bigr)^{r^2}\gamma
\le\bigl(\Bover{17\delta}4\bigr)^{r^2}\gamma$,}

\noindent which is impossible.\ \Bang\Qed

\medskip

{\bf (f)} Accordingly, setting $W_n=U_{2^{-n}\delta_1}$,
$\sequencen{W_n}$ is a $B$-sequence in $X$.
}%end of proof of 446N


\leader{*446O}{Theorem} Let $X$ be a locally compact Hausdorff
topological group.   Then it has an open subgroup $Y$ which has a
compact normal subgroup $Z$ such that $Y/Z$ has no small subgroups.

\proof{{\bf (a)} Let $U$ be a compact neighbourhood of the identity $e$
of $X$.   Then there are a subgroup $Y_0$ of $X$, included in $U$, and a
neighbourhood $W_0$ of $e$ such that every subgroup of $X$ included in
$W_0$ is also included in $Y_0$.

\medskip\Prf {\bf (i)} To begin with (down to the end of (iii)) let us
suppose that $X$ is metrizable.   Let $\sequencen{V_n}$ be a
non-increasing sequence of closed
symmetric neighbourhoods of $e$ running over a base of
neighbourhoods of $e$, and such that $V_1^2\subseteq V_0\subseteq U$.
For each $n\in\Bbb N$, set
$A_n=\{x:x^i\in V_n$ for every $i\in\Bbb N\}$.

\medskip

\quad{\bf (ii)}\Quer\ Suppose, if possible, that
$\bigcup_{k\in\Bbb N}A_n^k\not\subseteq U$ for every $n\in\Bbb N$.   For
each $n\in\Bbb N$ let $k(n)\in\Bbb N$ be such that
$A_n^{k(n)}\subseteq U$,
$A_n^{k(n)+1}\not\subseteq U$.   Then $\sequencen{A_n}$ and $U$ satisfy
the conditions of 446F (because $A_m\subseteq V_n$ whenever $m\ge n$,
and $\{V_n:n\in\Bbb N\}$ is a base of neighbourhoods of $e$).   Let
$\Cal F$ be a non-principal ultrafilter on $\Bbb N$ and set
$Q=\lim_{n\to\Cal F}A_n^{k(n)}$ in the space $\Cal C$
of closed subsets of $X$ with the Fell topology.

If $W$ is any neighbourhood of $e$, there is an $n\in\Bbb N$ such that
$V_n\subseteq W$, so that $x^i\in W$ whenever $x\in\bigcup_{m\ge n}A_m$ and
$i\in\Bbb N$.   Thus (ii) of 446F is not true, and $Q$ must be a closed
subgroup of $X$ included in $U$ and meeting the boundary of $U$.

By 446C, there are an $r\in\Bbb N$ and a continuous homomorphism
$\phi:Q\to GL(r,\Bbb R)$ such that the kernel $Z$ of $\phi$ is included
in $\interior V_1$.   Let $G\subseteq X$ be an open set including $Z$
and with closure disjoint from
$(X\setminus\interior V_1)\cup\{x:x\in Q,\,\|\phi(x)-I\|\ge\bover16\}$;
such can be found because $Z$ is compact and $X$ is regular (4A2F(h-ii)).
Then $Z\subseteq G$, and any subgroup $Z'$ of $Q$ included in
$\overline{G}$ has
$\|\phi(x)^i-I\|\le\bover16$ for every $x\in Z'$ and $i\in\Bbb N$,
so that $Z'\subseteq Z$, by 4A6N.   Set $V=\overline{G}$.

Since $V\subseteq U$ and $A_n^{k(n)+1}\not\subseteq U$ for every $n$, we
can find $j(n)\le k(n)$ such that $A_n^{j(n)}\subseteq V$ and
$A_n^{j(n)+1}\not\subseteq V$ for every $n$.   Set
$Q'=\lim_{n\to\Cal F}A_n^{j(n)}$.   As before, (ii) of 446F cannot be
true of $Q'$, and $Q'$ must be a closed subgroup of $X$
meeting the boundary of $V$.   Because $e\in A_n$,
$A_n^{j(n)}\subseteq A_n^{k(n)}$ for every $n$, and $Q'\subseteq Q$,
because $\{(E,F):E\subseteq F\}$ is closed in $\Cal C$ (4A2T(e-i));  also
$Q'\subseteq V$, so $Q'\subseteq Z$.   But $Z$ does not meet the
boundary of $V$.\ \Bang

\medskip

\quad{\bf (iii)} So there is some $n\in\Bbb N$ such that
$A_n^k\subseteq U$ for every $k\in\Bbb N$.   Because $A_n^{-1}=A_n$,
$Y_0=\bigcup_{k\in\Bbb N}A_n^k$ is a subgroup of $X$.
Any subgroup of $X$ included in $V_n$ is a subset of $A_n$ so is
included in $Y_0$.   Thus we have a pair $Y_0$, $W_0=V_n$ of the kind
required, at least when $X$ is metrizable.

\medskip

\quad{\bf (iv)} Now suppose that $X$ is $\sigma$-compact.   Let $U_1$ be
a neighbourhood of $e$ such that $U_1^2\subseteq U$.   Then there is a
closed normal subgroup $X_0$ of $X$ such that $X_0\subseteq U_1$ and
$X'=X/X_0$ is metrizable (4A5S).   By (i)-(iii), there are a subgroup
$Y'_0$ of $X'$, included in the image of $U_1$ in $X'$, and a
neighbourhood $W'_0$ of the identity in $X'$ such that any subgroup of
$X'$ included in $W'_0$ must also be included in $Y'_0$.   Write
$\pi:X\to X'$ for the canonical homomorphism and consider
$Y_0=\pi^{-1}[Y'_0]$, $W_0=\pi^{-1}[W'_0]$.   Then $W_0$ is a
neighbourhood of $e$ and $Y_0$ is a subgroup of $X$ included in

\Centerline{$\pi^{-1}[\pi[U_1]]=U_1X_0\subseteq U_1^2\subseteq U$.}

\noindent And if $Z$ is any subgroup of $X$ included in $W_0$, then
$\pi[Z]\subseteq W'_0$ so $\pi[Z]\subseteq Y'_0$ and $Z\subseteq Y_0$.
Thus in this case also we have the result.

\medskip

\quad{\bf (v)} Finally, for the general case, observe that $X$ has a
$\sigma$-compact open subgroup $X_1$ (4A5El).   So we can find a
subgroup
$Y_0$ of $X_1$, included in $U\cap X_1$, and a neighbourhood $W_0$ of
the identity in $X_1$ such that any subgroup of $X_1$ included in $W_0$
is also included in $Y_0$.   But of course $Y_0$ and $W_0$ also serve
for $X$ and $U$.

This completes the proof of (a).\ \Qed

\medskip

{\bf (b)} Of course $\overline{Y_0}$ is a subgroup of $X$ (4A5Em);
being included in $U$, it is compact.
By 446C, there is a finite-dimensional representation
$\phi:\overline{Y_0}\to GL(r,\Bbb R)$, for some $r\in\Bbb N$, such that the kernel
$Z$ of $\phi$ is included in $\interior W_0$.   Let $W_1$ be a
neighbourhood of $e$ in $X$ such that $\|\phi(x)-I\|\le\bover16$ for
every $x\in W_1\cap \overline{Y_0}$, and set $W=W_1Z\cap W_0$.   Note that if
$x\in W\cap\overline{Y_0}$, there is a $z\in Z$ such that $xz\in W_1\cap \overline{Y_0}$, so that
$\|\phi(x)-I\|=\|\phi(xz)-I\|\le\bover16$.   Of course
$Z\subseteq\interior W_1Z$, so $Z\subseteq\interior W$.

If $Y'$ is a subgroup of $\overline{Y_0}$ included in $W$, then
$\|\phi(x)^i-I\|\le\bover16$ for every $i\in\Bbb N$ and $x\in Y'$, so
$Y'\subseteq Z$.   Consequently any subgroup of $X$ included in $W$ is a
subgroup of $Z$, since by the choice of $\overline{Y_0}$ and $W_0$ it is a subgroup
of $\overline{Y_0}$.

Now let $Y$ be the normalizer of $Z$ in $X$.   $Z$ is compact, so
$G=\{x:xZx^{-1}\subseteq\interior W\}$ is open (4A5Ei), and contains $e$;  but
also $G\subseteq Y$, because if $x\in G$ then $xZx^{-1}$ is a subgroup
of $X$ included in $W$, and must be included in $Z$.
Accordingly $Y=GY$ is open.

Since any subgroup of $Y$ included in $W$ is a subgroup of $Z$, we see
that any subgroup of $Y/Z$ included in the image of $W$ is the trivial
subgroup, and $Y/Z$ has no small subgroups, as required.
}%end of proof of 446O

\vleader{72pt}{*446P}{Corollary} Let $X$ be a locally compact Hausdorff
topological group.   Then it has a chain
$\langle X_{\xi}\rangle_{\xi\le\kappa}$ of closed subgroups,
where $\kappa$ is an infinite cardinal, such that

\quad(i) $X_0$ is open,

\quad (ii) $X_{\xi+1}$ is a normal subgroup of $X_{\xi}$ for every
$\xi<\kappa$,

\quad (iii) $X_{\xi}$ is compact for $\xi\ge 1$,

\quad (iv) $X_{\xi}=\bigcap_{\eta<\xi}X_{\eta}$ for non-zero limit
ordinals $\xi\le\kappa$,

\quad (v) $X_{\xi}/X_{\xi+1}$ has a $B$-sequence for every $\xi<\kappa$,

\quad (vi) $X_{\kappa}=\{e\}$, where $e$ is the identity of $X$.

\proof{{\bf (a)} By 446O, $X$ has an open subgroup $X_0$ with a compact
normal subgroup $X_1$ such that $X_0/X_1$ has no small subgroups.   By
446M, $X_0/X_1$ has a $B$-sequence.

\medskip

{\bf (b)} Let $\Phi$ be the set of finite-dimensional representations of
$X_1$;   if we distinguish the trivial homomorphisms from $X_1$ to each
$GL(r,\Bbb R)$, $\Phi$ is infinite.   Set $\kappa=\#(\Phi)$ and let
$\langle\phi_{\xi}\rangle_{1\le\xi<\kappa}$ run over $\Phi$.
For $1\le\xi\le\kappa$, set

\Centerline{$X_{\xi}=\{x:x\in X_1,\,\phi_{\eta}(x)=I$ for
$1\le\eta<\xi\}$.}

\noindent   Then $\langle X_{\xi}\rangle_{\xi\le\kappa}$ satisfies
conditions (i)-(iv).
As for (v), I have already checked the case $\xi=0$, and
if $1\le\xi<\kappa$, then $\phi_{\xi}\restr X_{\xi}$ is a
finite-dimensional representation of $X_{\xi}$ with kernel $X_{\xi+1}$,
so $X_{\xi}/X_{\xi+1}$ has a faithful finite-dimensional representation
(446Ab), and therefore has a $B$-sequence (446N).

Finally, $X_{\kappa}=\{e\}$ by 446B;  if $x\in X_1$ and $x\ne e$, there
is a $\phi\in\Phi$ such that $\phi(x)\ne\phi(e)$, so that $x\notin
X_{\kappa}$.
}%end of proof of 446P

\exercises{\leader{446X}{Basic exercises (a)}
%\spheader 446Xa
Let $X$ be a locally compact Hausdorff abelian topological group.
Show that for every element $a$ of $X$, other than the identity, there
is a two-dimensional representation $\phi$ of $X$ such that
$\phi(a)\ne I$.   \Hint{445O.}
%446B

\spheader 446Xb Let $\sequencen{X_n}$ be any sequence of groups with 
their discrete topologies, and $X$ the product topological group 
$\prod_{n\in\Bbb N}X_n$.   Show that $X$ has a $B$-sequence.   
\Hint{set $V_n=\{x:x(i)=e(i)$ for $i<n\}$.}
%446L

\leader{446Y}{Further exercises (a)}
%\spheader 446Ya
Let $X$ be the countable group of all permutations of $\Bbb N$
which are
products of an even number of transpositions.   Give $X$ its discrete
topology, so that it is a locally compact topological group.   Show that
any finite-dimensional representation of $X$ is trivial.
\Hint{$X$ is simple and has many commuting involutions.}
%446Xa 446B a nontrivial representation must be faithful.
% If  T\in\GL(\Bbb R,r)  is an involution, then  
% \BbbR^r=N_{T-I}\oplus N_{T+I}  and  T  has lots of eigenvectors.
% If  T_0,\ldots,T_n  are commuting involutions then 
% \BbbR^r  is the linear span of  \bigcap_{i\le n}\{x:T_ix=\pm x\} .
% So there cannot be a set of more than  2^r  commuting involutions.

\spheader 446Yb Let $X$ be a compact Hausdorff topological group,
$f\in C(X)$ and $\epsilon>0$.
Show that there are a finite-dimensional representation
$\phi:X\to GL(r,\Bbb R)$ and $a$, $b\in\BbbR^r$ such that
$|f(x)-\innerprod{\phi(x)(a)}{b}|\le\epsilon$ for every $x\in X$.
%446B  Stone-Weierstrass:  need to know about tensor product of
% Hilbert spaces,
%\innerprod{u\otimes u'}{v\otimes v'}
%=\innerprod{u}{v}\cdot\innerprod{u'}{v'}

\spheader 446Yc Let $\kappa$ be an infinite cardinal,
and $(\frak B_{\kappa},\bar\nu_{\kappa})$ the measure
algebra of the usual measure on $\{0,1\}^{\kappa}$.   Give the group
$\Aut_{\bar\nu_{\kappa}}(\frak B_{\kappa})$ of measure-preserving
automorphisms of $\frak B_{\kappa}$ its topology of pointwise convergence.
Let $X$ be a compact Hausdorff topological group of weight
at most $\kappa$.   Show that there is a continuous injective homomorphism
from $X$ to $\Aut_{\bar\nu_{\kappa}}(\frak B_{\kappa})$.
%446B we have lots of homomorphisms into orthogonal groups which
% have Haar probability measures

%\spheader 446Yb Improve the constant $17^{r^2}$ in part (e) of the
%proof of 446N.  \Hint{\S263.}
%%446N
}%end of exercises

\endnotes{
\Notesheader{446} The ideas above are extracted from the structure
theory for locally compact groups, as described in
{\smc Montgomery \& Zippin 55}.
(A brisker and sometimes neater, but less complete,
exposition can be found in {\smc Kaplansky 71}.)   The full theory goes
very much deeper into the analysis of groups with no small subgroups.
One of the most important ideas, hidden away in 446I and part (c) of the
proof of 446K, is that of `one-parameter subgroup';  if $X$ is a group
with no small subgroups, there are enough continuous homomorphisms from
$\Bbb R$ to $X$ not only to provide a great deal of information on the
topological group structure of $X$, but even to set up a differential
structure ({\smc Kaplansky 71}, \S II.3).
For our purposes here, however, all we need to know is that groups with
no small subgroups have `$B$-sequences' (446L-446M), which can form the
basis of a theory corresponding to Vitali's theorem and Lebesgue's
Density Theorem in $\BbbR^r$ (447C-447D below).

There are four essential elements in the argument here.   Working from
the outside, the first step is 446O:  starting from a locally compact
Hausdorff group $X$, we can find an open subgroup $X_0$ of $X$ and a
compact normal subgroup $X_1$ of $X_0$ such that $X_0/X_1$ has no small
subgroups.   This depends on a subtle argument based on the first key
lemma, the dichotomy in 446F, which in turn uses the `smoothing'
construction in 446E and a careful analysis of inequalities involving
integrals.   (Naturally enough, the translation-invariance of the Haar
integral is a leitmotiv of this investigation.)   Note the remarkable
transition in 446H.   The sets $D_n(U)$ are defined solely in terms of
powers, while the sets $D_n(U)^n$ involve products.   We are able to
obtain information about products $x_1\ldots x_n$ from information about
the powers $x_j^i$ for $i$, $j\le n$.

Next, we need to find a chain of closed subnormal subgroups of $X_1$,
decreasing to $\{e\}$, such that the quotients all have faithful
finite-dimensional representations (in this context, this means that
they are isomorphic to compact subgroups of $GL(r,\Bbb R)$).   This step
depends on the older ideas in 446B-446C, where we use the theory of
compact operators on Hilbert spaces to show that a compact group has
many representations as actions on finite-dimensional subspaces of its
$L^2$ space.   (Observe that in this section I revert to real-valued,
rather than complex-valued, functions.)   This can be thought of as a
development of the result of 445O.   If $X$
is a locally compact abelian group, its characters separate its points
(cf.\ 446Xa);  if $X$ is compact but not necessarily abelian, its
finite-dimensional representations separate its points.   (But if $X$ is
neither compact nor abelian, there are further difficulties;  see
446Ya.)

The other two necessary facts are that both groups with no small
subgroups, and groups with faithful finite-dimensional representations,
have $B$-sequences.   The latter is reasonably straightforward (446N);
any complications are due entirely to the fact that the natural measure
on $GL(r,\Bbb R)$, inherited from $\BbbR^{r^2}$, is not quite invariant
under multiplication, so we have to manipulate some inequalities.   For
groups with no small subgroups (446M) we have much more to do.   The
proof I give here depends on a second key lemma, 446K, refining the
methods of 446F;  a slightly stronger version of this result is the
basis of the analysis of one-parameter subgroups in the general theory
(compare {\smc Montgomery \& Zippin 55}, \S3.8).

}%end of notes

\discrpage

