\frfilename{mt41.tex} 
\versiondate{17.4.10} 
\copyrightdate{1997} 
      
\def\chaptername{Topologies and Measures I} 
\newchapter{41} 
\def\chaptername{Topologies and measures I} 
      
I begin this volume with an introduction to some of the most important 
ways in which topologies and measures can interact, and with a 
description of 
the forms which such constructions as subspaces and product spaces take 
in such contexts.   By far the most important concept is that of Radon 
measure (411Hb, \S416).   In Radon measure spaces we find both the 
richest combinations of ideas and the most important applications. 
But, as usual, we are led both by analysis of these ideas and by other 
interesting examples to consider wider classes of topological measure 
space, and the 
greater part of the chapter, by volume, is taken up by a description of 
the many properties of Radon measures individually and in partial 
combinations. 
      
I begin the chapter with a short section of definitions (\S411), 
including a handful of more or less elementary examples.   The two 
central properties 
of a Radon measure are `inner regularity' (411B) and `$\tau$-additivity' 
(411C).   The former is an idea of great versatility which I look at in 
an abstract setting in \S412.   I take a section (\S413) to describe 
some methods of constructing measure spaces, extending the rather 
limited range of constructions offered in earlier volumes.   There are 
two sections on $\tau$-additive measures, \S\S414 and 417;  the former 
covers the elementary ideas, and the latter looks at product measures, 
where it turns out that we need a new technique to supplement the purely 
measure-theoretic constructions of Chapter 25.   On the way to Radon 
measures in \S416, I 
pause over `quasi-Radon' measures (411Ha, \S415), where inner regularity 
and $\tau$-additivity first come effectively together. 
      
The possible interactions of a topology and a measure on the same space 
are so varied that even a brief account makes a long chapter;  and this 
is with hardly any mention of results associated with particular types 
of topological space, most of which must wait for later chapters.   But 
I include one section on the two most important classes of functions 
acting between topological measure spaces (\S418), and another 
describing some examples to demonstrate special phenomena (\S419). 

\discrpage 
