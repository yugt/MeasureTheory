\frfilename{mt436.tex} 
\versiondate{9.5.11} 
\copyrightdate{2002} 
      
\def\chaptername{Topologies and measures II} 
\def\sectionname{Representation of linear functionals} 
      
\newsection{436} 
      
I began this treatise with the three steps which make measure theory, as 
we know it, possible:  a construction of Lebesgue measure, a 
definition of an integral from a measure, and a proof of the 
convergence theorems.   I used what I am sure is the best route: 
Lebesgue measure from Lebesgue 
outer measure, and integrable functions from simple functions.   But of 
course there are many other paths to the same ends, and some of them 
show us slightly different aspects of the subject.   In this section I 
come -- rather later than many authors would -- to an account of a 
procedure for constructing measures from integrals. 
      
I start with three fundamental theorems, the first and third being the 
most important.   A positive linear functional on a truncated Riesz 
space of functions is an integral iff it is sequentially smooth (436D); 
a smooth linear functional corresponds to a quasi-Radon measure (436H); 
and if $X$ is a compact Hausdorff space, any positive linear functional 
on $C(X)$ corresponds to a Radon measure (436J-436K). 
      
\leader{436A}{Definition} Let $X$ be a set, $U$ a Riesz subspace of 
$\Bbb R^X$, and $f:U\to\Bbb R$ a positive linear functional.   I say 
that $f$ is {\bf sequentially smooth} if whenever $\sequencen{u_n}$ is a 
non-increasing sequence in $U$ such that $\lim_{n\to\infty}u_n(x)=0$ for 
every $x\in X$, then $\lim_{n\to\infty}f(u_n)=0$. 
      
If $(X,\Sigma,\mu)$ is a measure space and $U$ is a Riesz subspace of 
the space of real-valued $\mu$-integrable functions defined everywhere 
on $X$, then $\int\,d\mu:U\to\Bbb R$ is sequentially smooth\cmmnt{, by 
Fatou's Lemma or Lebesgue's Dominated Convergence Theorem}. 
      
\cmmnt{\medskip 
      
\noindent{\bf Remark} It is essential to distinguish between 
`sequentially smooth', as defined here, and `sequentially 
order-continuous', as in 313Hb or 355G.   In the context here, a 
positive linear operator $f:U\to\Bbb R$ is sequentially order-continuous 
if $\lim_{n\to\infty}f(u_n)=0$ whenever $\sequencen{u_n}$ is a 
non-increasing sequence in $U$ such that $0$ is the greatest lower bound 
for $\{u_n:n\in\Bbb N\}$ in $U$;  while $f$ is sequentially smooth if 
$\lim_{n\to\infty}f(u_n)=0$ whenever $\sequencen{u_n}$ is a 
non-increasing sequence in $U$ such that $0$ is the greatest lower bound 
for $\{u_n:n\in\Bbb N\}$ in $\Bbb R^X$.   So there can be sequentially 
smooth functionals which are not sequentially order-continuous, as in 
436Xi.   A sequentially order-continuous positive linear functional is 
of course sequentially smooth. 
}%end of comment 
      
\leader{436B}{Definition} Let $X$ be a set.   I will say that a Riesz 
subspace $U$ of $\Bbb R^X$ is {\bf truncated} (or satisfies {\bf Stone's 
condition}) if $u\wedge\chi X\in U$ for every $u\in U$. 
      
In this case, $u\wedge\gamma\chi X\in U$ for every $\gamma\ge 0$ and 
$u\in U$\cmmnt{ (being $-u^-$ if $\gamma=0$, 
$\gamma(\gamma^{-1}u\wedge\chi X)$ otherwise)}. 
      
\leader{436C}{Lemma} Let $X$ be a set and $U$ a truncated Riesz subspace 
of $\Bbb R^X$.   Write $\Cal K$ for the family of sets of the form 
$\{x:x\in X,\,u(x)\ge 1\}$ as $u$ runs over $U$.   Let $f:U\to\Bbb R$ be 
a sequentially smooth positive linear functional, and $\mu$ a measure on 
$X$ such 
that $\mu K$ is defined and equal to $\inf\{f(u):\chi K\le u\in U\}$ for 
every $K\in\Cal K$.   Then $\int u\,d\mu$ exists and is equal to $f(u)$ 
for every $u\in U$. 
      
\proof{ It is enough to deal with the case $u\ge 0$, since $U=U^+-U^+$ 
and both $f$ and $\int$ are linear.   Note that if $v\in U$, 
$K\in\Cal K$ and $v\le\chi K$, then $v\le w$ whenever 
$\chi K\le w\in U$, so $f(v)\le\mu K$.   For $k$, $n\in\Bbb N$ set 
      
\Centerline{$K_{nk}=\{x:u(x)\ge  2^{-n}k\}$, 
\quad$u_{nk}=u\wedge 2^{-n}k\chi X$.} 
      
\noindent Then, for $k\ge 1$, 
      
\Centerline{$K_{nk}=\{x:\Bover{2^n}{k}u\ge 1\}\in\Cal K$,} 
      
\Centerline{$2^n(u_{n,k+1}-u_{nk})\le\chi K_{nk} 
\le 2^n(u_{nk}-u_{n,k-1})$.} 
      
\noindent So 
      
\Centerline{$2^nf(u_{n,k+1}-u_{nk})\le\mu K_{nk} 
\le 2^nf(u_{nk}-u_{n,k-1})$,} 
      
\noindent and 
      
\Centerline{$f(u_{n,4^n+1}-u_{n1})\le\sum_{k=1}^{4^n}2^{-n}\mu K_{nk} 
\le f(u_{n,4^n}\le f(u)$.} 
      
\noindent But setting $w_n=u_{n,4^n+1}-u_{n1}$, $\sequencen{w_n}$ is a 
non-decreasing sequence of functions in $U$ and $\sup_{n\in\Bbb 
N}w_n(x)=u(x)$ for every $x$, so 
$\lim_{n\to\infty}f(u-w_n)=0$ and $\lim_{n\to\infty}f(w_n)=f(u)$. 
Also, setting $v_n=\sum_{k=1}^{4^n}2^{-n}\chi K_{nk}$, we have 
$w_n\le v_n\le u$ and $f(w_n)\le\int v_n\le f(u)$ for each $n$, so 
      
\Centerline{$\biggerint u=\lim_{n\to\infty}\int v_n=f(u)$} 
      
\noindent by B.Levi's theorem. 
}%end of proof of 436C 
      
\leader{436D}{Theorem} Let $X$ be a set and $U$ a truncated Riesz 
subspace of $\Bbb R^X$.   Let $f:U\to\Bbb R$ be a positive linear 
functional.   Then the following are equiveridical: 
      
(i) $f$ is sequentially smooth; 
      
(ii) there is a measure $\mu$ on $X$ such that $\int u\,d\mu$ is defined 
and equal to $f(u)$ for every $u\in U$. 
      
\proof{ I remarked in 436A that (ii)$\Rightarrow$(i) is a 
consequence of Fatou's Lemma.   So the argument here is devoted to 
proving that (i)$\Rightarrow$(ii). 
      
\medskip 
      
{\bf (a)} Let $\Cal K$ be the family of sets $K\subseteq X$ such that 
$\chi K=\inf_{n\in\Bbb N}u_n$ for some sequence $\sequencen{u_n}$ in 
$U$, taking the infimum in $\Bbb R^X$, so that 
$(\inf_{n\in\Bbb N}u_n)(x)=\inf_{n\in\Bbb N}u_n(x)$ for every $x\in X$. 
Then $\Cal K$ 
is closed under finite unions and countable intersections.   \Prf\ (i) 
If $K$, $K'\in\Cal K$ take sequences $\sequencen{u_n}$, 
$\sequencen{u'_n}$ in $U$ such that $\chi K=\inf_{n\in\Bbb N}u_n$ and 
$\chi K'=\inf_{n\in\Bbb N}u'_n$; then 
$\chi(K\cup K')=\inf_{m,n\in\Bbb N}u_m\vee u'_n$, so 
$K\cup K'\in\Cal K$.   (ii) If $\sequencen{K_n}$ is a sequence in 
$\Cal K$, then for each $n\in\Bbb N$ we can choose a sequence 
$\sequence{i}{u_{ni}}$ in $U$ such 
that $\chi K_n=\inf_{i\in\Bbb N}u_{ni}$;  now 
$\chi(\bigcap_{n\in\Bbb N}K_n)=\inf_{n,i\in\Bbb N}u_{ni}$, so 
$\bigcap_{n\in\Bbb N}K_n\in\Cal K$.\ \Qed 
      
Note that $\emptyset\in\Cal K$ because $0\in U$. 
      
\medskip 
{\bf (b)} We need to know that if $u\in U$ then $K=\{x:u(x)\ge 1\}$ 
belongs to $\Cal K$.   \Prf\ Set 
      
\Centerline{$u_n=2^n((u\wedge\chi X)-(u\wedge(1-2^{-n})\chi X))$.} 
      
\noindent Because $U$ is truncated, every $u_n$ belongs to $U$, and it 
is easy to 
check that $\inf_{n\in\Bbb N}u_n=\chi K$.\ \QeD\  It follows that 
      
\Centerline{$\{x:u(x)\ge\alpha\}=\{x:\Bover1\alpha u(x)\ge 1\} 
\in\Cal K$} 
      
\noindent whenever $u\in U$ and $\alpha>0$. 
      
\medskip 
      
{\bf (c)} For $K\in\Cal K$, set 
$\phi_0K=\inf\{f(u):u\in U,\,u\ge\chi K\}$.   Then $\phi_0$ satisfies 
the conditions of 413I.   \Prf\  I have 
already checked ($\dagger$) and ($\ddagger$) of 413I. 
      
\medskip 
      
\quad\grheada\ Fix $K$, $L\in\Cal K$ with $L\subseteq K$.   Set 
$\gamma=\sup\{\phi_0K':K'\in\Cal K,\,K'\subseteq K\setminus L\}$. 
      
\medskip 
      
\qquad{\bf (i)} Suppose that $K'\in\Cal K$ is included in $K\setminus L$,  
and $\epsilon>0$.   Let $\sequencen{u_n}$, $\sequencen{u'_n}$ be 
sequences in $U$ such that $\chi L=\inf_{n\in\Bbb N}u_n$ and  
$\chi K'=\inf_{n\in\Bbb N}u'_n$, and let $u\in U$ be such that $u\ge\chi K$ 
and $f(u)\le\phi_0K+\epsilon$.   Set $v_n=u\wedge\inf_{i\le n}u_i$, 
$v'_n=u\wedge\inf_{i\le n}u'_i$ for each $n$.   Then 
$\sequencen{v_n\wedge v'_n}$ is a non-increasing sequence in $U$ with  
infimum $\chi L\wedge\chi K'=0$, so there is an $n$ such that  
$f(v_n\wedge v'_n)\le\epsilon$.   In this case 
      
$$\eqalign{\phi_0L+\phi_0K' 
&\le f(v_n)+f(v'_n) 
=f(v_n+v'_n)\cr 
&=f(v_n\vee v'_n)+f(v_n\wedge v'_n) 
\le f(u)+\epsilon 
\le\phi_0K+2\epsilon.\cr}$$ 
      
\noindent As $\epsilon$ is arbitrary, $\phi_0L+\phi_0K'\le \phi_0K$. 
As $K'$ is arbitrary, $\phi_0L+\gamma\le\phi_0K$. 
      
\medskip 
      
\qquad{\bf (ii)} Next, given $\epsilon\in\ooint{0,1}$, there are $u$, 
$v\in U$ such that $u\ge\chi K$, $v\ge\chi L$ and 
$f(v)\le\phi_0L+\epsilon$.   Consider 
      
\Centerline{$K'=\{x:x\in K,\,\min(1,u(x))-v(x)\ge\epsilon\} 
\subseteq K\setminus L$.} 
      
\noindent By (b), $K'\in\Cal K$.   If $w\in U$ and 
$w\ge\chi K'$, then $v(x)+w(x)\ge 1-\epsilon$ for every $x\in K$, so 
      
\Centerline{$\phi_0K 
\le\Bover1{1-\epsilon}f(v+w) 
\le\Bover1{1-\epsilon}(\phi_0L+\epsilon+f(w))$.} 
      
\noindent As $w$ is arbitrary, 
      
\Centerline{$(1-\epsilon)\phi_0K\le\phi_0 
L+\epsilon+\phi_0K'\le\phi_0L+\epsilon+\gamma$.} 
      
\noindent As $\epsilon$ is arbitrary, $\phi_0K\le\phi_0L+\gamma$ and we 
have equality, as required by ($\alpha$) in 413I. 
      
\medskip 
      
\quad\grheadb\ Now suppose that $\sequencen{K_n}$ is a non-increasing 
sequence in $\Cal K$ with empty intersection.   For each $n\in\Bbb N$ 
let $\sequence{i}{u_{ni}}$ be a sequence in $U$ with infimum $\chi K_n$ 
in $\Bbb R^X$.   Set $v_n=\inf_{i,j\le n}u_{ji}$ for each $n$;  then 
$\sequencen{v_n}$ is a non-increasing sequence in $U$ with infimum 
$\inf_{n\in\Bbb N}\chi K_n=0$, so $\inf_{n\in\Bbb N}f(v_n)=0$.   But 
      
\Centerline{$v_n\ge\inf_{j\le n}\chi K_j=\chi K_n$, 
\quad$\phi_0K_n\le f(v_n)$} 
      
\noindent for every $n$, so $\inf_{n\in\Bbb N}\phi_0K_n=0$, as required 
by ($\beta$) of 413I.\ \Qed 
      
\medskip 
      
{\bf (d)} By 413I, there is a complete locally determined measure $\mu$ 
on $X$, inner regular with respect to $\Cal K$, extending $\phi_0$. 
By 436C, $f(u)=\int u\,d\mu$ for every $u\in U$, as required. 
}%end of proof of 436D 
      
\leader{436E}{Proposition} Let $X$ be any topological space, 
and $C_b=C_b(X)$ the space of bounded continuous 
real-valued functions on 
$X$.   Then there is a one-to-one correspondence between totally finite 
Baire measures $\mu$ on $X$ and sequentially smooth positive linear 
functionals $f:C_b\to\Bbb R$, given by the formulae 
      
\Centerline{$f(u)=\biggerint u\,d\mu$ for every $u\in C_b$,} 
      
\Centerline{$\mu Z=\inf\{f(u):\chi Z\le u\in C_b\}$ for every zero 
set 
$Z\subseteq X$.} 
      
\proof{{\bf (a)} If $\mu$ is a totally finite Baire measure on $X$, then 
every continuous bounded real-valued function is integrable, and 
$f=\int d\mu$ is a sequentially smooth positive linear operator on 
$C_b$, by Fatou's Lemma, as usual. 
      
\medskip 
      
{\bf (b)} If $f:C_b\to\Bbb R$ is a sequentially smooth positive 
linear 
operator, then 436D tells us that there is a measure $\mu_0$ on $X$ such 
that $\int u\,d\mu_0$ is defined and equal to $f(u)$ for every $u\in 
C_b$.   By the construction in 436D, or otherwise, we may suppose 
that 
$\mu_0$ is complete, so that every $u\in C_b$ is 
$\Sigma$-measurable, where $\Sigma$ is the domain of $\mu_0$.   It 
follows by the definition of the Baire $\sigma$-algebra $\CalBa$ of 
$X$ (4A3K) that $\CalBa\subseteq\Sigma$, so that 
$\mu=\mu_0\restr\Sigma$ is a Baire measure;  of course we still have 
$f(u)=\int u\,d\mu$ for every $u\in C_b$.   Also, if $Z\subseteq X$ 
is a 
zero set, $\mu Z=\inf\{f(u):\chi Z\le u\in C_b\}$.   \Prf\  Express 
$Z$ 
as $\{x:v(x)=0\}$ where $v:X\to[0,1]$ is continuous.   Set 
      
\Centerline{$u_n=(\chi X-2^nv)^+$} 
      
\noindent for $n\in\Bbb N$;  then $\sequencen{u_n}$ is a non-increasing 
sequence in $C_b$ and $\sequencen{u_n(x)}\to(\chi Z)(x)$ for every 
$x\in X$, so 
      
$$\eqalign{\mu Z 
&\le\inf\{\int u\,d\mu:\chi Z\le u\in C_b\} 
=\inf\{f(u):\chi Z\le u\in C_b\}\cr 
&\le\inf_{n\in\Bbb N}f(u_n) 
=\lim_{n\to\infty}\int u_nd\mu 
=\mu Z.\text{ \Qed}\cr}$$ 
      
\medskip 
      
{\bf (c)} The argument of (b) shows that if two totally finite Baire 
measures give the same integrals to every member of $C_b$, then they 
must agree on all zero sets.   By the Monotone Class Theorem (136C) they 
agree on the $\sigma$-algebra generated by the zero sets, that is, 
$\CalBa$, and are therefore equal.   Thus the operator 
$\mu\mapsto\int d\mu$ from the set of totally finite Baire measures on 
$X$ to the set of sequentially smooth positive linear operators on 
$C_b$ 
is a bijection, and if $f=\int d\mu$ then $\mu Z=\inf\{f(u):\chi Z\le 
u\in C_b\}$ for every zero set $Z$, as required. 
}%end of proof of 436E 
      
\leader{436F}{}\cmmnt{ Corresponding to 434R, we have the 
following construction for product Baire measures, applicable to a 
slightly larger class of spaces. 
      
\medskip 
      
\noindent}{\bf Proposition} Let $X$ be a sequential space, $Y$ a 
topological space, and $\mu$, $\nu$ totally finite Baire measures on 
$X$, $Y$ respectively.   Then there is a Baire measure $\lambda$ on 
$X\times Y$ such that 
      
\Centerline{$\lambda W=\biggerint\nu W[\{x\}]\mu(dx)$, 
\quad$\int fd\lambda=\biggeriint f(x,y)\nu(dy)\mu(dx)$} 
      
\noindent for every Baire set $W\subseteq X\times Y$ and every bounded 
continuous function $f:X\times Y\to\Bbb R$. 
      
\proof{{\bf (a)} $\phi(f)=\iint f(x,y)dydx$ is defined in $\Bbb R$ for 
every bounded continuous function $f:X\times Y\to\Bbb R$.   \Prf\ For 
each $x\in X$, $g(x)=\int f(x,y)dy$ is defined because $y\mapsto f(x,y)$ 
is continuous.    If $\sequencen{x_n}$ is any sequence in $X$ converging 
to $x\in X$, then 
      
\Centerline{$g(x)=\biggerint f(x,y)dy=\int\lim_{n\to\infty}f(x_n,y)dy 
=\lim_{n\to\infty}\int f(x_n,y)dy=\lim_{n\to\infty}g(x_n)$} 
      
\noindent by Lebesgue's Dominated Convergence Theorem.   So $g$ is 
sequentially continuous;  because $X$ is sequential, $g$ is continuous 
(4A2Kd).   So $\iint f(x,y)dydx=\int g(x)dx$ is defined in 
$\Bbb R$.\ \Qed 
      
\medskip 
      
{\bf (b)} Of course $\phi$ is a positive linear functional on 
$C_b(X\times Y)$, and B.Levi's theorem shows that it is sequentially 
smooth.   By 436E, there is a Baire measure $\lambda$ on 
$X\times Y$ such that $\int fd\lambda=\phi(f)$ for every 
$f\in C_b(X\times Y)$. 
      
\medskip 
      
{\bf (c)} If $W\subseteq X\times Y$ is a zero set, there is a 
non-increasing sequence $\sequencen{f_n}$ in $C_b(X\times Y)$ such that 
$\chi W=\inf_{n\in\Bbb N}f_n$.   So B.Levi's theorem tells us that 
      
\Centerline{$\biggerint\nu W[\{x\}]dx 
=\lim_{n\to\infty}\int f_n(x,y)dydx 
=\lim_{n\to\infty}\int f_nd\lambda 
=\lambda W$.} 
      
\noindent Now the Monotone Class Theorem (136B) tells us that 
      
\Centerline{$\{W:W\subseteq X\times Y$ is Baire, $\int\nu W[\{x\}]dx$ 
exists $=\lambda W\}$} 
      
\noindent includes the $\sigma$-algebra generated by the zero sets, that 
is, contains every Baire set in $X\times Y$.   So $\lambda$ has the 
required properties. 
}%end of proof of 436F 
      
\leader{436G}{Definition} Let $X$ be a set, $U$ a Riesz subspace of 
$\Bbb R^X$, and $f:U\to\Bbb R$ a positive linear functional.   I say 
that $f$ is {\bf smooth} if whenever $A$ is a non-empty 
downwards-directed family in 
$U$ such that $\inf_{u\in A}u(x)=0$ for every $x\in X$, then 
$\inf_{u\in A}f(u)=0$. 
      
Of course a smooth functional is sequentially smooth.   If 
$(X,\frak T,\Sigma,\mu)$ is an effectively locally finite 
$\tau$-additive 
topological measure space and $U$ is a Riesz subspace of $\Bbb R^X$ 
consisting of integrable continuous functions, then 
$\int\,d\mu:U\to\Bbb R$ is 
smooth\cmmnt{, by 414Bb}.   \cmmnt{Corresponding to the remark in 
436A, note that an order-continuous positive linear functional must 
be smooth, but that a smooth positive linear functional need not be 
order-continuous.} 
      
\leader{436H}{Theorem} Let $X$ be a set and $U$ a truncated Riesz 
subspace of $\Bbb R^X$.   Let $f:U\to\Bbb R$ be a positive linear 
functional.   Then the following are equiveridical: 
      
(i) $f$ is smooth; 
      
(ii) there are a topology $\frak T$ and a measure $\mu$ on $X$ such that 
$\mu$ is a quasi-Radon measure with respect to $\frak T$, 
$U\subseteq C(X)$ and $\int u\,d\mu$ is defined and equal to $f(u)$ for 
every $u\in U$; 
      
(iii) writing $\frak S$ for the coarsest topology on $X$ for which every 
member of $U$ is continuous, there is a 
measure $\mu$ on $X$ such that $\mu$ is a quasi-Radon measure with 
respect to $\frak S$, 
and $\int u\,d\mu$ is defined and equal to $f(u)$ for every $u\in U$. 
      
\proof{ As remarked in 436G, in a fractionally more general context, 
(ii)$\Rightarrow$(i) is a 
consequence of 414B.   Of course (iii)$\Rightarrow$(ii).   So the 
argument here is devoted to proving that 
(i)$\Rightarrow$(iii).   Except for part (b) it is a simple adaptation 
of 
the method of 436D. 
      
\medskip 
      
{\bf (a)} Let $\Cal K$ be the family of sets $K\subseteq X$ such that 
$\chi K=\inf A$ in $\Bbb R^X$ for some non-empty set $A\subseteq U$. 
Then $\Cal K$ is closed under finite unions. 
\Prf\ If $K$, $K'\in\Cal K$ take $A$, $A'\subseteq U$ such that 
$\chi K=\inf A$, $\chi K'=A'$;  then 
$\chi(K\cup K')=\inf\{u\vee u':u\in A,\,u'\in A'\}$, so 
$K\cup K'\in\Cal K$.\ \Qed 
      
Note that $\emptyset\in\Cal K$ because $0\in U$. 
      
As in part (b) of the proof of 436D, $\{x:u(x)\ge\alpha\}\in\Cal K$ 
whenever $\alpha>0$ and $u\in U$. 
      
\medskip 
      
{\bf (b)} Every member of 
$\Cal K$ is closed for $\frak S$, being of the form 
$\{x:u(x)\ge 1$ for every $u\in A\}$ for some $A\subseteq U$.   We need 
to know that if $K\in\Cal K$ and $G\in\frak S$, then 
$K\setminus G\in\Cal K$.   \Prf\ Take a 
non-empty set $B\subseteq U$ such that $\chi K=\inf B$. 
Because $\Cal K$ is closed under 
finite unions and arbitrary intersections, 
$\frak S_K=\{G:G\subseteq X,\,K\setminus G\in\Cal K\}$ is a topology on 
$X$.  (i) If $u\in U$ and $G=\{x:u(x)>0\}$, then 
$\chi(K\setminus G)=\inf\{(v-ku)^+:v\in B,\,k\in\Bbb N\}$ so 
$K\setminus G\in\Cal K$ and $G\in\frak S_K$. 
(ii) If $u\in U$ and $\alpha>0$, then 
$\{x:u(x)>\alpha\}=\{x:(u-u\wedge\alpha\chi X)(x)>0\}$ belongs to 
$\frak S_K$, by (i). (iii) If $u\in U$ and $\alpha>0$, set 
$G=\{x:u(x)<\alpha\}$.   Then 
      
\Centerline{$K\setminus G 
=K\cap\{x:u(x)\ge\alpha\}\in\Cal K$,} 
      
\noindent so $G\in\frak S_K$. 
(iv) Thus every member of $U^+$ is $\frak S_K$-continuous (2A3Bc), so 
every member of $U$ is $\frak S_K$-continuous (2A3Be), and 
$\frak S\subseteq\frak S_K$, that is, 
$K\setminus G\in\Cal K$ for every $G\in\frak S$.\ \Qed 
      
\medskip 
      
{\bf (c)} For $K\in\Cal K$, set 
$\phi_0K=\inf\{f(u):u\in U,\,u\ge\chi K\}$.   Then $\phi_0$ satisfies 
the conditions of 415K.   \Prf\  I have 
already checked ($\dagger$) and ($\ddagger$) of 415K. 
      
\medskip 
      
\quad\grheada\ Fix $K$, $L\in\Cal K$ with $L\subseteq K$.   Set 
$\gamma=\sup\{\phi_0K':K'\in\Cal K,\,K'\subseteq K\setminus L\}$. 
      
\medskip 
      
\qquad{\bf (i)} Suppose that $K'\in\Cal K$ is included in $K\setminus L$ 
and $\epsilon>0$.   Set $A=\{u:\chi L\le u\in U\}$, $A'=\{u:\chi K'\le 
u\in 
U\}$, so that $\chi L=\inf A$ and $\chi K'=\inf A'$, and let $v\in U$ be 
such that $v\ge\chi K$ and $f(v)\le\phi_0K+\epsilon$.   Then $\{u\wedge 
u':u\in A,\,u'\in A'\}$ is a downwards-directed family with infimum $0$ 
in 
$\Bbb R^X$, so (because $f$ is smooth) there are $u\in A$, $u'\in A'$ 
such 
that $f(u\wedge u')\le\epsilon$.   In this case 
      
\Centerline{$\phi_0L+\phi_0K' 
\le f(v\wedge u)+f(v\wedge u') 
=f(v\wedge(u\vee u'))+f(v\wedge u\wedge u') 
\le\phi_0K+2\epsilon$.} 
      
\noindent As $\epsilon$ is arbitrary, $\phi_0L+\phi_0K'\le \phi_0K$. 
As 
$K'$ is arbitrary, $\phi_0L+\gamma\le\phi_0K$. 
      
\medskip 
      
\qquad{\bf (ii)} Next, given $\epsilon\in\ooint{0,1}$, there are $u$, 
$v\in 
U$ such that $u\ge\chi K$, $v\ge\chi L$ and $f(v)\le\phi_0L+\epsilon$. 
Consider 
      
\Centerline{$K'=\{x:x\in K,\,\min(1,u(x))-v(x)\ge\epsilon\}$.} 
      
\noindent By the last remark in (a), $K'\in\Cal K$.   If $w\in U$ and 
$w\ge\chi K'$, then $v(x)+w(x)\ge 1-\epsilon$ for every $x\in K$, so 
      
\Centerline{$\phi_0K 
\le\Bover1{1-\epsilon}f(v+w) 
\le\Bover1{1-\epsilon}(\phi_0L+\epsilon+f(w))$.} 
      
\noindent As $w$ is arbitrary, 
      
\Centerline{$(1-\epsilon)\phi_0K\le\phi_0 
L+\epsilon+\phi_0K'\le\phi_0L+\epsilon+\gamma$.} 
      
\noindent As $\epsilon$ is arbitrary, $\phi_0K\le\phi_0L+\gamma$ and we 
have equality, as required by ($\alpha$) in 415K. 
      
\medskip 
      
\quad\grheadb\ Now suppose that $\Cal K'$ is a non-empty 
downwards-directed subset of $\Cal K$ with empty intersection.   Set 
      
\Centerline{$A=\bigcup_{K\in\Cal K'}\{u:\chi K\le u\in U\}$.} 
      
\noindent Then $A$ is a downwards-directed subset of $U$ and $\inf A=0$ 
in $\Bbb R^X$.   Because $f$ is smooth, 
      
\Centerline{$0=\inf_{u\in A}f(u)=\inf_{K\in\Cal K'}\phi_0K$.} 
      
\noindent Thus ($\beta$) of 415K is satisfied. 
      
\medskip 
      
\quad\grheadc\ If $K\in\Cal K$ and $\phi_0K>0$, take $u\in U$ such that 
$u\ge\chi K$, and consider $G=\{x:u(x)>\bover12\}$.   Then 
$K\subseteq G$, while $G\subseteq\{x:2u(x)\ge 1\}$, so 
      
\Centerline{$\sup\{\phi_0K':K'\in\Cal K,\,K'\subseteq G\} 
\le 2f(u)<\infty$.} 
      
\noindent Thus $\phi_0$ satisfies ($\gamma$) of 415K.\ \Qed 
      
\medskip 
      
{\bf (d)} By 415K, there is a quasi-Radon measure $\mu$ on $X$ extending 
$\phi_0$.   By 436C, $f(u)=\int u\,d\mu$ for every $u\in U$. 
}%end of proof of 436H 
      
\medskip 
      
\noindent{\bf Remark}\cmmnt{ It is worth noting explicitly that} $\mu$, as 
constructed here, is inner regular with respect to the family $\Cal K$ 
of sets 
$K\subseteq X$ such that $\chi K=\inf A$ for some set $A\subseteq U$. 
      
\leader{436I}{Lemma} Let $X$ be a topological space.   Let $C_0=C_0(X)$ 
be the space of continuous functions $u:X\to\Bbb R$ which `vanish at 
infinity' in the sense that $\{x:|u(x)|\ge\epsilon\}$ is compact for 
every $\epsilon>0$. 
      
(a) $C_0$ is a norm-closed solid linear subspace of $C_b=C_b(X)$, so is 
a Banach lattice in its own right. 
      
(b) $C_0^*=C_0^{\sim}$ is an $L$-space\cmmnt{ (definition: 354M)}. 
      
(c) If $A\subseteq C_0$ is a non-empty downwards-directed set such 
that $\inf_{u\in A}u(x)=0$ for every $x\in X$, then 
$\inf_{u\in A}\|u\|_{\infty}=0$. 
      
\proof{{\bf (a)(i)} If $u\in C_0$, then $K=\{x:|u(x)|\ge 1\}$ is 
compact, so $\|u\|_{\infty}\le\sup(\{1\}\cup\{|u(x)|:x\in K\})$ is 
finite, and $u\in C_b$. 
      
\medskip 
      
\quad{\bf (ii)} If $u$, $v\in C_0$ and $\alpha\in\Bbb R$ and 
$w\in C_b$ and $|w|\le|u|$, then for any $\epsilon>0$ 
      
\Centerline{$\{x:|u(x)+v(x)|\ge\epsilon\} 
\subseteq\{x:|u(x)|\ge\Bover12\epsilon\} 
\cup\{x:|v(x)|\ge\Bover12\epsilon\}$,} 
      
\Centerline{$\{x:|\alpha u(x)|\ge\epsilon\} 
\subseteq\{x:|u(x)|\ge\Bover{\epsilon}{1+|\alpha|}\}$,} 
      
\Centerline{$\{x:|w(x)|\ge\epsilon\} 
\subseteq\{x:|u(x)|\ge\epsilon\}$} 
      
\noindent are closed relatively compact sets, so are compact, and $u+v$, 
$\alpha u$, $w$ belong to $C_0$.   Thus $C_0$ is a solid linear 
subspace 
of $C_b$. 
      
\medskip 
      
\quad{\bf (iii)} If $\sequencen{u_n}$ is a sequence in $C_0$ which 
$\|\,\|_{\infty}$-converges to $u\in C_b$, then for any $\epsilon>0$ 
there is an $n\in\Bbb N$ such that 
$\|u-u_n\|_{\infty}\le\bover12\epsilon$, so that 
      
\Centerline{$\{x:|u(x)|\ge\epsilon\} 
\subseteq\{x:|u_n(x)|\ge\Bover12\epsilon\}$} 
      
\noindent is compact, and $u\in C_0$.   Thus $C_0$ is norm-closed 
in 
$C_b$. 
      
\medskip 
      
\quad{\bf (iv)} Being a norm-closed Riesz subspace of a Banach lattice, 
$C_0$ is itself a Banach lattice. 
      
\medskip 
      
{\bf (b)} By 356Dc, $C_0^*=C_0^{\sim}$ is a Banach lattice.   Now 
$\|f+g\|=\|f\|+\|g\|$ for all non-negative $f$, $g\in C_0^*$.   \Prf\ 
Of course $\|f+g\|\le\|f\|+\|g\|$.   On the other hand, for any 
$\epsilon>0$ there are $u$, $v\in C_0$ such that $\|u\|_{\infty}\le 
1$, 
$\|v\|_{\infty}\le 1$ and 
$|f(u)|\ge\|f\|-\epsilon$, $|g(v)|\ge\|g\|-\epsilon$.   Set 
$w=|u|\vee|v|$;  then $w\in C_0$ and 
      
\Centerline{$\|w\|_{\infty} 
=\max(\|u\|_{\infty},\|v\|_{\infty})\le 1$.} 
      
\noindent So 
      
\Centerline{$\|f+g\|\ge(f+g)(w) 
\ge f(|u|)+g(|v|)\ge|f(u)|+|g(v)|\ge\|f\|+\|g\|-2\epsilon$.} 
      
\noindent As $\epsilon$ is arbitrary, $\|f+g\|\ge\|f\|+\|g\|$.\ \Qed 
      
So $C_0^*$ is an $L$-space. 
      
\medskip 
      
{\bf (c)} Let $\epsilon>0$.   For $u\in A$ set 
$K_u=\{x:u(x)\ge\epsilon\}$.   Then $\{K_u:u\in A\}$ is a 
downwards-directed family of closed compact sets with empty 
intersection, so there must be some $u\in A$ such that $K_u=\emptyset$, 
and $\|u\|_{\infty}\le\epsilon$.   As $\epsilon$ is arbitrary, we have 
the result. 
}%end of proof of 436I 
      
\cmmnt{\medskip 
      
\noindent{\bf Remark} (c) is a version of {\bf Dini's theorem}. 
} 
      
\vleader{48pt}{436J}{Riesz Representation Theorem (first form)} Let 
$(X,\frak T)$ be a locally compact Hausdorff space, and $C_k=C_k(X)$ the 
space of continuous real-valued functions on $X$ with compact 
support.   If $f:C_k\to\Bbb R$ is any positive linear functional, 
there is a unique Radon measure $\mu$ on $X$ such that 
$f(u)=\int u\,d\mu$ for every $u\in C_k$. 
      
\proof{{\bf (a)} The point is that $f$ is smooth.   \Prf\ Suppose that 
$A\subseteq C_k$ is non-empty and downwards-directed and that 
$\inf A=0$ 
in $\Bbb R^X$.   Fix $u_0\in A$ and set $K=\overline{\{x:u_0(x)>0\}}$, 
so that $K$ is compact.   Because $X$ is locally compact, there is an 
open relatively compact set $G\supseteq K$.   Now there is a continuous 
function $u_1:X\to[0,1]$ such that $u_1(x)=1$ for $x\in K$ and 
$u_1(x)=0$ for $x\in X\setminus G$ (4A2F(h-iii)).   Because $G$ is 
relatively compact, $u_1\in C_k$. 
      
Take any $\epsilon>0$.   By 436Ic, 
there is a $v\in A$ such that $\|v\|_{\infty}\le\epsilon$.   Now there 
is a $v'\in A$ such that $v'\le v\wedge u_0$, so that $v'(x)\le\epsilon$ 
for every $x\in K$ and $v'(x)=0$ for $x\notin K$.   In this case 
$v'\le\epsilon u_1$, and 
      
\Centerline{$\inf_{u\in A}f(u)\le f(v')\le\epsilon f(u_1)$.} 
      
\noindent As $\epsilon$ is arbitrary, $\inf_{u\in A}f(u)=0$;  as $A$ is 
arbitrary, $f$ is smooth.\ \Qed 
      
\medskip 
      
{\bf (b)} Note that because $\frak T$ is locally compact, it is the 
coarsest topology on $X$ for which every function in $C_k$ is 
continuous (4A2G(e-ii)).   Also $C_k$ is a truncated Riesz subspace 
of $\BbbR^X$. 
So 436H tells us that there is a quasi-Radon measure $\mu$ 
on $X$ such that $f(u)=\int u\,d\mu$ for every $u\in C_k$. 
And $\mu$ is locally finite.   \Prf\ If $x_0\in X$, then (as in (a) 
above) there is a $u_1\in C_k^+$ such that $u_1(x_0)=1$;  now 
$G=\{x:u_1(x)>\bover12\}$ is an open set containing $x_0$, and 
$\mu G\le 2f(u_1)$ is finite.\ \Qed 
      
By 416G, or otherwise, $\mu$ is a Radon measure. 
      
\medskip 
      
{\bf (c)} By 416E(b-v), $\mu$ is unique. 
}%end of proof of 436J 
      
\leader{436K}{Riesz Representation Theorem (second form)} Let 
$(X,\frak T)$ be a locally compact Hausdorff space.   If 
$f:C_0(X)\to\Bbb R$ is any positive linear functional, 
there is a unique totally finite Radon measure $\mu$ on $X$ such that 
$f(u)=\int u\,d\mu$ for every $u\in C_0=C_0(X)$. 
      
\proof{{\bf (a)} As noted in 436Ib, $C_0^*=C_0^{\sim}$, so $f$ is 
$\|\,\|_{\infty}$-continuous.   $C_k(X)$ is a linear subspace of 
$C_0$, and $f\restr C_k(X)$ is a positive linear functional;  so by 
436J 
there is a unique Radon measure $\mu$ on $X$ such that 
$f(u)=\int u\,d\mu$ for every $u\in C_k(X)$.   Now $\mu$ is totally 
finite.   \Prf\ By 414Ab, 
      
\Centerline{$\mu X=\sup\{f(u):u\in C_k(X),\,0\le u\le\chi X\} 
\le\|f\|<\infty$.  \Qed} 
      
\medskip 
      
{\bf (b)} Accordingly $\int u\,d\mu$ is defined for every $u\in C_b(X)$, 
and in particular for every $u\in C_0$.   Next, $C_k=C_k(X)$ is 
norm-dense in 
$C_0$.   \Prf\ If $u\in C_0^+$, then $u_n=(u-2^{-n}\chi X)^+$ 
belongs to 
$C_k$ and $\|u-u_n\|_{\infty}\le 2^{-n}$ for every $n\in\Bbb N$, so 
$u\in\overline{C}_k$;  accordingly $C_0=C_0^+-C_0^+$ is 
included in 
$\overline{C}_k$.\ \QeD\   Since $\int\,d\mu$, regarded as a linear 
functional on $C_0$, is positive, therefore continuous, and agrees 
with $f$ on $C_k$, it must be identical to $f$.   Thus 
$f(u)=\int u\,d\mu$ for every $u\in C_0$. 
      
\medskip 
      
{\bf (c)} Because there is only one Radon measure giving the right 
integrals to members of $C_k$ (436J), $\mu$ is unique. 
}%end of proof of 436K 
      
\leader{*436L}{}\cmmnt{ The results here, by opening a path between 
measure theory and the study of linear functionals on spaces of 
continuous functions, provide an enormously powerful tool for the 
analysis of dual spaces $C(X)^*$ and their relatives.   I will explore 
some of these ideas in the next section.   Here I will give only a 
sample pair of facts to show how measure theory can tell us things about 
Banach lattices which seem difficult to reach by other methods. 
      
\medskip 
      
\noindent}{\bf Proposition} Let $X$ be a topological space;  write 
$C_b$ for $C_b(X)$.   Suppose that $U$ is a 
norm-closed linear subspace of $C_b^*$ such that the functional 
$u\mapsto f(u\times v):C_b\to\Bbb R$ belongs to $U$ whenever $f\in U$ 
and $v\in C_b$.   Then $U$ is a band in the $L$-space $C_b^*$. 
      
\proof{{\bf (a)} Let $e=\chi X$ be the standard order unit of 
$C_b$, and if 
$f\in C_b^*$ and $u$, $v\in C_b$ write $f_v(u)$ for 
$f(u\times v)$.   By 356Na, $C_b^*=C_b^{\sim}$ is an $L$-space. 
      
\medskip 
      
{\bf (b)} I show first that $U$ is a Riesz subspace of $C_b^{\sim}$. 
\Prf\ If $f\in U$ and $\epsilon>0$, there is a $v\in C_b$ such that 
$|v|\le e$ and $f(v)\ge|f|(e)-\epsilon$ (356B).   Now $f_v\le|f|$ and 
      
\Centerline{$\||f|-f_v\|=(|f|-f_v)(e)\le\epsilon$} 
      
\noindent (356Nb), while 
$f_v\in U$.   As $\epsilon$ is arbitrary, $|f|\in\overline{U}=U$;  as 
$f$ is arbitrary, $U$ is a Riesz subspace of $C_b^*$ (352Ic).\ \Qed 
      
\medskip 
      
{\bf (c)} Now suppose that $X$ is a compact Hausdorff space.   Then $U$ 
is a solid linear subspace of $C_b^{\sim}=C(X)^{\sim}$.   \Prf\ 
Suppose that $f\in U$ and that $0\le g\le f$.   Let $\epsilon>0$.   By 
either 436J or 436K, there are Radon measures $\mu$, $\nu$ on $X$ such 
that $f(u)=\int u\,d\mu$ and $g(u)=\int u\,d\nu$ for every 
$u\in C(X)$.   By 416Ea, $\nu\le\mu$ in the sense of 234P, 
so $\nu$ is an indefinite-integral measure over 
$\mu$ (415Oa, or otherwise);  let $w:X\to[0,1]$ be such that 
$\int_Ew\,d\mu=\nu E$ for every $E\in\dom\nu$.    There is a continuous 
function $v:X\to\Bbb R$ such that $\int|w-v|d\mu\le\epsilon$ 
(416I), and now 
      
$$\eqalignno{|g(u)-f_v(u)| 
&=|\int u\,d\nu-\int u\times v\,d\mu| 
=|\int u\times(w-v)d\mu|\cr 
\displaycause{235K} 
&\le\|u\|_{\infty}\int|w-v|d\mu 
\le\epsilon\|u\|_{\infty}\cr}$$ 
      
\noindent for every $u\in C(X)$, so $\|g-f_v\|\le\epsilon$, while 
$f_v\in U$.   As $\epsilon$ is arbitrary, $g\in U$;  as $f$ and $g$ are 
arbitrary (and $U$ is a Riesz subspace of $C(X)^*$), $U$ is a solid 
linear subspace of $C(X)^*$.\ \Qed 
      
Since every norm-closed solid linear subspace of an $L$-space is a band 
(354Eg), it follows that (provided $X$ is a compact Hausdorff space) 
$U$ is actually a band. 
      
\medskip 
      
{\bf (d)} For the general case, let $Z$ be the set of all Riesz 
homomorphisms $z:C_b\to\Bbb R$ such that $z(e)=1$.   Then $Z$ is a 
weak*-closed subset of the unit ball of $C_b^*$ so is a compact 
Hausdorff space.   We have a 
Banach lattice isomorphism $T:C_b\to C(Z)$ given by the formula 
$(Tu)(z)=z(u)$ for $u\in C_b$, $z\in Z$ (see the proofs of 353M and 
354K).   But note also that $T$ is multiplicative (353Pd), and 
$T':C(Z)^*\to C_b^*$ is a Banach lattice isomorphism.   Let 
$V$ be $(T')^{-1}[U]\subseteq C(Z)^*$;   
then $V$ is a closed linear subspace 
of $C(Z)^*$.   If $g\in V$ and $v$, $w\in C(Z)$, then 
      
\Centerline{$g(v\times w)=(T'g)(T^{-1}v\times T^{-1}w)$,} 
      
\noindent so $g_w$, defined in $C(Z)^*$ by the convention of 
(a) above, is just $(T')^{-1}((T'g)_{T^{-1}w})$, and belongs to $V$. 
By (b), $V$ is a band in $C(Z)^*$ so $U$ is a band in $C_b^*$, as 
required. 
}%end of proof of 436L 
      
\leader{*436M}{Corollary} Let $\frak A$ be a Boolean algebra, and 
$M(\frak A)$ the $L$-space of bounded finitely additive functionals on 
$\frak A$\cmmnt{ (362B)}.   Let $U\subseteq M(\frak A)$ be a norm-closed linear 
subspace such that $a\mapsto\nu(a\Bcap b)$ belongs to $U$ whenever 
$\nu\in U$ and $b\in\frak A$.   Then $U$ is a band in $M(\frak A)$. 
 
\proof{{\bf (a)} 
Let $Z$ be the Stone space of $\frak A$, so that $C(Z)$ is the 
$M$-space $L^{\infty}(\frak A)$ (363A), and we have an $L$-space 
isomorphism $T:M(\frak A)\to C(Z)^*$ defined by saying that  
$(T\nu)(\chi\widehat{a})=\nu a$ whenever $\nu\in M(\frak A)$, $a\in\frak A$ 
and $\widehat{a}$ is the open-and-closed subset of $Z$ corresponding to 
$a$ (363K).   Now $V=T[U]$ is a norm-closed linear subspace of $C(Z)^*$.
 
\medskip 
 
{\bf (b)} $V$ satisfies the condition of 436L.   \Prf\ Suppose that  
$f\in V$ and $v\in C(Z)$;  set $f_v(u)=f(u\times v)$ for $u\in C(Z)$, and 
$\nu=T^{-1}f\in U$.    
(i) If $v$ is of the form $\chi\widehat{b}$, 
where $b\in\frak A$, then $\nu_b\in U$, where 
$\nu a=\nu(a\Bcap b)$ for $a\in\frak A$.   Now $T\nu_b\in V$ and 
 
\Centerline{$(T\nu_b)(\chi\widehat{a})=\nu_ba=\nu(a\Bcap b) 
=f(\chi(\widehat{a\Bcap b}))=f(\chi\widehat{a}\times\chi\widehat{b}) 
=f_v(\chi\widehat{a})$} 
 
\noindent for every $a\in\frak A$, so $f_v=T\nu_b$ belongs to $U$. 
(ii) If $v$ is of the form $\sum_{i=0}^n\alpha_i\chi\widehat{b}_i$, where 
$b_0,\ldots,b_n\in\frak A$ and $\alpha_0,\ldots,\alpha_n\in\Bbb R$, then 
$f_v=\sum_{i=0}^n\alpha_if_{v_i}$, where $v_i=\chi\widehat{b}_i$ for each 
$i$;   as $f_{v_i}\in V$ for each $i$, $f_v\in V$, because $V$ is a linear 
subspace.  (iii) In general, given 
$v\in C(Z)=L^{\infty}(\frak A)$ and $\epsilon>0$, there is a  
$w\in C(Z)$, expressible in the form of (ii), such that 
$\|v-w\|_{\infty}\le\epsilon$ (363C).   In this case $f_w\in V$, by (ii), 
while 
 
\Centerline{$|f_v(u)-f_w(u)| 
=|f(u\times(v-w))|\le\|f\|\|u\|_{\infty}\epsilon$} 
 
\noindent for every $u\in C(Z)$, and $\|f_v-f_w\|\le\epsilon\|f\|$.   As 
$\epsilon$ is arbitrary and $V$ is closed, $f_v\in V$.\ \Qed 
 
\medskip 
 
{\bf (c)} By 436L, $V$ is a band in $C(Z)^*$;  as $T$ is a Riesz space 
isomorphism, $U$ is a band in $M(\frak A)$. 
}%end of proof of 436M 
      
\exercises{\leader{436X}{Basic exercises $\pmb{>}$(a)} 
%\spheader 436Xa 
Let $(X,\Sigma,\mu_0)$ be a measure space, and $U$ the set of 
$\mu_0$-integrable $\Sigma$-measurable real-valued functions defined 
everywhere on $X$.   For $u\in U$ set $f(u)=\int u\,d\mu_0$.   Show that 
$U$ and $f$ satisfy the conditions of 436D, and that the measure $\mu$ 
constructed from $f$ by the procedure there is just the c.l.d.\ version 
of $\mu_0$. 
%436D 
      
\spheader 436Xb Let $\mu$ and $\nu$ be two complete locally determined 
measures on a set $X$, and suppose that $\int f\,d\mu=\int f\,d\nu$ for 
every function $f:X\to\Bbb R$ for which either integral is defined in 
$\Bbb R$.   Show that $\mu=\nu$. 
%436D 
      
\sqheader 436Xc Let $X$ be a set, $U$ a truncated Riesz subspace of 
$\Bbb R^X$, and $f:U\to\Bbb R$ a sequentially smooth positive linear 
functional.   For $A\subseteq X$ set 
      
$$\eqalign{\theta A 
&=\inf\{\sup_{n\in\Bbb N}f(u_n): 
  \sequencen{u_n}\text{ is a non-decreasing sequence in }U^+,\cr 
&\qquad\qquad\qquad\qquad\qquad\qquad\qquad 
  \lim_{n\to\infty}u_n(x)=1\text{ for every }x\in A\},\cr}$$ 
      
\noindent taking $\inf\emptyset=\infty$ if need be.   Show that $\theta$ 
is an outer measure on $X$.   Let $\mu_0$ be the measure defined from 
$\theta$ by \Caratheodory's method.   Show that $f(u)=\int u\,d\mu_0$ 
for every $u\in U$.   Show that the measure $\mu$ constructed in 436D is 
the c.l.d.\ version of $\mu_0$. 
%436D 
      
\spheader 436Xd Let $X$ be a set and $U$ a truncated Riesz subspace of 
$\Bbb R^X$.   Let $\tau:U\to\coint{0,\infty}$ be a seminorm such that 
(i) $\tau(u)\le\tau(v)$ whenever $|u|\le|v|$ (ii) 
$\lim_{n\to\infty}\tau(u_n)=0$ whenever $\sequencen{u_n}$ is a 
non-increasing sequence in $U$ and $\lim_{n\to\infty}u_n(x)=0$ for every 
$x\in X$.   Show that for any $u_0\in U^+$ there is a measure $\mu$ on 
$x$ such that $\int u\,d\mu$ is defined, and less than or equal to 
$\tau(u)$, for every $u\in U$, and $\int u_0d\mu=\tau(u_0)$.   \Hint{put 
the Hahn-Banach theorem together with 436D.} 
%436D 
      
\spheader 436Xe\dvArevised{2010}(i) 
Let $X$ be any topological space.   Show that every 
positive linear functional on $C(X)$ is sequentially smooth (compare 
375A), so corresponds to a totally finite Baire measure on $X$. 
(ii) Let $X$ be a regular Lindel\"of space.   Show that every positive 
linear functional on $C(X)$ is smooth, so corresponds to a totally finite 
quasi-Radon measure on $X$.   (iii) Let $X$ be a K-analytic 
Hausdorff space.   Show that every positive linear functional on $C(X)$ 
corresponds to at least one totally finite Radon measure on $X$. 
%436E 
      
\spheader 436Xf Let $X$ be a completely regular space.   Show that it is 
measure-compact iff every sequentially smooth positive linear functional 
on $C_b(X)$ is smooth.   \Hint{436Xj.} 
%4{}35D 436E 
      
\spheader 436Xg A completely regular topological space 
$X$ is called {\bf realcompact} 
if every Riesz homomorphism from $C(X)$ to $\Bbb R$ is of the form 
$u\mapsto\alpha u(x)$ for some $x\in X$ and $\alpha\ge 0$.    
(i) Show that, for any topological space $X$, any Riesz 
homomorphism from $C(X)$ to $\Bbb R$ is representable by a Baire measure 
on $X$ which takes at most two values.  (ii) Show that 
a completely regular space $X$ is realcompact iff every $\{0,1\}$-valued 
Baire measure on $X$ is of the form $E\mapsto\chi E(x)$.   (iii) Show 
that a completely regular space $X$ is realcompact iff every purely 
atomic totally finite Baire measure on $X$ is $\tau$-additive.   (iii) 
Show that a measure-compact completely regular space is realcompact. 
(iv) Show that the discrete topology on $[0,1]$ is realcompact. 
\Hint{if $\nu$ is a Baire measure taking only the values $0$ and $1$, 
set $x_0=\sup\{x:\nu[x,1]=1\}$.}   (v) Show that any product of 
realcompact completely regular spaces is realcompact.   (vi) Show that 
a Souslin-F subset of a realcompact completely regular space is 
realcompact.   (For realcompact spaces which are not measure-compact, 
see 439Xn.) 
%refers to 439K, 439L, 439P, 439Q 
%4{}35D 436E 
      
\spheader 436Xh In 436F, suppose that $\mu$ and $\nu$ are 
$\tau$-additive.   Let $\tilde\mu$ and $\tilde\nu$ be the corresponding 
quasi-Radon measures (415N), and $\tilde\lambda$ the quasi-Radon product 
of $\tilde\mu$ and $\tilde\nu$ (417N).   Show that $\lambda$ is the 
restriction of $\tilde\lambda$ to the Baire $\sigma$-algebra of 
$X\times Y$. 
%436F 
      
\sqheader 436Xi For $u\in C([0,1])$ let $f(u)$ be the Lebesgue integral 
of $u$.   Show that $f$ is smooth (therefore sequentially smooth) but 
not sequentially order-continuous (therefore not order-continuous). 
\Hint{enumerate $\Bbb Q\cap[0,1]$ as $\sequencen{q_n}$, and set 
$u_n(x)=\min_{i\le n}2^{i+2}|x-q_i|$ for $n\in\Bbb N$, $x\in[0,1]$; 
show that $\inf_{n\in\Bbb N}u_n=0$ in $C([0,1])$ but 
$\lim_{n\to\infty}f(u_n)>0$.} 
%436G 
      
\spheader 436Xj In 436E, show that $\mu$ is $\tau$-additive iff $f$ is 
smooth. 
%436G 
      
\spheader 436Xk Suppose that $X$ is a set, $U$ is a truncated Riesz 
subspace of $\Bbb R^X$ and $f:U\to\Bbb R$ is a smooth positive linear 
functional.   For $A\subseteq X$ set 
      
$$\eqalign{\theta A=\inf\{\sup_{u\in B}f(u):B\text{ is} 
&\text{ an upwards-directed family in }U^+\cr 
&\text{ such that }\sup_{u\in B}u(x)=1\text{ for every }x\in A\},\cr}$$ 
      
\noindent taking $\inf\emptyset=\infty$ if need be.   Show that $\theta$ 
is an outer measure on $X$.   Let $\mu_0$ be the measure defined from 
$\theta$ by \Caratheodory's method.   Show that $f(u)=\int u\,d\mu_0$ 
for every $u\in U$.   Show that the measure $\mu$ constructed in 436H is 
the c.l.d.\ version of $\mu_0$. 
%436H 
      
\spheader 436Xl Let $X$ be a completely regular topological space and 
$f$ a smooth positive linear functional on $C_b(X)$.   Show that there 
is a unique totally finite quasi-Radon measure $\mu$ on $X$ such that 
$f(u)=\int u\,d\mu$ for every $u\in C_b(X)$. 
%436H 
      
\sqheader 436Xm For $u\in C([0,1])$ let $f(u)$ be the Riemann integral 
of $u$ (134K).   Show that the Radon measure on $[0,1]$ constructed by 
the method of 436J is just Lebesgue measure on $[0,1]$.   Explain how to 
construct Lebesgue measure on $\Bbb R$ from an appropriate version of 
the Riemann integral on $\Bbb R$. 
%436J 
      
\spheader 436Xn Let $X$ be a topological space.   Let 
$f:C_b(X)\to\Bbb R$ be a linear functional.   (i) Show that the 
following are equiveridical:  ($\alpha$) $f$ is {\bf tight},  
that is, for every $\epsilon>0$ there is a closed compact 
$K\subseteq X$ such that $|f(u)|\le\epsilon$ whenever 
$0\le u\le\chi(X\setminus K)$ ($\beta$) there is a tight totally finite 
Borel measure  
$\mu$ on $X$ such that $|f(u)|\le\int|u|\,d\mu$ for every $u\in C_b(X)$.   
\Hint{show that a positive tight functional is smooth.} 
(ii)\dvAnew{2009}  
Show that the set of tight functionals on $C_b(X)$ is a band in 
$C_b(X)^{\sim}$. 
%436J 
      
\sqheader 436Xo Let $(X,\frak T,\Sigma,\mu)$ and 
$(Y,\frak S,\Tau,\nu)$ be locally compact Radon measure spaces.   (i) 
Show that the 
function $x\mapsto\int w(x,y)\nu(dy)$ belongs to $C_k(X)$ for every 
$w\in C_k(X\times Y)$, so we have a 
positive linear functional $h:C_k(X\times Y)\to\Bbb R$ defined by 
setting 
      
\Centerline{$h(w)=\biggeriint w(x,y)\nu(dy)\mu(dx)$} 
      
\noindent for $w\in C_k(X\times Y)$.   (ii) Show that the corresponding 
Radon measure on $X\times Y$ is just the product Radon 
measure as defined in 417P/417R. 
      
\sqheader 436Xp Let $\frak A$ be a Boolean algebra and $Z$ its Stone 
space;  identify $L^{\infty}(\frak A)$ with $C(Z)$, as in 363A.   Let 
$\nu$ be a non-negative finitely additive functional on $\frak A$,  $f$ 
the corresponding positive linear functional on $L^{\infty}(\frak A)$ 
(363K), and $\mu$ the corresponding Radon measure on $Z$ (416Qb).   Show 
that $f(u)=\int u\,d\mu$ for every $u\in L^{\infty}(\frak A)$. 
%436J/436K 
      
\spheader 436Xq Let $X$ be a locally compact Hausdorff space.   Show 
that a sequence $\sequencen{u_n}$ in $C_0(X)$ converges to $0$ for the 
weak topology on $C_0(X)$ iff $\sup_{n\in\Bbb N}\|u_n\|_{\infty}$ is 
finite and $\lim_{n\to\infty}f_n(x)=0$ for every $x\in X$. 
%436K 
      
\spheader 436Xr Let $X$ be a non-empty compact Hausdorff space, and 
$\phi:X\to X$ a continuous function.   Show that there is a Radon 
probability measure $\mu$ on $X$ such that 
$\phi$ is \imp\ for $\mu$.   \Hint{let $\Cal F$ be a non-principal 
ultrafilter on $X$, $x_0$ any point of $X$, and define $\mu$ by the 
formula 
$\int ud\mu=\lim_{n\to\Cal F}\Bover1{n+1}\sum_{k=0}^nu(\phi^k(x_0))$ for 
every $u\in C(X)$.   Use 416E(b-v) to show that $\mu\phi^{-1}=\mu$.} 
 
\spheader 436Xs Let $X$ be a locally compact Hausdorff space.    
(i) Write $M^{\infty+}_{\text{R}}$ for the set of all Radon measures on 
$X$.   For $\mu\in M^{\infty+}_{\text{R}}$, let $S\mu$ be the  
corresponding 
functional on $C_k(X)$, defined by setting $(S\mu)(u)=\int u\,d\mu$ for 
every $u\in C_k(X)$.   Show that 
$S(\mu+\nu)=S\mu+S\nu$ and $S(\alpha\mu)=\alpha S\mu$ 
whenever $\mu$, $\nu\in M^{\infty+}_{\text{R}}$ and 
$\alpha\ge 0$, where addition and scalar multiplication of measures 
are defined as in 234G and 234Xf. 
(ii) Write $M^+_{\text{R}}$ for the set of totally finite Radon measures on 
$X$.   For $\mu\in M^+_{\text{R}}$, let $T\mu$ be the  
corresponding 
functional on $C_b(X)$, defined by setting $(T\mu)(u)=\int u\,d\mu$ for 
every $u\in C_b(X)$.   Show that 
$T(\mu+\nu)=T\mu+T\nu$ and $T(\alpha\mu)=\alpha T\mu$ 
whenever $\mu$, $\nu\in M^+_{\text{R}}$ and $\alpha\ge 0$. 
%436I 436J out of order query 
 
      
\leader{436Y}{Further exercises (a)} 
%\spheader 436Ya 
Let $X$ be a set, $U$ a Riesz subspace of $\Bbb R^X$, and $f:U\to\Bbb R$ 
a sequentially smooth positive linear functional.   (i) Write 
$U_{\sigma}$ for the set of functions from $X$ to $[0,\infty]$ 
expressible as the 
supremum of a non-decreasing sequence $\sequencen{u_n}$ in $U^+$ such 
that $\sup_{n\in\Bbb N}f(u_n)<\infty$.   Show that there is a functional 
$f_{\sigma}:U_{\sigma}\to\coint{0,\infty}$ such that 
$f_{\sigma}(u)=\sup_{n\in\Bbb N}f(u_n)$ whenever $\sequencen{u_n}$ is a 
non-decreasing sequence in $U^+$ with supremum $u\in U_{\sigma}$. 
(Compare 122I.)   (ii) Show that $u+v\in U_{\sigma}$ and 
$f_{\sigma}(u+v)=f_{\sigma}(u)+f_{\sigma}(v)$ for all $u$, $v\in 
U_{\sigma}$.   (iii) Suppose that $u$, $v\in U_{\sigma}$, $u\le v$ and 
$u(x)=v(x)$ whenever $v(x)$ is finite.   Show that 
$f_{\sigma}(u)=f_{\sigma}(v)$.   \Hint{take non-decreasing sequences 
$\sequencen{u_n}$, $\sequencen{v_n}$ with suprema $u$, $v$.   Consider 
$\sequencen{f(v_k-u_n-\delta v_n)^+}$ where $k\in\Bbb N$, $\delta>0$.} 
(iv) Let $V$ be the set of functions $v:X\to\Bbb R$ such that there are 
$u_1$, $u_2\in U_{\sigma}$ such that $v(x)=u_1(x)-u_2(x)$ whenever 
$u_1(x)$, $u_2(x)$ are both finite.   Show that $V$ is a linear subspace 
of $\Bbb R^X$ and that there is a linear functional $g:V\to\Bbb R$ 
defined by setting $g(v)=f_{\sigma}(u_1)-f_{\sigma}(u_2)$ whenever 
$v=u_1-u_2$ in the sense of the last sentence.   (v) Show that $V$ is a 
Riesz subspace of $\Bbb R^X$.   (vi) Show that if $\sequencen{v_n}$ is a 
non-decreasing sequence in $V$ and $\gamma=\sup_{n\in\Bbb N}g(v_n)$ is 
finite, then there is a $v\in V$ such that $g(v)=\sup_{n\in\Bbb 
N}g(v\wedge v_n)=\gamma$.   (This is a version of 
the {\bf Daniell integral}.) 
%436D 
      
\spheader 436Yb Develop further the theory of 436Ya, finding a version 
of Lebesgue's Dominated Convergence Theorem, a concept of `negligible' 
subset of $X$, and an $L$-space of equivalence classes of `integrable' 
functions. 
%436Ya, 436D 
      
\spheader 436Yc Let $X$ be a countably compact topological space. 
Show that every positive linear functional on $C_b(X)$ is sequentially 
smooth, so corresponds to a totally finite Baire measure on $X$. 
%436Xe 436E 
      
\spheader 436Yd Let $X$ be a sequential space, $Y$ a topological space, 
$\mu$ a semi-finite Baire measure on $X$ and $\nu$ a $\sigma$-finite 
Baire measure on $Y$.   Let $\tilde\mu$ be the c.l.d.\ version of $\mu$. 
Show that there is a semi-finite Baire measure $\lambda$ on $X\times Y$ 
such that 
      
\Centerline{$\lambda W=\biggerint\nu W[\{x\}]\tilde\mu(dx)$, 
\quad$\int fd\lambda=\biggeriint f(x,y)\nu(dy)\tilde\mu(dx)$} 
      
\noindent for every Baire set $W\subseteq X\times Y$ and every 
non-negative continuous function $f:X\times Y\to\Bbb R$.   Show that the 
c.l.d.\ version of $\lambda$ extends the c.l.d.\ product measure of 
$\mu$ and $\nu$. 
%436F 
      
\spheader 436Ye Let $X_0,\ldots,X_n$ be sequential spaces and $\mu_i$ a 
totally finite Baire measure on $X_i$ for each $i$.   (i) Show that if 
$f:X_0\times\ldots\times X_n\to\Bbb R$ is a bounded separately 
continuous function, then 
      
\Centerline{$\phi(f)=\biggerint\ldots\int 
f(x_0,\ldots,x_n)\mu_n(dx_n)\ldots\mu_0(dx_0)$} 
      
\noindent is defined, so that we have a corresponding Baire product 
measure on $X_0\times\ldots\times X_n$.   (ii) Show that this product is 
associative. 
%436F 
      
\spheader 436Yf Let $X$ and $Y$ be compact Hausdorff spaces.   (i) Show 
that there is a unique bilinear operator 
$\phi:C(X)^*\times C(Y)^*\to C(X\times Y)^*$ which is separately 
continuous for the weak* topologies and such that 
$\phi(\delta_x,\delta_y)=\delta_{(x,y)}$ for all $x\in X$ and $y\in Y$, 
setting $\delta_x(f)=f(x)$ for $f\in C(X)$ and $x\in X$.   (ii) Show 
that if $\mu$ and $\nu$ are Radon measures on $X$, $Y$ respectively with 
Radon measure product $\lambda$, then 
$\phi(\int d\mu,\int d\nu)=\int d\lambda$.   (iii) Show that 
$\|\phi\|\le 1$ (definition:  253Ab). 
%436Xo, 436J 
 
\spheader 436Yg Let $X$ be any topological space.   (i) Let $C_k$ be the 
set of continuous functions $u:X\to\Bbb R$ such that 
$\overline{\{x:u(x)\ne 0\}}$ is compact, and $f:C_k\to\Bbb R$ a positive 
linear functional.   Show that there is a tight quasi-Radon measure $\mu$  
on $X$ such that $f(u)=\int u\,d\mu$ for every $u\in C_k$. 
(ii) Let $\tilde C_k$ be the 
set of continuous functions $u:X\to\Bbb R$ such that 
$\{x:u(x)\ne 0\}$ is relatively compact, and $f:\tilde C_k\to\Bbb R$ a  
positive 
linear functional.   Show that there is a tight quasi-Radon measure $\mu$  
on $X$ such that $f(u)=\int u\,d\mu$ for every $u\in C_k$. 
%436J
}%end of exercises 
      
\endnotes{ 
\Notesheader{436} 
From the beginning, integration has been at the centre of measure 
theory.   My own view -- implicit in the arrangement of this treatise, 
from Chapter 
11 onward -- is that `measure' and `integration' are not quite the same 
thing.   I freely acknowledge that my treatment of `integration' is 
distorted by my presentation of it as part of measure theory;  on the 
other side of the argument, I hold that regarding `measure' as a concept 
subsidiary to `integral', as many authors do, seriously interferes with 
the development of truly penetrating intuitions for the former.   But it 
is undoubtedly true that every complete locally determined measure can 
be derived from its associated integral (436Xb).   Moreover, it is 
clearly of the highest importance that we should be able to recognise 
integrals when we see them;  I mean, given a linear functional on a 
linear space of functions, then if it can be expressed as an integral 
with respect to a 
measure this is something we need to know at once.   And thirdly, 
investigation of linear functionals frequently leads us to measures of 
great importance and interest. 
      
Concerning the conditions in 436D, an integral must surely be `positive' 
(because measures in this treatise are always non-negative) and 
`sequentially smooth' (because measures are supposed to be countably 
additive).   But it is not clear that we are forced to restrict our 
attention to Riesz subspaces of $\Bbb R^X$, and even less clear that 
they have to be `truncated'.   In 439I below I give an example to show 
that this last condition is essential for 436D and 436H as stated. 
However it is not necessary for large parts of the theory.   In many 
cases, if $U\subseteq\Bbb R^X$ is a Riesz subspace which is not 
truncated, we can take an element $e\in U^+$ and look at 
$Y=\{x:e(x)>0\}$, 
$V_e=\{u:u(x)=0$ for every $x\in X\setminus Y\}\cong W_e$, where 
$W_e=\{u/e:u\in V_e\}$ is a truncated Riesz subspace of $\BbbR^Y$. 
But there are applications in 
which this approach is unsatisfactory and a more radical revision of the 
basic theory of integration, as in 436Ya, is useful. 
      
I have based the arguments of this section on the inner measure 
constructions of \S413.   Of course it is also possible to approach them 
by means of outer measures (436Xc, 436Xk). 
      
I emphasize the exercise 436Xo because it is prominent in `Bourbakist' 
versions of the 
theory of Radon measures, in which (following {\smc Bourbaki 65} rather 
than {\smc Bourbaki 69}) Radon measures are regarded as linear 
functionals on spaces of continuous functions.   By 436J, this is a 
reasonably effective approach as long as we are interested only in 
locally compact spaces, and there are parts of the theory of topological 
groups (notably the duality theory of \S445 below) in which it even has 
advantages.   The construction of 436Xo shows that we can find a direct 
approach to the tensor product of linear functionals which does not 
require any attempt to measure sets rather than integrate functions. 
I trust that the prejudices I am 
expressing will not be taken as too sweeping a disparagement of 
such methods.   Practically all correct arguments in mathematics (and 
not a few incorrect ones) are valuable in some way, suggesting new 
possibilities for investigation.   In particular, one of the challenges 
of measure theory (not faced in this treatise) is that of devising 
effective theories of vector-valued measures.   Typically this is much 
easier with Riemann-type integrals, and any techniques for working 
directly with these should be noted. 
      
436L revisits ideas from Chapter 35, and the result would be easier to 
find if it were in \S356.   I include it here as an example of the way 
in which familiar material from measure theory (in particular, the 
Radon-Nikod\'ym theorem) can be drafted to serve functional analysis. 
I should perhaps remark that there are alternative routes which do not 
use measure theory explicitly, and while longer are (in my view) more 
illuminating. 
}%end of notes 
      
\discrpage 
      
      
