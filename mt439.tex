\frfilename{mt439.tex}
\versiondate{24.9.04}
\copyrightdate{2002}

\def\chaptername{Topologies and measures II}
\def\sectionname{Examples}
\def\NN{\BbbN^{\Bbb N}}

\newsection{439}

As in Chapter 41, I end this chapter with a number of examples,
exhibiting some of the boundaries around the results in the rest of the
chapter, and filling in a gap with basic facts about Lebesgue measure
(439E).   The first three examples (439A) are measures defined on
$\sigma$-subalgebras of the Borel $\sigma$-algebra of $[0,1]$ which have
no extensions to the whole Borel algebra.   The next part of the section
(439B-439G) deals with `universally negligible' sets;  I use properties
of these to show that Hausdorff measures are generally not semi-finite
(439H), closing some unfinished business from \S\S264, and that
smooth linear functionals may fail to be representable by integrals in
the absence of Stone's condition (439I).   In 439J-439R I set out some
examples relevant to \S\S434-435, filling out the classification schemes
of 434A and 435A, with spaces which just miss being Radon (439K) or
measure-compact (439N, 439P, 439Q).   In 439S I present the canonical
example of a non-Prokhorov topological space, answering an obvious
question from \S437.

\leader{439A}{Example} Let $\Cal B$ be the Borel $\sigma$-algebra of
$[0,1]$.
There is a probability measure $\nu$ defined on a $\sigma$-subalgebra
$\Tau$ of $\Cal B$ which has no extension to a measure on $\Cal B$.

\prooflet{
\medskip

\noindent{\bf first construction}
Let $A\subseteq[0,1]$ be an analytic set which is not Borel (423L).
Let $\Cal I$ be the family of sets of the form $E\cup F$ where $E$, $F$
are Borel sets, $E\subseteq A$ and $F\subseteq[0,1]\setminus A$.   Then
$\Cal I$ is a $\sigma$-ideal of $\Cal B$ not containing $[0,1]$.   Set
$\Tau=\Cal I\cup\{[0,1]\setminus H:H\in\Cal I\}$, and define
$\nu:\Tau\to\{0,1\}$ by setting $\nu H=0$, $\nu([0,1]\setminus H)=1$ for
every $H\in\Cal I$;  then $\nu$ is a probability measure (cf.\
countable-cocountable measures (211R) or Dieudonn\'e's measure (411Q)).

\Quer\ If $\mu:\Cal B\to[0,1]$ is a measure extending $\nu$, then its
completion $\hat\mu$ measures $A$ (432A).   Also $\hat\mu$ is a Radon
measure (433Cb).   Now every compact subset of $A$ belongs to $\Cal I$, so

\Centerline{$\hat\mu A
=\sup\{\hat\mu K:K\subseteq A$ is compact$\}
=\sup\{\nu K:K\subseteq A$ is compact$\}
=0$.}

\noindent Similarly $\hat\mu([0,1]\setminus A)=0$, which is absurd.\
\Bang

\medskip

\noindent{\bf second construction} This time, let $\Cal I$ be the family
of meager Borel sets in $[0,1]$.   As before, let $\Tau$ be $\Cal
I\cup\{[0,1]\setminus E:E\in\Cal I\}$, and set $\nu E=0$,
$\nu([0,1]\setminus E)=1$ for $E\in\Cal I$.   \Quer\ If $\mu$ is a Borel
measure extending $\nu$, then $\mu([0,1]\setminus\Bbb Q)=1$, and $\mu$
is tight (that is, inner regular with respect to the compact sets), so there is a closed subset $F$
of $[0,1]\setminus\Bbb Q$ such that $\mu F>0$.   But $F$ is nowhere
dense, so $\nu F=0$.\ \Bang

\medskip

\noindent{\bf third construction}\cmmnt{\footnote{I am grateful to
M.Laczkovich and D.Preiss for showing this to me.}{}}
There is a function
$f:[0,1]\to\{0,1\}^{\frak c}$ which is $(\Cal B,\CalBa)$-measurable,
where $\CalBa$ is the Baire $\sigma$-algebra of $\{0,1\}^{\frak c}$, and
such that $f[\,[0,1]\,]$ meets every non-empty member of
$\CalBa$.
\Prf\ Set $X=C([0,1])^{\Bbb N}$ with the product of the norm
topologies,
so that $X$ is an uncountable Polish space (4A2Pe, 4A2Qc), and
$([0,1],\Cal B)$ is isomorphic to $(X,\Cal B(X))$, where $\Cal B(X)$ is
the Borel $\sigma$-algebra of $X$ (424Da).   Define
$g:X\to\{0,1\}^{[0,1]}$ by saying that $g(\sequence{i}{u_i})(t)=1$ iff
$\lim_{i\to\infty}u_i(t)=1$.   For each $t\in[0,1]$,

$$\{\sequence{i}{u_i}:\lim_{i\to\infty}u_i(t)=1\}
=\bigcap_{m\in\Bbb N}\bigcup_{n\in\Bbb N}\{\sequence{i}{u_i}:
|u_i(t)-1|\le 2^{-m}\text{ for every }i\ge n\}$$

\noindent is a Borel subset of $X$, so $g$ is
$(\Cal B(X),\CalBa(\{0,1\}^{[0,1]}))$-measurable, where
$\CalBa(\{0,1\}^{[0,1]})$ is the Baire $\sigma$-algebra of
$\{0,1\}^{[0,1]}$ (4A3Ne).   If $E\in\CalBa(\{0,1\}^{[0,1]})$ is
non-empty, there is a countable set $I\subseteq[0,1]$ such that $E$ is
determined by coordinates in $I$ (4A3Nb), so that
$E\supseteq\{w:w\restr I=z\}$ for some $z\in\{0,1\}^I$.   Now we can
find a sequence $\sequence{i}{u_i}$ in $C([0,1])$ such that
$\lim_{i\to\infty}u_i(t)=z(t)$ for every $t\in I$ (if
$I\subseteq\{t_j:j\in\Bbb N\}$, take $u_i$ such that
$|u_i(t_j)-z(t_j)|\le 2^{-i}$ whenever $j\le i$), and in this case
$g(\sequence{i}{u_i})\in E$.

Because $(X,\Cal B(X))\cong([0,1],\Cal B)$ and
$(\{0,1\}^{[0,1]},\CalBa(\{0,1\}^{[0,1]}))
\cong(\{0,1\}^{\frak c},\CalBa)$, we can copy $g$ to a function $f$ with
the required properties.\ \QeD\

In particular, $f[\,[0,1]\,]$ has full outer measure for the usual measure
$\nu_{\frakc}$ on $\{0,1\}^{\frak c}$,
because $\nu_{\frakc}$ is completion regular (415E).   Setting
$\Tau=\{f^{-1}[H]:H\in\CalBa\}$, we have a measure $\nu$ with domain $\Tau$
such that $f$ is \imp\ for $\nu$ and $\nu_{\frak c}$
(234F\formerly{1{}32G}).   The map
$H^{\ssbullet}\mapsto f^{-1}[H]^{\ssbullet}$ from the measure algebra of
$\nu_{\frakc}$ to the measure algebra of $\nu$ is measure-preserving;
since
it is surely surjective, the measure algebras are isomorphic, and $\nu$
has Maharam type $\frak c$.

However, any probability measure on the whole algebra $\Cal B$ has
countable Maharam type (433A), so cannot extend $\nu$.
}%end of prooflet

\cmmnt{\medskip

\noindent{\bf Remark} Compare 433J-433K.
}%end of comment

\leader{439B}{Definition} Let $X$ be a Hausdorff space.   I will call
$X$ {\bf universally negligible} if there is no Borel probability
measure $\mu$ defined on $X$ such that $\mu\{x\}=0$ for every $x\in X$.
A subset of $X$ will be `universally negligible' if it is universally
negligible in its subspace topology.

\vleader{48pt}{439C}{Proposition} Let $X$ be a Hausdorff space.

(a) If $A$ is a subset of $X$, the following are equiveridical:

\inset{(i) $A$ is universally negligible;

(ii) $\mu^*A=0$ whenever $\mu$ is a Borel probability measure on
$X$ such that $\mu\{x\}=0$ for every $x\in X$;

(iii) $\mu^*A=0$ whenever $\mu$ is a $\sigma$-finite topological
measure on $X$ such that $\mu\{x\}=0$ for every $x\in A$;

(iv) for every $\sigma$-finite topological measure $\mu$ on $X$ there is
a countable set $B\subseteq A$ such that $\mu^*A=\mu B$;

(v) $A$ is a Radon space and every compact subset of $A$ is scattered.}

\noindent In particular, countable subsets of $X$ are universally
negligible.

(b) The family of universally
negligible subsets of $X$ is a $\sigma$-ideal.

(c) Suppose that $Y$ is a universally
negligible Hausdorff space and $f:X\to Y$ a Borel measurable function
such that $f^{-1}[\{y\}]$ is universally negligible for every $y\in Y$.
Then $X$ is universally negligible.

(d) If the topology on $X$ is discrete, $X$ is universally negligible
iff $\#(X)$ is measure-free.

\proof{{\bf (a)(i)$\Rightarrow$(iii)} If $A$ is universally
negligible and $\mu$ is a $\sigma$-finite topological measure on $X$
such that $\mu\{x\}=0$ for every $x\in A$, let $\mu_A$ be the subspace
measure on $A$.
\Quer\ If $\mu^*A=\alpha>0$, then (because $\mu$ is $\sigma$-finite)
there is a measurable set $E\subseteq X$ such that
$\gamma=\mu^*(E\cap A)$ is finite and non-zero.   The subspace measure
$\mu_{E\cap A}$ is a topological measure on $E\cap A$;  set
$\nu F=\gamma^{-1}\mu_{E\cap A}(E\cap F)$ for relatively Borel sets
$F\subseteq A$;  then $\nu$ is a
Borel probability measure on $A$ which is zero on singletons.\ \BanG\
So $\mu^*A=0$.

\medskip

\quad{\bf (iii)$\Rightarrow$(iv)} If (iii) is true and $\mu$ is a
$\sigma$-finite topological measure on $X$, set
$B=\{x:x\in X,\,\mu\{x\}>0\}$.   Because $\mu$ is $\sigma$-finite, $B$
must be countable, therefore measurable, and if we set
$\nu E=\mu(E\setminus B)$ for every Borel set $E\subseteq X$, $\nu$ is a
$\sigma$-finite Borel measure on $X$ and $\nu\{x\}=0$ for every
$x\in X$.   By (iii), $\nu^*A=0$, that is, there is a Borel set
$E\supseteq A$ such that $\mu(E\setminus B)=0$;  in which case

\Centerline{$\mu^*A
\le\mu(E\setminus(B\setminus A))=\mu(E\setminus B)+\mu(A\cap B)
=\mu(A\cap B)\le\mu^*A$,}

\noindent so $\mu^*A=\mu(A\cap B)$.   As $\mu$ is
arbitrary, (iv) is true.

\medskip

\quad{\bf (iv)$\Rightarrow$(ii)} is trivial.

\medskip

\quad{\bf not-(i)$\Rightarrow$not-(ii)} If $A$
is not universally negligible, let $\mu$ be a Borel probability measure
on $A$ which is zero on singletons.   Set $\nu E=\mu(E\cap A)$ for any
Borel set $E\subseteq X$;  then $\nu$ is a Borel probability measure on
$X$ which is zero on singletons, and $\nu^*A=1$.

\medskip

\quad{\bf (i)$\Rightarrow$(v)} Suppose that $A$ is universally
negligible.   Let $\mu$ be a totally finite Borel measure on $A$.
Applying (i)$\Rightarrow$(iv) with $X=A$, we see that there is a
countable set $B\subseteq A$ such that $\mu B=\mu A$;  but this means
that $\mu$ is inner regular with respect to the finite subsets of $B$,
which of course are compact.   As $\mu$ is arbitrary, $A$ is a Radon
space.

\Quer\ Suppose, if possible, that $A$ has a compact set $K$ which is not
scattered.   In this case there is a continuous surjection $f:K\to[0,1]$
(4A2G(j-iv)).   Now there is a Radon probability measure $\nu$ on $K$
such that $f$ is \imp\ for $\nu$ and Lebesgue measure on $[0,1]$ and
induces an isomorphism of the measure algebras, so that $\nu$ is
atomless (418L).   Accordingly we have a Borel probability measure $\mu$
on $A$
defined by setting $\mu E=\nu(K\cap E)$ for every relatively Borel set
$E\subseteq A$, and $\mu\{x\}=0$ for every $x\in A$, so $A$ is not
universally negligible.\ \BanG\  Thus all compact subsets of $A$ are
scattered, and (v) is true.

\medskip

\quad{\bf (v)$\Rightarrow$(i)} Now suppose that (v) is true and that
that $\mu$ is a Borel probability measure on $A$.   Then $\mu$ has an
extension to a Radon measure $\tilde\mu$ (434F(a-iii)).   Let
$K\subseteq A$ be a non-empty compact set which is self-supporting for
$\tilde\mu$ (416Dc).   $K$ is scattered, so has an isolated point
$\{x\}$;  because $K$ is self-supporting, $\mu\{x\}=\tilde\mu\{x\}>0$.
As $\mu$ is arbitrary, $A$ is universally negligible.

\medskip

{\bf (b)} This is immediate from (a-ii).

\medskip

{\bf (c)} Let $\mu$ be a Borel probability measure on $X$.   Then
$F\mapsto\nu f^{-1}[F]$ is a Borel probability measure on $Y$.   Because
$Y$ is universally negligible, there must be a $y\in Y$ such that
$\mu f^{-1}[\{y\}]>0$.   Set $E=f^{-1}[\{y\}]$ and let $\mu_E$ be the
subspace measure on $E$.   Then $\mu_E$ is a non-zero totally finite
Borel measure on $E$.   Since $E$ is supposed to be universally
negligible, there must be some $x\in E$ such that
$0<\mu_E\{x\}=\mu\{x\}$.

\medskip

{\bf (d)} This is just a re-phrasing of the definition in 438A.
}%end of proof of 439C


\leader{439D}{Remark\dvro{}{s (a)}}\cmmnt{ The following will be useful when
interpreting the definition in 439B.}   Let $X$ be a hereditarily
Lindel\"of Hausdorff space and $\mu$ a topological probability measure
on $X$ such that $\mu\{x\}=0$ for every $x\in X$.   Then $\mu$ is
atomless.

\prooflet{\Prf\ Suppose that $\mu H>0$.   Write

\Centerline{$\Cal G=\{G:G\subseteq X$ is open, $\mu(G\cap H)=0\}$.}

\noindent Then there is a countable $\Cal G_0\subseteq\Cal G$ such that
$\bigcup\Cal G_0=\bigcup\Cal G$ (4A2H(c-i)), so

\Centerline{$\mu(H\cap\bigcup\Cal G)=\mu(H\cap\bigcup\Cal G_0)=0$,}

\noindent and $\mu(H\setminus\bigcup\Cal G)>0$.   Because $\mu$ is zero
on singletons, $H\setminus\bigcup\Cal G$ has at least two points $x$,
$y$ say.   Now there are disjoint open sets $G_0$, $G_1$ containing $x$,
$y$ respectively, and neither belongs to $\Cal G$, so $H\cap G_0$,
$H\cap G_1$ are disjoint subsets of $H$ of positive measure.   Thus $H$
is not an atom.   As $H$ is arbitrary, $\mu$ is atomless.\ \Qed
}%end of prooflet

\cmmnt{\spheader 439Db The obvious applications of (a) are when $X$
is separable and metrizable;  but, more generally, we can use it on any
Hausdorff space with a countable network, e.g., on any analytic space.
}%end of comment

\leader{439E}{Lemma} (a) Let $E$, $B\subseteq\Bbb R$ be such that $E$ is
measurable and $\mu_LE$, $\mu_L^*B$ are both greater than $0$, where
$\mu_L$ is Lebesgue measure.   Then $E-B=\{x-y:x\in E,\,y\in B\}$
includes a non-trivial interval.

(b) If $A\subseteq\Bbb R$ and $\mu_L^*A>0$, then $A+\Bbb Q$ is
of full outer measure in $\Bbb R$.

\proof{{\bf (a)} By 223B or 261Da, there are $a\in E$, $b\in B$ such that

$$\eqalign{\lim_{\delta\downarrow 0}\Bover1{2\delta}
  \mu(E\cap[a-\delta,a+\delta])
=\lim_{\delta\downarrow 0}\Bover1{2\delta}
  \mu^*(B\cap[b-\delta,b+\delta])
=1.\cr}$$

\noindent Let $\gamma>0$ be such that

\Centerline{$\mu_L(E\cap[a-\delta,a+\delta])>\Bover32\delta$,
\quad$\mu_L^*(B\cap[b-\delta,b+\delta])>\Bover32\delta$}

\noindent whenever $0<\delta\le\gamma$.   Now suppose that
$0<\delta\le\gamma$.   Then

$$\eqalign{\mu_L((E+b)\cap[a+b,a+b+\delta])
&=\mu_L(E\cap[a,a+\delta])\cr
&\ge\mu_L(E\cap[a-\delta,a+\delta])-\delta
>\Bover12\delta,\cr}$$

\noindent and similarly

$$\eqalign{\mu_L^*((B+a+\delta)\cap[a+b,a+b+\delta])
&=\mu_L^*(B\cap[b-\delta,b])\cr
&\ge\mu_L^*(B\cap[b-\delta,b+\delta])-\delta
>\Bover12\delta.\cr}$$

\noindent But this means that
$(E+b)\cap(B+a+\delta)$ cannot be empty.   If
$u\in(E+b)\cap(B+a+\delta)$, then $u-b\in E$, $u-a-\delta\in B$ so

\Centerline{$a-b+\delta=(u-b)-(u-a-\delta)\in E-B$.}

As $\delta$ is arbitrary, $E-B$ includes the interval
$\ocint{a-b,a-b+\gamma}$.

\medskip

{\bf (b)} \Quer\ Suppose, if possible, otherwise;  that there is a
measurable set $E\subseteq\Bbb R$ such that $\mu_LE>0$ and
$E\cap(A+\Bbb Q)=\emptyset$.   Then $E-A$ does not meet $\Bbb Q$ and
cannot include any non-trivial interval.\ \Bang
}%end of proof of 439E

\cmmnt{\medskip

\noindent{\bf Remark} There will be a dramatic generalization of (a) in
443Db.
}%end of comment

\leader{439F}{Proposition} Let $\kappa$ be the least cardinal of any set of
non-zero Lebesgue outer measure in $\Bbb R$.

(a) There is a set $X\subseteq[0,1]$ of cardinal $\kappa$ and full outer
Lebesgue measure.

(b) If $(Z,\Tau,\nu)$ is any atomless complete locally determined
measure space and $A\subseteq Z$ has cardinal less than $\kappa$, then
$\nu^*A=0$.

(c)\cmmnt{ ({\smc Grzegorek 81})} There is a universally negligible
set $Y\subseteq[0,1]$ of cardinal $\kappa$.

\proof{{\bf (a)} Take any set $A\subseteq\Bbb R$ such that
$\#(A)=\kappa$ and $\mu_L^*A>0$, where $\mu_L$ is Lebesgue measure.
Set $B=A+\Bbb Q$.   Then $(\mu_L)_*(\Bbb R\setminus B)=0$, by 439Eb.
Set $X=[0,1]\cap B$;  then $\mu_L^*X=1$ while $\#(X)\le\#(B)=\kappa$.
By the definition of $\kappa$, $\#(X)$ must be exactly $\kappa$.

\medskip

{\bf (b)} \Quer\ Otherwise, by 412Jc, there is a set $F\subseteq Z$ such
that $\nu F<\infty$ and $\nu^*(F\cap A)>0$.   By 343Cc, there is
a function $f:F\to[0,\nu F]$ which is \imp\ for
the subspace measure $\nu_F$ and Lebesgue measure on
$[0,\nu F]$.   But $f[A\cap F]$ has cardinal less than $\kappa$, so
$\mu_Lf[A\cap F]=0$ and

\Centerline{$0<\nu^*(A\cap F)\le\nu f^{-1}[f[A\cap F]]=0$,}

\noindent which is absurd.\ \Bang

\medskip

{\bf (c)(i)} Enumerate $X$ as $\langle x_{\xi}\rangle_{\xi<\kappa}$.
For each $\xi<\kappa$, $A_{\xi}=\{x_{\eta}:\eta\le\xi\}$ has cardinal
less than $\kappa$, so is Lebesgue negligible;  let
$\sequencen{I_{\xi n}}$ be a sequence of intervals covering $A_{\xi}$ with $\sum_{n=0}^{\infty}\mu_LI_{\xi n}<\bover12$.   Enlarging the intervals
slightly if necessary, we may suppose that every $I_{\xi n}$ has
rational endpoints;  let $\sequence{m}{J_m}$ enumerate the family of
intervals in $\Bbb R$ with rational endpoints.

Set

\Centerline{$C_{mn}=\{\xi:\xi<\kappa,\,I_{\xi n}=J_m\}$}

\noindent for each $m$, $n\in\Bbb N$.

\medskip

\quad{\bf (ii)} If $\nu$ is an atomless totally finite measure on
$\kappa$ which measures every $C_{mn}$, then $\nu\kappa=0$.   \Prf\ Note
first that (by (b), applied to the completion of $\nu$) $\nu^*\xi=0$ for
every $\xi<\kappa$.   Let $\lambda$
be the (c.l.d.) product of $\mu_X$, the subspace measure on $X$,
with $\nu$.   Set

\Centerline{$B=\bigcup_{m,n\in\Bbb N}((X\cap J_m)\times C_{mn})
\subseteq X\times\kappa$.}

\noindent Then $B$ is measured by $\lambda$, so, by Fubini's theorem,

\Centerline{$\int\nu B[\{x\}]\mu_X(dx)
=\int\mu_X B^{-1}[\{\xi\}]\nu(d\xi)$}

\noindent (252D).

Now look at the sectional measures $\nu B[\{x\}]$,
$\mu_XB^{-1}[\{\xi\}]$.   (Because $B$ is actually a countable union of
measurable rectangles, these are always defined.)   For any $x\in X$,
there is an $\eta<\kappa$ such that $x=x_{\eta}$, and now

$$\eqalign{B[\{x\}]
&=\{\xi:\text{ there are }m,\,n\in\Bbb N
  \text{ such that }x\in J_m\text{ and }\xi\in C_{mn}\}\cr
&=\{\xi:\text{ there are }m,\,n\in\Bbb N
  \text{ such that }x\in J_m\text{ and }I_{\xi n}=J_m\}\cr
&=\{\xi:\text{ there is an }n\in\Bbb N
  \text{ such that }x\in I_{\xi n}\}
\supseteq\kappa\setminus\eta\cr}$$

\noindent by the choice of the $I_{\xi n}$.   But as $\nu^*\eta=0$, this
means that $\nu B[\{x\}]=\nu\kappa$.

On the other hand, if $\xi<\kappa$, then

$$\eqalign{B^{-1}[\{\xi\}]
&=\{x:\text{ there are }m,\,n\in\Bbb N\text{ such that }x\in J_m
  \text{ and }\xi\in C_{mn}\}\cr
&=\{x:\text{ there are }m,\,n\in\Bbb N\text{ such that }x\in J_m
  \text{ and }I_{\xi n}=J_m\}\cr
&=\{x:\text{ there is an }n\in\Bbb N
  \text{ such that }x\in I_{\xi n}\}
=X\cap\bigcup_{n\in\Bbb N}I_{\xi n},\cr}$$

\noindent so that

\Centerline{$\mu_XB^{-1}[\{\xi\}]
\le\sum_{n=0}^{\infty}\mu_LI_{\xi n}\le\Bover12$.}

Returning to the integrals, we have

\Centerline{$\nu\kappa
=\int\nu B[\{x\}]\mu_X(dx)
=\int\mu_X B^{-1}[\{\xi\}]\nu(d\xi)
\le\Bover12\nu\kappa$,}

\noindent so that $\nu\kappa$ must be $0$, as claimed.\ \Qed

\medskip

\quad{\bf (iii)} Now there is an injective function $g:\kappa\to[0,1]$
such that $g[C_{mn}]$ is relatively Borel in $g[\kappa]$ for every $m$,
$n\in\Bbb N$.   \Prf\ Define
$h:\kappa\to\{0,1\}^{\Bbb N\times\Bbb N\times\Bbb N}$ by setting

$$\eqalign{h(\xi)(m,n,k)&=1\text{ if }\xi\in C_{mn}
  \text{ and }x_{\xi}\in J_k,\cr
&=0\text{ otherwise}.\cr}$$

\noindent Then $h$ is injective (because if $\xi\ne\eta$ then
$x_{\xi}\ne x_{\eta}$, so there is some $k$ such that $x_{\xi}\in J_k$
and $x_{\eta}\notin J_k$), and

\Centerline{$h[C_{mn}]=h[\kappa]\cap\{w:\text{ there is some }k
  \text{ such that }w(m,n,k)=1\}$}

\noindent is relatively Borel in $h[\kappa]$ for every $m$,
$n\in\Bbb N$.   But now recall that
$\{0,1\}^{\Bbb N\times\Bbb N\times\Bbb N}\cong\{0,1\}^{\Bbb N}$ is
homeomorphic to the Cantor set $C\subseteq[0,1]$ (4A2Uc).   If
$\phi:\{0,1\}^{\Bbb N\times\Bbb N\times\Bbb N}\to C$ is any
homeomorphism,
then $\phi h$ has the required properties.\ \Qed

\medskip

\quad{\bf (iv)} Set $Y=g[\kappa]$.   Because $g$ is injective,
$\#(Y)=\kappa$.   Also $Y$ is universally negligible.   \Prf\ Suppose
that $\tilde\nu$ is a Borel measure on $Y$ which is zero on singletons.
Then it is atomless, because $Y$ is separable and metrizable (439D).
So its copy $\nu=\tilde\nu(g^{-1})^{-1}$ on $\kappa$ is atomless.
Because $g[C_{mn}]$ is a Borel subset of $Y$, $\nu$ measures $C_{mn}$
for all $m$, $n\in\Bbb N$, so $\tilde\nu Y=\nu\kappa=0$, by (ii)
above.\ \Qed
}%end of proof of 439F

\leader{439G}{Corollary} A metrizable continuous image of a universally
negligible metrizable space need not be universally negligible.

\proof{ Take $X$ and $Y$ from 439Fa and 439Fc above, and let $f:X\to Y$
be any bijection.   Let $\Gamma$ be the graph of $f$.   The projection
map $(x,y)\mapsto y:\Gamma\to Y$ is continuous and injective, so
$\Gamma$ is universally negligible, by 439Cc.   On the other hand, the
projection map $(x,y)\mapsto x:\Gamma\to X$ is continuous and
surjective, and $X$ is
surely not universally negligible, since it is not Lebesgue negligible.
}%end of proof of 439G

\leader{439H}{Corollary} One-dimensional Hausdorff measure on $\BbbR^2$
is not semi-finite.

\proof{ Let $\mu_{H1}$ be one-dimensional Hausdorff measure on
$\BbbR^2$.    Let $X$, $\Gamma$ be the sets described in 439F and the
proof of 439G.

\medskip

{\bf (a)} $\mu_{H1}^*\Gamma>0$.   \Prf\ The first-coordinate map
$\pi_1:\BbbR^2\to\Bbb R$ is 1-Lipschitz, so, writing $\mu_L$ for
Lebesgue measure on $\Bbb R$,

\Centerline{$1=\mu_L^*X=\mu_L^*\pi_1[\Gamma]\le\mu_{H1}^*\Gamma$}

\noindent by 264G/471J and 264I.\ \Qed

\medskip

{\bf (b)} If $E\subseteq\BbbR^2$ and $\mu_{H1}E<\infty$, then
$\mu_{H1}(E\cap\Gamma)=0$, because $E\cap\Gamma$ is universally
negligible (439Cb) and $\mu_{H1}$ is a topological measure (264E/471Da)
which is zero on singletons.

\medskip

{\bf (c)} \Quer\ Suppose, if possible, that $\Gamma$ is not measured by
$\mu_{H1}$.   Then there is a set $A\subseteq\BbbR^2$ such that
$\mu^*_{H1}A<\mu^*_{H1}(A\cap\Gamma)+\mu^*_{H1}(A\setminus\Gamma)$
(264C/471A, 264Fb/471Dc).   Let $E$ be a Borel set including $A$ such that
$\mu_{H1}E=\mu_{H1}^*A$ (264Fa/471Db);  then $\mu_{H1}(E\cap\Gamma)=0$, so

\Centerline{$\mu^*_{H1}(A\cap\Gamma)+\mu^*_{H1}(A\setminus\Gamma)
\le\mu_{H1}(E\cap\Gamma)+\mu_{H1}^*A=\mu^*_{H1}A$.  \Bang}

\medskip

{\bf (d)} Since $\Gamma$ is measurable, not negligible, and meets every
measurable set of finite measure in a negligible set, it is purely
infinite, and $\mu_{H1}$ is not semi-finite.
}%end of proof of 439H

\leader{439I}{Example} There are a set $X$, a Riesz subspace $U$ of
$\Bbb R^X$ and a smooth positive linear functional $h:U\to\Bbb R$ which
is not expressible as an integral.

\proof{ Take $X$ and $Y$ from 439F.   Replacing $Y$ by $Y\setminus\{0\}$
if need be, we may suppose that $0\notin Y$.   Let $f:X\to Y$ be any
bijection.

Let $U$ be the Riesz subspace
$\{u\times f:u\in C_b\}\subseteq\Bbb R^X$, where $C_b$ is the space of
bounded continuous functions from $X$ to $\Bbb R$.   Because $f$ is
strictly positive, $u\mapsto u\times f:C_b\to U$ is a bijection,
therefore a Riesz space isomorphism;
moreover, for a non-empty set $A\subseteq C_b$, $\inf_{u\in A}u(x)=0$
for every $x\in X$ iff $\inf_{u\in A}u(x)f(x)=0$ for every $x\in X$.
We therefore have a smooth linear functional $h:U\to\Bbb R$ defined by
setting $h(u\times f)=\int u\,d\mu_X$ for every $u\in C_b$, where
$\mu_X$ is the subspace measure on $X$ induced by Lebesgue measure.
(By 415B, $\mu_X$ is quasi-Radon, so the integral it defines on $C_b$ is
smooth, as noted in 436H.)

\Quer\ But suppose, if possible, that $h$ is the integral with respect
to some measure $\nu$ on $X$.   Since $f\in U$, it must be
$\Tau$-measurable, where $\Tau$ is the domain of the completion
$\hat\nu$ of $\nu$.   Note that $\hat\nu\{x\}=0$ for every $x\in X$.
\Prf\ Set $u_n(y)=\max(0,1-2^n|y-x|)$ for $y\in X$.   Then

$$\eqalign{f(x)\hat\nu\{x\}
&=\lim_{n\to\infty}\int u_n\times f\,d\nu
=\lim_{n\to\infty}h(u_n\times f)\cr
&=\lim_{n\to\infty}\int u_n\,d\mu_X
=\mu_X\{x\}
=0,\cr}$$

\noindent so $\hat\nu\{x\}=0$.\ \Qed

For Borel sets $E\subseteq[0,1]$ set
$\lambda E=\hat\nu f^{-1}[E]$.   Then the completion $\hat\lambda$ of
$\lambda$ is a Radon measure on $[0,1]$ (433Cb or 256C).   If $t\in[0,1]$
then $f^{-1}[\{t\}]$ contains at most one point, so
$\hat\lambda\{t\}=\lambda\{t\}=0$.   But $Y$ is supposed to be
universally negligible, so $\lambda^*Y=\hat\lambda^*Y=0$ (439Ca), that
is, there is a Borel set $E\supseteq Y$ with $\lambda E=0$;  in which
case $\nu X=\hat\nu f^{-1}[E]=0$, which is impossible.\ \Bang

Thus $h$ is not an integral, despite being a smooth linear functional on
a Riesz subspace of $\Bbb R^X$.
}%end of proof of 439I

\cmmnt{\medskip

\noindent{\bf Remark} This example is adapted from
{\smc Fremlin \& Talagrand 78}.
}%end of comment

\leader{439J}{Example} Assume that there is some cardinal $\kappa$ which
is not measure-free.   Give $\kappa$ its discrete topology, and let
$\mu$ be a probability measure with domain $\Cal P\kappa$ such that
$\mu\{\xi\}=0$ for every $\xi<\kappa$.   Now every subset of $\kappa$ is
open-and-closed, so $\mu$ is simultaneously a Baire probability measure
and a completion regular Borel probability measure.    Of course it is
not $\tau$-additive.   \cmmnt{In the classification schemes of 434A
and 435A, we have a measure which is of type B$_1$ as a Borel measure
and type E$_3$ as a Baire measure.}
%4{}35A (type E_3) %434A (type B_1)

\leader{439K}{Example} There is a first-countable compact Hausdorff
space which is not Radon.

\proof{ The construction starts from a compact metrizable space
$(Z,\frak S)$ with an atomless Radon probability measure $\mu$.   The
obvious candidate is $[0,1]$ with Lebesgue measure;  but for technical
convenience in a later application I will instead use
$Z=\{0,1\}^{\Bbb N}$ with its usual product topology and measure (254J).

\medskip

{\bf (a)} There is a topology $\frak T_{\frak c}$ on $Z$ such that

\quad($\alpha$) $\frak S\subseteq\frak T_{\frak c}$;

\quad($\beta$) every point of $Z$ belongs to a countable set which is
compact and open for $\frak T_{\frak c}$;

\quad($\gamma$) if $\sequencen{F_n}$ is a sequence of
$\frak T_{\frak c}$-closed sets with empty intersection, then
$\bigcap_{n\in\Bbb N}\overline{F}_n^{\frak S}$ is countable, where I
write $\overline{F}^{\frak S}$ for the $\frak S$-closure of $F$.

\medskip

\Prf{\bf (i)} $\frak T_{\frak c}$ will be the last in a family
$\langle\frak T_{\xi}\rangle_{\xi\le\frak c}$ of topologies.   We must
begin by enumerating $Z$ as
$\langle z_{\xi}\rangle_{\xi<\frak c}$
and taking a family $\langle\sequencen{I_{\xi n}}\rangle_{\xi<\frak c}$
running over $([Z]^{\le\omega})^{\Bbb N}$ with cofinal repetitions,
where $[Z]^{\le\omega}$ is the family of countable subsets of
$Z$.   (This can be done because
$\#([Z]^{\le\omega})=\frak c$, by 2A1Hb.)
Together with $\langle\frak T_{\xi}\rangle_{\xi\le\frak c}$ we choose
simultaneously families $\langle x_{\xi}\rangle_{\xi<\frak c}$,
$\langle y_{\xi}\rangle_{\xi<\frak c}$ of points in $Z$, and the
inductive hypothesis will be

\inset{$\frak T_{\xi}$ is a topology on
$X_{\xi}=\{x_{\eta}:\eta<\xi\}\cup\{y_{\eta}:\eta<\xi\}$ finer than the
topology on $X_{\xi}$ induced by $\frak S$;}

\inset{if $\eta<\xi\le\frak c$, then $X_{\eta}\in\frak T_{\xi}$ and
$\frak T_{\eta}$ is the subspace topology on $X_{\eta}$ induced by
$\frak T_{\xi}$;}

\inset{every point of $X_{\xi}$ belongs to a countable set which is
compact and open for $\frak T_{\xi}$.}

\noindent The induction starts with $X_0=\emptyset$,
$\frak T_0=\{\emptyset\}$.

\medskip

\quad{\bf (ii)} {\it Inductive step to a successor ordinal} Suppose that
we have found $X_{\xi}$ and $\frak T_{\xi}$ where $\xi<\frak c$.

\medskip

\qquad\grheada\ Start by picking $y_{\xi}\in Z\setminus X_{\xi}$ such
that $y_{\xi}=z_{\xi}$ if $z_{\xi}\notin X_{\xi}$.
Examine the sequence $\sequencen{I_{\xi n}}$.   If {\it either}
$\bigcup_{n\in\Bbb N}I_{\xi n}\not\subseteq X_{\xi}$ {\it or}
$\bigcap_{n\in\Bbb N}\overline{I}_{\xi n}^{\frak S}$ is countable, take
$x_{\xi}$ to be any point of $Z\setminus(X_{\xi}\cup\{y_{\xi}\})$
and set $K_m=\emptyset$ for every $m$ before proceeding to ($\gamma$)
below.

\medskip

\qquad\grheadb\ If $I_{\xi n}\subseteq X_{\xi}$ for every $n$ and
$\bigcap_{n\in\Bbb N}\overline{I}_{\xi n}^{\frak S}$ is uncountable, it
must have cardinal $\frak c$, by 423K, so cannot be included in
$X_{\xi}\cup\{y_{\xi}\}$.    Take any
$x_{\xi}\in\bigcap_{n\in\Bbb N}\overline{I}_{\xi n}^{\frak S}
\setminus(X_{\xi}\cup\{y_{\xi}\})$.   Let $\sequence{m}{t_m}$ be a
sequence in $Z$ such that $t_m\restr m=x_{\xi}\restr m$ for every
$m\in\Bbb N$ and $t_m\in I_{\xi n}$ whenever $r\in\Bbb N$, $n\le 2r$ and
$m=r^2+n$.   (Thus, for each $n$, $t_m\in I_{\xi n}$ for infinitely many
$m$, while $\sequence{m}{t_m}\to x_{\xi}$ in the ordinary sense.)   By
the inductive hypothesis, we can find countable sets
$K_m\subseteq X_{\xi}$,
compact and open for $\frak T_{\xi}$, such that $t_m\in K_m$ for each
$m$.   Because $\{t:t\in X_{\xi},\,t\restr m=x_{\xi}\restr m\}$ is
open-and-closed for $\frak T_{\xi}$ and contains $t_m$, we may suppose
that $t\restr m=x_{\xi}\restr m$ for every $t\in K_m$.

\medskip

\qquad\grheadc\ Let $\frak T_{\xi+1}$ be the topology on
$X_{\xi+1}=X_{\xi}\cup\{x_{\xi},y_{\xi}\}$ generated by

\Centerline{$\frak T_{\xi}\cup\{\{y_{\xi}\}\}\cup\{L_n:n\in\Bbb N\}$,}

\noindent where $L_n=\{x_{\xi}\}\cup\bigcup_{m\ge n}K_m$ for each $n$.

\medskip

\qquad\grheadd\ Because $\frak T_{\xi}\subseteq\frak T_{\xi+1}$,
$X_{\xi}$ will be open in $X_{\xi+1}$.   Because the $K_m$ are always
$\frak T_{\xi}$-open, and $x_{\xi}$, $y_{\xi}$ are distinct points of
$Z\setminus X_{\xi}$, the topology on $X_{\xi}$ induced by
$\frak T_{\xi+1}$ is just $\frak T_{\xi}$.   Consequently (by the
inductive
hypothesis) the topology on $X_{\eta}$ induced by $\frak T_{\xi+1}$ is
$\frak T_{\eta}$ for every $\eta\le\xi$.   We have
$t\restr n=x_{\xi}\restr n$ for every $t\in L_n$, so $\frak T_{\xi+1}$
is finer than the usual topology on $X_{\xi+1}$.

If $x\in X_{\xi}$, then there is a countable $\frak T_{\xi}$-open
$\frak T_{\xi}$-compact set containing $x$, which is still
$\frak T_{\xi+1}$-open and $\frak T_{\xi+1}$-compact.   Of course
$\{y_{\xi}\}$
is a countable $\frak T_{\xi+1}$-open $\frak T_{\xi+1}$-compact set
containing $y_{\xi}$.   As for $x_{\xi}$, $L_0$ is surely countable and
$\frak T_{\xi+1}$-open.   To see that it is $\frak T_{\xi+1}$-compact,
observe that any ultrafilter containing $L_0$ either contains every
$L_n$, and converges to $x_{\xi}$, or contains some $K_m$ and converges
to a point of $K_m$.

Thus the induction proceeds at successor stages.

\medskip

\quad{\bf (iii)} {\it Inductive step to a limit ordinal} If
$\xi\le\frak c$ is a non-zero limit ordinal, then we have
$X_{\xi}=\bigcup_{\eta<\xi}X_{\eta}$, and can take $\frak T_{\xi}$ to be
the topology generated by $\bigcup_{\eta<\xi}\frak T_{\eta}$.   It is
easy to check that this works (because the topologies $\frak T_{\eta}$
are consistent with each other).

\medskip

\quad{\bf (iv)} At the end of the induction, we have $X_{\frak c}=Z$
because $z_{\xi}\in X_{\xi+1}\subseteq X_{\frak c}$ for every $\xi$.
The final topology $\frak T_{\frak c}$ on $Z$ will have the properties
($\alpha$) and ($\beta$) required.   \Quer\ Now suppose, if possible,
that $\sequencen{F_n}$ is a sequence of $\frak T_{\frak c}$-closed sets
with empty intersection, and that
$\bigcap_{n\in\Bbb N}\overline{F}^{\frak S}_n$ is uncountable.   For
each $n\in\Bbb N$, let
$J_n\subseteq F_n$ be a countable $\frak S$-dense set.   Then there is
some $\zeta<\frak c$ such that
$\bigcup_{n\in\Bbb N}J_n\subseteq X_{\zeta}$ (because
$\cf\frak c>\omega$, see 4A1A(c-iii)).
Let $\xi\ge\zeta$ be such that $J_n=I_{\xi n}$ for every $n\in\Bbb N$.
Then in the construction of $\frak T_{\xi+1}$ we must be in case
($\beta$) of (ii) above.   Taking
$\sequence{m}{t_m}$ as described there, we have
$\sequence{m}{t_m}\to x_{\xi}$ for $\frak T_{\xi+1}$, and therefore for
$\frak T_{\frak c}$.
But for any $n\in\Bbb N$, $t_m\in J_n\subseteq F_n$ for infinitely many
$m$, so $x_{\xi}\in F_n$.   Thus $x_{\xi}\in\bigcap_{n\in\Bbb N}F_n$;
but this is impossible.\ \Bang

So we have a topology of the type required.\ \Qed

\medskip

{\bf (b)} There is a probability measure $\nu$ on $Z$, extending the
usual measure $\mu$, such that with respect to
$\frak T_{\frak c}\,\,\nu$ is a topological measure inner regular with
respect to the closed sets, but is not $\tau$-additive.

\medskip

\Prf\ Let $\Cal K$ be the family of $\frak T_{\frak c}$-closed
subsets of $Z$.   For $F\in\Cal K$, set
%$\phi F=\mu^*F$.
$\phi F=\mu\overline{F}^{\frak S}$.

\medskip

\quad{\bf (i)} If $E$, $F$ are disjoint $\frak T_{\frak c}$-closed
sets, then $\overline{E}^{\frak S}\cap\overline{F}^{\frak S}$ must be
countable (take $F_{2n}=E$, $F_{2n+1}=F$ in (a-$\gamma$)).   So

$$\eqalign{\phi(E\cup F)
%=\mu^*(E\cup F)
%&=\mu^*((E\cup F)\cap\overline{E}^{\frak S})
%  +\mu^*((E\cup F)\setminus\overline{E}^{\frak S})\cr
%&=\mu^*E+\mu^*F
&=\mu\overline{E\cup F}^{\frak S}
=\mu\overline{E}^{\frak S}+\mu\overline{F}^{\frak S}
   -\mu(\overline{E}^{\frak S}\cap\overline{F}^{\frak S})\cr
&=\mu\overline{E}^{\frak S}+\mu\overline{F}^{\frak S}
=\phi E+\phi F.\cr}$$

\medskip

\quad{\bf (ii)} If $E$, $F\in\Cal K$, $E\subseteq F$ and $\epsilon>0$,
there is an $\frak S$-open set
%$G\supseteq E$
$G\supseteq\overline{E}^{\frak S}$
such that
%$\mu G\le\mu^*E+\epsilon$.
$\mu G\le\mu\overline{E}^{\frak S}+\epsilon$.
Now $F\setminus G\in\Cal K$ and

\Centerline{$\phi F
%=\mu^*(F\cap G)+\mu^*(F\setminus G)
=\mu(\overline{F}^{\frak S}\cap G)+\mu(\overline{F}^{\frak S}\setminus G)
\le\mu G+\phi(F\setminus G)
\le\phi E+\phi(F\setminus G)+\epsilon$.}

\noindent Putting this together with (i), we see that

\Centerline{$\phi F
=\phi E+\sup\{\phi E':E'\in\Cal K,\,E'\subseteq F\setminus E\}$.}

\medskip

\quad{\bf (iii)} If $\sequencen{F_n}$ is a non-increasing sequence in
$\Cal K$ with empty intersection, then

\Centerline{$\lim_{n\to\infty}\phi F_n
%\le\lim_{n\to\infty}\mu\overline{F}_n^{\frak S}
=\lim_{n\to\infty}\mu\overline{F}_n^{\frak S}
=\mu(\bigcap_{n\in\Bbb N}\overline{F}_n^{\frak S})
=0$.}

\medskip

\quad{\bf (iv)} Thus $\Cal K$ and $\phi$ satisfy all the conditions of
413I, and there is a measure $\nu$, extending $\phi$, which is defined
on every member of $\Cal K$ and inner regular with respect to $\Cal K$,
and therefore is (for $\frak T_{\frak c}$) a topological measure inner
regular with respect to the closed sets.

If we write $\Cal V$ for the family of $\frak T_{\frak c}$-compact
$\frak T_{\frak c}$-open countable subsets of $Z$, then for any
$K\in\Cal V$

\Centerline{$\nu K=\phi K=\mu K=0$,}

\noindent while $\Cal V$ is upwards-directed and has union $Z$;  so that
$\nu$ is not $\tau$-additive.\ \Qed

\medskip

{\bf (c)} So far we seem to have very little more than is provided by
$\omega_1$ with the order topology and Dieudonn\'e's measure.   The
point of doing all this work is the next step.   Set
$X=Z\times\{0,1\}$ and give $X$ the topology $\frak T$ generated by

\Centerline{$\{G\times\{0,1\}:G\in\frak S\}
\cup\{H\times\{1\}:H\in\frak T_{\frak c}\}
\cup\{X\setminus(K\times\{1\}):K$ is $\frak T_{\frak c}$-compact$\}$.}

\medskip

\quad{\bf (i)} $\frak T$ is Hausdorff.   \Prf\ If $w$, $z$ are distinct
points of $X$, then either their first coordinates differ and they are
separated by sets of the form $G_0\times\{0,1\}$, $G_1\times\{0,1\}$
where $G_0$, $G_1$ belong to $\frak S$, or they are of the form $(x,1)$,
$(x,0)$ and are separated by open sets of the form $K\times\{1\}$,
$X\setminus(K\times\{1\})$ for some set $K$ which is compact and open
for $\frak T_{\frak c}$.\ \Qed

\medskip

\quad{\bf (ii)} $\frak T$ is compact.   \Prf\ Let $\Cal F$ be an
ultrafilter on $X$.   Writing $\pi_1(x,0)=\pi_1(x,1)=x$ for $x\in Z$,
$\pi_1[[\Cal F]]$ is $\frak S$-convergent, to $x_0$ say.   If
$K\times\{1\}\in\Cal F$ for some $\frak T_{\frak c}$-compact set $K$,
then $\Cal F$ is $\frak T$-convergent to $(x,1)$;  otherwise, it is
$\frak T$-convergent to $(x,0)$ (using 4A2B(a-iv)).\ \Qed
%characterizing convergent filters given a base

\medskip

\quad{\bf (iii)} $\frak T$ is first-countable.   \Prf\ If $x\in Z$,
then $\{(x,0),(x,1)\}=\pi_1^{-1}[\{x\}]$ is a G$_{\delta}$ set in $X$
because $\{x\}$ is a G$_{\delta}$ set in $Z$ and $\pi_1$ is
continuous (4A2C(a-iii)).   Now $\{(x,0)\}$ and $\{(x,1)\}$ are
relatively open in $\{(x,0),(x,1)\}$, so are G$_{\delta}$ sets in $X$
(4A2C(a-iv)).   Thus singletons are G$_\delta$ sets.   Because $\frak T$
is compact and Hausdorff, it is first-countable (4A2Kf).\ \Qed

\medskip

\quad{\bf (iv)} $(X,\frak T)$ is not a Radon space.   \Prf\
$Z\times\{1\}$ is an open subset of $X$, homeomorphic to $Z$
with the topology $\frak T_{\frak c}$.   But the measure $\nu$ of (b)
above (or, if you prefer, its restriction to the
$\frak T_{\frak c}$-Borel
algebra) witnesses that $\frak T_{\frak c}$ is not a Radon topology, so
$\frak T$ also cannot be a Radon topology, by 434Fc.\ \Qed
}%end of proof of 439K

\cmmnt{\medskip

\noindent{\bf Remark} Aficionados will recognise $\frak T_{\frak c}$ as
a kind of `JKR-space', derived from the construction in
{\smc Juh\'asz Kunen \& Rudin 76}.
}%end of comment

\leader{439L}{Example} Suppose that $\kappa$ is a cardinal which is not
measure-free;  let $\mu$ be a probability measure with domain
$\Cal P\kappa$ which is zero on singletons.   Give $\kappa$ its discrete
topology\cmmnt{, so that $\mu$ is a Borel measure and $\kappa$ is
first-countable}.   Let $\nu$ be the restriction of the usual measure on
$Y=\{0,1\}^{\kappa}$ to the algebra $\Cal B$ of Borel subsets of $Y$, so
that $\nu$ is a $\tau$-additive probability measure,
and $\lambda$ the product measure on $\kappa\times Y$ constructed by the
method of 434R.   Then

\Centerline{$W=\{(\xi,y):\xi<\kappa$, $y(\xi)=1\}
=\bigcup_{\xi<\kappa}\{\xi\}\times\{y:y(\xi)=1\}$}

\noindent is open in $\kappa\times Y$.

If $W'\in\Cal P\kappa\tensorhat\Cal B$ then
$\lambda(W\symmdiff W')=\bover12$.   \prooflet{\Prf\  There is
a countable set $\Cal E\subseteq\Cal B$ such that $W'$ belongs to the
$\sigma$-algebra
generated by $\{A\times E:A\subseteq\kappa$, $E\in\Cal E\}$ (331Gd).
For $J\subseteq\kappa$, write $\pi_J(y)=y\restr J$ for $y\in Y$, let
$\nu_J$ be the usual measure on $\{0,1\}^J$ and $\Tau_J$ its domain,
and let $\Tau'_J$ be the family of sets
$E\subseteq Y$ such that there are $H$, $H'\in\Tau_J$ such that
$\pi_J^{-1}[H]\subseteq E\subseteq\pi_J^{-1}[H']$ and
$\nu_J(H'\setminus H)=0$.   Then $\Tau'_J\subseteq\Tau'_K$
whenever
$J\subseteq K\subseteq\kappa$, and every set measured by $\nu$ belongs
to $\Tau'_J$ for some countable $J$ (254Ob).   There is therefore a
countable set $J\subseteq\kappa$ such that $\Cal E\subseteq\Tau'_J$.
Also, of course, $\Tau'_J$ is a $\sigma$-algebra of subsets of $Y$.

The set

\Centerline{$\{V:V\subseteq\kappa\times Y,\,V[\{\xi\}]\in\Tau'_J$ for
every $\xi<\kappa\}$}

\noindent is a $\sigma$-algebra of subsets of $\kappa\times Y$
containing $A\times E$ whenever $A\subseteq\kappa$ and $E\in\Cal E$, so
contains $W'$.   But this means that if $\xi\in\kappa\setminus J$,
$W[\{\xi\}]$ and $W'[\{\xi\}]$ are stochastically independent, and
$\nu(W[\{\xi\}]\symmdiff W'[\{\xi\}])=\bover12$.   Since
$\mu(\kappa\setminus J)=1$,

\Centerline{$\lambda(W\symmdiff W')
=\int\nu(W[\{\xi\}]\symmdiff W'[\{\xi\}])\mu(d\xi)
=\Bover12$,}

\noindent as claimed.\ \Qed
}%end of prooflet

In particular, $W^{\ssbullet}$ in the the measure algebra of $\lambda$
cannot be represented by a member of $\Cal P\kappa\tensorhat\Cal B$.

\leader{439M}{Example} There is a first-countable locally compact
Hausdorff space $X$ with a
Baire probability measure $\mu$ which is not $\tau$-additive and
has no extension to a Borel
measure.   \cmmnt{In the classification of 435A, $\mu$ is of type
E$_0$.}

\proof{ Let $\Omega$ be the set of non-zero countable limit ordinals,
and for each $\xi\in\Omega$ let $\sequence{i}{\theta_{\xi}(i)}$ be a
strictly increasing sequence of ordinals with supremum $\xi$.   Set
$X=\omega_1\times(\omega+1)$, and define a topology $\frak T$ on $X$ by
saying that $G\subseteq X$ is open iff

\inset{$\{\xi:(\xi,n)\in G\}$ is open in the order topology of
$\omega_1$ for every $n\in\Bbb N$,}

\inset{whenever $\xi\in\Omega$ and $(\xi,\omega)\in G$ then there is
some $n<\omega$ such that $(\eta,i)\in G$ whenever $n\le i<\omega$ and
$\theta_{\xi}(i)<\eta\le\xi$.}

\noindent This is finer than the product of the order topologies, so is
Hausdorff.   For every $\xi<\omega_1$ and $n\in\Bbb N$,
$(\xi+1)\times\{n\}$ is a countable compact open set containing
$(\xi,n)$;  for
every $\xi\in\omega_1\setminus\Omega$, $\{(\xi,\omega)\}$ is a countable
compact open set containg $(\xi,\omega)$;  and for every $\xi\in\Omega$,

\Centerline{$\{(\xi,\omega)\}
\cup\{(\eta,i):i<\omega,\,\theta_{\xi}(i)<\eta\le\xi\}$}

\noindent is a countable compact open subset of $X$ containing
$(\xi,\omega)$.   Thus $\frak T$ is locally compact, and every singleton
subset of $X$ is G$_{\delta}$, so $\frak T$ is first-countable (4A2Kf).

If $f:X\to\Bbb R$ is continuous, then for every $n\in\Bbb N$ there is a
$\zeta_n<\omega_1$ such that $f$ is constant on
$\{(\xi,n):\zeta_n\le\xi<\omega_1\}$ (4A2S(b-iii)).   Setting
$\zeta=\sup_{n\in\Bbb N}\zeta_n$, $f$ must be constant on
$\{(\xi,\omega):\xi\in\Omega,\,\xi>\zeta\}$.   \Prf\ If $\xi$,
$\eta\in\Omega\setminus(\zeta+1)$, then
$f(\xi,\omega)=\lim_{i\to\infty}f(\theta_{\xi}(i)+1,i)$ and
$f(\eta,\omega)=\lim_{i\to\infty}f(\theta_{\eta}(i)+1,i)$.   But there
is some $n$ such that both $\theta_{\xi}(i)$ and $\theta_{\eta}(i)$ are
greater than $\zeta$ for every $i\ge n$, so that
$f(\theta_{\xi}(i)+1,i)=f(\theta_{\eta}(i)+1,i)$ for every $i\ge n$ and
$f(\xi,\omega)=f(\eta,\omega)$.\ \Qed

Writing $\Sigma$ for the family of subsets $E$ of $X$ such that
$\{\xi:\xi\in\Omega,\,(\xi,\omega)\in E\}$ is either countable or
cocountable in $\Omega$, $\Sigma$ is a $\sigma$-algebra of subsets of
$X$ such that every continuous function is $\Sigma$-measurable, so every
Baire set belongs to $\Sigma$.   We therefore have a Baire measure
$\mu_0$ on $X$ defined by saying that $\mu_0E=0$ if
$E\cap(\Omega\times\{\omega\})$ is countable, $1$ otherwise.
$\{(\xi+1)\times(\omega+1):\xi<\omega_1\}$ is a cover of $X$ by
negligible open-and-closed sets, so $\mu_0$ is not $\tau$-additive.

\Quer\ Suppose, if possible, that $\mu$ were a Borel measure on $X$
extending $\mu_0$.   Then we must have
$\mu(\omega_1\times\{n\})=\mu_0(\omega_1\times\{n\})=0$ for every
$n\in\Bbb N$, so $\mu(\omega_1\times\{\omega\})=1$.   Let $\lambda$ be
the subspace measure on $\omega_1\times\{\omega\}$ induced by $\mu$.
If $A\subseteq\omega_1$, $(A\times\{\omega\})\cup(\omega_1\times\omega)$
is an open set, so $\lambda$ is defined on every subset of
$\omega_1\times\{\omega\}$;  and if $\xi<\omega_1$, then
$\mu_0((\xi+1)\times(\omega+1))=0$, so $\lambda$ is zero on singletons.
And this contradicts Ulam's theorem (419G, 438Cd).\ \Bang
}%end of proof of 439M
%4{}35A (type E_0)

\leader{439N}{Example} Give $\omega_1$ its order topology.

(i) $\omega_1$ is a normal Hausdorff space which is not measure-compact.

(ii) There is a Baire probability measure $\mu_0$ on $\omega_1$ which is
not $\tau$-additive and has a unique extension to a Borel measure, which
is not completion regular\cmmnt{;  that is, $\mu_0$ is of type E$_2$
in the classification of 435A}.

\proof{{\bf (a)} As noted in 4A2Rc, order topologies are always normal
and Hausdorff.

\medskip

{\bf (b)} Let $\mu$ be Dieudonn\'e's measure on $\omega_1$, and
$\mu_0$ its restriction to the Baire $\sigma$-algebra.   Then $\mu$ is
the only Borel measure extending $\mu_0$.   \Prf\ Let $\nu$ be any Borel
measure extending $\mu_0$.   Every set $[0,\xi]=\coint{0,\xi+1}$, where
$\xi<\omega_1$, is open-and-closed, so

\Centerline{$\nu[0,\xi]=\mu_0[0,\xi]=\mu[0,\xi]=0$;}

\noindent also, of course, $\nu\omega_1=1$.   Let $F\subseteq\omega_1$
be any closed
set.   If $F$ is countable, then it is included in some initial segment
$[0,\xi]$, so $\nu F=\mu F=0$.   Now suppose that $F$ is uncountable.
Set $G=\omega_1\setminus F$.   For each $\xi\in F$, set
$\zeta_{\xi}=\min\{\eta:\xi<\eta\in F\}$ and
$G_{\xi}=\ooint{\xi,\zeta_{\xi}}$.   Then
$\langle G_{\xi}\rangle_{\xi\in F}$ is a disjoint family of open sets.
By 438Bb and 419G/438Cd,

\Centerline{$\nu(\bigcup_{\xi\in F}G_{\xi})
=\sum_{\xi\in F}\nu G_{\xi}=0$.}

\noindent But now

\Centerline{$1=\nu\omega_1
=\nu F+\nu\coint{0,\min F}+\nu(\bigcup_{\xi\in F}G_{\xi})
=\nu F=\mu F$.}

Thus $\mu$ and $\nu$ agree on the family $\Cal E$ of closed sets.   By
the Monotone Class Theorem (136C), they agree on the $\sigma$-algebra
generated by $\Cal E$, which is their common domain;  so they are
equal.\ \Qed

\medskip

{\bf (c)} I have already remarked in 411Q-411R that $\mu$ and $\mu_0$
are not $\tau$-additive and $\mu$ is not completion regular.   So of
course $\omega_1$ is not measure-compact.
}%end of proof of 439N
%4{}35A (type E_2)

\leader{439O}{}\cmmnt{ In 439M I described a Baire measure with no
extension to a Borel measure.   In view of \Marik's theorem
(435C), it is natural to ask whether this can be done with a normal
space.   This leads us into relatively deep water, and the only examples
known need special assumptions.

\medskip

\noindent}{\bf Example} Assume Ostaszewski's $\clubsuit$.
Then there is a normal
Hausdorff space with a Baire probability measure $\mu$ which is not
$\tau$-additive and not extendable to
a Borel measure.   \cmmnt{(In the classification of 435A, $\mu$ is of
type E$_0$.)}

\proof{{\bf (a)} $\clubsuit$ implies that there is a family
$\langle C_{\xi}\rangle_{\xi<\omega_1}$ of sets such that (i)
$C_{\xi}\subseteq\xi$ for every $\xi<\omega_1$ (ii) $C_{\xi}\cap\eta$ is
finite whenever $\eta<\xi<\omega_1$ (iii) for any uncountable sets $A$,
$B\subseteq\omega_1$ there is a $\xi<\omega_1$ such that $A\cap C_{\xi}$
and $B\cap C_{\xi}$ are both infinite (4A1N).   For
$A\subseteq\omega_1$, set $A'=\{\xi:\xi<\omega_1$, $A\cap C_{\xi}$ is
infinite$\}$;  then $A'\cap B'$ is non-empty whenever $A$,
$B\subseteq\omega_1$ are uncountable.   But this means that $A'\cap B'$
is actually uncountable for uncountable $A$, $B$, since $A'\cap
B'\setminus\gamma\supseteq(A\setminus\gamma)'\cap(B\setminus\gamma)'$ is
non-empty for every $\gamma<\omega_1$.

Set $X=\omega_1\times\Bbb N$.   For
$x=(\xi,n)\in X$, say that

$$\eqalign{I_x&=C_{\xi}\times\{n-1\}\text{ if }n\ge 1,\cr
&=\emptyset\text{ otherwise }.\cr}$$

\medskip

{\bf (b)} Define a topology $\frak
T$ on $X$ by saying  that a set $G\subseteq X$ is open iff $I_x\setminus
G$ is finite for every $x\in G$.

The form of the construction ensures that $\frak T$ is T$_1$.   In fact,
$I_x\cap I_y$ is finite whenever $x\ne y$ in $X$.   \Prf\
Express $x$ as $(\xi,m)$ and $y$ as
$(\eta,n)$ where $\eta\le\xi$.   If either $m=0$ or $n=0$ or $m\ne n$,
$I_x\cap I_y=\emptyset$.   If
$n\ge 1$ and $\eta<\xi$, then

\Centerline{$I_x\cap I_y
\subseteq(C_{\xi}\cap\eta)\times\{n-1\}$}

\noindent is finite.   Similarly, $I_x\cap I_y$ is finite if $m\ge 1$
and $\xi<\eta$.\ \QeD\   Consequently $\{x\}\cup J$
is closed for every $x\in X$, $J\subseteq I_x$.

Observe that $(\xi+1)\times\Bbb N$ is open and closed for every
$\xi<\omega_1$, again because $C_{\eta}\cap(\xi+1)$ is finite
whenever $\eta\in\Omega$ and $\eta>\xi$, while $C_{\xi}\subseteq\xi$ for
every $\xi$.

\medskip

{\bf (c)} The next step is to understand the uncountable closed
subsets of $X$.   First, if $F\subseteq X$ is closed and $n\in\Bbb N$,
then
$F^{-1}[\{n\}]'$, as defined in (a), is a subset of $F^{-1}[\{n+1\}]$,
since if
$\xi\in F^{-1}[\{n\}]'$ then $I_{(\xi,n+1)}\cap F$ is infinite.   If $F$
is uncountable, there is some $n\in\Bbb N$ such that $F^{-1}[\{n\}]$ is
uncountable, so that (inducing on $m$) $F^{-1}[\{m\}]$ is uncountable
for every $m\ge n$.   Finally, this means that if $E$, $F\subseteq X$
are uncountable closed sets, there is an $m\in\Bbb N$ such that
$E^{-1}[\{m\}]$ and $F^{-1}[\{m\}]$ are both uncountable, so that
$E^{-1}[\{m\}]'\cap F^{-1}[\{m\}]'$ is non-empty and $E\cap F$ is
non-empty.

\medskip

{\bf (d)} It follows that $X$ is normal.   \Prf\ Let $E$ and $F$ be
disjoint closed sets in $X$.   By (c), at least one of them is
countable;  let us take it that $E\subseteq\zeta\times\Bbb N$ where
$\zeta<\omega_1$.   Enumerate the open-and-closed set
$W=(\zeta+1)\times\Bbb N$ as $\sequencen{x_n}$.   Choose
$\sequencen{U_n}$, $\sequencen{V_n}$ inductively, as follows.   $U_0=E$,
$V_0=F\cap W$.   If $x_n\in U_n$, then $U_{n+1}=U_n\cup
(I_{x_n}\setminus V_n)$ and $V_{n+1}=V_n$;  if $x_n\notin U_n$, then
$U_{n+1}=U_n$ and $V_{n+1}=V_n\cup\{x_n\}\cup(I_{x_n}\setminus U_n)$.
An easy induction shows that, for every $n$, ($\alpha$) $U_n\cap
V_n=\emptyset$ ($\beta$) $U_n\cup V_n\subseteq W$ ($\gamma$)
$I_x\cap(U_n\cup V_n)$ is finite for every $x\in X\setminus(U_n\cup
V_n)$ ($\delta$) $I_x\cap V_n$ is finite for every $x\in U_n$
($\epsilon$) $I_x\cap U_n$ is finite for every $x\in V_n$.

At the end of the induction, set $G=\bigcup_{n\in\Bbb N}U_n$,
$H=\bigcup_{n\in\Bbb N}V_n\cup(X\setminus W)$.   Then $E\subseteq G$,
$F\subseteq H$ and $G\cap H=\emptyset$.   If $x\in G$, it is of the form
$x_n$ for some $n$, in which case $x_n\in U_n$ (because $x_n\notin
V_{n+1}$) and $I_x\setminus U_{n+1}=I_x\cap V_n$ is finite;  thus $G$ is
open.   If $x\in H\cap W$, again it is of the form $x_n$ where this time
$x_n\notin U_n$, so that $I_x\setminus V_{n+1}=I_x\cap U_n$ is finite;
so $H$ is open.

Thus $E$ and $F$ are separated by open sets;  since $E$ and $F$ are
arbitrary, $X$ is normal.\ \Qed

Being T$_1$ (see (b)), $X$ is also Hausdorff.

\medskip

{\bf (e)} Because disjoint closed sets in $X$ cannot both be uncountable
((c) above), any bounded continuous function on $X$ must be constant
on a cocountable set.   (Compare 4A2S(b-iii).)
%same for usual topology on $\omega_1$.
The countable-cocountable measure $\mu_0$ is therefore a Baire measure
on $X$ (cf.\ 411R).   But it has no extension to a Borel measure.
\Prf\ The point is that if $A$ is any subset of $\omega_1$, and
$n\in\Bbb N$, then

\Centerline{$(A\times\{n\})\cup(\omega_1\times n)$,
\quad$\omega_1\times n$}

\noindent are both open, so $A\times\{n\}$ is Borel;  accordingly every
subset of $X$ is a Borel set.   But $\omega_1$ is measure-free
(419G, 438Cd), so there can be no Borel probability measure on $X$ which
is zero on singletons.\ \Qed

Of course $\mu_0$ is not $\tau$-additive, because $\{(\xi+1)\times\Bbb
N:\xi<\omega_1\}$ is a cover of $X$ by open-and-closed negligible sets.
}%end of proof of 439O

\cmmnt{\medskip

\noindent{\bf Remark} Thus in Ma\v r\'\i k's theorem we really do
need `countably paracompact' as well as `normal', at least if we want a
theorem valid in ZFC.

Observe that any example of this phenomenon must involve a {\bf Dowker
space}, that is, a normal Hausdorff space which is not countably
paracompact.   The one here
is based on {\smc de Caux 76}.   Such spaces are hard to come by in ZFC
if we do not allow
ourselves to use special principles like $\clubsuit$.   `Real' Dowker
spaces have been
described by {\smc Rudin 71} and {\smc Balogh 96};  for a survey, see
{\smc Rudin 84}.   I do not know if either of these can be adapted to
provide a ZFC example to replace the one above.
}%end of comment
\leaveitout{In {\smc Simon 71} it is noted that Rudin's Dowker space
gives a Baire probability measure $\mu$ which has no extension to a
Borel measure which is inner regular with respect to the closed sets.
}

\allowmorestretch{468}{
\leader{439P}{Example}\cmmnt{ (cf.\ {\smc Moran 68})}
$\BbbN^{\frak c}$ is
not Borel-measure-compact, therefore not Borel-measure-complete,
measure-compact or Radon.
}

\proof{ Consider the topology $\frak T_{\frak c}$ on
$Z=\{0,1\}^{\Bbb N}$, as constructed in 439K.   Then
$(Z,\frak T_{\frak c})$ is homeomorphic to a closed subset of
$\BbbN^Z\times\{0,1\}^{\Bbb N}$, where in this product the second factor
$\{0,1\}^{\Bbb N}$ is given
its usual topology $\frak S$.   \Prf\ For each
$x\in Z$, let  $L_x$ be a
$\frak T_{\frak c}$-open $\frak T_{\frak c}$-compact subset of $Z$.
The first thing to observe is that if $x\in Z$, and we write
$V_{xm}=\{y:y\in Z,\,y\restr m=x\restr m\}$ for each $m\in\Bbb N$, then
$\Cal U_x=\{L_x\cap V_{xm}:m\in\Bbb N\}$ is a downwards-directed family
of compact open neighbourhoods of $x$ with intersection $\{x\}$, so is a
base of neighbourhoods of $x$ (4A2Gd);  thus
$\Cal U=\{L_x:x\in Z\}\cup\frak S$ generates $\frak T_{\frak c}$.   Now,
for $x\in Z$, define $\phi_x:Z\to\Bbb N$ by setting

$$\eqalign{\phi_{x}(y)&=0\text{ if }y\in L_x,\cr
&=m+1\text{ if }y\in V_{xm}\setminus(L_x\cup V_{x,m+1}).\cr}$$

\noindent Then every $\phi_x$ is $\frak T_{\frak c}$-continuous, so we
have a $\frak T_{\frak c}$-continuous function
$\phi:Z\to\BbbN^Z\times\{0,1\}^{\Bbb N}$ defined by setting
$\phi(y)=(\family{z}{Z}{\phi_z(y)},y)$ for $y\in Z$.   Because every
element of $\Cal U$ is of the form $\phi^{-1}[H]$ for some open set
$H\subseteq\BbbN^Z\times\{0,1\}^{\Bbb N}$, $Z$ is homeomorphic to its
image $\phi[Z]$.

Now suppose that $(w,z)\in\overline{\phi[Z]}$.   In this case, there is
a filter $\Cal G$ containing $\phi[Z]$ which converges to $(w,z)$ (4A2Bc).
Let $\Cal F$ be an ultrafilter on $Z$ including
$\{\phi^{-1}[A]:A\in\Cal G\}$;  then $\phi[[\Cal F]]$ includes $\Cal G$ so
converges to $(w,z)$, and $\Cal F\to z$ for $\frak S$.
\Quer\ If $z$ is not the
$\frak T_{\frak c}$-limit of $\Cal F$, then $\Cal F$ can have no
$\frak T_{\frak c}$-limit, and can contain no
$\frak T_{\frak c}$-compact set (2A3R).   In particular,
$L_z\notin\Cal F$;  but in this case $V_{zm}\setminus L_z\in\Cal F$ for
every $m$, so that $\{(v,y):v(x)>m\}\in\phi[[\Cal F]]$ for every $m$,
and $w(x)>m$ for every $m$, which is impossible.\ \BanG\   Thus
$\Cal F\to z$, and (as $\phi$ is continuous) $(w,z)=\phi(z)$.

This shows that $\phi[Z]$ is closed, so we have the required
homeomorphism between $Z$ and a closed subset of
$\BbbN^Z\times\{0,1\}^{\Bbb N}$.\ \Qed

Of course $\BbbN^Z\times\{0,1\}^{\Bbb N}$ is a closed subset of
$\BbbN^Z\times\BbbN^{\Bbb N}\cong\BbbN^{\frak c}$.   So $Z$ is
homeomorphic to a closed subset of $\BbbN^{\frak c}$.  But $Z$, with
$\frak T_c$, carries a Borel probability measure $\nu$ which is inner
regular with respect to the closed sets and is not $\tau$-additive
(439Kb).   So $(Z,\frak T_{\frak c})$ is not Borel-measure-compact.
By 434Hc, $\BbbN^{\frak c}$ is not Borel-measure-compact.   By 434Ic,
$\BbbN^{\frak c}$ is not
Borel-measure-complete;  by 434Ka, it is not Radon;  by 435Fd, it is not
measure-compact.
}%end of proof of 439P

\leader{439Q}{Example} Let $X$ be the real line with the {\bf
right-facing Sorgenfrey topology}, generated by sets of the form
$\coint{a,b}$ where $a<b$ in $\Bbb R$.
Then $X$ is measure-compact but $X^2$ is not.

\proof{{\bf (a)} Note that every set $\coint{a,b}$ is open-and-closed in
$X$, so that the topology is zero-dimensional, therefore completely
regular;  and it is finer than the usual topology of $\Bbb R$, so is
Hausdorff.

$X$ is Lindel\"of.   \Prf\ Let $\Cal G$ be an open cover of $X$.
For each $q\in\Bbb Q$, set

\Centerline{$A_q=\{a:a\in\ooint{-\infty,q}$ and there is some
$G\in\Cal G$ such that $\coint{a,q}\subseteq G\}$.}

\noindent Then $\bigcup_{q\in\Bbb Q}A_q=\Bbb R$.   For each
$q\in\Bbb Q$, there is a countable set $A'_q\subseteq A_q$ such that
$\inf A'_q=\inf A_q$ in $[-\infty,\infty]$ and $A'_q$ contains
$\min A_q$ if $A_q$ has a least element.   Now, for each pair $(a,q)$
where $q\in\Bbb Q$ and $a\in A'_q$, choose $G_{aq}\in\Cal G$ such that
$\coint{a,q}\subseteq G_{aq}$.   It is easy to see that
$\bigcup\{G_{aq}:a\in A'_q\}\supseteq A_q$, so that the countable family
$\{G_{aq}:q\in\Bbb Q,\,a\in A'_q\}$ covers $X$.   As $\Cal G$ is
arbitrary, $X$ is Lindel\"of.\ \Qed

It follows that $X$ is measure-compact (435Fb).

\medskip

{\bf (b)} Let $\frak S$ be the usual topology on $\BbbR^2$, and $\frak
T$ the product topology on $X^2$.

\medskip

\quad{\bf (i)} Whenever $G$, $H$ are disjoint $\frak T$-open sets, there
is an $\frak S$-Borel set $E$ such that $G\subseteq E\subseteq
X^2\setminus H$.   \Prf\ For $n\in\Bbb N$, set

\Centerline{$A_n
=\{(a,b):\coint{a,a+2^{-n}}\times\coint{b,b+2^{-n}}\subseteq G\}$.}

\noindent\Quer\ Suppose, if possible, that there is a point
$(x,y)\in\overline{A}_n^{\frak S}\cap H$, where I write
$\overline{\phantom{A}}^{\frak S}$ to denote closure for the topology
$\frak
S$.   Let $\delta>0$ be such that
$\coint{x,x+2\delta}\times\coint{y,y+2\delta}\subseteq H$ and
$2\delta<2^{-n}$.   Then there must be $(a,b)\in A_n$ such that
$|a-x|\le\delta$ and
$|b-y|\le\delta$.   In this case, $a\le x+\delta<a+2^{-n}$ and $b\le
y+\delta<b+2^{-n}$, so $(x+\delta,y+\delta)\in G$;  while $\delta$ was
chosen so that $(x+\delta,y+\delta)$ would belong to $H$.\ \Bang

Accordingly $E=\bigcup_{n\in\Bbb N}\overline{A}_n^{\frak S}$ is an
$\frak S$-Borel set disjoint from $H$.   But $G=\bigcup_{n\in\Bbb
N}A_n$, so $G\subseteq E$.\ \Qed

\medskip

\quad{\bf (ii)} Consequently every $\frak T$-continuous real-valued
function is $\frak S$-Borel measurable.   \Prf\ If $f:X^2\to\Bbb R$ is
$\frak T$-continuous and $\alpha\in\Bbb R$, then there is an
$\frak S$-Borel set $E_{\alpha}$ such that

\Centerline{$\{(x,y):f(x,y)<\alpha\}
\subseteq E_{\alpha}\subseteq\{(x,y):f(x,y)\le\alpha\}$.}

\noindent But this means that
$\{(x,y):f(x,y)<\alpha\}=\bigcup_{n\in\BbbN}E_{\alpha-2^{-n}}$ is
$\frak S$-Borel.\ \Qed

\medskip

\quad{\bf (iii)} It follows that every $\frak T$-Baire set is
$\frak S$-Borel.   We therefore have a $\frak T$-Baire probability
measure $\nu$ on $X^2$ defined by setting

\Centerline{$\nu E=\mu_L\{t:t\in[0,1],\,(t,1-t)\in E\}$}

\noindent for every $\frak T$-Baire subset of $X^2$, where $\mu_L$ is
Lebesgue measure on $\Bbb R$.   In this case
every point $(x,y)$ of $X^2$ belongs to a $\frak T$-open set of zero
measure for $\nu$.   \Prf\ Set $K=\{(t,1-t):t\in[0,1]\}$.   Then $K$ is
$\frak S$-closed, therefore $\frak T$-closed, and $\nu(X^2\setminus
K)=0$, so if $(x,y)\notin K$ then we can stop.   If $(x,y)\in K$, then
$\coint{x,x+1}\times\coint{y,y+1}$ is a $\frak T$-open $\frak T$-closed
set meeting $K$
in the single point $(x,y)$, so is a negligible $\frak T$-neighbourhood
of $(x,y)$.\ \Qed

Thus $\nu$ is not $\tau$-additive and $X^2$ is not measure-compact.
}%end of proof of 439Q

\cmmnt{\medskip

\noindent{\bf Remark} Contrast this with 438Xr.
}%end of comment

\leader{439R}{Example} There are first-countable completely regular
Hausdorff spaces $X$, $Y$ with Baire probability measures $\mu$, $\nu$
such that the Baire measures $\lambda$, $\lambda'$ on $X\times Y$
defined by the formulae

\Centerline{$\int fd\lambda=\biggeriint f(x,y)\nu(dy)\mu(dx)$,
\quad$\int fd\lambda'=\biggeriint f(x,y)\mu(dx)\nu(dy)$}

\noindent\cmmnt{(436F) }are different.

\proof{ Let $X$, $Y$ be disjoint stationary subsets of $\omega_1$
(4A1Cd).   Give each the topology induced by the order topology of
$\omega_1$.   Let $\tilde\mu$ be Dieudonn\'e's measure on $\omega_1$,
and $\tilde\mu_X$, $\tilde\mu_Y$ the subspace measures
induced on $X$ and $Y$ by $\tilde\mu$;  let $\mu$ and $\nu$ be the
restrictions of $\tilde\mu_X$, $\tilde\nu_Y$ to the Baire
$\sigma$-algebras of $X$, $Y$ respectively.   Then

\Centerline{$\mu X=\tilde\mu_XX=\tilde\mu^*X=1$}

\noindent because $X$ meets every cofinal closed set in $\omega_1$;
similarly,
$\nu Y=1$.

Set

\Centerline{$W=\{(x,y):x\in X,\,y\in Y,\,x<y\}
=\{(x,y):x\in X,\,y\in Y,\,x\le y\}$.}

\noindent Then $W$ is open-and-closed in $X\times Y$ (use 4A2Rl),
so that $f=\chi W$ is continuous.   But

\Centerline{$\biggeriint f(x,y)\nu(dy)\mu(dx)
=\int\nu\{y:y\in Y,\,x<y\}\mu(dx)
=1$,}

\Centerline{$\biggeriint f(x,y)\mu(dx)\nu(dy)
=\int\mu\{x:x\in X,\,x<y\}\nu(dy)
=0$.}
}%end of proof of 439R

\cmmnt{\medskip

\noindent{\bf Remark} Contrast this with 434Xx and 439Yi.
}%end of comment

\leader{439S}{}\cmmnt{ The results of 437V leave open the question of
which familiar spaces, beyond \v{C}ech-complete spaces, can be
Prokhorov.   In fact rather
few are.   The basis of any further investigation must be the following
result.

\medskip

\noindent}{\bf Theorem}\cmmnt{ ({\smc Preiss 73})} $\Bbb Q$ is not a
Prokhorov space.

\proof{{\bf (a)} There is a non-decreasing sequence $\sequence{k}{X_k}$
of non-empty compact subsets of $X=\Bbb Q\cap[0,1]$, with union $X$, such
that whenever $k\in\Bbb N$, $x\in X_k$
and $\delta>0$, then $X_{k+1}\cap[x-\delta,x+\delta]$ is infinite.
\Prf\ Start by enumerating $X$ as $\sequence{k}{q_k}$.   Set
$X_0=\{q_0\}$.   Given
that $X_k\subseteq X$ is compact, then for each $m\in\Bbb N$ let
$\Cal E_m$ be a finite cover of $X_k$ by open intervals of length at most
$2^{-m}$ all meeting
$X_k$, and
let $I_{km}$ be a finite subset of $X\setminus X_k$ meeting every member
of $\Cal E_m$;  set
$X_{k+1}=X_k\cup\{q_{k+1}\}\cup\bigcup_{m\in\Bbb N}I_{km}$.
If $\Cal H$ is any cover of $X_{k+1}$ by open sets in $\Bbb R$, then
there is a finite $\Cal H_0\subseteq\Cal H$ covering $X_k$.   There must
be an $m\in\Bbb N$ such that
$[x-2^{-m},x+2^{-m}]\subseteq\bigcup\Cal H_0$ for every $x\in X_k$
(2A2Ed), so that $I_{kl}\subseteq\bigcup\Cal H_0$ for every
$l\ge m$, and $X_{k+1}\setminus\bigcup\Cal H_0$ is finite;  accordingly
there is a finite $\Cal H_1\subseteq\Cal H$ covering $X_{k+1}$.   As
$\Cal H$ is arbitrary,
$X_{k+1}$ is compact, and the induction can proceed.   If $x\in X_k$ and
$\delta>0$, then for every $m\in\Bbb N$ there is an $x'\in
X_{k+1}\setminus X_k$ such that
$|x'-x|\le 2^{-m}$, so that $[x-\delta,x+\delta]\cap X_{k+1}$ must be
infinite.\ \Qed

\medskip

{\bf (b)} If $\sequence{k}{\epsilon_k}$ is any sequence in
$\ooint{0,\infty}$, and $F\subseteq[0,1]$ is a countable closed set,
then there is an $x^*\in X\setminus F$ such that
$\rho(x^*,X_k)<\epsilon_k$ for every $k\in\Bbb N$.   \Prf\ We can
suppose that $\lim_{k\to\infty}\epsilon_k=0$.
Define $\sequence{k}{H_k}$ inductively, as follows.   $H_0=\Bbb R$.
Given $H_k$, set
$H_{k+1}=H_k\cap\{x:\rho(x,X_k\cap H_k)<\epsilon_k\}$, where
$\rho(x,A)=\inf_{y\in A}|x-y|$ for $x\in\Bbb R$, $A\subseteq\Bbb R$.
Observe that every $H_k$ is an
open subset of $\Bbb R$ and that $X_k\cap H_k\subseteq H_{k+1}\subseteq
H_k$ for every $k$;  consequently, setting $E=\bigcap_{k\in\Bbb N}H_k$,
$E$ is a
G$_{\delta}$ subset of $\Bbb R$ and $X_k\cap H_k\subseteq E$ for every
$k$.   In particular, $E\cap X$ contains $q_0$ and is not empty.   Next,
for each $k$,
$\rho(x,E\cap X_k)<\epsilon_k$ for every
$x\in H_{k+1}$ and therefore for every $x\in E$;  accordingly $E\cap X$
is dense in $E$.   Moreover, if $x\in E\cap X$, there is
a $k\in\Bbb N$ such that $x\in X_k$;  we must have $x\in H_k$, and in
this case $H_{k+1}$ is a neighbourhood of $x$.   So every neighbourhood
of $x$ contains
infinitely many points of $H_{k+1}\cap X_{k+1}\subseteq E\cap X$.   Thus
$E\cap X$ has no isolated points;  it follows that $E$ has no isolated
points.   By 4A2Mc and 4A2Me, $E$ is uncountable.

There is therefore a point $z\in E\setminus F$.   Let $m\in\Bbb N$ be
such that $\rho(z,F)\ge\epsilon_m$.   As $z\in H_{m+1}$, there is an
$x^*\in H_m\cap X_m$ such that
$|z-x^*|<\epsilon_m$ and $x^*\notin F$.   Let $k\in\Bbb N$.   If
$k\ge m$ then certainly $\rho(x^*,X_k)=0<\epsilon_k$.
If $k<m$ then $x^*\in H_{k+1}$ so
$\rho(x^*,X_k)\le\rho(x^*,H_k\cap X_k)<\epsilon_k$.   So we have a
suitable $x^*$.\ \Qed

\medskip

{\bf (c)} For $n$, $k\in\Bbb N$ set

\Centerline{$G_{kn}=\{x:x\in\Bbb R\setminus X_k$,
$\rho(x,X_n)>2^{-k}\}$.}

\noindent Then $G_{kn}$ is an open subset of $\Bbb R$.   Let $A$ be the
set of Radon probability
measures $\mu$ on $X$ such that $\mu(G_{kn}\cap X)\le 2^{-n}$ for all
$n$,
$k\in\Bbb N$.

\medskip

{\bf (d)} Write $\tilde A$ for the set of Radon probability measures
$\mu$ on $[0,1]$ such that $\mu(G_{kn}\cap[0,1])\le 2^{-n}$ for all
$k$, $n\in\Bbb N$.   Then $\tilde A$ is a narrowly closed subset of the
set of Radon probability measures on $[0,1]$, which is itself narrowly
compact (437R(f-ii)).
Also $\mu([0,1]\setminus X)=0$ for every $\mu\in\tilde A$.   \Prf\ Let
$K\subseteq [0,1]\setminus X$ be compact, and $n\in\Bbb N$.   Then
$K$ and
$X_n$ are disjoint compact sets, so there is some $k\in\Bbb N$ such that
$|x-y|>2^{-k}$ for every $x\in X_n$ and $y\in K$.   In this case
$K\subseteq G_{kn}$ so
$\mu K\le 2^{-n}$.   As $n$ is arbitrary, $\mu K=0$;  as $K$ is
arbitrary, $\mu([0,1]\setminus X)=0$.\ \Qed

$A$ is compact in the narrow topology.   \Prf\
The identity map $\phi:X\to[0,1]$ induces a map
$\tilde\phi:M^+_{\text{R}}(X)\to M^+_{\text{R}}([0,1])$ which is a 
homeomorphism between $M^+_{\text{R}}(X)$ and
$\{\mu:\mu\in M^+_{\text{R}}([0,1])$, $\mu([0,1]\setminus X)=0\}$ (437Nb).
The definition of $A$ makes it plain that it is
$\tilde\phi^{-1}[\tilde A]$;  since
$\tilde A\subseteq\{\mu:\mu\in M^+_{\text{R}}([0,1])$,
$\mu([0,1]\setminus X)=0\}$, $\tilde\phi\restr A$ is a homeomorphism
between $A$ and $\tilde A$, and $A$ is compact.\ \Qed

\medskip

{\bf (e)} $A$, regarded as a subset of $M^+_{\text{R}}(X)$, is not
uniformly tight.
\Prf\ Let $K\subseteq X$ be compact.   Consider the set
$C$ of those $w\in[0,1]^X$ such that $w(x)=0$ for every $x\in K$,
$\sum_{x\in X}w(x)\le 1$ and $\sum_{x\in G_{kn}\cap X}w(x)\le 2^{-n}$
for all $k$, $n\in\Bbb N$.
Then $C$ is a compact subset of $[0,1]^X$.   If $D\subseteq C$ is any
non-empty upwards-directed set, then $\sup D$, taken in $[0,1]^X$,
belongs to $C$.   By Zorn's
Lemma, $C$ has a maximal member $w$ say.   \Quer\ Suppose, if possible,
that $\sum_{x\in X}w(x)=\gamma<1$.
For each $n\in\Bbb N$, let $L_n\subseteq X$ be a finite set such that
$\sum_{x\in L_n}w(x)\ge\gamma-2^{-n-1}$, and $m_n\in\Bbb N$ such that
$L_n\subseteq X_{m_n}$.
By (b), there is an $x^*\in X\setminus K$
such that $\rho(x^*,X_n)<2^{-m_n}$ for every $n\in\Bbb N$.   Let
$r\in\Bbb N$ be such that $x^*\in X_r$ and $\gamma+2^{-r}\le 1$, and set
$w'(x^*)=w(x^*)+2^{-r}$,
$w'(x)=w(x)$ for every $x\in X\setminus\{x^*\}$.   Then certainly
$w'\in[0,1]^X$ and $\sum_{x\in X}w'(x)\le 1$.   If $k$, $n\in\Bbb N$ and
$x^*\notin G_{kn}$, then $\sum_{x\in G_{kn}\cap X}w'(x)=\sum_{x\in
G_{kn}\cap X}w(x)\le 2^{-n}$.   If $x^*\in G_{kn}$, then $n<r$ and
$2^{-k}<\rho(x^*,X_n)<2^{-m_n}$, so $m_n<k$ and $L_n\subseteq X_k$ and

\Centerline{$\sum_{x\in G_{kn}\cap X}w(x)\le\sum_{x\in X\setminus
X_k}w(x)\le\sum_{x\in X\setminus L_n}w(x)\le 2^{-n-1}$,}

\Centerline{$\sum_{x\in G_{kn}\cap X}w'(x)\le 2^{-n-1}+2^{-r}
\le 2^{-n}$.}

\noindent Thus $w'\in C$ and $w$ was not maximal.\ \Bang

Accordingly $\sum_{x\in X}w(x)=1$ and the point-supported measure $\mu$
defined by $w$ is a probability measure on $X$.   By the definition of
$C$, $\mu\in A$ and
$\mu(X\setminus K)=1$.   As $K$ is arbitrary, $A$ cannot be uniformly
tight.\ \Qed

\medskip

{\bf (f)} Thus $A$ witnesses that $X=\Bbb Q\cap[0,1]$ is not a Prokhorov
space.   Since $X$ is a closed subset of $\Bbb Q$, 437Vb tells us that
$\Bbb Q$ is not a Prokhorov space.
}%end of proof of 439S

\exercises{
\leader{439X}{Basic exercises (a)}
%\spheader 439Xa
(i) Show that there is a set $A\subseteq[0,1]$ such that $\mu_L^*A=1$,
where $\mu_L$ is Lebesgue measure, and every member of $[0,1]$ is
uniquely expressible as $a+q$ where $a\in A$, $q\in\Bbb Q$.   \Hint{134B.}
(ii) Define $f:[0,1]\to A$ by setting $f(x)=a$
when $x\in a+\Bbb Q$.   Show that the image measure
$\mu_Lf^{-1}$ takes only the values $0$ and $1$.   ({\smc Aldaz 95}.
Compare 342Xg.)
%439A

\spheader 439Xb
Let $X$ be a Radon Hausdorff space and $A$ a subset of
$X$.   Show that $A$ is universally negligible iff $\mu A=0$ for every
atomless Radon measure on $X$.
%439C

\sqheader 439Xc Let $X$ be a Hausdorff space.   Show that a set
$A\subseteq X$ is universally negligible iff $\mu A=0$ whenever $\mu$ is
a complete locally determined topological measure on $X$ such that
$\mu\{x\}=0$ for every $x\in X$.
%439C

\spheader 439Xd Let $X$ be a Hausdorff space.   Show that any
universally negligible subset of $X$ is universally measurable in the
sense of 434D.
%439C

\spheader 439Xe(i) Show that there is an analytic set
$A\subseteq\Bbb R$ such that for any Borel subset $E$ of
$\Bbb R\setminus A$ there is an uncountable Borel subset of
$\Bbb R\setminus(A\cup E)$.   \Hint{423Qb, part (c) of the proof of
423L.}   (ii) Show that $A$ is universally measurable, but there is no
Borel set $E$ such that $A\symmdiff E$ is universally negligible.
%439C

\spheader 439Xf Show that a first-countable compact Hausdorff space is
universally negligible iff it is scattered iff it is countable.
%439C

\spheader 439Xg Show that the product of two universally negligible
Hausdorff spaces is universally negligible.
%439C

\spheader 439Xh Let us say that a Hausdorff space $X$ is {\bf
universally $\tau$-negligible} if there is no $\tau$-additive Borel
probability measure on $X$ which is zero on singletons.   (i) Show that
if $X$ is a Hausdorff space and $A\subseteq X$, then $A$ is universally
$\tau$-negligible iff $\mu^*A=0$ for every $\tau$-additive Borel
probability measure on $X$ such that $\mu\{x\}=0$ for every $x\in X$.
(ii) Show that if $X$ is a regular Hausdorff space, then a subset $A$ of
$X$ is universally $\tau$-negligible iff $\mu A=0$ for every atomless
quasi-Radon measure on $X$.   (iii) Show that if $X$ is a completely
regular Hausdorff space, it is universally $\tau$-negligible iff
whenever $\mu$ is an atomless Radon measure on a space $Z$, and
$X'\subseteq Z$ is homeomorphic to $X$, then $\mu X'=0$.   (iv) Show
that a Hausdorff space $X$ is universally negligible iff it is
Borel-measure-complete and universally $\tau$-negligible.   (v) Show
that if $X$ is a Hausdorff space, $Y$ is a universally $\tau$-negligible
Hausdorff space, and $f:X\to Y$ is a continuous function such that
$f^{-1}[\{y\}]$ is universally $\tau$-negligible for every $y\in Y$,
then $X$ is universally $\tau$-negligible.   (vi) Show that the product
of two universally $\tau$-negligible Hausdorff spaces is universally
$\tau$-negligible.   (vi) Show that a scattered Hausdorff space (in
particular, any discrete space) is universally $\tau$-negligible.
(vii) Show that a compact
Hausdorff space is universally $\tau$-negligible iff it is scattered.
%439C

\sqheader 439Xi Let $\kappa$ be the smallest cardinal of any subset of
$\Bbb R$ which is not Lebesgue negligible.   Show that if $(Z,\Tau,\nu)$
is any complete locally determined atomless measure space and
$A\subseteq Z$ has cardinal less than $\kappa$, then $\nu A=0$.
%439E

\spheader 439Xj Let $(X,\le)$ be any well-ordered set and $\mu$ a
non-zero $\sigma$-finite measure on $X$ such that every
singleton is negligible.   Show that $\{(x,y):x\le y\}$ is not
measured by the (c.l.d.) product measure on $X\times X$.
\Hint{Reduce to the case in which $\mu$ is complete and
totally finite, $X=\zeta$ is an ordinal and $\mu\xi=0$ for every
$\xi<\zeta$.   You will probably need 251Q.}
%439F

\sqheader 439Xk Show that 439Fc, or any of the examples of 439A, can be
regarded as an example of a probability space $(X,\mu)$ and
a function $f:X\to[0,1]$ such that there is no extension of $\mu$ to a
measure $\nu$ such that $f$ is $\dom\nu$-measurable;  and accordingly can
provide an example of a probability space $(X,\mu)$ with a countable
totally ordered family $\Cal A$ of subsets of $X$ such that there is no
extension of $\mu$ to a measure measuring every member of $\Cal A$.
Contrast with 214P, 214Xm-214Xn and 214Yb.
%439F 439A

\spheader 439Xl Show that $1$-dimensional Hausdorff measure on $\BbbR^2$
is not inner regular with respect to the closed sets.   \Hint{439H.}
%439H

\spheader 439Xm Show that $\BbbN^I$ is not pre-Radon for any uncountable
set $I$.   \Hint{417Xq.}
%439P

\spheader 439Xn(i) Suppose that $X$ is a completely regular Hausdorff
space and there is a continuous function $f$ from $X$ to a separable
metrizable space $Z$ such that $f^{-1}[\{z\}]$ is Lindel\"of for every
$z\in Z$.   Show that $X$ is realcompact (definition:  436Xg).   (ii)
Show that the spaces $X$ of 439K and $X^2$ of 439Q are realcompact.
(iii) Show that $\frak c$ with its discrete topology, and
$\BbbN^{\frak c}$ with the product topology, are realcompact.
%439Q

\spheader 439Xo Show that the one-point compactification of the space
$(Z,\frak T_{\frak c})$ described in 439K is a scattered compact Hausdorff
space with an atomless Borel probability measure.
%439K out of order query

\spheader 439Xp Let $X$ be a Polish space, $A\subseteq X$
an analytic set which is not
Borel (423Qb, 423Ye), and $\ofamily{\xi}{\omega_1}{E_{\xi}}$ a
family of Borel
constituents of $X\setminus A$ (423P).   Suppose that
$x_{\xi}\in E_{\xi}\setminus\bigcup_{\eta<\xi}E_{\eta}$ for every
$\xi<\omega_1$.   Show that $\{x_{\xi}:\xi<\omega_1\}$ is universally
negligible.   Hence show that any probability measure with domain
$\Cal P\omega_1$ is point-supported.
%439C out of order query

\leader{439Y}{Further exercises (a)}
%\spheader 439Ya
Show that a subset $A$ of $\Bbb R$ is universally
negligible iff $f[A]$ is Lebesgue negligible for every continuous
injective function $f:\Bbb R\to\Bbb R$.   \Hint{if $\nu$ is an atomless
Borel probability measure on $\Bbb R$, set
$f(x)=x+\nu[0,x]$ for $x\ge 0$, and show that $\mu_Lf[E]=\mu E+\nu E$ for
every Borel set $E\subseteq\coint{0,\infty}$.}
%439C

\spheader 439Yb For this exercise only, let us say that a `universally
negligible measurable space' is a pair $(X,\Sigma)$ where $X$ is a set
and $\Sigma$ a $\sigma$-algebra of subsets of $X$ containing every
countable subset of $X$ such that there is no probability measure $\mu$
with domain $\Sigma$ such that $\mu\{x\}=0$ for every $x\in X$.
(i) Let $X$ be a set, $\Sigma$ a $\sigma$-algebra of subsets of $X$
containing all countable subsets of $X$, $A\subseteq X$ and $\Sigma_A$
the subspace $\sigma$-algebra.   Show that $(A,\Sigma_A)$ is universally
negligible iff $\mu^*A=0$ whenever $\mu$ is a probability measure with
domain $\Sigma$ which is zero on singletons.   Show that if $(X,\Sigma)$
is universally negligible so is $(A,\Sigma_A)$.   (ii) Let $X$ and $Y$
be sets, $\Sigma$ and $\Tau\,\,\sigma$-algebras of subsets of $X$ and
$Y$ containing all appropriate countable sets, and $f:X\to Y$ a
$(\Sigma,\Tau)$-measurable function.   Suppose that $(Y,\Tau)$ and
$(f^{-1}[\{y\}],\Sigma_{f^{-1}[\{y\}]})$ are universally negligible for
every $y\in Y$.   Show that $(X,\Sigma)$ is universally negligible.
(iii) Let $X$ be a set and $\Sigma$ a $\sigma$-algebra of subsets of $X$
containing all countable subsets of $X$.   Show that the set of those
$A\subseteq X$ such that $(A,\Sigma_A)$ is universally negligible is a
$\sigma$-ideal of subsets of $X$.
%439C

\spheader 439Yc Let $X$ be an analytic Hausdorff
space and $A$ an analytic subset
of $X$.   Show that $X\setminus A$ is universally negligible iff all the
constituents of $X\setminus A$ (for any Souslin scheme defining $A$) are
countable.
%439C

\spheader 439Yd(i) Let $X$ be a metrizable space such that $f[X]$ is
Lebesgue negligible for every continuous function $f:X\to\Bbb R$.   Show
that $X$ is universally negligible.   (ii) Let $X$ be a completely
regular Hausdorff space such that $f[X]$ is Lebesgue negligible for
every continuous function $f:X\to\Bbb R$.   Show that $X$ is universally
$\tau$-negligible.
%439G

\spheader 439Ye Let $\frak T_{\frak c}$ be the topology on
$\{0,1\}^{\Bbb N}$
constructed in the proof of 439K.   (i) Show that it is normal and
countably paracompact.   (ii)
Show that any $\frak T_{\frak c}$-zero set is an $\frak S$-Borel set.
%439K

\spheader 439Yf Show that there is no atomless Borel probability measure
on $\omega_1$ endowed with its order topology.   \Hint{411R, 439N.}
%439N

\spheader 439Yg Show that the space of 439O is locally compact and locally
countable, therefore first-countable.
%439O

\spheader 439Yh Show that the Sorgenfrey right-facing topology on
$\Bbb R$ is hereditarily Lindel\"of, but that its square is not
Lindel\"of.
%439Q

\spheader 439Yi Show that if $\omega_1$ is given its order topology, and
$f:\omega_1^2\to\Bbb R$ is continuous, then there is a $\zeta<\omega_1$
such that $f$ is constant on $(\omega_1\setminus\zeta)^2$.
\Hint{4A2S(b-iii).}   Show that if $\mu$ and $\nu$ are Baire probability
measures on $\omega_1$, then the Baire probability measures
$\mu\times\nu$, $\nu\times\mu$ on $\omega_1^2$ defined by the formulae
of 436F coincide.
%439R

\spheader 439Yj Let $X\subseteq[0,1]$ be a dense set with no
uncountable compact subset.   Show that $X$ is not a Prokhorov space.
%439S

\spheader 439Yk(i) A pair
$(\ofamily{\xi}{\omega_1}{a_{\xi}},\ofamily{\xi}{\omega_1}{b_{\xi}})$ of
families of subsets
of $\Bbb N$ is a {\bf Hausdorff gap} if $a_{\xi}\setminus a_{\eta}$,
$a_{\xi}\setminus b_{\xi}$ and $b_{\eta}\setminus b_{\xi}$ are finite whenever
$\xi\le\eta<\omega_1$, $a_{\eta}\setminus a_{\xi}$ and
$b_{\xi}\setminus b_{\eta}$ are infinite whenever
$\xi<\eta<\omega_1$, and moreover
$\{\xi:\xi<\eta$, $a_{\xi}\subseteq b_{\eta}\cup n\}$ is finite for every
$\eta<\omega_1$.   (For a construction of a Hausdorff gap, see
{\smc Fremlin 84}, 21L.)   Show that in this case there is no $c\subseteq\Bbb N$
such that $a_{\xi}\setminus c$ and $c\setminus b_{\xi}$ are finite for every
$\xi<\omega_1$, and that
$\{a_{\xi}:\xi<\omega_1\}\cup\{b_{\xi}:\xi<\omega_1\}$ is universally
negligible in $\Cal P\Bbb N$.   (ii) Let
$\phi:(\Cal P\Bbb N)^{\Bbb N}\to\Cal P\Bbb N$ be a homeomorphism.
For $0<\xi<\omega_1$ let
$\sequencen{\theta(\xi,n)}$ be a sequence running over $\xi$.   Set
$a_0=\emptyset$ and for $0<\xi<\omega_1$ set
$a_{\xi}=\phi(\sequencen{a_{\theta(\xi,n)}})$.   Show that
$\{a_{\xi}:\xi<\omega_1\}$ is universally negligible.
%439F
}%end of exercises

\cmmnt{\Notesheader{439} I give three separate constructions in 439A
because the phenomenon here is particularly important.   For two
chapters I have, piecemeal, been offering theorems on the extension of
measures.   The principal ones so far seem to be 413O, 415L, 416N,
417C, 417E and 435C, and I have used methods reflecting my belief
that the essential feature on which each such theorem depends is inner
regularity of an appropriate kind.   I think we should simultaneously
seek to develop an intuition for measures which do {\it not} extend, and
those in 439A are especially significant because they refer to the Borel
algebra of the unit interval, which in so many other contexts is
comfortably clear of the obstacles which beset more exotic structures.

Note that because the Borel $\sigma$-algebra of $\Bbb R$ is countably
generated, the examples here are examples of measures which cannot be
extended to measure every member of a countable family of sets.
Recall that in 214P I showed that measures can be extended to
measure the sets in arbitrary {\it well-ordered} families.

Outside the context of Polish spaces, the terms `universally measurable'
and `universally negligible' are not properly settled.   I have tried to
select definitions which lead to a reasonable pattern.   At least
a universally negligible subset of a Hausdorff space is
universally measurable (439Xd), and both concepts can be expressed in
terms of sets with $\sigma$-algebras, as in 439Yb.   It is important to
notice, in 439B,
that I write `$\mu\{x\}=0$ for every $x\in X$', not `$\mu$ is atomless'.
For instance, Dieudonn\'e's measure shows that $\omega_1$, with
its order topology, is not universally negligible on the definition
here;  but it is easy to show that there is no atomless Borel
probability measure on $\omega_1$ (439Yf).   In many cases, of course,
we do not need to make this distinction (439D).

The cardinal $\kappa$ of 439F (the `uniformity' of the Lebesgue null
ideal) is one of a large family of cardinals which will be
examined in Chapter 52 in the next volume.

% 439J rvmc  439O cofinal closed
In some of the arguments above (439J, 439L, 439O) I appeal to
(different) principles (`there is a cardinal which is not measure-free',
$\clubsuit$) which are not
theorems according to the rules I follow in this book.   Such examples
would in some ways fit better into Volume 5, where I mean to investigate
such principles properly.   I include the examples here because they do
at least exhibit bounds on what can be proved in ZFC.   I should not
want anyone to waste her time trying to show, for instance, that all
completion regular Borel measures are $\tau$-additive.   Nevertheless,
the absence of a `real' counter-example (obviously we want a probability
measure on a completely regular Hausdorff space) remains in my view a
significant gap.   It remains conceivable that there is a mathematical
world in which no such space exists.   Clearly the discovery of such a
world is likely to require familiarity with the many worlds already
known, and I am not going to embark on any such exploration in this
volume.   On the other hand, it is also very possible that all we need
is a bit of extra ingenuity to construct a counter-example in ZFC.   In
this section we have two examples of successes of this kind.   In
439F-439H, for instance, we have results which were long known as
consequences of the continuum hypothesis;  the particular insight of
{\smc Grzegorek 81} was the observation that they depended on
determinate properties of the cardinal $\kappa$ of 439F, and that its
indeterminate position between $\omega_1$ and $\frak c$ was unimportant.
In 439K I show how a
re-working of ideas in {\smc Juh\'asz Kunen \& Rudin 76}, where a
similar example was constructed (for an entirely different purpose)
assuming the continuum hypothesis, provides us with an interesting space
(a first-countable non-Radon compact Hausdorff space) in ZFC.   Let me
emphasize that these ideas were originally set out in a framework
supported by an extra axiom, where some technical details were easier
and the prize aimed at (a non-Lindel\"of hereditarily separable space)
more important.

The examples in 439K-439R are mostly based on constructions more or less
familiar from general topology.   I have already mentioned the origins
of 439K.   439M is related to the Tychonoff and Dieudonn\'e planks
({\smc Steen \& Seebach 78}, \S\S86-89).   439N and 439Q revisit yet
again $\omega_1$ and the Sorgenfrey line.   439O is adapted from one of
the standard constructions of Dowker spaces.   Products of disjoint
stationary sets (439R) have also been used elsewhere.
}%end of notes

\frnewpage

