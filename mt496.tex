\frfilename{mt496.tex}
\versiondate{27.5.09}

\def\chaptername{Further topics}
\def\sectionname{Maharam submeasures}

\newsection{496\dvAnew{2008}}

The old problem of characterizing measurable algebras led, among other
things, to the concepts of `Maharam submeasure' and `Maharam algebra'
(\S393).   It is known that these can be very different from measures
(\S394), but the differences are not well understood.   In this section I
will continue the work of \S393 by showing that some, at least, of the
ways in which topologies and measures interact apply equally to Maharam
submeasures.   The most important of these interactions are associated with
the concept of `Radon measure', so the first step is to find a
corresponding notion of `Radon submeasure' (496C, 496Y).   In
496D-496K %496D 496E 496F 496G 496H 496I 496J 496K
I run
through a handful of theorems which parallel results in \S\S416 and
431-433.  %431 432 433
Products of submeasures remain problematic, but something can be
done (496L-496M).

\leader{496A}{Definitions}\cmmnt{ As
we have hardly had `submeasures' before
in this volume, I repeat the essential definitions from Chapter 39.}
If $\frak A$ is a Boolean algebra, a {\bf submeasure} on $\frak A$ is a
functional $\mu:\frak A\to[0,\infty]$
such that $\mu 0 = 0$ and $\mu a\le\mu(a\Bcup b)\le\mu a+\mu b$ for all
$a$, $b\in\frak A$\cmmnt{ (392A)}.
$\mu$ is {\bf strictly positive} if $\mu a>0$ for every
$a\in\frak A\setminus\{0\}$\cmmnt{ (392Ba)}, {\bf exhaustive} if
$\lim_{n\to\infty}\mu a_n=0$ for every disjoint sequence $\sequencen{a_n}$
in $\frak A$\cmmnt{ (392Bb)},
{\bf totally finite} if $\mu 1<\infty$\cmmnt{ (392Bd)},
a {\bf Maharam submeasure} if it is totally finite and
$\lim_{n\to\infty}\mu a_n=0$ for every
non-increasing sequence $\sequencen{a_n}$ in $\frak A$ with zero
infimum\cmmnt{ (393A)}.
A Maharam submeasure is sequentially
order-continuous\cmmnt{ (393Ba)}.
If $\mu$ and $\nu$ are two submeasures on a Boolean algebra $\frak A$, then
$\mu$ is {\bf absolutely continuous} with respect to $\nu$ if for every
$\epsilon>0$ there is a $\delta>0$ such that $\mu a\le\epsilon$ whenever
$\nu a\le\delta$\cmmnt{ (392Bg)}.
A {\bf Maharam algebra} is a Dedekind $\sigma$-complete Boolean algebra
which carries a strictly positive Maharam
submeasure\cmmnt{ (393E)}.

\leader{496B}{Basic facts }\cmmnt{I list some elementary ideas for future
reference.

\medskip

}{\bf (a)} Let $\mu$ be a submeasure on a Boolean algebra $\frak A$.

\medskip

\quad{\bf (i)} Set $I=\{a:a\in\frak A$, $\mu a=0\}$.
\cmmnt{Clearly }$I$ is an
ideal of $\frak A$;  write $\frak C$ for the quotient Boolean algebra
$\frak A/I$.   Then we have a strictly positive
submeasure $\bar\mu$ on $\frak C$ defined by setting
$\bar\mu a^{\ssbullet}=\mu a$ for every $a\in\frak A$.   \prooflet{\Prf\
If $a^{\ssbullet}=b^{\ssbullet}$ then

\Centerline{$\mu(a\Bsetminus b)=\mu(b\Bsetminus a)=\mu(a\Bsymmdiff b)=0$,
\quad$\mu a=\mu(a\Bcap b)=\mu b$;}

\noindent so $\bar\mu$ is well-defined.   The formulae defining
`submeasure' transfer directly from $\mu$ to $\bar\mu$.   If
$\bar\mu a^{\ssbullet}=0$ then $\mu a=0$, $a\in I$ and $a^{\ssbullet}=0$,
so $\bar\mu$ is strictly positive.\ \Qed}

\medskip

\quad{\bf (ii)} If $\mu$ is exhaustive, so is $\bar\mu$.
\prooflet{\Prf\ If $\sequencen{a_n}$ is a
sequence in $\frak A$ such that $\sequencen{a_n^{\ssbullet}}$ is disjoint
in $\frak A/I$, set $b_n=a_n\Bsetminus\sup_{i<n}a_i$ for each $n$;  then
$\sequencen{b_n}$ is disjoint so

\Centerline{$\lim_{n\to\infty}\bar\mu a_n^{\ssbullet}
=\lim_{n\to\infty}\bar\mu b_n^{\ssbullet}
=\lim_{n\to\infty}\mu b_n=0$;}

\noindent thus $\bar\mu$ is exhaustive.\ \Qed}

\medskip

\quad{\bf (iii)} If $\frak A$ is Dedekind $\sigma$-complete and $\mu$ is a
Maharam submeasure, then $\frak C$
is a Maharam algebra.  \prooflet{\Prf\ As $\mu$ is sequentially
order-continuous, $I$ is a $\sigma$-ideal and $\frak C$ is Dedekind
$\sigma$-complete (314C).   Now suppose that $\sequencen{a_n}$ is a
sequence in $\frak A$ such that $\sequencen{a_n^{\ssbullet}}$ is
non-increasing and has zero infimum in $\frak C$.   Set
$b_n=\inf_{i\le n}a_i$ for each $n$, and $a=\inf_{n\in\Bbb N}a_n$;  then
$a^{\ssbullet}=0$ so $\mu a=0$ and (again because $\mu$ is sequentially
order-continuous)

\Centerline{$\lim_{n\to\infty}\bar\mu a_n^{\ssbullet}
=\lim_{n\to\infty}\bar\mu b_n^{\ssbullet}
=\lim_{n\to\infty}\mu b_n
=\mu a=0$.}

\noindent Since we already know that $\bar\mu$ is a strictly positive
submeasure, it is a strictly positive Maharam submeasure and $\frak C$ is a
Maharam algebra.\ \Qed}

In this context I will say that $\frak C$ is
{\bf the Maharam algebra of} $\mu$.

\spheader 496Bb If $\mu$ is a strictly positive totally finite submeasure
on a Boolean algebra $\frak A$, there is an associated metric
$(a,b)\mapsto\mu(a\Bsymmdiff b)$\cmmnt{
(392H)};
the corresponding metric completion $\widehat{\frak A}$ admits a continuous
extension of $\mu$ to a strictly positive submeasure $\hat\mu$ on
$\widehat{\frak A}$.   If $\mu$ is exhaustive, then $\hat\mu$ is a Maharam
submeasure and $\widehat{\frak A}$ is a Maharam
algebra\cmmnt{ (393H)}.
A Maharam algebra is ccc, therefore Dedekind
complete, and \wsid\cmmnt{ (393Eb)}.

\spheader 496Bc If $\mu$ is a submeasure defined on an algebra
$\Sigma$ of subsets of a set $X$, I will say that the {\bf null ideal}
$\Cal N(\mu)$ of $\mu$ is
the ideal of subsets of $X$ generated by $\{E:E\in\Sigma$, $\mu E=0\}$.
If $\Cal N(\mu)\subseteq\Sigma$ I will say that $\mu$ is {\bf complete}.
Generally,
the {\bf completion} of $\mu$ is the functional $\hat\mu$ defined by saying
that $\hat\mu(E\symmdiff A)=\mu E$ whenever $E\in\Sigma$ and
$A\in\Cal N(\mu)$;\cmmnt{  it is elementary to check that} $\hat\mu$ is a
complete submeasure.

\spheader 496Bd If $\frak A$ is a Maharam algebra, and $\mu$, $\nu$ are two
strictly positive Maharam submeasures on $\frak A$, then each is absolutely
continuous with respect to the other\cmmnt{ (393F)}.
Consequently the metrics associated with them\cmmnt{ are uniformly
equivalent, and} induce the same topology, the {\bf Maharam-algebra
topology} of $\frak A$\cmmnt{ (393G)}.

\leader{496C}{Radon submeasures} Let $X$ be a Hausdorff space.
A {\bf totally finite Radon submeasure} on $X$ is a complete totally finite submeasure
$\mu$ defined on a $\sigma$-algebra $\Sigma$ of subsets of $X$
such that (i) $\Sigma$ contains every open set (ii)
$\inf\{\mu(E\setminus K):K\subseteq E$ is compact$\}=0$
for every $E\in\Sigma$.

In this context I
will say that a set $E\in\Sigma$ is {\bf self-supporting} if
$\mu(E\cap G)>0$ whenever $G\subseteq X$ is open and $G\cap E\ne\emptyset$.

\leader{496D}{Proposition}
Let $\mu$ be a totally finite Radon submeasure on a Hausdorff space $X$ with domain
$\Sigma$.

(a) $\mu$ is a Maharam submeasure.

(b) $\inf\{\mu(G\setminus E):G\supseteq E$ is open$\}=0$ for every
$E\in\Sigma$.

(c) If $E\in\Sigma$ there is a relatively closed $F\subseteq E$ such that
$F$ is self-supporting and $\mu(E\setminus F)=0$.

(d) If $E\in\Sigma$ and $\epsilon>0$ there is a compact self-supporting
$K\subseteq E$ such that $\mu(E\setminus K)\le\epsilon$.

\proof{{\bf (a)}
Let $\sequencen{E_n}$ be a non-increasing
sequence in $\Sigma$ with empty intersection.   \Quer\ If
$\inf_{n\in\Bbb N}\mu E_n=\epsilon>0$, then for each $n\in\Bbb N$ choose a
compact set $K_n\subseteq E_n$ such that
$\mu(E_n\setminus K_n)\le 2^{-n-2}\epsilon$.   For each $n\in\Bbb N$,

\Centerline{$\mu(E_n\setminus\bigcap_{i\le n}K_i)
\le\sum_{i=0}^n\mu(E_i\setminus K_i)<\epsilon\le\mu E_n$,}

\noindent so $\bigcap_{i\le n}K_i\ne\emptyset$.   There is therefore a
point in $\bigcap_{n\in\Bbb N}K_n\subseteq\bigcap_{n\in\Bbb N}E_n$.\ \BanG\
As $\sequencen{E_n}$ is arbitrary, $\mu$ is a Maharam submeasure.

\medskip

{\bf (b)} We have only to observe that

$$\eqalign{&\inf\{\mu(G\setminus E):G\supseteq E\text{ and }G
  \text{ is open}\}\cr
&\mskip150mu
\le\inf\{\mu((X\setminus E)\setminus K):K\subseteq X\setminus E
  \text{ is compact}\}
=0.\cr}$$

\medskip

{\bf (c)} Let $\Cal G$ be the family of open subsets $G$ of $X$ such that
$\mu(E\cap G)=0$, and $H=\bigcup\Cal G$.   Then $\Cal G$ is
upwards-directed.   If $\epsilon>0$ there is a compact set
$K\subseteq E\cap H$ such that $\mu(E\cap H\setminus K)\le\epsilon$;  now
there is a $G\in\Cal G$ such that $K\subseteq G$ and
$\mu(E\cap H)\le\epsilon+\mu K=\epsilon$.   As $\epsilon$ is arbitrary,
$H\in\Cal G$;  set $F=E\setminus H$.

\medskip

{\bf (d)} There is a compact $K_0\subseteq E$ such that
$\mu(E\setminus K_0)\le\epsilon$;  by (c), there is a closed
self-supporting $K\subseteq K_0$ such that $\mu(K_0\setminus K)=0$.
}%end of proof of 496D

\vleader{84pt}{496E}{Theorem} Let $X$ be a Hausdorff space and $\Cal K$ the
family of
compact subsets of $X$.   Let $\phi:\Cal K\to\coint{0,\infty}$ be a
bounded functional such that

\inset{($\alpha$) $\phi\emptyset=0$ and
$\phi K\le\phi(K\cup L)\le\phi K+\phi L$ for all $K$,
$L\in\Cal K$;

($\beta$) whenever $K\in\Cal K$ and $\epsilon>0$ there is an $L\in\Cal K$
such that $L\subseteq X\setminus K$ and $\phi K'\le\epsilon$ whenever
$K'\in\Cal K$ is disjoint from $K\cup L$;

($\gamma$) whenever $K$, $L\in\Cal K$ and $K\subseteq L$ then
$\phi L\le\phi K+\sup\{\phi K':K'\in\Cal K$, $K'\subseteq L\setminus K\}$.
}

\noindent Then there is a unique totally finite Radon submeasure on $X$ extending $\phi$.

\proof{{\bf (a)} For $A\subseteq X$ write
$\phi_*A=\sup\{\phi K:K\subseteq A$ is compact$\}$.   Then $\phi_*$ extends
$\phi$, by ($\alpha$).  Also
$\phi_*(\bigcup_{n\in\Bbb N}G_n)\le\sum_{n=0}^{\infty}\phi_*G_n$ for every
sequence $\sequencen{G_n}$ of open subsets of $X$.
\Prf\ If $K\subseteq\bigcup_{n\in\Bbb N}G_n$ is compact, it is
expressible as $\bigcup_{i\le n}K_i$ where $n\in\Bbb N$
and $K_i\subseteq G_i$ is compact for every $i\le n$ (4A2Fj).   Now

\Centerline{$\phi K\le\sum_{i=0}^n\phi K_i
\le\sum_{i=0}^{\infty}\phi_*G_i$.}

\noindent As $K$ is arbitrary,
$\phi_*(\bigcup_{n\in\Bbb N}G_n)\le\sum_{n=0}^{\infty}\phi_*G_n$.\ \QeD\
In particular, because $\phi\emptyset=0$,
$\phi_*(G\cup H)\le\phi_*G+\phi_*H$ for all open $G$, $H\subseteq X$.

\medskip

{\bf (b)} Let $\Sigma$ be the family of subsets $E$
of $X$ such that for every $\epsilon>0$ there is a $K\subseteq X$ such that
$K\cap E$ and $K\setminus E$ are both compact and
$\phi_*(X\setminus K)\le\epsilon$.   Then $\Sigma$ is an algebra of subsets
of $X$ including $\Cal K$.   \Prf\ (i) Of course $X\setminus E\in\Sigma$
whenever $E\in\Sigma$.   (ii) If $E$, $F\in\Sigma$ and $\epsilon>0$, let
$K$, $L\subseteq X$ be such that $K\cap E$, $K\setminus E$, $L\cap F$ and
$L\setminus F$ are all compact and $\phi_*(X\setminus K)$,
$\phi_*(X\setminus L)$ are both at most $\bover12\epsilon$.   Then
$(K\cap L)\cap(E\cup F)$ and $(K\cap L)\setminus(E\cup F)$ are both
compact, and

\Centerline{$\phi_*(X\setminus(K\cap L))
\le\phi_*(X\setminus K)+\phi_*(X\setminus L)\le\epsilon$.}

\noindent As $\epsilon$ is
arbitrary, $E\cup F\in\Sigma$.   (iii) By hypothesis ($\beta$),
$\Cal K\subseteq\Sigma$.\ \Qed

\medskip

{\bf (c)} $\Sigma$ is a $\sigma$-algebra of subsets of $X$.   \Prf\ Let
$\sequencen{E_n}$ be a sequence in $\Sigma$ with intersection $E$, and
$\epsilon>0$.   For each $n\in\Bbb N$ let $K_n\subseteq X$ be such that
$K_n\cap E_n$ and $K_n\setminus E_n$ are compact and
$\phi_*(X\setminus K_n)\le 2^{-n}\epsilon$;  set
$K=\bigcap_{n\in\Bbb N}K_n$.   Set $L=\bigcap_{n\in\Bbb N}K_n\cap E_n$, so
that $L\subseteq E$ is compact, and let $L'\subseteq X\setminus L$ be a
compact set such that $\phi_*(X\setminus(L\cup L'))\le\epsilon$;  set
$K'=K\cap(L\cup L')$.   Then $\phi_*(X\setminus K')\le 3\epsilon$.
As $L'\cap L=\emptyset$ there is an $n\in\Bbb N$ such that
$L'\cap\bigcap_{i\le n}K_i\cap E_i$ is empty.   Now

\Centerline{$K\cap L'
\subseteq\bigcup_{i\le n}(X\setminus(K_i\cap E_i))\cap\bigcap_{i\le n}K_i
\subseteq\bigcup_{i\le n}X\setminus E_i
\subseteq X\setminus E$,}

\noindent so $K'\cap E=K\cap L$ and $K'\setminus E=K\cap L'$ are compact.
As $\epsilon$ is arbitrary, $E\in\Sigma$.\ \Qed

\medskip

{\bf (d)} Set $\mu=\phi_*\restr\Sigma$.   Then $\mu$ is subadditive.
\Prf\ Suppose that $E$, $F\in\Sigma$ and $K\subseteq E\cup F$ is compact.
Let $\epsilon>0$.  Then there are $L_1$, $L_2\in\Cal K$ such that
$L_1\cap E$, $L_1\setminus E$, $L_2\cap F$ and $L_2\setminus F$ are all
compact, while $\phi_*(X\setminus L_1)$ and $\phi_*(X\setminus L_2)$ are
both at most $\epsilon$.   Set $K_1=L_1\cap E$ and
$K_2=L_2\cap F$, so that

$$\eqalignno{\phi K
&\le\phi(K\cup K_1\cup K_2)
\le\phi(K_1\cup K_2)+\phi_*(K\setminus(K_1\cup K_2))\cr
\displaycause{by hypothesis ($\gamma$)}
&\le\phi K_1+\phi K_2+\phi_*(X\setminus(L_1\cap L_2))
\le\phi_*E+\phi_*F+2\epsilon.\cr}$$

\noindent As $\epsilon$ and $K$ are
arbitrary, $\phi_*(E\cup F)\le\phi_*E+\phi_*F$.\ \Qed

\medskip

{\bf (e)} Every open set belongs to $\Sigma$.   \Prf\ Let $G\subseteq X$ be
open, and $\epsilon>0$.   Applying ($\beta$) with $K=\emptyset$ we have an
$L\in\Cal K$ such that $\phi_*(X\setminus L)\le\epsilon$.   Next, there is
an $L'\in\Cal K$, disjoint from $L\setminus G$, such that
$\phi_*(X\setminus((L\setminus G)\cup L'))\le\epsilon$.   Set
$L''=L\cap((L\setminus G)\cup L')$.   Then $L''\cap G=L\cap L'$ and
$L''\setminus G=L\setminus G$ are compact and
$\phi_*(X\setminus L'')\le 2\epsilon$.\ \Qed

\medskip

{\bf (f)} If $E\subseteq F\in\Sigma$ and $\mu F=0$ then $E\in\Sigma$.
\Prf\ Let $\epsilon>0$.   Let $K\subseteq X$ be such that $K\cap F$ and
$K\setminus F$ are both compact and $\phi_*(X\setminus K)\le\epsilon$.
Then $(K\setminus F)\cap E$ and $(K\setminus F)\setminus E$
are both compact, and

\Centerline{$\phi_*(X\setminus(K\setminus F))
=\mu(X\setminus(K\setminus F))
\le\mu(X\setminus K)+\mu F=\phi_*(X\setminus K)\le\epsilon$.}

\noindent As $\epsilon$ is arbitrary, $E\in\Sigma$.\ \Qed

\medskip

{\bf (g)} If
$E\in\Sigma$ and $\epsilon>0$, there is a compact $K\subseteq E$ such that
$\mu(E\setminus K)\le\epsilon$.   \Prf\ Let $K_0\subseteq X$ be such that
$K_0\cap E$
and $K_0\setminus E$ are both compact and
$\phi_*(X\setminus K_0)\le\epsilon$.   Set $K=E\cap K_0$.
If $L\in\Cal K$ and $L\subseteq E\setminus K$ then
$\phi L\le\phi_*(X\setminus K_0)\le\epsilon$;  so
$\mu(E\setminus K)\le\epsilon$.\ \Qed

\medskip

{\bf (h)} So $\mu$ is a totally finite Radon submeasure.
To see that it is unique, let
$\mu'$ be another totally finite Radon submeasure with the same
properties, and $\Sigma'$
its domain.   By condition (ii) of 496C, $\mu'=\phi_*\restr\Sigma'$.
If $E\in\Sigma$ there are sequences $\sequencen{K_n}$,
$\sequencen{L_n}$ of compact sets such that $K_n\subseteq E$,
$L_n\subseteq X\setminus E$ and
$\mu(E\setminus K_n)+\mu((X\setminus E)\setminus L_n)\le 2^{-n}$
for every $n$.   Set $F=\bigcup_{n\in\Bbb N}K_n$ and
$F'=\bigcup_{n\in\Bbb N}L_n$;  then $F\cup F'$ belongs to
$\Sigma\cap\Sigma'$ and

$$\eqalign{\mu'(X\setminus(F\cup F'))
&=\phi_*(X\setminus(F\cup F'))
=\mu(X\setminus(F\cup F'))\cr
&\le\inf_{n\in\Bbb N}\mu(X\setminus(K_n\cup L_n))
=0.\cr}$$

\noindent Consequently $E\setminus F\in\Sigma'$ and $E\in\Sigma'$.

The same works with $\mu$ and $\mu'$ interchanged, so $\Sigma=\Sigma'$ and
$\mu'=\phi_*\restr\Sigma=\mu$.
}%end of proof of 496E

\leader{496F}{Theorem} Let $X$ be a zero-dimensional compact Hausdorff
space
and $\Cal E$ the algebra of open-and-closed subsets of $X$.   Let
$\nu:\Cal E\to\coint{0,\infty}$ be an exhaustive submeasure.   Then there
is a unique totally finite Radon submeasure on $X$ extending $\nu$.

\proof{{\bf (a)} Let $\Cal K$ be the family of compact subsets of
$X$ and for $K\in\Cal K$ set $\phi K=\inf\{\nu E:K\subseteq E\in\Cal E\}$.
Then $\phi$ satisfies the conditions of 496E.

\medskip

\Prf\grheada\
Of course $\phi\emptyset=0$ and $\phi K\le\phi L$ whenever $K\subseteq L$
in $\Cal K$.   If $K\subseteq E\in\Cal E$ and $L\subseteq F\in\Cal E$,
then $K\cup L\subseteq E\cup F\in\Cal E$ and
$\nu(E\cup F)\le\nu E+\nu F$, so $\phi$ is subadditive.

\medskip

\quad\grheadb\ The point is that for every $K\in\Cal K$ and $\epsilon>0$
there is an $E\in\Cal E$ such that $K\subseteq E$ and $\nu F\le\epsilon$
whenever $F\in\Cal E$ and $F\subseteq E\setminus K$;  since otherwise we
could find a disjoint sequence $\sequencen{F_n}$ in $\Cal E$ with
$\nu F_n\ge\epsilon$ for every $n$.   But now $L=X\setminus E$ is compact
and disjoint from $K$, and every compact subset of
$X\setminus(K\cup L)=E\setminus K$ is included in a member of $\Cal E$
included in $E\setminus K$;  so
$\sup\{\phi K':K'\subseteq X\setminus(K\cup L)$ is compact$\}\le\epsilon$.

\medskip

\quad\grheadc\ If $K$ and $L$ are compact and $K\subseteq L$ and
$\epsilon>0$, take
$E\in\Cal E$ such that $K\subseteq E$ and $\nu E\le\phi K+\epsilon$.
Set $K'=L\setminus E$.   If $F\in\Cal E$ and $F\supseteq K'$, then
$E\cup F\supseteq L$, so

\Centerline{$\phi L\le\nu(E\cup F)\le\nu E+\nu F\le\phi K+\epsilon+\nu F$.}

\noindent As $F$ is arbitrary, $\phi L\le\phi K+\phi K'+\epsilon$.\ \Qed

\medskip

There is therefore a totally finite Radon submeasure $\mu$ extending $\phi$ and
$\nu$.

\medskip

{\bf (b)} If $\mu'$ is another totally finite Radon submeasure extending $\nu$, then
$\mu'\restr\Cal K=\phi$.   \Prf\ Of course $\mu'K\le\phi K$ for every
$K\in\Cal K$.   \Quer\ If $K\in\Cal K$ and $\epsilon>0$ and
$\mu'K+\epsilon<\phi K$, let $E\in\Cal E$ be such that $K\subseteq E$ and
$\phi L\le\epsilon$ whenever $L\subseteq E\setminus K$ is compact, as in
(a-$\beta$) above.   Then

$$\eqalign{\mu'(E\setminus K)
&=\sup\{\mu'L:L\subseteq E\setminus K\text{ is compact}\}\cr
&\le\sup\{\phi L:L\subseteq E\setminus K\text{ is compact}\}
\le\epsilon\cr}$$

\noindent and

\Centerline{$\nu E=\mu'E\le\epsilon+\mu'K<\mu K\le\mu E=\nu E$.  \Bang\Qed}

\noindent By the guarantee of uniqueness in 496E, $\mu'=\mu$.
}%end of proof of 496F

\vleader{48pt}{496G}{Theorem} Let $\frak A$ be a Maharam algebra, and
$\mu$ a strictly positive Maharam submeasure on $\frak A$.
Let $Z$ be the Stone space
of $\frak A$, and write $\widehat{a}$ for the open-and-closed subset of $Z$
corresponding to each $a\in\frak A$.
Then there is a unique totally finite Radon
submeasure $\nu$ on $Z$ such that
$\nu\widehat{a}=\mu a$ for every $a\in\frak A$.
The domain of $\nu$ is the Baire-property algebra $\widehat{\Cal B}$
of $Z$, and the null ideal of $\nu$ is the nowhere dense ideal of $Z$.

\proof{ Let $\Cal E$ be the algebra of open-and-closed subsets of $Z$, and
$\Cal M$ the ideal of meager subsets of $Z$.
Because $\frak A$ is Dedekind complete (393Eb/496Bb),
$\Cal E$ is the regular open algebra of $Z$ (314S).   By 496R(b-ii),
$\widehat{\Cal B}=\{E\symmdiff F:E\in\Cal E$, $F\in\Cal M\}$.

For $a\in\frak A$, let $\widehat{a}$ be the corresponding member of
$\Cal E$.   By 314M, we have an isomorphism
$\theta:\frak A\to\widehat{\Cal B}/\Cal M$ defined by setting
$\theta(a)=\widehat{a}^{\ssbullet}$ for every $a\in\frak A$.
For $E\in\widehat{\Cal B}$, set $\nu E=\mu(\theta^{-1}E^{\ssbullet})$.
Because
$E\mapsto E^{\ssbullet}$ is a sequentially order-continuous Boolean
homomorphism (313P(b-ii)), $\nu$ is a Maharam submeasure on
$\widehat{\Cal B}$.   Because $\mu$ is strictly positive, the null ideal of
$\nu$ is $\Cal M$.

Because $\frak A$ is \wsid\ (393Eb/496Bb), $\Cal M$ is the ideal of nowhere
dense subsets of $Z$ (316I).   If $E\in\widehat{\Cal B}$,
consider $B=\{b:b\in\frak A$, $\widehat{b}\subseteq E\}$;  set $a=\sup B$
in $\frak A$.
Now $E\setminus\widehat{a}$ is nowhere dense.   \Prf\Quer\ Otherwise, there
is a non-zero $c\in\frak A$ such that
$F=\widehat{c}\setminus(E\setminus\widehat{a})$ is nowhere dense.   In this
case, the non-empty open set $\widehat{c}\setminus\overline{F}$
is included in $E\setminus\widehat{a}$ and there is a non-zero
$b\in\frak A$ such that $\widehat{b}\subseteq E\setminus\widehat{a}$.
But in this case $b\in B$ and $\widehat{b}\subseteq\widehat{a}$, which is
absurd.\ \Bang\Qed

Set $D=\{a\Bsetminus b:b\in B\}$.   Then $D$ is downwards-directed and
has infimum $0$.   Because $\mu$ is sequentially order-continuous and
$\frak A$ is ccc, $\mu$ is order-continuous (316Fc), and
$\inf_{d\in D}\mu d=0$.   Accordingly

$$\eqalign{\inf\{\nu(E\setminus K):K\subseteq E\text{ is compact}\}
&\le\inf_{b\in B}\nu(E\setminus\widehat{b})
=\inf_{b\in B}\nu(\widehat{a}\setminus\widehat{b})\cr
&=\inf_{b\in B}\mu(a\Bsetminus b)
=0.\cr}$$

\noindent Thus condition (ii) of 496C is satisfied and $\nu$ is a totally finite Radon
measure.

By 496F, $\nu$ is unique.
}%end of proof of 496G

\leader{496H}{Theorem} Let $X$ be a Hausdorff space, $\Sigma_0$ an algebra
of subsets of $X$, and $\mu_0:\Sigma_0\to\coint{0,\infty}$ an exhaustive
submeasure such that $\inf\{\mu_0(E\setminus K):K\in\Sigma_0$ is compact,
$K\subseteq E\}=0$ for every $E\in\Sigma_0$.
Then $\mu_0$ has an extension to a totally finite Radon
submeasure $\mu_1$ on $X$.

\proof{{\bf (a)}
Let $P$ be the set of all submeasures $\mu$, defined on algebras of
subsets of $X$, which extend $\mu_0$, and have the properties

\inset{($\alpha$) $\inf\{\mu(E\setminus K):K\in\dom\mu$ is compact,
$K\subseteq E\}=0$ for every $E\in\dom\mu$,

(*) for every $E\in\dom\mu$ and $\epsilon>0$ there is an $F\in\Sigma_0$
such that $\mu(E\symmdiff F)\le\epsilon$.}

\noindent Order $P$ by extension of functions, so that $P$ is a partially
ordered set.

\medskip

{\bf (b)} If $\mu\in P$, then $\mu$ is exhaustive.   \Prf\Quer\ Otherwise,
let $\sequencen{E_n}$ be a disjoint sequence in $\dom\mu$ such that
$\epsilon=\inf_{n\in\Bbb N}\mu E_n$ is greater than $0$.   For each
$n\in\Bbb N$, let $F_n\in\Sigma_0$ be such that
$\mu(E_n\symmdiff F_n)\le 2^{-n-2}\epsilon$;  set
$G_n=F_n\setminus\bigcup_{i<n}F_i$ for each $n$.   Then

\Centerline{$E_n\subseteq G_n\cup\bigcup_{i\le n}(E_i\symmdiff F_i)$,
\quad$\epsilon\le\mu G_n+\sum_{i=0}^n2^{-i-2}\epsilon
\le\mu_0G_n+\Bover12\epsilon$}

\noindent and $\mu_0G_n\ge\bover12\epsilon$ for every $n$.   But
$\sequencen{G_n}$ is disjoint and $\mu_0$ is supposed to be exhaustive.\
\Bang\Qed

\medskip

{\bf (c)} Suppose that $\mu\in P$ has domain $\Sigma$, and that
$V\subseteq X$ is such that

\doubleinset{$\ddagger(V,\mu)$:  for every $\epsilon>0$ there is a
$K\in\Sigma$ such that $K\cap V$ is compact and
$\mu(X\setminus K)\le\epsilon$.}

\medskip

\quad{\bf (i)} Set $\Cal H=\{H:V\subseteq H\in\Sigma\}$.   Then
$\Cal H$ is downwards-directed.   If $\epsilon>0$ there is an $H\in\Cal H$
such that $\mu(H\setminus H')\le\epsilon$ for every $H'\in\Cal H$.
\Prf\Quer\ Otherwise, there would be a non-increasing sequence
$\sequencen{H_n}$ in $\Cal H$ such that
$\mu(H_n\setminus H_{n+1})\ge\epsilon$ for every $n$;  but $\mu$ is
exhaustive, by (b).\ \Bang\Qed

\medskip

\quad{\bf (ii)} Let $\Cal F$ be the filter on $\Cal H$
generated by sets of the form $\{H':H'\in\Cal H$, $H'\subseteq H\}$ for
$H\in\Cal H$.   Then
$\lim_{H\to\Cal F}\mu((E\cap H)\cup(F\setminus H))$ is defined for all
$E$, $F\in\Sigma$.   \Prf\ Given $\epsilon>0$, then (i) tells us that there
is an $H_0\in\Cal H$ such that
$\mu(H\symmdiff H')\le\mu(H_0\setminus(H\cap H'))\le\epsilon$ whenever $H$,
$H'\in\Cal H$ are included in $H_0$.   Now, for such $H$ and $H'$,

\Centerline{
$((E\cap H)\cup(F\setminus H))\symmdiff((E\cap H')\cup(F\setminus H'))
\subseteq H\symmdiff H'$,}

\noindent so

\Centerline{$|\mu((E\cap H)\cup(F\setminus H))
-\mu((E\cap H')\cup(F\setminus H'))|\le\epsilon$.  \Qed}

\medskip

\quad{\bf (iii)} If $E$, $F$, $E'$, $F'\in\Sigma$ and
$(E\cap V)\cup(F\setminus V)=(E'\cap V)\cup(F'\setminus V)$, then

\Centerline{$\lim_{H\to\Cal F}\mu((E\cap H)\cup(F\setminus H))
=\lim_{H\to\Cal F}\mu((E'\cap H)\cup(F'\setminus H))$.}

\noindent\Prf\ Given $\epsilon>0$, there is an $H_0\in\Cal H$ such that
$\mu G\le\epsilon$ whenever $G\in\Sigma$ and $G\subseteq H_0\setminus V$,
by (i).   Now if $H\in\Cal H$ and $H\subseteq H_0$,

\Centerline{$G=((E\cap H)\cup(F\setminus H))
\symmdiff((E'\cap H)\cup(F'\setminus H))
\subseteq H\setminus V$,}

\noindent so

\Centerline{$|\mu((E\cap H)\cup(F\setminus H))
-\mu((E'\cap H)\cup(F'\setminus H))|\le\mu G\le\epsilon$.}

\noindent As $\epsilon$ is arbitrary, the limits are equal.\ \Qed

\medskip

\quad{\bf (iv)} Consequently, taking $\Sigma'$ to be the algebra
$\{(E\cap V)\cup(F\setminus V):E$, $F\in\Sigma\}$ of subsets of $X$
generated by $\Sigma\cup\{V\}$, we have a
functional $\mu':\Sigma'\to\coint{0,\infty}$ defined by saying that

\Centerline{$\mu'((E\cap V)\cup(F\setminus V))
=\lim_{H\to\Cal F}\mu((E\cap H)\cup(F\setminus H))$}

\noindent whenever $E$, $F\in\Sigma$.

\medskip

\quad{\bf (v)} $\mu'$ is a submeasure extending $\mu$.
\Prf\ If $E\in\Sigma$, then

\Centerline{$\mu'E=\mu'((E\cap V)\cup(E\setminus V))
=\lim_{H\to\Cal F}\mu((E\cap H)\cup(E\setminus H))=\mu E$,}

\noindent so $\mu'$ extends $\mu$.   If $E_1$, $E_2$, $F_1$,
$F_2\in\Sigma$, set $E=E_1\cup E_2$, $F=F_1\cup F_2$;  then

\Centerline{$((E_1\cap A)\cup(F_1\setminus A))
\cup((E_2\cap A)\cup(F_2\setminus A))
=((E\cap A)\cup(F\setminus A))$}

\noindent for every set $A$, so

$$\eqalign{\mu'\bigl(((E_1\cap V)\cup(&F_1\setminus V))
\cup((E_2\cap V)\cup(F_2\setminus V))\bigr)\cr
&=\mu'(((E\cap V)\cup(F\setminus V))\cr
&=\lim_{H\to\Cal F}\mu((E\cap H)\cup(F\setminus H))\cr
&=\lim_{H\to\Cal F}\mu\bigl(((E_1\cap H)\cup(F_1\setminus H))
   \cup((E_2\cap H)\cup(F_2\setminus H))\bigr)\cr
&\le\lim_{H\to\Cal F}\mu((E_1\cap H)\cup(F_1\setminus H))
   +\mu((E_2\cap H)\cup(F_2\setminus H))\cr
&=\mu'((E_1\cap V)\cup(F_1\setminus V))
   +\mu'((E_2\cap V)\cup(F_2\setminus V)).\cr}$$

\noindent Thus $\mu'$ is subadditive;  monotonicity is easier.\ \Qed

\medskip

\quad{\bf (vi)} $\mu'$ has the property ($\alpha$).   \Prf\ Suppose that
$E$, $F\in\Sigma$ and that $\epsilon>0$.   Let $H_0\in\Cal H$ be such that
$\mu(H_0\setminus H)\le\epsilon$ whenever $H\in\Cal H$ and
$H\subseteq H_0$.   Let $K_0\in\Sigma$ be such that
$\mu(X\setminus K_0)\le\epsilon$ and $K_0\cap V$ is compact.
Let $K_1\subseteq E$ and $K_2\subseteq F\setminus H_0$
be compact sets, belonging to $\Sigma$, such that
$\mu(E\setminus K_1)\le\epsilon$ and
$\mu((F\setminus H_0)\setminus K_2)\le\epsilon$.   Set
$K=(K_1\cap K_0\cap V)\cup K_2$, so that $K$ is a compact set belonging
to $\Sigma'$
and $K\subseteq(E\cap V)\cup(F\setminus V)$.   Now if $H\in\Cal H$ and
$H\subseteq H_0$,

$$\eqalign{\mu\bigl(((E\setminus(&K_1\cap K_0))\cap H)
   \cup((F\setminus K_2)\setminus H)\bigr)\cr
&\le\mu(E\setminus K_1)+\mu(X\setminus K_0)
   +\mu((F\setminus H_0)\setminus K_2)+\mu(H_0\setminus H)
\le 4\epsilon.\cr}$$

\noindent Taking the limit along $\Cal F$,

\Centerline{$\mu'\bigl(((E\cap V)\cup(F\setminus V))\setminus K\bigr)
=\mu'\bigl(((E\setminus(K_1\cap K_0))\cap V)
   \cup((F\setminus K_2)\setminus V)\bigr)
\le 4\epsilon$.}

\noindent As $E$, $F$ and $\epsilon$ are arbitrary, we have the result.\
\Qed

\medskip

\quad{\bf (vii)} $\mu'$ has the property (*).   \Prf\
Suppose that
$E$, $F\in\Sigma$ and that $\epsilon>0$.   Let $H_0\in\Cal H$ be such that
$\mu(H_0\setminus H)\le\epsilon$ whenever $H\in\Cal H$ and
$H\subseteq H_0$.   Set $G=(E\cap H_0)\cup(F\setminus H_0)\in\Sigma$.
Then

\Centerline{$((E\cap V)\cup(F\setminus V))\symmdiff G
\subseteq H_0\setminus V$,}

\noindent so

\Centerline{$\mu'\bigl(((E\cap V)\cup(F\setminus V))\symmdiff G\bigr)
\le\mu'(H_0\setminus V)
=\lim_{H\to\Cal F}\mu(H_0\setminus H)\le\epsilon$.  \Qed}

\medskip

{\bf (d)(i)} If $\mu\in P$ and $V\in\Cal N(\mu)$, then $\ddagger(V,\mu)$
is true.   \Prf\ Let $\epsilon>0$.
There is an $E\in\dom\mu$, including $V$, such that
$\mu E=0$;   now there is a compact $K\in\dom\mu$, included in
$X\setminus E$, such that

\Centerline{$\epsilon\ge\mu((X\setminus E)\setminus K)=\mu(X\setminus K)$,}

\noindent while $K\cap V=\emptyset$ is compact.\ \Qed

\medskip

\quad{\bf (ii)} If $\mu\in P$ and $V\subseteq X$ is closed, then
$\ddagger(V,\mu)$ is true.   \Prf\ For every $\epsilon>0$, there is a
compact $K\in\dom\mu$ such that $\mu(X\setminus K)\le\epsilon$,
and now $K\cap V$ is compact.\ \Qed

\medskip

\quad{\bf (iii)} Now suppose that $\mu\in P$ is such that every compact
subset of $X$ belongs to the domain $\Sigma$ of $\mu$, and that
$\sequencen{E_n}$ is a sequence in $\Sigma$ with intersection $V$.   Then
$\ddagger(V,\mu)$ is true.   \Prf\ Let $\epsilon>0$.   For each
$n\in\Bbb N$, there are compact sets
$K_n\subseteq E_n$, $K'_n\subseteq X\setminus E_n$ such that

\Centerline{$\mu(E_n\setminus K_n)+\mu((X\setminus E_n)\setminus K'_n)
\le 2^{-n-1}\epsilon$.}

\noindent Set $K=\bigcap_{n\in\Bbb N}K_n\cup K'_n$;  then $K$ is compact,
so belongs to $\Sigma$.   If $L\subseteq X\setminus K$ is compact, then
there is an $n\in\Bbb N$ such that $L\cap\bigcap_{i\le n}K_i\cup K'_i$
is empty, so that

\Centerline{$\mu L\le\sum_{i=0}^n\mu(X\setminus(K_i\cup K'_i))
\le\sum_{i=0}^n2^{-n-1}\epsilon\le\epsilon$.}

\noindent As $L$ is arbitrary, $\mu(X\setminus K)\le\epsilon$.   Finally,

\Centerline{$K\cap V=\bigcap_{n\in\Bbb N}(K_n\cup K'_n)\cap E_n
=\bigcap_{n\in\Bbb N}K_n$}

\noindent is compact.\ \Qed

\medskip

{\bf (e)} If $Q\subseteq P$ is a non-empty totally ordered subset of $P$,
$\bigcup Q\in P$.   So $P$ has a maximal element $\mu_1$,
which is a submeasure,
satisfying ($\alpha$), and extending $\mu_0$.
Setting $\Sigma_1=\dom\mu_1$, (c) tells us that $V\in\Sigma_1$ whenever
$V\subseteq X$ and $\ddagger(V,\mu_1)$ is true.   By (d-i),
$\Cal N(\mu_1)\subseteq\Sigma_1$ and $\mu_1$ is complete.   By (d-ii),
every closed set, and therefore every open set, belongs to $\Sigma_1$.
So (d-iii) tells us that
$\bigcap_{n\in\Bbb N}E_n\in\Sigma_1$ for every sequence $\sequencen{E_n}$
in $\Sigma_1$, and $\Sigma_1$ is a $\sigma$-algebra.   Putting these
together, all the conditions of 496C are satisfied, and $\mu_1$ is a totally finite Radon
submeasure.
}%end of proof of 496H

\leader{496I}{Theorem} Let $X$ be a set, $\Sigma$ a $\sigma$-algebra of
subsets of $X$, and $\mu$ a complete Maharam submeasure on
$\Sigma$.

(a) $\Sigma$ is closed under Souslin's operation.

(b) If $A$ is the kernel of a Souslin scheme
$\family{\sigma}{S}{E_{\sigma}}$ in $\Sigma$, and $\epsilon>0$, there is a
$\psi\in\NN$ such that

\Centerline{$\mu(A\setminus\bigcup_{\phi\in\NN,\phi\le\psi}
   \bigcap_{n\ge 1}E_{\phi\restr n})\le\epsilon$.}

\proof{{\bf (a)} Let $\Cal N(\mu)$ be the null ideal of $\mu$.
Because $\mu$ is exhaustive, every disjoint sequence in
$\Sigma\setminus\Cal N(\mu)$ is countable, so 431G tells us that
$\Sigma$ is closed under Souslin's operation.

\medskip

{\bf (b)} The argument of 431D applies, with trifling modifications in its
expression.   For $\sigma\in S^*=\bigcup_{k\in\Bbb N}\BbbN^k$, set
$A_{\sigma}
=\bigcup_{\sigma\subseteq\phi\in\NN}\bigcap_{n\ge 1}E_{\phi\restr n}$;
then $A_{\sigma}\in\Sigma$, by (a).   Given $\epsilon>0$, let
$\family{\sigma}{S^*}{\epsilon_{\sigma}}$ be a family of strictly positive
real numbers such that $\sum_{\sigma\in S^*}\epsilon_{\sigma}\le\epsilon$.
For each $\sigma\in S^*$, let $m_{\sigma}$ be such that
$\mu(A_{\sigma}\setminus\bigcup_{i\le m_{\sigma}}
  A_{\sigma^{\smallfrown}\fraction{i}})\le\epsilon_{\sigma}$.
Set

\Centerline{$\psi(k)
=\max\{m_{\sigma}:\sigma\in\BbbN^k$, $\sigma(i)\le\psi(i)$
for every $i<k\}$}

\noindent for $k\in\Bbb N$;  then

\Centerline{$A\setminus\bigcup_{\phi\in\NN,\phi\le\psi}
    \bigcap_{n\ge 1}E_{\phi\restr n}
\subseteq\bigcup_{\sigma\in S^*}\bigl(A_{\sigma}
  \setminus\bigcup_{i\le m_{\sigma}}A_{\sigma^{\smallfrown}\fraction{i}}
  \bigr)$}

\noindent has submeasure at most $\epsilon$.
}%end of proof of 496I

\leader{496J}{Theorem} Let $X$ be a K-analytic
Hausdorff space and $\mu$ a
Maharam submeasure defined on the Borel $\sigma$-algebra of $X$.   Then

\Centerline{$\inf\{\mu(X\setminus K):K\subseteq X$ is compact$\}=0$.}

\proof{ Again, we have only to re-use the ideas of 432B.
Let $\hat\mu$ be the completion of $\mu$ (496A) and $\Sigma$ the
domain of $\hat\mu$.   Let $R\subseteq\NN\times X$ be an usco-compact
relation such that $R[\NN]=X$.   For
$\sigma\in S^*=\bigcup_{n\ge 1}\BbbN^n$
set $I_{\sigma}=\{\phi:\sigma\subseteq\phi\in\NN\}$,
$F_{\sigma}=\overline{R[I_{\sigma}]}$;  then $X$ is the kernel of the
Souslin scheme $\family{\sigma}{S^*}{F_{\sigma}}$.   Now, given $\epsilon>0$,
496Ib tells us that
there is a $\psi\in\NN$ such that $\mu(X\setminus F)\le\epsilon$, where
$F=\bigcup_{\phi\in\NN,\phi\le\psi}\bigcap_{n\ge 1}F_{\phi\restr n}$;
but $F=R[K]$ where $K$ is the compact set
$\{\phi:\phi\in\NN$, $\phi\le\psi\}$, so $F$ is compact.
}%end of proof of 496J

\leader{496K}{Proposition} Let $\mu$ be a Maharam submeasure
on the Borel $\sigma$-algebra of an analytic Hausdorff space $X$.
Then the completion of $\mu$ is a totally finite Radon submeasure on $X$.

\proof{ If $E\subseteq X$ is Borel, then it is K-analytic (423Eb);
applying 496J to $\mu\restr\Cal PE$,
we see that $\inf\{\mu(E\setminus K):K\subseteq E$ is compact$\}=0$.
Consequently, writing $\Sigma$ for the domain of the completion $\hat\mu$
of $\mu$,
$\inf\{\hat\mu(E\setminus K):K\subseteq E$ is compact$\}=0$ for every
$E\in\Sigma$.   Condition (i) of the definition 496C is surely
satisfied by $\hat\mu$, so $\hat\mu$ is a totally finite Radon submeasure.
}%end of proof of 496K

\leader{496L}{Free products of Maharam algebras} If $\frak A$, $\frak B$
are Boolean algebras with submeasures $\mu$, $\nu$ respectively, we have a
submeasure $\mu\ltimes\nu$ on the free product
$\frak A\otimes\frak B$\cmmnt{ (392K)}.
It is easy to see\cmmnt{, in 392K,} that
if $\mu$ and $\nu$ are strictly positive so is $\mu\ltimes\nu$;  moreover,
if $\mu$ and $\nu$ are exhaustive so is $\mu\ltimes\nu$\cmmnt{ (392Ke)}.

Now suppose that $\familyiI{\frak A_i}$ is a family of
Maharam algebras, where $I$ is a finite totally ordered set.   Then
we can take a strictly positive Maharam submeasure $\mu_i$ on
each $\frak A_i$, form an exhaustive submeasure $\lambda$ on
$\frak C_I=\bigotimes_{i\in I}\frak A_i$, and use $\lambda$ to
construct a metric completion $\widehat{\frak C}_I$ which is a
Maharam algebra\cmmnt{, as in 393H/496Bb}.
\cmmnt{(If $I=\{i_0,\ldots,i_n\}$ where $i_0<\ldots<i_n$, then
$\lambda=(.(\mu_{i_0}\ltimes\mu_{i_1})\ltimes\ldots)\ltimes\mu_{i_n}$
(392Kf).   By 392Kc, the product is associative, so the
arrangement of the brackets is immaterial.)}
If we change each $\mu_i$ to $\mu'_i$, where $\mu'_i$ is
another strictly positive Maharam submeasure on
$\frak A_i$, then every $\mu'_i$ is absolutely continuous with
respect to $\mu_i$\cmmnt{ (393F/496Bd)},
so the corresponding $\lambda'$ will be
absolutely continuous with respect to $\lambda$, and vice
versa\cmmnt{ (392Kd)};
in which case the metrics on $\frak C_I$ are uniformly equivalent
and we get the same metric completion
$\widehat{\frak C}_I$ up to Boolean algebra isomorphism.
We can therefore think of $\widehat{\frak C}_I$ as `the' {\bf Maharam
algebra free product} of the family $\familyiI{\frak A_i}$ of Boolean
algebras;  \cmmnt{as in 392Kf,} we shall have an isomorphism between
$\widehat{\frak C}_{J\cup K}$ and the Maharam algebra
free product of $\widehat{\frak C}_J$ and $\widehat{\frak C}_K$
whenever $J$, $K\subseteq I$ and $j<k$ for every $j\in J$ and $k\in K$.

\dvro{If}{From 392Kg we see that if} $(\frak A,\mu)$ and $(\frak B,\nu)$
are probability algebras, then
their Maharam algebra free product, regarded as a Boolean algebra, is
isomorphic to their probability algebra free product\cmmnt{ as defined in
\S325}.

\leader{496M}{}{\bf Representing products of Maharam algebras:
Theorem} Let $X$
and $Y$ be sets, with $\sigma$-algebras $\Sigma$ and $\Tau$ and Maharam
submeasures $\mu$ and $\nu$ defined on $\Sigma$, $\Tau$ respectively.
Let $\frak A$, $\frak B$ be their Maharam algebras and write $\bar\mu$,
$\bar\nu$ for the strictly positive Maharam submeasures on $\frak A$ and
$\frak B$ induced by $\mu$ and $\nu$\cmmnt{ as in 496Ba above}.   Let
$\Sigma\tensorhat\Tau$ be the $\sigma$-algebra of subsets of $X\times Y$
generated by $\{E\times F:E\in\Sigma$, $F\in\Tau\}$.

(a)\cmmnt{ (Compare 418T.)}
Give $\frak B$ its Maharam-algebra topology\cmmnt{ (393G/496Bd)}.
If $W\in\Sigma\tensorhat\Tau$ then $W[\{x\}]\in\Tau$ for every $x\in X$
and the function $x\mapsto W[\{x\}]^{\ssbullet}:X\to\frak B$ is
$\Sigma$-measurable and has separable range.   \cmmnt{Consequently}
$x\mapsto\nu W[\{x\}]:X\to\coint{0,\infty}$ is $\Sigma$-measurable.

(b) For $W\in\Sigma\tensorhat\Tau$ set

\Centerline{$\lambda W
=\inf\{\epsilon:\epsilon>0$,
  $\mu\{x:\nu W[\{x\}]>\epsilon\}\le\epsilon\}$.}

\noindent Then $\lambda$ is a Maharam submeasure on $\Sigma\tensorhat\Tau$,
and

\Centerline{$\lambda^{-1}[\{0\}]
=\{W:W\in\Sigma\tensorhat\Tau$,
   $\{x:W[\{x\}]\notin\Cal N(\nu)\}\in\Cal N(\mu)\}$.}

(c) Let $\frak C$ be the Maharam algebra of $\lambda$.   Then
$\frak A\otimes\frak B$ can be embedded in $\frak C$ by mapping
$E^{\ssbullet}\otimes F^{\ssbullet}$ to $(E\times F)^{\ssbullet}$ for all
$E\in\Sigma$ and $F\in\Tau$.

(d) This embedding identifies $(\frak C,\bar\lambda)$ with the metric
completion of $(\frak A\otimes\frak B,\bar\mu\ltimes\bar\nu)$.

\proof{{\bf (a)} Write $\Cal W$ for the set of those
$W\subseteq X\times Y$ such that $W[\{x\}]\in\Tau$ for every $x\in X$ and
$x\mapsto W[\{x\}]^{\ssbullet}:X\to\frak B$ is
$\Sigma$-measurable and has separable range.
Then $\Sigma\otimes\Tau$
(identified with the algebra of subsets of $X\times Y$ generated by
$\{E\times F:E\in\Sigma$, $F\in\Tau\}$) is included in $\Cal W$.

If $\sequencen{W_n}$ is a non-decreasing sequence in $\Cal W$ with union
$W$, then $W\in\Cal W$.   \Prf\ Of course
$W[\{x\}]=\bigcup_{n\in\Bbb N}W_n[\{x\}]$ belongs to $\Tau$ for every
$x\in X$.  Set $f_n(x)=W_n[\{x\}]^{\ssbullet}$ for $n\in\Bbb N$ and
$x\in X$.   For each $x\in X$, $W[\{x\}]\setminus W_n[\{x\}]$ is a
non-increasing sequence with empty intersection, so
$\lim_{n\to\infty}\nu(W[\{x\}]\setminus W_n[\{x\}])=0$ and
$\sequencen{f_n(x)}$
converges to $f(x)=W[\{x\}]^{\ssbullet}$ in $\frak B$.
By 418Ba, $f$ is measurable.  Also
$D=\overline{\{f_n(x):x\in X,\,n\in\Bbb N\}}$
is a separable subspace of $\frak B$
including $f[X]$.   So $W\in\Cal W$.\ \Qed

Similarly, $\bigcap_{n\in\Bbb N}W_n\in\Cal W$ for any non-increasing
sequence $\sequencen{W_n}$ in $\Cal W$.   $\Cal W$ therefore includes the
$\sigma$-algebra generated
by $\Sigma\otimes\Tau$ (136G), which is $\Sigma\tensorhat\Tau$.

Now $x\mapsto\nu W[\{x\}]=\bar\nu W[\{x\}]^{\ssbullet}$ is measurable
because $\bar\nu:\frak B\to\Bbb R$ is continuous.

\medskip

{\bf (b)} Of course $\lambda\emptyset=0$ and
$\lambda W\le\lambda W'$ if $W$, $W'\in\Sigma\tensorhat\Tau$
and $W\subseteq W'$.   If $W_1$, $W_2\in\Sigma\tensorhat\Tau$ have union
$W$, $\lambda W_1=\alpha_1$ and
$\lambda W_2=\alpha_2$, then

\Centerline{$\{x:\nu W[\{x\}]>\alpha_1+\alpha_2\}
\subseteq\{x:\nu W_1[\{x\}]>\alpha_1\}\cup\{x:\nu W_2[\{x\}]>\alpha_2\}$,}

\noindent so, setting $\alpha=\alpha_1+\alpha_2$,

\Centerline{$\mu\{x:\nu W[\{x\}]>\alpha\}
\le\mu\{x:\nu W_1[\{x\}]>\alpha_1\}+\mu\{x:\nu W_2[\{x\}]>\alpha_2\}
\le\alpha_1+\alpha_2=\alpha$,}

\noindent and $\lambda W\le\alpha$.   Thus $\lambda$ is monotonic and subadditive.

If now $\sequencen{W_n}$ is a non-increasing sequence in $\Sigma\tensorhat\Tau$ with
empty intersection, and $\epsilon>0$, set $E_n=\{x:\nu W_n[\{x\}]\ge\epsilon\}$ for
each $n$.   Then $\sequencen{E_n}$ is non-increasing;  moreover, for any $x\in X$,
$\sequencen{W_n[\{x\}]}$ is a non-increasing sequence in $\Tau$ with empty
intersection, so $\lim_{n\to\infty}\nu W_n[\{x\}]=0$ and
$x\notin\bigcap_{n\in\Bbb N}E_n$.   There is therefore an $n$ such that
$\mu E_n\le\epsilon$ and $\lambda W_n\le\epsilon$.   As $\sequencen{W_n}$ and
$\epsilon$ are arbitrary, $\lambda$ is a Maharam submeasure.

Finally, for $W\in\Sigma\tensorhat\Tau$,

$$\eqalign{\lambda W=0
&\iff\mu\{x:\nu W[\{x\}]\ge 2^{-n}\}\le 2^{-n}\text{ for every }n\in\Bbb N\cr
&\iff\mu\{x:\nu W[\{x\}]\ge 2^{-m}\}\le 2^{-n}\text{ for every }m,\,n\in\Bbb N\cr
&\iff\mu\{x:\nu W[\{x\}]>0\}\le 2^{-n}\text{ for every }n\in\Bbb N\cr
&\iff\mu\{x:\nu W[\{x\}]>0\}=0
\iff\{x:W[\{x\}]\notin\Cal N(\nu)\}\in\Cal N(\mu).\cr}$$

\medskip

{\bf (c)} If $E\in\Sigma$, then $\lambda(E\times F)=\min(\mu E,\nu F)$ for
all $E\in\Sigma$ and $F\in\Tau$.   So $\lambda(E\times F)=0$ iff
$E^{\ssbullet}\otimes F^{\ssbullet}=0$ in $\frak A\otimes\frak B$.
Consequently we have injective Boolean homomorphisms
from $\frak A$ to $\frak C$ and from $\frak B$ to $\frak C$ defined by the
formulae

\Centerline{$E^{\ssbullet}\mapsto(E\times Y)^{\ssbullet}$ for $E\in\Sigma$,
\quad$F^{\ssbullet}\mapsto(X\times F)^{\ssbullet}$ for $F\in\Tau$;}

\noindent by 315J and 315Kb\formerly{3{}15I-3{}15J}, 
we have an injective Boolean homomorphism from
$\frak A\otimes\frak B$ to $\frak C$ which maps
$E^{\ssbullet}\otimes F^{\ssbullet}$ to $(E\times F)^{\ssbullet}$ whenever
$E\in\Sigma$ and $F\in\Tau$.

\medskip

{\bf (d)} $\bar\lambda(\phi e)=(\mu\ltimes\nu)(e)$ for
every $e\in\frak A\otimes\frak B$.   \Prf\ Express $e$ as
$\sup_{i\in I}a_i\otimes b_i$ where $\familyiI{a_i}$ is a finite partition of unity in
$\frak A$ and $b_i\in\frak B$ for each $i$.   For each $i$, we can express
$a_i$, $b_i$ as $E_i^{\ssbullet}$, $F_i^{\ssbullet}$ where $E_i\in\Sigma$
and $F_i\in\Tau$;  moreover, we can do this in such a way that
$\familyiI{E_i}$ is a
partition of $X$.   In this case, $\phi e=W^{\ssbullet}$ where
$W=\bigcup_{i\in I}E_i\times F_i$, so that, for $\epsilon>0$,

\Centerline{$\mu\{x:\nu W[\{x\}]>\epsilon\}
=\mu(\bigcup\{E_i:i\in I,\,\nu F_i>\epsilon\})
=\bar\mu(\sup\{a_i:i\in I,\,\bar\nu b_i>\epsilon\})$.}

\noindent Accordingly

$$\eqalign{(\mu\ltimes\nu)(e)
&=\inf\{\epsilon:\bar\mu(\sup\{a_i:i\in I,\,\bar\nu b_i>\epsilon\})\le\epsilon\}\cr
&=\inf\{\epsilon:\mu\{x:\nu W[\{x\}]>\epsilon\}\le\epsilon\}
=\lambda W
=\bar\lambda W^{\ssbullet}
=\bar\lambda(\phi e). \text{ \Qed}\cr}$$

Next, $\phi[\frak A\otimes\frak B]$ is dense in $\frak C$ for the metric induced by
$\bar\lambda$.   \Prf\ Let $\frak D$ be the metric closure of
$\phi[\frak A\otimes\frak B]$ and set
$\Cal V=\{V:V\in\Sigma\otimes\Tau$, $V^{\ssbullet}\in\frak D\}$.   Then
$\Cal V$ includes $\Sigma\otimes\Tau$ and is closed under unions and
intersections of monotonic sequences, so is the whole of
$\Sigma\tensorhat\Tau$, and $\frak D=\frak C$, as required.\ \QeD\   But
this means that we can identify $\frak C$ with the metric
completions of $\phi[\frak A\otimes\frak B]$ and
$\frak A\otimes\frak B$.
}%end of proof of 496M

\exercises{\leader{496X}{Basic exercises (a)}
%\spheader 496Xa
Let $X$ and $Y$ be Hausdorff spaces, $\mu$ a totally finite Radon
submeasure on $X$, and $f:X\to Y$ a function which is almost continuous in
the sense that for every $\epsilon>0$ there is a compact $K\subseteq X$
such that $f\restr K$ is continuous and
$\mu(X\setminus K)\le\epsilon$.   Show that the image submeasure
$\mu f^{-1}$, defined on $\{F:F\subseteq Y$, $f^{-1}[F]\in\dom\mu\}$,
is a totally finite Radon submeasure on $Y$.
%496C

\spheader 496Xb Let $X$ be a Hausdorff space and $\mu$ a totally finite
Radon submeasure on $X$.   For $A\subseteq X$, set
$\mu^*A=\inf\{\mu E:A\subseteq E\in\dom\mu\}$.   Show that $\mu^*$ is an
outer regular Choquet capacity on $X$.
%496C

\spheader 496Xc Let $X$ and $Y$ be compact Hausdorff spaces,
$f:X\to Y$ a continuous surjection, and $\nu$ a totally finite Radon
submeasure on $Y$.
Show that there is a totally finite Radon submeasure $\mu$ on $X$ such
that $\nu$ is the image submeasure $\mu f^{-1}$.
%496H

\spheader 496Xd Let $X$ be a
regular K-analytic Hausdorff space, and $\mu$ a
Maharam submeasure on the Borel $\sigma$-algebra of $X$
which is $\tau$-additive in the sense that whenever $\Cal G$ is a non-empty
upwards-directed family of open sets in $X$ with union $H$, then
$\inf_{G\in\Cal G}\mu(H\setminus G)=0$.   Show that the completion of $\mu$
is a totally finite Radon submeasure on $X$.  \Hint{let $\Sigma_0$ be the algebra of
subsets of $X$ generated by the compact sets;  show that there is a totally finite Radon
submeasure extending $\mu\restr\Sigma_0$.}
%496H, 496J

\leader{496Y}{Further exercises}
In the following exercises, I will say that a {\bf Radon submeasure} is a
complete submeasure $\mu$ on a
Hausdorff space $X$ such that (i) the domain $\Sigma$ of $\mu$ is a
$\sigma$-algebra of subsets of $X$ containing every open set (ii) every
point of $X$ belongs to an open set $G$ such that $\mu G<\infty$
(iii)($\alpha$) $\mu E=\sup\{\mu K:K\subseteq E$ is compact$\}$ for every
$E\in\Sigma$ ($\beta$)
$\inf\{\mu(E\setminus K):K\subseteq E$ is compact$\}=0$ whenever
$E\in\Sigma$ and $\mu E<\infty$ (iv) if $E\subseteq X$ is such that
$E\cap K\in\Sigma$ for every compact $K\subseteq X$, then $E\in\Sigma$.

\spheader 496Ya Let $\mu$ be a Radon submeasure with domain $\Sigma$ and
null ideal $\Cal N(\mu)$.   Show that $\Sigma/\Cal N(\mu)$ is Dedekind
complete.

\spheader 496Yb
Let $X$ be a Hausdorff space, $Y$ a metrizable space,
$\mu$ a Radon submeasure on $X$ with domain $\Sigma$, and $f:X\to Y$ a
$\Sigma$-measurable
function.   Let $\Cal H$ be the family of those $H\in\Sigma$ such that
$f\restr H$ is continuous.   Show that ($\alpha$)
$\mu E=\sup\{\mu H:H\in\Cal H$, $H\subseteq E\}$ for every $E\in\Sigma$
($\beta$) $\inf\{\mu(E\setminus H):H\in\Cal H$, $H\subseteq E\}=0$ whenever
$E\in\Sigma$ and $\mu E<\infty$.
%496C

\spheader 496Yc
Let $X$ and $Y$ be Hausdorff spaces, $\mu$ a Radon
submeasure on $X$ with domain $\Sigma$, and $f:X\to Y$ a function.
Let $\Cal F$ be the family of those $F\in\Sigma$ such that
$f\restr F$ is continuous, and suppose that ($\alpha$)
$\mu E=\sup\{\mu F:F\in\Cal F$, $F\subseteq E\}$ for every $E\in\Sigma$
($\beta$) $\inf\{\mu(E\setminus F):F\in\Cal F$, $F\subseteq E\}=0$ whenever
$E\in\Sigma$ and $\mu E<\infty$.
(i) Show that the image submeasure
$\nu=\mu f^{-1}$, defined on $\{F:F\subseteq Y$, $f^{-1}[F]\in\Sigma\}$,
is a submeasure on $Y$ defined on a $\sigma$-algebra of sets
containing every open subset of $Y$.
(ii) Show that if $\nu$ is locally finite in the sense that
$Y=\bigcup\{H:H\subseteq Y$ is open, $\nu H<\infty\}$, then $\nu$ is a
Radon submeasure.
%496Xa 496C

\spheader 496Yd Let $X$ be a Hausdorff space and $\mu$ a Radon submeasure
on $X$ which is either submodular or supermodular.   Show that there is a
Radon measure on $X$ with the same domain and null ideal as $\mu$.
\Hint{413Yf.}
%496C

\spheader 496Ye Let $X$ be a topological space, $\Cal G$ the family of
cozero subsets of $X$, $\CalBa$ the Baire $\sigma$-algebra of $X$ and
$\psi:\Cal G\to\coint{0,\infty}$ a functional.   Show that $\psi$ can be
extended to a Maharam submeasure with domain $\CalBa$ iff

\inset{($\alpha$)
$\psi G\le\psi H$ whenever $G$, $H\in\Cal G$ and $G\subseteq H$,

($\beta$)
$\psi(\bigcup_{n\in\Bbb N}G_n)\le\sum_{n=0}^{\infty}\psi G_n$ for every
sequence $\sequencen{G_n}$ in $\Cal G$,

($\gamma$) $\lim_{n\to\infty}\psi G_n=0$ for every non-increasing
sequence $\sequencen{G_n}$ in $\Cal G$ with empty intersection,}

\noindent and that in this case the extension is unique.   \Hint{consider
the family of sets $E\subseteq X$ such that for every $\epsilon>0$ there
are a cozero set $G\supseteq E$ and a zero set $F\subseteq E$ such that
$\psi(G\setminus F)\le\epsilon$.}
%496E mt49bits

\spheader 496Yf Let $X$ be a Hausdorff space and $\Cal K$ the
family of
compact subsets of $X$.   Let $\phi:\Cal K\to\coint{0,\infty}$ be a
functional such that

\inset{($\alpha$) $\phi\emptyset=0$ and
$\phi K\le\phi(K\cup L)\le\phi K+\phi L$ for all $K$,
$L\in\Cal K$;

($\beta$) for every disjoint sequence $\sequencen{K_n}$ in $\Cal K$, either
$\lim_{n\to\infty}\phi K_n=0$ or
$\lim_{n\to\infty}\phi(\bigcup_{i\le n}K_i)=\infty$;

($\gamma$) whenever $K$, $L\in\Cal K$ and $K\subseteq L$ then
$\phi L\le\phi K+\sup\{\phi K':K'\in\Cal K$, $K'\subseteq L\setminus K\}$;

($\delta$) for every $x\in X$ there is an open set $G$ containing $x$ such
that $\sup\{\phi K:K\in\Cal K$, $K\subseteq G\}$ is finite.
}

\noindent Show that there is a unique
Radon submeasure on $X$ extending $\phi$.
%496E
}%end of exercises

\endnotes{
\Notesheader{496} `Submeasures' turn up in all sorts of places, if you are
looking out for them;  so, as always, I have tried to draw my definitions
as wide as practicable.   When we come to `Maharam' and `Radon'
submeasures, however, we certainly want to begin with results corresponding
to the familiar properties of totally finite measures, and the new language
is complex enough without troubling with infinite submeasures.   For the
main part of this section, therefore,
I look only at totally finite submeasures.

I have tried here to give a sample of the ideas from
the present volume which can be applied to submeasures as well as to
measures.   I think they go farther than most of us would take for granted.
One key point concerns the definition of inner regularity:  to the familiar
`$\mu E=\sup\{\mu K:K\in\Cal K$, $K\subseteq E\}$' we need to add
`if $\mu E$ is finite, then
$\inf\{\mu(E\setminus K):K\in\Cal K$, $K\subseteq E\}=0$' (496C, 496Y;  see
also condition ($\beta$) of 496Yf).   Using this refinement, we
can repeat a good proportion of the arguments of measure theory which are
based on topology and orderings rather than on arithmetic identities.
}%end of endnotes

\discrpage

