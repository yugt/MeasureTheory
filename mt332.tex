\frfilename{mt332.tex}
\versiondate{19.3.05}
\copyrightdate{1996}

\def\chaptername{Maharam's theorem}
\def\sectionname{Classification of localizable measure algebras}

\newsection{332}

In this section I present what I call `Maharam's theorem', that every
localizable measure algebra is expressible as a weighted simple product
of measure algebras of spaces of the form $\{0,1\}^{\kappa}$ (332B).
Among its many
consequences is a complete description of the isomorphism classes of
localizable measure algebras (332J).   This description needs the
concepts of `cellularity' of a Boolean algebra (332D) and its
refinement, the `magnitude' of a measure algebra (332G).   I end this
section with a discussion of those pairs of measure algebras for which
there is a measure-preserving homomorphism from one to the other
(332P-332Q), and a general formula for the Maharam type of a localizable
measure algebra (332S).

\leader{332A}{Lemma} Let $\frak A$ be any Boolean algebra.   Writing
$\frak A_a$ for the principal ideal generated by $a\in\frak A$,
\cmmnt{the set }
$\{a:a\in\frak A,\,\frak A_a$ is \Mth$\}$ is order-dense in $\frak A$.

\proof{ Take any $a\in\frak A\setminus\{0\}$.   Then
$A=\{\tau(\frak A_b):0\ne b\Bsubseteq a\}$ has a least member;
take $c\Bsubseteq a$ such that $c\ne 0$ and $\tau(\frak A_c)=\min A$.
If $0\ne b\Bsubseteq c$, then $\tau(\frak A_b)\le\tau(\frak A_c)$, by 331Hc,
while $\tau(\frak A_b)\in A$, so
$\tau(\frak A_c)\le\tau(\frak A_b)$.   Thus
$\tau(\frak A_b)=\tau(\frak A_c)$ for every non-zero $b\Bsubseteq c$, and
$\frak A_c$ is \Mth.
}%end of proof of 332A


\leader{332B}{Maharam's theorem} Let $(\frak A,\bar\mu)$ be a localizable
measure algebra.   Then it is isomorphic to the simple product of a
family $\langle(\frak A_i,\bar\mu_i)\rangle_{i\in I}$ of measure
algebras, where for each $i\in I\,\,(\frak A_i,\bar\mu_i)$ is isomorphic,
up to a re-normalization of
the measure, to the measure algebra of the usual measure on
$\{0,1\}^{\kappa_i}$, where $\kappa_i$ is either $0$ or an infinite
cardinal.

\proof{{\bf (a)} For $a\in\frak A$, let $\frak A_a$ be the principal
ideal of $\frak A$ generated by $a$.   Then

\Centerline{$D=\{a:a\in\frak A,\,0<\bar\mu a<\infty,\,\frak A_a$ is
\Mth$\}$}

\noindent is order-dense in $\frak A$.    \Prf\  If
$a\in\frak A\setminus\{0\}$, then (because $(\frak A,\bar\mu)$ is
semi-finite)
there is a $b\Bsubseteq a$ such that $0<\bar\mu b<\infty$;  now by 332A
there is
a non-zero $d\Bsubseteq b$ such that $\frak A_d$ is \Mth.\   \Qed

\medskip

{\bf (b)} By 313K, there is a partition of unity
$\langle e_i\rangle_{i\in I}$ consisting of members of $D$;
by 322L(d-i), $(\frak A,\bar\mu)$ is isomorphic, as measure algebra,
to the simple product of the principal ideals $\frak A_i=\frak A_{e_i}$.

\medskip

{\bf (c)} For each $i\in I$, $(\frak A_i,\bar\mu_i)$ is a non-trivial
totally finite \Mth\ measure algebra, writing
$\bar\mu_i=\bar\mu\restrp\frak A_i$.   Take
$\gamma_i=\bar\mu_i(1_{\frak A_i})=\bar\mu e_i$, and set
$\bar\mu_i'=\gamma_i^{-1}\bar\mu_i$.   Then $(\frak A_i,\bar\mu'_i)$
is a \Mth\ probability algebra, so by 331L is isomorphic
to the measure algebra $(\frak B_{\kappa_i},\bar\nu_{\kappa_i})$ of the
usual measure on $\{0,1\}^{\kappa_i}$, where $\kappa_i$ is either $0$
or an infinite cardinal.   Thus $(\frak A_i,\bar\mu_i)$ is isomorphic,
up to a scalar multiple of the measure, to
$(\frak B_{\kappa_i},\bar\nu_{\kappa_i})$.
}%end of proof of 332B

\cmmnt{\medskip

\noindent{\bf Remark} For the case of totally finite measure algebras,
this is Theorem 2 of {\smc Maharam 42}.
}

\leader{332C}{Corollary} Let $(\frak A,\bar\mu)$ be a localizable
measure algebra.   For any cardinal $\kappa$, write $\nu_{\kappa}$ for
the usual
measure on $\{0,1\}^{\kappa}$, and $\Tau_{\kappa}$ for its domain.
Then we can find families $\langle\kappa_i\rangle_{i\in I}$, $\langle
\gamma_i\rangle_{i\in I}$ such that every $\kappa_i$ is either $0$ or
an infinite cardinal, every $\gamma_i$ is a strictly positive real
number, and $(\frak A,\bar\mu)$ is isomorphic to the measure algebra of
$(X,\Sigma,\nu)$, where

\Centerline{$X=\{(x,i):i\in I,\,x\in\{0,1\}^{\kappa_i}\}$,}

\Centerline{$\Sigma=\{E:E\subseteq X,\,
\{x:(x,i)\in E\}\in\Tau_{\kappa_i}$ for every $i\in I\}$,}

\Centerline{$\nu E
=\sum_{i\in I}\gamma_i\nu_{\kappa_i}\{x:(x,i)\in E\}$}

\noindent for every $E\in\Sigma$.

\proof{ Take the family $\langle\kappa_i\rangle_{i\in I}$ from the last
theorem, take the $\gamma_i=\bar\mu e_i$ to be the normalizing factors
of the proof there, and apply 322Lb to identify the simple product of
the measure algebras of
$(\{0,1\}^{\kappa_i},\Tau_{\kappa_i},\gamma_i\nu_{\kappa_i})$ with
the measure algebra of their direct sum $(X,\Sigma,\nu)$.
}%end of proof of 332C


\vleader{72pt}{332D}{The cellularity of a Boolean algebra}\cmmnt{ In
order to  properly describe
non-sigma-finite measure algebras, we need the following concept.}   If
$\frak A$ is any Boolean algebra, write

\Centerline{$c(\frak A)=\sup\{\#(C):C\subseteq\frak A\setminus\{0\}$ is
disjoint$\}$,}

\noindent the {\bf cellularity} of $\frak A$.
\cmmnt{(If $\frak  A=\{0\}$, take $c(\frak A)=0$.)   Thus $\frak A$ is
ccc\cmmnt{ (316A)} iff $c(\frak A)\le\omega$.}

\leader{332E}{Proposition} Let $(\frak A,\bar\mu)$ be any semi-finite
measure algebra, and $C$ any partition of unity in $\frak A$ consisting
of elements of finite measure.   Then
$\max(\omega,\#(C))=\max(\omega,c(\frak A))$.

\proof{ Of course $\#(C\setminus\{0\})\le c(\frak A)$, because
$C\setminus\{0\}$ is disjoint, so

\Centerline{$\max(\omega,\#(C))
=\max(\omega,\#(C\setminus\{0\})\le\max(\omega,c(\frak A))$.}

Now suppose that $D$ is any disjoint set in $\frak A\setminus\{0\}$.
For $c\in C$, $\{d\Bcap c:d\in D\}$ is a disjoint set in the
principal ideal $\frak A_c$ generated by $c$.   But $\frak A_c$ is ccc
(322G), so $\{d\Bcap c:d\in D\}$ must be countable, and 
$D_c=\{d:d\in D,\,d\Bcap c\ne 0\}$ is countable.   Because $\sup C=1$,
$D=\bigcup_{c\in C}D_c$, so

\Centerline{$\#(D)\le\max(\omega,\#(C),\sup_{c\in C}\#(D_c))=\max(\omega,\#(C))$.}

\noindent As $D$ is arbitrary, $c(\frak A)\le\max(\omega,\#(C))$ and
$\max(\omega,c(\frak A))=\max(\omega,\#(C))$.
}%end of proof of 332E

\leader{332F}{Corollary} Let $(\frak A,\bar\mu)$ be any semi-finite
measure
algebra.   Then there is a disjoint set in $\frak A\setminus\{0\}$ of
cardinal $c(\frak A)$.

\proof{ Start by taking any partition of unity $C$ consisting of
non-zero elements of finite measure.   If $\#(C)=c(\frak A)$ we can
stop, because $C$ is a disjoint set in $\frak A\setminus\{0\}$.
Otherwise, by 332E, we must have $C$ finite and $c(\frak
A)=\omega$.   Let $A$ be the set of atoms in $\frak A$.   If $A$ is
infinite, it is a disjoint set of cardinal $\omega$, so we can stop.
Otherwise, since there is certainly a disjoint
set $D\subseteq\frak A\setminus\{0\}$ of cardinal greater than $\#(A)$,
and since each member of $A$ can meet at most one member of $D$, there
must be a member $d$ of $D$ which does not include any atom.
Accordingly we can choose inductively a sequence $\sequencen{d_n}$ such
that $d_0=d$, $0\ne d_{n+1}\Bsubset d_n$ for every $n$.   Now
$\{d_n\Bsetminus d_{n+1}:n\in\Bbb N\}$ is a disjoint set in $\frak
A\setminus\{0\}$ of cardinal $\omega=c(\frak A)$.
}%end of proof of 332F

\leader{332G}{Definitions}\cmmnt{ For the next theorem, it will be
convenient to have some special terminology.

\medskip

}{\bf (a)}\cmmnt{ The first word I wish to introduce is
a variant of the idea of
`cellularity', adapted to measure algebras.}   If $(\frak A,\bar\mu)$ is
a semi-finite measure algebra, let us say that the {\bf magnitude} of an
$a\in\frak A$ is $\bar\mu a$ if $\bar\mu a$ is finite, and otherwise is
the cellularity of the principal ideal $\frak A_a$ generated by $a$.
\cmmnt{(This
is necessarily infinite, since any partition of $a$ into sets of finite
measure must be infinite.)}
If we take it that any real number is less
than any infinite cardinal, then the class of possible magnitudes is
totally ordered.

I shall sometimes speak of the {\bf magnitude} of the measure algebra
$(\frak A,\bar\mu)$ itself, meaning the magnitude of $1_{\frak A}$.
Similarly, if $(X,\Sigma,\mu)$ is a semi-finite measure space, the {\bf
magnitude} of $(X,\Sigma,\mu)$, or of $\mu$, is the magnitude of its
measure algebra.

\header{332Gb}{\bf (b)} Next, for any Dedekind complete Boolean algebra
$\frak A$, and any cardinal $\kappa$, we can look at the element

\Centerline{$e_{\kappa}=\sup\{a:a\in\frak A\setminus\{0\},\,\frak A_a$
is \Mth\ with Maharam type $\kappa\}$,}

\noindent writing $\frak A_a$ for the principal ideal of $\frak A$
generated by $a$, as usual.   I will call this the
{\bf Maharam-type-$\kappa$ component} of $\frak A$.   \cmmnt{Of course
$e_{\kappa}\Bcap e_{\lambda}=0$ whenever $\lambda$, $\kappa$ are
distinct cardinals.
\prooflet{\Prf\ $a\Bcap b=0$ whenever $\frak A_a$, $\frak A_b$
are \Mth\ of different Maharam types, since
$\tau(\frak A_{a\Bcap b})$
cannot be equal simultaneously to $\tau(\frak A_a)$ and
$\tau(\frak A_b)$.\  \Qed}
}%end of comment

\cmmnt{Also} $\{e_{\kappa}:\kappa$ is a cardinal$\}$ is a partition of
unity in $\frak A$\cmmnt{, because

\Centerline{$\sup\{e_{\kappa}:\kappa$ is a cardinal$\}
=\sup\{a:\frak A_a$ is \Mth$\}=1$}

\noindent by 332A}.   \cmmnt{Note that there is no claim that
$\frak A_{e_{\kappa}}$ itself is homogeneous;  but we do have a useful
result in this direction.}

\leader{332H}{Lemma} Let $\frak A$ be a Dedekind complete Boolean
algebra and $\kappa$ an
infinite cardinal.   Let $e$ be the Maharam-type-$\kappa$ component of
$\frak A$.   If $0\ne d\Bsubseteq e$ and the principal ideal $\frak A_d$
generated by $d$ is ccc, then it is \Mth\ with Maharam
type $\kappa$.

\proof{{\bf (a)} The point is that $\tau(\frak A_d)\le\kappa$.   \Prf\
Set

\Centerline{$A=\{a:a\in\frak A\setminus\{0\},\,\frak A_a$ is \Mth\ of
Maharam type $\kappa\}$.}

\noindent Then $d=\sup\{a\Bcap d:a\in A\}$.   Because $\frak A_d$ is
ccc, there is a sequence $\sequencen{a_n}$ in $A$ such that
$d=\sup_{n\in\Bbb N}d\Bcap a_n$ (316E);  set $b_n=d\Bcap a_n$.   We have
$\tau(\frak A_{b_n})\le\tau(\frak A_{a_n})=\kappa$ for each $n$;  let
$D_n$ be a subset of $\frak A_{b_n}$, of cardinal at most $\kappa$,
which $\tau$-generates $\frak A_{b_n}$.   Set

\Centerline{$D=\bigcup_{n\in\Bbb N}D_n\cup\{b_n:n\in\Bbb
N\}\subseteq\frak A_d$.}

\noindent   If $\frak C$ is the
order-closed subalgebra of $\frak A_d$ generated by $D$, then
$\frak C\cap\frak A_{b_n}$ is an order-closed subalgebra of
$\frak A_{b_n}$
including $D_n$, so is equal to $\frak A_{b_n}$, for every $n$.   But
$a=\sup_{n\in\Bbb N}a\Bcap b_n$ for every $a\in\frak A_d$, so $\frak
C=\frak A_d$.   Thus $D\,\,\tau$-generates $\frak A_d$, and


\Centerline{$\tau(\frak A_d)\le\#(D)\le\max(\omega,\sup_{n\in\Bbb
N}\#(D_n))=\kappa$.   \Qed}

\medskip

{\bf (b)} If now $b$ is any non-zero member of $\frak A_d$, there is
some $a\in A$ such that $b\Bcap a\ne 0$, so that

\Centerline{$\kappa=\tau(\frak A_{b\Bcap a})
\le\tau(\frak A_b)\le\tau(\frak A_d)\le\kappa$.}

\noindent Thus we must have $\tau(\frak A_b)=\kappa$ for every non-zero
$b\in\frak A_d$, and $\frak A_d$ is \Mth\ with type
$\kappa$, as claimed.
}%end of proof of 332H

\leader{332I}{Lemma} Let $(\frak A,\bar\mu)$ be an atomless semi-finite
measure algebra which is not totally finite.   Then it has a partition
of unity consisting of elements of measure $1$.

\proof{ Let $A$ be the set $\{a:\bar\mu a=1\}$, and $\Cal C$ the family
of disjoint subsets of $A$.   By Zorn's lemma, $\Cal C$ has a maximal
member $C_0$ (compare the proof of 313K).
Set $D=\{d:d\in\frak A,\,d\Bcap c=0$ for every $c\in C_0\}$.   Then $D$
is upwards-directed.   If $d\in D$, then $\bar\mu a\ne 1$ for every
$a\Bsubseteq d$, so $\bar\mu d<1$, by 331C.   So $d_0=\sup D$ is defined
in $\frak A$ (321C);  of course $d_0\in D$, so $\bar\mu d_0<1$.
Observe that $\sup C_0=1\Bsetminus d_0$.

Because $\bar\mu 1=\infty$, $C_0$ must be
infinite;  let $\sequencen{a_n}$ be any sequence of distinct elements of
$C_0$.   For each $n\in\Bbb N$, use 331C again to choose an
$a'_n\Bsubseteq a_n$ such that $\bar\mu a'_n=\bar\mu d_0$.   Set

\Centerline{$b_0=d_0\Bcup(a_0\Bsetminus a'_0)$,
\quad $b_n=a'_{n-1}\Bcup(a_n\Bsetminus a'_n)$}

\noindent for every $n\ge 1$.   Then $\sequencen{b_n}$ is a disjoint
sequence of elements of measure $1$ and $\sup_{n\in\Bbb
N}b_n=\sup_{n\in\Bbb N}a_n\Bcup d_0$.    Now

\Centerline{$(C_0\setminus\{a_n:n\in\Bbb N\})\cup\{b_n:n\in\Bbb N\}$}

\noindent is a partition of unity consisting of elements of measure $1$.
}%end of proof of 332I

\leader{332J}{}\cmmnt{ Now I can formulate a complete classification
theorem for localizable measure algebras, refining the expression in
332B.

\medskip

\noindent}{\bf Theorem} Let $(\frak A,\bar\mu)$ and $(\frak B,\bar\nu)$
be localizable measure algebras.   For each cardinal $\kappa$, let
$e_{\kappa}$, $f_{\kappa}$ be the Maharam-type-$\kappa$ components of
$\frak A$, $\frak B$ respectively.   Then $(\frak A,\bar\mu)$ and
$(\frak B,\bar\nu)$ are isomorphic, as measure algebras, iff (i)
$e_{\kappa}$ and
$f_{\kappa}$ have the same magnitude for every infinite cardinal
$\kappa$ (ii) for every $\gamma\in\ooint{0,\infty}$, $(\frak A,\bar\mu)$
and $(\frak B,\bar\nu)$ have the same number of atoms of measure
$\gamma$.

\proof{ Throughout the proof, write $\frak A_a$ for the principal ideal
of $\frak A$ generated by $a$, and $\bar\mu_a$ for the restriction of
$\bar\mu$ to $\frak A_a$;  and define $\frak B_b$, $\bar\nu_b$ similarly
for $b\in\frak B$.

\medskip

{\bf (a)} If $(\frak A,\bar\mu)$ and $(\frak B,\bar\nu)$ are isomorphic,
then of course the isomorphism matches their Maharam-type components
together and retains their magnitudes, and matches atoms of the same
measure together;  so the conditions are surely satisfied.


\medskip

{\bf (b)} Now suppose that the conditions are satisfied.   Set

\Centerline{$K
=\{\kappa:\kappa$ is an infinite cardinal, $e_{\kappa}\ne 0\}
=\{\kappa:\kappa$ is an infinite cardinal, $f_{\kappa}\ne 0\}$.}

\noindent For $\gamma\in\ooint{0,\infty}$, let $A_{\gamma}$ be the set
of atoms of measure $\gamma$ in $\frak A$, and set $e_{\gamma}=\sup
A_{\gamma}$.   Write $I=K\cup\ooint{0,\infty}$.   Then $\langle
e_i\rangle_{i\in I}$ is a partition of unity in $\frak A$, so $(\frak
A,\bar\mu)$ is isomorphic to the simple product of $\langle(\frak
A_{e_i},\bar\mu_{e_i})\rangle_{i\in I}$, writing $\frak A_{e_i}$ for the
principal ideal generated by $e_i$ and $\bar\mu_{e_i}$ for the
restriction $\bar\mu\restrp\frak A_{e_i}$.

In the same way, writing
$B_{\gamma}$ for the set of atoms of measure $\gamma$ in $\frak B$,
$f_{\gamma}$ for $\sup B_{\gamma}$,  $\frak B_{f_i}$ for the principal
ideal generated by $f_i$ and $\bar\nu_{f_i}$ for the restriction of
$\bar\nu$
fo $\frak B_{f_i}$, we have $(\frak B,\bar\nu)$ isomorphic to the simple
product of $\langle(\frak B_{f_i},\bar\nu_{f_i})\rangle_{i\in I}$.

\medskip

{\bf (c)} It will therefore be enough if I can show that
$(\frak A_{e_i},\bar\mu_{e_i})\cong(\frak B_{f_i},\bar\nu_{f_i})$
for every $i\in I$.

\medskip

\quad{\bf (i)} For $\kappa\in K$, the hypothesis is that $e_{\kappa}$
and $f_{\kappa}$ have the same magnitude.   If they are both of finite
magnitude, that is, $\bar\mu e_{\kappa}=\bar\nu f_{\kappa}<\infty$, then
both $(\frak A_{e_{\kappa}},\bar\mu_{e_{\kappa}})$ and
$(\frak B_{f_{\kappa}},\bar\nu_{f_{\kappa}})$ are homogeneous and of
Maharam type $\kappa$, by 332H.   So 331I tells us that they are
isomorphic.   If they are both of infinite magnitude $\lambda$, then
332I tells us that both $\frak A_{e_{\kappa}}$, $\frak B_{f_{\kappa}}$
have partitions of unity $C$, $D$ consisting of sets of measure $1$.
So $(\frak A_{e_{\kappa}},\bar\mu_{e_{\kappa}})$ is isomorphic to the
simple product of $\langle(\frak A_c,\bar\mu_c)\rangle_{c\in C}$, while
$(\frak B_{f_{\kappa}},\bar\nu_{f_{\kappa}})$ is isomorphic to the
simple product of
$\langle(\frak B_d,\bar\nu_d)\rangle_{d\in D}$.  But we know also that
every $(\frak A_c,\bar\mu_c)$, $(\frak B_d,\bar\nu_d)$ is a homogeneous
probability
algebra with Maharam type $\kappa$, by 332H again, so by Maharam's theorem
again they are all isomorphic.   Since $C$, $D$ and $\lambda$ are all
infinite,

\Centerline{$\#(C)=c(\frak A_{e_{\kappa}})=\lambda=c(\frak
B_{f_{\kappa}})=\#(D)$}

\noindent by 332E.   So we are taking the same number of factors in each
product and $(\frak A_{e_{\kappa}},\bar\mu_{e_{\kappa}})$ must be
isomorphic
to $(\frak B_{f_{\kappa}},\bar\nu_{f_{\kappa}})$.

\medskip

\quad{\bf (ii)} For $\gamma\in\ooint{0,\infty}$, our hypothesis is that
$\#(A_{\gamma})=\#(B_{\gamma})$.   Now $A_{\gamma}$ is a partition of
unity in $\frak A_{e_{\gamma}}$, so
$(\frak A_{e_{\gamma}},\bar\mu_{e_{\gamma}})$ is isomorphic to the
simple product of
$\langle(\frak A_a,\bar\mu_a)\rangle_{a\in A_{\gamma}}$.   Similarly,
$(\frak B_{f_{\gamma}},\bar\nu_{f_{\gamma}})$ is isomorphic to the
simple product of
$\langle(\frak B_b,\bar\nu_b)\rangle_{b\in B_{\gamma}}$.
Since every $(\frak A_a,\bar\mu_a)$, $(\frak B_b,\bar\nu_b)$ is just a
simple atom of
measure $\gamma$, these are all isomorphic;  since we are taking the
same number of factors in each product,
$(\frak A_{e_{\gamma}},\bar\mu_{e_{\gamma}})$ must be isomorphic to
$(\frak B_{f_{\gamma}},\bar\nu_{f_{\gamma}})$.

\medskip

\quad{\bf (iii)} Thus we have the full set of required isomorphisms, and
$(\frak A,\bar\mu)$ is isomorphic to $(\frak B,\bar\nu)$.
}%end of proof of 332J

\cmmnt{
\leader{332K}{Remarks (a)} The partition of unity
$\{e_i:i\in I\}$ of
$\frak A$ used in the above theorem is in some sense canonical.   (You
might feel it more economical to replace $I$ by
$K\cup\{\gamma:A_{\gamma}\ne\emptyset\}$.)   The further partition of
the atomic part into individual atoms (part (c-ii) of the proof) is also
canonical.   But of course the partition of the $e_{\kappa}$ of infinite
magnitude into elements of measure $1$ requires a degree of arbitrary
choice.

The value of the expressions in 332C is that the parameters $\kappa_i$,
$\gamma_i$ there are sufficient to identify the measure algebra up to
isomorphism.   For, amalgamating the language of 332C and 332J, we see
that the magnitude of $e_{\kappa}$ in 332J is just
$\sum_{\kappa_i=\kappa}\gamma_i$ if this is finite,
$\#(\{i:\kappa_i=\kappa\})$ otherwise (using 332E, as usual);  while
the number of atoms of measure $\gamma$ is
$\#(\{i:\kappa_i=0,\,\gamma_i=\gamma\})$.

\header{332Kb}{\bf (b)} The classification which Maharam's theorem gives
us is not merely a listing.   It involves a real insight into the nature
of the algebras, enabling us to answer a very wide variety of natural
questions.   I give the next couple of results as a sample of what we
can expect these methods to do for us.
}%end of comment

\leader{332L}{Proposition} Let $(\frak A,\bar\mu)$ be a measure algebra,
and $a$, $b\in\frak A$ two elements of finite measure.   Suppose that
$\pi:\frak A_a\to\frak A_b$ is a measure-preserving isomorphism, where
$\frak A_a$, $\frak A_b$ are the principal ideals generated by $a$ and
$b$.   Then there is a measure-preserving automorphism
$\phi:\frak A\to\frak A$ which extends $\pi$.

\proof{ The point is that $\frak A_{b\Bsetminus a}$ is isomorphic, as
measure algebra, to $\frak A_{a\Bsetminus b}$.   \Prf\ Set $c=a\Bcup b$.
For each infinite cardinal $\kappa$, let $e_{\kappa}$ be the
Maharam-type-$\kappa$ component of $\frak A_c$.   Then
$e_{\kappa}\Bcap a$ is the Maharam-type-$\kappa$ component of
$\frak A_a$, because if $d\Bsubseteq c$ and $\frak A_d$ is Maharam
homogeneous with Maharam type $\kappa$, then $\frak A_{d\Bcap a}$ is
either $\{0\}$ or again Maharam-type-homogeneous with Maharam type $\kappa$.
Similarly, $e_{\kappa}\Bsetminus a$ is the Maharam-type-$\kappa$
component of $\frak A_{c\setminus a}=\frak A_{b\setminus a}$, $e_{\kappa}\Bcap b$ is the
Maharam-type-$\kappa$ component of $\frak A_b$ and
$e_{\kappa}\Bsetminus b$ is the Maharam-type-$\kappa$ component of
$\frak A_{a\setminus b}$.   Now $\pi:\frak A_a\to\frak A_b$ is an
isomorphism, so $\pi(e_{\kappa}\cap a)$ must be $e_{\kappa}\cap b$, and

$$\eqalign{\bar\mu(e_{\kappa}\setminus a)
&=\bar\mu e_{\kappa}-\bar\mu(e_{\kappa}\cap a)
=\bar\mu e_{\kappa}-\bar\mu\pi(e_{\kappa}\cap a)\cr
&=\bar\mu e_{\kappa}-\bar\mu(e_{\kappa}\cap b)
=\bar\mu(e_{\kappa}\setminus b).\cr}$$

In the same way, if we write $n_{\gamma}(d)$ for the number of atoms of
measure $\gamma$ in $\frak A_d$, then

\Centerline{$n_{\gamma}(b\Bsetminus a)
=n_{\gamma}(c)-n_{\gamma}(a)
=n_{\gamma}(c)-n_{\gamma}(b)
=n_{\gamma}(a\Bsetminus b)$}

\noindent for every $\gamma\in\ooint{0,\infty}$.   By 332J, there is a
measure-preserving isomorphism
$\pi_1:\frak A_{b\Bsetminus a}\to\frak A_{a\setminus b}$.\ \Qed

If we now set

\Centerline{$\phi d
=\pi(d\Bcap a)\Bcup\pi_1(d\Bcap b\Bsetminus a)\Bcup(d\Bsetminus c)$}

\noindent for every $d\in\frak A$, $\phi:\frak A\to\frak A$ is a
measure-preserving isomorphism which agrees with $\pi$ on $\frak A_a$.
}%end of proof of 332L
% doesn't need Maharam's theorem 386Xf

\leader{332M}{Lemma} Suppose that $(\frak A,\bar\mu)$ and
$(\frak B,\bar\nu)$
are homogeneous measure algebras, with  $\tau(\frak A)\le\tau(\frak B)$
and $\bar\mu 1=\bar\nu 1<\infty$.   Then there is a
measure-preserving Boolean homomorphism from $\frak A$ to $\frak B$.

\proof{ The case $\tau(\frak A)=0$ is trivial.   Otherwise, considering
normalized versions of the measures, we are reduced to the case
$\bar\mu 1=\bar\nu 1=1$, $\tau(\frak A)=\kappa\ge\omega$,
$\tau(\frak B)=\lambda\ge\kappa$, so that $(\frak A,\bar\mu)$ is isomorphic to the
measure algebra $(\frak B_{\kappa},\bar\nu_{\kappa})$ of the usual
measure $\bar\nu_{\kappa}$ on $\{0,1\}^{\kappa}$;   and similarly
$(\frak B,\bar\nu)$ is isomorphic to the measure algebra
$(\frak B_{\lambda},\bar\nu_{\lambda})$ of the usual measure on
$\{0,1\}^{\lambda}$.   Now (identifying the cardinals $\kappa$, $\lambda$ with von
Neumann ordinals, as usual),
$\kappa\subseteq\lambda$, so we have an inverse-measure-preserving map
$x\mapsto x\restr\kappa:\{0,1\}^{\lambda}\to\{0,1\}^{\kappa}$ (254Oa),
which induces a measure-preserving Boolean homomorphism from
$\frak B_{\kappa}$ to $\frak B_{\lambda}$ (324M), and hence a
measure-preserving homomorphism from $\frak A$ to $\frak B$.
}%end of proof of 332M

\leader{332N}{Lemma} If $(\frak A,\bar\mu)$ is a probability algebra and
$\kappa\ge\max(\omega,\tau(\frak A))$, then there is a
measure-preserving Boolean
homomorphism from $(\frak A,\bar\mu)$ to the measure
algebra $(\frak B_{\kappa},\bar\nu_{\kappa})$ of the usual measure $\nu$ on
$\{0,1\}^{\kappa}$;  that is, $(\frak A,\bar\mu)$ is isomorphic to a
closed subalgebra of $(\frak B_{\kappa},\bar\nu_{\kappa})$.

\proof{ Let $\familyiI{c_i}$ be a partition of unity in $\frak A$ such
that every principal ideal $\frak A_{c_i}$ is homogeneous and no $c_i$
is zero.   Then $I$ is countable and $\sum_{i\in I}\bar\mu c_i=1$.
Accordingly there is a partition of unity $\familyiI{d_i}$ in
$\frak B_{\kappa}$ such that $\bar\nu d_i=\bar\mu c_i$ for every $i$.
\Prf\ Because $I$ is countable, we may suppose that it is either
$\Bbb N$ or an initial segment of $\Bbb N$.   In this case, choose
$\familyiI{d_i}$
inductively such that $d_i\Bsubseteq 1\Bsetminus\sup_{j<i}d_j$ and
$\bar\nu d_i=\bar\mu d_i$ for each $i\in I$, using 331C.\ \Qed

If $i\in I$, then $\tau(\frak A_{c_i})\le\kappa=\tau((\frak B_{\kappa})_{d_i})$, so
there is a measure-preserving Boolean homomorphism
$\pi_i:\frak A_{c_i}\to(\frak B_{\kappa})_{d_i}$.   Setting
$\pi a=\sup_{i\in I}\pi_i(a\Bcap c_i)$ for $a\in\frak A$, we have a
measure-preserving Boolean homomorphism $\pi:\frak A\to\frak B_{\kappa}$.   By
324Kb, $\pi[\frak A]$ is a closed subalgebra of $\frak B_{\kappa}$, and of course
$(\pi[\frak A],\bar\nu_{\kappa}\restr\pi[\frak A])$ is isomorphic to
$(\frak A,\bar\mu)$.
}%end of proof of 332N

\leader{332O}{Lemma} Let $(\frak A,\bar\mu)$ and $(\frak B,\bar\nu)$ be
localizable measure algebras.   For each infinite cardinal $\kappa$ let
$e_{\kappa}$,
$f_{\kappa}$ be their Maharam-type-$\kappa$ components, and for
$\gamma\in\ooint{0,\infty}$ let $e_{\gamma}$, $f_{\gamma}$ be the
suprema of the atoms of measure $\gamma$ in $\frak A$, $\frak B$
respectively.   If there is a measure-preserving Boolean homomorphism
from $\frak A$ to $\frak B$, then the magnitude of
$\sup_{\kappa\ge\lambda}e_{\kappa}$ is not greater than the magnitude of
$\sup_{\kappa\ge \lambda}f_{\kappa}$ whenever $\lambda$ is an infinite
cardinal, while the magnitude of
$\sup_{\kappa\ge\omega}e_{\kappa}\Bcup\sup_{\gamma\le\delta}e_{\gamma}$
is not greater than the magnitude of
$\sup_{\kappa\ge\omega}f_{\kappa}\Bcup\sup_{\gamma\le\delta}f_{\gamma}$
for any $\delta\in\ooint{0,\infty}$.

\proof{ Suppose that $\pi:\frak A\to\frak B$ is a measure-preserving
Boolean homomorphism.   For infinite cardinals $\lambda$, set
$e^*_{\lambda}=\sup_{\kappa\ge\lambda}e_{\kappa}$,
$f^*_{\lambda}=\sup_{\kappa\ge\lambda}f_{\kappa}$, while for
$\delta\in\ooint{0,\infty}$ set $e^*_{\delta}
=\sup_{\kappa\ge\omega}e_{\kappa}\Bcup\sup_{\gamma\le\delta}e_{\gamma}$,
$f^*_{\delta}
=\sup_{\kappa\ge\omega}f_{\kappa}\Bcup\sup_{\gamma\le\delta}f_{\gamma}$.
Let $\langle c_i\rangle_{i\in I}$ be a partition of unity in
$\frak A$ such that all the principal ideals $\frak A_{c_i}$ are totally
finite and homogeneous, as in 332B.   Then $c_i\Bsubseteq e_{\kappa}$
whenever $\kappa=\tau(\frak A_{c_i})$ is infinite, and $c_i\Bsubseteq
e_{\gamma}$ if
$c_i$ is an atom of measure $\gamma$.   Take $v$ to be either an
infinite cardinal or a strictly positive real
number.   Set

\Centerline{$J=\{i:i\in I,\,c_i\Bsubseteq e^*_v\}$;}

\noindent then $e^*_v=\sup_{i\in J}c_i$.

Now the point is that if $i\in J$ then $\pi c_i\Bsubseteq f^*_{{v}}$.
\Prf\   We need to consider two cases.   (i) If $c_i$ is an atom, then
$v\in\ooint{0,\infty}$ and $\bar\mu c_i\le v$.   So we need only observe
that $1\Bsetminus f^*_v$ is just the supremum in $\frak B$ of the atoms
of measure greater than $v$, none of which can meet $\pi c_i$, since
this has measure at most $v$.   (ii)   Now suppose that $\frak A_{c_i}$
is atomless, with $\tau(\frak A_{c_i})=\kappa\ge v$.   If $0\ne
b\Bsubseteq\pi c_i$, then
$a\mapsto b\Bcap\pi a:\frak A_{c_i}\to\frak B_b$ is an order-continuous
Boolean homomorphism, while $\frak A_{c_i}$ is isomorphic (as Boolean
algebra) to the measure algebra of $\{0,1\}^{\kappa}$, so 331J tells us
that $\tau(\frak B_b)\ge\kappa$.   This means, first, that $b$ cannot be
an atom, so that $\pi c_i$ cannot meet
$\sup_{\gamma\in\ooint{0,\infty}}f_{\gamma}$;  and also that $b$ cannot
be included in $f_{\kappa'}$ for any infinite $\kappa'<\kappa$, so that
$\pi c_i$ cannot meet $\sup_{\omega\le\kappa'<\kappa}f_{\kappa}$.   Thus
$\pi c_i$ must be included in
$\sup_{\kappa'\ge\kappa}f_{\kappa}\Bsubseteq f^*_v$.\ \Qed

Of course $\langle\pi c_i\rangle_{i\in J}$ is disjoint.   So if
$e^*_{{v}}$ has finite magnitude, the magnitude of $f^*_{{v}}$ is at
least

\Centerline{$\sum_{i\in J}\bar\nu\pi c_i=\sum_{i\in J}\bar\mu
c_i=\bar\mu
e^*_{{v}}$,}

\noindent the magnitude of $e^*_{{v}}$.   While if $e^*_{{v}}$ has
infinite magnitude, this is $\#(J)$, by 332E, which is not greater than
the magnitude of $f^*_{{v}}$.
}%end of proof of 332O

\leader{332P}{Proposition} Let $(\frak A,\bar\mu)$ and $(\frak B,\bar\nu)$
be atomless totally finite measure algebras.   For each infinite
cardinal
$\kappa$ let $e_{\kappa}$, $f_{\kappa}$ be their Maharam-type-$\kappa$
components.   Then the following are equiveridical:

(i) $(\frak A,\bar\mu)$ is isomorphic to a closed subalgebra of a
principal ideal of $(\frak B,\bar\nu)$;

(ii) for every cardinal $\lambda$,

\Centerline{$\bar\mu(\sup_{\kappa\ge\lambda}e_{\kappa})
\le\bar\nu(\sup_{\kappa\ge\lambda}f_{\kappa})$.}

\proof{{\bf (a)(i)$\Rightarrow$(ii)} Suppose that $\pi:\frak A\to\frak
B_d$ is a measure-preserving isomorphism between $\frak A$ and a closed
subalgebra of a principal ideal $\frak B_d$ of $\frak B$.   The
Maharam-type-$\kappa$ component of $\frak B_d$ is just $d\Bcap
f_{\kappa}$, so 332O tells us that

\Centerline{$\bar\mu(\sup_{\kappa\ge\lambda}e_{\kappa})
\le\bar\nu(\sup_{\kappa\ge\lambda}d\Bcap f_{\kappa})
\le\bar\nu(\sup_{\kappa\ge\lambda}f_{\kappa})$}


\noindent for every $\lambda$.

\medskip

{\bf (b)(ii)$\Rightarrow$(i)} Now suppose that the condition is
satisfied.

\medskip

\quad\grheada\ Let $P$ be the set of all measure-preserving Boolean
homomorphisms $\pi$ from principal ideals  $\frak A_{c_{\pi}}$ of
$\frak A$ to principal ideals $\frak B_{d_{\pi}}$ of $\frak B$ such that

\Centerline{$\bar\mu(\sup_{\kappa\ge\lambda}
  e_{\kappa}\Bsetminus c_{\pi})
\le\bar\nu(\sup_{\kappa\ge\lambda}
  \bar\nu f_{\kappa}\Bsetminus d_{\pi})$}

\noindent for every cardinal $\lambda\ge\omega$.
Then the trivial homomorphism from $\frak A_0$ to $\frak B_0$ belongs to
$P$, so $P$ is not empty.   Order $P$ by saying that $\pi\le\pi'$ if
$\pi'$ extends $\pi$, that is, if $c_{\pi}\Bsubseteq c_{\pi'}$ and
$\pi'a=\pi a$ for every $a\in\frak A_{c_{\pi}}$.   Then $P$ is a
partially ordered set.

\medskip

\quad\grheadb\
If $Q\subseteq P$ is non-empty and totally ordered, it is bounded above
in $P$.   \Prf\ Set $c^*=\sup_{\pi\in Q}c_{\pi}$,
$d^*=\sup_{\pi\in Q}d_{\pi}$.   For $a\Bsubseteq c^*$ set
$\pi^*a=\sup_{\pi\in Q}\pi(a\Bcap c_{\pi})$.   Because $Q$ is totally
ordered, $\pi^*$ extends all the
functions in $Q$.   It is also easy to check that $\pi^*0=0$,
$\pi^*(a\Bcap a')=\pi^*a\Bcap\pi^*a'$ and
$\pi^*(a\Bcup a')=\pi^*a\Bcup\pi^*a'$ for all $a$, $a'\in\frak A_{c^*}$,
$\pi^*c^*=d^*$ and that $\bar\nu\pi^*a=\bar\mu a$ for every
$a\in\frak A_{c^*}$; so that $\pi^*$ is a measure-preserving Boolean
homomorphism from $\frak A_{c^*}$ to $\frak B_{d^*}$.

Now suppose that $\lambda$ is any cardinal;  then

$$\eqalign{\bar\mu(\sup_{\kappa\ge\lambda}e_{\kappa}\Bsetminus c^*)
&=\inf_{\pi\in Q}
   \bar\mu(\sup_{\kappa\ge\lambda}e_{\kappa}\Bsetminus c_{\pi})
\le\inf_{\pi\in Q}
   \bar\nu(\sup_{\kappa\ge\lambda}f_{\kappa}\Bsetminus d_{\pi})
=\bar\nu(\sup_{\kappa\ge\lambda}f_{\kappa}\Bsetminus d^*).\cr}$$

\noindent So $\pi^*\in P$ and is the required upper bound of $Q$.\
\Qed

\medskip

\quad\grheadc\  By Zorn's Lemma, $P$ has a maximal element $\tilde\pi$
say.   Now $c_{\tilde\pi}=1$.   \Prf\Quer\ If not, then let $\kappa_0$
be the least cardinal such that
$e_{\kappa_0}\Bsetminus c_{\tilde\pi}\ne 0$.  Then

\Centerline{$0
<\bar\mu(\sup_{\kappa\ge\kappa_0}e_{\kappa}\Bsetminus c_{\tilde\pi})
\le\bar\nu(\sup_{\kappa\ge\kappa_0}\bar\nu f_{\kappa}\Bsetminus d_{\tilde\pi})$,}

\noindent so there is a least $\kappa_1\ge\kappa_0$ such that
$f_{\kappa_1}\Bsetminus d_{\tilde\pi}\ne 0$.   Set
$\delta=\min(\bar\mu(e_{\kappa_0}\Bsetminus c_{\tilde\pi}),
\bar\nu(f_{\kappa_1}\Bsetminus d_{\tilde\pi}))>0$.   Because $\frak A$
and $\frak B$ are atomless, there are
$a\Bsubseteq e_{\kappa_0}\Bsetminus c_{\tilde\pi}$ and
$b\Bsubseteq f_{\kappa_1}\Bsetminus d_{\tilde\pi}$ such
that $\bar\mu a=\bar\nu b=\delta$ (331C).   Now $\frak A_a$ is
homogeneous with
Maharam type $\kappa_0$, while $\frak B_b$ is homogeneous with Maharam
type $\kappa_1$ (332H), so there is a
measure-preserving Boolean homomorphism $\phi:\frak A_a\to\frak B_b$
(332M).   Set

\Centerline{$c^*=c_{\tilde\pi}\Bcup a$, \quad $d^*=d_{\tilde\pi}\Bcup
b$,}

\noindent and define $\pi^*:\frak A_{c^*}\to\frak B_{d^*}$ by setting
$\pi^*(g)=\tilde\pi(g\Bcap c_{\tilde\pi})\Bcup\phi(g\Bcap a)$ for every
$g\Bsubseteq c^*$.  It is easy to check that $\pi^*$ is a
measure-preserving Boolean homomorphism.

If $\lambda$ is a cardinal and $\lambda\le\kappa_0$,

$$\eqalign{\bar\mu(\sup_{\kappa\ge\lambda}e_{\kappa}\Bsetminus c^*)
&=\bar\mu(\sup_{\kappa\ge\lambda}e_{\kappa}\Bsetminus c_{\tilde\pi})
-\delta
\le\bar\nu(\sup_{\kappa\ge\lambda}f_{\kappa}\Bsetminus d_{\tilde\pi})
-\delta
=\bar\nu(\sup_{\kappa\ge\lambda}\bar\nu f_{\kappa}\Bsetminus d^*).\cr}$$

\noindent If $\kappa_0<\lambda\le\kappa_1$,

$$\eqalignno{\bar\mu(\sup_{\kappa\ge\lambda}e_{\kappa}\Bsetminus c^*)
&=\bar\mu(\sup_{\kappa\ge\lambda}e_{\kappa}\Bsetminus c_{\tilde\pi})
\le\bar\mu(\sup_{\kappa\ge\kappa_0}e_{\kappa}\Bsetminus c_{\tilde\pi})
-\bar\mu(e_{\kappa_0}\Bsetminus c_{\tilde\pi})\cr
&\le\bar\mu(\sup_{\kappa\ge\kappa_0}e_{\kappa}\Bsetminus c_{\tilde\pi})
-\delta
\le\bar\nu(\sup_{\kappa\ge\kappa_0}f_{\kappa}\Bsetminus d_{\tilde\pi})
-\delta\cr
&=\bar\nu(\sup_{\kappa\ge\kappa_1}f_{\kappa}\Bsetminus d_{\tilde\pi})
-\delta\cr
\noalign{\noindent (by the choice of $\kappa_1$)}
&=\bar\nu(\sup_{\kappa\ge\kappa_1}f_{\kappa}\Bsetminus d^*)
\le\bar\nu(\sup_{\kappa\ge\lambda}f_{\kappa}\Bsetminus d^*).\cr}$$

\noindent If $\lambda>\kappa_1$,

$$\eqalign{\bar\mu(\sup_{\kappa\ge\lambda}e_{\kappa}\Bsetminus c^*)
&=\bar\mu(\sup_{\kappa\ge\lambda}e_{\kappa}\Bsetminus c_{\tilde\pi})
\le\bar\nu(\sup_{\kappa\ge\lambda}f_{\kappa}\Bsetminus d_{\tilde\pi})
=\bar\nu(\sup_{\kappa\ge\lambda}f_{\kappa}\Bsetminus d^*).\cr}$$

\noindent But this means that $\pi^*\in P$, and evidently it is a proper
extension of $\tilde\pi$, which is supposed to be impossible.\
\Bang\Qed

\medskip

\quad\grheadd\ Thus $\tilde\pi$ has domain $\frak A$ and is the required
measure-preserving homomorphism from $\frak A$ to the principal ideal
$\frak B_{d_{\tilde\pi}}$ of $\frak B$.
}%end of proof of 332P

\leader{332Q}{Proposition} Let $(\frak A,\bar\mu)$ and
$(\frak B,\bar\nu)$ be totally finite
measure algebras, and suppose that there are
measure-preserving Boolean homomorphisms $\pi_1:\frak A\to\frak B$ and
$\pi_2:\frak B\to\frak A$.   Then $(\frak A,\bar\mu)$ and
$(\frak B,\bar\nu)$ are isomorphic.

\proof{ Writing $e_{\kappa}$, $f_{\kappa}$ for their
Maharam-type-$\kappa$ components, 332O (applied to both $\pi_1$ and
$\pi_2$) tells us that

\Centerline{$\bar\mu(\sup_{\kappa\ge\lambda}e_{\kappa})
=\bar\nu(\sup_{\kappa\ge\lambda}f_{\kappa})$}

\noindent for every $\lambda$.   Because all these measures are finite,

$$\eqalign{\bar\mu e_{\lambda}
&=\bar\mu(\sup_{\kappa\ge\lambda}e_{\kappa})
  -\bar\mu(\sup_{\kappa>\lambda}e_{\kappa})\cr
&=\bar\nu(\sup_{\kappa\ge\lambda}f_{\kappa})
  -\bar\nu(\sup_{\kappa>\lambda}f_{\kappa})
=\bar\nu f_{\lambda}\cr}$$

\noindent for every $\lambda$.

Similarly, writing $e_{\gamma}$, $f_{\gamma}$ for the suprema in $\frak
A$, $\frak B$ of the atoms of measure $\gamma$, 332O tells us that

\Centerline{$\bar\mu(\sup_{\gamma\le\delta}e_{\gamma})
=\bar\nu(\sup_{\gamma\le\delta}f_{\gamma})$}

\noindent for every $\delta\in\ooint{0,\infty}$, and hence that $\bar\mu
e_{\gamma}=\bar\nu f_{\gamma}$ for every $\gamma$, that is, that $\frak
A$
and $\frak B$ have the same number of atoms of measure $\gamma$.

So $(\frak A,\bar\mu)$ and $(\frak B,\bar\nu)$ are isomorphic, by 332J.
}%end of proof of 332Q

\leader{332R}{}\cmmnt{ 332J tells us that if we know the magnitudes of
the Maharam-type-$\kappa$ components of a localizable measure algebra,
we shall have specified the algebra completely, so that all its
properties are determined.   The calculation of its Maharam type is
straightforward and useful, so I give the details.

\medskip

\noindent}{\bf Lemma} Let $(\frak A,\bar\mu)$ be a semi-finite measure
algebra.   Then $c(\frak A)\le 2^{\tau(\frak A)}$.

\proof{ Let $C\subseteq\frak A\setminus\{0\}$ be a disjoint set, and
$B\subseteq\frak A$ a $\tau$-generating set of size $\tau(\frak A)$.

\medskip

{\bf (a)} If $\frak A$ is purely atomic, then for each $c\in C$ choose
an atom $c'\Bsubseteq c$, and set $f(c)=\{b:b\in B,\,c'\Bsubseteq b\}$.
If $c_1$, $c_2$ are distinct members of $C$, the set

\Centerline{$\{a:a\in\frak A,\,c'_1\Bsubseteq a\iff c'_2\Bsubseteq a\}$}

\noindent is an order-closed subalgebra of $\frak A$ not containing
either $c'_1$ or $c'_2$, so cannot include $B$, and $f(c_1)\ne f(c_2)$.
Thus $f$ is injective, and

\Centerline{$\#(C)\le\#(\Cal PB)=2^{\tau(\frak A)}$.}

\medskip

{\bf (b)} Now suppose that $\frak A$ is not purely atomic;  in this case
$\tau(\frak A)$ is infinite.    For each $c\in C$ choose an element
$c'\Bsubseteq c$ of non-zero finite measure.   Let $\frak B$ be the
subalgebra of $\frak A$ generated by $B$.   Then the topological closure
of $\frak B$ is $\frak A$ itself (323J), and $\#(\frak B)=\tau(\frak A)$
(331Gc).   For $c\in C$ set

\Centerline{$f(c)=\{b:b\in\frak B,
  \,\bar\mu(b\Bcap c')\ge\Bover12\bar\mu c'\}$.}

\noindent Then $f:C\to\Cal P\frak B$ is injective.   \Prf\ If $c_1$,
$c_2$ are distinct members of $C$, then (because $\frak B$ is
topologically dense in $\frak A$) there is a $b\in\frak B$ such that

\Centerline{$\bar\mu((c'_1\Bcup c'_2)\Bcap(c'_1\Bsymmdiff b))
\le\Bover13\min(\bar\mu c'_1,\bar\mu c'_2)$.}

\noindent But in this case

\Centerline{$\bar\mu(c'_1\Bsetminus b)\le\Bover13\bar\mu c'_1$,
\quad$\bar\mu(c'_2\Bcap b)\le\Bover13\bar\mu c'_2$,}

\noindent and $b\in f(c_1)\symmdiff f(c_2)$, so $f(c_1)\ne f(c_2)$.\
\QeD\
Accordingly $\#(C)\le 2^{\#(\frak B)}=2^{\tau(\frak A)}$ in this case
also.

As $C$ is arbitrary, $c(\frak A)\le 2^{\tau(\frak A)}$.
}%end of proof of 332R

\leader{332S}{Theorem} Let $(\frak A,\bar\mu)$ be a localizable measure
algebra.   Then $\tau(\frak A)$ is the least cardinal $\lambda$ such
that ($\alpha$) $c(\frak A)\le 2^{\lambda}$ ($\beta$)
$\tau(\frak A_a)\le\lambda$ for every \Mth\ principal ideal
$\frak A_a$ of $\frak A$.

\proof{ Fix $\lambda$ as the least cardinal satisfying ($\alpha$) and
($\beta$).

\medskip

{\bf (a)} By 331Hc, $\tau(\frak A_a)\le\tau(\frak A)$ for every
$a\in\frak A$, while $c(\frak A)\le 2^{\tau(\frak A)}$ by 332R;  so
$\lambda\le\tau(\frak A)$.

\medskip

{\bf (b)}  Let $C$ be a partition of unity in $\frak A$ consisting of
elements of non-zero finite measure generating \Mth\
principal ideals (as in the proof of 332B);  then
$\#(C)\le c(\frak A)\le 2^{\lambda}$, and there is an injective function
$f:C\to\Cal P\lambda$.   For each $c\in C$, let $B_c\subseteq\frak A_c$ be a
$\tau$-generating set of cardinal $\tau(\frak A_c)$, and
$f_c:B_c\to\lambda$ an injection.   Set

\Centerline{$b_{\xi}=\sup\{c:c\in C,\,\xi\in f(c)\}$,}

\Centerline{$b'_{\xi}=\sup\{b:$ there is some $c\in C$ such that
$b\in B_c$ and $f_c(b)=\xi\}$}

\noindent for $\xi<\lambda$.   Set
$B=\{b_{\xi}:\xi<\lambda\}\cup\{b'_{\xi}:\xi<\lambda\}$ if $\lambda$ is
infinite, $\{b_{\xi}:\xi<\lambda\}$ if $\lambda$ is finite;  then
$\#(B)\le\lambda$.   Note that if $c\in C$ and $b\in B_c$ there is a
$b'\in B$ such that $b=b'\Bcap c$.   \Prf\ Since $B_c\ne\emptyset$,
$\tau(\frak A_c)>0$;  but this means that $\tau(\frak A_c)$ is infinite
(see 331H) so $\lambda$ is infinite and $b'_{\xi}\in B$, where
$\xi=f_c(b)$;  now $b=b'_{\xi}\Bcap c$.\   \Qed

Let $\frak B$ be the
closed subalgebra of $\frak A$ generated by $B$.   Then
$C\subseteq\frak B$.   \Prf\ For $c\in C$, we surely have
$c\Bsubseteq b_{\xi}$ if
$\xi\in f(c)$;  but also, because $C$ is disjoint, $c\Bcap b_{\xi}=0$ if
$\xi\in\lambda\setminus f(c)$.   Consequently

\Centerline{$c^*
=\inf_{\xi\in f(c)}b_{\xi}
  \Bcap\inf_{\xi\in\lambda\setminus f(c)}(1\Bsetminus b_{\xi})$}

\noindent includes $c$.   On the other hand, if $d$ is any other member
of $C$, there is some $\xi\in f(c)\symmdiff f(d)$, so that

\Centerline{$d^*\Bcap c^*\Bsubseteq b_{\xi}\Bcap(1\Bsetminus
b_{\xi})=0$.}

\noindent Since $\sup C=1$, it follows that $c=c^*$;  but $c^*\in\frak
B$, so $c\in\frak B$.\  \Qed

For any $c\in C$, look at $\{b\Bcap c:b\in\frak B\}\subseteq\frak B$.
This is a closed subalgebra of $\frak A_c$
(314F(a-i)) including $B_c$, so must be the whole of $\frak A_c$.   Thus
$\frak A_c\subseteq\frak B$ for every $c\in C$.   But $\sup C=1$, so
$a=\sup_{c\in C}a\Bcap c\in\frak B$ for every $a\in\frak A$, and $\frak
A=\frak B$.   Consequently $\tau(\frak A)\le\#(B)\le\lambda$, and
$\tau(\frak A)=\lambda$.
}%end of proof of 332S

\leader{332T}{Proposition} Let $(\frak A,\bar\mu)$ be a localizable
measure algebra and $\frak B$ a closed subalgebra of $\frak A$.   Then

(a) there is a function $\bar\nu:\frak B\to[0,\infty]$ such that
$(\frak B,\bar\nu)$ is a localizable measure algebra;

(b) $\tau(\frak B)\le\tau(\frak A)$.

\proof{{\bf (a)} Let $D$ be the set of those $b\in\frak B$ such that the
principal ideal $\frak B_b$ has Maharam type at most $\tau(\frak A)$ and
is a totally finite measure algebra when endowed with an appropriate
measure.   Then $D$ is order-dense in $\frak B$.   \Prf\ Take any
non-zero $b_0\in\frak B$.   Then there is an $a\in\frak A$ such that
$a\Bsubseteq b_0$ and $0<\bar\mu a<\infty$.   Set
$c=\upr(a,\frak B)=\min\{b:b\in\frak B,\,a\Bsubseteq b\}$;
then $c\in\frak B$ and $a\Bsubseteq c\Bsubseteq b_0$.
If $0\ne b\in\frak B_c$, then $c\Bsetminus b$ belongs to $\frak B$ and is properly
included in $c$, so cannot include $a$;  accordingly
$a\Bcap b\ne 0$.    For $b\in\frak B_c$, set
$\bar\nu b=\bar\mu(a\Bcap b)$.   Because the map $b\mapsto a\Bcap b$ is an injective
order-continuous Boolean homomorphism, $\bar\nu$ is countably additive
and strictly positive, that is, $(\frak B_c,\bar\nu)$ is a measure
algebra.  It is totally finite because $\bar\nu c=\bar\mu a<\infty$.

Let $d\in\frak B_c\setminus\{0\}$ be such that $\frak B_d$ is \Mth;
suppose that its Maharam type is $\kappa$.   The map
$b\mapsto b\Bcap a$ is a measure-preserving Boolean homomorphism from
$\frak B_d$ to $\frak A_{a\Bcap d}$, so by 332O $\frak A_{a\Bcap d}$
must have a non-zero Maharam-type-$\kappa'$ component for some
$\kappa'\ge\kappa$;  but this means that

\Centerline{$\tau(\frak B_d)\le\kappa\le\kappa'
\le\tau(\frak A_{a\Bcap d})\le\tau(\frak A)$.}

\noindent Thus $d\in D$, while $0\ne d\Bsubseteq c\Bsubseteq b_0$.   As
$b_0$ is arbitrary, $D$ is order-dense.\  \Qed

Accordingly there is a partition of unity $C$ in $\frak B$ such that
$C\subseteq D$.   For each $c\in C$ we have a functional
$\bar\nu_c$ such that $(\frak B_c,\bar\nu_c)$ is a totally finite
measure algebra with Maharam type at most $\tau(\frak A)$;  define
$\bar\nu:\frak B\to[0,\infty]$ by setting
$\bar\nu b=\sum_{c\in C}\bar\nu_c(b\Bcap c)$ for every $b\in\frak B$.
It is easy to check
that $(\frak B,\bar\nu)$ is a measure algebra (compare 322La);  it is
localizable because $\frak B$ (being
order-closed in a Dedekind complete partially ordered set) is Dedekind
complete.

\medskip

{\bf (b)} The construction above ensures that every homogeneous
principal ideal of $\frak B$ can have Maharam type at most
$\tau(\frak A)$, since it must share a principal ideal with some $\frak B_c$ for
$c\in C$.   Moreover, any disjoint set in $\frak B$ is also a disjoint
set in $\frak A$, so $c(\frak B)\le c(\frak A)$.   So 332S tells us that
$\tau(\frak B)\le\tau(\frak A)$.
}%end of proof of 332T

\cmmnt{\medskip

\noindent{\bf Remark} I think the only direct appeal I shall make to
this result will be when $(\frak A,\bar\mu)$ is a probability algebra,
in which case (a) above becomes trivial, and the proof of (b) can be
shortened to some extent, though I think we still need some of the ideas
of 332S.
}%end of comment

%332L out of order

\exercises{
\leader{332X}{Basic exercises (a)} Let $\frak A$ be a Dedekind complete
Boolean algebra.   Show that it is isomorphic to a simple
product of \Mth\ Boolean algebras.
%332A

\header{332Xb}{\bf (b)} Let $\frak A$ be a Boolean algebra of finite
cellularity.   Show that $\frak A$ is purely atomic.
%332D

\header{332Xc}{\bf (c)} Let $\frak A$ be a purely atomic Boolean
algebra.   Show that $c(\frak A)$ is the number of atoms in $\frak A$.
%332D

\header{332Xd}{\bf (d)} Let $\frak A$ be any Boolean algebra, and $Z$
its Stone space.   Show that $c(\frak A)$ is equal to

\Centerline{$c(Z)=\sup\{\#(\Cal G):G$ is a disjoint family of non-empty
open subsets of $Z\}$,}

\noindent the {\bf cellularity} of the topological space $Z$.
%332D

\header{332Xe}{\bf (e)} Let $X$ be a topological space, and $\RO(X)$
its regular open algebra.   Show that $c(\RO(X))=c(X)$ as defined in
332Xd.
%332Xe, 332D

\header{332Xf}{\bf (f)} Let $\frak A$ be a Boolean algebra, and
$\frak B$ any subalgebra of $\frak A$.   Show that
$c(\frak B)\le c(\frak A)$.
%332D

\spheader 332Xg Let $\langle\frak A_i\rangle_{i\in I}$ be any
family of Boolean algebras, with simple product $\frak A$.   Show that
the cellularity of $\frak A$ is at most
$\max(\omega,\#(I),\sup_{i\in I}c(\frak A_i))$.   Devise an
elegant expression of a necessary and sufficient condition for equality.
%332D

\spheader 332Xh Let $\frak A$ be any Boolean algebra, and
$a\in\frak A$;  let $\frak A_a$ be the principal ideal generated by $a$.
Show that $c(\frak A_a)\le c(\frak A)$.
%332D

\header{332Xi}{\bf (i)} Let $(\frak A,\bar\mu)$ be a semi-finite measure
algebra.   Show that it has a partition of unity of cardinal $c(\frak
A)$.
%332F

\spheader 332Xj Let $(\frak A,\bar\mu)$ and $(\frak B,\bar\nu)$ be
localizable measure algebras.   For each cardinal $\kappa$ let
$e_{\kappa}$, $f_{\kappa}$ be their Maharam-type-$\kappa$ components,
and $\frak A_{e_{\kappa}}$, $\frak B_{f_{\kappa}}$ the corresponding
principal ideals.   Show that $\frak A$ and $\frak B$ are isomorphic, as
Boolean algebras, iff $c(\frak A_{e_{\kappa}})=c(\frak B_{f_{\kappa}})$
for every $\kappa$.
%332J

\spheader 332Xk Let $\zeta$ be an ordinal, and
$\langle\alpha_{\xi}\rangle_{\xi<\zeta}$,
$\langle\beta_{\xi}\rangle_{\xi<\zeta}$ two families of non-negative
real numbers such that $\sum_{\theta\le\xi<\zeta}\alpha_{\xi}
\le\sum_{\theta\le\eta<\zeta}\beta_{\eta}<\infty$ for every
$\theta\le\zeta$.   Show that there is a family
$\langle\gamma_{\xi\eta}\rangle_{\xi\le\eta<\zeta}$ of non-negative real
numbers such that
$\alpha_{\xi}=\sum_{\xi\le\eta<\zeta}\gamma_{\xi\eta}$ for every
$\xi<\zeta$ and
$\beta_{\eta}\ge\sum_{\xi\le\eta}\gamma_{\xi\eta}$ for every
$\eta<\zeta$.   (If only finitely many of the $\alpha_{\xi}$,
$\beta_{\xi}$ are non-zero, this is an easy special case of the max-flow
min-cut theorem;  see {\smc Bollob\'as 79}, {\S}III.1 or {\smc Anderson 87}, 12.3.1;
there is a statement of the theorem in 4A4N in the next volume.)
Show that the $\gamma_{\xi\eta}$ can be chosen in such a way that if
$\xi<\xi'\le\eta'<\eta$ then at least one of $\gamma_{\xi\eta}$,
$\gamma_{\xi'\eta'}$ is zero.
%for 332Xl

\spheader 332Xl Use 332Xk and 332M to give another proof of 332P.
%332Xk, 332P

\spheader 332Xm For each cardinal $\kappa$, write
$(\frak B_{\kappa},\bar\nu_{\kappa})$ for the measure algebra of the usual
measure on $\{0,1\}^{\kappa}$.   Let $(\frak A,\bar\mu)$ be the simple
product of $\sequencen{(\frak B_{\omega_n},\bar\nu_{\omega_n})}$ and
$(\frak B,\bar\nu)$ the simple product of $(\frak A,\bar\mu)$ with
$(\frak B_{\omega_{\omega}},\bar\nu_{\omega_{\omega}})$.   (See 3A1E if
you are puzzled by the names $\omega_n$, $\omega_{\omega}$.)   Show that there is a
measure-preserving Boolean homomorphism from $\frak A$ to $\frak B$, but
that no such homomorphism can be order-continuous.
%332P

\spheader 332Xn For each cardinal $\kappa$, write
$(\frak B_{\kappa},\bar\nu_{\kappa})$ for the measure algebra of the usual
measure on $\{0,1\}^{\kappa}$.  Let $(\frak A,\bar\mu)$ be the simple product of
$\sequencen{(\frak B_{\kappa_{n}},\bar\nu_{\kappa_{n}})}$ and
$(\frak B,\bar\nu)$ the simple product of
$\sequencen{(\frak B_{\lambda_{n}},\bar\nu_{\lambda_{n}})}$, where
$\kappa_n=\omega$ for
even $n$, $\omega_n$ for odd $n$, while $\lambda_n=\omega$ for odd $n$,
$\omega_n$ for even $n$.   Show that there are
order-continuous measure-preserving Boolean homomorphisms from $\frak A$
to $\frak B$ and from $\frak B$ to $\frak A$, but that these two measure
algebras are not isomorphic.
%332Q, 332Xm

\spheader 332Xo Let $\frak C$ be a Boolean algebra.    Show that
the following are equiveridical:  (i) $\frak C$ is isomorphic (as Boolean
algebra) to a closed subalgebra of a localizable measure algebra;  (ii)
there is a $\bar\mu$ such that $(\frak C,\bar\mu)$ is itself a
localizable measure algebra;   (iii) $\frak C$ is Dedekind complete and
for every non-zero $c\in\frak C$ there is a completely additive
real-valued functional
$\nu$ on $\frak C$ such that $\nu c\ne 0$.   ({\it Hint for
(iii)$\Rightarrow$(ii)\/}:  show that the set of supports of
non-negative
completely additive functionals is order-dense in $\frak C$, so includes
a partition of unity.)
%332T

\leader{332Y}{Further exercises (a)}
%\spheader 332Ya
Let $(\frak A,\bar\mu)$ and
$(\frak B,\bar\nu)$ be atomless localizable measure algebras.   For each
infinite cardinal $\kappa$ let $e_{\kappa}$, $f_{\kappa}$ be their
Maharam-type-$\kappa$ components.   Show that the following are
equiveridical:  (i) $(\frak A,\bar\mu)$ is isomorphic to a closed
subalgebra of a principal ideal of $(\frak B,\bar\nu)$;
(ii) for every cardinal $\lambda$,
the magnitude of $\sup_{\kappa\ge\lambda}e_{\kappa}$ is not greater than
the magnitude of $\sup_{\kappa\ge\lambda}f_{\kappa}$.
%332P

\header{332Yb}{\bf (b)} Let $(\frak A,\bar\mu)$ and $(\frak B,\bar\nu)$
be any semi-finite measure algebras, and $(\widehat{\frak A},\hat\mu)$,
$(\widehat{\frak B},\hat\nu)$ their localizations (322P-322Q).   Let
$\langle e_i\rangle_{i\in I}$, $\langle f_j\rangle_{j\in J}$ be
partitions of
unity in $\frak A$, $\frak B$ respectively into elements of finite
measure generating homogeneous principal ideals $\frak A_{e_i}$,
$\frak B_{f_j}$.   For each infinite cardinal $\kappa$ set
$I_{\kappa}=\{i:\tau(\frak A_{e_i})=\kappa\}$,
$J_{\kappa}=\{j:\tau(\frak B_{f_j})=\kappa\}$;  for
$\gamma\in\ooint{0,\infty}$, set $I_{\gamma}=\{i:e_i$ is an atom,
$\bar\mu e_i=\gamma\}$, $J_{\gamma}=\{j:f_j$ is an atom,
$\bar\nu f_j=\gamma\}$.   Show that $(\widehat{\frak A},\hat\mu)$ and
$(\widehat{\frak B},\hat\nu)$ are isomorphic iff for each $u$, either
$\sum_{i\in I_u}\bar\mu e_i=\sum_{j\in J_u}\bar\nu f_j<\infty$
or $\sum_{i\in I_u}\bar\mu e_i=\sum_{j\in J_u}\bar\nu f_j=\infty$ and
$\#(I_u)=\#(J_u)$.

\spheader 332Yc Let $(\frak A,\bar\mu)$ and $(\frak B,\bar\nu)$ be
non-zero localizable measure algebras;  let $e_{\kappa}$, $f_{\kappa}$
be their Maharam-type-$\kappa$ components.   Show that the following are
equiveridical:  (i) $\frak A$ is isomorphic, as Boolean algebra, to an
order-closed subalgebra of a principal ideal of $\frak B$;   (ii)
$c(\frak A^*_{\lambda})\le c(\frak B^*_{\lambda})$ for every cardinal
$\lambda$, where $\frak A^*_{\lambda}$, $\frak B^*_{\lambda}$ are the
principal ideals generated by $\sup_{\kappa\ge\lambda}e_{\kappa}$ and
$\sup_{\kappa\ge\lambda}f_{\kappa}$ respectively.

}%end of exercises

\cmmnt{
\Notesheader{332}
Maharam's theorem tells us that all localizable measure algebras -- in
particular, all $\sigma$-finite measure algebras -- can be obtained from
the basic algebra $\frak A=\{0,a,1\Bsetminus a,1\}$, with $\bar\mu
a=\bar\mu(1\Bsetminus a)=\bover12$, by combining the constructions of
probability algebra free products, scalar multiples of measures and
simple products.   But what is much more important is the fact that we
get a description of our measure algebras in terms sufficiently explicit
to make a very wide variety of questions resolvable.   The description I
offer in 332J hinges on the complementary concepts of `Maharam type' and
`magnitude'.   If you like, the magnitude of a measure algebra is a
measure of its width, while its Maharam type is a measure of its depth.
The latter is more important just because, for localizable algebras, we
have such a simple decomposition into algebras of finite magnitude.   Of
course there is a good deal of scope for further complications if we
seek to consider non-localizable semi-finite algebras.   For these, the
natural starting point is a proper description of their localizations,
which is not difficult (332Yb).

Observe that 332C gives a representation of a localizable measure
algebra as the measure algebra of a measure space which is completely
different from the Stone representation in 321K.   It is less canonical
(since there is a degree of choice about the partition
$\langle e_i\rangle_{i\in I}$) but very much more informative, since the
$\kappa_i$, $\gamma_i$ carry enough information to identify the measure
algebra up to isomorphism (332K).

`Cellularity' is the second cardinal function I have introduced in this
chapter.   It refers as much to topological spaces as to Boolean
algebras (see 332Xd-332Xe).   There is an interesting question in this
context.   If $\frak A$ is an arbitrary Boolean algebra, is there
necessarily a disjoint set in $\frak A$ of cardinal $c(\frak A)$?   This
is believed to be undecidable from the ordinary axioms of set theory
(including the axiom of choice);  see the `Erd\H{o}s-Tarski theorem' in
Volume 5.
%511Db 513B 513Ya 514Nc
But for semi-finite measure algebras we have a definite answer (332F).

Maharam's classification not only describes the isomorphism classes of
localizable measure algebras, but also tells us when to expect Boolean
homomorphisms between them (332P, 332Yc).   I have given 332P only for
atomless totally finite measure algebras because the non-totally-finite
case (332Ya, 332Yc) seems to require a new idea, while atoms introduce
combinatorial complications.

I offer 332T as an example of the kind of result which these methods
make very simple.
It fails for general Boolean algebras;  in fact, there is for any $\kappa$ a
countably $\tau$-generated Dedekind complete Boolean algebra $\frak A$ with
cellularity $\kappa$ (514Yb in Volume 5, or {\smc Koppelberg 89}, 13.1),
so that $\Cal P\kappa$ is isomorphic to an order-closed subalgebra of
$\frak A$,
and if $\kappa>\frak c$ then $\tau(\Cal P\kappa)>\omega$ (332R).

For totally finite measure algebras we have a kind of weak
Schr\"oder-Bernstein theorem:  if we have two of them, each isomorphic
to a closed subalgebra of the other, they are isomorphic (332Q).   This
fails for $\sigma$-finite algebras (332Xn).   I call it a `weak'
Schr\"oder-Bernstein theorem because it is not clear how to build the
isomorphism from the two injections;  `strong'
Schr\"oder-Bernstein theorems include definite recipes for constructing
the isomorphisms declared to exist (see, for instance, 344D below).
}%end of notes

\discrpage


