\frfilename{mt433.tex}
\versiondate{27.6.10}
\copyrightdate{2003}
\def\NN{\BbbN^{\Bbb N}}
\def\tildeTau{\tilde{\text{T}}}

\def\chaptername{Topologies and measures II}
\def\sectionname{Analytic spaces}

\newsection{433}

We come now to the special properties of measures on `analytic'
spaces\cmmnt{,
that is, continuous images of $\NN$, as described in \S423}.   I start
with a couple of facts about spaces with countable networks.

\leader{433A}{Proposition} Let $(X,\frak T)$ be a topological space with
a countable network, and $\mu$ a localizable topological measure on $X$
which is inner regular with respect to the Borel sets.   Then $\mu$ has
countable Maharam type.

\proof{ Let $\tilde\mu$ be the c.l.d.\ version of $\mu$ (213E).   Then
the measure algebra $\frak A$ of $\tilde\mu$ can be identified with the
measure algebra of $\mu$ (322D(b-iii)).
Also $\tilde\mu$ is complete, locally determined and localizable, so
every subset of $X$ has a measurable envelope with respect to
$\tilde\mu$ (213J, 213L).    Let $\tilde\Sigma$ be the domain of
$\tilde\mu$, and $\Cal E$ a countable network for $\frak T$.   For each
$E\in\Cal E$, let $F_E\in\tilde\Sigma$ be a measurable envelope of $E$.

Let $\frak B$ be the order-closed subalgebra of $\frak A$ generated by
$\{F_E^{\ssbullet}:E\in\Cal E\}$, and set
$\Tau=\{F:F\in\tilde\Sigma,\,F^{\ssbullet}\in\frak B\}$.   Because
$\frak B$ is an order-closed subalgebra of $\frak A$, $\Tau$ is a
$\sigma$-subalgebra of $\tilde\Sigma$.   Now $\frak T\subseteq\Tau$.
\Prf\ If $G\in\frak T$, set $\Cal E_0=\{E:E\in\Cal E,\,E\subseteq G\}$.
Set $F=\bigcup_{E\in\Cal E_0}F_E$, so that $F\in\Tau$ and
$G\subseteq F$.   For each
$E\in\Cal E_0$, $F_E\setminus G$ is negligible, so $F\setminus G$ is
negligible, and $G^{\ssbullet}=F^{\ssbullet}\in\frak B$, so $G\in\Tau$.\
\Qed

It follows that $\Tau$ includes the Borel $\sigma$-algebra of $X$.
Because $\mu$ is inner regular with respect to the Borel sets, $\frak B$
is order-dense in $\frak A$, and $\frak B=\frak A$.   Thus the countable
set $\{F_E^{\ssbullet}:E\in\Cal E\}\,\,\tau$-generates $\frak A$, and
the Maharam
type of $\frak A$, which is the Maharam type of $\mu$, is countable.
}%end of proof of 433A

\leader{433B}{Lemma} If $(X,\frak T)$ is a Hausdorff space with a
countable network, then
any topological measure on $X$ is countably separated\cmmnt{ in the
sense of 343D}.

\proof{ By 4A2Nf, there is a countable family of open sets separating
the points of $X$.
}%end of proof of 433B used 4{}52, 4{}53, 5{}22

\vleader{48pt}{433C}{Theorem} Let $X$ be an analytic Hausdorff space,
and $\mu$ a Borel measure on $X$.

(a) If $\mu$ is semi-finite, it is tight.

(b) If $\mu$ is locally finite, its completion is a Radon measure on
$X$.

\proof{ $X$ is K-analytic (423C);  moreover, every open subset of $X$ is
again analytic (423Eb).   So 432C gives the result at once.
}%end of proof of 433C

\cmmnt{\medskip

\noindent{\bf Remark} Compare 256C.
}%end of comment

\leader{433D}{Theorem} Let $X$ and $Y$ be analytic Hausdorff spaces,
$\nu$ a totally finite Radon measure on $Y$ and $f:X\to Y$ a Borel
measurable function such that $f[X]$ is of full outer measure for $\nu$.
Then there is a Radon measure $\mu$ on $X$ such that $\nu=\mu f^{-1}$.

\proof{ By 423Ga, the graph $R$ of $f$ is an analytic set in
$X\times Y$, therefore K-analytic.   Set $\pi_1(x,y)=x$, $\pi_2(x,y)=y$
for $(x,y)\in R$, so that $\pi_1$ and $\pi_2$ are continuous.   Now
$\pi_2[R]=f[X]$ has full outer measure, so by 432G there is a Radon
measure $\lambda$ on $R$
such that $\nu=\lambda\pi_2^{-1}$.   Next, because $\pi_1$ is
continuous, the image $\mu=\lambda\pi_1^{-1}$ is a Radon measure on $X$,
by 418I.   But $\pi_2=f\pi_1$, so

\Centerline{$\mu f^{-1}=(\lambda\pi_1^{-1})f^{-1}
=\lambda(f\pi_1)^{-1}=\lambda\pi_2^{-1}=\nu$,}

\noindent as required.
}%end of proof of 433D

\leader{433E}{Proposition} Let $(X,\Sigma,\mu)$ be a semi-finite measure
space and $\frak T$ a topology on $X$ such that $\mu$ is inner regular
with respect to the closed sets.   Let $(Y,\frak S)$ be an analytic
Hausdorff space and $f:X\to Y$ a measurable function.   Then $f$ is almost
continuous.

\proof{ Take $E\in\Sigma$ and $\gamma<\mu E$.   Then there is an
$F\in\Sigma$ such that $F\subseteq E$ and $\gamma<\mu F<\infty$.   For
Borel sets $H\subseteq Y$, set $\nu H=\mu(F\cap f^{-1}[H])$.   Then
$\nu$ is a totally finite Borel measure on $Y$, so is tight (433C);
let $K\subseteq Y$ be a compact set
such that $\nu K>\gamma$, so that $\mu(F\cap f^{-1}[K])>\gamma$.   The
subspace measure on $L=F\cap f^{-1}[K]$ is still inner regular with
respect to the (relatively) closed sets (412Pc), and $f\restr L$ is
still measurable;
but $f\restr L$ is a function from $L$ to $K$, and $K$ is metrizable, by
423Dc.   So $f\restr L$ is almost continuous, by 418J, and there is
a set $F'\subseteq L$, of measure at least $\gamma$, such that
$f\restr F'$ is continuous.

As $E$ and $\gamma$ are arbitrary, $f$ is almost continuous.
}%end of proof of 433E

\cmmnt{\medskip

\noindent{\bf Remark} Compare 418Yg.}

\leader{433F}{}\cmmnt{ I give some simple corollaries of the von
Neumann-Jankow selection theorem (423M-423O).  %423M 423N 423O

\medskip

\noindent}{\bf Proposition} Let $(X,\frak T)$ and $(Y,\frak S)$ be
analytic Hausdorff spaces, and $f:X\to Y$ a Borel measurable surjection.
Let $\nu$ be a complete locally determined topological measure on $Y$,
and $\Tau$ its domain.   Then there is a $\Tau$-measurable function
$g:Y\to X$ such that $gf$ is the identity on $X$.

\proof{ By 423O we know that there is a function $g:Y\to X$ such that
$fg$ is the identity and $g$ is $\Tau_1$-measurable, where $\Tau_1$ is
the $\sigma$-algebra generated by the Souslin-F subsets of $Y$.   But
$\Tau$ contains every Souslin-F subset of $Y$, by 431B, therefore
includes $\Tau_1$, and $g$ is actually $\Tau$-measurable.
}%end of proof of 433F

\vleader{72pt}{433G}{Proposition} Let $(X,\frak T)$ be an analytic
Hausdorff
space, $(Y,\Tau,\nu)$ a complete locally determined measure
space, and $f:X\to Y$ a surjection.   Suppose that there is some
countable family $\Cal F\subseteq\Tau$ such that $\Cal F$ separates the
points of $Y$\cmmnt{ (that is, whenever $y$, $y'$ are distinct points
of $Y$ there is a member of $\Cal F$ containing one and not the other)}
and $f^{-1}[F]$ is a Borel subset of $X$ for every $F\in\Cal F$.   Then
there is a $\Tau$-measurable function $g:Y\to X$ such that $fg$ is the
identity on $Y$.

\proof{ Set $\Cal A=\Cal F\cup\{Y\setminus F:F\in\Cal F\}$.   The
topology $\frak T_1$ on $X$ generated by
$\frak T\cup\{f^{-1}[A]:A\in\Cal A\}$
is still analytic (423H).   If we take $\frak S$ to be the
topology on $Y$ generated by $\Cal A$, then $\frak S$ is Hausdorff and
$f$ is $(\frak T_1,\frak S)$-continuous, so $\frak S$ is analytic
(423Bb).

Because $\frak S$ is generated by a countable subset $\Cal A$ of $\Tau$,
it is second-countable, and
$\frak S\subseteq\Tau$ (4A3Da/4A3E).   So $\nu$ is a topological
measure with respect to $\frak S$.   By
433F, there is a function $g:Y\to X$, measurable for  $\Tau$ and the
topology $\frak T_1$, such that $gf$ is the identity on $X$;  and of
course $g$ is still measurable for $\Tau$ and the coarser original
topology $\frak T$ on $X$.
}%end of proof of 433G

\leader{433H}{Proposition}\dvAnew{2011}
Let $X$ be an analytic Hausdorff space, and $(Y,\Tau,\nu)$ a complete
locally determined measure space.   Suppose that
$W\subseteq X\times Y$ belongs to $\Cal S(\Cal B(X)\tensorhat\Tau)$,
where $\Cal B(X)$ is the Borel $\sigma$-algebra of $X$.   Then
$W[X]\in\Tau$ and there is a $\Tau$-measurable function
$g:W[X]\to X$ such that $(g(y),y)\in W$ for every $y\in W[X]$.

\proof{ Set 
$\Cal V=\Cal S(\{E\times F:E\subseteq X$ is closed, $F\in\Tau\})$.   Then 
$\Cal V$ contains $H\times Y$ for every Souslin-F
subset $H$ of $X$ (421Cb), and therefore for every $H\in\Cal B(X)$ (423Eb);
by 421F, it follows that $\Cal V$ includes $\Cal B(Y)\tensorhat\Tau$ and
therefore $W\in\Cal S(\Cal V)=\Cal V$ (421D).   By 423M,
$W[X]\in\Cal S(\Tau)$, which by 431A is just $\Tau$, and there is
a $\Tau$-measurable function which is a selector for $W^{-1}$.
}%end of proof of 431H

\leader{433I}{}\cmmnt{ Because analytic spaces have countable networks
(423C), and their compact subsets are therefore metrizable
(423Dc), their measure theory is very close to that of $\Bbb R$ or
$[0,1]$ or $\{0,1\}^{\Bbb N}$.   I give some simple manifestations of
this principle.

\medskip

\noindent}{\bf Proposition}\dvAformerly{4{}33H}
Let $\familyiI{X_i}$ be a family of analytic
Hausdorff spaces, and for each $i\in I$ let $\mu_i$ be a Radon probability measure
on $X_i$.   Let $\lambda$
be the ordinary product measure on $X=\prod_{i\in I}X_i$\cmmnt{, as
defined in \S254}.

(a) If $I$ is countable then $\lambda$ is a Radon measure.

(b) If every $\mu_i$ is strictly positive, then $\lambda$ is a
quasi-Radon measure.

\proof{{\bf (a)} In this case, $X$ is analytic (423Bc), therefore
hereditarily Lindel\"of (423Da).   Let $\Lambda$ be the domain of
$\lambda$ and
$\frak T$ the topology of $X$.  Then $\Lambda\cap\frak T$ is a base for
$\frak T$;  by 4A3Da, $\frak T\subseteq\Lambda$.   By 417Sb, $\lambda$
is the $\tau$-additive product measure on $X$;  by 417Q, this is a Radon
measure.

\medskip

{\bf (b)} By (a), the ordinary product measure on $\prod_{i\in J}X_i$ is
a Radon measure for every finite set $J\subseteq I$.   So 417Sc tells us
that $\lambda$ is the $\tau$-additive product measure on $X$;
by 417O, this is a quasi-Radon measure.
}%end of proof of 433I

\leader{433J}{Proposition}\dvAformerly{4{}33I}
Let $X$ be an analytic Hausdorff space, and
$\Tau$ a countably generated $\sigma$-subalgebra of the Borel
$\sigma$-algebra $\Cal B(X)$ of $X$.   Then
any locally finite measure with domain $\Tau$ has an extension
to a Radon measure on $X$.

\proof{ Let $\mu_0$ be a locally finite measure with domain $\Tau$.

\medskip

{\bf (a)} Consider first the case in which $\mu_0$ is totally
finite.   Let $\sequencen{F_n}$ be a sequence in $\Tau$ generating
$\Tau$ as $\sigma$-algebra.   Define $f:X\to\{0,1\}^{\Bbb N}$ by setting

\Centerline{$f(x)(n)=\chi F_n(x)$ for $n\in\Bbb N$, $x\in X$.}

\noindent Then $f$ is $\Tau$-measurable
(use 418Bd), so we have a Borel measure $\nu_0$
on $\{0,1\}^{\Bbb N}$ defined by setting $\nu_0E=\mu_0f^{-1}[E]$ for
every Borel set $E\subseteq\{0,1\}^{\Bbb N}$.   Now the completion $\nu$
of $\nu_0$ is a Radon
measure (433C).   Also $f[X]$ must be analytic, by 423Gb, because $f$ is
$\Cal B(X)$-measurable.    So $\nu$ measures $f[X]$ (432A), and

\Centerline{$\nu f[X]=\nu_0^*f[X]=\nu_0\{0,1\}^{\Bbb N}$,}

\noindent that is, $f[X]$ is $\nu$-conegligible.   By 433D, there is a
Radon measure $\mu$ on $X$ such that $\nu=\mu f^{-1}$.

Because every $F_n$ is expressible as $f^{-1}[E]$ for some
$E\in\Cal B(\{0,1\}^{\Bbb N})$,
so is every member of $\Tau$.   If $F\in\Tau$, take
$H\in\Cal B(\{0,1\}^{\Bbb N})$ such
that $F=f^{-1}[H]$;  then

\Centerline{$\mu F=\nu H=\nu_0H=\mu_0F$.}

\noindent Thus $\mu$ extends $\mu_0$ and $\mu\restr\Sigma$ will serve.

\medskip

{\bf (b)} In general, because $X$ is Lindel\"of and $\mu_0$ is locally
finite, $\mu_0$ is $\sigma$-finite.   Let $\sequencen{X_n}$ be a
partition of $X$ into members of $\Tau$ such that $\mu_0X_n$ is
finite for every $n$, and set $\mu^{(n)}_0F=\mu_0(F\cap X_n)$ for every
$n$ and every $F\in\Tau$;  then every $\mu^{(n)}_0$
has an extension to a Radon measure $\mu^{(n)}$.
Let $\mu$ be the sum $\sum_{n=0}^{\infty}\mu^{(n)}$
(234G\formerly{1{}12Ya}).
Of course $\mu$ extends $\mu_0=\sum_{n=0}^{\infty}\mu^{(n)}_0$.
Because $\mu_0$ is locally finite, so is $\mu$, and $\mu$ is a Radon
measure (416De).
}%end of proof of 433K

\leader{433K}{}\cmmnt{ I turn now to a brief mention of `standard
Borel spaces'.   From the point of view of this chapter, it is natural
to regard the following results as simple corollaries of theorems about
Polish spaces.   But, as remarked in \S424, there are cases in which a
standard Borel space is presented without any specific topology being
attached;  and in any case it is interesting to look at the ways in
which we can express these ideas as theorems about $\sigma$-algebras
rather than about topological spaces.

\medskip

\noindent}{\bf Proposition}\dvAformerly{4{}33J}
Let $(X,\Sigma)$ be a standard Borel space
and $\Tau$ a countably generated $\sigma$-subalgebra of $\Sigma$.   Then
any $\sigma$-finite measure with domain $\Tau$ has an extension
to $\Sigma$.

\proof{ Let $\mu_0$ be a $\sigma$-finite measure with domain $\Tau$.

\medskip

{\bf (a)} If $\mu_0$ is totally finite, give $X$ a Polish topology for
which $\Sigma$ is the Borel $\sigma$-algebra of $X$, and use 433J.

\medskip

{\bf (b)} In general, let $\sequencen{X_n}$ be a partition of $X$
into members of $\Tau$ such that $\mu_0X_n<\infty$ for every $n$, and
set $\mu^{(n)}_0F=\mu_0(F\cap X_n)$ for every $n$ and every $F\in\Tau$;
then every $\mu^{(n)}_0$
has an extension to a measure $\mu^{(n)}$ with domain $\Sigma$, and we
can set $\mu=\sum_{n=0}^{\infty}\mu^{(n)}$.
}%end of proof of 433K

\leader{433L}{Proposition}\dvAformerly{4{}33K}
Let $\sequencen{(X_n,\Sigma_n,\mu_n)}$ be a
sequence of probability spaces such that $(X_n,\Sigma_n)$ is a standard
Borel space for every $n$.   Suppose that for each $n\in\Bbb N$ we are
given an \imp\ function $f_n:X_{n+1}\to X_n$.   Then we can find a
standard Borel space $(X,\Sigma)$, a probability measure $\mu$ with
domain $\Sigma$, and \imp\ functions $g_n:X\to X_n$ such that
$f_ng_{n+1}=g_n$ for every $n$.

\proof{ For each $n$, choose a Polish topology $\frak T_n$ on $X_n$ such
that $\Sigma_n$ is the algebra of $\frak T_n$-Borel sets.  Let
$\hat\mu_n$ be the completion of $\mu_n$;  then $\hat\mu_n$ is a Radon
measure (433C).   Every $f_n$ is \imp\ for $\hat\mu_{n+1}$ and
$\hat\mu_n$, by 234Ba\formerly{2{}35Hc}, and almost continuous, by 418J.

By 418Q, we have a Radon measure $\hat\mu$ on

\Centerline{$X=\{x:x\in\prod_{n\in\Bbb N}X_n,\,f_n(x(n+1))=x(n)$ for
every $n\in\Bbb N\}$}

\noindent such that the continuous maps $x\mapsto x(n)=g_n(x):X\to X_n$
are \imp\ for every $n$.   Now $X$ is a Borel subset of
$Z=\prod_{n\in\Bbb N}X_n$.   \Prf\  For each $n\in\Bbb N$, let
$\Cal U_n$ be a countable base for $\frak T_n$.   Then

\Centerline{$Z\setminus X
=\bigcup_{n\in\Bbb N}\bigcup_{U,V\in\Cal U_n,U\cap V=\emptyset}
  \{z:z(n)\in U,\,f_n(z(n+1))\in V\}$}

\noindent is a countable union of Borel sets because $\{z:z(n)\in U\}$
is open and $\{z:z(n+1)\in f_n^{-1}[V]\}$ is Borel whenever $n\in\Bbb N$
and $U$, $V\in\Cal U_n$.   So $Z\setminus X$ and $X$ are Borel sets.\
\Qed

Accordingly $(X,\Sigma)$ is a standard Borel space, where $\Sigma$ is
the Borel $\sigma$-algebra of $X$ (424G).   So if we take
$\mu=\hat\mu\restr\Sigma$, we shall have a suitable measure on $X$.
}%end of proof of 433L

\exercises{
\leader{433X}{Basic exercises (a)}
%\spheader 433Xa
Let $(X,\frak T)$ be a topological space with a
countable network, and $\mu$ a topological measure on $X$ which is inner
regular with respect to the Borel sets and has the measurable envelope
property (213Xl).   Show that $\mu$ has countable Maharam type.
%433A

\spheader 433Xb Show that an effectively locally finite measure on a
hereditarily Lindel\"of space (in particular, on any analytic
Hausdorff space) is $\sigma$-finite.
%433B

\spheader 433Xc Let $X\subseteq[0,1]$ be a set of Lebesgue outer measure
$1$ and inner measure $0$.  Show that the subspace measure on $X$ is a
totally finite Borel measure which is not tight.
%433C

\spheader 433Xd Let $X$ be a Hausdorff space and $\mu$ a locally finite
measure on $X$, inner regular with respect to the Borel sets, such that
$\dom\mu$ includes a base for the topology of $X$.   Suppose that
$Y\subseteq X$ is an analytic set of full outer measure.   Show that
$\mu$ has a unique extension to a Radon measure $\tilde\mu$ on $X$, and
that $Y$ is $\tilde\mu$-conegligible.
%433C

\spheader 433Xe Let $(X,\Sigma)$ be a standard Borel space.   (i) Show
that any semi-finite measure with domain $\Sigma$ is a compact measure
(definition: 342Ac, or 451Ab below), therefore perfect.
\Hint{if $X$ is given a
suitable topology, the measure is tight.}
(ii) Show that any measure with domain
including $\Sigma$ is countably separated.
%433C

\sqheader 433Xf (i) Let $(X,\Sigma,\mu)$ and $(Y,\Tau,\nu)$ be atomless
probability spaces such that $(X,\Sigma)$ and $(Y,\Tau)$ are standard
Borel spaces.   Show that $(X,\Sigma,\mu)$ and $(Y,\Tau,\nu)$ are
isomorphic.   \Hint{by 344I, their completions are isomorphic;  by 344H,
they have negligible sets of cardinal $\frak c$;  show that any
isomorphism between the completions is $(\Sigma,\Tau)$-measurable on a
conegligible set;  use 424Da to match residual negligible sets.}
(ii) Let $X$ be a Polish space and $\mu$ an atomless Radon measure on $X$.
Show that there is a Borel isomorphism between $X$ and $[0,1]$ which
matches $\mu$ to Lebesgue measure on $[0,1]$.
%433Xe, 433C

\spheader 433Xg Let $X$ be $[0,1]\times\{0,1\}$, with its usual
topology, and $I^{\|}$ the split interval (419L);  define
$f:X\to I^{\|}$ by
setting $f(t,0)=t^-$, $f(t,1)=t^+$ for $t\in[0,1]$.   (i) Give $I^{\|}$
its usual Radon measure $\nu$ (343J, 419Lc).   Show that there is no
Radon measure
$\lambda$ on $X$ such that $\nu=\lambda f^{-1}$.   (ii) Let $\mu$ be the
product Radon probability measure on $X$, starting from Lebesgue measure
on $[0,1]$ and the usual fair-coin measure on $\{0,1\}$.   Show that $f$
is \imp\ for $\mu$ and
$\nu$.   Show that $f$ is not almost continuous.
%433E

\leader{433Y}{Further exercises (a)}
%\spheader 433Ya
Find a Hausdorff topological space $X$ with a countable network and a
semi-finite Borel measure on $X$ which does not have countable Maharam
type.
%433A %mt43bits

\spheader 433Yb
Let $X$ be an analytic Hausdorff space and $\mu$ an atomless
Radon measure on $X$.   Show that $(X,\mu)$ is isomorphic to Lebesgue
measure on some measurable subset of $\Bbb R$.   \Hint{344I.}
%

\spheader 433Yc Let $(X,\frak T)$ be a Polish space without isolated
points, and $\mu$ a $\sigma$-finite topological measure on $X$.
Show that there is a conegligible meager set.   \Hint{Show that every
non-empty open set is uncountable.   Find a countable dense negligible
set and a negligible G$_{\delta}$ set including it.}
%/

\spheader 433Yd Let $\sequencen{X_n}$ be a sequence of analytic
Hausdorff spaces
and for each $n\in\Bbb N$ let $\mu_n$ be a Borel probability measure on
$X_n$.   Suppose that for each $n\in\Bbb N$ we are
given an \imp\ function $f_n:X_{n+1}\to X_n$.   Show that we can find a
standard Borel space $(X,\Sigma)$, a probability measure $\mu$ with
domain $\Sigma$, and \imp\ functions $g_n:X\to X_n$ such that
$f_ng_{n+1}=g_n$ for every $n$.
%433L
}%end of exercises

\cmmnt{\Notesheader{433}
The measure-theoretic results of 433C-433E %433C 433D 433E
are of much the same
type as those in \S432.   A characteristic difference is that Borel
measurable functions between analytic spaces behave in many ways like
continuous functions.   (Compare 433D and 432G.)   You may feel that
423Yc offers some explanation of this.   For
any question which refers to the Borel algebra of an analytic space $X$,
or to the class of its analytic subsets, we can expect to be able to
suppose that $X$ is separable and metrizable (see 423Xd), and that any
single Borel measurable function appearing is continuous.   (424H is a
particularly remarkable instance of this principle.)

433I here amounts to spelling out a special case of ideas already
treated in 417S.   As this territory is relatively unfamiliar, I give
detailed examples
(423Xi, 433Xc, 433Xg, 439A, 439K) to show that the theorems of this
section are not generally valid for compact Hausdorff spaces (the
archetype of
K-analytic spaces which need not be analytic), nor for separable metric
spaces (the archetypical spaces with countable network).   They really
do depend on the particular combination of properties possessed by
analytic spaces.

For large parts of probability theory, standard Borel spaces provide an
adequate framework, and have a number of advantages;  some of the
technical problems concerning measurability which loom rather large in
this treatise disappear in such contexts.   Many authors accordingly
give them great prominence.   I myself believe that the simplifications
are an entrapment rather than a liberation, that sooner or later
everyone has to leave the comfortable environment of Borel algebras on
Polish spaces, and that it is better to be properly equipped with a
suitable general theory when one does.   But it is surely important to
know what the simplifications are, and the results in 433K-433L will I
hope show at least that there are wonderful ideas here, even if my own
presentation tends to leave them on one side.

}%end of notes

\discrpage

