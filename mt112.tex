\frfilename{mt112.tex}
\versiondate{20.2.05/20.8.08}

\def\chaptername{Measure spaces}
\def\sectionname{Measure spaces}
\copyrightdate{1994}

\newsection{112}

We are now, I hope, ready for the second major definition, the
definition on which all the work of this treatise is based.

\leader{112A}{Definition} A {\bf measure space} is a triple
$(X,\Sigma,\mu)$ where

\quad(i) $X$ is a set;

\quad(ii) $\Sigma$ is a $\sigma$-algebra of subsets of $X$;

\quad(iii) $\mu:\Sigma\to[0,\infty]$ is a function such that

\qquad($\alpha$) $\mu\emptyset=0$;

\qquad($\beta$) if $\langle E_n\rangle_{n\in\Bbb N}$ is a disjoint
sequence in $\Sigma$, then
$\mu(\bigcup_{n\in\Bbb N}E_n)=\sum_{n=0}^{\infty}\mu E_n$.

\noindent In this context, members of $\Sigma$ are called {\bf
measurable} sets, and $\mu$ is called a {\bf measure on $X$}.

\leader{112B}{Remarks}\cmmnt{ {\bf (a) The use of $\infty$} In (iii) of
the definition
above, I declare that $\mu$ is to be a function taking values in
`$[0,\infty]$', that is, the set comprising the non-negative real
numbers with `$\infty$' adjoined.   I expect that you have already
encountered various uses of the symbol $\infty$ in analysis;   I hope
you have realised that it means rather different things in different
contexts, and that it is necessary to establish clear conventions for
its use each time.   The `$\infty$ of measure' corresponds to the
notion of infinite length or area or volume.   The basic operation we
need to perform on it is addition:  $\infty+a=a+\infty=\infty$ for every
$a\in[0,\infty\roibr$ (that is, every real number $a\ge 0$), and
$\infty+\infty=\infty$.   This renders $[0,\infty]$ a semigroup under
addition.   It will be reasonably safe to declare
$\infty-a=\infty$ for every $a\in\Bbb R$;  but we must
absolutely decline to interpret the formula $\infty-\infty$.   As for
multiplication, it turns out that it is usually right to interpret
$\infty\cdot\infty$, $a\cdot\infty$ and $\infty\cdot a$ as $\infty$ for
$a>0$, while $0\cdot\infty=\infty\cdot 0$ can generally be taken as $0$.

We also have a natural total ordering of $[0,\infty]$, writing
$a<\infty$ for every $a\in\coint{0,\infty}$.   This gives an idea of
supremum and infimum of an arbitrary (non-empty) subset of $[0,\infty]$;
and it will often be right to interpret $\inf\emptyset$ as $\infty$, but
I will try to signal this particular convention each time it is
relevant.
We also have a notion of limit;  if
$\langle u_n\rangle_{n\in\Bbb N}$ is a sequence in $[0,\infty]$, then it converges to $u\in[0,\infty]$
if

\qquad {for every $v<u$ there is an $n_0\in\Bbb N$ such that $v\le
u_n$ for every $n\ge n_0$,}

\qquad {for every $v>u$ there is an $n_0\in\Bbb N$ such that $v\ge
u_n$ for every $n\ge n_0$.}

\noindent Of course if $u=0$ or $u=\infty$ then one of these clauses
will be vacuously satisfied.

(See also \S135.)

\header{112Bb}{\bf (b)} I should say plainly what I mean by a `disjoint'
sequence:
a sequence $\langle E_n\rangle_{n\in\Bbb N}$ is {\bf disjoint} if no
point belongs to more than one $E_n$, that is, if $E_m\cap
E_n=\emptyset$ for all distinct $m$, $n\in\Bbb N$.   Note that there is
no bar here on one, or many, of the $E_n$ being the empty set.

Similarly, if $\langle E_i\rangle_{i\in I}$ is a family of sets indexed
by an arbitrary set $I$, it is {\bf disjoint} if $E_i\cap E_j=\emptyset$
for all distinct $i$, $j\in I$.

\medskip

} {\bf (c)} In interpreting clause (iii-$\beta$) of the
definition above,
we need to assign values to sums $\sum_{n=0}^{\infty}u_n$ for arbitrary
sequences $\langle u_n\rangle_{n\in\Bbb N}$ in $[0,\infty]$.
\cmmnt{The
natural way to do this is to say that
$\sum_{n=0}^{\infty}u_n=\lim_{n\to\infty}\sum_{m=0}^nu_m$,
using the definitions sketched in (a).} %end of comment
If one of the $u_m$ is itself infinite, \cmmnt{say $u_k=\infty$,
then $\sum_{m=0}^nu_m=\infty$ for every
$n\ge k$, so of course} $\sum_{n=0}^{\infty}u_n=\infty$.
If all the $u_m$ are finite, then\cmmnt{, because they are all
non-negative,} the sequence $\langle\sum_{m=0}^nu_m\rangle_{n\in\Bbb N}$
of partial
sums is monotonic non-decreasing, and either has a finite limit
$\sum_{n=0}^{\infty}u_n\in\Bbb R$, or diverges to $\infty$;  in which
case we again interpret $\sum_{n=0}^{\infty}u_n$ as $\infty$.

\header{112Bd}{\bf (d)}\cmmnt{ Once again, the important examples of
measure spaces will have
to wait until \S\S114 and 115 below.   However, I can describe
immediately one particular class of measure space, which should always
be borne in
mind, though it does not give a good picture of the most important and
interesting parts of the subject.}   Let $X$ be any set, and let
$h:X\to[0,\infty]$ be any function.   For every $E\subseteq X$ write
$\mu E=\sum_{x\in E}h(x)$.   To interpret this sum, note that there is
no difficulty for
finite sets $E$ (taking $\sum_{x\in\emptyset}h(x)=0$), while for
infinite sets $E$ we can take
$\sum_{x\in E}h(x)=\sup\{\sum_{x\in I}h(x):I\subseteq E$ is finite$\}$,
because
every $h(x)$ is non-negative.   \cmmnt{(You may well prefer to think
about this at first with $X=\Bbb N$, so that
$\sum_{n\in E}h(n)=\lim_{n\to\infty}\sum_{m\in E,m\le n}h(m)$;
but I hope that a little thought will show you that the
general case, in which $X$ may even be uncountable, is not really more
difficult.)}   Now $(X,\Cal PX,\mu)$ is a measure space.

\cmmnt{We are very far from being ready for the specialized vocabulary
needed to
describe different kinds of measure space, but when the time comes} I will call
measures of this kind {\bf point-supported}.

Two particular cases recur often enough to be worth giving names to.
If $h(x)=1$ for every $x$, then $\mu E$
is just the number of points of $E$ if $E$ is finite, and is $\infty$ if
$E$ is infinite.   I will call this {\bf counting measure} on $X$.
If $x_0\in X$, we can set $h(x_0)=1$ and $h(x)=0$ for
$x\in X\setminus\{x_0\}$;  then $\mu E$ is $1$ if $x_0\in E$, and $0$ for
other $E$.   I will call this the
{\bf Dirac measure on $X$ concentrated at $x_0$}.
Another simple example is with $X=\Bbb N$, $h(n)=2^{-n-1}$ for
every $n$;  then $\mu X={1\over 2}+{1\over 4}+\ldots=1$.

\spheader 112Be If $(X,\Sigma,\mu)$ is a measure space\cmmnt{, then
$\Sigma$ is the domain of the function $\mu$, and $X$ is the largest
member of $\Sigma$.   We can therefore recover the whole triplet
$(X,\Sigma,\mu)$ from its final component $\mu$.   This is not a game
which is worth playing at this stage.   However, it is convenient on
occasion to introduce a measure without immediately giving a name to its
domain, and when I do this} I may say that `$\mu$ {\bf measures} $E$' or
`$E$ is {\bf measured by} $\mu$' to mean that $\mu E$ is
defined\cmmnt{, that is, that $E$ belongs to the $\sigma$-algebra
$\dom\mu$}.   \cmmnt{{\bf Warning!} Many authors use the phrase
`$\mu$-measurable set' to mean something a little different from what I am
discussing here.}

\leader{112C}{Elementary properties of measure spaces} Let
$(X,\Sigma,\mu)$ be a measure space.

(a) If $E$, $F\in\Sigma$ and $E\cap F=\emptyset$ then $\mu(E\cup F)=\mu
E+\mu F$.

(b) If $E$, $F\in\Sigma$ and $E\subseteq F$ then $\mu E\le \mu F$.

(c) $\mu(E\cup F)\le\mu E+\mu F$ for any $E$, $F\in\Sigma$.

(d) If $\langle E_n\rangle_{n\in\Bbb N}$ is any sequence in $\Sigma$,
then $\mu(\bigcup_{n\in\Bbb N}E_n)\le\sum_{n=0}^{\infty}\mu E_n$.

(e) If $\langle E_n\rangle_{n\in\Bbb N}$ is a non-decreasing sequence in
$\Sigma$ (that is, $E_n\subseteq E_{n+1}$ for every $n\in\Bbb N$) then

\Centerline{$\mu(\bigcup_{n\in\Bbb N}E_n)=\lim_{n\to\infty}\mu
E_n=\sup_{n\in\Bbb N}\mu E_n$.}

(f) If $\langle E_n\rangle_{n\in\Bbb N}$ is a non-increasing sequence in
$\Sigma$ (that is, $E_{n+1}\subseteq E_{n}$ for every $n\in\Bbb N$), and
if some $\mu E_n$ is finite, then

\Centerline{$\mu(\bigcap_{n\in\Bbb N}E_n)
=\lim_{n\to\infty}\mu E_n=\inf_{n\in\Bbb N}\mu E_n$.}

\proof{{\bf (a)} Set $E_0=E$, $E_1=F$, $E_n=\emptyset$ for $n\ge
2$;  then $\langle E_n\rangle_{n\in\Bbb N}$ is a disjoint sequence in
$\Sigma$ and $\bigcup_{n\in\Bbb N}E_n=E\cup F$, so

\Centerline{$\mu(E\cup
F)=\sum_{n=0}^{\infty}\mu E_n=\mu E+\mu F$}

\noindent (because $\mu\emptyset=0$).

\medskip

{\bf (b)} $F\setminus E\in\Sigma$ (111Dc) and
$\mu(F\setminus E)\ge 0$ (because all values of $\mu$ are in $[0,\infty]$);
so (using (a))

\Centerline{$\mu F=\mu E+\mu(F\setminus E)\ge\mu E$.}

\medskip

{\bf (c)} $\mu(E\cup F)=\mu E+\mu(F\setminus E)$, by (a), and
$\mu(F\setminus E)\le \mu F$, by (b).

\medskip

{\bf (d)} Set $F_0=E_0$, $F_n=E_n\setminus\bigcup_{i<n}E_i$ for $n\ge
1$;  then $\langle F_n\rangle_{n\in\Bbb N}$ is a disjoint sequence in
$\Sigma$, $\bigcup_{n\in\Bbb N}F_n=\bigcup_{n\in\Bbb N}E_n$ and
$F_n\subseteq E_n$ for every $n$.   By (b) just above, $\mu F_n\le\mu
E_n$ for each $n$;  so

\Centerline{$\mu(\bigcup_{n\in\Bbb N}E_n)
=\mu(\bigcup_{n\in\Bbb N}F_n)=\sum_{n=0}^{\infty}\mu
F_n\le\sum_{n=0}^{\infty}\mu E_n$.}

\medskip

{\bf (e)} Set $F_0=E_0$, $F_n=E_n\setminus E_{n-1}$ for $n\ge 1$;  then
$\langle F_n\rangle_{n\in\Bbb N}$ is a disjoint sequence in $\Sigma$ and
$\bigcup_{n\in\Bbb N}F_n=\bigcup_{n\in\Bbb N}E_n$.   Consequently
$\mu(\bigcup_{n\in\Bbb N}E_n)=\sum_{n=0}^{\infty}\mu F_n$.   But an easy
induction on $n$, using (a) for the inductive step, shows that $\mu
E_n=\sum_{m=0}^n\mu F_m$ for every $n$.   So

\Centerline{$\sum_{n=0}^{\infty}\mu
F_n=\lim_{n\to\infty}\sum_{m=0}^n\mu F_m=\lim_{n\to\infty}\mu E_n$.}

\noindent Finally, $\lim_{n\to\infty}\mu E_n=\sup_{n\in\Bbb N}\mu E_n$
because (by
(b)) $\langle\mu E_n\rangle_{n\in\Bbb N}$ is non-decreasing.

\medskip

{\bf (f)} Suppose that $\mu E_k<\infty$.   Set $F_n=E_k\setminus
E_{k+n}$ for $n\in\Bbb N$, $F=\bigcup_{n\in\Bbb N}F_n$;  then $\langle
F_n\rangle_{n\in\Bbb N}$ is a non-decreasing sequence in $\Sigma$, so
$\mu
F=\lim_{n\to\infty}\mu F_n$, by (e) just above.   Also, $\mu F_n+\mu
E_{k+n}=\mu E_k$;  because $\mu E_k<\infty$, we may safely write $\mu
F_n=\mu E_k-\mu E_{k+n}$, so that

\Centerline{$\mu F=\lim_{n\to\infty}(\mu E_k-\mu
E_{k+n})=\mu E_k-\lim_{n\to\infty}\mu E_n$.}

\noindent   Next, $F\subseteq E_k$, so
$\mu F+\mu(E_k\setminus F)=\mu E_k$, and (again because $\mu E_k$ is
finite) $\mu F=\mu E_k-\mu(E_k\setminus F)$.   Thus we must have
$\mu(E_k\setminus F)=\lim_{n\to\infty}\mu E_n$.   But $E_k\setminus F$
is just $\bigcap_{n\in\Bbb N}E_n$.

Finally, $\lim_{n\to\infty}\mu E_n=\inf_{n\in\Bbb N}\mu E_n$ because
$\langle\mu E_n\rangle_{n\in\Bbb N}$ is non-increasing.
}%end of proof of 112C

\medskip

\cmmnt{\noindent{\bf Remark} Observe that in (f) above it is essential
to have
$\inf_{n\in\Bbb N}\mu E_n<\infty$.   The construction in 112Bd is
already enough to show this.   Take $X=\Bbb N$ and let $\mu$ be counting
measure on $X$.   Set $E_n=\{i:i\in\Bbb N,\,i\ge n\}$ for each $n$.
Then $E_{n+1}\subseteq E_n$ for each $n$, but

\Centerline{$\mu(\bigcap_{n\in\Bbb N}E_n)
=\mu\emptyset=0<\infty=\lim_{n\to\infty}\mu E_n$.}
}%end of comment


\leader{112D}{Negligible sets} Let $(X,\Sigma,\mu)$ be any measure
space.

\header{112Da}{\bf (a)} A set $A\subseteq X$ is {\bf negligible} (or
{\bf null}) if
there is a set $E\subseteq\Sigma$ such that $A\subseteq E$ and
$\mu E=0$.   (If there seems to be a possibility of doubt about which measure is involved, I will write {\bf $\mu$-negligible}.)

\header{112Db}{\bf (b)} Let $\Cal N$ be the family of negligible subsets
of $X$.
Then (i) $\emptyset\in\Cal N$ (ii) if $A\subseteq B\in\Cal N$ then
$A\in\Cal N$ (iii) if $\langle A_n\rangle_{n\in\Bbb N}$ is any sequence
in $\Cal N$, $\bigcup_{n\in\Bbb N}A_n\in\Cal N$.   \prooflet{\Prf\ (i)
$\mu(\emptyset)=0$.   (ii) There is an $E\in\Sigma$ such that $\mu E=0$
and $B\subseteq E$;  now $A\subseteq E$.   (iii) For each $n\in\Bbb N$
choose an $E_n\in\Sigma$ such that $A_n\subseteq E_n$ and $\mu E_n=0$.
Now $E=\bigcup_{n\in\Bbb N}E_n\in\Sigma$ and
$\bigcup_{n\in\Bbb N}A_n\subseteq\bigcup_{n\in\Bbb N}E_n$, and
$\mu(\bigcup_{n\in\Bbb N}E_n)\le\sum_{n=0}^{\infty}\mu E_n$, by 112Cd,
so $\mu(\bigcup_{n\in\Bbb N}E_n)=0$.  \Qed}

I will call $\Cal N$ the {\bf null ideal} of the measure $\mu$.
(A family of sets satisfying the conditions (i)-(iii) here is called a
{\bf $\sigma$-ideal} of sets.)

\header{112Dc}{\bf (c)} A set $A\subseteq X$ is {\bf conegligible} if
$X\setminus A$
is negligible\cmmnt{;  that is, there is a measurable set
$E\subseteq A$
such that $\mu(X\setminus E)=0$}.   Note that (i) $X$ is conegligible
(ii) if $A\subseteq B\subseteq X$ and $A$ is conegligible then $B$ is
conegligible (iii) if $\langle A_n\rangle_{n\in\Bbb N}$ is a sequence of
conegligible sets, then $\bigcap_{n\in\Bbb N}A_n$ is conegligible.

\header{112Dd}{\bf (d)}\cmmnt{ It is convenient, and customary, to use
some relatively informal language concerning negligible sets.}   If
$P(x)$ is
some assertion applicable to members $x$ of the set $X$, we say that

\Centerline{`$P(x)$ for almost every $x\in X$'}

\noindent or

\Centerline{`$P(x)$ a.e.\ $(x)$'}

\noindent or

\Centerline{`$P$ almost everywhere',\quad `$P$ a.e.'}

\noindent or\cmmnt{, if it seems necessary to specify the measure
involved,}

\Centerline{`$P(x)$ for $\mu$-almost every $x$',
\quad`$P(x)\,\mu$-a.e.$(x)$',
\quad`$P\,\mu$-a.e.',}

\noindent to mean that

\Centerline{$\{x:x\in X,\,P(x)\}$}

\noindent is conegligible in $X$\cmmnt{, that is, that

\Centerline{$\{x:x\in X,\,P(x)$ is false$\}$}

\noindent is negligible}.   Thus\cmmnt{, for instance,} if
$f:X\to\Bbb R$ is a function, `$f>0$ a.e.' means that $\{x:f(x)\le 0\}$ is negligible.

\cmmnt{\spheader 112De The phrases `{\bf almost surely}' ({\bf
a.s.}), `{\bf presque partout}' ({\bf p.p.}) are also used for `almost
everywhere'.
}%end of comment

\cmmnt{\header{112Df}{\bf (f)} I should call your attention to the
fact that, on my
definitions, a negligible set need not itself be measurable, though it
must be included in some negligible measurable set.   (Measure spaces in
which all negligible sets are measurable are called
{\bf complete}.   I will return to this question in \S211.)
}%end of comment

\spheader 112Dg When $f$ and $g$ are real-valued functions defined on
conegligible subsets of a measure space, I will write $f\eae g$,
$f\leae g$ or $f\geae g$ to mean, respectively,

\Centerline{$f=g$ a.e., that is,
$\{x:x\in\dom(f)\cap\dom(g)$, $f(x)=g(x)\}$ is conegligible,}

\Centerline{$f\le g$ a.e., that is,
$\{x:x\in\dom(f)\cap\dom(g)$, $f(x)\le g(x)\}$ is conegligible,}

\Centerline{$f\ge g$ a.e., that is,
$\{x:x\in\dom(f)\cap\dom(g)$, $f(x)\ge g(x)\}$ is conegligible.}

\exercises{
\leader{112X}{Basic exercises $\pmb{>}$(a)}
%\sqheader 112Xa
Let $(X,\Sigma,\mu)$ be a
measure space.   Show that (i)
$\mu(E\cup F)+\mu(E\cap F)=\mu E+\mu F$ (ii)
$\mu(E\cup F\cup G)+\mu(E\cap F)+\mu(E\cap G)+\mu(F\cap G)=\mu E+\mu
F+\mu G+\mu(E\cap F\cap G)$ for all $E$, $F$, $G\in\Sigma$.   Generalize
these results to longer sequences of sets.   (You may prefer to begin
with the case in which $\mu E$, $\mu F$ and $\mu G$ are all finite.
But I hope you will be able to find arguments which deal with the
general case.)
%112C

\sqheader 112Xb Let $(X,\Sigma,\mu)$ be a measure space and
$\langle E_n\rangle_{n\in\Bbb N}$ a sequence in $\Sigma$.   Show that

\Centerline{$\mu(\bigcup_{n\in\Bbb N}\bigcap_{m\ge n}E_m)
\le\liminf_{n\to\infty}\mu E_n$.}
%112C

\spheader 112Xc Let $(X,\Sigma,\mu)$ be a measure space, and
$E$, $F\in\Sigma$;  suppose that $\mu E<\infty$.   Show that
$\mu(F\symmdiff E)=\mu F-\mu E+2\mu(E\setminus F)$.
%112C

\spheader 112Xd Let $(X,\Sigma,\mu)$ be a measure space and
$\sequencen{E_n}$ a sequence of measurable sets such that
$\mu(\bigcup_{n\in\Bbb N}E_n)<\infty$.   (i) Show that
$\limsup_{n\to\infty}\mu E_n
\le\mu(\bigcap_{n\in\Bbb N}\bigcup_{m\ge n}E_m)$.   (ii) Show that
if $\bigcap_{n\in\Bbb N}\bigcup_{m\ge n}E_m=E
=\bigcup_{n\in\Bbb N}\bigcap_{m\ge n}E_m$ then
$\lim_{n\to\infty}\mu E_n$ exists and is equal to $\mu E$.
%112Xb 112C

\sqheader 112Xe Let $(X,\Sigma,\mu)$ be a measure space, and $\eusm F$
the set of
real-valued functions whose domains are conegligible subsets of $X$.
(i) Show that $\{(f,g):f,\,g\in\eusm F,\,f\leae g\}$ and
$\{(f,g):f,\,g\in\eusm F,\,f\geae g\}$ are reflexive transitive relations
on $\eusm F$, each the inverse of the other.   (ii) Show that
$\{(f,g):f,\,g\in\eusm F,\,f\eae g\}$ is their intersection, and is an
equivalence relation on $\eusm F$.
%112D

\spheader 112Xf Let $(X,\Sigma,\mu)$ be a measure space, $Y$ a set, and
$\phi:X\to Y$ a function.   Set
$\Tau=\{F:F\subseteq Y$, $\phi^{-1}[F]\in\Sigma\}$ and
$\nu F=\mu\phi^{-1}[F]$ for $F\in\Tau$.   Show that $\nu$ is a measure on
$Y$.   ($\nu$ is called the {\bf image measure} on $Y$, and I will
generally denote it $\mu\phi^{-1}$.)
%111Xc

\leader{112Y}{Further exercises (a)}
%\spheader 112Ya
Let $X$ be a set and $\Sigma$ a
$\sigma$-algebra of subsets of $X$.   Let $\mu_1$ and $\mu_2$ be two
measures on $X$, both with domain $\Sigma$.   Set

\Centerline{$\mu E
=\inf\{\mu_1(E\cap F)+\mu_2(E\setminus F):F\in\Sigma\}$}

\noindent for each $E\in\Sigma$.   Show that $\mu$ is a measure on $X$,
and that it is the greatest measure, with domain $\Sigma$, such that
$\mu E\le\min(\mu_1E,\mu_2E)$ for every $E\in\Sigma$.
%problem with double use of symbol  \mu , here and in next exercise,
%see correspondence with Wallace Thompson

\spheader 112Yb Let $X$ be a set and $\Sigma$ a
$\sigma$-algebra of subsets of $X$.   Let $\mu_1$ and $\mu_2$ be two
measures on $X$, both with domain $\Sigma$.   Set

\Centerline{$\mu E
=\sup\{\mu_1(E\cap F)+\mu_2(E\setminus F):F\in\Sigma\}$}

\noindent for each $E\in\Sigma$.   Show that $\mu$ is a measure on $X$,
and that it is the least measure, with domain $\Sigma$, such that $\mu
E\ge\max(\mu_1E,\mu_2E)$ for every $E\in\Sigma$.

\spheader 112Yc Let $X$ be a set and $\Sigma$ a
$\sigma$-algebra of subsets of $X$.

\quad(i) Suppose that $\nu_0,\ldots,\nu_n$ are measures on $X$, all with domain
$\Sigma$.   Set

\Centerline{$\mu E=\inf\{\sum_{i=0}^n\nu_iF_i:F_0,\ldots,F_n\in\Sigma$,
$E\subseteq\bigcup_{i\le n}F_i\}$}

\noindent for $E\in\Sigma$.   Show that $\mu$ is a measure on $X$.

\quad(ii) Let $\Nu$ be a non-empty family of
measures on $X$, all with domain $\Sigma$.   Set

$$\eqalign{\mu E=\inf\{\sum_{n=0}^{\infty}\nu_nF_n:
&\sequencen{\nu_n}\text{ is a sequence in }\Nu,\cr
&\sequencen{F_n}\text{ is a sequence in }\Sigma,
\,E\subseteq\bigcup_{n\in\Bbb N}F_n\}\cr}$$

\noindent for $E\in\Sigma$.   Show that $\mu$ is a measure on $X$.

\quad(iii) Let $\Nu$ be a non-empty family of
measures on $X$, all with domain $\Sigma$, and suppose that there is some
$\nuprime\in\Nu$ such that $\nuprime X<\infty$.   Set

\Centerline{$\mu E=\inf\{\sum_{i=0}^n\nu_iF_i:n\in\Bbb N,
  \,\nu_0,\ldots,\nu_n\in\Nu,\,F_0,\ldots,F_n\in\Sigma,
  E\subseteq\bigcup_{i\le n}F_i\}$}

\noindent for $E\in\Sigma$.   Show that $\mu$ is a measure on $X$.

\quad(iv) Suppose, in (iii), that $\Nu$ is downwards-directed, that is,
for any $\nu_1$, $\nu_2\in\Nu$ there is a $\nu\in\Nu$ such that
$\nu E\le\min(\nu_1E,\nu_2E)$ for every $E\in\Sigma$.   Show that
$\mu E=\inf_{\nu\in\Nu}\nu E$ for every $E\in\Sigma$.

\quad(v) Show that in all the cases (i)-(iii) the measure constructed
is the greatest measure $\mu$ with domain $\Sigma$ such that
$\mu E\le\inf_{\nu\in\Nu}\nu E$ for every $E\in\Sigma$.

\spheader 112Yd Let $X$ be a set and $\Sigma$ a
$\sigma$-algebra of subsets of $X$.   Let $\Nu$ be a non-empty family of
measures on $X$, all with domain $\Sigma$.   Set

$$\eqalign{\mu E=\sup\{\sum_{i=0}^n\nu_iF_i:n\in\Bbb N,
  &\,\nu_0,\ldots,\nu_n\in\Nu,\cr
  &F_0,\ldots,F_n\text{ are disjoint subsets of }E
    \text{ belonging to }\Sigma\}\cr}$$

\noindent for $E\in\Sigma$.   (i) Show that

$$\eqalign{\mu E=\sup\{\sum_{n=0}^{\infty}\nu_nF_n:
&\sequencen{\nu_n}\text{ is a sequence in }\Nu,\cr
&\sequencen{F_n}\text{ is a disjoint sequence in }\Sigma,
\,\bigcup_{n\in\Bbb N}F_n\subseteq E\}\cr}$$

\noindent for every $E\in\Sigma$.
(ii) Show that $\mu$ is a measure on $X$, and that it is the
least measure, with domain $\Sigma$, such that
$\mu E\ge\sup_{\nu\in\Nu}\nu E$ for every $E\in\Sigma$.   (iii) Now
suppose that $\Nu$ is upwards-directed, that is, for any $\nu_1$,
$\nu_2\in\Nu$ there is a $\nu\in\Nu$ such that
$\nu E\ge\max(\nu_1E,\nu_2E)$ for every $E\in\Sigma$.   Show that
$\mu E=\sup_{\nu\in\Nu}\nu E$ for every $E\in\Sigma$.

\spheader 112Ye Let $(X,\Sigma,\mu)$ be a measure space and
$\sequencen{E_n}$ a sequence of measurable sets.   For each $k\in\Bbb N$
set $H_k=\{x:x\in X,\,\#(\{n:x\in E_n\})\ge k\}$, the set of points
belonging to $E_n$ for $k$ or more values of $n$.   (i) Show that each
$H_k$ is measurable.   (ii) Show that $\sum_{k=1}^{\infty}\mu
H_k=\sum_{n=0}^{\infty}\mu E_n$.   \Hint{start with the case in which
$E_n=\emptyset$ for $n\ge n_0$.}   (iii) Show that if
$\sum_{n=0}^{\infty}\mu E_n$ is finite, then almost every point of $X$
belongs to only finitely many $E_n$, and $\sum_{n=0}^{\infty}\mu
E_n=\sum_{k=0}^{\infty}k\mu G_k$, where

\Centerline{$G_k=H_k\setminus H_{k+1}=\{x:\#(\{n:x\in E_n\})=k\}$.}

\spheader 112Yf Let $X$ be a set and $\mu$, $\nu$ two measures on $X$, with
domains $\Sigma$, $\Tau$ respectively.   Set $\Lambda=\Sigma\cap\Tau$ and
define $\lambda:\Lambda\to[0,\infty]$ by setting $\lambda E=\mu E+\nu E$
for every $E\in\Lambda$.   Show that $(X,\Lambda,\lambda)$ is a measure
space.
%+
}%end of exercises

\endnotes{
\Notesheader{112} The calculations in such results as 112Ca-112Cc,
112Xa and 112Xc, involving only finitely many sets, are common to any
additive concept of measure;  you may have encountered them in
elementary probability theory, but of course I am now asking you to
consider also the possibility that one or more of the sets has measure
$\infty$.   I hope you will find that these results are entirely natural
and unsurprisin11g.    I recommend Venn diagrams in this context;   a
result of this kind involving only finitely many measurable sets and
only {\it addition}, with no {\it subtraction}, will be valid in general
if and only if it is valid for the area of simple geometric shapes in
the plane.   The requirement `$\mu E<\infty$' in 112Xc is necessary
only because we are subtracting $\mu E$;   the corresponding additive
result $\mu(F\symmdiff E)+\mu E=\mu F+2\mu(E\setminus F)$ is true for
all measurable $E$ and $F$.
Of course when sequences of sets enter the picture, we need to take a
bit more care;  the results 112Cd-112Cf are the basic ones to learn.   I
think however that the only trap is in the condition `some $\mu E_n$
is finite' in 112Cf.  As noted in the remark at the end of 112C, this is
essential, and for a decreasing sequence of measurable sets it is
possible for the measure of the limit to be strictly less than the limit
of the measures, though only when the latter is infinite.
}%end of notes

\discrpage

