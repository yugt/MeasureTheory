\frfilename{mt491.tex}
\versiondate{14.8.08}
\copyrightdate{2002}

\def\chaptername{Further topics}
\def\sectionname{Equidistributed sequences}

\newsection{491}

In many of the most important topological probability spaces, starting
with Lebesgue measure\cmmnt{ (491Xg)}, there are sequences which are
equidistributed in the sense that, in
the limit, they spend the right proportion of their time in each part
of the space\cmmnt{ (491Yf)}.   I give the basic results on existence
of equidistributed sequences in
491E-491H,  %491E 491F 491G 491H
491Q and 491R.   Investigating such sequences, we are led to some
interesting properties of the asymptotic
density ideal $\Cal Z$ and the quotient algebra
$\frak Z=\Cal P\Bbb N/\Cal Z$ (491A, 491I-491K,  %491I 491J 491K
491P).   For `effectively regular' measures (491L-491M), equidistributed
sequences lead to embeddings of measure algebras in $\frak Z$ (491N).

\leader{491A}{The asymptotic density ideal (a)} If $I$ is a subset of
$\Bbb N$, its {\bf upper asymptotic density} is
$d^*(I)=\limsup_{n\to\infty}\bover1n(I\cap n)$, and its
{\bf asymptotic density} is
$d(I)=\lim_{n\to\infty}\bover1n\#(I\cap n)$ if this is defined.
\cmmnt{It is easy to check that $d^*$ is a submeasure on
$\Cal P\Bbb N$\cmmnt{ (definition: 392A)}, so that}

\Centerline{$\Cal Z=\{I:I\subseteq\Bbb N$,
$d^*(I)=0\}=\{I:I\subseteq\Bbb N$, $d(I)=0\}$}

\noindent is an ideal, the {\bf asymptotic density ideal}.

\spheader 491Ab Note that

\Centerline{$\Cal Z=\{I:I\Bsubseteq\Bbb N$,
$\lim_{n\to\infty}2^{-n}\#(I\cap 2^{n+1}\setminus 2^n)=0\}$.}

\prooflet{\noindent\Prf\ If $I\subseteq\Bbb N$ and $d^*(I)=0$, then

\Centerline{$2^{-n}\#(I\cap 2^{n+1}\setminus 2^n)
\le 2\cdot 2^{-n-1}\#(I\cap 2^{n+1})\to 0$}

\noindent as $n\to\infty$.   In the other direction, if
$\lim_{n\to\infty}2^{-n}\#(I\cap 2^{n+1}\setminus 2^n)=0$, then for
any $\epsilon>0$ there is an $m\in\Bbb N$ such that
$\#(I\cap 2^{k+1}\setminus 2^k)\le 2^k\epsilon$ for every $k\ge m$.
In this case, for $n\ge 2^m$, take $k_n$ such that
$2^{k_n}\le n<2^{k_n+1}$, and see that

\Centerline{$\Bover1n\#(I\cap n)
\le 2^{-k_n}(\#(I\cap 2^m)+\sum_{k=m}^{k_n}2^k\epsilon)
\le 2^{-k_n}\#(I\cap 2^m)+2\epsilon\to 2\epsilon$}

\noindent as $n\to\infty$, and $d^*(I)\le 2\epsilon$;  as $\epsilon$
is arbitrary, $I\in\Cal Z$.\ \Qed}

\spheader 491Ac Writing $\Cal D$ for the domain of $d$,

$$\eqalign{\Cal D
&=\{I:I\subseteq\Bbb N,\,\limsup_{n\to\infty}\Bover1n\#(I\cap n)
=\liminf_{n\to\infty}\Bover1n\#(I\cap n)\}\cr
&=\{I:I\subseteq\Bbb N,\,d^*(I)=1-d^*(\Bbb N\setminus I)\},\cr}$$

\Centerline{$\Bbb N\in\Cal D$,
\quad if $I$, $J\in\Cal D$ and $I\subseteq J$ then
$J\setminus I\in\Cal D$,}

\Centerline{if $I$, $J\in\Cal D$ and $I\cap J=\emptyset$ then
$I\cup J\in\Cal D$ and $d(I\cup J)=d(I)+d(J)$.}

\noindent It follows that if $\Cal I\subseteq\Cal D$ and
$I\cap J\in\Cal I$ for all $I$, $J\in\Cal I$, then the subalgebra of
$\Cal P\Bbb N$ generated by $\Cal I$ is included in
$\Cal D$\cmmnt{ (313Ga)}.   \cmmnt{But note that $\Cal D$ itself
is
{\it not} a subalgebra of $\Cal P\Bbb N$ (491Xa).}

\spheader 491Ad \cmmnt{ The following elementary fact will be
useful.}
If $\sequencen{l_n}$ is a strictly increasing sequence in $\Bbb N$
such
that $\lim_{n\to\infty}l_{n+1}/l_n=1$,
and $I\subseteq\Bbb R$, then

\Centerline{$d^*(I)\le\limsup_{n\to\infty}
\Bover1{l_{n+1}-l_n}\#(I\cap l_{n+1}\setminus l_n)$.}

\prooflet{\noindent\Prf\   Set
$\gamma=\limsup_{n\to\infty}
\Bover1{l_{n+1}-l_n}\#(I\cap l_{n+1}\setminus l_n)$, and take
$\epsilon>0$.   Let $n_0$ be such that
$\#(I\cap l_{n+1}\setminus l_n)\le(\gamma+\epsilon)(l_{n+1}-l_n)$ and
$l_{n+1}-l_n\le\epsilon l_n$ for every $n\ge n_0$, and write $M$ for
$\#(I\cap l_{n_0})$.
If $m>l_{n_0}$, take $k$ such that $l_k\le m<l_{k+1}$;  then

$$\eqalign{\#(I\cap m)
&\le M+\sum_{n=n_0}^{k-1}\#(I\cap l_{n+1}\setminus l_n)+(m-l_k)\cr
&\le M+\sum_{n=n_0}^{k-1}(\gamma+\epsilon)(l_{n+1}-l_n)+l_{k+1}-l_k
\le M+m(\gamma+\epsilon)+\epsilon m,\cr}$$

\noindent so

\Centerline{$\Bover1m\#(I\cap m)\le\Bover{M}{m}+\gamma+2\epsilon$.}

\noindent  Accordingly $d^*(I)\le\gamma+2\epsilon$;  as $\epsilon$ is
arbitrary, $d^*(I)\le\gamma$.\ \Qed}%end of prooflet

\header{491Ae}{\bf *(e)}\cmmnt{ The following remark will not be used
directly in this section, but is one of the fundamental properties of
the ideal $\Cal Z$.}   If $\sequencen{I_n}$ is any sequence in $\Cal Z$,
there is an $I\in\Cal Z$ such that $I_n\setminus I$ is finite for
every
$n$.   \prooflet{\Prf\ Set $J_n=\bigcup_{j\le n}I_j$ for each $n$, so
that $\sequencen{J_n}$ is a non-decreasing sequence in $\Cal Z$.   Let
$\sequencen{l_n}$ be a strictly increasing
sequence in $\Bbb N$ such that, for each $n$,
$\#(J_n\cap k)\le 2^{-n}k$ for every $k\ge l_n$.
Set $I=\bigcup_{n\in\Bbb N}J_n\setminus l_n$.   Then
$I_n\setminus I\subseteq l_n$ is finite for each $n$.   Also, if
$n\in\Bbb N$ and $l_n\le k<l_{n+1}$,

\Centerline{$\#(I\cap k)\le\#(J_n\cap k)\le 2^{-n}k$,}

\noindent so $I\in\Cal Z$.\ \Qed}


\leader{491B}{Equidistributed sequences}
Let $X$ be a topological space and $\mu$ a probability measure on $X$.
I say that a sequence $\sequencen{x_n}$ in $X$
is {\bf (asymptotically) equidistributed}
if $\mu F\ge d^*(\{i:x_i\in F\})$ for every measurable closed set
$F\subseteq X$;   equivalently, if
$\mu G\le\liminf_{n\to\infty}\bover1n\#(\{i:i<n$, $x_i\in G\})$ for
every measurable open set $G\subseteq X$.

\cmmnt{\medskip

\noindent{\bf Remark} Equidistributed sequences are often called
{\bf uniformly distributed}.   Traditionally, such sequences have been
defined in terms of their action on continuous functions, as in 491Cf.
I have adopted the
definition here in order to deal both with Radon measures on spaces
which are not completely regular (so that we cannot identify the
measure with an integral) and with
Baire measures (so that there may be closed sets which are not
measurable).   Note that we cannot demand that the sets
$\{i:x_i\in F\}$ should have well-defined densities (491Xi).
}%end of comment

\leader{491C}{}\cmmnt{ I work through a list of basic facts.   The
technical
details (if we do not specialize immediately to metrizable or compact
spaces) are not quite transparent, so I set them out carefully.

\medskip

\noindent}{\bf Proposition} Let $X$ be a topological space, $\mu$ a
probability measure on $X$ and $\sequencen{x_n}$ a sequence in $X$.

(a) $\sequencen{x_n}$ is equidistributed iff
$\int fd\mu\le\liminf_{n\to\infty}\bover1{n+1}\sum_{i=0}^nf(x_i)$
for every measurable bounded lower semi-continuous function
$f:X\to\Bbb
R$.

(b) If $\mu$ measures every zero set and $\sequencen{x_n}$ is
equidistributed, then
$\lim_{n\to\infty}\bover1{n+1}\sum_{i=0}^nf(x_i)=\int fd\mu$ for every
$f\in C_b(X)$.

(c) Suppose that $\mu$ measures every zero set in $X$.
If $\lim_{n\to\infty}\bover1{n+1}\sum_{i=0}^nf(x_i)=\int fd\mu$ for
every $f\in C_b(X)$, then $d^*(\{n:x_n\in F\})\le\mu F$ for every zero
set $F\subseteq X$.

(d) Suppose that $X$ is normal and that $\mu$ measures every zero set
and is inner regular with respect to the closed sets.
If $\lim_{n\to\infty}\bover1{n+1}\sum_{i=0}^nf(x_i)=\int fd\mu$ for
every $f\in C_b(X)$, then
$\sequencen{x_n}$ is equidistributed.

(e) Suppose that $\mu$ is $\tau$-additive and there is a base $\Cal G$
for the topology of $X$, consisting of measurable sets and closed
under finite unions, such that
$\mu G\le\liminf_{n\to\infty}\bover1{n+1}\#(\{i:i\le n$, $x_i\in G\})$
for every $G\in\Cal G$.   Then $\sequencen{x_n}$ is equidistributed.

(f) Suppose that $X$ is completely regular and that $\mu$ measures
every zero set and is $\tau$-additive.
Then $\sequencen{x_n}$ is equidistributed iff the limit
$\lim_{n\to\infty}\bover1{n+1}\sum_{i=0}^nf(x_i)$ is defined and equal to
$\int fd\mu$ for every $f\in C_b(X)$.

(g) Suppose that $X$ is metrizable and that $\mu$ is a topological
measure.   Then $\sequencen{x_n}$ is equidistributed iff the limit
$\lim_{n\to\infty}\bover1{n+1}\sum_{i=0}^nf(x_i)$ is defined and equal to
$\int fd\mu$ for every $f\in C_b(X)$.

(h) Suppose that $X$ is compact, Hausdorff and zero-dimensional, and
that $\mu$ is a Radon measure on $X$.   Then $\sequencen{x_n}$ is
equidistributed iff
$d(\{n:x_n\in G\})=\mu G$ for every open-and-closed subset $G$ of $X$.

\proof{{\bf (a)(i)} Suppose that $\sequencen{x_n}$ is equidistributed.
Let $f:X\to[0,1]$ be a measurable lower semi-continuous function and
$k\ge 1$.   For each $j<k$ set
$G_j=\{x:f(x)>\bover{j}{k}\}$.   Then

\Centerline{$\liminf_{n\to\infty}\Bover1{n+1}\sum_{j=1}^n\chi G_j(x_i)
=\liminf_{n\to\infty}\Bover1{n+1}\#(\{i:i\le n,\,x_i\in G_j\})
\ge\mu G_j$}

\noindent because $\sequencen{x_n}$ is equidistributed and $G_j$ is a
measurable open set.
Also $f-\bover1k\chi X\le\bover1k\sum_{j=1}^k\chi G_j\le f$, so

$$\eqalign{\int fd\mu-\Bover1k
&\le\Bover1k\sum_{j=1}^k\mu G_j
\le\Bover1k\sum_{j=1}^k\liminf_{n\to\infty}\Bover1{n+1}\sum_{i=0}^n\chi
G_j(x_i)\cr
&\le\Bover1k\liminf_{n\to\infty}\Bover1{n+1}\sum_{i=0}^n\sum_{j=1}^k\chi
G_j(x_i)
\le\liminf_{n\to\infty}\Bover1{n+1}\sum_{i=0}^nf(x_i).\cr}$$

\noindent As $k$ is arbitrary, $\int
fd\mu\le\liminf_{n\to\infty}\bover1{n+1}\sum_{i=0}^nf(x_i)$.

The argument above depended on $f$ taking values in $[0,1]$.   But
multiplying by an appropriate positive scalar we see that
$\int fd\mu\le\liminf_{n\to\infty}\bover1{n+1}\sum_{i=0}^nf(x_i)$
for every bounded measurable lower semi-continuous
$f:X\to\coint{0,\infty}$, and adding a multiple of $\chi X$ we see
that
the same formula is valid for all bounded
measurable lower semi-continuous $f:X\to\Bbb R$.

\medskip

\quad{\bf (ii)}  Conversely, if $\int
fd\mu\le\liminf_{n\to\infty}\bover1{n+1}\sum_{i=0}^nf(x_i)$
for every bounded measurable lower semi-continuous $f:X\to\Bbb R$, and
$G\subseteq X$ is a
measurable open set, then $\chi G$ is lower semi-continuous, so
$\mu G\le\liminf_{n\to\infty}\bover1{n+1}\#(\{i:i\le n,\,x_i\in G\})$.
As $G$ is arbitrary, $\sequencen{x_n}$ is equidistributed.

\medskip

{\bf (b)} Apply (a) to the lower semi-continuous functions $f$ and
(Recall that if $\mu$ measures every zero set, then every
bounded continuous real-valued function is integrable, by 4A3L.)

\medskip

{\bf (c)} Let $F\subseteq X$ be a zero set, and
$\epsilon>0$.   Then there is a continuous $f:X\to\Bbb R$ such that
$F=f^{-1}[\{0\}]$.   Let $\delta>0$ be such that
$\mu\{x:0<|f(x)|\le\delta\}\le\epsilon$, and set
$g=(\chi X-\Bover1\delta|f|)^+$.   Then $g:X\to[0,1]$ is continuous
and
$\chi F\le g$, so

$$\eqalign{d^*(\{n:x_n\in F\})
&\le\limsup_{n\to\infty}\Bover1{n+1}\sum_{i=0}^ng(x_i)\cr
&=\int g\,d\mu\le\mu\{x:|f(x)|\le\delta\}
\le\mu F+\epsilon.\cr}$$

\noindent As $\epsilon$ and $F$ are arbitrary, we have the result.

\medskip

{\bf (d)} Let $F\subseteq X$ be a measurable closed set and
$\epsilon>0$.   Because $\mu$ is inner regular with respect to the
closed sets, there is a measurable closed set
$F'\subseteq X\setminus F$ such that
$\mu F'\ge\mu(X\setminus F)-\epsilon$.   Because $X$ is normal, there
is
a continuous function $f:X\to[0,1]$ such that
$\chi F\le f\le\chi(X\setminus F')$.   Now

\Centerline{$d^*(\{n:x_n\in
F\})\le\limsup_{n\to\infty}\Bover1{n+1}\sum_{i=0}^nf(x_i)=$$\biggerint
fd\mu\le\mu(X\setminus F')\le\mu F+\epsilon$.}

\noindent As $F$ and $\epsilon$ are arbitrary, $\sequencen{x_n}$ is
equidistributed.

\medskip

{\bf (e)} Let $G\subseteq X$ be a measurable open set, and $\epsilon>0$.
Then $\Cal H=\{H:H\in\Cal G$, $H\subseteq G\}$
is upwards-directed and has union
$G$;  since $\mu$ is $\tau$-additive, there is an $H\in\Cal H$ such that
$\mu H\ge\mu G-\epsilon$.   Now

$$\eqalign{\mu G
\le\epsilon+\mu H
&\le\epsilon+\liminf_{n\to\infty}\Bover1n\#(\{i:i<n,\,x_i\in H\})\cr
&\le\epsilon+\liminf_{n\to\infty}\Bover1n\#(\{i:i<n,\,x_i\in G\});\cr}$$

\noindent as $\epsilon$ and $G$ are arbitrary, $\sequencen{x_n}$ is
equidistributed.

\medskip

{\bf (f)}(i) If $\sequencen{x_n}$ is equidistributed then (b) tells us
that $\int fd\mu=\lim_{n\to\infty}\bover1{n+1}\sum_{i=0}^nf(x_i)$ for
every $f\in C_b(X)$.
(ii) Suppose that
$\int fd\mu=\lim_{n\to\infty}\bover1{n+1}\sum_{i=0}^nf(x_i)$ for every
$f\in C_b(X)$.
If $G\subseteq X$ is a cozero set, we can apply (c) to its complement
to see that
$\mu G\le\liminf_{n\to\infty}\bover1{n+1}\#(\{i:i\le n$, $x_i\in G\})$.
So applying (e) with $\Cal G$ the family of cozero sets we see that
$\sequencen{x_n}$ is equidistributed.

\medskip

{\bf (g)} Because every closed set is a zero set, this follows at once
from (b) and (c).

\medskip

{\bf (h)} If $\sequencen{x_n}$ is equidistributed and $G\subseteq X$
is
open-and-closed, then $d^*(\{n:x_n\in G\})\le\mu G$ because $G$ is
closed and $d^*(\{n:x_n\notin G\})\le 1-\mu G$
because $G$ is open;  so $d(\{n:x_n\in G\})=\mu G$.   If the condition
is satisfied, then (e) tells us that $\sequencen{x_n}$ is
equidistributed.
}%end of proof of 491C

\leader{491D}{}\cmmnt{ The next lemma provides a useful general
criterion for the existence of equidistributed sequences.

\medskip

\noindent}{\bf Lemma} Let $X$ be a topological space and $\mu$ a
probability measure on $X$.   Suppose that there is a sequence
$\sequencen{\nu_n}$ of point-supported
probability measures on $X$ such that
$\mu F\ge\limsup_{n\to\infty}\nu_nF$ for every measurable closed set
$F\subseteq X$.   Then $\mu$ has an equidistributed sequence.

\proof{ For each $n\in\Bbb N$, let $q_n:X\to[0,1]$ be such that
$\nu_nE=\sum_{x\in E}q_n(x)$ for every $E\subseteq X$.   Let
$q'_n:X\to[0,1]$ be such that $\sum_{x\in X}q'_n(x)=1$,
$K_n=\{x:q'_n(x)>0\}$ is finite, $q'_n(x)$ is rational for every $x$,
and $\sum_{x\in X}|q_n(x)-q'_n(x)|\le 2^{-n}$;  then
$\mu F\ge\limsup_{n\to\infty}\nu'_nF$ for every measurable
closed $F$, where $\nu'_n$ is defined from $q'_n$.   For each $n$, let
$s_n\ge 1$ be such that $r_n(x)=q'_n(x)s_n$ is an integer for every
$x\in K_n$.
Let $\ofamily{i}{s_n}{x_{ni}}$ be a family in $K_n$ such that
$\#(\{i:i<s_n$, $x_{ni}=x\})=r_n(x)$ for each $x\in K_n$;  then
$\nu'_nE=\Bover1{s_n}\#(\{i:i<s_n$, $x_{ni}\in E\})$ for
every $E\subseteq X$.

Let $\sequence{k}{m_k}$ be such that
$s_{k+1}\le 2^{-k}\sum_{j=0}^km_js_j$ for each $k$.   Set $l_0=0$.
Given $l_n$, take
the largest $k$ such that $\sum_{j=0}^{k-1}m_js_j\le l_n$;  set
$l_{n+1}=l_n+s_k$ and $x_i=x_{k,i-l_n}$ for $l_n\le i<l_{n+1}$;
continue.
By the choice of the $m_k$, $l_{n+1}/l_n\to 1$ as $n\to\infty$.   For
any $E\subseteq X$,
$\#(\{i:l_n\le i<l_{n+1},\,x_i\in E\})=\#(\{j:j<s_k,\,x_{kj}\in E\})$
whenever $\sum_{j=0}^{k-1}m_js_j\le l_n<\sum_{j=0}^km_js_j$.   So for
any measurable closed set $F\subseteq X$,

$$\eqalignno{d^*(\{i:x_i\in F\})
&\le\limsup_{k\to\infty}\Bover1{s_k}\#(\{j:j<s_k,\,x_{kj}\in F\})\cr
\displaycause{491Ad}
&=\limsup_{k\to\infty}\nu'_kF
\le\mu F.\cr}$$

\noindent As $F$ is arbitrary, $\sequencen{x_n}$ is an equidistributed
sequence for $\mu$.
}%end of proof of 491D

\leader{491E}{Proposition} (a)(i) Suppose that $X$ and $Y$ are
topological spaces, $\mu$ a probability measure on $X$ and
$f:X\to Y$ a continuous function.   If $\sequencen{x_n}$
is a sequence in $X$ which is equidistributed with respect to $\mu$,
then $\sequencen{f(x_n)}$ is equidistributed with respect to the image
measure $\mu f^{-1}$.

\quad(ii) Suppose that $(X,\mu)$ and $(Y,\nu)$ are topological
probability spaces and $f:X\to Y$ is a continuous \imp\ function.   If
$\sequencen{x_n}$
is a sequence in $X$ which is equidistributed with respect to $\mu$,
then $\sequencen{f(x_n)}$ is equidistributed with respect to $\nu$.

(b) Let $X$ be a topological space and $\mu$ a probability measure on
$X$, and suppose that $X$ has a countable network consisting of sets
measured by $\mu$.   Let $\lambda$ be the
ordinary product measure on $X^{\Bbb N}$.   Then $\lambda$-almost
every sequence $\sequencen{x_n}$ in $X^{\Bbb N}$ is $\mu$-equidistributed.

\proof{{\bf (a)(i)} Let $F\subseteq Y$ be a closed set which is measured
by $\mu f^{-1}$.   Then $f^{-1}[F]$ is a closed set in $X$ measured by
$\mu$.   So

\Centerline{$d^*(\{n:f(x_n)\in F\})=d^*(\{n:x_n\in f^{-1}[F]\})
\le\mu f^{-1}[F]$.}

\medskip

\quad{\bf (ii)} Replace `$\mu f^{-1}$' above by `$\nu$'.
\medskip

{\bf (b)} Let $\Cal A$ be a countable network for the given topology
$\frak S$ of $X$ consisting of measurable sets, and let $\Cal E$ be
the countable subalgebra of $\Cal PX$ generated
by $\Cal A$.   Let $\frak T\supseteq\frak S$ be the second-countable
topology generated by $\Cal E$;  then $\mu$ is a $\tau$-additive
topological measure with respect to $\frak T$,
and $\Cal E$ is a base for $\frak T$ closed under finite unions.
If $E\in\Cal E$, then $d(\{n:x_n\in E\})=\mu E$ for $\lambda$-almost
every sequence $\sequencen{x_n}$ in $X$, by the strong law of large
numbers (273J).   So

\Centerline{$d(\{n:x_n\in E\})=\mu E$ for every $E\in\Cal E$}

\noindent for $\lambda$-almost every $\sequencen{x_n}$.   Now 491Ce
tells us that any such sequence is equidistributed with respect to
$\frak T$ and therefore with respect to
$\frak S$.
}%end of proof of 491E

\leader{491F}{Theorem} Let
$\family{i}{I}{(X_i,\frak T_i,\Sigma_i,\mu_i)}$ be a family of
$\tau$-additive topological probability spaces,
each of which has an equidistributed sequence.
If $\#(I)\le\frak c$, then the $\tau$-additive product measure $\lambda$
on $X=\prod_{i\in I}X_i$\cmmnt{ (definition:  417G)} has an
equidistributed sequence.

\proof{{\bf (a)} For the time being (down to the end of (f)), let us
suppose that every $\mu_i$ is inner regular with respect to the Borel
sets.   (This will make it possible to use the theorems of \S417, in
particular 417H and 417J.)

%\allowmorestretch{468}{
The formulae of some of the arguments below will be simplified if we
immediately re-index the family
$\family{i}{I}{(X_i,\frak T_i,\penalty-100\Sigma_i,\mu_i)}$ as
$\family{A}{\Cal A}{(X_A,\frak T_A,\Sigma_A,\mu_A)}$ where
$\Cal A\subseteq\Cal P\Bbb N$.
For each $A\in\Cal A$, let $\sequencen{t_{An}}$ be an equidistributed
sequence in $X_A$;  for $n\in\Bbb N$, let $\nu_{An}$ be the
point-supported measure on $X_A$ defined by setting
$\nu_{An}E=\bover1{n+1}\#(\{i:i\le n$, $t_{Ai}\in E\})$ for
$E\subseteq X_A$.   For each finite set $\Cal I\subseteq\Cal A$, set
$Y_{\Cal I}=\prod_{A\in\Cal I}X_A$;  set
$\pi_{\Cal I}(x)=x\restr\Cal I\in Y_{\Cal I}$ for $x\in X$.
Let $\lambda_{\Cal I}$ be the $\tau$-additive product of
$\family{A}{\Cal I}{\mu_I}$ and, for each $n$, let
$\check\nu_{\Cal In}$ be the product of the measures
$\family{A}{\Cal I}{\nu_{An}}$.   (Because $\Cal I$ is finite, this is a
point-supported
probability measure.   I do not say `$\tau$-additive product' here
because I do not wish to assume that all singleton sets are Borel, so
the $\nu_{An}$ may not be inner regular with respect to Borel sets.)
%}%end of allowmorestretch

\medskip

{\bf (b)} Suppose that $\Cal I\subseteq\Cal A$ is finite and that
$W\subseteq Y_{\Cal I}$ is an open set.
Then $\lambda_{\Cal I}W\le\liminf_{n\to\infty}\check\nu_{\Cal In}W$.
\Prf\ Induce on $\#(\Cal I)$.   If $\Cal I=\emptyset$, $Y_{\Cal I}$
is a singleton and the result is trivial.   For the inductive step, if
$\Cal I\ne\emptyset$, take any $A\in\Cal I$ and set
$\Cal I'=\Cal I\setminus\{A\}$.   Then we can identify
$Y_{\Cal I}$ with $Y_{\Cal I'}\times X_A$, $\lambda_{\Cal I}$ with the
$\tau$-additive product of $\lambda_{\Cal I'}$ and $\mu_A$ (417J), and
each $\check\nu_{\Cal In}$ with the product of $\check\nu_{\Cal I'n}$
and $\nu_{An}$.

Let $\Cal V$ be the family of those subsets $V$ of $Y_{\Cal I}$ which
are expressible as a finite union of sets of the form $U\times H$
where $U\subseteq Y_{\Cal I'}$ and $H\subseteq X_A$
are open.   Then $\Cal V$ is a base for the topology of $Y_{\Cal I}$
closed under finite unions.   Let $\epsilon>0$.   Because
$\lambda_{\Cal I}$ is $\tau$-additive, there is a $V\in\Cal V$ such
that $\lambda_{\Cal I}V\ge\lambda W-\epsilon$.
The function $t\mapsto\lambda_{\Cal I'}V[\{t\}]:X_A\to[0,1]$ is lower
semi-continuous (417Ba), so 491Ca tells us that

$$\eqalign{\lambda_{\Cal I}V
&=\int\lambda_{\Cal I'}V[\{t\}]\mu_A(dt)\cr
&\le\liminf_{n\to\infty}\Bover1{n+1}
  \sum_{i=0}^n\lambda_{\Cal I'}V[\{t_{Ai}\}]
=\liminf_{n\to\infty}\int\lambda_{\Cal I'}V[\{t\}]\nu_{An}(dt).\cr}$$

\noindent At the same time, there are only finitely many sets of the
form $V[\{t\}]$, and for each of these we have
$\lambda_{\Cal I'}V[\{t\}]
\le\liminf_{n\to\infty}\check\nu_{\Cal I'n}V[\{t\}]$,
by the inductive hypothesis.   So there is an $n_0\in\Bbb N$ such that
$\lambda_{\Cal I'}V[\{t\}]\le\check\nu_{\Cal I'n}V[\{t\}]+\epsilon$
for every $n\ge n_0$ and every $t\in X_A$.   We must therefore have

$$\eqalign{\lambda_{\Cal I}W
&\le\lambda_{\Cal I}V+\epsilon
\le\liminf_{n\to\infty}\int\lambda_{\Cal
I'}V[\{t\}]\nu_{An}(dt)+\epsilon\cr
&\le\liminf_{n\to\infty}\int\check\nu_{\Cal
I'n}V[\{t\}]\nu_{An}(dt)+2\epsilon\cr
&=\liminf_{n\to\infty}\check\nu_{\Cal In}V+2\epsilon
\le\liminf_{n\to\infty}\check\nu_{\Cal In}W+2\epsilon.\cr}$$

\noindent As $\epsilon$ and $W$ are arbitrary, the induction
proceeds.\
\Qed

\medskip

{\bf (c)} For $K\subseteq m\in\Bbb N$, set
$\Cal A_{mK}=\{A:A\in\Cal A$, $A\cap m=K\}$
and $Z_{mK}=\prod_{A\in\Cal A_{mK}}X_A$.   (If $\Cal A_{mK}=\emptyset$
then $Z_{mK}=\{\emptyset\}$.)
Then for each $m\in\Bbb N$ we can identify $X$ with
the finite product $\prod_{K\subseteq m}Z_{mK}$.
%, and $\lambda$ will be %the $\tau$-additive product of the
%$\lambda_{mK}$ (417J).
For $K\subseteq m\in\Bbb N$ and $n\in\Bbb N$, define $z_{mKn}\in Z_{mK}$
by setting $z_{mKn}(A)=t_{An}$ for $A\in\Cal A_{mK}$;  let
$\tilde\nu_{mKn}$ be the point-supported measure
on $Z_{mK}$ defined by setting
$\tilde\nu_{mKn}W=\bover1{n+1}\#(\{i:i\le n$, $z_{mKi}\in W\})$ for
each $W\subseteq Z_{mK}$.
For $n\in\Bbb N$ let $\tilde\nu_n$ be the measure on $X$ which is the
product of the measures $\tilde\nu_{nKn}$ for $K\subseteq n$;  this
too is point-supported.

\medskip

{\bf (d)} If $\Cal I\subseteq\Cal A$ is finite, there is an $m\in\Bbb N$
such that
$\check\nu_{\Cal In}=\tilde\nu_n\pi_{\Cal I}^{-1}$ for every $n\ge m$.
\Prf\ Let $m$ be such that $A\cap m\ne A'\cap m$ for all distinct $A$,
$A'\in\Cal I$.   If $n\ge m$, then $\tilde\nu_n$ is the product of the
$\tilde\nu_{nKn}$
for $K\subseteq n$.   Now $\pi_{\Cal I}$, interpreted as a function
from $\prod_{K\subseteq n}Z_{nK}$ onto $Y_{\Cal I}$, is of the form
$\pi_{\Cal I}(\langle z_K\rangle_{K\subseteq n})
=\family{A}{\Cal I}{z_{A\cap n}(A)}$, so the image measure
$\tilde\nu_n\pi_{\Cal I}^{-1}$ is the product of the family
$\family{A}{\Cal I}{\tilde\nu_{n,A\cap n,n}\hat\pi_A^{-1}}$, writing
$\hat\pi_A(z)=z(A)$ when $A\cap n=K$ and $z\in Z_{nK}$ (254H, or
otherwise).   But, looking back at the definitions,

$$\eqalign{\tilde\nu_{n,A\cap n,n}\hat\pi_A^{-1}[E]
&=\Bover1{n+1}\#(\{i:i\le n,\,z_{n,A\cap
n,i}\in\hat\pi_A^{-1}[E]\})\cr
&=\Bover1{n+1}\#(\{i:i\le n,\,z_{n,A\cap n,i}(A)\in E\})\cr
&=\Bover1{n+1}\#(\{i:i\le n,\,t_{Ai}\in E\})
=\nu_{An}E\cr}$$

\noindent for every $E\subseteq X_A$.   So
$\tilde\nu_n\pi_{\Cal I}^{-1}$ is the product of
$\family{A}{\Cal I}{\nu_{An}}$, which is $\check\nu_{\Cal In}$.\ \Qed

\medskip

{\bf (e)} Let $\Cal W$ be the family of those open sets
$W\subseteq X$ expressible in the form $\pi_{\Cal I}^{-1}[W']$ for
some finite $\Cal I\subseteq\Cal A$ and
some open $W'\subseteq Y_{\Cal I}$.   If $W\in\Cal W$, then
$\lambda W\le\liminf_{n\in\Bbb N}\tilde\nu_nW$.   \Prf\ Take
$\Cal I\in[\Cal A]^{<\omega}$ and an open $W'\subseteq Y_{\Cal I}$
such that $W=\pi_{\Cal I}^{-1}[W']$.   Then

$$\eqalignno{\lambda W
&=\lambda_{\Cal I}W'\cr
\displaycause{417K}
&\le\liminf_{n\to\infty}\check\nu_{\Cal In}W'\cr
\displaycause{by (b) above}
&=\liminf_{n\to\infty}\tilde\nu_n\pi_{\Cal I}^{-1}[W']\cr
\displaycause{by (d)}
&=\liminf_{n\to\infty}\tilde\nu_nW. \text{ \Qed}\cr}$$

\medskip

{\bf (f)} If now $F\subseteq X$ is any closed set and $\epsilon>0$, then
(because $\Cal W$ is a base for the topology of $X$ closed under finite
unions) there is a $W\in\Cal W$
such that $W\subseteq X\setminus F$ and
$\lambda W\ge 1-\lambda F+\epsilon$.   In this case

\Centerline{$\limsup_{n\to\infty}\tilde\nu_nF
\le 1-\liminf_{n\to\infty}\tilde\nu_nW
\le 1-\lambda W
\le\lambda F+\epsilon$.}

\noindent As $\epsilon$ is arbitrary,
$\limsup_{n\to\infty}\tilde\nu_nF\le\lambda F$;  as $F$ is arbitrary,
491D tells us that there is an equidistributed sequence in
$X$.

\medskip

{\bf (g)} All this was on the assumption that every $\mu_i$ is inner
regular with respect to the Borel sets.   For the superficially more
general case enunciated, given only
that each $\mu_i$ is a $\tau$-additive topological measure with an
equidistributed sequence,
let $\mu'_i$ be the restriction of $\mu_i$ to the Borel
$\sigma$-algebra of $X_i$ for each $i\in I$.   Each $\mu'_i$ is still
$\tau$-additive and equidistributed sequences for $\mu_i$ are of
course equidistributed for $\mu'_i$.   If we take $\lambda'$ to be the
$\tau$-additive product of $\familyiI{\mu'_i}$, then it must agree
with $\lambda$ on the open sets of $X$ and therefore on the closed sets,
and an equidistributed sequence
for $\lambda'$ will be an equidistributed sequence for $\lambda$.
This completes the proof.
}%end of proof of 491F

\leader{491G}{Corollary} The usual measure of $\{0,1\}^{\frakc}$ has an
equidistributed sequence.
%491P

\proof{ The usual measure of $\{0,1\}$ of course has an equidistributed
sequence (just set $x_i=0$ for even $i$, $x_i=1$ for odd $i$), so
491F gives the result at once.
}%end of proof of 491G

\leader{491H}{Theorem}\cmmnt{ ({\smc Veech 71})} Any separable compact
Hausdorff topological group has an equidistributed sequence for its Haar
probability measure.

\proof{ Let $X$ be a separable compact Hausdorff topological group.
Recall that $X$ has exactly one Haar probability measure $\mu$,
which is both a left Haar measure and a right Haar measure (442Ic).

\medskip

{\bf (a)} We need some elementary facts about convolutions.

\medskip

\quad{\bf (i)} If $\nu_1$ and $\nu_2$ are point-supported probability
measures on $X$, then $\nu_1*\nu_2$ is point-supported.   \Prf\ If
$\nu_1E=\sum_{x\in E}q_1(x)$
and $\nu_2E=\sum_{x\in E}q_2(x)$ for every $E\subseteq X$, then

\Centerline{$(\nu_1*\nu_2)(E)=(\nu_1\times\nu_2)\{(x,y):xy\in E\}
=\sum_{xy\in E}q_1(x)q_2(y)=\sum_{z\in E}q(z)$}

\noindent where $q(z)=\sum_{x\in X}q_1(x)q_2(x^{-1}z)$ for $z\in X$.\
\Qed

\medskip

\quad{\bf (ii)} Let $\nu$, $\lambda$ be Radon probability measures on
$X$.   Suppose that $f\in C(X)$, $\alpha\in\Bbb R$ and $\epsilon>0$
are such that
$|\int f(yxz)\nu(dx)-\alpha|\le\epsilon$ for every $y$, $z\in X$.
Then $|\int f(yxz)(\lambda*\nu)(dx)-\alpha|\le\epsilon$ and
$|\int f(yxz)(\lambda*\nu)(dx)-\alpha|\le\epsilon$ for every $y$, $z\in X$.
\Prf\

$$\eqalignno{|\int f(yxz)(\lambda*\nu)&(dx)-\alpha|\cr
&=|\iint f(ywxz)\nu(dx)\lambda(dw)-\alpha|\cr
\displaycause{444C}
&\le\int|\int f(ywxz)\nu(dx)-\alpha|\lambda(dw)
\le\int\epsilon\lambda(dw)
=\epsilon,\cr
|\int f(yxz)(\nu*\lambda)&(dx)-\alpha|\cr
&=|\iint f(yxwz)\nu(dx)\lambda(dw)-\alpha|\cr
&\le\int|\int f(yxwz)\nu(dx)-\alpha|\lambda(dw)
\le\int\epsilon\lambda(dw)
=\epsilon.  \text{ \Qed}\cr}$$

\medskip

{\bf (b)} Let $A\subseteq X$ be a countable dense set.   Let $\Nu$ be
the set of point-supported probability measures $\nu$ on $X$ which are
defined by functions $q$ such that
$\{x:q(x)>0\}$ is a finite subset of $A$ and $q(x)$ is rational for
every $x$.   Then $\Nu$ is countable.   Now, for every $f\in C(X)$ and
$\epsilon>0$, there is a
$\nu\in\Nu$ such that $|\int f(yxz)\nu(dx)-\int fd\mu|\le\epsilon$ for
all $y$, $z\in X$.   \Prf\ Because $X$ is compact, $f$ is uniformly
continuous for the right uniformity of
$X$ (4A2Jf), so there is a neighbourhood $U$ of the identity $e$ such
that $|f(x')-f(x)|\le\bover12\epsilon$ whenever $x'x^{-1}\in U$.
Next, again because $X$ is compact, there
is a neighbourhood $V$ of $e$ such that $yxy^{-1}\in U$ whenever
$x\in V$ and $y\in X$ (4A5Ej).   Because $A$ is dense,
$V^{-1}x\cap A\ne\emptyset$ for every $x\in X$, that is,
$VA=X$;  once more because $X$ is compact, there are
$x_0,\ldots,x_n\in A$ such that $X=\bigcup_{i\le n}Vx_i$.   Set
$E_i=Vx_i\setminus\bigcup_{j<i}Vx_j$ for each $i\le n$.
Let $\alpha_0,\ldots,\alpha_n\in[0,1]\cap\Bbb Q$ be such that
$\sum_{i=0}^n\alpha_i=1$ and
$\|f\|_{\infty}\sum_{i=0}^n|\alpha_i-\mu E_i|\le\bover12\epsilon$,
and define $\nu\in\Nu$ by
setting $\nu E=\sum\{\alpha_i:i\le n$, $x_i\in E\}$ for every
$E\subseteq X$.

Let $y$, $z\in X$.   If $i\le n$ and $x\in E_i$, then $xx_i^{-1}\in V$
so $(yxz)(yx_iz)^{-1}=yxx_i^{-1}y^{-1}\in U$ and
$|f(yxz)-f(yx_iz)|\le\bover12\epsilon$.   Accordingly

$$\eqalignno{|\int f(yxz)\nu(dx)&-\int f(x)\mu(dx)|\cr
&=|\sum_{i=0}^n\alpha_i f(yx_iz)-\int f(yxz)\mu(dx)|\cr
\displaycause{because $X$ is unimodular}
&\le\sum_{i=0}^n|\alpha_i f(yx_iz)-\int_{E_i}f(yxz)\mu(dx)|\cr
&\le\sum_{i=0}^n|\alpha_i-\mu E_i||f(yx_iz)|
  +\sum_{i=0}^n|f(yx_iz)\mu E_i-\int_{E_i}f(yxz)\mu(dx)|\cr
&\le\|f\|_{\infty}\sum_{i=0}^n|\alpha_i-\mu E_i|
  +\sum_{i=0}^n\int_{E_i}|f(yxz)-f(yx_iz)|\mu(dx)\cr
&\le\Bover{\epsilon}2+\sum_{i=0}^n\Bover{\epsilon}2\mu E_i
=\epsilon.  \text{ \Qed}\cr}$$

\medskip

{\bf (c)} Let $\sequencen{\nu_n}$ be a sequence running over $\Nu$,
and set $\lambda_n=\nu_0*\nu_1*\ldots*\nu_n$ for each $n$.   (Recall from
444B that convolution is associative.)   Then each
$\lambda_n$ is a point-supported probability
measure on $X$, by (a-i).   Also
$\lim_{n\to\infty}\int fd\lambda_n=\int fd\mu$ for every $f\in C(X)$.
\Prf\ If $f\in C(X)$ and $\epsilon>0$, then (b) tells us that
there is an $m\in\Bbb N$ such that
$|\int f(yxz)\nu_m(dx)-\int fd\mu|\le\epsilon$ for all $y$, $z\in X$.
For any $n\ge m$, $\lambda_n$ is of the form $\lambda'*\nu_m*\lambda''$.
By (a-ii), used in both parts successively,
$|\int fd\lambda_n-\int fd\mu|\le\epsilon$.
As $\epsilon$ is arbitrary, we have the result.\ \Qed

\medskip

{\bf (d)} If $F\subseteq X$ is closed, then

$$\eqalign{\mu F
&=\inf\{\int fd\mu:\chi F\le f\in C(X)\}\cr
&=\inf_{\chi F\le f}\limsup_{n\to\infty}\int fd\lambda_n
\ge\limsup_{n\to\infty}\lambda_nF.\cr}$$

\noindent By 491D, $\mu$ has an equidistributed sequence.
}%end of proof of 491H

\leader{491I}{The quotient $\Cal P\Bbb N/\Cal Z$}\cmmnt{ I now return
to the asymptotic density ideal
$\Cal Z$, moving towards a striking relationship between the
corresponding quotient algebra and equidistributed
sequences.}    Since
$\Cal Z\normalsubgroup\Cal P\Bbb N$, we can form the quotient algebra
$\frak Z=\Cal P\Bbb N/\Cal Z$.   The functionals $d$
and $d^*$ descend naturally to $\frak Z$ if we set

\Centerline{$\bar d^*(I^{\ssbullet})=d^*(I)$,
\quad$\bar d(I^{\ssbullet})=d(I)$ whenever $d(I)$ is defined.}

\spheader 491Ia $\bar d^*$ is a strictly positive submeasure on
$\frak Z$.   \prooflet{\Prf\ $\bar d^*$ is a submeasure on $\frak Z$
because $d^*$ is a
submeasure on $\Cal P\Bbb N$.   $\bar d^*$ is strictly positive
because $\Cal Z\supseteq\{I:d^*(I)=0\}$.\ \Qed}

\spheader 491Ib Let $\bar\rho$ be the metric on $\frak Z$ defined by
saying that $\bar\rho(a,b)=\bar d^*(a\Bsymmdiff b)$ for
all $a$, $b\in\frak Z$.   Under $\rho$, the Boolean operations
$\Bcup$, $\Bcap$, $\Bsymmdiff$ and $\Bsetminus$ and the function
$\bar d^*:\frak Z\to[0,1]$
are uniformly continuous\cmmnt{ (392Hb\footnote{Formerly 3{}93Bb.})},
and $\frak Z$ is complete.
\prooflet{\Prf\ Let $\sequencen{c_n}$ be a sequence in $\frak Z$ such
that $\bar\rho(c_{n+1},c_n)\le 2^{-n}$ for every $n\in\Bbb N$;
then $\bar\rho(c_r,c_i)\le 2^{-i+1}$ for $i\le r$.
For each $n\in\Bbb N$ choose $C_n\subseteq\Bbb N$ such that
$C_n^{\ssbullet}=c_n$;  then $d^*(C_r\symmdiff C_i)\le 2^{-i+1}$ for
$i\le r$.   Choose a strictly increasing sequence
$\sequencen{k_n}$ in $\Bbb N$ such that $k_{n+1}\ge 2k_n$ for every
$n$ and, for each $n\in\Bbb N$,

\Centerline{$\Bover1m\#((C_n\symmdiff C_i)\cap m)\le 2^{-i+2}$
whenever $i\le n$, $m\ge k_n$.}

Set $C=\bigcup_{n\in\Bbb N}C_n\cap k_{n+1}\setminus k_n$, and
$c=C^{\ssbullet}\in\frak Z$.   If $n\in\Bbb N$ and $m\ge k_{n+1}$,
then take $r>n$ such that $k_r\le m<k_{r+1}$;  in this case
$k_i\le 2^{i-r}m$ for $i\le r$, so

$$\eqalignno{\#((C\symmdiff C_n)\cap m)
&\le k_n+\sum_{i=n}^{r-1}\#((C\symmdiff C_n)\cap k_{i+1}\setminus k_i)
    +\#((C\symmdiff C_n)\cap m\setminus k_r)\cr
&=k_n+\sum_{i=n}^{r-1}\#((C_i\symmdiff C_n)\cap k_{i+1}\setminus k_i)
    +\#((C_r\symmdiff C_n)\cap m\setminus k_r)\cr
&\le k_n+\sum_{i=n+1}^{r-1}\#((C_i\symmdiff C_n)\cap k_{i+1})
    +\#((C_r\symmdiff C_n)\cap m)\cr
&\le k_n+\sum_{i=n+1}^{r-1}2^{-n+2}k_{i+1}
    +2^{-n+2}m\cr
&\le k_n+\sum_{i=n+1}^{r-1}2^{-n+2}2^{i+1-r}m
    +2^{-n+2}m\cr
&\le k_n+2^{-n+3}m+2^{-n+2}m.\cr}$$

\noindent But this means that

$$\eqalign{\bar\rho(c,c_n)
=d^*(C\symmdiff C_n)
\le\lim_{m\to\infty}\Bover{k_n}m+2^{-n+3}+2^{-n+2}
\le 2^{-n+4}\cr}$$

\noindent for every $n$, and $\sequencen{c_n}$ converges to $c$ in
$\frak Z$.\ \Qed}%end of prooflet

%\def\spheader#1#2#3#4#5{\header{#1#2#3#4#5}{\bf (#5)}}
\header{491Ic}{\bf *(c)} If $\sequencen{a_n}$ is a non-increasing
sequence in $\frak Z$, there is an $a\in\frak Z$ such that
$a\Bsubseteq a_n$ for every $n$ and
$\bar d^*(a)=\inf_{n\in\Bbb N}\bar d^*(a_n)$.   \prooflet{\Prf\ For
each $n\in\Bbb N$, choose $I_n\subseteq\Bbb N$ such that
$I_n^{\ssbullet}=a_n$;  replacing $I_n$ by
$\bigcap_{j\le n}I_j$ if necessary, we can arrange that
$I_{n+1}\subseteq I_n$ for every $n$.   Set
$\gamma=\inf_{n\in\Bbb N}\bar d^*(a_n)=\inf_{n\in\Bbb N}d^*(I_n)$.
Let $\sequencen{k_n}$ be a strictly increasing sequence in $\Bbb N$
such that
$\#(I_n\cap k_n)\ge(\gamma-2^{-n})k_n$ for every $n$.   Set
$I=\bigcup_{n\in\Bbb N}I_n\cap k_n$ and $a=I^{\ssbullet}\in\frak Z$.   Then
$\#(I\cap k_n)\ge(\gamma-2^{-n})k_n$ for every $n$, so
$\bar d^*(a)=d^*(I)\ge\gamma$.
Also $I\setminus I_n\subseteq k_n$ is finite, so $a\Bsubseteq a_n$,
for every $n$.   Of course it follows
at once that $\bar d^*(a)=\gamma$ exactly, as required.\ \Qed}

\header{491Id}{\bf *(d)} $\bar d^*$ is a Maharam submeasure on
$\frak Z$.   \prooflet{(Immediate from (c).)}

\leader{491J}{Lemma} Let $\sequencen{a_n}$ be a non-decreasing
sequence in $\frak Z=\Cal P\Bbb N/\Cal Z$ such that
$\lim_{n\to\infty}\bar d^*(a_n)+\bar d^*(1\Bsetminus a_n)=1$.
Then $\sequencen{a_n}$ is topologically convergent to $a\in\frak Z$;
$a=\sup_{n\in\Bbb N}a_n$ in $\frak Z$ and
$d^*(a)+d^*(1\Bsetminus a)=1$.

\proof{{\bf (a)} The point is that if $m\le n$ then
$\bar d^*(a_n\Bsetminus a_m)\le\bar d^*(a_n)+\bar d^*(1\Bsetminus a_m)-1$.
\Prf\ Let $I$, $J\Bsubseteq\Bbb N$ be such that $I^{\ssbullet}=a_m$
and $J^{\ssbullet}=a_n$.   For any $k\ge 1$,

\Centerline{$\Bover1k\#(k\cap J)+\Bover1k\#(k\setminus I)
=\Bover1k\#(k\cap J\setminus I)+\Bover1k\#(k\setminus(I\setminus J))$,}

\noindent so

$$\eqalign{d^*(J)+d^*(\Bbb N\setminus I)
&=\limsup_{k\to\infty}\Bover1k\#(k\cap
J)+\limsup_{k\to\infty}\Bover1k\#(k\setminus I)\cr
&\ge\limsup_{k\to\infty}\Bover1k\#(k\cap J)+\Bover1k\#(k\setminus I)\cr
&=\limsup_{k\to\infty}\Bover1k\#(k\cap J\setminus I)
   +\Bover1k\#(k\setminus(I\setminus J))\cr
&\ge\limsup_{k\to\infty}\Bover1k\#(k\cap J\setminus I)
   +\liminf_{k\to\infty}\Bover1k\#(k\setminus(I\setminus J))
=d^*(J\setminus I)+1\cr}$$

\noindent because $a_m\Bsubseteq a_n$, so $I\setminus J\in\Cal Z$.
But
this means that

\Centerline{$\bar d^*(a_n\Bsetminus a_m)
=d^*(J\setminus I)
\le d^*(J)+d^*(\Bbb N\setminus I)-1
=\bar d^*(a_n)+\bar d^*(1\Bsetminus a_m)-1$.  \Qed}

\medskip

{\bf (b)} Accordingly

$$\eqalignno{\limsup_{m\to\infty}\sup_{n\ge m}\bar\rho(a_m,a_n)
&=\limsup_{m\to\infty}\sup_{n\ge m}\bar d^*(a_n\Bsetminus a_m)\cr
&\le\limsup_{m\to\infty}\sup_{n\ge m}\bar d^*(a_n)+\bar
d^*(1\Bsetminus
a_m)-1\cr
&=\limsup_{m\to\infty}\sup_{n\ge m}\bar d^*(a_n)-\bar d^*(a_m)\cr
\displaycause{because
$\lim_{m\to\infty}\bar d^*(a_m)+\bar d^*(1\Bsetminus a_m)=1$}
&=0,\cr}$$

\noindent and $\sequencen{a_n}$ is a Cauchy sequence in $\frak Z$.

\medskip

{\bf (c)} Because $\frak Z$ is complete, $a=\lim_{n\to\infty}a_n$ is
defined in $\frak Z$.   For each $m\in\Bbb N$,
$a_m\Bsetminus a=\lim_{n\to\infty}a_m\Bsetminus a_n=0$
(because $\Bsetminus$ is continuous), so $a_m\subseteq a$;  thus $a$
is an upper bound of $\{a_n:n\in\Bbb N\}$.   If $b$ is any upper bound of
$\{a_n:n\in\Bbb N\}$, then
$a\Bsetminus b=\lim_{n\to\infty}a_n\Bsetminus b=0$;  so
$a=\sup_{n\in\Bbb N}a_n$.   Finally,

\Centerline{$\bar d^*(a)+\bar d^*(1\Bsetminus a)=\lim_{n\to\infty}\bar
d^*(a_n)+\bar d^*(1\Bsetminus a_n)=1$.}
}%end of proof of 491J

\leader{491K}{Corollary} Set
$D=\dom\bar d=\{I^{\ssbullet}:I\subseteq\Bbb N$, $d(I)$ is defined$\}
\subseteq\frak Z=\Cal P\Bbb N/\Cal Z$.

(a) If $I\subseteq\Bbb N$ and $I^{\ssbullet}\in D$ then $d(I)$ is
defined.

(b) $D=\{a:a\in\frak Z$, $\bar d^*(a)+\bar d^*(1\Bsetminus a)=1\}$;
if $a\in D$ then $1\Bsetminus a\in D$;   if $a$, $b\in D$ and
$a\Bcap b=0$, then $a\Bcup b\in D$ and
$\bar d(a\Bcup b)=\bar d(a)+\bar d(b)$;
if $a$, $b\in D$ and $a\Bsubseteq b$ then
$b\Bsetminus a\in D$ and
$\bar d(b\Bsetminus a)=\bar d(b)-\bar d(a)$.

(c) $D$ is a topologically closed subset of $\frak Z$.

(d) If $A\subseteq D$ is upwards-directed, then $\sup A$ is defined in
$\frak Z$ and belongs to $D$;
moreover there is a sequence in $A$ with the same supremum
as $A$, and $\sup A$ belongs to the topological closure of $A$.

(e) Let $\frak B\subseteq D$ be a subalgebra of $\frak Z$.   Then the
following are equiveridical:

\quad (i) $\frak B$ is topologically closed in $\frak Z$;

\quad (ii) $\frak B$ is order-closed in $\frak Z$;

\quad (iii) setting $\bar\nu=\bar d^*\restr\frak B=\bar d\restr\frak B$,
$(\frak B,\bar\nu)$ is a probability algebra.

\noindent In this case, $\frak B$ is regularly embedded in $\frak Z$.

(f) If $I\subseteq D$ is closed under either $\Bcap$ or $\Bcup$, then
the topologically closed subalgebra of $\frak Z$ generated by $I$,
which is also the
order-closed subalgebra of $\frak Z$ generated by $I$, is included in
$D$.

\proof{{\bf (a)} There is a $J\subseteq\Bbb N$ such that $d(J)$ is
defined and $I\symmdiff J\in\Cal Z$.   But in this case
$d(I\symmdiff J)=0$, so $d(I\cup J)=d(J)+d(I\setminus J)$ is defined;
also $d(J\setminus I)=0$, so $d(I)=d(I\cup J)-d(J\setminus I)$ is
defined.

\medskip

{\bf (b)} These facts all follow directly from the corresponding
results
concerning $\Cal P\Bbb N$ and $d$ (491Ac).

\medskip

{\bf (c)} All we have to know is that $a\mapsto\bar d^*(a)$,
$a\mapsto 1\Bsetminus a$ are continuous;  so that
$\{a:\bar d^*(a)+\bar d^*(1\Bsetminus a)=1\}$ is closed.

\medskip

{\bf (d)} Because $A$ is upwards-directed, and $\bar d^*$ is a
non-decreasing functional on $\frak Z$, there is a non-decreasing
sequence $\sequencen{a_n}$ in $A$ such that
$\lim_{n\to\infty}\bar d^*(a_n)=\sup_{a\in A}\bar d^*(a)=\gamma$ say.
By 491J, $b=\lim_{n\to\infty}a_n=\sup_{n\in\Bbb N}a_n$ is defined in
$\frak Z$ and belongs to $D$.
If $a\in A$ and $\epsilon>0$, there is an $n\in\Bbb N$ such that
$\bar d^*(a_n)\ge\gamma-\epsilon$.   Let $a'\in A$ be a common upper
bound of $a$ and $a_n$.   Then

\Centerline{$\bar d^*(a\Bsetminus b)\le\bar d^*(a'\Bsetminus a_n)
=\bar d^*(a')-\bar d^*(a_n)\le\gamma-\bar d^*(a_n)\le\epsilon$.}

\noindent As $\epsilon$ is arbitrary, $a\Bsubseteq b$;  as $a$ is
arbitrary, $b$ is an upper bound of $A$;  as $b=\sup_{n\in\Bbb N}a_n$,
$b$ must be the supremum of $A$.

\medskip

{\bf (e)(i)$\Rightarrow$(ii)} Suppose that $\frak B$ is topologically
closed.   If $A\subseteq\frak B$ is a non-empty upwards-directed
subset with supremum $b\in\frak Z$, then
(d) tells us that $b\in\overline{A}\subseteq\frak B$.   It follows
that $\frak B$ is order-closed in $\frak Z$ (313E(a-i)).

\medskip

\quad{\bf (ii)$\Rightarrow$(iii)} Suppose that $\frak B$ is
order-closed in $\frak Z$.   If $A\subseteq\frak B$ is non-empty, then
$A'=\{a_0\Bcup\ldots\Bcup a_n:a_0,\ldots,a_n\in A\}$
is non-empty and upwards-directed, so has a supremum in $\frak Z$,
which must belong to $\frak B$, and must be the least upper bound of $A$
in $\frak B$.   Thus $\frak B$ is Dedekind ($\sigma$-)complete.   Now
let $\sequencen{a_n}$ be a disjoint sequence in $\frak B$ and set
$b_n=\sup_{i\le n}a_i$
for each $n$.   Then $\sequencen{b_n}$ is a non-decreasing sequence in
$D$ so has a limit and supremum $b\in D$, and $b\in\frak B$.   Also
$\bar d^*(b_n)=\sum_{i=0}^n\bar d^*(a_i)$ for each $n$ (induce on
$n$), so

\Centerline{$\bar\nu b=\bar d^*(b)=\lim_{n\to\infty}\bar d^*(b_n)
=\sum_{i=0}^{\infty}\bar d^*(a_i)=\sum_{i=0}^{\infty}\bar\nu a_i$.}

\noindent Since certainly $\bar\nu 0=0$, $\bar\nu 1=1$ and $\bar\nu b>0$
whenever $b\in\frak B\setminus\{0\}$, $(\frak B,\bar\nu)$ is a
probability algebra.

\medskip

\quad{\bf (iii)$\Rightarrow$(i)} Suppose that $(\frak B,\bar\nu)$ is a
probability algebra.   Then it is complete under its measure metric
(323Gc), which agrees on $\frak B$ with
the metric $\bar\rho$ of $\frak Z$;  so $\frak B$ must be
topologically closed in $\frak Z$.

We see also that $\frak B$ is regularly embedded in $\frak Z$.  \Prf\
(Compare 323H.)  If $A\subseteq\frak B$ is
non-empty and downwards-directed and has infimum $0$ in $\frak B$, and
$b\in\frak Z$ is any lower bound of $A$ in $\frak Z$, then

\Centerline{$\bar d^*(b)\le\inf_{a\in A}\bar d^*(a)
=\inf_{a\in A}\bar\nu a=0$}

\noindent (321F), so $b=0$.   Thus $\inf A=0$ in $\frak Z$.   As $A$ is
arbitrary, this is enough to show that the identity map from $\frak B$
to $\frak Z$ is
order-continuous (313Lb), that is, that $\frak B$ is regularly
embedded in $\frak Z$.\ \Qed

\medskip

{\bf (f)} Let $\frak B$ be the
order-closed subalgebra of $\frak Z$ generated by $I$.
If $I$ is closed under $\Bcap$, then (b), (d) and 313Gc tell us that
$\frak B\subseteq D$.   If $I$ is closed under $\Bcup$,
then $I'=\{1\Bsetminus a:a\in I\}$
is a subset of $D$ closed under $\Bcap$, while $\frak B$ is the
order-closed subalgebra generated by $I'$, so again $\frak B\subseteq D$.
By (e), $\frak B$ is in either case topologically closed.   So we see that
the topologically closed subalgebra generated by $I$ is included in $D$;
by (e) again, it is equal to $\frak B$.
}%end of proof of 491K

\leader{491L}{Effectively regular measures}\cmmnt{ The examples 491Xf
and 491Yc show that the definition in 491B is drawn a little too wide
for comfort, and allows some uninteresting
pathologies.   These do not arise in the measure spaces we care most
about, and the following definitions provide a fire-break.}
Let $(X,\Sigma,\mu)$ be a measure space, and $\frak T$ a topology on
$X$.

\spheader 491La I will say that
a measurable subset $K$ of $X$ of finite measure is {\bf regularly
enveloped} if for every $\epsilon>0$ there are an open measurable set
$G$ and a closed measurable set $F$ such that $K\subseteq G\subseteq F$
and $\mu(F\setminus K)\le\epsilon$.

\spheader 491Lb Note that the family\cmmnt{ $\Cal K$} of
regularly enveloped
measurable sets of finite measure is closed under finite unions and
countable intersections.   \prooflet{\Prf\
(i) If $K_1$, $K_2\in\Cal K$ and $*$ is either $\cup$ or $\cap$, let
$\epsilon>0$.   Take measurable open sets $G_1$, $G_2$ and measurable
closed sets $F_1$, $F_2$ such that $K_i\subseteq G_i\subseteq F_i$ and
$\mu(F_i\setminus K_i)\le\bover12\epsilon$ for both $i$.   Then
$G_1*G_2$ is a measurable open set, $F_1*F_2$ is a measurable closed
set, $K_1*K_2\subseteq G_1*G_2\subseteq F_1*F_2$ and
$\mu((F_1*F_2)\setminus(K_1*K_2))\le\epsilon$.   As $\epsilon$ is
arbitrary, $K_1*K_2\in\Cal K$.   (ii) If $\sequencen{K_n}$ is a
non-increasing sequence in $\Cal K$ with intersection $K$ and
$\epsilon>0$, let $n\in\Bbb N$ be such that $\mu K_n<\mu K+\epsilon$.
Then we can find a measurable open set $G$ and a measurable
closed set $F$ such that $K_n\subseteq G\subseteq F$ and
$\mu F\le\mu K+\epsilon$.   As $\epsilon$ is arbitrary, $K\in\Cal K$.
Together with (i), this is enough to show that
$\Cal K$ is closed under countable intersections.\ \Qed}

\spheader 491Lc \cmmnt{Now I say that} $\mu$ is {\bf effectively
regular} if it
is inner regular with respect to the regularly enveloped sets of
finite measure.

\leader{491M}{Examples (a)} Any totally finite Radon measure is
effectively regular.   \prooflet{\Prf\ Let $(X,\frak T,\Sigma,\mu)$
be a totally finite Radon measure space.
If $K\subseteq X$ is compact and $\epsilon>0$, let
$L\subseteq X\setminus K$ be a compact set such that
$\mu L\ge 1-\mu K+\epsilon$.   Let $G$, $H$ be disjoint open sets
including $K$, $L$ respectively (4A2F(h-i)).   Then
$K\subseteq G\subseteq X\setminus H$, $G$ is open, $X\setminus H$
is closed, both $G$ and $X\setminus H$ are measurable, and
$\mu((X\setminus H)\setminus K)\le\epsilon$.   This shows that every
compact set is regularly enveloped,
and $\mu$ is effectively regular.\ \Qed}

\spheader 491Mb Let $(X,\frak T,\Sigma,\mu)$ be a quasi-Radon measure
space such that $\frak T$ is a regular topology.   Then $\mu$ is
effectively regular.   \prooflet{\Prf\ Let
$E\in\Sigma$ and take $\gamma<\mu E$.   Choose sequences
$\sequencen{E_n}$ and $\sequencen{G_n}$ inductively, as follows.
$E_0\subseteq E$ is to be any measurable set
such that $\gamma<\mu E_0<\infty$.   Given that $\mu E_n>\gamma$, let
$G$ be an open set of finite measure such that $\mu(E_n\cap G)>\gamma$
(414Ea), and $F\subseteq G\setminus E_n$
a closed set such that $\mu F\ge\mu(G\setminus E_n)-2^{-n}$.   Let
$\Cal H$ be the family of open sets $H$ such that
$\overline H\subseteq G\setminus F$.   Then $\Cal H$ is
upwards-directed and covers $E_n$ (because $\frak T$ is regular), so
there is a $G_n\in\Cal H$ such that $\mu(E_n\cap G_n)>\gamma$ (414Ea
again).   Now $\mu(\overline{G}_n\setminus E_n)\le 2^{-n}$.   Set
$E_{n+1}=E_n\cap G_n$, and continue.

At the end of the induction, set $K=\bigcap_{n\in\Bbb N}E_n$.   For
each $n$, $K\subseteq G_n\subseteq\overline{G}_n$ and

\Centerline{$\lim_{n\to\infty}\mu(\overline{G}_n\setminus K)
\le\lim_{n\to\infty}2^{-n}+\mu(E_n\setminus K)=0$,}

\noindent so $K$ is regularly enveloped.   At the same time,
$K\subseteq E$ and $\mu K\ge\gamma$.   As $E$ and $\gamma$ are
arbitrary, $\mu$ is effectively regular.\ \Qed}

\spheader 491Mc Any totally finite Baire measure is effectively regular.
\prooflet{\Prf\ Let $\mu$ be a totally finite Baire measure on a
topological space $X$.   If $F\subseteq X$
is a zero set, let $f:X\to\Bbb R$ be a continuous function such that
$F=f^{-1}[\{0\}]$.   For each $n\in\Bbb N$, set
$G_n=\{x:|f(x)|<2^{-n}\}$, $F_n=\{x:|f(x)|\le 2^{-n}\}$;  then
$G_n$ is a measurable open set, $F_n$ is a measurable closed set,
$F\subseteq G_n\subseteq F_n$ for every $n$ and
$\lim_{n\to\infty}\mu F_n=\mu F$ (because $\mu$ is totally finite).
This shows that every zero set is regularly enveloped;  as $\mu$ is
inner regular with respect to the zero sets (412D), $\mu$ is
effectively regular.\ \Qed}

\spheader 491Md A totally finite completion regular topological measure
is effectively regular.   \prooflet{(As in (c), all zero sets are
regularly enveloped.)}

\leader{491N}{Theorem} Let $X$ be a topological space and $\mu$ an
effectively regular probability measure on $X$, with measure algebra
$(\frak A,\bar\mu)$.
Suppose that $\sequencen{x_n}$ is an equidistributed sequence in $X$.
Then we have a unique order-continuous Boolean
homomorphism $\pi:\frak A\to\frak Z=\Cal P\Bbb N/\Cal Z$ such that
$\pi G^{\ssbullet}\Bsubseteq\{n:x_n\in G\}^{\ssbullet}$ for every
measurable open set $G\subseteq X$,
and $\bar d^*(\pi a)=\bar\mu a$ for every $a\in\frak A$.

\proof{{\bf (a)} Define $\theta:\Cal PX\to\frak Z$ by setting
$\theta A=\{n:x_n\in A\}^{\ssbullet}$ for $A\subseteq X$;  then
$\theta$ is a Boolean homomorphism.
If $F\subseteq X$ is closed and measurable, then
$\bar d^*(\theta F)\le\mu F$, because $\sequencen{x_n}$ is
equidistributed.   Write $\Cal K$ for the family of
regularly enveloped measurable sets.

If $K\in\Cal K$, then $\pi_0K=\inf\{\theta G:K\subseteq G\in\Sigma\cap\frak T\}$
is defined in $\frak Z$, $\bar d^*(\pi_0K)=\mu K$ and $\pi_0K\in D$ as
defined in 491K.   \Prf\ For each $n\in\Bbb N$, let $G_n$,
$F_n\in\Sigma$ be such that $K\subseteq G_n\subseteq F_n$,
$G_n$ is open, $F_n$ is closed and $\mu(F_n\setminus K)\le 2^{-n}$.
Set $H_n=X\setminus\bigcap_{i\le n}G_i$.   Then

$$\eqalign{\bar d^*(\theta H_n)+\bar d^*(1\Bsetminus\theta H_n)
&\le\bar d^*(\theta H_n)+\bar d^*(\theta(\bigcap_{i\le n}F_i))
\le\mu H_n+\mu(\bigcap_{i\le n}F_i)\cr
&\le\mu(X\setminus K)+\mu F_n
\le 1+2^{-n}.\cr}$$

\noindent Also $\sequencen{\theta H_n}$ is a non-decreasing sequence in
$\frak Z$.   By 491J,
$a=\lim_{n\to\infty}\theta H_n=\sup_{n\in\Bbb N}\theta H_n$ is defined
in $\frak Z$ and belongs to $D$.   Set

\Centerline{$b=1\Bsetminus a
=\lim_{n\to\infty}1\Bsetminus\theta H_n
=\lim_{n\to\infty}\theta(\bigcap_{i\le n}G_i)$,}

\noindent so that $b$ also belongs to $D$.   If
$K\subseteq G\in\Sigma\cap\frak T$, then

\Centerline{$b\Bsetminus\theta G
=\lim_{n\to\infty}\theta(\bigcap_{i\le n}G_i)\Bsetminus\theta G
=\lim_{n\to\infty}\theta(\bigcap_{i\le n}G_i\setminus G)$,}

\noindent and

$$\eqalign{\bar d^*(b\Bsetminus\theta G)
&=\lim_{n\to\infty}\bar d^*(\theta(\bigcap_{i\le n}G_i\setminus G))\cr
&\le\lim_{n\to\infty}\bar d^*(\theta(\bigcap_{i\le n}F_i\setminus G))
\le\lim_{n\to\infty}\mu(\bigcap_{i\le n}F_i\setminus G)
=0.\cr}$$

\noindent This shows that $b\subseteq\theta G$ whenever
$K\subseteq G\in\Sigma\cap\frak T$.   On the other hand, any lower
bound of $\{\theta G:K\subseteq G\in\Sigma\cap\frak T\}$ is also a
lower bound of $\{\theta(\bigcap_{i\le n}G_i):n\in\Bbb N\}$, so is
included in $b$.   Thus
$b=\inf\{\theta G:K\subseteq G\in\Sigma\cap\frak T\}$ and $\pi_0(K)=b$ is
defined.

To compute $\bar d^*(b)$, observe first that
$b\Bsubseteq 1\Bsetminus\theta H_n\Bsubseteq\theta F_n$ for every $n$, so

\Centerline{$\bar d^*(b)\le\inf_{n\in\Bbb N}\bar d^*(\theta F_n)
\le\inf_{n\in\Bbb N}\mu F_n=\mu K$.}

\noindent On the other hand,

\Centerline{$\bar d^*(\theta(\bigcap_{i\le n}G_i))
\ge 1-\bar d^*(\theta H_n)\ge 1-\mu H_n\ge\mu K$}

\noindent for every $n$, so

\Centerline{$\bar d^*(b)
=\lim_{n\to\infty}\bar d^*(\theta(\bigcap_{i\le n}G_i))\ge\mu K$.}

\noindent Accordingly $\bar d^*(b)=\mu K$, and $\pi_0K$ has the required
properties.\ \Qed

\medskip

{\bf (b)} If $K$, $L\in\Cal K$, then $\pi_0(K\cap L)=\pi_0K\Bcap\pi_0L$.
\Prf\ We know that $K\cap L\in\Cal K$ (491Lb).   Now

$$\eqalignno{\pi_0K\Bcap\pi_0L
&=\inf\{\theta G:K\subseteq G\in\frak T\}
   \Bcap\inf\{\theta H:L\subseteq H\in\frak T\}\cr
&=\inf\{\theta G\Bcap\theta H:
  K\subseteq G\in\frak T,\,L\subseteq H\in\frak T\}\cr
&=\inf\{\theta(G\cap H):
  K\subseteq G\in\frak T,\,L\subseteq H\in\frak T\}
\Bsupseteq\pi_0(K\cap L).\cr}$$

\noindent Now suppose that $U\supseteq K\cap L$ is a measurable open
set
and $\epsilon>0$.   Let $G$, $G'$ be measurable open sets and $F$,
$F'$
measurable closed sets such that
$K\subseteq G\subseteq F$, $L\subseteq G'\subseteq F'$,
$\mu(F\setminus
K)\le\epsilon$ and $\mu(F'\setminus L)\le\epsilon$.   Then

$$\eqalign{\bar d^*(\pi_0K\Bcap\pi_0L\Bsetminus\theta U)
&\le\bar d^*(\theta G\Bcap\theta G'\Bsetminus\theta U)
=\bar d^*(\theta(G\cap G'\setminus U))\cr
&\le\bar d^*(\theta(F\cap F'\setminus U))
\le\mu(F\cap F'\setminus U)
\le 2\epsilon.\cr}$$

\noindent As $\epsilon$ is arbitrary,
$\pi_0K\Bcap\pi_0L\Bsubseteq\theta U$;  as $U$ is arbitrary,
$\pi_0K\Bcap\pi_0L\Bsubseteq\pi_0(K\cap L)$.\ \Qed

This means that $\{\pi_0K:K\subseteq X$ is a regularly embedded
measurable set$\}$ is a subset of $D$ closed under $\Bcap$.
By 491Kf, the topologically closed subalgebra $\frak B$ of $\frak Z$
generated by this family is included in $D$;
by 491Ke, $\frak B$ is order-closed and regularly embedded in $\frak Z$,
and $(\frak B,\bar d^*\restrp\frak B)$ is a probability algebra.

\medskip

{\bf (c)} Now observe that if we set
$Q=\{K^{\ssbullet}:K\in\Cal K\}\subseteq\frak A$, we have a function
$\pi:Q\to\frak B$ defined by setting $\pi K^{\ssbullet}=\pi_0K$
whenever $K\in\Cal K$.   \Prf\ Suppose that $K$, $L\in\Cal K$
and $\mu(K\symmdiff L)=0$.   Then

$$\eqalignno{\bar d^*(\pi_0K\Bsymmdiff\pi_0L)
&=\bar d^*(\pi_0K)+\bar d^*(\pi_0L)-2\bar d^*(\pi_0K\Bcap\pi_0L)\cr
\displaycause{because $\pi_0K$ and $\pi_0L$ belong to $\frak B\subseteq D$}
&=\bar d^*(\pi_0K)+\bar d^*(\pi_0L)-2\bar d^*(\pi_0(K\cap L))\cr
&=\mu K+\mu L-2\mu(K\cap L)
=0.\cr}$$

\noindent So $\pi_0K=\pi_0L$ and either can be used to define
$\pi K^{\ssbullet}$.\ \QeD\  Next, the same formulae show that
$\pi:Q\to\frak B$ is an isometry
when $Q$ is given the measure metric of $\frak A$, since if $K$,
$L$ belong to $\Cal K$,

\Centerline{$\bar\rho(\pi K^{\ssbullet},\pi L^{\ssbullet})
=\bar d^*(\pi_0K\Bsymmdiff\pi_0L)=\mu K+\mu L-2\mu(K\cap L)
=\mu(K\symmdiff L)=\bar\mu(K^{\ssbullet}\Bsymmdiff L^{\ssbullet})$.}

\noindent As $Q$ is dense in $\frak A$ (412N), there is a unique
extension of $\pi$ to an isometry from $\frak A$ to $\frak B$.

\medskip

{\bf (d)} Because

\Centerline{$\pi(K^{\ssbullet}\Bcap L^{\ssbullet})
=\pi(K\cap L)^{\ssbullet}=\pi_0(K\cap L)=\pi_0K\Bcap\pi_0L
=\pi K^{\ssbullet}\Bcap\pi L^{\ssbullet}$}

\noindent for all $K$, $L\in\Cal K$, $\pi(a\Bcap a')=\pi a\Bcap\pi a'$
for all $a$, $a'\in\frak A$.
It follows that $\pi$ is a Boolean homomorphism.   \Prf\ The point is
that $\bar d^*(\pi a)=\bar\mu a$ for every $a\in Q$, and therefore for
every $a\in\frak A$.   Now if $a\in\frak A$, $\pi(1\Bsetminus a)$ must
be disjoint from $\pi a$ (since certainly $\pi 0=0$), and has the same
measure as $1\Bsetminus\pi a$ (remember that we know that
$(\frak B,\bar d^*\restrp\frak B)$ is a measure algebra), so must be
equal to $1\Bsetminus\pi a$.
By 312H, $\pi$ is a Boolean homomorphism.\ \Qed

By 324G, $\pi$ is order-continuous when regarded as a function from
$\frak A$ to $\frak B$.   Because $\frak B$ is regularly embedded in
$\frak Z$, $\pi$ is order-continuous when regarded as a function from
$\frak A$ to $\frak Z$.

\medskip

{\bf (e)} Let $G\in\Sigma\cap\frak T$.   For any $\epsilon>0$, there
is
a $K\in\Cal K$ such that $K\subseteq G$ and $\mu(G\setminus
K)\le\epsilon$.
In this case, $\pi K^{\ssbullet}=\pi_0K\subseteq\theta G$.   So

\Centerline{$\bar d^*(\pi G^{\ssbullet}\Bsetminus\theta G)
\le\bar d^*(\pi G^{\ssbullet}\Bsetminus\pi K^{\ssbullet})
=\bar\mu(G^{\ssbullet}\Bsetminus K^{\ssbullet})=\mu(G\setminus K)
\le\epsilon$.}

\noindent As $\epsilon$ is arbitrary,
$\pi G^{\ssbullet}\subseteq\theta G$.

\medskip

{\bf (f)} This shows that we have a homomorphism $\pi$ with the
required
properties.   To see that $\pi$ is unique, suppose that
$\pi':\frak A\to\frak Z$ is any
homomorphism of the same kind.   In this case

\Centerline{$\bar d^*(1\Bsetminus\pi'a)=\bar d^*(\pi'(1\Bsetminus
a))=\bar\mu(1\Bsetminus a)=1-\bar\mu a=1-\bar d^*(\pi'a)$,}

\noindent so $\pi'a\in D$, for every $a\in\frak A$.   If $K\in\Cal K$,
then $\pi'K^{\ssbullet}\subseteq\theta G$ whenever $K\subseteq
G\in\Sigma\cap\frak T$, so
$\pi'K^{\ssbullet}\Bsubseteq\pi_0K=\pi K^{\ssbullet}$.   As both $\pi
K^{\ssbullet}$ and $\pi'K^{\ssbullet}$ belong to $D$,

\Centerline{$\bar d^*(\pi K^{\ssbullet}\Bsetminus\pi'K^{\ssbullet})
=\bar d^*(\pi K^{\ssbullet})-\bar d^*(\pi'K^{\ssbullet})
=\mu K-\mu K=0$,}

\noindent and $\pi K^{\ssbullet}=\pi'K^{\ssbullet}$.   As
$\{K^{\ssbullet}:K\in\Cal K\}$ is topologically dense in $\frak A$,
and both $\pi$ and $\pi'$ are continuous,
they must be equal.
}%end of proof of 491N

\leader{491O}{Proposition} Let $X$ be a topological space and $\mu$ an
effectively regular probability measure on $X$ which measures every zero
set, and suppose that
$\sequencen{x_n}$ is an equidistributed sequence in $X$.   Let $\frak A$
be the measure algebra of $\mu$ and
$\pi:\frak A\to\frak Z=\Cal P\Bbb N/\Cal Z$ the regular embedding
described in 491N;  let
$T_{\pi}:L^{\infty}(\frak A)\to L^{\infty}(\frak Z)$ be the corresponding
order-continuous Banach algebra embedding\cmmnt{ (363F)}.   Let
$S:\ell^{\infty}(X)\to\ell^{\infty}$ be the Riesz homomorphism defined
by setting $(Sf)(n)=f(x_n)$ for
$f\in\ell^{\infty}(X)$ and $n\in\Bbb N$, and
$R:\ell^{\infty}\to L^{\infty}(\frak Z)$ the Riesz homomorphism
corresponding to the Boolean homomorphism
$I\mapsto I^{\ssbullet}:\Cal P\Bbb N\to\frak Z$.   For
$f\in\eusm L^{\infty}(\mu)$ let $f^{\ssbullet}$ be the corresponding
member of $L^{\infty}(\mu)\cong L^{\infty}(\frak A)$
\cmmnt{ (363I)}.   Then $T_{\pi}(f^{\ssbullet})=RSf$ for every
$f\in C_b(X)$.

\proof{ To begin with, suppose that $f:X\to[0,1]$ is continuous and
$k\ge 1$.   For each $i\le k$ set $G_i=\{x:f(x)>\bover{i}{k}\}$,
$F_i=\{x:f(x)\ge\bover{i}{k}\}$.
Then $\bover1k\sum_{i=1}^k\chi F_i\le f\le\bover1k\sum_{i=0}^k\chi G_i$.
So

\Centerline{$\Bover1k\sum_{i=1}^k\chi(\pi F_i^{\ssbullet})
\le T_{\pi}f^{\ssbullet}\le\Bover1k\sum_{i=0}^k\chi(\pi G_i^{\ssbullet})$,}

\Centerline{$\Bover1k\sum_{i=1}^k\chi(\theta F_i)
\le RSf\le\Bover1k\sum_{i=0}^k\chi(\theta G_i)$}

\noindent where $\theta:\Cal PX\to\frak Z$ is the Boolean homomorphism
described in 491N, because $RS:\ell^{\infty}\to L^{\infty}(\frak Z)$
is the Riesz homomorphism corresponding
to $\theta$ (see 363Fa, 363Fg).   Now 491N tells us that
$\pi G^{\ssbullet}\Bsubseteq\theta G$ for every cozero set
$G\subseteq X$, so

$$\eqalign{T_{\pi}f^{\ssbullet}
&\le\Bover1k\sum_{i=0}^k\chi(\pi G_i^{\ssbullet})
\le\Bover1k\sum_{i=0}^k\chi(\theta G_i)\cr
&\le\Bover1k\sum_{i=0}^k\chi(\theta F_i)
=\Bover1ke+\sum_{i=1}^k\chi(\theta F_i)
\le\Bover1ke+RSf\cr}$$

\noindent where $e$ is the standard order unit of the $M$-space
$L^{\infty}(\frak Z)$.   But looking at complements we see that we
must have $\pi F^{\ssbullet}\Bsupseteq\theta F$
for every zero set $F\subseteq X$, so

$$\eqalign{RSf
&\le\Bover1k\sum_{i=0}^k\chi(\theta G_i)
\le\Bover1k\sum_{i=0}^k\chi(\theta F_i)\cr
&\le\Bover1k\sum_{i=0}^k\chi(\pi F_i^{\ssbullet})
=\Bover1ke+\sum_{i=1}^k\chi(\pi F_i^{\ssbullet})
\le\Bover1ke+T_{\pi}f^{\ssbullet}.\cr}$$

\noindent This means that $|T_{\pi}f^{\ssbullet}-RSf|\le\bover1ke$ for
every $k\ge 1$, so that $T_{\pi}f^{\ssbullet}=RSf$.   This is true
whenever $f\in C_b(X)$ takes values in
$[0,1]$;  as all the operators here are linear, it is true for every
$f\in C_b(X)$.
}%end of proof of 491O

\leader{491P}{Proposition} Any probability algebra $(\frak A,\bar\mu)$
of cardinal at most $\frak c$ can be regularly embedded as a
subalgebra of $\frak Z=\Cal P\Bbb N/\Cal Z$
in such a way that $\bar\mu$ is identified with the restriction of the
submeasure $\bar d^*$ to the image of $\frak A$.

\proof{ The usual measure of $\{0,1\}^{\frak c}$ is a totally finite
Radon measure, so is effectively regular (491Ma).   It has an
equidistributed sequence
(491G), so its measure algebra $(\frak B_{\frak c},\bar\nu_{\frak c})$
can be regularly embedded in $\frak Z$ in a way which matches
$\bar\nu_{\frak c}$ with $\bar d^*$ (491N).
Now if $(\frak A,\bar\mu)$ is any probability algebra of cardinal at
most $\frak c$, it can be regularly embedded (by a measure-preserving
homomorphism) in
$(\frak B_{\frak c},\bar\nu_{\frak c})$ (332N), and therefore in
$(\frak Z,\bar d^*)$.   This completes the proof.
}%end of proof of 491P

\leader{491Q}{Corollary} Every Radon probability measure on
$\{0,1\}^{\frak c}$ has an equidistributed sequence.

\proof{ Let $\mu$ be a Radon probability measure on $\{0,1\}^{\frak c}$,
and $(\frak A,\bar\mu)$ its measure algebra.   By 491P, there is a
measure-preserving embedding
$\pi:\frak A\to\frak Z$, and $\pi[\frak A]\subseteq D$ as defined in
491K.   For $\xi<\frak c$ let $a_{\xi}\in\frak A$ be the equivalence
class of $\{x:x(\xi)=1\}$,
and let $I_{\xi}\subseteq\Bbb N$ be such that
$I_{\xi}^{\ssbullet}=\pi a_{\xi}$ in $\frak Z$.   Define $x_n(\xi)$, for
$n\in\Bbb N$ and $\xi<\frak c$, by setting $x_n(\xi)=1$ if $n\in I_{\xi}$,
$0$ otherwise.
Now suppose that $E\subseteq\{0,1\}^{\frak c}$ is a basic open set of
the form $\{x:x(\xi)=1$ for $\xi\in K$, $0$ for $\xi\in L\}$, where
$K$, $L\subseteq\frak c$
are finite.   Set $b=\pi E^{\ssbullet}$ in $\frak Z$,

\Centerline{$I=\{n:x_n\in E\}=\Bbb N\cap\bigcap_{\xi\in
K}I_{\xi}\setminus\bigcup_{\xi\in L}I_{\xi}$.}

\noindent Then

$$\eqalign{b
&=\pi E^{\ssbullet}
=\pi(\inf_{\xi\in K}a_{\xi}\Bsetminus\sup_{\xi\in L}a_{\xi})\cr
&=\inf_{\xi\in K}\pi a_{\xi}\Bsetminus\sup_{\xi\in L}\pi a_{\xi}
=\inf_{\xi\in K}I_{\xi}^{\ssbullet}\Bsetminus\sup_{\xi\in
L}I_{\xi}^{\ssbullet}
=I^{\ssbullet}.\cr}$$

\noindent Since $b\in D$, $d(I)$ is defined and is equal to
$\bar d^*(b)=\bar\mu E^{\ssbullet}=\mu E$.

If we now take $E$ to be an open-and-closed subset of
$\{0,1\}^{\frak c}$, it can be expressed as a disjoint union of
finitely
many basic open sets of the type just considered;  because $d$ is
additive on disjoint sets, $d(\{n:x_n\in E\})$ is defined and equal to
$\mu E$.   But this is enough to ensure that $\sequencen{x_n}$ is
equidistributed, by 491Ch.
}%end of proof of 491Q

\leader{491R}{}\cmmnt{ In this section I have been looking at
probability measures with equidistributed sequences.   A standard
line of investigation is to ask which of our ordinary constructions,
applied to such measures, lead to others of the same kind, as in
491Ea and 491F.   We find that the language developed here enables us
to express another result of this type.

\medskip

\noindent}{\bf Proposition} Let $X$ be a topological space, $\mu$ an
effectively regular topological probability measure on $X$ which has
an equidistributed sequence, and $\nu$ a probability measure on $X$
which is an indefinite-integral measure over $\mu$.   Then $\nu$ has
an equidistributed sequence.

\proof{ Let $\Cal K$ be the family of regularly enveloped measurable
sets.

\medskip

{\bf (a)} Consider first the case in which $\nu$ has Radon-Nikod\'ym
derivative of the form $\Bover1{\mu K}\chi K$ for some $K\in\Cal K$
of non-zero measure.    For each $m\in\Bbb N$, we have an open set
$G_m\supseteq K$ such that $\mu(\overline{G}_m\setminus K)\le 2^{-m}$;
of course we can arrange that $G_{m+1}\subseteq G_m$ for
each $m$.   Let $\sequencen{x_n}$ be an equidistributed sequence for
$\mu$.   Then there is an $I\subseteq\Bbb N$ such that $d(I)=\mu K$
and $\{n:n\in I$, $x_n\notin G_m\}$ is finite for every $m$.   \Prf\ For
each $m\in\Bbb N$, set $I_m=\{n:x_n\in G_m\}$.   We know that
$\liminf_{n\to\infty}\Bover1n\#(I_m\cap n)\ge\mu G_m\ge\mu K$ for
each $m$, so we can find a strictly increasing sequence
$\sequence{m}{k_m}$ such that
$\Bover1n\#(I_m\cap n)\ge\mu K-2^{-m}$ whenever $m\in\Bbb N$ and
$n>k_m$.   Set $I=\bigcup_{m\in\Bbb N}I_m\cap k_{m+1}$.   If
$k_m<n\le k_{m+1}$,

\Centerline{$\Bover1n\#(I\cap n)\ge\Bover1n\#(I_m\cap n)
\ge\mu K-2^{-m}$;}

\noindent so $\liminf_{n\to\infty}\Bover1n\#(I\cap n)\ge\mu K$.   On
the other hand, for any $m\in\Bbb N$,

\Centerline{$\{n:n\in I$, $x_n\notin F_m\}\subseteq I\setminus I_m
\subseteq k_{m+1}$}

\noindent is finite, so

$$\eqalign{\limsup_{n\to\infty}\Bover1n\#(I\cap n)
&\le\limsup_{n\to\infty}\Bover1n\#(\{i:i<n,\,x_i\in F_m\})\cr
&\le\mu F_m\le\mu K+2^{-m}.\cr}$$

\noindent Accordingly $\limsup_{n\to\infty}\Bover1n\#(I\cap n)\le\mu K$ and
$d(I)$ is defined and equal to $\mu K$.\ \Qed

Let $\sequencen{j_n}$ be the increasing enumeration of $I$, and set
$y_n=x_{j_n}$ for each $n$.   Then $\sequencen{y_n}$ is
equidistributed for $\nu$.   \Prf\ Note first that

\Centerline{$\lim_{n\to\infty}\Bover{n}{j_n}
=\lim_{n\to\infty}\Bover1{j_n}\#(I\cap j_n)=\mu K$.}

\noindent Let $F\subseteq X$ be closed.
Then $\nu F=\Bover{\mu(F\cap K)}{\mu K}$.   On the other hand, for any
$m\in\Bbb N$,

$$\eqalignno{d^*(\{n:y_n\in F\})
&=\limsup_{n\to\infty}\Bover1n\#(\{i:i<n,\,x_{j_i}\in F\})\cr
&=\limsup_{n\to\infty}\Bover1n\#(\{i:i<j_n,\,i\in I,\,x_i\in F\})\cr
&=\limsup_{n\to\infty}
  \Bover1n\#(\{i:i<j_n,\,i\in I,\,x_i\in F\cap G_m\})\cr
\displaycause{because $\{i:i\in I$, $x_i\notin G_m\}$ is finite}
&\le\limsup_{n\to\infty}
  \Bover{j_n}{n}\Bover1{j_n}\#(\{i:i<j_n,\,x_i\in F\cap F_m\})\cr
&=\Bover1{\mu K}\limsup_{n\to\infty}
  \Bover1{j_n}\#(\{i:i<j_n,\,x_i\in F\cap F_m\})\cr
&\le\Bover1{\mu K}\limsup_{n\to\infty}
  \Bover1{n}\#(\{i:i<n,\,x_i\in F\cap F_m\})\cr
&\le\Bover1{\mu K}\mu(F\cap F_m)
\le\Bover1{\mu K}(\mu(F\cap K)+2^{-m})
=\nu F+\Bover1{2^m\mu K};\cr}$$

\noindent as $m$ is arbitrary, $d^*(\{n:y_n\in F\})\le\nu F$;  as $F$
is arbitrary, $\sequencen{y_n}$ is equidistributed for $\nu$.\ \Qed

\medskip

{\bf (b)} Now turn to the general case.   Let $f$ be a
Radon-Nikod\'ym derivative of $\nu$;  we may suppose that $f$ is
measurable and
non-negative.   Then there is a sequence $\sequence{m}{K_m}$ in $\Cal
K$ such that $f\eae\sum_{m=0}^{\infty}\bover1{m+1}\chi K_m$.
\Prf\ Choose $f_m$, $K_m$ inductively, as follows.   $f_0=f$.
Given that $f_m\ge 0$ is measurable, set
$E_m=\{x:f_m(x)\ge\bover1{m+1}\}$ and let $K_m\in\Cal K$ be such that
$K_m\subseteq E_m$ and $\mu(E_m\setminus K_m)\le 2^{-m}$;  set
$f_{m+1}=f_m-\bover1{m+1}\chi K_m$.   Then $\sequence{m}{f_m}$ is
non-increasing;  set $g=\lim_{m\to\infty}f_m$.   \Quer\ If $g$ is not
zero almost everywhere, let $r\in\Bbb N$ be such that $\mu
E>2^{-r+1}$ where $E=\{x:g(x)\ge\bover1{r+1}\}$.   Then $E\subseteq
E_m$ for every $m\ge r$, so $\mu(E\setminus K_m)\le 2^{-m}$ for every
$m\ge r$ and $F=E\cap\bigcap_{m\ge r}K_m$ is not empty.   Take $x\in
F$;  then
$f_{m+1}(x)\le f_m(x)-\bover1{m+1}$ for every $m\ge r$, which is
impossible.\ \BanG\  So $g=0$ a.e.\ and
$f\eae\sum_{m=0}^{\infty}\bover1{m+1}\chi K_m$.\ \Qed

By (a), we have for each $m$ a sequence $\sequencen{y_{mn}}$ in $X$
such that

\Centerline{$\mu(F\cap K_m)
\ge\mu K_m\cdot\limsup_{n\to\infty}\Bover1n\#(\{i:i<n,\,y_{mi}\in F\})$}

\noindent for
every closed $F\subseteq X$.   For $n\in\Bbb N$, let $\nu_n$ be the
point-supported measure on $X$ defined by setting

\Centerline{$\nu_nA=\sum_{m=0}^{\infty}
  \Bover{\mu K_m}{(n+1)(m+1)}\#(\{i:i\le n,\,y_{mi}\in A\})$}

\noindent for $A\subseteq X$;  because
$\sum_{m=0}^{\infty}\Bover{\mu K_m}{m+1}=\int fd\mu=1$, $\nu_n$ is a
probability measure.   If $F\subseteq X$ is closed,

$$\eqalignno{\limsup_{n\to\infty}\nu_nF
&\le\sum_{m=0}^{\infty}\Bover{\mu K_m}{m+1}
  \limsup_{n\to\infty}\Bover1{n+1}\#(\{i:i\le n,\,y_{mi}\in F\})\cr
\displaycause{because $\sum_{m=0}^{\infty}\Bover{\mu
K_m}{m+1}<\infty$}
&\le\sum_{m=0}^{\infty}\Bover1{m+1}\mu(F\cap K_m)
=\int_Ffd\mu
=\nu F.\cr}$$

\noindent So 491D tells us that there is an equidistributed sequence
for $\nu$, as required.
}%end of proof of 491R

\leader{491S}{The asymptotic density
filter}\dvAnew{2011} \cmmnt{Corresponding to the
asymptotic density ideal, of course we have a filter.
It is not surprising that convergence along this filter, in the sense of
2A3Sb, should be interesting and sometimes important.

\medskip

}{\bf (a)} Set

\Centerline{$\Cal F_d=\{\Bbb N\setminus I:I\in\Cal Z\}
=\{I:I\subseteq\Bbb N$, $\lim_{n\to\infty}\Bover1n\#(I\cap n)=1\}$.}

\noindent\cmmnt{ Then }$\Cal F_d$ is a filter on $\Bbb N$,
the {\bf (asymptotic) density filter}.

\spheader 491Sb For a
bounded sequence $\sequencen{\alpha_n}$ in $\Bbb C$, 
$\lim_{n\to\Cal F_d}\alpha_n=0$ iff
$\lim_{n\to\infty}\Bover1{n+1}\sum_{k=0}^n|\alpha_k|=0$.
\prooflet{\Prf\
Set $M=\sup_{k\in\Bbb N}|\alpha_k|$, and for $\epsilon>0$ set
$I_{\epsilon}=\{n:|\alpha_n|\le\epsilon\}$.   Then, for any $n\ge 1$,

\Centerline{$\Bover{\epsilon}{n+1}\#((n+1)\setminus I_{\epsilon})
\le\Bover1{n+1}\sum_{k=0}^n|\alpha_k|
\le\epsilon+\Bover{M}{n+1}\#((n+1)\setminus I_{\epsilon})$.}

\noindent So if
$\lim_{n\to\infty}\Bover1{n+1}\sum_{k=0}^n|\alpha_k|=0$, then
$\lim_{n\to\infty}\Bover1{n+1}\#((n+1)\setminus I_{\epsilon})=0$, that is,
$\Bbb N\setminus I_{\epsilon}\in\Cal Z$ and $I_{\epsilon}\in\Cal F_d$;  as
$\epsilon$ is arbitrary, $\lim_{n\to\Cal F_d}\alpha_n=0$.   While if
$\lim_{n\to\Cal F_d}\alpha_n=0$ then
$\Bbb N\setminus I_{\epsilon}\in\Cal Z$ and
$\limsup_{n\to\infty}\Bover1{n+1}\sum_{k=0}^n|\alpha_k|\le\epsilon$;
again, $\epsilon$ is arbitrary, so
$\lim_{n\to\infty}\Bover1{n+1}\sum_{k=0}^n|\alpha_k|=0$.\ \Qed}

\spheader 491Sc For any $m\in\Bbb N$ and $A\subseteq\Bbb N$,
$A+m\in\Cal F_d$ iff $A\in\Cal F_d$.   \prooflet{\Prf\ For any $n\ge m$,
$\#(n\cap(A+m))=\#((n-m)\cap A)$, so

\Centerline{$d(A+m)
=\lim_{n\to\infty}\Bover1n\#(n\cap(A+m))
=\lim_{n\to\infty}\Bover1n\#(n\cap A)
=d(A)$}

\noindent if either $d(A+m)$ or $d(A)$ is defined, in particular, if
either $A+m$ or $A$ belongs to $\Cal F_d$.\ \Qed}
Hence, or otherwise,
for any (real or complex) sequence $\sequencen{\alpha_n}$,
$\lim_{n\to\Cal F_d}\alpha_n=\lim_{n\to\Cal F_d}\alpha_{m+n}$ if either is
defined.




\exercises{\leader{491X}{Basic exercises (a)}
%\spheader 491Xa
(i) Show that if $I$, $J\in\Cal D=\dom d$ as defined in 491A, then
$I\cup J\in\Cal D$ iff $I\cap J\in\Cal D$ iff $I\setminus J\in\Cal D$
iff $I\symmdiff J\in\Cal D$.
(ii) Show that if $\Cal E\subseteq\Cal D$ is an algebra of sets, then
$d\restr\Cal E$ is additive.
(iii) Find $I$, $J\in\Cal D$ such that $I\cap J\notin\Cal D$.
%491A

\sqheader 491Xb Suppose that $I\subseteq\Bbb N$ and that
$f:\Bbb N\to\Bbb N$ is strictly increasing.   Show that
$d^*(f[I])\le d^*(I)d^*(f[\Bbb N])$, with equality if either $I$ or
$f[\Bbb N]$ has asymptotic density.
%491A

\spheader 491Xc Show that if $A\subseteq\Bbb N$ and
$0\le\alpha\le d^*(A)$ there is a $B\subseteq A$ such that
$d^*(B)=\alpha$ and $d^*(A\setminus B)=d^*(A)-\alpha$.
%491A 491Xb

\spheader 491Xd(i) Let $I\subseteq\Bbb N$ be such that
$d(J)=d^*(J\cap I)+d^*(J\setminus I)$ for every $J\subseteq\Bbb N$
such
that $d(J)$ is defined.
Show that either $I\in\Cal Z$ or $\Bbb N\setminus I\in\Cal Z$.  (ii)
Show that for every $\epsilon>0$ there is an $I\subseteq\Bbb N$ such
that $d^*(I)=\epsilon$ but $d(J)=1$ whenever $J\supseteq I$ and $d(J)$
is defined.
%491A   take $I=\bigcup_{n\in\Bbb N}\coint{(1-\epsilon)k_n,k_n}$
%491Xb

\spheader 491Xe Let $(X,\Sigma,\mu)$ be a probability space and
$\sequencen{E_n}$ a sequence in $\Sigma$.   For $x\in X$, set
$I_x=\{n:n\in\Bbb N$, $x\in E_n\}$.   Show that
$\int d^*(I_x)\mu(dx)\ge\liminf_{n\to\infty}\mu E_n$.
%491A

\spheader 491Xf Let $(X,\frak T,\Sigma,\mu)$ be a compact Radon
probability space.   Take any point $\infty$ not belonging to $X$, and
give $X\cup\{\infty\}$ the topology generated by
$\{G\cup\{\infty\}:G\in\frak T\}$.   Show that $X\cup\{\infty\}$ is
compact and that the image measure $\mu_{\infty}$
of $\mu$ under the identity map from $X$ to $X\cup\{\infty\}$ is a
quasi-Radon measure, inner regular with respect to the compact sets.
Show that if we set $x_n=\infty$ for every $n$, then $\sequencen{x_n}$
is equidistributed for $\mu_{\infty}$.
%491B

\sqheader 491Xg(i) Show that a sequence $\sequencen{t_n}$ in $[0,1]$ is
equidistributed with respect to Lebesgue measure iff
$\lim_{n\to\infty}\bover1{n+1}\penalty-100\#(\{i:i\le n$,
$t_i\le\beta\})=\beta$ for every $\beta\in[0,1]$.
(ii) Show that if $\alpha\in\Bbb R$ is irrational then the sequence
$\sequencen{\fraction{n\alpha}}$ of fractional parts of
multiples of $\alpha$ is equidistributed in $[0,1]$ with respect to
Lebesgue measure.   \Hint{281Yi.}
(iii) Show that a function $f:[0,1]\to\Bbb R$ is Riemann
integrable iff $\lim_{n\to\infty}\bover1{n+1}\sum_{i=0}^nf(x_i)$ is
defined in $\Bbb R$ for every sequence $\sequencen{x_n}$ in $[0,1]$
which is equidistributed for Lebesgue measure.
%491C

\spheader 491Xh Show that the usual measure on the split interval
(419L) has an equidistributed sequence.

\sqheader 491Xi Show that if $X$ is a Hausdorff space and
$f:\Bbb N\to X$ is injective, then there is an open set
$G\subseteq X$ such that $f^{-1}[G]$ does not have asymptotic density.
\Hint{show that if $A\subseteq\Bbb N$ and $d^*(A)>\gamma$, there is an
$n\in\Bbb N$ such that for every $m\ge n$ there is an open set
$G\supseteq f[m\setminus n]$ such that
$d^*(A\setminus f^{-1}[G])>\gamma$.   You may prefer to tackle the case of
metrizable $X$ first.}
%491C

\spheader 491Xj Let $X$ be a metrizable space, and $\mu$ a quasi-Radon
probability measure on $X$.   (i) Show that there is an
equidistributed
sequence for $\mu$.   (ii) Show that if the
support of $\mu$ is not compact, and $\sequencen{x_n}$ is an
equidistributed sequence for $\mu$, then there is a continuous
integrable function $f:X\to\Bbb R$ such that
$\lim_{n\to\infty}\bover1{n+1}\sum_{i=0}^nf(x_i)=\infty$.
%491E

\spheader 491Xk Let $\phi:\frak c\to\Cal P\Bbb N$ be an injective
function.   For each $n\in\Bbb N$ let $\lambda_n$ be the uniform
probability measure on $\Cal P(\Cal Pn)$, giving measure $2^{-2^n}$
to each singleton.
Define $\psi_n:\Cal P(\Cal Pn)\to\{0,1\}^{\frak c}$ by setting
$\psi_n(\Cal I)(\xi)=1$ if
$\phi(\xi)\cap n\in\Cal I$, $0$ otherwise, and let $\nu_n$ be the
image measure $\lambda_n\psi_n^{-1}$.   Show that $\nu_nE$ is the usual
measure of $E$ whenever
$E\subseteq\{0,1\}^{\frak c}$ is determined by coordinates in a finite
set on which the map $\xi\mapsto\phi(\xi)\cap n$ is injective.   Use
this with 491D to prove 491G.
%491G

\sqheader 491Xl(i) Let $Z$ be the Stone space of the measure algebra of
Lebesgue measure on $[0,1]$, with its usual measure.
Show that there is no equidistributed sequence in $Z$.   \Hint{meager
sets in $Z$ have negligible closures.}
(ii) Show that Dieudonn\'e's measure on $\omega_1$ (411Q) has no
equidistributed sequence.   (iii) Show that if $\#(I)>\frak c$ then the
usual measure on $\{0,1\}^I$ has no equidistributed sequence.   \Hint{if
$\sequencen{x_n}$ is any
sequence in $\{0,1\}^I$, there is an infinite $J\subseteq I$ such that
$\sequencen{x_n(\eta)}=\sequencen{x_n(\xi)}$ for all $\eta$, $\xi\in J$.}
(iv) Show that if $X$ is a topological group with a Haar probability
measure $\mu$, and $X$ is not separable, then $\mu$ has no
equidistributed sequence.   \Hint{use
443D to show that every separable subset is negligible.}
%491G

\spheader 491Xm Let $X$ be a compact Hausdorff abelian topological
group and $\mu$ its Haar probability measure.   Show that a sequence
$\sequencen{x_n}$ in $X$ is equidistributed for $\mu$ iff
$\lim_{n\to\infty}\bover1{n+1}\sum_{i=0}^n\chi(x_i)=0$ for every
non-trivial character $\chi:X\to S^1$.   \Hint{281G.}
%491H

\spheader 491Xn(i) Let $\sequencen{a_n}$ be a non-decreasing sequence
in $\frak Z=\Cal P\Bbb N/\Cal Z$.   Show that there is an $a\in\frak Z$
such that $a_n\Bsubseteq a$ for every $n\in\Bbb N$ and
$\bar d^*(a)=\sup_{n\in\Bbb N}\bar d^*(a_n)$.
(ii) Show that $\frak Z$ is not Dedekind $\sigma$-complete.
\Hint{393Bc\footnote{Formerly 3{}92Hc.}.}
%491I n99527

\spheader 491Xo Let $\frak Z$, $\bar d^*$ and $D$ be as in 491K.   Show
that if $a\in D\setminus\{0\}$ and $\frak Z_a$ is the principal ideal of
$\frak Z$ generated by $a$, then $(\frak Z_a,\bar d^*\restrp\frak Z_a)$
is isomorphic, up to a scalar multiple of the submeasure, to
$(\frak Z,\bar d^*)$.
%491K 491Xb

\spheader 491Xp Let $(X,\Sigma,\mu)$ be a semi-finite measure space
and
$\frak T$ a topology on $X$.   Show that $\mu$ is effectively regular
iff whenever $E\in\Sigma$,
$\mu E<\infty$ and $\epsilon>0$ there are a measurable open set $G$
and
a measurable closed set $F\supseteq G$ such that $\mu(F\setminus
E)+\mu(E\setminus G)\le\epsilon$.
%491L

\spheader 491Xq Let $X$ be a normal topological space and $\mu$ a
topological measure on $X$ which is inner regular with respect to the
closed sets and effectively locally finite.
Show that $\mu$ is effectively regular.
%491M

\spheader 491Xr Let $X$ be a topological space and $\mu$ an
effectively
regular measure on $X$.   (i) Show that the completion and c.l.d.\
version of $\mu$ are also effectively regular.   (ii) Show that if
$Y\subseteq X$ then the subspace measure is again effectively regular.
(iii) Show that any totally
finite indefinite-integral measure over $\mu$ is effectively regular.
%491M

\spheader 491Xs(i) Let $X_1$, $X_2$ be topological spaces with
effectively regular measures $\mu_1$, $\mu_2$.   Show that the c.l.d.\
product measure on $X_1\times X_2$ is
effectively regular with respect to the product topology.   \Hint{412R.}
(ii) Let $\familyiI{X_i}$ be a family of topological spaces and $\mu_i$
an effectively regular probability
measure on $X_i$ for each $i$.   Show that the product probability
measure on $\prod_{i\in I}X_i$ is effectively regular.
%491M

\spheader 491Xt Give $[0,1]$ the topology $\frak T$ generated by the
usual topology and $\{[0,1]\setminus A:A\subseteq\Bbb Q\}$.   Let
$\mu_L$ be Lebesgue measure on $[0,1]$,
and $\Sigma$ its domain.   For $E\in\Sigma$ set
$\mu E=\mu_LE+\#(E\cap\Bbb Q)$ if $E\cap\Bbb Q$ is finite, $\infty$
otherwise.   Show that $\mu$ is a $\sigma$-finite
quasi-Radon measure with respect to the topology $\frak T$, but is not
effectively regular.
%491M

\spheader 491Xu\dvAformerly{4{}91Xw, previously 4{}91Xy}
Let $\frak A$ be a countable Boolean algebra and $\nu$ a
finitely additive functional on $\frak A$ such that $\nu 1=1$.   Show
that there is a Boolean
homomorphism $\pi:\frak A\to\Cal P\Bbb N$ such that $d(\pi a)$ is
defined and equal to $\nu a$ for every $a\in\frak A$ (i) using 491Xc
(ii) using 392H\formerly{3{}93B}, 491P and 341Xc.
%491P

\spheader 491Xv Let $X$ be a dyadic space.   (i) Show that there is a
Radon probability measure on $X$ with support $X$.
(ii) Show that the following are equiveridical:  ($\alpha$)
$w(X)\le\frak c$;  ($\beta$) every Radon probability measure on $X$ has an
equidistributed sequence;  ($\gamma$) $X$ is separable.
\Hint{4A2Dd, 418L.}
%491Q

\spheader 491Xw(i) Give an example of a Radon probability space
$(X,\mu)$ with a closed conegligible set $F\subseteq X$ such that
$\mu$ has an equidistributed sequence but the subspace measure
$\mu_F$ does not.   \Hint{the Stone space of the measure algebra of
Lebesgue measure embeds into $\{0,1\}^{\frak c}$.}   (ii) Show that
if $X$ is a topological space with an effectively regular topological
probability measure $\mu$ which has an equidistributed sequence, and
$G\subseteq X$ is a non-negligible open set, then the normalized subspace
measure
$\Bover1{\mu G}\mu_G$ has an equidistributed sequence.   \Hint{491R.}
%491R  query:  example for (ii) if \mu not eff reg

%suppose we have  \sequencen{x_n}  such that
%d_*\{i:x_i\in H\}\ge\gamma\mu H  for every open  H .   Must there be
%an equidistributed sequence?

\spheader 491Xx\dvAnew{2011} Let $X$ be a topological space.   A sequence
$\sequencen{x_n}$ in $X$
is called {\bf statistically convergent} to $x\in X$ if
$d(\{n:x_n\in G\})=1$ for every open set $G$ containing $x$.
(i) Show that if
$X$ is first-countable then $\sequencen{x_n}$ is statistically convergent
to $x$ iff there is a set $I\subseteq\Bbb N$ such that $d(I)=1$ and
$\family{n}{I}{x_n}$ converges to $x$ in the sense that
$\{n:n\in I$, $x_n\notin G\}$ is finite for every open set $G$ containing
$x$.  (ii) Show that a bounded sequence $\sequencen{\alpha_n}$ in $\Bbb R$
is statistically convergent to $\alpha$ iff
$\lim_{n\to\infty}\Bover1n\sum_{i=0}^{n-1}|\alpha_i-\alpha|=0$.

\leader{491Y}{Further exercises (a)}
%\spheader 491Ya
Show that every subset $A$ of $\Bbb N$ is expressible in
the form $I_A\symmdiff J_A$ where $d(I_A)=d(J_A)=\bover12$ (i) by a
direct construction, with
$A\mapsto I_A$ a continuous function (ii) using 443D.
%491A

\spheader 491Yb Let $\frak A$ be a Boolean algebra, and
$\nu:\frak A\to[0,\infty]$ a submeasure.   Show that $\nu$ is uniformly
exhaustive iff whenever $\sequencen{a_n}$ is a sequence in $\frak A$ such
that $\inf_{n\in\Bbb N}\nu a_n>0$, there is a set $I\subseteq\Bbb N$ such
that $d^*(I)>0$ and $\inf_{i\in I\cap n}a_i\ne 0$ for every $n\in\Bbb N$.
%491Xe 491A

\spheader 491Yc
Find a topological space $X$ with a $\tau$-additive probability
measure
$\mu$ on $X$, a sequence $\sequencen{x_n}$ in $X$ and a base $\Cal G$
for the topology of
$X$, consisting of measurable sets and closed under finite
intersections, such that
$\mu G\le\liminf_{n\to\infty}\bover1{n+1}\#(\{i:i\le n$, $x_i\in G\})$
for every $G\in\Cal G$ but $\sequencen{x_n}$ is not equidistributed.
\Hint{take $\#(X)=4$.}
%491C

\spheader 491Yd Let $X$ be a compact Hausdorff space on which every
Radon probability measure has an equidistributed sequence.   Show that
the cylindrical $\sigma$-algebra of
$C(X)$ is the $\sigma$-algebra generated by sets of the form
$\{f:f\in C(X)$, $f(x)>\alpha\}$ where $x\in X$ and $\alpha\in\Bbb R$.
\Hint{436J, 491Cb.}
%491C

\spheader 491Ye Give $\omega_1+1$ and $[0,1]$ their usual compact
Hausdorff topologies.   Let $\sequence{i}{t_i}$ be a sequence in
$[0,1]$
which is equidistributed for
Lebesgue measure $\mu_L$, and set $Q=\{t_i:i\in\Bbb N\}$,
$X=(\omega_1\times([0,1]\Bsetminus Q))\cup(\{\omega_1\}\times Q)$,
with the subspace topology inherited from $(\omega_1+1)\times[0,1]$.
(i) Set $F=\{\omega_1\}\times Q$.   Show that
$F$ is a closed Baire set in the completely regular Hausdorff space
$X$.
(ii) Show that if $f\in C_b(X)$ then there are a $g_f\in C([0,1])$
and a
$\zeta<\omega_1$ such that
such that $f(\xi,t)=g_f(t)$ whenever $(\xi,t)\in X$ and
$\zeta\le\xi\le\omega_1$.   (iii) Show that there is a Baire measure
$\mu$ on $X$ such that $\int fd\mu=\int g_fd\mu_L$ for every
$f\in C_b(X)$.   (iv) Show that $\mu F=0$.   (v) Show that
$\int fd\mu=\lim_{n\to\infty}\bover1{n+1}\sum_{i=0}^nf(\omega_1,t_i)$
for every $f\in C_b(X)$, but that $\sequence{i}{(\omega_1,t_i)}$ is not
equidistributed with respect to $\mu$.
%491C

\spheader 491Yf Let $(X,\frak T,\Sigma,\mu)$ be a topological measure
space.   Let $\Cal E$ be the Jordan algebra of $X$ (411Yc).
(i) Suppose that $\mu$ is a complete probability measure on $X$ and
$\sequencen{x_n}$ an equidistributed sequence in $X$.
Show that the asymptotic density $d(\{n:x_n\in E\})$ is defined and
equal to $\mu E$ for every $E\in\Cal E$.  (ii) Suppose that $\mu$ is a
probability measure on $X$
and that $\sequencen{x_n}$ is a sequence in $X$ such that
$d(\{n:x_n\in E\})$ is defined and equal to $\mu E$ for every
$E\in\Cal E$.   Show that
$\lim_{n\to\infty}\bover1{n+1}\sum_{i=0}^nf(x_i)=\int fd\mu$ for every
$f\in C_b(X)$.
%491C

\spheader 491Yg Show that a sequence $\sequencen{x_n}$ in $[0,1]$
is equidistributed for Lebesgue measure iff there is some $r_0\in\Bbb N$
such that $\lim_{n\to\infty}\bover1{n+1}\sum_{i=0}^nx_i^r=\bover1{r+1}$
for every $r\ge r_0$.
%491C

\spheader 491Yh Let $Z$, $\mu$, $X=Z\times\{0,1\}$ and $\nu$ be as
described in 439K, so that $\mu$ is a Radon probability
measure on the compact metrizable space $Z$, $X$ has a compact
Hausdorff topology finer than the product topology,
and $\nu$ is a measure on $Z$ extending $\mu$.   (i) Show that if
$f\in C(X)$, then
$\{t:t\in Z$, $f(t,0)\ne f(t,1)\}$ is countable.   \Hint{4A2F(h-vii).}
(ii) Show that $\int f(t,0)\mu(dt)=\int f(t,1)\nu(dt)$ for every
$f\in C(X)$.
(iii) Let $\lambda$ be the measure $\nu g^{-1}$ on $Z$, where
$g(t,0)=g(t,1)=(t,1)$ for $t\in Z$.   Show that there is a
sequence $\sequencen{x_n}$ in $Z\times\{0\}$ such that
$\int fd\lambda=\lim_{n\to\infty}\bover1{n+1}\sum_{i=0}^nf(x_i)$ for
every $f\in C(X)$, but that $\sequencen{x_n}$ is not
$\lambda$-equidistributed.
%491C

\spheader 491Yi(i)\dvAformerly{4{}91Yg} Show that a Radon
probability measure on an extremally disconnected
compact Hausdorff space has an equidistributed sequence iff it is
point-supported.   \Hint{see the hint for 326Yo.}
(ii) Show that there is a separable compact Hausdorff space with a Radon
probability measure which has no equidistributed sequence.
%491G
%(ii) Stone space of regular open algebra of \Bbb R

\spheader 491Yj Show that there is a countable dense set
$D\subseteq[0,1]^{\frakc}$ such that no sequence in $D$ is
equidistributed for the usual measure on $[0,1]^{\frakc}$.
%491G 49bits

\spheader 491Yk Let $\frak Z=\Cal P\Bbb N/\Cal Z$ and
$\bar d^*:\frak Z\to[0,1]$ be as in 491I.   Show that $\bar d^*$ is
{\bf order-continuous on the left} in the sense that whenever
$A\subseteq\frak Z$ is non-empty and upwards-directed and has a supremum
$c\in\frak Z$, then $\bar d^*(c)=\sup_{a\in A}\bar d^*(a)$.
%491K n99527

\spheader 491Yl(i) Show that $\frak Z$ is \wsid.
(ii) Show that $\frak Z\cong\frak Z^{\Bbb N}$.
(iii) Show that $\frak Z$ has the $\sigma$-interpolation property, but is
not Dedekind $\sigma$-complete.
(iv) Show that $\frak Z$ has many involutions in the sense of 382O.
%491K 491Yk n99527

\spheader 491Ym Let $(X,\rho)$ be a separable metric space and $\mu$ a
Borel probability measure on $X$.   (i) Show that there is an
equidistributed sequence in $X$.
(ii) Show that if $\sequencen{x_n}$ is an equidistributed sequence in
$X$, and $\sequencen{y_n}$ is a sequence in $X$ such that
$\lim_{n\to\infty}\rho(x_n,y_n)=0$, then
$\sequencen{y_n}$ is equidistributed.
(iii) Show that if $f:X\to\Bbb R$ is a bounded function, then
$\lim_{n\to\infty}\bover1{n+1}\sum_{i=0}^nf(x_i)-f(y_i)=0$ for all
equidistributed
sequences $\sequencen{x_n}$, $\sequencen{y_n}$ in $X$ iff
$\{x:f$ is continuous at $x\}$ is conegligible, and in this case
$\lim_{n\to\infty}\bover1{n+1}\sum_{i=0}^nf(x_i)=\int fd\mu$
for every equidistributed sequence $\sequencen{x_n}$ in $X$.
\Hint{$\{x:f$ is continuous at $x\}=\bigcap_{m\in\Bbb N}G_m$, where
$G_m=\bigcup\{G:G\subseteq X$ is open,
$\sup_{x,y\in G}|f(x)-f(y)|\le 2^{-m}\}$.}   Compare 491Xg.
%491M

\spheader 491Yn Let $(X,\frak T)$ be a topological space, $\mu$ a
probability measure on $X$, and $\phi:X\to X$ an \imp\ function.
(i) Suppose that $\frak T$ has a countable network
consisting of measurable sets, and that $\phi$ is ergodic.   Show that
$\sequencen{\phi^n(x)}$ is equidistributed for almost every $x\in X$.
\Hint{372Qb.}   (ii)
Suppose that $\mu$ is {\it either} inner regular with respect to the
closed sets {\it or}
effectively regular, and that $\{x:\sequencen{\phi^n(x)}$ is
equidistributed$\}$ is not negligible.   Show that $\phi$ is ergodic.
%491M 491Xe

\spheader 491Yo Let $\ofamily{\xi}{\frak c}{X_{\xi}}$ be a family of
topological spaces with countable networks consisting of Borel sets, and
$\mu$ a $\tau$-additive topological probability measure on
$X=\prod_{\xi<\frak c}X_{\xi}$.   Show that $\mu$ has an equidistributed
sequence.
%491Q
%basic facts needed:  (i) if $\frak C\subseteq\frak Z$ a countable
%subalgebra, it has a lifting to a subalgebra of $\Cal P\Bbb N$ (ii)
%if $\Cal E$ a countable
% algebra of subsets of $X$, $\pi:\Cal E\to\Cal P\Bbb N$ an injective
%Boolean homomorphism, we can find $\sequencen{x_n}$ in $X$ such that
%$\{n:x_n\in\pi E\}\symmdiff E$
% is finite for every $E\in\Cal E$

\spheader 491Yp(i)
Show that there is a family $\ofamily{\xi}{\frakc}{a_{\xi}}$ in
$\frak Z$ such that
$\inf_{\xi\in I}a_{\xi}=0$ and $\sup_{\xi\in I}a_{\xi}=1$ for every
infinite $I\subseteq\frak c$.   (ii) Show that if
$B\subseteq\frak Z\setminus\{0\}$ has cardinal less than $\frak c$
then
there is an $a\in\frak Z$ such that $b\Bcap a$ and $b\Bsetminus a$ are
non-zero for every $b\in B$.
%491P

\spheader 491Yq Let $(X,\frak T,\Sigma,\mu)$ be a $\tau$-additive
topological probability space.   A sequence $\sequencen{x_n}$ in $X$ is
{\bf completely equidistributed} if, for every $r\ge 1$, the sequence
$\sequencen{\ofamily{i}{r}{x_{n+i}}}$ is equidistributed for some
(therefore any) $\tau$-additive extension of the c.l.d.\ product measure
$\mu^r$ on $X^r$.   (i) Show that if there is an equidistributed sequence
in $X$, then there is a completely equidistributed sequence in $X$.   (ii)
Show that if $\frak T$ is second-countable, then $\mu^{\Bbb N}$-almost
every sequence in $X$ is completely equidistributed.   (iii) Show that
if $X$ has two disjoint open sets of non-zero measure, then no sequence
which is well-distributed in the sense of 281Ym can
be completely equidistributed.

\spheader 491Yr Suppose,
in 491O, that $\mu$ is a topological measure.   Show
that $T_{\pi}f^{\ssbullet}\le RSf$ for every bounded lower semi-continuous
$f:X\to\Bbb R$.
%491O

\spheader 491Ys\dvAnew{2009.} (M.Elekes)
Let $(X,\Sigma,\mu)$ be a $\sigma$-finite measure
space and $\sequencen{E_n}$ a sequence in $\Sigma$ such that
$\bigcap_{n\in\Bbb N}\bigcup_{m\ge n}E_m$ is conegligible.
Show that there is an $I\in\Cal Z$ such that
$\bigcap_{n\in\Bbb N}\bigcup_{m\in I\setminus n}E_m$ is conegligible.
%491A out of order query
}%end of exercises

\leader{491Z}{Problem} It is known that for almost every $x>1$ the
sequence $\sequencen{\fraction{x^n}}$ of fractional parts of powers of $x$ is
equidistributed for Lebesgue measure on
$[0,1]$\cmmnt{ ({\smc Kuipers \& Niederreiter 74}, p.\ 35)}.   But is
$\sequencen{\fraction{(\bover32)^n}}$ equidistributed?

% question:  must a Radon probability measure on a hereditarily
% separable space have an equidistributed sequence?

% possibly useful fact:  if $\mu$ is inner regular for $\{E:\mu_E$ has
% equidistr seq for normalized subspace measure$\}$, then $\mu$ has
% equidistributed sequence

\endnotes{
\Notesheader{491} The notations $d^*$, $d$ (491A) are standard, and
usefully suggestive.   But coming from measure theory we have to
remember that $d^*$, although a submeasure,
is not an outer measure, the domain of $d$ is not an algebra of sets
(491Xa), and $d$ and $d^*$ are related by only one of the formulae we
expect to connect a
measure with an outer measure (491Ac, 491Xd).   The limits
$\lim_{n\to\infty}\bover1n\sum_{i=0}^nf(x_i)$ are the {\bf Ces\`aro means}
of the sequences $\sequence{n}{f(x_n)}$.
The delicacy of the arguments here arises from the fact that the family
of (bounded) sequences with Ces\`aro means, although a norm-closed linear
subspace of $\ell^{\infty}$, is neither a sublattice nor a subalgebra.
When we turn to the quotient algebra $\frak Z=\Cal P\Bbb N/\Cal Z$, we
find ourselves with a natural submeasure to which we can apply ideas
from \S392 to good effect (491I;  see also 491Yk and 491Yl).
What is striking is that
equidistributed sequences induce regular embeddings of measure algebras
in $\frak Z$ which can be thought of as measure-preserving (491N).

Most authors have been content to define an `equidistributed sequence'
to be one such that the integrals of bounded continuous functions are
correctly specified (491Cf, 491Cg);
that is, that the point-supported measures
$\bover1{n+1}\sum_{i=0}^n\delta_{x_i}$ converge to $\mu$ in the vague
topology on an appropriate class of measures (437J).
I am going outside this territory in order to cover some ideas I find
interesting.   491Yh shows that it makes a difference;  there are Borel
measures on compact Hausdorff spaces
which have sequences which give the correct Ces\`aro means for continuous
functions, but lie within negligible closed sets;  and the same can
happen with Baire measures (491Ye).
It seems to be difficult, in general, to determine whether
a topological probability space -- even a compact Radon probability
space -- has an equidistributed sequence.   In the proofs of
491D-491G %491D 491E 491F 491H 491G
I have tried to collect the principal techniques for showing that spaces
do have equidistributed sequences.   In the other direction, it is
obviously impossible for a space to
have an equidistributed sequence if every separable subspace is
negligible (491Xl).   For an example of a separable compact Hausdorff
space with a Radon measure which does not have
an equidistributed sequence, we seem to have to go deeper (491Yi).

491Z is a famous problem.   It is not clear that it is a problem in
measure theory, and there is no reason to suppose that any of the ideas
of this treatise beyond 491Xg(i) are
relevant.   I mention it because I think everyone should know that it
is there.
}%end of notes

\discrpage

