\frfilename{mt553.tex}
\versiondate{3.5.14}
\copyrightdate{2005}

\def\chaptername{Possible worlds}
\def\sectionname{Random reals II}

\Loadfourteens

\def\BbbPk{\Bbb P_{\kappa}}
\long\def\doubleinset#1{\inset{\inset{\parindent=-20pt #1}}}
\def\Fn{\mathop{\text{Fn}}\nolimits}
\def\rank{\mathop{\text{rank}}}
\def\VVdash{\mskip5mu\vrule height 7.5pt depth 2.5pt width 0.5pt
  \mskip2.5mu\vrule height 7.5pt depth 2.5pt width 0.5pt
  \vrule height 2.75pt depth -2.25pt width 4pt\mskip2mu}
\def\VVdPk{\VVdash_{\Bbb P_{\kappa}}}
\def\VVdP{\VVdash_{\Bbb P}}

\newsection{553}

In this section I collect some further properties of random real models
which seem less directly connected with the main topics of this book than
those treated in \S552.   The first concerns strong measure zero or
`Rothberger's property'\cmmnt{ (534E)} and
gives a bound for the sizes of sets with this property.
The second relates
perfect sets in $V^{\Bbb P_{\kappa}}$ to negligible F$_{\sigma}$ sets in
the original universe;  it shows that a random real model can have
properties relevant to a question in \S531 (553F).
Following these, I discuss properties of ultrafilters and partially
ordered sets which are not obviously connected with measure theory,
but where the arguments needed to establish the truth of sentences in
$V^{\Bbb P_{\kappa}}$ involve interesting properties of measure algebras
(553G-553M).   I conclude with notes on medial limits (553N) and
universally measurable sets (553O).

%553B 553C Rothberger's property
%553E 553F perfect sets meet ground-model negl sets
%553G 553H rapid p-point ufilters
%553I 553J ccc posets
%553K 553L 553M special Aronszajn trees
%553N medial limits

\leader{553A}{Notation}\cmmnt{ I repeat some formulae from 552A.}
For any set $I$, $\nu_I$ will be
the usual measure on $\{0,1\}^I$, $\Tau_I$ its domain,
$\Cal N(\nu_I)$ its null ideal and
$(\frak B_I,\bar\nu_I)$
its measure algebra.   $\CalBa_I$ will be the
Baire $\sigma$-algebra of $\{0,1\}^I$.   For a cardinal $\kappa$,
$\BbbPk$ will be the forcing notion
$\frak B_{\kappa}^+$, active downwards.

\leader{553B}{Lemma} If $A\subseteq\{0,1\}^{\Bbb N}$ has Rothberger's
property,
then for any $f:\Bbb N\to\Bbb N$ there is a sequence $\sequencen{z_n}$ such
that $z_n\in\{0,1\}^{f(n)}$ for each $n$ and
$A\subseteq\bigcap_{n\in\Bbb N}
\bigcup_{m\ge n}\{x:z_m\subseteq x\in\{0,1\}^{\Bbb N}\}$.

\proof{ By 534Fd, $A$ has strong measure zero with respect to the metric
$\rho$ on $\{0,1\}^{\Bbb N}$ defined by saying that

\Centerline{$\rho(x,y)=2^{-n}$ if $x\restr n=y\restr n$
and $x(n)\ne y(n)$.}

\noindent So
for each $n\in\Bbb N$ we can find a sequence $\sequence{i}{A_{ni}}$ of
subsets of $\{0,1\}^{\Bbb N}$ such that
$A\subseteq\bigcup_{i\in\Bbb N}A_{ni}$ and
$\diam A_{ni}\le 2^{-f(2^n(2i+1))}$ for each $i$.   Take $x_{ni}\in A_{ni}$
if $A_{ni}$ is non-empty, any member of $\{0,1\}^{\Bbb N}$ otherwise, and
set $z_m=x_{ni}\restr f(2^n(2i+1))$ if $m=2^n(2i+1)$;  take $z_0$ to be any
member of $\{0,1\}^{f(0)}$.   Then $z_m\in\{0,1\}^{f(m)}$ for every $m$,
and if $n\in\Bbb N$ then

$$\eqalign{A
&\subseteq\bigcup_{i\in\Bbb N}A_{ni}
\subseteq\bigcup_{i\in\Bbb N}\{x:\rho(x,x_{ni})\le 2^{-f(2^n(2i+1))}\}\cr
&=\bigcup_{i\in\Bbb N}\{x:x\supseteq z_{2^n(2i+1)}\}
\subseteq\bigcup_{m\ge n}\{x:x\supseteq z_m\},\cr}$$

\noindent as required.
}%end of proof of 553B

\leader{553C}{Proposition} Let $\kappa$ be any cardinal.
Then

\Centerline{$\VVdPk$ every subset of
$\{0,1\}^{\Bbb N}$ with Rothberger's property has size at most
$\check{\frak c}$.}

\proof{ (See {\smc Bartoszy\'nski \& Judah 95}, 8.2.11.)

\medskip

{\bf (a)}
Let $\dot A$ be a $\BbbPk$-name for a subset of $\{0,1\}^{\Bbb N}$
with Rothberger's property.   Take any $f\in\BbbN^{\Bbb N}$.
Applying 553B in the forcing language, we must have

\doubleinset{$\VVdPk$ there is a sequence $\sequencen{z_n}$
such that $z_n\in\{0,1\}^{\check f(n)}$ for every $n\in\Bbb N$ and
$\dot A\subseteq\bigcap_{n\in\Bbb N}\bigcup_{m\ge n}\{x:x\supseteq z_n\}$.}

\noindent Let $\sequencen{\dot z_f(n)}$ be a sequence of $\BbbPk$-names
such that

\Centerline{$\VVdPk\dot z_f(n)\in\{0,1\}^{\check f(n)}$ for every
$n\in\Bbb N$,}

\Centerline{$\VVdPk\dot A\subseteq\bigcap_{n\in\Bbb N}\bigcup_{m\ge n}
\{x:x\supseteq\dot z_f(n)\}$.}

\medskip

{\bf (b)} Let $\ofamily{\xi}{\kappa}{e_{\xi}}$ be the standard generating
family in $\frak B_{\kappa}$ (525A).
Let $J\subseteq\kappa$ be a set of size at
most $\frak c$ such that $\Bvalue{\dot z_f(n)=\check z}$ belongs to the
closed subalgebra $\frak C_J$ generated by $\{e_{\xi}:\xi\in J\}$ for every
$f\in\BbbN^{\Bbb N}$ and $z\in\bigcup_{n\in\Bbb N}\{0,1\}^{f(n)}$.
Let $P_J:L^{\infty}(\frak B_{\kappa})\to L^{\infty}(\frak C_J)$ be the
conditional expectation operator (365Q\formerly{3{}65R}).

Observe that $\frak C_J$ and $\frak C_J^{\Bbb N}$ have
cardinal at most $\frak c^{\omega}=\frak c$.   So we have a family
$\ofamily{\eta}{\frak c}{\dot y_{\eta}}$ of $\BbbPk$-names for members of
$\{0,1\}^{\Bbb N}$ such that
whenever $\sequencen{d_n}$ is a sequence in $\frak C_J$ there is an
$\eta<\frak c$ such that $\Bvalue{\dot y_{\eta}(\check n)=1}=d_n$ for every
$n\in\Bbb N$.

\medskip

{\bf (c)}
Let $\dot x$ be a $\BbbPk$-name for a member of $\{0,1\}^{\Bbb N}$, and
suppose that $a=\Bvalue{\dot x\in\dot A}$ is non-zero.
For $z\in\bigcup_{n\in\Bbb N}\{0,1\}^n$ set
$b_z=\Bvalue{\dot x\supseteq\check z}$.   For $m$, $n\in\Bbb N$, set

\Centerline{$c_{nm}=\sup_{z\in\{0,1\}^m}
   \Bvalue{P_J(\chi(a\Bcap b_z))>2^{-n}}\in\frak C_J$.}

\noindent Note that if $k\le m$ and $z\in\{0,1\}^m$ then
$b_z\Bsubseteq b_{z\restr k}$, so

\Centerline{$\Bvalue{P_J(\chi(a\Bcap b_z))>2^{-n}}
\Bsubseteq\Bvalue{P_J(\chi(a\Bcap b_{z\restr k}))>2^{-n}}$,}

\noindent and $c_{nm}\Bsubseteq c_{nk}$;  set $c_n=\inf_{m\in\Bbb N}c_{nm}$.
\Quer\ Suppose, if possible, that $c_n=0$ for every $n$.
Let $f\in\BbbN^{\Bbb N}$ be such
that $\sum_{n=0}^{\infty}\bar\nu_{\kappa}c_{n,f(n)}<\bar\nu_{\kappa}a$, and
set

\Centerline{$d=\sup_{n\in\Bbb N}(\sup_{z\in\{0,1\}^{f(n)}}
   \Bvalue{P_J(\chi(a\Bcap b_z))>2^{-n}})
\in\frak C_J$.}

\noindent Then $\bar\nu_{\kappa}d<\bar\nu_{\kappa}a$, so
$a\Bsetminus d\ne 0$;  while

$$\eqalignno{
\bar\nu_{\kappa}((a\Bsetminus d)\Bcap\Bvalue{\dot x\supseteq\dot z_f(n)})
&=\sum_{z\in\{0,1\}^{f(n)}}
  \bar\nu_{\kappa}(a\Bcap b_z
    \Bcap\Bvalue{\dot z_f(n)=\check z}\Bsetminus d)\cr
\displaycause{because $\VVdPk\dot z_f(n)\in\{0,1\}^{f(n)\var2spcheck}$}
&=\sum_{z\in\{0,1\}^{f(n)}}
  \int_{\Bvalue{\dot z_f(n)=\check z}\Bsetminus d}P_J(\chi(a\Bcap b_z))\cr
&\le 2^{-n}\sum_{z\in\{0,1\}^{f(n)}}
  \bar\nu_{\kappa}\Bvalue{\dot z_f(n)=\check z}\cr
\displaycause{because $d$ includes
$\Bvalue{P_J(\chi(a\Bcap b_z))>2^{-n}}$ for every $z\in\{0,1\}^{f(n)}$}
&=2^{-n}\cr}$$

\noindent for every $n$.   Consequently

\Centerline{$(a\Bsetminus d)\Bcap\inf_{n\in\Bbb N}\sup_{m\ge n}
  \Bvalue{\dot x\supseteq\dot z_f(m)}=0$,}

\noindent that is,

\Centerline{$a\Bsetminus d\VVdPk
  \dot x\supseteq\dot z_f(n)$ for only finitely many $n$.}

\noindent But this implies that

\Centerline{$a\Bsetminus d\VVdPk
  \dot x\notin\dot A$,}

\noindent contrary to hypothesis.\ \Bang

\medskip

{\bf (d)} Continuing from (c), we find that there are an
$a'\in\frak B_{\kappa}^+$, stronger than $a$, and an $\eta<\frak c$ such that
$a'\VVdPk\dot x=\dot y_{\eta}$.

\Prf\ Let $n\in\Bbb N$ be such that $c_n$, as defined in (c),
is non-zero.   For $m\in\Bbb N$ and $w\in\{0,1\}^m$, set

\Centerline{$d_w=\inf_{k\ge m}\sup_{z\in\{0,1\}^k,z\supseteq w}
  \Bvalue{P_J(\chi(a\Bcap b_z))>2^{-n}}\in\frak C_J$.}

\noindent Then

\Centerline{$\sup_{w\in\{0,1\}^m}d_w
  =\inf_{k\ge m}\sup_{z\in\{0,1\}^k}\Bvalue{P_J(\chi(a\Bcap b_z))>2^{-n}}
=c_n$}

\noindent because

\Centerline{$\langle\sup_{z\in\{0,1\}^k,z\supseteq w}
  \Bvalue{P_J(\chi(a\Bcap b_z))>2^{-n}}\rangle_{k\ge m}$}

\noindent is non-increasing for each $w$.    In particular,
$d_{\emptyset}=c_n$.   Also
$d_w=d_{w\cup\{(m,0)\}}\Bcup d_{w\cup\{(m,1)\}}$ for every $w\in\{0,1\}^m$.
So if we set

\Centerline{$d'_{\emptyset}=1$,}

\Centerline{$d'_{w\cup\{(m,0)\}}=d'_w\Bcap d_{w\cup\{(m,0)\}}$,
\quad$d'_{w\cup\{(m,1)\}}=d'_w\Bsetminus d_{w\cup\{(m,0)\}}$}

\noindent for every $m\in\Bbb N$ and $w\in\{0,1\}^m$, $d'_w\in\frak C_J$
and $d_w=d'_w\Bcap c_n$ for every $w$,
and there must be an $\eta<\frak c$ such that

\Centerline{$\Bvalue{\dot y_{\eta}(\check n)=1}
=\sup_{w\in\{0,1\}^{n+1},w(n)=1}d'_w$}

\noindent for every $n\in\Bbb N$, in which case

\Centerline{$\Bvalue{\check w\subseteq\dot y_{\eta}}=d'_w$ for every
$w\in\bigcup_{n\in\Bbb N}\{0,1\}^n$.}

If $m\in\Bbb N$, then

$$\eqalignno{\bar\nu_{\kappa}
  (a\Bcap\Bvalue{\dot x\restr\check m=\dot y_{\eta}\restr\check m})
&=\sum_{w\in\{0,1\}^m}\bar\nu_{\kappa}(a\Bcap b_w\Bcap d'_w)\cr
&=\sum_{w\in\{0,1\}^m}\int_{d'_w}P_J(\chi(a\Bcap b_w))
\ge\sum_{w\in\{0,1\}^m}2^{-n}\bar\nu_{\kappa}(c_n\Bcap d'_w)\cr
\displaycause{because $c_n\Bcap d'_w=d_w
\Bsubseteq\Bvalue{P_J(\chi(a\Bcap b_w))>2^{-n}}$ for every $w$}
&=2^{-n}\bar\nu_{\kappa}c_n.\cr}$$

\noindent So if we set $a'=a\Bcap\Bvalue{\dot x=\dot y_{\eta}}$, then

$$\eqalign{\bar\nu_{\kappa}a'
&=\bar\nu_{\kappa}
  (\inf_{m\in\Bbb N}a
    \Bcap\Bvalue{\dot x\restr\check m=\dot y_{\eta}\restr\check m})\cr
&=\inf_{m\in\Bbb N}\bar\nu_{\kappa}
  (a\Bcap\Bvalue{\dot x\restr\check m=\dot y_{\eta}\restr\check m})
\ge 2^{-n}\bar\nu_{\kappa}c_n
>0,\cr}$$

\noindent and $a'\ne 0$, while $a'\Bsubseteq a$ and
$a'\VVdPk\dot x=\dot y_{\eta}$.   So we have a suitable pair $a'$,
$\eta$.\ \Qed

\medskip

\wheader{553C}{0}{0}{0}{36pt}
{\bf (e)} Putting (c) and (d) together, we see that for any
name $\dot x$ for a member of $\{0,1\}^{\Bbb N}$,

\Centerline{$\Bvalue{\dot x\in\dot A}
\Bsubseteq\sup_{\eta<\frak c}\Bvalue{\dot x=\dot y_{\eta}}$.}

\noindent But this means that

\Centerline{$\VVdPk\dot A\subseteq\{\dot y_{\eta}:\eta<\check{\frak c}\}$}

\noindent and

\Centerline{$\VVdPk\#(\dot A)\le\check{\frak c}$.}
}%end of proof of 553C

\medskip

\cmmnt{
\leader{553D}{Remark} If $\kappa>\frak c$ then

\doubleinset{$\VVdPk$ for any countable family of subsets of
$\{0,1\}^{\omega}$
there is an extension of $\nu_{\omega}$ measuring every member of
the family}

\noindent (552N).   By 552Oa and 552Gc we see that in this case

\doubleinset{$\VVdPk$ any universally negligible subset of
$\{0,1\}^{\omega}$ has cardinal less than $\check{\frak c}$.}

\noindent The proposition here tells us that

\doubleinset{$\VVdPk$ any subset of
$\{0,1\}^{\omega}$ with Rothberger's property
has cardinal at most $\check{\frak c}$}

\noindent without restriction on $\kappa$.
}%end of comment

\vleader{72pt}{553E}{Proposition}
Let $\kappa$ and $\lambda$ be infinite cardinals,
and $\dot K$ a $\BbbPk$-name such that

\Centerline{$\VVdPk\,\dot K$ is a compact
subset of $\{0,1\}^{\check\lambda}$ which is not scattered.}

\noindent Then there is a
negligible F$_{\sigma}$ set $G\subseteq\{0,1\}^{\lambda}$ such that

\Centerline{$\VVdPk\,\dot K\cap\tilde G\ne\emptyset$}

\noindent where
$\tilde G\cmmnt{\mskip5mu=(\{0,1\}^{\kappa}\times G)^{\sspvec}}$
is the $\BbbPk$-name for an F$_{\sigma}$ set in
$\{0,1\}^{\check\lambda}$ corresponding to $G$\cmmnt{ as described in
551K}.

\woddheader{553E}{0}{0}{0}{30pt}

\proof{{\bf (a)} If $a\in\frak B_{\kappa}^+$,
$\dot A$ is a $\BbbPk$-name such that

\Centerline{$a\VVdPk\,\dot A$ is an infinite subset of
$\{0,1\}^{\check\lambda}$}

\noindent and $\epsilon>0$, then there is an open-and-closed subset $H$ of
$\{0,1\}^{\lambda}$ such that $\nu_{\lambda}H\le\epsilon$ and
$\bar\nu_{\kappa}(a\Bcap\Bvalue{\dot A\cap\tilde H=\emptyset})\le\epsilon$.
\Prf\ We may suppose that $\epsilon=2^{-k}$ for some $k\in\Bbb N$.
Let $\sequence{i}{\dot y_i}$ be a sequence of $\BbbPk$-names such that

\Centerline{$a\VVdPk\,\dot y_i\in\dot A$ and $\dot y_i\ne\dot y_j$}

\noindent whenever $i$, $j\in\Bbb N$ are distinct.
Let $N\in\Bbb N$ be so large that $e^{-\epsilon N}<\bover12\epsilon$.
Then

\Centerline{$a\VVdPk$ there is a finite $J\subseteq\check\lambda$ such that
$\dot y_i\restr J\ne\dot y_j\restr J$ whenever $i<j<\check N$,}

\noindent that is,

\Centerline{$\sup_{J\in[\lambda]^{<\omega}}
\Bvalue{\dot y_i\restr\check J\ne\dot y_j\restr\check J
  \text{ whenever }i<j<\check N}
\Bsupseteq a$,}

\noindent and there is a finite set $J\subseteq\lambda$ such that

\Centerline{$\bar\nu_{\kappa}(a\Bsetminus
\Bvalue{\dot y_i\restr\check J\ne\dot y_j\restr\check J
  \text{ whenever }i<j<\check N})
\le\Bover12\epsilon$;}

\noindent enlarging $J$ if necessary,
we can suppose that $m=\#(J)$ is such that $m\ge k$ and
\penalty-100$(1-\Bover{N}{2^m})^{2^m\epsilon}\le\Bover12\epsilon$.

For each $i\in\Bbb N$, let $f_i:\{0,1\}^{\kappa}\to\{0,1\}^{\lambda}$ be a
$(\Tau_{\kappa},\CalBa_{\lambda})$-measurable function such that
(in the language of 551Cc) $a\VVdPk\,\dot y_i=\vec f_i$ .   Set

\Centerline{$E=\{x:f_i(x)\restr J\ne f_j(x)\restr J$ whenever
$i<j<N\}$;}

\noindent then

\Centerline{$E^{\ssbullet}
=\Bvalue{\dot y_i\restr\check J\ne\dot y_j\restr\check J
\text{ whenever }i<j<\check N}$,}

\noindent and
$\bar\nu_{\kappa}(a\Bsetminus E^{\ssbullet})\le\bover12\epsilon$.

Let $L$ be a subset of $\{0,1\}^J$ obtained by a
stochastic process in which
we pick $2^{m-k}=2^m\epsilon$ points independently with the uniform
distribution, and take $L$ to be the set of these points.
For any $x\in E$,

\Centerline{$\Pr(f_i(x)\restr J\notin L\Forall i<N)
=\Pr((L\cap\{f_i(x)\restr J:i<N\})=\emptyset)
=(1-\Bover{N}{2^m})^{2^m\epsilon}
\le\Bover12\epsilon$.}

\noindent By Fubini's theorem, there must be an $L\subseteq\{0,1\}^J$ such that
$\#(L)\le 2^m\epsilon$ and

\Centerline{$\nu_{\kappa}\{x:x\in E$,
$f_i(x)\restr J\notin L\Forall i<N\}
\le\Bover12\epsilon\nu_{\kappa}E\le\Bover12\epsilon$.}

Set $H=\{y:y\in\{0,1\}^{\lambda}$, $y\restr J\in L\}$ and
$b=\Bvalue{\dot A\cap\tilde H=\emptyset}$.   Then $H$ is open-and-closed,
$\nu_{\lambda}H=2^{-m}\#(L)\le\epsilon$ and

$$\eqalign{a\Bcap b
&\Bsubseteq a\Bcap\Bvalue{\dot y_i\notin\tilde H\Forall i<\check N}
=a\Bcap\{x:f_i(x)\notin H\Forall i<N\}^{\ssbullet}\cr
&=a\Bcap\{x:f_i(x)\restr J\notin L\text{ for every }i<N\}^{\ssbullet}\cr
&\Bsubseteq(a\Bsetminus E^{\ssbullet})
\Bcup\{x:x\in E,\,
f_i(x)\restr J\notin L\text{ for every }i<N\}^{\ssbullet}\cr}$$

\noindent has measure at most $\epsilon$, as required.\ \Qed

\medskip

{\bf (b)} Because every non-scattered space has a non-empty closed
subset with no isolated points, we may suppose that

\Centerline{$\VVdPk\,\dot K$ has no isolated points.}

\noindent For any $\epsilon\in\ooint{0,1}$
there is a compact negligible set $F\subseteq\{0,1\}^{\lambda}$ such that

\Centerline{$\bar\nu_{\kappa}\Bvalue{\dot K\cap\tilde F\ne\emptyset}
\ge 1-\epsilon$.}

\noindent\Prf\ Choose $\sequencen{a_n}$, $\sequencen{H_n}$
and $\sequencen{\dot K_n}$ inductively, as
follows.   $a_0=1$ and $\dot K_0=\dot K$.   Given that

\Centerline{$a_n\VVdPk\,\dot K_n$ is a non-empty compact set in
$\{0,1\}^{\check\lambda}$ without isolated points}

\allowmorestretch{468}{
\noindent and $\bar\nu_{\kappa}a_n>1-\epsilon$,
let $H_n\subseteq\{0,1\}^{\lambda}$ be an open-and-closed set of
measure at most $2^{-n}$ such that
$a_{n+1}=\penalty-100a_n\Bcap\Bvalue{\dot K_n\cap\tilde H_n\ne\emptyset}$
has measure greater than $1-\epsilon$.
Now let $\dot K_{n+1}$ be a $\BbbPk$-name such that
$\VVdPk\,\dot K_{n+1}=\dot K_n\cap\tilde H_n$.   Because
}

\doubleinset{$a_{n+1}\VVdPk\,\dot K_n$ is a compact set without
isolated points, $\tilde H_n$ is
open-and-closed and $\dot K_n\cap\tilde H_n\ne\emptyset$, so
$\dot K_{n+1}$ is a non-empty compact set without isolated points,}

\noindent the induction continues.

At the end of the induction, set $F=\bigcap_{n\in\Bbb N}H_n$ and
$a=\inf_{n\in\Bbb N}a_n$.   Then $\bar\nu_{\kappa}a\ge 1-\epsilon$ and
$\VVdPk\,\tilde F=\bigcap_{n\in\Bbb N}\tilde H_n$, so

\doubleinset{$a\VVdPk\,\tilde F\cap\dot K$ is the intersection of the
non-increasing sequence $\sequencen{\dot K_n}$ of non-empty compact sets,
so is not empty.}

\noindent So

\Centerline{$\bar\nu_{\kappa}\Bvalue{\tilde F\cap\dot K\ne\emptyset}
\ge\bar\nu_{\kappa}a\ge 1-\epsilon$.}

\noindent Also, of course, $\nu_{\kappa}F=0$, as required.\ \Qed

\medskip

\wheader{553E}{0}{0}{0}{48pt}
{\bf (c)} Finally, let $\sequencen{F_n}$ be a sequence of compact negligible
sets such that

\Centerline{$\bar\nu_{\kappa}\Bvalue{\dot K\cap\tilde F_n\ne\emptyset}
\ge 1-2^{-n}$}

\noindent for every $n$, and set $G=\bigcup_{n\in\Bbb N}F_n$;  this works.
}%end of proof of 553E

\leader{553F}{Corollary} Suppose that $\cf\Cal N(\nu_{\omega})=\omega_1$
and that $\kappa\ge\omega_2$ is a cardinal.   Then

\doubleinset{$\VVdPk\,\omega_1$ is a precaliber of every measurable
algebra but does not have Haydon's property.}

\proof{ By 523N,

\Centerline{$\cf\Cal N(\nu_{\omega_1})
=\max(\cf\Cal N(\nu_{\omega}),\cff[\omega_1]^{\le\omega})=\omega_1$;}

\noindent let
$\ofamily{\xi}{\omega_1}{H_{\xi}}$ be a cofinal family in
$\Cal N(\nu_{\omega_1})$.   Now 552Ga and 525J tell us that

\doubleinset{$\VVdPk\,\cov\Cal N(\nu_{\lambda})>\omega_1$ for every
infinite cardinal $\lambda$, so $\omega_1$ is a precaliber of every
measurable algebra.}

\noindent Next, defining $\tilde H_{\xi}$ from $H_{\xi}$ as in 551K and
553E,

\Centerline{$\VVdPk\,\tilde H_{\xi}\in\Cal N(\nu_{\omega_1})$
for every $\xi<\omega_1$}

\noindent (551Kd;  remember that in this context we do not need to
distinguish between $\omega_1$ and $\check\omega_1$, by 5A3Nb), while

\Centerline{$\VVdPk$ if $K\subseteq\{0,1\}^{\omega_1}$ is a non-scattered
compact set then $K$ meets $\bigcup_{\xi<\omega}\tilde H_{\xi}$.}

\noindent\Prf\ Suppose that $a\in\frak B_{\kappa}^+$ and $\dot K$ is a
$\BbbPk$-name such that

\Centerline{$a\VVdPk\,\dot K\subseteq\{0,1\}^{\omega_1}$ is a non-scattered
compact set.}

\noindent If $a=1$ set $\dot K'=\dot K$;  otherwise let
$\dot K'$ be a $\BbbPk$-name such that

\Centerline{$a\VVdPk\,\dot K'=\dot K$,
\quad$1\Bsetminus a\VVdPk\,\dot K'=\{0,1\}^{\omega_1}$.}

\noindent By 553E, there is a negligible set $G\subseteq\{0,1\}^{\omega_1}$ such
that $\VVdPk\,\dot K'\cap\tilde G\ne\emptyset$.
Now there is a $\xi<\omega_1$ such that $G\subseteq H_{\xi}$, so that

\Centerline{$a\VVdPk\,\dot K\cap\tilde H_{\xi}
\supseteq\dot K'\cap\tilde G\ne\emptyset$.  \Qed}

\noindent By 531N,

\Centerline{$\VVdPk\,\omega_1$ does not have Haydon's property.}
}%end of proof of 553F

\leader{553G}{Lemma} Let $(\frak A,\bar\mu)$ be a probability algebra,
$\frak C$ a subalgebra of $\frak A$, and $\sequencen{e_n}$ a sequence in
$\frak A$ stochastically independent of each other and of $\frak C$.   Let
$I\subseteq\frak A$ be a finite set and $\frak C_I$ the subalgebra of
$\frak A$ generated by $\frak C\cup I$.   Then for every $\epsilon>0$ there
is an $n_0\in\Bbb N$ such that
$|\bar\mu(b\Bcap e_n)-\bar\mu b\cdot\bar\mu e_n|\le\epsilon\bar\mu b$
whenever $b\in\frak C_I$ and $n\ge n_0$.

\proof{{\bf (a)} The first step is to show that if
$u\in L^1(\frak A,\bar\mu)$ then

\inset{\noindent
for every $\epsilon>0$ there is an $n_0\in\Bbb N$ such that
$|\int_{b\Bcap e_n}u-\bar\mu e_n\cdot\int_bu|\le\epsilon\int|u|$
whenever $b\in\frak C$ and $n\ge n_0$.}

\noindent\Prf\ Consider the set
$U$ of those $u\in L^1(\frak A,\bar\mu)$ for which this true.
This is a linear subspace of $L^1(\frak A,\bar\mu)$.   Also it is
$\|\,\|_1$-closed, because if $\int|v-u|\le\bover13\int|u|$ and
$|\int_{b\Bcap e_n}v-\bar\mu e_n\cdot\int_bv|\le\bover14\epsilon\int|v|$
then $|\int_{b\Bcap e_n}u-\bar\mu e_n\cdot\int_bu|\le\epsilon\int|u|$.
If we take $\frak D_m$ to be the subalgebra of $\frak A$
generated by $\frak C\cup\{e_n:n\le m\}$, then
$\bar\mu(a\Bcap b\Bcap e_n)=\bar\mu(a\Bcap b)\cdot\bar\mu e_n$ whenever
$a\in\frak D_m$, $b\in\frak C$ and $n\ge m$,
so $\chi a\in U$ for every $a\in\frak D_m$.
Consequently $\chi a\in U$ for every $a\in\frak D$, where $\frak D$ is the
metric closure of $\bigcup_{m\in\Bbb N}\frak D_m$ in $\frak A$.
Identifying
$L^1(\frak D,\bar\mu\restrp\frak D)$ with the closed linear subspace of
$L^1(\frak A,\bar\mu)$ generated by $\{\chi a:a\in\frak D\}$
(365Q, 365F),
we see that $U\supseteq L^1(\frak D,\bar\mu\restrp\frak D)$.   Now suppose
that $u$ is any member of $L^1(\frak A,\bar\mu)$.   Then we have a
conditional expectation $Pu$ of $u$ in $L^1(\frak D,\bar\mu\restrp\frak D)$
(365Q), and

\Centerline{$|\int_{b\Bcap e_n}u-\bar\mu b\cdot\int_bu|
=|\int_{b\Bcap e_n}Pu-\bar\mu e_n\cdot\int_bPu|$}

\noindent for every $b\in\frak C$ and $n\in\Bbb N$, while $|Pu|\le P|u|$,
so $u\in U$ because $Pu\in U$.\ \Qed

\medskip

{\bf (b)} I show now, by induction on $\#(I)$, that if $a\in\frak A$ then

\inset{\noindent
for every $\epsilon>0$ there is an $n_0\in\Bbb N$ such that
$|\bar\mu(a\Bcap b\Bcap e_n)-\bar\mu(a\Bcap b)\cdot\bar\mu e_n|
\le\epsilon\mu a$ whenever $b\in\frak C_I$ and $n\ge n_0$.}

\noindent\Prf\ If $I$ is empty, we can apply (a) with $u=\chi a$.   For the
inductive step to $\#(I)=k+1$, express $I$ as $J\cup\{c\}$ where $\#(J)=k$.
Take $a\in\frak A$.   Let $n_0\in\Bbb N$ be such that

\Centerline{$|\bar\mu((a\Bcap c)\Bcap b\Bcap e_n)
  -\bar\mu((a\Bcap c)\Bcap b)\cdot\bar\mu e_n|
\le\epsilon\mu(a\Bcap c)$,}

\Centerline{$|\bar\mu((a\Bsetminus c)\Bcap b\Bcap e_n)
  -\bar\mu((a\Bsetminus c)\Bcap b)\cdot\bar\mu e_n|
\le\epsilon\mu(a\Bsetminus c)$}

\noindent whenever
$b\in\frak C_J$ and $n\ge n_0$.   Now take $b\in\frak C_I$ and $n\ge n_0$.
There are $b'$, $b''\in\frak C_J$ such that
$b=(b'\Bcap c)\Bcup(b''\Bsetminus c)$, so that

$$\eqalign{|\bar\mu(a\Bcap b\Bcap e_n)-\bar\mu(a\Bcap b)\cdot\bar\mu e_n|
&=|\bar\mu((a\Bcap c)\Bcap b'\Bcap e_n)
  -\bar\mu((a\Bcap c)\Bcap b')\cdot\bar\mu e_n\cr
&\mskip50mu+\bar\mu((a\Bsetminus c)\Bcap b''\Bcap e_n)
  -\bar\mu((a\Bsetminus c)\Bcap b'')\cdot\bar\mu e_n|\cr
&\le|\bar\mu((a\Bcap c)\Bcap b'\Bcap e_n)
  -\bar\mu((a\Bcap c)\Bcap b')\cdot\bar\mu e_n|\cr
&\mskip50mu+|\bar\mu((a\Bsetminus c)\Bcap b''\Bcap e_n)
  -\bar\mu((a\Bsetminus c)\Bcap b'')\cdot\bar\mu e_n|\cr
&\le\epsilon\bar\mu(a\Bcap c)+\epsilon\bar\mu(a\Bsetminus c)
=\epsilon\bar\mu a.\cr}$$

\noindent Thus the induction proceeds.\ \Qed

\medskip

{\bf (c)} Now the result as stated is just the case $a=1$ in (b).
}%end of proof of 553G

\leader{553H}{Theorem} If $\kappa>\frak c$, then

\Centerline{$\VVdPk$ there are no rapid $p$-point
ultrafilters, therefore no Ramsey filters on $\Bbb N$.}

%do we need "ultra"? yes, see 553Xc

\proof{ (See {\smc Jech 78}, \S38.)

\medskip

{\bf (a)} Let $\langle e_{\xi n}\rangle_{\xi<\kappa,n\in\Bbb N}$ be a
re-indexing of the standard generating family in $\frak B_{\kappa}$.
Let $\dot\Cal F$ be a $\BbbPk$-name for an ultrafilter, and set
$\hat a=\Bvalue{\dot\Cal F\text{ is a rapid }p\text{-point ultrafilter}}$.
\Quer\ Suppose, if possible, that $\hat a\ne 0$.
For each $f\in\BbbN^{\Bbb N}$,

\Centerline{$\hat a\VVdPk$ there is a $D\in\dot\Cal F$ such that
$\#(D\cap\check f(k))\le k$ for every $k$}

\noindent(538Ad);   let $\dot D_f$ be a $\BbbPk$-name for a subset of
$\Bbb N$ such that

\Centerline{$\hat a\VVdPk\dot D_f\in\dot\Cal F$}

\noindent and

\Centerline{$\hat a\VVdPk\#(\dot D_f\cap f(k)\var2spcheck)\le\check k$}

\noindent for every $k\in\Bbb N$.
Let $J\subseteq\kappa$ be a set of size
at most $\frak c$ such that $\Bvalue{\check n\in\dot D_f}$ belongs to the closed
subalgebra $\frak C$ generated by
$\{e_{\xi i}:\xi\in J$, $i\in\Bbb N\}$ for
every $f\in\Bbb N^{\Bbb N}$ and every $n\in\Bbb N$, and $\hat a$ also
belongs to $\frak C$.

\medskip

{\bf (b)} Let $\zeta<\kappa$ be such that the ordinal sum $\zeta+k$ does
not belong to $J$ for any $k\in\Bbb N$.
For each $k\in\Bbb N$ let $\dot C_k$ be a $\BbbPk$-name for a subset of
$\Bbb N$ such that $\Bvalue{\check n\in\dot C_k}=e_{\zeta+k,n}$ for every
$n\in\Bbb N$.
Set $c_k=\Bvalue{\dot C_k\notin\dot\Cal F}$ and let $\dot A_k$ be a
$\BbbPk$-name for a subset of $\Bbb N$ such that

\Centerline{$c_k\VVdPk\dot A_k=\Bbb N\setminus\dot C_k\in\dot\Cal F$,
\quad$1\Bsetminus c_k\VVdPk\dot A_k=\dot C_k\in\dot\Cal F$.}

\noindent Then $\VVdPk\dot A_k\in\dot\Cal F$ for every $k$, and
$\Bvalue{\check n\in\dot A_k}=c_k\Bsymmdiff e_{\zeta+k,n}$ for every
$n\in\Bbb N$.

\medskip

{\bf (c)} For $k$, $n\in\Bbb N$ set

\Centerline{$b_{kn}=\Bvalue{\check n\in\bigcap_{i<\check k}\dot A_i}
=\inf_{i<k}\Bvalue{\check n\in\dot A_i}
=\inf_{i<k}c_i\Bsymmdiff e_{\zeta+i,n}$.}

\noindent Then we have a non-decreasing $f:\Bbb N\to\Bbb N$
such that
$\bar\nu_{\kappa}(c\Bcap b_{kn})\le(2^{-k+1}-2^{-2k})\bar\nu_{\kappa}c$
whenever $c\in\frak C$, $k\in\Bbb N$ and $n\ge f(k)$.   \Prf\  Define
$f$ inductively, as follows.   If
$k=0$ then (interpreting $\inf\emptyset$ as $1$) we have $b_{kn}=1$ for every
$n$ so we can take $f(0)=0$.   For the inductive step to $k+1$,
let $\frak C_k$ be the closed subalgebra of $\frak B_{\kappa}$ generated
by $\frak C\cup\{e_{\zeta+i,n}:i<k$, $n\in\Bbb N\}$ and $\frak D_k$ the
subalgebra generated by $\frak C_k\cup\{c_i:i\le k\}$.
Then $\frak C_k$ and
$\sequencen{e_{\zeta+k,n}}$ are stochastically independent, so Lemma 553G tells
us that there is an $f(k+1)\ge f(k)$ such that

\Centerline{$|\bar\nu_{\kappa}(d\Bcap e_{\zeta+k,n})-\Bover12\bar\nu_{\kappa}d|
\le\Bover1{24}\cdot 2^{-k}\bar\nu_{\kappa}d$ whenever $d\in\frak D_k$ and
$n\ge f(k+1)$.}

\noindent Take $n\ge f(k+1)$ and $c\in\frak C$.   Then

$$\eqalignno{\bar\nu_{\kappa}(c\Bcap b_{k+1,n})
&=\bar\nu_{\kappa}(c\Bcap b_{kn}\Bcap(c_k\Bsymmdiff e_{\zeta+k,n}))\cr
&=\bar\nu_{\kappa}(c\Bcap b_{kn}\Bcap c_k)
  -2\bar\nu_{\kappa}(c\Bcap b_{kn}\Bcap c_k\Bcap e_{\zeta+k,n})
  +\bar\nu_{\kappa}(c\Bcap b_{kn}\Bcap e_{\zeta+k,n})\cr
&\le 2|\bar\nu_{\kappa}(c\Bcap b_{kn}\Bcap c_k\Bcap e_{\zeta+k,n})
    -\Bover12\bar\nu_{\kappa}(c\Bcap b_{kn}\Bcap c_k)|\cr
&\mskip100mu+|\bar\nu_{\kappa}(c\Bcap b_{kn}\Bcap e_{\zeta+k,n})
    -\Bover12\bar\nu_{\kappa}(c\Bcap b_{kn})|
    +\Bover12\bar\nu_{\kappa}(c\Bcap b_{kn})\cr
&\le\Bover1{12}\cdot 2^{-k}\bar\nu_{\kappa}(c\Bcap b_{kn}\Bcap c_k)
    +\Bover1{24}\cdot 2^{-k}\bar\nu_{\kappa}(c\Bcap b_{kn})
    +\Bover12\bar\nu_{\kappa}(c\Bcap b_{kn})\cr
\displaycause{because all the elements
$c\Bcap b_{kn}$ and $c\Bcap b_{kn}\Bcap c_k$ belong to $\frak D_k$}
&\le(2^{-k-3}+\Bover12)\bar\nu_{\kappa}(c\Bcap b_{kn})\cr
&\le(2^{-k-3}+\Bover12)(2^{-k+1}-2^{-2k})\bar\nu_{\kappa}c\cr
\displaycause{because $n\ge f(k)$}
&\le(2^{-k}-2^{-2k-2})\bar\nu_{\kappa}c.\cr}$$

\noindent So the construction proceeds.\ \Qed

\medskip

{\bf (d)} Because

\Centerline{$\hat a\VVdPk\dot{\Cal F}$ is a $p$-point ultrafilter and
$\dot A_k\in\dot{\Cal F}$ for every $k$,}

\noindent there are a $\BbbPk$-name $\dot A$ for a subset of $\Bbb N$ and a
$\BbbPk$-name $\dot g$ for a function from $\Bbb N$ to itself such that

\Centerline{$\hat a\VVdPk\dot A\in\dot{\Cal F}$ and
$\dot A\setminus\dot A_i\subseteq\dot g(k)$ whenever $i<k\in\Bbb N$.}

\noindent Let $g$ (in the ordinary universe) be a non-decreasing
function such that $f(k)\le g(k)$ and
$\bar\nu_{\kappa}(\Bvalue{\dot g(\check k)>g(k)\var2spcheck})
\penalty-100\le 2^{-k-2}\bar\nu_{\kappa}\hat a$ for every $k$.   Set
$\hat a_1=\hat a\Bcap\Bvalue{\dot g\le\check g}$;  then
$\bar\nu_{\kappa}\hat a_1\ge\bover12\bar\nu_{\kappa}\hat a$.

\medskip

{\bf (e)} Take the function $g$ from (d) and the
name $\dot D_g$ from (a), and set
$d_n=\hat a\Bcap\Bvalue{\check n\in\dot D_g}\in\frak C$ for
every $n$.   Then

\Centerline{$\sum_{n=g(k)}^{g(k+1)-1}\bar\nu_{\kappa}(d_n\Bcap b_{kn})
\le 2^{-k+1}(k+1)$}

\noindent for every $k\in\Bbb N$.   \Prf\ Set $K=g(k+1)\setminus g(k)$.
We have

$$\eqalignno{\sum_{n=g(k)}^{g(k+1)-1}\bar\nu_{\kappa}(d_n\Bcap b_{kn})
&=\sum_{n\in K}\sum_{I\subseteq K}
  \bar\nu_{\kappa}(d_n\Bcap b_{kn}
    \Bcap\Bvalue{\check I=\dot D_g\cap\check K})\cr
&=\sum_{I\subseteq K}\sum_{n\in I}
  \bar\nu_{\kappa}(d_n\Bcap b_{kn}
    \Bcap\Bvalue{\check I=\dot D_g\cap\check K})\cr
\displaycause{because
$d_n\Bcap\Bvalue{\check I=\dot D_g\cap\check K}
\Bsubseteq\Bvalue{\check n\in\dot D_g}
  \Bcap\Bvalue{\check I=\dot D_g\cap\check K}$ is zero if
$n\notin I$}
&=\sum_{I\in[K]^{\le k+1}}\sum_{n\in I}
  \bar\nu_{\kappa}(d_n\Bcap b_{kn}
     \Bcap\Bvalue{\check I=\dot D_g\cap\check K})\cr
\displaycause{because
$d_n\Bcap\Bvalue{\check I=\dot D_g\cap K}
\Bsubseteq\hat a
  \Bcap\Bvalue{\check I\subseteq\dot D_g\cap g(k+1)\var2spcheck}$
is zero if $\#(I)>k+1$}
&\le 2^{-k+1}\sum_{I\in[K]^{\le k+1}}\sum_{n\in I}
  \bar\nu_{\kappa}
  (d_n\Bcap\Bvalue{\check I=\dot D_g\cap\check K})\cr
\displaycause{because
$d_n\Bcap\Bvalue{\check I=\dot D_g\cap\check K}\in\frak C$
for every $n$ and $I$, and we are looking only at $n\ge g(k)\ge f(k)$}
&\le 2^{-k+1}(k+1)\sum_{I\in[K]^{\le k+1}}
  \bar\nu_{\kappa}\Bvalue{\check I=\dot D_g\cap\check K}\cr
&\le 2^{-k+1}(k+1).  \text{  \Qed}\cr}$$

\woddheader{553H}{4}{2}{2}{30pt}

{\bf (f)} As $\hat a\ne 0$, $\hat a_1\ne 0$.   Let $m$ be such that
$\sum_{k=m}^{\infty}2^{-k+1}(k+1)$ is less than
$\bar\nu_{\kappa}\hat a_1$;  then

\Centerline{$\hat a_2
=\hat a_1\Bsetminus\sup_{k\ge m}\sup_{g(k)\le n<g(k+1)}(d_n\Bcap b_{kn})$}

\noindent is non-zero.   Let $\dot B$
be a $\BbbPk$-name for a subset of $\Bbb N$ such that
$\VVdPk\dot B=\dot A\Bcap\dot D_g\setminus g(m)\var2spcheck$.   Then
$\hat a\VVdPk\dot B\in\dot{\Cal F}$.   But
$\hat a_2\VVdPk\dot B=\emptyset$.   \Prf\ Take any $n\in\Bbb N$.   If
$n<g(m)$ then $\VVdPk n\notin\dot B$.   If $k\ge m$ and $g(k)\le n<g(k+1)$,
then

\Centerline{$\hat a_1\VVdPk\dot g(\check k)\le g(k)\var2spcheck$,
\quad$\hat a\VVdPk\dot A\setminus\dot A_i\subseteq\dot g(\check k)$
for every $i<\check k$,}

\noindent so

\Centerline{$\hat a_1\Bcap\Bvalue{\check n\in\dot B}
\Bsubseteq\hat a\Bcap\Bvalue{\check n\in\dot A\setminus\dot g(\check k)}
\Bsubseteq\Bvalue{\check n\in\bigcap_{i<\check k}\dot A_i}=b_{kn}$.}

\noindent Also, of course, $\VVdPk\dot B\subseteq\dot D_g$, so
$\hat a_1\Bcap\Bvalue{\check n\in\dot B}\Bsubseteq d_n\Bcap b_{kn}$ is
disjoint from $\hat a_2$.   But this means that
$\hat a_2\VVdPk\check n\notin\dot B$.
As $n$ is arbitrary, $\hat a_2\VVdPk\dot B=\emptyset$.\ \QeD\   Now

\Centerline{$\hat a_2\VVdPk\emptyset\in\dot{\Cal F}$,}

\noindent which is impossible.\ \Bang

\medskip

{\bf (g)} So $\hat a=0$, that is,

\Centerline{$\VVdPk\dot{\Cal F}$ is not a rapid $p$-point ultrafilter.}

\noindent As $\dot{\Cal F}$ is arbitrary,

\Centerline{$\VVdPk$ there are no rapid $p$-point ultrafilters.}

\medskip

{\bf (h)} Finally, by 538Fa,

\Centerline{$\VVdPk$ there are no Ramsey ultrafilters.}
}%end of proof of 553H

\vleader{72pt}{553I}{Lemma}
Suppose that $S\subseteq\omega_1^2$ is a set such that
whenever $n\in\Bbb N$ and
$\ofamily{\xi}{\omega_1}{I_{\xi}}$ is a family in $[\omega_1]^n$ such that
$I_{\xi}\cap\xi=\emptyset$ for every $\xi<\omega_1$, there are
$\xi<\omega_1$ and $\eta<\xi$ such that
$I_{\xi}\times I_{\eta}\subseteq S$.   Let $P$ be the set

\Centerline{$\{I:I\in[\omega_1]^{<\omega}$, $I\cap\xi\subseteq S[\{\xi\}]$ for
every $\xi\in I\}$,}

\noindent ordered by $\subseteq$.   Then $P$ is upwards-ccc.

\proof{ Let $\ofamily{\xi}{\omega_1}{J_{\xi}}$ be any family in $P$.
Then there are distinct $\xi$, $\eta<\omega_1$ such that
$J_{\xi}\cup J_{\eta}\in P$.   \Prf\ By the
$\Delta$-system Lemma (4A1Db), there is an uncountable set
$A_0\subseteq\omega_1$
such that $\family{\xi}{A_0}{J_{\xi}}$ is a $\Delta$-system with root $J$ say;
next, there is an $n\in\Bbb N$ such that $A_1=\{\xi:\xi\in A_0$,
$\#(J_{\xi}\setminus J)=n\}$ is uncountable.    If $n=0$ then
$J_{\xi}\cup J_{\eta}=J$ belongs to $P$ for any $\xi$, $\eta\in A_1$ and we can
stop.   Otherwise, there is an uncountable $A_2\subseteq A_1$ such that whenever
$\xi$, $\eta\in A_2$ and $\eta<\xi$ then
$\max J_{\eta}<\min(J_{\xi}\setminus J)$.
Re-enumerate $\family{\xi}{A_2}{J_{\xi}\setminus J}$ in
increasing order to get a family
$\ofamily{\xi}{\omega_1}{I_{\xi}}$ in $[\omega_1]^n$ such that
$\min I_{\xi}\ge\xi$ for every $\xi$.   Our hypothesis tells us that there are
$\eta<\xi$ such that $I_{\xi}\times I_{\eta}\subseteq S$.   Let $\xi'$,
$\eta'<\omega_1$ be such that $I_{\xi}=J_{\xi'}\setminus J$ and
$I_{\eta}=J_{\eta'}\setminus J$, and consider
$I=J\cup I_{\xi}\cup I_{\eta}$.   If $\alpha\in I$ and
$\beta\in I\cap\alpha$,

\inset{----- {\it either} $\alpha$, $\beta$ both belong to $J_{\eta'}$ so
$(\alpha,\beta)\in S$

----- {\it or} $\alpha$, $\beta$ both belong to $J_{\xi'}$ so
$(\alpha,\beta)\in S$

----- {\it or} $\alpha\in I_{\xi}$ and $\beta\in I_{\eta}$ so
$(\alpha,\beta)\in S$.}

\noindent So $J_{\xi'}\cup J_{\eta'}=I$ belongs to $S$.\ \Qed

Thus $P$ has no uncountable up-antichains and is upwards-ccc.
}%end of proof of 553I

\leader{553J}{Theorem} Let $\kappa$ be an infinite cardinal.   Then

\doubleinset{$\VVdPk$ there are two upwards-ccc partially ordered sets
whose product is not upwards-ccc.}

\cmmnt{\medskip

\noindent{\bf Remark} If $\kappa>\omega$ this is immediate from 552E, 537F
and 537G.   So we have a new result only if $\kappa=\omega$.}

\proof{{\bf (a)} Let $\family{\xi}{\kappa}{e_{\xi}}$ be the standard
generating family in $\frak B_{\kappa}$.   For $J\subseteq\kappa$ let
$\frak C_J$ be the
closed subalgebra of $\frak B_{\kappa}$ generated by
$\{e_{\xi}:\xi\in J\}$.   For $\xi<\omega_1$ let
$h_{\xi}:\xi\to\Bbb N$ be an injective function.

\medskip

{\bf (b)} Let $\dot S_0$ be a $\BbbPk$-name for a
subset of $\omega_1^2$ such that

$$\eqalign{\Bvalue{(\check\xi,\check\eta)\in\dot S_0}
&=e_{h_{\xi}(\eta)}\text{ if }\eta<\xi,\cr
&=0\text{ otherwise}.\cr}$$

\noindent Then

\doubleinset{$\VVdPk$ whenever $n\in\Bbb N$ and
$\ofamily{\xi}{\omega_1}{I_{\xi}}$ is a family in $[\omega_1]^n$ such that
$I_{\xi}\cap\xi=\emptyset$ for every $\xi<\omega_1$, there are
$\xi<\omega_1$ and $\eta<\xi$ such that
$I_{\xi}\times I_{\eta}\subseteq\dot S_0$.}

\noindent\Prf\Quer\ Suppose, if possible, otherwise.   Then we have an
$n\in\Bbb N$, an $a\in\frak B_{\kappa}^+$ and a family
$\ofamily{\xi}{\omega_1}{\dot I_{\xi}}$ of $\BbbPk$-names such that

\Centerline{$a\VVdPk\,\dot I_{\eta}\in[\omega_1\setminus\eta]^{\check n}$
and $\dot I_{\xi}\times\dot I_{\eta}\not\subseteq\dot S_0$ whenever
$\eta<\xi<\omega_1$.}

\noindent For each $\xi<\omega_1$ there are $a_{\xi}\in\frak B_{\kappa}^+$,
stronger than $a$, and $I_{\xi}\in[\omega_1\setminus\xi]^n$ such that
$a_{\xi}\VVdPk\,\dot I_{\xi}=\check I_{\xi}$.   By 525Tc
we can find an uncountable set
$A_0\subseteq\omega_1$ and an $\epsilon>0$ such that
$\bar\nu_{\kappa}(a_{\xi}\Bcap a_{\eta})\ge\epsilon$
whenever $\xi$, $\eta\in A_0$.  Next, there is an uncountable
set $A_1\subseteq A_0$ such that $I_{\eta}\subseteq\xi$ whenever $\xi\in A_1$
and $\eta\in\xi\cap A_1$;  consequently $I_{\eta}\cap I_{\xi}=\emptyset$
whenever $\xi$, $\eta\in A_1$ are distinct, and
$\bar\nu_{\kappa}
 \Bvalue{\check I_{\xi}\times\check I_{\eta}\subseteq\dot S_0}
=2^{-n^2}$.

Let $\delta>0$ be such that $2^{-n^2}(\epsilon-2\delta)-2\delta>0$.
For each $\xi\in A_1$ we can find a finite set $J_{\xi}\subseteq\kappa$ and an
$a'_{\xi}\in\frak C_{J_{\xi}}$ such that
$\bar\nu_{\kappa}(a_{\xi}\Bsymmdiff a'_{\xi})\le\delta$.   Let $m\in\Bbb N$ be
such that

\Centerline{$A_2=\{\xi:\xi\in A_1$, $J_{\xi}\cap\omega\subseteq m\}$}

\noindent is uncountable.   Let $\xi\in A_2$ be such that $A_2\cap\xi$ is
infinite.   In this case, $\family{\eta}{A_2\cap\xi}{h_{\zeta}[I_{\eta}]}$ is
disjoint for each $\zeta\in I_{\xi}$, so we have an $\eta\in A_2\cap\xi$ such
that $h_{\zeta}[I_{\eta}]\cap m=\emptyset$ for every $\zeta\in I_{\xi}$.
It follows that
$\Bvalue{\check I_{\xi}\times\check I_{\eta}\subseteq\dot S_0}
\in\frak C_{\omega\setminus m}$, while
$a'_{\xi}\Bcap a'_{\eta}\in\frak C_{m\cup(\kappa\setminus\omega)}$, so

$$\eqalign{\bar\nu_{\kappa}(a'_{\xi}\Bcap a'_{\eta}
\Bcap\Bvalue{\check I_{\xi}\times\check I_{\eta}\subseteq\dot S_0})
&=\bar\nu_{\kappa}(a'_{\xi}\Bcap a'_{\eta})
\cdot\bar\nu_{\kappa}
  \Bvalue{\check I_{\xi}\times\check I_{\eta}\subseteq\dot S_0}\cr
&=2^{-n^2}\bar\nu_{\kappa}(a'_{\xi}\Bcap a'_{\eta}).\cr}$$

\noindent If we set
$b=a_{\xi}\Bcap a_{\eta}
  \Bcap\Bvalue{\check I_{\xi}\times\check I_{\eta}\subseteq\dot S_0}$,

$$\eqalign{\bar\nu_{\kappa}b
&\ge 2^{-n^2}\bar\nu_{\kappa}(a'_{\xi}\Bcap a'_{\eta})-2\delta
\ge 2^{-n^2}(\bar\nu_{\kappa}(a_{\xi}\Bcap a_{\eta})-2\delta)-2\delta\cr
&\ge 2^{-n^2}(\epsilon-2\delta)-2\delta
>0.\cr}$$

\noindent But now we have $b\Bsubseteq a$ and

\Centerline{$b\VVdPk\,\dot I_{\xi}\times\dot I_{\eta}
=\check I_{\xi}\times\check I_{\eta}\subseteq\dot S_0$,}

\noindent which is supposed to be impossible.\ \Bang\Qed

\medskip

{\bf (c)} Let $\dot P_0$ be a $\BbbPk$-name for a partially ordered set defined
from $\dot S_0$ by the process of 553I, so that for a finite set
$I\subseteq\omega_1$

\Centerline{$\Bvalue{\check I\in\dot P_0}
=\inf_{\xi,\eta\in I,\eta<\xi}e_{h_{\xi}(\eta)}$.}

\noindent By 553I and (b) above,

\Centerline{$\VVdPk\,\dot P_0$ is upwards-ccc.}

\wheader{553J}{6}{2}{2}{48pt}

{\bf (d)} Similarly, if $\dot S_1$ is a $\BbbPk$-name for a
subset of $\omega_1^2$ such that

$$\eqalign{\Bvalue{(\check\xi,\check\eta)\in\dot S_1}
&=1\Bsetminus e_{h_{\xi}(\eta)}\text{ if }\eta<\xi,\cr
&=0\text{ otherwise},\cr}$$

\noindent and $\dot P_1$ is a $\BbbPk$-name for a partially ordered set
defined from $\dot S_1$ by the process of 553I, then

\Centerline{$\VVdPk\,\dot P_1$ is upwards-ccc.}

\noindent (The point is just that
$\ofamily{\xi}{\kappa}{1\Bsetminus e_{\xi}}$ also is a stochastically
independent family of elements of measure $\bover12$.)
But now observe that if $\eta<\xi<\omega_2$ then

\Centerline{$\Bvalue{\{\check\xi,\check\eta\}\in\dot P_0\cap\dot P_1}
=\Bvalue{(\check\xi,\check\eta)\in\dot S_0\cap\dot S_1}
=e_{h_{\xi}(\eta)}\Bcap(1\Bsetminus e_{h_{\xi}(\eta)})=0$.}

\noindent So

\doubleinset{$\VVdPk\,\{\{\{\xi\},\{\xi\}\}:\xi<\omega_1\}$
is an up-antichain in
$\dot P_0\times\dot P_1$, and $\dot P_0\times\dot P_1$ is not upwards-ccc.}

\noindent Thus we have the required example.
}%end of proof of 553J

\leader{553K}{}\cmmnt{ I extract an elementary step from the proof of the next
lemma.

\medskip

\noindent}{\bf Lemma} Let $\frak A$ be a Boolean algebra and
$\nu:\frak A\to\coint{0,\infty}$ a non-negative additive functional.   Then

\Centerline{$\sum_{i=0}^n\nu a_i
\le\nu(\sup_{i\le n}a_i)+\sum_{i<j\le n}\nu(a_i\Bcap a_j)$}

\noindent whenever $a_0,\ldots,a_n\in\frak A$.

\proof{ Let $d$ be any atom of the subalgebra of $\frak A$ generated by
$a_0,\ldots,a_n$.   Suppose that $\#(\{i:i\le n$, $d\Bsubseteq a_i\})=m$.
Then

$$\eqalign{\nu(d\Bcap\sup_{i\le n}a_i)+\sum_{i<j\le n}&\nu(d\Bcap a_i\Bcap a_j)
  -\sum_{i=0}^n\nu(d\Bcap a_i)\cr
&=0\text{ if }m\le 1,\cr
&=1+\Bover{m(m-1)}2-m=\Bover12(m-1)(m-2)\ge 0\text{ otherwise}.\cr}$$

\noindent Summing over $d$, we have the result.
}%end of proof of 553K

\vleader{72pt}{553L}{Lemma} Let $(\frak A,\bar\mu)$ be a
probability algebra, $I$ an uncountable set, $X$ a non-empty set and
$\Cal F$ an ultrafilter on $X$.
Let $\langle a_{ix}\rangle_{i\in I,x\in X}$ be a
family in $\frak A$ such that
$\inf_{i\in I}\lim_{x\to\Cal F}\bar\mu a_{ix}>0$.
Then there are an uncountable set $S\subseteq I$ and a family
$\family{i}{S}{b_i}$ in $\frak A\setminus\{0\}$ such that

\Centerline{$b_i\Bcap b_j
\Bsubseteq\sup_{x\in F}a_{ix}\Bcap a_{jx}$}

\noindent for all $i$, $j\in S$ and $F\in\Cal F$.

\proof{{\bf (a)} We can suppose that $I=\omega_1$.   For each
$\xi<\omega_1$ set $u_{\xi}=\lim_{x\to\Cal F}\chi a_{\xi x}$,
the limit being taken for the weak topology on $L^2(\frak A,\bar\mu)$
(\S366), so that

\Centerline{$\int_au_{\xi}
=\lim_{x\to\Cal F}\bar\mu(a\Bcap a_{\xi x})$}

\noindent for every $a\in\frak A$.   In particular,
$\int u_{\xi}\ge\epsilon$, where
$\epsilon=\inf_{\xi<\omega_1}\lim_{x\to\Cal F}\bar\mu a_{\xi x}>0$;  set
$b'_{\xi}=\Bvalue{u_{\xi}>\bover12\epsilon}$, so that $b'_{\xi}\ne 0$.

\medskip

{\bf (b)} For $\xi$, $\eta<\omega_1$ set

\Centerline{$c_{\xi\eta}
=\inf_{F\in\Cal F}\sup_{x\in F}a_{\xi x}\Bcap a_{\eta x}$.}

\noindent For $K\subseteq\omega_1$ set

\Centerline{$d_K
=\inf_{\xi\in K}b'_{\xi}\Bsetminus
\sup_{\xi,\eta\in K\text{ are distinct}}c_{\xi\eta}$.}

\noindent If $d_K\ne 0$ then
$\epsilon\#(K)<3$.   \Prf\ We may suppose that $K$ is finite and not
empty; set $n=\#(K)$.
We have $\int_{d_K}u_{\xi}>\bover12\epsilon\bar\mu d_K$ for every
$\xi\in K$, so

\Centerline{$F_0
=\{x:x\in X$,
$\bar\mu(d_K\Bcap a_{\xi x})\ge\bover12\epsilon\bar\mu d_K$ for every
$\xi\in K\}$}

\noindent belongs to $\Cal F$.   Let $F\in\Cal F$ be such that, setting
$c'_{\xi\eta}=\sup_{x\in F}a_{\xi x}\Bcap a_{\eta x}$,
$\bar\mu(c'_{\xi\eta}\Bsetminus c_{\xi\eta})\le\Bover{\bar\mu d_K}{n^2}$
for all $\xi$, $\eta\in K$.   Take any $x\in F\cap F_0$.
If $\xi$, $\eta\in K$ are distinct,

\Centerline{$\bar\mu(d_K\Bcap a_{\xi x}\Bcap a_{\eta x})
\le\bar\mu(c'_{\xi\eta}\Bsetminus c_{\xi\eta})\le\Bover{\bar\mu d_K}{n^2}$,}

\noindent so

$$\eqalignno{\Bover{n\epsilon}2\bar\mu d_K
&\le\sum_{\xi\in K}\bar\mu(d_K\Bcap a_{\xi x})
\le\bar\mu d_K+\sum_{\xi,\eta\in K,\xi<\eta}
\bar\mu(d_K\Bcap a_{\xi x}\Bcap a_{\eta x})\cr
\displaycause{553K}
&\le\bar\mu d_K+\Bover{n(n-1)}2\cdot\Bover{\bar\mu d_K}{n^2}
<\Bover32\bar\mu d_K\cr}$$

\noindent and $n\epsilon<3$.\ \Qed

\medskip

{\bf (c)} For each infinite $\xi<\omega_1$ there is therefore a maximal
subset $K_{\xi}$ of $\xi$ such that
$b_{\xi}=d_{K_{\xi}\cup\{\xi\}}$ is
non-zero.   Every $K_{\xi}$ is finite, so there is a
$K\in[\omega_1]^{<\omega}$ such that
$S=\{\xi:\omega\le\xi<\omega_1$, $K_{\xi}=K\}$ is
stationary.   \Prf\ By the Pressing-Down Lemma (4A1Cc), there is a
$\zeta<\omega_1$ such that
$\{\xi:\xi<\omega_1$, $\sup K_{\xi}=\zeta\}$ is stationary.
As $[\zeta+1]^{<\omega}$ is countable, there will be a $K\subseteq\zeta+1$
such that $\{\xi:K_{\xi}=K\}$ is stationary.\ \QeD\
Now suppose that $\eta$, $\xi\in S$ and $\eta<\xi$.
Then

\Centerline{$b_{\eta}\Bcap b_{\xi}\Bsetminus c_{\eta\xi}
=d_{K_{\eta}\cup\{\eta\}}\Bcap d_{K_{\xi}\cup\{\xi\}}
  \Bsetminus c_{\xi\eta}
=d_{K\cup\{\eta,\xi\}}=0$}

\noindent because $K\cup\{\eta\}$ is a subset
of $\xi$ properly including $K_{\xi}$.   So we have an appropriate
family $\family{\xi}{S}{b_{\xi}}$.
}%end of proof of 553L

\leader{553M}{Proposition}\cmmnt{ ({\smc Laver 87})}
If $\frak m>\omega_1$ and $\kappa$ is any infinite cardinal, then

\Centerline{$\VVdPk$ every Aronszajn tree is special, so Souslin's
hypothesis is true.}

\proof{{\bf (a)} By 5A1D(b-ii) and 5A1D(d-ii), it is enough to show that

\Centerline{$\VVdPk$ every Aronszajn tree ordering of $\omega_1$ included
in the usual ordering is special.}

\noindent
Let $\dot\preccurlyeq$ be a $\BbbPk$-name for an Aronszajn tree ordering of
$\omega_1$ included in the usual ordering of $\omega_1$.
For $\alpha$, $\beta<\omega_1$ set
$a_{\alpha\beta}=\Bvalue{\check\alpha\dot\preccurlyeq\check\beta}$;
note that
$a_{\alpha\alpha}=1$, $a_{\alpha\beta}=0$ if $\beta<\alpha$ and
$a_{\alpha\beta}\Bsupseteq a_{\alpha\gamma}\Bcap a_{\beta\gamma}$
whenever $\alpha\le\beta\le\gamma<\omega_1$.

If $\Cal F$ is an ultrafilter on $\omega_1$ containing
$\omega_1\setminus\zeta$ for every $\zeta<\omega_1$, then
$\lim_{\xi\to\Cal F}\bar\nu_{\kappa}a_{\alpha\xi}=0$
for all but countably many
$\alpha<\omega_1$.   \Prf\Quer\ Otherwise, there is an $\epsilon>0$ such
that $I=\{\alpha:\alpha<\omega_1$,
$\lim_{\xi\to\Cal F}\bar\mu a_{\alpha\xi}\ge\epsilon\}$ is uncountable.
By 553L, there are an uncountable $S\subseteq I$ and a family
$\family{\alpha}{S}{b_{\alpha}}$ in $\frak A\setminus\{0\}$ such that

\Centerline{$b_{\alpha}\Bcap b_{\beta}
\Bsubseteq\sup_{\xi\ge\beta}a_{\alpha\xi}\Bcap a_{\beta\xi}
\Bsubseteq a_{\alpha\beta}$}

\noindent whenever $\alpha$, $\beta\in S$ and $\alpha<\beta$.
Set $c=\inf_{\alpha<\omega_1}\sup_{\beta\in S\setminus\alpha}b_{\beta}$,
so that $c\ne 0$.   Let $\dot Y$ be a $\BbbPk$-name
for a subset of $\omega_1$
such that $\Bvalue{\check\alpha\in\dot Y}=b_{\alpha}$ for $\alpha\in S$,
$\Bvalue{\check\alpha\in\dot Y}=0$ for other $\alpha$.   Then

\Centerline{$\VVdPk\alpha\mskip5mu\dot\preccurlyeq\mskip5mu\beta$ whenever $\alpha$,
$\beta\in\dot Y$ and $\alpha<\beta$,}

\Centerline{$c\VVdPk\dot Y$ is uncountable;}

\noindent so

\Centerline{$c\VVdPk\dot Y$ is an uncountable branch in the Aronszajn
tree,}

\noindent which is impossible.\ \Bang\Qed

\medskip

{\bf (b)} Let $\ofamily{\xi}{\kappa}{e_{\xi}}$ be the standard
generating family in $\frak B_{\kappa}$.   Choose inductively a
non-decreasing family $\ofamily{\alpha}{\omega_1}{J_{\alpha}}$ of
countably infinite
subsets of $\kappa$ such that $a_{\beta\alpha}$ belongs to the
closed subalgebra $\frak C_{J_{\alpha}}$
of $\frak B_{\kappa}$ generated by
$\{e_{\xi}:\xi\in J_{\alpha}\}$ whenever $\beta\le\alpha<\omega_1$.
%Let $\frak E_{\alpha}$ be the (countable) subalgebra of $\frak B_{\kappa}$
%generated by $\{e_{\xi}:\xi\in J_{\alpha}\}$.

Let $P$ be the partially ordered set of functions $f$ such that

\inset{$\dom f$ is a finite subset of $\omega_1\times\omega$,

for every $(\alpha,n)\in\dom f$, $f(\alpha,n)\in\frak C_{J_{\alpha}}$ and
$\bar\nu_{\kappa}f(\alpha,n)>\bover12$,

$f(\alpha,n)\Bcap f(\beta,n)\Bcap a_{\beta\alpha}=0$
whenever $(\alpha,n)$, $(\beta,n)\in\dom f$ and $\beta<\alpha$.}

\noindent Say that $f\le g$ if $\dom f\subseteq\dom g$ and
$g(\alpha,n)\Bsubseteq f(\alpha,n)$
for every $(\alpha,n)\in\dom f$.   Then $\le$ is a partial order on $P$.

$P$ is upwards-ccc.   \Prf\ Let $\ofamily{\xi}{\omega_1}{f_{\xi}}$ be a
family in $P$.   Let $A_0\subseteq\omega_1$ be an uncountable set such that
$\family{\xi}{A_0}{\dom f_{\xi}}$ is a
$\Delta$-system with root $K$
say;  let $\epsilon>0$, $m\in\Bbb N$ be such that

$$\eqalign{A_1
&=\{\xi:\xi\in A_0,\,\#(\dom f_{\xi})=m+\#(K),\cr
&\mskip100mu
\bar\nu_{\kappa}f_{\xi}(\alpha,n)\ge\Bover12+2\epsilon
  \text{ whenever }(\alpha,n)\in\dom f_{\xi}\}\cr}$$

\noindent is uncountable.   Let $A_2\subseteq A_1$ be an uncountable set
such that
$\bar\mu(f_{\eta}(\alpha,n)\Bsymmdiff f_{\xi}(\alpha,n))\le\epsilon$
whenever $\xi$, $\eta\in A_2$ and $(\alpha,n)\in K$;  such a set exists
because $\frak C_{J_{\alpha}}$ is metrically separable for each $\alpha$.
Let $A_3\subseteq A_2$ be an uncountable set such that
$\beta<\alpha$ whenever $\eta\in A_3$, $\xi\in A_3$, $\eta<\xi$,
$(\beta,m)\in\dom f_{\eta}$ and
$(\alpha,n)\in(\dom f_{\xi})\setminus K$.

For $\xi\in A_3$, enumerate $(\dom f_{\xi})\setminus K$ as
$\ofamily{i}{m}{(\alpha_{\xi i},n_{\xi i})}$.   Let $\Cal F$ be an
ultrafilter on $\omega_1$ containing $A_3\setminus\zeta$ for every
$\zeta<\omega_1$, and for $i<m$ let $\Cal F_i$ be the ultrafilter
$\{F:F\subseteq\omega_1$, $\{\xi:\alpha_{\xi i}\in F\}\in\Cal F\}$.
By (a), we have an uncountable $A_4\subseteq A_3$ such that

\Centerline{$\lim_{\xi\to\Cal F_i}
  \bar\nu_{\kappa}a_{\alpha_{\eta j},\xi}=0$}

\noindent for every $i$, $j<m$ and every $\eta\in A_4$;  that is,

\Centerline{$\lim_{\xi\to\Cal F}
  \bar\nu_{\kappa}a_{\alpha_{\eta j},\alpha_{\xi i}}=0$}

\noindent whenever $i$, $j<m$ and $\eta\in A_4$.   But this means that we
can find $\eta\in A_4$ and $\xi\in A_3$ such that $\eta<\xi$ and
$\bar\nu_{\kappa}a_{\alpha_{\eta j},\alpha_{\xi i}}
\le\Bover{\epsilon}{m+1}$ for all $i$, $j<m$.   Now consider the function
$g$ with domain $\dom f_{\eta}\cup\dom f_{\xi}$ such that

$$\eqalign{g(\alpha,n)
&=f_{\eta}(\alpha,n)\Bcap f_{\xi}(\alpha,n)
  \text{ if }(\alpha,n)\in K,\cr
&=f_{\eta}(\alpha,n)
  \text{ if }(\alpha,n)\in\dom f_{\eta}\setminus K,\cr
&=f_{\xi}(\alpha_{\xi i},n_{\xi i})
  \Bsetminus\sup_{j<m}a_{\alpha_{\eta j},\alpha_{\xi i}}
  \text{ if }i<m\text{ and }(\alpha,n)=(\alpha_{\xi i},n_{\xi i}).\cr}$$

\noindent Then $g(\alpha,n)\in\frak C_{J_{\alpha}}$
and $\bar\nu_{\kappa}g(\alpha,n)\ge\bover12+\epsilon$ for every
$(\alpha,n)\in\dom g$.   If $(\alpha,n)$ and $(\beta,n)$ belong to $\dom g$
and $\beta<\alpha$, then

\inset{----- if both $(\beta,n)$ and $(\alpha,n)$ belong to
$\dom f_{\eta}$, then

\Centerline{$g(\beta,n)\Bcap g(\alpha,n)\Bcap a_{\beta\alpha}
\Bsubseteq f_{\eta}(\beta,n)\Bcap f_{\eta}(\alpha,n)\Bcap a_{\beta\alpha}
=0$;}

----- if both $(\beta,n)$ and $(\alpha,n)$ belong to
$\dom f_{\xi}$, then

\Centerline{$g(\beta,n)\Bcap g(\alpha,n)\Bcap a_{\beta\alpha}
\Bsubseteq f_{\xi}(\beta,n)\Bcap f_{\xi}(\alpha,n)\Bcap a_{\beta\alpha}
=0$;}

----- if $(\beta,n)=(\alpha_{\eta j},n_{\eta j})$ and
$(\alpha,n)=(\alpha_{\xi i},n_{\xi i})$ then
$g(\alpha,n)$ is disjoint from
$a_{\alpha_{\eta j},\alpha_{\xi i}}=a_{\beta\alpha}$ so
\ifnum\stylenumber=12\break\fi
$g(\beta,n)\Bcap g(\alpha,n)\Bcap a_{\beta\alpha}=0$.}

\noindent So $g\in P$ and is an upper bound for $f_{\eta}$ and $f_{\xi}$.
Thus $\ofamily{\xi}{\omega_1}{f_{\xi}}$ is not an up-antichain in $P$;
as $\ofamily{\xi}{\omega_1}{f_{\xi}}$ is arbitrary, $P$ is upwards-ccc.\
\Qed

\medskip

{\bf (c)} For each $\alpha<\omega_1$ let $C_{\alpha}$ be a countable
metrically dense subset of
$\{c:c\in\frak C_{J_{\alpha}}$, $\bar\nu_{\kappa}c\le\bover12\}$.
For $\alpha<\omega_1$ and $c\in C_{\alpha}$, set

$$\eqalign{Q_{\alpha c}
&=\{f:f\in P\text{ and there is some }n\in\Bbb N\text{ such that }
(\alpha,n)\in\dom f,\cr
&\mskip250mu c\Bsubseteq f(\alpha,n)\text{ and }
\bar\nu_{\kappa}f(\alpha,n)=\Bover12+\Bover13\bar\nu_{\kappa}c\}.\cr}$$

\noindent Then $Q_{\alpha c}$ is cofinal with $P$.   \Prf\ Because
$J_{\alpha}$ is infinite, $\frak C_{J_{\alpha}}$ is atomless and there is
an $a\in\frak C_{J_{\alpha}}$ such that $c\Bsubseteq a$ and
$\bar\nu_{\kappa}a=\bover12+\bover13\bar\nu_{\kappa}c$.   Now take $n$ so
large that $i<n$ whenever $(\alpha,i)\in\dom f$, and set
$g=f\cup\{((\alpha,n),a)\}$;  then $f\le g\in Q_{\alpha c}$.\ \Qed

\medskip

{\bf (d)} Because $\frak m>\omega_1$, there is an upwards-directed set
$R\subseteq P$ meeting $Q_{\alpha c}$ whenever $\alpha<\omega_1$ and
$c\in C_{\alpha}$.   Now, for $n\in\Bbb N$, let $\dot A_n$ be a
$\BbbPk$-name for a subset of $\omega_1$ such that, for every
$\alpha<\omega_1$,

$$\eqalign{\Bvalue{\check\alpha\in\dot A_n}
&=\inf\{f(\alpha,n):f\in R,\,(\alpha,n)\in\dom f\}
  \text{ if }(\alpha,n)\in\bigcup_{f\in R}\dom f,\cr
&=0\text{ otherwise}\cr}$$

\noindent Then $\dot A_n$ is a name for an up-antichain for the tree order
$\dot\preccurlyeq$.   \Prf\ If $\beta<\alpha<\omega_1$, then either
$\Bvalue{\check\beta\in\dot A_n}=0$ or
$\Bvalue{\check\alpha\in\dot A_n}=0$ or there
are $f$, $g\in R$ such that $(\alpha,n)\in\dom f$ and $(\beta,n)\in\dom g$.
In this case, because $R$ is upwards-directed, there is an $h\in R$ such
that both $(\alpha,n)$ and $(\beta,n)$ belong to $\dom h$, so that

\Centerline{$\Bvalue{\check\alpha\in\dot A_n}
  \Bcap\Bvalue{\check\beta\in\dot A_n}
  \Bcap\Bvalue{\check\beta\dot\preccurlyeq\check\alpha}
\Bsubseteq h(\alpha,n)\Bcap h(\beta,n)\Bcap a_{\beta\alpha}
=0$.}

\noindent Thus

\Centerline{$\VVdPk$ if $\alpha$, $\beta\in\dot A_n$ then they are
$\dot\preccurlyeq$-incompatible upwards.}

\noindent As $\alpha$ and $\beta$ are arbitrary,

\Centerline{$\VVdPk\dot A_n$ is an up-antichain.  \Qed}

\wheader{553M}{6}{2}{2}{36pt}

{\bf (e)} Finally,

\Centerline{$\VVdPk\bigcup_{n\in\Bbb N}\dot A_n=\omega_1$.}

\noindent\Prf\Quer\ Otherwise, there is an $\alpha<\omega_1$ such that
$a=1\Bsetminus\sup_{n\in\Bbb N}\Bvalue{\check\alpha\in\dot A_n}\ne 0$.
Observe at this point that
$\Bvalue{\check\alpha\in\dot A_n}\in\frak C_{J_{\alpha}}$ for every $n$.    So
$a\in\frak C_{J_{\alpha}}$.   Let $a'\in\frak C_{J_{\alpha}}$ be such that
$a'\Bsubseteq a$ and $0<\bar\nu_{\kappa}a'\le\bover12$, and let
$c\in C_{\alpha}$ be such that
$\bar\nu_{\kappa}(a'\Bsymmdiff c)\le\bover14\bar\nu_{\kappa}a'$, so that
$c\ne 0$ and
$\bar\nu_{\kappa}(c\Bsetminus a')\le\bover13\bar\nu_{\kappa}c$.
Since $R$ meets $Q_{\alpha c}$, there are $n\in\Bbb N$, $f\in R$ such that
$c\Bsubseteq f(\alpha,n)$ and
$\bar\nu_{\kappa}f(\alpha,n)=\bover12+\bover13\bar\nu_{\kappa}c$.

If $g\in P$ and $f\le g$, then $g(\alpha,n)\Bsubseteq f(\alpha,n)$ and
$\bar\nu_{\kappa}g(\alpha,n)>\bover12$, so

\Centerline{$\bar\nu_{\kappa}(c\Bsetminus g(\alpha,n))
\le\bar\nu_{\kappa}f(\alpha,n)-\bar\nu_{\kappa}g(\alpha,n)
\le\Bover13\bar\nu_{\kappa}c$.}

\noindent Because $R$ is upwards-directed,
$\{g(\alpha,n):g\in R$, $(\alpha,n)\in\dom g\}$ is downwards-directed, and

$$\eqalign{\bar\nu_{\kappa}(c\Bsetminus\Bvalue{\check\alpha\in\dot A_n})
&=\sup\{\bar\nu_{\kappa}(c\Bsetminus g(\alpha,n)):g\in R,\,
(\alpha,n)\in\dom g\}\cr
&=\sup\{\bar\nu_{\kappa}(c\Bsetminus g(\alpha,n)):g\in R,\,f\le g\}
\le\Bover13\bar\nu_{\kappa}c.\cr}$$

Accordingly

\Centerline{$\bar\nu_{\kappa}(a'\Bcap\Bvalue{\check\alpha\in\dot A_n})
\ge\bar\nu_{\kappa}(c\Bcap\Bvalue{\check\alpha\in\dot A_n})
   -\bar\nu_{\kappa}(c\Bsetminus a')
\ge\Bover23\bar\nu_{\kappa}c-\Bover13\bar\nu_{\kappa}c
>0$;}

\noindent but $a'\Bsubseteq a$ is supposed to be disjoint from
$\Bvalue{\check\alpha\in\dot A_n}$.\ \Bang\Qed

So $\sequencen{\dot A_n}$ is a name for a sequence of antichains covering
$\omega_1$, and

\Centerline{$\VVdPk(\omega_1,\dot\preccurlyeq)$ is special,}

\noindent as required.
}%end of proof of 553M

\leader{553N}{Proposition} Suppose that there is a medial limit\cmmnt{
(definition:  538Q)}, and that $\kappa$ is a cardinal.   Then

\Centerline{$\VVdPk$ there is a medial limit.}

\proof{{\bf (a)} Let $\theta:\Cal P\Bbb N\to[0,1]$ be a medial limit.
Let $Q$ be the
rationally convex hull of the usual basis of $\ell^1$, that is, the set of
functions $v:\Bbb N\to\Bbb Q\cap[0,1]$ such that $\{n:v(n)\ne 0\}$ is
finite and $\sum_{n=0}^{\infty}v(n)=1$.   Note that $Q$ is absolute in the
sense that

\Centerline{$\VVdP\,\check Q$ is the rationally convex hull of the usual
basis of $\ell^1$}

\noindent for every forcing notion $\Bbb P$.   Let $\Cal F$ be the filter
on $Q$ which is the trace of the weak* neighbourhood filter of $\theta$,
that is, the filter generated by sets of the form

\Centerline{$\{v:v\in Q$,
$|\sum_{n=0}^{\infty}v(n)u(n)-\dashint u(n)\theta(dn)|\le\epsilon\}$}

\noindent where $u\in\ell^{\infty}$ and $\epsilon>0$.   (Identifying
$Q\subseteq\ell^1$ with its image in $(\ell^{\infty})^*\cong(\ell^1)^{**}$,
the weak* closure of $Q$ is convex, so is equal to its bipolar (4A4Eg)
and is the set of positive linear functionals on $\ell^{\infty}$
taking the value $1$ on the order unit $\chi\Bbb N$.   See 363L and
538P for the notation $\dashint\ldots\theta(dn)$.)
Let $\vec{\Cal F}$ be the $\BbbPk$-name derived from $\Cal F$ and
$(\{0,1\}^{\kappa},\Tau_{\kappa},\Cal N_{\kappa})$ by the method of 551Rb,
so that

\Centerline{$\VVdPk\,\vec{\Cal F}$ is a filter on $\check Q$.}

\noindent Let $\dot\nu$ be a $\BbbPk$-name such that

\doubleinset{$\VVdPk\,\dot\nu$ is a bounded additive functional on
$\Cal P\Bbb N$, and identifying $\check Q$ with a subset of
$(\ell^{\infty})^*$, itself identified with the space $M(\Cal P\Bbb N)$ of
bounded additive functionals on $\Cal P\Bbb N$,
$\dot\nu$ is a cluster point of $\vec{\Cal F}$ for the weak* topology.}

\medskip

{\bf (b)} Suppose that $a\in\frak B_{\kappa}^+$ and that
$\dot{\frak e}$ is a $\BbbPk$-name such that

\Centerline{$a\VVdPk\,\dot{\frak e}$ is a sequence of Borel subsets of
$\{0,1\}^{\Bbb N}$.}

\noindent By 551Fb, we have for each $n\in\Bbb N$ a set
$W_n\in\Tau_{\kappa}\tensorhat\CalBa_{\Bbb N}$ such that

\Centerline{$a\VVdPk\,\dot{\frak e}(\check n)=\vec W_n$,}

\noindent where $\vec W_n$ is defined as in 551D.   Let
$\lambda=\nu_{\kappa}\times\nu_{\omega}$ be the product measure on
$\{0,1\}^{\kappa}\times\{0,1\}^{\Bbb N}$.   Because
$\theta$ is a medial limit,
$\idashint\chi W_n(x,y)\theta(dn)\lambda(d(x,y))$ is defined and equal to
$\dashint\lambda W_n\theta(dn)$;  that is, there are a conegligible Baire set
$W\subseteq\{0,1\}^{\kappa}\times\{0,1\}^{\Bbb N}$ and a Baire measurable
function $\psi:\{0,1\}^{\kappa}\times\{0,1\}^{\Bbb N}\to[0,1]$ such that

\Centerline{$\psi(x,y)
=\dashint\chi W_n(x,y)\theta(dn)
=\lim_{v\to\Cal F}\sum_{n=0}^{\infty}v(n)\chi W_n(x,y)$}

\noindent whenever $(x,y)\in W$, and

\Centerline{$\int\psi\,d\lambda
=\dashint\lambda W_n\theta(dn)
=\lim_{v\to\Cal F}\sum_{n=0}^{\infty}v(n)\lambda W_n$.}

\noindent Let $\vec W$ and $\vec\psi$ be the corresponding $\BbbPk$-names,
as in 551D and 551M, so that

\Centerline{$\VVdPk\,\vec W\in\CalBa_{\Bbb N}$ and
$\vec\psi:\{0,1\}^{\Bbb N}\to\Bbb R$ is Baire measurable.}

\noindent Moreover, since $\nu_{\kappa}$-almost every
vertical section of $W$ must be $\nu_{\omega}$-conegligible,

\Centerline{$\VVdPk\,\nu_{\omega}\vec W=1$}

\noindent (551I).

\medskip

{\bf (c)} Now suppose that $\dot s$ is a $\BbbPk$-name and that
$b\in\frak B_{\kappa}^+$ is stronger than $a$ and such that

\Centerline{$b\VVdPk\,\dot s\in\vec W$.}

\medskip

\quad{\bf (i)} By 551Cc, there is a $\Tau_{\kappa}$-measurable
$f:\{0,1\}^{\kappa}\to\{0,1\}^{\Bbb N}$ such that

\Centerline{$b\VVdPk\,\dot s=\vec f$.}

\noindent Expressing $b$ as $E^{\ssbullet}$ where
$E\in\Tau_{\kappa}\setminus\Cal N_{\kappa}$,
$(x,f(x))\in W$ for $\nu_{\kappa}$-almost every $x\in E$, by 551Ea.

For each $m\in\Bbb N$, consider

$$\eqalign{C_m
&=\{(x,v):x\in\{0,1\}^{\kappa},\,v\in Q,\,(x,f(x))\in W,\cr
&\mskip50mu
|\psi(x,f(x))-\sum_{n=0}^{\infty}v(n)\chi W_n(x,f(x))|\le 2^{-m}\}.\cr}$$

\noindent Then $C_m\in\Tau_{\kappa}\tensorhat\Cal PQ$;  and if
$x\in\{0,1\}^{\kappa}$ is such that $(x,f(x))\in W$,
$C_m[\{x\}]\in\Cal F$.   Consequently

\Centerline{$b\VVdPk\,\vec C_m\in\vec{\Cal F}$,}

\noindent where in this formula $\vec C_m$ is the $\BbbPk$-name defined by
the method of 551Ra.   At the same time,

\Centerline{$b\VVdPk\,
|\vec\psi(\dot s)-\sum_{n=0}^{\infty}v(n)\chi(\dot{\frak e}(n))(\dot s)|
\le 2^{-\check m}$ for every $v\in\vec C_m$.}

\noindent\Prf\ Suppose we have a $c$ stronger than $b$ and a $\BbbPk$-name
$\dot v$ such that $c\VVdPk\,\dot v\in\vec C_m$.   Then there are a
$G\in\Tau_{\kappa}\setminus\Cal N_{\kappa}$ and a $v\in Q$ such that
$G^{\ssbullet}\Bsubseteq c$, $G^{\ssbullet}\VVdPk\,\dot v=\check v$,
and $(x,v)\in C_m$ for every $x\in G$.   Setting

\Centerline{$h(x)=\psi(x,f(x))$,
\quad$h_n(x)=\chi W_n(x,f(x))$}

\noindent for $x\in\{0,1\}^{\kappa}$ and $n\in\Bbb N$, and interpreting
$\vec h$, $\vec h_n$ as in 551B,

\Centerline{$b\VVdPk\,
\vec h=\vec\psi(\vec f)=\vec\psi(\dot s)$,}

\noindent and

$$\eqalignno{b\VVdPk\,\vec h_n
&=(\chi W_n)\sspvec(\dot s)=(\chi\vec W_n)(\dot s)\cr
\displaycause{551Nd}
&=(\chi\dot{\frak e}(\check n))(\dot s)\cr}$$

\noindent for $n\in\Bbb N$.
For $x\in G$, moreover,
$|h(x)-\sum_{n=0}^{\infty}v(n)h_n(x)|\le 2^{-m}$, so

\Centerline{$G^{\ssbullet}\VVdPk\,
|\vec\psi(\dot s)
   -\sum_{n=0}^{\infty}\dot v(n)\chi(\dot{\frak e}(n))(\dot s)|
=|\vec h-\sum_{n=0}^{\infty}\check v(n)\vec h_n|
\le 2^{-\check m}$.}

\noindent As $c$ and $\dot v$ are arbitrary, we have the result.\ \Qed

\medskip

\quad{\bf (ii)} As $m$ is arbitrary,

\Centerline{$b\VVdPk\,\{v:v\in\check Q,\,
|\vec\psi(\dot s)
   -\sum_{n=0}^{\infty}v(n)\chi(\dot{\frak e}(n))(\dot s)|
   \le\epsilon\}
\in\vec{\Cal F}$ for every $\epsilon>0$,}

\noindent that is,

\Centerline{$b\VVdPk\,
\vec\psi(\dot s)=\lim_{v\to\vec{\Cal F}}
   \sum_{n=0}^{\infty}v(n)\chi(\dot{\frak e}(n))(\dot s)$.}

\noindent As $b$ and $\dot s$ are arbitrary,

\Centerline{$a\VVdPk\,
\vec\psi(y)=\lim_{v\to\vec{\Cal F}}
   \sum_{n=0}^{\infty}v(n)\chi(\dot{\frak e}(n))(y)$
for every $y\in\vec W$;}

\noindent since $\VVdPk\,\vec W$ is conegligible,

\Centerline{$a\VVdPk\,
\vec\psi\eae\lim_{v\to\vec{\Cal F}}
   \sum_{n=0}^{\infty}v(n)\chi(\dot{\frak e}(n))$.}

\noindent Looking back at the choice of $\dot\nu$, we see that

\Centerline{$a\VVdPk\,
\vec\psi(y)=\dashint\chi(\dot{\frak e}(n))(y)\dot\nu(dn)$ for
$\nu_{\omega}$-almost every $y$.}

\medskip

{\bf (d)} As for the integral of $\vec\psi$, 551Nf tells us that

\Centerline{$\VVdPk\,\int\vec\psi\,d\nu_{\omega}=\vec h$,}

\noindent where I now set $h(x)=\int\psi(x,y)\nu_{\omega}(dy)$ for
$x\in\{0,1\}^{\kappa}$.   Similarly, setting
$h_n(x)=\nu_{\omega}W_n[\{x\}]$, we have

\Centerline{$a\VVdPk\,\nu_{\omega}\dot{\frak e}(\check n)
=\vec h_n$.}

Set

\Centerline{$H=\{x:x\in\{0,1\}^{\kappa}$, $W[\{x\}]$ is
conegligible in $\{0,1\}^{\Bbb N}\}$;}

\noindent then $H$ is conegligible in $\{0,1\}^{\kappa}$.
Now remember that $\theta$ is a medial limit.   If $x\in H$ we have
$\psi(x,y)=\dashint\chi W_n(x,y)\theta(dn)$ for every $y$ in the
conegligible set $W[\{x\}]$, so

$$\eqalign{h(x)
&=\int\psi(x,y)\nu_{\omega}(dy)
=\idashint\chi W_n(x,y)\theta(dn)\nu_{\omega}(dy)\cr
&=\dashiint\chi W_n(x,y)\nu_{\omega}(dy)\theta(dn)
=\dashint\nu_{\omega}W_n[\{x\}]\theta(dn)
=\lim_{v\to\Cal F}\sum_{n=0}^{\infty}v(n)h_n(x).\cr}$$

\noindent So if, for $m\in\Bbb N$, we set

\Centerline{$C'_m
=\{(x,v):x\in\{0,1\}^{\kappa}$, $v\in Q$,
   $|h(x)-\sum_{n=0}^{\infty}v(n)h_n(x)|
       \le 2^{-m}\}$,}

\noindent we shall again have $\VVdPk\,\vec C'_m\in\vec{\Cal F}$;  and if
$G\in\Tau_{\kappa}\setminus\Cal N_{\kappa}$ and $v\in Q$ are such that
$G^{\ssbullet}$ is stronger than $p$ and
$G^{\ssbullet}\VVdPk\,\check v\in\vec C'_m$, then

\Centerline{$G^{\ssbullet}\VVdPk\,
|\int\vec\psi\,d\nu_{\omega}
  -\sum_{n=0}^{\infty}\check v(n)\nu_{\omega}\dot{\frak e}(n)|
\le 2^{-\check m}$.}

\noindent So

\Centerline{$a\VVdPk\,\{v:|\int\vec\psi\,d\nu_{\omega}
  -\sum_{n=0}^{\infty}\check v(n)\nu_{\omega}\dot{\frak e}(n)|
\le 2^{-\check m}\}\in\vec{\Cal F}$}

\noindent for every $m$, and

\Centerline{$a\VVdPk\,\int\vec\psi\,d\nu_{\omega}
=\lim_{v\to\vec{\Cal F}}
  \sum_{n=0}^{\infty}\check v(n)\nu_{\omega}\dot{\frak e}(n)
=\dashint\nu_{\omega}\dot{\frak e}(n)\dot\nu(dn)$.}

\medskip

{\bf (e)} As $p$ and $\dot{\frak e}$ are arbitrary, we see that

\Centerline{$\VVdPk\,\dot\nu$ satisfies condition (iv) of 538P, so is a
medial functional.}

\noindent It is now easy to check that

\Centerline{$\VVdPk\,\dot\nu\ge 0$,
$\dot\nu\Bbb N=1$ and $\dot\nu\{n\}=0$ for every
$n\in\Bbb N$, so $\dot\nu$ is a medial limit.}

\noindent This completes the proof.
}%end of proof of 553N

\leader{553O}{}\dvAnew{2010}\cmmnt{ For the most familiar classes of
`small' set --
the Lebesgue null ideal, or the meager ideal of $\Bbb R$, for instance --
it is easy to calculate the number of sets in the class;  because there is
a nowhere dense Lebesgue negligible set with cardinal $\frak c$, there must
be exactly $2^{\frakc}$ meager Lebesgue negligible sets, and therefore
there are just $2^{\frakc}$ Lebesgue measurable subsets of $\BbbR^r$ for
any $r\ge 1$.   But when we come to the ideal
$\Cal N_{\text{universal}}\normalsubgroup\Cal P\Bbb R$
of universally negligible sets, or the algebra
$\Sigma_{\text{um}}\subseteq\Cal P\Bbb R$ of universally
measurable sets, the position is much less clear.   In general, since by
Grzegorek's theorem
(439F) we know that there is a universally negligible subset of $\Bbb R$ of
cardinal $\non\Cal N(\nu_{\omega})$, we can say that

\Centerline{$\frak c\le 2^{\non\Cal N(\nu_{\omega})}
\le\#(\Cal N_{\text{universal}})
\le\#(\Sigma_{\text{um}})\le 2^{\frakc}$.}

\noindent It turns out that in random real models these inequalities may
well collapse to the lower bound, as in (b) of the next theorem.

\wheader{553O}{4}{2}{2}{60pt}

\noindent}{\bf Theorem}\cmmnt{ ({\smc Larson Neeman \& Shelah 10})}
Let $\kappa$ be an infinite cardinal.

(a)

\vskip-14pt plus 0pt minus 1pt

\doubleinset{$\VVdPk$ every universally measurable subset of
$\{0,1\}^{\Bbb N}$ is expressible as the union of at most
$\check{\frak c}$ Borel sets.}

(b) If the cardinal power $\kappa^{\frak c}$ is equal to $\kappa$, then

\Centerline{$\VVdPk$ there are exactly $\frak c$ universally measurable
subsets of $\{0,1\}^{\Bbb N}$.}

\proof{{\bf (a)(i)} It will save a moment later if I note at once that we
need consider only the case $\kappa>\frak c$.   \Prf\ If
$\kappa\le\frak c$, then $\kappa^{\omega}=\frak c$, so

\Centerline{$\VVdPk$ $\frak c=\check\frak c$}

\noindent by 552B.   But since we surely have

\doubleinset{$\VVdPk$ every universally measurable subset of
$\{0,1\}^{\Bbb N}$ is expressible as the union of at most
$\frak c$ singleton sets,}

\noindent we get the result.\ \Qed

So henceforth I will take it that $\kappa>\frak c$.   It will save
time to have a local notation:  if $M\subseteq\kappa$ and
$V\subseteq\{0,1\}^{\kappa}\times\{0,1\}^{\Bbb N}$, I will say that
$V$ is {\bf $M$- } if $(\tilde\omega,\omega')\in V$ whenever
$(\omega,\omega')\in V$ and $\tilde\omega\in\{0,1\}^{\kappa}$ is such
that $\tilde\omega\restr M=\omega\restr M$.

\medskip

\quad{\bf (ii)} (The testing measures.)  If
$E\in\CalBa_{\kappa}\setminus\Cal N(\nu_{\kappa})$,
$g:\{0,1\}^{\kappa}\to\{0,1\}^{\Bbb N}$ is
a $\CalBa_{\kappa}$-measurable function, and $I\subseteq\kappa$ is a set,
write $Q_{IEg}$ for the set of pairs $(V,h)$ where
$V\in\CalBa_{\kappa}\tensorhat\CalBa_{\Bbb N}$ and
$h:\{0,1\}^{\kappa}\to[0,1]$ is defined by saying that

\Centerline{$h(\omega)=\nu_{\kappa\setminus I}
  \{\omega':(\omega\restr I)\cup\omega'\in E$,
    $(\omega,g((\omega\restr I)\cup\omega'))\in V\}$}

\noindent for every $\omega\in\{0,1\}^{\kappa}$.   Then
$h$ is $\CalBa_{\kappa}$-measurable for every $(V,h)\in Q_{IEg}$.
\Prf\ The function

\Centerline{$(\omega,\omega')
\mapsto(\omega\restr I)\cup\omega':
  \{0,1\}^{\kappa}\times\{0,1\}^{\kappa\setminus I}
  \to\{0,1\}^{\kappa}$}

\noindent is $(\CalBa_{\kappa}\tensorhat\CalBa_{\kappa\setminus I},
\CalBa_{\kappa})$-measurable, because
$\CalBa_{\kappa}=\Tensorhat_{\kappa}\Cal P(\{0,1\})$ (4A3Na).   So

\Centerline{$\{(\omega,\omega'):(\omega\restr I)\cup\omega'\in E$,
    $(\omega,g((\omega\restr I)\cup\omega'))\in V\}
  \in\CalBa_{\kappa}\tensorhat\CalBa_{\kappa\setminus I}$}

\noindent and we can apply 252P.\ \Qed

We can therefore set

\Centerline{$\dot\mu_{IEg}
=\{((\vec V,\vec h),\Bbbone):(V,h)\in Q_{IEg}\}$,}

\noindent and $\dot\mu_{IEg}$ will be a $\BbbPk$-name, subject to the
conventions I use concerning the interpretation of the brackets in the
formula $((\vec V,\vec h),\Bbbone)$.

\medskip

\quad{\bf (iii)} If $I$, $E$, $g$ and $\dot\mu_{IEg}$ are as in (ii),
then

\Centerline{$\VVdPk\,\dot\mu$ is a $[0,1]$-valued function with domain
$\CalBa_{\Bbb N}$.}

\noindent\Prf\ Suppose that $(V_0,h_0)$,
$(V_1,h_1)\in Q_{IEg}$ and $p\in\BbbPk$ are such that
$p\VVdPk\,\vec V_0=\vec V_1$.
Express $p$ as $F^{\ssbullet}$ where $F\in\CalBa_{\kappa}$.   By 551Gb,
$F'=F\setminus\{\omega:V_0[\{\omega\}]=V_1[\{\omega\}]\}$ is negligible.
But for $\omega\in F\setminus F'$,

$$\eqalign{\{\omega':(\omega\restr I)\cup\omega'\in E,
   \,(&\omega,g((\omega\restr I)\cup\omega'))\in V_0\}\cr
&=\{\omega':(\omega\restr I)\cup\omega'\in E,
   \,(\omega,g((\omega\restr I)\cup\omega'))\in V_1\}\cr}$$

\noindent and $h_0(\omega)=h_1(\omega)$.   So $p\VVdPk\,\vec h_0=\vec h_1$.

Thus $\dot\mu_{IEg}$ satisfies the condition (ii) of 5A3Ha, and

\doubleinset{$\VVdPk\,\dot\mu_{IEg}$ is a function with domain
$\{(\vec V,\Bbbone):V\in\CalBa_{\kappa}\tensorhat\CalBa_{\Bbb N}\}
=\CalBa_{\Bbb N}$}

\noindent where the second $\CalBa_{\Bbb N}$ is interpreted in the forcing
language (551F).   Since

\Centerline{$\VVdPk\,\vec h\in[0,1]$}

\noindent whenever $h:\{0,1\}^{\kappa}\to[0,1]$ is a
$\CalBa_{\kappa}$-measurable function (551B),

\Centerline{$\VVdPk\,\dot\mu_{IEg}$ takes values in $[0,1]$.\ \Qed}

Next,

\Centerline{$\VVdPk\,\dot\mu_{IEg}$ is countably additive, so is a Borel
measure on $\{0,1\}^{\Bbb N}$.}

\noindent\Prf\ (Compare 551M-551N.)
Use the formulae of 551E.   If $\dot{\pmb{G}}$ is a
$\BbbPk$-name such that

\Centerline{$\VVdPk$ $\dot{\pmb{G}}$ is a disjoint sequence in
$\CalBa_{\Bbb N}$,}

\noindent then there is a sequence $\sequencen{W_n}$ in
$\CalBa_{\kappa}\tensorhat\CalBa_{\Bbb N}$ such that

\Centerline{$\VVdPk$ $\dot{\pmb{G}}=\sequencen{\vec W_n}$.}

\noindent If $m\ne n$, then

\Centerline{$\VVdPk$ $(W_m\cap W_n)\sspvec=\vec W_m\cap\vec W_n=\emptyset$,}

\noindent so
$\nu_{\kappa}\{\omega:W_m[\{\omega\}]\cap W_n[\{\omega\}]\ne\emptyset\}=0$
(551Ga);  accordingly $\sequencen{W_n[\{\omega\}]}$ is disjoint for
$\nu_{\kappa}$-almost every $\omega$.   Setting
$W=\bigcup_{n\in\Bbb N}W_n$,

\Centerline{$h_n(\omega)=\nu_{\kappa\setminus I}
  \{\omega':(\omega\restr I)\cup\omega'\in E$,
    $(\omega,g((\omega\restr I)\cup\omega'))\in W_n\}$,}

\Centerline{$h(\omega)=\nu_{\kappa\setminus I}
  \{\omega':(\omega\restr I)\cup\omega'\in E$,
    $(\omega,g((\omega\restr I)\cup\omega'))\in W\}$}

\noindent for $\omega\in\{0,1\}^{\kappa}$ and $n\in\Bbb N$,
we see that whenever $\sequencen{W_n[\{\omega\}]}$ is disjoint then

\Centerline{$\sequencen{\{\omega':(\omega\restr I)\cup\omega'\in E$,
    $(\omega,g((\omega\restr I)\cup\omega'))\in W_n\}}$}

\noindent is disjoint, with union

\Centerline{$\{\omega':(\omega\restr I)\cup\omega'\in E$,
    $(\omega,g((\omega\restr I)\cup\omega'))\in W\}$,}

\noindent so $h(\omega)=\sum_{n=0}^{\infty}h_n(\omega)$.
Accordingly

$$\eqalignno{\VVdPk\dot\mu_{IEg}(\bigcup_{n\in\Bbb N}\vec W_n)
&=\dot\mu_{IEg}\vec W\cr
\displaycause{551Ed}
&=\vec h
=\sum_{n=0}^{\infty}\vec h_n\cr
\displaycause{5A3L(c-iii), 5A3Ld}
&=\sum_{n=0}^{\infty}\dot\mu_{IEg}\vec W_n.\cr}$$

\noindent As $\dot{\pmb G}$ is arbitrary,

\Centerline{$\VVdPk$ $\dot\mu_{IEg}$ is countably additive.  \Qed}

\medskip

\quad{\bf (iv)} Still supposing that $I$, $E$, $g$ and $\dot\mu_{IEg}$
are as in (ii), let $J$, $M\subseteq\kappa$ be
such that $E$ and $g$ are determined by coordinates in $J$, and
$J\cap M=I$.   If
$V\in\CalBa_{\kappa}\tensorhat\CalBa_{\Bbb N}$ is $M$-determined in the
sense of (i) above, and

\Centerline{$\VVdPk\,\dot\mu_{IEg}(\vec V)=0$,}

\noindent then

\Centerline{$E^{\ssbullet}\VVdPk\,\vec g\notin\vec V$.}

\noindent\Prf\ We can suppose that $J\cup M=\kappa$, so that
$(I,J\setminus I,M\setminus I)$ is a partition of $\kappa$.   Identifying
$\{0,1\}^{\kappa}$ with
$\{0,1\}^I\times\{0,1\}^{J\setminus I}\times\{0,1\}^{M\setminus I}$,
we can find
$E'\in\CalBa_I\tensorhat\CalBa_{J\setminus I}$, a
$\CalBa_I\tensorhat\CalBa_{J\setminus I}$-measurable function
$g':\{0,1\}^I\times\{0,1\}^{J\setminus I}\to\{0,1\}^{\Bbb N}$ and
a set
$V'\in\CalBa_I\tensorhat\CalBa_{M\setminus I}\tensorhat\CalBa_{\Bbb N}$
such that

\Centerline{$E=\{(\omega_0,\omega_1,\omega_2):(\omega_0,\omega_1)\in E'$,
  $\omega_2\in\{0,1\}^{M\setminus I}\}$,}

\Centerline{$g(\omega_0,\omega_1,\omega_2)=g'(\omega_0,\omega_1)$ for all
$\omega_0\in\{0,1\}^I$, $\omega_1\in\{0,1\}^{J\setminus I}$,
$\omega_2\in\{0,1\}^{M\setminus I}$,}

\Centerline{$V=\{((\omega_0,\omega_1,\omega_2),x):
((\omega_0,\omega_2),x)\in V'$, $\omega_1\in\{0,1\}^{J\setminus I}\}$.}

\noindent The hypothesis

\Centerline{$\VVdPk\,\dot\mu_{IEg}(\vec V)=0$}

\noindent translates into `$h=0\,\,\nu_{\kappa}$-a.e.', where

\Centerline{$h(\omega)=\nu_{\kappa\setminus I}
  \{\omega':(\omega\restr I)\cup\omega'\in E$,
    $(\omega,g((\omega\restr I)\cup\omega'))\in V\}$,}

\noindent that is, identifying $\{0,1\}^{\kappa\setminus I}$ with
$\{0,1\}^{J\setminus I}\times\{0,1\}^{M\setminus I}$,

\Centerline{$h(\omega_0,\omega_1,\omega_2)
=\nu_{\kappa\setminus I}\{(\omega'_1,\omega'_2):
  (\omega_0,\omega'_1)\in E',\,
    ((\omega_0,\omega_2),g'(\omega_0,\omega'_1))\in V'\}$.}

\noindent So we see that

\Centerline{$\nu_{\kappa\setminus I}\{(\omega'_1,\omega'_2):
  (\omega_0,\omega'_1)\in E',\,
  ((\omega_0,\omega_2),g'(\omega_0,\omega'_1))\in V'\}=0$}

\noindent for $\nu_{\kappa}$-almost every $(\omega_0,\omega_1,\omega_2)$.
It follows that

\Centerline{$\nu_{J\setminus I}\{\omega'_1:
  (\omega_0,\omega'_1)\in E',\,
  ((\omega_0,\omega_2),g'(\omega_0,\omega'_1))\in V'\}=0$}

\noindent for $\nu_{\kappa}$-almost every $(\omega_0,\omega_1,\omega_2)$,
and therefore for $\nu_M$-almost every $(\omega_0,\omega_2)$, here
identifying $\{0,1\}^M$ with $\{0,1\}^I\times\{0,1\}^{M\setminus I}$.
Consequently

\Centerline{$W=\{(\omega_0,\omega'_1,\omega_2):
  (\omega_0,\omega'_1)\in E',\,
    ((\omega_0,\omega_2),g'(\omega_0,\omega'_1))\in V'\}$}

\noindent is $\nu_{\kappa}$-negligible.   But $W$ is also

$$\eqalign{&\{(\omega_0,\omega_1,\omega_2):
  (\omega_0,\omega_1,\omega_2)\in E,\,
   ((\omega_0,\omega_1,\omega_2),g(\omega_0,\omega_1,\omega_2))\in V\}\cr
&\mskip200mu
=\{\omega:\omega\in E,\,(\omega,g(\omega))\in V\}.\cr}$$

By 551Ea,
$\Bvalue{\vec g\in\vec V}=\{\omega:(\omega,g(\omega))\in V\}^{\ssbullet}$;
we have just seen that this is disjoint from $E^{\ssbullet}$ in
$\frak B_{\kappa}$, so

\Centerline{$E^{\ssbullet}\VVdPk\,\vec g\notin\vec V$,}

\noindent as required.\ \Qed

\medskip

\quad{\bf (v)} (The key.)
Once again taking $I$, $E$, $g$ and $\dot\mu_{IEg}$
as in (ii), let $J$, $M\subseteq\kappa$ be such that $J$ is countable,
$E$ and $g$ are determined by coordinates in $J$, $J\cap M=I$
and $M\setminus I$ is infinite.   Then there are an
$E'\in\CalBa_{\kappa}\setminus\Cal N(\nu_{\kappa})$ and a
$\CalBa_{\kappa}$-measurable function
$g':\{0,1\}^{\kappa}\to\{0,1\}^{\Bbb N}$, both determined by coordinates in
$M$, such that

\Centerline{$\VVdPk$ $\dot\mu_{IEg}=\dot\mu_{IE'g'}$.}

\noindent\Prf\ Because $J$ is countable, $I\subseteq M$ and
$M\setminus I$ is infinite,
there is a permutation $\alpha$ of $\kappa$ such that $\alpha(\xi)=\xi$
for every $\xi\in I$ and $\alpha[J]\subseteq M$.   Set

\Centerline{$E'
=\{\omega:\omega\in\{0,1\}^{\kappa}$, $\omega\alpha\in E\}$,
\quad$g'(\omega)=g(\omega\alpha)$ for every $\omega\in\{0,1\}^{\Bbb N}$.}

\noindent Because $\omega\mapsto\omega\alpha$ is an autohomeomorphism of
$\{0,1\}^{\kappa}$, $E'$ is a Baire set and $g'$ is Baire measurable.
If $\omega$, $\omega'\in\{0,1\}^{\kappa}$ and
$\omega\restr M=\omega'\restr M$, then
$\omega\alpha\restr J=\omega'\alpha\restr J$;  so $E'$ and $g'$ are both
determined by coordinates in $M$.

Take any $V\in\CalBa_{\kappa}\tensorhat\CalBa_{\Bbb N}$ and set

\Centerline{$h(\omega)=\nu_{\kappa\setminus I}
  \{\omega':(\omega\restr I)\cup\omega'\in E$,
    $(\omega,g((\omega\restr I)\cup\omega'))\in V\}$,}

\Centerline{$h'(\omega)=\nu_{\kappa\setminus I}
  \{\omega':(\omega\restr I)\cup\omega'\in E'$,
    $(\omega,g'((\omega\restr I)\cup\omega'))\in V\}$}

\noindent for $\omega\in\{0,1\}^{\kappa}$.
Consider the permutation
$\beta=\alpha\restr\kappa\setminus I$ of $\kappa\setminus I$, and set
$\hat\beta\omega'=\omega'\beta$ for
$\omega'\in\{0,1\}^{\kappa\setminus I}$.   Then, for any $\omega$,

$$\eqalign{\{\omega':(\omega\restr I)\cup\omega'\in E',\,
    (&\omega,g'((\omega\restr I)\cup\omega'))\in V\}\cr
&=\{\omega':(\omega\restr I)\cup\hat\beta(\omega')\in E,\,
    (\omega,g((\omega\restr I)\cup\hat\beta(\omega')))\in V\}\cr
&=\hat\beta^{-1}[\{\omega':(\omega\restr I)\cup\omega'\in E,\,
    (\omega,g((\omega\restr I)\cup\omega'))\in V\}].\cr}$$

\noindent As $\hat\beta$ is an automorphism of
$(\{0,1\}^{\kappa\setminus I},\nu_{\kappa\setminus I})$,
$\{\omega':(\omega\restr I)\cup\omega'\in E',\,
    (\omega,g'((\omega\restr I)\cup\omega'))\in V\}$ and
$\{\omega':(\omega\restr I)\cup\omega'\in E,\,
    (\omega,g((\omega\restr I)\cup\omega'))\in V\}$ have the same measure.
Thus $h=h'$, and

\Centerline{$\VVdPk\,\dot\mu_{IEg}(\vec V)
=\vec h=\vec{h'}=\dot\mu_{IE'g'}(\vec V)$.}

\noindent As $V$ is arbitrary,

\Centerline{$\VVdPk\,\dot\mu_{IEg}=\dot\mu_{IE'g'}$.}

\noindent So we have an appropriate pair $E'$, $g'$.\ \Qed

\medskip

\quad{\bf (vi)} I come at last to universally measurable sets.
Let $\dot A$ be a $\Bbb P_{\kappa}$-name such that

\Centerline{$\VVdPk\,\dot A$ is a universally measurable
subset of $\{0,1\}^{\Bbb N}$.}

\noindent Then there is a set $M\subseteq\kappa$ with cardinal $\frak c$
such that whenever $I\in[M]^{\le\omega}$,
$E\in\CalBa_{\kappa}\setminus\Cal N(\nu_{\kappa})$ and a Baire measurable
$g:\{0,1\}^{\kappa}\to\{0,1\}^{\Bbb N}$ are determined by coordinates in
$M$, then there are $M$-determined sets
$F$, $V\in\CalBa_{\kappa}\tensorhat\CalBa_{\Bbb N}$ such that

\Centerline{$\VVdPk\,\dot A\symmdiff\vec F\subseteq\vec V$ and
$\dot\mu_{IEg}\vec V=0$.}

\noindent\Prf\ If $I\in[\kappa]^{\le\omega}$,
$E\in\CalBa_{\kappa}\setminus\Cal N(\nu_{\kappa})$ and
$g:\{0,1\}^{\kappa}\to\{0,1\}^{\Bbb N}$ is Baire measurable, then

\doubleinset{$\VVdPk$ the completion of $\dot\mu_{IEg}$ measures $\dot A$,
so there are Borel sets $G$, $H\subseteq\{0,1\}^{\Bbb N}$ such that
$\dot A\symmdiff G\subseteq H$ and $\dot\mu_{IEg}H=0$.}

\noindent By 551F, as usual, there must be $F$,
$V\in\CalBa_{\kappa}\tensorhat\CalBa_{\Bbb N}$ such that

\Centerline{$\VVdPk$
$\dot A\symmdiff\vec F\subseteq\vec V$ and $\dot\mu_{IEg}\vec V=0$.}

\noindent Now there will be a countable set $K(I,E,g)\subseteq\kappa$ such
that $F$ and $V$ are both $K(I,E,g)$-determined.
If we build inductively a non-decreasing family
$\langle M_{\xi}\rangle_{\xi\le\omega_1}$ of
subsets of $\kappa$ with cardinal $\frak c$ such that whenever
$\xi<\omega_1$, $I\in[M_{\xi}]^{\le\omega}$,
$E'\in\CalBa_{\kappa}\setminus\Cal N(\nu_{\kappa})$ and a Baire measurable
$g':\{0,1\}^{\kappa}\to\{0,1\}^{\Bbb N}$ are determined by coordinates in
$M_{\xi}$, then $K(I,E,g)\subseteq M_{\xi+1}$ (which is possible because if
$\#(M_{\xi})=\frak c$ then there are just $\frak c$ possibilities for $I$,
$E$ and $g$), we shall be able to set $M=M_{\omega_1}$ to get a set of the
type we need.\ \Qed

Enumerate

\Centerline{$\{W:W\in\CalBa_{\kappa}\tensorhat\CalBa_{\Bbb N}$,
$W$ is $M$-determined$\}$}

\noindent as $\ofamily{\xi}{\frak c}{W_{\xi}}$.
Then

\Centerline{$\VVdPk\,\dot A=\bigcup\{\vec W_{\xi}:\xi<\check\frak c$,
$\vec W_{\xi}\subseteq\dot A\}$.}

\noindent\Prf\ Suppose that
$E\in\CalBa_{\kappa}\setminus\Cal N(\nu_{\kappa})$ and a $\BbbPk$-name
$\dot x$ are such that

\Centerline{$E^{\ssbullet}\VVdPk$ $\dot x\in\dot A$.}

\noindent Then there is a Baire measurable
function $g:\{0,1\}^{\kappa}\to\{0,1\}^{\Bbb N}$ such that

\Centerline{$E^{\ssbullet}\VVdPk\,\dot x=\vec g$}

\noindent (551Cc).
Let $J\in[\kappa]^{\le\omega}$ be such that $E$ and $g$ are both
determined by coordinates in $J$, and set $I=J\cap M$.
By (v) above, there are $E'$ and $g'$, both
determined by coordinates in $M$, such that

\Centerline{$\VVdPk\,\dot\mu_{IEg}=\dot\mu_{IE'g'}$.}

\noindent We therefore have $M$-determined $F$,
$V\in\CalBa_{\kappa}\tensorhat\CalBa_{\Bbb N}$ such that

\Centerline{$\VVdPk\,\dot A\symmdiff\vec F\subseteq\vec V$ and
$\dot\mu_{IEg}\vec V=\dot\mu_{IE'g'}\vec V=0$.}

\noindent Let $\xi<\frak c$ be such that $F\setminus V=W_{\xi}$.
By (iv),

\Centerline{$E^{\ssbullet}\VVdPk\,\vec g\notin\vec V$ and
$\dot x=\vec g\in\vec W_{\xi}\subseteq\dot A$.}

\noindent
As $E$ and $\dot x$ are arbitrary, we have the result.\ \Qed

Thus

\Centerline{$\VVdPk$ $\dot A$ is expressible as the union of
$\check\frak c$ Borel sets.}

\noindent As $\dot A$ is arbitrary, (a) is proved.

\medskip

{\bf (b)} Now all we have to do is count.   We surely have

\Centerline{$\VVdPk$ there are at least $\frak c$ universally measurable
subsets of $\{0,1\}^{\Bbb N}$}

\noindent just because singletons are universally measurable.   In the
other direction, because $\kappa^{\frak c}=\kappa$,

\Centerline{$\VVdPk$ $\frak c^{\check\frak c}
=(2^{\omega})^{\check\frak c}=2^{\check\frak c}=\check\kappa
=(\kappa^{\omega})\var2spcheck=\frak c$}

\noindent (552B), while (a) tells us that

\Centerline{$\VVdPk$ the number of universally measurable subsets of
$\{0,1\}^{\Bbb N}$ is at most $\frak c^{\check\frak c}$.}
}%end of proof of 553O

\exercises{\vleader{60pt}{553X}{Basic exercises (a)}(i)
%\spheader 553Xa
Suppose that $A\subseteq\{0,1\}^{\Bbb N}$ has Rothberger's
property, and that $\kappa$ is a cardinal.   Show that

\Centerline{$\VVdPk\,\check A$ has Rothberger's property in
$\{0,1\}^{\Bbb N}$.}

\noindent (ii) Repeat with $\Bbb R$ in place of $\{0,1\}^{\Bbb N}$.
(iii)\dvAnew{2010} Suppose that $\frak m=\frak c>\omega_1$.   Show that

\Centerline{$\Vdash_{\Bbb P_{\omega_1}}$ there is a set of strong measure
zero in $\Bbb R$ with cardinal greater than $\frakmctbl$.}
%552O 553B

\spheader 553Xb Let $W\subseteq\{0,1\}^{\omega}\times\{0,1\}^{\omega}$ be the
set

\Centerline{$\{(x,y):x(2n)=y(2n)$ for every $n\in\Bbb N\}$.}

\noindent Show that, for every $y\in\{0,1\}^{\omega}$,

\Centerline{$\VVdash_{\Bbb P_{\omega}}\,\vec W$ is homeomorphic to
$\{0,1\}^{\omega}$ and $\check y\notin\vec W$.}
%553E

\spheader 553Xc(i) Suppose that $\Cal F$ is a $p$-point filter on
$\Bbb N$, and that $\Bbb P$ is a ccc forcing notion.   Show
that

\Centerline{$\VVdP$ the filter on $\Bbb N$ generated by $\check{\Cal F}$
is a $p$-point filter.}

\noindent (ii) Suppose that $\Cal F$ is a rapid filter on $\Bbb N$,
and that $\kappa$ is a cardinal.   Show that

\Centerline{$\VVdPk$ the filter on $\Bbb N$ generated by $\check{\Cal F}$
is a rapid filter.}
%553H out of order query

\spheader 553Xd Let $\frak A$ be a Boolean algebra and
$\nu:\frak A\to\coint{0,\infty}$ a non-negative additive functional.   Show that
if $\familyiI{a_i}$ is a finite family in $\frak A$ then

\Centerline{$\nu(\sup_{i\in I}a_i)
\le\sum_{k=1}^m(-1)^{k+1}\sum_{J\in[I]^k}\nu(\inf_{i\in J}a_i)$
if $m\ge 1$ is odd,}

\Centerline{$\nu(\sup_{i\in I}a_i)
\ge\sum_{k=1}^m(-1)^{k+1}\sum_{J\in[I]^k}\nu(\inf_{i\in J}a_i)$
if $m\ge 1$ is even.}
%553K

\spheader 553Xe Let $\Bbb P$ be a forcing notion which satisfies Knaster's
condition.   (i) Show that if
$(P,\le)$ is an upwards-ccc partially ordered set then

\Centerline{$\VVdP\,(\check P,\check{\le})$ is upwards-ccc.}

\noindent (ii) Show that if $(T,\le)$ is a Souslin tree then

\Centerline{$\VVdP\,(\check T,\check{\le})$ is a Souslin tree.}
%553M

\leader{553Y}{Further exercises (a)}
%\spheader 553Ya
Let $\kappa$ be a cardinal, $\dot G$ a $\BbbPk$-name and
$a\in\frak B_{\kappa}^+$ such that

\Centerline{$a\VVdPk\,\dot G$ is a dense open subset of
$\{0,1\}^{\omega}$.}

\noindent Show that there is a
$W\in\Tau_{\kappa}\tensorhat\CalBa_{\omega}$ such that every vertical
section of $W$ is a dense open set and $a\VVdPk\,\dot G=\vec W$.
%551E

\spheader 553Yb Let $\kappa$ be a cardinal and
$W\in\Tau_{\kappa}\tensorhat\CalBa_{\omega}$ a set such that every
vertical section of $W$ is a dense open set.   Let $C$ be the space of
continuous functions from $\{0,1\}^{\kappa}$ to $\{0,1\}^{\omega}$ with the
compact-open topology (definition:  441Yh).   Show that
$\{f:f\in C$, $\{x:(x,f(x))\in W\}$ is conegligible$\}$ is comeager in $C$.

\spheader 553Yc Show that

\Centerline{$\VVdash_{\Bbb P_{\omega}}
\,\frakmctbl\ge(\frakmctbl)\var2spcheck$.}

\noindent\Hint{work with the ideal of meager sets in the Polish space of
continuous functions from $\{0,1\}^{\omega}$ to itself.}
%553Ya 553Yb

\spheader 553Yd Let $\Cal K$ be the family of compact well-ordered subsets
of $\Bbb Q\cap\coint{0,\infty}$ containing $0$.
For $s$, $t\in\Cal K$ say that
$s\preccurlyeq t$ if $s=t\cap[0,\gamma]$ for some $\gamma\in\Bbb R$;  for
$s\in\Cal K$ and $\gamma\in\Bbb Q$, set
$A(s,\gamma)=\{t:t\in\Cal K$, $\max t=\gamma$, $s\preccurlyeq t\}$.
(i) Show that $(\Cal K,\preccurlyeq)$ is a tree, and that $\otp(t)=r(t)+1$ for
every $t\in\Cal K$.
(ii) Choose $\ofamily{\xi}{\omega_1}{\Cal K_{\xi}}$,
$\ofamily{\xi}{\omega_1}{T_{\xi}}$ inductively so that
$\Cal K_0=T_0=\{\{0\}\}$ and for $0<\xi<\omega_1$

\inset{$\Cal K_{\xi}=\{t:t\in \Cal K$, $r(t)=\xi$,
  $s\in\bigcup_{\eta<\xi}T_{\eta}$ whenever $s\prec t\}$,

$T_{\xi}\subseteq\Cal K_{\xi}$ is countable,

if $\eta<\xi$, $s\in T_{\eta}$ and $\gamma\in\Bbb Q$
are such that $\gamma>\max s$ and $A(s,\gamma)$ meets $\Cal K_{\xi}$, then
$A(s,\gamma)$ meets $T_{\xi}$.}

\noindent Show that if $\eta<\xi<\omega_1$, $s\in T_{\eta}$,
$\gamma\in\Bbb Q$ and $\gamma>\max s$, there is a $t\in T_{\xi}$ such that
$\max t=\gamma$ and $s\preccurlyeq t$.
(iii) Show that $T=\bigcup_{\xi<\omega_1}T_{\xi}$ is a
special Aronszajn tree.
%\Hint{for any $q\in\Bbb Q$, $\{t:\max t=q\}$ is an antichain.}
%553M

\spheader 553Ye Show that
$\VVdash_{\Bbb P_{\omega}}\,
\frak p\ge(\frak m_{\sigma\text{-linked}})\var2spcheck$.
%mt55bits
}%end of exercises

\leader{553Z}{Problem} Suppose that the generalized continuum hypothesis is
true.   Is it the case that

\Centerline{$\VVdash_{\Bbb P_{\omega_2}}$ there is a Borel lifting for
Lebesgue measure?}

\cmmnt{\noindent(Compare 554I.)}

\endnotes{
\Notesheader{553}
To my mind, the chief interest of the results of this section is that
they force us to explore aspects of the structures considered in new ways.
We know, for instance, that if a set has Rothberger's property
(in a separable metrizable space) this can be witnessed by a family of
$\frak d$ sequences.   The point of
553C is that (in random real models) any family of $\frak d$ sequences is
associated with a set $\dot Y$ of size at most the cardinal power
$\frak d^{\omega}=\frak c$ (taken in the ordinary universe $V$),
such that $\dot Y$ must include the given set with Rothberger's property.
Remember that

\Centerline{$\VVdPk\,(\BbbN^{\Bbb N})\var2spcheck$ is cofinal with
$\BbbN^{\Bbb N}$}

\noindent (see the proof of 552C),
so there is no point in looking at `new' members of
$\BbbN^{\Bbb N}$ in part (a) of the proof.

In 553E, we need to distinguish between the $\Bbb P_{\kappa}$-names
$\tilde G$ and $\check G$.   It is quite possible to have

\Centerline{$\VVdPk\,\dot K\cap(\{0,1\}^{\lambda})\var2spcheck=\emptyset$;}

\noindent that is, we might have $\VVdPk\,\dot K=\vec W$ where
$W\subseteq\{0,1\}^{\kappa}\times\{0,1\}^{\lambda}$ has negligible
horizontal sections (553Xb).
The name $\tilde G$ refers not to a copy of the
set $G$ but to a re-interpretation of one (or any) of its descriptions as
an F$_{\sigma}$ set.

In 553H and 553M, we have to look quite deeply into the structure of
measure algebras.   Lemmas 553G and 553L are already not obvious,
and the combinatorial measure theory of the
proof of 553H is delicate.   553J is easier.
The idea here is to `randomize' a construction from {\smc Galvin 80},
where the continuum hypothesis was used to
build complementary sets $S_0$, $S_1$ with the property of 553I.

I give a bit of space to `Aronszajn trees' because the results here express yet
another contrast between random and Cohen forcing.   Cohen forcing creates
Souslin trees (554Yc).
Random forcing preserves old Souslin trees (553Xe) but does not
necessarily produce new ones (553M).

}%end of notes

\leaveitout{\leader{}{} Do random reals give $\shr(\CalSmz)=\omega_1$ or
something?

\leader{}{Problem} If the Borel conjecture is true does it follow that

\Centerline{$\VVdPk$ the Borel conjecture is true}

\noindent for every infinite $\kappa$?  (equivalently, for
$\kappa=\omega$).

See {\smc Bartoszy\'nski \& Judah 95}, 8.3B.
}%end of leaveitout


\discrpage
