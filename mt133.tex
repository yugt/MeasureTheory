\frfilename{mt133.tex} 
\versiondate{29.3.10} 
\copyrightdate{1994} 
      
\def\chaptername{Complements} 
\def\sectionname{Wider concepts of integration} 
      
\newsection{133} 
      
There are various contexts in which it is useful to be able to assign a 
value to the integral of a function which is not quite covered by the 
basic definition in 122M.   In this section I offer suggestions 
concerning the assignment of the values $\pm\infty$ to integrals of 
real-valued functions (133A), the integration 
of complex-valued functions (133C-133H) and upper and lower integrals 
(133I-133L).   In \S135 below I will discuss a further elaboration of 
the ideas of Chapter 12. 
      
\leader{133A}{Infinite integrals} It is normal to restrict the phrase 
`$f$ is integrable' to functions $f$ to which a finite integral 
$\int f$ can 
be assigned\cmmnt{ (just as a series is called `summable' only when 
a finite 
sum can be assigned to it)}.   But for non-negative functions it is 
sometimes convenient to write `$\int f=\infty$' if, in some sense, the 
only way in which $f$ fails to be integrable is that the integral is too 
large;  that is, $f$ is defined almost everywhere, is $\mu$-virtually 
measurable, and either 
      
\Centerline{$\{x:x\in\dom f,\,f(x)\ge\epsilon\}$} 
      
\noindent includes a set of infinite measure for some $\epsilon>0$, or 
      
\Centerline{$\sup\{\int h:h\text{ is simple},\,h\leae f\}=\infty$.} 
      
\noindent\cmmnt{(Compare 122J.) }Under this rule,\cmmnt{ we shall still 
have}
      
\Centerline{$\int f_1+f_2=\int f_1+\int f_2$, 
\quad$\int cf=c\int f$} 
      
\noindent whenever $c\in\coint{0,\infty}$ and $f_1$, $f_2$, $f$ are 
non-negative functions for which $\int f_1$, $\int f_2$, $\int f$ are 
defined in $[0,\infty]$. 
      
We can therefore\cmmnt{ repeat the definition 122M and} say that 
      
\Centerline{$\int f_1-f_2=\int f_1-\int f_2$} 
      
\noindent whenever $f_1$, $f_2$ are real-valued functions such that 
$\int f_1$, $\int f_2$ are defined in $[0,\infty]$ and are not both 
infinite\cmmnt{;  the last condition being imposed to avoid the 
possibility of being asked to calculate $\infty-\infty$}. 
      
We still have the rules that 
      
\Centerline{$\int f+g=\int f+\int g$, 
\quad$\int(cf)=c\int f$, 
\quad$\int|f|\ge|\int f|$} 
      
\noindent at least when the right-hand-sides can be interpreted, 
allowing $0\cdot\infty=0$, but not allowing any interpretation of 
$\infty-\infty$;  and $\int f\le\int g$ whenever both integrals are 
defined and $f\leae g$.   \cmmnt{(But of course it is now possible 
to have $f\le g$ and $\int f=\int g=\pm\infty$ without $f$ and $g$  
being equal almost everywhere.)} 
      
Setting $f^+(x)=\max(f(x),0)$, $f^-(x)=\max(-f(x),0)$ for $x\in\dom f$, 
then 
      
\Centerline{$\int f=\infty\,\iff\,\int f^+=\infty$ and $f^-$ is 
integrable,} 
      
\Centerline{$\int f=-\infty\,\iff\,f^+$ is integrable and 
$\int f^-=\infty$.} 
 
\cmmnt{For further ideas in this direction, see \S135 below.} 
          
\leader{133B}{Functions with exceptional values} It is also convenient 
to allow as `integrable' functions $f$ which take occasional values 
which are not real\cmmnt{ -- typically, where a formula for $f(x)$ 
allows the 
value `$\infty$' on some convention}.   For such a function I will 
write $\int f=\int\tilde f$ if $\int\tilde f$ is defined, where 
      
\Centerline{$\dom\tilde f=\{x:x\in\dom f,\,f(x)\in\Bbb R\}$, 
\quad$\tilde f(x)=f(x)$ for $x\in\dom\tilde f$.} 
      
\cmmnt{\noindent Since in this convention I still require $\tilde f$ 
to be 
defined almost everywhere in $X$, the set $\{x:x\in\dom 
f,\,f(x)\notin\Bbb R\}$ will have to be negligible.} 
      
\cmmnt{ 
\leader{133C}{Complex-valued functions} All the theory of 
measurable and 
integrable functions so far developed has been devoted to real-valued 
functions.   There are no substantial new ideas required to deal with 
complex-valued functions, but perhaps I should spell out some of the 
details, since 
there are many applications in which complex-valued functions are the 
most natural context in which to work. 
}%end of comment 
      
\leader{133D}{Definitions (a)} Let $X$ be a set and $\Sigma$ a 
$\sigma$-algebra of subsets of $X$.   If $D\subseteq X$ and 
$f:D\to\Bbb C$ is a function, then we 
say that $f$ is {\bf measurable} if its real and imaginary parts 
$\Real f$, $\Imag f$ are measurable\cmmnt{ in the sense of 121B-121C}. 
      
\medskip 
      
{\bf (b)} Let $(X,\Sigma,\mu)$ be a measure space.   If $f$ is a 
complex-valued function defined on a conegligible 
subset of $X$, we say that $f$ is {\bf integrable} if its real and 
imaginary parts are integrable, and then 
      
\Centerline{$\int f=\int\Real f+i\int\Imag f$.} 
      
\medskip 
      
{\bf (c)} Let $(X,\Sigma,\mu)$ be a measure space, $H\in\Sigma$ and $f$ 
a complex-valued function defined on a subset of $X$.   Then 
$\int_Hf$ is $\int(f\restr H)d\mu_H$ if this is defined in the sense of 
(b), taking the subspace measure $\mu_H$ to be that of 131A-131B. 
      
\ifnum\stylenumber=11\ifresultsonly\eject\fi\fi 
      
\leader{133E}{Lemma} (a) If $X$ is a set, $\Sigma$ is a $\sigma$-algebra 
of subsets of $X$, and $f$ and $g$ are measurable complex-valued 
functions with domains $\dom f$, $\dom g\subseteq X$, then 
      
\quad(i) $f+g:\dom f\cap\dom g\to\Bbb C$ is measurable; 
      
\quad(ii) $cf:\dom f\to\Bbb C$ is measurable, for every $c\in\Bbb C$; 
      
\quad(iii) $f\times g:\dom f\cap\dom g\to\Bbb C$ is measurable; 
      
\quad(iv) $f/g:\{x:x\in\dom f\cap\dom g,\,g(x)\ne 0\}\to\Bbb C$ is 
measurable; 
      
\quad(v) $|f|:\dom f\to\Bbb R$ is measurable. 
      
{(b)} If $\sequencen{f_n}$ is a sequence of measurable 
complex-valued functions defined on subsets of $X$, then 
$f=\lim_{n\to\infty}f_n$ is measurable, if we take $\dom f$ to be 
      
$$\eqalign{\{x:x\in\bigcup_{n\in\Bbb N}&\bigcap_{m\ge n}\dom f_m, 
  \,\lim_{n\to\infty}f_n(x)\text{ exists in }\Bbb C\}\cr 
&=\dom(\lim_{n\to\infty}\Real f_n) 
  \cap\dom(\lim_{n\to\infty}\Imag f_n).\cr}$$ 
      
\proof{{\bf (a)} All are immediate from 121E, if you write down 
the formulae for the real and imaginary parts of $f+g,\ldots,|f|$ in 
terms of the real and imaginary parts of $f$ and $g$. 
      
\medskip 
      
{\bf (b)} Use 121Fa. 
}%end of proof of 133E 
      
\leader{133F}{Proposition} Let $(X,\Sigma,\mu)$ be a measure space. 
      
(a) If $f$ and $g$ are integrable 
complex-valued functions defined 
on conegligible subsets of $X$, then $f+g$ and $cf$ are integrable, 
$\int f+g=\int f+\int g$ and $\int cf=c\int f$, for 
every $c\in\Bbb C$.
      
{(b)} If $f$ is a complex-valued function defined on a conegligible 
subset of $X$, then $f$ is integrable iff $|f|$ is integrable and $f$ is 
$\mu$-virtually measurable\cmmnt{, that is, $\Real f$ and $\Imag f$ are
$\mu$-virtually measurable}. 
      
\proof{{\bf (a)} Use 122Oa-122Ob. 
      
\medskip 
      
{\bf (b)} The point is that $|\Real f|$, 
$|\Imag f|\le|f|\le|\Real f|+|\Imag f|$;  now we need only apply 122P an 
adequate number of times. 
}%end of proof of 133F 
      
\leader{133G}{Lebesgue's Dominated Convergence Theorem} Let 
$(X,\Sigma,\mu)$ 
be a measure space and $\langle f_n\rangle_{n\in\Bbb N}$ a sequence of 
integrable complex-valued functions on $X$ such that 
$f(x)=\lim_{n\to\infty}f_n(x)$ 
exists in $\Bbb C$ for almost every $x\in X$.   Suppose moreover that 
there is a real-valued 
integrable function $g$ on $X$ such that $|f_n|\leae g$ for each 
$n$.   Then $f$ is integrable and $\lim_{n\to\infty}\int f_n$ exists and is equal to $\int f$. 
      
\proof{Apply 123C to the sequences $\sequencen{\Real f_n}$ and 
$\sequencen{\Imag f_n}$. 
}%end of proof of 133G 
      
\leader{133H}{Corollary} Let $(X,\Sigma,\mu)$ be a measure space and 
$\ooint{a,b}$ a non-empty open interval in $\Bbb R$.   Let 
$f:X\times\ooint{a,b}\to\Bbb C$ be a function such that 
      
\inset{(i) the integral $F(t)=\int f(x,t)dx$ is defined for every 
$t\in\ooint{a,b}$;} 
      
\inset{(ii) the partial derivative $\pd{f}{t}$ of $f$ with respect to 
the second variable is defined everywhere in $X\times\ooint{a,b}$;} 
      
\inset{(iii) there is an integrable function $g:X\to\coint{0,\infty}$ 
such that $|\pd{f}{t}(x,t)|\le g(x)$ for every $x\in X$, $t\in\ooint{a,b}$.} 
      
\noindent Then the derivative $F'(t)$ and the integral 
$\biggerint\pd{f}{t}(x,t)dx$ exist for every $t\in\ooint{a,b}$, 
and are equal. 
      
\proof{Apply 123D to $\Real f$ and $\Imag f$.} 
      
      
\leader{133I}{Upper and lower integrals}\dvArevised{2007}\cmmnt{ I return  
now to real-valued functions.}   Let $(X,\Sigma,\mu)$ be a 
measure space and $f$ a real-valued function defined almost everywhere 
in $X$.   Its {\bf upper integral} is 
      
\Centerline{$\overlineint f 
=\inf\{\int g:\int g$ is defined in the sense of 133A and $f\leae g\}$,} 
      
\noindent allowing $\infty$ for $\inf\{\infty\}$ or $\inf\emptyset$ 
and $-\infty$ for 
$\inf\Bbb R$.   Similarly, the {\bf lower integral} of $f$ is 
      
\Centerline{$\underlineint f 
=\sup\{\int g:\int g$ is defined, $f\geae g\}$,} 
      
\noindent allowing $-\infty$ for $\sup\{-\infty\}$ or $\sup\emptyset$ 
and $\infty$ for $\sup\Bbb R$. 
      
\leader{133J}{Proposition}\dvArevised{2007} Let $(X,\Sigma,\mu)$ be a  
measure space. 
      
(a) Let $f$ be a real-valued function defined almost everywhere in $X$. 
      
\quad (i) If $\overline{\int}f$ is finite, then there is 
an integrable $g$ such that $f\leae g$ and $\int g=\overline{\int}f$. 
In this case, 
 
\Centerline{$\{x:x\in\dom f\cap\dom g$, $g(x)\le f(x)+g_0(x)\}$} 
 
\noindent has full outer measure for every measurable function 
$g_0:X\to\ooint{0,\infty}$. 
      
\quad (ii) If $\underline{\int}f$ is finite, 
then there is an integrable $h$ such that $h\leae f$ and 
$\int h=\underline{\int}f$.   In this case, 
 
\Centerline{$\{x:x\in\dom f\cap\dom h$, $f(x)\le h(x)+h_0(x)\}$} 
 
\noindent has full outer measure for every measurable function 
$h_0:X\to\ooint{0,\infty}$. 
      
{(b)} For any real-valued functions $f$, $g$ defined on conegligible 
subsets of $X$ and any $c\ge 0$, 
      
\vskip 2pt 
      
\quad(i) $\underline{\int}f\le\overline{\int}f$, 
      
\vskip2pt 
      
\quad (ii) $\overline{\int}f+g\le\overline{\int}f+\overline{\int}g$, 
      
\vskip2pt 
      
\quad (iii) $\overline{\int}cf=c\overline{\int}f$, 
      
\vskip2pt 
      
\quad (iv) $\underline{\int}(-f)=-\overline{\int}f$, 
      
\vskip2pt 
      
\quad (v) $\underline{\int}f+g\ge\underline{\int}f+\underline{\int}g$, 
      
\vskip2pt 
      
\quad (vi) $\underline{\int}cf=c\underline{\int}f$ 
      
\vskip2pt 
      
\noindent whenever the right-hand-sides do not involve adding $\infty$ 
to $-\infty$. 
      
(c) If $f\leae g$ then $\overline{\int}f\le\overline{\int}g$ and 
$\underline{\int}f\le\underline{\int}g$. 
      
(d) A real-valued function $f$ defined almost everywhere in $X$ is 
integrable iff 
      
\Centerline{$\overlineint f=\underlineint f=a\in\Bbb R$,} 
      
\noindent and in this case $\int f=a$. 
      
(e) $\mu^*A=\overline{\int}\chi A$ for every $A\subseteq X$. 
 
\proof{{\bf (a)(i)} For each 
$n\in\Bbb N$, choose a function $g_n$ such that $f\leae g_n$ 
and $\int g_n$ is defined and at most $2^{-n}+\overline{\int}f$;  as 
$\overline{\int}f\le\int g_n$, $\int g_n$ is finite,  
so $g_n$ is integrable.   Set  
$h_n=\inf_{i\le n}g_i$ for each $n$;  then $h_n$ is integrable (because 
$|h_n-g_0|\le\sum_{i=0}^n|g_i-g_0|$ on $\bigcap_{i\le n}\dom g_i$), and 
$f\leae h_n$, so 
      
\Centerline{$\overlineint f\le\int h_n 
\le\int g_n\le 2^{-n}+\overlineint f$.} 
      
\noindent By B.Levi's theorem (123A), applied to $\sequencen{-h_n}$, 
$g(x)=\inf_{n\in\Bbb N}h_n(x)\in\Bbb R$ for almost every $x$, and 
$\int g=\inf_{n\in\Bbb N}\int h_n=\overline{\int}f$;  also, of course, 
$f\leae g$. 
 
Now take a measurable function $g_0:X\to\ooint{0,\infty}$, 
and consider the set 
 
\Centerline{$A=\{x:x\in\dom f\cap\dom g$, $g(x)\le f(x)+g_0(x)\}$.} 
 
\noindent\Quer\ If $A$ does not have full outer measure, there is a 
non-negligible measurable set $F\subseteq X\setminus A$.   Since $g_0$ is 
strictly positive, $F=\bigcup_{n\in\Bbb N}F_n$ where  
$F_n=\{x:x\in F$, $g_0(x)\ge 2^{-n}\}$,  
and there is an $n\in\Bbb N$ such that $\mu F_n>0$.   Consider the 
function $g_1=g-2^{-n}\chi F$.   Then $f\leae g_1$.   Also  
$\int g_1=\int g-2^{-n}\mu F_n$ is 
strictly less than $\int g$, so $\overline{\int}f<\int g$.\ \Bang 
 
\medskip 
      
\quad{\bf (ii)} Argue similarly, or use (b-iv). 
      
\medskip 
      
{\bf (b)(i)} If either $\underline{\int}f=-\infty$ or 
$\overline{\int}f=\infty$ this is trivial.   Otherwise it follows at 
once from the fact that if $g\leae f\leae h$ 
then $\int g\le\int h$ if the integrals are defined (in the wide sense). 
      
\medskip 
      
\quad{\bf (ii)} If $a>\overline{\int}f+\overline{\int}g$, neither 
$\overline{\int}f$ nor $\overline{\int}g$ can be $\infty$, so there must be  
functions $f_1$, $g_1$ such that $f\leae f_1$, 
$g\leae g_1$ and $\int f_1+\int g_1\le a$.   Now $f+g\leae f_1+g_1$, so 
      
\Centerline{$ 
\overlineint f+g\le\int f_1+g_1\le a$.} 
      
\noindent As $a$ is arbitrary, we have the result. 
      
\medskip 
      
\quad{\bf (iii)($\pmb{\alpha}$)} If $c=0$ this is trivial.   {\bf 
($\pmb{\beta}$)} If $c>0$ and $a>c\overline{\int}f$, there must be an 
$f_1$ such that $f\leae f_1$ and $c\int f_1\le a$.   Now 
$cf\leae cf_1$ and $\int cf_1\le a$, so $\overline{\int}cf\le a$. 
As $a$ is arbitrary, $\overline{\int}cf\le c\overline{\int}f$. 
{\bf($\pmb{\gamma}$)}  Still supposing that $c>0$, we also have 
      
\Centerline{$c\overlineint f=c\overlineint c^{-1}cf 
\le cc^{-1}\overlineint cf=\overlineint cf$,} 
      
\noindent so we get equality. 
      
\medskip 
      
\quad{\bf (iv)} This is just because $\int(-f_1)=-\int f_1$ for any 
function $f_1$ for which either integral is defined. 
      
\medskip 
      
\quad{\bf (v)-(vi)} Use (iv) to turn $\underline{\int}$ into 
$\overline{\int}$, and apply (ii) or (iii). 
      
\medskip 
      
{\bf (c)} These are immediate from the definitions, because (for 
instance) if $g\leae h$ then $f\leae h$. 
      
\medskip 
      
{\bf (d)} If $f$ is integrable, then 
      
\Centerline{$\overlineint f=\int f=\underlineint f$} 
      
\noindent by 122Od.   If 
$\overline{\int}f=\underline{\int}f=a\in\Bbb R$, then, by (a), there are  
integrable $g$, $h$ such that $g\leae f\leae h$ 
and $\int g=\int h=a$, so that $g\eae h$, by 122Rc, $g\eae f\eae h$  
and $f$ is integrable, by 122Rb. 
 
\medskip 
 
{\bf (e)} If $E\supseteq A$ is measurable, then  
 
\Centerline{$\mu E=\int\chi E\ge\overlineint\chi A$;} 
 
\noindent as $E$ is arbitrary, $\mu^*A\ge\overline{\int}\chi A$.    
If $\int g$ is defined and $\chi A\leae g$, let $E\subseteq\dom g$ be a  
conegligible measurable set such that $g\restr E$ is measurable, and set  
$F=\{x:x\in E$, $g(x)\ge 1\}$.   Then $A\setminus F$ is negligible, so  
$\mu^*A\le\mu F\le\int g$;  as $g$ is arbitrary,  
$\mu^*A\le\overline{\int}\chi A$. 
}%end of proof of 133J 
      
\cmmnt{\medskip 
      
\noindent{\bf Remark} I hope that the formulae here remind you of 
$\limsup$, $\liminf$.} 
      
\leader{133K}{Convergence theorems for upper 
\dvrocolon{integrals}}\cmmnt{ We have the 
following versions of B.Levi's theorem and Fatou's Lemma. 
      
\medskip 
      
\noindent}{\bf Proposition} Let $(X,\Sigma,\mu)$ be a measure space, and 
$\sequencen{f_n}$ a sequence of real-valued functions defined almost 
everywhere in $X$. 
      
(a) If, for each $n$, $f_n\leae f_{n+1}$, and 
$-\infty<\sup_{n\in\Bbb N}\overline{\int}f_n<\infty$, then 
$f(x)=\sup_{n\in\Bbb N}f_n(x)$ is defined in $\Bbb R$ for almost every 
$x\in X$, and 
$\overline{\int}f=\sup_{n\in\Bbb N}\overline{\int}f_n$. 
      
(b) If, for each $n$, $f_n\ge 0$ a.e., and 
$\liminf_{n\to\infty}\overline{\int}f_n<\infty$, then 
$f(x)=\liminf_{n\to\infty}f_n(x)$ is defined in $\Bbb R$ for almost 
every $x\in X$, and 
$\overline{\int}f\le\liminf_{n\to\infty}\overline{\int}f_n$. 
      
\proof{{\bf (a)} Set $c=\sup_{n\in\Bbb N}\overline{\int}f_n$.    For 
each $n$, there is an integrable function $g_n$ such that $f_n\leae g_n$ 
and $\int g_n=\overline{\int}f_n$ (133J(a-i)).   
Set $g'_n=\min(g_n,g_{n+1})$; 
then $g'_n$ is integrable and $f_n\leae g'_n\leae g_n$, so 
      
\Centerline{$\overlineint f_n\le\int g'_n 
\le\int g_n=\overlineint f_n$} 
      
\noindent and $g'_n$ must be equal to $g_n$ a.e.   Consequently 
$g_n\leae g_{n+1}$, for each $n$, while 
$\sup_{n\in\Bbb N}\int g_n=c<\infty$. 
By B.Levi's theorem, $g=\sup_{n\in\Bbb N}g_n$ is defined, as a 
real-valued function, almost everywhere in $X$, and $\int g=c$.   Now of 
course $f(x)$ is defined, and not greater than $g(x)$, for any 
$x\in\dom g\cap\bigcap_{n\in\Bbb N}\dom f_n$ such that $f_n(x)\le g_n(x)$ 
for every $n$, that is, for almost every $x$;  so 
$\overline{\int}f\le\int g=c$.   On the other hand, $f_n\leae f$, so $\overline{\int}f_n\le\overline{\int}f$, 
for every $n\in\Bbb N$;  it follows that $\overline{\int}f$ must be at 
least $c$, and is therefore equal to $c$, as required. 
      
\medskip 
      
{\bf (b)} The argument follows that of 123B.   Set 
$c=\liminf_{n\to\infty}\overline{\int}f_n$.   For each $n$, set 
$g_n=\inf_{m\ge n}f_n$;  then  
$\overline{\int}g_n\le\inf_{m\ge n}\overline{\int}f_m\le c$.    
We have $g_n(x)\le g_{n+1}(x)$ for every 
$x\in\dom g_n$, that is, almost everywhere, for each $n$;  so, by (a), 
      
\Centerline{$\overlineint g 
=\sup_{n\in\Bbb N}\overlineint g_n\le c$,} 
      
\noindent where 
      
\Centerline{$g=\sup_{n\in\Bbb N}g_n\eae\liminf_{n\to\infty}f_n$,} 
      
\noindent and $\overline{\int}\liminf_{n\to\infty}f_n\le c$, as claimed. 
}%end of proof of 133K 
 
\leader{*133L}{}\dvAnew{2007}\cmmnt{ The following is at a less fundamental 
level than the results in 133J, but is still important. 
 
\medskip 
 
\noindent}{\bf Proposition}  
Let $(X,\Sigma,\mu)$ be a measure space and $f$ a 
real-valued function defined almost everywhere in $X$. 
Suppose that $h_1$, $h_2$ are non-negative virtually measurable 
functions defined almost everywhere in $X$.   Then  
 
\Centerline{$\overlineint f\times(h_1+h_2) 
=\overlineint f\times h_1+\overlineint f\times h_2$,}

\noindent where here, for once, we can interpret
$\infty+(-\infty)$ or $(-\infty)+\infty$ as $\infty$ if called for on
the right-hand side. 
 
\proof{{\bf (a)} If either $\overline{\int}f\times h_1=\infty$ or 
$\overline{\int}f\times h_2=\infty$ then
$\overline{\int}f\times(h_1+h_2)=\infty$.   \Prf\Quer\ Otherwise,
there is a $g$ such that $f\times(h_1+h_2)\leae g$ and $\int g<\infty$.   
In this case, 

\Centerline{$f\times h_1
\leae f^+\times h_1
\leae f^+\times(h_1+h_2)
=(f\times(h_1+h_2))^+
\leae g^+$}

\noindent so $\overline{\int}f\times h_1\le\int g^+<\infty$.   Similarly,
$\overline{\int}f\times h_2<\infty$;  contradicting our hypothesis.\
\Bang\QeD\   So in this case, under the local rule
$\infty+(-\infty)=(-\infty)+\infty=\infty$, we have the result.

\medskip

{\bf (b)} Now suppose that the upper integrals
$\overline{\int}f\times h_1$ and 
$\overline{\int}f\times h_2$ are both less than $\infty$, 
so that their sum can be interpreted by the usual rules.
By 133J(b-ii),  
$\overline{\int}f\times(h_1+h_2) 
\le\overline{\int}f\times h_1+\overline{\int}f\times h_2<\infty$.   
In the other direction, suppose that $g\geae f\times(h_1+h_2)$ 
and $\int g<\infty$.   For $i=1$, $2$ set 
 
$$\eqalign{g_i(x) 
&=\Bover{g(x)h_i(x)}{h_1(x)+h_2(x)} 
  \text{ if }x\in\dom g\cap\dom h_1\cap\dom h_2 
  \text{ and }h_1(x)+h_2(x)>0,\cr 
&=0\text{ for other }x\in X.\cr}$$ 
 
\noindent Then, for both $i$, $g_i$ is virtually measurable, 
$g_i^+\leae g^+$ and $g_i\geae f\times h_i$;  while $g\geae g_1+g_2$. 
\Prf\ The set 
 
\Centerline{$H=\{x:x\in\dom f\cap\dom g\cap\dom h_1\cap\dom h_2$, 
$g(x)\ge f(x)(h_1(x)+h_2(x))\}$} 
 
\noindent is conegligible, and for $x\in H$ 
 
$$\eqalign{g(x) 
&=g_1(x)+g_2(x)\text{ if }h_1(x)+h_2(x)>0,\cr 
&\ge 0=g_1(x)+g_2(x)\text{ if }h_1(x)+h_2(x)=0.  \text{ \Qed}\cr}$$ 
 
\noindent So 
 
\Centerline{$\overlineint f\times h_1+\overlineint f\times h_2 
\le\int g_1+\int g_2 
=\int g_1+g_2
\le\int g$} 
 
\noindent (because $\int g_1$ and $\int g_2$ are both at most 
$\int g^+<\infty$, so we can add them on the usual rules).   
As $g$ is arbitrary, 
$\overline{\int}f\times h_1+\overline{\int}f\times h_2 
\le\overline{\int}f\times(h_1+h_2)$ and we must have equality. 
}%end of proof of 133L 
 
\exercises{ 
\leader{133X}{Basic exercises $\pmb{>}$(a)} 
%\spheader 133Xa 
Let $(X,\Sigma,\mu)$ be a measure space, and $f:X\to\coint{0,\infty}$ a  
measurable function.   Show that 
      
$$\eqalign{\int fd\mu 
&=\sup_{n\in\Bbb N}2^{-n}\sum_{k=1}^{4^n}\mu\{x:f(x)\ge 2^{-n}k\}\cr 
&=\lim_{n\to\infty}2^{-n}\sum_{k=1}^{4^n}\mu\{x:f(x)\ge 2^{-n}k\}\cr}$$ 
      
\noindent in $[0,\infty]$. 
%133A 
      
\spheader 133Xb 
Let $(X,\Sigma,\mu)$ be a measure space and $f$ 
a complex-valued function defined on a subset of $X$.   (i) Show that if 
$E\in\Sigma$, then $f\restr E$ is $\mu_E$-integrable iff $\tilde f$ is 
$\mu$-integrable, writing $\mu_E$ for the subspace measure on $E$ and 
$\tilde f(x)=f(x)$ if $x\in E\cap\dom f$, $0$ if $x\in X\setminus E$; 
and in this case $\int_Efd\mu_E=\int \tilde fd\mu$.   (ii) Show that if 
$E\in\Sigma$ and $f$ is defined $\mu$-almost everywhere, then 
$f\restr E$ is 
$\mu_E$-integrable iff $f\times\chi E$ is $\mu$-integrable, and in this 
case $\int_Ef=\int f\times\chi E$.   (iii) Show that if $\int_Ef=0$ for 
every $E\in\Sigma$, then $f=0$ a.e. 
%133E 
      
\spheader 133Xc Suppose that $(X,\Sigma,\mu)$ is a measure space 
and that $G$ is an open subset of $\Bbb C$, that is, a set such that for 
every $w\in G$ there is a $\delta>0$ such that 
$\{z:|z-w|<\delta\}\subseteq G$.   Let $f:X\times G\to\Bbb C$ be a 
function, and suppose that the derivative $\pd{f}{z}$ of $f$ with 
respect to the 
second variable exists for all $x\in X$, $z\in G$.   Suppose 
moreover that (i) $F(z)=\int f(x,z)dx$ exists for every 
$z\in G$ (ii) there is an integrable function $g$ such that 
$|\pd{f}{z}(x,z)|\le g(x)$ for every $x\in X$, $z\in G$. 
Show that the derivative $F'$ of $F$ exists everywhere in $G$, and 
$F'(z)=\biggerint\pd{f}{z}(x,z)dx$ for every $z\in G$.   
({\it Hint\/}:  you will need to check that 
$|f(x,z)-f(x,w)|\le|z-w|g(x)$ whenever $x\in X$, $z\in G$ and $w$ is close to $z$.) 
%133F 
      
\sqheader 133Xd Let $f$ be a complex-valued function defined 
almost everywhere 
on $\coint{0,\infty}$, endowed as usual with Lebesgue measure.   Its 
{\bf Laplace transform} is the function $F$ defined by writing 
      
\Centerline{$F(s)=\int_0^{\infty}e^{-sx}f(x)dx$} 
      
\noindent for all those complex numbers $s$ for which the integral is 
defined in $\Bbb C$. 
      
\quad(i) Show that if $s\in\dom F$ and $\Real s'\ge\Real s$ then 
$s'\in\dom F$ (because $|e^{-s'x}e^{sx}|\le 1$ for all $x$). 
      
\quad(ii) Show that $F$ is analytic (that is, differentiable as a 
function of a complex variable) on the interior of its domain. 
\Hint{133Xc.} 
      
\quad(iii) Show that if $F$ is defined anywhere then $\lim_{\Real 
s\to\infty}F(s)=0$. 
      
\quad(iv) Show that if $f$, $g$ have Laplace transforms $F$, $G$ then 
the Laplace transform of $f+g$ is $F+G$, at least on $\dom F\cap \dom 
G$. 
%133Xc 133F 
      
\sqheader 133Xe Let $f$ be an integrable complex-valued function 
defined almost everywhere in $\Bbb R$, endowed as usual with Lebesgue 
measure.   Its {\bf Fourier transform} is the function $\varhatf$ 
defined by 
      
\Centerline{$\varhatf(s)=\Bover{1}{\sqrt{2\pi}} 
\int_{-\infty}^{\infty}e^{-isx}f(x)dx$} 
      
\noindent for all real $s$. 
      
\quad(i) Show that $\varhatf$ is continuous.   \Hint{use Lebesgue's 
Dominated Convergence Theorem on sequences of the form 
$f_n(x)=e^{-is_nx}f(x)$.} 
      
\quad(ii) Show that if $f$, $g$ have Fourier transforms $\varhatf$, 
$\varhat g$ then the Fourier transform of $f+g$ is $\varhatf+\varhat g$. 
      
\quad(iii) Show that if $\int xf(x)dx$ exists then $\varhatf$ is 
differentiable, with 
$\varhatf\vthsp'(s)=-\Bover{i}{\sqrt{2\pi}}\int xe^{-isx}f(x)dx$ for 
every $s$. 
%133Xc 133F 
      
\spheader 133Xf Let $(X,\Sigma,\mu)$ be a measure space and 
$\sequencen{f_n}$ a sequence of real-valued functions each defined 
almost everywhere in $X$.   Suppose that there is an integrable 
real-valued function $g$ such that $|f_n|\leae g$ for each $n$. 
Show that 
      
\Centerline{$\overlineint\liminf_{n\to\infty}f_n 
\le\liminf_{n\to\infty}\overlineint f_n$,\quad 
$\underlineint\limsup_{n\to\infty}f_n 
\ge\limsup_{n\to\infty}\underlineint f_n$.} 
%133K 
 
      
\leader{133Y}{Further exercises (a)} 
%\spheader 133Ya 
Use the ideas of 133C-133H to develop a theory of measurable and 
integrable functions taking values in $\BbbR^r$, where $r\ge 2$. 
%133H 
      
\spheader 133Yb Let $X$ be a set and $\Sigma$ a $\sigma$-algebra 
of subsets of $X$.   Let $Y$ be a subset of $X$ and $f:Y\to\Bbb C$ a 
$\Sigma_Y$-measurable function, where $\Sigma_Y=\{E\cap Y:E\in\Sigma\}$. 
Show that there is a $\Sigma$-measurable function $\tilde f:X\to\Bbb C$ 
extending $f$.  \Hint{121I.} 
%133F 
      
\spheader 133Yc Let $f$ be an integrable complex-valued function 
defined almost everywhere in $\BbbR^r$, endowed as usual with Lebesgue 
measure, where $r\ge 1$.   Its {\bf Fourier transform} is the function 
$\varhatf$ defined by 
      
\Centerline{$\varhatf(s)=\Bover{1}{{(\sqrt{2\pi})}^r} 
\int e^{-is\dotproduct x}f(x)dx$} 
      
\noindent for all $s\in\BbbR^r$, writing $s\dotproduct x$ for 
$\sigma_1\xi_1+\ldots+\sigma_r\xi_r$ if $s=(\sigma_1,\ldots,\sigma_r)$, 
$x=(\xi_1,\ldots,\xi_r)\in\BbbR^r$. 
      
\quad(i) Show that $\varhatf$ is continuous.   
      
\quad(ii) Show that if $f$, $g$ have Fourier transforms $\varhatf$, 
$\varhat g$ then the Fourier transform of $f+g$ is $\varhatf+\varhat g$. 
      
\quad (iii) Show that if $\int\|x\||f(x)|dx$ is finite
(taking $\|x\|=\sqrt{\xi_1^2+\ldots+\xi_r^2}$ if 
$x=(\xi_1,\dots,\xi_r)$), then $\varhatf$ is differentiable, with 
      
\Centerline{$\Pd{\varhatf}{\sigma_k}(s) 
=-\Bover{i}{{(\sqrt{2\pi})}^r}\int\xi_ke^{-is\dotproduct x}f(x)dx$} 
      
\noindent for every $s\in\BbbR^r$, $k\le r$. 
%133Xe 133F 
      
\spheader 133Yd Recall the definition of `quasi-simple' 
function from 122Yd.   Show that for any measure space $(X,\Sigma,\mu)$ 
and any real-valued function $f$ defined almost everywhere in $X$, 
      
\Centerline{$ 
\overlineint f=\inf\{\int g:g$ is quasi-simple, $f\leae g\}$,} 
      
\Centerline{$ 
\underlineint f=\sup\{\int g:g$ is quasi-simple, $f\geae g\}$,} 
      
\noindent allowing $\infty$ for $\inf\emptyset$ and $\sup\Bbb R$ and 
$-\infty$ for $\inf\Bbb R$ and $\sup\emptyset$. 
%133J 
      
\spheader 133Ye State and prove a similar result concerning the 
`pseudo-simple' functions of 122Ye. 
%133Yd 133J 
}%end of exercises 
      
\endnotes{ 
\Notesheader{133} I have spelt this section out in 
detail, even though there is nothing that can really be called a new 
idea in it, because it gives us an opportunity to review the previous 
work, and because the manipulations which are by now, I hope, becoming 
`obvious' to you are in fact justifiable only through difficult 
theorems, and I believe that it is at least some of the time right to 
look back to the exact points at which justifications were written out. 
      
You may have noticed similarities between results involving `upper 
integrals', as described here, and those of \S132 concerning `outer 
measure' (132Ae and 133Ka, for instance, or 132Xe and 133Kb).   These 
are not a coincidence;  an explanation of sorts can be found in 252Ym in 
Volume 2. 
}%end of notes 
      
\discrpage 
      
     
