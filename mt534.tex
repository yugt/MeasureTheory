\frfilename{mt534.tex}
\versiondate{21.11.05/1.8.10}
\copyrightdate{2003}

\def\chaptername{Topology and measure III}
\def\sectionname{Hausdorff measures and strong measure zero}
\def\czz{ CHECK\query\ }

\newsection{534}

In this section I look at constructions which are primarily metric
rather than topological.   I start with a note on Hausdorff
measures, spelling out connexions between Hausdorff $r$-dimensional
measure on a
separable metric space and the basic $\sigma$-ideal $\Cal N$ (534B).

The main part of the section section is a brief introduction to a
class of ideals which are of
great interest in set-theoretic analysis.   While the most important
ones are based on separable metric spaces, some of the ideas can be
expressed in more general contexts, and I give the definition and most
elementary properties in terms of uniformities (534C-534D).   An
associated topological notion is what I call `Rothberger's property'
(534E-534G).   %534E 534F 534G
A famous
characterization of sets of strong measure zero in $\Bbb R$ in terms
of translations of meager sets can also be represented as a theorem about
$\sigma$-compact groups (534H).   There are few elementary results
describing the cardinal functions of strong measure zero ideals, but I
give some information on their uniformities (534I) and additivities
(534K).   There seem to be some interesting questions concerning
spaces with isomorphic strong measure zero ideals, which I consider in
534L-534N.  %534L 534M 534N
A particularly important question, from the very beginning of the
topic in {\smc Borel 1919}, concerns the possible cardinals of sets of
strong measure zero;  in 534O-534P I give some sample facts and a pair of
illustrative examples.

\leader{534A}{}\cmmnt{ An elementary lemma will be useful.

\medskip

\noindent}{\bf Lemma} Let $(X,\rho)$ be a separable metric space.
Then there is a countable family $\Cal C$ of subsets of $X$ such that
whenever $A\subseteq X$ has finite diameter and $\eta>0$ then there
is a $C\in\Cal C$ such that $A\subseteq C$ and $\diam C\le\eta+2\diam A$.

\proof{ Let $D$ be a countable dense subset of $X$ and set
$\Cal C=\{\emptyset\}\cup\{B(x,q):x\in D$, $q\in\Bbb Q$, $q\ge 0\}$.
If $A\subseteq X$ has finite diameter and $\eta>0$, then if
$A=\emptyset$ we can take $C=\emptyset$.   Otherwise, take $y\in A$
and $q\in\Bbb Q$ such that
$\diam A+\bover14\eta\le q\le\diam A+\bover12\eta$.   Let $x\in D$ be
such that $\rho(x,y)\le\bover14\eta$;  then $C=B(x,q)\in\Cal C$,
$A\subseteq B(y,\diam A)\subseteq C$ and $\diam C\le 2q\le\eta+2\diam A$.
}%end of proof of 534A

\leader{534B}{Hausdorff \dvrocolon{measures}}\cmmnt{ There are
difficult questions concerning the cardinals associated with
even the most
familiar Hausdorff measures.   However we do have some easy results.

\medskip

\noindent}{\bf Theorem} Let $(X,\rho)$ be a metric space and
$r>0$.   Write $\mu_{Hr}$ for $r$-dimensional Hausdorff measure on
$X$, $\Cal N(\mu_{Hr})$ for its null ideal,
$\Cal N$ for the null ideal of Lebesgue measure on $\Bbb R$ and
$\Cal M$ for the ideal of meager subsets of $\Bbb R$.

(a) $\add\mu_{Hr}=\add\Cal N(\mu_{Hr})$.

(b) If $X$ is separable,
$\Cal N(\mu_{Hr})\prT\Cal N$, so that
$\add\mu_{Hr}\ge\add\Cal N$ and $\cf\Cal N(\mu_{Hr})\le\cf\Cal N$.

(c) If $X$ is separable,
$(X,\in,\Cal N(\mu_{Hr}))\prGT(\Cal M,\not\ni,\Bbb R)$, so that
$\cov\Cal N(\mu_{Hr})\le\non\Cal M$ and
$\frakmctbl\le\non\Cal N(\mu_{Hr})$.

(d) If $X$ is analytic and $\mu_{Hr}X>0$, then
$\add\mu_{Hr}=\add\Cal N$, $\cf\Cal N(\mu_{Hr})=\cf\Cal N$,
$\non\Cal N(\mu_{Hr})\le\non\Cal N$ and
$\cov\Cal N(\mu_{Hr})\ge\cov\Cal N$.

\proof{{\bf (a)} 521Ac.

\medskip

{\bf (b)(i)} Let $\Cal C$ be a countable family of subsets of $X$ such
that whenever $A\subseteq X$ has finite diameter and $\eta>0$ there
is a $C\in\Cal C$
such that $A\subseteq C$ and $\diam C\le\eta+2\diam A$ (534A).

If $A\subseteq X$, then $A\in\Cal N(\mu_{Hr})$ iff for
every $\epsilon>0$ there is a sequence $\sequencen{C_n}$ in $\Cal C$
such that $A\subseteq\bigcup_{n\in\Bbb N}C_n$ and
$\sum_{n=0}^{\infty}(\diam C_n)^r\le\epsilon$.   \Prf\ If $A$ is
negligible and $\epsilon>0$, then (by the definition in 471A) there
must be a sequence $\sequencen{A_n}$ of subsets of $X$ such that
$A\subseteq\bigcup_{n\in\Bbb N}A_n$ and
$\sum_{n=0}^{\infty}(\diam A_n)^r<2^{-r}\epsilon$.   Let
$\sequencen{\eta_n}$ be a sequence of strictly positive real numbers
such that $\sum_{n=0}^{\infty}(\eta_n+2\diam A_n)^r\le\epsilon$.
For each $n$ we
can find $C_n\in\Cal C_n$ such that $A_n\subseteq C_n$ and
$\diam C_n\le\eta_n+2\diam A_n$, so that
$\sum_{n=0}^{\infty}(\diam C_n)^r\le\epsilon$, while
$A\subseteq\bigcup_{n\in\Bbb N}C_n$.

On the other hand, if $A$ satisfies the condition, then for every
$\epsilon$, $\delta>0$ there is a sequence $\sequencen{C_n}$ of
subsets of $X$ such that $A\subseteq\bigcup_{n\in\Bbb N}C_n$ and
$\sum_{n=0}^{\infty}(\diam C_n)^r\le\min(\epsilon,\delta^r)$.   In
this case, $\diam C_n\le\delta$ for every $n$, so $\theta_{r\delta}A$, as
defined in 471A, is at most $\epsilon$.   As $\epsilon$ is arbitrary,
$\theta_{r\delta}A=0$;  as $\delta$ is arbitrary, $A$ is
$\mu_{Hr}$-negligible.\ \Qed

\medskip

\quad{\bf (ii)} It follows that
$(\Cal N(\mu_{Hr}),\subseteq,\Cal N(\mu_{Hr}))
\prGT(\BbbN^{\Bbb N},\subseteq^*,\Cal S)$, where
$(\BbbN^{\Bbb N},\subseteq^*,\Cal S)$ is the $\Bbb N$-localization relation
(522K).

\Prf\grheada\ For each $n\in\Bbb N$, let $\Cal I_n$ be the family of
finite subsets $I$ of $\Cal C$ such that
$\sum_{C\in I}(\diam C)^r\le 4^{-n}$.   Let $\sequence{j}{I_{nj}}$ be
a sequence running over $\Cal I_n$.   Now, given $A\in\Cal N(\mu_{Hr})$,
then for each $n\in\Bbb N$ let $\sequence{i}{C_{ni}}$ be a sequence in
$\Cal C$, covering $A$, such that
$\sum_{i=0}^{\infty}(\diam C_{ni})^r\le 2^{-n-1}$.   Let
$\sequence{i}{C_i}$ be a re-indexing of the family
$\langle C_{ni}\rangle_{n,i\in\Bbb N}$, so that $\sequence{i}{C_i}$
is a sequence in $\Cal C$, $\sum_{i=0}^{\infty}(\diam C_i)^r\le 1$, and
$A\subseteq\bigcap_{m\ge n}\bigcup_{i\ge m}C_i$.   Let
$\sequencen{k(n)}$ be a strictly increasing sequence in $\Bbb N$ such
that $k(0)=0$ and $\sum_{i=k(n)}^{\infty}(\diam C_i)^r\le 4^{-n}$ for
every $n$.   Now, for $n\in\Bbb N$, let $\phi(A)(n)$ be such that
$\{C_i:k(n)\le i<k(n+1)\}=I_{n,\phi(A)(n)}$.

This process defines a function $\phi:\Cal N(\mu_{Hr})\to\BbbN^{\Bbb N}$
such that

\Centerline{$A
\subseteq\bigcap_{m\in\Bbb N}\bigcup_{n\ge m}\bigcup I_{n,\phi(A)(n)}$}

\noindent for every $A\in\Cal N(\mu_{Hr})$.

\medskip

\qquad\grheadb\ For $S\in\Cal S$, set

\Centerline{$\psi(S)
=\bigcap_{m\in\Bbb N}\bigcup_{n\ge m}\bigcup_{i\in S[\{n\}]}\bigcup I_{ni}
\subseteq X$.}

\noindent If $n\in\Bbb N$, then

\Centerline{$\sum\{(\diam C)^r:C\in\bigcup_{i\in S[\{n\}]}I_{ni}\}
\le 2^n\cdot 4^{-n}=2^{-n}$,}

\noindent because $\#(S[\{n\}])\le 2^n$ and
$\sum\{(\diam C)^r:C\in I_{ni}\}\le 4^{-n}$ for every $i$.   But this
means that, for any $m\in\Bbb N$,

\Centerline{$\sum\{(\diam C)^r:C\in\bigcup_{n\ge m}
  \bigcup_{i\in S[\{n\}]}I_{ni}\}
\le 2^{-m+1}$,}

\noindent while

\Centerline{$\psi(S)
\subseteq\bigcup(\bigcup_{n\ge m}\bigcup_{i\in S[\{n\}]}I_{ni})$.}

\noindent So $\psi(S)\in\Cal N(\mu_{Hr})$ for every $S\in\Cal S$.

\medskip

\qquad\grheadc\ Suppose that $A\in\Cal N(\mu_{Hr})$ and
$\phi(A)\subseteq^*S\in\Cal S$.   Then there is some $m_0\in\Bbb N$
such
that $(n,\phi(A)(n))\in S$ for every $n\ge m_0$.   Now, for any
$m\in\Bbb N$, we have

$$\eqalign{A
&\subseteq\bigcup_{n\ge\max(m,m_0)}\bigcup I_{n,\phi(A)(n)}\cr
&\subseteq\bigcup_{n\ge\max(m,m_0)}\bigcup_{i\in S[\{n\}]}\bigcup
I_{ni}
\subseteq\bigcup_{n\ge m}\bigcup_{i\in S[\{n\}]}\bigcup I_{ni},\cr}$$

\noindent so $A\subseteq\psi(S)$.   This shows that $(\phi,\psi)$ is a
Galois-Tukey connection from
$(\Cal N(\mu_{Hr}),\subseteq\nobreak,\penalty-100\Cal N(\mu_{Hr}))$
to $(\BbbN^{\Bbb N},\subseteq^*,\Cal S)$, and
$(\Cal N(\mu_{Hr}),\subseteq,\Cal N(\mu_{Hr}))
\prGT(\BbbN^{\Bbb N},\subseteq^*,\Cal S)$.\ \Qed

\medskip

\quad{\bf (iii)} Since $(\Cal N,\subseteq,\Cal N)
\equivGT(\BbbN^{\Bbb N},\subseteq^*,\Cal S)$ (522M),
$(\Cal N(\mu_{Hr}),\subseteq,\Cal N(\mu_{Hr}))
\prGT(\Cal N,\subseteq\nobreak,\penalty-100\Cal N)$,
that is, $\Cal N(\mu_{Hr})\penalty-100\prT\Cal N$.

\medskip

\quad{\bf (iv)} By 513Ee, as usual, we can conclude that
$\add\Cal N(\mu_{Hr})\ge\add\Cal N$ and
$\cf\Cal N(\mu_{Hr})\le\cf\Cal N$.

\medskip

{\bf (c)(i)} If $\mu_{Hr}X=0$, the result is trivial.   \Prf\
Set $\phi(x)=\emptyset$ for $x\in X$, $\psi(t)=X$ for
$t\in\Bbb R$;  then $(\phi,\psi)$ is a Galois-Tukey connection from
$(X,\in,\Cal N(\mu_{Hr}))$ to $(\Cal M,\not\ni,\Bbb R)$.\ \QeD\  So
let
us suppose that $X$ is infinite.

\medskip

\quad{\bf (ii)} Let $F$ be the set of $1$-Lipschitz functions
$f:X\to[0,1]$.   Define $T:X\to\ell^{\infty}(F)$ by setting
$(Tx)(f)=f(x)$ for $f\in F$ and $x\in X$.   Then

\Centerline{$\|Tx-Ty\|_{\infty}
=\sup_{f\in F}|f(x)-f(y)|=\min(1,\rho(x,y))$}

\noindent for all $x$, $y\in X$.   \Prf\ Of course
$\sup_{f\in F}|f(x)-f(y)|\le\min(1,\rho(x,y))$, by the definition of
$F$.   On the other hand, given $x\in X$, we can set
$f(z)=\min(1,\rho(z,x))$ for every $z\in X$;  then $f\in F$ and
$|f(x)-f(y)|=\min(1,\rho(x,y))$.   So we have equality.\ \QeD\   Thus
$T$ is $1$-Lipschitz for $\rho$ and the usual metric on
$\ell^{\infty}(F)$, and $T[X]$ is a separable subset of
$\ell^{\infty}(F)$ (4A2B(e-iii)).   Let $V$ be the closed linear
subspace of $\ell^{\infty}(F)$ generated by $T[X]$;  then $V$ is
separable (4A4Bg).   Being a closed subset of the complete metric
space
$\ell^{\infty}(F)$, $V$ is a Polish space.   Since $X$ has more than
one
point, and $T$ is injective, $V$ is non-empty and has no isolated
points.

Let $\sequencen{v_n}$ enumerate a dense subset of $V$.   Set

\Centerline{$E
=\bigcap_{n\in\Bbb N}\bigcup_{i\ge n}U(v_i,2^{-i-1})$}

\noindent where $U(v,\delta)=\{u:u\in V$, $\|u-v\|_{\infty}<\delta\}$
for $v\in V$, $\delta>0$.   Then $E$ is the intersection of a sequence
of dense open sets in $V$, so is comeager, and $M=V\setminus E$
belongs to the ideal $\Cal M(V)$ of meager subsets of $V$.
For any $v\in V$, the map $u\mapsto u-v:V\to V$ is a
homeomorphism, so $M-v\in\Cal M(V)$.   Define $\phi:X\to\Cal M(V)$ by
setting $\phi(x)=M-Tx$ for $x\in X$.

In the other direction, define $\psi:V\to\Cal PX$ by setting
$\psi(v)=T^{-1}[E-v]$ for $v\in V$.   Then $\psi(v)\in\Cal
N(\mu_{Hr})$
for every $v\in V$.   \Prf\ If $v\in V$ and $\delta\le\bover12$, then
$\|u-u'\|_{\infty}<1$ for all $u$, $u'\in U(v,\delta)$, so
$\rho(x,x')\le\|Tx-Tx'\|_{\infty}$ whenever $x$,
$x'\in T^{-1}[U(v,\delta)]$.   Accordingly
$\diam T^{-1}[U(v_i-v,2^{-i-1})]\le 2^{-i}$ for every $i\in\Bbb N$.
This means that

$$\eqalign{\mu_{Hr}^*T^{-1}[E-v]
&=\mu_{Hr}^*(\bigcap_{n\in\Bbb N}\bigcup_{i\ge n}
  T^{-1}[U(v_i-v,2^{-i-1})]\cr
&\le\inf_{n\in\Bbb N}\sum_{i=n}^{\infty}(2^{-i})^r
=0.  \text{ \Qed}\cr}$$

\noindent So $\psi$ is a function from $V$ to $\Cal N(\mu_{Hr})$.   We
now see that

$$\eqalign{\phi(x)\not\ni v
&\Longrightarrow v\notin M-Tx
\Longrightarrow Tx\notin M-v\cr
&\Longrightarrow Tx\in E-v
\Longrightarrow x\in\psi(v).\cr}$$

\noindent Thus $(\phi,\psi)$ is a Galois-Tukey connection from
$(X,\in,\Cal N(\mu_{Hr}))$ to $(\Cal M(V),\not\ni,V)$ and

\Centerline{$(X,\in,\Cal N(\mu_{Hr}))\prGT(\Cal M(V),\not\ni,V)
\cong(\Cal M,\not\ni,\Bbb R)$}

\noindent (522Wb).

\medskip

\quad{\bf (iii)} Now

\Centerline{$\cov\Cal N(\mu_{Hr})=\cov(X,\in,\Cal N(\mu_{Hr}))
\le\cov(\Cal M,\not\ni,\Bbb R)=\non\Cal M$,}

\Centerline{$\non\Cal N(\mu_{Hr})=\add(X,\in,\Cal N(\mu_{Hr}))
\ge\add(\Cal M,\not\ni,\Bbb R)=\cov\Cal M=\frakmctbl$}

\noindent (512D, 512Ed, 522Sa).

\medskip

{\bf (d)} If $X$ is analytic and $\mu_{Hr}X>0$, then by Howroyd's
theorem (471S) there is a compact set $K\subseteq X$ such that
$0<\mu_{Hr}K<\infty$.   Now the subspace measure $\mu_{Hr}^{(K)}$ on $K$
is an atomless Radon measure (471E, 471Dg, 471F)
on a compact metric space, so

\Centerline{$\add\Cal N\le\add\Cal N(\mu_{Hr})
\le\add\Cal N(\mu_{Hr}^{(K)})=\add\Cal N$,}

\Centerline{$\cf\Cal N\ge\cf(X,\Cal N(\mu_{Hr}))
\ge\cf(K,\Cal N(\mu_{Hr}^{(K)}))=\cf\Cal N$,}

\Centerline{$\non\Cal N(\mu_{Hr})
\le\non\Cal N(\mu_{Hr}^{(K)})=\non\Cal N$,}

\Centerline{$\cov(X,\Cal N(\mu_{Hr}))
\ge\cov(K,\Cal N(\mu_{Hr}^{(K)}))=\cov\Cal N$}

\noindent by (b) above, 521F and 522Wa.
}%end of proof of 534B

\leader{534C}{Strong measure zero} Let $(X,\Cal W)$ be a uniform space.
I say that $X$ has {\bf strong measure zero} or {\bf property C} if
for any sequence $\sequencen{W_n}$ in $\Cal W$ there is a
cover $\sequencen{A_n}$ of $X$ such that $A_n\times A_n\subseteq W_n$
for every $n\in\Bbb N$.   A subset $A$ of $X$ has strong measure zero
if it has strong measure zero in its subspace uniformity\cmmnt{, that
is, if for any sequence $\sequencen{W_n}$ in $\Cal W$ there is a
cover $\sequencen{A_n}$ of $A$ such that $A_n\times A_n\subseteq W_n$
for every $n\in\Bbb N$}.   If $(X,\rho)$
is a metric space, it has strong measure zero if it has strong measure
zero for the uniformity associated with the metric\cmmnt{ (3A4B),
that is, for any
sequence $\sequencen{\epsilon_n}$ of strictly positive real numbers
there is a
cover $\sequencen{A_n}$ of $X$ such that $\diam A_n\le\epsilon_n$ for
every $n\in\Bbb N$}.

I will write $\CalSmz(X,\Cal W)$ or $\CalSmz(X,\rho)$ for the family of
sets of strong measure zero in a uniform space $(X,\Cal W)$ or a metric
space $(X,\rho)$.

\leader{534D}{Proposition} (a) If $(X,\Cal W)$ is a uniform space,
then $\CalSmz(X,\Cal W)$ is a $\sigma$-ideal
containing all the countable subsets of $X$.

(b) If $(X,\Cal W)$ is a uniform space with strong measure zero,
$(Y,\Cal V)$ is a uniform space, and $f:X\to Y$ is uniformly
continuous, then $f[X]\in\CalSmz(Y,\Cal V)$.

(c) Let $(X,\Cal W)$ be a uniform space and $A\subseteq X$.   Then
$A\in\CalSmz(X,\Cal W)$ iff $f[A]\in\CalSmz(Y,\rho)$ whenever $(Y,\rho)$ is
a metric space and $f:X\to Y$ is uniformly continuous.

(d) Let $(X,\Cal W)$ be a uniform space and $A\in\CalSmz(X,\Cal W)$.
If $B\subseteq X$ is such that $B\setminus G\in\CalSmz(X,\Cal W)$
whenever $G$ is an open set including $A$, then
$B\in\CalSmz(X,\Cal W)$.

(e) If $(X,\rho)$ is a metric space with
strong measure zero, then $X$ is separable and universally negligible.
%Consequently all compact subsets of $X$ are countable.

\proof{{\bf (a)} It is immediate from the definition that any subset of
a set in $\CalSmz(X,\Cal W)$ belongs to $\CalSmz(X,\Cal W)$,
and so does any countable set.   Now suppose that
$\sequencen{A_n}$ is a sequence in $\CalSmz(X,\Cal W)$.
Let $\sequencen{W_n}$ be any sequence in
$\Cal W$.   For each $k\in\Bbb N$, $\sequence{i}{W_{2^k(2i+1)}}$ is a
sequence in $\Cal W$, so there is a sequence $\sequence{i}{A_{ki}}$,
covering $A_k$, such that $A_{ki}\times A_{ki}\subseteq W_{2^k(2i+1)}$
for every $i$.   Set $B_0=\emptyset$, $B_n=A_{ki}$ if $n=2^k(2i+1)$
where $k$, $i\in\Bbb N$;  then $A\subseteq\bigcup_{n\in\Bbb N}B_n$ and
$B_n\times B_n\subseteq W_n$ for every $n$.   As $\sequencen{W_n}$ is
arbitrary, $A$ has strong measure zero;  as $\sequencen{A_n}$ is
arbitrary, $\CalSmz(X,\Cal W)$ is a $\sigma$-ideal.

\medskip

{\bf (b)} Let $\sequencen{V_n}$ be a sequence in $\Cal V$.   For each
$n\in\Bbb N$, there is a $W_n\in\Cal W$
such that $(f(x),f(x'))\in V_n$ whenever
$(x,x')\in W_n$.   Because $X\in\CalSmz(X,\Cal W)$, there is
a cover $\sequencen{A_n}$ of $X$ such that
$A_n\times A_n\subseteq W_n$
for every $n$;  now $f[A_n]\times f[A_n]\subseteq V_n$ for every $n$
and $\bigcup_{n\in\Bbb N}f[A_n]=f[X]$.   As $\sequencen{V_n}$ is
arbitrary, $f[X]\in\CalSmz(Y,\Cal V)$.

\medskip

{\bf (c)} If $A$ has strong measure zero, then of course $f[A]$ has
strong measure zero for any uniformly continuous function $f$ from $X$
to a metric space, by (b) applied to $f\restr A$.   Now suppose that
$A$ satisfies the
condition, and that $\sequencen{W_n}$ is a sequence in $\Cal W$.   Then
there is a pseudometric $\rho$ on $X$, compatible with the uniformity in
the sense that $\{(x,y):\rho(x,y)\le\epsilon\}\in\Cal W$ for every
$\epsilon>0$, such that $\{(x,y):\rho(x,y)<2^{-n}\}\subseteq W_n$ for
every $n$ (4A2Ja).   Set $\sim\mskip5mu=\{(x,y):\rho(x,y)=0\}$.   Then
$\sim$ is
an equivalence relation on $X$.   If $Y$ is the set of equivalence
classes, we have a metric $\tilde\rho$ on $Y$ defined by setting
$\tilde\rho(x^{\ssbullet},y^{\ssbullet})=\rho(x,y)$ for all $x$,
$y\in X$.   Setting $f(x)=x^{\ssbullet}$ for $x\in X$, $f:X\to Y$ is
uniformly continuous.   So $f[A]\in\CalSmz(Y,\tilde\rho)$.   Let
$\sequencen{B_n}$ be a cover of $f[A]$ such that $\diam B_n\le 2^{-n-1}$
for every $n$, and set $A_n=f^{-1}[B_n]$ for each $n$.   Then
$\sequencen{A_n}$ is a cover of $A$.   If $n\in\Bbb N$ and $x$,
$y\in A_n$, then $\rho(x,y)=\tilde\rho(f(x),f(y))\le 2^{-n-1}$, so
$(x,y)\in W_n$.   Thus $A_n\times A_n\subseteq W_n$.   As
$\sequencen{W_n}$ is arbitrary, $A\in\CalSmz(X,\Cal W)$.

\medskip

{\bf (d)} Let $\sequencen{W_n}$ be any sequence in $\Cal W$.   For each
$n\in\Bbb N$, let $V_n\in\Cal W$ be such that
$V_n\frsmallcirc V_n\frsmallcirc V_n^{-1}\subseteq W_{2n}$.   Then there
is a sequence $\sequencen{A_n}$, covering $A$, such that
$A_n\times A_n\subseteq V_n$ for every $n$.   Set
$B_{2n}=\interior V_n[A_n]$ for
each $n$, and $G=\bigcup_{n\in\Bbb N}B_{2n}$;  then
$B_{2n}\times B_{2n}\subseteq W_{2n}$ for every $n$ and $G$ is an open
set including $A$.   Accordingly $B\setminus G\in\CalSmz(X,\Cal W)$
and there is a sequence $\sequencen{B_{2n+1}}$, covering $B\setminus G$,
such that $B_{2n+1}\times B_{2n+1}\subseteq W_{2n+1}$ for every $n$.
Now $\sequencen{B_n}$ covers $B$ and $B_n\times B_n\subseteq W_n$ for
every $n$.   As $\sequencen{W_n}$ is arbitrary, $B$ has strong measure
zero.

\medskip

{\bf (e)} \Quer\ If $X$ is not separable, there is an uncountable
$A\subseteq X$ such that $\epsilon=\inf_{x,y\in A,x\ne y}\rho(x,y)$ is
greater than $0$ (5A4B(h-iii)).
Now there can be no cover $\sequencen{A_n}$ of $X$ by sets of
diameter less than $\epsilon$.\ \BanG\ Thus $X$ is separable.

Now suppose that $\mu$ is a Borel probability measure on $X$.   Then
there is a $\delta>0$ such that for every $n\in\Bbb N$ there is a
Borel
set $E_n$ with $\diam E_n\le 2^{-n}$ and $\mu E_n\ge\delta$.
\Prf\Quer\ Otherwise, we can find for each $n\in\Bbb N$ an
$\epsilon_n>0$ such that $\mu E\le 2^{-n-2}$ whenever $E$ is a Borel
set
and $\diam E\le\epsilon_n$.   Let $\sequencen{A_n}$ be a cover of $X$
such that $\diam A_n\le\epsilon_n$ for every $n$;  then
$\diam\overline{A}_n\le\epsilon_n$, so $\mu\overline{A}_n\le 2^{-n-2}$
for every $n$, and

\Centerline{$\mu X\le\sum_{n=0}^{\infty}\mu\overline{A}_n<1$.
\Bang\Qed}

Now consider $E=\bigcap_{n\in\Bbb N}\bigcup_{m\ge n}E_m$.   Since
$\mu E\ge\delta>0$, there is an $x\in E$.   For any $n\in\Bbb N$,
there
is an $m\ge n$ such that

\Centerline{$x\in E_m\subseteq B(x,2^{-m})\subseteq B(x,2^{-n})$,}

\noindent so

\Centerline{$\mu\{x\}=\inf_{n\in\Bbb N}\mu B(x,2^{-n})\ge\delta>0$.}

\noindent As $\mu$ is arbitrary, this shows that $X$ is universally
negligible.

%If $K\subseteq X$ is compact, it must be scattered (439C(a-v));
%because it is first-countable, it must be countable (4A2G(j-vi)).
}%end of proof of 534D

\leader{534E}{Rothberger's property}\cmmnt{ `Strong measure zero' is
a uniform-space property.   The topological notion closest to it seems
to be the following.}   If $X$ is a topological space and $A$ is a
subset of $X$, I will say that $A$ has {\bf Rothberger's property in}
$X$ if for every sequence
$\sequencen{\Cal G_n}$ of non-empty open covers of $X$ there is a
sequence $\sequencen{G_n}$ such that $G_n\in\Cal G_n$ for every
$n\in\Bbb N$ and $A\subseteq\bigcup_{n\in\Bbb N}G_n$.   \cmmnt{Note
that this is a
relative property of the pair $(A,X)$, not an intrinsic property of
the set $A$ (see 534Xe, 534Xo).}   I will write $\CalRbg(X)$ for the family
of subsets of $X$ with Rothberger's property.

\vleader{48pt}{534F}{Proposition} Let $X$ be a topological space.

(a) $\CalRbg(X)$ is
a $\sigma$-ideal containing all the countable subsets of $X$.

(b) If $Y$ is another topological space, $f:X\to Y$ is continuous and
$A\in\CalRbg(X)$, then $f[A]\in\CalRbg(Y)$.

(c) If $X$ is completely regular, and $\Cal W$ is any uniformity on
$X$ compatible with its topology, then
$\CalRbg(X)\subseteq\CalSmz(X,\Cal W)$.

(d) If $X$ is $\sigma$-compact and completely regular, and $\Cal W$ is
any uniformity on $X$ compatible with its topology, then
$\CalRbg(X)=\CalSmz(X,\Cal W)$.

\proof{{\bf (a)} (Cf.\ 534Da.)
It is immediate from the definition that any subset
of a set in $\CalRbg(X)$, and any countable subset of $X$, belong to
$\CalRbg(X)$.   Now
suppose that $\sequencen{A_n}$ is a sequence in $\CalRbg(X)$,
with union $A$.   Let $\sequencen{\Cal G_n}$ be any sequence of
non-empty open covers of $X$.   For each $k\in\Bbb N$,
$\sequence{i}{\Cal G_{2^k(2i+1)}}$ is a
sequence of open covers of $X$, so there is a sequence
$\sequence{i}{G_{ki}}$,
covering $A_k$, such that $G_{ki}\in\Cal G_{2^k(2i+1)}$
for every $i$.   Take $H_0$ to be any member of $\Cal G_0$, and set
$H_n=G_{ki}$ if $n=2^k(2i+1)$
where $k$, $i\in\Bbb N$;  then $A\subseteq\bigcup_{n\in\Bbb N}H_n$ and
$H_n\in\Cal G_n$ for every $n$.   As $\sequencen{\Cal G_n}$ is
arbitrary, $A$ has Rothberger's property in $X$.

\medskip

{\bf (b)} (Cf.\ 534Db.)
Let $\sequencen{\Cal H_n}$ be a sequence of non-empty open
covers of $Y$.   For each $n\in\Bbb N$, set
$\Cal G_n=\{f^{-1}[H]:H\in\Cal H_n\}$;  then $\Cal G_n$ is a non-empty
open cover of $X$.   Because $A\in\CalRbg(X)$, there
is a cover $\sequencen{G_n}$ of $A$ such that $G_n\in\Cal G_n$ for
every $n\in\Bbb N$.   Express $G_n$ as $f^{-1}[H_n]$,
where $H_n\in\Cal H_n$,
for each $n\in\Bbb N$;   then $f[A]\subseteq\bigcup_{n\in\Bbb N}H_n$.
As $\sequencen{\Cal H_n}$ is
arbitrary, $f[A]$ has Rothberger's property in $Y$.

\medskip

{\bf (c)} Suppose that $A\in\CalRbg(X)$,
and that $\sequencen{W_n}$ is any sequence in $\Cal W$.   For each
$n\in\Bbb N$, set
$\Cal G_n=\{G:G\subseteq X$ is open, $G\times G\subseteq W_n\}$;  then
$\Cal G_n$ is a non-empty open cover of $X$.   So we can find a cover
$\sequencen{G_n}$ of $A$ such that $G_n\in\Cal G_n$, that is,
$G_n\times G_n\subseteq W_n$,  for each $n$.   As $\sequencen{W_n}$ is
arbitrary, $A\in\CalSmz(X,\Cal W)$.

\medskip

{\bf (d)(i)} Let $K\subseteq X$ be compact and $\Cal G$ an open cover
of $X$.   Then there is a $W\in\Cal W$ such that whenever $x\in K$ there
is a $G\in\Cal G$ such that $W[\{x\}]\subseteq G$.   \Prf\ (Cf.\ 2A2Ed.)
Set

\Centerline{$Q
=\{(x,V):x\in X$, $V\in\Cal W$, $V[V[\{x\}]]\subseteq G$ for some
$G\in\Cal G\}$.}

\noindent Then for every $x\in X$ there are a $G\in\Cal G$ such that
$x\in G$ and a $V\in\Cal W$ such that $V[V[\{x\}]]\subseteq G$, and in
this case $(x,V)\in Q$ and $x\in\interior V[\{x\}]$.   So
$\{\interior V[\{x\}]:(x,V)\in Q\}$ is an open cover of $X$ and
there is a finite set $Q_0\subseteq Q$ such that
$K\subseteq\bigcup\{\interior V[\{x\}]:(x,V)\in Q_0\}$.
Let $W\in\Cal W$ be such
that $W\subseteq V$ whenever $(x,V)\in Q_0$.   If $x\in K$, there
is an $(x',V)\in Q_0$ such that $x\in V[\{x'\}]$;  and now there is a
$G\in\Cal G$ including $V[V[\{x'\}]]\supseteq W[\{x\}]$.\ \Qed

\medskip

\quad{\bf (ii)} Let $K\subseteq X$ be compact and $A\in\CalSmz(X,\Cal W)$.
Then $A\cap K\in\CalRbg(X)$.
\Prf\ Let $\sequencen{\Cal G_n}$ be a sequence of non-empty open
covers of $X$.   For each $n\in\Bbb N$ let $W_n\in\Cal W$ be such that
$\{W[\{x\}]:x\in K\}$ refines $\Cal G_n$.   Let $\sequencen{A_n}$ be a
cover of $A$ such that $A_n\times A_n\subseteq W_n$ for every $n$.
If
$n\in\Bbb N$ and $A_n\cap K=\emptyset$, take any $G_n\in\Cal G_n$.
Otherwise, take $x_n\in A_n\cap K$ and $G_n\in\Cal G_n$ such that
$W_n[\{x_n\}]\subseteq G_n$.   If $x\in A\cap K$, there is an
$n\in\Bbb N$ such that $x\in A_n$;  now
$(x_n,x)\in A_n\times A_n\subseteq W_n$ and
$x\in W_n[\{x_n\}]\subseteq G_n$.  As $x$ is arbitrary,
$A\cap K\subseteq\bigcup_{n\in\Bbb N}G_n$.   As $\sequencen{\Cal G_n}$
is arbitrary, $A\cap K$ has Rothberger's property in $X$.\ \Qed

\medskip

\quad{\bf (iii)} $\CalSmz(X,\Cal W)\subseteq\CalRbg(X)$.
\Prf\ If $A\in\CalSmz(X,\Cal W)$, let
$\sequencen{K_n}$ be a sequence of compact subsets of $X$ covering $X$.
By (iii), $A\cap K_n\in\CalRbg(X)$ for each $n$;
by (a) above, $A\in\CalRbg(X)$.\ \QeD\  By (c), we have equality.
}%end of proof of 534F

\leader{534G}{Corollary} If $X$ is a $\sigma$-compact completely
regular topological space, then any uniformity on $X$ compatible with
its topology gives the same sets of strong measure zero.

\proof{ Immediate from 534Fd.
}%end of proof of 534G

\vleader{48pt}{534H}{Theorem} Let $X$ be a $\sigma$-compact locally compact
Hausdorff topological group and $A$ a subset of $X$.   Then the
following are equiveridical:

(i) $A$ has Rothberger's property in $X$;

(ii) for any sequence $\sequencen{U_n}$ of neighbourhoods of the
identity $e$ of $X$, there is a sequence $\sequencen{x_n}$ in $X$ such
that $A\subseteq\bigcup_{n\in\Bbb N}U_nx_n$;

(iii) $FA\ne X$ for any nowhere dense set $F\subseteq X$;

(iv) $EA\ne X$ for any meager set $E\subseteq X$;

(v) $AF\ne X$ for any nowhere dense set $F\subseteq X$;

(vi) $AE\ne X$ for any meager set $E\subseteq X$.

\cmmnt{\medskip

\noindent{\bf Remark} For the general theory of topological groups see
\S4A5 and Chapter 44.   Readers unfamiliar with this theory, or
impatient with the extra discipline needed to deal with
non-commutative
groups, may prefer to start by assuming that $X=\BbbR^2$, so that
every $xU$ becomes $x+U$, and every $V^{-1}V$ becomes $V-V$.
}%end of comment

\proof{{\bf (i)$\Rightarrow$(ii)} Suppose that (i) is true, and that
$\sequencen{U_n}$ is any sequence of neighbourhoods of $e$.   Then
$\Cal G_n=\{\interior U_nx:x\in X\}$ is an open cover of $X$, so there
is a sequence $\sequencen{x_n}$ such that
$A\subseteq\bigcup_{n\in\Bbb N}U_nx_n$.


\medskip

{\bf (ii)$\Rightarrow$(i)} Suppose that (ii) is true, and that
$\sequencen{W_n}$ is any sequence in the right uniformity of $X$.   Then
for each $n\in\Bbb N$ there is a neighbourhood $U_n$ of $e$ such that
$W_n\supseteq\{(x,y):xy^{-1}\in U_n\}$;  let $V_n$ be a neighbourhood
of $e$ such that $V_nV_n^{-1}\subseteq U_n$.   By (ii), there is a
sequence $\sequencen{x_n}$ such
that $A\subseteq\bigcup_{n\in\Bbb N}V_nx_n$.   Set $A_n=V_nx_n$ for
each $n$.   Then $A_nA_n^{-1}=V_nV_n^{-1}\subseteq U_n$, so
$A_n\times A_n\subseteq W_n$, for each $n$, while $\sequencen{A_n}$
covers $A$.   As $\sequencen{W_n}$ is arbitrary, $A$ has strong
measure zero.   By 534Fd, $A\in\CalRbg(X)$.

\medskip

{\bf (ii)$\Rightarrow$(iv)} Suppose that $A$ satisfies (ii), and that
$E\subseteq X$ is meager.

\medskip

\quad\grheada\
If $K\subseteq X$ is compact and nowhere dense, then
there is a sequence $\sequencen{U_n}$ of neighbourhoods of $e$ such
that $K'=\bigcap_{n\in\Bbb N}U_nK$ is still compact and nowhere dense.
\Prf\ By 443N(ii), there is a nowhere dense zero set $F\supseteq K$.
Now $F$ is a G$_{\delta}$ set;  suppose that $F=\bigcap_{n\in\Bbb N}G_n$
where $G_n$ is open for each $n$.   As $K\subseteq G_n$,
the open set $U'_n=\{x:xK\subseteq G\}$ (4A5Ei) contains $e$;  let
$U_n$ be a compact
neighbourhood of $e$ included in $U'_n$.   Then $U_nK\subseteq G_n$
for every $n$, so $K'=\bigcap_{n\in\Bbb N}U_nK\subseteq F$ is nowhere
dense, while $K'$ is compact (use 4A5Ef).\ \Qed

\medskip

\quad\grheadb\ Let $K\subseteq X$ be compact and nowhere dense and
$U$ a neighbourhood of $e$.   Then there is a neighbourhood $V$ of $e$ such
that for every $x\in X$ there is an $x'\in Ux$ such that
$Vx'\cap K=\emptyset$.   \Prf\ Let $\sequencen{U_n}$ be a sequence of
neighbourhoods of $e$ such that $K'=\bigcap_{n\in\Bbb N}U_nK$ is
compact and nowhere dense (($\alpha$) above).   Choose a sequence
$\sequencen{V_n}$ of compact neighbourhoods of $e$ such that
$V_0\subseteq U$ and $V_{n+1}V_{n+1}^{-1}\subseteq U_n\cap V_n$ for
each $n\in\Bbb N$.   Then $Y=\bigcap_{n\in\Bbb N}V_n$ is a compact subgroup
of $X$ (see the proof of 4A5S), and $YK=\bigcap_{n\in\Bbb N}V_nK$
(4A5Eh).   \Quer\ If for every $n\in\Bbb N$ there is an
$x_n\in X$ such that $V_n^{-1}x'\cap K\ne\emptyset$ for every
$x'\in Ux_n$, then, in particular, $V_n^{-1}x_n\cap K\ne\emptyset$, so
$x_n\in V_nK$.   Since $\sequencen{V_nK}$ is a non-increasing sequence
of compact sets, $\sequencen{x_n}$ has a cluster point

\Centerline{$x^*\in\bigcap_{n\in\Bbb N}V_nK=YK\subseteq K'$.}

Because $K'$ is nowhere dense, $V_1x^*\not\subseteq K'$;  take
$x\in V_1x^*\setminus K'$.   Let $W$ be an open
neighbourhood of $e$ such that $Wx\cap K'=\emptyset$.   Then $Wx$ is
disjoint from $YK=Y^{-1}YK$ so $YWx\cap YK=\emptyset$.   Now
$YW$ is an open set including $Y=\bigcap_{n\in\Bbb N}V_n$, and
all the $V_n$ are compact, so there is an $m\ge 1$ such that
$V_m\subseteq Y W$ and $V_mx\cap YK=\emptyset$.

But observe that there is an $n>m$ such that
$x_n\in V_1x^*$, so that

\Centerline{$x\in V_1V_1^{-1}x_n\subseteq V_0x_n\subseteq Ux_n$,}

\noindent while $V_n^{-1}x\cap K\subseteq V_mx\cap YK$ is empty.\
\Bang

Thus we can take $V=V_n^{-1}$ for some $n$.\ \Qed

\medskip

\quad\grheadc\ Because $X$ is
$\sigma$-compact, any F$_{\sigma}$ set in $X$ is actually K$_{\sigma}$,
and there is
a sequence $\sequencen{K_n}$ of nowhere dense compact sets covering $E$;
we can suppose that $\sequencen{K_n}$ is non-decreasing.   Choose
inductively sequences $\sequencen{U_n}$, $\sequencen{V_n}$,
$\sequencen{V'_n}$ and $\sequencen{V''_n}$ of neighbourhoods of $e$
such that

\inset{$U_0$ is any compact neighbourhood of $e$,

given $U_n$, $V_n$ is to be a neighbourhood of $e$ such that
$V_nV_n\subseteq U_n$,

given $V_n$, $V'_n$ is to be a neighbourhood of $e$ such that for
every $y\in X$ there is a $z\in V_ny$ such that $V'_nz\cap
K_{n+1}=\emptyset$}

\noindent (using ($\beta$)),

\inset{given $V'_n$, $V''_n$ is to be an open neighbourhood of $e$
such that $(V''_n)^{-1}V''_n\subseteq V'_n$,

given $V''_n$, $U_{n+1}$ is to be a compact neighbourhood of $e$,
included in $V_n\cap V''_n$, such that
$K_{n+1}U_{n+1}\subseteq V''_nK_{n+1}$.}

\noindent (This last is possible by 4A5Ei, because $V''_nK_{n+1}$ is
an open set including $K_{n+1}$,
so $\{x:K_{n+1}x\subseteq V''_nK_{n+1}\}$ is
an open set containing $e$.)

\medskip

\quad\grheadd\ For each $k\in\Bbb N$, $\sequence{i}{U_{2^k(2i+1)}}$
is a sequence of neighbourhoods of $e$, so there must be a sequence
$\sequence{i}{x^{(k)}_i}$ such that
$A\subseteq\bigcup_{i\in\Bbb N}U_{2^k(2i+1)}x^{(k)}_i$.
Set $x_0=e$ and $x_n=x^{(k)}_i$ if $n=2^k(2i+1)$.   For any $k\in\Bbb N$,

\Centerline{$A\subseteq\bigcup_{i\in\Bbb N}A_{ki}
\subseteq\bigcup_{i\in\Bbb N}U_{2^k(2i+1)}x^{(k)}_i
\subseteq\bigcup_{n\ge 2^k}U_nx_n\subseteq\bigcup_{n\ge k}U_nx_n$.}

\noindent This means that $EA\subseteq\bigcup_{n\ge 1}K_nU_nx_n$.
\Prf\ If $z\in EA$, we can express it as $xy$ where $x\in E$ and
$y\in A$.   There are a $k\ge 1$ such that $x\in K_k$ and an
$n\ge k$ such that $y\in U_nx_n$, in which case $z\in K_nU_nx_n$.\
\Qed

\medskip

\quad\grheade\ Now choose $\sequencen{y_n}$, $\sequencen{z_n}$ as
follows.   Start from $y_0=e$.   Given $y_n$, let
$z_n\in V_ny_nx_{n+1}^{-1}$ be such that
$V'_nz_n\cap K_{n+1}=\emptyset$;  this is possible by the choice of
$V'_n$.   Now set $y_{n+1}=z_nx_{n+1}$, and continue.

For each $n$,

$$\eqalignno{U_{n+1}y_{n+1}
&\subseteq V_ny_{n+1}\cr
\displaycause{by the choice of $U_{n+1}$}
&=V_nz_nx_{n+1}
\subseteq V_nV_ny_nx_{n+1}^{-1}x_{n+1}\cr
\displaycause{by the choice of $z_n$}
&\subseteq U_ny_n\cr}$$

\noindent by the choice of $V_n$.   Consequently,
$U_{n+1}y_{n+1}\cap K_{n+1}U_{n+1}x_{n+1}=\emptyset$.   \Prf\ We chose
$z_n$ such that $V'_nz_n\cap K_{n+1}=\emptyset$.   Because
$(V''_n)^{-1}V''_n\subseteq V'_n$,
$V''_nz_n\cap V''_nK_{n+1}=\emptyset$.   Because
$K_{n+1}U_{n+1}\subseteq V''_nK_{n+1}$ and $U_{n+1}\subseteq V''_n$,
$U_{n+1}z_n\cap K_{n+1}U_{n+1}=\emptyset$, that is,
$U_{n+1}y_{n+1}\cap K_{n+1}U_{n+1}x_{n+1}=\emptyset$.\ \Qed

\medskip

\quad{\bf ($\pmb{\zeta}$)} From ($\epsilon$) we see that
$\sequencen{U_ny_n}$ is a non-increasing sequence of compact sets, so
has non-empty intersection.   Take any $x\in\bigcap_{n\in\Bbb
N}U_ny_n$.
Then $x\notin K_{n+1}U_{n+1}x_{n+1}$ for any $n$, so
$x\notin\bigcup_{n\ge 1}K_nU_nx_n\supseteq EA$.   Thus $EA\ne X$.   As
$E$ is arbitrary, (iv) is true.

\medskip

{\bf (iv)$\Rightarrow$(iii)} is trivial.

\medskip

{\bf (iii)$\Rightarrow$(ii)} Suppose that (iii) is true.   Let
$\sequencen{U_n}$ be any sequence of open neighbourhoods of $e$.
Then there is a sequence $\sequencen{x_n}$ in $X$ such that
$G=\bigcup_{n\in\Bbb N}x_nU_n^{-1}$ is dense.   \Prf\ Let
$\sequencen{V_n}$ be a sequence of neighbourhoods of $e$ such that
$V_{n+1}V_{n+1}^{-1}\subseteq V_n\cap U_n^{-1}$ for every $n\in\Bbb N$.
Then there is a compact normal subgroup $Y$ of $X$ such that
$Y\subseteq\bigcap_{n\in\Bbb N}V_n$ and $X/Y$ is metrizable (4A5S).
The canonical map $x\mapsto x^{\ssbullet}:X\to X/Y$ is continuous, so
$X/Y$ is $\sigma$-compact, therefore separable (4A2P(a-ii)).   Let
$\sequencen{x_n}$ be a sequence in $X$ such that
$\{x_n^{\ssbullet}:n\in\Bbb N\}$ is dense in $X/Y$.   Set
$G_0=\bigcup_{n\in\Bbb N}x_nV_{n+1}Y$.   \Quer\ If
$H=X\setminus\overline{G}_0$ is non-empty, then
$\{x^{\ssbullet}:x\in H\}$
is open (4A5Ja) so contains $x_n^{\ssbullet}$ for some $n$.   But
$x_nY\subseteq x_nV_{n+1}Y\subseteq G_0$, so there can be no $x\in H$
such that $x^{\ssbullet}=x_n^{\ssbullet}$.\ \BanG\  Thus $G_0$ is
dense. But, for any $n\in\Bbb N$,
$Y\subseteq V_{n+1}^{-1}$ so $V_{n+1}Y\subseteq U_n^{-1}$, and
$G=\bigcup_{n\in\Bbb N}x_nU_n^{-1}$ includes $G_0$.   Thus $G$ is dense,
as required.\ \Qed

Accordingly $F=X\setminus G$ is nowhere dense, and
$FA\ne X$;  suppose $x\in X\setminus FA$.   Then
$F\cap xA^{-1}=\emptyset$, that is,
$xA^{-1}\subseteq\bigcup_{n\in\Bbb N}x_nU_n^{-1}$, that is,
$A^{-1}\subseteq\bigcup_{n\in\Bbb N}x^{-1}x_nU_n^{-1}$, that is,
$A\subseteq\bigcup_{n\in\Bbb N}U_nx_n^{-1}x$.   As $\sequencen{U_n}$ is
arbitrary, (ii) is true.

\medskip

{\bf (i)$\Leftrightarrow$(v)$\Leftrightarrow$(vi)} Because $x\mapsto
x^{-1}$ is a
homeomorphism,

$$\eqalignno{A\in\CalRbg(X)
&\mskip-6mu\iff A^{-1}\in\CalRbg(X)\cr
& \Longrightarrow EA^{-1}\ne X
  \text{ whenever }E\subseteq X\text{ is meager}\cr
& \Longrightarrow E^{-1}A^{-1}\ne X
  \text{ whenever }E\subseteq X\text{ is meager}\cr
\displaycause{because $E^{-1}$ is meager if $E$ is}
&\mskip-6mu\iff AE\ne X\text{ whenever }E\subseteq X
  \text{ is meager}\cr
& \Longrightarrow AF^{-1}\ne X\text{ whenever }F\subseteq X
  \text{ is nowhere dense}\cr
\displaycause{because $F^{-1}$ is nowhere dense if $F$ is}
& \Longrightarrow FA^{-1}\ne X\text{ whenever }F\subseteq X
  \text{ is nowhere dense}\cr
& \Longrightarrow A^{-1}\in\CalRbg(X).\cr}$$
}%end of proof of 534H

\cmmnt{\medskip

\noindent{\bf Remark} The case $X=\Bbb R$ is due to {\smc Galvin
Mycielski \& Solovay 79}.
}%end of comment

\leader{534I}{Proposition} (a) Let $X$ be a Lindel\"of space.
Then $\non\CalRbg(X)\ge\frakmctbl$.

(b)\cmmnt{ ({\smc Fremlin \& Miller 88})} Give $\NN$ the metric
$\rho$ defined by setting

\Centerline{$\rho(x,y)
=\inf\{2^{-n}:n\in\Bbb N$, $x\restr n=y\restr n\}$.}

\noindent Then $\non\CalSmz(\NN,\rho)=\frakmctbl$.

\proof{{\bf (a)} Suppose that $A\subseteq X$ and $\#(A)<\frakmctbl$.
Let $\sequencen{\Cal G_n}$ be a sequence of non-empty open covers of
$X$.   Because $X$ is Lindel\"of, we can choose for each $n$ a
non-empty
countable $\Cal G'_n\subseteq\Cal G_n$ covering $X$.
Let $P$ be the set of finite sequences $p=\ofamily{i}{n}{p(i)}$ such
that $p(i)\in\Cal G'_i$ for every $i<n$;  say
that $p\le q$ in $P$ if $q$ extends $P$.   Then $P$ is a countable
partially ordered set.   For each $x\in A$, the set
$Q_x=\{p:x\in p(i)$
for some $i<\#(p)\}$ is cofinal with $P$.   \Prf\ Given $p\in P$, set
$n=\#(p)$;  let $G\in\Cal G'_n$ be such that $x\in G$;  set
$q=p\cup\{(n,G)\}$;  then $p\le q\in Q_x$.\ \Qed

Because $\#(A)<\frakmctbl\le\frak m^{\uparrow}(P)$ (517Pc), there is
an upwards-directed family $R\subseteq P$ meeting every $Q_x$ (517B(iv)).
Now $p^*=\bigcup R$ is a function;  its domain $I$ is either $\Bbb N$ or
a natural number;  in either case, $A\subseteq\bigcup_{i\in I}p^*(i)$
and $p^*(i)\in\Cal G_i$ for every $i\in I$.   As
$\sequencen{\Cal G_n}$ is arbitrary, $A$ has Rothberger's property in
$X$;  as $A$ is arbitrary, $\non\CalRbg(X)\ge\frakmctbl$.

\medskip

{\bf (b)} By (a) and 534Fc, $\non\CalSmz(\NN,\rho)\ge\frakmctbl$.   By
522Sb, there is a set
$A\subseteq\NN$, with cardinal $\frakmctbl$, such that for every $y\in\NN$
there is an $x\in A$ such that $x(n)\ne y(n)$ for every $n$.   \Quer\
If $A\in\CalSmz(\NN,\rho)$, take a sequence $\sequencen{y_n}$ in $\NN$ such that
$A\subseteq\bigcup_{n\in\Bbb N}B(y_n,2^{-n-1})$.   Set $y(n)=y_n(n)$
for every $n$.   Then there is an $x\in A$ such that $x(n)\ne y(n)$ for
every $n$.   But in this case $x(n)\ne y_n(n)$ and
$x\restr n+1\ne y_n\restr n+1$ and $x\notin B(y_n,2^{-n-1})$ for every
$n$.\ \Bang

Thus $A$ witnesses that $\non\CalSmz(\NN,\rho)\le\frakmctbl$.
}%end of proof of 534I

\cmmnt{\medskip

\noindent{\bf Remark} Observe that $(\NN,\rho)$, as described in (b)
above, is isometric with $\BbbZ^{\Bbb N}$ with the metric defined by
the same formula, and that this metric is translation-invariant, so
induces
the topological group uniformity of $\BbbZ^{\Bbb N}$\cmmnt{ (4A5He)}.
}

%re  cov(\CalSmz) see Barto & Judah 8.4.7

\leader{534J}{Proposition}\cmmnt{ ({\smc Fremlin 91})} Let
$(X,\rho)$ be a separable metric space.
Then $\CalSmz(X,\rho)\prT\Cal N^{\frak d}$\cmmnt{,
where $\Cal N$ is the null ideal
of Lebesgue measure on $\Bbb R$ and $\frak d$ is the dominating
number (522A)}.

\proof{{\bf (a)} By 534A, there is a countable family $\Cal C$ of
subsets of $X$ such that whenever $A\subseteq X$ has finite diameter
and $\eta>0$, there is a $C\in\Cal C$ such that $A\subseteq C$ and
$\diam C\le\eta+2\diam A$.   For each $i\in\Bbb N$, let
$\sequence{j}{C_{ij}}$ be a sequence running over $\{C:C\in\Cal C$,
$\diam C\le 2^{-i}\}$.   Let $(\NN,\subseteq^*,\Cal S)$ be the
$\Bbb N$-localization relation.

\medskip

{\bf (b)} Let $D\subseteq\NN$ be a cofinal set with cardinal $\frak d$.
For each $d\in D$ we can find a function $\phi_d:\CalSmz(X,\rho)\to\NN$ such
that $A\subseteq\bigcap_{n\in\Bbb N}\bigcup_{i\ge n}
C_{d(i),\phi_d(A)(i)}$ for every $A\in\CalSmz(X,\rho)$.   \Prf\ For
$A\in\CalSmz(X,\rho)$ and $k\in\Bbb N$, choose a sequence
$\sequence{i}{A_{ki}}$
of sets covering $A$ such that $2\diam A_{ki}<2^{-d(2^k(2i+1))}$ for
every $i\in\Bbb N$.   For $n=2^k(2i+1)$, let $A_n\in\Cal C$ be such
that $A_{ki}\subseteq A_n$ and $\diam A_n\le 2^{-d(n)}$;  choose
$\phi_d(A)(n)$ such that $A_n=C_{d(n),\phi_d(A)(n)}$.\ \QeD\  Define
$\phi:\CalSmz(X,\rho)\to(\NN)^D$ by setting
$\phi(A)=\family{d}{D}{\phi_d(A)}$
for $A\in\CalSmz(X,\rho)$.

\medskip

{\bf (c)} For $S\in\Cal S$ and $d\in D$, define

\Centerline{$\psi_d(S)
=\bigcap_{n\in\Bbb N}\bigcup_{i\ge n}\bigcup_{j\in S[\{i\}]}C_{d(i),j}
\subseteq X$.}

\noindent For $\family{d}{D}{S_d}\in\Cal S^D$ set
$\psi(\family{d}{D}{S_d})=\bigcap_{d\in D}\psi_d(S_d)$.   Then
$A=\psi(\family{d}{D}{S_d})$ has strong measure zero.   \Prf\ Let
$\sequence{i}{\epsilon_i}$ be any family of strictly positive real
numbers.   Let $d\in D$ be such that $2^{-d(k)}\le\epsilon_i$ whenever
$k\in\Bbb N$ and $i<2^{k+1}$.   For each $k\in\Bbb N$,
$\#(S_d[\{k\}])\le 2^k$, so we can find a sequence $\sequence{i}{A_i}$
such that
$\langle A_i\rangle_{2^k\le i<2^{k+1}}$ is a re-enumeration of
$\family{j}{S[\{k\}]}{C_{d(k),j}}$ supplemented by empty sets if
necessary.   This will ensure that if $2^k\le i<2^{k+1}$ then $\diam
A_i\le 2^{-d(k)}\le\epsilon_i$, while

\Centerline{$A\subseteq\psi_d(S_d)
\subseteq\bigcup_{(k,j)\in S_d}C_{d(k),j}
=\bigcup_{i\in\Bbb N}A_i$.}

\noindent As $\sequence{i}{\epsilon_i}$ is arbitrary, $A\in\CalSmz(X,\rho)$.\
\Qed

\medskip

{\bf (d)} Now $(\phi,\psi)$ is a Galois-Tukey connection from
$(\CalSmz(X,\rho),\subseteq,\CalSmz(X,\rho))$ to
$(\NN,\penalty-100\subseteq^*\nobreak,\penalty-100\Cal S)^D$, that is,
$((\NN)^D,T,\penalty-100\Cal S^D)$,
where $T$ is the simple product relation as defined in 512H.   \Prf\
$\phi:\CalSmz(X,\rho)\to(\NN)^D$
and $\psi:\Cal S^D\to\CalSmz(X,\rho)$ are functions.   Suppose that
$A\in\CalSmz(X,\rho)$ and $\family{d}{D}{S_d}$ are such that
$(\phi(A),\family{d}{D}{S_d})\in T$, that is,
$\phi_d(A)\subseteq^*S_d$
for every $d$.   Fix $d\in D$ for the moment.   Then there is an
$n\in\Bbb N$ such that $(i,\phi_d(A)(i))\in S_d$ for every $i\ge n$.
Now, for any $m\ge n$,

\Centerline{$A\subseteq\bigcup_{i\ge m}C_{d(i),\phi_d(A)(i)}
\subseteq\bigcup_{i\ge m}\bigcup_{j\in S_d[\{i\}]}C_{d(i),j}$.}

\noindent Thus

\Centerline{$A\subseteq
\bigcap_{m\in\Bbb N}\bigcup_{i\ge m}\bigcup_{j\in
S_d[\{i\}]}C_{d(i),j)}
=\psi_d(S_d)$.}

\noindent This is true for every $d$, so
$A\subseteq\psi(\family{d}{D}{S_d})$.   As $A$ and
$\family{d}{D}{S_d}$
are arbitrary, $(\phi,\psi)$ is a Galois-Tukey connection.\ \Qed

\medskip

{\bf (e)} Thus
$(\CalSmz(X,\rho),\subseteq,\CalSmz(X,\rho))
\prGT(\NN,\subseteq^*,\Cal S)^D$.   But
$(\NN,\subseteq^*,\Cal S)
\equivGT(\Cal N,\penalty-100\subseteq\nobreak,\penalty-100\Cal N)$ (522M), so
$(\NN,\subseteq^*,\penalty-100\Cal S)^D
\equivGT(\Cal N,\subseteq,\Cal N)^D$ (512Hb) and

\Centerline{$(\CalSmz(X,\rho),\subseteq,\CalSmz(X,\rho))
\prGT(\Cal N,\subseteq,\Cal N)^D=(\Cal N^D,\le,\Cal N^D)$}

\noindent where $\le$ is the natural partial order of the product
partially ordered set $\Cal N^D$.   Accordingly
$\CalSmz(X,\rho)\prT\Cal N^D\cong\Cal N^{\frak d}$, as claimed.
}%end of proof of 534J

\leader{534K}{Corollary} Let $(X,\Cal W)$ be a Lindel\"of uniform
space.   Then $\add\CalSmz(X,\Cal W)\ge\add\Cal N$,
where $\Cal N$ is the null ideal of Lebesgue measure on $\Bbb R$.

\proof{{\bf (a)} Suppose to begin with that $\Cal W$ is metrizable.
Then $X$ is separable (4A2Pd), so $\CalSmz(X,\rho)\prT\Cal N^{\frak d}$,
and $\add\Cal N=\add\Cal N^{\frak d}\le\add\CalSmz(X,\rho)$ by 513E(e-ii).

\medskip

{\bf (b)} For the general case, suppose that
$\Cal A\subseteq\CalSmz(X,\Cal W)$
and $\#(\Cal A)<\add\Cal N$, and set $A^*=\bigcup\Cal A$.   Let $f$ be a
uniformly continuous function from $X$ to a metric space $Y$.   Then
$f[X]$ is Lindel\"of (5A4Bc), and $f[A]$ has strong measure zero in
$f[X]$ for every $A\in\Cal A$ (534Db), so
$f[A^*]=\bigcup_{A\in\Cal A}f[A]$ has strong measure zero, by (a).
As $f$ is arbitrary, $A^*$ has strong measure zero, by 534Dc;  as $\Cal A$
is arbitrary, $\add\CalSmz(X,\Cal W)\ge\add\Cal N$.
}%end of proof of 534K

\leader{534L}{$\CalSmz$-equivalence (a)} If $(X,\Cal V)$ and
$(Y,\Cal W)$ are uniform spaces, \cmmnt{I say that} they are
{\bf $\CalSmz$-equivalent} if there is a bijection $f:X\to Y$ such
that
a set $A\subseteq X$ has strong measure zero in $X$ iff $f[A]$ has
strong measure zero in $Y$.   \cmmnt{Of course this is an
equivalence
relation on the class of uniform spaces.}

\spheader 534Lb If $(X,\Cal V)$ and $(Y,\Cal W)$ are uniform
spaces,\cmmnt{ I
say that} $X$ is {\bf $\CalSmz$-embeddable} in $Y$ if it is
$\CalSmz$-equivalent to a subspace of $Y$\cmmnt{ (with the subspace
uniformity, of course)}.   Evidently this is transitive in the sense
that if $X$ is $\CalSmz$-embeddable in $Y$ and $Y$ is
$\CalSmz$-embeddable in $Z$ then $X$ is $\CalSmz$-embeddable in $Z$.

\spheader 534Lc When $X$ and $Y$ are topological spaces,\cmmnt{ I
will say that} they are {\bf $\CalRbg$-equivalent} if there is a
bijection $f:X\to Y$ such that
$A\subseteq X$ has Rothberger's property in $X$ iff $f[A]$ has
Rothberger's property in $Y$.

\leader{534M}{Lemma} (a) Suppose that $(X,\Cal V$) and $(Y,\Cal W)$
are uniform spaces, and that $\sequencen{X_n}$, $\sequencen{Y_n}$ are
partitions of $X$, $Y$ respectively such that $X_n$ is
$\CalSmz$-equivalent to $Y_n$ for every $n$.   Then $X$ is
$\CalSmz$-equivalent to $Y$.

(b) Suppose that $(X,\Cal V)$ and $(Y,\Cal W)$ are uniform spaces, and
that $X$ is $\CalSmz$-embeddable in $Y$ and $Y$ is
$\CalSmz$-embeddable
in $X$.   Then $X$ and $Y$ are $\CalSmz$-equivalent.

\proof{{\bf (a)} For each $n\in\Bbb N$, let $f_n:X_n\to Y_n$ be a
bijection identifying the ideals of sets with strong measure zero.
Then $f=\bigcup_{n\in\Bbb N}f_n$ is a bijection identifying
$\CalSmz(X,\Cal V)$ and $\CalSmz(Y,\Cal W)$.

\medskip

{\bf (b)} (Compare 344D.) Let $X_1\subseteq X$ and $Y_1\subseteq Y$ be
$\CalSmz$-equivalent to $Y$, $X$ respectively;  let $f:X\to Y_1$ and
$g:Y\to X_1$ be bijections identifying the ideals of strong measure
zero in each pair.   Set $X_0=X$, $Y_0=Y$, $X_{n+1}=g[Y_n]$ and
$Y_{n+1}=f[X_n]$ for
each $n\in\Bbb N$;  then $\sequencen{X_n}$ is a non-increasing sequence
of subsets of $X$ and $\sequencen{Y_n}$ is a non-increasing sequence of
subsets of $Y$.   Set $X_{\infty}=\bigcap_{n\in\Bbb N}X_n$,
$Y_{\infty}=\bigcap_{n\in\Bbb N}Y_n$.   Then
$f\restr X_{2k}\setminus X_{2k+1}$ is an $\CalSmz$-equivalence between
$X_{2k}\setminus X_{2k+1}$ and $Y_{2k+1}\setminus Y_{2k+2}$, while
$g\restrp Y_{2k}\setminus Y_{2k+1}$ is an $\CalSmz$-equivalence
between
$Y_{2k}\setminus Y_{2k+1}$ and $X_{2k+1}\setminus X_{2k+2}$;  and
$g\restrp Y_{\infty}$ is an $\CalSmz$-equivalence between $Y_{\infty}$
and $X_{\infty}$.    So (a) gives the required $\CalSmz$-equivalence
between $X$ and $Y$.
}%end of proof of 534M

\leader{534N}{Proposition} $\BbbR^r$, $[0,1]^r$ and $\{0,1\}^{\Bbb N}$
are $\CalRbg$-equivalent for every integer $r\ge 1$.

\proof{{\bf (a)} Give $\Bbb R$ its usual uniformity.   Of course the
identity map is an $\CalSmz$-embedding of
$[0,1]$ in $\Bbb R$.   In the other direction, any homeomorphism from
$\Bbb R$ to
$\ooint{0,1}$ is an $\CalRbg$-equivalence between $\Bbb R$ and
$\ooint{0,1}$ and therefore, because $\Bbb R$ and
$\ooint{0,1}$ are $\sigma$-compact, an $\CalSmz$-embedding of $\Bbb R$ in
$[0,1]$ (534Fd).   By 534Mb, $\Bbb R$ and $[0,1]$ and $\ooint{0,1}$ are
$\CalSmz$-equivalent.

\medskip

{\bf (b)} Give $\{0,1\}^{\Bbb N}$ the metric $\rho$ defined by saying
that

\Centerline{$\rho(x,y)
=\inf\{2^{-n}:n\in\Bbb N$, $x\restr n=y\restr n\}$}

\noindent for $x$, $y\in\{0,1\}^{\Bbb N}$.   Define
$f:\{0,1\}^{\Bbb N}\to[0,1]$ by setting
$f(x)=\sum_{n=0}^{\infty}2^{-n-1}x(n)$ for $x\in\{0,1\}^{\Bbb N}$.
Then $f$ is continuous, therefore uniformly continuous, so $f[A]$ has
strong measure zero in $[0,1]$ whenever $A\subseteq\{0,1\}^{\Bbb N}$
has strong measure zero in $\{0,1\}^{\Bbb N}$.   It is also the case that
$f^{-1}[B]$ has strong measure zero whenever $B\subseteq[0,1]$ does.
\Prf\ Let $\sequencen{\epsilon_n}$ be any sequence of strictly
positive
numbers.   Then there is a sequence $\sequencen{B_n}$, covering $B$,
such that $\diam B_n<\bover12\min(1,\epsilon_{2n},\epsilon_{2n+1})$
for every $n$.   Fix $n$ for the moment and consider $f^{-1}[B_n]$.   If
$k$ is such that $2^{-k-1}\le\diam B_n<2^{-k}$, then $B_n$ can meet at
most two intervals of the type $I_{ki}=[2^{-k}i,2^{-k}(i+1)]$.   So
$f^{-1}[B_n]$
can meet at most two sets of the type $\{x:x\restr k=z\}$, and we can
express it as $A_{2n}\cup A_{2n+1}$ where

\Centerline{$\max(\diam A_{2n},\diam A_{2n+1})
\le 2^{-k}\le 2\diam B_n\le\min(\epsilon_{2n},\epsilon_{2n+1})$.}

\noindent Putting these together, we have a cover $\sequencen{A_n}$ of
$\bigcup_{n\in\Bbb N}f^{-1}[B_n]\supseteq f^{-1}[B]$ such that
$\diam A_n\le\epsilon_n$ for every $n$;  as $\sequencen{\epsilon_n}$
is arbitrary, $f^{-1}[B]$ has strong measure zero.\ \Qed

Of course $f$ is not a bijection, so it is not in itself an
$\CalSmz$-equivalence.   But if we set

\Centerline{$D_1
=\{x:x\in\{0,1\}^{\Bbb N}$, $x$ is eventually constant$\}$,}

\Centerline{$D_2=\{2^{-k}i:k\in\Bbb N$, $i\le 2^k\}$,}

\noindent then $D_1\subseteq\{0,1\}^{\Bbb N}$ and $D_2\subseteq[0,1]$
are countably infinite, and $f\restr\{0,1\}^{\Bbb N}\setminus D_1$ is
an $\CalSmz$-equivalence between $\{0,1\}^{\Bbb N}\setminus D_1$ and
$[0,1]\setminus D_2$.   Putting this together with any bijection
between $D_1$ and $D_2$, we have an $\CalSmz$-equivalence between
$\{0,1\}^{\Bbb N}$ and $[0,1]$.

\medskip

{\bf (c)(i)} I show by induction on $r$ that $[0,1]^r$ is
$\CalSmz$-equivalent to $\Bbb R$ and therefore to $[0,1]$.   The case
$r=1$ is covered by (a).   For the inductive step to $r\ge 2$, I adapt
the method of (b).   Give $\{0,1\}^{\Bbb N\times r}$ the metric $\rho$
defined by setting

\Centerline{$\rho(x,y)
=\inf\{2^{-n}:n\in\Bbb N$, $x\restr(n\times r)=y\restr(n\times r)\}$}

\noindent for $x$, $y\in\{0,1\}^{\Bbb N\times r}$.   Define
$f:\{0,1\}^{\Bbb N\times r}\to[0,1]^r$ by setting

\Centerline{$f(x)=\ofamily{j}{r}{\sum_{i=0}^{\infty}2^{-i-1}x(i,j)}$}

\noindent for $x\in \{0,1\}^{\Bbb N\times r}$.   Then $f$ is uniformly
continuous, so $f[A]$ has strong measure zero in $[0,1]^r$ whenever
$A$
has strong measure zero in $\{0,1\}^{\Bbb N\times r}$.   Moreover, we
find once again that $f^{-1}[B]$ has strong measure zero whenever
$B\subseteq[0,1]^r$ has strong measure zero.   \Prf\ Let
$\sequencen{\epsilon_n}$ be a sequence of strictly positive real
numbers.   This time, set $m=2^r$ and let $\sequence{n}{B_n}$ be a
cover of $B$ such that
$\diam B_n<\bover12\min(1,\inf_{mn\le i<mn+m}\epsilon_i)$ for every
$n$.   (For definiteness, let me say that I am giving $[0,1]^r$ its
Euclidean
metric.)   In this case, if $2^{-k-1}\le\diam B_n<2^{-k}$, $B_n$ can
meet at most $2^r$ intervals of the form
$[2^{-k}\pmb{n},2^{-k}(\pmb{n}+\pmb{1})]$ where $\pmb{n}\in\BbbN^r$ and
$\pmb{1}=(1,\ldots,1)$.   So $f^{-1}[B_n]$ can meet at most $2^r=m$ sets
of the form $\{x:x\restr(k\times r)=z\}$, and can be covered by $m$
sets $\langle A_j\rangle_{mn\le j<mn+m}$ where

\Centerline{$\diam A_j\le 2^{-k}\le 2\diam B_n\le\epsilon_j$}

\noindent for every $j$.   Putting these together, we have a cover
$\sequence{j}{A_j}$ of $f^{-1}[B]$ such that $\diam A_j\le\epsilon_j$
for every $j$;  as $\sequencen{\epsilon_n}$ is arbitrary, $f^{-1}[B]$
has strong measure zero.\ \Qed

The function $f$ here is very far from being one-to-one.   But if we
set

\Centerline{$D_1^*=\bigcup_{j<r}\{x:x\in\{0,1\}^{\Bbb N\times r}$,
$\sequence{i}{x(i,j)}\in D_1\}$,}

\Centerline{$D_2^*=\bigcup_{j<r}\{z:z\in[0,1]^r$, $z(j)\in D_2\}$,}

\noindent where $D_1\subseteq\{0,1\}^{\Bbb N}$, $D_2\subseteq[0,1]$
are defined as in the proof of (b), then $f$ is a bijection between
$\{0,1\}^{\Bbb N\times r}\setminus D^*_1$ and
$[0,1]^r\setminus D^*_2$,
so is an $\CalSmz$-equivalence between these.   Accordingly
$[0,1]^r\setminus D^*_2$ is $\CalSmz$-embeddable in
$\{0,1\}^{\Bbb N\times r}$, which is homeomorphic, therefore uniformly
equivalent, to $\{0,1\}^{\Bbb N}$, which is in turn
$\CalSmz$-equivalent
to $\ooint{0,1}$;  so  $[0,1]^r\setminus D^*_2$ is
$\CalSmz$-embeddable in $\ooint{0,1}$.

Now consider $D_2^*$.   This is a countable union of sets which are
isometric, therefore $\CalSmz$-equivalent, to $[0,1]^{r-1}$ and
therefore to $\ooint{0,1}$, by the inductive hypothesis.   We can
therefore express $D_2^*$ as $\bigcup_{n\in\Bbb N}X_n$ where
$\sequencen{X_n}$ is disjoint and every $X_n$ is $\CalSmz$-embeddable
in $\ooint{0,1}$ and therefore in $\ooint{n+1,n+2}$.   Assembling these
with the $\CalSmz$-equivalence between $[0,1]^r\setminus D_2^*$ and
$\ooint{0,1}$ we have already found, we have a $\CalSmz$-embedding
from $[0,1]^r$ to $\Bbb R$.   In the other direction, we certainly have an
isometric embedding of $[0,1]$ in $[0,1]^r$ and therefore a
$\CalSmz$-embedding of $\Bbb R$ in $[0,1]^r$;  so $\Bbb R$ and
$[0,1]^r$ are $\CalSmz$-equivalent.   Thus the induction proceeds.

\medskip

\quad{\bf (ii)} As for $\BbbR^r$, we have a homeomorphism between
$\BbbR^r$ and $\ooint{0,1}^r$, which (because these again are
$\sigma$-compact) is a $\CalSmz$-equivalence and therefore a
$\CalSmz$-embedding of $\BbbR^r$ in $[0,1]^r$.   So 534Mb, once more,
tells us that $\BbbR^r$ and $[0,1]^r$ and $[0,1]$ are
$\CalSmz$-equivalent.

\medskip

{\bf (d)} So $\BbbR^r$, $[0,1]^r$ and
$\{0,1\}^{\Bbb N}$ are $\CalSmz$-equivalent, for their usual
uniformities.
Now 534Fd tells us that they are also $\CalRbg$-equivalent.
}%end of proof of 534N

\leader{534O}{Large sets with strong measure
\dvrocolon{zero}}\cmmnt{
It is a remarkable fact that it is relatively consistent with ZFC to
suppose that the only subsets of $\Bbb R$ with strong measure zero are
the countable sets ({\smc Laver 76}, {\smc Ihoda 88} or
{\smc Bartoszy\'nski \& Judah 95}, \S8.3).   We therefore find
ourselves
investigating constructions of non-trivial sets with strong measure
zero
under special axioms.   I start by clearing the ground a little.

\medskip

\noindent}{\bf Proposition} (a) If $(X,\rho)$ is a separable metric
space and
$A\subseteq X$ has cardinal less than $\frak c$, there is a Lipschitz
function $f:X\to\Bbb R$ such that $f\restr A$ is injective.

(b)\cmmnt{ ({\smc Carlson 93})} For any cardinal $\kappa<\frak c$,
if there is any separable metric space with a set of size $\kappa$ which
is of strong measure zero, then there is a subset of $\Bbb R$ of size
$\kappa$ which has strong measure zero.

(c)\dvArevised{2010}(i) If
$\cf(\frakmctbl)=\frak b$ there is a
subset of $\Bbb R$ of size $\frakmctbl$ which has strong measure zero.

\quad(ii) If $\frakmctbl=\frak d$ there is a
subset of $\Bbb R$ of size $\frakmctbl$ which has strong measure zero.

\quad(iii)\cmmnt{ ({\smc Rothberger 41})}
If $\frak b=\omega_1$ there is a subset of $\Bbb R$ of size
$\omega_1$ which has strong measure zero.

\proof{{\bf (a)} If $X=\emptyset$ this is trivial.   Otherwise, let
$\sequencen{x_n}$ run over a dense sequence in $X$, and for $x\in X$
define $g_x:\Bbb R\to\Bbb R$ by setting

\Centerline{$g_x(t)
=\sum_{n=0}^{\infty}\Bover{\min(1,\rho(x,x_n))}{n!}t^n$}

\noindent for $t\in\Bbb R$.   Then $g_x$ is a real-entire function
(5A5A).   If $x$, $y\in X$ are distinct, then there must be some $n$
such that $\min(1,\rho(x,x_n))\ne\min(1,\rho(y,x_n))$, so that one of
the coefficients of $g_x-g_y$ is non-zero, and $\{t:g_x(t)=g_y(t)\}$
is countable (5A5A).   So if $A\subseteq X$ and $\#(A)<\frak c$, we can
find a $t\ge 0$ such that $g_x(t)\ne g_y(t)$ for all distinct $x$,
$y\in A$.   Set $f(x)=g_x(t)$ for $x\in X$;  then $f:X\to\Bbb R$ is a
function such that $f\restr A$ is injective.   If $x$, $y\in X$ then

$$\eqalign{|f(x)-f(y)|
&=|\sum_{n=0}^{\infty}(\min(1,\rho(x,x_n))-\min(1,\rho(y,y_n))
  \Bover{t^n}{n!}|\cr
&\le\exp(t)\sup_{n\in\Bbb N}|\rho(x,x_n)-\rho(y,y_n)|
\le\exp(t)\rho(x,y),\cr}$$

\noindent so that $f$ is Lipschitz.

\medskip

{\bf (b)} Let $(X,\rho)$ be a separable metric space with a set
$A\in[X]^{\kappa}$ of strong measure zero.   Then (a) tells us that we
have a uniformly continuous function $f:X\to\Bbb R$ which is injective
on $A$, so that $f[A]\in[\Bbb R]^{\kappa}$ has strong measure zero
(534Db).

\medskip

{\bf (c)(i)} Let
$\ofamily{\xi}{\frak b}{x_{\xi}}$ be a family
in $\NN$ which is increasing and
unbounded for the pre-order $\le^*$ of 522C(i).   Let
$C\subseteq\frakmctbl$ be a closed cofinal set with cardinal $\frak b$
(5A1Ad), and
$\ofamily{\xi}{\frak b}{\zeta_{\xi}}$ the increasing enumeration of $C$;
let $\ofamily{\eta}{\frakmctbl}{y_{\eta}}$ be a family of distinct elements
of $\NN$ such that $y_{\eta}\ge x_{\xi}$ whenever $\xi<\frak b$ and
$\zeta_{\xi}\le\eta<\zeta_{\xi+1}$.

If $K\subseteq\NN$ is compact, then $\{\eta:y_{\eta}\in K\}$ has cardinal
strictly less than $\frakmctbl$.   \Prf\ Set $x(n)=\sup_{y\in K}y(n)$ for
each $n\in\Bbb N$ (I pass over the trivial case $K=\emptyset$).   Then
there is a $\xi<\frak b$ such that $x_{\xi}\not\le^*x$.   If
$\zeta_{\xi}\le\eta<\frakmctbl$, there is a $\xi'\ge\xi$ such that
$\zeta_{\xi'}\le\eta<\zeta_{\xi'+1}$ (this is where we need to know that
$C$ is closed), and now

\Centerline{$y_{\eta}\ge x_{\xi'}\ge^*x_{\xi}$,
\quad$y_{\eta}\not\le x$,
\quad$y_{\eta}\notin K$.}

\noindent So $\{\eta:y_{\eta}\in K\}\subseteq\zeta_{\xi}$ has cardinal less
than $\frakmctbl$.\ \Qed

Let $f:\NN\to[0,1]\setminus\Bbb Q$ be any homeomorphism (4A2Ub), and
consider $A=\{f(y_{\eta}):\eta<\frakmctbl\}$.   Then $\#(A)=\frakmctbl$.
Also $A$ has strong measure zero.   \Prf\ Of course $\Bbb Q$, being
countable, has strong measure zero.   Let $G\subseteq\Bbb R$ be an
open set including $\Bbb Q$.   Then $[0,1]\setminus G$ and
$K=f^{-1}[\,[0,1]\setminus G]$ are compact.   Now
$\#(A\setminus G)=\#(\{\eta:y_{\eta}\in K\})<\frakmctbl$, so
$A\setminus G$ has Rothberger's property (534Ia) and strong measure zero
(534Fc).   By
534Dd, this is enough to show that $A$ has strong measure zero.\ \Qed

Thus we have a set of the required kind.

\medskip

\quad{\bf (ii)} The argument is similar.   This time, let
$\ofamily{\xi}{\frak d}{x_{\xi}}$ be a cofinal family in $\NN$.   For each
$\xi<\frak d$, let $y_{\xi}\in\NN$ be such that $y_{\xi}\ge x_{\xi}$ and
$y_{\xi}\not\le x_{\eta}$ for any $\eta<\xi$.   Again, if
$K\subseteq\NN$ is compact, then $\{\eta:y_{\eta}\in K\}$ has cardinal
strictly less than $\frakmctbl$.   \Prf\ Taking $x=\sup K$ as before,
there is a $\xi<\frak d=\frakmctbl$
such that $x\le x_{\xi}$;  now for any $\eta>\xi$
we know that $y_{\eta}\not\le x_{\xi}$ so $y_{\eta}\notin K$.\ \QeD\
The rest of the proof proceeds as before.   (The set
$\{y_{\eta}:\eta<\frak d\}$ has cardinal $\frak d$ because it is cofinal
with $\NN$.)

\medskip

\quad{\bf (iii)} If $\frakmctbl=\omega_1$, this is immediate from (i);
otherwise it is a consequence of 534Ia and 534Fc, as used in the arguments
above.
}%end of proof of 534O

\leader{534P}{}{ Subject to the continuum hypothesis we have many
ways of building sets with strong measure zero, in addition to that in
534Oc.   I give one example to
suggest what can be done with a weak form of Martin's axiom.

\medskip

\noindent}{\bf Example} Suppose that $\frakmctbl=\frak c$.   Then
there
is a set $A\subseteq\Bbb R\setminus\Bbb Q$, with Rothberger's property
in $\Bbb R$, such that

(i) $A+A=\Bbb R$,

(ii) $A\times A\notin\CalRbg(\BbbR^2)$,

(iii) there is a continuous surjection from $A$ onto $\Bbb R$,

(iv) $A\notin\CalRbg(\Bbb R\setminus\Bbb Q)$.

\proof{{\bf (a)} For $x\in\NN$, define $\psi(x)\in\{0,1\}^{\Bbb N}$ by
setting $\psi(x)(n)=0$ if $x(n)$ is even, $1$ if $x(n)$ is odd.
Then $\psi:\NN\to\{0,1\}^{\Bbb N}$ is a continuous surjection.   Let
$\phi:\NN\to[0,1]\setminus\Bbb Q$ be a homeomorphism (4A2Ub again).
Enumerate $\NN$ as
$\ofamily{\xi}{\frak c}{x_{\xi}}$ and $\Bbb R$ as
$\ofamily{\xi}{\frak c}{t_{\xi}}$ and $\Bbb Q$ as $\sequencen{q_n}$.
For $\xi\le\frak C$, set $K_{\xi}=\{x:x\in\NN$, $x\le x_{\xi}\}$, so that
$K_{\xi}$ is compact, and $\phi[K_{\eta}]$ is a compact subset of
$[0,1]\setminus\Bbb Q$, therefore nowhere dense in $\Bbb R$.
Write $\Cal M$ for the ideal of meager subsets of $\Bbb R$, as in
\S522.

Choose $\ofamily{\xi}{\frak c}{a_{\xi}}$,
$\ofamily{\xi}{\frak c}{b_{\xi}}$ and $\ofamily{\xi}{\frak
c}{c_{\xi}}$
as follows.   For each $\xi<\frak c$, $\{x_{\eta}:\eta\le\xi\}$ is not
cofinal with $\NN$, because

\Centerline{$\cf\NN=\frak d\ge\cov\Cal M=\frakmctbl=\frak c$}

\noindent (522I, 522Sa), so we can find a $y_{\xi}\in\NN$ such that
$y_{\xi}\not\le x_{\eta}$ for any $\eta\le\xi$;   raising $y_{\xi}$ if
need be, we can arrange that
$\psi(y_{\xi})=\psi(x_{\xi})$.   Set $a_{\xi}=\phi(y_{\xi})$.   Consider

\Centerline{$\Cal E_{\xi}
=\{\phi[K_{\eta}]+\Bbb Z:\eta\le\xi\}
 \cup\{t_{\xi}-\phi[K_{\eta}]+\Bbb Z:\eta\le\xi\}
 \cup\{\Bbb Q\}\cup\{t_{\xi}-\Bbb Q\}$.}

\noindent This is a family of fewer than $\frak c=\frakmctbl$ meager
subsets of $\Bbb R$, so does not cover $\Bbb R$ (522Sa once more).
Take any $b_{\xi}\in\Bbb R\setminus\bigcup\Cal E_{\xi}$;  then neither
$b_{\xi}$ nor $c_{\xi}=t_{\xi}-b_{\xi}$ belongs to
$\Bbb Q\cup\bigcup_{\eta\le\xi}(\Bbb Z+\phi[K_{\eta}])$.

\medskip

{\bf (b)} At the end of the process, set

\Centerline{$A_0=\{a_{\xi}:\xi<\frak c\}\cup\{b_{\xi}:\xi<\frak c\}
\cup\{c_{\xi}:\xi<\frak c\}$,
\quad$A=A_0+\Bbb Z$.}

\noindent Then $A\cap[0,1]$ has strong measure zero.   \Prf\ Let
$G\subseteq\Bbb R$ be any open set including $\Bbb Q$.    Then
$[0,1]\setminus G$ is a compact subset of $[0,1]\setminus\Bbb Q$, and
$K=\phi^{-1}[\,[0,1]\setminus G]$ is a compact subset of $\NN$.
There is therefore some $\xi<\frak c$ such that
$K\subseteq\{x:x\le x_{\xi}\}$
and $[0,1]\setminus G\subseteq \phi[K_{\xi}]$.   Now if $\eta\ge\xi$,
neither
$b_{\eta}$ nor $c_{\eta}$ belongs to $\phi[K_{\xi}]+\Bbb Z$;  and also
$a_{\eta}\in\ooint{0,1}\setminus \phi[K_{\xi}]$, so $a_{\eta}$ also
does
not belong to $\phi[K_{\xi}]+\Bbb Z$.   What this means is that

\Centerline{$A\cap[0,1]\setminus G
\subseteq(\{a_{\eta}:\eta\le\xi\}\cup\{b_{\eta}:\eta\le\xi\}
\cup\{c_{\eta}:\eta\le\xi\})+\Bbb Z$}

\noindent has cardinal less than $\frak c=\frakmctbl$.   By 534Ia
and 534Fc, it has
strong measure zero.   As $G$ is arbitrary, $A\cap[0,1]$ has strong
measure zero, by 534Dd.\ \Qed

Because $A+\Bbb Z=A$,

\Centerline{$A=\bigcup_{n\in\Bbb Z}(A\cap[0,1])+n$}

\noindent also has strong measure zero, therefore has Rothberger's
property (534Fd).

\medskip

{\bf (c)} Let us consider the properties (i)-(iv).   Because no
$a_{\xi}$, $b_{\xi}$ or $c_{\xi}$ belongs to $\Bbb Q$,
$A\cap\Bbb Q=\emptyset$.   For every $\xi<\frak c$,
$t_{\xi}=b_{\xi}+c_{\xi}\in A+A$;  as
$\ofamily{\xi}{\frak c}{t_{\xi}}$
is an enumeration of $\Bbb R$, $A+A=\Bbb R$.   Since
$+:\BbbR^2\to\Bbb R$ is continuous, and $+[A\times A]=A+A$
does not have Rothberger's property, nor does $A\times A$ (534Fb).
Thus we have (i) and (ii).

Let $h:\{0,1\}^{\Bbb N}\to[0,1]$ be a continuous surjection, and let
$f:\Bbb R\setminus\Bbb Q\to\Bbb R$ be the continuous
function
defined by setting $f(a)=h\psi\phi^{-1}(a-n)+n$ if
$a\in\ooint{n,n+1}\setminus\Bbb Q$ where $n\in\Bbb Z$.   For every
$\xi<\frak c$ such that $x_{\xi}\in\{0,1\}^{\Bbb N}$,

\Centerline{$f(a_{\xi})=h\psi\phi^{-1}(a_{\xi})=h(x_{\xi})$;}

\noindent because
$\{x_{\xi}:\xi<\frak c\}=\NN\supseteq\{0,1\}^{\Bbb N}$, $f[A]=[0,1]$.

Thus $A$ satisfies (iii).   As for (iv), define a metric $\rho$ on
$\Bbb R\setminus\Bbb Q$ by setting $\rho(s,t)=|s-t|+|f(s)-f(t)|$ for
$s$, $t\in\Bbb R\setminus\Bbb Q$.   Because $f$ is continuous, this
defines the usual topology of $\Bbb R\setminus\Bbb Q$.   But $f$ is
uniformly continuous for $\rho$ and the usual metric of $\Bbb R$, and
$f[A]$ does not have strong measure zero, so $A$ cannot have strong
measure zero for the metric $\rho$ (534Db), and does not have Rothberger's
property in $\Bbb R\setminus\Bbb Q$ (534Fc).   This completes the proof.
}%end of proof of 534P

\exercises{\leader{534X}{Basic exercises (a)}%
%\spheader 534Xa
(i) Let $(X,\rho)$ be a metric space, $r>0$ and $A\subseteq X$ a set
with strong measure zero.   Show that $A$ has zero Hausdorff
$r$-dimensional measure.   (ii) Find a subset of $\BbbR^2$ which is
universally
negligible but does not have strong measure zero (for the usual metric
on $\BbbR^2$).   \Hint{439H.}   (iii)
Find a subset of $\{0,1\}^{\Bbb N}$ which is universally negligible
but does not have strong measure zero for the metric of 534Ib.
%534D

\spheader 534Xb Let $r$, $s\ge 1$ be integers.   Let
$A\subseteq\BbbR^r$
be a set with strong measure zero, and $f:A\to\BbbR^s$ a function which
is differentiable relative to its domain at every point of $A$.   Show
that $f[A]$ has strong measure zero.  \Hint{262N.}
%534D

\spheader 534Xc Let $(X,\Cal W)$ be a Hausdorff uniform space with
strong measure zero.   Show that $X$ is universally negligible iff it
is a Radon space.
%534D

\spheader 534Xd Give $\omega_1+1$ its order topology and the
corresponding uniformity (4A2Jg).   Show that it has strong measure zero
but is not universally negligible.
%534D

\spheader 534Xe(i) Show that a topological space which has
Rothberger's property in itself must be Lindel\"of.
(ii) Give $X=\omega_1+1$ its order topology.   Show that $\omega_1$
has Rothberger's property in $\omega_1+1$ but not in itself.
%534E

\spheader 534Xf Let $X$ be a topological space and $A\subseteq X$.
Show that $A\in\CalRbg(X)$ iff $A\in\CalRbg(\overline{A})$.
%534F

\spheader 534Xg Let $X$ be a $\sigma$-compact topological space
which is either Hausdorff or regular, and $A\subseteq X$.   Show that
$A\in\CalRbg(X)$ iff for every sequence
$\sequencen{\Cal G_n}$ of finite open
covers of $X$, there is a sequence $\sequencen{G_n}$, covering $A$,
such that $G_n\in\Cal G_n$ for every $n$.
%534F

\spheader 534Xh Let $X$ be a topological space and $A\in\CalRbg(X)$.
Suppose that $B\subseteq X$
is such that $B\setminus G\in\CalRbg(X)$ for every
open subset $G$ of $X$ including $A$.   Show that $B\in\CalRbg(X)$.
%534F

\spheader 534Xi Let $X$ be a paracompact Hausdorff space, and $A$ a subset
of $X$.   Show that the following are equiveridical:  (i)
$A\in\CalRbg(X)$;  (ii) $f[A]\in\CalRbg(Y)$
whenever $Y$ is a metrizable space and $f:X\to Y$ is continuous;
(iii) $f[A]\in\CalSmz(Y,\rho)$ whenever $(Y,\rho)$ is a metric space
and $f:X\to Y$ is continuous;  (iv) $A\in\CalSmz(X,\Cal W)$ whenever
$\Cal W$ is a uniformity on $X$ inducing the topology of $X$.
\Hint{5A4Fb.}
%534F

\spheader 534Xj(i) Let $X$ be a Hausdorff topological space.   Show
that if $A\in\CalRbg(X)$ then $A$ is
universally $\tau$-negligible (definition: 439Xh).   (ii) Let $X$ be a
Hausdorff uniform space.   Show that if $X$ has strong measure zero
then it is universally $\tau$-negligible.
%534F

\spheader 534Xk Let $X$ be a locally compact Hausdorff topological
group.   Show that a subset of $X$ has Rothberger's property in $X$
iff it has strong measure zero for the right uniformity of $X$
iff it has strong measure zero for the bilateral uniformity of $X$.
%534H

\spheader 534Xl Show that $\non\CalRbg(\NN)=\frakmctbl$.
%534I  \ge 534Ia, \le 534Ib

\spheader 534Xm Let $(X,\Cal W)$ be a Lindel\"of uniform space.   Show
that there is some $\kappa$ such that
$\CalSmz(X,\Cal W)\prT\Cal N^{\kappa}$,
where $\Cal N$ is the null ideal of Lebesgue measure on $\Bbb R$.
%534K   idea of "uniform weight" would specify \kappa

\spheader 534Xn Show that every separable metric space $(X,\rho)$ is
uniformly equivalent to a subspace of $[0,1]^{\Bbb N}$ and is
therefore $\CalSmz$-embeddable in $[0,1]^{\Bbb N}$.
%534L

\spheader 534Xo Suppose that $\frak d=\omega_1$.   Show that there is a
set $A\subseteq\Bbb R\setminus\Bbb Q$ such that $A$ has Rothberger's
property in $\Bbb R$ but not in $\Bbb R\setminus\Bbb Q$.
%534O, 534E

\spheader 534Xp Let $A$ be the set constructed in 534P on the assumption
that $\frakmctbl=\frak c$.   Show that $A\cup\Bbb Q$ has Rothberger's
property in itself, but $A$ does not have Rothberger's property in
itself.
%534P

\leader{534Y}{Further exercises (a)}
%\spheader 534Ya
Let $(X,\rho)$ be an analytic metric space and $\mu_{Hr}$ Hausdorff
$r$-dimensional measure on $X$, where $r>0$;  suppose that
$\mu_{Hr}X>0$.   Let $\Cal I$ be the
$\sigma$-ideal of subsets of $X$ generated by
$\{A:\mu_{Hr}^*A<\infty\}$.   Show that

$$\eqalign{\non\Cal N(\mu_{Hr})
=\min(\non\Cal N,\non\Cal I)
&=\non\Cal N\text{ if }\mu_{Hr}\text{ is }\sigma\text{-finite},\cr
&=\non\Cal I\text{ otherwise}.\cr}$$
%\noindent\Hint{if $\mu_{Hr}$ is not $\sigma$-finite, then by 471S there
%is an uncountable disjoint family of Borel sets of non-zero finite
%measure.}
%534B mt53bits

\spheader 534Yb (i) Set
$\Cal I=\{\coint{4^{-m}i,4^{-m}(i+1)}:m\in\Bbb N$, $i\in\Bbb Z\}$.   For
$A\subseteq\Bbb R$ set
$\theta(A)=\inf\{\sum_{I\in\Cal I'}\sqrt{\diam I}:
\Cal I'\subseteq\Cal I$ covers $A\}$.
Show that if $\mu^{(1)}_{H,1/2}$ is Hausdorff $\bover12$-dimensional
measure on $\Bbb R$, then $\mu^{(1)}_{H,1/2}(A)=0$ iff $\theta(A)=0$.
(ii) Set $\Cal J
=\{\coint{2^{-m}i,2^{-m}(i+1)}\times\coint{2^{-m}j,2^{-m}(j+1)}:
  m\in\Bbb N$, $i$, $j\in\Bbb Z\}$, and for $A\subseteq\BbbR^2$ set
$\theta'(A)=\inf\{\sum_{J\in\Cal J'}\diam J:\Cal J'\subseteq\Cal J$
covers $A\}$.
Show that if $\mu^{(2)}_{H1}$ is Hausdorff $1$-dimensional measure on
$\BbbR^2$, then $\mu^{(2)}_{H1}(A)=0$ iff $\theta'(A)=0$.
(iii) Show that the null ideals $\Cal N(\mu^{(1)}_{H,1/2})$ and
$\Cal N(\mu^{(2)}_{H1})$ are isomorphic.
%534B

\spheader 534Yc Show that if {\it either}
$\non\Cal N=\cf\Cal N$ {\it or} $\non\Cal N<\cov\Cal N$, where $\Cal N$ is
the null ideal of Lebesgue measure on $\Bbb R$,
then Hausdorff one-dimensional measure on $\BbbR^2$ does not have the
measurable envelope property.
%Problem:  is it consistent to suppose that \mu_{H1} *does* have the
%measurable envelope property?
%534B mt53bits

\spheader 534Yd Let $X$ be a Hausdorff space.   Show that a compact
subset of $X$ has Rothberger's property in $X$ iff it is scattered.
%534F induce on Cantor-Bendixson rank

\spheader 534Ye Suppose that $\frakmctbl=\frak c$.   Let $X$ be the
group of all permutations of $\Bbb N$, regarded as the isometry group of
$\Bbb N$ with its $\{0,1\}$-valued metric, so that $X$ is a Polish group
(441Xp-441Xq).   Show that there is a subset $A$ of $X$ such that $A$
has strong measure zero for the right uniformity of $X$ but $A^{-1}$
does not.
%534H mt53bits

\spheader 534Yf
Let $\frak G$ be a collection of families of sets.   Let us say that a
set $A$ has the
{\bf $\frak G$-Rothberger property} if for every sequence
$\sequencen{\Cal G_n}$ in $\frak G$ there is a cover $\sequencen{G_n}$
of $A$ such that $G_n\in\Cal G_n$ for every $n\in\Bbb N$.   (i) Show
that the family $\Cal I$ of sets with the $\frak G$-Rothberger
property
is a $\sigma$-ideal of sets containing every countable subset of
$\bigcap_{\Cal G\in\frak G}\bigcup\Cal G$.
(ii) Show that if $\frak H$ is another collection of families of sets,
and $f$ is a function such that for every $\Cal H\in\frak H$ there is
a member of $\frak G$ refining $\{f^{-1}[H]:H\in\Cal H\}$, then $f[A]$
has the $\frak H$-Rothberger property whenever $A\in\Cal I$.   (iii)
Suppose that $\frak G$ is a collection of families of open subsets of a
topological space  $X$, that $A\in\Cal I$ has the
$\frak G$-Rothberger property, and that $B\subseteq X$ is such that
$B\setminus G\in\Cal I$ for every open set $G\supseteq A$.   Show that
$B\in\Cal I$.   (iv) Suppose that $X=\bigcup\Cal G$ for every
$\Cal G\in\frak G$, and that every member of $\frak G$ is countable.
Show that $\non(\Cal I,X)\ge\frakmctbl$.
%534K

\spheader 534Yg Let $X$ be a Lindel\"of topological space.   Show that
$\add\CalRbg(X)\ge\add\Cal N$, where
$\Cal N$ is the null ideal of Lebesgue measure on $\Bbb R$.
%534K

\leaveitout{
\spheader 534Y? Let $(X,\Cal W)$ be a uniform space.
Consider the following infinite game for two players, `Cover' and
`Noncover'.   (Compare 451V.)
Noncover chooses $W_0\in\Cal W$;  Cover chooses
$A_0$ such that $A_0\times A_0\subseteq W_0$;
Noncover chooses $W_1\in\Cal W$;  Cover chooses
$A_1$ such that $A_1\times A_1\subseteq W_1$;  and so on.   At the end of
the game, Cover wins iff $X\subseteq\bigcup_{n\in\Bbb N}A_n$.   $X$ is {\bf
strategically of strong measure zero} if Cover has a winning strategy in
this game.   (i) Show that $\omega_1+1$, with its usual uniformity, is
strategically of strong measure zero.   (ii) Show that a metric space is
strategically of strong measure zero iff it is countable.
}%end of leaveitout
}%end of exercises

\leader{534Z}{Problems (a)}
%\spheader 534Za
Let $\mu^{(2)}_{H1}$ be one-dimensional
Hausdorff measure on $\BbbR^2$.
Is the covering number $\cov\Cal N(\mu^{(2)}_{H1})$ necessarily equal
to $\cov\Cal N$?
\cmmnt{As observed in 534Bc-534Bd, we have
$\cov\Cal N\le\cov\Cal N(\mu^{(2)}_{H1})\le\non\Cal M$.
We can ask the same question for $r$-dimensional Hausdorff
measure on $\BbbR^n$ whenever $0<r<n$;
in particular, for $r$-dimensional Hausdorff measure on $[0,1]$, where
$0<r<1$, and these questions are strongly connected (534Yb).
{\smc Shelah \& Stepr\=ans 05} show that
$\non\Cal N(\mu^{(2)}_{H1})$ can be less than $\non\Cal N$;
of course this is possible only because $\mu^{(2)}_{H1}$ is not
semi-finite (439H, 521Xg).
}%end of comment

\spheader 534Zb Can $\cf\CalRbg(\Bbb R)$ be $\omega_1$?

\spheader 534Zc How many types of Polish spaces under
$\CalSmz$-equivalence can there be?   If we give $\NN$ the metric of
534Ib, can it fail to be $\CalSmz$-equivalent to $[0,1]^{\Bbb N}$?

\spheader 534Zd Suppose that there is a separable metric space
of size $\frak c$ with strong measure zero.   Must
there be a subset of $\Bbb R$ of size $\frak c$ with
strong measure zero in $\Bbb R$?

\endnotes{
\Notesheader{534}
I have very little to say about Hausdorff measures, and 534B is here
only because it would seem even lonelier in a section by itself.   All I
have tried to do is to run through the obvious questions connecting \S471
with Chapter 52.   But at the next level there is surely much more to be
done (534Za).

`Strong measure zero' has attracted a great deal of attention, starting
with the work of E.Borel, who suggested that every subset of
$\Bbb R$ with strong measure zero must be countable;  this is the
{\bf Borel conjecture}.   It turns out that this is undecidable in
ZFC, and that if the Borel conjecture is true then there are no uncountable
sets of strong measure zero in any separable metric space (534O).
So we have some questions of a new kind:  in the ideals
$\CalSmz(X,\Cal W)$ of sets
of strong measure zero, in addition to the standard cardinals $\add$,
$\non$, $\cov$ and $\cf$, we find ourselves asking for the possible
cardinals of sets belonging to the ideal.

The next point is that strong measure zero is not (or rather, not
always) either a topological property or a metric property;  it is a
property of uniform spaces.   We must therefore be prepared to examine
uniformities, even if we are happy to stay with metrizable ones.   In
534P(iv), using an axiom which is a consequence of the continuum
hypothesis, I show that we can have a set which has strong measure
zero for one of two equivalent metrics and not for the other.
{\smc Goldstern Judah \& Shelah 93} describe a model in which the
ideal $\CalRbg(\Bbb R)$ has
$\add\CalRbg(\Bbb R)=\non\CalRbg(\Bbb R)=\omega_2$ while
$\frakmctbl=\omega_1$.   So in this
case 534Ib tells us that $\NN$, with the metric described there, is
not even $\CalSmz$-embeddable in
$\Bbb R$.   Of course in models of set theory in which the Borel
conjecture is true we do have topologically determined structures, for
trivial reasons.

Note that for any uncountable complete separable metric space $X$, there
is a subset of $X$ homeomorphic to $\{0,1\}^{\Bbb N}$ (423J), and the
homeomorphism must be a uniform equivalence;  so that $\{0,1\}^{\Bbb N}$
and its companions $[0,1]^r$, $\BbbR^r$ must be $\CalSmz$-embeddable
in $X$.   In this sense they are the `simplest' uncountable complete
metric spaces.   In the same sense,
$[0,1]^{\Bbb N}$ is the most complex (534Xn).

For $\sigma$-compact spaces, strong measure zero becomes a topological
property (534Fd, 534G), corresponding to what I call
`Rothberger's property'
(534E).   {\smc Rothberger 38b} investigated subsets of $\Bbb R$
which have Rothberger's property in themselves, under the name `property
C$'$'.   The ideas of 534D and 534J-534K can all be re-presented as
theorems about Rothberger's property (534F,
534Xh, 534Xl, 534Yg);  the machinery of 534Yf is supposed to
suggest a reason for this.   It is natural to be attracted to a
topological concept, but there is a difficulty in that Rothberger's
property is not hereditary in the usual way (534Xe, 534Xo, 534Xp).
I note that while 534N is stated in terms of
$\CalRbg$-equivalence, isomorphism of the ideals of sets with the
appropriate Rothberger's property, the concept of strong measure zero
seems to be necessary in the Schr\"oder-Bernstein arguments based on 534M.

For a fuller discussion of strong measure zero in $\Bbb R$, see
{\smc Bartoszy\'nski \& Judah 95}, chap.\ 8, from which much of the
material of this section is taken.
}%end of notes

\discrpage

\leaveitout{
What about \shr(\CalSmz)?
}
