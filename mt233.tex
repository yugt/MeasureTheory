\frfilename{mt233.tex}
\versiondate{16.6.02}
\copyrightdate{1994}

\def\chaptername{The Radon-Nikod\'ym theorem}
\def\sectionname{Conditional expectations}

\newsection{233}

I devote a section to a first look at one of the principal applications
of the Radon-Nikod\'ym theorem.   It is one of the most
vital ideas of measure theory, and will appear repeatedly in one form or
another.   Here I give the definition and most basic properties of
conditional expectations as they arise in abstract probability theory,
with notes on convex functions and a version of Jensen's inequality
(233I-233J).

\leader{233A}{$\sigma$-subalgebras} Let $X$ be a set and $\Sigma$ a
$\sigma$-algebra of subsets of $X$.   A {\bf $\sigma$-subalgebra} of
$\Sigma$ is a $\sigma$-algebra $\Tau$ of subsets of $X$ such that
$\Tau\subseteq\Sigma$.   If $(X,\Sigma,\mu)$ is a measure space and
$\Tau$ is a $\sigma$-subalgebra of $\Sigma$, then
$(X,\Tau,\mu\restrp\Tau)$ is again a measure space\cmmnt{;  this is
immediate from the definition (112A).   Now we have the
following straightforward lemma.   It is a special case of 235G below,
but I give a separate proof in case you do not wish as yet
to embark on the general investigation pursued in \S235}.

\leader{233B}{Lemma} Let $(X,\Sigma,\mu)$ be a measure space and $\Tau$
a $\sigma$-subalgebra of $\Sigma$.   A real-valued function $f$ defined
on a subset of $X$
is $\mu\restrp\Tau$-integrable iff (i) it is $\mu$-integrable (ii)
$\dom f$ is $\mu\restrp\Tau$-conegligible (iii) $f$ is
$\mu\restrp\Tau$-virtually measurable;  and in this case
$\int fd(\mu\restrp\Tau)=\int fd\mu$.

\proof{{\bf (a)}  Note first that if $f$ is a $\mu\restrp\Tau$-simple
function, that is, is expressible as $\sum_{i=0}^na_i\chi E_i$ where
$a_i\in\Bbb R$, $E_i\in\Tau$ and $(\mu\restrp\Tau)E_i<\infty$ for each
$i$, then $f$ is $\mu$-simple and

\Centerline{$\int f d\mu=\sum_{i=0}^na_i\mu E_i
=\int f d(\mu\restrp\Tau)$.}

\medskip

{\bf (b)} Let $U_{\mu}$ be the set of non-negative
$\mu$-integrable functions and $U_{\mu\restrp\Tau}$ the set of
non-negative $\mu\restrp\Tau$-integrable functions.

Suppose $f\in U_{\mu\restrp\Tau}$.   Then there is a
non-decreasing sequence $\sequencen{f_n}$ of
$\mu\restrp\Tau$-simple functions such that
$f(x)=\lim_{n\to\infty}f_n\,\mu\restrp\Tau$-a.e.\ and

\Centerline{$\int fd(\mu\restrp\Tau)
=\lim_{n\to\infty}\int f_nd(\mu\restrp\Tau)$.}

\noindent  But now
every $f_n$ is also $\mu$-simple, and
$\int f_nd\mu=\int f_nd(\mu\restrp\Tau)$ for every $n$,  and
$f=\lim_{n\to\infty}f_n\,\mu$-a.e.   So $f\in U_{\mu}$ and
$\int fd\mu=\int fd(\mu\restrp\Tau)$.

\medskip

{\bf (c)} Now suppose that $f$ is
$\mu\restrp\Tau$-integrable.   Then it is the difference of two members
of $U_{\mu\restrp\Tau}$, so is $\mu$-integrable, and
$\int fd\mu=\int fd(\mu\restrp\Tau)$.   Also conditions (ii) and (iii)
are satisfied,
according to the conventions established in Volume 1 (122Nc, 122P-122Q).

\medskip

{\bf (d)} Suppose that $f$ satisfies conditions
(i)-(iii).   Then $|f|\in U_{\mu}$, and  there is a conegligible set
$E\subseteq\dom f$ such that $E\in\Tau$ and $f\restr E$ is
$\Tau$-measurable.   Accordingly $|f|\restr E$ is $\Tau$-measurable.
Now, if $\epsilon>0$, then

\Centerline{$(\mu\restrp\Tau)\{x:x\in E,\,|f|(x)\ge\epsilon\}
=\mu\{x:x\in E,\,|f|(x)\ge\epsilon\}
\le{1\over\epsilon}\int|f|d\mu<\infty$;}

\noindent moreover,

$$\eqalign{\sup\{\int g\,d(\mu\restrp\Tau):g&\text{ is a }
  \mu\restrp\Tau\text{-simple function},
  \,g\le |f|\,\mu\restrp\Tau\text{-a.e.}\}\cr
&=\sup\{\int g\,d\mu:g\text{ is a }
  \mu\restrp\Tau\text{-simple function},
  \,g\le |f|\,\mu\restrp\Tau\text{-a.e.}\}\cr
&\le\sup\{\int g\,d\mu:g\text{ is a }
  \mu\text{-simple function},\,g\le |f|\,\mu\text{-a.e.}\}\cr
&\le\int |f|d\mu<\infty.\cr}$$

\noindent By the criterion of 122Ja, $|f|\in U_{\mu\restrp\Tau}$.
Consequently $f$, being $\mu\restrp\Tau$-virtually $\Tau$-measurable,
is $\mu\restrp\Tau$-integrable, by 122P.   This completes the proof.
}%end of proof of 233B

\cmmnt{
\leader{233C}{Remarks (a)} My argument just above is detailed
to the point of
pedantry.   I think, however, that while I can be accused of wasting
paper by writing everything down, every element of the argument is
necessary to the result.   To be sure, some of the details are needed
only because I use such a wide notion of `integrable function';  if
you restrict the notion of `integrability' to measurable functions
defined on the whole measure space, there are simplifications at this
stage, to be paid for later when you discover that many of the principal
applications are to functions defined by formulae which do not apply on
the whole underlying space.

The essential point which does have to be grasped is that while a
$\mu\restrp\Tau$-negligible set is always $\mu$-negligible, a
$\mu$-negligible set need not be $\mu\restrp\Tau$-negligible.

\header{233Cb}{\bf (b)} As the simplest possible example of the problems
which can arise, I offer the following.   Let $(X,\Sigma,\mu)$ be
$[0,1]^2$ with Lebesgue measure.   Let $\Tau$ be the set of those
members of $\Sigma$ expressible as $F\times[0,1]$ for some
$F\subseteq[0,1]$;  it is easy to see that $\Tau$ is a
$\sigma$-subalgebra of $\Sigma$.   Consider $f$, $g:X\to[0,1]$ defined
by saying that

\Centerline{$f(t,u)=1$ if $u>0$, $0$ otherwise,}

\Centerline{$g(t,u)=1$ if $t>0$, $0$ otherwise.}

\noindent Then both $f$ and $g$ are $\mu$-integrable, being constant
$\mu$-a.e.   But only $g$ is $\mu\restrp\Tau$-integrable, because any
non-negligible $E\in\Tau$ includes a complete vertical section
$\{t\}\times[0,1]$, so that $f$ takes both values $0$ and $1$ on $E$.
If we set

\Centerline{$h(t,u)=1$ if $u>0$, undefined otherwise,}

\noindent then again (on the conventions I use) $h$ is
$\mu$-integrable but not $\mu\restrp\Tau$-integrable, as there is no
conegligible member of $\Tau$ included in the domain of $h$.

\header{233Cc}{\bf (c)} If $f$ is defined everywhere in $X$, and
$\mu\restrp\Tau$ is complete, then of course $f$ is
$\mu\restrp\Tau$-integrable iff it is $\mu$-integrable and
$\Tau$-measurable.   But note that in the example just above, which is
one of the archetypes for this topic, $\mu\restrp\Tau$ is not complete,
as singleton sets are negligible but not measurable.
}%end of comment

\leader{233D}{Conditional expectations} Let $(X,\Sigma,\mu)$ be a
probability
space\cmmnt{, that is, a measure space with $\mu X=1$}.
\cmmnt{(Nearly all the ideas here work
perfectly well for any totally finite measure space, but there seems
nothing to be gained from the extension, and the traditional phrase
`conditional expectation' demands a probability space.)}
Let $\Tau\subseteq\Sigma$ be a $\sigma$-subalgebra.
\dvro{ If $f$ is a $\mu$-integrable real-valued function, there is a
$\mu\restrp\Tau$-integrable function $g$ such that $\int_Fg=\int_Ff$ for
every $F\in\Tau$;  such  a function is {\bf a conditional expectation}
of $f$ on $\Tau$.}{}

\cmmnt{\header{233Da}{\bf (a)} For any
$\mu$-integrable real-valued function $f$ defined on a conegligible
subset of $X$, we have a corresponding indefinite integral
$\nu_f:\Sigma\to\Bbb R$ given by the formula $\nu_fE=\int_Ef$ for every
$E\in\Sigma$.   We know that $\nu_f$ is countably additive and truly
continuous with respect to $\mu$, which in the present context is the
same as saying that it is absolutely continuous (232Bc-232Bd).   Now
consider the restrictions $\mu\restrp\Tau$, $\nu_f\restrp\Tau$ of $\mu$
and $\nu_f$ to the $\sigma$-algebra $\Tau$.   It follows
directly from the definitions of `countably additive' and
`absolutely continuous' that
$\nu_f\restrp\Tau$ is countably additive and absolutely continuous with
respect to $\mu\restrp\Tau$, therefore truly continuous with respect to
$\mu\restrp\Tau$.   Consequently, the
Radon-Nikod\'ym theorem (232E) tells us that there is a
$\mu\restrp\Tau$-integrable function $g$ such that
$(\nu_f\restrp\Tau)F=\int_Fg\,d(\mu\restrp\Tau)$ for every $F\in\Tau$.
}%end of comment

\cmmnt{\header{233Db}{\bf (b)} Let us define {\bf a conditional
expectation of $f$ on} $\Tau$ to be such a function; that is, a
$\mu\restrp\Tau$-integrable
function $g$ such that $\int_Fg\,d(\mu\restrp\Tau)=\int_Ffd\mu$ for
every $F\in\Tau$.   Looking back at 233B, we see that for such a $g$ we
have

\Centerline{$\int_Fg\,d(\mu\restrp\Tau)
=\int g\times\chi F\,d(\mu\restrp\Tau)
=\int g\times\chi F\, d\mu=\int_Fg\,d\mu$}

\noindent for every $F\in\Tau$;  also, that $g$ is almost everywhere
equal to a $\Tau$-measurable function defined everywhere in $X$ which is
also a conditional expectation of $f$ on $\Tau$ (232He).
}%end of comment

\cmmnt{\header{233Dc}{\bf (c)} I set the word `a' of the phrase
`a conditional
expectation' in bold type to emphasize that there is nothing unique
about the function $g$.   In 242J I will return to this
point, and describe an object which could properly be called `the'
conditional expectation of $f$ on $\Tau$.   $g$ is `essentially
unique' only in the sense that if $g_1$, $g_2$ are both conditional
expectations of $f$ on $\Tau$ then $g_1=g_2\,\,\mu\restrp\Tau$-a.e.\
(131Hb).   This does of course
mean that a very large number of its properties -- for instance, the
distribution function $G(a)=\hat\mu\{x:g(x)\le a\}$, where $\hat\mu$ is
the completion of $\mu$ (212C) -- are independent of which $g$ we
take.
}%end of comment

\cmmnt{\header{233Dd}{\bf (d)} A word of explanation of the phrase
`conditional expectation' is in order.   This derives from the standard
identification of probability with measure, due to Kolmogorov, which I
will discuss more fully in Chapter 27.   A real-valued random
variable may be regarded as a measurable, or virtually measurable,
function $f$ on a probability space $(X,\Sigma,\mu)$;  its
`expectation' becomes identified with $\int f d\mu$, supposing that this
exists.   If $F\in\Sigma$ and $\mu F>0$ then the `conditional
expectation of $f$ given $F$' is ${1\over{\mu F}}\int_Ff$.   If
$F_0,\ldots,F_n$ is a partition of $X$ into measurable sets of
non-zero measure, then the function $g$ given by

\Centerline{$g(x)=\Bover1{\mu F_i}\int_{F_i}f$ if $x\in F_i$}

\noindent is a kind of anticipated conditional expectation;  if we are
one day told that $x\in F_i$, then $g(x)$ will be our subsequent
estimate of the expectation of $f$.   In the terms of the definition
above, $g$ is a conditional expectation of $f$ on the finite algebra
$\Tau$ generated by $\{F_0,\ldots,F_n\}$.   An appropriate intuition for
general $\sigma$-algebras $\Tau$ is that they consist of the events
which we shall be able to observe at some stated future time $t_0$,
while the whole algebra $\Sigma$ consists of all events, including those
not observable until times later than $t_0$, if ever.
}%end of comment

\leader{233E}{}\cmmnt{ I list some of the elementary facts concerning
conditional expectations.

\medskip

\noindent}{\bf Proposition} Let $(X,\Sigma,\mu)$ be a probability space
and $\Tau$ a
$\sigma$-subalgebra of $\Sigma$.   Let $\sequencen{f_n}$ be a sequence
of $\mu$-integrable real-valued functions, and for each $n$ let $g_n$ be
a conditional expectation of $f_n$ on
$\Tau$.   Then

(a) $g_1+g_2$ is a conditional expectation of $f_1+f_2$ on $\Tau$;

(b) for any $c\in\Bbb R$, $cg_0$ is a conditional expectation of
$cf_0$ on $\Tau$;

(c) if $f_1\leae f_2$ then $g_1\leae g_2$;

(d) if $\sequencen{f_n}$ is non-decreasing a.e.\ and
$f=\lim_{n\to\infty}f_n$ is $\mu$-integrable, then
$\lim_{n\to\infty}g_n$ is a conditional expectation of $f$ on $\Tau$;

(e) if $f=\lim_{n\to\infty}f_n$ is defined a.e.\ and there is a
$\mu$-integrable function $h$ such that $|f_n|\leae h$ for every
$n$,
then $\lim_{n\to\infty}g_n$ is a conditional expectation of $f$ on
$\Tau$;

(f) if $F\in\Tau$ then $g_0\times \chi F$ is a conditional
expectation of $f_0\times\chi F$ on $\Tau$;

(g) if $h$ is a bounded, $\mu\restrp\Tau$-virtually measurable
real-valued function
defined $\mu\restrp\Tau$-almost everywhere in $X$, then $g_0\times h$ is
a conditional expectation of $f_0\times h$ on $\Tau$;

(h) if $\Upsilon$ is a $\sigma$-subalgebra of $\Tau$, then a function
$h_0$ is a conditional expectation of $f_0$ on $\Upsilon$ iff it is a
conditional expectation of $g_0$ on $\Upsilon$.

\proof{{\bf (a)-(b)} We have only to observe that

\Centerline{$\int_Fg_1+g_2d(\mu\restrp\Tau)
=\int_Fg_1d(\mu\restrp\Tau)+\int_Fg_2d(\mu\restrp\Tau)
=\int_Ff_1d\mu+\int_Ff_2d\mu=\int_Ff_1+f_2d\mu$,}

\Centerline{$\int_Fcg_0d(\mu\restrp\Tau)
=c\int_Fg_0d(\mu\restrp\Tau)=c\int_Ff_0d\mu=\int_Fcf_0d\mu$}

\noindent for every $F\in\Tau$.

\medskip

{\bf (c)} If $F\in\Tau$ then

\Centerline{$\int_Fg_1d(\mu\restrp\Tau)=\int_Ff_1d\mu\le\int_Ff_2d
\mu
=\int_Fg_2d(\mu\restrp\Tau)$}

\noindent for every $F\in\Tau$;  consequently
$g_1\le g_2\,\mu\restrp\Tau$-a.e.\ (131Ha).

\medskip

{\bf (d)} By (c), $\sequencen{g_n}$ is non-decreasing
$\mu\restrp\Tau$-a.e.;   moreover,

\Centerline{$\sup_{n\in\Bbb N}\int g_nd(\mu\restrp\Tau)
=\sup_{n\in\Bbb N}\int f_nd\mu
=\int fd\mu<\infty$.}

\noindent By B.Levi's theorem, $g=\lim_{n\to\infty}g_n$ is defined
$\mu\restrp\Tau$-almost everywhere, and

\Centerline{$\int_Fg\,d(\mu\restrp\Tau)
=\lim_{n\to\infty}\int_Fg_nd(\mu\restrp\Tau)
=\lim_{n\to\infty}\int_Ff_nd\mu
=\int_Ffd\mu$}

\noindent for every $F\in\Tau$, so $g$ is a conditional expectation of
$f$ on $\Tau$.

\medskip

{\bf (e)} Set $f'_n=\inf_{m\ge n}f_m$, $f''_n=\sup_{m\ge n}f_m$ for each
$n\in\Bbb N$.   Then we have

\Centerline{$-h\leae f'_n\le f_n\le f''_n\leae h$,}

\noindent and $\sequencen{f'_n}$, $\sequencen{f''_n}$ are
almost-everywhere-monotonic sequences of functions both converging
almost everywhere to $f$.   For each $n$, let $g'_n$, $g''_n$ be
conditional expectations of $f'_n$, $f''_n$ on $\Tau$.   By (iii) and
(iv), $\sequencen{g'_n}$ and $\sequencen{g''_n}$ are
almost-everywhere-monotonic sequences converging almost everywhere to
conditional expectations $g'$, $g''$ of $f$.   Of course
$g'=g''\,\,\mu\restrp\Tau$-a.e.\ (233Dc).   Also, for each $n$,
$g'_n\leae g_n\leae g''_n$, so $\sequencen{g_n}$ converges to
$g'\,\,\mu\restrp\Tau$-a.e., and $g=\lim_{n\to\infty}g_n$ is defined
almost everywhere and is a conditional expectation of $f$ on $\Tau$.

\medskip

{\bf (f)} For any $H\in\Tau$,

\Centerline{$\int_Hg_0\times\chi F\,d(\mu\restrp\Tau)
=\int_{H\cap F}g_0d(\mu\restrp\Tau)
=\int_{H\cap F}f_0d\mu
=\int_Hf_0\times\chi F\,d\mu$.}

\medskip

{\bf (g)}(i) If $h$ is actually $(\mu\restrp\Tau)$-simple, say
$h=\sum_{i=0}^na_i\chi F_i$ where $F_i\in\Tau$ for each $i$, then

\Centerline{$\int_Fg_0\times h\,d(\mu\restrp\Tau)
=\sum_{i=0}^na_i\int_Fg_0\times\chi F_id(\mu\restrp\Tau)
=\sum_{i=0}^na_i\int_Ff\times\chi F_i\,d\mu
=\int_Ff\times h\,d\mu$}

\noindent for every $F\in\Tau$.   (ii) For the general case, if
$h$ is $\mu\restrp\Tau$-virtually measurable and
$|h(x)|\le M\,\,\mu\restrp\Tau$-almost
everywhere, then there is a sequence $\sequencen{h_n}$ of
$\mu\restrp\Tau$-simple functions converging to $h$ almost everywhere,
and with $|h_n(x)|\le M$ for every $x$, $n$.   Now
$f_0\times h_n\to f_0\times h$ a.e.\ and $|f_0\times h_n|\leae M|f_0|$
for each $n$,
while $g_0\times h_n$ is a conditional expectation of $f_0\times h_n$
for every $n$, so by (e) we see that $\lim_{n\to\infty}g_0\times h_n$
will be a conditional expectation of $f_0\times h$;  but this is equal
almost everywhere to $g_0\times h$.

\medskip

{\bf (h)} We need note only that
$\int_Hg_0d(\mu\restrp\Tau)=\int_Hf_0d\mu$ for every $H\in\Upsilon$, so

$$\eqalign{\int_Hh_0&d(\mu\restrp\Upsilon)=\int_Hg_0d(\mu\restrp\Tau)
\text{ for every }H\in\Upsilon\cr
&\iff\int_Hh_0d(\mu\restrp\Upsilon)=\int_Hf_0d\mu
\text{ for every }H\in\Upsilon.\cr}$$
}%end of proof of 233E

\cmmnt{
\leader{233F}{Remarks} Of course the results above are
individually nearly trivial (though I think (e) and (g) might give you
pause for
thought if they were offered without previous preparation of the
ground).   Cumulatively they amount to some quite strong properties.
In \S242 I will restate them in language which is syntactically
more direct, but relies on a deeper level of abstraction.

As an illustration of the power of conditional
expectations to surprise us, I offer the next proposition, which depends
on the concept of `convex' function.
}%end of comment

\leader{233G}{Convex functions} Recall that a real-valued function
$\phi$ defined on an interval $I\subseteq\Bbb R$ is {\bf convex} if

\Centerline{$\phi(tb+(1-t)c)\le t\phi(b)+(1-t)\phi(c)$}

\noindent whenever $b$, $c\in I$ and $t\in[0,1]$.

\cmmnt{\medskip

\noindent{\bf Examples} The formulae $|x|$, $x^2$, $e^{\pm x}\pm x$
define convex functions on $\Bbb R$;  on $\ooint{-1,1}$ we have
$1/(1-x^2)$; on $\ooint{0,\infty}$ we have $1/x$ and $x\ln x$;  on
$[0,1]$ we have the function which is zero on $\ooint{0,1}$ and $1$ on
$\{0,1\}$.
}%end of comment

\leader{233H}{}\cmmnt{ The general theory of convex functions is both
extensive and important;  I list a few of their more salient properties
in 233Xe.
For the moment the following lemma covers what we need.

\medskip

\noindent}{\bf Lemma} Let $I\subseteq\Bbb R$ be a non-empty open
interval\cmmnt{ (bounded or unbounded)} and $\phi:I\to\Bbb R$ a convex
function.

(a) For every $a\in I$ there is a $b\in\Bbb R$ such that
$\phi(x)\ge\phi(a)+b(x-a)$ for every $x\in I$.

(b) If we take, for each $q\in I\cap\Bbb Q$, a $b_q\in\Bbb R$ such that
$\phi(x)\ge\phi(q)+b_q(x-q)$ for every $x\in I$, then

\Centerline{$\phi(x)=\sup_{q\in I\cap\Bbb Q}\phi(q)+b_q(x-q)$}

\noindent for every $x\in I$.

(c) $\phi$ is Borel measurable.

\proof{{\bf (a)} If $c$, $c'\in I$ and $c<a<c'$,
then $a$ is expressible as $dc+(1-d)c'$ for some $d\in\ooint{0,1}$, so
that $\phi(a)\le d\phi(c)+(1-d)\phi(c')$ and

$$\eqalign{{\Bover{\phi(a)-\phi(c)}{a-c}}
&\le{\Bover{d\phi(c)+(1-d)\phi(c')-\phi(c)}{dc+(1-d)c'-c}}
={\Bover{(1-d)(\phi(c')-\phi(c))}{(1-d)(c'-c)}}\cr
&={\Bover{d(\phi(c')-\phi(c))}{d(c'-c)}}
={\Bover{\phi(c')-d\phi(c)-(1-d)\phi(c')}{c'-dc-(1-d)c'}}
\le{\Bover{\phi(c')-\phi(a)}{c'-a}}.\cr}$$

\noindent This means that

\Centerline{$b=\sup_{c<a,c\in I}\Bover{\phi(a)-\phi(c)}{a-c}$}

\noindent is finite, and $b\le\Bover{\phi(c')-\phi(a)}{c'-a}$ whenever
$a<c'\in I$;  accordingly $\phi(x)\ge\phi(a)+b(x-a)$ for every $x\in I$.

\medskip

{\bf (b)}  By the choice of the $b_q$, $\phi(x)\ge\sup_{q\in\Bbb
Q}\phi_q(x)$.
On the other hand, given $x\in I$, fix $y\in I$ such that $x<y$ and let
$b\in\Bbb R$ be such that $\phi(z)\ge\phi(x)+b(z-x)$ for every $z\in I$.
If $q\in\Bbb Q$ and $x<q<y$, we have $\phi(y)\ge\phi(q)+b_q(y-q)$, so
that
$b_q\le\bover{\phi(y)-\phi(q)}{y-q}$ and

$$\eqalign{\phi(q)+b_q(x-q)
&=\phi(q)-b_q(q-x)
\ge\phi(q)-\Bover{\phi(y)-\phi(q)}{y-q}(q-x)\cr
&=\Bover{y-x}{y-q}\phi(q)-\Bover{q-x}{y-q}\phi(y)
\ge\Bover{y-x}{y-q}(\phi(x)+b(q-x))-\Bover{q-x}{y-q}\phi(y).\cr}$$

\noindent Now


$$\eqalign{\phi(x)
&=\lim_{q\downarrow x}
   \Bover{y-x}{y-q}(\phi(x)+b(q-x))-\Bover{q-x}{y-q}\phi(y)\cr
&\le\sup_{q\in\Bbb Q\cap\ooint{x,y}}
   \Bover{y-x}{y-q}(\phi(x)+b(q-x))-\Bover{q-x}{y-q}\phi(y)\cr
&\le\sup_{q\in\Bbb Q\cap\ooint{x,y}}\phi(q)+b_q(x-q)
\le\sup_{q\in\Bbb Q\cap I}\phi(q)+b_q(x-q).\cr}$$

\medskip

{\bf (c)} Writing $\phi_q(x)=\phi(q)+b_q(x-q)$ for every 
$q\in\Bbb Q\cap I$, every $\phi_q$ is a Borel measurable function, and 
$\phi=\sup_{q\in I\cap\Bbb Q}\phi_q$ is the supremum of a countable 
family of Borel measurable functions, so is Borel measurable.
}%end of proof of 233H

\leader{233I}{Jensen's inequality} Let $(X,\Sigma,\mu)$ be a measure
space and $\phi:\Bbb R\to\Bbb R$ a convex function.

(a) Suppose that $f$ and $g$ are real-valued $\mu$-virtually measurable
functions defined almost everywhere in $X$ and that 
$g\ge 0$ almost everywhere,
$\int g=1$ and $g\times f$ is integrable.   Then
$\phi(\int g\times f)\le\int g\times\phi f$,
where we may need to interpret the right-hand integral as
$\infty$.

(b) In particular, if $\mu X=1$ and $f$ is a real-valued function which
is integrable over $X$, then $\phi(\int f)\le\int\phi f$.

\proof{{\bf (a)} For each $q\in\Bbb Q$ take $b_q$ such that
$\phi(t)\ge\phi_q(t)=\phi(q)+b_q(t-q)$ for every $t\in\Bbb R$ (233Ha).
Because $\phi$ is Borel measurable (233Hc), $\phi f$ is $\mu$-virtually
measurable (121H), so $g\times\phi f$ also is;   since $g\times\phi f$
is defined almost everywhere and almost everywhere greater than or equal
to the integrable function $g\times\phi_0f$, $\int g\times\phi f$ is
defined in $\ocint{-\infty,\infty}$.   Now

$$\eqalign{\phi_q\bigl(\intop g\times f\bigr)
&=\phi(q)+b_q\int g\times f-b_qq\cr
&=\int g\times(b_qf+(\phi(q)-b_qq)\chi X)
=\int g\times\phi_qf
\le\int g\times\phi f,\cr}$$

\noindent because $\int g=1$ and $g\ge 0$ a.e.   By 233Hb,

\Centerline{$\phi(\intop g\times f)
=\sup_{q\in\Bbb Q}\phi_q(\intop g\times f)
\le\int g\times\phi f$.}

\medskip

{\bf (b)} Take $g$ to be the constant function with value $1$.
}%end of proof of 233I

\leader{233J}{}\cmmnt{ Even the special case 233Ib of Jensen's
inequality is already very useful.   It can be extended as follows.

\medskip

\noindent}{\bf Theorem} Let $(X,\Sigma,\mu)$ be a
probability space and $\Tau$ a $\sigma$-subalgebra of $\Sigma$.   Let
$\phi:\Bbb R\to\Bbb R$ be a convex function and $f$ a $\mu$-integrable
real-valued function defined almost everywhere in $X$ such that the
composition $\phi f$ is also
integrable.   If $g$ and $h$ are conditional expectations on $\Tau$ of
$f$, $\phi f$ respectively, then $\phi g\leae h$.   Consequently
$\int\phi g\le\int\phi f$.

\proof{ We use the same ideas as in 233I.   For each $q\in\Bbb Q$ take a
$b_q\in\Bbb R$ such
that $\phi(t)\ge\phi_q(t)=\phi(q)+b_q(t-q)$ for every $t\in\Bbb R$, so
that $\phi(t)=\sup_{q\in\Bbb Q}\phi_q(t)$ for every $t\in\Bbb R$.   Now
setting

\Centerline{$\psi_q(x)=\phi(q)+b_q(g(x)-q)$}

\noindent for $x\in\dom g$, we see that $\psi_q=\phi_q g$ is a
conditional expectation of $\phi_q f$, and as
$\phi_q f\leae\phi f$ we must have $\psi_q\leae h$.   But also 
$\phi g=\sup_{q\in\Bbb Q}\psi_q$ wherever $g$ is
defined, so $\phi g\leae h$, as claimed.

It follows at once that $\int\phi g\le\int h=\int\phi f$.
}%end of proof of 233J

\leader{233K}{}\cmmnt{ I give the following proposition, an
elaboration of
233Eg, in a very general form, as its applications can turn up anywhere.

\medskip

\noindent}{\bf Proposition} Let $(X,\Sigma,\mu)$ be a probability space,
and $\Tau$ a $\sigma$-subalgebra of $\Sigma$.   Suppose that $f$ is a
$\mu$-integrable function and $h$ is a $(\mu\restrp\Tau)$-virtually
measurable real-valued function defined $(\mu\restrp\Tau)$-almost
everywhere in $X$.   Let $g$, $g_0$ be conditional expectations of $f$
and $|f|$ on $\Tau$.   Then $f\times h$ is integrable iff $g_0\times h$
is integrable, and in this case $g\times h$ is a conditional expectation
of $f\times h$ on $\Tau$.

\proof{{\bf (a)} Suppose that $h$ is a $\mu\restrp\Tau$-simple function.
Then surely $f\times h$ and $g_0\times h$ are integrable, and $g\times
h$ is a conditional expectation of $f\times h$ as in 233Eg.

\medskip

{\bf (b)} Now suppose that $f$, $h\ge 0$.   Then $g=g_0\ge 0$ a.e.
(233Ec).  Let $\tilde h$ be a non-negative
$\Tau$-measurable function defined everywhere in $X$ such that
$h\eae\tilde h$.    For each $n\in\Bbb N$ set

$$\eqalign{h_n(x)
&=2^{-n}k\text{ if }0\le k<4^n
  \text{ and }2^{-n}k\le\tilde h(x)<2^{-n}(k+1),\cr
&=2^n\text{ if }\tilde h(x)\ge 2^{-n}.\cr}$$

\noindent Then $h_n$ is a $(\mu\restrp\Tau)$-simple function, so
$g\times h_n$ is a conditional expectation of $f\times h_n$.
Both $\sequencen{f\times h_n}$ and $\sequencen{g\times h_n}$ are almost
everywhere non-decreasing sequences of integrable functions, with limits
$f\times h$ and $g\times h$ respectively.   By B.Levi's theorem,

$$\eqalignno{f\times h\text{ is integrable}
&\iff\,f\times\tilde h\text{ is integrable}\cr
&\iff\,\sup_{n\in\Bbb N}\int f\times h_n<\infty
\iff\,\sup_{n\in\Bbb N}\int g\times h_n<\infty\cr
\noalign{\noindent (because $\int g\times h_n=\int f\times h_n$ for each
$n$)}
&\iff\,g\times h\text{ is integrable}
\iff\,g_0\times h\text{ is integrable}.\cr}$$

\noindent Moreover, in this case

$$\eqalign{\int_Ef\times h
&=\int_Ef\times\tilde h
=\lim_{n\to\infty}\int_Ef\times h_n\cr
&=\lim_{n\to\infty}\int_Eg\times h_n
=\int_Eg\times\tilde h
=\int_Eg\times h\cr}$$

\noindent for every $E\in\Tau$, while $g\times h$ is
$(\mu\restrp\Tau)$-virtually measurable, so $g\times h$ is a
conditional expectation of $f\times h$.

\medskip

{\bf (c)} Finally, consider the general case of integrable $f$ and
virtually measurable $h$.   Set $f^+=f\vee 0$, $f^-=(-f)\vee 0$,
so that $f=f^+-f^-$ and $0\le f^+,\,f^-\le|f|$;  similarly, set
$h^+=h\vee 0$, $h^-=(-h)\vee 0$.   Let $g_1$, $g_2$ be conditional
expectations of $f^+$, $f^-$ on $\Tau$.   Because $0\le f^+$,
$f^-\le|f|$, $0\le g_1$, $g_2\leae g_0$,  while $g\eae g_1-g_2$.

We see that

$$\eqalign{f\times h\text{ is integrable}
&\iff\,|f|\times|h|=|f\times h|\text{ is integrable}\cr
&\iff\,g_0\times |h|\text{ is integrable}\cr
&\iff\,g_0\times h\text{ is integrable}.\cr}$$

\noindent And in this case all four of
$f^+\times h^+,\ldots,f^-\times h^-$ are integrable, so

\Centerline{$(g_1-g_2)\times h
=g_1\times h^+-g_2\times h^+-g_1\times h^-+g_2\times h^-$}

\noindent is a conditional expectation of

\Centerline{$f^+\times h^+-f^-\times h^+-f^+\times h^-+f^-\times h^- %
=f\times h$.}

\noindent Since $g\times h\eae(g_1-g_2)\times h$, this also is a
conditional expectation of $f\times h$, and we're done.
}%end of proof of 233K

\exercises{
\leader{233X}{Basic exercises (a)}
%\spheader 233Xa
Let $(X,\Sigma,\mu)$ be a probability space and $\Tau$ a
$\sigma$-subalgebra of $\Sigma$.   Let $\sequencen{f_n}$ be a sequence
of non-negative $\mu$-integrable functions and suppose that $g_n$ is a
conditional expectation of $f_n$ on $\Tau$ for each $n$.   Suppose that
$f=\liminf_{n\to\infty}f_n$ is integrable and has a conditional
expectation $g$.   Show that $g\leae\liminf_{n\to\infty}g_n$.
%233E

\spheader 233Xb Let $I\subseteq\Bbb R$ be an interval, and
$\phi:I\to\Bbb R$ a function.   Show that $\phi$ is convex iff
$\{x:x\in I,\,\phi(x)+bx\le c\}$ is an interval for every $b$,
$c\in\Bbb R$.
%233G

\sqheader 233Xc Let $I\subseteq\Bbb R$ be an open interval and
$\phi:I\to\Bbb R$ a function.   (i) Show that if $\phi$ is
differentiable then it is convex iff $\phi'$ is non-decreasing.   (ii)
Show that if $\phi$ is absolutely continuous on every bounded closed
subinterval of $I$ then $\phi$ is convex iff
$\phi'$ is non-decreasing on its domain.

\spheader 233Xd For any $r\ge 1$, a subset $C$ of $\BbbR^r$ is
{\bf convex} if $tx + (1-t)y\in C$ for all $x$, $y\in C$ and
$t\in[0,1]$.   If $C\subseteq\BbbR^r$ is convex, then a function
$\phi:C\to\Bbb R$ is {\bf convex} if
$\phi(tx+(1-t)y)\le t\phi(x)+(1-t)\phi(y)$ for all $x$, $y\in C$ and
$t\in[0,1]$.

Let $C\subseteq\BbbR^r$ be a convex set and $\phi:C\to\Bbb R$ a
function.   Show that the following are equiveridical:   (i) the
function $\phi$ is convex; (ii) the set
$\{(x,t):x\in C,\,t\in\Bbb R,\,t\ge\phi(x)\}$ is convex in
$\BbbR^{r+1}$;  (iii) the set
$\{x:x\in C,\,\phi(x)+b\dotproduct x\le c\}$ is convex in $\BbbR^r$ for
every $b\in\BbbR^r$ and $c\in\Bbb R$, writing
$b\dotproduct x=\sum_{i=1}^r\beta_i\xi_i$ if
$b=(\beta_1,\ldots,\beta_r)$ and $x=(\xi_1,\ldots,\xi_r)$.
%233G

\spheader 233Xe Let $I\subseteq\Bbb R$ be an interval
and $\phi:I\to\Bbb R$ a convex function.

\quad (i) Show that if $a$, $d\in I$ and $a<b\le c<d$ then

\Centerline{$\Bover{\phi(b)-\phi(a)}{b-a}
\le\Bover{\phi(d)-\phi(c)}{d-c}$.}

\quad (ii) Show that $\phi$ is continuous at every interior point of
$I$.

\quad (iii) Show that {\it either} $\phi$ is monotonic on $I$ {\it or}
there is a $c\in I$ such that $\phi(c)=\min_{x\in I}\phi(x)$ and $\phi$
is non-increasing on $I\cap\ocint{-\infty,c}$, monotonic
non-decreasing on $I\cap\coint{c,\infty}$.

\quad (iv) Show that $\phi$ is differentiable at all but countably many
points of $I$, and that its derivative is non-decreasing in
the sense that $\phi'(x)\le\phi'(y)$ whenever $x$, $y\in\dom\phi'$ and
$x\le y$.

\quad (v) Show that if $I$ is closed and bounded and $\phi$ is
continuous then $\phi$ is absolutely continuous.

\quad (vi) Show that if $I$ is closed and bounded and $\psi:I\to\Bbb R$
is absolutely continuous with a non-decreasing derivative then
$\psi$ is convex.
%233H

\spheader 233Xf Show that if $I\subseteq\Bbb R$ is an interval and
$\phi$, $\psi:I\to\Bbb R$ are convex functions so is $a\phi+b\psi$ for
any $a$, $b\ge 0$.
%233H

\spheader 233Xg In the context of 233K, give an example in which
$g\times h$ is integrable but $f\times h$ is not.  \Hint{take
$X$, $\mu$, $\Tau$ as in 233Cb, and arrange for $g$ to be $0$.}
%233K

\spheader 233Xh Let $I\subseteq\Bbb R$ be an interval and $\Phi$ a
non-empty family of convex real-valued functions on $I$ such that
$\psi(x)=\sup_{\phi\in\Phi}\phi(x)$ is finite for every $x\in I$.   Show
that $\psi$ is convex.
%233H

\leader{233Y}{Further exercises (a)} If $I\subseteq\Bbb R$ is an
interval, a function $\phi:I\to\Bbb R$ is {\bf mid-convex} if
$\phi(\bover{x+y}2)\le\bover12(\phi(x)+\phi(y))$ for all $x$, $y\in I$.
Show that a mid-convex function which is bounded on any non-trivial
subinterval of $I$ is convex.
%233G

\spheader 233Yb Generalize 233Xd to arbitrary normed spaces in
place of $\BbbR^r$.
%233G

\spheader 233Yc Let $(X,\Sigma,\mu)$ be a probability space and
$\Tau$ a $\sigma$-subalgebra of $\Sigma$.   Let $\phi$ be a convex
real-valued function with domain an interval $I\subseteq\Bbb R$, and $f$
an integrable real-valued function on $X$ such that $f(x)\in I$ for
almost every $x\in X$ and $\phi f$ is integrable.   Let $g$,
$h$ be conditional expectations on $\Tau$ of $f$, $\phi f$
respectively.   Show that $g(x)\in I$ for almost every $x$ and that
$\phi g\leae h$.
%233J

\spheader 233Yd(i) Show that if $I\subseteq\Bbb R$ is a bounded
interval, $E\subseteq I$ is Lebesgue measurable, and
$\mu E>\bover23\mu I$ where $\mu$ is Lebesgue measure, then for every
$x\in I$ there are $y$, $z\in E$ such that $z=\bover{x+y}2$.   \Hint{by
134Ya/263A, $\mu(x+E)+\mu(2E)>\mu(2I)$.}   (ii) Show that if
$f:[0,1]\to\Bbb R$ is a mid-convex Lebesgue measurable function
(definition:  233Ya), $a>0$, and $E=\{x:x\in[0,1]$, $a\le f(x)<2a\}$ is
not negligible, then there is a non-trivial interval $I\subseteq[0,1]$
such that $f(x)>0$ for every $x\in I$.   \Hint{223B.}   (iii) Suppose
that $f:[0,1]\to\Bbb R$ is a mid-convex function such that $f\le 0$
almost everywhere in $[0,1]$.   Show that $f\le 0$ everywhere in
$\ooint{0,1}$.   \Hint{for every $x\in\ooint{0,1}$,
$\max(f(x-t),f(x+t))\le 0$ for almost every $t\in[0,\min(x,1-x)]$.}
(iv) Suppose that $f:[0,1]\to\Bbb R$ is a mid-convex Lebesgue measurable
function such that $f(0)=f(1)=0$.   Show that $f(x)\le 0$ for every
$x\in[0,1]$.   \Hint{show that $\{x:f(x)\le 0\}$ is dense in $[0,1]$,
use (ii) to show that it is conegligible in $[0,1]$ and apply (iii).}
(v) Show that if $I\subseteq\Bbb R$ is an interval and $f:I\to\Bbb R$ is
a mid-convex Lebesgue measurable function then it is convex.
%233Ya 233G not in 2001 edition

\spheader 233Ye Let $(X,\Sigma,\mu)$ be a probability space, $\Tau$ a
$\sigma$-subalgebra of subsets of $X$, and $f:X\to[0,\infty]$ a
$\Sigma$-measurable function.   Show that (i) there is a $\Tau$-measurable
$g:X\to[0,\infty]$ such that $\int_Fg=\int_Ff$ for every $F\in\Tau$
(ii) any two such functions are equal a.e.

\spheader 233Yf Suppose that $r\ge 1$ and $C\subseteq\BbbR^r\setminus\{0\}$
is a convex set.   Show that there is a non-zero $b\in\BbbR^r$ 
such that $\varinnerprod{b}{z}\ge 0$ for every $z\in C$.
\Hint{if $r=2$, identify $\BbbR^2$ with $\Bbb C$;  reduce to the case in
which $C$ contains no points which are real and negative;  set
$\theta=\sup\{\arg z:z\in C\}$ and $b=-ie^{i\theta}$.   Now induce on $r$.}
%233Xd

\spheader 233Yg Suppose that $r\ge 1$, $C\subseteq\BbbR^r$ is a convex set
and $\phi:C\to\Bbb R$ is a convex function.   Show that there is a function
$h:\BbbR^r\to\coint{-\infty,\infty}$ such that
$\phi(z)=\sup\{h(y)+\varinnerprod{z}{y}:y\in\BbbR^r\}$ for every $z\in C$.
\Hint{try $h(y)=\inf\{\phi(z)-\varinnerprod{z}{y}:z\in C\}$, and apply
233Yf to a translate of $\{(z,t):\phi(z)\le t\}$.}
%233Yf 233Xd 233H

\spheader 233Yh Let $(X,\Sigma,\mu)$ be a probability space, $r\ge 1$ an
integer and $C\subseteq\BbbR^r$ a convex set.   
Let $f_1,\ldots,f_r$ be $\mu$-integrable real-valued functions
and suppose that $\{x:x\in\bigcap_{j\le r}\dom f_j$,
$(f_1(x),\ldots,f_r(x))\in C\}$ is a conegligible subset of $X$.
Show that $(\int f_1,\ldots,\int f_r)\in C$.
\Hint{induce on $r$.}
%233Yg 233H

\spheader 233Yi Let $(X,\Sigma,\mu)$ be a probability space, $r\ge 1$ an
integer, $C\subseteq\BbbR^r$ a convex set and $\phi:C\to\BbbR^r$ a convex
function.   Let $f_1,\ldots,f_r$ be $\mu$-integrable real-valued functions
and suppose that $\{x:x\in\bigcap_{j\le r}\dom f_j$,
$(f_1(x),\ldots,f_r(x))\in C\}$ is a conegligible subset of $X$.
Show that $\phi(\int f_1,\ldots,\int f_r)\le\int\phi(f_1,\ldots,f_r)$.
%233Yg 233Yh 233H

\spheader 233Yj Let $(X,\Sigma,\mu)$ be a measure space with $\mu X>0$, 
$r\ge 1$ an
integer, $C\subseteq\BbbR^r$ a convex set such that $tz\in C$ whenever
$z\in C$ and $t>0$, and $\phi:C\to\Bbb R$ a convex function.
Let $f_1,\ldots,f_r$ be $\mu$-integrable real-valued functions
and suppose that $\{x:x\in\bigcap_{j\le r}\dom f_j$,
$(f_1(x),\ldots,f_r(x))\in C\}$ is a conegligible subset of $X$.
Show that $(\int f_1,\ldots,\int f_r)\in C$ and that
$\phi(\int f_1,\ldots,\int f_r)\le\int\phi(f_1,\ldots,f_r)$.
\Hint{putting 215B(viii) and 235K below together, show that there are a
probability measure $\nu$ on $X$ and a function $h:X\to\coint{0,\infty}$
such that $\int f_jd\mu=\int f_j\times h\,d\nu$ for every $j$.}
%233Yh 233Yi 233H

}%end of exercises

\endnotes{
\Notesheader{233} The concept of `conditional expectation' is
fundamental in probability theory, and will reappear in Chapter 27 in
its natural context.   I hope that even as an exercise in technique,
however, it will strike you as worth taking note of.

I introduced 233E as a `list of elementary facts', and they are indeed
straightforward.   But below the surface there are some remarkable ideas
waiting for expression.   If you take $\Tau$ to be the trivial algebra
$\{\emptyset,X\}$, so that the (unique) conditional expectation of an
integrable function $f$ is the constant function $(\int f)\chi X$, then
233Ed and 233Ee become versions of B.Levi's theorem and Lebesgue's
Dominated Convergence Theorem.   (Fatou's Lemma is in 233Xa.)   Even
233Eg can be thought of as a generalization of the result that
$\int cf=c\int f$, where the constant $c$ has been replaced by a bounded
$\Tau$-measurable function.   A recurrent theme in the later parts of
this treatise will be the search for `conditioned' versions of theorems.
The proof of 233Ee is a typical example of an argument which has been
translated from a proof of the original `unconditioned' result.

I suggested that 233I-233J are surprising, and I think
that most of us find them so, even applied to the list of convex
functions given in 233G.   But I should remark that in a way 233J has
very little to
do with conditional expectations.   The only properties of conditional
expectations used in the proof are (i) that if $g$ is a conditional
expectation of $f$, then $a\chi X+bg$ is a conditional expectation of
$a\chi X+bf$ for all real $a$, $b$ (ii) if $g_1$, $g_2$ are conditional
expectations of $f_1$, $f_2$ and $f_1\leae f_2$, then $g_1\leae g_2$.   
See 244Xm below.   Jensen's inequality has an interesting extension to the
multidimensional case, explored in 
233Yf-233Yj.   %233Yf 233Yg 233Yh 233Yi 233Yj
If you have encountered `geometric' forms of the Hahn-Banach theorem (see
3A5C in Volume 3) you will find 233Yf and 233Yg very natural, and you may
notice that the finite-dimensional case is slightly different from the
infinite-dimensional case you have probably been taught.   
I think that in fact the most
delicate step is in 233Yh.

Note that 233Ib can be regarded as the special case of 233J in which
$\Tau=\{\emptyset,X\}$.   In fact 233Ia can be derived from 233Ib
applied to the measure $\nu$ where $\nu E=\int_Eg$ for every
$E\in\Sigma$.

Like 233B, 233K seems to have rather a lot of technical detail in the
argument.   The point of this result is that we can deduce the
integrability of $f\times h$ from that of $g_0\times h$ (but not from
the integrability of $g\times h$;  see 233Xg).
Otherwise it should be routine.
}%end of notes

\discrpage

