\frfilename{mt451.tex}
\versiondate{8.11.07}
\copyrightdate{2002}

\def\undphi{\underline{\phi}\vthsp}

\def\chaptername{Perfect measures, disintegrations and processes}
\def\sectionname{Perfect, compact and countably compact measures}

\newsection{451}

In \S\S342-343 I introduced `compact' and `perfect' measures as part of
a study of the representation of homomorphisms of measure algebras by
functions between measure spaces.   An intermediate class of `countably
compact' measures\cmmnt{ (the `compact' measures of
{\smc Marczewski 53})} has appeared in the exercises.   It is now time to collect these
ideas together in a more systematic way.   In this section I run through
the standard properties of compact, countably compact and perfect
measures (451A-451J), with a couple of simple examples of their
interaction with topologies (451M-451P).   An example of a
perfect measure space which is not countably compact is in 451U.   Some
new ideas, involving non-trivial set theory, show that measurable
functions from compact totally finite measure spaces to metrizable
spaces have `essentially separable ranges' (451R);  consequently, any
measurable function from a Radon measure space to a metrizable space is
almost continuous (451T).

\cmmnt{
\leader{451A}{}\cmmnt{ Let me begin by recapitulating the principal
facts already covered.

\medskip

}{\bf (a)} A family $\Cal K$ of sets is a {\bf compact class} if
$\bigcap\Cal K'\ne\emptyset$ whenever $\Cal K'\subseteq\Cal K$ has the
finite intersection property.
If $\Cal K\subseteq\Cal PX$, then
$\Cal K$ is a compact class iff there is a compact topology on $X$ for
which every member of $\Cal K$ is closed\cmmnt{ (342D)}.   A subfamily
of a compact class is compact\cmmnt{ (342Ab)}.

\spheader 451Ab A measure on a set $X$ is {\bf compact} if it is inner
regular with respect to some compact class of sets;  equivalently, if it
is inner regular with respect to the closed sets for some compact
topology on $X$\cmmnt{ (342F)}.   All Radon measures are compact
measures\cmmnt{ (416Wa)}.   If $(X,\Sigma,\mu)$ is a semi-finite
compact measure space with measure algebra $\frak A$, $(Y,\Tau,\nu)$ is
a complete strictly localizable measure space with measure algebra
$\frak B$, and $\pi:\frak A\to\frak B$ is an order-continuous Boolean
homomorphism, there is a function $g:Y\to X$ such that
$g^{-1}[E]\in\Tau$ and $g^{-1}[E]^{\ssbullet}=\pi(E^{\ssbullet})$ for
every $E\in\Sigma$\cmmnt{ (343B)}.

\spheader 451Ac A family $\Cal K$ of sets is a {\bf countably compact
class} if $\bigcap_{n\in\Bbb N}K_n\ne\emptyset$ whenever
$\sequencen{K_n}$ is a sequence in $\Cal K$ such that
$\bigcap_{i\le n}K_i\ne\emptyset$ for every $n\in\Bbb N$.   Any
subfamily of a
countably compact class is countably compact.   If $\Cal K$ is a
countably compact class, then there is a countably compact class
$\Cal K^*\supseteq\Cal K$ which is closed under finite unions and
countable intersections\cmmnt{ (413R)}.

\spheader 451Ad A measure space $(X,\Sigma,\mu)$ is {\bf perfect} if
whenever $f:X\to\Bbb R$ is measurable, $E\in\Sigma$ and $\mu E>0$, there
is a compact set $K\subseteq f[E]$ such that $\mu f^{-1}[K]>0$.
A countably separated semi-finite measure space is compact iff it is
perfect\cmmnt{ (343K)}.   A measure space $(X,\Sigma,\mu)$ is
isomorphic to the unit interval with Lebesgue measure iff it is an
atomless complete countably separated perfect probability
space\cmmnt{ (344Ka)}.
}%end of comment

\leader{451B}{}\cmmnt{ Now for the new class of measures.

\medskip

\noindent}{\bf Definition} Let $(X,\Sigma,\mu)$ be a measure space.
Then $(X,\Sigma,\mu)$, or $\mu$, is {\bf
countably compact} if $\mu$ is inner regular with respect to some
countably compact class of sets.

\cmmnt{Evidently compact measures are
also countably compact.   A simple example of a countably compact
measure which is not compact is the countable-cocountable measure on an
uncountable set (342M).   For an example of a perfect
measure which is not countably compact, see 451U.

Note that if $\mu$ is inner regular with respect to a countably compact
class $\Cal K$, then it is also inner regular with respect to
$\Cal K\cap\Sigma$ (411B), and $\Cal K\cap\Sigma$ is still countably
compact.}

\leader{451C}{Proposition}\cmmnt{ ({\smc Ryll-Nardzewski 53})} Any
semi-finite countably compact measure is perfect.

\proof{ The central idea is the same as in 342L, but we need to refine
the second half of the argument.

\medskip

{\bf (a)} Let $(X,\Sigma,\mu)$ be a countably compact measure space,
$f:X\to\Bbb R$ a measurable function, and $E\in\Sigma$ a set of positive
measure.  Let $\Cal K$ be a countably compact class such that $\mu$ is
inner regular with respect to $\Cal K$;  by 451Ac, we may suppose that
$\Cal K$ is closed under finite unions and countable intersections.

Because $\mu$ is semi-finite, there is a measurable set $F\subseteq E$
such that $0<\mu F<\infty$;  replacing $F$ by a set of the form $F\cap
f^{-1}[[-n,n]]$ if necessary, we may suppose that $f[F]$ is bounded;
finally, we may suppose that $F\in\Cal K$.
Let $\langle\epsilon_q\rangle_{q\in\Bbb Q}$ be a family of strictly
positive real numbers such that $\sum_{q\in\Bbb Q}\epsilon_q<\bover12\mu
F$.   For each $q\in\Bbb Q$,
set $E_q=\{x:x\in F,\,f(x)\le q\}$, $E'_q=\{x:x\in F,\,f(x)>q\}$, and
choose $K_q$, $K'_q\in\Cal K\cap\Sigma$ such that $K_q\subseteq E_q$,
$K'_q\subseteq E'_q$
and $\mu(E_q\setminus K_q)\le\epsilon_q$, $\mu(E'_q\setminus
K'_q)\le\epsilon_q$.   Then $K=\bigcap_{q\in\Bbb Q}(K_q\cup K'_q)\in\Cal
K\cap\Sigma$, $K\subseteq F$ and

\Centerline{$\mu(F\setminus K)\le\sum_{q\in\Bbb Q}\mu(E_q\setminus
K_q)+\mu(E'_q\setminus K'_q)<\mu F$, }

\noindent so $\mu K>0$.

\medskip

{\bf (b)} Take any $t\in\overline{f[K]}$.   Enumerate $\Bbb Q$ as
$\sequencen{q_n}$ and define $\sequencen{L_n}$ in $\Cal K$ by the rule

$$\eqalign{L_n&=K_{q_n}\text{ if }t<q_n,\cr
&=K'_{q_n}\text{ if }t>q_n,\cr
&=F\text{ if }t=q_n.\cr}$$

\noindent Now $\bigcap_{i\le n}L_i\ne\emptyset$ for every $n\in\Bbb N$.
\Prf\ Because $t\in\overline{f[K]}$, there must be some $s\in f[K]$ such
that $s<q_i$ whenever $i\le n$ and $t<q_i$, while $s>q_i$ whenever $i\le
n$ and $t>q_i$.   Let $x\in K$ be such that $f(x)=s$.   Then, for any
$i\le n$,

\inset{{\it either} $t<q_i$, $f(x)<q_i$ so $x\notin K'_{q_i}$ and $x\in
K_{q_i}=L_i$}

\inset{{\it or} $t>q_i$, $f(x)>q_i$ so $x\notin K_{q_i}$ and $x\in
K'_{q_i}=L_i$,}

\inset{{\it or} $t=q_i$ and $x\in F=L_i$.}

\noindent So $x\in\bigcap_{i\le n}L_i$.\ \Qed

As $\Cal K$ is a countably compact class, there must be some
$x\in\bigcap_{n\in\Bbb N}L_n$.   But this means that, for any $n\in\Bbb
N$,

\inset{if $t>q_n$ then $x\in K'_{q_n}$ and $f(x)>q_n$,}

\inset{if $t<q_n$ then $x\in K_{q_n}$ and $f(x)<q_n$.}

\noindent So in fact $f(x)=t$.   Accordingly $t\in f[K]$.

\medskip

{\bf (c)} What this shows is that $\overline{f[K]}\subseteq f[K]$ and
$f[K]$ is closed.   Because (by the choice of $F$) it is also bounded,
it is compact (2A2F).   Of course we now have $f[K]\subseteq f[E]$,
while $\mu f^{-1}[f[K]]\ge\mu K>0$.   As $f$ and $E$ are arbitrary,
$\mu$ is perfect.
}%end of proof of 451C

\leader{451D}{Proposition} Let $(X,\Sigma,\mu)$ be a measure space, and
$E\in\Sigma$;  let $\mu_E$ be the subspace measure on $E$.

(a) If $\mu$ is compact, so is $\mu_E$.

(b) If $\mu$ is countably compact, so is $\mu_E$.

(c) If $\mu$ is perfect, so is $\mu_E$.

\proof{{\bf (a)-(b)} Let $\Cal K$ be a (countably) compact class such
that $\mu$ is inner regular with respect to $\Cal K$.   Then $\mu_E$ is
inner regular with respect to $\Cal K$ (412Oa), so is (countably)
compact.

\medskip

{\bf (c)} Suppose that $f:E\to\Bbb R$ is $\Sigma_E$-measurable, where
$\Sigma_E=\Sigma\cap\Cal PE$ is the subspace $\sigma$-algebra, and
$F\subseteq E$ is such that $\mu F>0$.   Set

$$\eqalign{g(x)&=\arctan f(x)\text{ if }x\in E,\cr
&=2\text{ if }x\in X\setminus E.\cr}$$

\noindent Then $g$ is $\Sigma$-measurable, so there is a compact set
$K\subseteq g[F]$ such that $\mu g^{-1}[K]>0$.   Set
$L=\{\tan t:t\in K\}$;  then $L\subseteq f[F]$ is compact and
$f^{-1}[L]=g^{-1}[K]$ has non-zero measure.   As $f$ and $F$ are
arbitrary, $\mu_E$ is perfect.
}%end of proof of 451D

\leader{451E}{Proposition} Let $(X,\Sigma,\mu)$ be a perfect measure
space.

(a) If $(Y,\Tau,\nu)$ is another measure space and $f:X\to Y$ is an
\imp\ function, then $\nu$ is perfect.

(b) In particular, $\mu\restrp\Tau$ is perfect for any
$\sigma$-subalgebra $\Tau$ of $\Sigma$.

\proof{{\bf (a)} Suppose that $g:Y\to\Bbb R$ is $\Tau$-measurable and
$F\in\Tau$ is such that $\nu F>0$.   Then $gf:X\to\Bbb R$ is
$\Sigma$-measurable and $\mu f^{-1}[F]>0$.   So there is a compact set
$K\subseteq(gf)[f^{-1}[F]]$ such that $\mu(gf)^{-1}[K]>0$.   But now
$K\subseteq g[F]$ and $\nu g^{-1}[K]>0$.    As $g$ and $F$ are
arbitrary, $\nu$ is perfect.

\medskip

{\bf (b)} Apply (a) to $Y=X$, $\nu=\mu\restrp\Tau$ and $f$ the identity
function.
}%end of proof of 451E

\cmmnt{\medskip

\noindent{\bf Remark} We shall see in 452R that there is a similar
result for countably compact measures;  but for compact measures, there
is not (342Xf, 451Xh).
}%end of comment

\leader{451F}{Lemma}\cmmnt{ ({\smc Sazonov 66})} Let $(X,\Sigma,\mu)$
be a semi-finite measure
space.   Then the following are equiveridical:

(i) $\mu$ is perfect;

(ii) $\mu\restrp\Tau$ is compact for every
countably generated $\sigma$-subalgebra $\Tau$ of $\Sigma$;

(iii) $\mu\restrp\Tau$ is perfect for every countably generated
$\sigma$-subalgebra $\Tau$ of $\Sigma$;

(iv) for every countable set $\Cal E\subseteq\Sigma$ there is a
$\sigma$-algebra $\Tau\supseteq\Cal E$ such that $\mu\restrp\Tau$ is
perfect.

\proof{{\bf (a)(i)$\Rightarrow$(ii)} Suppose that $\mu$ is perfect, and
that $\Tau$ is a countably generated $\sigma$-subalgebra of $\Sigma$.
Let $\sequencen{E_n}$ be a sequence in $\Tau$ which $\sigma$-generates
it, and define $f:X\to\Bbb R$ by setting
$f(x)=\sum_{n=0}^{\infty}3^{-n}\chi E_n(x)$ for every $x\in X$.
Then $f$ is measurable.   Set $\Cal
K=\{f^{-1}[L]:L\subseteq f[X]$ is compact$\}$.   Then $\Cal K$ is a
compact class.   \Prf\ If $\Cal K'\subseteq\Cal K$ is non-empty and has
the finite intersection property, then $\Cal L'=\{L:L\subseteq f[X]$ is
compact, $f^{-1}[L]\in\Cal K'\}$ is also a non-empty family with the
finite intersection property.   So there is an $\alpha\in\bigcap\Cal L'$;
since $\alpha\in f[X]$, there is an $x$ such that $f(x)=\alpha$,
and now $x\in\bigcap\Cal K'$.   As $\Cal K'$ is arbitrary, $\Cal K$ is a
compact class.\ \Qed

Observe next that, for any $n\in\Bbb N$,

\Centerline{$E_n=\{x:\,\exists\,I\subseteq n,\,
  \sum_{i\in I}3^{-i}+3^{-n}\le f(x)<\sum_{i\in I}3^{-i}+3^{-n+1}\}$.}

\noindent So $\Tau'=\{f^{-1}[F]:F\subseteq\Bbb R\}$ contains
every $E_n$;  as it is a
$\sigma$-algebra of subsets of $X$, it includes $\Tau$.

Now $\mu\restrp\Tau$ is inner regular with respect to $\Cal K$.   \Prf\
If $E\in\Tau$ and $\mu E>0$, there is a set $F\subseteq\Bbb R$ such that
$E=f^{-1}[F]$.   Because $f$ is $\Sigma$-measurable
and $\mu$ is perfect, there is a compact set $L\subseteq f[E]$ such that
$\mu f^{-1}[L]>0$.   But now $f^{-1}[L]\in\Cal K\cap\Tau$, and
$f^{-1}[L]\subseteq E$ because $L\subseteq F$.
Because $\Cal K$ is closed under finite unions, this is enough to show
that $\mu\restrp\Tau$ is inner regular with respect to $\Cal K$.\ \Qed

Thus $\Cal K$ witnesses that $\mu\restrp\Tau$ is a compact measure.

\medskip

{\bf (b)(ii)$\Rightarrow$(i)} Now suppose that $\mu\restrp\Tau$ is
compact for every countably generated $\sigma$-algebra
$\Tau\subseteq\Sigma$, that $f:X\to\Bbb R$ is a measurable function,
and that $\mu E>0$.   Let $F\subseteq E$ be a
measurable set of non-zero finite measure, and $\Tau$ the
$\sigma$-algebra generated by
$\{F\}\cup\{f^{-1}[\,\ooint{-\infty,q}\,]:q\in\Bbb Q\}$, so that $\Tau$
is countably generated and $f$ is $\Tau$-measurable.   Because
$\mu\restrp\Tau$ is compact, so is the subspace measure
$(\mu\restrp\Tau)_F$ (451Da);  but this is now perfect (342L or 451C),
while $F\in\Tau$ and $\mu F>0$, so there is a compact set
$L\subseteq f[F]\subseteq f[E]$ such that $\mu f^{-1}[L]>0$.   As $f$
and $E$ are arbitrary, $\mu$ is perfect.

\medskip

{\bf (c)(i)$\Rightarrow$(iv)} is trivial.

\medskip

{\bf (d)(iv)$\Rightarrow$(iii)} If (iv) is true, and $\Tau$ is a
countably generated $\sigma$-subalgebra of $\Sigma$, let $\Cal E$ be a
countable set generating it.   Then there is a $\sigma$-algebra
$\Tau_1\supseteq\Cal E$ such that $\mu\restrp\Tau_1$ is perfect.
By 451Eb, $\mu\restrp\Tau=(\mu\restrp\Tau_1)\restrp\Tau$ is compact,
therefore perfect.

\medskip

{\bf (e)(iii)$\Rightarrow$(ii)} If (iii) is true, and $\Tau$ is a
countably generated $\sigma$-subalgebra of $\Sigma$, then
$\mu\restrp\Tau$ is perfect;  but as (i)$\Rightarrow$(ii), and $\Tau$ is
a countably generated $\sigma$-subalgebra of itself, $\mu\restrp\Tau$ is
compact.
}%end of proof of 451F

\leader{451G}{Proposition} Let $(X,\Sigma,\mu)$ be a
measure space.   Let $(X,\hat\Sigma,\hat\mu)$ be its completion and
$(X,\tilde\Sigma,\tilde\mu)$ its c.l.d.\ version.   Then

(a)(i) if $\mu$ is compact, so are $\hat\mu$ and $\tilde\mu$;

\quad(ii) if $\mu$ is semi-finite and either $\hat\mu$ or $\tilde\mu$ is
compact, then $\mu$ is compact.

(b)(i) If $\mu$ is countably compact, so are $\hat\mu$ and $\tilde\mu$;

\quad(ii) if $\mu$ is semi-finite and either $\hat\mu$ or $\tilde\mu$ is
countably compact, then $\mu$ is countably compact.

(c)(i) If $\mu$ is perfect, so are $\hat\mu$ and $\tilde\mu$;

\quad(ii) if $\hat\mu$ is perfect, then $\mu$ is perfect;

\quad(iii) if $\mu$ is semi-finite and $\tilde\mu$ is perfect, then
$\mu$ is perfect.

\proof{{\bf (a)-(b)}  The arguments for $\hat\mu$ and $\tilde\mu$ run
very closely together.   Write $\check\mu$ for either of them, and
$\check\Sigma$ for its domain.

\medskip

\quad{\bf (i)} If $\mu$ is inner regular with respect to $\Cal K$, so is
$\check\mu$ (412Ha).   So if $\mu$ is (countably) compact, so is
$\check\mu$.

\medskip

\quad{\bf (ii)} Now suppose that $\mu$ is semi-finite.   The point is
that if $\Cal K$ is closed under countable intersections and $\check\mu$
is inner regular with respect to $\Cal K$, so is $\mu$.   \Prf\ Suppose
that $E\in\Sigma$ and that $\mu E>\gamma$.   Choose sequences
$\sequencen{E_n}$ in $\Sigma$ and $\Cal K_n$ in $\Cal K$ inductively, as
follows.   $E_0$ is to be such that $E_0\subseteq E$ and
$\gamma<\mu E_0<\infty$.   Given that $\gamma<\mu E_n<\infty$, let
$K_n\in\Cal K\cap\check\Sigma$ be such that $K_n\subseteq E_n$ and
$\check\mu K_n>\gamma$;  now take $E_{n+1}\in\Sigma$ such that
$E_{n+1}\subseteq K_n$ and $\mu E_{n+1}=\check\mu K_n$ (212C or 213Fc),
and continue.
At the end of the induction, $\bigcap_{n\in\Bbb N}K_n=\bigcap_{n\in\Bbb
N}E_n$ is a member of
$\Sigma\cap\Cal K$ included in $E$ and of measure at least $\gamma$.
As $E$ and $\gamma$ are arbitrary, $\mu$ is inner regular with respect
to $\Cal K$.\ \Qed

It follows that if $\check\mu$ is compact or countably compact, so is
$\mu$.  \Prf\ Let $\Cal K$ be a (countably) compact class such that
$\check\mu$ is inner regular with respect to $\Cal K$;  by 451Aa or
451Ac, there is a (countably) compact class $\Cal K^*$, including $\Cal
K$, which is closed under countable intersections, so that $\mu$ is
inner regular with respect to $\Cal K^*$, and is itself (countably)
compact.\ \Qed

\medskip

{\bf (c)(i)}\grheada\ Let $f:X\to\Bbb R$ be $\hat\Sigma$-measurable, and
$E\in\hat\Sigma$ such that $\hat\mu E>0$.   Then there are a
$\mu$-conegligible set $F_0\in\Sigma$ such that $f\restr F_0$ is
$\Sigma$-measurable (212Fa), and an $F_1\in\Sigma$ such that
$F_1\subseteq E$ and $\hat\mu(E\setminus F_1)=0$.  Set $F=F_0\cap F_1$.
By 451Dc, the subspace measure $\mu_F$ is perfect, while $f\restr F$ is
$\Sigma_F$-measurable;  so there is a compact set $K\subseteq f[F]$
such that $\mu(F\cap f^{-1}[K])>0$.   But now $K\subseteq f[E]$ and
$\hat\mu f^{-1}[K]>0$.   As $f$ and $E$ are arbitrary, $\hat\mu$ is
perfect.

\medskip

\qquad\grheadb\ Let $f:X\to\Bbb R$ be $\tilde\Sigma$-measurable,
and $E\in\tilde\Sigma$ such
that $\tilde\mu E>0$.   Then there is a set $F\in\Sigma$ such that
$\mu F<\infty$ and $\hat\mu(F\cap E)$ is defined and greater than $0$
(213D).   In this case, $\hat\mu$ and $\tilde\mu$ induce the same
subspace measure
$\hat\mu_F$ on $F$.   Accordingly $f\restr F$ is
$\hat\Sigma$-measurable.   Because $\hat\mu$ is perfect (by ($\alpha$)
just above), so is $\hat\mu_F$ (451Dc), and there is a compact set
$K\subseteq f[F\cap E]$ such that $\hat\mu_F(f\restr F)^{-1}[K]>0$.
But now, of course, $K\subseteq f[E]$ and $\tilde\mu f^{-1}[K]>0$.   As
$f$ and $E$ are arbitrary, $\tilde\mu$ is perfect.

\medskip

\quad{\bf (ii)} Suppose that $\hat\mu$ is perfect.   Since
$\mu=\hat\mu\restr\Sigma$, $\mu$ is perfect, by 451Eb.

\medskip

\quad{\bf (iii)} Similarly, if $\tilde\mu$ is perfect and $\mu$ is
semi-finite, then $\mu=\tilde\mu\restr\Sigma$, by 213Hc, so
$\mu$ is perfect.
}%end of proof of 451G

\leader{451H}{Lemma} Let $\familyiI{X_i}$ be a family of sets with
product $X$.   Suppose that $\Cal K_i\subseteq\Cal PX_i$ for each $i\in
I$, and set $\Cal K=\{\pi_i^{-1}[K]:i\in I,\,K\in\Cal K_i\}$, where
$\pi_i:X\to X_i$ is the coordinate map for each $i\in I$.   Then

(a) if every $\Cal K_i$ is a compact class, so is $\Cal K$;

(b) if every $\Cal K_i$ is a countably compact class, so is $\Cal K$.

\proof{{\bf (a)} For each $i\in I$, let $\frak T_i$ be a compact
topology on $X_i$ such that every
member of $\Cal K_i$ is closed.   Then the product topology $\frak T$ on
$X$ is compact (3A3J), and every member of $\Cal K$ is $\frak T$-closed,
so $\Cal K$ is a compact class.

\medskip

{\bf (b)} If $\sequencen{K_n}$ is a sequence in
$\Cal K$ such that $\bigcap_{k\le n}K_k\ne\emptyset$ for every
$n\in\Bbb N$, then we must be able to express each $K_n$ as
$\pi_{j_n}^{-1}[L_n]$, where $j_n\in I$ and $L_n\in\Cal K_{j_n}$
for every $n$.   Now, for $i\in I$,
$\Cal L_i=\{K_{j_n}:n\in\Bbb N,\,j_n=i\}$ is a countable subset of
$\Cal K_i$, and any finite
subfamily of $\Cal L_i$ has non-empty intersection.   Since
$K_0\ne\emptyset$, $X_i\ne\emptyset$;  so, whether $\Cal L_i$ is empty
or not, $X_i\cap\bigcap\Cal L_i$ is non-empty.   Accordingly

\Centerline{$\bigcap_{k\in\Bbb N}K_k
=\prod_{i\in I}(X_i\cap\bigcap\Cal L_i)$}

\noindent is not empty.   As $\sequencen{K_n}$ is arbitrary, $\Cal K$ is
countably compact.
}%end of proof of 451H

\leader{451I}{Theorem} Let $(X,\Sigma,\mu)$ and $(Y,\Tau,\nu)$ be
measure spaces, with c.l.d.\ product $(X\times Y,\Lambda,\lambda)$.

(a) If $\mu$ and $\nu$ are compact, so is $\lambda$.

(b) If $\mu$ and $\nu$ are countably compact, so is $\lambda$.

(c) If $\mu$ and $\nu$ are perfect, so is $\lambda$.

\proof{{\bf (a)-(b)} Let $\Cal K\subseteq\Cal PX$,
$\Cal L\subseteq\Cal PY$ be
(countably) compact classes such that $\mu$ is inner regular with
respect to $\Cal K$ and $\nu$ is inner regular with respect to $\Cal L$.
Set $\Cal M_0=\{K\times Y:K\in\Cal K\}\cup\{X\times L:L\in\Cal L\}$.
Then $\Cal M_0$ is (countably) compact, by 451H.   By 451Aa/451Ac, there
is a (countably) compact class $\Cal M\supseteq\Cal M_0$ which is closed
under finite unions and countable intersections.   By 412R, $\lambda$ is
inner regular with respect to $\Cal M$, so is (countably) compact.

\medskip

{\bf (c)(i)} Let $f:X\times Y\to\Bbb R$ be $\Lambda$-measurable, and
$V\in\Lambda$ a set of positive measure.   Then there are $G\in\Sigma$,
$H\in\Tau$ such that $\mu G$, $\nu H$ are both finite and
$\lambda(V\cap(G\times H))>0$.   Recall that the subspace measure
$\lambda_{G\times H}$ on $G\times H$ is just the product of the subspace
measures $\mu_G$ and $\mu_H$ (251P(ii-$\alpha$)), and is the completion
of its restriction $\theta$ to the $\sigma$-algebra
$\Sigma_G\tensorhat\Tau_H$ generated by $\{E\times F:E\in\Sigma_G$,
$F\in\Tau_H\}$, where $\Sigma_G$ and $\Tau_H$ are the subspace
$\sigma$-algebras on $G$, $H$ respectively, the domains of $\mu_G$ and
$\mu_H$ (251K).
Next, for any $W\in\Sigma_G\tensorhat\Tau_H$, there are
countable families $\Cal E\subseteq\Sigma_G$, $\Cal F\subseteq\Tau_H$
such that $W$ belongs to the $\sigma$-algebra of subsets of $G\times H$
generated by $\{E\times F:E\in\Cal E,\,F\in\Cal F\}$ (331Gd).

\medskip

\quad{\bf (ii)} The point is that $\theta$ is perfect.   \Prf\ Let
$\Lambda'$ be any countably generated $\sigma$-subalgebra of
$\Sigma_G\tensorhat\Tau_H$;  let $\sequencen{W_n}$ be a sequence in
$\Lambda'$ generating it.   Then there are countable families
$\Cal E\subseteq\Sigma_G$, $\Cal F\subseteq\Tau_H$ such that every $W_n$
belongs to the $\sigma$-algebra generated by
$\{E\times F:E\in\Cal E,\,F\in\Cal F\}$.   Let $\Sigma'$, $\Tau'$ be the
$\sigma$-algebras of
subsets of $G$ and $H$ generated by $\Cal E$ and $\Cal F$ respectively;
then every $W_n$ belongs to $\Sigma'\tensorhat\Tau'$, so
$\Lambda'\subseteq\Sigma'\tensorhat\Tau'$.   Let $\lambda'$ be the
product of the measures $\mu\restr\Sigma'=\mu_G\restr\Sigma'$ and
$\nu\restrp\Tau'$.   Then $\lambda'$ is the completion of its
restriction to $\Sigma'\tensorhat\Tau'$.

Now trace through the results above.   $\mu_G$ and $\nu_H$ are perfect
(451Dc), so $\mu_G\restr\Sigma'$ and $\nu_H\restrp\Tau'$ are compact
(451F), so $\lambda'$ is
compact ((a) of this theorem), so $\lambda'$ is perfect
(342L or 451C again).
But $\theta$ must agree with $\lambda'$ on $\Lambda'$, by Fubini's
theorem (252D), or otherwise, so $\theta\restr\Lambda'$ is a restriction
of $\lambda'$, and is perfect (451Eb).

Thus $\theta\restr\Lambda'$ is perfect for every countably generated
$\sigma$-subalgebra $\Lambda'$ of $\dom\theta$.   By 451F, $\theta$ is
perfect.\ \Qed

\medskip

\quad{\bf (iii)} By 451G(c-i), $\lambda_{G\times H}$ is perfect.   Now
$f\restr G\times H$ is measurable, and $\lambda_{G\times
H}(V\cap(G\times H))>0$, so there is a compact set $K\subseteq
f[V\cap(G\times H)]$ such that
$\lambda_{G\times H}((G\times H)\cap f^{-1}[K])>0$;  in which case
$K\subseteq f[V]$ and $\lambda f^{-1}[K]>0$.

As $f$ and $V$ are arbitary, $\lambda$ is perfect.
}%end of proof of 451I

\leader{451J}{Theorem} Let $\familyiI{(X_i,\Sigma_i,\mu_i)}$ be a family
of probability spaces, with product $(X,\Sigma,\mu)$.

(a) If every $\mu_i$ is compact, so is $\mu$.

(b) \cmmnt{({\smc Marczewski 53})}
If every $\mu_i$ is countably compact, so is $\mu$.

(c) If every $\mu_i$ is perfect, so is $\mu$.

\proof{ The same strategy as in 451I is again effective.

\medskip

{\bf (a)-(b)} For each $i\in I$, let $\Cal K_i\subseteq\Cal PX_i$ be a
(countably) compact class such that $\mu_i$ is inner regular with
respect to $\Cal K_i$.   Set
$\Cal M_0=\{\pi_i^{-1}[K]:i\in I,\,K\in\Cal K_i\}$, so that
$\Cal M_0$ is (countably) compact.   Let $\Cal M\supseteq\Cal M_0$ be a
(countably)
compact class which is closed under finite unions and countable
intersections.   By 412T, $\mu$ is inner regular with respect to
$\Cal M$, so is (countably) compact.

\medskip

{\bf (c)} Let $\Lambda'$ be a countably generated $\sigma$-subalgebra of
$\Tensorhat_{i\in I}\Sigma_i$, the $\sigma$-algebra of subsets of $X$
generated by the sets $\{x:x(i)\in E\}$ for $i\in I$ and $E\in\Sigma_i$.
Then $\lambda\restr\Lambda'$ is perfect.   \Prf\ For every
$W\in\Tensorhat_{i\in I}\Sigma_i$, we must be able to find countable
subsets $\Cal E_i$ of $\Sigma_i$ such that $W$ is in the
$\sigma$-algebra generated by $\{\pi_i^{-1}[E]:i\in I,\,E\in\Sigma_i\}$;
so there are in fact countable sets $\Cal E_i\subseteq\Sigma_i$ such
that the $\sigma$-algebra generated by $\{\pi_i^{-1}[E]:i\in
I,\,E\in\Sigma_i\}$ includes $\Lambda'$.   Let $\Tau_i$ be the
$\sigma$-subalgebra of $\Sigma_i$ generated by $\Cal E_i$, so that
$\mu_i\restrp\Tau_i$ is compact.   Let $\lambda'$ be the product of
$\familyiI{\mu_i\restrp\Tau_i}$;  then $\lambda'$ is compact, by (a)
above, therefore perfect.   Now $\lambda$ is an extension of $\lambda'$,
by 254G or otherwise, so $\lambda'$ is an extension of
$\lambda\restr\Lambda'$, and $\lambda\restr\Lambda'$ is perfect.\ \QeD\
As $\Lambda'$ is arbitrary, $\lambda\restr\Tensorhat_{i\in I}\Sigma_i$
is perfect, and its completion $\lambda$ (254Ff) also is perfect.
}%end of proof of 451J

\cmmnt{\medskip

\noindent{\bf Remark} This theorem is generalized in 454Ab.
}

\leader{451K}{}\cmmnt{ The following result is interesting because it
can be reached from an unexpectedly weak hypothesis;  it will be useful in
\S455.

\medskip

\noindent}{\bf Proposition} Let $\familyiI{X_i}$ be a family of sets
with product $X$, and
$\Sigma_i$ a $\sigma$-algebra of subsets of $X_i$ for each $i$.   Let
$\lambda$ be a perfect totally finite measure with domain
$\Tensorhat_{i\in I}\Sigma_i$.   Set $\pi_J(x)=x\restr J$ for
$x\in X$ and $J\subseteq I$.

(a) Let $\Cal K$ be the set
$\{V:V\subseteq X$, $\pi_J[V]\in\Tensorhat_{i\in J}\Sigma_i$ for every
$J\subseteq I\}$.   Then $\lambda$ is inner regular with respect to
$\Cal K$.

(b) Let $\hat\lambda$ be the completion of $\lambda$.

\quad(i) For any $J\subseteq I$, the completion of the image measure
$\lambda\pi_J^{-1}$ on $\prod_{i\in J}X_i$ is the image measure
$\hat\lambda\pi_J^{-1}$.

\quad(ii) If $W$ is measured by $\hat\lambda$ and $W$ is determined by
coordinates in $J\subseteq I$, then there is a
$V\in\Tensorhat_{i\in I}\Sigma_i$ such that
$V\subseteq W$, $V$ is determined by coordinates in $J$ and $W\setminus V$
is $\lambda$-negligible.

\proof{{\bf (a)(i)} Take $W\in\Tensorhat_{i\in I}\Sigma_i$.
Then we can find a family $\familyiI{\Tau_i}$ such
that $\Tau_i$ is a countably generated $\sigma$-subalgebra of $\Sigma_i$
for each $i$ and $W\in\Tensorhat_{i\in I}\Tau_i$.
For each $i\in I$ and $E\in\Tau_i$ set
$\lambda_iE=\lambda\{x:x\in X$, $x(i)\in E\}$;  then $\lambda_i$ is perfect
(451Ea).   Because $\Tau_i$ is countably generated, $\lambda_i$ is compact
(451F);  let $\Cal K_i$ be a compact class such that $\lambda_i$ is inner
regular with respect to $\Cal K_i$.   By 342D, we may suppose that
$\Cal K_i$ is the family of closed sets for a compact topology $\frak T_i$
on $X_i$.

\medskip

\quad{\bf (ii)} Let $\Cal V$ be the family of all sets $V\subseteq X$
expressible in the form

\Centerline{$V
=\bigcap_{n\in\Bbb N}\bigcup_{i\in J_n}\{x:x\in X$, $x(i)\in K_{ni}\}$}

\noindent where $\sequencen{J_n}$ is a sequence of finite subsets of $I$
and $K_{ni}\in\Cal K_i\cap\Tau_i$ whenever $n\in\Bbb N$ and $i\in J_n$.
Given $V$ expressed in this form, set
$V_n=\bigcap_{m\le n}\bigcup_{i\in J_m}\{x:x(i)\in K_{mi}\}$ for each $n$.
Then $\pi_J[V]=\bigcap_{n\in\Bbb N}\pi_J[V_n]$ for every
$J\subseteq I$.   \Prf\ The product topology
$\frak T$ on $X$ is compact, and all the $V_n$ are $\frak T$-closed.
If $z\in\bigcap_{n\in\Bbb N}\pi_J[V_n]$, then for each $n\in\Bbb N$ there
is an $x_n\in V_n$ such that $\pi_J(x)=z$.   Let $x$
be a cluster point of $\sequencen{x_n}$.
The topologies are not Hausdorff, so we do not know at once that
$\pi_J(x)=z$;  but if we define $x'$ by saying that

$$\eqalign{x'(i)
&=z(i)\text{ if }i\in J,\cr
&=x(i)\text{ if }i\in I\setminus J,\cr}$$

\noindent then any neighbourhood $U$
of $x'$ must include a neighbourhood of
the form $\{y:y(i)\in U_i$ for $i\in K\}$ where $K\subseteq I$ is finite
and $U_i$ is a neighbourhood of $x'(i)$ for each $i\in K$.   In this case,
$\{y:y\in U_i$ for $i\in K\setminus J\}$ is a neighbourhood of $x$, so

\Centerline{$\{n:x_n\in U\}
\supseteq\{n:x_n(i)\in U_i$ for $i\in K\setminus J\}$}

\noindent is infinite.   Thus $x'$ also is a cluster point of
$\sequencen{x_n}$, while $\pi_J(x')=z$.   Since
$x'\in\overline{\{x_m:m\ge n\}}\subseteq V_n$ for every $n$, $x\in V$,
and $z\in\pi_J[V]$.   Thus
$\bigcap_{n\in\Bbb N}\pi_J[V_n]\subseteq\pi_J[V]$.   Since surely
$\pi_J[V]\subseteq\bigcap_{n\in\Bbb N}\pi_J[V_n]$, we have equality.\ \Qed

It follows that $V\in\Cal K$.   \Prf\ If $J\subseteq I$ and
$n\in\Bbb N$, then $V_n$ belongs to the  algebra of
subsets of $X$ generated by sets of the form $\{x:x(i)\in H\}$ where
$i\in I$ and $H\in\Sigma_i$, which we can identify with the free product
$\bigotimes_{i\in I}\Sigma_i$ (315Ma\formerly{3{}15L}).   
This means that $V_n$ can be
expressed as a finite union of cylinder sets of the form
$C=\prod_{i\in I}H_i$ where $H_i\in\Sigma_i$ for every $i$ and
$\{i:H_i\ne X_i\}$ is finite (315Kb\formerly{3{}15J}).   
But in this case $\pi_J[C]$ is
either empty or $\prod_{i\in J}H_i$, and in either case belongs to
$\Tensorhat_{i\in J}\Sigma_i$.   So $\pi_J[V_n]$, being a finite union
of such sets, also belongs to $\Tensorhat_{i\in J}\Sigma_i$.   As this
is true for every $n\in\Bbb N$,
$\pi_J[V]=\bigcap_{n\in\Bbb N}\pi_J[V_n]$ belongs to
$\Tensorhat_{i\in J}\Sigma_i$.  As $J$ is arbitrary, $V\in\Cal K$.\ \Qed

\medskip

\quad{\bf (iii)} Observe next that $\Cal V$ is closed under finite unions.
\Prf\ If $V'$, $V''\in\Cal V$, express them as

\Centerline{$V'
=\bigcap_{n\in\Bbb N}\bigcup_{i\in J'_n}\{x:x(i)\in K'_{ni}\}$}

\Centerline{$V''
=\bigcap_{n\in\Bbb N}\bigcup_{i\in J''_n}\{x:x(i)\in K''_{ni}\}$}

\noindent where, for each $n$, $J'_n$, $J''_n\subseteq I$ are finite,
$K'_{ni}\in\Cal K_{ni}\cap\Sigma_i$ for $i\in J'_n$ and
$K''_{ni}\in\Cal K_{ni}\cap\Sigma_i$ for $i\in J''_n$.
For $m$, $n\in\Bbb N$, set $J_{mn}=J'_m\cup J''_n$ and

$$\eqalign{K_{mni}
&=K'_{mi}\cup K''_{ni}\text{ if }i\in J'_m\cap J''_n,\cr
&=K'_{mi}\text{ if }i\in J'_m\setminus J''_n,\cr
&=K''_{ni}\text{ if }i\in J''_n\setminus J''_m.\cr}$$

\noindent Then

\Centerline{$V'\cap V''
=\bigcap_{m,n\in\Bbb N}\bigcup_{i\in J_{mn}}\{x:x(i)\in K_{mni}\}
\in\Cal V$.  \Qed}

\noindent We see also, immediately from its definition, that $\Cal V$ is
closed under countable intersections.

\medskip

\quad{\bf (iv)} Now consider the family $\Cal A$ of sets of the form
$\{x:x(i)\in E\}$ where $i\in I$ and $E\in\Tau_i$.   If
$A\in\Cal A$ is expressed in this form, then

\Centerline{$\sup\{\lambda V:V\in\Cal V$, $V\subseteq A\}
\ge\sup\{\lambda_iK:K\in\Cal K_i\cap\Tau_i$, $K\subseteq E\}
=\lambda_iE=\lambda A$.}

\noindent By 412C, $\lambda\restr\Tensorhat_{i\in I}\Tau_i$ is inner
regular with respect to $\Cal V$.   In particular, returning to our
original set $W$,

\Centerline{$\mu W=\sup\{\lambda V:V\in\Cal V$, $V\subseteq W\}
=\sup\{\lambda K:K\in\Cal K$, $K\subseteq W\}$.}

\noindent As $W$ is arbitrary, $\lambda$ is inner regular with respect to
$\Cal K$.

\medskip

{\bf (b)(i)} Write $\lambda_J=\lambda\pi_J^{-1}$ and $\hat\lambda_J$
for its completion.
Since $\pi_J:X\to\prod_{i\in J}X_J$ is \imp\ for $\lambda$ and $\lambda_J$,
it is \imp\ for $\hat\lambda$ and $\hat\lambda_J$
(234Ba\formerly{2{}35Hc}), that is,
$\hat\lambda\pi_J^{-1}$ extends $\hat\lambda_J$.   Now suppose that
$V$ is measured by $\hat\lambda\pi_J^{-1}$.   Since $\lambda$ is inner
regular with respect to $\Cal K$, so is $\hat\lambda$ (412Ha again), so

$$\eqalign{\hat\lambda\pi_J^{-1}[V]
&=\sup\{\lambda K:K\in\Cal K,\,K\subseteq\pi_J^{-1}[V]\}\cr
&\le\sup\{\lambda\pi_J^{-1}[\pi_J[K]]:K\in\Cal K,\,
  K\subseteq\pi_J^{-1}[V]\}\cr
&\le\sup\{\lambda\pi_J^{-1}[F]:F\in\Tensorhat_{i\in J}\Sigma_i,\,
   F\subseteq V\}.\cr}$$

\noindent As $V$ is arbitrary,
$\hat\lambda\pi_J^{-1}$ is inner regular with respect to
$\Tensorhat_{i\in J}\Sigma_i$.   By 412L (or otherwise),
$\hat\lambda\pi_J^{-1}=\hat\lambda_J$.

\medskip

\quad{\bf (ii)} Because $\pi_J^{-1}[\pi_J[W]]=W$, $\pi_J[W]$ is measured by
$\hat\lambda\pi_J^{-1}=\hat\lambda_J$.   So there is a
$V'\subseteq\pi_J[W]$, measured by $\lambda_J$, such that

\Centerline{$0=\hat\lambda_J(\pi_J[W]\setminus V')
=\hat\lambda(W\setminus\pi_J^{-1}[V'])$,}

\noindent and we can take $V=\pi_J^{-1}[V']$.
}%end of proof of 451K

\leader{*451L}{}\cmmnt{ The next result is sometimes useful, as a
fractionally weaker sufficient condition for compactness or countable
compactness of a measure.

\medskip

\noindent}{\bf Proposition}\cmmnt{ 
({\smc Borodulin-Nadzieja \& Plebanek 05})} 
Let $(X,\Sigma,\mu)$ be a strictly localizable
measure space.   Let us say
that a family $\Cal E\subseteq\Sigma$ is {\bf $\mu$-centered} if
$\mu(\bigcap\Cal E_0)>0$ for every non-empty finite
$\Cal E_0\subseteq\Cal E$.

(i) Suppose that $\mu$ is inner regular with respect to some
$\Cal K\subseteq\Sigma$ such that every $\mu$-centered subset of $\Cal K$
has non-empty intersection.   Then $\mu$ is compact.

(ii) Suppose that $\mu$ is inner regular with respect to some
$\Cal K\subseteq\Sigma$ such that every
countable $\mu$-centered subset of $\Cal K$
has non-empty intersection.   Then $\mu$ is countably compact.

\proof{ I take the two arguments together, as follows.   The case
$\mu X=0$ is trivial;  suppose henceforth that $\mu X>0$.
Let $\hat\mu$ be
the completion of $\mu$.   Then $\hat\mu$ is still strictly localizable
(212Gb) so has a lifting $\phi:\Sigma\to\Sigma$ (341K).   Let $\Cal K_1$ be
the set of all those $K\in\Sigma$ for which there is some sequence
$\sequencen{K_n}$ in $\Cal K$ such that

\Centerline{$K=\bigcap_{n\in\Bbb N}K_n
\subseteq\bigcap_{n\in\Bbb N}\phi K_n$.}

\noindent Then $\mu$ is inner regular with respect to $\Cal K_1$.   \Prf\
Suppose that $E\in\Sigma$ and $0\le\gamma<\mu E$.   Because $\mu$ is
semi-finite, there is an $F\in\Sigma$ such that $F\subseteq E$ and
$\gamma<\mu F<\infty$.   Choose $\sequencen{K_n}$ in $\Cal K$ inductively,
as follows.   $K_0$ is to be such that $K_0\subseteq F$ and
$\mu K_0>\gamma$.   Given that $\mu K_n>\gamma$, then
$\hat\mu(K_n\cap\phi K_n)=\hat\mu K_n>\gamma$;  also $\hat\mu$ is inner
regular with respect to $\Cal K$ (412Ha once more),
so there is a $K_{n+1}\in\Cal K$
such that $K_{n+1}\subseteq K_n\cap\phi K_n$ and
$\mu K_{n+1}=\hat\mu K_{n+1}>\gamma$.   Continue.   At the end of the
induction,

\Centerline{$K=\bigcap_{n\in\Bbb N}K_n
\subseteq\bigcap_{n\in\Bbb N}\phi K_n$}

\noindent belongs to $\Cal K_1$, is included in $E$ and has measure at
least $\gamma$.\ \Qed

Now $\Cal K_1$ is (countably) compact.   \Prf\ Let
$\Cal K'\subseteq\Cal K_1$ be a [countable] set with the finite
intersection
property.   For each $K\in\Cal K'$, let $\Cal E_K\subseteq\Cal K$ be a
countable set such that
$K=\bigcap\Cal E_K\subseteq\bigcap\{\phi E:E\in\Cal E_K\}$;  set
$\Cal E=\bigcup_{K\in\Cal K'}\Cal E_K$.   If $\Cal E_0\subseteq\Cal E$ is
finite and not empty, then
$\phi(\bigcap\Cal E_0)=\bigcap_{E\in\Cal E_0}\phi E$ includes the
intersection of a finite subfamily of $\Cal K'$, so is not empty, and
$\mu(\bigcap\Cal E_0)=\hat\mu(\bigcap\Cal E_0)$ is non-zero.   Thus
$\Cal E\subseteq\Cal K$ is a [countable] $\mu$-centered set and must have
non-empty intersection.   But now $\bigcap\Cal K'=\bigcap\Cal E$ is
non-empty.   As $\Cal K'$ is arbitrary, $\Cal K_1$ is (countably) compact.\
\Qed

So $\Cal K_1$ witnesses that $\mu$ is (countably) compact, as claimed.
}%end of proof of 451L

\leader{451M}{}\cmmnt{ The following is one of the basic ways in which
we can find ourselves with a compact measure.

\medskip

\noindent}{\bf Proposition} Let $(X,\Sigma)$ be a standard Borel space.
Then any semi-finite measure $\mu$ with domain $\Sigma$ is compact,
therefore perfect.

\proof{ If $\frak T$ is a Polish topology on $X$ with respect to which
$\Sigma$ is the Borel $\sigma$-algebra, then $\mu$ is inner regular with
respect to the family $\Cal K$ of $\frak T$-compact sets (433Ca), which
is a compact class.
}%end of proof of 451M

\vleader{48pt}{451N}{Proposition} Let $(X,\Sigma,\mu)$ be a perfect
measure
space and $\frak T$ a T$_0$ topology on $X$ with a countable network
consisting of measurable sets.
\cmmnt{(For instance, $\mu$ might be a topological measure on a
regular space with a countable network (4A2Ng), or a second-countable
space.   In particular, $X$ might be a separable metrizable space.)}
Then $\mu$ is inner regular with respect to the compact sets.

\proof{ This is a refinement of 343K.
Let $\sequencen{E_n}$ be a sequence in $\Sigma$ running over a
network for $\frak T$.   Define $g:X\to\Bbb R$ by setting
$g=\sum_{n=0}^{\infty}3^{-n}\chi H_n$ (cf.\ 343E).   Then $g$ is
measurable, because every $\chi E_n$ is.
Writing $\alpha_I=\sum_{i\in I}3^{-i}$
for $I\subseteq\Bbb N$, and

\Centerline{$H_n=\bigcup_{I\subseteq n}
  \ooint{\alpha_I+\bover123^{-n},\alpha_I+3^{-n+1}}$,}

\noindent we see that $E_n=g^{-1}[H_n]$ for each $n\in\Bbb N$.
This shows that $g$ is injective, because if $x$, $y$ are distinct
points in $X$ there is an open set containing one but not the other, and
now there is an $n\in\Bbb N$ such that $E_n$ contains that one and not
the other, so that just one of $g(x)$, $g(y)$ belongs to $H_n$.
Also $g^{-1}:g[X]\to X$ is continuous, since
$(g^{-1})^{-1}[E_n]=g[E_n]=H_n\cap g[X]$ is relatively open in $g[X]$
for every $n\in\Bbb N$ (4A2B(a-ii)).

Now suppose that $E\in\Sigma$ and $\mu E>0$.   Then there is a compact
set $K\subseteq g[E]$ such that $\mu g^{-1}[K]>0$.   But as $g$ is
injective, $g^{-1}[K]\subseteq E$, and as $g^{-1}$ is continuous,
$g^{-1}[K]$ is compact.   By 412B, this is enough to show that $\mu$ is
inner regular with respect to the compact sets.
}%end of proof of 451N

\leader{451O}{Corollary}\dvAformerly{4{}51N} Let
$(X,\Sigma,\mu)$ be a complete perfect
measure space, $Y$ a Hausdorff space with a countable network
consisting of Borel sets and $f:X\to Y$ a measurable function.   If
the image measure $\mu f^{-1}$ is locally finite, it is a Radon
measure.

\proof{ Because $f$ is measurable, $\mu f^{-1}$ is a topological
measure;  by 451Ea, it is perfect;  by 451N, it is tight;  and it is
complete because $\mu$ is.
Because $Y$ has a countable network, it is Lindel\"of (4A2Nb), and
$\mu f^{-1}$ is $\sigma$-finite (411Ge), therefore locally determined.
So it is a Radon measure.
}%end of proof of 451O

\leader{451P}{Corollary} Let $(X,\Sigma,\mu)$ be a perfect measure
space, $Y$ a separable metrizable space, and $f:X\to Y$ a measurable
function.

(a) If $E\in\Sigma$ and $\gamma<\mu E$, there is a compact set
$K\subseteq f[E]$ such that $\mu(E\cap f^{-1}[K])\ge\gamma$.

(b) If $\nu=\mu f^{-1}$ is the image measure, then
$\mu_*f^{-1}[B]=\nu_*B$ for every $B\subseteq Y$.

(c) If moreover $\mu$ is $\sigma$-finite, then $\mu^*f^{-1}[B]=\nu^*B$
for every $B\subseteq Y$.

\proof{{\bf (a)} Consider the subspace measure $\mu_E$, the measurable
function
$f\restr E$ from $E$ to the separable metrizable space $f[E]$, and the
image measure $\nuprime=\mu_E(f\restr E)^{-1}$ on $f[E]$.   By
451Dc, 451Ea and 451N, this is tight, while $\nuprime f[E]=\mu E$;
so there is a compact set $K\subseteq f[E]$ such that
$\nuprime K\ge\gamma$, and this serves.

\medskip

{\bf (b)}(i) If $F\in\dom\nu$ and $F\subseteq B$, then

\Centerline{$\nu F=\mu f^{-1}[F]\le\mu_*f^{-1}[B]$;}

\noindent as $F$ is arbitrary, $\mu_*f^{-1}[B]\ge\nu_*B$.  (ii) If
$E\in\Sigma$ and $E\subseteq f^{-1}[B]$ and $\gamma<\mu E$, then (a)
tells us that there is a compact set $K\subseteq f[E]$ such that
$\mu(E\cap f^{-1}[K])\ge\gamma$, in which case

\Centerline{$\nu_*B\ge\nu K\ge\gamma$.}

\noindent As $E$ and $\gamma$ are arbitrary, $\nu_*B\ge\mu_*f^{-1}[B]$.

\medskip

{\bf (c)}(i) If $F\in\dom\nu$ and $F\supseteq B$, then

\Centerline{$\nu F=\mu f^{-1}[F]\ge\mu^*f^{-1}[B]$;}

\noindent as $F$ is arbitrary, $\mu^*f^{-1}[B]\le\nu^*B$.  (ii) If
$\mu^*f^{-1}[B]=\infty$, then of course $\mu^*f^{-1}[B]=\nu^*B$.
Otherwise, because $\mu$ is $\sigma$-finite, we can find a disjoint
sequence $\sequencen{E_n}$ of subsets of $X$ of finite measure, covering
$X$, such that $E_0\supseteq f^{-1}[B]$ and $\mu E_0=\mu^*f^{-1}[B]$.
Let $\epsilon>0$.   For each $n\ge 1$, (a) tells us that there is a
compact set $K_n\subseteq f[E_n]$ such that
$\mu f^{-1}[E_n\setminus K_n]\le 2^{-n}\epsilon$.   Set
$H=Y\setminus\bigcup_{n\ge 1}K_n$;  then $\nu H\le\mu E+\epsilon$, and
$B\subseteq H$.   So

\Centerline{$\nu^*B\le\nu H\le\mu E+\epsilon=\mu^*f^{-1}[B]+\epsilon$.}

\noindent As $\epsilon$ is arbitrary, $\nu^*B\le\mu^*f^{-1}[B]$.
}%end of proof of 451P

\leader{451Q}{}\cmmnt{ I turn now to a remarkable extension of the
idea above to general metric spaces $Y$.

\medskip

\noindent}{\bf Lemma} Let $(X,\Sigma,\mu)$ be a semi-finite compact
measure space, and $\familyiI{E_i}$ a disjoint family of subsets of $X$
such that $\bigcup_{i\in J}E_i\in\Sigma$ for every $J\subseteq I$.
Then $\mu(\bigcup_{i\in I}E_i)=\sum_{i\in I}\mu E_i$.

\proof{{\bf (a)} To begin with (down to the end of part (d) of the
proof) assume that $\mu$ is complete and totally finite and that every $E_i$ is
negligible.
Set $X_0=\bigcup_{i\in I}E_i$, and let $\mu_0$ be the
subspace measure on $X_0$.   Define $f:X_0\to I$
by setting $f(x)=i$ if $i\in I$, $x\in E_i$, and let $\nu$ be the image
measure $\mu_0f^{-1}$, so that $\nu J=\mu(\bigcup_{i\in J}E_i)$ for
$J\subseteq I$;
then $(I,\Cal PI,\nu)$ is a totally finite measure space.

\medskip

{\bf (b)} $\nu$ is purely atomic.   \Prf\Quer\ Suppose, if possible,
otherwise;  that there is a $K\subseteq I$ such that $\nu K>0$ and the
subspace measure $\nu\restrp\Cal PK$ is atomless.   In this case there
is an \imp\ function $g:K\to[0,\gamma]$, where
$\gamma=\nu K$ and $[0,\gamma]$ is given
Lebesgue measure (343Cc);  write $\lambda$ for Lebesgue measure on
$[0,\gamma]$.   Set $X_1=f^{-1}[K]=\bigcup_{i\in K}E_i$ and let $\mu_1$
be the subspace measure on $X_1$.   Now $gf:X_1\to[0,\gamma]$ is \imp\
for $\mu_1$ and $\lambda$.
Because $\mu$ is compact, so is $\mu_1$ (451Da), so $\mu_1$ is perfect
(342L or 451C once more).
By 451O, the image measure $\lambda_1=\mu_1(gf)^{-1}$
is a Radon measure.   But $\lambda_1$ must be an extension of Lebesgue
measure $\lambda$, because $gf$ is \imp\ for $\mu_1$ and $\lambda$, and
$\lambda_1$ and $\lambda$ must agree on all compact sets.   By
416E(b-ii), $\lambda_1$ and
$\lambda$ are identical, and, in particular, have the same domains.
Now for any set $A\subseteq [0,\gamma]$,
$(gf)^{-1}[A]=\bigcup_{i\in J}E_i\in\Sigma$, where
$J=g^{-1}[A]\subseteq I$;  so
$A\in\dom\lambda_1=\dom\lambda$.   But we know from 134D or 419I that
not every subset of $[0,\gamma]$ can be Lebesgue measurable.\ \Bang\Qed

\medskip

{\bf (c)} But $\nu$ is also atomless.  \Prf\Quer\ Suppose, if possible,
that $M\subseteq I$ is an atom
for $\nu$.   Set $\gamma=\nu M=\mu(\bigcup_{i\in M}E_i)$,

\Centerline{$\Cal F=\{F:F\subseteq M,\,\nu(M\setminus F)=0\}$.}

\noindent Because $\nu F$ is defined for every $F\subseteq M$, and $M$
is an atom, $\Cal F$ is an ultrafilter on $M$;  and because $\nu$ is
countably additive, the intersection of any sequence in $\Cal F$ belongs
to $\Cal F$, that is, $\Cal F$ is $\omega_1$-complete (definition:
4A1Ib).   Also $\Cal F$ must be non-principal, because we are supposing
that $\nu\{i\}=0$ for every $i\in M$.   By 4A1K, there are a regular
uncountable cardinal $\kappa$ and a
function $h:M\to\kappa$ such that the image filter $\Cal H=h[[\Cal F]]$
is normal.

For each $\xi<\kappa$, $\kappa\setminus\xi\in\Cal H$, so

\Centerline{$G_{\xi}=(hf)^{-1}[\kappa\setminus\xi]
=\bigcup\{E_i:h(i)\ge\xi\}\in\Sigma$,
\quad$\mu G_{\xi}=\nu h^{-1}[\kappa\setminus\xi]=\gamma>0$.}

\noindent At this point I apply the full strength of the hypothesis that
$\mu$ is a compact measure.   Let $\Cal K\subseteq\Sigma$ be a compact
class such that $\mu$ is inner regular with respect to $\Cal K$, and for
each $\xi<\kappa$ choose $K_{\xi}\in\Cal K$ such that
$K_{\xi}\subseteq
G_{\xi}$ and $\mu K_{\xi}\ge\bover12\gamma$.   Let
$S\subseteq[\kappa]^{<\omega}$ be the family of those finite sets
$L\subseteq\kappa$ such that $\bigcap_{\xi\in L}K_{\xi}=\emptyset$.
Because $\Cal H$ is a normal
ultrafilter, there is an $H\in\Cal H$ such that,
for every $n\in\Bbb N$, $[H]^n$ is either a subset of $S$ or disjoint
from $S$ (4A1L).

If we look at $\{G_{\xi}:\xi\in H\}$, we see that it has empty
intersection,
because $h(f(x))\ge\xi$ for every $x\in G_{\xi}$, and $\sup H=\kappa$.
So $\bigcap_{\xi\in H}K_{\xi}=\emptyset$.   Because all the $K_{\xi}$
belong to
the compact class $\Cal K$, there must be a finite set $L_0\subseteq H$
such that $\bigcap_{\xi\in L_0}K_{\xi}=\emptyset$, that is, $L_0\in S$.
But this means that $[H]^n\cap S\ne\emptyset$, where $n=\#(L_0)$, so
that $[H]^n\subseteq S$, by the choice of $H$.
However, $H$ is surely infinite, so we can find distinct
$\xi_0,\ldots,\xi_{2n}$ in $H$.   If we now look at
$K_{\xi_0},\ldots,K_{\xi_{2n}}$, we see that
$\#(\{i:i\le 2n,\,x\in K_{\xi_i}\})<n$ for every $x\in X$, so

\Centerline{$\sum_{i=0}^{2n}\chi K_{\xi_i}\le(n-1)\chi G_0$,
\quad$\sum_{i=0}^{2n}\int\chi K_{\xi_i}\ge\Bover12\gamma(2n+1)$,}

\noindent which is impossible, because $\mu G_0=\gamma$.\ \Bang\Qed

\medskip

{\bf (d)} Thus $\nu$ is simultaneously atomless and purely atomic, which
means that $\nu I=0$, that is, that
$\mu(\bigcup_{i\in I}E_i)=0=\sum_{i\in I}\mu E_i$.

\medskip

{\bf (e)} Now let us return to the general case.   Of course

\Centerline{$\sum_{i\in I}\mu E_i
=\sup_{J\subseteq I\text{ is finite}}\sum_{i\in J}\mu E_i
\le\mu(\bigcup_{i\in I}E_i)$.}

\noindent\Quer\ Suppose, if possible, that
$\sum_{i\in I}\mu E_i<\mu(\bigcup_{i\in I}E_i)$.
Because $\mu$ is semi-finite, there is
a set $F\subseteq\bigcup_{i\in I}E_i$ such that
$\sum_{i\in I}\mu E_i<\mu F<\infty$.   Set $L=\{i:i\in I,\,\mu E_i>0\}$;
then $L$ must be
countable, so $\mu(\bigcup_{i\in J}E_i)=\sum_{i\in J}\mu E_i<\mu F$, and
$\mu G>0$, where $G=F\setminus\bigcup_{i\in L}E_i$.   Set
$E'_i=G\cap E_i$ for every $i\in I$, and let $\hat\mu_G$ be the
completion of the subspace measure $\mu_G$ on $G$.
Then $\hat\mu_G$ is compact (451Da, 451G(a-i)) and totally finite,
$\hat\mu_GE'_i=0$ for
every $i\in I$, $\bigcup_{i\in J}E'_i=G\cap\bigcup_{i\in J}E_i$ is
measured by
$\hat\mu_G$ for every $J\subseteq I$, every $E'_i$ is
$\hat\mu_G$-negligible,
but $\hat\mu_G(\bigcup_{i\in I}E'_i)=\mu G$ is not zero;
which contradicts the result of (a)-(d) above.\ \Bang

So $\sum_{i\in I}\mu E_i=\mu(\bigcup_{i\in I}E_i)$, as required.
}%end of proof of 451Q

\leader{451R}{Lemma} Let $(X,\Sigma,\mu)$ be a totally finite compact
measure space, $Y$ a metrizable space, and $f:X\to Y$ a measurable
function.  Then there is a closed separable subspace $Y_0$ of $Y$ such
that $f^{-1}[Y\setminus Y_0]$ is negligible.

\proof{{\bf (a)} (Cf.\ 438D.) By 4A2L(g-ii), there is a
$\sigma$-disjoint base $\Cal U$ for the topology of $Y$.   Express
$\Cal U$ as
$\bigcup_{n\in\Bbb N}\Cal U_n$ where $\Cal U_n$ is disjoint for each
$n$.   Then $\family{U}{\Cal U_n}{f^{-1}[U]}$ is disjoint, so
$\sum_{U\in\Cal U_n}\mu f^{-1}[U]\le\mu X$ is finite, and
$\Cal V_n=\{V:V\in\Cal U_n,\,\mu f^{-1}[V]>0\}$ is
countable for each $n$.

If $\Cal W\subseteq\Cal U_n\setminus\Cal V_n$, then

\Centerline{$\mu(\bigcup_{U\in\Cal W}f^{-1}[U])=f^{-1}[\bigcup\Cal W]$}

\noindent is measurable.   By 451Q,

\Centerline{$\mu f^{-1}[\bigcup(\Cal U_n\setminus\Cal V_n)]
=\mu(\bigcup_{U\in\Cal U_n\setminus\Cal V_n}f^{-1}[U])
=\sum_{U\in\Cal U_n\setminus\Cal V_n}\mu f^{-1}[U]=0$.}

Set

\Centerline{$\Cal V=\bigcup_{n\in\Bbb N}\Cal V_n$,
\quad$Y_0=Y\setminus\bigcup(\Cal U\setminus\Cal V)$.}

\noindent Then $Y_0$ is closed, and

\Centerline{$f^{-1}[Y\setminus Y_0]\subseteq
\bigcup_{n\in\Bbb N}f^{-1}[\bigcup(\Cal U_n\setminus\Cal V_n)]$}

\noindent is negligible, so $f^{-1}[Y_0]$ is conegligible.   On the
other hand, $Y_0$ is
separable.   \Prf\ Because $\Cal U$ is a base for the topology of $X$,
$\{Y\cap U:U\in\Cal U\}$ is a base for the topology of $Y$ (4A2B(a-vi)).
But this is included in the countable family
$\{Y\cap V:V\in\Cal V\}\cup\{\emptyset\}$, so $Y$ is second-countable,
therefore separable (4A2Oc).\ \Qed

So we have found an appropriate $Y_0$.
}%end of proof of 451R

\leader{451S}{Proposition} Let $(X,\Sigma,\mu)$ be a semi-finite compact
measure space, $Y$ a metrizable space and $f:X\to Y$ a measurable
function.

(a) The image measure $\nu=\mu f^{-1}$ is tight.

(b) If $\nu$ is locally finite and $\mu$ is complete and locally
determined, $\nu$ is a Radon measure.

\proof{{\bf (a)} Take $F\subseteq Y$ such that $\nu F>0$.   Then
$\mu f^{-1}[F]>0$.   Because $\mu$ is semi-finite, there is an
$E\in\Sigma$
such that $E\subseteq f^{-1}[F]$ and $0<\mu E<\infty$.

Consider the subspace measure $\mu_E$ and the restriction $f\restr E$.
$\mu_E$ is a totally finite compact measure and $f\restr E$ is
measurable,
so 451R tells us that there is a closed separable subspace
$Y_0\subseteq Y$ such that $\mu(E\setminus f^{-1}[Y_0])=0$.   Set
$E_1=E\cap f^{-1}[Y_0]$, so that $\mu E_1>0$.   Again, the subspace
measure $\mu_{E_1}$ is a totally finite compact measure, therefore
perfect, while
$f[E_1]\subseteq Y_0$.   So the image measure
$\mu_{E_1}(f\restr E_1)^{-1}$ on $Y_0$ is perfect (451Ea), therefore
tight (451N), and there is
a compact set $K\subseteq Y_0\cap F$ such that $\nu K=\mu f^{-1}[K]>0$.
By 412B, this is enough to show that $\nu$ is tight.

\medskip

{\bf (b)} $\nu$ is complete because $\mu$ is.   Now
suppose that $H\subseteq Y$ is such that $H\cap F$ belongs to the domain
$\Tau$ of $\nu$ whenever $\mu F<\infty$.   In this case $\mu$ is inner
regular with respect to
$\Cal E=\{E:E\in\Sigma,\,E\cap f^{-1}[H]\in\Sigma\}$.   \Prf\ Suppose
that $E\in\Sigma$ and that $\mu E>0$.   Applying (a) to $\mu_E$ and
$f\restr E$,
there is a compact set $K\subseteq f[E]$ such that $\mu f^{-1}[K]>0$.
Now $\nu K<\infty$, because $\nu$ is locally finite, so $K\cap H\in\Tau$
and $f^{-1}[K]\cap f^{-1}[H]\in\Sigma$.   Thus $f^{-1}[K]$ is a
non-negligible
member of $\Cal E$ included in $E$.   Since $\Cal E$ is closed under
finite unions, this is enough to show that $\mu$ is inner regular with
respect to $\Cal E$.\ \Qed

Accordingly $f^{-1}[H]\in\Sigma$, by 412Ja.   As $H$ is
arbitrary, $\nu$ is locally determined, therefore a Radon measure.
}%end of proof of 451S

\leader{451T}{Theorem}\cmmnt{ ({\smc Fremlin 81},
{\smc Koumoullis \& Prikry 83})} Let $(X,\frak T,\Sigma,\mu)$ be a
Radon measure space and $Y$ a metrizable space.
Then a function $f:X\to Y$ is measurable iff it is almost continuous.

\proof{ If $f$ is almost continuous it is surely measurable, by 418E.
Now suppose that $f$ is measurable and that $E\in\Sigma$ and
$\gamma<\mu E$.  Let $E_0\subseteq E$ be such that $E_0\in\Sigma$ and
$\gamma<\mu E_0<\infty$.
Applying 451R to the subspace measure $\mu_{E_0}$ and the restricted
function $f\restr E_0$, we see that there is a closed separable subspace
$Y_0$ of $Y$ such that $\mu(E_0\setminus f^{-1}[Y_0])=0$.   Set
$E_1=E_0\cap f^{-1}[Y_0]$;  then $\mu E_1>\gamma$.   Applying 418J to
$\mu_{E_1}$ and
$f\restr E_1:E_1\to Y_0$, we can find a measurable set $F\subseteq E_1$
such that $f\restr F$ is continuous and $\mu F\ge\gamma$.   As $E$ and
$\gamma$ are arbitrary, $f$ is almost continuous.
}%end of proof of 451T

\leader{451U}{Example}\cmmnt{ ({\smc Vinokurov \& Makhkamov 73},
{\smc Musia{\l} 76})} There is a perfect
completion regular quasi-Radon probability space which is not countably
compact.

\proof{{\bf (a)} Let $\Omega$ be the set of non-zero countable limit
ordinals.   For each $\xi\in\Omega$, let $\sequencen{\theta_{\xi}(n)}$
be a strictly increasing sequence in $\xi$ with supremum $\xi$, and set

\Centerline{$Q_{\xi}=\{x:x\in\{0,1\}^{\omega_1},\,x(\theta_{\xi}(n))=0$
for every $n\in\Bbb N\}$.}

\noindent Write

\Centerline{$X=\{0,1\}^{\omega_1}\setminus\bigcup_{\xi\in\Omega}Q_{\xi}$.}

\noindent Let $\nu_{\omega_1}$ be the usual measure on
$\{0,1\}^{\omega_1}$, and
$\Tau_{\omega_1}$ its domain;  let $\mu$ be the subspace measure on $X$, and
$\Sigma=\dom\mu$.

\medskip

{\bf (b)} It is convenient to note immediately the following fact:  for
every countable set $J\subseteq\omega_1$, the set $\pi_J[X]$ is
conegligible in $\{0,1\}^J$, where $\pi_J(x)=x\restr J$ for
$x\in\{0,1\}^{\omega_1}$.   \Prf\ Set

\Centerline{$A=\{\xi:\xi\in\Omega,\,\theta_{\xi}(n)\in J$ for every
$n\in\Bbb N\}$.}

\noindent Then $A$ is countable, because $\xi\le\sup J$ for every
$\xi\in A$.   So

\Centerline{$D=\bigcup_{\xi\in
A}\{y:y\in\{0,1\}^J,\,y(\theta_{\xi}(n))=0$
for every $n\in\Bbb N\}$}

\noindent is negligible in $\{0,1\}^J$, being a countable union of
negligible sets.   If $y\in\{0,1\}^J\setminus D$, define
$x\in\{0,1\}^{\omega_1}$ by setting $x(\eta)=y(\eta)$ for $\eta\in J$,
$x(\eta)=1$ for $\eta\in\omega_1\setminus J$.   Then $x\notin Q_{\xi}$
for any $\xi\in A$, because $x\restr J=y\restr J$, while $x\notin Q_{\xi}$
for any $\xi\in\Omega\setminus A$ by the definition of $A$.
So $x\in X$.   As $y$ is arbitrary,
$\pi_J[X]\supseteq\{0,1\}^J\setminus D$ is conegligible.\ \Qed

\medskip

{\bf (c)} $\mu$ is a completion regular quasi-Radon measure
because $\nu_{\omega_1}$ is (415E, 415B, 412Pd).   Also $\mu X=1$.
\Prf\  Let $F\in\Tau_{\omega_1}$ be a measurable envelope for $X$.
Then there is a countable
$J\subseteq\omega_1$ such that $\nu_J\pi_J[F]$ is defined and equal to
$\nu_{\omega_1} F$ (254Od), where $\nu_J$ is the usual measure on
$\{0,1\}^J$.   But we know that $\nu_J\pi_J[X]=1$, so

\Centerline{$\mu X=\nu_{\omega_1}^*X=\nu_{\omega_1}F=\nu_J\pi_JF=1$. \Qed}

\medskip

{\bf (d)} $\mu$ is perfect.   \Prf\ Take $E\in\Sigma$ such that
$\mu E>0$, and a measurable function $f:E\to\Bbb R$.   Set
$f_1(x)=\Bover{f(x)}{1+|f(x)|}$ for $x\in E$, $1$ for
$x\in X\setminus E$;  then $f_1:X\to\Bbb R$ is measurable.   Let
$g:\{0,1\}^{\omega_1}\to\Bbb R$ be a measurable function extending
$f_1$.  By 254Pb, there are a countable set $J\subseteq\omega_1$, a
conegligible set
$W\subseteq\{0,1\}^J$, and a measurable $h:W\to\Bbb R$ such that $g$
extends $h\pi_J$.   By (b), $W'=W\cap\pi_J[X]$ is conegligible, while
$W''=\{z:z\in W'$, $h_1(z)<1\}$ is measurable and not
negligible.   Because $W''$ is a non-negligible measurable subset of the
perfect measure space $\{0,1\}^J$, there is a compact set
$K_1\subseteq h[W'']$ such that $\nu_Jh^{-1}[K_1]>0$.   Set
$K=\{\Bover{t}{1-|t|}:t\in K_1\}$;  then $K$ is compact, and we have

\Centerline{$K_1
\subseteq h[W'']
=h[W\cap\pi_J[X]]\cap\ooint{-\infty,1}
\subseteq g[X]\cap\ooint{-\infty,1}
=f_1[X]\cap\ooint{-\infty,1}
=f_1[E]$,}

\Centerline{$K\subseteq f[E]$,}

\noindent while $f_1$, $g$ and $h\pi_J$ all agree on the
$\mu$-conegligible set $X\cap\pi_J^{-1}[W]$, so

$$\eqalignno{\mu f^{-1}[K]
&=\mu f_1^{-1}[K_1]
=\mu(X\cap(h\pi_J)^{-1}[K_1])\cr
&=\nu_{\omega_1}^*(X\cap(h\pi_J)^{-1}[K_1])
=\nu_{\omega_1}(h\pi_J)^{-1}[K_1]\cr
\displaycause{because $\nu_{\omega_1}^*X=1$ and $(h\pi_J)^{-1}[K_1]$ is
measurable}
&=\nu_Jh^{-1}[K_1]
>0.\cr}$$

\noindent As $f$ is arbitrary, $\mu$ is perfect.\ \Qed

\medskip

{\bf (e)} \Quer\ Suppose, if possible, that $\mu$ is countably compact.
Let $\Cal K$ be a countably compact class of sets such that $\mu$ is
inner regular with respect to $\Cal K$;  we may suppose that
$\Cal K\subseteq\Sigma$.

\medskip

\quad{\bf (i)} For $I\subseteq\omega_1$ set

\Centerline{$U(I)=\{x:x\in X,\,x(\eta)=0$ for every $\eta\in I\}$.}

\noindent It will be helpful to know that if $E\in\Sigma$ and $\mu E>0$,
there is a $\gamma<\omega_1$ such that $\mu(E\cap U(I))>0$ for every
finite $I\subseteq\omega_1\setminus\gamma$.   \Prf\ Express $E$ as
$X\cap F$ where $F\in\Tau_{\omega_1}$.
Let $J\subseteq\omega_1$ be a countable
set such that $\nu_{\omega_1}(F'\setminus F)=0$,
where $F'=\pi_J^{-1}[\pi_J[F]]$
(254Od again), and $\gamma<\omega_1$ such that $J\subseteq\gamma$.   If
$I\subseteq\omega_1\setminus\gamma$ is finite, then $I\cap J=\emptyset$,
while $U(I)$ is determined by coordinates in $I$ and $F'$ is determined
by coordinates in $J$;  so

$$\eqalign{\mu(E\cap U(I))
&=\nu_{\omega_1}^*(X\cap F\cap U(I))
=\nu_{\omega_1}(F\cap U(I))\cr
&=\nu_{\omega_1}(F'\cap U(I))
=\nu_{\omega_1} F'\cdot\nu_{\omega_1} U(I)
=\mu E\cdot\nu_{\omega_1} U(I)
>0.\cr}$$

\noindent Thus this $\gamma$ serves.\ \Qed

\medskip

\quad{\bf (ii)} Let $\Cal M$ be the family of countable subsets $M$ of
$\omega_1\cup\Cal K$ such that

\inset{($\alpha$) if $I\subseteq M\cap\omega_1$ is finite there is a
$K\in M\cap\Cal K$ such that $K\subseteq U(I)$ and $\mu K>0$;}

\inset{($\beta$) if $K\in M\cap\Cal K$, $I\subseteq M\cap\omega_1$ is
finite and $\mu(K\cap U(I))>0$, then there is a $K'\in M\cap\Cal K$ such
that $K'\subseteq K\cap U(I)$ and $\mu K'>0$;}

\inset{($\gamma$) if $\gamma\in M\cap\omega_1$ then
$\gamma\subseteq M$;}

\inset{($\delta$) if $K\in M\cap\Cal K$ and $\mu K>0$ then there is a
$\gamma\in M\cap\omega_1$ such that $\mu(K\cap U(I))>0$ whenever
$I\subseteq\omega_1\setminus\gamma$ is finite.}

\noindent Then every countable $M\subseteq\omega_1\cup\Cal K$ is
included in some member $M'$ of $\Cal M$.

\Prf\ Choose $\sequencen{N_n}$ as follows.   $N_0=M$.   Given
that $N_n$ is a countable subset of $\omega_1\cup\Cal K$ then
let $N_{n+1}\subseteq\omega_1\cup\Cal K$ be a countable set such that

\inset{($\alpha$) if $I\subseteq N_n\cap\omega_1$ is finite there is a
$K\in N_{n+1}\cap\Cal K$ such that $K\subseteq U(I)$ and $\mu K>0$;}

\inset{($\beta$) if $K\in N_n\cap\Cal K$, $I\subseteq N_n\cap\omega_1$
is finite and $\mu(K\cap U(I))>0$, then there is a $K'\in
N_{n+1}\cap\Cal K$ such that $K'\subseteq K\cap U(I)$ and $\mu K'>0$;}

\inset{($\gamma$) if $\gamma\in N_n\cap\omega_1$ then
$\gamma\subseteq N_{n+1}$;}

\inset{($\delta$) if $K\in N_n\cap\Cal K$ and $\mu K>0$ then there is a
$\gamma\in N_{n+1}\cap\omega_1$ such that $\mu(K\cap U(I))>0$ whenever
$I\subseteq\omega_1\setminus\gamma$ is finite;}

\inset{($\epsilon$) $N_n\subseteq N_{n+1}$.}

\noindent On completing the induction, set $M'=\bigcup_{n\in\Bbb N}N_n$;
this serves (because every finite subset of $M'$ is a subset of some
$N_n$).\ \Qed

\medskip

\quad{\bf (iii)} Choose a sequence $\sequencen{M_n}$ in $\Cal M$ such
that, for each $n$, $M_n\cup\{\sup(M_n\cap\omega_1)+1\}\subseteq
M_{n+1}$.   Set $\gamma_n=\sup(M_n\cap\omega_1)$ for each $n$.   Note
that $\gamma_n\subseteq M_n$, because if $\eta<\gamma_n$ then there is
some $\xi\in M_n$ such that $\eta<\xi$;  now $\xi\subseteq M_n$
because $M_n\in\Cal M$, so $\eta\in M_n$.   Also
$\gamma_n+1\in M_{n+1}$ for each $n$, so $\sequencen{\gamma_n}$ is
strictly increasing, and $\xi=\sup_{n\in\Bbb N}\gamma_n$ belongs to
$\Omega$.

Set $J=\{\theta_{\xi}(n):n\in\Bbb N\}$.   Then $J\cap\eta$ is finite for
every $\eta<\xi$, and in particular $J\cap\gamma_n$ is finite for every
$n$.   Set $I_0=J\cap\gamma_0$ and
$I_n=J\cap\gamma_n\setminus\gamma_{n-1}$ for $n\ge 1$.   Then
$\bigcap_{n\in\Bbb N}U(I_n)=Q_{\xi}$ is disjoint from $X$.

Choose a sequence $\sequencen{K_n}$ in $\Cal K$ as follows.   Because
$I_0$ is a finite subset of $M_0\cap\omega_1$, there is a $K_0\in
M_0\cap\Cal K$ such that
$K_0\subseteq U(I_0)$ and $\mu K_0>0$.    Given that $K_n\in M_n\cap\Cal
K$ and $\mu K_n>0$, then there is a $\beta\in M_n\cap\omega_1$ such that
$\mu(K_n\cap U(I))>0$ for every finite
$I\subseteq\omega_1\setminus\beta$;  now $\beta\le\gamma_n$ and
$I_{n+1}\cap\gamma_n=\emptyset$, so $\mu(K_n\cap U(I_{n+1}))>0$.   But
$K_n\in M_{n+1}\cap\Cal K$ and $I_{n+1}$ is a finite subset of
$M_{n+1}\cap\omega_1$, so there is a $K_{n+1}\in M_{n+1}\cap\Cal K$ such
that $K_{n+1}\subseteq K_n\cap U(I_{n+1})$ and $\mu K_{n+1}>0$.
Continue.

In this way we find a non-increasing sequence $\sequencen{K_n}$ in $\Cal
K$ such that $K_n\subseteq U(I_n)$ for every $n$ and no $K_n$ is empty.
But in this case $\bigcap_{i\le n}K_i=K_n$ is non-empty for every $n$,
while $\bigcap_{n\in\Bbb N}K_n\subseteq X\cap\bigcap_{n\in\Bbb N}U(I_n)$
is empty.   So $\Cal K$ is not a countably compact class.\ \Bang

\medskip

{\bf (f)} Thus $\mu$ is not countably compact, and has all the
properties claimed.
}%end of proof of 451U (Musial)

\leader{*451V}{Weakly $\alpha$-favourable spaces}\cmmnt{ There is
an interesting variation on the concept of `countably compact' measure
space, as follows.}   For any measure space $(X,\Sigma,\mu)$ we can
imagine an infinite game for two players, whom I will call `Empty' and
`Nonemepty'.   Empty chooses a
non-negligible measurable set $E_0$;  Nonempty chooses a
non-negligible measurable set $F_0\subseteq E_0$;  Empty chooses a
non-negligible measurable set $E_1\subseteq F_0$;  Nonempty chooses a
non-negligible measurable set $F_1\subseteq E_1$, and so on.   At the
end of the game, Empty wins if
$\bigcap_{n\in\Bbb N}E_n=\bigcap_{n\in\Bbb N}F_n$ is empty;  otherwise
Nonempty wins.   (If $\mu X=0$, so that Empty has no legal initial move,
I declare Nonempty the winner by default.)
\cmmnt{If you have seen `Banach-Mazur' games, you will recognise this
as a similar construction, in which open sets are replaced by
non-negligible measurable sets.}

A {\bf strategy} for Nonempty is a\cmmnt{ rule to determine his
moves in terms of the preceding moves for Empty;  that is, a}
function $\sigma:\bigcup_{n\in\Bbb N}(\Sigma\setminus\Cal
N)^{n+1}\to\Sigma\setminus\Cal N$, where $\Cal N$ is the ideal of
negligible sets, such that\cmmnt{ $\sigma(E_0,E_1,\ldots,E_n)\subseteq
E_n$, at least whenever $E_0,\ldots,E_n\in\Sigma\setminus\Cal N$ are
such that $E_{k+1}\subseteq\sigma(E_0,\ldots,E_k)$ for every $k<n$;
since it never matters what Nonempty does if Empty has already
broken the rules, we usually just demand that}
$\sigma(E_0,\ldots,E_n)\subseteq E_n$ for all
$E_0,\ldots,E_n\in\Sigma\setminus\Cal N$.   $\sigma$ is a {\bf winning
strategy} if $\bigcap_{n\in\Bbb N}E_n\ne\emptyset$ whenever
$\sequencen{E_n}$ is a sequence in $\Sigma\setminus\Cal N$ such that
$E_{n+1}\subseteq\sigma(E_0,\ldots,E_n)$ for every $n\in\Bbb N$.
\cmmnt{In terms of the game, we interpret this as saying that Nonempty
will win if he plays $F_n=\sigma(E_0,\ldots,E_n)$ whenever faced
with the position $(E_0,F_0,E_1,F_1,\ldots,F_{n-1},E_n)$.   (Since it is
supposed that Nonempty will use the same strategy throughout the game,
the moves $F_0,\ldots,F_{n-1}$ are determined by $E_0,\ldots,E_{n-1}$
and there is no advantage in taking them separately into account when
choosing $F_n$.)}

Now we say that the measure space $(X,\Sigma,\mu)$ is {\bf weakly
$\alpha$-favourable} if there is\cmmnt{ such} a winning strategy for
Nonempty.

\cmmnt{It turns out that the class of weakly $\alpha$-favourable
spaces behaves in much the same way as the class of countably compact
spaces.   For the moment, however, I leave the details to the exercises
(451Yh-451Yr).   See {\smc Fremlin 00}.}

\exercises{\leader{451X}{Basic exercises (a)}
%\spheader 451Xa
(i) Show that any purely atomic measure space is perfect.  (ii) Show
that any strictly localizable purely atomic measure space is countably
compact.   (iii) Show that the space of 342N is not countably
compact.
%451B

\sqheader 451Xb Show that a compact measure space in which singleton
sets are negligible is atomless.
%451B

\sqheader 451Xc Let $(X,\Sigma,\mu)$ be a measure space, and $\nu$ an
indefinite-integral measure over $\mu$\cmmnt{ (234J\formerly{2{}34B})}.
Show that
$\nu$ is compact, or countably compact, or perfect if $\mu$ is.
%451B

\spheader 451Xd  In 413Xo, show that $\mu$ is a countably compact
measure.   \Hint{show that the algebra $\Sigma$ there is a countably
compact class.}
%451B

\spheader 451Xe Let $\familyiI{(X_i,\Sigma_i,\mu_i)}$ be a family of
measure spaces, with direct sum $(X,\Sigma,\mu)$.   Show that $\mu$ is
compact, or countably compact, or perfect iff every $\mu_i$ is.
%451B

\spheader 451Xf Let $(X,\Sigma,\mu)$ be a measure space and $\Cal K$ a
family of subsets of $X$ such that whenever $E\in\Sigma$ and $\mu E>0$
there is a $K\in\Cal K$ such that $K\subseteq E$ and $\mu_*K>0$.
(i) Show that if $\Cal K$ is a compact class then $\mu$ is a
compact measure.   (ii) Show that if $\Cal K$ is a countably compact class
then $\mu$ is a countably compact measure.
%451B 413R out of order query

\spheader 451Xg Let $(X,\Sigma,\mu)$ be a measure space.   For
$A\subseteq X$, write $\mu_A$ for the subspace measure on $A$.  Suppose
that whenever $E\in\Sigma$ and $\mu E>0$ there is a set $A\subseteq X$
such that $\mu_A$ is perfect and $\mu^*(A\cap E)>0$.   Show that $\mu$
is perfect.
%451D

\spheader 451Xh(i) Give an example of a compact probability space
$(X,\Sigma,\mu)$ and a $\sigma$-subalgebra $\Tau$ of $\Sigma$ such that
$\mu\restrp\Tau$ is not compact.   (ii) Give an example of a compact
probability space $(X,\Sigma,\mu)$, a set $Y$ and a function $f:X\to Y$
such that the image measure $\mu f^{-1}$ is not compact.   \Hint{342M,
342Xf, 439Xa.}
%451E

\spheader 451Xi Let $\familyiI{X_i}$ be a family of sets, with product
$X$.   Suppose that $\Cal K_i\subseteq\Cal PX_i$ for each $i$, and set
$\Cal K=\{\prod_{i\in I}K_i:K_i\in\Cal K_i$ for each $i\}$.   (i) Show
that if $\Cal K_i$ is a compact class for each $i$, so is $\Cal K$.
(ii) Show that if $\Cal K_i$ is a countably compact class for each $i$,
so is $\Cal K$.
%451H

\spheader 451Xj Let $A\subseteq[0,1]$ be a set with outer
Lebesgue measure $1$ and inner measure $0$.   Show that there is a
Borel measure $\lambda$ on $A\times[0,1]$ such that
$\lambda$ is not inner regular with respect to sets which have Borel
measurable projections on the factor spaces.
%451K

\spheader 451Xk Let $X$ be a Polish space and $E$ a subset of $X$.
Show that the following are equiveridical:  (i) $E$ is universally
measurable;  (ii) every Borel probability measure on $E$ is perfect;
(iii) every $\sigma$-finite Borel measure on $E$ is compact;  (iv)
$f[E]$ is universally measurable in $\Bbb R$ for every Borel measurable
function $f:X\to\Bbb R$.
%451M

\spheader 451Xl In 451N, show that $\mu$ is a compact measure.
%451N can omit

\spheader 451Xm Find a Radon measure space $(X,\frak T,\Sigma,\mu)$,
a continuous function $f:X\to[0,1]$ and a set
$B\subseteq[0,1]$ such that $\mu^*(f^{-1}[B])<(\mu f^{-1})^*B$.
%451P mt45bits

\spheader 451Xn
Let $(X,\Sigma,\mu)$ be a $\sigma$-finite measure space.   Show that it
is perfect iff whenever $f:X\to\Bbb R$ is measurable there is a
K$_{\sigma}$ set $H\subseteq f[X]$ such that $f^{-1}[H]$ is
conegligible.
%451P

\spheader 451Xo Let $X$ be a metrizable space, and $\mu$ a semi-finite
topological measure on $X$ which (regarded as a measure) is compact.
Show that $\mu$ is $\tau$-additive.
%451Q

\spheader 451Xp Let $(X,\Sigma,\mu)$ be a compact strictly localizable
measure space (e.g., any Radon measure space), $(Y,\Tau,\nu)$ a
$\sigma$-finite measure space, and $f:X\to L^0(\nu)$ a function.   Show
that the following are equiveridical:  (i) $f$ is measurable, when
$L^0(\nu)$ is given its topology of convergence in measure;
(ii) there is a function $h\in\eusm L^0(\lambda)$, where $\lambda$ is
the c.l.d.\ product measure on $X\times Y$, such that
$f(x)=h_x^{\ssbullet}$ for
almost every $x\in X$, where $h_x(y)=h(x,y)$.   \Hint{418R.}
%451R

\sqheader 451Xq Let $(X,\frak T,\Sigma,\mu)$ be a Radon measure space.
Show that $\Sigma=\Cal PX$ iff $\mu$ is purely atomic.   \Hint{if
$\Sigma=\Cal PX$, apply 451T with $Y=X$, the discrete topology on $Y$
and the identity function from $X$ to $Y$.}
%451T

\spheader 451Xr Let $(X,\frak T,\Sigma,\mu)$ be a Radon measure space
and $U$ a normed space.   Show that if $f$, $g:X\to U$ are measurable
functions, then $f+g$ is measurable.   (Cf.\ 418Xj.)
%451T

%new 2008
\spheader 451Xs Show that in all three of the constructions of 439A, the
measure $\nu$ is
countably compact.   \Hint{for the `third construction', consider
$\{f^{-1}[F]:F\subseteq\{0,1\}^{\frakc}$ is a zero set$\}$.}
%451Xa 451Xd 451B out of order

\leader{451Y}{Further exercises (a)}
%\spheader 451Ya
Show that for any probability space $(X,\Sigma,\mu)$, there is a compact
probability space $(Y,\Tau,\nu)$ with a subspace isomorphic to
$(X,\Sigma,\mu)$.

\spheader 451Yb
Let $\familyiI{X_i}$ be a family of sets, and $\Sigma_i$ a
$\sigma$-algebra of subsets of $X_i$ for each $i$.   Suppose that for
each finite $J\subseteq I$ we are given a finitely additive functional
$\nu_J$ on $X_J=\prod_{i\in J}X_i$, with domain the algebra
$\Tau_J=\bigotimes_{i\in J}\Sigma_i$
generated by sets of the form $\{x:x\in X_J,\,x(i)\in E\}$ for $i\in J$,
$E\in\Sigma_i$, and that ($\alpha$)
$\nu_K\{x:x\in X_K,\,x\restr J\in W\}=\nu_JW$ whenever
$J\subseteq K\in[I]^{<\omega}$ and $W\in\Tau_J$
($\beta$) $\mu_i=\nu_{\{i\}}$ is a countably compact probability measure
for every $i\in I$.   Show that there is a countably compact measure $\mu$
on $X=X_I$ such that $\mu\{x:x\in X,\,x\restr J\in W\}=\nu_JW$ whenever
$J\in[I]^{<\omega}$ and $W\in\Tau_J$.  \Hint{454D.}  (Compare 418M.)
%451J

\spheader 451Yc Describe $\mu$ in the case of 451Yb in which $I=[0,1]$,
$X_i=[0,1]\setminus\{i\}$, $\Sigma_i$ is the algebra of Lebesgue
measurable subsets of $X_i$, and $\nu_JE=\mu_L\{t:z_{Jt}\in E\}$ for
every $E\in\bigotimes_{i\in J}\Sigma_i$, where $z_{Jt}(i)=t$ for
$i\in J$, $t\in [0,1]$.   Contrast this with the difficulty encountered
in 418Xu.
%451J, 451Yb

\spheader 451Yd Let $(X,\Sigma,\mu)$ be a semi-finite compact measure
space, and $\familyiI{E_i}$ a point-finite family of measurable subsets
of $X$ such that $\bigcup_{i\in J}E_i\in\Sigma$ for every
$J\subseteq I$.   Show that $\mu(\bigcup_{i\in I}E_i)
=\sup_{J\subseteq I\text{ is finite}}\mu(\bigcup_{i\in J}E_i)$.
\Hint{438Ya.}
%451Q

\spheader 451Ye Let $X$ be a
hereditarily metacompact space, and $\mu$ a semi-finite
topological measure on $X$ which (regarded as a measure) is compact.
Show that $\mu$ is $\tau$-additive.
%451Yd 451Xo 451Q

\spheader 451Yf
Let $(X,\Sigma,\mu)$ be a compact measure space, $V$ a Banach space and
$f:X\to V$ a measurable function such that $\|f\|:X\to\coint{0,\infty}$
is integrable.   Show that $f$ is Bochner integrable (253Yf).
%451T

\spheader 451Yg Let $(X,\frak T,\Sigma,\mu)$ be a Radon measure
space.   Suppose that $Y$ is a separable
metrizable space and $Z$ is a metrizable space, and that
$f:X\times Y\to Z$ is a function such that $x\mapsto f(x,y)$ is measurable
for every $y\in Y$ and $y\mapsto f(x,y)$ is continuous for every $x\in X$.
Show that $\mu$ is inner regular
with respect to $\{F:F\subseteq X$, $f\restr F\times Y$ is continuous$\}$.
\Hint{418Yo.}
%451T

\spheader 451Yh Show that any purely atomic measure space is weakly
$\alpha$-favourable, so that the space of 342N is weakly
$\alpha$-favourable but not countably compact.
%451V/451Xa

\spheader 451Yi Show that the direct sum of a family of
weakly $\alpha$-favourable measure spaces is weakly $\alpha$-favourable.
%451V/451Xe

\spheader 451Yj Show that an indefinite-integral measure over a
weakly $\alpha$-favourable measure is weakly $\alpha$-favourable.
%451V/451Xc

\spheader 451Yk(i) Show that a countably compact measure space is
weakly $\alpha$-favourable.  (ii) Show that a semi-finite weakly
$\alpha$-favourable measure space is perfect.
%451V/451C

\spheader 451Yl Show that any measurable subspace of a weakly
$\alpha$-favourable measure space is weakly $\alpha$-favourable.
%451V/451D

\spheader 451Ym Let $(X,\Sigma,\mu)$ be a weakly $\alpha$-favourable
measure space, $(Y,\Tau,\nu)$ a semi-finite measure space, and $f:X\to
Y$ a $(\Sigma,\Tau)$-measurable function such that $f^{-1}[F]$ is
negligible whenever $F\subseteq Y$ is negligible.   Show that
$(Y,\Tau,\nu)$ is weakly $\alpha$-favourable.
%451V/451E

\spheader 451Yn(i) Show that a measure space is weakly
$\alpha$-favourable iff its completion is
weakly $\alpha$-favourable.   (ii) Show that a semi-finite measure space
is weakly $\alpha$-favourable iff its c.l.d.\ version is weakly
$\alpha$-favourable.
%451V/451G

\spheader 451Yo Show that the c.l.d.\ product of two weakly
$\alpha$-favourable measure spaces is weakly $\alpha$-favourable.
%451V/451I

\spheader 451Yp Show that the product of
any family of weakly $\alpha$-favourable probability measures is weakly
$\alpha$-favourable.
%451V/451J

\spheader 451Yq Show that the space of 451U is not weakly
$\alpha$-favourable.
%451V/451U

\spheader 451Yr Let $(X,\Sigma,\mu)$ be a complete locally determined
measure space and $\undphi$ a lower density for $\mu$ such that
$\undphi X=X$;  let $\frak T$ be the corresponding density topology
(414P).   Show that $(X,\Sigma,\mu)$ is weakly $\alpha$-favourable iff
$(X,\frak T)$ is weakly $\alpha$-favourable (definition: 4A2A).
%451V

\spheader 451Ys Let $X$ be a set, and $\familyiI{\mu_i}$ a family of weakly
$\alpha$-favourable measures on $X$ with sum $\mu$
(234G\formerly{1{}12Ya}).   Show that if $\mu$ is
semi-finite, it is weakly $\alpha$-favourable.
%451Ym 451V

\spheader 451Yt\dvAnew{2009}
Let $X$ and $Y$ be locally compact Hausdorff groups
and $\phi:X\to Y$ a group homomorphism which is Haar
measurable in the sense of 411L, that is, $\phi^{-1}[H]$ is Haar measurable
for every open $H\subseteq Y$.   Show that $\phi$ is continuous.
%Y_0 an open sigma-compact subgp of  Y ;  451Q  to show that
%\phi^{-1}[Y_0]  not negligible;  hence  \phi^{-1}[V]  not negligible for
%any nhd  V  of identity in  Y  ;  use  443D .
}%end of exercises

\endnotes{
\Notesheader{451} For a useful survey of results on countably compact
and perfect measures, with historical notes, see {\smc Ramachandran 02}.

The concepts of `compact', `countably compact' and
`perfect' measure space can all be regarded as attempts to understand
and classify the special properties of Lebesgue measure on $[0,1]$,
regarded as a measure space.   Because a countably separated perfect
probability space is very nearly isomorphic to Lebesgue measure (451Ad),
we can think of a perfect measure space as one in which the
countably-generated $\sigma$-subalgebras look like Lebesgue measure
(451F).   The arguments of 451Ic and 451Jc already hint at the kind of
results we can hope for.   When we form a product measure, each
measurable set in the product will depend, in effect, on sequences of
measurable sets in the factors, and therefore can be studied in terms of
countably generated subalgebras;  so that many results about products of
perfect measures will be derivable, if we wish to take that route, from
results about products of copies of Lebesgue measure.   Of course my
normal approach in this treatise is to go straight for the general
result;  but like anyone else I often start from a picture based on the
familiar special case.   In the next section we shall have some theorems
for which countable compactness, rather than perfectness, seems to be
the relevant property.

The first half of the section (down to 451P) is essentially a matter of
tidying up the theory of compact and perfect measures, and showing that
the same ideas will cover the new class of countably compact measures.
(You may like to go back to 342G, in which I worked through the basic
properties of compact measures, and contrast the arguments used there
with the slightly more sophisticated ones above.)   In 451Q-451T I enter
new territory, showing that for compact measures (and therefore for
Radon measures) the theory of measurable functions into metric spaces is
particularly simple, without making any assumptions about measure-free
cardinals.
}%end of notes

\discrpage

