\frfilename{mt38.tex} 
\versiondate{15.8.08} 
\copyrightdate{2003} 
 
\def\chaptername{Automorphism groups} 
 
\newchapter{38} 
 
As with any mathematical structure, every measure algebra has an 
associated symmetry group, the group of all measure-preserving 
automorphisms.   In this chapter I set out to describe some of the 
remarkable features of these groups.   I begin with elementary results 
on automorphisms of general Boolean algebras (\S381), introducing 
definitions for the rest of the chapter.   In \S382 I give a general 
theorem on the expression of an automorphism as the product of 
involutions (382M), with a description of the normal subgroups of 
certain groups of automorphisms (382R). 
Applications of these ideas to measure algebras are in \S383. 
I continue 
with a discussion of circumstances under which these automorphism groups 
determine the underlying algebras and/or have few outer automorphisms 
(\S384). 
 
One of the outstanding open problems of the subject is the `isomorphism 
problem', the classification of 
automorphisms of measure algebras up to conjugacy in the automorphism 
group.   I offer two sections on `entropy', the most important numerical 
invariant enabling us to distinguish some non-conjugate automorphisms 
(\S\S385-386).   For Bernoulli shifts on the Lebesgue measure algebra 
(385Q-385S), %385Q 385R 385S 
the isomorphism problem is solved by Ornstein's theorem (387I, 387K);  I 
present a complete proof of this theorem in \S\S386-387.   Finally, in 
\S388, I give Dye's theorem, describing the full subgroups generated by 
single automorphisms of measure algebras of countable Maharam type. 
 
\discrpage 
 
