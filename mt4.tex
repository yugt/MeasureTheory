\frfilename{mt4.tex}
\versiondate{14.12.06}
\copyrightdate{2001}

\def\volumename{Topological Measure Spaces}

\newvolume{4}

I return in this volume to the study of measure {\it spaces} rather than
measure {\it algebras}.   For fifty years now measure theory has been
intimately connected with general topology.   Not only do a very large
proportion of the measure spaces arising in applications carry
topologies related in interesting ways to their measures, but many
questions in abstract measure theory can be effectively studied by
introducing suitable topologies.   Consequently any course in measure
theory at this level must be frankly dependent on a substantial
knowledge of topology.   With this proviso, I hope that the present
volume will be accessible to graduate students, and will lead them to
the most important ideas of modern abstract measure theory.

The first and third chapters of the volume seek to provide a thorough
introduction into the ways in which topologies and measures can
interact.   They are divided by a short chapter on descriptive set
theory, on the borderline between set theory, logic, real analysis and
general topology, which I single out for detailed exposition because I
believe that it forms an indispensable part of the background of any
measure theorist.   Chapter 41 is dominated by the concepts of inner
regularity and $\tau$-additivity, coming together in Radon measures
(\S416).   Chapter 43 concentrates rather on questions concerning
properties of a
topological space which force particular relationships with measures on
that space.   But plenty of side-issues are treated in both, such as
Lusin measurability (\S418), the definition of measures from linear
functionals (\S436) and
measure-free cardinals (\S438).   Chapters 45 and 46 continue some of
the same themes, with particular investigations into `disintegrations'
or regular conditional probabilities (\S\S452-453),
stochastic processes (\S\S454-456), %\S454 \S455 \S456
Talagrand's theory of
stable sets (\S465) and the theory of measures on normed
spaces (\S\S466-467).

In contrast with the relatively amorphous structure of Chapters 41, 43,
45 and 46, four chapters of this volume have definite topics.
I have already said that Chapter 42 is an introduction to descriptive
set theory;  like Chapters 31 and 35 in the preceding volume, it is a 
kind of
appendix brought into the main stream of the argument.   Chapter 44
deals with topological groups.   Most of it is of course devoted to Haar
measure, giving the Pontryagin-van Kampen duality theorem (\S445) and
the Ionescu Tulcea theorem on the existence of translation-invariant
liftings (\S447).   But there are also sections on Polish groups
(\S448) and amenable groups (\S449), and some of the general theory of
measures on measurable groups
(\S444).   Chapter 47 is a second excursion, after Chapter 26, into
geometric measure theory.   It starts with Hausdorff measures (\S471),
gives a proof of the Di Giorgio-Federer
Divergence Theorem (\S475), and then examines a number of examples of
`concentration of measure' (\S476).   In the second half of the chapter,
\S\S477-479, I decribe Brownian motion and use it as a basis of 
the theory of Newtonian capacity.
In Chapter 48, I set out the elementary theory of gauge integrals, with
sections on the Henstock and Pfeffer integrals (\S\S483-484).
Finally, in Chapter 49, I give notes on seven special topics:
equidistributed sequences (\S491),
combinatorial forms of concentration of measure (\S492),
extremely amenable groups and groups of measure-preserving automorphisms
(\S\S493-494),
Poisson point processes (\S495),
submeasures (\S496), Szemer\'edi's theorem (\S497) and
subproducts in product spaces (\S498).

I had better mention prerequisites, as usual.   To embark on this
material you will certainly need a solid foundation in measure theory.
Since I do of course use my own exposition as my principal source of
references to the elementary ideas, I advise readers to ensure that they
have easy access to all three previous volumes before starting serious
work on this one.   But you may not need to read very much of them.   It
might be prudent to glance through the detailed contents of Volume 1 and
the first five chapters of Volume 2 to
check that most of the material there is more or less familiar.   I think
\S417 might be difficult to read without at least the results-only version
of Chapter 25 to hand.   But
Volume 3, and the last three chapters of Volume 2, can probably be left
on one side for the moment.   Of course you will need the
Lifting Theorem (Chapter 34) for \S\S447, 452 and 453, and Chapter
26 is essential background for Chapter 47, while Chapter 28 (on Fourier
analysis) may help to make sense of Chapter 44, and parts of Chapter 27
(on probability theory) are necessary for \S\S455-456 and 458-459.
You will certainly need some Fourier analysis for \S479.
And measure algebras
are mentioned in every chapter except (I think) Chapter 48;  but I hope
that the cross-references are precise enough to lead you to what
you need to know at any particular point.   Even Maharam's theorem is
hardly used in this volume.

What you will need, apart from any knowledge of measure theory, is a
sound background in general topology.   This volume calls on a great
many miscellaneous facts from general topology, and the list in \S4A2 is
not a good place to start if continuity and compactness and the
separation axioms are unfamiliar.   My primary reference for topology is
{\smc Engelking 89}.   I do not insist that you should have read this
book (though of course I hope you will do so sometime);  but I do think
you should make sure that you can use it.

In the general introduction to this treatise, I wrote `I make no attempt
to describe the history of the subject', and I have generally been
casual -- some would say negligent -- in my attributions of results to
their discoverers.   Through much of the first three volumes I did at
least have the excuse that the history exists in print in far
more detail than I am qualified to describe.   In the present volume I
find my position more uncomfortable, in that I have been watching
the evolution of the subject relatively closely over the last forty
years, and ought to be able to say something about it.   Nevertheless I
remain reluctant to make definite
statements crediting one person rather than another with originating an
idea.   My more intimate knowledge of the topic makes me even more
conscious than elsewhere of the danger of error and of the breadth of
reading that would be necessary to produce a balanced account.   In some
cases I do attach a result to a specific published paper, but these
attributions should never be regarded as an assertion that any
particular author has priority;  at most, they declare that a historian
should examine the source cited before coming to any decision.   I
assure my friends and colleagues that my omissions are not intended to
slight either them or those we all honour.   What I have tried to do is
to include in
the bibliography to this volume all the published work which (as far as
I am consciously aware) has influenced me while writing it, so that
those who wish to go into the matter will have somewhere to start their
investigations.

\frnewpage

