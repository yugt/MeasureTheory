\frfilename{mt47.tex}
\versiondate{8.4.13}
\copyrightdate{2002}

\def\chaptername{Geometric measure theory}
\def\sectionname{Introduction}

\def\varinnerprod#1#2{#1\dotproduct#2}

\newchapter{47}

I offer a chapter on geometric measure theory,
continuing from Chapter 26.   The greater part of it is
directed specifically at two topics:
a version of the Divergence Theorem (475N) and the elementary theory
of Newtonian capacity and potential (\S479).
I do not attempt to provide a balanced view of the subject, 
for which I must refer you to {\smc Mattila 95},
{\smc Evans \& Gariepy 92} and {\smc Federer 69}.
However \S472, at least, deals with something which must be central to
any approach, Besicovitch's Density Theorem for Radon measures on
$\BbbR^r$ (472D).   In \S473 I examine Lipschitz functions, and give crude
forms of some fundamental ineqalities relating integrals
$\int\|\grad f\|d\mu$ with other measures of the variation of a function
$f$ (473H, 473K).   
In \S474 I introduce perimeter measures $\lambda^{\partial}_E$
and outward-normal functions $\psi_E$ as those for which the Divergence
Theorem, in the form $\int_E\diverg\phi\,d\mu
=\int\varinnerprod{\phi}{\psi_E}\,d\lambda^{\partial}_E$, will be valid
(474E), and give the geometric description of $\psi_E(x)$ as the Federer
exterior normal to $E$ at $x$ (474R).   In \S475 I show that
$\lambda^{\partial}_E$ can be identified with normalized Hausdorff
$(r-1)$-dimensional measure on the essential boundary of $E$.

\S471 is devoted to Hausdorff measures on general metric spaces, extending
the ideas introduced in \S264 for Euclidean space, up to basic results on
densities (471P) and Howroyd's theorem (471S).
In \S476 I turn to a different topic, the problem of finding the subsets
of $\BbbR^r$ on which Lebesgue measure is most `concentrated' in some
sense.   I present a number of classical results, the deepest being
the Isoperimetric Theorem (476H):  among sets with a given measure,
those with the smallest perimeters are the balls.

The last three sections are different again.   Classical electrostatics led
to a vigorous theory of capacity and potential, based on the idea of
`harmonic function'.   It turns out that `Brownian motion' in
$\BbbR^r$ (\S477)
gives an alternative and very powerful approach to the subject.   I have
brought Brownian motion and Wiener measure to this chapter because I wish
to use them to illuminate the geometry of $\BbbR^r$;  but much of \S477
(in particular, the strong Markov property, 477G) is
necessarily devoted to adapting ideas developed in the more general
contexts of L\'evy and Gaussian processes, as described in \S\S455-456.   
In \S478 I give the
most elementary parts of the theory of harmonic and superharmonic
functions, building up to a definition of `harmonic
measures' based on Brownian motion (478P).   In \S479
I use these techniques to describe Newtonian capacity and its extension
Choquet-Newton capacity (479C) on Euclidean space of three or more
dimensions, and establish their basic properties (479E,
479F, 479N, 479P, 479U).

\discrpage


