\frfilename{mt325.tex}
\versiondate{30.8.06}
\copyrightdate{1999}

\def\chaptername{Measure algebras}
\def\sectionname{Free products and product measures}

\newsection{325}

In this section I aim to describe the measure algebras of product
measures as defined in Chapter 25.   This will involve the concept of
`free product' set out in \S315.   It turns out that we cannot determine
the measure algebra of a product measure from the measure algebras of
the factors (325B), unless we are told that
the product measure is localizable;  but that
there is nevertheless a general construction of `localizable measure
algebra free product', applicable to any pair of semi-finite measure
algebras (325D), which represents the measure algebra of the product
measure in the most important cases (325Eb).   In the second part of the
section (325I-325M) I deal with
measure algebra free products of probability algebras, corresponding to
the products of probability spaces treated in \S254.

\leader{325A}{Theorem} Let $(X,\Sigma,\mu)$ and $(Y,\Tau,\nu)$ be
measure spaces, with measure algebras $(\frak A,\bar\mu)$ and
$(\frak B,\bar\nu)$.   Let $\lambda$ be the c.l.d.\ product measure on
$X\times Y$, and $\Lambda$ its domain;  let $(\frak C,\bar\lambda)$ be
the corresponding measure algebra.

(a)(i) The map $E\mapsto E\times Y:\Sigma\to\Lambda$ induces an
order-continuous Boolean homomorphism from $\frak A$ to $\frak C$.

\quad(ii) The map $F\mapsto X\times F:\Tau\to\Lambda$ induces an
order-continuous Boolean homomorphism from $\frak B$ to $\frak C$.

(b) The map $(E,F)\mapsto E\times F:\Sigma\times\Tau\to\Lambda$ induces
a Boolean homomorphism $\psi:\frak A\otimes\frak B\to\frak C$.

(c) $\psi[\frak A\otimes\frak B]$ is topologically dense in $\frak C$ for
the measure-algebra topology of $\frak C$.

(d) For every $c\in\frak C$,

\Centerline{$\bar\lambda c
=\sup\{\bar\lambda(c\Bcap\psi(a\otimes b)):
a\in\frak A,\,b\in\frak B,\,\bar\mu a<\infty,\,\bar\nu b<\infty\}$.}

(e) If $\mu$ and $\nu$ are semi-finite, $\psi$ is injective and
$\bar\lambda\psi(a\otimes b)=\bar\mu a\cdot\bar\mu b$ for every
$a\in\frak A$, $b\in\frak B$.

\proof{{\bf (a)} $E\times Y\in\Lambda$ for every $E\in\Sigma$ (251E),
and $\lambda(E\times Y)=0$ whenever $\mu E=0$ (251Ia).   Thus
$E\mapsto(E\times Y)^{\ssbullet}:\Sigma\to\frak C$ is a Boolean
homomorphism with kernel including $\{E:\mu E=0\}$, so descends to a
Boolean homomorphism $\varepsilon_1:\frak A\to\frak C$.

To see that $\varepsilon_1$ is order-continuous,
let $A\subseteq\frak A$ be a non-empty downwards-directed set with
infimum $0$.   \Quer\ If there is a non-zero lower bound $c$ of
$\varepsilon_1[A]$, express $c$ as $W^{\ssbullet}$ where $W\in\Lambda$.
We have $\lambda(W)>0$;  by the definition of $\lambda$ (251F), there
are $G\in\Sigma$, $H\in\Tau$ such that $\mu G<\infty$, $\nu H<\infty$
and $\lambda(W\cap(G\times H))>0$.   Of course
$\inf_{a\in A}a\Bcap G^{\ssbullet}=0$ in $\frak A$, so
$\inf_{a\in A}\bar\mu(a\Bcap G^{\ssbullet})=0$, by 321F;
let $a\in A$ be such that
$\bar\mu(a\Bcap G^{\ssbullet})\cdot\nu H<\lambda(W\cap(G\times H))$.
Express $a$ as $E^{\ssbullet}$, where $E\in\Sigma$.   Then
$\lambda(W\setminus(E\times Y))=0$.   But this means that

\Centerline{$\lambda(W\cap(G\times H))
\le\lambda((E\cap G)\times H)
=\mu(E\cap G)\cdot\nu H
=\bar\mu(a\Bcap G^{\ssbullet})\cdot\nu H$,}

\noindent contradicting the choice of $a$.\ \Bang\   Thus
$\inf\varepsilon_1[A]=0$ in $\frak C$;  as $A$ is arbitrary,
$\varepsilon_1$ is order-continuous.

Similarly $\varepsilon_2:\frak B\to\frak C$, induced by $F\mapsto
X\times F:\Tau\to\Lambda$, is order-continuous.

\medskip

{\bf (b)} Now there must be a corresponding Boolean homomorphism
$\psi:\frak A\otimes\frak B\to\frak C$ such that
$\psi(a\otimes b)=\varepsilon_1a\Bcap\varepsilon_2b$ for every
$a\in\frak A$ and $b\in\frak B$, that is,

\Centerline{$\psi(E^{\ssbullet}\otimes F^{\ssbullet})
=(E\times Y)^{\ssbullet}\Bcap(X\times F)^{\ssbullet}
=(E\times F)^{\ssbullet}$}

\noindent for every $E\in\Sigma$, $F\in\Tau$ (315Jb).

\medskip

{\bf (c)} Suppose that $c$, $e\in\frak C$, $\bar\lambda e<\infty$
and $\epsilon>0$.   Express $c$, $e$ as $U^{\ssbullet}$, $W^{\ssbullet}$
where $U$, $W\in\Lambda$.   By 251Ie, there are
$E_0,\ldots,E_n\in\Sigma$, $F_0,\ldots,F_n\in\Tau$, all of finite
measure, such that
$\lambda((U\cap W)\symmdiff\bigcup_{i\le n}E_i\times F_i)\le\epsilon$.
Set

\Centerline{$c_1=(\bigcup_{i\le n}E_i\times
F_i)^{\ssbullet}\in\psi[\frak A\otimes\frak B]$;}

\noindent then

\Centerline{$\bar\lambda(e\Bcap(c\Bsymmdiff c_1))
=\lambda(W\cap(U\symmdiff\bigcup_{i\le n}E_i\times F_i))
\le\epsilon$.}

\noindent As $c$, $e$ and $\epsilon$ are arbitrary,
$\psi[\frak A\otimes\frak B]$ is topologically dense in $\frak C$.

\medskip

{\bf (d)} By the definition of $\lambda$, we have

\Centerline{$\lambda W=\sup\{\lambda(W\cap(E\times F)):
E\in\Sigma,\,F\in\Tau,\,\mu E<\infty,\,\nu F<\infty\}$}

\noindent for every $W\in\Lambda$;   so all we have to do is express $c$
as $W^{\ssbullet}$.

\medskip

{\bf (e)} Now suppose that $\mu$ and $\nu$ are semi-finite.   Then
$\lambda(E\times F)=\mu E\cdot\nu F$
for any $E\in\Sigma$, $F\in\Tau$ (251J), so
$\bar\lambda\psi(a\otimes b)=\bar\mu a\cdot\bar\nu b$ for every
$a\in\frak A$ and $b\in\frak B$.

To see that $\psi$ is injective, take any non-zero
$c\in\frak A\otimes\frak B$;  then there must be non-zero $a\in\frak A$,
$b\in\frak B$ such that $a\otimes b\Bsubseteq c$ (315Kb), so that

\Centerline{$\bar\lambda\psi c\ge\bar\lambda\psi(a\otimes b)
=\bar\mu a\cdot\bar\nu b>0$}

\noindent and $\psi c\ne 0$.
}%end of proof of 325A

\leader{325B}{Characterizing the measure algebra of a
product \dvrocolon{space}}\cmmnt{ A
very natural question to ask is, whether it is possible to define a
`measure algebra free product' of two abstract measure algebras in a way
which will correspond to one of the constructions above.  I give an
example to show the difficulties involved.

\medskip

\noindent}{\bf Example} There are complete locally determined
localizable measure spaces $(X,\mu)$, $(X',\mu')$, with isomorphic
measure algebras, and a probability space $(Y,\nu)$ such that the
measure algebras of the c.l.d.\ product measures on $X\times Y$,
$X'\times Y$ are not isomorphic.

\proof{ Let $(X,\Sigma,\mu)$ be the complete
locally determined localizable not-strictly-localizable measure space
described in 216E.   Recall that, for $E\in\Sigma$,
$\mu E=\#(\{\gamma:\gamma\in C,\,f_{\gamma}\in E\})$ if this is finite,
$\infty$ otherwise (216Eb), where $C$ is a set of cardinal greater than
$\frak c$.   The map
$E\mapsto\{\gamma:f_{\gamma}\in E\}:\Sigma\to\Cal PC$ is surjective
(216Ec), so descends to an isomorphism between
$\frak A$, the measure algebra of $\mu$, and $\Cal PC$.   Let
$(X',\Sigma',\mu')$ be $C$ with counting measure, so that its measure
algebra $(\frak A',\bar\mu')$ is isomorphic to $(\frak A,\bar\mu)$,
while $\mu'$ is of course strictly localizable.

Let $(Y,\Tau,\nu)$ be $\{0,1\}^C$ with its usual measure.   Let
$\lambda$, $\lambda'$ be the c.l.d.\ product measures on $X\times Y$,
$X'\times Y$ respectively, and $(\frak C,\bar\lambda)$,
$(\frak C',\bar\lambda')$ the corresponding measure algebras.   Then
$\lambda$ is not localizable (254U), so $(\frak C,\bar\lambda)$ is not
localizable (322Be).   On the other hand, $\lambda'$, being the c.l.d.\
product of strictly localizable measures, is strictly localizable
(251O), therefore localizable, so $(\frak C',\bar\lambda')$ is
localizable, and is not isomorphic to $(\frak C,\bar\lambda)$.
}%end of proof of 325Bb

\leader{325C}{}\cmmnt{ Thus there can be no universally applicable
method of identifying the measure algebra of a product measure from the
measure algebras of
the factors.   However, you have no doubt observed that the example
above involves non-$\sigma$-finite spaces, and conjectured that this is
not an accident.   In contexts in which we know that the algebras
involved are localizable, there are positive results available, such as
the following.

\medskip

\noindent}{\bf Theorem} Let $(X_1,\Sigma_1,\mu_1)$ and
$(X_2,\Sigma_2,\mu_2)$ be semi-finite measure spaces, with measure
algebras $(\frak A_1,\bar\mu_1)$ and $(\frak A_2,\bar\mu_2)$.   Let
$\lambda$ be the c.l.d.\ product measure on $X_1\times X_2$, and
$(\frak C,\bar\lambda)$ the corresponding measure algebra.   Let
$(\frak B,\bar\nu)$ be a localizable measure algebra, and
$\phi_1:\frak A_1\to\frak B$, $\phi_2:\frak A_2\to\frak B$
order-continuous Boolean homomorphisms such that
$\bar\nu(\phi_1(a_1)\Bcap\phi_2(a_2))=\bar\mu_1a_1\cdot\bar\mu_2a_2$ for
all $a_1\in\frak A_1$, $a_2\in\frak A_2$.   Then there is a unique
order-continuous measure-preserving Boolean homomorphism
$\phi:\frak C\to\frak B$ such that
$\phi(\psi(a_1\otimes a_2))=\phi_1(a_1)\Bcap\phi_2(a_2)$ for all
$a_1\in\frak A_1$,
$a_2\in\frak A_2$, writing $\psi:\frak A_1\otimes\frak A_2\to\frak C$
for the canonical map\cmmnt{ described in 325A}.

\proof{{\bf (a)} Because $\psi$ is injective, it is an isomorphism
between $\frak A_1\otimes\frak A_2$ and its image in $\frak C$.  I trust
it will cause no confusion if I abuse notation slightly and treat
$\frak A_1\otimes\frak A_2$ as actually a subalgebra of $\frak C$.
Now the
Boolean homomorphisms $\phi_1$, $\phi_2$ correspond to a Boolean
homomorphism $\theta:\frak A_1\otimes\frak A_2\to\frak B$.   The point
is that $\bar\nu\theta c=\bar\lambda c$ for every
$c\in\frak A_1\otimes\frak A_2$.
\Prf\ By 315Kb, every member of $\frak A_1\otimes\frak A_2$ is
expressible as $\sup_{i\le n}a_i\otimes a'_i$, where $a_i\in\frak A_1$,
$a'_i\in\frak A_2$ for each $i$
and $\langle a_i\otimes a'_i\rangle_{i\le n}$ is
disjoint.   Now for each $i$ we have

\Centerline{$\bar\nu\theta(a_i\otimes a'_i)
=\bar\nu(\phi_1(a_i)\Bcap\phi_2(a'_i))
=\bar\mu_1a_i\cdot\bar\mu_2a'_i
=\bar\lambda(a_i\otimes a'_i)$,}

\noindent by 325Ae.   So

\Centerline{$\bar\nu\theta(c)
=\sum_{i=0}^n\bar\nu\theta(a_i\otimes a'_i)
=\sum_{i=0}^n\bar\lambda(a_i\otimes a'_i)
=\bar\lambda c$.   \Qed}

\medskip

{\bf (b)} The following fact will underlie many of the arguments below.
If $e\in\frak B$, $\bar\nu e<\infty$ and $\epsilon>0$, there are
$e_1\in\frak A_1^f$, $e_2\in\frak A_2^f$ such that
$\bar\nu(e\Bsetminus\theta(e_1\otimes e_2))\le\epsilon$,
writing $\frak A_i^f$ for $\{a:\bar\mu_i a<\infty\}$.
\Prf\ Because $(\frak A_1,\bar\mu_1)$ is semi-finite, $\frak A_1^f$
has supremum $1$ in
$\frak A_1$;  because $\phi_1$ is order-continuous,
$\sup\{\phi_1(a):a\in\frak A_1^f\}=1$ in $\frak B$, and
$\inf\{e\Bsetminus\phi_1(a):a\in\frak A_1^f\}=0$ (313Aa).   Because
$\frak A_1^f$ is upwards-directed,
$\{e\Bsetminus\phi_1(a):a\in\frak A_1^f\}$ is downwards-directed, so
$\inf\{\bar\nu(e\Bsetminus\phi(a)):a\in\frak A_1^f\}=0$ (321F again).   Let
$e_1\in\frak A_1^f$ be such that
$\bar\nu(e\Bsetminus\phi_1(e_1))\le\bover12\epsilon$.

In the same way, there is an $e_2\in\frak A_2^f$ such that
$\bar\nu(e\Bsetminus\phi_2(e_2))\le\bover12\epsilon$.   Consider
$e'=e_1\otimes e_2\in\frak C$.   Then

\Centerline{$\bar\nu(e\Bsetminus\theta e')
=\bar\nu(e\Bsetminus(\phi_1(e_1)\Bcap\phi_2(e_2)))
\le\bar\nu(e\Bsetminus\phi_1(e_1))+\bar\nu(e\Bsetminus\phi_2(e_2))
\le\epsilon$.  \Qed}

\medskip

{\bf (c)} The next step is to check that $\theta$ is uniformly
continuous for the measure-algebra uniformities defined by $\bar\nu$
and $\bar\lambda$.   \Prf\ Take any $e\in\frak B^f$ and $\epsilon>0$.
By (b), there are $e_1$, $e_2$ such that $\bar\lambda(e_1\otimes
e_2)<\infty$ and $\bar\nu(e\Bsetminus\theta(e_1\otimes
e_2))\le\bover12\epsilon$.   Set $e'=e_1\otimes e_2$.
Now suppose that $c$, $c'\in\frak A_1\otimes\frak A_2$ and
$\bar\lambda((c\Bsymmdiff c')\Bcap e')\le\bover12\epsilon$.   Then

\Centerline{$\bar\nu((\theta(c)\Bsymmdiff\theta(c'))\Bcap e)
\le\bar\nu\theta((c\Bsymmdiff c')\Bcap e')+\bar\nu(e\Bsetminus\theta e')
\le\bar\lambda((c\Bsymmdiff c')\Bcap e')+\Bover12\epsilon
\le\epsilon$.}

\noindent By 3A4Cc, $\theta$ is uniformly continuous for the
subspace uniformity on $\frak A_1\otimes\frak A_2$.\ \Qed

\medskip

{\bf (d)} Recall that $\frak A_1\otimes\frak A_2$ is topologically dense
in $\frak C$ (325Ac), while $\frak B$ is complete for its uniformity
(323Gc).   So there is a uniformly continuous function
$\phi:\frak C\to\frak B$ extending $\theta$ (3A4G).

\medskip

{\bf (e)} Because $\theta$ is a Boolean homomorphism, so is $\phi$.
\Prf\ (i) The functions $c\mapsto\phi(1\Bsetminus c)$,
$c\mapsto1\Bsetminus\phi(c)$ are continuous and the topology of
$\frak B$ is Hausdorff, so
$\{c:\phi(1\Bsetminus c)=1\Bsetminus\phi(c)\}$ is
closed;   as it includes $\frak A_1\otimes\frak A_2$, it must be the
whole of $\frak C$.   (ii) The functions $(c,c')\mapsto\phi(c\Bcup c')$,
$(c,c')\mapsto\phi(c)\Bcup\phi(c')$ are continuous, so
$\{(c,c'):\phi(c\Bcup c')=\phi(c)\Bcup\phi(c')\}$ is closed in
$\frak C\times\frak C$;  as it includes
$(\frak A_1\otimes\frak A_2)\times(\frak A_1\otimes\frak A_2)$, it must be the whole of $\frak C\times\frak C$.\ \Qed

\medskip

{\bf (f)} Because $\theta$ is measure-preserving, so is $\phi$.   \Prf\
Take any $e_1\in\frak A_1^f$, $e_2\in\frak A_2^f$.   Then the functions
$c\mapsto\bar\lambda(c\Bcap(e_1\otimes e_2))$,
$c\mapsto\bar\nu\phi(c\Bcap(e_1\otimes e_2))$ are continuous and equal
on $\frak A_1\otimes\frak A_2$, so are equal on $\frak C$.   The
argument of (b) shows that for any $b\in\frak B$,

$$\eqalign{\bar\nu b
&=\sup\{\bar\nu(b\Bcap e):e\in\frak B^f\}\cr
&=\sup\{\bar\nu(b\Bcap\phi(e_1\otimes e_2)):e_1\in\frak
A_1^f,\,e_2\in\frak
A_2^f\},\cr}$$

\noindent so that

$$\eqalign{\bar\nu\phi(c)
&=\sup\{\bar\nu\phi(c\Bcap(e_1\otimes e_2)):e_1\in\frak
A_1^f,\,e_2\in\frak
A_2^f\}\cr
&=\sup\{\bar\lambda(c\Bcap(e_1\otimes e_2)):e_1\in\frak
A_1^f,\,e_2\in\frak A_2^f\}
=\bar\lambda c\cr}$$

\noindent for every $c\in\frak C$.\ \Qed

\medskip

{\bf (g)} To see that $\phi$ is order-continuous, take any non-empty
downwards-directed set $C\subseteq\frak C$ with infimum $0$.   \Quer\ If
$\phi[C]$ has a non-zero lower bound $b$ in $\frak B$, let
$e\Bsubseteq b$ be such that $0<\bar\nu e<\infty$.   Let $e'\in\frak C$
be such that
$\bar\lambda e'<\infty$ and $\bar\nu(e\Bsetminus\phi(e'))<\bar\nu e$, as
in (b) above, so that $\bar\nu(e\Bcap\phi(e'))>0$.   Now, because
$\inf C=0$, there is a $c\in C$ such that
$\bar\lambda(c\Bcap e')<\bar\nu(e\Bcap\phi(e'))$.   But this means that

\Centerline{$\bar\nu(b\Bcap\phi(e'))\le\bar\nu\phi(c\Bcap e')
=\bar\lambda(c\Bcap e')<\bar\nu(e\Bcap\phi(e'))
\le\bar\nu(b\Bcap\phi(e'))$,}

\noindent which is absurd.\ \BanG   Thus $\inf\phi[C]=0$ in $\frak B$.
As $C$ is arbitrary, $\phi$ is order-continuous.

\medskip

{\bf (h)} Finally, to see that $\phi$ is unique, observe that any
order-continuous Boolean homomorphism from $\frak C$ to $\frak B$ must
be continuous (324Fc);  so that if it agrees with $\phi$ on
$\frak A_1\otimes\frak A_2$ it must agree with $\phi$ on $\frak C$.
}%end of proof of 325C

\vleader{72pt}{325D}{Theorem} Let $(\frak A_1,\bar\mu_1)$ and
$(\frak A_2,\bar\mu_2)$ be semi-finite measure algebras.

(a) There is a localizable measure algebra $(\frak C,\bar\lambda)$,
together with order-continuous Boolean homomorphisms
$\varepsilon_1:\frak A_1\to\frak C$ and $\varepsilon_2:\frak A_2\to\frak C$, such that
whenever $(\frak B,\bar\nu)$ is a localizable measure algebra, and
$\phi_1:\frak A_1\to\frak B$, $\phi_2:\frak A_2\to\frak B$ are
order-continuous Boolean homomorphisms and
$\bar\nu(\phi_1(a_1)\Bcap\phi_2(a_2))=\bar\mu_1a_1\cdot\bar\mu_2a_2$ for
all $a_1\in\frak A_1$, $a_2\in\frak A_2$, then there is a unique
order-continuous measure-preserving Boolean homomorphism
$\phi:\frak C\to\frak B$ such that $\phi\varepsilon_j=\phi_j$ for both $j$.

(b) The structure $(\frak C,\bar\lambda,\varepsilon_1,\varepsilon_2)$ is determined
up to isomorphism by this property.

(c)(i) The Boolean homomorphism
$\psi:\frak A_1\otimes\frak A_2\to\frak C$ defined from $\varepsilon_1$ and
$\varepsilon_2$ is injective, and $\psi[\frak A_1\otimes\frak A_2]$ is
topologically dense in $\frak C$.

\quad(ii) The closed subalgebra of $\frak C$ generated by
$\psi[\frak A_1\otimes\frak A_2]$ is the whole of $\frak C$.

(d) If $j\in\{1,2\}$ and $(\frak A_j,\bar\mu_j)$ is localizable, then
$\varepsilon_j[\frak A_j]$ is a closed subalgebra of $(\frak C,\bar\lambda)$.

\proof{{\bf (a)(i)} We may regard $(\frak A_1,\bar\mu_1)$ as the
measure algebra of $(Z_1,\Sigma_1,\mu_1)$ where $Z_1$ is the Stone space
of $\frak A_1$, $\Sigma_1$ is the algebra of subsets of $Z_1$ differing
from an open-and-closed set by a meager set, and $\mu_1$ is an
appropriate measure (321K).   Note that in this representation,
each $a\in\frak A_1$ becomes identified with $\widehat{a}^{\ssbullet}$,
where $\widehat{a}$ is the open-and-closed subset of $Z_1$ corresponding
to $a$.   Similarly, we may think of $(\frak A_2,\bar\mu_2)$ as the measure algebra of $(Z_2,\Sigma_2,\mu_2)$,
where $Z_2$ is the Stone space of $\frak A_2$.

\medskip

\quad{\bf (ii)} Let $\lambda$ be the c.l.d.\ product measure on
$Z_1\times Z_2$.   The point is that $\lambda$ is strictly localizable.
\Prf\ By 322Ea, both $\frak A_1$ and $\frak A_2$ have partitions of unity
consisting of elements of finite measure;  let
$\langle c_i\rangle_{i\in I}$, $\langle d_j\rangle_{j\in J}$ be such partitions.
Then $\langle\widehat{c}_i\times\widehat{d}_j\rangle_{i\in I,j\in J}$ is
a disjoint family of sets of finite measure in $Z_1\times Z_2$.   If
$W\subseteq Z_1\times Z_2$ is such that $\lambda W>0$, there must be sets $E_1$, $E_2$ of finite measure such that
$\lambda(W\cap(E_1\times E_2))>0$.   Because
$E_1^{\ssbullet}=\sup_{i\in I}E_1^{\ssbullet}\Bcap c_i$, we must have

\Centerline{$\mu_1E_1=\bar\mu_1E_1^{\ssbullet}
=\sum_{i\in I}\bar\mu_1(E_1^{\ssbullet}\Bcap c_i)
=\sum_{i\in I}\mu_1(E_1\cap\widehat{c}_i)$.}

\noindent Similarly,
$\mu_2E_2=\sum_{i\in J}\mu_2(E_2\cap\widehat{d}_j)$.
But this means that there must be finite $I'\subseteq I$,
$J'\subseteq J$ such that

\Centerline{$\sum_{i\in I',j\in J'}
  \mu_1(E_1\cap\widehat{c}_i)\mu_2(E_2\cap\widehat{d}_j)
>\mu_1E_1\cdot\mu_2E_2-\lambda(W\cap(E_1\times E_2))$,}

\noindent so that there have to be $i\in I'$, $j\in J'$ such that
$\lambda(W\cap(\widehat{c}_i\times\widehat{d}_j))>0$.

Now this means that
$\langle \widehat{c}_i\times\widehat{d}_j\rangle_{i\in I,j\in J}$
satisfies the conditions of 213O.   Because $\lambda$ is surely
complete and locally determined, it is strictly localizable.\ \Qed

\medskip

\quad{\bf (iii)} We may therefore take $(\frak C,\bar\lambda)$ to be
just the measure algebra of $\lambda$.   The maps $\varepsilon_1$, $\varepsilon_2$
will be the canonical maps described in 325Aa, inducing the map
$\psi:\frak A_1\otimes\frak A_2\to\frak C$ referred to in 325C;  and
325C now gives the result.

\medskip

{\bf (b)} This is nearly obvious.   Suppose we had an alternative
structure $(\frak C',\bar\lambda',\varepsilon'_1,\varepsilon'_2)$
with the same
property.   Then we must have an order-continuous measure-preserving
Boolean homomorphism $\phi:\frak C\to\frak C'$ such that
$\phi\varepsilon_j=\varepsilon'_j$ for both $j$;  and similarly we have an
order-continuous measure-preserving Boolean homomorphism
$\phi':\frak C'\to\frak C$ such that $\phi'\varepsilon'_j=\varepsilon_j$
for both $j$.   Now $\phi'\phi:\frak C\to\frak C$ is an
order-continuous measure-preserving Boolean homomorphism such that
$\phi'\varepsilon_j=\varepsilon_j$ for both $j$.   By the uniqueness assertion in
(a), applied with $\frak B=\frak C$, $\phi'\phi$ must be the
identity on $\frak C$.   In the same way, $\phi\phi'$ is the
identity on $\frak C'$.   So $\phi$ and $\phi'$ are the two halves
of the required isomorphism.

\medskip

{\bf (c)} In view of the construction for $\frak C$ offered in part (a)
of the proof, (i) is just a consequence of 325Ac and 325Ae.   Now (ii)
follows by 323J.

\medskip

{\bf (d)} If $\frak A_j$ is Dedekind complete then $\varepsilon_j[\frak A_j]$
is order-closed in $\frak C$ because $\varepsilon_j$ is order-continuous
(314F(a-i)).
}%end of proof of 325D

\leader{325E}{Remarks}\cmmnt{ \bf (a)}  We could say that a
measure algebra $(\frak C,\bar\lambda)$, together with embeddings
$\varepsilon_1$ and $\varepsilon_2$, as
described in 325D, is a {\bf localizable measure algebra free product}
of $(\frak A_1,\bar\mu_1)$ and $(\frak A_2,\bar\mu_2)$\cmmnt{;  and
its uniqueness up to isomorphism makes it safe, most of the time, to
call it `the' localizable measure algebra free product.   Observe that
it can equally well be regarded as the uniform space completion of the
algebraic free product;  see 325Yc}.

\cmmnt{\spheader 325Eb As the example in
325B shows, the localizable
measure algebra free product of the measure algebras of given measure
spaces need not appear directly as the measure algebra of their product.
But there is one context in which it must so appear:  if
the product measure is localizable, 325C tells us at once
that it has the right measure algebra.   For $\sigma$-finite
measure algebras, of course, any corresponding measure spaces have to
be strictly localizable, so again we can use the product measure
directly.
}%end of comment

\leader{325F}{}\cmmnt{ I ought not to proceed to the next topic
without giving
another pair of examples to show the subtlety of the concept of `measure
algebra free product'.

\medskip

\noindent}{\bf Example} Let $(\frak A,\bar\mu)$ be the measure algebra
of Lebesgue measure $\mu$ on $[0,1]$, and $(\frak C,\bar\lambda)$ the
measure algebra of Lebesgue measure $\lambda$ on $[0,1]^2$.   Then
$(\frak C,\bar\lambda)$ can be regarded as the localizable measure
algebra free
product of $(\frak A,\bar\mu)$ with itself\cmmnt{, by 251N and 325Eb}.
Let $\psi:\frak A\otimes\frak A\to\frak C$ be the canonical map\cmmnt{, as described
in 325A}.   Then $\psi[\frak A\otimes\frak A]$ is not order-dense
in $\frak C$, and $\psi$ is not order-continuous.

\proof{{\bf (a)} Let $\sequencen{\epsilon_n}$ be a sequence in $[0,1]$
such that $\sum_{n=0}^{\infty}\epsilon_n=\infty$, but
$\sum_{n=0}^{\infty}\epsilon_n^2<1$;  for instance, we could take
$\epsilon_n=\bover1{n+2}$.   Let $\sequencen{E_n}$ be a stochastically
independent sequence of measurable subsets of $[0,1]$ such that
$\mu E_n=\epsilon_n$ for each $n$.   In $\frak A$ set
$a_n=E_n^{\ssbullet}$,
and consider $c_n=\sup_{i\le n}a_i\otimes a_i\in\frak A\otimes\frak A$
for each $n$.

\medskip

{\bf (b)} We have $\sup_{n\in\Bbb N}c_n=1$ in $\frak A\otimes\frak A$.
\Prf\Quer\ Otherwise, there is a non-zero $a\in\frak A\otimes\frak A$
such that $a\Bcap(a_n\otimes a_n)=0$ for every $n$, and now there are
non-zero $b$, $b'\in\frak A$ such that $b\otimes b'\Bsubseteq a$.   Set
$I=\{n:a_n\Bcap b=0\}$, $J=\{n:a_n\Bcap b'\}=0$.   Then
$\langle E_n\rangle_{n\in I}$ is an independent family and
$\mu(\bigcup_{n\in I}E_i)\le 1-\bar\mu b<1$, so
$\sum_{n\in I}\mu E_n<\infty$, by the
Borel-Cantelli lemma (273K).   Similarly $\sum_{n\in J}\mu E_n<\infty$.
Because $\sum_{n\in\Bbb N}\mu E_n=\infty$, there must be some
$n\in\Bbb N\setminus(I\cup J)$.   Now $a_n\Bcap b$ and $a_n\Bcap b'$ are
both non-zero, so

\Centerline{$0\ne(a_n\Bcap b)\otimes(a_n\Bcap b')
=(a_n\otimes a_n)\Bcap(b\otimes b')=0$,}

\noindent which is absurd.\ \Bang\Qed

\medskip

{\bf (c)} On the other hand,

\Centerline{$\sum_{n=0}^{\infty}\bar\lambda\psi(c_n)
\le\sum_{n=0}^{\infty}(\bar\mu a_n)^2
=\sum_{n=0}^{\infty}\epsilon_n^2<1$,}

\noindent by the choice of the $\epsilon_n$.   So
$\sup_{n\in\Bbb N}\psi(c_n)$ cannot be $1$ in $\frak C$.

Thus $\psi$ is not order-continuous.

\medskip

{\bf (d)} By 313P(a-ii) and 313O, $\psi[\frak A\otimes\frak A]$ cannot
be order-dense in $\frak C$;  alternatively, (b) shows that there can be
no non-zero member of $\psi[\frak A\otimes\frak A]$ included in
$1\Bsetminus\sup_{n\in\Bbb N}\psi(c_n)$.   (Both these arguments rely
tacitly on the fact that $\psi$ is injective, as noted in 325Ae.)
}%end of proof of 325F

\leader{325G}{}\cmmnt{ Since 325F shows that the free product
and the localizable measure algebra free product are very different
constructions, I had better repeat an idea from \S315 in the new
context.

\medskip

\noindent}{\bf Example} Again, let $(\frak A,\bar\mu)$ be the measure
algebra of Lebesgue measure on $[0,1]$, and $(\frak C,\bar\lambda)$
the measure algebra of Lebesgue measure on $[0,1]^2$.   Then there is no
order-continuous Boolean homomorphism $\phi:\frak C\to\frak A$ such that
$\phi(a\otimes b)=a\Bcap b$ for all $a$, $b\in\frak A$.
\prooflet{\Prf\ Let $\phi:\frak C\to\frak A$ be a Boolean homomorphism
such that $\phi(a\otimes b)=a\Bcap b$ for all $a$, $b\in \frak A$.   For
$i<2^n$ let $a_{ni}$ be the equivalence class in $\frak A$ of the
interval $[2^{-n}i,2^{-n}(i+1)]$, and set $c_n=\sup_{i<2^n}a_{ni}\otimes
a_{ni}$.   Then $\phi c_n=1$ for every $n$, but
$\bar\lambda c_n=2^{-n}$ for each $n$, so $\inf_{n\in\Bbb N}c_n=0$ in
$\frak C$;  thus $\phi$ cannot be order-continuous.\ \Qed}
\cmmnt{(Compare 315Q.)}

\leader{*325H}{Products of more than two factors} \cmmnt{We can of
course extend the ideas of 325A, 325C and 325D to products of any finite
number of factors.   No new ideas are needed, so I spell the results out
without proofs.

\medskip

}{\bf (a)} Let $\langle(\frak A_i,\bar\mu_i)\rangle_{i\in I}$ be a
non-empty finite family of semi-finite measure algebras.
Then there is a localizable measure algebra
$(\frak C,\bar\lambda)$, together with order-continuous Boolean
homomorphisms $\varepsilon_i:\frak A_i\to\frak C$ for $i\in I$, such that
whenever $(\frak B,\bar\nu)$ is a localizable measure algebra, and
$\phi_i:\frak A_i\to\frak B$ are
order-continuous Boolean homomorphisms such that
$\bar\nu(\inf_{i\in I}\phi_i(a_i))=\prod_{i\in I}\bar\mu_ia_i$ whenever
$a_i\in\frak A_i$ for each $i$, then there is a unique
order-continuous measure-preserving Boolean homomorphism
$\phi:\frak C\to\frak B$ such that $\phi\varepsilon_i=\phi_i$ for every $i$.

\spheader 325Hb The structure
$(\frak C,\bar\lambda,\langle\varepsilon_i\rangle_{i\in I})$ is determined up to
isomorphism by this property.

\spheader 325Hc The Boolean homomorphism
$\psi:\bigotimes_{i\in I}\frak A_i\to\frak C$ defined from the
$\varepsilon_i$
is injective, and $\psi[\bigotimes_{i\in I}\frak A_i]$ is topologically
dense in $\frak C$.

\spheader 325Hd Write $\Locmafp_{i\in I}(\frak A_i,\bar\mu_i)$
for (a particular version of) the localizable measure algebra free
product described in (a).   If
$\langle\frak(A_i,\bar\mu_i)\rangle_{i\in I}$ is a
finite family of semi-finite measure algebras and
$\langle I_k\rangle_{k\in K}$ is a partition of $I$ into non-empty sets,
then $\Locmafp_{i\in I}(\frak A_i,\bar\mu_i)$ is isomorphic,
in a canonical way, to
$\Locmafp_{k\in K}\bigl(\Locmafp_{i\in I_k}(\frak A_i,\bar\mu_i)\bigr)$.

\spheader 325He Let $\langle(X_i,\Sigma_i,\mu_i)\rangle_{i\in I}$
be a finite family of semi-finite measure spaces, and write
$(\frak A_i,\bar\mu_i)$ for the measure algebra of $(X_i,\Sigma_i,\mu_i)$.
Let $\lambda$ be the c.l.d.\ product measure on
$\prod_{i\in I}X_i$\cmmnt{ (251W)}, and
$(\frak C,\bar\lambda)$ the corresponding measure algebra.   Then there
is a canonical order-continuous measure-preserving embedding of
$(\frak C,\bar\lambda)$ into the localizable measure algebra free
product of the $(\frak A_i,\bar\mu_i)$.   If each $\mu_i$ is strictly
localizable, this embedding is an isomorphism.

\leader{325I}{Infinite \dvrocolon{products}}\cmmnt{ Just as in \S254,
we can now turn to products of infinite families of probability
algebras.

\wheader{325I}{4}{2}{2}{96pt}

\noindent}{\bf Theorem}  Let
$\langle(X_i,\Sigma_i,\mu_i)\rangle_{i\in I}$ be any family of probability spaces, with measure algebras
$(\frak A_i,\bar\mu_i)$.   Let $\lambda$ be the product measure on
$X=\prod_{i\in I}X_i$, and $(\frak C,\bar\lambda)$ the corresponding
measure algebra.   For each $i\in I$, we have a measure-preserving
homomorphism $\varepsilon_i:\frak A_i\to\frak C$ corresponding to the
inverse-measure-preserving function $x\mapsto x(i):X\to X_i$.   Let
$(\frak B,\bar\nu)$ be a probability algebra, and
$\phi_i:\frak A_i\to\frak B$ Boolean homomorphisms such that $\bar\nu(\inf_{i\in J}\phi_i(a_i))=\prod_{i\in J}\bar\mu_ia_i$ whenever $J\subseteq I$ is
a finite set and $a_i\in\frak A_i$ for every $i$.   Then there is a
unique measure-preserving Boolean homomorphism $\phi:\frak C\to\frak B$
such that $\phi\varepsilon_i=\phi_i$ for every $i\in I$.

\proof{{\bf (a)} As remarked in 254Fb, all the maps $x\mapsto x(i)$ are
inverse-measure-preserving, so correspond to measure-preserving
homomorphisms $\varepsilon_i:\frak A_i\to\frak C$ (324M).   It will be helpful
to use some notation from \S254.   Write $\Cal C$ for the family of
measurable cylinders in $X$ expressible in the form

\Centerline{$E=\{x:x\in X,\,x(i)\in E_i$ for every $i\in J\}$,}

\noindent where $J\subseteq I$ is finite and $E_i\in\Sigma_i$ for every
$i\in J$.   Note that in this case

\Centerline{$E^{\ssbullet}=\inf_{i\in J}\varepsilon_i(E_i^{\ssbullet})$.}

\noindent Set

\Centerline{$C=\{E^{\ssbullet}:E\in\Cal C\}\subseteq\frak C$,}

\noindent so that $C$ is precisely the family of elements of $\frak C$
expressible in the form $\inf_{i\in J}\phi_i(a_i)$ where $J\subseteq I$
is finite and $a_i\in \frak A_i$ for each $i$.

The homomorphisms $\varepsilon_i:\frak A_i\to\frak C$ define a Boolean
homomorphism $\psi:\bigotimes_{i\in I}\frak A_i\to\frak C$ (315J),
which is injective.   \Prf\ If
$c\in\bigotimes_{i\in I}\frak A_i$ is
non-zero, there must be a finite set $J\subseteq I$ and a family
$\langle a_i\rangle_{i\in J}$ such that $a_i\in\frak A_i\setminus\{0\}$
for each $i$ and $c\Bsupseteq\inf_{i\in J}\tilde\varepsilon_i(a_i)$, where
for the moment I write $\tilde\varepsilon_i$ for the canonical map from
$\frak A_i$ to $\bigotimes_{i\in I}\frak A_i$ (315Kb).
Express each $a_i$ as $E_i^{\ssbullet}$, where $E_i\in\Sigma_i$.   Then

\Centerline{$E=\{x:x\in X,\,x(i)\in E_i$ for each $i\in J\}$}

\noindent has measure

\Centerline{$\lambda E=\prod_{i\in J}\mu E_i
=\prod_{i\in J}\bar\mu a_i\ne 0$,}

\noindent while

\Centerline{$E^{\ssbullet}
=\psi(\inf_{i\in J}\tilde\varepsilon_i(a_i))\Bsubseteq\psi(c)$,}

\noindent so $\psi(c)\ne 0$.   As $c$ is arbitrary, $\psi$ is
injective.\ \Qed

\medskip

{\bf (b)} Because $\psi$ is injective, it is an isomorphism between
$\bigotimes_{i\in I}\frak A_i$ and its image in $\frak C$.  I trust it
will cause no confusion if I abuse notation slightly and treat
$\bigotimes_{i\in I}\frak A_i$ as actually a subalgebra of $\frak C$, so
that $\varepsilon_j:\frak A_j\to\frak C$ becomes identified with
$\tilde\varepsilon_j:\frak A_j\to\bigotimes_{i\in I}\frak A_i$.   Now the
Boolean homomorphisms $\phi_i:\frak A_i\to\frak B$ correspond to a
Boolean homomorphism $\theta:\bigotimes_{i\in I}\frak A_i\to\frak B$.
The point is that $\bar\nu\theta(c)=\bar\lambda c$ for every
$c\in\bigotimes_{i\in I}\frak A_i$.   \Prf\ Suppose to begin with that
$c\in C$.   Then we have $c=E^{\ssbullet}$, where $E=\{x:x(i)\in
E_i\Forall i\in J\}$ and $E_i\in\Sigma_i$ for each $i\in J$.   So

$$\eqalign{\bar\lambda c
&=\lambda E
=\prod_{i\in J}\mu E_i
=\prod_{i\in J}\bar\mu_iE_i^{\ssbullet}
=\bar\nu(\inf_{i\in J}\phi a_i)\cr
&=\bar\nu(\inf_{i\in J}\theta\varepsilon_i(a_i))
=\bar\nu\theta(\inf_{i\in J}\varepsilon_i(a_i))
=\bar\nu\theta(c).\cr}$$

\noindent Next, any $c\in\bigotimes_{i\in I}\frak A_i$
is expressible as the supremum of a
finite disjoint family $\langle c_k\rangle_{k\in K}$ in $C$ (315Kb),  so

\Centerline{$\bar\nu\theta(c)=\sum_{k\in K}\bar\nu\theta(c_k)
=\sum_{k\in K}\bar\lambda(c_k)=\bar\lambda c$.   \Qed}


\medskip

{\bf (c)} It follows that $\theta$ is uniformly continuous for the
measure metrics defined by $\bar\nu$ and $\bar\lambda$, since

\Centerline{$\bar\nu(\theta(c)\Bsymmdiff\theta(c'))
=\bar\nu\theta(c\Bsymmdiff c')
=\bar\lambda(c\Bsymmdiff c')$}

\noindent for all $c$, $c'\in\bigotimes_{i\in I}\frak A_i$.

\medskip

{\bf (d)} Next, $\bigotimes_{i\in I}\frak A_i$ is topologically dense in
$\frak C$.   \Prf\ Let $c\in\frak C$, $\epsilon>0$.   Express $c$ as
$W^{\ssbullet}$.  Then by 254Fe there are $H_0,\ldots,H_k\in\Cal C$ such
that $\lambda(W\symmdiff\bigcup_{j\le k}H_j)\le\epsilon$.   Now
$c_j=H_j^{\ssbullet}\in C$ for each $j$, so

\Centerline{$c'=\sup_{j\le k}c_j=(\bigcup_{j\le
k}H_j)^{\ssbullet}\in\bigotimes_{i\in I}\frak A_i$,}

\noindent and $\bar\lambda(c\Bsymmdiff c')\le\epsilon$.\ \Qed

Since $\frak B$ is complete for its uniformity (323Gc), there is a
uniformly continuous function $\phi:\frak C\to\frak B$ extending
$\theta$ (3A4G).

\medskip

{\bf (e)} Because $\theta$ is a Boolean homomorphism, so is $\phi$.
\Prf\ (i) The functions $c\mapsto\phi(1\Bsetminus c)$,
$1\Bsetminus\phi(c)$ are continuous and the topology of $\frak B$ is
Hausdorff, so  $\{c:\phi(1\Bsetminus c)=1\Bsetminus\phi(c)\}$ is closed;
as it includes $\bigotimes_{i\in I}\frak A_i$, it must be the whole of
$\frak C$.   (ii) The functions $(c,c')\mapsto\phi(c\Bcup c')$,
$(c,c')\mapsto\phi(c)\Bcup\phi(c')$ are continuous, so
$\{(c,c'):\phi(c\Bcup c')=\phi(c)\Bcup\phi(c')\}$ is closed in
$\frak C\times\frak C$;  as it includes
$\bigotimes_{i\in I}\frak A_I\times\bigotimes_{i\in I}\frak A_i$,
it must be the whole of $\frak C\times\frak C$.\ \Qed

\medskip

{\bf (f)} Because $\theta$ is measure-preserving, so is $\phi$.   \Prf\
The functions $c\mapsto\bar\lambda c$, $c\mapsto\bar\nu\phi(c)$ are
continuous and equal on $\bigotimes_{i\in I}\frak A_i$, so are equal on
$\frak C$.\ \Qed

\medskip

{\bf (g)} Finally, to see that $\phi$ is unique, observe that any
measure-preserving Boolean homomorphism from $\frak C$ to $\frak B$ must
be continuous,  so that if it agrees with $\phi$ on
$\bigotimes_{i\in I}\frak A_i$ it must agree with $\phi$ on $\frak C$.
}%end of proof of 325I

\leader{325J}{}\cmmnt{ Of course this leads at once to a result
corresponding to 325D.

\medskip

\noindent}{\bf Theorem} Let $\familyiI{(\frak A_i,\bar\mu_i)}$ be a
family of probability algebras.

(a) There is a probability algebra $(\frak C,\bar\lambda)$, together
with measure-preserving Boolean homomorphisms
$\varepsilon_i:\frak A_i\to\frak C$ for $i\in I$,
such that whenever $(\frak B,\bar\nu)$ is a probability algebra, and
$\phi_i:\frak A_i\to\frak B$ are
Boolean homomorphisms such that
$\bar\nu(\inf_{i\in J}\phi_i(a_i))=\prod_{i\in J}\bar\mu_ia_i$ whenever
$J\subseteq I$ is
finite and $a_i\in\frak A_i$ for each $i\in J$, then there is a unique
measure-preserving Boolean homomorphism $\phi:\frak C\to\frak B$ such
that $\phi\varepsilon_i=\phi_i$ for every $i\in I$.

(b) The structure $(\frak C,\bar\lambda,\familyiI{\varepsilon_i})$ is
determined up to isomorphism by this property.

(c) The Boolean homomorphism $\psi:\bigotimes_{i\in I}\frak A_i\to\frak C$
defined from the $\varepsilon_i$ is injective, and
$\psi[\bigotimes_{i\in I}\frak A_i]$ is topologically dense in $\frak C$.

\proof{ For (a) and (c), all we have to do is represent each $(\frak
A_i,\bar\mu_i)$ as the measure algebra of a probability space, and
apply 325I.   The uniqueness of $\frak C$ and the $\varepsilon_i$ follows from
the uniqueness of the homomorphisms $\phi$, as in 325Db.
}%end of proof of 325J

\leader{325K}{Definition} As in 325Ea, we can say that
$(\frak C,\bar\lambda,\langle\varepsilon_i\rangle_{i\in I})$
is\dvro{ the}{ a, or the,} {\bf probability algebra free product} of
$\langle(\frak A_i,\bar\mu_i)\rangle_{i\in I}$.

\leader{325L}{Independent subalgebras} If $(\frak A,\bar\mu)$ is a
probability algebra, we say that a family
$\langle\frak B_i\rangle_{i\in I}$ of subalgebras of $\frak A$ is
{\bf stochastically independent} if
$\bar\mu(\inf_{i\in J}b_i)=\prod_{i\in J}\bar\mu b_i$ whenever
$J\subseteq I$ is finite and $b_i\in\frak B_i$ for each $i$.
\cmmnt{(Compare
272Ab.)}   If every $\frak B_i$ is closed, so that
$(\frak B_i,\bar\mu\restrp\frak B_i)$ is a probability algebra,
the identity maps $\iota_i:\frak B_i\to\frak A$
satisfy the conditions of the universal mapping theorem 325Ja,
so we have a probability algebra free product
$(\frak C,\bar\mu\restrp\frak C,\familyiI{\iota_i})$ of
$\langle(\frak B_i,\bar\mu\restrp\frak B_i)\rangle_{i\in I}$, where
$\frak C=\bigvee_{i\in I}\frak B_i$ is the closed subalgebra
of $\frak A$ generated by $\bigcup_{i\in I}\frak B_i$.

Conversely, if $\familyiI{(\frak A_i,\bar\mu_i)}$ is any family of
probability algebras with probability algebra free product
$(\frak C,\bar\lambda,\familyiI{\varepsilon_i})$, then
$\familyiI{\varepsilon_i[\frak A_i]}$ is an independent family of
closed subalgebras of $\frak C$.   \cmmnt{(Compare 272J, 315Xp.)}


\leader{325M}{}\cmmnt{ We can now make a general trawl
through Chapters 25 and 27 seeking results which can be expressed in
the language of this section.   I give some in
325Xf-325Xi.   %325Xf 325Xg 325Xh 325Xi
Some ideas from \S254 which are thrown into sharper relief by a
reformulation are in the following theorem.

\medskip

\noindent}{\bf Theorem} Let
$\langle(\frak A_i,\bar\mu_i)\rangle_{i\in I}$ be a family of
probability algebras and
$(\frak C,\bar\lambda,\langle\varepsilon_i\rangle_{i\in I})$ their probability
algebra free product.   For
$J\subseteq I$ let $\frak C_J=\bigvee_{i\in J}\varepsilon_i[\frak A_i]$
be the closed subalgebra of $\frak C$
generated by $\bigcup_{i\in J}\varepsilon_i[\frak A_i]$.

(a) For any $J\subseteq I$,
$(\frak C_J,\bar\lambda\restrp\frak C_J,
   \langle\varepsilon_i\rangle_{i\in J})$
is a probability algebra free product of
$\langle(\frak A_i,\bar\mu_i)\rangle_{i\in J}$.

(b)(i)  For any $c\in\frak C$, there is a unique smallest $J_c\subseteq I$
such that $c\in\frak C_{J_c}$, and this $J_c$ is countable.

\quad(ii)\dvAnew{2008}
If $c$, $d\in\frak C$ and $c\Bsubseteq d$, then there is an
$e\in\frak C_{J_c\cap J_d}$ such that $c\Bsubseteq e\Bsubseteq d$.

(c) For any non-empty family $\Cal J\subseteq\Cal PI$,
$\bigcap_{J\in\Cal J}\frak C_J=\frak C_{\bigcap\Cal J}$.

\proof{{\bf (a)} If $(\frak B,\bar\nu,\langle\phi_i\rangle_{i\in J})$ is
any probability algebra free product of
$\langle(\frak A_i,\bar\mu_i)\rangle_{i\in J}$,
then we have a measure-preserving
homomorphism $\psi:\frak B\to\frak C$ such that $\psi\phi_i=\varepsilon_i$ for
every $i\in J$.   Because the subalgebra $\frak B_0$ of $\frak B$
generated by $\bigcup_{i\in J}\phi_i[\frak A_i]$ is topologically dense
in $\frak B$ (325Jc), and $\psi$ is continuous (324Kb),
$\bigcup_{i\in J}\varepsilon_i[\frak A_i]$ is topologically dense in
$\psi[\frak B]$;  also
$\psi[\frak B]$ is closed in $\frak C$ (324Kb again).   But this means
that $\psi[\frak B]$ is just the topological closure of
$\bigcup_{i\in I}\varepsilon_i[\frak A_i]$ and must be $\frak C_J$.
Thus $\psi$ is an isomorphism, and

\Centerline{$(\frak C_J,\bar\lambda\restrp\frak C_J,
   \family{i}{J}{\varepsilon_i})
=(\psi[\frak B],\bar\nu\psi^{-1},\family{i}{J}{\psi\phi_i})$}

\noindent also is a probability algebra free product of
$\langle(\frak A_i,\bar\mu_i)\rangle_{i\in J}$.

\medskip

{\bf (b)} As in 325J, we may suppose that each $(\frak A_i,\bar\mu_i)$
is the measure algebra of a probability space $(X_i,\Sigma_i,\mu_i)$,
and that $\frak C$ is the measure algebra of their product
$(X,\Lambda,\lambda)$.   For $J\subseteq I$ let $\Lambda_J$ be
the set of members of $\Lambda$ which are determined by coordinates
in $J$.   Then
$\{x:x(i)\in E\}\in\Lambda_J$ for every $i\in J$ and $E\in\Sigma_i$;  so
$\{U^{\ssbullet}:U\in\Lambda_J\}$ is a closed subalgebra of $\frak C$
including $\varepsilon_i[\frak A_i]$ for every $i\in J$, and therefore
including $\frak C_J$.   On the other hand, as observed in 254Ob,
any member of $\Lambda_J$ is approximated, in measure, by sets in the
$\sigma$-algebra $\Tau_J$ generated by sets of the form
$\{x:x(i)\in E\}$ where $i\in J$ and $E\in\Sigma_i$.   Of course
$\Tau_J\subseteq\Lambda_J$, so
$\{W^{\ssbullet}:W\in\Lambda_J\}=\{W^{\ssbullet}:W\in\Tau_J\}$ is the
closed subalgebra of $\frak C$ generated by
$\bigcup_{i\in J}\varepsilon_i[\frak A_i]$, which is $\frak C_J$.

\medskip

\quad{\bf (i)} Let $W\in\Lambda$ be such that $c=W^{\ssbullet}$.
By 254Rd, there is a smallest $J_c\subseteq I$ such that
$W\symmdiff U$ is negligible for some $U\in\Lambda_{J_c}$,
and $J_c$ is countable.   By the remarks above, $J_c$ is also the
unique smallest subset of $I$ such that $c\in\frak C_{J_c}$.

\medskip

\quad{\bf (ii)} Let $U\in\Lambda_{J_c}$, $V\in\Lambda_{J_d}$ be such that
$c=U^{\ssbullet}$ and $d=V^{\ssbullet}$.   We can think of
$\lambda$ as a product $\lambda'\times\lambda''$ where $\lambda'$ is the
product measure on $X'=\prod_{i\in J_d}X_i$ and $\lambda''$ is the product
measure on $X''=\prod_{i\in I\setminus J_d}X_i$ (254N).   Express $V$ as
$V_0\times X''$ where $V_0\subseteq X'$ belongs to the domain of
$\lambda'$ (254Ob).   Consider

\Centerline{$W_0=\{z:z\in X'$, $\{w:w\in X''$, $(z,w)\in U\}$ is not
$\lambda''$-negligible$\}$;}

\noindent then $W_0$ is measured by $\lambda'$, by Fubini's theorem
(252B or 252D).
Because $c\Bsubseteq d$, $U\setminus V$ is $\lambda$-negligible
and $W_0\setminus V_0$ is $\lambda'$-negligible, while $W_0$ is determined
by coordinates in $J_c\cap J_d$.   So $W=W_0\times X''$ also is determined
by coordinates in $J_c\cap J_d$, while $U\setminus W$ and $W\setminus V$
are $\lambda$-negligible.   We can therefore take $e=W^{\ssbullet}$.


\medskip

{\bf (c)} Of course $\frak C_K\subseteq\frak C_J$ whenever $K\subseteq
J\subseteq I$, so
$\bigcap_{J\in\Cal J}\frak C_J\supseteq\frak C_{\bigcap\Cal J}$.   On
the other hand, suppose that $c\in\bigcap_{J\in\Cal J}\frak C_J$;  then
by (b-i) there is some $K\subseteq\bigcap\Cal J$ such that $c\in\frak
C_K\subseteq\frak C_{\bigcap\Cal J}$.   As $c$ is arbitrary,
$\bigcap_{J\in\Cal J}\frak C_J=\frak C_{\bigcap\Cal J}$.
}%end of proof of 325M

\leader{*325N}{Notation} In this context, I will say that an element
$c$ of $\frak C$ is {\bf determined by coordinates in} $J$ if
$c\in\frak C_J$.

\exercises{
\leader{325X}{Basic exercises (a)}
%\spheader 325Xa
Let $(\frak A_1,\bar\mu_1)$, $(\frak A_2,\bar\mu_2)$ be two semi-finite
measure algebras, and suppose that
for each $j$ we are given a closed subalgebra $\frak B_j$ of $\frak A_j$
such that $(\frak B_j,\bar\nu_j)$ also is semi-finite, where
$\bar\nu_j=\bar\mu_j\restrp\frak B_j$.   Show that the localizable
measure algebra free product
$(\frak B_1,\bar\nu_1)\locmafp(\frak B_2,\bar\nu_2)$ can be thought of as
a closed subalgebra of
$(\frak A_1,\bar\mu_1)\locmafp(\frak A_2,\bar\mu_2)$.
%325E

\spheader 325Xb Let $(\frak A_1,\bar\mu_1)$ and
$(\frak A_2,\bar\mu_2)$ be two semi-finite measure algebras,
and suppose that
for each $j$ we are given a principal ideal $\frak B_j$ of $\frak A_j$.
Set $\bar\nu_j=\bar\mu_j\restrp\frak B_j$.   Show that the
localizable measure algebra free product
$(\frak B_1,\bar\nu_1)\locmafp(\frak B_2,\bar\nu_2)$ can be thought of as
a principal ideal of $(\frak A_1,\bar\mu_1)\locmafp(\frak A_2,\bar\mu_2)$.
%325E

\spheader 325Xc Let $(\frak A,\bar\mu)$ and $(\frak B,\bar\nu)$ be
semi-finite measure algebras with localizations
$(\widehat{\frak A},\tilde\mu)$ and $(\widehat{\frak B},\tilde\nu)$.
Show that the localizable measure algebra free products
$(\frak A,\bar\mu)\locmafp(\frak B,\bar\nu)$ and
$(\widehat{\frak A},\tilde\mu)\locmafp(\widehat{\frak B},\tilde\nu)$ are
isomorphic.
%325E

\sqheader 325Xd Let $\langle(\frak A_i,\bar\mu_i)\rangle_{i\in I}$ and
$\langle(\frak B_j,\bar\nu_j)\rangle_{j\in J}$ be families of
semi-finite measure algebras, with simple products $(\frak A,\bar\mu)$
and $(\frak B,\bar\nu)$ (322L).   Show that the localizable measure
algebra free product $(\frak A,\bar\mu)\locmafp(\frak B,\bar\nu)$ can be
identified with the simple product of the family
$\langle(\frak A_i,\bar\mu_i)\locmafp(\frak B_j,\bar\nu_j)
 \rangle_{i\in I,j\in J}$.
%325E, 325H

\sqheader 325Xe Let $\langle(\frak A_i,\bar\mu_i)\rangle_{i\in I}$
and $\familyiI{(\frak A_i',\bar\mu'_i)}$ be two
families of probability algebras, and
$(\frak C,\bar\lambda,\langle\varepsilon_i\rangle_{i\in I})$,
$(\frak C',\bar\lambda',\langle\varepsilon'_i\rangle_{i\in I})$ their
probability algebra free products.   Suppose that
for each $i\in I$ we are given a measure-preserving Boolean homomorphism
$\pi_i:\frak A_i\to\frak A'_i$.   Show that there is a unique
measure-preserving Boolean homomorphism $\pi:\frak C\to\frak C'$ such
that $\pi\varepsilon_i=\varepsilon'_i\pi_i$ for every $i\in I$.
%325K

\sqheader 325Xf Let $(\frak A,\bar\mu)$ be a probability
algebra.   We say that a family $\langle a_i\rangle_{i\in I}$ in
$\frak A$ is {\bf stochastically independent} if
$\bar\mu(\inf_{i\in J}a_i)=\prod_{i\in J}\bar\mu a_i$ for every non-empty
finite $J\subseteq I$.   Show that this is so iff
$\langle\frak A_i\rangle_{i\in I}$ is stochastically
independent, where $\frak A_i=\{0,a_i,1\Bsetminus a_i,1\}$ for each $i$.
(Compare 272F.)
%325L

\sqheader 325Xg Let $(\frak A,\bar\mu)$ be a probability
algebra, and $\langle\frak A_i\rangle_{i\in I}$ a stochastically
independent family of closed subalgebras of $\frak A$.   Let
$\langle J(k)\rangle_{k\in K}$ be a disjoint family of subsets of $I$,
and for each $k\in K$ let $\frak B_k=\bigvee_{i\in J(k)}\frak A_i$
be the closed subalgebra of $\frak A$
generated by $\bigcup_{i\in J(k)}\frak A_i$.   Show that
$\langle\frak B_k\rangle_{k\in K}$ is
stochastically independent.   (Compare 272K.)
%325L

\spheader 325Xh Let $(\frak A,\bar\mu)$ be a probability
algebra, and $\langle\frak A_i\rangle_{i\in I}$ a stochastically
independent family of closed subalgebras of $\frak A$.   For
$J\subseteq I$ set $\frak B_J=\bigvee_{i\in J}\frak A_i$.
Show that
$\bigcap\{\frak B_{I\setminus J}:J$ is a finite subset of $I\}=\{0,1\}$.
\Hint{For
$J\subseteq I$, show that $\bar\mu(b\Bcap c)=\bar\mu b\cdot\bar\mu c$
for every $b\in\frak B_{I\setminus J}$ and $c\in \frak B_J$.   Compare
272O, 325M.}
%325L

\spheader 325Xi Let
$\langle(\frak A_i,\bar\mu_i)\rangle_{i\in I}$ be a family of
probability algebras with probability algebra free
product $(\frak C,\bar\lambda,\langle\varepsilon_i\rangle_{i\in I})$.   For
$J\subseteq I$ set $\frak C_J=\bigvee_{i\in J}\varepsilon_i[\frak A_i]$.
Show that for any $J$, $K\subseteq I$ and $c\in\frak C$,
$\frak C_J\cap\frak C_K=\frak C_{J\cap K}$ and the upper envelope
$\upr(c,\frak C_{J\cap K})$ is equal to
$\upr(\upr(c,\frak C_J),\frak C_K)$.
%325M

\leader{325Y}{Further exercises (a)}
%\spheader 325Ya
Let $\mu$ be counting measure on $X=\{0\}$, $\mu'$ the
countable-cocountable measure on $X'=\omega_1$, and $\nu$ counting
measure on $Y=\omega_1$.   Show that the measure algebras of the
primitive product measures on $X\times Y$, $X'\times Y$ are not
isomorphic.
%325B

\spheader 325Yb Let $(\frak A_1,\bar\mu_1)$, $(\frak A_2,\bar\mu_2)$,
$(\frak A'_1,\bar\mu'_1)$ and $(\frak A'_2,\bar\mu'_2)$ be semi-finite
measure algebras with localizable measure algebra free products
$(\frak C,\bar\lambda,\varepsilon_1,\varepsilon_2)$ and
$(\frak C',\bar\lambda',\varepsilon'_1,\varepsilon'_2)$.   Suppose that
$\pi_1:\frak A_1\to\frak A'_1$ and $\pi_2:\frak A_2\to\frak A'_2$ are
measure-preserving Boolean homomorphisms.   Show that there is a
measure-preserving Boolean homomorphism $\pi:\frak C\to\frak C'$ such that
$\pi\varepsilon_i=\varepsilon'_i\pi_i$ for both $i$, but that $\pi$ is not necessarily
unique.
%325C

\spheader 325Yc
Let $\frak A$ be a Boolean algebra, and
$\mu:\frak A\to[0,\infty]$ a functional such that $\mu 0=0$, $\mu a>0$ for
every $a\ne 0$, and
$\mu(a\Bcup b)=\mu a+\mu b$ whenever $a$, $b\in \frak A$ and
$a\Bcap b=0$;   suppose that $\frak A^f=\{a:\mu a<\infty\}$ is
order-dense in $\frak A$.  For $e\in\frak A^f$, $a$, $b\in\frak A$ set
$\rho_e(a,b)=\mu(e\Bcap(a\Bsymmdiff b))$.   Give $\frak A$
the uniformity defined by $\{\rho_e:\mu e<\infty\}$.   (i) Show that the
completion $\widehat{\frak A}$ of $\frak A$ under this uniformity has a
measure $\hat\mu$, extending
$\mu$, under which it is a localizable measure algebra.   (ii) Show that
if $a\in\widehat{\frak A}$, $\hat\mu a<\infty$ and $\epsilon>0$, there
is a $b\in\frak A$ such that $\hat\mu(a\Bsymmdiff b)\le\epsilon$.
(iii) Show that for every $a\in\widehat{\frak A}$ there is a sequence
$\sequencen{a_n}$ in $\frak A$ such that
$a\Bsupseteq\sup_{n\in\Bbb N}\inf_{m\ge n}a_m$ and
$\hat\mu a=\hat\mu(\sup_{n\in\Bbb N}\inf_{m\ge n}a_m)$.
(iv) In particular, the set of infima in $\widehat{\frak A}$
of sequences in $\frak A$ is
order-dense in $\widehat{\frak A}$.   (v) Explain the relevance of this
construction to the embedding
$\frak A_1\otimes\frak A_2\embedsinto\frak C$ in 325D.
%325D

\spheader 325Yd In 325F, set $W=\bigcup_{n\in\Bbb N}E_n\times E_n$.   Show that if $A$, $B$ are any non-negligible subsets
of $[0,1]$, then $W\cap(A\times B)$ is not negligible.
%325F

\spheader 325Ye Let $(\frak A,\bar\mu)$ be the measure algebra of
Lebesgue measure on $[0,1]$.   Show that $\frak A\otimes\frak A$ is ccc
but not \wsid.   \Hint{(i) $\frak A\otimes\frak A$ is embeddable as a
subalgebra of a probability algebra (ii) in the notation of 325F, look
at $c_{mn}=\sup_{m\le i\le n}e_i\otimes e_i$.}
%325F

\spheader 325Yf Repeat 325F-325G and 325Yd-325Ye with an arbitrary
atomless probability space in place of $[0,1]$.
%325G, 325Yd

\spheader 325Yg Let $(\frak A,\bar\mu)$ be a probability algebra and
$\langle a_i\rangle_{i\in I}$ a stochastically independent family in
$\frak A$.   Show that for any $a\in\frak A$ and $\epsilon>0$ the set
$\{i:i\in I,\,
  |\bar\mu(a\Bcap a_i)-\bar\mu a\cdot\bar\mu a_i|\ge\epsilon\}$ is finite,
so that
$\{i:\bar\mu(a\Bcap a_i)\ne\bar\mu a\cdot\bar\mu a_i\}$ is countable.
\Hint{272Ye\formerly{2{}72Yd}.}
%325M
}%end of exercises

\endnotes{
\Notesheader{325} 325B shows that the measure algebra of a product
measure may be irregular if we have factor measures which are not
strictly localizable.   But two facts lead the way
to the `localizable measure algebra free product' in 325D-325E.   The
first is that every semi-finite measure algebra is embeddable, in a
canonical way, in a localizable measure algebra (322P);  and the second
is that the Stone representation of a localizable measure algebra is
strictly localizable (322O).   It is a happy coincidence that we can
collapse these two facts together in the construction of 325D.   Another
way of looking at the localizable measure algebra free product of two
localizable measure algebras is to express it as the simple product of
measure algebra free products of totally finite measure algebras, using
325Xd and the fact that for $\sigma$-finite measure algebras there is
only one reasonable measure algebra free product, being that provided
by any representation of them as measure algebras of measure spaces
(325Eb).

Yet a third way of approaching measure algebra free products is as the
uniform space completions of algebraic free products, using 325Yc.
This gives the same result as the construction of 325D because the
algebraic free product appears as a topologically dense subalgebra of
the localizable measure algebra free product, which is complete
as uniform space (325Dc).
(I have to repeat such phrases as `topologically dense' because the
algebraic free product is emphatically {\it not} order-dense in the
measure algebra free product (325F).)   The results in 251I on
approximating measurable sets for a c.l.d.\ product measure by
combinations of measurable rectangles correspond to general facts about
completions of finitely-additive measures (325Yc(ii), 325Yc(iii)).
It is worth noting that the completion process can be regarded as made
up of two steps;  first take infima of sequences of sets of finite
measure, and then take arbitrary suprema (325Yc(iv)).

The idea of 325F appears in many guises, and this is only the first time
that I shall wish to call on it.   The point of the set
$W=\bigcup_{n\in\Bbb N}E_n\times E_n$ is that it is a measurable subset
of the square (indeed, by taking the $E_n$ to be open sets we can
arrange that $W$ should be open), of measure strictly less than $1$ (in
fact, as small as we wish), such that its complement does not include
any non-negligible `measurable rectangle' $G\times H$;   indeed,
$W\cap(A\times B)$ is non-negligible for any non-negligible sets $A$,
$B\subseteq[0,1]$ (325Yd).   I believe that the first published example
of such a set was by {\smc Erd\H{o}s \& Oxtoby 55} (a version of which is in
532N in Volume 5);  I learnt the method of 325F from R.O.Davies.

I include 325G as a kind of guard-rail.   The relationship between
preservation of measure and order-continuity is a subtle one, as I have
already tried to show in 324K, and it is often worth considering the
possibility that a result involving order-continuous measure-preserving
homomorphisms has a form applying to all order-continuous homomorphisms.
However, there is no simple expression of such an idea in the present
context.

In the context of infinite free products of probability algebras, there
is a degree of simplification, since there is only one algebra which can
plausibly be called the probability algebra free product, and this is
produced by any realization of the algebras as measure algebras of
probability spaces (325I-325K).   The examples 325F-325G apply equally,
of course, to this context.   At this point I mention the concept of
`stochastically independent' family (325L, 325Xf) because we have the
machinery to translate several results from \S272 into the language of
measure algebras (325Xf-325Xh).   I feel that I have to use the phrase
`stochastically independent' here because there is the much weaker
alternative concept of `Boolean independence' (315Xp) also present.
But I leave most of this as exercises, because the language of measure
algebras offers few ideas to the probability theory already covered in
Chapter 27.   All it can do is formalise the ever-present principle that
negligible sets often can and should be ignored.
}%end of notes

\discrpage

