\frfilename{mt111.tex}
\versiondate{26.1.05}
\copyrightdate{1994}

\def\chaptername{Measure spaces}
\def\sectionname{$\sigma$-algebras}

\newsection{111}

In the introduction to this chapter I remarked that a measure space is
`a set in which some (not, as a rule, all) subsets may be assigned a
measure'.   All ordinary concepts of `length' or `area' or `volume'
apply only to reasonably regular sets.   Modern measure theory
is remarkably powerful in that an extraordinary variety of sets are
regular enough to be measured;  but we must still expect some
limitation, and when studying any measure a proper understanding of the
class of sets which it measures will be central to our work.   The basic
definition here is that of `$\sigma$-algebra of sets';  all measures
in the standard theory are defined on such collections.   I
therefore begin with a statement of the definition, and a brief
discussion of the properties, of these classes.


\leader{111A}{Definition} Let $X$ be a set.   A {\bf $\sigma$-algebra of
subsets of $X$}\cmmnt{ (sometimes called a {\bf $\sigma$-field})} is a
family $\Sigma$ of subsets of $X$ such that

\qquad(i) $\emptyset\in\Sigma$;

\qquad(ii) for every $E\in\Sigma$, its complement $X\setminus E$ in $X$
belongs to $\Sigma$;

\qquad(iii) for every sequence $\langle E_n\rangle_{n\in\Bbb N}$ in
$\Sigma$, its union $\bigcup_{n\in\Bbb N}E_n$ belongs to $\Sigma$.


\cmmnt{
\leader{111B}{Remarks (a)} Almost any new subject in pure
mathematics is
likely to begin with definitions.   At this point there is no substitute
for rote learning.   These definitions encapsulate years, sometimes
centuries, of thought by many people;  you cannot expect that they will
always correspond to familiar ideas.


\header{111Bb}{\bf (b)} Nevertheless, you should always seek immediately
to find ways
of making new definitions more concrete by finding examples within your
previous mathematical experience.   In the case of `$\sigma$-algebra',
the really important examples, to be described below, are going to be
essentially new -- supposing, that is, that you need to read this
chapter at all.   However, two examples should be immediately accessible
to you, and you should bear these in mind henceforth:

\qquad(i) for any $X$, $\Sigma=\{\emptyset,X\}$ is a $\sigma$-algebra of
subsets of $X$;

\qquad(ii) for any $X$, $\Cal PX$, the set of all subsets of $X$, is a
$\sigma$-algebra of subsets of $X$.

\noindent These are of course the smallest and largest $\sigma$-algebras
of subsets of $X$, and while we shall spend little time with them, both
are in
fact significant.


\header{111Bc}{\bf *(c)} The phrase {\bf measurable space} is often
used to mean a pair $(X,\Sigma)$, where $X$ is a set and $\Sigma$ is a
$\sigma$-algebra of subsets of $X$;  but I myself prefer to avoid this
terminology, unless greatly pressed for time, as in fact many of the
most interesting examples of such
objects have no useful measures associated with them.
}%end of comment

\cmmnt{
\leader{111C}{Infinite unions and intersections} If you have not
seen infinite unions before, it is worth
pausing over the formula $\bigcup_{n\in\Bbb N}E_n$.   This is the set of
points belonging to one or more of the sets $E_n$;  we may write it as

$$\eqalign{\bigcup_{n\in\Bbb N}E_n
&=\{x:\exists\enskip n\in\Bbb N,\,x\in E_n\}\cr
&=E_0\cup E_1\cup E_2\cup\ldots.\cr}$$

\noindent (I write $\Bbb N$ for the set of natural numbers
$\{0,1,2,3,\ldots\}$.)   In the same way,

$$\eqalign{\bigcap_{n\in\Bbb N}E_n
&=\{x:x\in E_n\Forall n\in\Bbb N\}\cr
&=E_0\cap E_1\cap E_2\cap\ldots.\cr}$$

\noindent It is characteristic of the elementary theory of measure
spaces that it demands greater facility with the set-operations $\cup$,
$\cap$, $\setminus$ (`set difference':  $E\setminus F=\{x:x\in
E,\,x\notin F\}$), $\symmdiff$ (`symmetric difference':  $E\symmdiff
F=(E\setminus F)\cup(F\setminus E)=(E\cup F)\setminus(E\cap F)$) than
you have probably needed before, with the added complication of infinite
unions and intersections.   I strongly advise spending at least a little
time with Exercise 111Xa at some point.
}%end of comment

\leader{111D}{Elementary properties of $\sigma$-algebras} If $\Sigma$ is
a $\sigma$-algebra of subsets of $X$, then it
has the following properties.

\header{111Da}{\bf (a)} $E\cup F\in\Sigma$ for all $E$, $F\in\Sigma$.
\prooflet{\Prf\ For
if $E$, $F\in\Sigma$, set $E_0=E$, $E_n=F$ for $n\ge 1$;  then
$\langle E_n\rangle_{n\in\Bbb N}$ is a sequence in $\Sigma$ and $E\cup
F=\bigcup_{n\in\Bbb N}E_n\in\Sigma$.\ \Qed}


\header{111Db}{\bf (b)} $E\cap F\in\Sigma$ for all $E$, $F\in\Sigma$.
\prooflet{\Prf\ By
(ii) of the definition in 111A, $X\setminus E$ and $X\setminus
F\in\Sigma$;  by (a) of this paragraph, $(X\setminus E)\cup(X\setminus
F)\in\Sigma$;  by 111A(ii) again, $X\setminus((X\setminus
E)\cup(X\setminus F))\in\Sigma$;  but this is just $E\cap F$.\ \Qed}


\header{111Dc}{\bf (c)} $E\setminus F\in\Sigma$ for all $E$,
$F\in\Sigma$.
\prooflet{\Prf\ $E\setminus F=E\cap(X\setminus F)$.\ \Qed}


\header{111Dd}{\bf (d)} Now suppose that
$\langle E_n\rangle_{n\in\Bbb N}$ is a
sequence in $\Sigma$, and consider

$$\eqalign{\bigcap_{n\in\Bbb N}E_n
&=\{x:x\in E_n\Forall n\in\Bbb N\}\cr
&=E_0\cap E_1\cap E_2\cap\ldots\cr
&=X\setminus\bigcup_{n\in\Bbb N}(X\setminus E_n);\cr}$$

\noindent this also belongs to $\Sigma$.


\cmmnt{
\leader{111E}{More on infinite unions and
intersections (a)} So far I
have considered infinite unions and intersections only in the context of
sequences $\langle
E_n\rangle_{n\in\Bbb N}$ indexed by the set $\Bbb N$ of natural numbers
itself.   Many others will arise more or less naturally in the pages
ahead.   Consider, for instance, sets of the form

\Centerline{$\bigcup_{n\ge 4}E_n=E_4\cup E_5\cup E_6\cup\ldots$,}

\Centerline{$\bigcup_{n\in\Bbb Z}E_n
=\{x:\exists\enskip n\in\Bbb Z,\,x\in E_n\}
=\ldots\cup E_{-2}\cup E_{-1}\cup E_0\cup E_1\cup E_2\cup\ldots$,}

\Centerline{$\bigcup_{q\in\Bbb Q}E_q
=\{x:\exists\enskip q\in\Bbb Q,\,x\in E_q\}$,}

\noindent where I write $\Bbb Z$ for the set of all integers and
$\Bbb Q$
for the set of rational numbers.   If every $E_n$, $E_q$ belongs to a
$\sigma$-algebra $\Sigma$, so will these unions.   On the other hand,

\Centerline{$\bigcup_{t\in[0,1]}E_t
=\{x:\exists\enskip t\in [0,1],\,x\in E_t\}$}

\noindent may fail to belong to a $\sigma$-algebra containing every
$E_t$, and it is of the greatest importance to develop an intuition for
those index sets, like $\Bbb N$, $\Bbb Z$ and $\Bbb Q$, which are `safe'
in this context, and those which are not.

\spheader 111Eb I rather hope that you
have seen enough of Cantor's theory of infinite sets to make the
following remarks a restatement of familiar material;  but if not, I
hope that they can stand as a first, and very partial, introduction to
these ideas.   The point about the first three examples is that we
can re-index the families of sets involved as simple sequences of sets.
For the first one,
this is elementary;  write $E'_n=E_{n+4}$ for $n\in\Bbb N$, and see that
$\bigcup_{n\ge 4}E_n=\bigcup_{n\in\Bbb N}E'_n\in\Sigma$.
For the other
two, we need to know something about the sets $\Bbb Z$ and
$\Bbb Q$.   We can find sequences $\langle k_n\rangle_{n\in\Bbb N}$ of
integers, and $\langle q_n\rangle_{n\in\Bbb N}$ of rational numbers,
such that every integer appears (at least once) as a $k_n$, and
every rational number appears (at least once) as a $q_n$;  that is, the
functions $n\mapsto k_n:\Bbb N\to\Bbb Z$ and
$n\mapsto q_n:\Bbb N\to\Bbb Q$ are surjective.   \prooflet{\Prf\ There are many ways of doing this;  one is to set

$$\eqalign{k_n&=\Bover{n}2\text{ for even }n,\cr
&=-\Bover{n+1}2\text{ for odd }n,\cr
q_n&=\Bover{n-m^3-m^2}{m+1}
  \text{ if }m\in\Bbb N\text{ and }m^3\le n<(m+1)^3.\cr}$$

\noindent (You should check carefully that these formulae do indeed do
what I claim they do.) \QeD}
Now, to deal with $\bigcup_{n\in\Bbb Z}E_n$, we can set

\Centerline{$E'_n=E_{k_n}\in\Sigma$}

\noindent for $n\in\Bbb N$, so that

\Centerline{$\bigcup_{n\in\Bbb Z}E_n=\bigcup_{n\in\Bbb N}E_{k_n}
=\bigcup_{n\in\Bbb N}E'_n\in\Sigma$,}

\noindent while for the other case we have

\Centerline{$\bigcup_{q\in\Bbb Q}E_q
=\bigcup_{n\in\Bbb N}E_{q_n}\in\Sigma$.}

Note that the first case $\bigcup_{n\ge 4}E_n$ can be thought of
as an
application of the same principle;  the map $n\mapsto n+4$ is a
surjection from $\Bbb N$ onto $\{4,5,6,7,\ldots\}$.
}%end of comment

\vleader{48pt}{111F}{Countable sets (a)}\cmmnt{ The common feature of the
sets
$\{n:n\ge 4\}$, $\Bbb Z$ and
$\Bbb Q$ which makes this procedure possible is that they are `countable'.
For our purposes here, the most natural definition of
countability is the following:} \dvro{A}{a} set $K$ is {\bf countable} if
either it is empty or there is a surjection from $\Bbb N$ onto $K$.   In
this case, if $\Sigma$ is a $\sigma$-algebra of sets and
$\langle  E_k\rangle_{k\in K}$ is a family in $\Sigma$ indexed by $K$, then
$\bigcup_{k\in K}E_k\in\Sigma$.   \prooflet{\Prf\ For if
$n\mapsto k_n:\Bbb N\to K$ is a surjection, then $E'_n=E_{k_n}\in\Sigma$ for every
$n\in\Bbb N$, and $\bigcup_{k\in K}E_k=\bigcup_{n\in\Bbb N}E'_n\in\Sigma$.   This
leaves out the case $K=\emptyset$;  but in this case the natural
interpretation of $\bigcup_{k\in K}E_k$ is

\Centerline{$\{x:\exists\enskip k\in \emptyset,\,x\in E_k\}$}

\noindent which is itself $\emptyset$, and therefore belongs to $\Sigma$
by clause (i) of 111A.\ \QeD}   \cmmnt{(In a sense this treatment of
$\emptyset$
is a conventional matter;
but there are various contexts in which we shall wish to discuss
$\bigcup_{k\in K}E_k$ without checking whether $K$ actually has any
members, and we need to be clear about what we will do in such cases.)}

\header{111Fb}{\bf (b)}\cmmnt{ There is an extensive, and enormously
important, theory concerning countable sets.   The only fragments which
I think we must have explicit at this point are the following.
(In \S1A1 I add a few words to link this presentation to conventional
approaches.)

\medskip

\quad}{\bf (i)}  If $K$ is countable and
$L\subseteq K$, then $L$ is countable.   \prooflet{\Prf\ If
$L=\emptyset$, this is
immediate.   Otherwise, take any $l^*\in L$, and a surjection
$n\mapsto k_n:\Bbb N\to K$ (of course $K$ also is not empty, as $l^*\in K$);  set
$l_n=k_n$ if $k_n\in L$, $l^*$ otherwise;  then $n\mapsto l_n:\Bbb N\to
L$ is a surjection.\ \Qed}

\medskip

\quad{\bf (ii)} The Cartesian product $\Bbb N\times\Bbb N=\{(m,n):m$,
$n\in\Bbb N\}$ is countable.   \prooflet{\Prf\ For each $n\in\Bbb N$,
let $k_n$,
$l_n\in\Bbb N$ be such that $n+1=2^{k_n}(2l_n+1)$;  that is, $k_n$ is
the power of $2$ in the prime factorisation of $n+1$, and $2l_n+1$ is
the (necessarily odd) number $(n+1)/2^{k_n}$.   Now
$n\mapsto(k_n,l_n)$ is a
surjection from $\Bbb N$ to $\Bbb N\times\Bbb N$.\ \Qed}\
\cmmnt{It will be
important to us later to know that $n\mapsto(k_n,l_n)$ is actually a
bijection, as is readily checked.}

\medskip

\quad{\bf (iii)} It follows that if $K$ and $L$ are countable sets, so
is $K\times L$.   \prooflet{\Prf\ If either $K$ or $L$ is empty, so is
$K\times
L$, so in this case $K\times L$ is certainly countable.   Otherwise, let
$\phi:\Bbb N\to K$ and $\psi:\Bbb N\to L$ be surjections;  then we have
a surjection $\theta:\Bbb N\times\Bbb N\to K\times L$ defined by setting
$\theta(m,n)=(\phi(m),\psi(n))$ for all $m$, $n\in\Bbb N$.   Now we know
from (ii) just above that there is also a surjection
$\chi:\Bbb N\to\Bbb N\times\Bbb N$, so that
$\theta\chi:\Bbb N\to K\times L$ is a
surjection, and $K\times L$ must be countable.\ \Qed}

\medskip

\quad{\bf (iv)}\dvro{ If}{ An induction on $r$ now shows
us that if} $K_1$, $K_2,\ldots,K_r$ are countable sets, so is
$K_1\times\ldots\times K_r$.   In particular,\cmmnt{ such sets as}
$\Bbb Q^r\times\Bbb Q^r$ will be countable, for any integer $r\ge 1$.


\header{111Fc}{\bf (c)}\dvro{ If}{ Putting 111Dd above together with
these
ideas, we see that if} $\Sigma$ is a $\sigma$-algebra of sets, $K$ is a
non-empty countable
set, and $\langle E_k\rangle_{k\in K}$ is a family in $\Sigma$, then

\Centerline{$\bigcap_{k\in K}E_k=\{x:x\in E_k\Forall k\in
K\}$}

\noindent belongs to $\Sigma$.   \prooflet{\Prf\ Let
$n\mapsto k_n:\Bbb N\to K$ be
a surjection;  then
$\bigcap_{k\in K}E_k=\bigcap_{n\in\Bbb N}E_{k_n}\in\Sigma$, as in 111Dd.\ \Qed}

\cmmnt{Note that there is a difficulty with the notion of
$\bigcap_{k\in
K}E_k$ if $K=\emptyset$;  the natural interpretation of this formula is
to read it as the universal class.   So ordinarily, when there is any
possibility that $K$ might be empty, one needs some such formulation as
$X\cap\bigcap_{k\in K}E_k$.
}%end of comment

\cmmnt{\header{111Fd}{\bf (d)} As an example of the way in which these
ideas will be used, consider the
following.   Suppose that $X$ is a set, $\Sigma$ is a $\sigma$-algebra
of subsets of $X$, and $\langle E_{qn}\rangle_{q\in\Bbb Q,n\in\Bbb N}$
is a family in $\Sigma$.   Then

\Centerline{$E
=\bigcap_{q\in\Bbb Q,q<\sqrt{2}}\bigcup_{m\in\Bbb N}\bigcap_{n\ge m}E_{qn}
=\bigcap_{q\in\Bbb Q,q<\sqrt{2}}(\bigcup_{m\in\Bbb N}(\bigcap_{n\ge m}E_{qn}))
\in\Sigma$.}

\noindent\vthsp\prooflet{\Prf\ Set
$F_{qm}=\bigcap_{n\ge m}E_{qn}=\bigcap_{n\in\Bbb N}E_{q,m+n}$ for $q\in\Bbb Q$
and $m\in\Bbb N$;
then every $F_{qm}$ belongs to $\Sigma$, by 111Dd or (c) above.   Set
$G_q=\bigcup_{m\in\Bbb N}F_{qm}$ for $q\in\Bbb Q$;  then every $G_q$
belongs to $\Sigma$, by 111A(iii).   Set
$K=\{q:q\in\Bbb Q,\,q<\sqrt{2}\}$;
then $K$ is countable, by 111E and (b-i) of this paragraph.   So
$\bigcap_{q\in K}G_q$ belongs to $\Sigma$, by (c).   But
$E=\bigcap_{q\in K}G_q$.\ \Qed}
}%end of comment

\cmmnt{\header{111Fe}{\bf (e)} And one final remark, which I give
without proof
here -- though many proofs will be implicit in the work below, and I
spell one out in 1A1Ha --

\medskip

\centerline{\bf The set $\Bbb R$ of real numbers is not countable.}

\medskip

\noindent So you must resist any temptation to look for a list $a_0$,
$a_1,\ldots$ running over the whole set of real numbers.
}%end of comment

\leader{111G}{Borel sets }\cmmnt{I can describe here one type of
non-trivial
$\sigma$-algebra;  the formulation is rather abstract, but the technique
is important
and the terminology is part of the basic vocabulary of measure theory.

\medskip

}{\bf (a)} Let $X$ be a set, and let $\frak S$ be any
non-empty family of
$\sigma$-algebras of subsets of $X$.   \cmmnt{(Thus a {\it member} of
$\frak S$ is itself a {\it family} of sets;
$\frak S\subseteq\Cal P(\Cal PX)$.)}   Then

\Centerline{$\bigcap\frak S=\{E:E\in\Sigma$ for every $\Sigma\in\frak
S\}$,}

\noindent the intersection of all the $\sigma$-algebras belonging to
$\frak S$, is a $\sigma$-algebra of subsets of $X$.   \prooflet{\Prf\
(i) By hypothesis, $\frak S$ is not empty;  take $\Sigma_0\in\frak S$;
then $\bigcap\frak S\subseteq\Sigma_0\subseteq\Cal PX$, so every member
of $\bigcap\frak S$ is a subset of $X$.   (ii) $\emptyset\in\Sigma$ for
every $\Sigma\in\frak S$,
so $\emptyset\in\bigcap\frak S$.   (iii) If
$E\in\bigcap\frak S$ then $E\in\Sigma$ for every $\Sigma\in\frak S$, so
$X\setminus E\in\Sigma$ for every $\Sigma\in\frak S$ and $X\setminus
E\in\bigcap\frak S$.   (iv) Let $\langle E_n\rangle_{n\in\Bbb N}$ be any
sequence in $\bigcap\frak S$.   Then for every $\Sigma\in\frak S$,
$\langle E_n\rangle_{n\in\Bbb N}$ is a sequence in $\Sigma$, so
$\bigcup_{n\in\Bbb N}E_n\in\Sigma$;  as $\Sigma$ is arbitrary,
$\bigcup_{n\in\Bbb N}E_n\in\bigcap\frak S$.\ \Qed}

\header{111Gb}{\bf (b)} Now let $\Cal A$ be any family of subsets of
$X$.   Consider

\Centerline{$\frak S=\{\Sigma:\Sigma$ is a $\sigma$-algebra of subsets
of $X$, $\Cal A\subseteq\Sigma\}$.}

\noindent\cmmnt{By definition, $\frak S$ is a family of
$\sigma$-algebras of
subsets of $X$;  also, it is not empty, because $\Cal PX\in\frak S$.
So} $\Sigma_{\Cal A}=\bigcap\frak S$ is a $\sigma$-algebra of subsets of
$X$\cmmnt{.   Because $\Cal A\subseteq\Sigma$ for every
$\Sigma\in\frak S$, $\Cal A\subseteq\Sigma_{\Cal A}$;  thus
$\Sigma_{\Cal A}$ itself
belongs to $\frak S$};  it is the smallest $\sigma$-algebra of subsets
of $X$ including $\Cal A$.

We say that $\Sigma_{\Cal A}$ is the $\sigma$-algebra of subsets of $X$
{\bf generated by} $\Cal A$.

\medskip

\noindent{\bf Examples (i)} For any $X$, the $\sigma$-algebra of subsets
of $X$ generated by $\emptyset$ is $\{\emptyset,X\}$.

\medskip

\quad{\bf (ii)} The
$\sigma$-algebra of subsets of $\Bbb N$ generated by
$\{\{n\}:n\in\Bbb N\}$ is $\Cal P\Bbb N$.

\header{111Gc}{\bf (c)(i)} We say that a set $G\subseteq\Bbb R$ is {\bf
open} if for
every $x\in G$ there is a $\delta>0$ such that the open interval
$\ooint{x-\delta,x+\delta}$ is included in $G$.

\medskip

\quad{\bf (ii)} Similarly, for any $r\ge 1$, we say that a set
$G\subseteq\BbbR^r$ is {\bf open} in $\BbbR^r$ if for every $x\in G$
there is a $\delta>0$ such that $\{y:\|y-x\|<\delta\}\subseteq G$,
where for $z=(\zeta_1,\ldots,\zeta_r)\in\BbbR^r$ I write
$\|z\|=\sqrt{\sum_{i=1}^r|\zeta_i|^2}$\cmmnt{;  thus $\|y-x\|$ is just
the ordinary Euclidean distance from $y$ to $x$}.

\header{111Gd}{\bf (d)} Now the {\bf Borel sets} of $\Bbb R$, or of
$\BbbR^r$, are
just the members of the $\sigma$-algebra of subsets of $\Bbb R$ or
$\BbbR^r$ generated by the family of open sets of $\Bbb R$ or $\BbbR^r$;
the $\sigma$-algebra itself is called the {\bf Borel $\sigma$-algebra}
in each case.

\cmmnt{\header{111Ge}{\bf (e)} Some readers will rightly feel that the
development here gives
very little idea of what a Borel set is `really' like.   (Open sets are
much easier;  see 111Ye.)   In fact the
importance of the concept derives largely from the fact that there are
alternative, more explicit, and in a sense more concrete, ways of
describing Borel sets.   I shall return to this topic in Chapter 42 in Volume 4.
}%end of comment

\exercises{
\leader{111X}{Basic exercises $\pmb{>}$(a)} Practise the algebra of
infinite unions and
intersections until you can confidently interpret such formulae as

\Centerline{$E\cap(\bigcup_{n\in\Bbb N}F_n)$,
  \qquad$\bigcup_{n\in\Bbb N}(E_n\setminus F)$,
  \qquad$E\cup(\bigcap_{n\in\Bbb N}F_n)$,}

\Centerline{$\bigcup_{n\in\Bbb N}(E\setminus F_n)$,
  \qquad$E\setminus(\bigcup_{n\in\Bbb N}F_n)$,
  \qquad$\bigcap_{n\in\Bbb N}(E_n\setminus F)$,}

\Centerline{$E\setminus(\bigcap_{n\in\Bbb N}F_n)$,
  \qquad$\bigcap_{n\in\Bbb N}(E\cup F_n)$,
  \qquad$(\bigcup_{n\in\Bbb N}E_n)\setminus F$,}

\Centerline{$\bigcup_{n\in\Bbb N}(E\cap F_n)$,
  \qquad$(\bigcap_{n\in\Bbb N}E_n)\setminus F$,
  \qquad$\bigcap_{n\in\Bbb N}(E\setminus F_n)$,}

\Centerline{$(\bigcup_{n\in\Bbb N}E_n)\cap(\bigcup_{n\in\Bbb N}F_n)$,
  \qquad$\bigcap_{m,n\in\Bbb N}(E_m\setminus F_n)$,
  \qquad$(\bigcap_{n\in\Bbb N}E_n)\cup(\bigcap_{n\in\Bbb N}F_n)$,}

\Centerline{$\bigcap_{m,n\in\Bbb N}(E_m\cup F_n)$,
  \qquad$(\bigcap_{n\in\Bbb N}E_n)\setminus(\bigcup_{n\in\Bbb N}F_n)$,
  \qquad$\bigcup_{m,n\in\Bbb N}(E_m\cap F_n)$,}

\noindent and, in particular, can identify the nine pairs into which
these formulae naturally fall.

\sqheader 111Xb In $\Bbb R$, show that all `open intervals'
$\ooint{a,b}$, $\ooint{-\infty,b}$, $\ooint{a,\infty}$ are open sets,
and that all intervals (bounded or unbounded, open, closed or half-open)
are Borel sets.

\sqheader 111Xc Let $X$ and $Y$ be sets and $\Sigma$ a
$\sigma$-algebra of subsets of $X$.   Let $\phi:X\to Y$ be a function.
Show that $\{F:F\subseteq Y,\,\phi^{-1}[F]\in\Sigma\}$ is a
$\sigma$-algebra of subsets of $Y$.   (See 1A1B for the notation here.)

\sqheader 111Xd Let $X$ and $Y$ be sets and $\Tau$ a
$\sigma$-algebra of subsets of $Y$.   Let $\phi:X\to Y$ be a function.
Show that $\{\phi^{-1}[F]:F\in\Tau\}$ is a
$\sigma$-algebra of subsets of $X$.

\spheader 111Xe Let $X$ be a set, $\Cal A$ a family of subsets of $X$,
and $\Sigma$ the $\sigma$-algebra of subsets of $X$ generated by $\Cal
A$.   Suppose that $Y$ is another set and $\phi:Y\to X$ a function.   Show
that $\{\phi^{-1}[E]:E\in\Sigma\}$ is the $\sigma$-algebra of subsets of
$Y$ generated by $\{\phi^{-1}[A]:A\in\Cal A\}$.

\spheader 111Xf Let $X$ be a set, $\Cal A$ a family of subsets of $X$,
and $\Sigma$ the $\sigma$-algebra of subsets of $X$ generated by $\Cal
A$.   Suppose that $Y\subseteq X$.   Show that $\{E\cap Y:E\in\Sigma\}$
is the $\sigma$-algebra of subsets of $Y$ generated by 
$\{A\cap Y:A\in\Cal A\}$.

\leader{111Y}{Further exercises (a)} In $\BbbR^r$, where $r\ge 1$, show
that $G+a=\{x+a:x\in G\}$ is open whenever $G\subseteq\BbbR^r$ is open
and $a\in\BbbR^r$.   Hence show that $E+a$ is a Borel set whenever
$E\subseteq\BbbR^r$ is a Borel set and $a\in\BbbR^r$.    \Hint{show that
$\{E:E+a\text{ is a Borel set}\}$ is a $\sigma$-algebra
containing all open sets.}

\header{111Yb}{\bf (b)} Let $X$ be a set, $\Sigma$ a $\sigma$-algebra of
subsets of $X$ and $A$ any subset of $X$.   Show that $\{(E\cap
A)\cup(F\setminus A):E,\,F\in\Sigma\}$ is a $\sigma$-algebra of subsets
of $X$, the $\sigma$-algebra generated by $\Sigma\cup\{A\}$.

\header{111Yc}{\bf (c)} Let $G\subseteq\BbbR^2$ be an open set.   Show
that all the horizontal and vertical sections

\Centerline{$\{\xi:(\xi,\eta)\in G\}$,
\quad$\{\xi:(\eta,\xi)\in G\}$}

\noindent of $G$ are open subsets of $\Bbb R$.

\header{111Yd}{\bf (d)} Let $E\subseteq\BbbR^2$ be a Borel set.   Show
that all the horizontal and vertical sections

\Centerline{$\{\xi:(\xi,\eta)\in E\}$,
\quad$\{\xi:(\eta,\xi)\in E\}$}

\noindent of $E$ are Borel subsets of $\Bbb R$.   \Hint{show
that the family of subsets of $\BbbR^2$ whose sections are all Borel
sets is a $\sigma$-algebra of subsets of $\BbbR^2$ containing all the
open sets.}

\spheader 111Ye Let $G\subseteq\Bbb R$ be an open set.   Show that $G$
is uniquely expressible as the union of a countable (possibly empty)
family $\Cal I$ of open intervals (the `components' of $G$) no two of
which have any point in common.   \Hint{for $x$,
$y\in G$ say that $x\sim y$ if every point between $x$ and $y$ belongs
to $G$.   Show that $\sim$ is an equivalence relation.   Let $\Cal I$ be
the set of equivalence classes.}
}%end of exercises

\endnotes{
\Notesheader{111} I suppose that the most important concept in this
section is the one introduced tangentially in 111E-111F, the idea of
`countable' set.   While it is possible to avoid much of the formal
theory of infinite sets for the time being, I do not think it is
possible to make sense of this chapter without a firm notion of the
difference between `finite' and `infinite', and some intuitions
concerning `countability'.   In particular, you must remember that
infinite sets are not, in general, countable, and that $\sigma$-algebras
are not, in general, closed under arbitrary unions.

The next thing to be sure of is that you can cope with the set-theoretic
manipulations here, so that such formulae as
$\bigcap_{n\in\Bbb N}E_n=X\setminus\bigcup_{n\in\Bbb N}(X\setminus E_n)$ (111Dd) are, if
not yet transparent, at least not alarming.   A large proportion of
the volume will be expressed in this language, and reasonable fluency is
essential.

Finally, for those who are looking for an actual idea to work on
straight away, I offer the concept of $\sigma$-algebra `generated' by a
collection $\Cal A$ (111G).   The point of the definition here is that
it involves consideration of a family $\frak S\in\Cal P(\Cal P(\Cal PX))$,
even though both $\Cal A$ and $\Sigma_{\Cal A}$ are subsets of
$\Cal PX$;  we need to work a layer or two up in the hierarchy of power sets.
You may have seen, for instance, the concept of `linear subspace $U$
generated by vectors $u_1,\ldots,u_n$'.   This can be defined as the
intersection of all the linear subspaces containing the vectors
$u_1,\ldots,u_n$, which is the method corresponding to that of 111Ga-b;
but it also has an `internal' definition, as the set of vectors
expressible as $\alpha_1u_1+\ldots+\alpha_nu_n$ for scalars $\alpha_i$.
For $\sigma$-algebras, however, there is no such simple `internal'
definition available (though there is a great deal to be said in this
direction which I think we are not yet ready for;  some ideas may be
found in \S136).   This is primarily
because of (iii) in the definition 111A;  a $\sigma$-algebra must be
closed under an infinitary operation, that is, the operation of union
applied to infinite sequences of sets.   By contrast, a linear subspace
of a vector space need be closed only under the finitary operations of
scalar multiplication and addition, each involving at most two vectors
at a time.
}%end of notes

\discrpage


