\frfilename{mt213.tex}
\versiondate{13.9.13}

\copyrightdate{2000}
\def\chaptername{Taxonomy of measure spaces}
\def\sectionname{Semi-finite, locally determined and localizable spaces}

\newsection{213}

In this section I collect a variety of useful facts concerning these
types of measure space.   I start with the characteristic properties of
semi-finite spaces (213A-213B), and continue with complete locally
determined spaces (213C) and the concept of
`c.l.d.\ version' (213D-213H),
%213D 213E 213F 213G 213H
the most powerful of the universally
available methods of modifying a measure space into a better-behaved
one.   I briefly discuss `locally determined negligible sets'
(213I-213L),
%213I 213J 213K 213L
and measurable envelopes (213L-213M),
and end with results on localizable spaces (213N) and strictly
localizable spaces (213O).

\leader{213A}{Lemma} Let $(X,\Sigma,\mu)$ be a semi-finite measure
space.   Then

\Centerline{$\mu E
=\sup\{\mu F:F\in\Sigma,\,F\subseteq E,\,\mu F<\infty\}$}

\noindent for every $E\in\Sigma$.

\proof{ Set
$c=\sup\{\mu F:F\in\Sigma,\,F\subseteq E,\,\mu F<\infty\}$.   Then
surely $c\le\mu E$, so if
$c=\infty$ we can stop.   If $c<\infty$, let $\sequencen{F_n}$
be a sequence of measurable subsets of $E$, of finite measure, such that
$\lim_{n\to\infty}\mu F_n=c$;  set $F=\bigcup_{n\in\Bbb N}F_n$.
For each $n\in\Bbb N$, $\bigcup_{k\le n}F_k$ is a measurable set of
finite measure included in $E$, so $\mu(\bigcup_{k\le n}F_k)\le c$,
and

\Centerline{$\mu F=\lim_{n\to\infty}\mu(\bigcup_{k\le n}F_k)\le c$.}

\noindent Also


\Centerline{$\mu F\ge\sup_{n\in\Bbb N}\mu F_n\ge c$,}

\noindent so $\mu F=c$.

If $F'$ is a measurable subset of $E\setminus F$ and $\mu F'<\infty$,
then $F\cup F'$ has finite measure and is included in $E$, so has
measure at most $c=\mu F$;  it follows that $\mu F'=0$.   But this
means that $\mu(E\setminus F)$ cannot be infinite, since then, because
$(X,\Sigma,\mu)$ is semi-finite, it would have to include a measurable
set of non-zero finite measure.   So $E\setminus F$ has finite measure,
and is therefore in fact negligible;  and $\mu E=c$, as claimed.
}%end of proof of 213A

\vleader{72pt}{213B}{Proposition} Let $(X,\Sigma,\mu)$ be a semi-finite measure
space.   Let $f$ be a $\mu$-virtually measurable $[0,\infty]$-valued
function defined almost everywhere in $X$.   Then

$$\eqalign{\int f
&=\sup\{\int g:g\text{ is a simple function},\,g\leae f\}\cr
&=\sup_{F\in\Sigma,\mu F<\infty}\int_Ff}$$

\noindent in $[0,\infty]$.

\proof{{\bf (a)} For any measure space $(X,\Sigma,\mu)$, a
$[0,\infty]$-valued function defined on a subset of $X$ is
integrable iff there is a conegligible set $E$ such that

\inset{($\alpha$) $E\subseteq\dom f$ and $f\restr E$ is measurable,}

\inset{($\beta$)
$\sup\{\int g:g$ is a simple function, $g\leae f\}$ is finite,}

\inset{($\gamma$) for every $\epsilon>0$,
$\{x:x\in E,\,f(x)\ge\epsilon\}$ has finite measure,}

\inset{($\delta$) $f$ is finite almost everywhere}

\noindent (see 122Ja, 133B).   But if $\mu$ is semi-finite, ($\gamma$)
and ($\delta$) are consequences of the rest.   \Prf\ Let $\epsilon>0$.
Set

\Centerline{$E_{\epsilon}=\{x:x\in E,\,f(x)\ge\epsilon\}$,}

\Centerline{$c
=\sup\{\int g:g$ is a simple function, $g\leae f\}$;}

\noindent we are supposing that $c$ is finite.   If
$F\subseteq E_{\epsilon}$ is measurable and $\mu F<\infty$, then
$\epsilon\chi F$ is a
simple function and $\epsilon\chi F\leae f$, so

\Centerline{$\epsilon\mu F=\int\epsilon\chi F\le c$,
\quad$\mu F\le\Bover{c}{\epsilon}$.}

\noindent As $F$ is arbitrary, 213A tells us that
$\mu E_{\epsilon}\le\bover{c}{\epsilon}$ is finite.   As $\epsilon$ is
arbitrary, ($\gamma$) is satisfied.

As for ($\delta$), if $F=\{x:x\in E,\,f(x)=\infty\}$ then $\mu F$ is
finite (by ($\gamma$)) and $n\chi F\leae f$, so $n\mu F\le c$, for
every $n\in\Bbb N$, so $\mu F=0$.\ \Qed

\medskip

{\bf (b)} Now suppose that $f:D\to[0,\infty]$ is a $\mu$-virtually
measurable function, where $D\subseteq X$ is conegligible, so that
$\int f$ is defined in $[0,\infty]$ (135F).   Then (a) tells us that

\leaveitout{$$\eqalignno{\int f
&=\sup_{g\text{ is simple},g\le f\text{ a.e.}}\int g\cr
\displaycause{if either is finite, and therefore also if either is
infinite}
&=\sup_{g\text{ is simple},g\le f\text{ a.e.},\mu F<\infty}\int_Fg
\le\sup_{\mu F<\infty}\int_Ff
\le\int f,\cr}$$}

$$\eqalignno{\int f
&=\sup_{\Atop{g\text{ is simple}}{g\le f\text{ a.e.}}}\int g\cr
\displaycause{if either is finite, and therefore also if either is
infinite}
&=\sup_{\Atop{g\text{ is simple}}{\Atop{g\le f\text{ a.e.}}
{\mu F<\infty}}}\int_Fg
\le\sup_{\mu F<\infty}\int_Ff
\le\int f,\cr}$$

\noindent so we have the equalities we seek.
}%end of proof of 213B

\leader{*213C}{Proposition} Let $(X,\Sigma,\mu)$ be a complete
locally determined measure space, and $\mu^*$ the outer measure derived
from $\mu$\cmmnt{ (132A-132B)}.   Then the measure defined from
$\mu^*$ by \Caratheodory's method is $\mu$ itself.

\proof{ Write $\check\mu$ for the measure defined by \Caratheodory's
method from $\mu^*$, and $\check\Sigma$ for its domain.

\medskip

{\bf (a)} If $E\in\Sigma$ and $A\subseteq X$ then
$\mu^*(A\cap E)+\mu^*(A\setminus E)=\mu^*A$ (132Af), so
$E\in\check\Sigma$.   Now
$\check\mu E=\mu^*E=\mu E$ (132Ac).   Thus $\Sigma\subseteq\check\Sigma$
and $\mu=\check\mu\restr\Sigma$.

\medskip

{\bf (b)} Now suppose that $H\in\check\Sigma$.   Let $E\in\Sigma$ be
such that $\mu E<\infty$.   Then $H\cap E\in\Sigma$.   \Prf\ Let $E_1$,
$E_2\in\Sigma$ be measurable envelopes of $E\cap H$, $E\setminus H$
respectively, both included in $E$ (132Ee).   Because $H\in\check\Sigma$,

\Centerline{$\mu E_1+\mu E_2=\mu^*(E\cap H)+\mu^*(E\setminus H)
=\mu^*E=\mu E$.}

\noindent As $E_1\cup E_2=E$,

\Centerline{$\mu(E_1\cap E_2)=\mu E_1+\mu E_2-\mu E=0$.}

\noindent Now $E_1\setminus(E\cap H)\subseteq E_1\cap E_2$;  because
$\mu$ is complete, $E_1\setminus(E\cap H)$ and $E\cap H$ belong to
$\Sigma$.\ \Qed

As $E$ is arbitrary, and $\mu$ is locally determined, $H\in\Sigma$.   As
$H$ is arbitrary, $\check\Sigma=\Sigma$ and $\check\mu=\mu$.
}%end of proof of 213C

\leader{213D}{C.l.d.\ versions:  Proposition} Let $(X,\Sigma,\mu)$ be a
measure space.    Write $(X,\hat\Sigma,\hat\mu)$ for its
completion\cmmnt{ (212C)} and $\Sigma^f$ for
$\{E:E\in\Sigma,\,\mu E<\infty\}$.   Set

\Centerline{$\tilde\Sigma=\{H:H\subseteq X,\,H\cap E\in\hat\Sigma$ for
every $E\in\Sigma^f\}$,}

\noindent and for $H\in\tilde\Sigma$ set

\Centerline{$\tilde\mu H=\sup\{\hat\mu(H\cap E):E\in\Sigma^f\}$.}

\noindent Then
$(X,\tilde\Sigma,\tilde\mu)$ is a complete locally determined
measure space.

\proof{{\bf (a)} I check first that $\tilde\Sigma$ is a
$\sigma$-algebra.   \Prf\ {(i)} $\emptyset\cap E=\emptyset\in\hat\Sigma$
for every $E\in\Sigma^f$, so $\emptyset\in\tilde\Sigma$.   {(ii)} If
$H\in\tilde\Sigma$ then

\Centerline{$(X\setminus H)\cap E
=E\setminus(E\cap H)\in\hat\Sigma$}

\noindent for every $E\in\Sigma^f$, so $X\setminus H\in\tilde\Sigma$.
{(iii)} If $\sequencen{H_n}$ is a sequence in $\tilde\Sigma$ with union
$H$, then

\Centerline{$H\cap E=\bigcup_{n\in\Bbb N}H_n\cap E\in\hat\Sigma$}

\noindent for every
$E\in\Sigma^f$, so $H\in\tilde\Sigma$.\ \Qed

\medskip

{\bf (b)} It is obvious that $\tilde\mu \emptyset=0$.   If
$\sequencen{H_n}$ is a disjoint sequence in $\tilde\Sigma$ with union
$H$, then

$$\eqalign{\tilde\mu H
&=\sup\{\hat\mu(H\cap E):E\in\Sigma^f\}\cr
&=\sup\{\sum_{n=0}^{\infty}\hat\mu(H_n\cap E):E\in\Sigma^f\}
\le\sum_{n=0}^{\infty}\tilde\mu H_n.\cr}$$

\noindent On the other hand, given $a<\sum_{n=0}^{\infty}\tilde\mu H_n$,
there is an $m\in\Bbb N$ such that $a<\sum_{n=0}^m\tilde\mu H_n$;  now
we can find $E_0,\ldots,E_m\in\Sigma^f$ such that
$a\le\sum_{n=0}^m\hat\mu(H_n\cap E_n)$.   Set $E=\bigcup_{n\le
m}E_n\in\Sigma^f$;  then

\Centerline{$\tilde\mu H\ge\hat\mu(H\cap E)
=\sum_{n=0}^{\infty}\hat\mu(H_n\cap E)
\ge\sum_{n=0}^m\hat\mu(H_n\cap E_n)
\ge a$.}

\noindent As $a$ is arbitrary, $\tilde\mu
H\ge\sum_{n=0}^{\infty}\tilde\mu H_n$ and $\tilde\mu
H=\sum_{n=0}^{\infty}\tilde\mu H_n$.

\medskip

{\bf (c)} Thus $(X,\tilde\Sigma,\tilde\mu)$ is a measure space.   To
see that it is semi-finite, note first that
$\hat\Sigma\subseteq\tilde\Sigma$ (because if $H\in\hat\Sigma$ then
surely $H\cap E\in\hat\Sigma$ for every $E\in\Sigma^f$), and that
$\tilde\mu H=\hat\mu H$ whenever $\hat\mu H<\infty$ (because then, by
the definition in 212Ca, there is an $E\in\Sigma^f$ such that
$H\subseteq E$, so that
$\tilde\mu H=\hat\mu(H\cap E)=\hat\mu H$).   Now suppose that
$H\in\tilde\Sigma$ and that $\tilde\mu H=\infty$.   There is surely an
$E\in\Sigma^f$ such that $\hat\mu(H\cap E)>0$;  but now $0<\hat\mu(H\cap
E)<\infty$, so $0<\tilde\mu(H\cap E)<\infty$.

\medskip

{\bf (d)} Thus $(X,\tilde\Sigma,\tilde\mu)$ is a semi-finite measure space.
To see that it is locally determined, let $H\subseteq X$ be such that
$H\cap G\in\tilde\Sigma$ whenever $G\in\tilde\Sigma$ and
$\tilde\mu G<\infty$.   Then, in particular, we must have
$H\cap E\in\tilde\Sigma$
for every $E\in\Sigma^f$.   But this means in fact that
$H\cap E\in\hat\Sigma$ for every $E\in\Sigma^f$, so that
$H\in\tilde\Sigma$.
As $H$ is arbitrary, $(X,\Sigma,\mu)$ is locally determined.

\medskip

{\bf (e)} To see that $(X,\tilde\Sigma,\tilde\mu)$ is complete,
suppose that $A\subseteq H\in\tilde\Sigma$ and that $\tilde\mu H=0$.
Then for every $E\in\Sigma^f$ we must have $\hat\mu(H\cap E)=0$.
Because $(X,\hat\Sigma,\hat\mu)$ is complete, and
$A\cap E\subseteq H\cap E$, $A\cap E\in\hat\Sigma$.   As $E$ is
arbitrary, $A\in\tilde\Sigma$.
}%end of proof of 213D

\leader{213E}{Definition} For any measure space $(X,\Sigma,\mu)$, I will
call $(X,\tilde\Sigma,\tilde\mu)$, as constructed in 213D, the
{\bf c.l.d.\ version} (`complete locally determined version') of $(X,\Sigma,\mu)$;  and $\tilde\mu$ will be the
{\bf c.l.d.\ version} of $\mu$.

\leader{213F}{}\cmmnt{ Following the same pattern as in
212E-212G, %212E 212F 212G
I start with some elementary remarks to facilitate manipulation of this
construction.

\medskip

\noindent}{\bf Proposition} Let $(X,\Sigma,\mu)$ be any measure space
and $(X,\tilde\Sigma,\tilde\mu)$ its c.l.d.\ version.

(a) $\Sigma\subseteq\tilde\Sigma$ and $\tilde\mu E=\mu E$ whenever
$E\in\Sigma$ and $\mu E<\infty$ -- in fact, if $(X,\hat\Sigma,\hat\mu)$
is the completion of $(X,\Sigma,\mu)$, $\hat\Sigma\subseteq\tilde\Sigma$
and $\tilde\mu E=\hat\mu E$ whenever $\hat\mu E<\infty$.

(b) Writing $\tilde\mu^*$ and $\mu^*$ for the outer measures defined
from $\tilde\mu$ and $\mu$ respectively, $\tilde\mu^*A\le\mu^*A$ for
every $A\subseteq X$, with equality if
$\mu^*A$ is finite.   In particular, $\mu$-negligible sets are
$\tilde\mu$-negligible;   consequently, $\mu$-conegligible sets are
$\tilde\mu$-conegligible.

(c) If $H\in\tilde\Sigma$,

\quad (i) $\tilde\mu H=\sup\{\mu F:E\in\Sigma$, $\mu F<\infty$,
$F\subseteq H\}$;

\quad (ii) there is an $E\in\Sigma$ such that
$E\subseteq H$ and $\mu E=\tilde\mu H$, so that
if $\tilde\mu H<\infty$ then $\tilde\mu(H\setminus E)=0$.

\proof{{\bf (a)} This is already covered by remarks in the proof of
213D.

\medskip

{\bf (b)} If $\mu^*A=\infty$ then surely $\tilde\mu^*A\le\mu^*A$.   If
$\mu^*A<\infty$, take $E\in\Sigma$ such that $A\subseteq E$ and
$\mu E=\mu^*A$ (132Aa).   Then

\Centerline{$\tilde\mu^*A\le\tilde\mu E=\mu E=\mu^*A$.}

\noindent On the other hand, if $A\subseteq H\in\tilde\Sigma$, then

\Centerline{$\tilde\mu H\ge\hat\mu(H\cap E)\ge\hat\mu^*A=\mu^*A$,}

\noindent using 212Ea.   So $\mu^*A\le\tilde\mu^*A$ and
$\mu^*A=\tilde\mu^*A$.

\medskip

{\bf (c)} Write $\Sigma^f$ for $\{E:E\in\Sigma,\,\mu E<\infty\}$;  then,
by the definition in 213D,
$\tilde\mu H=\sup\{\hat\mu(H\cap E):E\in\Sigma^f\}$.   Let
$\sequencen{E_n}$ be a sequence in $\Sigma^f$
such that $\tilde\mu H=\sup_{n\in\Bbb N}\hat\mu(H\cap E_n)$.   For each
$n\in\Bbb N$ there is an $F_n\in\Sigma$ such that
$F_n\subseteq H\cap E_n$ and
$\mu F_n=\hat\mu(H\cap E_n)$ (212C).   Set $E=\bigcup_{n\in\Bbb N}F_n$.
Then $E\in\Sigma$, $E\subseteq H$ and

\Centerline{$\tilde\mu H
=\sup_{n\in\Bbb N}\mu F_n
\le\lim_{n\to\infty}\mu(\bigcup_{i\le n}F_i)
=\mu E
=\lim_{n\to\infty}\tilde\mu(\bigcup_{i\le n}F_i)
\le\tilde\mu H,$}

\noindent so $\mu E=\tilde\mu H$, and if $\tilde\mu H<\infty$ then
$\tilde\mu(H\setminus E)=0$.    At the same time,

$$\eqalignno{\tilde\mu H
&=\sup_{n\in\Bbb N}\mu F_n
\le\sup_{F\in\Sigma^f,F\subseteq H}\mu F
=\sup_{F\in\Sigma^f,F\subseteq H}\tilde\mu F\cr
\displaycause{by (a) again}
&\le\tilde\mu H,\cr}$$

\noindent so we have equality here too.
}%end of proof of 213F

\leader{213G}{}\cmmnt{ The next step is to look at functions which are
measurable or integrable with respect to $\tilde\mu$.

\medskip

\noindent}{\bf Proposition} Let $(X,\Sigma,\mu)$ be a measure space, and
$(X,\tilde\Sigma,\tilde\mu)$ its c.l.d.\ version.

(a) If a real-valued function $f$ defined on a subset of $X$ is
$\mu$-virtually measurable, it is $\tilde\Sigma$-measurable.
%should we have $[-\infty,\infty]$-valued here?

(b) If a real-valued function is $\mu$-integrable, it is
$\tilde\mu$-integrable with the same integral.

(c) If $f$ is a $\tilde\mu$-integrable real-valued function, there is a
$\mu$-integrable real-valued function which is equal to
$f\,\,\tilde\mu$-almost everywhere.

\proof{ Write $\Sigma^f$ for $\{E:E\in\Sigma,\,\mu E<\infty\}$.   By
213Fa, $\tilde\mu$ and $\mu$ agree on $\Sigma^f$.

\medskip

{\bf (a)} By 212Fa, $f$ is $\hat\Sigma$-measurable, where
$\hat\Sigma$ is the domain of the completion of $\mu$;  but since
$\hat\Sigma\subseteq\tilde\Sigma$, $f$ is $\tilde\Sigma$-measurable.

\medskip

{\bf (b)(i)} If $f$ is a $\mu$-simple function it is
$\tilde\mu$-simple,
and $\int fd\mu=\int fd\tilde\mu$, because $\tilde\mu E=\mu E$ for every
$E\in\Sigma^f$.

\medskip

\quad{\bf (ii)} If $f$ is a non-negative $\mu$-integrable function,
there is a non-decreasing sequence $\sequencen{f_n}$ of $\mu$-simple
functions converging to $f\,\,\mu$-almost everywhere;  now (by 213Fb)
$\mu$-negligible sets are $\tilde\mu$-negligible, so $\sequencen{f_n}$
converges to $f\,\,\tilde\mu$-a.e.\ and (by B.Levi's theorem, 123A) $f$
is $\tilde\mu$-integrable, with

\Centerline{$\int fd\tilde\mu=\lim_{n\to\infty}\int f_nd\tilde\mu
=\lim_{n\to\infty}\int f_nd\mu=\int fd\mu$.}

\medskip

\quad{\bf (iii)} In general, if $\int fd\mu$ is defined in
$\Bbb R$, we have

\Centerline{$\int fd\tilde\mu
=\int f^+d\tilde\mu-\int f^-d\tilde\mu
=\int f^+d\mu-\int f^-d\mu=\int fd\mu$,}

\noindent writing $f^+$ for $f\vee 0$ and $f^-$ for $(-f)\vee 0$.

\medskip

{\bf (c)(i)} Let $f$ be a $\tilde\mu$-simple function.   Express it as
$\sum_{i=0}^na_i\chi H_i$ where $\tilde\mu H_i<\infty$ for each $i$.
Choose $E_0,\ldots,E_n\in\Sigma$ such that $E_i\subseteq H_i$ and
$\tilde\mu(H_i\setminus E_i)=0$ for each $i$ (using 213Fc above).   Then
$g=\sum_{i=0}^na_i\chi E_i$ is $\mu$-simple,
$g=f\,\,\tilde\mu$-a.e., and $\int g\,d\mu=\int fd\tilde\mu$.

\medskip

\quad{\bf (ii)} Let $f$ be a non-negative $\tilde\mu$-integrable
function.   Let $\sequencen{f_n}$ be a non-decreasing sequence of
$\tilde\mu$-simple functions converging $\tilde\mu$-almost everywhere to
$f$.   For each $n$, choose a $\mu$-simple function $g_n$ equal
$\tilde\mu$-almost everywhere to $f_n$.   Then $\{x:g_{n+1}(x)<g_n(x)\}$
belongs to $\Sigma^f$ and is $\tilde\mu$-negligible, therefore
$\mu$-negligible.   So $\sequencen{g_n}$ is non-decreasing $\mu$-almost
everywhere.   Because

\Centerline{$\lim_{n\to\infty}\int g_nd\mu
=\lim_{n\to\infty}\int f_nd\tilde\mu=\int fd\tilde\mu$,}

\noindent B.Levi's theorem tells us that $\sequencen{g_n}$ converges
$\mu$-almost everywhere to a $\mu$-integrable function $g$;  because
$\mu$-negligible sets are $\tilde\mu$-negligible,

$$\eqalign{(X\setminus\dom f)
&\cup(X\setminus\dom g)\cr
&\cup\bigcup_{n\in\Bbb N}\{x:f_n(x)\ne g_n(x)\}\cr
&\cup\{x:x\in\dom
f,\,f(x)\ne\sup_{n\in\Bbb N}f_n(x)\}\cr
&\cup\{x:x\in\dom
g,\,g(x)\ne\sup_{n\in\Bbb N}g_n(x)\}\cr}$$

\noindent is $\tilde\mu$-negligible, and $f=g\,\,\tilde\mu$-a.e.

\medskip

\quad{\bf (iii)} If $f$ is $\tilde\mu$-integrable, express it as
$f_1-f_2$ where $f_1$ and $f_2$ are $\tilde\mu$-integrable and
non-negative;  then there
are $\mu$-integrable functions $g_1$, $g_2$ such that $f_1=g_1$,
$f_2=g_2\,\,\tilde\mu$-a.e., so that $g=g_1-g_2$ is $\mu$-integrable and
equal to $f\,\,\tilde\mu$-a.e.
}%end of proof of 213G

\leader{213H}{}\cmmnt{ Thirdly, I turn to the effect of the
construction here on the other properties being considered in this
chapter.

\medskip

\noindent}{\bf Proposition} Let $(X,\Sigma,\mu)$ be a measure space,
$(X,\hat\Sigma,\hat\mu)$ its completion and $(X,\tilde\Sigma,\tilde\mu)$
its c.l.d.\ version.

(a) If $(X,\Sigma,\mu)$ is a probability space, or totally finite, or
$\sigma$-finite, or strictly localizable, so is
$(X,\tilde\Sigma,\tilde\mu)$, and in all these cases
$\tilde\mu=\hat\mu$;

(b) if $(X,\Sigma,\mu)$ is localizable, so is
$(X,\tilde\Sigma,\tilde\mu)$, and for every $H\in\tilde\Sigma$ there is
an $E\in\Sigma$ such that $\tilde\mu(E\symmdiff H)=0$;

(c) $(X,\Sigma,\mu)$ is semi-finite iff $\tilde\mu F=\mu F$ for every
$F\in\Sigma$, and in this case $\int fd\tilde\mu=\int fd\mu$ whenever
the latter is defined in $[-\infty,\infty]$;

(d) a set $H\in\tilde\Sigma$ is an atom for $\tilde\mu$ iff there is an
atom $E$ for $\mu$ such that $\mu E<\infty$ and
$\tilde\mu(H\symmdiff E)=0$;

(e) if $(X,\Sigma,\mu)$ is atomless or purely atomic, so is
$(X,\tilde\Sigma,\tilde\mu)$;

(f) $(X,\Sigma,\mu)$ is complete and locally determined iff
$\tilde\mu=\mu$.


\proof{{\bf (a)(i)} I start by showing that if $(X,\Sigma,\mu)$ is
strictly localizable, then
$\tilde\mu=\hat\mu$.   \Prf\ Let $\langle X_i\rangle_{i\in I}$ be a
decomposition of $X$ for $\mu$;  then it is also a decomposition for
$\hat\mu$ (212Gb).   If $H\in\tilde\Sigma$, we shall have
$H\cap X_i\in\hat\Sigma$ for every $i$, and therefore $H\in\hat\Sigma$;
moreover,

$$\eqalign{\hat\mu H
&=\sum_{i\in I}\hat\mu(H\cap X_i)
=\sup\{\sum_{i\in J}\hat\mu(H\cap X_i):J\subseteq I
\text{ is finite}\}\cr
&\le\sup\{\hat\mu(H\cap E):E\in\Sigma,\,\mu E<\infty\}
=\tilde\mu H
\le\hat\mu H.\cr}$$

\noindent So $\hat\mu H=\tilde\mu H$ for every $H\in\tilde\Sigma$ and
$\hat\mu=\tilde\mu$.\ \Qed

\medskip

\quad{\bf (ii)} Consequently, if $(X,\Sigma,\mu)$ is a probability
space, or totally
finite, or $\sigma$-finite, or strictly localizable, so is
$(X,\tilde\Sigma,\tilde\mu)$, using 212Ga-212Gb to see that
$(X,\hat\Sigma,\hat\mu)$ has the property involved.

\medskip

{\bf (b)} If $(X,\Sigma,\mu)$ is localizable, let $\Cal H$ be any subset
of $\tilde\Sigma$.   Set

\Centerline{$\Cal E=\{E:E\in\Sigma^f,\,\exists\enskip
H\in\Cal H,\,E\subseteq H\}$}

\noindent where $\Sigma^f=\{E:\mu E<\infty\}$ as usual.
Working in $(X,\Sigma,\mu)$, let $F\in\Sigma$ be an essential
supremum for $\Cal E$.

\medskip

\quad{\bf (i)} \Quer\ Suppose, if possible, that there is an
$H\in\Cal H$ such that $\tilde\mu(H\setminus F)>0$.   By 213F(c-i),
there is an
$E\in\Sigma^f$ such that $E\subseteq H\setminus F$ and $\mu E>0$.
This $E$ belongs to $\Cal E$
and $\mu(E\setminus F)=\mu E>0$;  which is impossible if $F$ is an
essential supremum of $\Cal E$.\ \Bang

\medskip

\quad{\bf (ii)} Thus $\tilde\mu(H\setminus F)=0$ for every $H\in\Cal H$.
Now take any $G\in\tilde\Sigma$ such that $\tilde\mu(H\setminus G)=0$
for every $H\in\Cal H$.   \Quer\ If $\tilde\mu(F\setminus G)>0$, there
is an $E_0\in\Sigma^f$ such that $E_0\subseteq F\setminus G$ and
$\mu E_0>0$.   If $E\in\Cal E$, there is an
$H\in\Cal H$ such that $E\subseteq H$, so that
$E\setminus(F\setminus E_0)\subseteq H\setminus(F\cap G)$, while
$\mu(E\setminus(F\setminus E_0))<\infty$;  so

\Centerline{$\mu(E\setminus(F\setminus E_0))
\le\tilde\mu(H\setminus(F\cap G))
\le\tilde\mu(H\setminus F)+\tilde\mu(H\setminus G)
=0$.}

\noindent Because $F$ is an essential supremum for $\Cal E$ in $\Sigma$,

\Centerline{$0
=\mu(F\setminus(F\setminus E_0))
=\mu E_0$.  \Bang}

\noindent This shows that $F$ is an essential supremum for $\Cal H$ in
$\tilde\Sigma$.   As $\Cal H$ is arbitrary, $(X,\tilde\Sigma,\tilde\mu)$
is localizable.

\medskip

\quad{\bf (iii)} The argument of (i)-(ii) shows in fact that if $\Cal
H\subseteq\tilde\Sigma$ then $\Cal H$ has an essential supremum $F$ in
$\tilde\Sigma$ such that $F$ actually belongs to $\Sigma$.   Taking
$\Cal H=\{H\}$, we see that if $H\in\tilde\Sigma$ there is an
$F\in\Sigma$ such that $\mu(H\symmdiff F)=0$.

\medskip

{\bf (c)} We already know that $\tilde\mu E\le\mu E$ for every
$E\in\Sigma$, with equality if $\mu E<\infty$, by 213Fa.

\medskip

\quad{\bf (i)} If $(X,\Sigma,\mu)$ is semi-finite, then 213A
and 213F(c-i) tell us that for any $F\in\Sigma$ we have

\Centerline{$\mu F
=\sup\{\mu E:E\in\Sigma,\,E\subseteq F,\,\mu E<\infty\}
=\tilde\mu F$.}

\medskip

\quad{\bf (ii)} Suppose that $\tilde\mu F=\mu F$ for every $F\in\Sigma$.
If $\mu F=\infty$, then $\tilde\mu F=\infty$ so (by 213F(c-i) again)
there must be an $E\in\Sigma^f$ such that $E\subseteq F$ and
$\mu E>0$;  as $F$ is arbitrary, $(X,\Sigma,\mu)$ is
semi-finite.

\medskip

\quad{\bf (iii)} If $f$ is non-negative and $\int fd\mu=\infty$, then
$f$ is $\mu$-virtually measurable, therefore $\tilde\Sigma$-measurable
(213Ga), and defined $\mu$-almost everywhere, therefore
$\tilde\mu$-almost everywhere.   Now

$$\eqalign{\int fd\tilde\mu
&=\sup\{\int g\,d\tilde\mu:g\text{ is }\tilde\mu\text{-simple},\,
   0\le g\le f\,\,\tilde\mu\text{-a.e.}\}\cr
&\ge\sup\{\int g\,d\mu:g\text{ is }\mu\text{-simple},\,
   0\le g\le f\,\,\mu\text{-a.e.}\}=\infty\cr}$$

\noindent by 213B.   With 213Gb, this shows that
$\int fd\tilde\mu=\int fd\mu$ whenever $f$ is non-negative and
$\int fd\mu$ is defined in $[0,\infty]$.   Applying this to the positive
and negative parts of $f$, we see that $\int fd\tilde\mu=\int fd\mu$
whenever the latter is defined in $[-\infty,\infty]$.

\medskip

{\bf (d)(i)} If $H\in\tilde\Sigma$ is an atom for $\tilde\mu$, then
there is an $E\in\Sigma^f$ such that $E\subseteq H$ and $0<\mu E<\infty$.
In this case, $\tilde\mu E>0$ so $\tilde\mu(H\setminus E)$ must be zero.
If $F\in\Sigma$ and $F\subseteq E$, then either
$\mu F=\tilde\mu F=0$ or $\mu(E\setminus F)=\tilde\mu(H\setminus F)=0$.
Thus $E\in\Sigma$ is an atom for $\mu$ with
$\tilde\mu(H\symmdiff E)=0$ and $\mu E<\infty$.

\medskip

\quad{\bf (ii)} If $H\in\tilde\Sigma$ and there is an atom $E$ for $\mu$
such that $\mu E<\infty$ and $\tilde\mu(H\symmdiff E)=0$, let
$G\in\tilde\Sigma$ be a subset of $H$ with $\tilde\mu G>0$.
We have $\tilde\mu(E\cap G)=\tilde\mu(H\cap G)>0$, so there is an
$F\in\Sigma$ such that $F\subseteq E\cap G$ and $\mu F>0$.   Now
$\mu(E\setminus F)$ must be zero, so

\Centerline{$\tilde\mu(H\setminus G)\le\tilde\mu(H\setminus F)
=\tilde\mu(E\setminus F)=\mu(E\setminus F)=0$.}

\noindent As $G$ is arbitrary, $H$ is an atom for $\tilde\mu$.

\medskip

{\bf (e)} If $(X,\Sigma,\mu)$ is atomless, then
$(X,\tilde\Sigma,\tilde\mu)$ must be atomless, by (d).

If $(X,\Sigma,\mu)$ is purely atomic, $H\in\tilde\Sigma$ and
$\tilde\mu H>0$, then there is an $E\in\Sigma^f$ such that
$E\subseteq H$ and $\mu E>0$.
There is an atom $F$ for $\mu$ such that $F\subseteq E$;   now
$\mu F<\infty$ so $F$ is an atom for $\tilde\mu$, by (d).   Also
$F\subseteq H$.   As $H$ is arbitrary, $(X,\tilde\Sigma,\tilde\mu)$ is
purely atomic.

\medskip

{\bf (f)} If $\mu=\tilde\mu$, then of course $(X,\Sigma,\mu)$ must be
complete and locally determined, because $(X,\tilde\Sigma,\tilde\mu)$
is.   If $(X,\Sigma,\mu)$ is complete and locally determined, then
$\hat\mu=\mu$ so (using the definition in 213D)
$\tilde\Sigma\subseteq\Sigma$ and $\tilde\mu=\mu$, by (c) above.
}%end of proof of 213H

\leader{213I}{Locally determined negligible \dvrocolon{sets}}\cmmnt{
The following simple idea is occasionally useful.

\medskip

\noindent}{\bf Definition} A measure space $(X,\Sigma,\mu)$ has
{\bf locally determined negligible sets} if for every non-negligible
$A\subseteq X$ there is an $E\in\Sigma$ such that $\mu E<\infty$ and
$A\cap E$ is not negligible.

\leader{213J}{Proposition} If a measure space $(X,\Sigma,\mu)$ is
{\it either} strictly localizable {\it or} complete and locally
determined, it has locally determined negligible sets.

\proof{ Let $A\subseteq X$ be a set such that $A\cap E$ is negligible
whenever $\mu E<\infty$;  I need to show that $A$ is negligible.

\medskip

{\bf (i)} If $\mu$ is strictly localizable, let $\familyiI{X_i}$ be a
decomposition of $X$.   For each $i\in I$, $A\cap X_i$ is negligible, so
there we can choose
a negligible $E_i\in\Sigma$ such that $A\cap X_i\subseteq E_i$.
Set $E=\bigcup_{i\in I}E_i\cap X_i$.   Then
$\mu E=\sum_{i\in I}\mu(E_i\cap X_i)=0$ and $A\subseteq E$, so $A$ is
negligible.

\medskip

{\bf (ii)} If $\mu$ is complete and locally determined, take any
measurable set $E$ of finite measure.   Then $A\cap E$ is negligible,
therefore measurable;  as $E$ is arbitrary, $A$ is measurable;  as $\mu$
is semi-finite, $A$ is negligible.
}%end of proof of 213J

\leader{*213K}{Lemma} If a measure space $(X,\Sigma,\mu)$ has locally
determined negligible sets, and $\Cal E\subseteq\Sigma$ has an essential
supremum $H\in\Sigma$\cmmnt{ in the sense of 211G}, then
$H\setminus\bigcup\Cal E$ is negligible.

\proof{ Set $A=H\setminus\bigcup\Cal E$.   Take any $F\in\Sigma$ such
that $\mu F<\infty$.   Then
$F\cap A$ has a measurable envelope $V$ say (132Ee again).
If $E\in\Cal E$, then

\Centerline{$\mu(E\setminus(X\setminus V))
=\mu(E\cap V)=\mu^*(E\cap F\cap A)=0$,}

\noindent so $H\cap V=H\setminus(X\setminus V)$ is negligible and
$F\cap A$ is negligible.   As $F$ is arbitrary and $\mu$ has locally
determined negligible sets, $A$ is negligible, as claimed.
}%end of proof of 213K

\leader{213L}{Proposition} Let $(X,\Sigma,\mu)$ be a localizable measure
space with locally determined negligible sets.
Then every subset $A$ of $X$ has a measurable envelope.

\proof{ Set

\Centerline{$\Cal E=\{E:E\in\Sigma,\,\mu^*(A\cap E)=\mu E<\infty\}$.}

\noindent Let $G$ be an essential supremum for $\Cal E$ in $\Sigma$.

\medskip

\quad{\bf (i)} $A\setminus G$ is negligible.   \Prf\ Let $F$ be any set
of finite measure for $\mu$.   Let $E$ be a measurable envelope of
$A\cap F$.   Then $E\in\Cal E$ so $E\setminus G$ is negligible.   But
$F\cap A\setminus G\subseteq E\setminus G$, so $F\cap A\setminus G$ is
negligible.   Because $\mu$ has locally determined negligible sets,
this is enough to show that $A\setminus G$ is negligible.\ \Qed

\medskip

\quad{\bf (ii)} Let $E_0$ be a negligible measurable set including
$A\setminus G$, and set $\tilde G=E_0\cup G$, so that
$\tilde G\in\Sigma$,
$A\subseteq\tilde G$ and $\mu(\tilde G\setminus G)=0$.   \Quer\
Suppose, if
possible, that there is an $F\in\Sigma$ such that
$\mu^*(A\cap F)<\mu(\tilde G\cap F)$.   Let $F_1\subseteq F$ be a
measurable envelope of $A\cap F$.   Set $H=X\setminus(F\setminus F_1)$;
then $A\subseteq H$.   If $E\in\Cal E$ then

\Centerline{$\mu E=\mu^*(A\cap E)\le\mu(H\cap E)$,}

\noindent so $E\setminus H$ is negligible;  as $E$ is arbitrary,
$G\setminus H$ is negligible and $\tilde G\setminus H$ is negligible.
But $\tilde G\cap F\setminus F_1\subseteq\tilde G\setminus H$ and

\Centerline{$\mu(\tilde G\cap F\setminus F_1)
=\mu(\tilde G\cap F)-\mu^*(A\cap F)>0$. \Bang}

This shows that $\tilde G$ is a measurable envelope of $A$, as required.
}%end of proof of 213L

\leader{213M}{Corollary} (a) If $(X,\Sigma,\mu)$ is $\sigma$-finite, then
every subset of $X$ has a measurable envelope for $\mu$.

(b) If $(X,\Sigma,\mu)$ is localizable, then every subset of $X$ has a
measurable envelope for the c.l.d.\ version of $\mu$.

\proof{{\bf (a)} Use 132Ee, or 213L, 211Lc and 213J.

\medskip

{\bf (b)} Use 213L and the fact that the c.l.d.\ version of $\mu$ is
localizable as well as being complete and locally determined (213Hb).
}%end of proof of 213M

\leader{213N}{}\cmmnt{ When we come to use the concept of
`localizability', it will frequently be through the following
property, which in fact characterizes localizable spaces (213Xm).

\medskip

\noindent}{\bf Theorem} Let $(X,\Sigma,\mu)$ be a localizable measure
space.   Suppose that $\Phi$ is a family of measurable real-valued
functions, all defined on measurable subsets of $X$, such that whenever
$f$, $g\in\Phi$ then $f=g$ almost everywhere in $\dom f\cap\dom g$.
Then there is a measurable function $h:X\to\Bbb R$ such that every
$f\in\Phi$ agrees with $h$ almost everywhere in $\dom f$.

\proof{ For $q\in\Bbb Q$, $f\in\Phi$ set

\Centerline{$E_{fq}=\{x:x\in\dom f,\,f(x)\ge q\}\in\Sigma$.}

\noindent For each $q\in\Bbb Q$, let $E_q$ be an essential supremum of
$\{E_{fq}:f\in\Phi\}$ in $\Sigma$.   Set

\Centerline{$h^*(x)=\sup\{q:q\in\Bbb Q,\,x\in E_q\}\in[-\infty,\infty]$}

\noindent for $x\in X$, taking $\sup\emptyset=-\infty$ if necessary.

If $f$, $g\in\Phi$ and $q\in\Bbb Q$, then

$$\eqalign{E_{fq}\setminus(X\setminus(\dom g\setminus E_{gq}))
&=E_{fq}\cap\dom g\setminus E_{gq}\cr
&\subseteq\{x:x\in\dom f\cap\dom g,\,f(x)\ne g(x)\}\cr}$$

\noindent is negligible;  as $f$ is arbitrary,

\Centerline{$E_q\cap\dom g\setminus E_{gq}=E_q\setminus(X\setminus(\dom
g\setminus E_{gq}))$}

\noindent is negligible.   Also $E_{gq}\setminus E_q$ is negligible, so
$E_{gq}\symmdiff(E_q\cap\dom g)$ is negligible.   Set
$H_g=\bigcup_{q\in\Bbb Q}E_{gq}\symmdiff(E_q\cap\dom g)$;  then $H_g$
is negligible.   But if $x\in\dom g\setminus H_g$, then, for every
$q\in\Bbb Q$, $x\in E_q\iff x\in E_{gq}$;  it follows that for such $x$,
$h^*(x)=g(x)$.   Thus $h^*=g$ almost everywhere in $\dom g$;  and this
is true for every $g\in\Phi$.

The function $h^*$ is not necessarily real-valued.   But it is
measurable, because
\Centerline{$\{x:h^*(x)>a\}=\bigcup\{E_q:q\in\Bbb Q,\,q>a\}\in\Sigma$}

\noindent for every real $a$.   So if we modify it by setting

$$\eqalign{h(x)&=h^*(x)\text{ if }h(x)\in\Bbb R,\cr
&=0\text{ if }h^*(x)\in\{-\infty,\infty\},\cr}$$

\noindent we shall get a measurable real-valued function $h:X\to\Bbb R$;
and for any $g\in\Phi$, $h(x)$ will be equal to $g(x)$ at least whenever
$h^*(x)=g(x)$, which is true for almost every $x\in\dom g$.   Thus $h$
is a suitable function.
}%end of proof of 213N

\leader{213O}{}\cmmnt{ There is an interesting and useful criterion
for a space to be strictly localizable which I introduce at this point,
though it will be used rarely in this volume.

\medskip

\noindent}{\bf Proposition} Let $(X,\Sigma,\mu)$ be a complete locally
determined space.

(a) Suppose that there is a disjoint family $\Cal E\subseteq\Sigma$
such that ($\alpha$) $\mu E<\infty$ for every $E\in\Cal E$
($\beta$) whenever $F\in\Sigma$ and $\mu F>0$ then there is an
$E\in\Cal E$
such that $\mu(E\cap F)>0$.   Then $(X,\Sigma,\mu)$ is strictly
localizable, $\bigcup\Cal E$ is conegligible, and
$\Cal E\cup\{X\setminus\bigcup\Cal E\}$ is a
decomposition of $X$.

(b) Suppose that $\langle X_i\rangle_{i\in I}$ is a partition of $X$
into
measurable sets of finite measure such that whenever $E\in\Sigma$ and
$\mu E>0$ there is an $i\in I$ such that $\mu(E\cap X_i)>0$.   Then
$(X,\Sigma,\mu)$ is strictly localizable, and
$\langle X_i\rangle_{i\in I}$ is a decomposition of $X$.

\proof{{\bf (a)(i)}
The first thing to note is that if $F\in\Sigma$ and
$\mu F<\infty$, there is a countable
$\Cal E'\subseteq\Cal E$ such that
$\mu(F\setminus\bigcup\Cal E')=0$.   \Prf\ Set

\Centerline{$\Cal E'_n=\{E:E\in\Cal E,\,\mu(F\cap E)\ge 2^{-n}\}$ for
each $n\in\Bbb N$,}

\Centerline{$\Cal E'=\bigcup_{n\in\Bbb N}\Cal E'_n=\{E:E\in\Cal
E,\,\mu(F\cap E)>0\}$.}

\noindent Because $\Cal E$ is disjoint, we must have

\Centerline{$\#(\Cal E'_n)\le 2^n\mu F$}

\noindent for every $n\in\Bbb N$, so that every $\Cal E'_n$ is finite
and $\Cal E'$, being the union of a sequence of countable sets, is
countable.   Set $E'=\bigcup\Cal E'$ and $F'=F\setminus E'$, so that
both $E'$ and $F'$ belong to $\Sigma$.   If $E\in\Cal E'$, then
$E\subseteq E'$ so $\mu(E\cap F')=\mu\emptyset=0$;  if $E\in\Cal
E\setminus\Cal E'$, then $\mu(E\cap F')=\mu(E\cap F)=0$.   Thus
$\mu(E\cap F')=0$ for every $E\in\Cal E$.   By the hypothesis ($\beta$)
on
$\Cal E$, $\mu F'=0$, so $\mu(F\setminus\bigcup\Cal E')=0$, as
required.\ \Qed

\medskip

\quad{\bf (ii)} Now suppose that $H\subseteq X$ is such that $H\cap
E\in\Sigma$ for every $E\in\Cal E$.   In this case $H\in\Sigma$.   \Prf\
Let $F\in\Sigma$ be such that $\mu F<\infty$.   Let $\Cal
E'\subseteq\Cal E$ be a countable set such that $\mu(F\setminus E')=0$,
where $E'=\bigcup\Cal E'$.  Then $H\cap(F\setminus E')\in\Sigma$
because $(X,\Sigma,\mu)$ is complete.   But also $H\cap
E'=\bigcup_{E\in\Cal E'}H\cap E\in\Sigma$.   So

\Centerline{$H\cap F=(H\cap(F\setminus E'))\cup(F\cap(H\cap
E'))\in\Sigma$.}

\noindent As $F$ is arbitrary and $(X,\Sigma,\mu)$ is locally
determined, $H\in\Sigma$.\ \Qed

\medskip

\quad{\bf (iii)} We find also that $\mu H=\sum_{E\in\Cal E}\mu(H\cap E)$
for every $H\in\Sigma$.   \Prf\ {($\alpha$)} Because $\Cal E$ is
disjoint, we
must have $\sum_{E\in\Cal E'}\mu(H\cap E)\le\mu H$ for every finite
$\Cal E'\subseteq\Cal E$, so

\Centerline{$\sum_{E\in\Cal E}\mu(H\cap E)
=\sup\{\sum_{E\in\Cal E'}
  \mu(H\cap E):\Cal E'\subseteq\Cal E$ is finite$\}\le\mu H$.}

\noindent ($\beta$) For the reverse inequality, consider first the case
$\mu H<\infty$.   By (i), there is a countable $\Cal E'\subseteq\Cal E$
such that $\mu(H\setminus\bigcup\Cal E')=0$, so that

\Centerline{$\mu H=\mu(H\cap\bigcup\Cal E')
=\sum_{E\in\Cal E'}\mu(H\cap E)\le\sum_{E\in\Cal E}\mu(H\cap E)$.}

\noindent{($\gamma$)} In general, because $(X,\Sigma,\mu)$ is
semi-finite,

$$\eqalign{\mu H
&=\sup\{\mu F:F\subseteq H,\,\mu F<\infty\}\cr
&\le\sup\{\sum_{E\in\Cal E}\mu(F\cap E):F\subseteq H,\,\mu F<\infty\}
\le\sum_{E\in\Cal E}\mu(H\cap E).\cr}$$

\noindent So in all cases we have
$\mu H\le\sum_{E\in\Cal E}\mu(H\cap E)$, and the two are equal.\ \Qed

\medskip

\quad{\bf (iv)} In particular, setting $E_0=X\setminus\bigcup\Cal E$,
$E_0\in\Sigma$ and $\mu E_0=0$;  that is, $\bigcup\Cal E$ is
conegligible.
Consider $\Cal E^*=\Cal E\cup\{E_0\}$.
This is a partition of $X$ into sets of finite measure (now using the
hypothesis ($\alpha$) on $\Cal E$).   If $H\subseteq X$ is such that
$H\cap E\in\Sigma$ for every $E\in\Cal E^*$, then $H\in\Sigma$ and

\Centerline{$\mu H=\sum_{E\in\Cal E}\mu(H\cap E)
=\sum_{E\in\Cal E^*}\mu(H\cap E)$.}

\noindent Thus $\Cal E^*$ (or, if you prefer, the indexed family
$\langle E\rangle_{E\in\Cal E^*}$) is a decomposition witnessing that
$(X,\Sigma,\mu)$ is strictly localizable.

\medskip

{\bf (b)} Apply (a) with $\Cal E=\{X_i:i\in I\}$, noting that $E_0$ in
(iv) is empty, so can be dropped.
}%end of proof of 213O

\exercises{
\leader{213X}{Basic exercises (a)}
%\spheader 213Xa
Let $(X,\Sigma,\mu)$ be any measure space, $\mu^*$ the outer measure
defined from $\mu$, and $\check\mu$ the measure defined by
\Caratheodory's method from $\mu^*$;  write $\check\Sigma$ for the
domain of $\check\mu$.   Show that
(i) $\check\mu$ extends the completion $\hat\mu$ of $\mu$;
(ii) if $H\subseteq X$ is such that
$H\cap F\in\check\Sigma$ whenever $F\in\Sigma$ and $\mu F<\infty$, then
$H\in\check\Sigma$;
(iii) $(\check\mu)^*=\mu^*$, so that the integrable functions
for $\check\mu$ and $\mu$ are the same (212Xb);  (iv) if $\mu$ is
strictly localizable then $\check\mu=\hat\mu$;  (v) if $\mu$ is defined
by \Caratheodory's method from another outer measure, then $\mu=\check\mu$
(cf.\ 132Xa).
%213C

\sqheader 213Xb
Let $\mu$ be counting measure restricted to the
countable-cocountable $\sigma$-algebra of a set $X$ (211R, 211Ye).
(i) Show that the c.l.d.\ version $\tilde\mu$ of $\mu$ is just counting
measure on $X$.   (ii) Show that $\check\mu$, as defined in 213Xa, is
equal to $\tilde\mu$, and in particular strictly extends the completion
of $\mu$ if $X$ is uncountable.
%213E

\spheader 213Xc Let $(X,\Sigma,\mu)$ be any measure space.   For
$E\in\Sigma$ set

\Centerline{$\mu_{\text{sf}}E
=\sup\{\mu(E\cap F):F\in\Sigma,\,\mu F<\infty\}$.}

\quad{(i)} Show that $(X,\Sigma,\mu_{\text{sf}})$ is a semi-finite measure
space, and is equal to $(X,\Sigma,\mu)$ iff $(X,\Sigma,\mu)$ is
semi-finite.

\quad{(ii)} Show that a $\mu$-integrable real-valued function $f$ is
$\mu_{\text{sf}}$-integrable, with the same integral.

\quad(iii) Show that if $E\in\Sigma$ then $E$
can be expressed as $E_1\cup E_2$ where $E_1$, $E_2\in\Sigma$,
$\mu E_1=\mu_{\text{sf}}E_1$ and $\mu_{\text{sf}}E_2=0$.

\quad{(iv)} Show that if $f$ is a $\mu_{\text{sf}}$-integrable real-valued
function on $X$, it is equal $\mu_{\text{sf}}$-almost everywhere to a
$\mu$-integrable function.

\quad{(v)} Show that if $(X,\Sigma,\mu_{\text{sf}})$ is complete, so is
$(X,\Sigma,\mu)$.

\quad{(vi)} Show that $\mu$ and $\mu_{\text{sf}}$ have identical c.l.d.\
versions.
%213F

\spheader 213Xd Let $(X,\Sigma,\mu)$ be any measure space.   Define
$\check\mu$ as in 213Xa.   Show that $(\check\mu)_{\text{sf}}$, as constructed
in 213Xc, is precisely the c.l.d.\ version $\tilde\mu$ of $\mu$, so that
$\check\mu=\tilde\mu$ iff $\check\mu$ is semi-finite.
%213F, 213Xa, 213Xc

\spheader 213Xe Let $(X,\Sigma,\mu)$ be a measure space.   For
$A\subseteq X$ set
$\mu_* A=\sup\{\mu E:E\in\Sigma,\,\mu E<\infty,\,E\subseteq A\}$, as in
113Yh.   (i) Show that the measure constructed from $\mu_*$
by the method of 113Yg/212Ya is just the c.l.d.\ version $\tilde\mu$ of
$\mu$.
(ii) Show that $\tilde\mu_*=\mu_*$.   (iii) Show that if $\nu$ is
another measure
on $X$, with domain $\Tau$, then $\tilde\mu=\tilde\nu$ iff
$\mu_*=\nu_*$.
%213F

\spheader 213Xf Let $X$ be a set and $\theta$ an outer measure on $X$.
Show that $\theta_{\text{sf}}$, defined by writing

\Centerline{$\theta_{\text{sf}}A=\sup\{\theta B:B\subseteq A,\,\theta
B<\infty\}$}

\noindent is also an outer measure on $X$.   Show that the measures
defined by \Caratheodory's method from $\theta$, $\theta_{\text{sf}}$ have the
same domains.
%213F

\spheader 213Xg Let $(X,\Sigma,\mu)$ be any measure space.   Set

\Centerline{$\mu^*_{\text{sf}}A
=\sup\{\mu^*(A\cap E):E\in\Sigma,\,\mu E<\infty\}$}

\noindent for every $A\subseteq X$.

\quad{(i)} Show that

\Centerline{$\mu^*_{\text{sf}}A=\sup\{\mu^*B:B\subseteq A,\,\mu^*B<\infty\}$}

\noindent for every $A$.

\quad{(ii)} Show that $\mu^*_{\text{sf}}$ is an outer measure.

\quad(iii) Show that if $A\subseteq X$ and $\mu^*_{\text{sf}}A<\infty$, there
is an $E\in\Sigma$ such that $\mu^*_{\text{sf}}A=\mu^*(A\cap E)=\mu E$,
$\mu^*_{\text{sf}}(A\setminus E)=0$.   \Hint{take a non-decreasing sequence
$\sequencen{E_n}$ of measurable sets of finite measure such that
$\mu^*_{\text{sf}}A=\lim_{n\to\infty}\mu^*(A\cap E_n)$, and let
$E\subseteq\bigcup_{n\in\Bbb N}E_n$ be a measurable envelope of
$A\cap\bigcup_{n\in\Bbb N}E_n$.}

\quad{(iv)} Show that the measure defined from $\mu^*_{\text{sf}}$ by
\Caratheodory's method is precisely the c.l.d.\ version $\tilde\mu$ of
$\mu$.

\quad{(v)} Show that $\mu^*_{\text{sf}}=\tilde\mu^*$, so that if $\mu$ is
complete and locally determined then $\mu^*_{\text{sf}}=\mu^*$.
%213F, 213Xc, 213Xf

\sqheader 213Xh Let $(X,\Sigma,\mu)$ be a measure space
with locally determined measurable sets.   Show that it is semi-finite.
%213I

\sqheader 213Xi Let $(X,\Sigma,\mu)$ be a measure space, $\hat\mu$ the
completion of $\mu$,
$\tilde\mu$ the c.l.d.\ version of $\mu$
and $\check\mu$ the measure defined by \Caratheodory's method
from $\mu^*$.   Show that the following are equiveridical:
(i) $\mu$ has
locally determined negligible sets;
(ii) $\mu$ and $\tilde\mu$ have the
same negligible sets;
(iii) $\check\mu=\tilde\mu$;
(iv) $\hat\mu$ and $\tilde\mu$ have the same sets of finite measure;
(v) $\mu$ and $\tilde\mu$ have the same integrable functions;
(vi) $\tilde\mu^*=\mu^*$;
(vii) the outer measure $\mu^*_{\text{sf}}$ of 213Xg is equal to $\mu^*$.
%213J 213Xi

\sqheader 213Xj Let $(X,\Sigma,\mu)$ be a strictly localizable measure
space with a decomposition $\langle X_i\rangle_{i\in I}$.   Show that
$\mu^*A=\sum_{i\in I}\mu^*(A\cap X_i)$ for every $A\subseteq X$.
%213L

\sqheader 213Xk Let $(X,\Sigma,\mu)$ be a complete locally determined
measure space, and let $A\subseteq X$ be such that
$\min(\mu^*(E\cap A),\mu^*(E\setminus A))<\mu E$ whenever $E\in\Sigma$
and $0<\mu E<\infty$.
Show that $A\in\Sigma$.   \Hint{given $\mu F<\infty$, consider
the intersection $E$ of measurable envelopes of $F\cap A$, $F\setminus
A$ to see that $\mu^*(F\cap A)+\mu^*(F\setminus A)=\mu F$.}
%213M

\spheader 213Xl Let us say that a measure space $(X,\Sigma,\mu)$ has the
{\bf measurable envelope property} if every subset of $X$ has a
measurable envelope.   (i) Show that a semi-finite space with the
measurable envelope property has locally determined negligible sets.
(ii) Show that a complete semi-finite space with
the measurable envelope property is locally determined.
%213M

\spheader 213Xm Let $(X,\Sigma,\mu)$ be a semi-finite measure space, and
suppose that it satisfies the conclusion of Theorem 213N.   Show that it
is localizable.   \Hint{given $\Cal E\subseteq\Sigma$, set
$\Cal F=\{F:F\in\Sigma,\,E\cap F$ is negligible for every
$E\in\Cal E\}$.
Let $\Phi$ be the set of functions $f$ from subsets of $X$ to $\{0,1\}$
such that $f^{-1}[\{1\}]\in \Cal E$ and $f^{-1}[\{0\}]\in\Cal F$.}
%213N

\spheader 213Xn Let $(X,\Sigma,\mu)$ be a measure space.   Show that its
c.l.d.\ version is strictly localizable iff there is a disjoint family
$\Cal E\subseteq\Sigma$ such that $\mu E<\infty$ for every $E\in\Cal E$
and whenever $F\in\Sigma$ and $0<\mu F<\infty$ there is an $E\in\Cal E$
such that $\mu(E\cap F)>0$.
%213O

\spheader 213Xo Show that the c.l.d.\ version of any point-supported
measure is point-supported.

\vleader{48pt}{213Y}{Further exercises (a)}
%\spheader 213Ya
Let $(X,\Sigma,\mu)$ be a measure space.   Show that $\mu$
is semi-finite iff there is a family $\Cal E\subseteq\Sigma$ such that
$\mu E<\infty$ for every $E\in\Cal E$ and
$\mu F=\sum_{E\in\Cal E}\mu(F\cap E)$ for every $F\in\Sigma$.
\Hint{take $\Cal E$ maximal subject to the intersection of any two elements
being negligible.}
%213A  would be useful in 552M, (i) => (iii)

\spheader 213Yb
Set $X=\Bbb N$, and for $A\subseteq X$ set

\Centerline{$\theta A=\sqrt{\#(A)}$ if $A$ is finite, $\infty$ if $A$ is
infinite.}

\noindent Show that $\theta$ is an outer measure on $X$, that
$\theta A=\sup\{\theta B:B\subseteq A,\,\theta B<\infty\}$ for every
$A\subseteq X$, but that the measure $\mu$ defined from $\theta$ by
\Caratheodory's
method is not semi-finite.   Show that if $\check\mu$ is the measure
defined by \Caratheodory's method from $\mu^*$ (213Xa), then
$\check\mu\ne\mu$.
%213C

\spheader 213Yc Set $X=[0,1]\times\{0,1\}$, and let
$\Sigma$ be the family of those subsets $E$ of $X$ such that

\Centerline{$\{x:x\in[0,1],\,E[\{x\}]\ne\emptyset,\,E[\{x\}]\ne\{0,1\}\}
$}

\noindent is countable, writing $E[\{x\}]=\{y:(x,y)\in E\}$ for each
$x\in[0,1]$.   Show that $\Sigma$ is a $\sigma$-algebra of subsets of
$X$.   For $E\in\Sigma$, set $\mu E=\#(\{x:(x,1)\in E\})$ if this is
finite, $\infty$ otherwise.   Show that $\mu$ is a complete semi-finite
measure.   Show that the measure $\check\mu$ defined from $\mu^*$ by
\Caratheodory's method (213Xa) is not semi-finite.   Show that the
domain of the c.l.d.\ version of $\mu$ is the whole of $\Cal PX$.
%213Xa, 213F

\spheader 213Yd Set $X=\Bbb N$, and for $A\subseteq X$ set

\Centerline{$\phi A=\#(A)^2$ if $A$ is finite, $\infty$ if $A$ is
infinite.}

\noindent Show that $\phi$ satisfies the conditions of 212Ya,
but that the measure defined from $\phi$ by the method there is not
semi-finite.
%213Xe, 213F

\spheader 213Ye Let $(X,\Sigma,\mu)$ be a complete locally
determined measure space.   Suppose that $D\subseteq X$ and that
$f:D\to\Bbb R$ is a function.   Show that the following are equiveridical:
(i) $f$ is measurable;  (ii)

\Centerline{$\mu^*\{x:x\in D\cap E,\,f(x)\le
a\}+\mu^*\{x:x\in D\cap E,\,f(x)\ge b\}\le\mu E$}

\noindent whenever $a<b$ in $\Bbb R$, $E\in\Sigma$ and $\mu E<\infty$ (iii)

\Centerline{$\min(\mu^*\{x:x\in D\cap E,\,f(x)\le
a\},\mu^*\{x:x\in D\cap E,\,f(x)\ge b\})<\mu E$}

\noindent  whenever $a<b$ in $\Bbb R$ and $0<\mu E<\infty$.  ({\it
Hint\/}:
for (iii)$\Rightarrow$(i), show that if $E\subseteq X$ then

\Centerline{$\mu^*\{x:x\in D\cap E,\,f(x)>a\}
=\sup_{b>a}\mu^*\{x:x\in D\cap E,\,f(x)\ge b\}$,}

\noindent and use 213Xk above.)
%213Xk, 213M

\spheader 213Yf Let $(X,\Sigma,\mu)$ be a complete locally determined
measure space and suppose that $\Cal E\subseteq\Sigma$ is such that
$\mu E<\infty$ for every $E\in\Cal E$ and whenever $F\in\Sigma$ and
$\mu F<\infty$ there is a countable $\Cal E_0\subseteq\Cal E$ such that
$F\setminus\bigcup\Cal E_0$, $F\cap\bigcup(\Cal E\setminus\Cal E_0)$ are
negligible.   Show that $(X,\Sigma,\mu)$ is strictly localizable.
%213O
}%end of exercises

\endnotes{
\Notesheader{213} I think it is fair to say that if the definition of
`measure space' were re-written to exclude all spaces which are not
semi-finite, nothing significant would be lost from the theory.   There
are solid reasons for not taking such a drastic step, starting with the
fact that it would confuse everyone (if you say to an unprepared
audience `let $(X,\Sigma,\mu)$ be a measure space', there is a danger
that some will imagine that you mean `$\sigma$-finite measure space',
but very few will suppose that you mean `semi-finite measure space').
But the whole point of measure theory is that we distinguish between
sets by their measures, and if every subset of $E$ is either
non-measurable, or negligible, or of infinite measure, the
classification is
too crude to support most of the usual ideas, starting, of course, with
ordinary integration.

Let us say that a measurable set $E$ is {\bf purely infinite} if $E$
itself and all its non-negligible measurable subsets have infinite
measure.   On the definition of the integral which I chose in Volume 1,
every simple function, and therefore every integrable function, must be
zero almost everywhere in $E$.   This means that the whole theory of
integration will ignore $E$ entirely.   Looking at the definition of
 `c.l.d.\ version' (213D-213E), you will see that the c.l.d.\ version
of the measure will render $E$ negligible, as does the `semi-finite
version' described in 213Xc.   These amendments do not, however, affect
sets of finite measure, and consequently leave integrable functions
integrable, with the same integrals.

The strongest reason we have yet seen for admitting non-semi-finite
spaces into consideration is that \Caratheodory's method does not
always produce semi-finite spaces.   (I give examples in 213Yb-213Yc;
more important ones are the Hausdorff measures of
\S\S264-265 below.)   In practice the right thing to do
is often to take the c.l.d.\ version of the measure produced by
\Caratheodory's construction.

It is a reasonable general philosophy, in measure theory, to say that we
wish to measure as many sets, and integrate as many functions, as we can
manage in a canonical way -- I mean, without making blatantly arbitrary
choices about the values we assign to our measure or integral.   The
revision of a measure $\mu$ to its c.l.d.\ version $\tilde\mu$ is about
as far as we can go with an arbitrary measure space in which
we have no other structure to guide our choices.

You will observe that $\tilde\mu$ is not as close to $\mu$ as the
completion $\hat\mu$ of $\mu$
is;  naturally so, because if $E\in\Sigma$ is purely infinite for
$\mu$ then we have to choose between
setting $\tilde\mu E=0\ne\mu E$ and finding some way of fitting many
sets of finite measure into $E$;  which if $E$ is a singleton will be
actually impossible, and in any case would be an arbitrary process.
However the integrable functions for $\tilde\mu$, while not always the
same as those for $\mu$ (since $\tilde\mu$ turns purely infinite sets
into negligible ones, so that their indicator functions become
integrable), are `nearly' the same, in the sense that any
$\tilde\mu$-integrable function can be changed into a $\mu$-integrable
function by adjusting it on a $\tilde\mu$-negligible set.   This
corresponds, of course, to the fact that any set of finite measure for
$\tilde\mu$ is the symmetric difference of a set of finite measure for
$\mu$ and a $\tilde\mu$-negligible set.   For sets of infinite measure
this can fail, unless $\mu$ is localizable (213Hb, 213Xb).

If $(X,\Sigma,\mu)$ is semi-finite, or localizable, or strictly
localizable, then of course it is correspondingly closer to
$(X,\tilde\Sigma,\tilde\mu)$, as detailed in 213Ha-c.

It is worth noting that while the measure $\check\mu$ obtained by
\Caratheodory's method directly from the outer measure $\mu^*$ defined
from $\mu$ may fail to be semi-finite, even when $\mu$ is (213Yc), a
simple modification of $\mu^*$ (213Xg) yields the c.l.d.\ version
$\tilde\mu$ of $\mu$, which can also be obtained from an appropriate
inner measure (213Xe).   The measure $\check\mu$ is of course related
in other ways to $\tilde\mu$;  see 213Xd.
}%end of notes

\discrpage


