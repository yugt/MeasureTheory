\frfilename{mt554.tex}
\versiondate{2.9.14}
\copyrightdate{2006}

\def\chaptername{Possible worlds}
\def\sectionname{Cohen reals}

\Loadfourteens

\def\BbbQk{\Bbb Q_{\kappa}}
\long\def\doubleinset#1{\inset{\inset{\parindent=-20pt #1}}}
\def\Fn{\mathop{\text{Fn}}\nolimits}
\def\VVdash{\mskip5mu\vrule height 7.5pt depth 2.5pt width 0.5pt
  \mskip2.5mu\vrule height 7.5pt depth 2.5pt width 0.5pt
  \vrule height 2.75pt depth -2.25pt width 4pt\mskip2mu}
\def\VVdP{\VVdash_{\Bbb P}}
\def\VVdQk{\VVdash_{\Bbb Q_{\kappa}}}

\newsection{554}

Parallel to the theory of random reals as described in \S\S552-553, we have
a corresponding theory based on category algebras rather than measure
algebras.   I start with the exactly matching result on cardinal arithmetic
(554B), and continue with Lusin sets (balancing the Sierpi\'nski
sets of 552E) and the cardinal functions of the
meager ideal of $\Bbb R$ (554C-554E, %554C 554D 554E
554F).   In the last third of the section I use the theory of Freese-Nation
numbers (\S518) to prove Carlson's theorem on Borel liftings (554I).

\leader{554A}{Notation} For any set $I$, I will write $\widehat{\Cal B}_I$
for the Baire-property algebra of $\{0,1\}^I$\cmmnt{, $\CalBa_I$ for the
Baire $\sigma$-algebra of $\{0,1\}^I$}, $\Cal M_I$ for the meager
ideal of $\{0,1\}^I$, $\frak G_I=\widehat{\Cal B}_I/\Cal M_I$ for the
category algebra of $\{0,1\}^I$, and $\Bbb Q_I$ for the forcing notion
$\frak G_I^+=\frak G_I\setminus\{\emptyset\}$ active downwards.
\cmmnt{$\Cal C_I$ will be the family of basic cylinder sets
$\{x:z\subseteq x\in\{0,1\}^I\}$ for $z\in\Fn_{<\omega}(I;\{0,1\})$, and
$C_I$ the corresponding set
$\{C^{\ssbullet}:C\in\Cal C_I\}\subseteq\frak G_I$;
then $C_I$ is order-dense in $\frak G_I$\cmmnt{ (because $\Cal C_I$ is a
$\pi$-base for the topology of $\{0,1\}^I$)}.   It follows that
$\tau(\frak G_I)\le\pi(\frak G_I)\le\max(\omega,\#(I))$.   (These
inequalities are of course equalities if $I$ is infinite.)}

\leader{554B}{Theorem} Suppose that $\lambda$ and $\kappa$ are infinite
cardinals.   Then

\Centerline{$\VVdQk\,2^{\check\lambda}=(\kappa^{\lambda})\var2spcheck$.}

\proof{ (Compare 552B.)

\medskip

{\bf (a)} Since $\frak G_{\kappa}$ is ccc and has an order-dense
subset $C_{\kappa}$ of size $\kappa$,
$\#(\frak G_{\kappa})$ is at most the cardinal power $\kappa^{\omega}$.

If $\dot A$ is a $\BbbQk$-name for a subset of
$\check\lambda$, then we have a corresponding family
$\ofamily{\eta}{\lambda}{\Bvalue{\check\eta\in\dot A}}$ of truth values;
and if $\dot A$, $\dot B$ are two such names, and
$\Bvalue{\check\eta\in\dot A}=\Bvalue{\check\eta\in\dot B}$ for every
$\eta<\lambda$, then

\Centerline{$\VVdQk\,\dot A=\dot B$.}

\noindent So

\Centerline{$\VVdQk\,2^{\check\lambda}=\#(\Cal P\check\lambda)
\le\#((\frak G_{\kappa}^{\lambda})\var2spcheck)
=(\kappa^{\lambda})\var2spcheck$.}

\medskip

{\bf (b)} Consider first the case in which $\lambda\le\kappa$.
Let $F$ be the set of all functions from $\lambda$ to $\kappa$,
so that $\#(F)=\kappa^{\lambda}$.   As in part (b) of the proof of 552B,
there is a set $G\subseteq F$ such that
$\#(G)=\kappa^{\lambda}$ and $\{\eta:\eta<\lambda$, $f(\eta)\ne g(\eta)\}$
is infinite whenever $f$, $g\in G$ are distinct.
Let $\langle\zeta_{\xi\eta}\rangle_{\xi<\kappa,\eta<\lambda}$ be a
family of distinct elements of $\kappa$ and set
$E_{\xi\eta}=\{x:x\in\{0,1\}^{\kappa}$, $x(\zeta_{\xi\eta})=1\}$ for
$\xi<\kappa$ and $\eta<\lambda$.
For $f\in G$ let $\dot A_f$ be a $\BbbQk$-name for
a subset of $\lambda$ such that

\Centerline{$\Bvalue{\check\eta\in\dot A_f}=E^{\ssbullet}_{f(\eta),\eta}$}

\noindent for every $\eta<\lambda$.   If $f$, $g\in G$ are distinct, set
$I=\{\eta:f(\eta)\ne g(\eta)\}$;  then

\Centerline{$\Bvalue{\dot A_f\ne\dot A_g}
=\sup_{\eta<\lambda}E_{f(\eta),\eta}^{\ssbullet}\Bsymmdiff
   E_{g(\eta),\eta}^{\ssbullet}=\Bbbone$}

\noindent because
$\bigcup_{\eta\in I}E_{f(\eta),\eta}\symmdiff E_{g(\eta),\eta}$
is a dense open set in $\{0,1\}^{\kappa}$.

Thus in the forcing language we have a name for an injective function from
$\check G$ to $\Cal P\lambda$, corresponding to the map $f\mapsto\dot A_f$
from $G$ to names of subsets of $\lambda$.   So

\Centerline{$\VVdQk\,2^{\check\lambda}\ge\#(\check G)
=(\kappa^{\lambda})\var2spcheck$.}

\noindent Putting this together with (a), we have

\Centerline{$\VVdQk\,2^{\check\lambda}=(\kappa^{\lambda})\var2spcheck$.}

\medskip

{\bf (c)} If $\lambda>\kappa$, then
$2^{\lambda}=\kappa^{\lambda}$.   Now

\Centerline{$\VVdQk\,
  (\Cal P\lambda)\var2spcheck\subseteq\Cal P\check\lambda$,}

\noindent so

\Centerline{$\VVdQk\,(\kappa^{\lambda})\var2spcheck
=\#((\Cal P\lambda)\var2spcheck)
\le\#(\Cal P\check\lambda)=2^{\check\lambda}$,}

\noindent and again we have

\Centerline{$\VVdQk\,2^{\check\lambda}=(\kappa^{\lambda})\var2spcheck$.}
}%end of proof of 554B

\leader{554C}{Definition} If $X$ is a topological space, a subset of $X$
is a {\bf Lusin set} if it is uncountable but meets every meager set
in a countable set\cmmnt{;  equivalently, if it is uncountable but meets
every nowhere dense set in a countable set}.

\leader{554D}{Proposition}
Let $\kappa$ be a cardinal such that $\Bbb R$ has a
Lusin set of size $\kappa$.

(a) Writing $\Cal M$ for the ideal of meager subsets of $\Bbb R$,
$\non\Cal M=\omega_1$ and $\frakmctbl\ge\kappa$.

(b) There is a point-countable family $\Cal A$ of
Lebesgue-conegligible subsets of $\Bbb R$ with $\#(\Cal A)=\kappa$.

(c) If $(\frak A,\bar\mu)$ is a semi-finite measure algebra which is not
purely atomic, $(\kappa,\omega_1)$ is not a precaliber pair of $\frak A$.

\proof{ Let $B\subseteq\Bbb R$ be a Lusin set of size $\kappa$.

\medskip

{\bf (a)} By 522Sa, $\frakmctbl=\cov\Cal M$.
Any uncountable subset of $B$ is non-meager, so
$\non\Cal M=\omega_1$.
If $\Cal E$ is a cover of $\Bbb R$ by meager sets,
then each member of $\Cal E$ meets $B$ in a countable set, so

\Centerline{$\kappa=\#(B)\le\max(\omega,\#(\Cal E))$}

\noindent and $\#(\Cal E)\ge\kappa$;  thus $\cov\Cal M\ge\kappa$.

\medskip

{\bf (b)}
Let $E\subseteq\Bbb R$ be a conegligible meager set containing $0$, and set
$\Cal A=\{x+E:x\in B\}$.   Then $\Cal A$ is a family of conegligible sets.
If $y\in\Bbb R$, then $y-E$ is meager so
$\{x:x\in B$, $y\in x+E\}=B\cap(y-E)$ is countable;  thus $\Cal A$ is
point-countable.   Also, each
member of $\Cal A$ is meager, so meets $B$ in a
countable set, and (because $B\subseteq\bigcup\Cal A$)

\Centerline{$\kappa=\#(B)\le\max(\omega,\#(\Cal A))\le\#(\Cal A)\le\#(B)$,}

\noindent so $\#(\Cal A)=\kappa$.

\medskip

{\bf (c)} Let $K\subseteq E$ be a compact set of non-zero measure.   If
$\Gamma\subseteq B$ is uncountable,
$\bigcap_{x\in\Gamma}x+K=\emptyset$, $\{x+K:x\in\Gamma\}$ does not have the
finite intersection property and $\{(x+K)^{\ssbullet}:x\in\Gamma\}$ is not
centered in the measure algebra $\frak A_L$ of Lebesgue measure.   Thus
$\family{x}{B}{(x+K)^{\ssbullet}}$ witnesses that $(\kappa,\omega_1)$ is
not a precaliber pair of $\frak A_L$.

Since $(\frak A,\bar\mu)$ is semi-finite and not purely atomic, there is a
subalgebra of a principal ideal of $\frak A$ which is isomorphic to
$\frak A_L$, and $(\kappa,\omega_1)$ is not a precaliber pair of $\frak A$,
by 516Sa.

}%end of proof of 554D

\leader{554E}{Theorem} Let $\kappa$ be an uncountable cardinal.   Then

\Centerline{$\VVdQk$ there is a Lusin set $A\subseteq\Bbb R$ with cardinal
$\check\kappa$.}

\proof{{\bf (a)} (Compare 552E.)   Write $\Bbb P$ for
$\Bbb Q_{\kappa\times\omega}$.   For each
$\xi<\kappa$, let $f_{\xi}:\{0,1\}^{\kappa\times\omega}\to\{0,1\}^{\omega}$
be given by setting $f_{\xi}(x)(n)=x(\xi,n)$ for every
$x\in\{0,1\}^{\kappa\times\lambda}$ and $n<\omega$;  then, taking
$\vec f_{\xi}$ to be the $\Bbb P$-name defined by the process of 551Cb,

\Centerline{$\VVdP\vec f_{\xi}\in\{0,1\}^{\omega}$.}

\noindent If $\xi$, $\xi'<\kappa$ are distinct, then, by 551Cd,

$$\eqalign{\Bvalue{\vec f_{\xi}=\vec f_{\xi'}}
&=\{x:f_{\xi}(x)=f_{\xi'}(x)\}^{\ssbullet}\cr
&=\{x:x(\xi,n)=x(\xi',n)\text{ for every }n\}^{\ssbullet}
=0\cr}$$

\noindent because $\{x:x(\xi,n)=x(\xi',n)\text{ for every }n\}$ is closed
and nowhere dense.   So, taking $\dot A$ to be the $\VVdP$-name
$\{(\vec f_{\xi},\Bbbone):\xi<\kappa\}$, we have

\Centerline{$\VVdP\,\dot A\subseteq\{0,1\}^{\omega}$ has
cardinal $\check\kappa$.}

\medskip

{\bf (b)} Now suppose that $\dot W$ is a $\Bbb P$-name such that

\Centerline{$\VVdP\,\dot W$ is a nowhere dense zero set in
$\{0,1\}^{\omega}$.}

\noindent By 551Fb there is a
$W\in\widehat{\Cal B}_{\kappa\times\omega}\tensorhat\CalBa_{\omega}$
such that, in the language of 551D, $\VVdP\,\dot W=\vec W$.   Now $W$ is
meager in $\{0,1\}^{\kappa\times\omega}$.   \Prf\ For
$z\in\Fn_{<\omega}(\omega;\{0,1\})$ set
$V_z=\{(x,y):x\in\{0,1\}^{\kappa\times\omega}$,
$z\subseteq y\in\{0,1\}^{\omega}\}$.   By 551Ee,

\Centerline{$\VVdP\,\vec V_z
=\{y:\check z\subseteq y\in\{0,1\}^{\omega}\}$;}

\noindent and as $\widehat{\Cal B}_{\kappa\times\omega}$ is closed under
Souslin's operation (431Fb),

\Centerline{$\Bvalue{\vec W\cap\vec V_z=\emptyset}
=\{x:W[\{x\}]\cap\{y:y\supseteq z\}
  =\emptyset\}^{\ssbullet}$}

\noindent (551Ga).   Now we have

$$\eqalign{1
&=\Bvalue{\vec W\text{ is nowhere dense}}\cr
&=\inf_{z\in\Fn_{<\omega}(\omega;\{0,1\})}
  \sup_{z'\in\Fn_{<\omega}(\omega;\{0,1\}),z'\supseteq z}
  \Bvalue{\vec W\cap\{y:\check z'\subseteq y\}=\emptyset}\cr
&=\inf_{z\in\Fn_{<\omega}(\omega;\{0,1\})}
  \sup_{z'\in\Fn_{<\omega}(\omega;\{0,1\}),z'\supseteq z}
  \{x:W[\{x\}]\cap\{y:y\supseteq z'\}=\emptyset\}^{\ssbullet}
  \cr
&=\bigl(\bigcap_{z\in\Fn_{<\omega}(\omega;\{0,1\})}
  \bigcup_{z'\in\Fn_{<\omega}(\omega;\{0,1\}),z'\supseteq z}
  \{x:W[\{x\}]\cap\{y:y\supseteq z'\}=\emptyset\}
  \bigr)^{\ssbullet}\cr
&=\{x:W[\{x\}]\text{ is nowhere dense}\}^{\ssbullet}.\cr}$$

\noindent So $\{x:W[\{x\}]$ is meager$\}$ is comeager.
Because $W$ has the Baire property in
$\{0,1\}^{\kappa\times\omega}\times\{0,1\}^{\omega}$ (5A4E(c-ii)),
it must be meager, by the Kuratowski-Ulam theorem (527D).\ \Qed

\medskip

{\bf (c)} Continuing from (b), there
is a meager Baire set $W'\supseteq W$ (5A4E(d-ii)).
Let $J\subseteq\kappa$ be
a countable set such that $W'$ is determined by coordinates in
$(J\times\omega)\mskip5mu\dot\cup\mskip5mu\omega$, that is, if
$(x,y)\in W'$, $x'\in\{0,1\}^{\kappa\times\omega}$ and
$x'\restr J\times\omega=x\restr J\times\omega$ then
$(x',y)\in W'$.   Take any $\xi\in\kappa\setminus J$.
Set $L=(\kappa\setminus\{\xi\})\times\omega$ and

\Centerline{$V=\{(x\restr L,y):(x,y)\in W'\}$;}

\noindent then $V\subseteq\{0,1\}^L\times\{0,1\}^{\omega}$ is meager
(applying 527D to

\Centerline{$V\times\{0,1\}^{\{\xi\}\times\omega}
\subseteq\{0,1\}^L\times\{0,1\}^{\{\xi\}\times\omega}
\cong\{0,1\}^{\kappa\times\omega}\times\{0,1\}^{\omega}$).}

Now consider the map
$\phi:\{0,1\}^{\kappa\times\omega}\to\{0,1\}^L\times\{0,1\}^{\omega}$
defined by setting $\phi(x)=(x\restr L,f_{\xi}(x))$ for
$x\in\{0,1\}^{\kappa\times\omega}$.
Looking back at the definition of $f_{\xi}$, we see that
this is a homeomorphism.   So $\phi^{-1}[V]$ must be meager, and

$$\eqalignno{\Bvalue{\vec f_{\xi}\in\vec W}
&\Bsubseteq\Bvalue{\vec f_{\xi}\in\vec W'}
=\{x:(x,f_{\xi}(x))\in W'\}^{\ssbullet}\cr
\displaycause{551Ea}
&=\{x:(x\restr L,f_{\xi}(x))\in V\}^{\ssbullet}
=(\phi^{-1}[V])^{\ssbullet}
=0,\cr}$$

\noindent that is, $\VVdP\,\vec f_{\xi}\notin\vec W$.

This is true for every $\xi\in\kappa\setminus J$.   So

\Centerline{$\VVdP\,\dot A\cap\dot W
\subseteq\{\vec f_{\xi}:\xi\in\check J\}$ is countable.}

\noindent As $\dot W$ is arbitrary,

\Centerline{$\VVdP\,\dot A$ has countable intersection with every nowhere
dense zero set.}

\noindent It follows at once that

\Centerline{$\VVdP\,\dot A$ has countable intersection with every nowhere
dense set, and is a Lusin set.}

\noindent As $\Bbb P$ and $\BbbQk$ are isomorphic,

\Centerline{$\VVdQk\,\{0,1\}^{\omega}$ has a Lusin set with cardinal
$\check\kappa$.}

\medskip

{\bf (d)} The statement of the proposition referred to $\Bbb R$ rather than
to $\{0,1\}^{\omega}$.   But, writing $\Cal M$ for the ideal of meager
subsets of
$\Bbb R$ and $\Cal M_{\omega}$ for the ideal of meager subsets of
$\{0,1\}^{\omega}$,
$(\Bbb R,\Cal M)$ and $(\{0,1\}^{\omega},\Cal M_{\omega})$ are isomorphic
(522Wb), and one will have Lusin sets iff the other does.   So

\Centerline{$\VVdQk\,\Bbb R$ has a Lusin set with cardinal $\check\kappa$.}
}%end of proof of 554E

\leader{554F}{Corollary} Let $\kappa$ be a cardinal which is equal to the
cardinal power $\kappa^{\omega}$.   Write $\Cal M$ for the ideal of meager
subsets of $\Bbb R$.   Then

\Centerline{$\VVdQk\,\non\Cal M=\omega_1$ and
$\frakmctbl=\frak c$.}

\proof{ By 554B, $\VVdQk\,\frak c=\check\kappa$;  so we have only to put
554E and 554Da together.
}%end of proof of 554F

\leader{554G}{Theorem} Let $\kappa$ be an infinite cardinal such that
$\FN(\frak G_{\kappa})=\omega_1$.   Then

\Centerline{$\VVdQk\,\FN(\Cal P\Bbb N)=\omega_1$.}

\proof{{\bf (a)} We need to know that $\frak G_{\kappa}$ is isomorphic to
the simple power
algebra $\frak G_{\kappa}^{\Bbb N}$.  \Prf\ The algebra $\Cal E$
of open-and-closed subsets of $\{0,1\}^{\kappa}$ is isomorphic to a free
product of two-element algebras, so is homogeneous (316Q);
$\frak G_{\kappa}$ is isomorphic to the Dedekind
completion of $\Cal E$, so is homogeneous (316P).
Now we have a partition
of unity $\sequencen{p_n}$ in $\frak G_{\kappa}$ consisting of non-zero
elements, so that $\frak G_{\kappa}$ is isomorphic to the simple product of
the corresponding principal ideals (315F) and to
$\frak G_{\kappa}^{\Bbb N}$.\ \QeD\  There is therefore a Freese-Nation
function $\theta:
\frak G_{\kappa}^{\Bbb N}\to[\frak G_{\kappa}^{\Bbb N}]^{\le\omega}$.

For $\xi<\kappa$, set $E_{\xi}=\{x:x\in\{0,1\}^{\kappa}$, $x(\xi)=1\}$;
for $J\subseteq\kappa$, let $\frak C_J$ be the order-closed subalgebra of
$\frak G_{\kappa}$ generated by $\{E_{\xi}^{\ssbullet}:\xi\in J\}$,
and let $C_J$ be the set of elements of $\frak C_J$ of the form
$\inf_{\xi\in K}E_{\xi}^{\ssbullet}
\Bsetminus\sup_{\xi\in L}E_{\xi}^{\ssbullet}$ where $K$, $L$ are disjoint
finite subsets of $J$.

For $v\in\frak G_{\kappa}^{\Bbb N}$ let $\vec v$ be the $\BbbQk$-name
$\{(\check n,v(n)):n\in\Bbb N$, $v(n)\ne 0\}$;  then
$\VVdQk\,\vec v\subseteq\Bbb N$, and $\Bvalue{\check n\in\vec v}=v(n)$
for every $n\in\Bbb N$.

For any $\BbbQk$-name $\dot u$, let $J(\dot u)$ be a countable subset of
$\kappa$ such that $\Bvalue{\check n\in\dot u}\in\frak C_{J(\dot u)}$
for every $n\in\Bbb N$.

\medskip

{\bf (b)} Let $\dot X$ be a discriminating $\BbbQk$-name such that
$\VVdQk\,\dot X=\Cal P\Bbb N$ (5A3Ka).   For $\sigma=(\dot u,p)\in\dot X$
set

\Centerline{$\theta_1(\sigma)
=\bigcup_{e\in C_{J(\dot u)}}
  \theta(\sequencen{\Bvalue{\check n\in\dot u}\Bcap e)})
\cup\bigcup_{e\in C_{J(\dot u)}}
  \theta(\sequencen{\Bvalue{\check n\in\dot u}\Bcup(1\Bsetminus e)})
\in[\frak G_{\kappa}^{\Bbb N}]^{\le\omega}$,}

\Centerline{$\theta_2(\sigma)=\{(\vec v,p):v\in\theta_1(\sigma)\}$,}

\noindent so that $\theta_2(\sigma)$ is a $\BbbQk$-name and

\Centerline{$\VVdQk\,\theta_2(\sigma)$ is a countable subset of
$\Cal P\Bbb N$.}

\medskip

{\bf (c)} Set

\Centerline{$\dot\theta
=\{((\dot u,\theta_2(\dot u,p)),p):(\dot u,p)\in\dot X\}$.}

\noindent By 5A3Kb,

\Centerline{$\VVdQk\,\dot\theta$ is a function with domain
$\dot X=\Cal P\Bbb N$.}

\noindent Next,

\Centerline{$\VVdQk\,\dot\theta$ takes values in
$[\Cal P\Bbb N]^{\le\omega}$.}

\noindent\Prf\ Suppose that $\dot x$ is a $\BbbQk$-name and $p\in\frak G_{\kappa}^+$ is
such that

\Centerline{$p\VVdQk\,\dot x$ is a value of $\dot\theta$.}

\noindent Then there are a $(\dot u,q)\in\dot X$ and a $p'$ stronger than
both $p$ and $q$ such that

\Centerline{$p'\VVdQk\,\dot x=(\dot u,\theta_2(\dot u,q))$ has second
member $\theta_2(\dot u,q)\in[\Cal P\Bbb N]^{\le\omega}$.}

\noindent As $p$ and $\dot x$ are arbitrary,

\doubleinset{$\VVdQk$ every value of $\dot\theta$, being the second member
of an element of $\dot\theta$, is a countable subset of $\Cal P\Bbb N$.
\Qed}

\medskip

{\bf (d)} In fact,

\Centerline{$\VVdQk\,\dot\theta$ is a Freese-Nation function on
$\Cal P\Bbb N$.}

\noindent\Prf\ Suppose that $\dot A_1$, $\dot A_2$ are $\BbbQk$-names and
$p\in\frak G_{\kappa}^+$ is such that

\Centerline{$p\VVdQk\,\dot A_1\subseteq\dot A_2\subseteq\Bbb N$.}

\noindent Because $\VVdQk\,\dot X=\Cal P\Bbb N$, there must be
$(\dot u_1,q_1)$ and $(\dot u_2,q_2)\in\dot X$ and a $p_1$
stronger than $p$, $q_1$ and $q_2$ such that

\Centerline{$p_1\VVdQk\,\dot u_1=\dot A_1$ and $\dot u_2=\dot A_2$.}

\noindent In this case, for both $i$,
$((\dot u_i,\theta_2(\dot u_i,q_i)),q_i)\in\dot\theta$,
so we have

\Centerline{$p_1\VVdQk\,\dot\theta(\dot A_i)=\dot\theta(\dot u_i)
=\theta_2(\dot u_i,q_i)$.}

\noindent Let $e\Bsubseteq p_1$ be a member of $C_{\kappa}$, that
is, a member of $\frak G_{\kappa}$ which is the equivalence class of a
basic cylinder set.   We have

\Centerline{$e\VVdQk\,\dot u_1=\dot A_1\subseteq\dot A_2=\dot u_2$,}

\noindent so
$e\Bcap\Bvalue{\check n\in\dot u_1}\Bsubseteq\Bvalue{\check n\in\dot u_2}$
for every $n\in\Bbb N$.   Express $e$ as $e_1\Bcap e_2\Bcap e_3$ where
$e_1\in C_{J(\dot u_1)}$, $e_2\in C_{J(\dot u_2)}$ and
$e_3\in C_{\kappa\setminus K}$, where $K=J(\dot u_1)\cup J(\dot u_2)$.
For each $n\in\Bbb N$,

\Centerline{$e_1\Bcap e_2\Bcap\Bvalue{\check n\in\dot u_1}
\Bsetminus\Bvalue{\check n\in\dot u_2}$}

\noindent belongs to $\frak C_K$ and is disjoint from
$e_3\in\frak C_{\kappa\setminus K}\setminus\{0\}$, so must be zero;
we therefore have

\Centerline{$e_1\Bcap\Bvalue{\check n\in\dot u_1}
\Bsubseteq\Bvalue{\check n\in\dot u_2}\Bcup(1\Bsetminus e_2)$}

\noindent for every $n$, that is,

\Centerline{$\sequencen{\Bvalue{\check n\in\dot u_1}\Bcap e_1}
\Bsubseteq\sequencen{\Bvalue{\check n\in\dot u_2}\Bcup(1\Bsetminus e_2)}$}

\noindent in $\frak G_{\kappa}^{\Bbb N}$.   Because $\theta$ is a
Freese-Nation function, there is a sequence

\Centerline{$\sequencen{a_n}
\in\theta(\sequencen{\Bvalue{\check n\in\dot u_1}\Bcap e_1})
\cap\theta(\sequencen{\Bvalue{\check n\in\dot u_2}\Bcup(1\Bsetminus e_2)})$}

\noindent such that

\Centerline{$\Bvalue{\check n\in\dot u_1}\Bcap e_1
\Bsubseteq a_n
\Bsubseteq\Bvalue{\check n\in\dot u_2}\Bcup(1\Bsetminus e_2)$}

\noindent for every $n$.   Now $v=\sequencen{a_n}$ belongs to
$\theta_1(\dot u_1,p_1)\cap\theta_1(\dot u_2,p_2)$, so
$(\vec v,p_i)\in\theta_2(\dot u_i,p_i)$ and

\Centerline{$p_i\VVdQk\,\vec v\in\theta_2(\dot u_i,p_i)
=\dot\theta(\dot u_i)$}

\noindent for both $i$.   Returning to $e$, we have

\Centerline{$e\Bcap\Bvalue{\check n\in\dot u_1}\Bsubseteq e\Bcap a_n
\Bsubseteq e\Bcap\Bvalue{\check n\in\dot u_2}$}

\noindent for every $n$, because $e\Bsubseteq e_1\Bcap e_2$.   So

\Centerline{$e\VVdQk\,\dot u_1\Bsubseteq\vec v\Bsubseteq\dot u_2$.}

\noindent Also $e$ is stronger than $p$ and

\Centerline{$e\VVdQk\,\vec v
\in\dot\theta(\dot u_1)\cap\dot\theta(\dot u_2)
=\dot\theta(\dot A_1)\cap\dot\theta(\dot A_2)$.}

\noindent As $p$, $\dot A_1$ and $\dot A_2$ are arbitrary,

\doubleinset{$\VVdQk$ for any $A$, $B\subseteq\Bbb N$ there is a
$C\in\dot\theta(A)\cap\dot\theta(B)$ such that $A\subseteq C\subseteq B$;
that is, $\dot\theta$ is a Freese-Nation function.  \Qed}

\medskip

{\bf (e)} Putting (c) and (d) together, we have

\Centerline{$\VVdQk\,\FN(\Cal P\Bbb N)\le\omega_1$;}

\noindent and since the Freese-Nation number of $\Cal P\Bbb N$ is surely
uncountable (522U), this is enough.
}%end of proof of 554G

\leader{554H}{Corollary} Suppose that $\FN(\Cal P\Bbb N)=\omega_1$ and that
$\kappa$ is an infinite cardinal such that

\inset{($\alpha$) $\cff[\lambda]^{\le\omega}\le\lambda^+$ for every
cardinal $\lambda\le\kappa$,

($\beta$) $\square_{\lambda}$ is true for every uncountable cardinal
$\lambda\le\kappa$ of countable cofinality.}

\noindent Then $\VVdQk\,\FN(\Cal P\Bbb N)=\omega_1$.

\proof{ Any countably generated order-closed subalgebra $\frak C$ of
$\frak G_{\kappa}$ is (in the language of part (a) of the proof of 554G)
included in $\frak C_J$ for some countable $J\subseteq\kappa$, which has a
countable $\pi$-base $C_J$;  so $\frak C_J$ and $\frak C$ are
$\sigma$-linked, and $\FN(\frak C)\le\FN(\Cal P\Bbb N)=\omega_1$,
by 518D.
By 518I, the conditions ($\alpha$) and ($\beta$), together with the fact
that $\tau(\frak G_{\kappa})\le\kappa$, now ensure that
$\FN(\frak G_{\kappa})\le\omega_1$, so 554G gives the result.
}%end of proof of 554H

\leader{554I}{Theorem}\cmmnt{ ({\smc Carlson Frankiewicz \& Zbierski 94})}
Suppose that the continuum hypothesis is true.   Then

\Centerline{$\VVdash_{\Bbb Q_{\omega_2}}\,\frak c=\omega_2$ and
Lebesgue measure has a Borel lifting.}

\proof{ Of course the cardinal power $\omega_2^{\omega}$ (in the ordinary
universe) is equal to
$\max(\frak c,\penalty-100\cff[\omega_2]^{\le\omega})=\omega_2$.
From 554H and 554B we see that

\Centerline{$\VVdash_{\Bbb Q_{\omega_2}}\,
\FN(\Cal P\Bbb N)=\omega_1$ and $\frak c=\omega_2$.}

\noindent So 535E(b-ii) tells us that

\Centerline{$\VVdash_{\Bbb Q_{\omega_2}}$
Lebesgue measure has a Borel lifting.}
}%end of proof of 554I

\exercises{\leader{554X}{Basic exercises (a)}
%\spheader 554Xa
Show that $\#(\frak G_{\kappa})=\kappa^{\omega}$ for every
infinite cardinal $\kappa$.
%554A

\spheader 554Xb\dvAnew{2014} Show that if $I$ is any set, every
regular uncountable cardinal is a precaliber of $\frak G_I$.
%554A 4A1D

\spheader 554Xc\dvAnew{2014} Let $I$ be any set.   (i)
Show that $(\Cal C_I,\supseteq)$ is isomorphic to
$(\Fn_{<\omega}(I;\{0,1\}),\subseteq)$ (definition:  552A).   (ii)
Show that
$\frak G_I$ can be identified with the regular open algebra
$\RO^{\uparrow}(\Fn_{<\omega}(I;\{0,1\}))$.
%554A out of order \query

\spheader 554Xd\dvAnew{2011}
Let $\kappa$ be an infinite cardinal such that $\Bbb R$ has a Lusin set
of size $\kappa$.   Show that there is a first-countable compact Hausdorff
space $X$ such that $\kappa\in\MahR(X)$.   \Hint{531O.}
%554D 531O out of order query

\spheader 554Xe Devise a definition of `strongly Lusin' set to match 537Ab,
and state and prove a result corresponding to 552E.
\Hint{527Xf.}
%554E

\spheader 554Xf\dvAnew{2010}
Describe Cicho\'n's diagram in the forcing universe
$V^{\Bbb Q_{\omega_2}}$ (i) if we start with $\frak c=\omega_1$
(ii) if we start with $\frak m=\frak c=\omega_2$.
%554E

\leader{554Y}{Further exercises (a)}
%\spheader 554Ya
For how many of the results of 552F-552J %552F 552G 552H 552I 552J
can you find equivalents with respect to Cohen real forcing?
\Hint{{\smc Bartoszy\'nski \& Judah 95}.}

\spheader 554Yb(i) Show that there is a family
$\ofamily{\xi}{\omega_1}{e_{\xi}}$ such that ($\alpha$) for each $\xi$,
$e_{\xi}\subseteq\xi\times\Bbb N$ is an injective function from $\xi$ to
$\Bbb N$ ($\beta$) if $\eta\le\xi<\omega_1$ then
$e_{\eta}\setminus e_{\xi}$ is finite.
\Hint{choose the $e_{\xi}$ inductively, taking care that
$\Bbb N\setminus e_{\xi}[\xi]$ is infinite for every $\xi$.}   (ii) Set
$T=\{e_{\xi}\restr\eta:\eta$, $\xi<\omega_1$ are successor ordinals$\}$.
Show that $T\cup\{\emptyset\}$,
ordered by $\subseteq$, is a special Aronszajn
tree.   \Hint{for any $n\in\Bbb N$, $\{t:t(\max(\dom t))=n\}$ is an
antichain.}
%553M

\spheader 554Yc ({\smc Todor\v{c}evi\'c 87}) %p 292
Let $\kappa$ be an infinite cardinal.
Take $\ofamily{\xi}{\omega_1}{e_{\xi}}$ as in 554Yb.
Let $\dot\preccurlyeq$ be the $\Bbb Q_{\kappa}$-name

\Centerline{$\{((\check\eta,\check\xi),p):
  \eta\le\xi<\omega_1,\,p\in\Bbb Q_{\kappa},\,
  p\Bsubseteq\{x:xe_{\eta}\subseteq xe_{\xi}\}^{\ssbullet}\}$.}

\noindent Show that
$\VVdQk\,(\omega_1,\dot\preccurlyeq)$ is a Souslin tree.
%554Yb mt55bits
}%end of exercises


\endnotes{
\Notesheader{554}
The original theories of Cohen and random reals were developed in
parallel;  see {\smc Kunen 84} for an account of the special properties of
null and meager ideals which made this possible.   Thus the Sierpi\'nski
sets of random real models become Lusin sets in Cohen real models, and
the horizontal gap which appears in Cicho\'n's diagram if we add random
reals becomes a vertical gap if we add Cohen reals
(552F-552I, %552F 552G 552H 552I
554F).   I give a
very much briefer account of Cohen reals because I am restricting attention
to results which have consequences in measure theory, as in 554Dc and 554I,
and (except in 554Yc/553M)
I make no attempt to look for reflections of the patterns in \S553,
which are mostly there for the illumination they throw on the structure of
measure algebras.
But I do not seek out the shortest route in every case.   In particular, I
spell out some of the theory of Freese-Nation numbers (554G-554H) for its
own sake as well as to provide a proof of Carlson's theorem 554I.   Let me
remind you that $\omega_2$ has a very special place in the arguments here;
see 518Rb and 535Zb.

I have written this section in terms of forcing with category algebras,
partly in order to emphasize the connexion with random reals, and partly to
be able to quote from \S551.   But of course it can equally be regarded as
a fragment of the theory of forcing with partially ordered sets
$\Fn_{<\omega}(I;\{0,1\})$ (554Xc), and there are many places
(e.g.\ 554Yc) where this simplifies the details.

}%end of notes
\discrpage



