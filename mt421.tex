\frfilename{mt421.tex}
\versiondate{14.12.07}
\copyrightdate{2001}

\def\NN{\BbbN^{\Bbb N}}

\def\chaptername{Descriptive set theory}
\def\sectionname{Souslin's operation}

\newsection{421}

I introduce Souslin's operation $\Cal S$ (421B) and show that it is
idempotent (421D).   I describe alternative characterizations of members
of $\Cal S(\Cal E)$, where $\Cal E\subseteq\Cal PX$, as projections of
sets in $\NN\times X$ (421G-421J).   I briefly mention Souslin-F sets
(421J-421L) and a special property of `inner Souslin kernels' (421M).
At the end of the section I set up an abstract theory of `constituents'
for kernels of Souslin schemes and their complements (421N-421Q).

\leader{421A}{Notation}\cmmnt{ Throughout this chapter, and frequently
in the next, I shall regard
a member of $\Bbb N$ as the set of its predecessors, so that a
finite power $X^k$
can be identified with the set of functions from $k$ to $X$, and if
$\phi\in X^{\Bbb N}$ and $k\in\Bbb N$, we can speak of the restriction
$\phi\restr k\in X^k$.   In the same spirit, identifying functions
with their graphs, I can write `$\sigma\subseteq\phi$' when
$\sigma\in X^k$,
$\phi\in X^{\Bbb N}$ and $\phi$ extends $\sigma$.   On occasion I may
write $\#(\sigma)$ for the `length' of a finite function $\sigma$ --
again identifying $\sigma$ with its graph -- so that $\#(\sigma)=k$ if
$\sigma\in X^k$.   And if $k=0$, identified with $\emptyset$, then the
only function from $k$ to $X$ is the empty function, so $X^0$ becomes
$\{\emptyset\}$.

I shall sometimes refer to the `usual topology of $\NN$';
this is the product topology if each copy of $\Bbb N$ is given its
discrete topology.}   $S$ will always be the set
$\bigcup_{k\in\Bbb N}\BbbN^k$, and $S^*=S\setminus\{\emptyset\}$ the set
$\bigcup_{k\ge 1}\BbbN^k$;  for $\sigma\in S$,
$I_{\sigma}$ will be
$\{\phi:\phi\in\BbbN^{\Bbb N},\,\phi\supseteq\sigma\}$.\cmmnt{  Then
$I_{\emptyset}=\NN$ and
$\{I_{\sigma}:\sigma\in S^*\}$ is a base for the topology of
$\BbbN^{\BbbN}$ consisting of open-and-closed sets.}
If $\sigma\in\BbbN^k$ and $i\in\Bbb N$ I write
$\sigma^{\smallfrown}\fraction{i}$ for
the member $\tau$ of $\BbbN^{k+1}$ such that $\tau(k)=i$ and
$\tau(j)=\sigma(j)$ for $j<k$.

\leader{421B}{Definition}
If $\Cal E$ is a family of sets, I write $\Cal S(\Cal E)$ for the family
of sets expressible in the form

$$\bigcup_{\phi\in\NN}\bigcap_{k\ge 1}E_{\phi\restr k}$$

\noindent for some family $\family{\sigma}{S^*}{E_{\sigma}}$ in $\Cal E$.

A family $\family{\sigma}{S^*}{E_{\sigma}}$ is called a {\bf Souslin
scheme};  the corresponding set
$\bigcup_{\phi\in\NN}\bigcap_{k\ge 1}E_{\phi\restr k}$ is its
{\bf kernel};  the operation

$$\family{\sigma}{S^*}{E_{\sigma}}
\mapsto\bigcup_{\phi\in\BbbN^{\Bbb N}}\bigcap_{k\ge 1}E_{\phi\restr k}$$

\noindent is {\bf Souslin's operation} or {\bf operation $\Cal A$}.
Thus $\Cal S(\Cal E)$ is the
family of sets obtainable from sets in $\Cal E$ by Souslin's operation.
If $\Cal E=\Cal S(\Cal E)$, we say that $\Cal E$ is {\bf closed under
Souslin's operation}.

\cmmnt{\medskip

\noindent{\bf Remark} I should perhaps warn you that some authors use
$\bigcup_{k\in\Bbb N}\BbbN^k$ here in place of $S^*$;  so that their
Souslin kernels are of the form
$\bigcup_{\phi\in\NN}\bigcap_{k\ge 0}E_{\phi\restr k}
\subseteq E_{\emptyset}$.   Consequently, for such
authors, any member of $\Cal S(\Cal E)$ is included in some member of
$\Cal E$.   If $\Cal E$ has a greatest member (or, fractionally more
generally, if any sequence in $\Cal E$ is bounded above in $\Cal E$)
this makes no difference;  but if, for instance, $\Cal E$ is the family
of compact subsets of a topological space, the two definitions of
$\Cal S$ may not quite coincide.
I believe that on this point, for once, I am following the majority.}

\leader{421C}{Elementary facts (a)} It is worth noting straight away
that if $\Cal E$ is any family of sets, then $\bigcup_{n\in\BbbN}E_n$
and $\bigcap_{n\in\BbbN}E_n$ belong to $\Cal S(\Cal E)$ for any sequence
$\sequencen{E_n}$ in $\Cal E$.   \prooflet{\Prf\ Set

\Centerline{$F_{\sigma}=E_{\sigma(0)}$ for every $\sigma\in S^*$,}

\Centerline{$G_{\sigma}=E_k$ whenever $k\in\Bbb N$, 
$\sigma\in\Bbb N^{k+1}$;}

\noindent then

\Centerline{$\bigcup_{n\in\BbbN}E_n=
\bigcup_{\phi\in\NN}\bigcap_{k\ge 1}F_{\phi\restr k}
\in\Cal S(\Cal E)$,}

\Centerline{$\bigcap_{n\in\BbbN}E_n=
\bigcup_{\phi\in\NN}\bigcap_{k\ge 1}G_{\phi\restr k}\in\Cal
S(\Cal E)$.  \Qed}
}%end of prooflet

In particular, $\Cal E\subseteq\Cal S(\Cal E)$.   \cmmnt{But note that
there is no reason why $E\setminus F$ should belong to $\Cal S(\Cal E)$
for $E$, $F\in\Cal E$.}

\spheader 421Cb Let $X$ and $Y$ be sets, and $f:X\to Y$ a function.
Let $\family{\sigma}{S^*}{F_{\sigma}}$ be a Souslin scheme in $\Cal PY$,
with kernel $B$.   Then $f^{-1}[B]$ is the kernel of the Souslin scheme
$\family{\sigma}{S^*}{f^{-1}[F_{\sigma}]}$.   \prooflet{\Prf\

\Centerline{$f^{-1}[B]
=f^{-1}[\bigcup_{\phi\in\NN}\bigcap_{n\ge 1}F_{\phi\restr n}]
=\bigcup_{\phi\in\NN}\bigcap_{n\ge 1}f^{-1}[F_{\phi\restr n}]$.\ \Qed}
}%end of prooflet

\spheader 421Cc Let $X$ and $Y$ be sets, and $f:X\to Y$ a function.
Let $\Cal F$ be a family of subsets of $Y$.   Then

\Centerline{$\{f^{-1}[B]:B\in\Cal S(\Cal F)\}
=\Cal S(\{f^{-1}[F]:F\in\Cal F\})$.}

\prooflet{\noindent\Prf\ For a set $A\subseteq X$, $A\in
\Cal S(\{f^{-1}[F]:F\in\Cal F\})$ iff there is some Souslin scheme
$\family{\sigma}{S^*}{E_{\sigma}}$ in $\{f^{-1}[F]:F\in\Cal F\}$ such that
$A$ is the kernel of $\family{\sigma}{S^*}{E_{\sigma}}$, that is, iff
there is some Souslin scheme $\family{\sigma}{S^*}{F_{\sigma}}$ in $\Cal
F$ such that $A$ is the kernel of
$\family{\sigma}{S^*}{f^{-1}[F_{\sigma}]}$, that
is, iff $A=f^{-1}[B]$ where $B$ is the kernel of some Souslin scheme in
$\Cal F$.\ \Qed
}%end of prooflet

\spheader 421Cd Let $X$ and $Y$ be sets, and $f:X\to Y$ a surjective
function.   Let $\Cal F$ be a family of subsets of $Y$.   Then

\Centerline{$\Cal S(\Cal F)
=\{B:B\subseteq Y,\,f^{-1}[B]\in\Cal S(\{f^{-1}[F]:F\in\Cal F\})\}$.}

\prooflet{\noindent\Prf\ If $B\in\Cal S(\Cal F)$, then 
$f^{-1}[B]\in\Cal S(\{f^{-1}[F]:F\in\Cal F\})$, by (c) above.   If $B\subseteq Y$ and
$f^{-1}[B]\in\Cal S(\{f^{-1}[F]:F\in\Cal F\})$, then there is a Souslin
scheme  $\family{\sigma}{S^*}{F_{\sigma}}$ in $\Cal F$ such that
$f^{-1}[B]$ is the kernel of
$\family{\sigma}{S^*}{f^{-1}[F_{\sigma}]}$, that is, $f^{-1}[B]=f^{-1}[C]$
where $C$ is the kernel of $\family{\sigma}{S^*}{F_{\sigma}}$.   Because
$f$ is surjective, $B=C\in\Cal S(\Cal F)$.\ \Qed
}%end of prooflet

\spheader 421Ce\cmmnt{ Souslin's operation can be thought of as a
projection operator, as follows.}   Let $\family{\sigma}{S^*}{E_{\sigma}}$
be a Souslin scheme with kernel $A$.   Set

\Centerline{$R=\bigcap_{n\ge
1}\bigcup_{\sigma\in\BbbN^n}I_{\sigma}\times E_{\sigma}$.}

\noindent Then $R[\NN]=A$.   \prooflet{\Prf\ For any $x$, and any
$\phi\in\NN$,

$$\eqalign{(\phi,x)\in R
&\iff\text{ for every }n\ge 1\text{ there is a }\sigma\in \BbbN^n
  \text{ such that }x\in E_{\sigma},\,\phi\in I_{\sigma}\cr
&\iff x\in E_{\phi\restr n}\text{ for every }n\ge 1.\cr}$$

\noindent But this means that

$$\eqalign{x\in R[\NN]
&\iff\text{ there is a }\phi\in\NN
  \text{ such that }(\phi,x)\in R\cr
&\iff\text{ there is a }\phi\in\NN
  \text{ such that }x\in\bigcap_{n\ge 1}E_{\phi\restr n}
\iff x\in A.\text{ \Qed}}$$
}%end of prooflet

\leader{421D}{}\cmmnt{ The first fundamental theorem is that the
operation $\Cal S$ is idempotent.

\medskip

\noindent}{\bf Theorem}\cmmnt{ ({\smc Souslin 1917})} For any family
$\Cal E$ of sets, $\Cal S(\Cal E)$ is closed under Souslin's operation.

\proof{{\bf (a)} Let $\family{\sigma}{S^*}{A_{\sigma}}$ be a family in
$\Cal S(\Cal E)$, and set
$A=\bigcup_{\phi\in\NN}\bigcap_{k\ge 1}A_{\phi\restr k}$;  I have to
show that $A\in\Cal S(\Cal E)$.   For each
$\sigma\in S$, let $\family{\tau}{S^*}{E_{\sigma\tau}}$ be a family in
$\Cal E$ such that $A_{\sigma}
=\bigcup_{\psi\in\NN}\bigcap_{m\ge 1}E_{\sigma,\psi\restr m}$.   Then

$$A
=\bigcup_{\phi\in\NN}\bigcap_{k\ge 1}\bigcup_{\psi\in\NN}
   \bigcap_{m\ge 1}E_{\phi\restr k,\psi\restr m}
=\bigcup_{\Atop{\phi\in\NN}{\pmb{\psi}\in(\NN)^{\BbbN\setminus\{0\}}}}
   \bigcap_{k,m\ge 1}E_{\phi\restr k,\psi_k\restr m},$$

\noindent writing $\pmb{\psi}=\langle\psi_k\rangle_{k\ge 1}$ for
$\pmb{\psi}\in(\NN)^{\BbbN\setminus\{0\}}$.
The idea of the proof is simply that
$\NN\times(\BbbN^{\Bbb N})^{\BbbN\setminus\{0\}}$ is essentially
identical to $\NN$,
so that all we have to do is to organize new names for the
$E_{\sigma\tau}$.   But as it is by no means a trivial matter to devise
a coding scheme which really works, I give the details at length.

\medskip

{\bf (b)} The first step is to note that $S^*$ and $(S^*)^2$ are countable, so
there is a sequence $\sequencen{H_n}$ running over
$\{E_{\sigma\tau}:\sigma$, $\tau\in S^*\}$.   Next, choose any injective
function $q:\BbbN\times\BbbN\to\BbbN\setminus\{0\}$ such that $q(0,0)=1$
and $q(0,1)=2$.   For $k$, $m\ge 1$ set
$J_{km}=\{(i,0):i<k\}\cup\{(i,k):i<m\}$, so that
$J_{11}=\{(0,0),(0,1)\}$,
and choose a family $\langle(k_n,m_n)\rangle_{n\ge 3}$ running over
$(\Bbb N\setminus\{0\})^2$ such that $q[J_{k_n,m_n}]\subseteq n$ for
every $n\ge 3$.   (The pairs $(k_n,m_n)$ need not all be distinct, so
this is easy to achieve.)

Now, for $\upsilon\in\BbbN^n$, where $n\ge 3$, set
$F_{\upsilon}=E_{\sigma\tau}$ where

\Centerline{$\sigma\in\BbbN^{k_n}$, $\sigma(i)=\upsilon(q(i,0))$ for
$i<k_n$,}

\Centerline{$\tau\in\BbbN^{m_n}$, $\tau(i)=\upsilon(q(i,k_n))$ for
$i<m_n$;}

\noindent these are well-defined because $q[J_{k_n,m_n}]\subseteq n$.
For
$\upsilon\in\BbbN^1\cup\BbbN^2$, set $F_{\upsilon}=H_{\upsilon(0)}$.

\medskip

{\bf (c)} This defines a Souslin scheme
$\family{\upsilon}{S^*}{F_{\upsilon}}$ in $\Cal E$.   Let $A'$ be its
kernel,
so that $A'\in\Cal S(\Cal E)$.   The point is that $A'=A$.

\medskip

\Prf\ {\bf (i)} If $x\in A$, there must be $\phi\in\NN$,
$\pmb{\psi}\in(\NN)^{\BbbN\setminus\{0\}}$ such that
$x\in\bigcap_{k,m\ge 1}E_{\phi\restr k,\psi_k\restr m}$.
Choose $\theta\in\NN$ such that

$$\eqalign{H_{\theta(0)}&=E_{\phi\restr 1,\psi_1\restr 1},\cr
\theta(q(i,0))&=\phi(i)\text{ for every }i\in\BbbN,\cr
\theta(q(i,k))&=\psi_k(i)\text{ for every }k\ge 1,\,i\in\BbbN.\cr}$$

\noindent (This is possible because $q:\BbbN^2\to\BbbN\setminus\{0\}$ is
injective.)   Now

\Centerline{$F_{\theta\restr 1}
=F_{\theta\restr 2}
=H_{\theta(0)}
=E_{\phi\restr 1,\psi_1\restr 1}$}

\noindent certainly contains $x$.   And for $n\ge 3$,
$F_{\theta\restr n}=E_{\sigma\tau}$ where $\sigma(i)=\theta(q(i,0))$ for
$i<k_n$, $\tau(i)=\theta(q(i,k_n))$ for $i<m_n$, that is,
$\sigma=\phi\restr k_n$ and $\tau=\psi_{k_n}\restr m_n$, so again
$x\in F_{\theta\restr n}$.   Thus

\Centerline{$x\in\bigcap_{n\ge 1}F_{\theta\restr n}\subseteq A'$.}

\noindent As $x$ is arbitrary, $A\subseteq A'$.

\medskip

\quad{\bf (ii)} Now take any $x\in A'$.   Let $\theta\in\NN$
be
such that $x\in\bigcap_{n\ge 1}F_{\theta\restr n}$.   Define
$\phi\in\Bbb
N^{\BbbN}$, $\pmb{\psi}\in(\NN)^{\BbbN\setminus\{0\}}$ by setting

$$\eqalign{\phi(i)&=\theta(q(i,0))\text{ for }i\in\BbbN,\cr
\psi_k(i)&=\theta(q(i,k))\text{ for }k\ge 1,\,i\in\BbbN.\cr}$$

\noindent If $k$, $m\ge 1$, let $n\ge 3$ be such that $k=k_n$, $m=m_n$.
Then $x\in F_{\theta\restr n}=E_{\sigma\tau}$, where

\Centerline{$\sigma(i)=\theta(q(i,0))$ for $i<k_n$,
\quad$\tau(i)=\theta(q(i,k_n))$ for $i<m_n$,}

\noindent that is, $\sigma=\phi\restr k_n=\phi\restr k$ and
$\tau=\psi_{k_n}\restr m_n=\psi_k\restr m$.   As $m$ and $n$ are
arbitrary,

\Centerline{$x\in\bigcap_{m,n\ge 1}E_{\phi\restr k,\psi_k\restr m}
\subseteq A$.}

\noindent As $x$ is arbitrary, $A'\subseteq A$.\ \Qed

Accordingly we must have $A\in\Cal S(\Cal E)$, and the proof is
complete.
}%end of proof of 421D


\leader{421E}{Corollary} For any family $\Cal E$ of sets,
$\Cal S(\Cal E)$ is closed under countable unions and intersections.

\proof{ For 421Ca tells us that the union and intersection of any
sequence in $\Cal S(\Cal E)$ will belong to
$\Cal S\Cal S(\Cal E)=\Cal S(\Cal E)$.
}%end of proof of 421E

\leader{421F}{Corollary} Let $X$ be a set and $\Cal E$ a family of
subsets of $X$.   Suppose that $X$ and $\emptyset$ belong to
$\Cal S(\Cal E)$ and that $X\setminus E\in\Cal S(\Cal E)$ for every
$E\in\Cal E$.   Then $\Cal S(\Cal E)$ includes the $\sigma$-algebra of subsets of $X$ generated by $\Cal E$.

\proof{ The set

\Centerline{$\Sigma
=\{F:F\in\Cal S(\Cal E),\,X\setminus F\in\Cal S(\Cal E)\}$}

\noindent is closed under complements (necessarily), contains
$\emptyset$
(because $\emptyset$ and $X$ belong to $\Cal S(\Cal E)$), and is also
closed under countable unions, by 421E.   So it is a $\sigma$-algebra;
but the hypotheses also ensure that $\Cal E\subseteq\Sigma$, so that the
$\sigma$-algebra generated by $\Cal E$ is included in $\Sigma$ and in
$\Cal S(\Cal E)$.
}%end of proof of 421F

\leader{421G}{Proposition} Let $\Cal E$ be a family of sets such that
$\emptyset\in\Cal E$.   Then

$$\eqalign{\Cal S(\Cal E)
&=\{R[\NN]:R\in\Cal S(\{I_{\sigma}\times E:\sigma\in S^*,
  \,E\in\Cal E\})\}\cr
&=\{R[\NN]:R\in\Cal S(\{I_{\sigma}\times E:\sigma\in S^*,
  \,E\in\Cal E\}),\,R^{-1}[\{x\}]\text{ is closed for every }x\}.\cr}$$

\proof{ Set $\Cal F=\{I_{\sigma}\times E:\sigma\in S^*,\,E\in\Cal E\}$.

\medskip

{\bf (a)} Suppose first that $A\in\Cal S(\Cal E)$.   Let
$\family{\sigma}{S^*}{E_{\sigma}}$ be a Souslin scheme in $\Cal E$ with
kernel $A$.   Set

\Centerline{$R=\bigcap_{k\ge 1}\bigcup_{\sigma\in\BbbN^k}
  I_{\sigma}\times E_{\sigma}$.}

\noindent Then $R\in\Cal S(\Cal F)$, by 421E, and $R[\NN]=A$,
by 421Ce.   Also

\Centerline{$R^{-1}[\{x\}]
=\bigcap_{k\ge 1}\bigcup\{I_{\sigma}:
  \sigma\in\BbbN^k$, $x\in E_{\sigma}\}$}

\noindent is closed, for every $x$.

\medskip

{\bf (b)} Now suppose that $A=R[\NN]$ for some
$R\in\Cal S(\Cal F)$.   Let
$\family{\sigma}{S^*}{I_{\tau(\sigma)}\times E_{\sigma}}$ be a
Souslin scheme in $\Cal F$ with kernel $R$.   For $k\ge 1$,
$\sigma\in\Bbb N^k$ set

$$\eqalign{F_{\sigma}
&=E_{\sigma}
  \text{ if }\bigcap_{1\le n\le k}I_{\tau(\sigma\restr n)}
    \ne\emptyset,\cr
&=\emptyset\text{ otherwise}.\cr}$$

\noindent Then $\family{\sigma}{S^*}{F_{\sigma}}$ is a Souslin scheme in
$\Cal E$, so its kernel $A'$ belongs to $\Cal S(\Cal E)$.

The point is that $A'=A$.   \Prf\ (i) If $x\in A$, there are a
$\phi\in\Bbb N^{\BbbN}$ such that $(\phi,x)\in R$ and a $\psi\in\NN$
such that $(\phi,x)\in
\bigcap_{n\ge 1}I_{\tau(\psi\restr n)}\times E_{\psi\restr n}$.   Now,
for any $k\ge 1$, we have

\Centerline{$\phi
\in\bigcap_{1\le n\le k}I_{\tau(\psi\restr n)}
=\bigcap_{1\le n\le k}I_{\tau((\psi\restr k)\restr n)}$,}

\noindent so that
$F_{\psi\restr k}=E_{\psi\restr k}$ contains $x$;  thus
$x\in\bigcap_{k\ge 1}F_{\psi\restr k}\subseteq A'$.   As $x$ is
arbitrary, $A\subseteq A'$.
(ii) If $x\in A'$, take $\psi\in\NN$ such that
$x\in\bigcap_{n\ge 1}F_{\psi\restr n}$.   In this case we must have
$F_{\psi\restr k}\ne\emptyset$, so
$\bigcap_{1\le n\le k}I_{\tau(\psi\restr n)}\ne\emptyset$, for every
$k\ge 1$.   But what this means is that, setting
$\tau_n=\tau(\psi\restr n)$ for each $n\ge 1$, $\tau_n(i)=\tau_m(i)$
whenever $i\in\Bbb N$ is such that both are defined.   So
$\{\tau_n:n\ge 1\}$ must have a common extension $\phi\in\NN$, and
$\phi\in\bigcap_{n\ge 1}I_{\tau(\psi\restr n)}$.   Now

\Centerline{$(\phi,x)\in\bigcap_{n\ge 1}I_{\tau(\psi\restr n)}\times
E_{\psi\restr n}\subseteq R$,}

\noindent so $x\in A$.   Thus $A'\subseteq A$ and the two are equal.\
\Qed

This shows that

\Centerline{$\{R[\NN]:R\in\Cal S(\Cal F)\}\subseteq\Cal S(\Cal E)$,}

\noindent and the proof is complete.
}%end of proof of 421G

\leader{421H}{}\cmmnt{ When the class $\Cal E$ is a $\sigma$-algebra,
the last proposition can be extended.

\medskip

\noindent}{\bf Proposition} Let $X$ be a set, and $\Sigma$ a
$\sigma$-algebra of subsets of $X$.   Let $\Cal B$ be the algebra of
Borel subsets of $\NN$.   Then

$$\eqalign{\Cal S(\Sigma)
&=\{R[\NN]:R\in\Cal B\tensorhat\Sigma\}\cr
&=\{R[\NN]:R\in\Cal S(\{I_{\sigma}\times E:
  \sigma\in S^*,\,E\in\Sigma\})\}\cr
&=\{R[\NN]:R\in\Cal S(\Cal B\tensorhat\Sigma)\}.\cr}$$

\cmmnt{\medskip

\noindent{\bf Notation} Recall that $\Cal B\tensorhat\Sigma$ is the
$\sigma$-algebra of subsets of $\NN\times X$ generated by
$\{H\times E:H\in\Cal B,\,E\in\Sigma\}$.
}%end of comment

\proof{{\bf (a)} Suppose first that $A\in\Cal S(\Sigma)$.   As in 421G,
let
$\family{\sigma}{S^*}{E_{\sigma}}$ be a Souslin scheme in $\Sigma$ with
kernel $A$, and set

\Centerline{$R=\bigcap_{k\ge
1}\bigcup_{\sigma\in\BbbN^k}I_{\sigma}\times
E_{\sigma}$,}

\noindent so that $A=R[\NN]$ (421Ce again).
Because every $I_{\sigma}$ is
an open-and-closed set in $\NN$, $R\in\Cal B\tensorhat\Sigma$.   Thus

\Centerline{$\Cal S(\Sigma)
\subseteq\{R[\NN]:R\in\Cal B\tensorhat\Sigma\}$.}

\medskip

{\bf (b)} Set $\Cal F=\{I_{\sigma}\times E:\sigma\in S^*,\,E\in\Sigma\}$.
Then $\Cal S(\Cal B\tensorhat\Sigma)=\Cal S(\Cal F)$.   \Prf\ If
$E\in\Sigma$ and $\sigma\in\BbbN^k$ then

\Centerline{$(\NN\times X)\setminus(I_{\sigma}\times E)
=(I_{\sigma}\times(X\setminus E))\cup\bigcup_{\tau\in\Bbb
N^k,\tau\ne\sigma}I_{\tau}\times X\in\Cal S(\Cal F)$.}

\noindent Also

\Centerline{$\NN\times X=\bigcup_{\sigma\in\Bbb
N^1}I_{\sigma}\times X$,
\quad $\emptyset=I_{\tau}\times\emptyset$}

\noindent (where $\tau$ is any member of $S^*$) belong to $\Cal S(\Cal F)$.
By 421F, $\Cal S(\Cal F)$ includes the $\sigma$-algebra $\Lambda$ of
sets
generated by $\Cal F$.   Now if $E\in\Sigma$ and $H\subseteq\BbbN^{\Bbb
N}$ is open, $H=\bigcup_{\sigma\in T}I_{\sigma}$ for some 
$T\subseteq S^*$;  as $T$ is necessarily countable,

\Centerline{$H\times E
=\bigcup_{\sigma\in T}I_{\sigma}\times E\in\Lambda$.}

\noindent Since $\{F:F\subseteq\NN,\,F\times E\in\Lambda\}$ is
a $\sigma$-algebra of subsets of $\NN$, and we have just seen
that it contains all the open sets, it must include $\Cal B$;  thus
$F\times E\in\Lambda$ for every $F\in\Cal B$, $E\in\Sigma$.   So 
$\Cal B\tensorhat\Sigma\subseteq\Lambda\subseteq\Cal S(\Cal F)$, and

\Centerline{$\Cal S(\Cal F)
\subseteq\Cal S(\Cal B\tensorhat\Sigma)
\subseteq\Cal S\Cal S(\Cal F)=\Cal S(\Cal F)$}

\noindent (421D).\ \Qed

\medskip

{\bf (c)} Now we have

$$\eqalignno{\Cal S(\Sigma)
&\subseteq\{R[\NN]:R\in\Cal B\tensorhat\Sigma\}\cr
\noalign{\noindent (by (a))}
&\subseteq\{R[\NN]:R\in\Cal S(\Cal B\tensorhat\Sigma)\}
=\{R[\NN]:R\in\Cal S(\Cal F)\}\cr
\noalign{\noindent (by (b))}
&=\Cal S(\Sigma)\cr}$$

\noindent by 421G.
}%end of proof of 421H

\leader{421I}{}\cmmnt{ There is a particularly simple description of
sets obtainable by Souslin's operation from closed sets in a topological
space.

\medskip

\noindent}{\bf Lemma} Let $X$ be a topological space and
$R\subseteq\NN\times X$ a closed set.   Then

\Centerline{$R[A]=\bigcup_{\phi\in A}\bigcap_{n\ge
1}\overline{R[I_{\phi\restr n}]}$.}

\noindent for any $A\subseteq\NN$.
In particular, $R[\NN]$ is the kernel of the Souslin scheme
$\family{\sigma}{S^*}{\overline{R[I_{\sigma}]}}$.

\wheader{421I}{0}{0}{0}{36pt}

\proof{ Set

\Centerline{$B=\bigcup_{\phi\in A}
  \bigcap_{n\ge 1}\overline{R[I_{\phi\restr n}]}$.}

\noindent(i) If $x\in R[A]$, there is a $\phi\in A$ such that
$(\phi,x)\in R$.   In this case, $\phi\in I_{\phi\restr n}$ so

\Centerline{$x\in R[I_{\phi\restr n}]
\subseteq\overline{R[I_{\phi\restr n}]}$}

\noindent for every $n$, and $x\in B$.   Thus $R[A]\subseteq B$.  (ii)
If $x\in B$, let $\phi\in A$ be such that $x\in\overline{R[I_{\phi\restr
n}]}$ for every $n\in\Bbb N$.   \Quer\ If $(\phi,x)\notin R$, then
(because $R$ is closed) there are a $\sigma\in S^*$ and an open
$G\subseteq X$ such that $\phi\in I_{\sigma}$, $x\in G$ and
$(I_{\sigma}\times G)\cap R=\emptyset$.   But this means that $G\cap
R[I_{\sigma}]=\emptyset$ so $G\cap \overline{R[I_{\sigma}]}=\emptyset$
and $x\notin\overline{R[I_{\sigma}]}$;  which is absurd, because
$\sigma=\phi\restr n$ for some $n\ge 1$.\ \BanG\  Thus $(\phi,x)\in R$
and $x\in R[A]$.   As $x$ is arbitrary, $B\subseteq R[A]$ and $B=R[A]$,
as required.
}%end of proof of 421I

\leader{421J}{Proposition} Let $X$ be a topological space,
and $\Cal F$ the family of closed subsets of $X$.   Then a set
$A\subseteq X$ belongs to $\Cal S(\Cal F)$ iff there is a closed set
$R\subseteq\NN\times X$ such that $A$ is the projection of $R$
on $X$.

\proof{{\bf (a)} Suppose that $A\in\Cal S(\Cal F)$.   Let
$\family{\sigma}{S^*}{F_{\sigma}}$ be a Souslin scheme in $\Cal F$ with
kernel $A$.   Set

\Centerline{$R=\bigcap_{n\ge 1}\bigcup_{\sigma\in\BbbN^n}
I_{\sigma}\times F_{\sigma}$.}

\noindent For each $n\ge 1$,

\Centerline{$\bigcup_{\sigma\in\BbbN^n}I_{\sigma}\times F_{\sigma}
=(\NN\times X)\setminus\bigcup_{\sigma\in\Bbb
N^n}I_{\sigma}\times(X\setminus F_{\sigma})$}

\noindent is closed in $\NN\times X$, so $R$ is closed;  and
the projection $R[\NN]$ is $A$, by 421Ce.

\medskip

{\bf (b)} Suppose that $R\subseteq\NN$ is a closed set with
projection $A$.   Then $A$ is the kernel of the Souslin scheme
$\family{\sigma}{S^*}{\overline{R[I_{\sigma}]}}$, by 421I, so belongs to
$\Cal S(\Cal F)$.
}%end of proof of 421J

\leader{421K}{Definition} Let $X$ be a topological space.   A subset of
$X$ is a {\bf Souslin-F} set in $X$ if it is obtainable from closed
subsets of $X$ by Souslin's operation\cmmnt{;  that is, is the
projection of a closed subset of $\NN\times X$}.

\cmmnt{For a subset of $\BbbR^r$, or, more generally, of any Polish
space, it is common to say \lq Souslin set' for \lq Souslin-F
set';  see 421Xl.
}%end of comment

\leader{421L}{Proposition} Let $X$ be any topological space.   Then
every Baire subset of $X$ is Souslin-F.

\proof{ Let $\Cal Z$ be the family of zero sets in $X$.   If
$F\in\Cal Z$ then $X\setminus F$ is a countable union of zero sets
(4A2C(b-vi)), so belongs to $\Cal S(\Cal Z)$.   By 421F, the
$\sigma$-algebra generated by
$\Cal Z$ is included in $\Cal S(\Cal Z)\subseteq\Cal S(\Cal F)$, where
$\Cal F$ is the family of closed subsets of $X$;  that is, every Baire
set is Souslin-F.
}%end of proof of 421L

\leader{421M}{Proposition} Let $\Cal E$ be any family of sets such that
$\emptyset\in\Cal E$ and $E\cup E'$, $\bigcap_{n\in\BbbN}E_n$
belong to $\Cal E$ for every $E$, $E'\in\Cal E$ and all sequences
$\sequencen{E_n}$ in $\Cal E$.   \cmmnt{(For instance, $\Cal E$ could
be the family of closed subsets of a topological space, or a
$\sigma$-algebra of sets.)}   Let $\family{\sigma}{S^*}{E_{\sigma}}$ be
a Souslin scheme in $\Cal E$, and $K\subseteq\NN$ a set which
is compact for the usual topology on $\NN$.   Then
$\bigcup_{\phi\in K}\bigcap_{n\ge 1}E_{\phi\restr n}\in\Cal E$.

\proof{ Set $A=\bigcup_{\phi\in K}\bigcap_{n\ge 1}E_{\phi\restr n}$.
For $k\in\Bbb N$, set $K_k=\{\phi\restr k:\phi\in K\}$;  note that
$K_k\subseteq\BbbN^k$ is compact, because $\phi\mapsto\phi\restr k$ is
continuous, therefore finite, because the topology of $\BbbN^k$ is
discrete.   Set

\Centerline{$H=\bigcap_{k\ge 1}\bigcup_{\phi\in K_k}
  \bigcap_{1\le n\le k}E_{\phi\restr n}$.}

\noindent Because $\Cal E$ is closed under finite unions and countable
intersections, $H\in\Cal E$.   Now $A=H$.   \Prf\ (i) If $x\in A$, take
$\phi\in K$ such that $x\in E_{\phi\restr n}$ for every $n\ge 1$;  then
$\phi\restr k\in K_k$ and $x\in\bigcap_{1\le n\le k}E_{(\phi\restr
k)\restr n}$ for every $k\ge 1$, so $x\in H$.   Thus $A\subseteq H$.
(ii) If $x\in
H$, then for each $k\in\Bbb N$ we have a $\sigma_k\in K_k$ such that
$x\in\bigcap_{1\le n\le k}E_{\sigma_k\restr n}$.   Choose $\phi_k\in K$
such that $\phi_k\restr k=\sigma_k$ for each $k$.   Now $K$ is supposed
to be compact, so the sequence $\sequence{k}{\phi_k}$ has a cluster
point $\phi$ in $K$.

If $n\ge 1$, then $I_{\phi\restr n}$ is a neighbourhood of $\phi$ in
$\NN$, so must
contain $\phi_k$ for infinitely many $k$;  let $k\ge n$ be such that
$\phi_k\restr n=\phi\restr n$.   In this case

\Centerline{$x\in E_{\sigma_k\restr n}=E_{\phi_k\restr n}
=E_{\phi\restr n}$.}

\noindent As $n$ is arbitrary,

\Centerline{$x\in\bigcap_{n\ge 1}E_{\phi\restr n}\subseteq A$.}

\noindent As $x$ is arbitrary, $H\subseteq A$ and $H=A$, as claimed.\
\Qed

So $A\in\Cal E$.
}%end of proof of 421M

\leader{*421N}{}\cmmnt{ I now embark on preparations for the theory of
`constituents' of analytic and coanalytic sets.   It turns out that much
of the work can be done in the abstract context of this section.

\medskip

\noindent}{\bf Trees and derived trees (a)} Let $\Cal T$ be the family
of subsets $T$ of $S^*\cmmnt{\mskip5mu=\bigcup_{n\ge 1}\BbbN^n}$ such
that $\sigma\restr k\in T$ whenever
$\sigma\in T$ and $1\le k\le\#(\sigma)$.   Note that the intersection
and union of any non-empty family of members of $\Cal T$ again belong
to $\Cal T$.   \cmmnt{Members of $\Cal T$ are often called {\bf
trees}.}

\spheader 421Nb For $T\in\Cal T$, set

\Centerline{$\partial T
=\{\sigma:\sigma\in S^*,\,\Exists i\in\Bbb N$,
$\sigma^{\smallfrown}\fraction{i}\in T\}$,}

\noindent so that $\partial T\in\Cal T$ and $\partial T\subseteq T$.
\cmmnt{Of course} $\partial T_0\subseteq\partial T_1$ whenever $T_0$,
$T_1\in\Cal T$ and $T_0\subseteq T_1$.

\spheader 421Nc For $T\in\Cal T$, define
$\langle\partial^{\xi}T\rangle_{\xi<\omega_1}$ inductively by setting
$\partial^0T=T$ and, for $\xi>0$,
$\partial^{\xi}T=\bigcap_{\eta<\xi}\partial(\partial^{\eta}T)$.
\cmmnt{An easy induction shows that} $\partial^{\xi}T\in\Cal T$,
$\partial^{\xi}T\subseteq\partial^{\eta}T$ and
$\partial^{\xi+1}T=\partial(\partial^{\xi}T)$ whenever
$\eta\le\xi<\omega_1$.

\spheader 421Nd For any $T\in\Cal T$, there is a $\xi<\omega_1$ such
that $\partial^{\xi}T=\partial^{\eta}T$ whenever $\xi\le\eta<\omega_1$.
\prooflet{\Prf\ Set $T_1=\bigcap_{\xi<\omega_1}\partial^{\xi}T$.   For
each $\sigma\in S^*\setminus T_1$, there is a $\xi_{\sigma}<\omega_1$
such that $\sigma\notin\partial^{\xi_{\sigma}}T$.   Set
$\xi=\sup\{\xi_{\sigma}:\sigma\in S^*\setminus T_1\}$;  because $S^*$ is
countable, $\xi<\omega_1$, and we must now have $\partial^{\xi}T=T_1$,
so that $\partial^{\xi}T=\partial^{\eta}T$ whenever
$\xi\le\eta<\omega_1$.\ \Qed}

\spheader 421Ne For $T\in\Cal T$, its {\bf rank} is the first ordinal
$r(T)<\omega_1$ such that $\partial^{r(T)}T=\partial^{r(T)+1}T$;
\cmmnt{of course} $\partial^{r(T)}T=\partial^{\eta}T$ whenever
$r(T)\le\eta<\omega_1$, and
$\partial(\partial^{r(T)}T)=\partial^{r(T)}T$.

\spheader 421Nf For $T\in\Cal T$, the following are equiveridical:
($\alpha$) $\partial^{r(T)}T\ne\emptyset$;  ($\beta$) there is a
$\phi\in\NN$ such that $\phi\restr n\in T$ for every $n\ge 1$.
\prooflet{\Prf\ (i) If $\sigma\in\partial^{r(T)}T$ then
$\sigma\in\partial(\partial^{r(T)}T)$ so there is an $i\in\Bbb N$ such
that $\sigma^{\smallfrown}\fraction{i}\in\partial^{r(T)}T$.
We can therefore choose
$\sequencen{\sigma_n}$ inductively so that
$\sigma_n\in\partial^{r(T)}T$ and $\sigma_{n+1}$ properly extends
$\sigma_n$ for every $n$.   At the end of the induction,
$\phi=\bigcup_{n\in\Bbb N}\sigma_n$ belongs to $\NN$ and

\Centerline{$\phi\restr n=\sigma_n\restr n\in\partial^{r(T)}T\subseteq T$}

\noindent for every $n\ge 1$.   (ii) If $\phi\in\NN$ is such that
$\phi\restr n\in T$ for every $n\ge 1$, then an easy induction shows
that
$\phi\restr n\in\partial^{\xi}T$ for every $\xi<\omega_1$ and every
$n\ge 1$, so that $\partial^{r(T)}T$ is non-empty.\ \Qed}

\spheader 421Ng Now suppose that $\family{\sigma}{S^*}{A_{\sigma}}$ is a
Souslin scheme.   For any $x$ we have a tree $T_x\in\Cal T$ defined by
saying that

\Centerline{$T_x=\{\sigma:\sigma\in S^*,
  \,x\in\bigcap_{1\le i\le\#(\sigma)}A_{\sigma\restr i}\}$.}

\noindent Now the kernel of $\family{\sigma}{S^*}{A_{\sigma}}$ is just

$$\eqalign{A&=\{x:\,\Exists\phi\in\NN,\,
  x\in\bigcap_{n\ge 1}A_{\phi\restr n}\}\cr
&=\{x:\,\Exists\phi\in\NN,\,\phi\restr n\in T_x\Forall n\ge 1\}
=\{x:\partial^{r(T)}T\ne\emptyset\}\dvro{.}{}\cr}$$

\cmmnt{\noindent by (f).}

The sets

\Centerline{$\{x:x\in X\setminus A,\,r(T_x)=\xi\}
=\{x:x\in X,\,r(T_x)=\xi,\,\partial^{\xi}T_x=\emptyset\}$,}

\noindent for $\xi<\omega_1$, are called {\bf constituents} of
$X\setminus A$.   \cmmnt{(Of course they should properly be called
`the constituents of the Souslin scheme
$\family{\sigma}{S^*}{A_{\sigma}}$'.)}

\vleader{72pt}{*421O}{Theorem} Let $X$ be a set and $\Sigma$ a
$\sigma$-algebra
of subsets of $X$.   Let $\family{\sigma}{S^*}{A_{\sigma}}$ be a Souslin
scheme in $\Sigma$ with kernel $A$, and for $x\in X$ set

\Centerline{$T_x=\{\sigma:\sigma\in S^*,\,
  x\in\bigcap_{1\le i\le\#(\sigma)}
  A_{\sigma\restr i}\}\dvro{.}{\mskip5mu \in\Cal T}$}

\cmmnt{\noindent as in 421Ng.}

(a) For every $\xi<\omega_1$ and $\sigma\in S^*$,
$\{x:x\in X,\,\sigma\in\partial^{\xi}T_x\}\in\Sigma$.

(b) For every $\xi<\omega_1$, $\{x:x\in A,\,r(T_x)\le\xi\}$ and
$\{x:x\in X\setminus A,\,r(T_x)\le\xi\}$ belong to $\Sigma$.
In particular, all the constituents of $X\setminus A$ belong to
$\Sigma$.

\proof{{\bf (a)} Induce on $\xi$.   For $\xi=0$, we have

\Centerline{$\{x:x\in X,\,\sigma\in\partial^0T_x\}
=\{x:x\in X,\,\sigma\in T_x\}
=\bigcap_{1\le i\le\#(\sigma)}A_{\sigma\restr i}\in\Sigma$.}

\noindent For the inductive step to $\xi>0$, we have

$$\eqalign{\{x:\sigma\in\partial^{\xi}T_x\}
&=\{x:\sigma\in\bigcap_{\eta<\xi}\partial(\partial^{\eta}T_x)\}\cr
&=\bigcap_{\eta<\xi}\bigcup_{i\in\Bbb N}
  \{x:\sigma^{\smallfrown}\fraction{i}\in\partial^{\eta}T_x\}
\in\Sigma\cr}$$

\noindent because $\xi$ is countable and all the sets
$\{x:\sigma^{\smallfrown}\fraction{i}\in\partial^{\eta}T_x\}$ belong to
$\Sigma$ by the inductive hypothesis.

\medskip

{\bf (b)} Now, given $\xi<\omega_1$, we see that $r(T_x)\le\xi$ iff
$\partial^{\xi+1}T_x\supseteq\partial^{\xi}T_x$, so that if we set
$E_{\xi}=\{x:x\in X,\,r(T_x)\le\xi\}$ then

\Centerline{$E_{\xi}
=\bigcap_{\sigma\in S^*}\{x:x\in X,\,\sigma\in\partial^{\xi+1}T_x
  \text{ or }\sigma\notin\partial^{\xi}T_x\}$}

\noindent belongs to $\Sigma$.   If $x\in E_{\xi}$, so that
$\partial^{r(T_x)}T_x=\partial^{\xi}T_x$, 421Ng tells us that $x\in A$
iff $\partial^{\xi}T_x\ne\emptyset$;  so that

\Centerline{$E_{\xi}\cap A
=E_{\xi}\cap\bigcup_{\sigma\in S^*}\{x:\sigma\in\partial^{\xi}T_x\}$}

\noindent and $E_{\xi}\setminus A$ both belong to $\Sigma$.

Now the constituents of $X\setminus A$ are the sets
$(E_{\xi}\setminus A)\setminus\bigcup_{\eta<\xi}E_{\eta}$ for
$\xi<\omega_1$, which all belong to $\Sigma$.
}%end of proof of 421O

\leader{*421P}{Corollary} Let $X$ be a set and $\Sigma$ a
$\sigma$-algebra of subsets of $X$.   If $A\in\Cal S(\Sigma)$ then both
$A$ and $X\setminus A$ can be expressed as the union of at most
$\omega_1$ members of $\Sigma$.

\proof{ In the language of 421O, we have

\Centerline{$A=\bigcup_{\xi<\omega_1}E_{\xi}\cap A$,
\quad$X\setminus A=\bigcup_{\xi<\omega_1}E_{\xi}\setminus A$.}
}%end of proof of 421P

\leader{*421Q}{Lemma} Let $X$ be a set and
$\family{\sigma}{S^*}{A_{\sigma}}$ and $\family{\sigma}{S^*}{B_{\sigma}}$
two Souslin schemes of subsets of $X$.   Suppose that whenever $\phi$,
$\psi\in\NN$ there is an $n\ge 1$ such that
$\bigcap_{1\le i\le n}A_{\phi\restr i}\cap B_{\psi\restr i}=\emptyset$.
For $x\in X$ set

\Centerline{$T_x=\bigcup_{n\ge 1}\{\sigma:
  \sigma\in\BbbN^n,\,x\in\bigcap_{1\le i\le n}A_{\sigma\restr i}\}$}

\noindent\cmmnt{as in 421Ng, }and let $B$ be the kernel of
$\family{\sigma}{S^*}{B_{\sigma}}$.   Then $\sup_{x\in B}r(T_x)<\omega_1$.

\proof{ For $\sigma\in S^*$ set
$A'_{\sigma}=\bigcap_{1\le i\le\#(\sigma)}A_{\sigma\restr i}$,
$B'_{\sigma}=\bigcap_{1\le i\le\#(\sigma)}B_{\sigma\restr i}$.   Then
$T_x=\{\sigma:\sigma\in S^*,\,x\in A'_{\sigma}\}$ for each $x\in X$, $B$
is the kernel of $\family{\sigma}{S^*}{B'_{\sigma}}$, and for every
$\phi$, $\psi\in\NN$ there is an $n\in\Bbb N$ such that
$A'_{\phi\restr n}\cap B'_{\psi\restr n}=\emptyset$.

Define $\ofamily{\xi}{\omega_1}{Q_{\xi}}$ inductively by setting

\Centerline{$Q_0=\{(\sigma,\tau):\sigma,\,\tau\in S^*,
  \,A'_{\sigma}\cap B'_{\tau}\ne\emptyset\}$,}

\noindent and, for $0<\xi<\omega_1$,

\Centerline{$Q_{\xi}
=\bigcap_{\eta<\xi}\{(\sigma,\tau):
  \sigma,\,\tau\in S^*,\,\Exists i,\,j\in\Bbb N,
  \,(\sigma^{\smallfrown}\fraction{i},\tau^{\smallfrown}\fraction{j})
  \in Q_{\eta}\}$.}

\noindent Then the same arguments as in 421Na-421Nd show that there is a
$\zeta<\omega_1$ such that $Q_{\zeta+1}=Q_{\zeta}$.   \Quer\ If
$Q_{\zeta}\ne\emptyset$, then, just as in 421Nf, there must be $\phi$,
$\psi\in\NN$ such that
$(\phi\restr m,\psi\restr n)\in Q_{\zeta}\subseteq Q_0$ for every $m$,
$n\ge 1$;  but this means that
$A'_{\phi\restr n}\cap B'_{\psi\restr n}\ne\emptyset$ for every
$n\ge 1$, which is supposed to be impossible.\ \Bang

Now suppose that $x\in B$.   Then there is a $\psi\in\NN$ such that
$x\in B'_{\psi\restr n}$ for every $n\ge 1$.   But this means that
$(\sigma,\psi\restr n)\in Q_0$ for every $\sigma\in T_x$ and every
$n\ge 1$.  An easy induction shows that
$(\sigma,\psi\restr n)\in Q_{\xi}$ whenever $\xi<\omega_1$,
$\sigma\in\partial^{\xi}T_x$ and $n\ge 1$.   But as
$Q_{\zeta}=\emptyset$ we must have $\partial^{\zeta}T_x=\emptyset$ and
$r(T_x)\le\zeta$.   Thus $\sup_{x\in B}r(T_x)\le\zeta<\omega_1$, and the
proof is complete.
}%end of proof of 421Q

\exercises{\leader{421X}{Basic exercises (a)}
%\spheader 421Xa
Let $X$ be a set and $\Cal E$ a family of subsets of $X$.
(i) Show that $\emptyset\in\Cal S(\Cal E)$ iff there is a sequence in
$\Cal E$ with empty intersection.   (ii) Show that $X\in\Cal S(\Cal E)$
iff there is a sequence in $\Cal E$ with union $X$.
%421C

\spheader 421Xb Let $\Cal E$ be a family of sets and $F$ any set.   Show
that

\Centerline{$\Cal S(\{E\cap F:E\in\Cal E\})
=\{A\cap F:A\in\Cal S(\Cal E)\}$,}

\Centerline{$\Cal S(\{E\cup F:E\in\Cal E\})
=\{A\cup F:A\in\Cal S(\Cal E)\}$.}
%421C


\spheader 421Xc Suppose that $\Cal E$ is a family of sets with $\#(\Cal
E)\le\frak c$.   Show that $\#(\Cal S(\Cal E))\le\frak c$.
\Hint{$\#(\Cal E^{S^*})\le\#((\Cal P\BbbN)^{S^*})
=\#(\Cal P(\BbbN\times S^*))$.}
%421C

\spheader 421Xd Let $\Cal E$ be the family of half-open intervals
$\coint{2^{-n}k,2^{-n}(k+1)}$, where $n\in\Bbb N$, $k\in\Bbb Z$;  let
$\Cal G$ be the set of open subsets of $\Bbb R$;  let $\Cal F$ be the
set of closed subsets of $\Bbb R$;  let $\Cal K$ be the set of compact
subsets of $\Bbb R$;  let $\Cal B$ be the Borel $\sigma$-algebra of
$\Bbb R$.
Show that $\Cal S(\Cal E)=\Cal S(\Cal F)=\Cal S(\Cal G)=\Cal S(\Cal
K)=\Cal S(\Cal B)$.
\Hint{421F.}
%421F

\spheader 421Xe Let $\Cal I$ be the family
$\{I_{\sigma}:\sigma\in\bigcup_{k\in\BbbN}\BbbN^k\}$ (421A);   let $\Cal
G$ be the set of open subsets of $\NN$;  let $\Cal F$ be the
set of closed subsets of $\NN$;  let $\Cal K$ be the set of
compact subsets of $\NN$;  let $\Cal B$ be the Borel $\sigma$-algebra
of $\NN$.   Show that $\Cal S(\Cal I)=\Cal S(\Cal F)=\Cal
S(\Cal G)=\Cal S(\Cal B)$, but that $\Cal S(\Cal K)$ is strictly smaller
than these.   \Hint{if $\sequencen{K_n}$ is any sequence in $\Cal K$,
set $\phi(i)=1+\sup_{\psi\in K_i}\psi(i)$ for each $i\in\Bbb N$, so that
$\phi\notin\bigcup_{n\in\BbbN}K_n$;  hence show that $\BbbN^{\Bbb
N}\notin\Cal S(\Cal K)$.}
%421F, 421Xd

\spheader 421Xf Let $X$ be a separable metrizable space with at least
two points;  let $\Cal U$ be any base for its topology, and $\Cal B$ its
Borel $\sigma$-algebra.   Show that $\Cal S(\Cal U)=\Cal S(\Cal B)$.
What can happen if $\#(X)\le 1$?   What about hereditarily Lindel\"of
spaces?
%421F, 421Xd

\spheader 421Xg Let $X$ be a topological space;  let $\Cal Z$ be the set
of zero sets in $X$, $\Cal G$ the set of cozero sets, and $\CalBa$ the
Baire $\sigma$-algebra.   Show that
$\Cal S(\Cal Z)=\Cal S(\Cal G)=\Cal S(\CalBa)$.
%421F, 421Xd

\spheader 421Xh Let $X$ be a set, $\Cal E$ a family of subsets of $X$,
and $\Sigma$ the $\sigma$-algebra of subsets of $X$ generated by
$\Cal E$.   Show that if $\#(\Cal E)\le\frak c$ then
$\#(\Sigma)\le\frak c$.
\Hint{$\#(\frak c^{S^*})=\#(\Cal P(\Bbb N\times\Bbb N))=\frak c$ and
$\Sigma\subseteq\Cal S(\Cal E\cup\{X\setminus E:E\in\Cal E\})$.}
%421F, 421Xc

\spheader 421Xi Let $X$ be a topological space such that every open set
is Souslin-F.   Show that every Borel set is Souslin-F.
%421F

\spheader 421Xj Let $X$ be a topological space and $\Cal B(X)$ its Borel
$\sigma$-algebra.   Show that $\Cal S(\Cal B(X))$ is just the set of
projections on $X$ of Borel subsets of $\NN\times X$.   \Hint{4A3G.}
%421H

\spheader 421Xk Let $X$ and $Y$ be topological spaces, $f:X\to Y$ a
continuous function and $F\subseteq Y$ a Souslin-F set.   Show that
$f^{-1}[F]$ is a Souslin-F set in $X$.
%421K

\spheader 421Xl Let $X$ be any perfectly normal topological space (e.g.,
any metrizable space);  let $\Cal G$ be the
set of open subsets of $X$, $\Cal F$ the set of closed subsets, and
$\Cal B$ the Borel $\sigma$-algebra.   Show that
$\Cal S(\Cal G)=\Cal S(\Cal F)=\Cal S(\Cal B)$.
%421L, 421Xd

\spheader 421Xm Let us say that a Souslin scheme
$\family{\sigma}{S^*}{E_{\sigma}}$ is {\bf regular} if
$E_{\sigma}\subseteq E_{\tau}$ whenever $\sigma$, $\tau\in S^*$,
$\#(\tau)\le\#(\sigma)$ and
$\sigma(i)\le\tau(i)$ for every $i<\#(\sigma)$.   Let $\Cal E$ be a
family of sets such that $E\cup F$ and $E\cap F$ belong to
$\Cal E$ for all $E$,
$F\in\Cal E$.   Show that every member of $\Cal S(\Cal E)$ can be
expressed as the kernel of a regular Souslin scheme in $\Cal E$.
\Hint{if $\family{\sigma}{S^*}{E_{\sigma}}$ is any Souslin scheme in 
$\Cal E$ with kernel $A$, set
$F_{\sigma}=\bigcap_{\tau\subseteq\sigma}E_{\tau}$,
$G_{\sigma}=\bigcup_{\tau\le\sigma}F_{\tau}$, where $\tau\le\sigma$ if
$\tau(i)\le\sigma(i)$ for $i<\#(\tau)=\#(\sigma)$;  show that $A$ is the
kernel of $\family{\sigma}{S^*}{F_{\sigma}}$ and of
$\family{\sigma}{S^*}{G_{\sigma}}$, using an idea from 421M for the
latter.}
%421M

\sqheader 421Xn Let $X$ be a Hausdorff topological space and
$\family{\sigma}{S^*}{K_{\sigma}}$ a Souslin scheme in which every
$K_{\sigma}$ is a compact subset in $X$.   Show that
$\bigcup_{\phi\in K}\bigcap_{n\ge 1}K_{\phi\restr n}$ is compact for any
compact $K\subseteq\NN$.
%421M

\leader{421Y}{Further exercises (a)}
%\spheader 421Ya
Let $X$ be a topological space, $Y$ a Hausdorff space and
$f:X\to Y$ a continuous function.   Let $\Cal K$ be the family of closed
countably compact subsets of $X$.   Show that for any
$\Cal E\subseteq\Cal K$ such that $E\cap F\in\Cal E$ for all $E$,
$F\in\Cal E$,

\Centerline{$\{f[A]:A\in\Cal S(\Cal E)\}=\Cal S(\{f[E]:E\in\Cal E\})$.}
%421C

\spheader 421Yb Let $\Cal E$ be a family of sets and $F$ any set.   Show
that

$$\eqalign{\Cal S(\Cal E\cup\{F\})
&=\{F\}\cup\{A\cap F:A\in\Cal S(\Cal E)\}
  \cup\{B\cup F:B\in\Cal S(\Cal E)\}\cr
&\qquad\qquad\qquad\cup\{(A\cap F)\cup B:A,\,B\in\Cal S(\Cal E)\}.\cr}$$
%421C  mt42bits

\spheader 421Yc Let $X$ be a topological space, and $\CalBa$ its Baire
$\sigma$-algebra.   Show that $\Cal S(\CalBa)$ is just the
family of sets expressible as $f^{-1}[B]$ where $f$ is a continuous
function from $X$ to some metrizable space $Y$ and
$B\subseteq Y$ is Souslin-F.
%421F, 421Xg

\spheader 421Yd Let $X$ be a set, $\Cal E$ a family of subsets of $X$,
and $\Sigma$ the smallest $\sigma$-algebra of subsets of $X$ including
$\Cal E$ and closed under Souslin's operation.   Show that if
$\#(\Cal E)\le\frak c$ then $\#(\Sigma)\le\frak c$.   \Hint{define
$\ofamily{\xi}{\omega_1}{\Cal E_{\xi}}$ by setting
$\Cal E_{\xi}
=\Cal S(\{X\setminus E:E\in\Cal E\cup\bigcup_{\eta<\xi}\Cal E_{\xi}\})$
for each $\xi$.   Show that $\#(\Cal E_{\xi})\le\frak c$ for every $\xi$
and that $\Sigma=\bigcup_{\xi<\omega_1}\Cal E_{\xi}$.}

\spheader 421Ye Let $X$ be a compact space and $A$ a Souslin-F set in
$X$.   Show that there is a family $\ofamily{\xi}{\omega_1}{F_{\xi}}$ of
Borel sets such that $X\setminus A=\bigcup_{\xi<\omega}F_{\xi}$ and
whenever $B\subseteq X\setminus A$ is a Souslin-F set there is a
$\xi<\omega_1$ such that $B\subseteq F_{\xi}$.   \Hint{take
$F_{\xi}=\{x:r(T_x)\le\xi\}\setminus A$ as in 421Ob, and apply 421Q.}
}%end of exercises

\endnotes{
\Notesheader{421} In 111G, I defined the Borel sets of $\Bbb R$ to be
the members of the
smallest $\sigma$-algebra containing every open set.   In 114E, I
defined a set to be Lebesgue measurable if it behaves in the right way
with respect to Lebesgue outer measure.   The latter formulation, at
least, provides some sort of testing principle to determine whether a
set is Lebesgue measurable.   But the definition of `Borel set' does
not.   The only tool so far available for proving that a set
$E\subseteq\Bbb R$ is {\it not}
Borel is to find a $\sigma$-algebra containing all open sets and not
containing $E$;  conversely, the only method we have for proving
properties of Borel sets is to show that a property is possessed by
every member of some $\sigma$-algebra containing every open set.   The
revolutionary insight of {\smc Souslin 1917} was a construction which
could build every Borel set from rational intervals.   (See 421Xd.)
For fundamental reasons, no construction of this kind can provide all
Borel sets without also producing other sets, and to actually
characterize the Borel $\sigma$-algebra a
further idea is needed (423Fa);  but the class of analytic sets, being
those constructible by Souslin's operation from rational intervals (or
open sets, or closed sets, or Borel sets -- the operation is robust
under such variations), turns out to have remarkable properties which
make it as important in modern real analysis as the Borel algebra
itself.

The guiding principle of `descriptive set theory' is that
the properties of a
set may be analysed in the light of a construction for that set.   Thus
we can think of a closed set $F\subseteq\Bbb R$ as

\Centerline{$\Bbb R\setminus\bigcup_{(q,q')\in I}\ooint{q,q'}$}

\noindent where $I\subseteq\Bbb Q\times\Bbb Q$.   The principle can be
effective because we often have such descriptions in terms of objects
fundamentally simpler than the set being described.   In the formula
above, for instance, $\Bbb Q\times\Bbb Q$ is simpler than the set $F$,
being a countable set with a straightforward description from $\Bbb N$.
The set $\Cal P(\Bbb Q\times\Bbb Q)$ is relatively complex;  but a
single subset $I$ of $\Bbb Q\times\Bbb Q$ can easily be coded as a
single subset of $\Bbb N$ (taking some more or less natural enumeration
of $\Bbb Q^2$ as a
sequence $\sequencen{(q_n,q'_n)}$, and matching $I$ with
$\{n:(q_n,q'_n)\in I\}$).   So, subject to an appropriate coding, we
have a description of
closed subsets of $\Bbb R$ in terms of subsets of $\Bbb N$.   At the
most elementary level, this shows that there are at most $\frak c$
closed subsets of $\Bbb R$.   But we can also set out to analyse such
operations as intersection, union, closure in terms of these
descriptions.   The
details are complex, and I shall go no farther along this path until
Chapter 56 in Volume 5;
but investigations of this kind are at the heart of some of the most
exciting developments of twentieth-century real analysis.

The particular descriptive method which concerns us in the present
section is Souslin's operation.   Starting from a relatively simple
class $\Cal E$, we proceed to the larger class $\Cal S(\Cal E)$.   The
most fundamental property of $\Cal S$ is 421D:
$\Cal S\Cal S(\Cal E)=\Cal S(\Cal E)$.
This means, for instance, that if $\Cal E\subseteq\Cal S(\Cal F)$ and
$\Cal F\subseteq\Cal S(\Cal E)$, then $\Cal S(\Cal E)$ will be equal to
$\Cal S(\Cal F)$;  consequently, different classes of sets will often
have the same Souslin closures, as in 421Xd-421Xg.   After a little
practice you will find that it is often easy to see when two classes
$\Cal E$ and $\Cal F$ are at the same level in this sense;  but watch
out for traps like the
class of compact subsets of $\NN$ (421Xe) and odd technical
questions (421Xf).

Souslin's operation, and variations on it, will be the basis of much of
the next chapter;  it has dramatic applications in general topology and
functional analysis as well as in real analysis and measure theory.   An
important way of looking at the kernel of a Souslin scheme
$\family{\sigma}{S^*}{E_{\sigma}}$ is to regard it as the projection on
the second coordinate of the corresponding set
$R=\bigcap_{k\ge 1}\bigcup_{\sigma\in\BbbN^k}I_{\sigma}\times E_{\sigma}$
(421Ce).   We find that many other sets $R\subseteq\NN\times X$ will
also have projections in $\Cal S(\Cal E)$ (421G, 421H).   Let me remark
that it is essential here that the first coordinate should be of the
right type.   In one sense, indeed, $\NN$ is the only thing that
will do;  but its virtue transfers to analytic spaces, as we shall see in
423M-423O below.  %423M 423N 423O
We shall often
want to deal with members of $\Cal S(\Cal E)$ which are most naturally
defined in terms of some such auxiliary space.

I have moved into slightly higher gear for 421N-421Q
%421N 421O 421P 421Q
because these are not essential for most of the work of the next
chapter.   From the point of view of this section 421P is very striking
but the significance of 421Q is unlikely to be apparent.   It becomes
important in contexts in which the condition

\Centerline{$\Forall\phi$, $\psi\in\NN\Exists n\ge 1,\,
\bigcap_{1\le i\le n}A_{\phi\restr i}\cap B_{\psi\restr i}=\emptyset$}

\noindent is satisfied for natural reasons.   I will expand on these in
the next two sections.   In the meantime, I offer 421Ye as an example
of what 421O and 421Q together can tell us.
}%end of notes

\discrpage

