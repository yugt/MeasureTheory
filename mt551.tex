\frfilename{mt551.tex}
\versiondate{2.12.13}
\copyrightdate{2007}

\def\chaptername{Possible worlds}
\def\sectionname{Forcing with quotient algebras}

\def\BbbPk{\Bbb P_{\kappa}}
\def\VVdPk{\VVdash_{\Bbb P_{\kappa}}}


\newsection{551}

In preparation for the discussion of random real forcing in the next two
sections, I
introduce some techniques which can be applied whenever a forcing
notion is described in terms of a Loomis-Sikorski representation of 
its regular
open algebra.   The first step is just a translation of the correspondence
between names for real numbers in the forcing language and members of
$L^0(\RO(\Bbb P))$, as described in 5A3L, when $L^0(\RO(\Bbb P))$ can be
identified with a quotient of a space $L^0(\Sigma)$ of measurable
functions.   More care is needed,
but we can find a similar formulation of names for members
of $\{0,1\}^I$ for any set $I$ (551C).   Going a step farther,
it turns out that there are very useful descriptions
of Baire subsets of $\{0,1\}^I$ (551D-551F), Baire
measurable functions (551N),
the usual measure on $\{0,1\}^I$ (551I-551J) and its measure
algebra (551P).   In some special cases, these methods can be used to
represent iterated forcing notions (551Q).
I end with a construction for a forcing
extension of a filter on a countable set (551R).

\leader{551A}{Definition (a)} A {\bf measurable space with negligibles} is
a triple $(\Omega,\Sigma,\Cal I)$ where $\Omega$ is a set, $\Sigma$ is a
$\sigma$-algebra of subsets of $\Omega$ and $\Cal I$ is a $\sigma$-ideal of
subsets of $\Omega$ generated by $\Sigma\cap\Cal I$.
In this case $\frak A=\Sigma/\Sigma\cap\Cal I$ is a
Dedekind $\sigma$-complete Boolean algebra\cmmnt{ (314C)}.

\cmmnt{In this context I will use the phrase
`$\Cal I$-almost everywhere' to mean `except on a
set belonging to $\Cal I$'.}

\spheader 551Ab\cmmnt{ I will say that} $(\Omega,\Sigma,\Cal I)$ is
{\bf non-trivial} if $\Omega\notin\Cal I$, so that $\frak A\ne\{0\}$.
In this case, the forcing notion $\Bbb P$
{\bf associated} with $(\Omega,\Sigma,\Cal I)$ is
$(\frak A^+,\Bsubseteqshort,\Omega^{\ssbullet},\downarrow)$\cmmnt{
(5A3Ab)}.   If $\frak A$ is Dedekind complete we can identify
$\frak A$ with the regular open algebra
$\RO(\Bbb P)$\cmmnt{ (514Sb, 5A3M)}.

\spheader 551Ac\cmmnt{ I will say that} $(\Omega,\Sigma,\Cal I)$ is {\bf
$\omega_1$-saturated} if $\Sigma\cap\Cal I$ is $\omega_1$-saturated in
$\Sigma$\cmmnt{ in the sense of 541A, that is, if there is no
uncountable disjoint family in
$\Sigma\setminus\Cal I$, that is, if $\frak A$ and $\Bbb P$ are ccc}.
In this case, $\frak A$ is Dedekind complete\cmmnt{ (316Fa, 541B)}.

\spheader 551Ad\cmmnt{ I will say that} $(\Omega,\Sigma,\Cal I)$ is
{\bf complete} if $\Cal I\subseteq\Sigma$\cmmnt{ (cf.\ 211A)}.

\cmmnt{\medskip

\noindent{\bf Remark} For an account of the general theory of measurable
spaces with negligibles, see {\smc Fremlin 87}.
}

\leader{551B}{Definition} Let $(\Omega,\Sigma,\Cal I)$ be a
non-trivial measurable space
with negligibles with a Dedekind complete quotient algebra $\frak A$,
and $\Bbb P$ its associated forcing notion.
\cmmnt{Recall from 364Ib
that }$L^0(\frak A)$ can be regarded as a quotient of
the space of $\Sigma$-measurable functions from $\Omega$ to
$\Bbb R$.   If $h:\Omega\to\Bbb R$ is $\Sigma$-measurable, write
$\vec h=(h^{\ssbullet})\sspvec$ where $h^{\ssbullet}$ is the equivalence
class of $h$ in $L^0(\frak A)$, identified with $L^0(\RO(\Bbb P))$, and
$(h^{\ssbullet})\sspvec$ is the $\Bbb P$-name for a real number as
defined in 5A3L.   Then

\Centerline{$\VVdP\,\vec h$ is a real number,}

\noindent and for any $\alpha\in\Bbb Q$

\Centerline{$\Bvalue{\vec h>\check\alpha}
=\Bvalue{(h^{\ssbullet})\sspvec>\check\alpha}
=\Bvalue{h^{\ssbullet}>\alpha}
=\{\omega:h(\omega)>\alpha\}^{\ssbullet}$.}

\noindent From 5A3Lc, we see that if $h_0$, $h_1$ are
$\Sigma$-measurable real-valued functions on $\Omega$, then

\Centerline{$\VVdP\,(h_0+h_1)\sspvec=\vec h_0+\vec h_1$,
$(h_0\times h_1)\sspvec=\vec h_0\times\vec h_1$,}

\noindent and that if $\sequencen{h_n}$ is a sequence of measurable
functions with limit $h$,

\Centerline{$\VVdP\,\vec h=\lim_{n\to\infty}\vec h_n$ in
$\Bbb R$.}

\leader{551C}{Definition} Let $(\Omega,\Sigma,\Cal I)$ be a
non-trivial measurable space
with negligibles with a Dedekind complete quotient algebra $\frak A$,
and $\Bbb P$ its associated forcing notion.

\spheader 551Ca If $f:\Omega\to\{0,1\}$ is $\Sigma$-measurable, let
$\vec f$ be the $\Bbb P$-name

\Centerline{$\{(\varchecki,f^{-1}[\{i\}]^{\ssbullet}):i\in\{0,1\}$,
$f^{-1}[\{i\}]\notin\Cal I\}$.}

\noindent\cmmnt{Then }$\VVdP\,\vec f\in\{0,1\}$ and
$\Bvalue{\vec f=\varchecki}=f^{-1}[\{i\}]^{\ssbullet}$ for both $i$.
\cmmnt{(I will try always
to make it clear when this definition of $\vec f$ is intended to overrule
the definition in 551B;  but we see from 551Xf that any confusion is
unlikely to matter.)}

Observe that if a $\Bbb P$-name $\dot x$ and $p\in\frak A^+$ are
such that $p\VVdP\,\dot x\in\{0,1\}$, then there is a measurable
$f:\Omega\to\{0,1\}$ such that $p\VVdP\,\dot x=\vec f$\prooflet{;
take $f=\chi E$ where $E\in\Sigma$ is such that
$E^{\ssbullet}=\Bvalue{\dot x=1}$ in $\frak A$}.

\spheader 551Cb Now let $I$ be any set, and $f:\Sigma\to\{0,1\}^I$ a
$(\Sigma,\CalBa_I)$-measurable function, where
$\CalBa_I=\CalBa(\{0,1\}^I)$ is
the Baire $\sigma$-algebra of $\{0,1\}^I$\cmmnt{,
that is, the $\sigma$-algebra of
subsets of $\{0,1\}^I$ generated by sets of the form
$\{x:x\in\{0,1\}^I$, $x(i)=1\}$ for $i\in I$ (4A3Na)}.   For each
$i\in I$, set $f_i(\omega)=f(\omega)(i)$ for $\omega\in\Omega$;
then\cmmnt{ $f_i:\Omega\to\{0,1\}$ is measurable, so} we have a
$\Bbb P$-name $\vec f_i$\cmmnt{ as in (a)}.   Let $\vec f$ be
the $\Bbb P$-name
$\{(\family{i}{\check I}{\vec f_i},\Bbbone)\}$\cmmnt{ (interpreting
the subformula $\family{i}{\check I}{\ldots}$ in the forcing
language, of course, by the convention of 5A3Eb)}.   \cmmnt{Then}

\Centerline{$\VVdP\,\vec f\in\{0,1\}^{\check I}$,}

\noindent and for every $i\in I$

\Centerline{$\VVdP\,\vec f(\varchecki)=\vec f_i$.}

\spheader 551Cc In the other direction, if a $\Bbb P$-name $\dot x$
and $p\in\frak A^+$ are such that
$p\VVdP\,\dot x\in\{0,1\}^{\check I}$, then for each $i\in I$ we have a
$\Bbb P$-name $\dot x(\varchecki)$ and a measurable
$f_i:\Omega\to\{0,1\}$ such that
$p\VVdP\,\dot x(\varchecki)=\vec f_i$;  setting
$f(\omega)=\familyiI{f_i(\omega)}$ for $\omega\in\Omega$, $f$ is
$(\Sigma,\CalBa_I)$-measurable and $p\VVdP\,\vec f=\dot x$.

\cmmnt{\spheader 551Cd I ought to remark that there is a problem with
equality for the $\Bbb P$-names $\vec f$.
If, in the context of (b)-(c) above, we have two
$(\Sigma,\CalBa_I)$-measurable functions $f$ and $g$, and if
$p\in\frak A^+$, then

$$\eqalign{p\VVdP\,\vec f=\vec g
&\iff\text{ for every }i\in I,\,p\VVdP\,\vec f_i=\vec g_i\cr
&\iff\text{ for every }i\in I,\,
p\Bsubseteq\{\omega:f_i(\omega)=g_i(\omega)\}^{\ssbullet}
  \text{ in }\frak A.\cr}$$

\noindent In particular, $\VVdP\,\vec f=\vec g$ iff $f_i=g_i$ 
$\Cal I$-a.e.\ for
every $i\in I$.   If $I$ is uncountable we can easily have
$\VVdP\,\vec f=\vec g$
while $f(\omega)\ne g(\omega)$ for every $\omega\in\Omega$.   But if $I$ is
countable then we shall have

\Centerline{$p\VVdP\,\vec f=\vec g
\iff p\Bsubseteq\{\omega:f(\omega)=g(\omega)\}^{\ssbullet}$.}

\noindent For a context in which these considerations are vital, see (a-ii) of
the proof of 551E.
}%end of comment

\spheader 551Ce Suppose that $x$ is any point of $\{0,1\}^I$.
Then we have a
corresponding $\Bbb P$-name $\check x$, and
$\VVdP\,\check x\in\{0,1\}^{\check I}$.   For each $i\in I$,
$\VVdP\,\check x(\varchecki)=x(i)\var2spcheck\in\{0,1\}$.
If we set $e_x(\omega)=x$ for
every $\omega\in\Omega$, then\cmmnt{ $e_x(\omega)(i)=x(i)$
for every $i\in I$ and
$\omega\in\Omega$, so $\VVdP\,\vec e_x(\varchecki)=x(i)\var2spcheck$
for every $i\in I$, and} $\VVdP\,\vec e_x=\check x$.

\vleader{60pt}{551D}{Definition} Let $(\Omega,\Sigma,\Cal I)$ be a
non-trivial measurable space
with negligibles with a Dedekind complete quotient algebra,
and $\Bbb P$ its associated forcing notion.   Let $I$ be any set.
If $W\subseteq\Omega\times\{0,1\}^I$, let $\vec W$ be the
$\Bbb P$-name

$$\eqalign{\{(\vec f,E^{\ssbullet}):E&\in\Sigma\setminus\Cal I,\,
f:\Omega\to\{0,1\}^I\text{ is }(\Sigma,\CalBa_I)\text{-measurable},\cr
&\mskip170mu(\omega,f(\omega))\in W\text{ for every }\omega\in E\},\cr}$$

\noindent interpreting $\vec f$ as in 551C.
\cmmnt{I give the definition for arbitrary sets $W$, but
it is useful primarily when $W\in\Sigma\tensorhat\CalBa_I$,
as in most of the
next proposition.   Perhaps I can note straight away that}

\Centerline{$\VVdP\,\vec W\subseteq\{0,1\}^{\check I}$}

\noindent and\cmmnt{ that} if $W=\Omega\times\{0,1\}^I$ then

\Centerline{$\VVdP\,\vec W=\{0,1\}^{\check I}$\ifresultsonly{.}\fi}

\cmmnt{\noindent (using 551Cb-551Cc).}

\leader{551E}{Proposition} Let $(\Omega,\Sigma,\Cal I)$ be a
non-trivial measurable space
with negligibles with a Dedekind complete quotient 
algebra\cmmnt{ $\frak A$},
$\Bbb P$ its associated forcing notion, and $I$ a set.

(a) If $W\in\Sigma\tensorhat\CalBa_I$ and $f:\Omega\to\{0,1\}^I$ is
$(\Sigma,\CalBa_I)$-measurable, then
$\{\omega:(\omega,f(\omega))\in W\}$ belongs to $\Sigma$, and
$\Bvalue{\vec f\in\vec W}=\{\omega:(\omega,f(\omega))\in W\}^{\ssbullet}$.

(b) If $V$, $W\in\Sigma\tensorhat\CalBa_I$ then

\doubleinset{$\VVdP\,\vec V\cap\vec W=(V\cap W)\sspvec$,
$\vec V\cup\vec W=(V\cup W)\sspvec$,
$\vec V\setminus\vec W=(V\setminus W)\sspvec$ and
$\vec V\symmdiff\vec W=(V\symmdiff W)\sspvec$.}

(c) If $V$, $W\subseteq\Omega\times\{0,1\}^I$ and $V\subseteq W$ then

\Centerline{$\VVdP\,\vec V\subseteq\vec W$.}

(d) If $\sequencen{W_n}$ is a sequence in $\Sigma\tensorhat\CalBa_I$
with union $W$ and intersection $V$, then

\Centerline{$\VVdP\,\bigcup_{n\in\Bbb N}\vec W_n=\vec W$
and $\bigcap_{n\in\Bbb N}\vec W_n=\vec V$.}

(e) Suppose that $J\subseteq I$ is countable, $z\in\{0,1\}^J$, $E\in\Sigma$
and

\Centerline{$W
=\{(\omega,x):\omega\in E$, $x\in\{0,1\}^I$, $x\restr J=z\}$.}

\noindent Then

\Centerline{$E^{\ssbullet}
=\Bvalue{\vec W=\{x:x\in\{0,1\}^{\check I}$, $\check z\subseteq x\}}$,}

\Centerline{$1\Bsetminus E^{\ssbullet}
=\Bvalue{\vec W=\emptyset}$.}

\proof{{\bf (a)(i)} Let $\Cal W$ be the family of subsets of
$\Omega\times\{0,1\}^I$ such that
$F_W=\{\omega:(\omega,f(\omega))\in W\}\in\Sigma$.
Then $\Cal W$ is a Dynkin class of subsets of $\Omega\times\{0,1\}^I$, just
because $\Sigma$ is a $\sigma$-algebra.
If $H\in\Sigma$, $J\subseteq I$ is finite, $z\in\{0,1\}^J$ and
$W=\{(\omega,x):\omega\in H$, $z\subseteq x\in\{0,1\}^I\}$ then
$F_W=H\cap\{\omega:f(\omega)(i)=z(i)$ for every $i\in J\}$ belongs to
$\Sigma$ because $f$ is $(\Sigma,\CalBa_I)$-measurable, so $W\in\Cal W$.
By the Monotone Class Theorem (136B),
$\Cal W$ includes the $\sigma$-algebra generated by sets of this
form, which is just $\Sigma\tensorhat\CalBa_I$.

\medskip

\quad{\bf (ii)} Now suppose that $W\in\Sigma\tensorhat\CalBa_I$.
If $F_W\in\Cal I$ then
surely $F_W^{\ssbullet}=0\Bsubseteq\Bvalue{\vec f\in\vec W}$.   If
$F_W\notin\Cal I$
then $(\vec f,F_W^{\ssbullet})\in\vec W$,
$F_W^{\ssbullet}\VVdP\,\vec f\in\vec W$
and again $F_W^{\ssbullet}\Bsubseteq\Bvalue{\vec f\in\vec W}$.

\Quer\ I wish to apply 5A3H.
If $F_W^{\ssbullet}\ne\Bvalue{\vec f\in\vec W}$, set
$p=\Bvalue{\vec f\in\vec W}\Bsetminus F_W^{\ssbullet}$.   Since
$p\VVdP\,\vec f\in\vec W$ there must be a $q\in\frak A^+$ and a
$\Bbb P$-name
$\dot x$ and an $r$ stronger than both $p$ and $q$ such that

\Centerline{$r\VVdP\,\dot x=\vec f$ and $(\dot x,q)\in\vec W$.}

\noindent Now there must be a
$(\Sigma,\CalBa_I)$-measurable function $g$ and an
$E\in\Sigma\setminus\Cal I$
such that $\dot x=\vec g$, $q=E^{\ssbullet}$
and $(\omega,g(\omega))\in W$ for
every $\omega\in E$.   In this case, $r=G^{\ssbullet}$ for some
$G\subseteq E\setminus F_W$, and $r\VVdP\,\vec f=\vec g$.

Because $W\in\Sigma\tensorhat\CalBa_I$,
there is a countable set $J\subseteq I$ such that
$W$ factors through $\Omega\times\{0,1\}^J$.   For each $i\in J$, we have
$r\VVdP\,\vec f(\varchecki)=\vec g(\varchecki)$, that is,

\Centerline{$r\Bsubseteq\Bvalue{\vec f(\varchecki)=\vec g(\varchecki)}
=\{\omega:f(\omega)(i)=g(\omega)(i)\}^{\ssbullet}$.}

\noindent So $f(\omega)(i)=g(\omega)(i)$ for
$\Cal I$-almost every $\omega\in G$.   This
is true for every $i\in J$, so $f(\omega)\restr J=g(\omega)\restr J$ for
$\Cal I$-almost
every $\omega\in G$.   But this means that, for
$\Cal I$-almost every $\omega\in G$,
$(\omega,f(\omega))\in W$ iff $(\omega,g(\omega))\in W$.   However,
$G\subseteq E\setminus F_W$, so $(\omega,g(\omega))\in W$ and
$(\omega,f(\omega))\notin W$ for every $\omega\in G$.\ \Bang

So we must have $F_W^{\ssbullet}=\Bvalue{\vec f\in\vec W}$, as claimed.

\medskip

{\bf (b)} These are now elementary.   The point is that if a $\Bbb P$-name
$\dot x$ and $p\in\frak A^+$ are such that $p\VVdP\,\dot x\in\vec V\cap\vec W$,
then $p\VVdP\,\dot x\in\{0,1\}^{\check I}$, so there is a
$(\Sigma,\CalBa_I)$-measurable $f:\Omega\to\{0,1\}^I$ such that
$p\VVdP\,\dot x=\vec f$, and $p\VVdP\,\vec f\in\vec V\cap\vec W$.
Now (a) shows that

$$\eqalign{\Bvalue{\vec f\in(V\cap W)\sspvec}
&=\{\omega:(\omega,f(\omega))\in V\cap W\}^{\ssbullet}\cr
&=\{\omega:(\omega,f(\omega))\in V\}^{\ssbullet}
  \Bcap\{\omega:(\omega,f(\omega))\in W\}^{\ssbullet}\cr
&=\Bvalue{\vec f\in\vec V}\Bcap\Bvalue{\vec f\in\vec W}
\Bsupseteq p\cr}$$

\noindent and

\Centerline{$p\VVdP\,\dot x=\vec f\in(V\cap W)\sspvec$.}

\noindent As $p$ and $\dot x$ are arbitrary,

\Centerline{$\VVdP\,\vec V\cap\vec W\subseteq(V\cap W)\sspvec$.}

\noindent The other seven inequalities are equally straightforward.

\medskip

{\bf (c)} This is immediate from the definition in 551D, since we actually
have $\vec V\subseteq\vec W$.

\medskip

{\bf (d)} We can repeat the method of (b).   If a $\Bbb P$-name
$\dot x$ and $p\in\frak A^+$ are such that
$p\VVdP\,\dot x\in\bigcap_{n\in\Bbb N}\vec W_n$,
then there is a
$(\Sigma,\CalBa_I)$-measurable $f:\Omega\to\{0,1\}^I$ such that
$p\VVdP\,\dot x=\vec f$, and $p\VVdP\,\vec f\in\vec W_n$ for every $n$.
Now

$$\eqalignno{\Bvalue{\vec f\in\vec V}
&=\{\omega:(\omega,f(\omega))\in\bigcap_{n\in\Bbb N}W_n\}^{\ssbullet}\cr
&=(\bigcap_{n\in\Bbb N}\{\omega:(\omega,f(\omega))\in W_n\})^{\ssbullet}
=\inf_{n\in\Bbb N}\{\omega:(\omega,f(\omega))\in W_n\}^{\ssbullet}\cr
\displaycause{because $\Sigma\cap\Cal I$ is a $\sigma$-ideal of $\Sigma$,
so $E\mapsto E^{\ssbullet}$ is sequentially order-continuous, by 313Qb}
&\Bsupseteq p\cr}$$

\noindent and

\Centerline{$p\VVdP\,\dot x=\vec f\in\vec V$.}

\noindent As $p$ and $\dot x$ are arbitrary,

\Centerline{$\VVdP\,\bigcap_{n\in\Bbb N}\vec W_n\subseteq\vec V$.}

\noindent On the other hand, (c) tells us that

\Centerline{$\VVdP\,\vec V\subseteq\bigcap_{n\in\Bbb N}\vec W_n$,
so we have equality.}

Putting this together with (b) (and recalling that
$\VVdP\,(\Omega\times\{0,1\}^I)\sspvec=\{0,1\}^{\check I}$), we get

\Centerline{$\VVdP\,\bigcup_{n\in\Bbb N}\vec W_n=\vec W$.}

\medskip

{\bf (e)(i)} Suppose that $p\in\frak A^+$ and that $\dot x$ is a
$\Bbb P$-name such that $p\VVdP\,\dot x\in\vec W$.
Let $f:\Omega\to\{0,1\}^I$ be
a $(\Sigma,\CalBa_I)$-measurable function such that
$p\VVdP\,\dot x=\vec f$ (551Cc).   Then

\Centerline{$p\Bsubseteq\Bvalue{\vec f\in\vec W}
=\{\omega:\omega\in E$, $z\subseteq f(\omega)\}^{\ssbullet}$}

\noindent by (a) above;  that is, $p\Bsubseteq E^{\ssbullet}$ and

\Centerline{$p\VVdP\,\vec f(\varchecki)=z(i)\var2spcheck
=\check z(\varchecki)$}

\noindent for every $i\in J$, so

\Centerline{$p\VVdP\,\check z\subseteq\vec f$.}

\noindent As $p$ and $\dot x$ are arbitrary,

\Centerline{$\Bvalue{\vec W\ne\emptyset}\subseteq E^{\ssbullet}$}

\noindent and

\Centerline{$\VVdP\,\vec W
\subseteq\{x:\check z\subseteq x\in\{0,1\}^{\check I}\}$.}

\wheader{551E}{6}{2}{2}{48pt}

\quad{\bf (ii)} If $E\in\Cal I$ then
$\VVdP\,\vec W=\emptyset$ and we can stop.
Otherwise, suppose that $p\in\frak A^+$ is stronger than
$E^{\ssbullet}$ and that $\dot x$ is a $\Bbb P$-name such that

\Centerline{$p\VVdP\,\check z\subseteq\dot x\in\{0,1\}^{\check I}$.}

\noindent Let $f$ be a $(\Sigma,\CalBa_I)$-measurable function such that
$p\VVdP\,\dot x=\vec f$.   Then $p\VVdP\,\vec f_i=z(i)\var2spcheck$
for each $i\in J$, where $f_i(\omega)=f(\omega)(i)$ for every $\omega$, so
$p\Bsubseteq\{\omega:z\subseteq f(\omega)\}^{\ssbullet}$.   But also
$p\Bsubseteq E^{\ssbullet}$, so

\Centerline{$p
\Bsubseteq\{\omega:\omega\in E$, $z\subseteq f(\omega)\}^{\ssbullet}
=\Bvalue{\vec f\in\vec W}$,}

\noindent and $p\VVdP\,\dot x\in\vec W$.   As $p$ and
$\dot x$ are arbitrary,

\Centerline{$\Bvalue{\{x:\check z\subseteq x\}\subseteq\vec W}
\Bsupseteq E^{\ssbullet}$}

\noindent and we have

\Centerline{$\Bvalue{\{x:\check z\subseteq x\}=\vec W}
=E^{\ssbullet}$,
\quad$\Bvalue{\emptyset=\vec W}
=1\Bsetminus E^{\ssbullet}$.}
}%end of proof of 551E

\leader{551F}{Proposition} Let $(\Omega,\Sigma,\Cal I)$ be a
non-trivial measurable space
with negligibles with a Dedekind complete quotient 
algebra\cmmnt{ $\frak A$},
$\Bbb P$ its associated forcing notion, and $I$ a set.

(a) If $W\in\Sigma\tensorhat\CalBa_I$ then

\Centerline{$\VVdP\,\vec W\in\CalBa_{\check I}$.}

(b) Suppose that $(\Omega,\Sigma,\Cal I)$ is $\omega_1$-saturated,
$p\in\frak A^+$, and that
$\dot W$ is a $\Bbb P$-name such that

\Centerline{$p\VVdP\,\dot W\in\CalBa_{\check I}$.}

\noindent Then there is a $W\in\Sigma\tensorhat\CalBa_I$ such that

\Centerline{$p\VVdP\,\dot W=\vec W$.}

\proof{{\bf (a)} Let $\Cal W$ be the family of those
$W\in\Sigma\tensorhat\CalBa_I$
such that $\VVdP\,\vec W\in\CalBa_{\check I}$.
551Eb and 551Ed tell us that $\Cal W$ is a $\sigma$-subalgebra of
$\Sigma\tensorhat\CalBa_I$, and 551Ee tells us that $E\times H\in\Cal W$
whenever $E\in\Sigma$ and $H$ is a basic cylinder set in $\{0,1\}^I$.
So $\Cal W$ must be the whole of $\Sigma\tensorhat\CalBa_I$.

\medskip

{\bf (b)(i)} Suppose that $p\in\frak A^+$ and that $\dot W$ is a
$\Bbb P$-name such that

\Centerline{$p\VVdP\,\dot W$ is a basic cylinder set in
$\{0,1\}^{\check I}$.}

\noindent Then there is a $W\in\Sigma\tensorhat\CalBa_I$ such that
$p\VVdP\,\dot W=\vec W$.   \Prf\ We know that

\Centerline{$p\VVdP$ there is a $z\in\Fn_{<\omega}(\check I;\{0,1\})$
such that $\dot W=\{x:z\subseteq x\in\{0,1\}^{\check I}\}$.}

\noindent So there is a $\Bbb P$-name $\dot z$ such that

\Centerline{$p\VVdP\,\dot z\in\Fn_{<\omega}(\check I;\{0,1\})$ and
$\dot W=\{x:\dot z\subseteq x\}$;}

\noindent adjusting $\dot z$ if necessary, we can suppose that

\Centerline{$\VVdP\,\dot z\in\Fn_{<\omega}(\check I;\{0,1\})$.}

\noindent But this means that there is a maximal antichain (that is, a
partition of unity) $C\subseteq\frak A^+$
and a family $\family{c}{C}{z_c}$ in
$\Fn_{<\omega}(I;\{0,1\})$ such that

\Centerline{$c\VVdP\,\dot z=\check z_c$}

\noindent for every $c\in C$.   Because $\Cal I$ is $\omega_1$-saturated,
$\frak A$ is ccc and $C$ is countable.   We can therefore find a partition
$\family{c}{C}{E_c}$ of $\Omega$ into members of $\Sigma$ such that
$E_c^{\ssbullet}=c$ for every $c\in C$.   Consider

\Centerline{$W_c=E_c\times\{x:z_c\subseteq x\in\{0,1\}^I\}$ for $c\in C$,
\quad$W=\bigcup_{c\in C}W_c$.}

\noindent Of course $W\in\Sigma\tensorhat\CalBa_I$.   By 551Ee,

\Centerline{$c\VVdP\,\vec W_c=\{x:\check z_c\subseteq x\}
=\{x:\dot z\subseteq x\}$,
\quad$c\VVdP\,\vec W_d=\emptyset$}

\noindent whenever $c$, $d\in C$ are distinct.   
Because $C$ is countable, 551Ed tells us that

\Centerline{$c\VVdP\,\vec W
=\bigcup_{d\in\check C}\vec W_d=\{x:\dot z\subseteq x\}$}

\noindent for every $c\in C$;  because $C$ is a maximal antichain,

\Centerline{$\VVdP\,\vec W=\{x:\dot z\subseteq x\}$}

\noindent and

\Centerline{$p\VVdP\,\vec W=\dot W$.   \Qed}

\medskip

\quad{\bf (ii)} Suppose that $p\in\frak A^+$ and that $\dot W$ is a
$\Bbb P$-name such that

\Centerline{$p\VVdP\,\dot W$ is a cozero set in $\{0,1\}^{\check I}$.}

\noindent Then there is a $W\in\Sigma\tensorhat\CalBa_I$ such that
$p\VVdP\,\dot W=\vec W$.   \Prf\ Set $p'=\Bvalue{\dot W=\emptyset}$.
If $p\Bsubseteq p'$ we can take $W=\emptyset$ and stop.   Otherwise,
let $E\in\Sigma$ be such that $E^{\ssbullet}=1\Bsetminus p'$.   We have

\Centerline{$p\Bsetminus p'\VVdP\,
\dot W$ is the union of a sequence of basic cylinder sets,}

\noindent so there is a sequence $\sequencen{\dot W_n}$ of $\Bbb P$-names
such that

\Centerline{$p\Bsetminus p'\VVdP\,\dot W_n$
is a basic cylinder set for every $n$ and
$\dot W=\bigcup_{n\in\Bbb N}\dot W_n$.}

\noindent By (i), we have for each $n\in\Bbb N$ a
$W_n\in\Sigma\tensorhat\CalBa_I$ such that
$p\Bsetminus p'\VVdP\,\dot W_n=\vec W_n$;
now $V=\bigcup_{n\in\Bbb N}W_n$ belongs to
$\Sigma\tensorhat\CalBa_I$ and
$\VVdP\,\vec V=\bigcup_{n\in\Bbb N}\vec W_n$, so
$p\Bsetminus p'\VVdP\,\vec V=\dot W$.   Finally, setting
$W=(E\times\{0,1\}^I)\cap V$, $p\VVdP\,\vec W=\dot W$.\ \Qed

\medskip

\quad{\bf (iii)} Suppose that $p\in\frak A^+$, $\alpha<\omega_1$ and that
$\dot W$ is a $\Bbb P$-name such that

\Centerline{$p\VVdP\,\dot W\in\CalBa_{\check\alpha}(\{0,1\}^{\check I})$,}

\noindent defining $\CalBa_{\alpha}$ as in 5A4Ga.
Then there is a $W\in\Sigma\tensorhat\CalBa_I$ such that
$p\VVdP\,\dot W=\vec W$.   \Prf\ Induce on $\alpha$.   The case $\alpha=0$
is (ii) above.   For the inductive step to $\alpha>0$, we have

\Centerline{$p\VVdP\,\dot W\in\CalBa_{\check\alpha}(\{0,1\}^{\check I})$,}

\noindent so

\doubleinset{$p\VVdP$ there is a sequence $\sequencen{W_n}$ in
$\bigcup_{\beta<\check\alpha}\CalBa_{\beta}(\{0,1\}^{\check I})$
such that $\dot W=\bigcup_{n\in\Bbb N}\{0,1\}^{\check I}\setminus W_n$;}

\noindent let $\sequencen{\dot W_n}$ be a sequence of $\Bbb P$-names
such that

\Centerline{$p\VVdP\,
\dot W_n\in\bigcup_{\beta<\check\alpha}\CalBa_{\beta}(\{0,1\}^{\check I})$
for every $n\in\Bbb N$ and
$\dot W=\bigcup_{n\in\Bbb N}\{0,1\}^{\check I}\setminus\dot W_n$.}

\noindent For $n\in\Bbb N$, $\beta<\alpha$ set

\Centerline{$b_{n\beta}
=\Bvalue{\dot W_n\in\CalBa_{\check\beta}(\{0,1\}^{\check I})
\setminus\bigcup_{\gamma<\check\beta}\CalBa_{\gamma}(\{0,1\}^{\check I})}$,}

\noindent and choose $E_{n\beta}\in\Sigma$ such that
$E_{n\beta}^{\ssbullet}=b_{n\beta}$.
Writing $A_n=\{\beta:\beta<\alpha$, $b_{n\beta}\ne 0\}$,
$p\Bsubseteq\sup_{\beta\in A_n}b_{n\beta}$.  If $\beta\in A_n$, then

\Centerline{$b_{n\beta}
 \VVdP\,\dot W_n\in\CalBa_{\check\beta}(\{0,1\}^{\check I})$,}

\noindent so by the inductive hypothesis there is a
$W_{n\beta}\in\Sigma\tensorhat\CalBa_I$ such that
$b_{n\beta}\VVdP\,\dot W_n=\vec W_{n\beta}$.   For
$\beta\in\alpha\setminus A_n$ set $W_{n\beta}=\emptyset$.

Set $W_n=\bigcup_{\beta<\alpha}(E_{n\beta}\times\{0,1\}^I)\cap W_{n\beta}$.
Then

\Centerline{$\VVdP\,\vec W_n
=\bigcup_{\beta<\check\alpha}(E_{n\beta}\times\{0,1\}^I)\sspvec
  \cap\vec W_{n\beta}$,}

\noindent so if $\beta\in A_n$

\Centerline{$b_{n\beta}\VVdP\,\vec W_n=\vec W_{n\beta}=\dot W_n$}

\noindent because

\Centerline{$b_{n\beta}\VVdP\,(E_{n\gamma}\times\{0,1\}^I)\sspvec
 =\emptyset$}

\noindent if $\gamma<\alpha$ and $\gamma\ne\beta$, and

\Centerline{$b_{n\beta}\VVdP\,
(E_{n\beta}\times\{0,1\}^I)\sspvec=\{0,1\}^{\check I}$.}

\noindent As $p\Bsubseteq\sup_{\beta\in A_n}b_{n\beta}$,

\Centerline{$p\VVdP\,\vec W_n=\dot W_n$.}

This is true for every $n\in\Bbb N$.   So if we set
$W=\bigcup_{n\in\Bbb N}(\Omega\times\{0,1\}^I)\setminus W_n$,
we shall have $W\in\Sigma\tensorhat\CalBa_I$ and

\Centerline{$p\VVdP\,\vec W
=\bigcup_{n\in\Bbb N}\{0,1\}^{\check I}\setminus\vec W_n
=\bigcup_{n\in\Bbb N}\{0,1\}^{\check I}\setminus\dot W_n
=\dot W$.   \Qed}

\medskip

\quad{\bf (iv)} Finally, suppose that $p\in\frak A^+$ and that $\dot W$
is a $\Bbb P$-name such that

\Centerline{$p\VVdP\,\dot W\in\CalBa_{\check I}$.}

\noindent Then there is a $W\in\Sigma\tensorhat\CalBa_I$ such that
$p\VVdP\,\dot W=\vec W$.   \Prf\ Because $\Bbb P$ is ccc,

\Centerline{$\VVdP\,\check\omega_1$ is the first uncountable ordinal}

\noindent (5A3Nb), so

\Centerline{$\VVdP\,\CalBa(\{0,1\}^{\check I})
=\bigcup_{\alpha<\check\omega_1}\CalBa_{\alpha}(\{0,1\}^{\check I})$.}

\noindent For $\alpha<\omega_1$ set

\Centerline{$b_{\alpha}
=\Bvalue{\dot W\in\CalBa_{\check\alpha}(\{0,1\}^{\check I})}$.}

\noindent Then
$p\Bsubseteq\sup_{\alpha<\omega_1}b_{\alpha}$.   Again because $\frak A$ is
ccc, there is a $\gamma<\omega_1$ such that
$p\Bsubseteq\sup_{\alpha<\gamma}b_{\alpha}$.   If $\alpha<\gamma$ and
$c_{\alpha}=b_{\alpha}\Bsetminus\sup_{\beta<\alpha}b_{\beta}$ is non-zero,
choose $W_{\alpha}\in\Sigma\tensorhat\CalBa_I$ such that
$c_{\alpha}\VVdP\,\vec W_{\alpha}=\dot W$;  for other $\alpha<\gamma$
set $W_{\alpha}=\emptyset$.   Choose $F_{\alpha}\in\Sigma$ such that
$F_{\alpha}^{\ssbullet}=c_{\alpha}$ for each $\alpha$.
Set $W=\bigcup_{\alpha<\gamma}(F_{\alpha}\times\{0,1\}^I)\cap W_{\alpha}
\in\Sigma\tensorhat\CalBa_I$.   As in (iii) just above,

\Centerline{$c_{\alpha}\VVdP\,\vec W=\vec W_{\alpha}=\dot W$}

\noindent whenever $c_{\alpha}\ne 0$, so

\Centerline{$p\VVdP\,\vec W=\dot W$.  \Qed}
}%end of proof of 551F

\vleader{48pt}{551G}{}\cmmnt{ I noted above that there are difficulties in
computing
$\Bvalue{\vec f=\vec g\,}$ for functions $f$, $g:\Sigma\to\{0,1\}^I$.
For $W$, $V\in\Sigma\tensorhat\CalBa_I$ the corresponding question about
$\Bvalue{\vec W=\vec V}$ turns out to be simpler, at least in some
important cases.

\medskip

\wheader{551G}{0}{0}{0}{72pt}
\noindent}{\bf Proposition} Let $(\Omega,\Sigma,\Cal I)$ be a non-trivial
measurable space with negligibles with a Dedekind complete quotient
algebra $\frak A$,
$\Bbb P$ the associated forcing notion and $I$ a set.   Suppose
that $\Sigma$ is closed under Souslin's operation.

(a) If $W\in\Sigma\tensorhat\CalBa_I$ then
$F=\{\omega:W[\{\omega\}]\ne\emptyset\}$ belongs to $\Sigma$ and
$\Bvalue{\vec W\ne\emptyset}=F^{\ssbullet}$ in $\frak A\cong\RO(\Bbb P)$.

(b) If $W$, $V\in\Sigma\tensorhat\CalBa_I$ then
$\Bvalue{\vec W=\vec V}=\{\omega:W[\{\omega\}]=V[\{\omega\}]\}^{\ssbullet}$.

\proof{{\bf (a)(i)} The point is that there is a $\Sigma$-measurable
function
$f:\Omega\to\{0,1\}^I$ such that $(\omega,f(\omega))\in W$ for every
$\omega\in F$.

\medskip

\Prf\grheada\ Suppose first that $I$ is countable.   Let
$\Cal V$ be the family of subsets of $\Omega\times\{0,1\}^I$ obtainable by
Souslin's operation $\Cal S$ from
$\{E\times H:E\in\Sigma$, $H\subseteq\{0,1\}^I$ is closed$\}$.   The family
$\Cal W=\{V:V\in\Cal V$, $(\Omega\times\{0,1\}^I)\setminus V\in\Cal V\}$ is a
$\sigma$-algebra and contains $E\times H$ whenever $E\in\Sigma$ and
$H\subseteq\{0,1\}^I$ is open-and-closed, so
$\Cal W\supseteq\Sigma\tensorhat\CalBa_I$ and $W\in\Cal W\subseteq\Cal V$.
By 423M, there is a selector $g$ for $W$ which is measurable for the
$\sigma$-algebra $\Tau$ of subsets of $\Omega$ generated by
$\Cal S(\Sigma)$;
but we are supposing that this is just $\Sigma$.   Also
$F=\dom g$ belongs to $\Tau=\Sigma$.   If $f$ is any extension of $g$ to a
$\Sigma$-measurable function from $\Omega$ to $\{0,1\}^I$, then $f$ has the
required property.

\medskip

\qquad\grheadb\ For the general case, note that $W\in\Sigma\tensorhat\CalBa_I$
factors through $\Omega\times\{0,1\}^J$ for some countable $J\subseteq I$, that
is, there is a $W_1\in\Sigma\tensorhat\CalBa_J$ such that

\Centerline{$W
=\{(\omega,x):\omega\in\Omega$, $x\in\{0,1\}^I$, $(\omega,x\restr J)\in W_1\}$.}

\noindent Now ($\alpha$) tells us that
$F_1=\{\omega:W_1[\{\omega\}]\ne\emptyset\}$ belongs to $\Sigma$ and that
there is a $\Sigma$-measurable $f_1:\Omega\to\{0,1\}^J$ such that
$(\omega,f_1(\omega))\in W_1$ for every $\omega\in F_1$.   Of course $F_1=F$,
and if we set

$$\eqalign{f(\omega)(i)
&=f_1(\omega)(i)\text{ for }\omega\in\Omega,\,i\in J,\cr
&=0\text{ for }\omega\in\Omega,\,i\in I\setminus J,\cr}$$

\noindent then $f$ is $(\Sigma,\CalBa_I)$-measurable and
$(\omega,f(\omega))\in W$ for every $\omega\in F$.\ \Qed

\medskip

\quad{\bf (ii)} Using 551Ea, it follows that

\Centerline{$\Bvalue{\vec W\ne\emptyset}\Bsupseteq\Bvalue{\vec f\in\vec W}
=\{\omega:(\omega,f(\omega))\in W\}^{\ssbullet}=F^{\ssbullet}$.}

\noindent On the other hand, if $a=\Bvalue{\vec W\ne\emptyset}$ is non-zero,
then there is a $\Bbb P$-name $\dot x$ such that $a\VVdP\,\dot x\in\vec W$.   By
551Cc, there is a $(\Sigma,\CalBa_I)$-measurable $g$ such that
$a\VVdP\,\dot x=\vec g$, in which case

\Centerline{$a\Bsubseteq\Bvalue{\vec g\in\vec W}
=\{\omega:(\omega,g(\omega))\in W\}^{\ssbullet}\Bsubseteq F^{\ssbullet}$.}

\noindent So $\Bvalue{\vec W\ne\emptyset}=F^{\ssbullet}$ exactly.

\medskip

{\bf (b)} Apply (a) to $W\symmdiff V$ (using 551Eb, as usual).
}%\end of proof of 551G

\leader{551H}{Examples} \cmmnt{Cases in which a $\sigma$-algebra is
closed under Souslin's operation, so that the conditions of 551G can be
satisfied, include the following.

\medskip

}{\bf (a)} If $(X,\Sigma,\mu)$ is a complete locally
determined measure space, then $\Sigma$ is closed under Souslin's
operation\prooflet{ (431A)}.

\spheader 551Hb If $(\Omega,\Sigma,\Cal I)$ is a complete
$\omega_1$-saturated
measurable space with negligibles, then $\Sigma$ is closed under Souslin's
operation\prooflet{ (431G)}.

\spheader 551Hc If $X$ is any topological space, then its Baire-property
algebra $\widehat{\Cal B}(X)$ is closed under Souslin's
operation\cmmnt{ (431Fb)}.

\leader{551I}{Theorem} Let $(\Omega,\Sigma,\Cal I)$ be a
non-trivial measurable space
with negligibles with a Dedekind complete quotient algebra,
$\Bbb P$ its associated forcing notion, and $I$ a set.
Let $W$ be any member of $\Sigma\tensorhat\CalBa_I$.  Then

(i) $h(\omega)=\nu_IW[\{\omega\}]$ is defined for every
$\omega\in\Omega$, where $\nu_I$ is the usual measure of $\{0,1\}^I$;

(ii) $h:\Omega\to[0,1]$ is $\Sigma$-measurable;

(iii) $\VVdP\,\nu_{\check I}\vec W=\vec h$,

\noindent where\cmmnt{ in this formula} $\vec h$ is the
$\Bbb P$-name for a real number defined from $h$ as in 551B, 
and $\nu_{\check I}$ is an
abbreviation for `the usual measure on $\{0,1\}^{\check I}$'.

\proof{ I follow the method of 551Ea and 551Fa.

\medskip

{\bf (a)} Suppose that $W$ is of the form
$E\times\{x:z\subseteq x\in\{0,1\}^I\}$, where $z\in\{0,1\}^J$ for some finite
$J\subseteq I$.   Then (using 551Ee)

\Centerline{$E^{\ssbullet}
=\Bvalue{\vec W=\{x:\check z\subseteq x\in\{0,1\}^{\check I}\}}
\subseteq\Bvalue{\nu_{\check I}\vec W=2^{-\#(\check J)}}$,}

\Centerline{$1\setminus E^{\ssbullet}
=\Bvalue{\vec W=\emptyset}
\subseteq\Bvalue{\nu_{\check I}\vec W=0}$;}

\noindent while also $h=2^{-\#(J)}\chi E$ so

\Centerline{$E^{\ssbullet}=\Bvalue{\vec h=2^{-\#(\check J)}}$,
\quad$1\Bsetminus E^{\ssbullet}=\Bvalue{\vec h=0}$.}

\noindent So in this case

\Centerline{$\VVdP\,\nu_{\check I}\vec W=\vec h$.}

\medskip

{\bf (b)} Now 551E shows that the set of those $W\in\CalBa_I$ for which
(i)-(iii) are true is a Dynkin class, so by the Monotone Class Theorem once more
we have the result.
}%\end of proof of 551I

\vleader{72pt}{551J}{Corollary} Let $(\Omega,\Sigma,\Cal I)$ be a
non-trivial $\omega_1$-saturated measurable space with negligibles,
$\Bbb P$ its associated forcing notion, 
$P\cmmnt{\mskip5mu=(\Sigma/\Sigma\cap\Cal I)^+}$ 
the partially ordered set underlying $\Bbb P$, and $I$ a set.
If $p\in P$ and $\dot W$ is a $\Bbb P$-name such that

\Centerline{$p\VVdP\,\dot W\subseteq\{0,1\}^{\check I}$ is 
$\nu_{\check I}$-negligible,}

\noindent then there is a $W\in\Sigma\tensorhat\CalBa_I$
such that $\nu_IW[\{\omega\}]=0$ for every $\omega\in\Omega$ and

\Centerline{$p\VVdP\,\dot W\subseteq\vec W$.}

\proof{ Because

\Centerline{$\VVdP$ the usual measure on $\{0,1\}^{\check I}$
is a completion regular Radon measure,}

\noindent we know that

\Centerline{$p\VVdP$ there is a $\nu_{\check I}$-negligible
member of $\CalBa_{\check I}$ including $\dot W$.}

\noindent Let $\dot V$ be a $\Bbb P$-name such that

\Centerline{$p\VVdP\,\dot W\subseteq\dot V\in\CalBa_{\check I}$ and
$\nu_{\check I}\dot V=0$.}

\noindent By 551Fb, there is a $V\in\Sigma\tensorhat\CalBa_I$ such that
$p\VVdP\,\dot V=\vec V$.   Set $h(\omega)=\nu_IV[\{\omega\}]$ for
$\omega\in\Omega$;  then

\Centerline{$p\VVdP\,\vec h=\nu_{\check I}\vec V=0$}

\noindent (551I), so $p\Bsubseteq E^{\ssbullet}$, where $E=h^{-1}[\{0\}]$.
Set $W=(E\times\{0,1\}^I)\cap V$.   Then
$W\in\Sigma\tensorhat\CalBa_I$, $\nu_IW[\{\omega\}]=0$
for every $\omega$, and (using 551Gb)

\Centerline{$p\VVdP\,\vec W=\vec V=\dot V\supseteq\dot W$,}

\noindent as required.
}%end of proof of 551J

\leader{551K}{}\cmmnt{ We have been looking here at general sets
$W\in\Sigma\tensorhat\CalBa_I$.
A special case of obvious importance is
when $W$ is of the form $\Omega\times H$ where $H\in\CalBa_I$.   For
these it is worth refining the results slightly.

\medskip

\noindent}{\bf Proposition} Let $(\Omega,\Sigma,\Cal I)$ be a
non-trivial measurable space with negligibles with a Dedekind complete
quotient algebra, $\Bbb P$ the associated forcing notion, and $I$ a set.
For $H\subseteq\{0,1\}^I$ set
$\tilde H=(\Omega\times H)\sspvec$\cmmnt{ as defined in 551D}.

(a) If $H=\{x:z\subseteq x\in\{0,1\}^I\}$, where
$z\in\Fn_{<\omega}(I;\{0,1\})$, then

\Centerline{$\VVdP\,\tilde H=\{x:\check z\subseteq x\in\{0,1\}^{\check I}\}$.}

(b)(i) If $G$, $H\in\CalBa_I$ then

$$\eqalign{\VVdP\,\tilde G\cup\tilde H&=(G\cup H)\ssptilde,\,
\tilde G\cap\tilde H=(G\cap H)\ssptilde,\cr
&\tilde G\setminus\tilde H=(G\setminus H)\ssptilde,\,
\tilde G\symmdiff\tilde H=(G\symmdiff H)\ssptilde.\cr}$$

\quad(ii) If $\sequencen{H_n}$ is any sequence in $\CalBa_I$ then

\Centerline{$\VVdP\,
\bigcup_{n\in\Bbb N}\tilde H_n=(\bigcup_{n\in\Bbb N}H_n)\ssptilde$,
$\bigcap_{n\in\Bbb N}\tilde H_n=(\bigcap_{n\in\Bbb N}H_n)\ssptilde$.}

(c) If $\alpha<\omega_1$ and
$H\in\CalBa_{\alpha}(\{0,1\}^I)$,\cmmnt{ once again defining
$\CalBa_{\alpha}$ as in 5A4Ga,} then

\Centerline{$\VVdP\,\tilde H\in\CalBa_{\check\alpha}(\{0,1\}^{\check I})$.}

(d) If $H$ is measured by the usual measure $\nu_I$ of $\{0,1\}^I$, then

\Centerline{$\VVdP\,\nu_{\check I}\tilde H=(\nu_IH)\var2spcheck$.}

\proof{{\bf (a)} This is covered by 551Ee.

\medskip

{\bf (b)} This is a special case of parts (b) and (d) of 551E.

\medskip

{\bf (c)} A subset of $\{0,1\}^I$ is a cozero set iff it is
empty or expressible as
the union of a sequence of basic cylinder sets, so if $H$ is a cozero set then
(a) and (b-ii) tell us that

\Centerline{$\VVdP\,\tilde H$ is a cozero set in $\{0,1\}^{\check I}$.}

\noindent Now an induction on $\alpha$ shows that if
$H\in\CalBa_{\alpha}(\{0,1\}^I)$ then

\Centerline{$\VVdP\,\tilde H\in\CalBa_{\check\alpha}(\{0,1\}^{\check I})$.}

\medskip

{\bf (d)} We have $H_0$, $H_1\in\CalBa_I$ such that
$H_0\subseteq H\subseteq H_1$ and $\nu_IH_0=\nu_IH=\nu_IH_1$.
Applying 551I(iii) to $\Omega\times H_0$ and $\Omega\times H_1$,

\Centerline{$\VVdP\,\nu_{\check I}\tilde H_0=\nu_{\check I}\tilde H_1
=(\nu_IH)\var2spcheck$,}

\noindent while of course
$\VVdP\,\tilde H_0\subseteq\tilde H\subseteq\tilde H_1$ (551Ec), so

\Centerline{$\VVdP\,\nu_{\check I}\tilde H=(\nu_IH)\var2spcheck$.}
}%end of proof of 551K

\cmmnt{
\leader{551L}{Remark} If I ask you to think of your favourite Baire set in
$\{0,1\}^I$, it is likely to come with a definition;  for instance, the set $H$
of those $x\in\{0,1\}^{\Bbb N}$ such that
$\lim_{n\to\infty}\Bover1{n+1}\sum_{i=1}^nx(i)=\Bover12$.   
The point of 551K is that we shall automatically get

\Centerline{$\VVdP\,\tilde H
=\{x:x\in\{0,1\}^{\Bbb N}$,
$\lim_{n\to\infty}\Bover1{n+1}\sum_{i=1}^nx(i)=\Bover12\}$.}

\noindent\Prf\

\Centerline{$H
=\bigcap_{n\in\Bbb N}\bigcup_{m\in\Bbb N}\bigcap_{k\ge m}
  \bigcup_{z\in L_{nk}}\{x:z\subseteq x\in\{0,1\}^{\Bbb N}\}$,}

\noindent where

\Centerline{$L_{nk}=\{z:z\in\{0,1\}^{k+1}$,
$|\Bover1{k+1}\sum_{i=0}^kz_i-\Bover12|\le\Bover1{n+1}\}$.}

\noindent So 551K tells us that

\Centerline{$\VVdP\,\tilde H
=\bigcap_{n\in\Bbb N}\bigcup_{m\in\Bbb N}\bigcap_{k\ge m}
\bigcup_{z\in\check L_{nk}}
\{x:z\subseteq x\in\{0,1\}^{\Bbb N}\}$,}

\noindent and of course

\Centerline{$\VVdP\,\check L_{nk}=\{z:z\in\{0,1\}^{\check k+1}$,
$|\Bover1{\check k+1}\sum_{i=0}^{\check k}z_i-\Bover12|
  \le\Bover1{\check n+1}\}$.\ \Qed}

What I am trying to say here is that the process
$H\mapsto(\Omega\times H)\sspvec=\tilde H$ builds a $\Bbb P$-name for the
`right' subset of $\{0,1\}^I$, in the sense that any adequately concrete
definition of $H$ will also, when interpreted in $V^{\Bbb P}$,
be a definition of $\tilde H$.
}%end of comment

\leader{551M}{}\cmmnt{ We can go still farther.

\medskip

\noindent}{\bf Definition} Let $(\Omega,\Sigma,\Cal I)$ be a non-trivial
measurable space with negligibles, and $\Bbb P$ its associated forcing
notion.   Let $I$ be any set.   If $\psi:\Omega\times\{0,1\}^I\to\Bbb R$ is
$(\Sigma\tensorhat\CalBa_I)$-measurable, let $\vec\psi$ be the
$\Bbb P$-name

$$\eqalign{\{((\vec f,\vec h),\Bbbone):
f&\text{ is a }(\Sigma,\CalBa_I)\text{-measurable function from }
  \Omega\text{ to }\{0,1\}^I,\cr
&h:\Omega\to\Bbb R\text{ is }\Sigma\text{-measurable},\,
h(\omega)=\psi(\omega,f(\omega))\text{ for every }\omega\in\Omega\},\cr}$$

\noindent where in this formula $\vec f$ is to be interpreted as a
$\Bbb P$-name for a member of $\{0,1\}^{\check I}$, as in 551C, and
$\vec h$ as a $\Bbb P$-name for a real number, as in 551B.

\leader{551N}{Proposition} Let $(\Omega,\Sigma,\Cal I)$ be a non-trivial
measurable space with negligibles with a Dedekind complete quotient
algebra $\frak A$, $\Bbb P$ its associated forcing
notion, and $I$ a set.   Suppose that
$\psi:\Omega\times\{0,1\}^I\to\Bbb R$ is
$(\Sigma\tensorhat\CalBa_I)$-measurable, and define $\vec\psi$ as in 551M.

(a) $\VVdP\,\vec\psi$ is a real-valued
function on $\{0,1\}^{\check I}$.

(b) If $\phi:\Omega\times\{0,1\}^I\to\Bbb R$ is another
$(\Sigma\tensorhat\CalBa_I)$-measurable function, and $\alpha\in\Bbb R$,
then

\Centerline{$\VVdP\,(\phi+\psi)\sspvec=\vec\phi+\vec\psi$,
$(\alpha\phi)\sspvec=\check\alpha\vec\phi$.}

(c) If $\sequencen{\psi_n}$ is a sequence of
$(\Sigma\tensorhat\CalBa_I)$-measurable real-valued functions on
$\Omega\times\{0,1\}^I$ and
\ifnum\stylenumber=12 we set \fi
$\psi(\omega,x)=\penalty-100\lim_{n\to\infty}\psi_n(\omega,x)$ for every
$\omega\in\Omega$ and $x\in\{0,1\}^I$, then

\Centerline{$\VVdP\,\vec\psi(x)=\lim_{n\to\infty}\vec\psi_n(x)$ for every
$x\in\{0,1\}^{\check I}$.}

\wheader{551N}{0}{0}{0}{30pt}
(d) If $W\in\Sigma\tensorhat\CalBa_I$, then

\Centerline{$\VVdP\,(\chi W)\sspvec=\chi\vec W$.}

(e) $\VVdP\,\vec\psi$ is $\CalBa_{\check I}$-measurable.

(f) If $h(\omega)=\int\psi(\omega,x)\nu_I(dx)$ is defined for every
$\omega\in\Omega$, then

\Centerline{$\VVdP\,\int\vec\psi\,d\nu_{\check I}$ is defined and equal to
$\vec h$.}

\proof{{\bf (a)(i)} Suppose that we have two members
$((\vec f_0,\vec h_0),\Bbbone)$
and $((\vec f_1,\vec h_1),\Bbbone)$ of $\vec\psi$, and that
$E\in\Sigma\setminus\Cal I$ is such that
$E^{\ssbullet}\VVdP\,\vec f_0=\vec f_1$.   Then
$E^{\ssbullet}\VVdP\,\vec h_0=\vec h_1$.   \Prf\ Let $J\subseteq I$ be a
countable set such that $\psi$ factors through $\Omega\times\{0,1\}^J$, in
the sense that $\psi(\omega,x)=\psi(\omega,y)$ whenever $\omega\in\Omega$
and $x$, $y\in\{0,1\}^I$ are such that $x\restr J=y\restr J$.   For each
$i\in J$,

\Centerline{$E^{\ssbullet}\VVdP\,
\vec f_0(\varchecki)=\vec f_1(\varchecki)$,}

\noindent so that $f_0(\omega)(i)=f_1(\omega)(i)$ for
$\Cal I$-almost every
$\omega\in E$.   Consequently $f_0(\omega)\restr J=f_1(\omega)\restr J$ and

\Centerline{$h_0(\omega)=\psi(\omega,f_0(\omega))=\psi(\omega,f_1(\omega))
=h_1(\omega)$}

\noindent for $\Cal I$-almost every $\omega\in E$;  that is,
$E^{\ssbullet}\VVdP\,\vec h_0=\vec h_1$.\ \Qed

It follows that

\Centerline{$\VVdP\,\vec\psi$ is a function}

\noindent (5A3H).

\medskip

\quad{\bf (ii)} By the constructions in 551Cb and 551B,

\Centerline{$\VVdP\,\vec\psi\subseteq\{0,1\}^{\check I}\times\Bbb R$.}

\medskip

\quad{\bf (iii)} If $\dot x$ is a $\Bbb P$-name and $p\in\frak A^+$ is such
that $p\VVdP\,\dot x\in\{0,1\}^{\check I}$, then there
is a $(\Sigma,\CalBa_I)$-measurable $f:\Omega\to\{0,1\}^I$ such that
$p\VVdP\,\dot x=\vec f$ (551Cc again).   
Setting $h(\omega)=\psi(\omega,f(\omega))$ for $\omega\in\Omega$,
$((\vec f,\vec h),\Bbbone)\in\vec\psi$, so

\Centerline{$p\VVdP\,\dot x=\vec f$ and $(\vec f,\vec h)\in\vec\psi$,
so $\dot x\in\dom(\vec\psi)$.}

\noindent As $p$ and $\dot x$ are arbitrary,

\Centerline{$\VVdP\,\dom(\vec\psi)=\{0,1\}^{\check I}$.}

\medskip

{\bf (b)} This is easy.   If $p\in\frak A^+$ and $\dot x$ is a
$\Bbb P$-name such that $p\VVdP\,\dot x\in\{0,1\}^{\check I}$, take a
$(\Sigma,\CalBa_I)$-measurable $f:\Omega\to\{0,1\}^I$ such that
$p\VVdP\,\dot x=\vec f$;  set

\Centerline{$h_0(\omega)=\phi(\omega,f(\omega))$,
\quad$h_1(\omega)=\psi(\omega,f(\omega))$}

\noindent for $\omega\in\Omega$;  then

$$\eqalign{p\VVdP\,
(\phi+\psi)\sspvec(\dot x)
&=(\phi+\psi)\sspvec(\vec f\,)
=(h_0+h_1)\sspvec\cr
&=\vec h_0+\vec h_1
=\vec\phi(\vec f\,)+\vec\psi(\vec f\,)
=\vec\phi(\dot x)+\vec\psi(\dot x),\cr
(\alpha\phi)\sspvec(\dot x)
&=(\alpha h_0)\sspvec
=\check\alpha\vec h_0
=\check\alpha\vec\phi(\dot x)\cr}$$

\noindent by 5A3Lc.   As $p$ and $\dot x$ are arbitrary,

\Centerline{$\VVdP\,(\phi+\psi)\sspvec=\vec\phi+\vec\psi$,
$(\alpha\phi)\sspvec=\check\alpha\vec\phi$.}

\medskip

{\bf (c)} In the same way,
if $p\in\frak A^+$ and $\dot x$ is a
$\Bbb P$-name such that $p\VVdP\,\dot x\in\{0,1\}^{\check I}$, take a
$(\Sigma,\CalBa_I)$-measurable $f:\Omega\to\{0,1\}^I$ such that
$p\VVdP\,\dot x=\vec f$.  Set

\Centerline{$h_n(\omega)=\psi_n(\omega,f(\omega))$,
\quad$h(\omega)=\psi(\omega,f(\omega))$}

\noindent for $\omega\in\Omega$ and $n\in\Bbb N$;  then
$h=\lim_{n\to\infty}h_n$, so

\Centerline{$p\VVdP\,\vec\psi(\dot x)=\vec h
=\lim_{n\to\infty}\vec h_n=\lim_{n\to\infty}\vec\psi_n(\dot x)$.}

\medskip

{\bf (d)} Take $p\in\frak A^+$  and a $\Bbb P$-name $\dot x$ such that
$p\VVdP\,\dot x\in\{0,1\}^{\check I}$.   Let $f:\Omega\to\{0,1\}^I$ be a
$(\Sigma,\CalBa_I)$-measurable function such that
$p\VVdP\,\dot x=\vec f$;  set $h(\omega)=\chi W(\omega,f(\omega))$
for $\omega\in\Omega$, so that

\Centerline{$\VVdP\,\vec h=(\chi W)\sspvec(\vec f\,)$,
\quad$p\VVdP\,\vec h=(\chi W)\sspvec(\dot x)$.}

\noindent If $p=E^{\ssbullet}$ where $E\in\Sigma\setminus\Cal I$,

$$\eqalignno{p\VVdP\,&(\chi W)\sspvec(\dot x)=1\cr
&\iff p\VVdP\,\vec h=1\cr
&\iff h(\omega)=1\text{ for }\Cal I\text{-almost every }\omega\in E\cr
&\iff (\omega,f(\omega))\in W\text{ for }\Cal I\text{-almost every }
  \omega\in E\cr
&\iff p\Bsubseteq\Bvalue{\vec f\in\vec W}\cr
\displaycause{551Ea}
&\iff p\VVdP\,\dot x\in\vec W;\cr}$$

\noindent similarly,

$$\eqalignno{p\VVdP\,&(\chi W)\sspvec(\dot x)=0\cr
&\iff h(\omega)=0\text{ for }\Cal I\text{-almost every }\omega\in E\cr
&\iff (\omega,f(\omega))\notin W\text{ for }\Cal I\text{-almost every }
   \omega\in E\cr
&\iff p\VVdP\,\dot x\notin\vec W.\cr}$$

\wheader{551N}{0}{0}{0}{36pt}
\noindent As $p$ and $\dot x$ are arbitrary,

\Centerline{$\VVdP\,
(\chi W)\sspvec=\chi\vec W$.}

\medskip

{\bf (e)} Assembling (a)-(d), and recalling 551Fa,
we see that the result is true when $\psi$ is
a linear multiple of the indicator function
of a set in $\Sigma\tensorhat\CalBa_I$, whenever $\psi$ is a sum
of such functions, and whenever $\psi$ is the limit of a sequence of such
sums;  that is, whenever $\psi$ is $(\Sigma\tensorhat\CalBa_I)$-measurable.

\medskip

{\bf (f)} Similarly, (d) and 551I tell us that the result is true for the
indicator function of a member of $\Sigma\tensorhat\CalBa_I$.   Once
again, we can move to a linear combination of such functions,
using (b), and thence to a non-negative
$(\Sigma\tensorhat\CalBa_I)$-measurable function,
using (c);  finally, with (b) again, we get the general case.
}%end of proof of 551N

\leader{551O}{Measure algebras}\cmmnt{ With a little more effort we can
get a representation of the standard measure algebras in the same style.}
Let $I$ be a set, $\nu_I$ the usual measure on $\{0,1\}^I$ and
$(\frak B_I,\bar\nu_I)$ its measure algebra.   It will be important to
appreciate that these are abbreviations for formulae in set theory with a
single parameter $I$;   so that if we have a forcing notion $\Bbb P$ and a
$\Bbb P$-name $\tau$, we shall have $\Bbb P$-names $\frak B_{\tau}$ and
$\bar\nu_{\tau}$, uniquely defined as soon as we have settled on the
exact formulations we wish to apply when interpreting the basic
constructions $\{\ldots\}$, $\Cal P$ in the forcing language.
Similarly, if we write
$\Bbb P_I=(\frak B_I^+,\Bsubseteqshort,1,\downarrow)$
for the forcing notion based on the Boolean algebra $\frak B_I$, this also
is a formula which can be interpreted in forcing languages.

\leader{551P}{Theorem} Let $(\Omega,\Sigma,\Cal I)$ be a non-trivial
$\omega_1$-saturated measurable space with negligibles 
such that $\Sigma$ is closed under Souslin's operation.
Let $\Bbb P$ be the associated forcing notion, 
$P\cmmnt{\mskip5mu=(\Sigma/\Sigma\cap\Cal I)^+}$ 
its underlying partially ordered set, and $I$ a set.   Set

\Centerline{$\Lambda=\Sigma\tensorhat\CalBa_I$,
\quad$\Cal J=\{W:W\in\Lambda$, $\nu_IW[\{\omega\}]=0$ for
$\Cal I$-almost every $\omega\in\Omega\}$;}

\noindent then $\Cal J$ is a $\sigma$-ideal of $\Lambda$\cmmnt{ (cf.\
527B)};  let $\frak C$ be the quotient algebra $\Lambda/\Cal J$.   For
$W\in\Lambda$ and $\omega\in\Omega$
set $h_W(\omega)=\nu_IW[\{\omega\}]$.
For $a\in\frak C$ let $\vec a$ be the $\Bbb P$-name

\Centerline{$\{(\vec W,\Bbbone):W\in\Lambda$, $W^{\ssbullet}=a\}$}

\noindent where the $\Bbb P$-names $\vec W$ are defined as in 551D.
Consider the $\Bbb P$-names

\Centerline{$\dot{\frak D}=\{(\vec a,\Bbbone):a\in\frak C\}$,
\quad$\dot\pi
=\{(((W^{\ssbullet})\sspvec,(\vec W)^{\ssbullet}),\Bbbone):
W\in\Lambda\}$.}

(a) $\VVdP\,\dot\pi$ is a bijection between $\dot{\frak D}$ and
$\frak B_{\check I}$.

\wheader{551P}{0}{0}{0}{48pt}

(b) If $a$, $b\in\frak C$, $V\in\Lambda$ and
$V^{\ssbullet}=a$, then

\Centerline{$\VVdP\,
\dot\pi(a\Bsymmdiff b)\sspvec=\dot\pi\vec a\Bsymmdiff\dot\pi\vec b$,
\quad$\dot\pi(a\Bcap b)\sspvec=\dot\pi\vec a\Bcap\dot\pi\vec b$,
\quad$\bar\nu_{\check I}(\dot\pi\vec a)=\vec h_V$,}

\noindent defining $h_V$ and $\vec h_V$ as in 551I.

(c) Let $\varepsilon:\Sigma/\Sigma\cap\Cal I\to\frak C$ be the canonical
map defined by the formula

\Centerline{$\varepsilon(E^{\ssbullet})=(E\times\{0,1\}^I)^{\ssbullet}$
for $E\in\Sigma$.}

\noindent If $p\in(\Sigma/\Sigma\cap\Cal I)^+$ and $a$,
$b\in\frak C$, then

\Centerline{$p\VVdP\,\dot\pi\vec a=\dot\pi\vec b$}

\noindent iff $a\Bcap\varepsilon(p)=b\Bcap\varepsilon(p)$.

\cmmnt{\medskip

\noindent{\bf Remarks} Note that in the formula

\Centerline{$\{(((W^{\ssbullet})\sspvec,(\vec W)^{\ssbullet}),\Bbbone):
W\in\Lambda\}$}

\noindent the first $^{\ssbullet}$ is
interpreted in the ordinary universe as the canonical map from
$\Lambda$ to $\frak C$, and the second is interpreted in the forcing
language as the canonical map from $\CalBa_{\check I}$ to
$\frak B_{\check I}$;  while among the brackets $(\ldots)$, some are just
separators, some are to be interpreted as an ordered-pair construction in
the ordinary universe, and some are to be interpreted as and ordered-pair
construction in the forcing language.    Similarly, in the formula

\Centerline{$\VVdP\,
\dot\pi(a\Bsymmdiff b)\sspvec=\dot\pi\vec a\Bsymmdiff\dot\pi\vec b$}

\noindent the first $\Bsymmdiff$ is to be interpreted in the ordinary
universe as symmetric difference in the algebra $\frak C$,
while the second is to be interpreted in the forcing language as symmetric
difference in $\frak B_{\check I}$.
}

\proof{{\bf (a)(i)}
$\VVdP\,\dot\pi$ is a function with domain $\dot{\frak D}$ and
$\dot\pi[\dot{\frak D}]=\dot B$.

\Prf\Quer\ Suppose, if possible, that $V$, $W\in\Lambda$ and
$E\in\Sigma\setminus\Cal I$ are such that

\Centerline{$E^{\ssbullet}\VVdP\,
(V^{\ssbullet})\sspvec=(W^{\ssbullet})\sspvec$,
$\vec V^{\ssbullet}\ne\vec W^{\ssbullet}$.}

\noindent By 551I(iii) and 551Eb,

\Centerline{$E^{\ssbullet}\VVdP\,\vec h_{V\symmdiff W}
=\nu_{\check I}(V\symmdiff W)\sspvec\ne 0$.}

\noindent On the other hand,

\Centerline{$E^{\ssbullet}\VVdP\,\vec V\in(V^{\ssbullet})\sspvec
=(W^{\ssbullet})\sspvec$,}

\noindent so there must be a $W_1\in\Lambda$ and an
$F\in\Sigma\setminus\Cal I$ such that $F^{\ssbullet}$ is stronger than
$E^{\ssbullet}$,
$W_1\symmdiff W\in\Cal J$ and $F^{\ssbullet}\VVdP\,\vec V=\vec W_1$.
Now, calculating in $\Sigma/\Sigma\cap\Cal I$,

$$\eqalignno{F^{\ssbullet}
&\Bsubseteq\{\omega:V[\{\omega\}]=W_1[\{\omega\}]\}^{\ssbullet}\cr
\displaycause{551Gb}
&\Bsubseteq
\{\omega:\nu_I(V[\{\omega\}]\symmdiff W_1[\{\omega\}])=0\}^{\ssbullet}
=\{\omega:\nu_I(V[\{\omega\}]\symmdiff W[\{\omega\}])=0\}^{\ssbullet}\cr
\displaycause{because $W\symmdiff W_1\in\Cal J$, so
$\nu_I(W[\{\omega\}]\symmdiff W_1[\{\omega\}])=0$ for
$\Cal I$-almost every $\omega$}
&=\Bvalue{\vec h_{V\symmdiff W}=0}\cr}$$

\noindent (551B);  which is impossible, because
$E^{\ssbullet}\Bsubseteq\Bvalue{\vec h_{V\symmdiff W}\ne 0}$.\ \BanG\  
So 5A3H tells us that

\Centerline{$\VVdP\,\dot\pi$ is a function with domain $\dot{\frak D}$ 
and $\dot\pi[\dot{\frak D}]=\dot B$,}

\noindent where $\dot B$ is the $\Bbb P$-name
$\{(\vec W^{\ssbullet},\Bbbone):W\in\Lambda\}$.

\medskip

\quad{\bf (ii)} Now $\VVdP\,\dot\pi$ is injective.
\Prf\ I aim to use the condition in 5A3Hb.  I
take the argument in two bites.

\medskip

\qquad\grheada\Quer\
Suppose, if possible, that $V$, $W\in\Lambda$ and 
$p\in P$ are such that $p\VVdP\,\vec V^{\ssbullet}=\vec W^{\ssbullet}$ but
$p\notVVdash_{\Bbb P}\,
(V^{\ssbullet})\sspvec\subseteq(W^{\ssbullet})\sspvec$.   Then there are 
a $q\in P$, stronger than $p$, and a $\Bbb P$-name $\dot x$ such that

\Centerline{$q\VVdP\,
\dot x\in(V^{\ssbullet})\sspvec\setminus(W^{\ssbullet})\sspvec$.}

\noindent By the definition in 5A3Cb, there are an $r\in P$, stronger than
$q$, and a $V_1\in\Lambda$ such that
$V_1^{\ssbullet}=V^{\ssbullet}$ and $r\VVdP\,\dot x=\vec V_1$.
Let $E\in\Sigma\setminus\Cal I$ be such that $E^{\ssbullet}=r$, and set

\Centerline{$W_1=(V_1\cap(E\times\{0,1\}^I))
  \cup(W\cap((\Omega\setminus E)\times\{0,1\}^I))\in\Lambda$.}

\noindent Then $E^{\ssbullet}\Bsubseteq\Bvalue{\vec W_1=\vec V_1}$
(551Gb again).   At the same time, 

\Centerline{$E^{\ssbullet}
\Bsubseteq\Bvalue{\vec V^{\ssbullet}=\vec W^{\ssbullet}}
=\Bvalue{\vec h_{V\symmdiff W}=0}$}

\noindent as in (i) just above.   Now

$$\eqalign{&\{\omega:\omega\in\Omega,\,
  \nu_I(W_1[\{\omega\}]\symmdiff W[\{\omega\}])>0\}
\cr&\mskip80mu
=\{\omega:\omega\in E,\,
  \nu_I(V_1[\{\omega\}]\symmdiff W[\{\omega\}])>0\}
\cr&\mskip80mu
\subseteq\{\omega:\omega\in\Omega,\,
   \nu_I(V_1[\{\omega\}]\symmdiff V[\{\omega\}])>0\}
 \cup\{\omega:\omega\in E,\,h_{V\symmdiff W}(\omega)>0\}
\cr&\mskip80mu
\in\Cal I.\cr}$$

\noindent But this means that $W_1^{\ssbullet}=W^{\ssbullet}$ and
$(\vec W_1,\Bbbone)\in(W^{\ssbullet})\sspvec$, so that

\Centerline{$\VVdP\,\vec W_1\in(W^{\ssbullet})\sspvec$,
\quad$r\VVdP\,\dot x=\vec V_1=\vec W_1\in(W^{\ssbullet})\sspvec$;}

\noindent but $r$ is stronger than $q$ and

\Centerline{$q\VVdP\,\dot x\notin(W^{\ssbullet})\sspvec$,}

\noindent so we have a contradiction.\ \Bang

\medskip

\qquad\grheadb\ Thus if $V$, $W\in\Lambda$ and 
$p\in P$ are such that $p\VVdP\,\vec V^{\ssbullet}=\vec W^{\ssbullet}$,
we must have 
$p\VVdP\,(V^{\ssbullet})\sspvec\subseteq(W^{\ssbullet})\sspvec$.
Similarly,
$p\VVdP\,(W^{\ssbullet})\sspvec\subseteq(V^{\ssbullet})\sspvec$ and
$p\VVdP\,(W^{\ssbullet})\sspvec=(V^{\ssbullet})\sspvec$. 
Accordingly the conditions of 5A3Hb are satisfied and
$\VVdP\,\dot\pi$ is injective.\ \Qed

\medskip

\quad{\bf (iii)} We need to check that

\Centerline{$\VVdP\,\dot B=\frak B_{\check I}$.}

\medskip

\Prf\grheada\ Suppose that
$E\in\Sigma\setminus\Cal I$ and a $\Bbb P$-name $\dot x$ are such that
$E^{\ssbullet}\VVdP\,\dot x\in\dot B$.   Then there must be an
$F\in\Sigma\setminus\Cal I$ and a $W\in\Lambda$
such that $F^{\ssbullet}$ is stronger than $E^{\ssbullet}$ and
$F^{\ssbullet}\VVdP\,\dot x=\vec W^{\ssbullet}\in\frak B_{\check I}$;
as $E^{\ssbullet}$ and $\dot x$ are arbitrary,

\Centerline{$\VVdP\,\dot B\subseteq\frak B_{\check I}$.}

\medskip

\qquad\grheadb\ Suppose that
$E\in\Sigma\setminus\Cal I$ and a $\Bbb P$-name $\dot x$ are such that
$E^{\ssbullet}\VVdP\,\dot x\in\frak B_{\check I}$.   Then there must be a
$\Bbb P$-name $\dot W$ such that

\Centerline{$E^{\ssbullet}\VVdP\,\dot W\in\CalBa_{\check I}$ and
$\dot x=\dot W^{\ssbullet}$.}

\noindent By 551Fb, there is a $W\in\Lambda$ such that

\Centerline{$E^{\ssbullet}\VVdP\,\vec W=\dot W$,
so $\dot x=\vec W^{\ssbullet}\in\dot B$;}

\noindent as $E$ and $\dot x$ are arbitrary,

\Centerline{$\VVdP\,\frak B_{\check I}\subseteq\dot B$
and $\dot B=\frak B_{\check I}$.\ \Qed}

\medskip

\quad{\bf (iv)} Putting these together,

\Centerline{$\VVdP\,\dot\pi$ is a bijection between $\dot{\frak D}$ and
$\frak B_{\check I}$.}

\medskip

{\bf (b)} This is now easy.   If $V$, $W\in\Lambda$, $a=V^{\ssbullet}$ and
$b=W^{\ssbullet}$, then


$$\eqalignno{\VVdP\,\dot\pi(a\Bsymmdiff b)\sspvec
&=\dot\pi((V\symmdiff W)^{\ssbullet})\sspvec
=((V\symmdiff W)\sspvec)^{\ssbullet}
=(\vec V\symmdiff\vec W)^{\ssbullet}\cr
\displaycause{551Eb}
&=\vec V^{\ssbullet}\Bsymmdiff\vec W^{\ssbullet}
=\dot\pi\vec a\Bsymmdiff\dot\pi\vec b,\cr}$$

\noindent and similarly

\Centerline{$\VVdP\,\dot\pi(a\Bcap b)\sspvec
=\dot\pi\vec a\Bcap\dot\pi\vec b$.}

\noindent Finally,

\Centerline{$\VVdP\,\bar\nu_{\check I}(\dot\pi\vec a)
=\bar\nu_{\check I}\vec V^{\ssbullet}
=\nu_{\check I}\vec V
=\vec h_V$}

\noindent by 551I(iii) again.

\medskip

{\bf (c)} Let $E\in\Sigma\setminus\Cal I$ and $V$, $W\in\Lambda$ be such
that $E^{\ssbullet}=p$, $V^{\ssbullet}=a$ and $W^{\ssbullet}=b$.   Then
(b) tells us that

\Centerline{$\VVdP\,
\bar\nu_{\check I}(\dot\pi\vec a\Bsymmdiff\dot\pi\vec b)
=\bar\nu_{\check I}\dot\pi(a\Bsymmdiff b)\sspvec
=\bar\nu_{\check I}\dot\pi((V\Bsymmdiff W)^{\ssbullet})\sspvec
=\vec h_{V\symmdiff W}$.}

\noindent So

$$\eqalignno{p\VVdP\,\dot\pi\vec a=\dot\pi\vec b
&\iff p\VVdP\,\bar\nu_{\check I}(\dot\pi\vec a\Bsymmdiff\dot\pi\vec b)=0\cr
&\iff p\VVdP\,\vec h_{V\symmdiff W}=0\cr
&\iff h_{V\symmdiff W}(\omega)=0\text{ for }\Cal I\text{-almost every }
   \omega\in E\cr
\displaycause{551B}
&\iff \nu_I(V[\{\omega\}]\symmdiff W[\{\omega\}])=0
   \text{ for }\Cal I\text{-almost every }\omega\in E\cr
&\iff (E\times\{0,1\}^I)\cap(V\symmdiff W)\in\Cal J\cr
&\iff\varepsilon(p)\Bcap(a\Bsymmdiff b)=0
\iff a\Bcap\varepsilon(p)=b\Bcap\varepsilon(p).\cr}$$
}%end of proof of 551P

\leader{551Q}{Iterated \dvrocolon{forcing}}\cmmnt{ The machinery just
developed can
be used to establish one of the most important properties of
random real forcing.

\medskip

\noindent}{\bf Theorem} Let $(\Omega,\Sigma,\Cal I)$ be a non-trivial
$\omega_1$-saturated measurable space with negligibles
such that $\Sigma$ is closed under Souslin's operation,
$\Bbb P$ its associated forcing notion, and $I$ a set.    As in
551P, set $\Lambda=\Sigma\tensorhat\CalBa_I$,

\Centerline{$\Cal J=\{W:W\in\Lambda$, $\nu_IW[\{\omega\}]=0$ for
$\Cal I$-almost every $\omega\in\Omega\}$}

\noindent and $\frak C=\Lambda/\Cal J$.   Then

\Centerline{$\frak C\cong\RO(\Bbb P*\Bbb P_{\check I})$,}

\noindent where the $\Bbb P$-name $\Bbb P_{\check I}$ is defined as in
551O\cmmnt{, and $\Bbb P*\Bbb P_{\check I}$ is the iterated forcing notion
defined in 5A3O}.

\proof{{\bf (a)} Since I wish to follow {\smc Kunen 80} as closely as
possible, I should perhaps start with a remark on the interpretation of
names for forcing notions.   There is, strictly speaking, a distinction to
be made between a name for a forcing notion, which is a name for a
quadruplet of the form $(P,\le,\Bbbone,\updownarrows)$, and
a quadruplet of names, the first for a set, the second for a pre-order on
that set, and so on;  and the latter is easier to work with
({\smc Kunen 80}, \S VIII.5, and 5A3O).
In the present case, we do not need
any new manoeuvres, since the construction of
the name $\Bbb P_{\check I}$ is based
on $\Bbb P$-names for $\frak B_{\check I}$,
$\Bsubseteqshort_{\frak B_{\check I}}$ and
$1_{\frak B_{\check I}}$.

As usual in this section, I will write $\frak A$ for
$\Sigma/\Sigma\cap\Cal I$.

\medskip

{\bf (b)} Now $\Bbb P*\Bbb P_{\check I}$ is based on the set $P$ of pairs
$(p,\dot b)$ where $p\in\frak A^+$, $\dot b\in B$ and
$p\VVdP\,\dot b\in\frak B_{\check I}^+$;  here $B$ is the
domain of the $\Bbb P$-name $\frak B_{\check I}^+$ (5A3Ba).
If we say that
$(p,\dot b)\le(p',\dot b')$ if $p\Bsubseteq p'$ and
$p\VVdP\,\dot b\Bsubseteq\dot b'$, then $P$ is pre-ordered by $\le$ and
$\Bbb P*\Bbb P_{\check I}$ is active downwards.

We have a unique function $\theta:P\to\frak C^+$ such that

\Centerline{$\theta(p,\dot b)\Bsubseteq\varepsilon(p)$,
\quad$p\VVdP\,\dot\pi\theta(p,\dot b)\sspvec=\dot b$,}

\noindent whenever $(p,\dot b)\in P$, where $\varepsilon$, $\dot\pi$ and
$\vec a$, for $a\in\frak C$, are defined as
in 551P.    \Prf\ If $(p,\dot b)\in P$, so that
$p\VVdP\,\dot b\in\frak B_{\check I}$,
there is a $\Bbb P$-name $\dot b_1$ such that

\Centerline{$\VVdP\,\dot b_1\in\frak B_{\check I}$,}

\Centerline{$p\VVdP\,\dot b_1=\dot b$.}

\noindent Next, there is an
$a_0\in\frak C$ such that $\VVdP\,\dot\pi\vec a_0=\dot b_1$
(551Pa).   Set $a=a_0\Bcap\varepsilon(p)$.   Then 551Pc tells us that

\Centerline{$p\VVdP\,\dot\pi\vec a=\dot\pi\vec a_0
=\dot b_1=\dot b\ne 0$,}

\noindent and $a\ne 0$.
To see that $a$ is unique, observe that if $c\in\frak C$ is
such that $p\VVdP\,\dot\pi\vec c=\dot b$, then
$c\Bcap\varepsilon(p)=a\Bcap\varepsilon(p)$, by 551Pc again;  so if
$c\Bsubseteq\varepsilon(p)$, $c=a$.   We therefore can, and must,
take $a$ for $\theta(p,\dot b)$.\ \Qed

\wheader{551Q}{4}{2}{2}{42pt}

{\bf (c)(i)} If $(p,\dot b)$, $(p',\dot b')\in P$ and $(p,\dot b)$ is
stronger than $(p',\dot b')$, then $p\Bsubseteq p'$ and
$p\VVdP\,\dot b\Bsubseteq\dot b'$.   In this case,
\ifnum\stylenumber=12

\Centerline{$p\VVdP\,\dot\pi\theta(p',\dot b')\sspvec=\dot b'$ and
$\dot\pi(\theta(p,\dot b)\Bcap\theta(p',\dot b'))\sspvec
=\dot\pi\theta(p,\dot b)\sspvec\Bcap\dot\pi\theta(p',\dot b')\sspvec
=\dot b\Bcap\dot b'=\dot b$}
\else

\doubleinset{$p\VVdP\,\dot\pi\theta(p',\dot b')\sspvec=\dot b'$ and
$\dot\pi(\theta(p,\dot b)\Bcap\theta(p',\dot b'))\sspvec
=\dot\pi\theta(p,\dot b)\sspvec\Bcap\dot\pi\theta(p',\dot b')\sspvec
=\dot b\Bcap\dot b'=\dot b$,}
\fi

\noindent while
$\theta(p,\dot b)\Bcap\theta(p',\dot b')\Bsubseteq\varepsilon(p)$, so
$\theta(p,\dot b)\Bcap\theta(p',\dot b')=\theta(p,\dot b)$ and
$\theta(p,\dot b)\Bsubseteq\theta(p',\dot b')$.

\medskip

\allowmorestretch{468}{
\quad{\bf (ii)}
If $(p,\dot b)$ and $(p',\dot b')\in P$ are incompatible, then
$\theta(p,\dot b)\Bcap\theta(p',\dot b')=0$.   \Prf\Quer\ Otherwise,
writing $a$ for $\theta(p,\dot b)\Bcap\theta(p',\dot b')$,
}

\Centerline{$\varepsilon(p\Bcap p')=\varepsilon(p)\Bcap\varepsilon(p')
\Bsupseteq a\ne 0$}

\noindent so $p\Bcap p'\ne 0$ and

\Centerline{$p\Bcap p'\VVdP\,\dot\pi\vec a
\Bsubseteq\dot\pi\theta(p,\dot b)\sspvec
  \Bcap\dot\pi\theta(p',\dot b')\sspvec
=\dot b\Bcap\dot b'
=0$;}

\noindent as $a\Bsubseteq\varepsilon(p\Bcap p')$, $a$ must be $0$;  which
is absurd.\ \Bang\Qed

\medskip

\quad{\bf (iii)} If $a\in\frak C^+$, there is a
$(p,\dot b)\in P$ such that $\theta(p,\dot b)\Bsubseteq a$.   \Prf\
By 551Pc, $\notVVdash_{\Bbb P}\dot\pi\vec a=0$, that is, there is a
$p_0\in\frak C^+$ such that $p_0\VVdP\,\dot\pi\vec a\ne 0$.
Now there must be a
$p\in\frak C^+$, stronger than $p_0$,
and a $\dot b\in B$ such that
$p\VVdP\,\dot b=\dot\pi\vec a$, in which case $(p,\dot b)\in P$ and
$p\VVdP\,\dot\pi\theta(p,\dot b)\sspvec=\dot\pi\vec a$.   Accordingly

\Centerline{$\theta(p,\dot b)=\theta(p,\dot b)\Bcap\varepsilon(p)
=a\Bcap\varepsilon(p)\Bsubseteq a$.   \Qed}

\medskip

{\bf (d)} Observe now that $\frak C$ is ccc (527L), therefore Dedekind
complete, and (c) tells us that
$\theta:P\to\frak C^+$ satisfies the conditions
of 514Sa.     So
$\RO(\Bbb P*\Bbb P_{\check I})=\RO^{\downarrow}(P)$ is
isomorphic to $\frak C$.
}%end of proof of 551Q

\leader{551R}{Extending \dvrocolon{filters}}\cmmnt{ The following
device will be useful in \S553.
%553N

\medskip

\noindent}{\bf Proposition} Let $(\Omega,\Sigma,\Cal I)$ be a non-trivial
$\omega_1$-saturated
measurable space with negligibles, $\frak A$ its quotient algebra,
$\Bbb P$ the associated forcing notion,
$I$ a countable set and $\Cal F$ a filter on $I$.

(a) For $H\in\Sigma\tensorhat\Cal PI$,
write $\vec H$ for the $\Bbb P$-name
$\{(\varchecki,H^{-1}[\{i\}]^{\ssbullet}):
i\in I$, $H^{-1}[\{i\}]\notin\Cal I\}$.

\quad(i) $\VVdP\,\vec H\subseteq\check{I}$.

\quad(ii) If $\dot F$ is a $\Bbb P$-name and $p\in\frak A^+$ is such that
$p\VVdP\,\dot F\subseteq\check{I}$, then there is an
$H\in\Sigma\tensorhat\Cal P I$ such that
$p\VVdP\,\dot F=\vec H$.

(b) Write $\vec{\Cal F}$ for the $\Bbb P$-name

\Centerline{$\{(\vec H,E^{\ssbullet}):H\in\Sigma\tensorhat\Cal P I$,
$E\in\Sigma\setminus\Cal I$, $H[\{\omega\}]\in\Cal F$ for every
$\omega\in E\}$.}

\noindent Then

\Centerline{$\VVdP\,\vec{\Cal F}$ is a filter on $\check I$.}

\proof{{\bf (a)(i)} is elementary, just because

\Centerline{$\VVdP\,\varchecki\in\check{I}$}

\noindent for every $i\in I$.

\medskip

\quad{\bf (ii)} Because $(\Omega,\Sigma,\Cal I)$ is $\omega_1$-saturated,
$\frak A$ is Dedekind complete and can be identified with
$\RO(\Bbb P)$.   We therefore have, for each $i\in I$, an
$E_i\in\Sigma$ such that $E_i^{\ssbullet}$ can be identified with
$\Bvalue{\varchecki\in\dot F}$.   Set $H=\bigcup_{i\in I}E_i\times\{i\}$.
Then

\Centerline{$\Bvalue{\varchecki\in\dot F}
=H^{-1}[\{i\}]^{\ssbullet}=\Bvalue{\varchecki\in\vec H}$}

\noindent for every $i\in I$, so

\Centerline{$p\VVdP\,\dot F=\dot F\cap\check{I}=\vec H\cap\check{I}=\vec H$.}

\medskip

{\bf (b)(i)} By (a-i), $\VVdP\,\vec{\Cal F}\subseteq\check{I}$.

\medskip

\quad{\bf (ii)}
Since $((\Omega\times I)\sspvec,\Bbbone)\in\vec{\Cal F}$ and

\Centerline{$\VVdP\,(\Omega\times I)\sspvec=\check{I}$}

\noindent (551D), we have

\Centerline{$\VVdP\,\check{I}\in\vec{\Cal F}$.}

\medskip

\quad{\bf (iii)} If $(\vec H,p)\in\vec{\Cal F}$ then
$p\VVdP\,\vec H\ne\emptyset$.   \Prf\ Express $p$ as $E^{\ssbullet}$ where
$E\in\Sigma$ and $H[\{\omega\}]\in\Cal F$ for every $\omega\in E$.
Then $E\subseteq\bigcup_{i\in I}H^{-1}[\{i\}]$.   So if $q\in\frak A^+$
is stronger than $p$, there must be an $i\in I$ such that
$r=q\Bcap H^{-1}[\{i\}]^{\ssbullet}$ is non-zero;  in which case $r$ is
stronger than $q$ and

\Centerline{$r\VVdP\,\varchecki\in\vec H$, so $\vec H\ne\emptyset$.}

\noindent As $q$ is arbitrary, $p\VVdP\,\vec H\ne\emptyset$.\ \Qed

It follows at once that $\VVdP\,\emptyset\notin\vec{\Cal F}$.

\medskip

\quad{\bf (iv)} If $\dot F_0$, $\dot F_1$ are $\Bbb P$-names and
$p\in\frak A^+$ is such that

\Centerline{$p\VVdP\,\dot F_0$, $\dot F_1\in\vec{\Cal F}$,}

\noindent then

\Centerline{$p\VVdP\,\dot F_0\cap\dot F_1\in\vec{\Cal F}$.}

\noindent\Prf\ If $q\in\frak A^+$ is stronger than $p$ there must be
$(\vec H_0,E_0^{\ssbullet})$, $(\vec H_1,E_1^{\ssbullet})\in\vec{\Cal F}$
and an $r$ stronger than $q$, $E_0^{\ssbullet}$ and
$E_1^{\ssbullet}$ such that

\Centerline{$r\VVdP\,\dot F_0=\vec H_0$ and $\dot F_1=\vec H_1$.}

\noindent Now
$((H_0\cap H_1)\sspvec,(E_0\cap E_1)^{\ssbullet})\in\vec{\Cal F}$ and

\Centerline{$r\VVdP\,\dot F_0\cap\dot F_1=\vec H_0\cap\vec H_1
=(H_0\cap H_1)\sspvec\in\vec{\Cal F}$.}

\noindent As $q$ is arbitrary,

\Centerline{$p\VVdP\,\dot F_0\cap\dot F_1\in\vec{\Cal F}$.\ \Qed}

Accordingly

\Centerline{$\VVdP\,\vec{\Cal F}$ is closed under $\cap$.}

\medskip

\quad{\bf (v)} Suppose that $\dot F_0$, $\dot F_1$ are $\Bbb P$-names and
$p\in\frak A^+$ is such that

\Centerline{$p\VVdP\,\dot F_0\subseteq\dot F_1\subseteq\check{I}$,
$\dot F_0\in\vec{\Cal F}$.}

\noindent By (a-ii), there is an $H_1\in\Sigma\tensorhat\Cal P I$ such
that $p\VVdP\,\dot F_1=\vec H_1$.   If $q$ is stronger than $p$, there are
an $(\vec H_0,E_0^{\ssbullet})\in\vec{\Cal F}$ and an $r$ stronger than
both $q$ and $E_0^{\ssbullet}$ such that

\Centerline{$r\VVdP\,\vec H_0=\dot F_0\subseteq\dot F_1=\vec H_1$.}

\noindent Expressing $r$ as $E^{\ssbullet}$ where
$E\in\Sigma\setminus\Cal I$, we have

\Centerline{$E\cap H_0^{-1}[\{i\}]\setminus H_1^{-1}[\{i\}]\in\Cal I$}

\noindent for every $i\in I$.   Set

\Centerline{$E_1
=E\setminus\bigcup_{i\in I}(H_0^{-1}[\{i\}]\setminus H_1^{-1}[\{i\}])$;}

\noindent then $(\vec H_1,E_1^{\ssbullet})\in\vec{\Cal F}$, so

\Centerline{$r=E_1^{\ssbullet}\VVdP\,\dot F_1=\vec H_1\in\vec{\Cal F}$.}

\noindent As $q$ is arbitrary,

\Centerline{$p\VVdP\,\dot F_1\in\vec{\Cal F}$.}

\noindent As $p$, $\dot F_0$ and $\dot F_1$ are arbitrary,

\Centerline{$\VVdP\,\vec{\Cal F}$ is a filter on $\check{I}$.}
}%end of proof of 551R

\exercises{\leader{551X}{Basic exercises (a)}
%\spheader 551Xa
Let $(\Omega,\Sigma,\Cal I)$ be any measurable space with
negligibles.   Set $\hat\Sigma=\{E\symmdiff F:E\in\Sigma$, $F\in\Cal I\}$.
(i) Show that $(\Omega,\hat\Sigma,\Cal I)$ is a complete measurable space
with negligibles;  we may call it the {\bf completion} of
$(\Omega,\Sigma,\Cal I)$.
(ii) Show that the algebras $\Sigma/\Sigma\cap\Cal I$ and
$\hat\Sigma/\Cal I$ are canonically isomorphic (cf.\ 322Da).
(iii) Show that $(\Omega,\hat\Sigma,\Cal I)$ is $\omega_1$-saturated iff
$(\Omega,\Sigma,\Cal I)$ is.
%551A

\spheader 551Xb
Let $(\Omega,\Sigma,\mu)$ be a measure space
and $\Cal N(\mu)$ the null ideal of $\mu$.  (i) Show that
$(\Omega,\Sigma,\Cal N(\mu))$ is a measurable space with negligibles.
(ii) Show that if the completion of $(\Omega,\Sigma,\mu)$ (212C) is
$(\Omega,\hat\Sigma,\hat\mu)$,
then $(\Omega,\hat\Sigma,\Cal N(\hat\mu))$
is the completion of $(\Omega,\Sigma,\Cal N(\mu))$.
%551A

\spheader 551Xc Let $X$ be a topological space, $\Cal B(X)$ the Borel
$\sigma$-algebra of $X$, $\widehat{\Cal B}(X)$ the Baire-property algebra of $X$
 and $\Cal M$ the ideal of meager subsets of $X$.
(i) Show that $(X,\Cal B(X),\Cal M)$ is a measurable space with negligibles.
(ii) Show that its completion is $(X,\widehat{\Cal B}(X),\Cal M)$.
%551A

\spheader 551Xd Let $(\Omega,\Sigma,\Cal I)$ be a non-trivial measurable
space with negligibles, and $\Bbb P$ the associated forcing notion.
(i) Show that
the regular open algebra of $\Bbb P$ can be identified with the Dedekind
completion of $\Sigma/\Sigma\cap\Cal I$.   (ii) Show that if
$\Cal E$ is any coinitial subset of $\Sigma\setminus\Cal I$ containing
$\Omega$, then the forcing notion $\Cal E$, active downwards,
has regular open algebra isomorphic to $\RO(\Bbb P)$.
%551A

\spheader 551Xe Let $(\Omega,\Sigma,\Cal I)$ and $(\Upsilon,\Tau,\Cal J)$
be measurable spaces with negligibles.   Show that
$(\Omega\times\Upsilon,\Sigma\tensorhat\Tau,
\Cal I\ltimes_{\Sigma\tensorhat\Tau}\Cal J)$,
as defined in 527Bc, is a measurable space with negligibles.
%551A

\spheader 551Xf Let $(\Omega,\Sigma,\Cal I)$ be a non-trivial measurable
space with negligibles, and $\Bbb P$ the associated forcing notion.   Let
$f:\Omega\to\{0,1\}$ be a measurable function, and define $\Bbb P$-names
$\dot x$, $\dot y$ by saying that $\dot x$ is the $\Bbb P$-name for a 
real number as defined from $f$ in 551B, while
$\dot y$ is the $\Bbb P$-name for a member of $\{0,1\}$
as defined in 551Ca.   Show that

\Centerline{$\VVdP$ regarding $0$ and $1$ as real numbers,
$\dot x=\dot y$.}
%551C

\spheader 551Xg Let $(\Omega,\Sigma,\Cal I)$ be a non-trivial measurable
space
with negligibles, $\Bbb P$ the associated forcing notion, and $I$ a set.
Suppose that $f:\Omega\to\{0,1\}^I$ is a $(\Sigma,\CalBa_I)$-measurable
function, and that $\Gamma_f\subseteq\Omega\times\{0,1\}^I$ is its graph
(for once, I distinguish between $f$ and $\Gamma_f$).   Let $\vec f$ and
$\vec\Gamma_f$ be the $\Bbb P$-names for a point of $\{0,1\}^{\check I}$
and a subset of $\{0,1\}^{\check I}$ defined by the formulae in
551C and 551D respectively.   Show that
$\VVdP\,\vec\Gamma_f=\{\vec f\,\}$.
%551D

\spheader 551Xh Let $(\Omega,\Sigma,\Cal I)$ be a non-trivial measurable
space with negligibles, and $(\Omega,\hat\Sigma,\Cal I)$ its completion;
write $\Bbb P$, $\hat{\Bbb P}$ for the associated forcing notions, so that
$\Bbb P$ and $\hat{\Bbb P}$ are canonically isomorphic.
Let $I$ be a set, $W$ a member of
$\Sigma\tensorhat\{0,1\}^I\subseteq\hat\Sigma\tensorhat\{0,1\}^I$ and
$\vec W$ the $\Bbb P$-name, $\hat{\vec W}$ the $\hat{\Bbb P}$-name defined
by the formula in 551D.   Explain what it ought to mean to say that
$\VVdash_{\hat{\Bbb P}}\,\hat{\vec W}=\vec W$, and why this is true.
%551D 551Xa

\spheader 551Xi Let $(\Omega,\Sigma,\Cal I)$ be a non-trivial
$\omega_1$-saturated measurable space with negligibles with a Dedekind
complete quotient algebra,
$\Bbb P$ the associated forcing notion, and $I$ a set.   Suppose that
$W\in\Sigma\tensorhat\CalBa(\{0,1\}^I)$ is such that
$\VVdP\,\vec W\in\CalBa_{\check\alpha}(\{0,1\}^{\check I})$,
where $\alpha<\omega_1$.
Show that $W[\{\omega\}]\in\CalBa_{\alpha}(\{0,1\}^I)$ for
$\Cal I$-almost every $\omega\in\Omega$.
%551K

\leader{551Y}{Further exercises (a)}
%\spheader 551Ya
Investigate the difficulties which arise if we try to represent
names for Borel subsets of $\{0,1\}^{\check I}$ as members of
$\Sigma\tensorhat\{0,1\}^I$, when $I$ is uncountable.
Show that some of these are
resolvable if $\Omega$ is actually the Stone space of $\RO(\Bbb P)$.
%551F

\spheader 551Yb Let $(\Omega,\Sigma,\Cal I)$ be a non-trivial 
measurable space
with negligibles, $\Bbb P$ the associated forcing notion and
$P$ the partially ordered set underlying $\Bbb P$.
(i) Let $p\in P$
and a $\Bbb P$-name $\dot G$ be such that

\Centerline{$p\VVdP\,\dot G$ is a dense open subset of $\{0,1\}^{\omega}$.}

\noindent Show that there is a $W\in\Sigma\tensorhat\CalBa_{\omega}$ such that
every vertical section of $W$ is a dense open set and $p\VVdP\,\dot G=\vec W$.
(ii) Let $p\in P$
and a $\Bbb P$-name $\dot A$ be such that

\Centerline{$p\VVdP\,\dot A$ is a meager subset of $\{0,1\}^{\omega}$.}

\noindent Show that there is a $W\in\Sigma\tensorhat\CalBa_{\omega}$ such that
every vertical section of $W$ is a meager set and
$p\VVdP\,\dot A\subseteq\vec W$.
%551F

}%end of exercises

\endnotes{
\Notesheader{551}
There are real metamathematical difficulties in forcing, and we need to
find new compromises between formal rigour and intuitive accessibility.
In the formulae
of this section I am taking a path with rather more explicit
declarations than is customary.   The definitions
of $\vec u$ in 5A3L, $\vec f$ in 551Ca and $\vec W$ in 551D are
supposed to be $\Bbb P$-names in the exact sense used in {\smc Kunen 80}.
This leads to rather odd sentences of the form

\Centerline{$(\vec f,p)\in\vec W$ so $p\VVdP\,\vec f\in\vec W$}

\noindent (as in (a-ii) of the proof of 551E, for example), 
in which the symbol $\in$ is
being used in different ways in the two halves;
but it has the advantage that
we can move from $W$ to $\vec W$ without further explanation, as in the
statements of 551E-551J.  %551E 551F 551I 551J
But you will observe that elsewhere I allow such terms as
$\CalBa$ and $\nu_{\ldots}$ to enter sentences in the forcing language,
since these correspond to definitions which can be expanded there.
Note that I am being less strict than usual in requiring formulae to
be unambiguous (see 551Xf and 551Xg).

There is always room for variation in the matter of which terms
should be decorated with $\var2spcheck$s when they appear in expressions of the
forcing language, and I have tried to be reasonably consistent;  but there
are particular difficulties with transferring names for families (5A3Eb),
which appear here in such formulae as
`$\VVdP\,\lim_{n\to\infty}\vec h_n=\lim_{n\to\infty}\vec\psi_n(\dot x)$'
(part (c) of the proof of 551N).

I hope that it is not too confusing
to have the formula $\Bvalue{\ldots}$ used in two different ways, not
infrequently in the same sentence:  sometimes as a
`Boolean value' in the forcing sense, and sometimes in the sense of
Chapter 36.   If you look back at the definitions in \S364 you will see
that the expression $f^{\ssbullet}$
also shifts in interpretation as we move between the
formally distinct algebras $\frak A$ and $\RO(\Bbb P)$.   There are some
particularly difficult formulae to parse in 551P;  following the statement
of the theorem I offer a remark on 
the expression $(((W^{\ssbullet})\sspvec,(\vec W)^{\ssbullet}),\Bbbone)$, 
and some of the same difficulties arise in the line

\Centerline{$\VVdP\,\dot\pi(a\Bsymmdiff b)\sspvec
=\dot\pi((V\symmdiff W)^{\ssbullet})\sspvec
=((V\symmdiff W)\sspvec)^{\ssbullet}
=(\vec V\symmdiff\vec W)^{\ssbullet}$}

\noindent in part (b) of the proof, where as well as the ambiguities in
$^{\ssbullet}$ and $(\ldots)$ we have the symbols $\Bsymmdiff$ and
$\symmdiff$ being used first for an operation in the ground-model
Boolean algebra $\frak C$, then for symmetric difference in the ordinary
universe, and finally for symmetric
difference in the forcing language.
}%end of notes

\discrpage



