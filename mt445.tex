\frfilename{mt445.tex}
\versiondate{20.3.08}
\copyrightdate{1998}

\def\varhatbarf{{\setover{\,\,\lower 0.4ex
              \hbox{$\scriptscriptstyle\wedge$}}{\bar f}}}

\def\chaptername{Topological groups}
\def\sectionname{The duality theorem}

\newsection{445}

In this section I present a proof of the Pontryagin-van Kampen duality
theorem (445U).   As in Chapter 28, and for the same reasons, we need to
use complex-valued functions;  the relevant formulae in \S\S443 and 444
apply
unchanged, and I shall not repeat them here, but you may wish to re-read
parts of those sections taking functions to be complex- rather than
real-valued.   (It {\it is} possible to avoid complex-valued measures,
which I relegate to the exercises.)
The duality theorem itself applies only to abelian locally compact
Hausdorff groups, and it would be reasonable, on first reading, to take
it for granted that all groups here are of this type, which simplifies
some of the proofs a little.

My exposition is based on that of {\smc Rudin 67}.   I start with the
definition of `dual group', including a description of a topology on the
dual (445A), and the simplest examples (445B), with a mention of
Fourier-Stieltjes transforms of measures (445C-445D).   The elementary
special properties of dual groups of groups carrying Haar measures are
in 445E-445G;  in particular, in these cases, the bidual of a group
begins to make sense, and we can start talking about Fourier transforms
of functions.

Serious harmonic analysis begins with the identification of the dual
group with the maximal ideal space of $L^1$ (445H-445K).   The next idea
is that of `positive definite' function (445L-445M).   Putting these
together, we get the first result here which asserts that the
dual group of an abelian group $X$ carrying Haar measures is
sufficiently large to effectively describe functions on $X$ (Bochner's
theorem, 445N).   It is now easy to establish that $X$ can be faithfully
embedded in its bidual (445O).   We also have most of the machinery
necessary to describe the correctly normalized Haar measure of the dual
group, with a first step towards identifying functions whose Fourier
transforms will have inverse Fourier transforms (the Inversion Theorem,
445P).   This leads directly to the Plancherel Theorem, identifying the
$L^2$ spaces of $X$ and its dual (445R).   At this point it is clear
that the bidual $\frak X$ cannot be substantially larger than $X$, since
they must have essentially the same $L^2$ spaces.   A little
manipulation of
shifts and convolutions in $L^2$ (445S-445T) shows that $X$ must be
dense in $\frak X$, and a final appeal to local compactness
shows that $X$ is closed in $\frak X$.

\leader{445A}{Dual groups} Let $X$ be any topological group.

\spheader 445Aa A {\bf character} on $X$ is a continuous group
homomorphism from $X$ to $S^1=\{z:z\in\Bbb C,\,|z|=1\}$.   It is easy to
see that the set $\Cal X$ of all characters on $X$ is a subgroup of the
group $(S^1)^X$\ifwithproofs{, just because $S^1$ is an abelian
topological
group.   (If $\chi$, $\theta\in\Cal X$, then $x\mapsto\chi(x)\theta(x)$
is continuous, and

\Centerline{$(\chi\theta)(xy)
=\chi(xy)\theta(xy)
=\chi(x)\chi(y)\theta(x)\theta(y)
=\chi(x)\theta(x)\chi(y)\theta(y)
=(\chi\theta)(x)(\chi\theta)(y)$.)}

\noindent}\else{.  }\fi   So $\Cal X$ itself is an abelian group.

\spheader 445Ab Give $\Cal X$ the topology of uniform convergence on
subsets of $X$ which are totally bounded for the bilateral uniformity on
$X$\cmmnt{ (4A5Hb, 4A5O)}.   \cmmnt{(If $\Cal E$ is the set of
totally bounded subsets of $X$, then the topology of $\Cal X$ is
generated by the pseudometrics $\rho_E$, where
$\rho_E(\chi,\theta)=\sup_{x\in E}|\chi(x)-\theta(x)|$ for $E\in\Cal E$
and $\chi$, $\theta\in\Cal X$.
It will be useful, in this formula, to interpret $\sup\emptyset=0$, so
that $\rho_{\emptyset}$ is the zero pseudometric.   Note that $\Cal E$
is closed under finite unions, so $\{\rho_E:E\in\Cal E\}$ is
upwards-directed, as in 2A3Fe.)}   Then $\Cal X$ is a Hausdorff
topological group.   \prooflet{(If $x\in E\in\Cal E$ and $\chi$,
$\chi_0$, $\theta$, $\theta_0\in\Cal X$,

$$\eqalign{|(\chi\theta)(x)-(\chi_0\theta_0)(x)|
&=|\chi(x)(\theta(x)-\theta_0(x))+\theta_0(x)(\chi(x)-\chi_0(x))|\cr
&\le|\theta(x)-\theta_0(x)|+|\chi(x)-\chi_0(x)|,\cr}$$

\Centerline{$|\chi^{-1}(x)-\chi_0^{-1}(x)|
=|\overline{\chi(x)}-\overline{\chi_0(x)}|=|\chi(x)-\chi_0(x)|$,}

\noindent so

\Centerline{$\rho_E(\chi\theta,\chi_0\theta_0)
\le\rho_E(\chi,\chi_0)+\rho_E(\theta,\theta_0)$,
\quad$\rho_E(\chi^{-1},\chi_0^{-1})=\rho_E(\chi,\chi_0)$.}

\noindent If $\chi\ne\theta$ then there is an $x\in X$ such that
$\chi(x)\ne\theta(x)$, and now $\{x\}\in\Cal E$ and
$\rho_{\{x\}}(\chi,\theta)>0$.)
}%end of prooflet

\spheader 445Ac Note that if $X$ is locally compact, then\cmmnt{ its
totally bounded sets are just its relatively compact sets (4A5Oe),
so} the topology of $\Cal X$ is the topology of uniform convergence on
compact subsets of $X$.

\spheader 445Ad If $X$ is compact, then $\Cal X$ is discrete.
\prooflet{\Prf\ $X$ itself is totally bounded, so
$U=\{\chi:|\chi(x)-1|\le 1$ for every $x\in X\}$ is a neighbourhood of
the identity $\iota$ in $\Cal X$.   But if $\chi\in U$ and $x\in X$ then
$|\chi(x)^n-1|\le 1$ for every $n\in\Bbb N$, so $\chi(x)=1$.   Thus
$U=\{\iota\}$ and $\iota$ is an isolated point of $\Cal X$;  it follows
that every point of $\Cal X$ is isolated.\ \Qed}

\spheader 445Ae If $X$ is discrete then $\Cal X$ is compact.
\prooflet{\Prf\ The only totally bounded sets in $X$ are the finite
sets, so the topology of $\Cal X$ is just that induced by its embedding
in $(S^1)^X$.   On the other hand, every homomorphism from $X$ to $S^1$ is
continuous, so $\Cal X$ is a closed set in $(S^1)^X$, which is compact
by Tychonoff's theorem.\ \Qed}

\cmmnt{\spheader 445Af I ought to remark that to most group theorists
the word `character' means
something rather different.   For a finite abelian
group $X$ with its discrete topology, the `characters' on $X$, 
as defined in (a) above, are just the group homomorphisms from
$X$ to $\Bbb C\setminus\{0\}$, which in this context
can be identified with the characters of the irreducible 
complex representations of $X$.}

\leader{445B}{Examples (a)} If $X=\Bbb R$ with addition, then $\Cal X$
can also be identified with the additive group $\Bbb R$, if we write
$\chi_y(x)=e^{iyx}$ for $x$, $y\in\Bbb R$.

\prooflet{\Prf\ It is easy to check that every $\chi_y$, so defined, is
a character on $\Bbb R$, and that $y\mapsto\chi_y:\Bbb R\to\Cal X$ is a
homomorphism.   On the other hand, if $\chi$ is a character, then
(because it is continuous) there is a $\delta\ge 0$ such that
$|\chi(x)-1|\le 1$ whenever $|x|\le\delta$.   $\chi(\delta)$ is uniquely
expressible as $e^{i\alpha}$ where $|\alpha|\le\bover{\pi}2$.   Set
$y=\alpha/\delta$, so that $\chi(\delta)=\chi_y(\delta)$.   Now
$\chi(\bover12\delta)$ must be one of the square roots of
$\chi(\delta)$, so is $\pm\chi_y(\bover12\delta)$;  but as
$|\chi(\bover12\delta)-1|\le 1$, it must be $+\chi_y(\bover12\delta)$.
Inducing on $n$, we see that $\chi(2^{-n}\delta)=\chi_y(2^{-n}\delta)$
for every $n\in\Bbb N$, so that
$\chi(2^{-n}k\delta)=\chi_y(2^{-n}k\delta)$ for every $k\in\Bbb Z$,
$n\in\Bbb N$;  as $\chi$ and $\chi_y$ are continuous, $\chi=\chi_y$.
Thus the map $y\mapsto\chi_y$ is surjective and is a group isomorphism
between $\Bbb R$ and $\Cal X$.

As for the topology of $\Cal X$, $\Bbb R$ is a locally compact
topological group, so
the totally bounded sets are just the relatively compact sets 
(4A5Oe again),
that is, the bounded sets in the usual sense (2A2F).   Now a
straightforward calculation shows that for any $\alpha\ge 0$ in
$\Bbb R$ and $\epsilon\in\ooint{0,2}$,

\Centerline{$\rho_{[-\alpha,\alpha]}(\chi_y,\chi_z)\le\epsilon
\iff\alpha|y-z|\le 2\arcsin\Bover{\epsilon}2$,}

\noindent so that the topology of $\Cal X$ agrees with that of
$\Bbb R$.\ \Qed}%end of prooflet

\spheader 445Bb Let $X$ be the group $\Bbb Z$ with its discrete
topology.   Then we
may identify its dual group $\Cal X$ with $S^1$ itself, writing
$\chi_{\zeta}(n)=\zeta^n$ for $\zeta\in S^1$, $n\in\Bbb Z$.
\prooflet{\Prf\ Once again, it is elementary to check that every
$\chi_{\zeta}$ is a character, and that $\zeta\mapsto\chi_{\zeta}$ is an
injective group homomorphism from $S^1$ to $\Cal X$.   If $\chi\in\Cal
X$,
set $\zeta=\chi(1)$;  then $\chi=\chi_{\zeta}$.   So $\Cal X\cong S^1$.
And because the only totally bounded sets in $X$ are finite,
$\chi\mapsto\chi_{\zeta}$ is continuous, therefore a
homeomorphism.\ \Qed}%end of prooflet

\spheader 445Bc On the other hand, if $X=S^1$ with its usual topology,
then we may identify its dual group $\Cal X$ with $\Bbb Z$, writing
$\chi_n(\zeta)=\zeta^n$ for $n\in\Bbb Z$, $\zeta\in S^1$.
\prooflet{\Prf\ The verification follows the same lines as in (a) and
(b).   As usual, the key step is to show that the map
$n\mapsto\chi_n:\Bbb Z\to\Cal X$ is surjective.   We can do this by applying (a).   If
$\chi\in\Cal X$, then $x\mapsto\chi(e^{ix})$ is a character of $\Bbb R$,
so there is a $y\in\Bbb R$ such that $\chi(e^{ix})=e^{iyx}$ for every
$x\in\Bbb R$.   In particular, $e^{2iy\pi}=\chi(1)=1$, so $y\in\Bbb Z$,
and $\chi=\chi_y$.   Concerning the topology of $\Cal X$, we know from
445Ad that it must be discrete, so that also matches the usual topology
of $\Bbb Z$.\ \Qed}

\spheader 445Bd Let $\family{j}{J}{X_j}$ be any family of topological
groups, and $X$ their product\cmmnt{ (4A5G)}.   For each $j\in J$ let
$\Cal X_j$ be the dual group of $X_j$.   Then the dual group of $X$ can
be identified with the subgroup $\Cal X$ of $\prod_{j\in I}\Cal X_j$
consisting
of those $\chi\in\prod_{j\in J}\Cal X_j$ such that $\{j:\chi(j)$ is not
the identity$\}$ is finite;  the action of $\Cal X$ on $X$ is defined by
the formula

\Centerline{$\chi\action x=\prod_{j\in J}\chi(j)(x(j))$.}

\noindent\cmmnt{(This is well-defined because only finitely many
terms in the product are not equal to $1$.) }If $I$ is finite, so that
$\Cal X=\prod_{j\in I}\Cal X_j$, the topology of $\Cal X$ is the product
topology.

\prooflet{\Prf\ As usual, it is easy to check that $\action$, as defined
above, defines an injective homomorphism from $\Cal X$ to the dual group
of $X$.   If $\theta$ is any character on $X$, then for each $j\in I$ we
have a continuous group homomorphism $\varepsilon_j:X_j\to X$ defined by
setting $\varepsilon_j(\xi)(j)=\xi$, $\varepsilon_j(\xi)(k)=e_k$, the
identity of $X_k$, for every $k\ne j$.   Setting
$\chi(j)=\theta\varepsilon_j$ for each $j$, we obtain
$\chi\in\prod_{j\in I}\Cal X_j$.   Now there is a neighbourhood $U$ of
the
identity of $X$ such that $|\theta(x)-1|\le 1$ for every $x\in U$, and
we may suppose that $U$ is of the form $\{x:x(j)\in G_j$ for every $j\in
J\}$, where $J\subseteq I$ is finite and $G_j$ is a neighbourhood of
$e_j$ for every $j\in J$.   If $k\in I\setminus J$,
$\varepsilon_k(\xi)\in U$ for every $\xi\in X_k$, so that
$|\chi(k)(\xi)-1|\le 1$ for every $\xi$, and $\chi(k)$ must be the
identity character on $X_k$;  this shows that $\chi\in\Cal X$.
If $x\in X$ and $x(j)=e_j$ for $j\in J$, then again $|\theta(x^n)-1|\le
1$ for every $n\in\Bbb N$, so $\theta(x)=\chi\action x=1$.   For any
$x\in X$, we can express it as a finite product $y\prod_{j\in
J}\varepsilon_j(x(j))$ where $y(j)=e_j$ for every $j\in J$, so that

\Centerline{$\theta(x)=\theta(y)\prod_{j\in J}\theta\varepsilon_j(x(j))
=\prod_{j\in J}\chi(j)(x(j))
=\chi\action x$.}

\noindent Thus $\action$ defines an isomorphism between $\Cal X$ and the
dual group of $X$.

As for the topology of $\Cal X$, a subset of $X$ is totally bounded iff
it is included in a product of totally bounded sets (4A5Od).
If $E=\prod_{j\in I}E_j$ is such a product (and not empty), then for
$\chi$, $\theta\in\Cal X$

\Centerline{$\sup_{j\in I}\rho_{E_j}(\chi(j),\theta(j))
\le\rho_E(\chi,\theta)
\le\sum_{j\in I}\rho_{E_j}(\chi(j),\theta(j))$,}

\noindent so (if $I$ is finite) the topology on $\Cal X$ is just the
product topology.\ \Qed
}%end of prooflet

\leader{445C}{Fourier-Stieltjes transforms} Let $X$ be a topological
group, and $\Cal X$ its dual group.
For any totally finite topological measure $\nu$ on $X$, we can form its
`characteristic function' or {\bf Fourier-Stieltjes transform}
$\varhat{\nu}:\Cal X\to\Bbb C$ by writing
$\varhat{\nu}(\chi)=\int\chi(x)\nu(dx)$.   \cmmnt{ (This generalizes the
`characteristic functions' of \S285.)}

\leader{445D}{Theorem} Let $X$ be a topological group, and $\Cal X$ its
dual group.   If $\lambda$ and $\nu$ are totally finite quasi-Radon
measures on $X$, then 
$(\lambda*\nu)\varsphat=\varhat{\lambda}\times\varhat{\nu}$.

\proof{ If $\chi\in\Cal X$, then, by 444C,

$$\eqalign{(\lambda*\nu)\varsphat(\chi)
&=\int\chi\,d(\lambda*\nu)
=\iint\chi(xy)\lambda(dx)\nu(dy)\cr
&=\iint\chi(x)\chi(y)\lambda(dx)\nu(dy)
=\int\chi(x)\lambda(dx)\cdot\int\chi(y)\nu(dy)
=\varhat{\lambda}(\chi)\varhat{\nu}(\chi).\cr}$$

}%end of proof of 445D


\leader{445E}{}\cmmnt{ Let us turn now to groups carrying Haar
measures.   I start with three welcome properties.

\medskip

\noindent}{\bf Proposition} Let $X$ be a topological group with a
neighbourhood of the identity which is totally bounded for the bilateral
uniformity on $X$, and $\Cal X$ its dual group\cmmnt{, with its dual
group topology}.

(a) The map $(\chi,x)\mapsto \chi(x):\Cal X\times X\to S^1$ is
continuous.

(b) Let $\frak X$ be the dual group of $\Cal X$\cmmnt{, again with its
dual group topology, the topology of uniform convergence on totally
bounded subsets of $\Cal X$}.   Then we have a continuous homomorphism
$x\mapsto\hat x:X\to\frak X$ defined by setting $\hat x(\chi)=\chi(x)$
for $x\in X$ and $\chi\in\Cal X$.

(c) For any totally finite quasi-Radon measure $\nu$ on $X$, its
Fourier-Stieltjes
transform $\varhat{\nu}:\Cal X\to\Bbb C$ is uniformly continuous.

\cmmnt{\medskip

\noindent{\bf Remark} Note that the condition here is satisfied by any
topological group $X$ carrying Haar measures (443H).}

\proof{ Fix an open totally bounded set $U_0$ containing the identity.

\medskip

{\bf (a)} Let $\chi_0\in\Cal X$, $x_0\in X$ and $\epsilon>0$.   Then
$x_0U_0$ is totally bounded, so

\Centerline{$V=\{\chi:|\chi(y)-\chi_0(y)|\le\Bover12\epsilon$ for every
$y\in x_0U_0\}$}

\noindent is a neighbourhood of $\chi_0$.   Also

\Centerline{$U=\{x:x\in x_0U_0$,
$|\chi_0(x)-\chi_0(x_0)|\le\Bover12\epsilon\}$}

\noindent is a neighbourhood of $x_0$.   And if $\chi\in V$, $x\in U$ we
have

\Centerline{$|\chi(x)-\chi_0(x_0)|
\le|\chi(x)-\chi_0(x)|+|\chi_0(x)-\chi_0(x_0)|\le\epsilon$.}

\noindent As $\chi_0$, $x_0$ and $\epsilon$ are arbitrary,
$(\chi,x)\mapsto\chi(x)$ is continuous.

\medskip

{\bf (b)(i)} It is easy to check that $\hat x$, as defined above, is
always a homomorphism from $\Cal X$ to $S^1$, and that $x\mapsto\hat
x:X\to(S^1)^{\Cal X}$ is a homomorphism.
Because $\rho_{\{x\}}$ is always one of the defining pseudometrics for
the topology of $\Cal X$ (445Ab), $\hat x$ is always continuous, so
belongs to $\frak X$.

\medskip

\quad{\bf (ii)} To see that \hbox{$\sphat\,$} is continuous, I argue as
follows.   Take an open set $H\subseteq\frak X$ and $x_0\in X$ such that
$\hat x_0\in H$.   Then there are a totally bounded set
$F\subseteq\Cal X$ and
an $\epsilon>0$ such that $\frak x\in H$ whenever $\frak x\in\frak X$
and $\rho_F(\frak x,\hat x_0)\le\epsilon$.   Now $x_0U_0$ is a totally
bounded neighbourhood of $x_0$, so

\Centerline{$V=\{\theta:\theta\in\Cal X$,
$|\theta(y)-1|\le\Bover12\epsilon$ for every $y\in x_0U_0\}$}

\noindent is a neighbourhood of the identity in $\Cal X$.   There are
therefore $\chi_0,\ldots,\chi_n\in\Cal X$ such that
$F\subseteq\bigcup_{k\le n}\chi_kV$.   Set

\Centerline{$U=\{x:x\in x_0U_0$,
$|\chi_k(x)-\chi_k(x_0)|<\Bover12\epsilon$ for every $k\le n\}$.}

\noindent Then $U$ is an open neighbourhood of $x_0$ in $X$.

If $x\in U$ and $\chi\in F$ then there is a $k\le n$ such that
$\theta=\chi_k^{-1}\chi\in V$, so that

$$\eqalign{|\chi(x)-\chi(x_0)|
&=|\chi_k(x)\theta(x)-\chi_k(x_0)\theta(x_0)|\cr
&\le|\chi_k(x)-\chi_k(x_0)|+|\theta(x)-\theta(x_0)|
\le\Bover12\epsilon+\Bover12\epsilon
=\epsilon.\cr}$$

\noindent But this shows that $\rho_F(\hat x,\hat x_0)\le\epsilon$, so
$\hat x\in H$.

So we have $x_0\in U\subseteq\{x:\hat x\in H\}$.   As $x_0$ is
arbitrary, $\{x:\hat x\in H\}$ is open;  as $H$ is arbitrary,
$x\mapsto\hat x$ is continuous.

\medskip

{\bf (c)} Let $\epsilon>0$.   Because $\nu$ is $\tau$-additive, there
are $x_0,\ldots,x_n\in X$ such that $\nu(X\setminus\bigcup_{k\le
n}x_kU_0)\le\bover13\epsilon$.   Set $E=\bigcup_{k\le n}x_kU_0$;  then
$E$ is totally bounded.   So

\Centerline{$V=\{\theta:|\theta(x)-1|\le\Bover{\epsilon}{1+3\nu X}$
for every $x\in E\}$}

\noindent is a neighbourhood of the identity in $\Cal X$.   If $\chi$,
$\chi'\in\Cal X$ are such that $\theta=\chi^{-1}\chi'$ belongs to $V$,
then

$$\eqalign{|\varhat{\nu}(\chi)-\varhat{\nu}(\chi')|
&\le\int|\chi(x)-\chi'(x)|\nu(dx)
=\int|1-\theta(x)|\nu(dx)\cr
&\le 2\nu(X\setminus E)+\Bover{\epsilon\nu E}{1+3\nu X}
\le\epsilon.\cr}$$

\noindent Since $\epsilon$ is arbitrary (and $\Cal X$ is abelian), this
is enough to show that $\varhat{\nu}$ is uniformly continuous.
}%end of proof of 445E

\leader{445F}{Fourier transforms of functions} Let $X$ be a topological
group with a left Haar measure $\mu$.   For any
$\mu$-integrable complex-valued function $f$, define its {\bf Fourier
transform} $\varhatf:\Cal X\to\Bbb C$ by setting
$\varhatf(\chi)=\int f(x)\chi(x)\mu(dx)$ for every character $\chi$ of
$X$.  \cmmnt{(Compare 283A.   If $f$ is real and non-negative, then
$\varhatf=(f\mu)\varsphat$ as defined in 445C, where $f\mu$ is the
indefinite-integral measure, as in 444K.)}
\cmmnt{Note that} $\varhatf=\varhat{g}$ whenever $f\eae g$, so we
can\cmmnt{ equally
well} write $\varhat{u}(\chi)=\varhatf(\chi)$ whenever $u=f^{\ssbullet}$
in $L_{\Bbb C}^1(\mu)$.

\leader{445G}{Proposition} Let $X$ be a topological group with a left
Haar measure $\mu$.   Then for any $\mu$-integrable complex-valued
functions $f$ and $g$,
$(f*g)\varsphat=\varhatf\times\varhat{g}$;  \cmmnt{that
is,} $(u*v)\varsphat=\varhat{u}\times\varhat{v}$ for all $u$, $v\in
L_{\Bbb C}^1(\mu)$.

\proof{ For any character $\chi$ on $X$,

$$\eqalign{\int\chi(x)(f*g)(x)dx
&=\iint\chi(xy)f(x)g(y)dxdy\cr
&=\iint\chi(x)\chi(y)f(x)g(y)dxdy
=\varhatf(\chi)\varhat{g}(\chi)\cr}$$
%too long for Centerline in full size

\noindent (using 444Od).
}%end of proof of 445G

\leader{445H}{Theorem} Let $X$ be a topological group with a left Haar
measure $\mu$;  let $\Cal X$ be its dual group and let $\Phi$ be the set
of non-zero multiplicative linear functionals on the complex Banach
algebra $\cmmnt{L_{\Bbb C}^1=}L_{\Bbb C}^1(\mu)$\cmmnt{ (444Sb)}.
Then there is a one-to-one
correspondence between $\Cal X$ and $\Phi$, defined by the formulae

\Centerline{$\phi(f^{\ssbullet})
=\int f\times\chi\,d\mu=\varhatf(\chi)$
for every $f\in\eusm L_{\Bbb C}^1=\eusm L_{\Bbb C}^1(\mu)$,}

\Centerline{$\phi(a\action_lu)=\chi(a)\phi(u)$ for every
$u\in L_{\Bbb C}^1$, $a\in X$,}

\noindent for $\chi\in\Cal X$ and $\phi\in\Phi$.

\cmmnt{\medskip

\noindent{\bf Remark} I follow 443G in writing
$a\action_lf^{\ssbullet}=(a\action_lf)^{\ssbullet}$, where
$(a\action_lf)(x)=f(a^{-1}x)$ for $f\in\eusm L_{\Bbb C}^1$ and $a$,
$x\in X$, as in 4A5Cc.
}%end of comment

\proof{{\bf (a)} If $\chi\in\Cal X$ then we can think of its equivalence
class $\chi^{\ssbullet}$ as a member of
$L_{\Bbb C}^{\infty}=L_{\Bbb C}^{\infty}(\mu)$, so
that we can define $\phi_{\chi}\in(L_{\Bbb C}^1)^*$ by writing
$\phi_{\chi}(u)=\int\chi^{\ssbullet}\times u$ for every
$u\in L_{\Bbb C}^1$;
that is, $\phi_{\chi}(f^{\ssbullet})=\int f\times\chi=\varhatf(\chi)$
for every $f\in\eusm L_{\Bbb C}^1$.
445G tells us that $\phi_{\chi}$ is multiplicative.   To see that it is
non-zero, recall that $\mu$ is strictly positive (442Aa) and that $\chi$
is continuous.   Let $G$ be an open set containing the identity $e$ of
$X$ such that $|\chi(x)-1|\le\bover12$ for every $x\in G$;  then
$\Real(\chi(x))\ge\bover12$ for every $x\in G$, so

\Centerline{$|\int_G\chi(x)dx|\ge\Real\int_G\chi(x)dx
=\int_G\Real(\chi(x))dx\ge\bover12\mu G>0$.}

\noindent Accordingly $\phi_{\chi}(\chi G)^{\ssbullet}\ne 0$ and
$\phi_{\chi}\ne 0$.   (I hope that no confusion will arise if I continue
occasionally to write $\chi E$ for the indicator function of a set
$E$, even if the symbol $\chi$ is already active in the sentence.)

\medskip

{\bf (b)} Now suppose that $\phi$ is a non-zero multiplicative linear
functional on $L_{\Bbb C}^1$.   Fix on some $g_0\in\eusm L_{\Bbb C}^1$
such that $\phi(g_0^{\ssbullet})=1$.
Let $\Delta$ be the left modular function of $X$.   (If you are reading
this proof on the assumption that $X$ is abelian, then
$a\action_rf=a^{-1}\action_lf$
and $\Delta\equiv 1$, so the argument below simplifies usefully.)

\medskip

\quad{\bf (i)} For any $u\in L_{\Bbb C}^1$ and $a\in X$,
$\phi(a^{-1}\action_lu)
=\Delta(a)\phi(a\action_ru)$.
\Prf\ Let $f\in\eusm L^1_{\Bbb C}$ be such that $f^{\ssbullet}=u$.
Take any $\epsilon>0$.   Then for any sufficiently small open
neighbourhood $U$
of the identity, if we set $h=\Bover1{\mu U}\chi U$, we shall have

\Centerline{$\|(a\action_rf)*h-a\action_rf\|_1\le\epsilon$,
\quad$\|h*f-f\|_1\le\epsilon$}

\noindent (444T, with 444P;
see 444U).   Setting $w=h^{\ssbullet}$, we have

\Centerline{$\|(a\action_ru)*w-a\action_ru\|_1\le\epsilon$,
\quad$\|w*u-u\|_1\le\epsilon$,}

$$\eqalignno{
|\phi(a^{-1}\action_lu)-\Delta(a)\phi(a\action_ru)|
&\le|\phi(a^{-1}\action_lu)-\phi(u*(a^{-1}\action_lw))|\cr
\displaycause{444Sa}
&\mskip100mu +|\Delta(a)\phi((a\action_ru)*w)
     -\Delta(a)\phi(a\action_ru)|\cr
&=|\phi(a^{-1}\action_lu)
  -\phi(u)\phi(a^{-1}\action_lw)|\cr
&\mskip100mu
  +\Delta(a)|\phi((a\action_ru)*w)-\phi(a\action_ru)|\cr
&\le|\phi(a^{-1}\action_lu)
  -\phi(a^{-1}\action_lw)\phi(u)|\cr
&\mskip100mu  +\Delta(a)\|(a\action_ru)*w-a\action_ru\|_1\cr
\displaycause{because $\|\phi\|\le 1$ in $(L_{\Bbb C}^1)^*$, by 4A6F}
&\le|\phi(a^{-1}\action_lu)
  -\phi((a^{-1}\action_lw)*u)|+\epsilon\Delta(a)\cr
&\le\|a^{-1}\action_lu-(a^{-1}\action_lw)*u\|_1
  +\epsilon\Delta(a)\cr
&=\|a^{-1}\action_l(u-w*u)\|_1
  +\epsilon\Delta(a)\cr
\displaycause{by another of the formulae in 444Sa}
&=\|u-w*u\|_1
  +\epsilon\Delta(a)\cr
\displaycause{443Ge}
&\le(1+\Delta(a))\epsilon.\cr}$$

\noindent As $\epsilon$ is arbitrary, we have the result. \Qed

\medskip

\quad{\bf (ii)} For any $u$, $v\in L_{\Bbb C}^1$ and $a\in X$,
$\phi(a\action_lu)\phi(v)
=\phi(u)\phi(a\action_lv)$.   \Prf\

$$\eqalign{\phi(a\action_lu)\phi(v)
&=\phi((a\action_lu)*v)
=\phi(a\action_l(u*v))
=\Delta(a^{-1})\phi(a^{-1}\action_r(u*v))\cr
&=\Delta(a^{-1})\phi(u*a^{-1}\action_rv)
=\phi(u)\Delta(a^{-1})\phi(a^{-1}\action_rv)
=\phi(u)\phi(a\action_lv),\cr}$$

\noindent using (i) for the third and sixth equalities, and 444Sa for
the second and fourth.\ \Qed

\medskip

\quad{\bf (iii)} Let $v_0$ be $g_0^{\ssbullet}$, so that $\phi(v_0)=1$,
and set $\chi(a)=\phi(a\action_lv_0)$ for
every $a\in X$.   Then if $a$, $b\in X$,

$$\eqalignno{\chi(ab)
&=\phi(ab\action_lv_0)
=\phi(a\action_l(b\action_lv_0))\cr
\noalign{\noindent (because $\action_l$ is an action of $X$ on
$L_{\Bbb C}^1$, as noted in 443Ge for $L^1$)}
&=\phi(a\action_l(b\action_lv_0))\phi(v_0)
=\phi(b\action_lv_0)\phi(a\action_lv_0)\cr
\noalign{\noindent (by (ii) above)}
&=\chi(a)\chi(b).\cr}$$

\noindent So $\chi:X\to\Bbb C$ is a group homomorphism.   Moreover,
because $\action_l$ is continuous (443Gf), and $\phi$ also is continuous
(indeed, of norm at most $1$), $\chi$ is continuous.   Finally,

\Centerline{$|\chi(a)|\le\|a\action_lv_0\|_1=\|v_0\|_1$}

\noindent for every $a\in X$, by 443Gb;  it follows at once that
$\{\chi(a)^n:n\in\Bbb Z\}$ is bounded, so that $|\chi(a)|=1$, for every
$a\in X$.   Thus $\chi\in\Cal X$.   Moreover, for any $u\in L_{\Bbb C}^1$,

\Centerline{$\phi(a\action_lu)=\phi(a\action_lu)\phi(v_0)
=\phi(u)\phi(a\action_lv_0)=\chi(a)\phi(u)$.}

\medskip

\quad{\bf (iv)} Now $\phi=\phi_{\chi}$.   \Prf\  Because
$\phi\in(L_{\Bbb C}^1)^*$, there is some
$h\in\eusm L_{\Bbb C}^{\infty}(\mu)$ such that
$\phi(f^{\ssbullet})=\int h(x)f(x)dx$ for every $f\in\eusm L_{\Bbb C}^1$
(243Gb/243K;
recall that by the rules of 441D, $\mu$ is suppose to be a quasi-Radon
measure, therefore strictly localizable, by 415A).   In this case, for
any $f\in\eusm L_{\Bbb C}^1$,

$$\eqalignno{\phi(f^{\ssbullet})
&=\phi(f*g_0)^{\ssbullet}
=\int h(x)(f*g_0)(x)dx
=\iint h(xy)f(x)g_0(y)dydx\cr
\noalign{\noindent (444Od)}
&=\iint h(y)f(x)g_0(x^{-1}y)dydx
=\int\phi(x\action_lg_0)^{\ssbullet}f(x)dx\cr
&=\int\chi(x)f(x)dx
=\phi_{\chi}(f^{\ssbullet}).\text{ \Qed}\cr}$$

\medskip

{\bf (c)} Thus we see that the formulae announced do define a surjection
from $\Cal X$ onto $\Phi$.   We have still to confirm that it is
injective.   But if $\chi$, $\theta$ are distinct members of $\Cal X$,
then $\{x:\chi(x)\ne\theta(x)\}$ is a non-empty open set, so has
positive measure, because $\mu$ is strictly positive;  because
$\mu$ is semi-finite, they represent different linear functionals on
$L_{\Bbb C}^1$, and $\phi_{\chi}\ne\phi_{\theta}$.

This completes the proof.
}%end of proof of 445H

\leader{445I}{The topology of the dual group:  Proposition} Let $X$ be a
topological group with a left Haar measure $\mu$, and $\Cal X$ its dual
group.   For $\chi\in\Cal X$, let $\chi^{\ssbullet}$ be its equivalence
class in $L_{\Bbb C}^0=L_{\Bbb C}^0(\mu)$, and $\phi_{\chi}\in(L_{\Bbb
C}^1)^*=(L_{\Bbb C}^1(\mu))^*$ the
multiplicative linear functional corresponding to $\chi$\cmmnt{, as in
445H}.   Then the
maps $\chi\mapsto\chi^{\ssbullet}$ and $\chi\mapsto\phi_{\chi}$ are
homeomorphisms between $\Cal X$ and its images in $L_{\Bbb C}^0$ and
$(L_{\Bbb C}^1)^*$,
if we give $L_{\Bbb C}^0$ the topology of convergence in
measure\cmmnt{ (245A/245M)}
and $(L_{\Bbb C}^1)^*$ the weak* topology\cmmnt{ (2A5Ig)}.

\proof{{\bf (a)} Note that $\chi\mapsto\chi^{\ssbullet}$ is injective
because $\mu$ is strictly positive, so that if $\chi$, $\theta$ are
distinct members of $\Cal X$ then the non-empty open set
$\{x:\chi(x)\ne\theta(x)\}$ has non-zero measure;  and that
$\chi\mapsto\phi_{\chi}$ is injective by 445H.   So we have one-to-one
correspondences between $\Cal X$ and its images in $L_{\Bbb C}^0$ and
$(L_{\Bbb C}^1)^*$.

Write $\frak T$ for the topology of $\Cal X$ as defined in 445Ab, $\frak
T_{m}$ for the topology induced by its identification with its image in
$L_{\Bbb C}^0$, and $\frak T_{w}$ for the topology induced by its
identification
with its image in $(L_{\Bbb C}^1)^*$.   Let $\Cal E$ be the family of
non-empty
totally bounded subsets of $X$ and $\Sigma^f$ the set of measurable sets
of finite measure;  for $E\in\Cal E$, $F\in\Sigma^f$ and $f\in\eusm
L_{\Bbb C}^1=\eusm L_{\Bbb C}^1(\mu)$ set

\Centerline{$\rho_E(\chi,\theta)=\sup_{x\in E}|\chi(x)-\theta(x)|$,}

\Centerline{$\rho'_F(\chi,\theta)
=\int_F\min(1,|\chi(x)-\theta(x)|)\mu(dx)$,}

\Centerline{$\rho''_f(\chi,\theta)
=|\int f(x)\chi(x)\mu(dx)-\int f(x)\theta(x)\mu(dx)|$}

\noindent for $\chi$, $\theta\in\Cal X$.   Then $\frak T$ is generated
by the pseudometrics $\{\rho_E:E\in\Cal E\}$, $\frak T_m$ is generated by
$\{\rho'_F:F\in\Sigma^f\}$ and $\frak T_w$ is generated by
$\{\rho''_f:f\in\eusm L_{\Bbb C}^1\}$.

\medskip

{\bf (b)} $\frak T_m\subseteq\frak T$.   \Prf\ Suppose that
$F\subseteq X$ is a measurable set of finite measure, and $\epsilon>0$.
There is a
non-empty totally bounded open set $U\subseteq X$ (443H).   Since
$\{xU:x\in X\}$ is an open cover of $X$ and $\mu$ is
$\tau$-additive, there are $y_0,\ldots,y_n\in X$ such that
$\mu(F\setminus\bigcup_{j\le n}y_jU)\le\bover13\epsilon$;  set
$E=\bigcup_{j\le n}y_jU$.   Then $E$ is totally bounded, and
$\rho'_F(\chi,\theta)\le\epsilon$ whenever
$\rho_E(\chi,\theta)\le\Bover{\epsilon}{1+3\mu E}$.   As $F$ and
$\epsilon$ are
arbitrary, the identity map $(\Cal X,\frak T)\to(\Cal X,\frak T_m)$ is
continuous (2A3H), that is, $\frak T_m\subseteq\frak T$.\ \Qed

\medskip

{\bf (c)} $\frak T_w\subseteq\frak T_m$.   \Prf\
If $f\in\eusm L_{\Bbb C}^1$ and $\epsilon>0$ let $F\in\Sigma^f$, $M>0$
be such that
$\int(|f|-M\chi F)^+d\mu\le\bover14\epsilon$.   If $\chi$,
$\theta\in\Cal X$ and $\rho'_F(\chi,\theta)\le\Bover{\epsilon}{4M}$,
then

$$\eqalign{\rho''_f(\chi,\theta)
&=\bigl|\int(\chi-\theta)\times f\bigr|
\le\int|\chi-\theta|\times|f|\cr
&\le 2\int(|f|-M\chi F)^++M\int_F|\chi-\theta|\cr
&\le\Bover12\epsilon+2M\int_F\min(1,|\chi-\theta|)
=\Bover12\epsilon+2M\rho'_F(\chi,\theta)
\le\epsilon.\cr}$$

\noindent As $f$ and $\epsilon$ are arbitrary, this shows that
$\frak T_w\subseteq\frak T_m$.\ \Qed

\medskip

{\bf (d)} Finally, $\frak T\subseteq\frak T_w$.   \Prf\ Fix
$\chi\in\Cal X$, $E\in\Cal E$ and $\epsilon>0$.   Let
$u\in L_{\Bbb C}^1$ be such
that $\phi_{\chi}(u)=1$, and represent $u$ as $f^{\ssbullet}$ where
$f\in\eusm L_{\Bbb C}^1(\mu)$.   Set
$U=\{a:a\in X,\,\|a\action_lu-u\|_1<\bover14\epsilon\}$;  then $U$ is an
open neighbourhood of the identity $e$ of $X$, because
$a\mapsto a\action_lu$ is continuous (443Gf).   Because $E$ is totally
bounded, there are $y_0,\ldots,y_n\in X$ such that
$E\subseteq\bigcup_{k\le n}y_kU$.
Set $f_k=y_k\action_lf$, so that $f_k^{\ssbullet}=y_k\action_lu$ for each $k\le n$.

Now suppose that $\theta\in\Cal X$ is such that

\Centerline{$\rho''_f(\theta,\chi)\le\Bover{\epsilon}4$,
\quad$\rho''_{f_k}(\theta,\chi)\le\Bover{\epsilon}4$ for every
$k\le n$.}

\noindent Take any $x\in E$.   Then there is a $k\le n$ such that
$x\in y_kU$, so that $y_k^{-1}x\in U$ and

\Centerline{$\|x\action_lu-y_k\action_lu\|_1
=\|y_k\action_l(y_k^{-1}x\action_lu-u)\|_1
=\|y_k^{-1}x\action_lu-u\|_1
\le\Bover{\epsilon}4$}

\noindent (using 443Ge for the second equality).   Now
$\phi_{\chi}(x\action_lu)=\chi(x)$ (445H), so

$$\eqalign{|\phi_{\theta}(x\action_lu)-\chi(x)|
&\le|\phi_{\theta}(x\action_lu-y_k\action_lu)|
   +|\phi_{\theta}(y_k\action_lu)-\phi_{\chi}(y_k\action_lu)|
   +|\phi_{\chi}(y_k\action_lu-x\action_lu)|\cr
&\le 2\|x\action_lu-y_k\action_lu\|_1+\rho''_{f_k}(\theta,\chi)
\le \Bover34\epsilon.\cr}$$

\noindent On the other hand,

\Centerline{$|\theta(x)-\phi_{\theta}(x\action_lu)|
=|\theta(x)||1-\phi_{\theta}(u)|
=\rho''_f(\theta,\chi)\le\Bover{\epsilon}4$.}

\noindent So $|\theta(x)-\chi(x)|\le\epsilon$.   As $x$ is arbitrary,
$\rho_E(\theta,\chi)\le\epsilon$.

As $\chi$, $E$ and $\epsilon$ are arbitrary, this shows that
$\frak T\subseteq\frak T_w$.\ \Qed
}%end of proof of 445I

\leader{445J}{Corollary} For any topological group $X$ carrying Haar
measures, its dual group $\Cal X$ is locally compact and Hausdorff.

\proof{ Let $\Phi$ be the set of non-zero multiplicative linear
functionals on $L_{\Bbb C}^1=L_{\Bbb C}^1(\mu)$, for some left Haar
measure $\mu$ on $X$,
and give $\Phi$ its weak* topology.   Then
$\Phi\cup\{0\}\subseteq(L_{\Bbb C}^1)^*$ is the set of all
multiplicative linear
functionals on $L_{\Bbb C}^1$, and is closed for the weak* topology,
because

\Centerline{$\{\phi:\phi\in(L_{\Bbb C}^1)^*,\,
\phi(u*v)=\phi(u)\phi(v)\}$}

\noindent is closed for all $u$, $v\in L_{\Bbb C}^1$.   Because the unit
ball of $(L_{\Bbb C}^1)^*$ includes $\Phi$ (4A6F again), and
is a compact Hausdorff space for the weak* topology (3A5F),
so is $\Phi\cup\{0\}$.   So $\Phi$ itself is an open subset of a compact
Hausdorff space and is a locally compact Hausdorff space (3A3Bg).
Since the topology on $\Cal X$ can be identified with the weak*
topology on $\Phi$ (445I), $\Cal X$ also is locally compact and
Hausdorff.
}%end of proof of 445J

\leader{445K}{Proposition} Let $X$ be a topological group and $\mu$ a
left Haar measure on $X$.   Let $\Cal X$ be the dual group of $X$, and
write $C_0=C_0(\Cal X;\Bbb C)$ for the Banach algebra of continuous
functions $h:\Cal X\to\Bbb C$ such that $\{\chi:|h(\chi)|\ge\epsilon\}$
is compact for every $\epsilon>0$.

(a) For any $u\in L_{\Bbb C}^1=L_{\Bbb C}^1(\mu)$, its Fourier transform
$\varhat{u}$ belongs to $C_0$.

(b) The map $u\mapsto\varhat{u}:L_{\Bbb C}^1\to C_0$ is a multiplicative
linear
operator, of norm at most $1$.

(c) Suppose that $X$ is abelian.   For
$f\in\eusm L_{\Bbb C}^1=\eusm L_{\Bbb C}^1(\mu)$,
set $\tilde f(x)=\overline{f(x^{-1})}$ whenever this is defined.   Then
$\tilde f\in\eusm L_{\Bbb C}^1$ and $\|\tilde f\|_1=\|f\|_1$.   For
$u\in L_{\Bbb C}^1$,
we may define $\tilde u\in L_{\Bbb C}^1$ by setting $\tilde u=\tilde
f^{\ssbullet}$ whenever $u=f^{\ssbullet}$.   Now $\varhat{\tilde u}$ is
the complex conjugate of $\varhat{u}$, so 
$(u*\tilde u)\varsphat=|\varhat{u}|^2$.

(d) Still supposing that $X$ is abelian,
$\{\varhat{u}:u\in L_{\Bbb C}^1\}$ is a
norm-dense subalgebra of $C_0$, and $\|\varhat{u}\|_{\infty}=r(u)$, the
spectral radius of $u$\cmmnt{ (4A6G)}, for every $u\in L_{\Bbb C}^1$.

\proof{{\bf (a)} As in 445H and 445J, let $\Phi$ be the set of non-zero
multiplicative linear functionals on $L_{\Bbb C}^1$, so that
$\Phi\cup\{0\}$ is
compact for the weak* topology of $(L_{\Bbb C}^1)^*$, and
$\varhat{u}(\chi)=\phi_{\chi}(u)$
for every $\chi\in\Cal X$.   By the definition of the weak* topology,
$\phi\mapsto\phi(u)$ is continuous;  since we can identify the weak*
topology on $\Phi$ with the dual group topology of $\Cal X$ (445I),
$\varhat{u}$ is continuous.   Also, for any $\epsilon>0$,

\Centerline{$\{\chi:\chi\in\Cal X,\,|\varhat{u}(\chi)|\ge\epsilon\}
\cong\{\phi:\phi\in\Phi,\,|\phi(u)|\ge\epsilon\}$,}

\noindent which is a closed subset of $\Phi\cup\{0\}$, therefore
compact.

\medskip

{\bf (b)} It is immediate from the definition of $\varsphat$ that it is
a linear operator from $L_{\Bbb C}^1$ to $\Bbb C^{\Cal X}$, and therefore
from $L_{\Bbb C}^1$ to $C_0$;   it is multiplicative
by 445G, and of norm at most $1$ because all the multiplicative linear
functionals $u\mapsto\varhat{u}(\chi)$ must be of norm at most $1$.

\woddheader{445K}{4}{2}{2}{30pt}

{\bf (c)} Now suppose that $X$ is abelian.   If
$f\in\eusm L_{\Bbb C}^1$, then

\Centerline{$\int\tilde f(x)dx
=\overline{\intop f(x^{-1})dx}
=\overline{\intop f(x)dx}$}

\noindent by 442Kb, so $\tilde f\in\eusm L_{\Bbb C}^1$;  the same
formulae tell us that $\|\tilde f\|_1=\|f\|_1$.   If $f\eae g$ then
$\tilde f\eae\tilde g$ (442G, or otherwise), so $\tilde u$ is
well-defined.   If $\chi\in\Cal X$, and $u=f^{\ssbullet}$, then

$$\eqalign{\varhat{\tilde u}(\chi)
&=\int\tilde f(x)\chi(x)dx
=\int\overline{f(x^{-1})}\chi(x)dx
=\int\overline{f(x)}\chi(x^{-1})dx\cr
&=\int\overline{f(x)\chi(x)}dx
=\overline{\int f(x)\chi(x)dx}
=\overline{\varhat{u}(\chi)},\cr}$$

\noindent so $\varhat{\tilde u}$ is the complex conjugate of
$\varhat{u}$, and

\Centerline{$(u*\tilde u)\varsphat
=\varhat{u}\times\varhat{\tilde u}=|\varhat{u}|^2$.}

\medskip

{\bf (d)} To see that $A=\{\varhat{u}:u\in L_{\Bbb C}^1\}$ is dense in
$C_0$, we
can use the Stone-Weierstrass theorem in the form 4A6B.   $A$ is a
subalgebra of $C_0$;  it separates the points (because the canonical map
from $\Cal X$ to $(L_{\Bbb C}^1)^*$ is injective);  if $\chi\in\Cal X$,
there is
an $h\in A$ such that $h(\chi)\ne 0$ (because elements of $\Cal X$ act
on $L_{\Bbb C}^1$ as non-zero functionals);  and the complex conjugate
of any function in $A$ belongs to $A$, by (c) above.

Accordingly $A$ is dense in $C_0$, by 4A6B.

The calculation of $\|\varhat{u}\|_{\infty}$ is an immediate consequence
of the characterization of $r(u)$ as
$\max\{|\phi(u)|:\phi\in\Phi\}$ (4A6K) and the identification of $\Phi$
with $\Cal X$.
}%end of proof of 445K

\cmmnt{\medskip

\noindent{\bf Remark} This is the first point in this section where we
really need to know whether or not our group is abelian.
}%end of comment

\leader{445L}{Positive definite functions} Let $X$ be a group.

\spheader 445La A function $h:X\to\Bbb C$ is called {\bf positive
definite} if

\Centerline{$\sum_{j,k=0}^n\zeta_j\bar\zeta_k h(x_k^{-1}x_j)\ge 0$}

\noindent for all $\zeta_0,\ldots,\zeta_n\in\Bbb C$ and
$x_0,\ldots,x_n\in X$.

\spheader 445Lb Suppose that $h:X\to\Bbb C$ is positive definite.
Then, writing $e$ for the identity of $X$,

\quad(i) $|h(x)|\le h(e)$ for every $x\in X$;

\quad(ii) $h(x^{-1})=\overline{h(x)}$ for every $x\in X$.

\noindent\prooflet{\Prf\ If $\zeta\in\Bbb C$ and $x\in X$, take $n=1$,
$x_0=e$, $x_1=x$, $\zeta_0=1$ and $\zeta_1=\zeta$ in the definition in
(a) above, and observe that

\Centerline{$(1+|\zeta|^2)h(e)+\zeta h(x)+\bar\zeta h(x^{-1})
=h(e^{-1}e)+\zeta h(e^{-1}x)+\bar\zeta h(x^{-1}e)
  +\zeta\bar\zeta h(x^{-1}x)
\ge 0$.}

\noindent Taking $\zeta=0$, $x=e$ we get $h(e)\ge 0$.   Taking $\zeta=1$
we see that $h(x)+h(x^{-1})$ is real, and taking $\zeta=i$, we see that
$h(x)-h(x^{-1})$ is purely imaginary;   that is,
$h(x^{-1})=\overline{h(x)}$, for any $x$.   Taking $\zeta$ such that
$|\zeta|=1$, $\zeta h(x)=-|h(x)|$ we get $2h(e)-2|h(x)|\ge 0$, that is,
$|h(x)|\le h(e)$ for every $x\in X$.\ \Qed}

\spheader 445Lc If $h:X\to\Bbb C$ is positive definite and
$\chi:X\to S^1$ is a homomorphism, then $h\times\chi$ is positive definite.   \prooflet{\Prf\
If $\zeta_0,\ldots,\zeta_n\in\Bbb C$ and $x_0,\ldots,x_n\in X$ then

\Centerline{$\sum_{j,k=0}^n\zeta_j\bar\zeta_k(h\times\chi)(x_k^{-1}x_j)
=\sum_{j,k=0}^n\zeta_j\chi(x_j)\overline{\zeta_k\chi(x_k)}
  h(x_k^{-1}x_j)\ge 0$.  \Qed}
}%end of prooflet

\spheader 445Ld If $X$ is an abelian topological group and $\mu$ a Haar
measure on $X$, then for any $f\in\eusm L_{\Bbb C}^2(\mu)$ the
convolution
$f*\tilde f:X\to\Bbb C$ is continuous and positive definite, where
$\tilde f(x)=\overline{f(x^{-1})}$ whenever this is defined.
\prooflet{\Prf\ As in 444Rc, $f*\tilde f$ is defined everywhere on
$X$ and is continuous.   (The definition of $\ssptilde$ has shifted since
\S444, but the argument there applies unchanged to the present
situation.)   Now, if $x_0,\ldots,x_n\in X$ and
$\zeta_0,\ldots,\zeta_n\in\Bbb C$,

$$\eqalign{\sum_{j,k=0}^n\zeta_j\bar\zeta_k(f*\tilde f)(x_k^{-1}x_j)
&=\sum_{j,k=0}^n\zeta_j\bar\zeta_k\int
  f(y)\tilde f(y^{-1}x_k^{-1}x_j)dy\cr
&=\int\sum_{j,k=0}^n\zeta_j\bar\zeta_k
  f(y)\overline{f(x_j^{-1}x_ky)}dy\cr
&=\int\sum_{j,k=0}^n\zeta_j\bar\zeta_k
  f(x_jy)\overline{f(x_j^{-1}x_kx_jy)}dy\cr
&=\int\sum_{j,k=0}^n\zeta_j\bar\zeta_k
  f(x_jy)\overline{f(x_ky)}dy\cr
&=\int|\sum_{j=0}^n\zeta_j f(x_jy)|^2dy
\ge 0.\cr}$$

\noindent So $f*\tilde f$ is positive definite.\ \Qed
}%end of prooflet

\leader{445M}{Proposition} Let $X$ be a topological group and $\nu$ a
quasi-Radon measure on $X$.   If $h:X\to\Bbb C$ is a continuous positive
definite function, then
$\iint h(y^{-1}x)f(x)\overline{f(y)}\nu(dx)\nu(dy)$\vthsp$\ge 0$ for
every $\nu$-integrable function $f$.

\proof{{\bf (a)} Extend $f$, if necessary, to the whole of $X$;  since
the hypothesis implies that $\dom f$ is conegligible, this does not
affect the integrals.  Let $\lambda$ be the product quasi-Radon measure
on $X\times X$;  because $h$ is continuous (by hypothesis) and bounded
(by 445L(b-i)), the function $(x,y)\mapsto
h(y^{-1}x)f(x)\overline{f(y)}$ is
$\lambda$-integrable, and (because $\{x:f(x)\ne 0\}$ can be covered by a
sequence of sets of finite measure)

\Centerline{$I=\iint h(y^{-1}x)f(x)\overline{f(y)}\nu(dx)\nu(dy)
=\int h(y^{-1}x)f(x)\overline{f(y)}\lambda(d(x,y))$}

\noindent (417H).

\medskip

{\bf (b)} Let $\epsilon>0$.   Set $\gamma=\sup_{x\in X}|h(x)|=h(e)$
(445L(b-i)).   Let
$F\subseteq X$ be a non-empty measurable set of finite measure for $\nu$
such that $\gamma\int_{(X\times X)\setminus(F\times
F)}|f(x)\overline{f(y)}|\lambda(d(x,y))\le\bover12\epsilon$ and $f$ is
bounded on $F$;  say $|f(x)|\le M$ for every $x\in F$.  Let $\delta>0$
be such that

\Centerline{$\delta(M^2+2M\gamma)(\nu F)^2+2M^2\gamma\delta
\le\bover12\epsilon$.}

Let $\Cal G$ be the set

$$\eqalign{\{G\times H:G,\,H\subseteq X\text{ are open},\,
&|h(y^{-1}x)-h(y_1^{-1}x_1)|\le\delta\cr
&\text{ whenever }x,\,x_1\in G,\,y,\,y_1\in H\}.\cr}$$

\noindent Because $h$ is continuous, $\Cal G$ is a cover of $X\times X$.
Because $\lambda$ is $\tau$-additive, there is a finite set $\Cal
G_0\subseteq\Cal G$ such that
$\lambda((F\times F)\setminus\bigcup\Cal G_0)\le\delta$;  we may suppose
that $\Cal G_0$ is non-empty.   Set
$W=(F\times F)\cap\bigcup\Cal G_0$.   Enumerate $\Cal G_0$ as
$\langle G_i\times H_i\rangle_{i\le n}$.

Let $\Cal F$ be a finite partition of $F$ into measurable sets such that
$|f(x)-f(x')|\le\delta$ whenever $x$, $x'$ belong to the same member of
$\Cal F$.   Let $\Cal E$ be the partition of $F$ generated by
$\Cal F\cup\{F\cap G_j:j\le n\}\cup\{F\cap H_j:j\le n\}$.   Enumerate
$\Cal E$ as $\langle E_k\rangle_{k\le m}$;  for each $k\le m$ choose
$x_k\in E_k$.   Set
$J=\{(j,k):j\le m,\,k\le m,\,E_j\times E_k\subseteq W\}$;  then
$W=\bigcup_{(j,k)\in J}E_j\times E_k$.

\medskip

{\bf (c)} If $(j,k)\in J$, $x\in E_j$ and $y\in E_k$ then

\Centerline{$|h(y^{-1}x)f(x)\overline{f(y)}
   -h(x_k^{-1}x_j)f(x_j)\overline{f(x_k)}|
\le\delta(M^2+2M\gamma)$,}

\noindent because there must be some $r\le n$ such that
$E_j\times E_k\subseteq G_r\times H_r$, so that
$|h(y^{-1}x)-h(x_k^{-1}x_j)|\le\delta$, while there are members of
$\Cal F$ including $E_j$ and $E_k$, so that $|f(x)-f(x_j)|\le\delta$ and
$|\overline{f(y)}-\overline{f(x_k)}|\le\delta$;  at the same time,

\Centerline{$|f(x)\overline{f(y)}|\le M^2$,
\quad$|h(x_k^{-1}x_j)||\overline{f(y)}|\le M\gamma$,
\quad$|h(x_k^{-1}x_j)||f(x_j)|\le M\gamma$}

\noindent because $x$, $y$, $x_j$ and $x_k$ all belong to $F$.

\woddheader{445M}{4}{2}{2}{72pt}

{\bf (d)} Set $\zeta_j=f(x_j)\nu E_j$ for $j\le m$, so that
$\bar\zeta_j=\overline{f(x_j)}\nu E_j$.   Now consider

$$\eqalign{\bigl|\int_W h(&y^{-1}x)f(x)\overline{f(y)}\lambda(d(x,y))
  -\sum_{(j,k)\in J}\zeta_j\bar\zeta_kh(x_k^{-1}x_j)\bigr|\cr
&=\bigl|\sum_{(j,k)\in J}\int_{E_j\times E_k}
      h(y^{-1}x)f(x)\overline{f(y)}
  -h(x_k^{-1}x_j)f(x_j)\overline{f(x_k)}\lambda(d(x,y))\bigr|\cr
&\le\sum_{(j,k)\in J}\int_{E_j\times E_k}|h(y^{-1}x)f(x)\overline{f(y)}
  -h(x_k^{-1}x_j)f(x_j)\overline{f(x_k)}|\lambda(d(x,y))\cr
&\le\sum_{(j,k)\in J}\delta(M^2+2M\gamma)\nu E_j\nu E_k\cr
&\le\delta(M^2+2M\gamma)\lambda W
\le\delta(M^2+2M\gamma)(\nu F)^2.\cr}$$

\noindent On the other hand,

$$\eqalign{\bigl|\int_{(X\times X)\setminus W}
   h(y^{-1}x)f(x)\overline{f(y)}\lambda(d(x,y))\bigr|
&\le\gamma\int_{(X\times X)\setminus(F\times
F)}|f(x)\overline{f(y)}|\lambda(d(x,y))\cr
&\qquad\qquad+\gamma\int_{(F\times F)\setminus W}|f(x)
     \overline{f(y)}|\lambda(d(x,y))\cr
&\le\Bover12\epsilon+\gamma\delta M^2,\cr}$$

\noindent and

$$\eqalign{\bigl|\sum_{j\le m,k\le m,(j,k)\notin J}
    h(x_k^{-1}x_j)f(x_j)\tilde f(x_k)\bigr|
&\le\gamma M^2\sum_{j\le m,k\le m,(j,k)\notin J}\nu E_j\nu E_k\cr
&=\gamma M^2\lambda((F\times F)\setminus W)
\le\gamma M^2\delta.\cr}$$

\noindent Putting these together,

$$\eqalign{|I-\sum_{j,k=0}^m\zeta_j\bar\zeta_kh(x_k^{-1}x_j)|
&\le\delta(M^2+2M\gamma)(\nu F)^2
+\Bover12\epsilon+\gamma\delta M^2
+\gamma M^2\delta
\le\epsilon.\cr}$$

\noindent But $\sum_{j,k=0}^m\zeta_j\bar\zeta_kh(x_k^{-1}x_j)\ge 0$,
because $h$ is positive definite.   As $\epsilon$ is arbitrary,
$I\ge 0$, as required.
}%end of proof of 445M

\leader{445N}{Bochner's theorem}\cmmnt{ ({\smc Herglotz 1911},
{\smc Bochner 33}, {\smc Weil 40})} Let $X$ be an abelian topological group
with a Haar measure $\mu$, and $\Cal X$ its dual group.   Then for
any continuous positive definite function $h:X\to\Bbb C$ there is a
unique totally finite Radon measure $\nu$ on $\Cal X$ such that

\Centerline{$\int h\times f\,d\mu=\int\varhatf\,d\nu$
for every $f\in\eusm L_{\Bbb C}^1=\eusm L_{\Bbb C}^1(\mu)$,}

\Centerline{$h(x)=\int\chi(x)\nu(d\chi)$ for every $x\in X$.}

\proof{{\bf (a)} If $h(e)=0$, where $e$ is the identity in $X$, then
$h=0$, by 445L(b-i), and we can take $\nu$ to be the zero measure.
Otherwise, since
multiplying $h$ by a positive scalar leaves $h$ positive definite and
does not affect the result, we may suppose that $h(e)=1$.   For $f$,
$g\in\eusm L_{\Bbb C}^1=\eusm L_{\Bbb C}^1(\mu)$ set

\Centerline{$(f|g)
=\biggeriint f(x)\overline{g(y)}h(y^{-1}x)dxdy
=\iint f(x)\overline{g(y^{-1})}h(yx)dxdy$}

\noindent (by 442Kb, since $X$ is unimodular).   Then, by 445M,
$(f|f)\ge 0$ for every $f\in\eusm L_{\Bbb C}^1$.   Also
$(f_1+f_2|g)=(f_1|g)+(f_2|g)$, $(\zeta f|g)=\zeta(f|g)$ and
$(g|f)=\overline{(f|g)}$
for all $f$, $g$, $f_1$, $f_2\in\eusm L_{\Bbb C}^1$ and
$\zeta\in\Bbb C$.  \Prf\
Only the last is anything but trivial, and for this we have

$$\eqalignno{(g|f)
&=\iint g(x)\overline{f(y)}h(y^{-1}x)dxdy\cr
&=\iint g(x)\overline{f(y)}h(y^{-1}x)dydx\cr
\noalign{\noindent (by 417Ha, because
$(x,y)\mapsto g(x)\overline{f(y)}h(y^{-1}x)$ is integrable for the
product measure and
zero off the square of a countable union of sets of finite measure)}
&=\iint g(x)\overline{f(y)}\overline{h(x^{-1}y)}dydx\cr
\noalign{\noindent (using 445L(b-ii))}
&=\overline{\iint f(y)\overline{g(x)}
    h(x^{-1}y)dydx}\cr
&=\overline{(f|g)}.\text{ \Qed}\cr}$$

\medskip

{\bf (b)} If $f$, $g\in\eusm L_{\Bbb C}^1$, $|(f|g)|^2\le(f|f)(g|g)$.
\Prf\
(Really this is just Cauchy's inequality.)   For
any $\alpha$, $\beta\in\Bbb C$,

\Centerline{$|\alpha|^2(f|f)+2\Real(\alpha\bar\beta(f|g))+|\beta|^2(g|g)
=(\alpha f+\beta g|\alpha f+\beta g)
\ge 0$.}

\noindent If $(f|f)=0$ we have $2\Real(\alpha(f|g))+(g|g)\ge 0$ for
every $\alpha\in\Bbb C$ so in this case $(f|g)=0$;  similarly
$(f|g)=0$ if $(g|g)=0$;  otherwise we can find non-zero $\alpha$,
$\beta$ such that $|\alpha|^2=(g|g)$, $|\beta|^2=(f|f)$ and
$\alpha\bar\beta(f|g)=-|\alpha\beta(f|g)|$, in which case the inequality
simplifies
to $|(f|g)|\le|\alpha\beta|$ and $|(f|g)|^2\le(f|f)(g|g)$, as
required.\ \Qed

\medskip

{\bf (c)} Now consider the functional
$\psi\in(L_{\Bbb C}^1)^*=(L_{\Bbb C}^1(\mu))^*$
corresponding to $h$, so that $\psi(f^{\ssbullet})=\int h\times f\,d\mu$
for every $f\in\eusm L_{\Bbb C}^1$.   Then
$|\psi(f^{\ssbullet})|^2\le(f|f)$ for
every $f\in\eusm L_{\Bbb C}^1$.   \Prf\ Let $\epsilon>0$.   Then there
is an open neighbourhood $U$ of $e$ such that $U=U^{-1}$ and

\Centerline{$|h(y^{-1}x)-h(e)|\le\epsilon$ whenever $x$, $y\in U$,}

\Centerline{$\|a\action_lf-f\|_1\le\epsilon$ for every $a\in U$}

\noindent where $(a\action_lf)(x)=f(a^{-1}x)$ whenever this is defined, as
usual (443Gf).   Shrinking $U$ if need be, we may suppose that
$\mu U<\infty$, and of course $\mu U>0$.   Set
$g=\Bover1{\mu U}\chi U\in\eusm L_{\Bbb C}^1$.   Then

$$\eqalign{|(f|g)-\psi(f^{\ssbullet})|
&=\Bover1{\mu U}\bigl|\int_U\int_Xf(x)h(y^{-1}x)dxdy
  -\int_U\int_Xf(x)h(x)dxdy\bigr|\cr
&=\Bover1{\mu U}\bigl|\int_U\int_Xf(yx)h(x)dxdy
  -\int_U\int_Xf(x)h(x)dxdy\bigr|\cr
&=\Bover1{\mu U}\bigl|\int_U\int_X
  ((y^{-1}\action_lf)(x)-f(x))h(x)dxdy\bigr|\cr
&\le\Bover1{\mu U}\int_U\int_X
  |(y^{-1}\action_lf)(x)-f(x)||h(x)|dxdy\cr
&\le\Bover1{\mu U}\int_U\|y^{-1}\action_lf-f\|_1\mu(dy)
\le\epsilon.\cr}$$

\noindent Also

$$\eqalign{|(g|g)-1|
&=\bigl|\Bover1{\mu U^2}\int_U\int_U
   (h(y^{-1}x)-1)dxdy\bigr|\cr
&\le\Bover1{\mu U^2}\int_U\int_U|h(y^{-1}x)-1|dxdy
\le\epsilon.\cr}$$

\noindent So

\Centerline{$\max(0,|\psi(f^{\ssbullet})|-\epsilon)^2
\le|(f|g)|^2
\le(f|f)(g|g)
\le(1+\epsilon)(f|f)$.}

\noindent Letting $\epsilon\downarrow 0$ we have the result.\ \Qed

\woddheader{445N}{4}{2}{2}{40pt}

{\bf (d)} If we look at $(f|f)$, however, and apply 444Od, we see that

$$\eqalign{(f|f)
&=\iint f(x)\overline{f(y)}h(y^{-1}x)dxdy\cr
&=\iint f(x)\overline{f(y^{-1})}h(yx)dxdy
=\int h(x)(f*\tilde f)(x)dx,\cr}$$

\noindent where $\tilde f(x)=\overline{f(x^{-1})}$ whenever this is
defined;  that is,

\Centerline{$(f|f)=\psi(f*\tilde f)^{\ssbullet}$.}

\noindent (Note that $\tilde f\in\eusm L_{\Bbb C}^1$ because $X$ is
unimodular,
as in part (c) of the proof of 445K.)    So (c) tells us that

\Centerline{$|\psi(f^{\ssbullet})|^2\le\psi(f*\tilde f)^{\ssbullet}$}

\noindent for every $f\in\eusm L_{\Bbb C}^1$, that is,
$|\psi(u)|^2\le\psi(u*\tilde u)$ for every $u\in L_{\Bbb C}^1$, defining
$\tilde
u$ as in 445Kc.

\medskip

{\bf (e)} In fact

\Centerline{$|\psi(u)|\le\|\varhat{u}\|_{\infty}$}

\noindent for every $u\in L_{\Bbb C}^1$.  \Prf\ Set $u_0=u$ and
$u_{k+1}=u_k*\tilde u_k$ for every $k\in\Bbb N$.   We need to know that
$u_k=\tilde u_k$ for $k\ge 1$.   To see this, represent $u_{k-1}$ as
$f^{\ssbullet}$ where $f\in\eusm L_{\Bbb C}^1$, so that $u_k=(f*\tilde
f)^{\ssbullet}$.   Now

$$\eqalign{(f*\tilde f)\ssptilde(x)
&=\overline{(f*\tilde f)(x^{-1})}
=\overline{\int f(x^{-1}y)\tilde f(y^{-1})dy}\cr
&=\int\overline{f(x^{-1}y)}f(y)dy
=\int\tilde f(y^{-1}x)f(y)dy
=(f*\tilde f)(x)\cr}$$

\noindent for every $x$, so $(f*\tilde f)\ssptilde=f*\tilde f$ and
$\tilde u_k=u_k$.   Accordingly $u_{k+1}=u_k*u_k$ for $k\ge 1$ and we
have $u_k=(u_1)^{2^{k-1}}$ for every $k\ge 1$.

At the same time, we have $|\psi(u_k)|^2\le\psi(u_{k+1})$ for every $k$,
by (d), so that, for $k\ge 1$,

\Centerline{$|\psi(u)|^{2^k}\le\psi(u_k)\le\|u_k\|_1
=\|u_1^{2^{k-1}}\|_1$,}

\Centerline{$|\psi(u)|\le\|u_1^{2^{k-1}}\|_1^{1/2^k}$.}

\noindent Letting $k\to\infty$, $|\psi(u)|\le\sqrt{r(u_1)}$, where
$r(u_1)$ is the spectral radius of $u_1$.

At this point, recall that $r(u_1)=\|\varhat{u}_1\|_{\infty}$ (445Kd),
while $|\varhat{u}|^2=\varhat{u}_1$ (445Kc), so
$r(u_1)=\|\varhat{u}\|_{\infty}^2$ and
$|\psi(u)|\le\|\varhat{u}\|_{\infty}$.\ \Qed

\medskip

{\bf (f)} Now consider $\varsphat$ as a linear operator from
$L_{\Bbb C}^1$ to
$C_0=C_0(\Cal X;\Bbb C)$, as in 445K.   If $\varhat{u}=0$ then
$\psi(u)=0$, by (e), so setting $A=\{\varhat{u}:u\in L_{\Bbb C}^1\}$ we
have a
linear functional $\psi_0:A\to\Bbb C$ defined by saying that
$\psi_0(\varhat{u})=\psi(u)$ for every $u\in L_{\Bbb C}^1$.   By (e),
$\|\psi_0\|\le 1$.   Now
$\psi_0$ has an extension to a bounded linear operator $\psi_1$,
still of norm at most $1$, from $C_0$ to $\Bbb C$ (3A5Ab).

\medskip

{\bf (g)} Suppose that $q\in C_0$ and $0\le q\le\tbf{1}$,
writing $\tbf{1}$
for the constant function with value $1$;  set $\alpha=\psi_1(q)$.
Then for any $\zeta\in\Bbb C$ and $\gamma\ge 0$ we have
$|\zeta-\gamma\alpha|\le\max(|\zeta|,|\gamma|,|\zeta-\gamma|)$.   \Prf\
Let $\epsilon>0$.   Set $V=\{x:|1-h(x)|<\epsilon\}$;  then $V$ is an
open neighbourhood of $e$;  set $f=\bover1{\mu V}\chi V$ and
$u=f^{\ssbullet}$, so that

\Centerline{$\|\varhat{u}\|_{\infty}=r(u)\le\|u\|_1=1$,}

\Centerline{$|1-\psi(u)|=|\Bover1{\mu V}\int_V(1-h(x))dx|
\le\epsilon$.}

\noindent Set $v=u*\tilde u$;  then

\Centerline{$\psi_1(\varhat{v})=\psi(v)
\ge|\psi(u)|^2\ge(1-\epsilon)^2$,}

\noindent using part (d) for the central inequality.
But $\varhat{v}=|\varhat{u}|^2$, so that $0\le\varhat{v}\le\tbf{1}$ and
$\psi_1(\varhat{v})\le 1$.

Now consider $\|\zeta\varhat{v}-\gamma q\|_{\infty}$.   If $\chi\in\Cal
X$, then $\zeta\varhat{v}(\chi)$ and $\gamma q(\chi)$ both lie in the
triangle with vertices $0$, $\zeta$ and $\gamma$, because
$0\le\varhat{v}\le\tbf{1}$ and $0\le q\le\tbf{1}$.   So

\Centerline{$|\zeta\varhat{v}(\chi)-\gamma q(\chi)|
\le\max(|\gamma|,|\zeta|,|\gamma-\zeta|)$.}

\noindent As $\chi$ is arbitrary,

\Centerline{$\|\zeta\varhat{v}-\gamma q\|_{\infty}
\le\max(|\gamma|,|\zeta|,|\gamma-\zeta|)$.}

\noindent Accordingly

$$\eqalign{|\zeta-\gamma\alpha|
&\le|\zeta-\zeta\psi_1(\varhat{v})|
  +|\psi_1(\zeta\varhat{v}-\gamma q)|\cr
&\le|\zeta|(1-(1-\epsilon)^2)+\|\zeta\varhat{v}-\gamma q\|_{\infty}
\le 2\epsilon|\zeta|+\max(|\zeta|,|\gamma|,|\zeta-\gamma|).\cr}$$

\noindent As $\epsilon$ is arbitrary, we have the result.\ \Qed

Taking $\zeta=\gamma=1$ we see that
$|1-\alpha|\le 1$, so that $\Real\alpha\ge 0$.   Taking $\zeta=\pm i$,
we see that $|i\pm\gamma\alpha|\le\sqrt{1+\gamma^2}$ for every
$\gamma\ge 0$, so that $\Imag\alpha=0$.   Thus $\psi_1(q)\ge 0$;  and
this is true whenever $0\le q\le\tbf{1}$ in $C_0$.

\medskip

{\bf (h)} It follows at once that $\psi_1(q)\ge 0$ whenever $q\ge 0$
in $C_0$.   Applying the Riesz Representation Theorem, in the form
436K, to the restriction of $\psi_1$ to $C_0(\Cal X;\Bbb R)$, we
see that there is a totally finite Radon measure $\nu$ on
$\Cal X$ such that $\psi_1(q)=\int q\,d\nu$ for every real-valued
$q\in C_0$;
of course it follows that $\psi_1(q)=\int q\,d\nu$ for every $q\in C_0$.
Unwrapping the definition of $\psi_1$, we see that

\Centerline{$\int h(x)f(x)dx
=\psi(f^{\ssbullet})=\psi_1(\varhatf)=\int\varhatf\,d\nu$}

\noindent for every $f\in\eusm L_{\Bbb C}^1(\mu)$.

\medskip

{\bf (i)} For the second formula, argue as follows.   Given
$f\in\eusm L_{\Bbb C}^1(\mu)$, consider the function
$(x,\chi)\mapsto f(x)\chi(x):X\times\Cal X\to\Bbb C$.   Because
$(\chi,x)\mapsto\chi(x)$ is
continuous (445Ea), this is $\Lambda$-measurable, where
$\Lambda$ is the domain of the product quasi-Radon
measure $\mu\times\nu$ on $X\times\Cal X$.   It is integrable because
$\nu\Cal X<\infty$ and $|\chi(x)|=1$ for every $\chi$, $x$;  moreover,
it is zero off the set $\{x:f(x)\ne 0\}\times\Cal X$, which is a
countable
union of products of sets of finite measure.   Note also that because
$\chi\mapsto\chi(x)$ is continuous and bounded for every $x\in X$,
$h_1(x)=\int\chi(x)\nu(d\chi)$ is defined, and $|h_1(x)|\le\nu X$,  for
every $x\in X$.   What is more, $h_1$ is continuous.   \Prf\ Let
$\frak X$ be the dual group of $\Cal X$, and for $x\in X$ let $\hat x$
be the corresponding member of $\frak X$.   Then, in the language of
445C, applied to the topological group $\Cal X$,

\Centerline{$h_1(x)=\int\hat x\,d\nu=\varhat{\nu}(\hat x)$}

\noindent for every $x\in X$.   But $\varhat{\nu}:\frak X\to\Bbb C$ is
continuous, by 445Ec, and $x\mapsto\hat x:X\to\frak X$ is continuous, by
445Eb;  so $h_1$ also is continuous.\ \Qed\

We may therefore apply Fubini's theorem (417H) to see that

$$\eqalign{\int f(x)h_1(x)\mu(dx)
&=\iint f(x)\chi(x)\nu(d\chi)\mu(dx)
=\iint f(x)\chi(x)\mu(dx)\nu(d\chi)\cr
&=\int\varhatf(\chi)\nu(d\chi)
=\int f(x)h(x)\mu(dx).\cr}$$

\noindent Since this is true for every $f\in\eusm L_{\Bbb C}^1$,
$h_1\eae h$;   since both are continuous, $h_1=h$, as required.

\medskip

{\bf (j)} Finally, to see that $\nu$ is uniquely defined, note that
$\{\varhat{f}:f\in\eusm L^1_{\Bbb C}\}$ is $\|\,\|_{\infty}$-dense in $C_0$
(445Kd), so 436K tells us that there can be at most one totally finite
Radon measure $\nu$ on $\Cal X$ such that
$\int h\times f\,d\mu=\int\varhatf\,d\nu$ for every
$f\in\eusm L^1_{\Bbb C}$.
}%end of proof of 445N

\leader{445O}{Proposition} Let $X$ be a Hausdorff abelian topological
group carrying Haar measures.   Then the map $x\mapsto\hat x$ from $X$
to its bidual group $\frak X$ is a homeomorphism between $X$ and its
image in $\frak X$.   In particular, the dual group $\Cal X$ of $X$
separates the points of $X$.

\proof{ We already know that \hbox{$\sphat\,$} is continuous (445Eb) and
that $\Cal X$ is locally compact and Hausdorff (445J).
Now let $U$ be any neighbourhood of the identity $e$ of $X$.   Let
$V\subseteq U$ be an open neighbourhood of $e$ such that
$VV^{-1}\subseteq U$ and
$\mu V<\infty$.   Then $f=\chi V\in\eusm L_{\Bbb C}^2(\mu)$, so
$f*\tilde f$ is positive definite and continuous (445Ld) and there is a
totally finite Radon measure
$\nu$ on $\Cal X$ such that $(f*\tilde f)(x)=\int\chi(x)\nu(d\chi)$ for
every $x\in X$ (445N).   Note that, writing $\frak e$ for the identity
of $\frak X$ and $\varhat{\nu}:\frak X\to\Bbb C$ for the Fourier-Stieltjes
transform of $\nu$,

$$\eqalign{\varhat{\nu}(\frak e)
&=\int\frak e(\chi)\nu(d\chi)
=\int\chi(e)\nu(d\chi)
=(f*\tilde f)(e)\cr
&=\int f(y)\tilde f(y^{-1})\mu(dy)
=\int|f(y)|^2\mu(dy)
\ne 0.\cr}$$

\noindent Now $\varhat{\nu}$ is continuous (445Ec), so
$W=\{\frak x:\varhat{\nu}(\frak x)\ne 0\}$ is a neighbourhood of
$\frak e$.   If $x\in X$ and $\hat x\in W$, then

\Centerline{$(f*\tilde f)(x)=\int\chi(x)\nu(d\chi)
=\int\hat x(\chi)\nu(d\chi)=\varhat{\nu}(\hat x)\ne 0$,}

\noindent so there is some $y\in X$ such that
$f(y)\tilde f(y^{-1}x)\ne 0$, that is, $f(y)\ne 0$ and
$f(x^{-1}y)\ne 0$, that is, $y$ and
$x^{-1}y$ both belong to $V$;  in which case $x\in VV^{-1}\subseteq U$.

Thus $U\supseteq\{x:\hat x\in W\}$.   This means that, writing $\frak S$
for $\{\{x:\hat x\in H\}:H\subseteq\frak X$ is open$\}$, every
neighbourhood of $e$ for the original topology $\frak T$ of $X$ is a
neighbourhood of $e$ for $\frak S$.   But (it is easy to check)
$(X,\frak S)$ is a topological group because $\frak X$ is a topological
group and \hbox{$\sphat\,$} is a homomorphism.   So
$\frak T\subseteq\frak S$ (4A5Fb).    As we know already that
$\frak S\subseteq\frak T$, the two topologies are equal.

It follows at once that if $\frak T$ is Hausdorff, then (because
$\frak S$ is Hausdorff) the map \hbox{$\sphat\,$} is an injection and is a
homeomorphism between $X$ and its image in $\frak X$.
}%end of proof of 445O

\vleader{60pt}{445P}{The Inversion Theorem} 
Let $X$ be an abelian topological
group and $\mu$ a Haar measure on $X$.   Then there is a unique Haar
measure $\lambda$ on the dual group $\Cal X$ of $X$ such that whenever
$f:X\to\Bbb C$ is continuous, $\mu$-integrable and positive definite,
then $\varhatf:\Cal X\to\Bbb C$ is $\lambda$-integrable and

\Centerline{$f(x)
=\int\varhatf(\chi)\overline{\chi(x)}\lambda(d\chi)$}

\noindent for every $x\in X$.

\proof{{\bf (a)} Write $P$ for the set of $\mu$-integrable positive
definite continuous functions $h:X\to\Bbb C$.   For $h\in P$, let
$\nu_h$ be the corresponding totally finite Radon measure on $\Cal X$
defined in 445N, so that

\Centerline{$\int f\times h\,d\mu=\int\varhatf\,d\nu_h$}

\noindent for every $f\in\eusm L_{\Bbb C}^1=\eusm L_{\Bbb C}^1(\mu)$.

\medskip

{\bf (b)} The basis of the argument is the following fact.   If
$f\in\eusm L_{\Bbb C}^1$ and $h_1$, $h_2\in P$, then

\Centerline{$\int\varhatf\times\varhat{\bar h}_2d\nu_{h_1}
=\int\varhatf\times\varhat{\bar h}_1d\nu_{h_2}$.}


$$\eqalignno{\text{\Prf }\int\varhatf\times\varhat{\bar h}_1d\nu_{h_2}
&=\int(f*\bar h_1)\varsphat d\nu_{h_2}\cr
\noalign{\noindent (445G)}
&=\int h_2(x)(f*\bar h_1)(x)\mu(dx)
=\int \bar h_2(x^{-1})(f*\bar h_1)(x)\mu(dx)\cr
\noalign{\noindent (by 445Lb)}
&=((f*\bar h_1)*\bar h_2)(e)
=((f*\bar h_2)*\bar h_1)(e)\cr
\noalign{\noindent (because $*$ is associative and commutative, by 444Oe
and 444Og)}
&=\int\varhatf\times\varhat{\bar h}_2d\nu_{h_1}.\text{ \Qed}\cr}$$

Now because $\varhat{\bar h}_1$ and $\varhat{\bar h}_2$ are both bounded
(by $\int|h_1|d\mu$ and $\int|h_2|d\mu$ respectively), and $\nu_{h_1}$
and $\nu_{h_2}$ are both totally finite measures, and
$\{\varhatf:f\in\eusm L_{\Bbb C}^1(\mu)\}$ is $\|\,\,\|_{\infty}$-dense
in
$C_0=C_0(\Cal X;\Bbb C)$ (445Kd), we must have

\Centerline{$\int p\times\varhat{\bar h}_2d\nu_{h_1}=\int
p\times\varhat{\bar h}_1d\nu_{h_2}$}

\noindent for every $p\in C_0$.

\wheader{445P}{6}{2}{2}{48pt}

{\bf (c)} Let $\Cal K$ be the family of compact subsets of $\Cal X$.
For $K\in\Cal K$ set

\Centerline{$P_K=\{h:h\in P,\,\varhat{\bar h}(\chi)>0$ for every
$\chi\in K\}$.}

\noindent Then $P_K$ is non-empty.   \Prf\ Set

\Centerline{$U=\{x:x\in X$, $|1-\chi(x)|\le\bover12$ for every $\chi\in
K\}$.}

\noindent Then $U$ is a neighbourhood of the identity $e$ of $X$, by
445Eb.   Let $V$ be an open neighbourhood of $e$, of finite measure,
such that $VV^{-1}\subseteq U$, set $g=\bover1{\mu V}\chi V$, and try
$h=g*\tilde g$.   Then $h$ is continuous and positive definite (445Ld),
real-valued and non-negative (because $g$ and $\tilde g$ are),
zero outside $U$ (because
$VV^{-1}\subseteq U$, as in the proof of 445O),  and

\Centerline{$\int h\,d\mu=\int g\,d\mu\cdot\int\tilde g\,d\mu=1$}

\noindent (444Qb).   Next,

\Centerline{$\varhat{\bar h}
=\varhat{h}=|\varhat{g}|^2$}

\noindent (445Kc) is non-negative, and if $\chi\in K$ then

\Centerline{$|1-\varhat{h}(\chi)|
=|\int h(x)-h(x)\chi(x)\mu(dx)|
\le\int h(x)|1-\chi(x)|\mu(dx)
\le\bover12$}

\noindent because $|1-\chi(x)|\le\bover12$ if $x\in U$ and $h(x)=0$ if
$x\in X\setminus U$.   So

\Centerline{$\varhat{\bar h}(\chi)=\varhat{h}(\chi)\ge\bover12$}

\noindent for every $\chi\in K$, and $h\in P_K$.\ \Qed

\medskip

{\bf (d)} Because $\Cal K$ is upwards-directed, $\{P_K:K\in\Cal K\}$ is
downwards-directed and generates a filter $\Cal F$ on $P$.   Let
$C_k=C_k(\Cal X;\Bbb C)$ be the space of continuous complex-valued
functions on $\Cal X$ with compact support.   If $q\in C_k$, then

\Centerline{$\phi(q)
=\lim_{h\to\Cal F}\int\bover{q}{\varhat{\bar h}}\,d\nu_h$}

\noindent is defined in $\Bbb C$, where in the division
$q/\varhat{\bar h}$ we interpret $0/0$ as $0$ if necessary.   \Prf\
Setting
$K=\overline{\{\chi:q(\chi)\ne 0\}}$, we see in fact that for any $h_1$,
$h_2\in P_K$ we may define a function $p\in C_k$ by setting

$$\eqalign{p(\chi)
&=\bover{q(\chi)}{\varhat{\bar h}_1(\chi)\varhat{\bar h}_2(\chi)}
   \text{ if }\chi\in K,\cr
&=0\text{ if }q(\chi)=0,\cr}$$

\noindent so that

$$\eqalignno{\int\bover{q}{\varhat{\bar h}_1}\,d\nu_{h_1}
&=\int p\times\varhat{\bar h}_2\,d\nu_{h_1}
=\int p\times\varhat{\bar h}_1\,d\nu_{h_2}\cr
\noalign{\noindent (by (b) above)}
&=\int\bover{q}{\varhat{\bar h}_2}\,d\nu_{h_2}.\cr}$$

\noindent So this common value must be $\phi(q)$.\ \Qed

If $q$, $q'\in C_k$ and $\alpha\in\Bbb C$, then

$$\int\bover{q+q'}{\varhat{\bar h}}\,d\nu_h
=\int\bover{q}{\varhat{\bar h}}\,d\nu_h
  +\int\bover{q'}{\varhat{\bar h}}\,d\nu_h,$$

$$\int\bover{\alpha q}{\varhat{\bar h}}\,d\nu_h
=\alpha\int\bover{q}{\varhat{\bar h}}\,d\nu_h$$

\noindent whenever $h\in P_K$, where
$K=\overline{\{\chi:|q(\chi)|+|q'(\chi)|>0\}}$;  so
$\phi(q+q')=\phi(q)+\phi(q')$ and $\phi(\alpha q)=\alpha\phi(q)$.
Moreover, if $q\ge 0$, then
$q/\varhat{\bar h}\ge 0$ for every $h\in P_K$, so $\phi(q)\ge 0$.

\medskip

{\bf (e)} By the Riesz Representation Theorem (in the form 436J) there
is a Radon measure $\lambda$ on $\Cal X$ such that
$\int q\,d\lambda=\phi(q)$ for any continuous function $q$ of compact
support.   (As in part (h) of the proof of 445N, the shift from
real-valued $q$ to complex-valued $q$ is elementary.)

\medskip

{\bf (f)} Now $\lambda$ is translation-invariant.   \Prf\ Take
$\theta\in\Cal X$ and $q\in C_k$.   Set
$K=\overline{\{\chi:q(\chi)\ne 0\}}$ and $L=\theta^{-1}K$, and
take any $h\in P_K$.  Set $h_1(x)=h\times\theta^{-1}$.   Then
$h_1$ is positive definite (445Lc);  of course it is continuous and
$\mu$-integrable;  and for any $\chi\in L$,

\Centerline{$\varhat{\bar h}_1(\chi)
=\int\overline{h(x)}\theta(x)\chi(x)\mu(dx)
=\varhat{\bar h}(\theta\chi)
>0$.}

\noindent So $h_1\in P_L$.

To relate $\nu_{h_1}$ to $\nu_h$, observe that if
$f\in\eusm L_{\Bbb C}^1$ then

\Centerline{$\varhatf(\theta\chi)
=\int f(x)\theta(x)\chi(x)\mu(dx)
=(f\times\theta)\varsphat(\chi)$,}

\noindent so

\Centerline{$\int\varhatf(\theta\chi)\nu_{h_1}(d\chi)
=\int (f\times\theta)(x)h_1(x)\mu(dx)
=\int f(x)h(x)\mu(dx)
=\int\varhatf(\chi)\nu_h(d\chi)$.}

\noindent So we see that the equation

\Centerline{$\int p(\theta\chi)\nu_{h_1}(d\chi)
=\int p(\chi)\nu_h(d\chi)$}

\noindent is valid whenever $p$ is of the form $\varhatf$, for some
$f\in\eusm L_{\Bbb C}^1$, and therefore for every $p\in C_0$.

Set $q_1(\chi)=q(\theta\chi)$ for every $\chi\in\Cal X$, so that $q_1\in
C_k$ and $L=\overline{\{\chi:q_1(\chi)\ne 0\}}$.   Accordingly

$$\eqalign{\phi(q_1)
&=\int\bover{q_1(\chi)}{\varhat{\bar h}_1(\chi)}\nu_{h_1}(d\chi)
=\int\bover{q(\theta\chi)}{\varhat{\bar h}(\theta\chi)}
  \nu_{h_1}(d\chi)\cr
&=\int\bover{q(\chi)}{\varhat{\bar h}(\chi)}
  \nu_h(d\chi)
=\phi(q).\cr}$$

\noindent So

\Centerline{$\int q(\theta\chi)\lambda(d\chi)
 =\int q(\chi)\lambda(d\chi)$.}

\noindent As $q$ and $\theta$ are arbitrary, $\lambda$ is
translation-invariant (441L).\ \Qed

\medskip

{\bf (g)} Thus $\lambda$ is either zero or a Haar measure on $\Cal X$.
I have still to confirm that

\Centerline{$f(x)
=\int\varhatf(\chi)\overline{\chi(x)}\lambda(d\chi)$}

\noindent whenever $f$ is continuous, positive definite and
$\mu$-integrable, and $x\in X$.   But recall the formula from (b) above.
If $q\in C_k$, $K=\overline{\{\chi:q(\chi)\ne 0\}}$ and $h\in P_K$, then
we must have

$$\int q\times\varhatbarf d\lambda
=\int\bover{q\times\varhatbarf}{\varhat{\bar h}}d\nu_h
=\int\bover{q\times\varhat{\bar h}}{\varhat{\bar h}}d\nu_f
=\int q\,d\nu_f.$$

\noindent In particular, $\int q\times\varhatbarf d\lambda\ge 0$ whenever
$q\ge 0$;  since $\varhatbarf$ is continuous (445Ka), and $\lambda$,
being a Radon measure, is strictly positive, this shows that
$\varhatbarf\ge 0$.   Also

\Centerline{$\int\varhatbarf d\lambda
=\sup\{\int q\times\varhatbarf d\lambda:q\in C_k,\,0\le q\le\tbf{1}\}
=\nu_f(\Cal X)<\infty$,}

\noindent so we have an indefinite-integral measure
$\varhatbarf\lambda$;  since this is a Radon measure (416S), and acts on
$C_k$ in the same way as $\nu_f$, it is actually equal to $\nu_f$
(by the uniqueness guaranteed in 436J).   In particular, for any $x\in X$,

\Centerline{$f(x)=\int\chi(x)\nu_f(d\chi)
=\int\chi(x)\varhatbarf(\chi)\lambda(d\chi)$}

\noindent by the second formula in 445N, and 235K.   But

$$\eqalignno{\varhatbarf(\chi)
&=\int\overline{f(x)}\chi(x)\mu(dx)
=\int f(x^{-1})\chi(x)\mu(dx)\cr
\noalign{\noindent (by 445Lb)}
&=\int f(x)\chi(x^{-1})\mu(dx)\cr
\noalign{\noindent (because $X$ is abelian, therefore unimodular)}
&=\varhatf(\chi^{-1})\cr}$$

\noindent for every $\chi\in\Cal X$.   So

$$\eqalignno{f(x)
&=\int\chi(x)\varhatf(\chi^{-1})\lambda(d\chi)
=\int\chi^{-1}(x)\varhatf(\chi)\lambda(d\chi)\cr
\noalign{\noindent (because $\Cal X$ is abelian)}
&=\int\varhatf(\chi)\overline{\chi(x)}\lambda(d\chi).\cr}$$

\medskip

{\bf (h)} We should check that $\lambda$ is non-zero
and unique.   But the construction in part (c) of the proof shows that
there are many $f\in P$ such that $f(e)\ne 0$, and for any such $f$ we have
$f(e)=\int\varhatf\,d\lambda$.   This shows simultaneously that $\lambda$
is non-zero, therefore a Haar measure;  and as all the Haar measures on
$\Cal X$ are scalar multiples of each other, there is at most one suitable
$\lambda$.
}%end of proof of 445P

\leader{445Q}{Remark}\cmmnt{ We can extract the following useful fact
from part (g) of the proof above.}   If $h:X\to\Bbb C$ is
$\mu$-integrable, continuous and positive definite, then
$\varhat{\bar h}$ is non-negative and $\lambda$-integrable, and the
Radon measure $\nu_h$ of 445N is just the indefinite-integral measure
$\varhat{\bar h}\lambda$.

\cmmnt{Note also that} $\lambda$ is\cmmnt{ actually} a Radon
measure\cmmnt{;  of
course it has to be, because $\Cal X$ is locally compact and Hausdorff
(445J)}.

\leader{445R}{The Plancherel Theorem} Let $X$ be an abelian topological
group with a
Haar measure $\mu$, and $\Cal X$ its dual group.   Let $\lambda$ be the
Haar measure on $\Cal X$ corresponding to $\mu$\cmmnt{ (445P)}.
Then there is a normed space isomorphism
$T:L_{\Bbb C}^2(\mu)\to L_{\Bbb C}^2(\lambda)$
defined by setting $T(f^{\ssbullet})=\varhatf\vthsp^{\ssbullet}$
whenever
$f\in\eusm L_{\Bbb C}^1(\mu)\cap\eusm L_{\Bbb C}^2(\mu)$.

\proof{{\bf (a)} Since $\varhatf=\varhat{g}$ whenever $f\eae g$, the
formula certainly defines an operator $T$ from
$L_{\Bbb C}^1(\mu)\cap L_{\Bbb C}^2(\mu)$
to $L_{\Bbb C}^0(\lambda)$, and of course it is linear.

If $f\in\eusm L_{\Bbb C}^1(\mu)\cap\eusm L_{\Bbb C}^2(\mu)$,
$h=f*\tilde f$ is continuous and positive definite (445Ld) and integrable,
and $\varhat{h}=|\varhatf|^2$ (445Kc).   Now

$$\eqalignno{\int|\varhatf|^2\,d\lambda
&=\int\varhat{h}\,d\lambda
=h(e)\cr
\noalign{\noindent (445P)}
&=\int f(x)\tilde f(x^{-1})\mu(dx)
=\int|f|^2d\mu.\cr}$$

\allowmorestretch{468}{
\noindent Thus $\|Tu\|_2=\|u\|_2$ whenever
$u\in L_{\Bbb C}^1(\mu)\cap L_{\Bbb C}^2(\mu)$;
since $L_{\Bbb C}^1(\mu)\cap L_{\Bbb C}^2(\mu)$ is $\|\,\,\|_2$-dense in
$L_{\Bbb C}^2(\mu)$
(244Ha/244Pb, or otherwise), we have a unique isometry
$T:L_{\Bbb C}^2(\mu)\to L_{\Bbb C}^2(\lambda)$ agreeing with the given
formula on $L_{\Bbb C}^1(\mu)\cap L_{\Bbb C}^2(\mu)$
(2A4I).
}%end of allowmorestretch

\medskip

{\bf (b)} The meat of the theorem is of course the proof that $T$ is
surjective.   \Prf\Quer\ Suppose, if possible, that
$W=T[L_{\Bbb C}^2(\mu)]$ is not equal to $L_{\Bbb C}^2(\lambda)$.
Because $T$ is linear, $W$ is a linear
subspace of $L_{\Bbb C}^2(\lambda)$;  because $T$ is an isometry, $W$ is
isometric to $L_{\Bbb C}^2(\mu)$, and in particular is complete,
therefore closed
in $L_{\Bbb C}^2(\lambda)$ (3A4Fd).   There is therefore a non-zero
continuous
linear functional on $L_{\Bbb C}^2(\lambda)$ which is zero on $W$
(3A5Ad), and this is of the form $u\mapsto\int u\times v$ for some
$v\in L_{\Bbb C}^2(\lambda)$ (244J/244Pc).   What this means is that
there is
a $g\in\eusm L_{\Bbb C}^2(\lambda)$ such that $g^{\ssbullet}\ne 0$ in
$L_{\Bbb C}^2(\lambda)$ but $\int\varhatf\times g\,d\lambda=0$ for every
$f\in\eusm L_{\Bbb C}^1(\mu)\cap\eusm L_{\Bbb C}^2(\mu)$.

Suppose that $f\in\eusm L_{\Bbb C}^1(\mu)$ and that $h$ is any
$\mu$-integrable
continuous positive definite function on $X$.   Then $h$ is bounded
(445Lb), so $|h|^2$ also is $\mu$-integrable, and
$f*\bar h\in\eusm L_{\Bbb C}^1(\mu)\cap\eusm L_{\Bbb C}^2(\mu)$ (444Ra).
Accordingly

\Centerline{$\int g\times\varhatf\times\varhat{\bar h}\,d\lambda
=\int g\times(f*\bar h)\varsphat d\lambda=0$.}

\noindent Thus $\int g\times\varhatf\,d\nu_h=0$, where
$\nu_h=\varhat{\bar h}\lambda$ is the
Radon measure on $\Cal X$ corresponding to $h$ constructed in 445N (see
445Q).   And this is true for every $f\in\eusm L_{\Bbb C}^1(\mu)$.   But
as $\{\varhatf:f\in\eusm L_{\Bbb C}^1(\mu)\}$ is
$\|\,\,\|_{\infty}$-dense in
$C_0(\Cal X;\Bbb C)$ (445Kd), and $g$ is $\nu_h$-integrable (because
$\varhat{\bar h}\in\eusm L_{\Bbb C}^2(\lambda)$, by (a), so
$\int|g\times\varhat{\bar h}|d\lambda<\infty$), $g$ must be zero
$\nu_h$-a.e., that is,
$g\times\varhat{\bar h}=0\,\,\lambda$-a.e.   Now recall (from part (c)
of the proof of 445P, for instance) that for every compact set
$K\subseteq\Cal X$ there is a $\mu$-integrable continuous positive
definite $h$ such that $\varhat{\bar h}(\chi)\ne 0$ for every
$\chi\in K$, so $g=0$ a.e.\ on $K$.   Since $\lambda$ is tight,
$g=0$ a.e.\ (412Jc), which is impossible.\ \Bang\Qed

Thus $T$ is surjective and we have the result.
}%end of proof of 445R

\leader{445S}{}\cmmnt{ While we do not have a direct definition of
$\varhatf$ for $f\in\eusm L_{\Bbb C}^2\setminus\eusm L_{\Bbb C}^1$, the
map $T:L_{\Bbb C}^2(\mu)\to L_{\Bbb C}^2(\lambda)$ does correspond to
the map $f\mapsto\varhatf$ in many
ways.   In particular, we have the following useful properties.

\medskip

\noindent}{\bf Proposition}   Let $X$ be an abelian topological group
with a Haar measure $\mu$, $\Cal X$ its dual group, $\lambda$ the
associated Haar measure on $\Cal X$ and
$T:L_{\Bbb C}^2(\mu)\to L_{\Bbb C}^2(\lambda)$ the
standard isometry\cmmnt{ described in 445R}.   Suppose that $f_0$,
$f_1\in\eusm L_{\Bbb C}^2(\mu)$ and $g_0$,
$g_1\in\eusm L_{\Bbb C}^2(\nu)$
are such that $Tf_0^{\ssbullet}=g_0^{\ssbullet}$ and
$Tf_1^{\ssbullet}=g_1^{\ssbullet}$, and take any $\theta\in\Cal X$.
Then

(a) setting $f_2=\bar f_0$, $g_2(\chi)=\overline{g_0(\chi^{-1})}$
whenever this is defined, $Tf_2^{\ssbullet}=g_2^{\ssbullet}$;

(b) setting $f_3=f_1\times\theta$, $g_3(\chi)=g_1(\theta\chi)$ whenever
this is defined, $Tf_3^{\ssbullet}=g_3^{\ssbullet}$;

(c) setting $f_4=f_0\times f_1\in\eusm L_{\Bbb C}^1(\mu)$,
$\varhatf_4(\theta)=(g_0*g_1)(\theta)$.

\proof{{\bf (a)} We have isometries $R_1:L_{\Bbb C}^2(\mu)\to L_{\Bbb
C}^2(\mu)$ and
$R_2:L_{\Bbb C}^2(\lambda)\to L_{\Bbb C}^2(\lambda)$ defined by setting
$R_1f^{\ssbullet}=(\bar f)^{\ssbullet}$ for every $f\in\eusm L_{\Bbb
C}^2(\mu)$
and $R_2g^{\ssbullet}=(\tilde g)^{\ssbullet}$ for every $g\in\eusm
L_{\Bbb C}^2(\lambda)$, where $\tilde g(\chi)=\overline{g(\chi^{-1})}$
whenever
this is defined.   Now if $f\in\eusm L_{\Bbb C}^1(\mu)$, then

\Centerline{$\varhatbarf(\chi)
=\int\overline{f(x)}\chi(x)\mu(dx)
=\overline{\intop f(x)\chi^{-1}(x)\mu(dx)}
=\tilde{\varhatf}(\chi)$}

\noindent for every $\chi\in\Cal X$.   So $TR_1u=R_2Tu$ for every
$u\in L_{\Bbb C}^1(\mu)\cap L_{\Bbb C}^2(\mu)$;  as
$L_{\Bbb C}^1\cap L_{\Bbb C}^2$ is dense in $L_{\Bbb C}^2$,
$TR_1=R_2T$, which is what we need to know.

\medskip

{\bf (b)} This time, set $R_1f^{\ssbullet}=(f\times\theta)^{\ssbullet}$
for every $f\in\eusm L_{\Bbb C}^2(\mu)$ and
$R_2g^{\ssbullet}=(\theta^{-1}\action_lg)^{\ssbullet}$ for
$g\in\eusm L_{\Bbb C}^2(\lambda)$, where
$(\theta\action_lg)(\chi)=g(\theta^{-1}\chi)$ whenever this
is defined.   Once again, $R_1:L_{\Bbb C}^2(\mu)\to L_{\Bbb C}^2(\mu)$
and $R_2:L_{\Bbb C}^2(\lambda)\to L_{\Bbb C}^2(\lambda)$ are isometries.
If $f\in\eusm L_{\Bbb C}^1(\mu)$, then

\Centerline{$(f\times\theta)\varsphat(\chi)
=\int f(x)\theta(x)\chi(x)\mu(dx)
=\varhatf(\theta\chi)$}

\noindent for every $\chi$, so $TR_1f^{\ssbullet}=R_2Tf^{\ssbullet}$;
once again, this is enough to prove that $TR_1=R_2T$, as required.

\medskip

{\bf (c)} We have

$$\eqalignno{(g_0*g_1)(\theta)
&=\int g_0(\chi^{-1})g_1(\theta\chi)\lambda(d\chi)
=\int\overline{g_2(\chi)}g_3(\chi)\lambda(d\chi)
=(g_3^{\ssbullet}|g_2^{\ssbullet})\cr
\noalign{\noindent (where $(\,\,|\,\,)$ is the standard inner product of
$L_{\Bbb C}^2(\lambda)$)}
&=(Tf_3^{\ssbullet}|Tf_2^{\ssbullet})\cr
\noalign{\noindent (using (a) and (b))}
&=(f_3^{\ssbullet}|f_2^{\ssbullet})\cr
\noalign{\noindent (because linear isometries of Hilbert space preserve
inner products, see 4A4Jc)}
&=\int f_3(x)\overline{f_2(x)}\mu(dx)
=\int f_1(x)\theta(x)f_0(x)\mu(dx)
=(f_0\times f_1)\varsphat(\theta).\cr}$$

}%end of proof of 445S

\leader{445T}{Corollary} Let $X$ be an abelian topological group with a
Haar measure
$\mu$, and $\lambda$ the corresponding Haar measure on the dual group
$\Cal X$ of $X$\cmmnt{ (445P)}.   Then for any non-empty open set
$H\subseteq\Cal X$, there is an $f\in\eusm L_{\Bbb C}^1(\mu)$ such that
$\varhatf\ne 0$ and $\varhatf(\chi)=0$ for $\chi\in\Cal X\setminus H$.

\proof{ Let $V_1$ and $V_2$ be non-empty open sets of finite measure
such that $V_1V_2\subseteq H$, and let $g_1$, $g_2$ be their
indicator functions.   Then there are $f_1$, $f_2\in\eusm L_{\Bbb
C}^2(\mu)$
such that $Tf_j^{\ssbullet}=g_j^{\ssbullet}$ for both $j$, where
$T:L_{\Bbb C}^2(\mu)\to L_{\Bbb C}^2(\lambda)$ is the isometry of 445R.
In this case, by
445Sc, $(f_1\times f_2)\varsphat=g_1*g_2$.   But it is easy to check
that $g_1*g_2$ is non-zero and zero outside $H$.
}%end of proof of 445T

\leader{445U}{The Duality Theorem}\cmmnt{ ({\smc Pontryagin 34}, 
{\smc Kampen 35})} Let $X$ be a locally compact Hausdorff abelian
topological group.   Then the canonical map $x\mapsto\hat x$ from $X$ to
its bidual $\frak X$\cmmnt{ (445E)} is an isomorphism between $X$ and
$\frak X$ as topological groups.

\proof{ By 445O, \hbox{$\sphat\,$} is a homeomorphism between $X$ and its
image $\hat X\subseteq\frak X$.   Accordingly $\hat X$ is itself, with
its subspace topology, a locally compact topological group, and is
closed in $\frak X$ (4A5Mc).

\Quer\ Suppose, if possible, that $\hat X\ne\frak X$.   Let $\mu$ be a
Haar measure on $X$ (441E) and $\lambda$ the associated Haar measure on
the dual $\Cal X$ of $X$ (445P).   By 445T, there is a
$g\in\eusm L_{\Bbb C}^1(\lambda)$ such that $\varhat{g}$ is zero on
$\hat X$ but not zero
everywhere, so that $g$ is not zero a.e.   We may suppose that $g$ is
defined everywhere on $\Cal X$.   In this case we have

\Centerline{$0=\varhat{g}(\hat x)
=\int g(\chi)\hat x(\chi)\lambda(d\chi)
=\int g(\chi)\chi(x)\lambda(d\chi)$}

\noindent for every $x\in X$.

By 418J, $g:\Cal X\to\Bbb C$ is almost continuous.   Let
$K\subseteq\{\chi:\chi\in\Cal X$, $g(\chi)\ne 0\}$ be a compact set such
that $\int_K|g|d\lambda\ge\bover34\int_{\Cal X}|g|d\lambda$ and
$g\restr K$ is continuous.   Set $q(\chi)=\overline{g(\chi)}/|g(\chi)|$
for $\chi\in K$.   Now consider the linear span $A$ of
$\{\hat x:x\in X\}$ as a linear space of complex-valued functions on
$\Cal X$.   Since
$\widehat{xy}=\hat x\times\hat y$ for all $x$, $y\in X$, $A$ is a
subalgebra
of $C_b=C_b(\Cal X;\Bbb C)$;  since $\widehat{x^{-1}}=\bar{\hat x}$ for
every $x\in X$, $\bar h\in A$ for every $h\in A$;  the constant function
$\hat e$ belongs to $A$;  and $A$ separates the points of
$\Cal X$.   By the Stone-Weierstrass theorem, in the form 281G, there is
an $h\in A$ such that $|h(\chi)-q(\chi)|\le\bover12$ for every
$\chi\in K$ and $|h(\chi)|\le 1$ for every $\chi\in\Cal X$.   Of course
$\int g\times h\,d\lambda=0$ for every $h\in A$ because
$\int g\times\hat x\,d\lambda=0$ for every $x\in X$.


Now, however,

$$\eqalign{\int_K|g|d\lambda
&=\int_Kg\times q\,d\lambda
\le\bigl|\int_Kg\times h\,d\lambda\bigr|+\int_K|g|\times|h-q|d\lambda\cr
&\le\bigl|\int_{\Cal X\setminus K}g\times h\,d\lambda\bigr|
  +\Bover12\int_K|g|d\lambda
\le\int_{\Cal X\setminus K}|g|d\lambda+\Bover12\int_K|g|d\lambda
<\int_K|g|d\lambda,\cr}$$

\noindent which is impossible.\ \Bang

Thus $\hat X=\frak X$ and the proof is complete.
}%end of proof of 445U

\exercises{\leader{445X}{Basic exercises (a)}
%\spheader 445Xa
Consider the additive group $\Bbb Q$ with its usual topology.   Show
that its dual group can be identified with the additive group $\Bbb R$
with its usual topology.
%445B

\spheader 445Xb Let $X$ be any topological group, and $\Cal X$ its dual
group.   Show that if $\nu$ is a totally finite Radon measure on $X$,
then its Fourier-Stieltjes
transform $\varhat{\nu}:\Cal X\to\Bbb C$ is continuous.
%445E

\spheader 445Xc Let $X$ be a topological group carrying Haar measures
and $\Cal X$ its dual group.   For a totally finite quasi-Radon measure
$\nu$ on $\Cal X$ set $\varhat{\nu}(x)=\int\chi(x)\nu(d\chi)$ for every
$x\in X$.   (i) Show that $\varhat{\nu}:X\to\Bbb C$ is continuous.
(ii) Show that
for any totally finite quasi-Radon measure $\mu$ on $X$ with
Fourier-Stieltjes
transform $\hat\mu$, $\int\varhat{\mu}\,d\nu=\int\varhat{\nu}\,d\mu$.
%445E

\sqheader 445Xd Let $X$ be a group and $h_1$, $h_2:X\to\Bbb C$ positive
definite functions.   Show that $h_1+h_2$, $\alpha h_1$ and $\bar h_1$
are also positive definite for any $\alpha\ge 0$.
%445L

\spheader 445Xe(i) Let $X$ be a group, $Y$ a subgroup of $X$ and
$h:Y\to\Bbb C$ a positive definite function.   Set $h_1(x)=h(x)$ if
$x\in Y$, $h_1(x)=0$ if $x\in X\setminus Y$.   Show that $h_1$ is
positive definite.   (ii) Let $X$ and $Y$ be groups, $\phi:X\to Y$ a
group homomorphism and $h:Y\to\Bbb C$ a positive definite function.
Show that $h\phi:X\to\Bbb C$ is positive definite.

\spheader 445Xf Let $X$ be a topological group and $\Cal X$ its dual
group.   (i) Let $\nu$ be any totally finite topological measure on
$\Cal X$ and set $h(x)=\int\chi(x)\nu(d\chi)$ for $x\in X$.   Show that
$h:X\to\Bbb C$ is positive definite.   (ii) Let $\nu$ be any totally
finite topological measure on $X$.   Show that its Fourier transform
$\varhat{\nu}:\Cal X\to\Bbb C$ is positive definite.
%445L

\sqheader 445Xg Let $X$ be a topological group with a left Haar measure,
and $h:X\to\Bbb C$ a bounded continuous function.   Show that $h$ is
positive definite iff $\int h(x^{-1}y)f(y)\overline{f(x)}dxdy\ge 0$ for
every integrable function $f$.
%445M

\sqheader 445Xh Let $\phi:\BbbR^r\to\Bbb C$ be a function.   Show that
it is the characteristic function of a probability distribution on
$\Bbb R^r$ iff it is continuous and positive definite and $\phi(0)=1$.
%445N

\sqheader 445Xi Let $X$ be a compact Hausdorff abelian topological
group, and $\mu$ the Haar probability measure on $X$.   Show that the
corresponding Haar measure on the dual group $\Cal X$ of $X$ is just
counting measure on $\Cal X$.
%445P

\sqheader 445Xj Let $X$ be an abelian group with its discrete topology,
and $\mu$ counting measure on $X$.   Let $\Cal X$ be the dual group and
$\lambda$ the corresponding Haar measure on $\Cal X$.   Show that
$\lambda\Cal X=1$.
%445P

\spheader 445Xk Let $X$ be the topological group $\Bbb R$, and
$\mu=\Bover1{\sqrt{2\pi}}\mu_L$, where $\mu_L$ is Lebesgue measure.
(i) Show that if we identify the dual group $\Cal X$ of $X$ with
$\Bbb R$, writing $\chi(x)=e^{-i\chi x}$ for $x$, $\chi\in\Bbb R$, then the
Haar measure on $\Cal X$ corresponding to the Haar measure $\mu$ on $X$
is $\mu$ itself.   (ii) Show that if we change the action of $\Bbb R$ on
itself by setting $\chi(x)=e^{-2\pi i\chi x}$, then the Haar measure on
$\Cal X$ corresponding to $\mu_L$ is $\mu_L$.
%445P

\spheader 445Xl Let $X_0,\ldots,X_n$ be abelian topological groups with
Haar measures $\mu_0,\ldots,\mu_n$, and let $X=X_0\times\ldots\times
X_n$ be the product group with its Haar measure
$\mu=\mu_0\times\ldots\times\mu_n$.   For each $k\le n$ let $\Cal X_k$
be the dual group of $X_k$ and $\lambda_k$ the Haar measure on $\Cal
X_k$ corresponding to $\mu_k$.   Show that if we identify
$\Cal X=\Cal X_0\times\ldots\times\Cal X_n$ with the dual group of $X$,
then the Haar measure on $\Cal X$ corresponding to $\mu$ is just the
product measure $\lambda_0\times\ldots\times\lambda_n$.
%445P

\sqheader 445Xm Let $X$ be a compact Hausdorff abelian topological
group, with dual group $\Cal X$, and $\mu$ the Haar probability measure
on $X$.   (i) Show that $\int\chi\,d\mu=0$ for every $\chi\in\Cal X$
except the identity.   (ii) Show that $\{\chi^{\ssbullet}:\chi\in\Cal
X\}$ is an orthonormal basis of the Hilbert space $L_{\Bbb C}^2(\mu)$.
\Hint{$(u|\chi^{\ssbullet})=(Tu|\varhat{\chi}\vthsp^{\ssbullet})$
where $T$ is the operator of 445R.}
%445R

\spheader 445Xn Let $X$ be an abelian topological group with a Haar
measure $\mu$, $\lambda$ the associated Haar measure on the dual group
$\Cal X$ and $T:L_{\Bbb C}^2(\mu)\to L_{\Bbb C}^2(\lambda)$ the standard
isometry.
Suppose that $u$, $v\in L_{\Bbb C}^2(\mu)$.   Suppose that $f_0$,
$f_1\in\eusm L_{\Bbb C}^2(\mu)$, $g_0$, $g_1\in\eusm L_{\Bbb C}^2(\nu)$
are such that
$Tf_0^{\ssbullet}=g_0^{\ssbullet}$ and
$Tf_1^{\ssbullet}=g_1^{\ssbullet}$.   Show that
$(f_0*f_1)(x)=\int g_0(\chi)g_1(\chi)\overline{\chi(x)}\lambda(d\chi)$
for any $x\in X$.
%445S

\spheader 445Xo Let $X$ be a locally compact Hausdorff abelian
topological group and $\Cal X$ its dual group.   Show that a function
$h:\Cal X\to\Bbb C$ is the Fourier-Stieltjes transform of a totally
finite Radon measure on $X$ iff it is continuous and positive definite.
%445U, 445N

\def\pltwadic{+_{\text{2adic}}}

\spheader 445Xp(i) Show that we can define a binary operation
$\pltwadic$ on $X=\{0,1\}^{\Bbb N}$ by setting $x\pltwadic y=z$ whenever
$x$, $y$, $z\in X$ and $\sum_{i=0}^k2^i(x(i)+y(i)-z(i))$ is divisible by
$2^{k+1}$ for every $k$.   (ii) Show that if we give $X$ its usual
topology then $(X,\pltwadic)$ is an abelian topological group.  (iii)
Show that the usual measure on $X$ is the Haar probability measure for
this group operation.   (iv) Show that
$G=\{\zeta:\zeta\in\Bbb C,\,\Exists n\in\Bbb N,\,\zeta^{2^n}=1\}$ is a
subgroup of $\Bbb C$.   (v) Show that the dual of $(X,\pltwadic)$ is
$\{\chi_{\zeta}:\zeta\in G\}$ where
$\chi_{\zeta}(x)=\prod_{i=0}^{\infty}\zeta^{2^ix(i)}$ for $\zeta\in G$
and $x\in X$.   (vi) Show that the functions $f$, $g:X\to X$ described
in 388E are of the form $x\mapsto x\pm_{\text{2adic}}x_0$ for a certain
$x_0\in X$.
% +

\spheader 445Xq Let $X$ be a locally compact Hausdorff abelian topological
group.   Show that if two totally finite Radon measures on
$X$ have the same Fourier-Stieltjes transform, they are equal.  
\Hint{281G.}
%445O out of order query

\leader{445Y}{Further exercises (a)}
%\spheader 445Ya
Let $X$ be any Hausdorff topological group.   Let $\widehat{X}$ be its
completion under its bilateral uniformity.   Show that the dual
groups of $X$ and $\widehat{X}$ can be identified as groups.   Show that
they can be identified as topological groups if {\it either} $X$ is
metrizable {\it or} $X$ has a totally bounded neighbourhood of the
identity.
%445B

\spheader 445Yb Let $\family{j}{I}{X_j}$ be a countable family of
topological groups, with product $X$;  let $\Cal X_j$ be the dual group
of each $X_j$, and $\Cal X$ the dual group of $X$.   Show that the
topology of $\Cal X$ is generated by sets of the form $\Cal
X\cap\prod_{j\in I}H_j$
where $H_j\subseteq\Cal X_j$ is open for each $j$.
%445B

\spheader 445Yc Let $X$ be a real linear topological space, with
addition as its group operation.   Show that its dual group is just the
set of functionals $x\mapsto e^{if(x)}$ where $f:X\to\Bbb R$ is a
continuous linear functional.   Hence show that there are abelian groups
with trivial duals.
%445B

\spheader 445Yd Let $X$ be the group of rotations of $\BbbR^3$, that
is, the group of orthogonal real $3\times 3$ matrices with determinant
$1$, and give $X$ its usual topology, corresponding to its embedding as
a subspace of $\BbbR^9$.   Show that the only character on $X$ is the
constant function $1$.   \Hint{(i) show that two rotations through the
same angle
are conjugate in $X$;  (ii) show that if $0<\theta\le\bover{\pi}2$ then
the product of two rotations through the angle $\theta$ about
orthogonal axes is not a rotation through an angle $2\theta$.}
%445B

\spheader 445Ye Let $X$ be a finite abelian group, endowed with its
discrete topology.   Show that its dual is isomorphic to $X$.
\Hint{$X$ is a product of cyclic groups.}
%445B

\spheader 445Yf Show that if $I$ is any uncountable set, then there is a
quasi-Radon probability measure $\nu$ on the topological group
$\Bbb R^I$ such that its Fourier-Stieltjes transform
$\varhat{\nu}$ is not continuous.   \Hint{take $\nu$ to be a power of a
suitably widely spread probability distribution on $\Bbb R$.}
%445E

\spheader 445Yg Let $X$ be an abelian topological group with a Haar
measure $\mu$, and $\lambda$ the associated Haar measure on the dual
group $\Cal X$ of $X$.   Let
$T:L_{\Bbb C}^2(\mu)\to L_{\Bbb C}^2(\lambda)$ be the
standard isomorphism.   Suppose that $f\in\eusm L_{\Bbb C}^2(\mu)$ and
$g\in\eusm L_{\Bbb C}^1(\lambda)\cap\eusm L_{\Bbb C}^2(\lambda)$ are such
that $Tf^{\ssbullet}=g^{\ssbullet}$.   Show that
$f(x)=\int g(\chi)\overline{\chi(x)}\lambda(d\chi)$ for almost every
$x\in X$.   \Hint{look first at locally compact Hausdorff $X$.}
%445S

\spheader 445Yh Let $X$ be a locally compact Hausdorff abelian
topological group
with dual group $\Cal X$, $P_{\text{R}}$ the set of Radon probability
measures on $X$, $\sequencen{\nu_n}$ a sequence in $P_{\text{R}}$ and
$\nu$ a member of $P_{\text{R}}$.
Show that the following are equiveridical:  (i)
$\sequencen{\nu_n}\to\nu$ for the narrow topology on $P_{\text{R}}$;
(ii) $\lim_{n\to\infty}\varhat{\nu}_n(\chi)=\varhat{\nu}(\chi)$ for every
$\chi\in\Cal X$.   \Hint{compare 285L.   For the critical step, showing
that $\{\nu_n:n\in\Bbb N\}$ is uniformly tight, use the formulae in
445N to show that there is an integrable $f:\Cal X\to\Bbb C$ such that
$0\le\varcheckf\le\tbf{1}$ and
$\int\varcheckf(x)\nu(dx)\ge 1-\bover12\epsilon$.}
%445U

\spheader 445Yi Let $X$ be a topological group carrying Haar measures
and $\Cal X$ its dual group.   Let $M_{\tau}$ be the complex Banach
space of signed totally finite $\tau$-additive Borel measures on
$\Cal X$ (put the ideas of 437F and 437Yb together).   Show that $X$
separates
the points of $M_{\tau}$ in the sense that if $\nu\in M_{\tau}$ is
non-zero, there is an $x\in X$ such that $\int\chi(x)\nu(d\chi)\ne 0$,
if the integral is appropriately interpreted.
\Hint{use the method in the proof of 445U.}
%445U

\spheader 445Yj Let $X$ be an abelian topological group carrying Haar
measures.   Let $M_{\tau}$ be the complex Banach space of signed totally
finite $\tau$-additive Borel measures on $X$.   Show that the dual
$\Cal X$ of $X$ separates the points of $M_{\tau}$ in the sense that if
$\nu\in M_{\tau}$ is non-zero, there is a $\chi\in\Cal X$ such that
$\int\chi(x)\nu(dx)\ne 0$.   \Hint{use 443L to reduce to the case in
which $X$ is locally compact and Hausdorff;  now use 445U and 445Yi.}
%445U, 445Yi

\spheader 445Yk Let $X$ be an abelian topological group and $\mu$ a Haar
measure on $X$.   Show that the spectral radius of any non-zero element
of $L_{\Bbb C}^1(\mu)$ is non-zero.   \Hint{445Yj, 445Kd.}
%445U, 445Yj

\spheader 445Yl Show that for any integer $p\ge 2$ there is an operation
$+_{p\text{adic}}$ on $\{0,\ldots,p-1\}^{\Bbb N}$ with properties
similar to those of the operation $\pltwadic$ of 445Xp.
%445Xp

\spheader 445Ym Let $\mu$ be Lebesgue measure on $\coint{0,\infty}$.
(i) For $f$, $g\in\eusm L^1(\mu)$ define
$(f*g)(x)=\int_0^xf(y)g(x-y)\mu(dy)$
whenever the integral is defined.   Show that $f*g\in\eusm L^1(\mu)$.
(ii) Show that we can define a bilinear operator $*$ on $L^1(\mu)$ by
setting $f^{\ssbullet}*g^{\ssbullet}=(f*g)^{\ssbullet}$ for $f$,
$g\in\eusm L^1(\mu)$, and that under this multiplication $L^1(\mu)$ is a
Banach algebra.   (iii) Show that if $\phi:L^1(\mu)\to\Bbb R$ is a
multiplicative linear operator then there is some $s\ge 0$ such that
$\phi(f^{\ssbullet})=\int_0^{\infty}f(x)e^{-sx}\mu(dx)$ for every
$f\in\eusm L^1(\mu)$.
%445H out of order
}%end of exercises

\endnotes{
\Notesheader{445} I repeat that this section is intended to be a more or
less direct attack on the duality theorem.   At every point the clause
`let $X$ be a locally compact Hausdorff abelian topological group' is
present in
spirit.   The actual statement of each theorem involves some subset of
these properties, purely in accordance with the principle of omission
of irrelevant hypotheses, not because I expect to employ the results in
any more general setting.

In 445Ab I describe a topology on the dual group in a context so
abstract that we have rather a lot of choice.   For groups carrying Haar
measures, the alternative descriptions of the topology on the dual
(445I) make it plain that this must be the first topology to study.   By
445E it is already becoming fairly convincing.
But it is not clear that there is any such pre-eminent topology in the
general case.

Fourier-Stieltjes transforms hardly enter into the arguments of this
section;  I mention them mostly because they form the obvious
generalization of the ideas in \S285.   But I note that the principal
theorem of \S285 (that sequential convergence of characteristic
functions corresponds to sequential convergence of distributions, 285L)
generalizes directly to the context here (445Yh).

I have tried to lay this treatise out in such a way that we periodically
return to themes from past chapters at a higher level of sophistication.
There seem to be four really important differences between this section
and the previous treatment in Chapter 28.   (i) The first is the obvious
one;  we are dealing with general locally compact Hausdorff abelian
groups, rather than with $\Bbb R$ and $S^1$ and $\Bbb Z$.   Of course
this puts much heavier demands on our technique, and, to begin with,
leaves our imaginations unfocused.   (ii) The second concerns
differentiation, or rather its absence;  since we no longer have any
differential structure on our groups, a substantial part of the theory
evaporates, and we are forced to employ new tactics in the rest.   (iii)
The third concerns the normalization of the measure on the dual group.
As soon as we know that $\Cal X$ is a locally compact group (445J) we
know that it carries Haar measures.   The problem is to describe the
particular one we need in appropriate terms.   In the case of the dual
pairs $(\Bbb R,\Bbb R)$ or $(S^1,\Bbb Z)$, we have measures already
presented (counting measure on $\Bbb Z$, Lebesgue measure on
$\ocint{-\pi,\pi}$ and $\Bbb R$).   (They are not in fact dual in the
sense we need here, at least not if we use the simplest formulae for the
duality, and have to be corrected in each case by a factor of $2\pi$.
See 445Xk.)   But since we do have dual pairs already to hand, we can
simultaneously develop theories of Fourier transforms and inverse
Fourier transforms (for the pair $(S^1,\Bbb Z)$ the inverse Fourier
transform is just summation of trigonometric series), and the problem is
to successfully match operations which have independent existences.
(iv) The final change concerns an interesting feature of $\Bbb Z$ and
$\Bbb R$.   Repeatedly, in \S\S282-283, the formulae invoked symmetric
limits $\lim_{n\to\infty}\sum_{k=-n}^n$ or $\lim_{a\to\infty}\int_{-a}^a$
to approach some conditionally convergent sum or integral.   Elsewhere
one sometimes deals with singularities by examining `Cauchy principal
values';  if $\int_{-1}^1f$ is undefined, try
$\lim_{a\downarrow 0}(\int_{-1}^{-a}f+\int_a^1f)$.   This particular
method seems to disappear in the general context.   But the general
challenge of the subject remains the same:  to develop a theory of the
transform $u\mapsto\varhat{u}$ which will apply to the largest possible
family of objects $u$ and will enable us to justify, in the widest
possible contexts, the manipulations listed in the notes to \S284.   The
calculations in 445S and 445Xn, treating `shift' and `convolution' in
$L^2$, are typical.

In terms of the actual proofs of the results here, `test functions'
(284A) have gone, and in their place we take a lot more trouble over the
Banach algebra $L^1$.   This algebra is the key to one of the magic
bits, which turns up in rather undignified corners in 445Kd and part (e)
of the proof of 445N.   Down to 445O, the dominating problem is that we
do not know that the dual group $\Cal X$ of a group $X$ is large enough
to tell us anything interesting about $X$.   (After that, the problem
reverses;  we have to show that $\Cal X$ is not too big.)   We find that
(under rather specially arranged circumstances) we are able to say
something useful about the spectral radius of a member of $L^1$, and we
use this to guarantee that it has a non-trivial Fourier transform.   If
we identify $\Cal X$ with the maximal ideal space of $L^1$ (445H), then
the Fourier transform on $L^1$ becomes the `Gelfand map', a general
construction of great power in the theory of commutative Banach
algebras.

There is one similarity between the methods of this section and those of
\S284.   In both cases we have isomorphisms between $L_{\Bbb C}^2(\mu)$
and $L_{\Bbb C}^2(\lambda)$ (the Plancherel Theorem), but cannot define
the Fourier
transform of a function in $\eusm L_{\Bbb C}^2$ in any direct way;
indeed, while
the Fourier transform of a function in $\eusm L_{\Bbb C}^1$, or even of
a (totally finite) measure, can really be thought of as a (continuous)
function, the transform of a function in $L^2$ is at best a member of
$L^2$, not a function at all.   We manoeuvre around this difficulty by
establishing that our (genuine) Fourier transforms match dense subspaces
isometrically.   In \S284 I used test functions, and in the present
section I use $\eusm L^1\cap\eusm L^2$.   Test functions are easier
partly because the Fourier transform of a test function is again a test
function, and all the formulae we need are easy to establish for such
functions.

Searching for classes of functions which will be readily manageable in
general locally compact abelian groups, we come to the `positive
definite' functions.   The phrase is unsettling, since the functions
themselves are in no obvious sense positive (nor even, as a rule,
real-valued).   Also their natural analogues in the theory of bilinear
forms are commonly called `positive semi-definite'.   However, their
Fourier transforms, whether regarded as
measures (445N) or as functions (445Q), are positive, and, as a bonus,
we get a characterization of the Fourier transforms of measures (445Xf,
445Xh), answering a question left hanging in 285Xr.
}%end of notes

\discrpage

