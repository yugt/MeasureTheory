\frfilename{mt476.tex}
\versiondate{10.11.07}
\copyrightdate{2001}

\def\varinnerprod#1#2{#1\dotproduct#2}

\def\chaptername{Geometric measure theory}
\def\sectionname{Concentration of measure}

\newsection{476}

Among the myriad special properties of Lebesgue measure, a particularly
interesting one is `concentration of measure'.   For a set of given
measure in the plane, it is natural to feel that it is most
`concentrated' if it is a disk.   There are many ways of defining
`concentration', and I examine three of them in this section (476F, 476G
and 476H);  all lead us to Euclidean balls as the `most concentrated'
shapes.   On the sphere the same criteria lead us to caps
(476K\cmmnt{, 476Xe}).

All the main theorems of this section will be based on the fact that
semi-continuous functions on compact spaces attain their bounds.   The
compact spaces in question will be spaces of subsets, and I begin with
some general facts concerning the topologies introduced in 4A2T
(476A-476B).
The particular geometric properties of Euclidean
space which make all these results possible are described in 476D-476E,
where I describe concentrating operators based on reflections.   The
actual theorems 476F-476H %476F 476G 476H
and 476K can now almost be mass-produced.

\leader{476A}{Proposition} Let $X$ be a topological space,
$\Cal C$ the
family of closed subsets of $X$, $\Cal K\subseteq\Cal C$ the family of
closed compact sets and $\mu$ a topological measure on $X$.

(a) Suppose that $\mu$ is inner regular with respect to the closed sets.

\quad(i) If $\mu$ is outer regular with respect to the open
sets\cmmnt{ (for instance, if $\mu$ is totally finite)} then
$\mu\restr\Cal C:\Cal C\to\Bbb R$ is upper semi-continuous
with respect to the Vietoris topology on
$\Cal C$.

\quad(ii) If $\mu$ is locally finite then
$\mu\restr\Cal K$ is upper semi-continuous
with respect to the Vietoris topology.

\quad(iii) If $f$ is a non-negative $\mu$-integrable real-valued function
then $F\mapsto\int_Ffd\mu:\Cal C\to\Bbb R$ is upper semi-continuous with
respect to the Vietoris topology.

(b) Suppose that $\mu$ is tight.

\quad(i) If $\mu$ is totally finite then
$\mu\restr\Cal C$ is upper semi-continuous
with respect to the Fell topology on $\Cal C$.

\quad(ii) If $f$ is a non-negative $\mu$-integrable real-valued function
then $F\mapsto\int_Ffd\mu:\Cal C\to\Bbb R$ is upper semi-continuous with
respect to the Fell topology.

(c) Suppose that $X$ is metrizable, and that $\rho$ is a metric on $X$
defining its topology;  let $\tilde\rho$ be the Hausdorff metric on
$\Cal C\setminus\{\emptyset\}$.

\quad(i) If $\mu$ is totally finite, then
$\mu\restr\Cal C\setminus\{\emptyset\}$ is upper semi-continuous with
respect to $\tilde\rho$.

\quad(ii) If $\mu$ is locally finite, then
$\mu\restr\Cal K\setminus\{\emptyset\}$ is upper semi-continuous with
respect to $\tilde\rho$.

\quad(iii) If $f$ is a non-negative
$\mu$-integrable real-valued function, then
$F\mapsto\int_Ffd\mu:\Cal C\setminus\{\emptyset\}\to\Bbb R$ is upper
semi-continuous with respect to $\tilde\rho$.

\proof{{\bf (a)(i)} Suppose that $F\in\Cal C$ and that $\mu F<\alpha$.
Because $\mu$ is outer regular with respect to the
open sets, there is an open set
$G\supseteq F$ such that $\mu G<\alpha$.
Now $\Cal V=\{E:E\in\Cal C,\,E\subseteq G\}$ is an open set
for the Vietoris topology containing $F$,
and $\mu E<\alpha$ for every $E\in\Cal V$.   As $F$ and
$\alpha$ are arbitrary, $\mu\restr\Cal C$ is upper semi-continuous for the
Vietoris topology.

\medskip

\quad{\bf (ii)} Given that $K\in\Cal K$ and $\mu K<\alpha$, then,
because $\mu$ is locally finite, there is an open set $G$ of
finite measure including $K$ (cf.\ 411Ga).   Now there is a closed set
$F\subseteq G\setminus K$ such that $\mu F>\mu G-\alpha$, so that
$\Cal V=\{L:L\in\Cal K$, $L\subseteq G\setminus F\}$ is a relatively open
subset of $\Cal K$ for the Vietoris topology containing $K$, and
$\mu L<\alpha$ for every $L\in\Cal V$.

\medskip

\quad{\bf (iii)} Apply (i) to the indefinite-integral measure over $\mu$
defined by $f$;  by 412Q this is still inner regular with respect to the
closed sets.

\medskip

{\bf (b)} If $F\in\Cal C$ and $\mu F<\alpha$, let
$K\subseteq X\setminus F$ be a compact set such that
$\mu K>\mu X-\alpha$.   Then
$\Cal V=\{E:E\in\Cal C$, $E\cap K=\emptyset\}$ is a neighbourhood of $F$
and $\mu E<\alpha$ for every $E\in\Cal V$.   This proves (i).
Now (ii) follows as in (a-iii) above.

\medskip

{\bf (c)(i)} If $F\in\Cal C\setminus\{\emptyset\}$ and
$\mu F<\alpha$, then for each $n\in\Bbb N$ set
$F_n=\{x:\rho(x,F)\le 2^{-n}\}$.   Since $\sequencen{F_n}$ is a
non-increasing sequence of closed sets with intersection $F$, and $\mu$
is totally finite, there is an $n$ such that $\mu F_n<\alpha$.   If now
we take $E\in\Cal C\setminus\{\emptyset\}$ such that
$\tilde\rho(E,F)\le 2^{-n}$, then $E\subseteq F_n$ so $\mu E<\alpha$.
As $F$ and $\alpha$ are arbitrary,
$\mu\restr\Cal C\setminus\{\emptyset\}$ is upper semi-continuous.

\medskip

\quad{\bf (ii)} If $K\in\Cal K\setminus\{\emptyset\}$ and $\mu K<\alpha$,
let $G\supseteq K$ be an open set of finite measure, as in (a-ii) above.
The function $x\mapsto\rho(x,X\setminus G)$ is continuous and
strictly positive on $K$, so has a non-zero lower bound on $K$, and
there is some $m\in\Bbb N$ such that $\rho(x,X\setminus G)>2^{-m}$ for
every $x\in K$.   If, as in (i) just above, we set
$F_n=\{x:\rho(x,K)\le 2^{-n}\}$ for each $n$, $F_m\subseteq G$ has finite
measure.   So, as in (i), we have an $n\ge m$ such that $\mu F_n<\alpha$,
and we can continue as before.

\medskip

\quad{\bf (iii)} Once again this follows at once from (i).
}%end of proof of 476A

\vleader{60pt}{476B}{Lemma} Let $(X,\rho)$ be a metric space, and $\Cal C$
the family of closed subsets of $X$, with its Fell topology.   For
$\epsilon>0$, set
$U(A,\epsilon)=\{x:x\in X$, $\rho(x,A)<\epsilon\}$ if
$A\subseteq X$ is not
empty;  set $U(\emptyset,\epsilon)=\emptyset$.   Then for any
$\tau$-additive topological measure $\mu$ on $X$, the function

\Centerline{$(F,\epsilon)\mapsto\mu U(F,\epsilon):
  \Cal C\times\ooint{0,\infty}\to[0,\infty]$}

\noindent is lower semi-continuous.

\proof{ Set
$Q=\{(F,\epsilon):F\in\Cal C$, $\epsilon>0$, $\mu U(F,\epsilon)>\gamma\}$,
where $\gamma\in\Bbb R$.   Take any $(F_0,\epsilon_0)\in Q$.
Note first that
$\mu U(F_0,\epsilon_0)=\sup_{\epsilon<\epsilon_0}\mu U(F,\epsilon)$, so
there is a $\delta\in\ooint{0,\bover12\epsilon_0}$ such that
$\mu U(F_0,\epsilon_0-2\delta)>\gamma$.
Next, $\{U(x,\epsilon_0-2\delta):x\in F_0\}$ is an open cover of
$U(F_0,\epsilon_0-2\delta)$;  because $\mu$ is $\tau$-additive, there is a
finite set $I\subseteq F_0$ such that
$\mu(\bigcup_{x\in I}U(x,\epsilon_0-2\delta))>\gamma$.   Consider

\Centerline{$\Cal V
=\{F:F\in\Cal C$, $F\cap U(x,\delta)\ne\emptyset$ for every $x\in I\}$.}

\noindent By the definition of the Fell topology, $\Cal V$ is open.
So $\Cal V\times\ooint{\epsilon_0-\delta,\infty}$ is an open
neighbourhood of $(F_0,\epsilon)$ in $\Cal C\times\Bbb R$.   If
$F\in\Cal V$ and $\epsilon>\epsilon_0-\delta$, then

\Centerline{$U(F,\epsilon)\supseteq\bigcup_{x\in I}U(x,\epsilon-\delta)
\supseteq\bigcup_{x\in I}U(x,\epsilon_0-2\delta)$}

\noindent has measure greater than $\gamma$ and $(F,\epsilon)\in Q$.   As
$(F_0,\epsilon_0)$ is arbitrary, $Q$ is open;  as $\gamma$ is arbitrary,
$(F,\epsilon)\mapsto\mu U(F,\epsilon)$ is lower semi-continuous.
}%end of proof of 476B

\cmmnt{\medskip

\noindent{\bf Remark} Recall that all `ordinary' topological measures on
metric spaces are $\tau$-additive;  see 438J.
}

\leader{476C}{Proposition} Let $(X,\rho)$ be a non-empty compact metric
space, and suppose that its isometry group $G$ acts transitively on $X$.
Then $X$ has a unique $G$-invariant Radon probability measure $\mu$,
which is strictly positive.

\proof{ By 441G, $G$, with its topology of pointwise convergence, is a
compact topological group, and the action of $G$ on $X$ is continuous.
So 443Ud gives the result.
}%end of proof of 476C

\leader{476D}{Concentration by partial reflection}\cmmnt{ The
following construction will be
used repeatedly in the rest of the section.}   Let $X$ be an inner
product space.   \cmmnt{(In this section, $X$ will be usually be
$\BbbR^r$, but in 493G below it
will be helpful to be able to speak of abstract Hilbert spaces.)}   For
any unit vector $e\in X$ and any $\alpha\in\Bbb R$, write
$R_{e\alpha}:X\to X$ for the reflection in the hyperplane
$V_{e\alpha}=\{x:x\in X,\,\innerprod{x}{e}=\alpha\}$, so that
$R_{e\alpha}(x)=x+2(\alpha-\innerprod{x}{e})e$ for every $x\in X$.
Next, for any
$A\subseteq X$, \cmmnt{we can} define \cmmnt{a set}
$\psi_{e\alpha}(A)$ by setting

$$\eqalign{\psi_{e\alpha}(A)
&=\{x:x\in A,\,\innerprod{x}{e}\ge\alpha\}
\cup\{x:x\in A,\innerprod{x}{e}<\alpha,\,R_{e\alpha}(x)\in A\}\cr
&\hskip10em\cup
\{x:x\in \BbbR^r\setminus A,\,\innerprod{x}{e}\ge\alpha,\,
R_{e\alpha}(x)\in A\}\cr
&=(W\cap(A\cup R_{e\alpha}[A]))
  \cup(A\cap R_{e\alpha}[A]),\cr}$$

\noindent where $W$ is the half-space
$\{x:\innerprod{x}{e}\ge\alpha\}$.   \cmmnt{Geometrically, we
construct $\psi_{e\alpha}(A)$ by moving those points of $A$ on the
`wrong' side of the hyperplane
$V_{e\alpha}$ to their reflections, provided those points are not
already occupied.   We have the following facts.}

\spheader 476Da For non-empty $A\subseteq X$ and $\epsilon>0$, set
$U(A,\epsilon)=\{x:\rho(x,A)<\epsilon\}$, where $\rho$ is the
standard metric on $X$.   \cmmnt{Now}
$U(\psi_{e\alpha}(A),\epsilon)\subseteq\psi_{e\alpha}(U(A,\epsilon))$.
\prooflet{\Prf\ Take $x\in U(\psi_{e\alpha}(A),\epsilon)$.   Then there
is a $y\in\psi_{e\alpha}(A)$ such that $\|x-y\|<\epsilon$.

\quad{\bf case 1} Suppose $\innerprod{x}{e}\ge\alpha$.   If
$x\in U(A,\epsilon)$ then certainly $x\in\psi_{e\alpha}(U(A,\epsilon))$.
Otherwise, because $\|x-y\|<\epsilon$, $y\notin A$, so
$R_{e\alpha}(y)\in A$.   But
$\|R_{e\alpha}(x)-R_{e\alpha}(y)\|=\|x-y\|<\epsilon$, so
$R_{e\alpha}(x)\in U(A,\epsilon)$ and
$x\in\psi_{e\alpha}(U(A,\epsilon))$.

\quad{\bf case 2a} Suppose $\innerprod{x}{e}<\alpha$,
$\innerprod{y}{e}\ge\alpha$.   Then

\Centerline{$\|R_{e\alpha}(x)-y\|=\|x-R_{e\alpha}(y)\|
\le\|R_{e\alpha}(x)-R_{e\alpha}(y)\|=\|x-y\|<\epsilon$.}

\noindent At least one of $y$, $R_{e\alpha}(y)$ belongs to $A$, so both
$x$ and $R_{e\alpha}(x)$ belong to $U(A,\epsilon)$ and
$x\in\psi_{e\alpha}(U(A,\epsilon))$.

\quad{\bf case 2b} Suppose $\innerprod{x}{e}<\alpha$ and
$\innerprod{y}{e}<\alpha$.   In this case, both $y$ and $R_{e\alpha}(y)$
belong to $A$, so both $x$ and $R_{e\alpha}(x)$ belong to
$U(A,\epsilon)$ and again $x\in\psi_{e\alpha}(U(A,\epsilon))$.

Thus $x\in\psi_{e\alpha}(U(A,\epsilon))$ in all cases;  as $x$ is
arbitrary, we have the result.\ \Qed}%end of prooflet

\spheader 476Db If $F\subseteq X$ is closed, then $\psi_{e\alpha}(F)$ is
closed.   \prooflet{\Prf\ Use the second formula for
$\psi_{e\alpha}(F)$.\ \Qed}

\vleader{60pt}{476E}{Lemma} 
Let $X$ be an inner product space, $e\in X$ a unit vector
and $\alpha\in\Bbb R$.   Let $R=R_{e\alpha}:X\to X$ be the reflection
operator, and $\psi=\psi_{e\alpha}:\Cal PX\to\Cal PX$ the associated
transformation, as described in 476D.   For $x\in A\subseteq X$, define

$$\eqalign{\phi_A(x)
&=x\text{ if }\innerprod{x}{e}\ge\alpha,\cr
&=x\text{ if }\innerprod{x}{e}<\alpha
  \text{ and }R(x)\in A,\cr
&=R(x)\text{ if }\innerprod{x}{e}<\alpha
  \text{ and }R(x)\notin A.\cr}$$

\noindent Let $\nu$ be a topological measure on $X$ which is
$R$-invariant\cmmnt{, that is, $\nu$ coincides with the image measure
$\nu R^{-1}$}.

(a) For any $A\subseteq X$, $\phi_A:A\to\psi(A)$ is a bijection.
If $\alpha<0$, then $\|\phi_A(x)\|\le\|x\|$ for every $x\in A$, with
$\|\phi_A(x)\|<\|x\|$ iff $\innerprod{x}{e}<\alpha$ and
$R(x)\notin A$.

(b) If $E\subseteq X$ is measured by $\nu$, then $\psi(E)$ is measurable
and $\nu\psi(E)=\nu E$;  moreover, $\phi_E$ is a measure space
isomorphism for the subspace measures on $E$ and $\psi(E)$.

(c) If $\alpha<0$ and $E\subseteq X$ is measurable, then
$\int_E\|x\|\nu(dx)\ge\int_{\psi(E)}\|x\|\nu(dx)$, with equality iff
$\{x:x\in E,\,\innerprod{x}{e}<\alpha,\,R(x)\notin E\}$ is negligible.

(d) Suppose that $X$ is separable.   Let $\lambda$ be the c.l.d.\
product measure of $\nu$ with itself on $X\times X$.   If
$E\subseteq X$ is measurable, then

\Centerline{$\int_{E\times E}\|x-y\|\lambda(d(x,y))
\ge\int_{\psi(E)\times\psi(E)}\|x-y\|\lambda(d(x,y))$.}

(e) Now suppose that $X=\BbbR^r$.   Then
$\nu(\partstar\psi(A))\le\nu(\partstar A)$ for every
$A\subseteq\BbbR^r$, where $\partstar A$ is the essential boundary of
$A$\cmmnt{ (definition:  475B)}.

\proof{{\bf (a)} That $\phi_A:A\to\psi(A)$ is a bijection is immediate
from the definitions of $\psi$ and $\phi_A$.   If $\alpha<0$, then for
any $x\in A$ either $\phi_A(x)=x$ or $\innerprod{x}{e}<\alpha$ and
$R(x)\notin A$.   In the latter case

$$\eqalignno{\|\phi_A(x)\|^2
&=\|R(x)\|^2=\|x+2\gamma e\|^2\cr
\displaycause{where $\gamma=\alpha-\innerprod{x}{e}>0$}
&=\|x\|^2+4\gamma\innerprod{x}{e}+4\gamma^2
=\|x\|^2+4\gamma\alpha
<\|x\|^2,\cr}$$

\noindent so $\|\phi_A(x)\|<\|x\|$.

\medskip

{\bf (b)} If we set

\Centerline{$E_1=\{x:x\in E,\,\innerprod{x}{e}\ge\alpha\}$,}

\Centerline{$E_2=\{x:x\in E,\innerprod{x}{e}<\alpha,\,
R_{e\alpha}(x)\in E\}$,}

\Centerline{$E_3=\{x:x\in E,\,\innerprod{x}{e}<\alpha,\,
R_{e\alpha}(x)\notin E\}$,}

\Centerline{$E_4=\{x:x\in\BbbR^r\setminus E,\,
\innerprod{x}{e}>\alpha,\,R_{e\alpha}(x)\in E\}$,}

\noindent then $E_1$, $E_2$, $E_3$ and $E_4$ are disjoint measurable
sets, $E=E_1\cup E_2\cup E_3$, $\psi(E)=E_1\cup E_2\cup E_4$ and
$\phi_E\restr E_3=R\restr E_3$ is a measure space isomorphism for the
subspace measures on $E_3$ and $E_4$.

\medskip

{\bf (c)} By (a),

\Centerline{$\int_E\|x\|\nu(dx)
\ge\int_E\|\phi_E(x)\|\nu(dx)
=\int_{\psi(E)}\|x\|\nu(dx)$}

\noindent by 235Gc, because $\phi_E:E\to\psi(E)$ is \imp, with equality
only when

\Centerline{$\{x:\|x\|>\|\phi_E(x)\|\}
=\{x:x\in E,\,\innerprod{x}{e}<\alpha,\,R(x)\notin E\}$}

\noindent is negligible.

\medskip

{\bf (d)} Note first that if $\Lambda$ is the domain of $\lambda$ then
$\Lambda$ includes the Borel algebra of $X\times X$ (because $X$ is
second-countable, so this is just the $\sigma$-algebra generated by
products of Borel sets, by 4A3G);  so that $(x,y)\mapsto\|x-y\|$ is
$\Lambda$-measurable, and the integrals are defined in $[0,\infty]$.   Now
consider the sets

\Centerline{$W_1=\{(x,y):x\in E,\,y\in E,\,R(x)\notin E,\,
R(y)\in E,\,\innerprod{x}{e}<\alpha,\,\innerprod{y}{e}<\alpha\}$,}

\Centerline{$W'_1=\{(x,y):x\in E,\,y\in E,\,R(x)\notin E,\,
R(y)\in E,\,\innerprod{x}{e}<\alpha,\,\innerprod{y}{e}>\alpha\}$,}

\Centerline{$W_2=\{(x,y):x\in E,\,y\in E,\,R(x)\in E,\,
R(y)\notin E,\,\innerprod{x}{e}<\alpha,\,\innerprod{y}{e}<\alpha\}$,}

\Centerline{$W'_2=\{(x,y):x\in E,\,y\in E,\,R(x)\in E,\,
R(y)\notin E,\,\innerprod{x}{e}>\alpha,\,\innerprod{y}{e}<\alpha\}$.}

\noindent Then $(x,y)\mapsto(x,R(y)):W'_1\to W_1$ is a measure space
isomorphism for the subspace measures induced on $W_1$ and $W'_1$ by
$\lambda$, so

$$\eqalign{\int_{W_1}\|\phi_E(x)-\phi_E(y)\|\lambda(d(x,y))
&=\int_{W_1}\|R(x)-y\|\lambda(d(x,y))\cr
&=\int_{W'_1}\|R(x)-R(y)\|\lambda(d(x,y))\cr
&=\int_{W'_1}\|x-y\|\lambda(d(x,y)).\cr}$$

\noindent Similarly,

$$\eqalign{\int_{W'_1}\|\phi_E(x)-\phi_E(y)\|\lambda(d(x,y))
&=\int_{W'_1}\|R(x)-y\|\lambda(d(x,y))\cr
&=\int_{W_1}\|R(x)-R(y)\|\lambda(d(x,y))\cr
&=\int_{W_1}\|x-y\|\lambda(d(x,y)).\cr}$$

\noindent So we get

\Centerline{$\int_{W_1\cup W'_1}
  \|\phi_E(x)-\phi_E(y)\|\lambda(d(x,y))
=\int_{W_1\cup W'_1}\|x-y\|\lambda(d(x,y))$.}

\noindent In the same way, $(x,y)\mapsto(R(x),y)$ is an isomorphism of
the subspace measures on $W_2$ and $W'_2$, and we have

\Centerline{$\int_{W_2\cup W'_2}
  \|\phi_E(x)-\phi_E(y)\|\lambda(d(x,y))
=\int_{W_2\cup W'_2}\|x-y\|\lambda(d(x,y))$.}

\noindent On the other hand, for all
$(x,y)\in(E\times E)\setminus(W_1\cup W'_1\cup W_2\cup W'_2)$, we have
$\|\phi_E(x)-\phi_E(y)\|\le\|x-y\|$.   (Either $x$ and $y$ are both left
fixed by $\phi$, or both are moved, or one is on the reflecting
hyperplane, or one is moved to the same side of the reflecting
hyperplane as the other.)   So we get

$$\eqalign{\int_{E\times E}\|x-y\|\lambda(d(x,y))
&\ge\int_{E\times E}\|\phi_E(x)-\phi_E(y)\|\lambda(d(x,y))\cr
&=\int_{\psi(E)\times\psi(E)}\|x-y\|\lambda(d(x,y))\cr}$$

\noindent because $(x,y)\mapsto(\phi_E(x),\phi_E(y))$ is an \imp\
transformation for the subspace measures on $E\times E$ and
$\psi(E)\times\psi(E)$.

\medskip

{\bf (e)(i)} Because $R$ is both an isometry and a
measure space automorphism, $\clstar R[A]=R[\clstar A]$ and
$\intstar R[A]=R[\intstar A]$, where $\clstar A$ and $\intstar A$ are
the essential closure and the essential interior of $A$, as in 475B.
Recall that $\clstar A$, $\intstar A$ and $\partstar A$ are all Borel
sets (475Cc), so that $\partstar A$ and $\partstar\psi(A)$ are
measurable.

\medskip

\quad{\bf (ii)} Suppose that $\varinnerprod{x}{e}=\alpha$.   Then
$x\in\partstar\psi(A)$
iff $x\in\partstar A$.   \Prf\ It is easy to check that
$B(x,\delta)\cap\psi(A)=\psi(B(x,\delta)\cap A)$ for any $\delta>0$, so
that $\nu^*(B(x,\delta)\cap\psi(A))=\nu^*(B(x,\delta)\cap A)$ for
every $\delta>0$ and $x\in\clstar A$ iff $x\in\clstar\psi(A)$.   If
$x=R(x)\in\intstar A$, then $x\in\intstar R[A]$ so
$x\in\intstar(A\cap R[A])$ (475Cd) and $x\in\intstar\psi(A)$.   If
$x\in\intstar\psi(A)$ then
$x\in\intstar(R[\psi(A)]\cap\psi(A))\subseteq\intstar A$.\ \Qed

\medskip

\quad{\bf (iii)} If $x\in\partstar\psi(A)\setminus\partstar A$ then
$R(x)\in\partstar A\setminus\partstar\psi(A)$.   \Prf\ By (ii),
$\varinnerprod{x}{e}\ne\alpha$.

\qquad{\bf case 1} Suppose that $\varinnerprod{x}{e}>\alpha$.   Setting
$\delta=\varinnerprod{x}{e}-\alpha$, we see that
$U(x,\delta)\cap\psi(A)=U(x,\delta)\cap(A\cup R[A])$, while
$U(R(x),\delta)\cap\psi(A)=U(R(x),\delta)\cap A\cap R[A]$.   Since
$x\notin\intstar\psi(A)$, $x\notin\intstar(A\cup R[A])$ and
$x\notin\intstar A$;  since $x$ also does not belong to $\partstar A$,
$x\notin\clstar A$.   However,
$x\in\clstar(A\cup R[A])=\clstar A\cup\clstar R[A]$ (475Cd), so
$x\in\clstar R[A]$ and $R(x)\in\clstar A$.   Next,
$x\notin\intstar R[A]$, so $R(x)\notin\intstar A$ and
$R(x)\in\partstar A$.   Since $x\notin\clstar A$,
$R(x)\notin\clstar R[A]$ and $R(x)\notin\clstar\psi(A)$;  so
$R(x)\in\partstar A\setminus\partstar\psi(A)$.

\qquad{\bf case 2} Suppose that $\varinnerprod{x}{e}<\alpha$.    This
time, set $\delta=\alpha-\varinnerprod{x}{e}$, so that
$U(x,\delta)\cap\psi(A)=U(x,\delta)\cap A\cap R[A]$ and
$U(R(x),\delta)\cap\psi(A)=U(R(x),\delta)\cap(A\cup R[A])$.   As
$x\in\clstar\psi(A)$, $x\in\clstar(A\cap R[A])$ and $R(x)\in\clstar A$.
Also $x\in\clstar A$;  as $x\notin\partstar A$, $x\in\intstar A$,
$R(x)\in\intstar R[A]$ and $R(x)\in\intstar\psi(A)$, so that
$R(x)\notin\partstar\psi(A)$.   Finally, we know that $x\in\intstar A$
but $x\notin\intstar(A\cap R[A])$ (because $x\notin\intstar\psi(A)$;  it
follows that $x\notin\intstar R[A]$ so $R(x)\notin\intstar A$ and
$R(x)\in\partstar A\setminus\partstar\psi(A)$.

Thus all possibilities are covered and we have the result.\ \Qed

\medskip

\quad{\bf (iv)} What this means is that if we set
$E=\partstar\psi(A)\setminus\partstar A$
then $R[E]\subseteq\partstar A\setminus\partstar\psi(A)$.   So

\Centerline{$\nu\partstar\psi(A)
=\nu E+\nu(\partstar\psi(A)\cap\partstar A)
=\nu R[E]+\nu(\partstar\psi(A)\cap\partstar A)
\le\nu\partstar A$,}

\noindent as required in (e).   This ends the proof of the lemma.
}%end of proof of 476E

\leader{476F}{Theorem} Let $r\ge 1$ be an integer, and let $\mu$ be
Lebesgue measure on $\BbbR^r$.   For non-empty $A\subseteq\BbbR^r$ and
$\epsilon>0$, write $U(A,\epsilon)$ for $\{x:\rho(x,A)<\epsilon\}$,
where $\rho$ is the Euclidean metric on $\BbbR^r$.   If $\mu^*A$ is
finite, then $\mu U(A,\epsilon)\ge\mu U(B_A,\epsilon)$, where $B_A$ is
the closed ball with centre $\tbf{0}$ and measure $\mu^*A$.

\proof{{\bf (a)} To begin with, suppose that $A$ is bounded.   Set
$\gamma=\mu^*A$ and $\beta=\mu U(A,\epsilon)$.   If $\gamma=0$ then
(because $A\ne\emptyset$)

\Centerline{$\mu U(A,\epsilon)\ge\mu U(\{\tbf{0}\},\epsilon)
=\mu U(B_A,\epsilon)$,}

\noindent and we can stop.   So let us suppose henceforth that
$\gamma>0$.   Let $M\ge 0$ be such that $A\subseteq B(\tbf{0},M)$, and
consider the family

\Centerline{$\Cal F
=\{F:F\in\Cal C$, $F\subseteq B(\tbf{0},M)$,
$\mu F\ge\gamma$, $\mu U(F,\epsilon)\le\beta\}$,}

\noindent where $\Cal C$ is the family of closed subsets of
$\BbbR^r$ with its Fell topology.   Because
$U(\overline{A},\epsilon)=U(A,\epsilon)$,
$\overline{A}\in\Cal F$ and $\Cal F$ is non-empty.   By the definition of
the Fell topology, $\{F:F\subseteq B(\tbf{0},M)\}$ is closed;
by 476A(b-ii) (applied to the functional
$F\mapsto\int_F\chi B(\tbf{0},M)d\mu$)
and 476B, $\Cal F$ is closed in $\Cal C$, therefore compact,
by 4A2T(b-iii).   Next, the function

\Centerline{$F\mapsto\int_F\max(0,M-\|x\|)\mu(dx):
\Cal C\to\coint{0,\infty}$}

\noindent is upper semi-continuous, by 476A(b-ii) again.
It therefore attains its
supremum on $\Cal F$ at some $F_0\in\Cal F$ (4A2Gl).   Let
$F_1\subseteq F_0$ be a closed self-supporting set of the same measure
as $F_0$;  then $U(F_1,\epsilon)\subseteq U(F_0,\epsilon)$ and
$\mu F_1=\mu F_0$, so $F_1\in\Cal F$;  also

\Centerline{$\int_{F_1}M-\|x\|\mu(dx)
=\int_{F_0}M-\|x\|\mu(dx)
\ge\int_{F}M-\|x\|\mu(dx)$}

\noindent for every $F\in\Cal F$.

\medskip

{\bf (b)} Now $F_1$ is a ball with centre $\tbf{0}$.   \Prf\Quer\
Suppose, if possible, otherwise.   Then there are $x_1\in F_1$ and
$x_0\in\BbbR^r\setminus F_1$ such that $\|x_0\|<\|x_1\|$.
Set $e=\Bover1{\|x_0-x_1\|}(x_0-x_1)$, so that $e$ is a unit vector, and
$\alpha=\Bover12\varinnerprod{e}{(x_0+x_1)}$;  then

\Centerline{$\alpha
=\Bover1{2\|x_0-x_1\|}\varinnerprod{(x_0-x_1)}{(x_0+x_1)}
=\Bover1{2\|x_0-x_1\|}(\|x_0\|^2-\|x_1\|^2)<0$.}

\noindent Define $R=R_{e\alpha}:\BbbR^r\to\BbbR^r$ and
$\psi=\psi_{e\alpha}$ as in 476D.   Set $F=\psi(F_1)$.   Then $F$ is
closed (476Db) and $\mu F=\mu F_1\ge\mu^*A$ (476Eb).   Also
$U(F,\epsilon)\subseteq\psi(U(F_1,\epsilon))$ (476Da), so

\Centerline{$\mu U(F,\epsilon)\le\mu(\psi(U(F_1,\epsilon))
=\mu U(F_1,\epsilon)\le\beta$}

\noindent and $F\in\Cal F$.   It follows that
$\int_FM-\|x\|\mu(dx)\le\int_{F_1}M-\|x\|\mu(dx)$;  as $\mu F=\mu F_1$,
$\int_F\|x\|\mu(dx)\ge\int_{F_1}\|x\|\mu(dx)$.   By 476Ec,
$G=\{x:x\in F_1,\,\varinnerprod{x}{e}<\alpha,\,R(x)\notin F_1\}$ is
negligible.   But $G$ contains $x_1$ and is relatively open in $F_1$,
and $F_1$ is supposed to be self-supporting;  so this is impossible.\
\Bang\Qed

\medskip

{\bf (c)} Since $\mu F_1\ge\gamma$, $F_1\supseteq B_A$, and

\Centerline{$\mu U(B_A,\epsilon)\le\mu U(F_1,\epsilon)\le\beta
=\mu U(A,\epsilon)$.}

\noindent So we have the required result for bounded $A$.   In general,
given an unbounded set $A$ of finite measure, let $\delta$ be the radius
of $B_A$;  then

$$\eqalign{\mu U(B_A,\epsilon)
&=\mu B(\tbf{0},\delta+\epsilon)
=\sup_{\alpha<\delta}\mu B(\tbf{0},\alpha+\epsilon)\cr
&\le\sup_{A'\subseteq A\text{ is bounded}}\mu U(B_{A'},\epsilon)
\le\sup_{A'\subseteq A\text{ is bounded}}\mu U(A',\epsilon)
=\mu U(A,\epsilon)\cr}$$

\noindent because $\{U(A',\epsilon):A'\subseteq A$ is bounded$\}$ is an
upwards-directed family of open sets with union $U(A,\epsilon)$, and
$\mu$ is $\tau$-additive.   So the theorem is true for unbounded $A$ as
well.
}%end of proof of 476F

\leader{476G}{Theorem} Let $r\ge 1$ be an integer, and let $\mu$ be
Lebesgue measure on $\BbbR^r$;  write $\lambda$ for the product measure
on $\BbbR^r\times\BbbR^r$.   For any measurable set $E\subseteq\BbbR^r$
of finite measure, write $B_E$ for the closed ball with centre $\tbf{0}$
and the same measure as $E$.   Then

\Centerline{$\int_{E\times E}\|x-y\|\lambda(d(x,y))
\ge\int_{B_E\times B_E}\|x-y\|\lambda(d(x,y))$.}

\proof{{\bf (a)} Suppose for the moment that $E$ is compact and not
empty, and that $\epsilon>0$.   Let $M\ge 0$ be such that $\|x\|\le M$
for every $x\in E$.   For a non-empty set $A\subseteq\BbbR^r$ set
$U(A,\epsilon)=\{x:\rho(x,A)<\epsilon\}$, where $\rho$ is Euclidean
distance on $\BbbR^r$.   Set
$\beta=\int_{U(E,\epsilon)\times U(E,\epsilon)}\|x-y\|\lambda(d(x,y))$.
Let $\Cal F$ be the family of non-empty closed subsets $F$ of the ball
$B(\tbf{0},M)=\{x:\|x\|\le M\}$ such that $\mu F\ge\mu E$ and
$\int_{U(F,\epsilon)\times U(F,\epsilon)}
\|x-y\|\lambda(d(x,y))\le\beta$.
Then $\Cal F$ is compact for the Fell topology on the family
$\Cal C$ of closed subsets of $\BbbR^r$.   \Prf\ We know from 4A2T(b-iii)
that $\Cal C$ is compact, and from 476A(b-ii) that
$\{F:\int_F\chi B(\tbf{0},M)d\mu\ge\mu E\}$ is closed;
also $\{F:F\subseteq B(\tbf{0},M)\}$ is closed.
Let $\sigma$ be the metric on $\BbbR^r\times\BbbR^r$
defined by setting $\sigma((x,y),(x',y'))=\max(\|x-x'\|,\|y-y'\|)$, and
$\nu$ the indefinite-integral measure over $\lambda$ defined by
the function $(x,y)\mapsto\|x-y\|$.   Then

\Centerline{$U(F,\epsilon)\times U(F,\epsilon)
=\{(x,y):\sigma((x,y),F\times F)<\epsilon\}=U(F\times F,\epsilon;\sigma)$}

\noindent for $F\in\Cal C$ and $\epsilon>0$.   Now, writing $\Cal C_2$ for
the family of closed sets in $\BbbR^r\times\BbbR^r$ with its Fell topology,
we know that

\inset{$F\mapsto F\times F:\Cal C\to\Cal C_2$
is continuous, by 4A2Td,

$E\mapsto\nu U(E,\epsilon;\sigma):\Cal C_2\to\Bbb R$ is
lower semi-continuous, by 476B;}

\noindent so $F\mapsto\nu(U(F,\epsilon)\times U(F,\epsilon))$ is
lower semi-continuous, and $\{F:
\int_{U(F,\epsilon)\times U(F,\epsilon)}\|x-y\|\lambda(d(x,y))\le\beta\}$
is closed.   Putting these together, $\Cal F$ is a closed subset of
$\Cal C$ and is compact.\ \Qed

\medskip

{\bf (b)} Since $E\in\Cal F$, $\Cal F$ is not empty.   By 476A(b-ii),
there is an $F_0\in\Cal F$ such that
$\int_{F_0}M-\|x\|\mu(dx)\ge\int_FM-\|x\|\mu(dx)$ for every
$F\in\Cal F$.   Let $F_1\subseteq F_0$ be a closed self-supporting set
of the same measure;  then $U(F_1,\epsilon)\subseteq U(F_0,\epsilon)$,
so $\int_{U(F_1,\epsilon)\times U(F_1,\epsilon)}\|x-y\|\lambda(d(x,y))
\le\beta$ and $F_1\in\Cal F$;  also

\Centerline{$\int_{F_1}M-\|x\|\mu(dx)
=\int_{F_0}M-\|x\|\mu(dx)\ge\int_FM-\|x\|\mu(dx)$}

\noindent for every $F\in\Cal F$.

Now $F_1$ is a ball with centre $\tbf{0}$.   \Prf\Quer\ Suppose, if
possible, otherwise.   Then (just as in the proof of 476G) there are
$x_1\in F_1$ and $x_0\in\BbbR^r\setminus F_1$ such that
$\|x_1\|>\|x_0\|$.   Set $e=\Bover1{\|x_0-x_1\|}(x_0-x_1)$ and
$\alpha=\Bover12\varinnerprod{e}{(x_0+x_1)}<0$.
Define $R=R_{e\alpha}:\BbbR^r\to\BbbR^r$ and $\psi=\psi_{e\alpha}$ as in
476D.   Set $F=\psi(F_1)$.   Then $F$ is closed and
$\mu F=\mu F_1\ge\mu E$ and
$U(F,\epsilon)\subseteq\psi(U(F_1,\epsilon))$.   So

$$\eqalignno{\int_{U(F,\epsilon)\times U(F,\epsilon)}
  \|x-y\|\lambda(d(x,y))
&\le\int_{\psi(U(F_1,\epsilon))\times\psi(U(F_1,\epsilon))}
  \|x-y\|\lambda(d(x,y))\cr
&\le\int_{U(F_1,\epsilon)\times U(F_1,\epsilon)}
  \|x-y\|\lambda(d(x,y))\cr
\displaycause{476Ed}
&\le\beta.\cr}$$

\noindent This means that $F\in\Cal F$.   Accordingly
$\int_{F_1}M-\|x\|\mu(dx)\ge\int_FM-\|x\|\mu(dx)$;  since
$\mu F=\mu F_1$, $\int_{F_1}\|x\|\mu(dx)\le\int_F\|x\|\mu(dx)$.
By 476Ec,
$G=\{x:x\in F_1,\,\varinnerprod{x}{e}<\alpha,\,R(x)\notin F_1\}$
must be negligible.   But $G$ contains $x_1$ and is relatively open in
$F_1$, and $F_1$ is supposed to be self-supporting;  so this is
impossible.\ \Bang\Qed

\medskip

{\bf (c)} Since $\mu F_1\ge\mu E$, $F_1\supseteq B_E$, and

$$\eqalign{\int_{B_E\times B_E}\|x-y\|\lambda(d(x,y))
&\le\int_{U(F_1,\epsilon)\times U(F_1,\epsilon)}
  \|x-y\|\lambda(d(x,y))\cr
&\le\beta
=\int_{U(E,\epsilon)\times U(E,\epsilon)}
  \|x-y\|\lambda(d(x,y)).\cr}$$

\noindent At this point, recall that $\epsilon$ was arbitrary.   Since
$E$ is compact,

$$\eqalign{\int_{E\times E}\|x-y\|\lambda(d(x,y))
&=\inf_{\epsilon>0}\int_{U(E,\epsilon)\times U(E,\epsilon)}
  \|x-y\|\lambda(d(x,y))\cr
&\ge\int_{B_E\times B_E}\|x-y\|\lambda(d(x,y)).\cr}$$

\medskip

{\bf (d)} Thus the result is proved for non-empty compact sets $E$.   In
general, given a measurable set $E$ of finite measure, then if $E$ is
negligible the result is trivial;  and otherwise, writing $\delta$ for
the radius of $B_E$,

$$\eqalign{\int_{B_E\times B_E}\|x-y\|\lambda(d(x,y))
&=\sup_{\alpha<\delta}\int_{B(\tbf{0},\alpha)\times B(\tbf{0},\alpha)}
  \|x-y\|\lambda(d(x,y))\cr
&\le\sup_{K\subseteq E\text{ is compact}}
  \int_{B_K\times B_K}\|x-y\|\lambda(d(x,y))\cr
&\le\sup_{K\subseteq E\text{ is compact}}
  \int_{K\times K}\|x-y\|\lambda(d(x,y))\cr
&=\int_{E\times E}\|x-y\|\lambda(d(x,y)),\cr}$$

\noindent so the proof is complete.
}%end of proof of 476G

\leader{476H}{The Isoperimetric Theorem} Let $r\ge 1$ be an integer, and
let $\mu$ be Lebesgue measure on $\BbbR^r$.   If $E\subseteq\BbbR^r$ is
a measurable set of finite measure, then $\per E\ge\per B_E$, where
$B_E$ is the closed ball with centre $\tbf{0}$ and the same measure as
$E$\cmmnt{, while $\per E$ is the perimeter of $E$ as defined in
474D}.

\proof{{\bf (a)} Suppose to begin with that $E\subseteq B(\tbf{0},M)$,
where $M\ge 0$, and that $\per E<\infty$.   Let $\Cal F$ be the family
of measurable sets $F\subseteq\BbbR^r$ such that
$F\setminus B(\tbf{0},M)$ is negligible, $\mu F\ge\mu E$ and
$\per F\le\per E$, with
the topology of convergence in measure (474T).   Then

$$\eqalign{\Cal F
&=\{F:\per F\le\per E,\,\mu(F\cap B(\tbf{0},M))\ge\mu E,\cr
&\hskip10em
\mu(F\cap B(\tbf{0},\alpha))\le\mu(F\cap B(\tbf{0},M))
  \text{ for every }\alpha\ge 0\}\cr}$$

\noindent is a closed subset of $\{F:\per F\le\per E\}$, which is
compact (474Tb), so $\Cal F$ is compact.   For $F\in\Cal F$ set
$h(F)=\int_F\|x\|\mu(dx)$;  then
$|h(F)-h(F')|\le M\mu((F\symmdiff F')\cap B(\tbf{0},M))$ for all $F$,
$F'\in\Cal F$, so $h$ is continuous.   There is therefore an
$F_0\in\Cal F$ such that $h(F_0)\le h(F)$ for every $F\in\Cal F$.   Set
$F_1=\clstar F_0$, so that $F_1\symmdiff F_0$ is negligible (475Cg),
$\per F_1=\per F_0$ (474F),
$F_1\in\Cal F$, $F_1\subseteq B(\tbf{0},M)$ and $h(F_1)=h(F_0)$.

\medskip

{\bf (b)} Writing $\delta=\sup_{x\in F_1}\|x\|$, we have
$U(\tbf{0},\delta)\subseteq F_1$.   \Prf\Quer\ Otherwise, there are
$x_0\in\BbbR^r\setminus F_1$ and $x_1\in F_1$ such that
$\|x_0\|<\|x_1\|$.   Set $e=\Bover1{\|x_0-x_1\|}(x_0-x_1)$,
$\alpha=\Bover12\varinnerprod{e}{(x_0+x_1)}<0$,
$R=R_{e\alpha}:\BbbR^r\to\BbbR^r$ and $\psi=\psi_{e\alpha}$.
Set $F=\psi(F_1)$ and let $\phi=\phi_{F_1}:F_1\to F$ be the function
described in 476E.   Then $\|\phi(x)\|\le\|x\|$ for every $x\in F_1$
(476Ea).   In particular, $F=\phi[F_1]\subseteq B(\tbf{0},M)$.   Now
$F$ is measurable and $\mu F=\mu F_1\ge\mu E$, by 476Eb.   Also, writing
$\nu$ for normalized $(r-1)$-dimensional Hausdorff measure on $\BbbR^r$,

\Centerline{$\per F=\nu(\partstar F)\le\nu(\partstar F_1)
=\per F_1\le\per E$,}

\noindent by 475Mb and 476Ee.   So $F\in\Cal F$, and

\Centerline{$\int_F\|x\|\mu(dx)\ge\int_{F_0}\|x\|\mu(dx)
=\int_{F_1}\|x\|\mu(dx)$.}

\noindent By 476Ec,
$G=\{x:x\in F_1,\,\varinnerprod{x}{e}<\alpha,\,R(x)\notin F_1\}$ is
negligible.   But now consider $G\cap U(x_1,\eta)$ for small $\eta>0$.
Since $x_1$ belongs to $F_1=\clstar F_0=\clstar F_1$, but $x_0$ does
not,

\Centerline{$\limsup_{\eta\downarrow 0}
\Bover{\mu(F_1\cap B(x_1,\eta))}{\mu B(x_1,\eta)}
>0
=\limsup_{\eta\downarrow 0}
\Bover{\mu(F_1\cap B(x_0,\eta))}{\mu B(x_0,\eta)}$.}

\noindent There must therefore be some $\eta>0$ such that
$\eta<\bover12\|x_1-x_0\|$ and
$\mu(F_1\cap B(x_0,\eta))<\mu(F_1\cap B(x_1,\eta))$.   In this case,
however,
$G\supseteq F_1\cap B(x_1,\eta)\setminus R[F_1\cap B(x_0,\eta)]$ has
measure at least $\mu(F_1\cap B(x_1,\eta))-\mu(F_1\cap B(x_0,\eta))>0$,
which is impossible.\ \Bang\Qed

\medskip

{\bf (c)} Thus
$U(\tbf{0},\delta)\subseteq F_1\subseteq B(\tbf{0},\delta)$ and
$\per F_1=\per B(\tbf{0},\delta)$.   Since $\mu F_1\ge\mu E$, the radius
of $B_E$ is at most $\delta$ and

\Centerline{$\per B_E\le\per B(\tbf{0},\delta)=\per F_1\le\per E$.}

\medskip

{\bf (d)} Thus the result is proved when $E$ is bounded and has finite
perimeter.   Of course it is trivial when $E$ has infinite perimeter.
Now suppose that $E$ is any measurable set with finite measure and
finite perimeter.   Set $E_{\alpha}=E\cap B(\tbf{0},\alpha)$ for
$\alpha\ge 0$;  then
$\per E=\liminf_{\alpha\to\infty}\per E_{\alpha}$ (475Mc, 475Xk).
%probably lim_{\alpha\to\infty}
By (a)-(c), $\per E_{\alpha}\ge\per B_{E_{\alpha}}$;  since
$\per B_{E_{\alpha}}\to\per B_E$ as $\alpha\to\infty$,
$\per E\ge\per B_E$ in this case also.
}%end of proof of 476H

\leader{476I}{Spheres in inner product spaces} For the rest of the
section I will use the following notation.
Let $X$ be a (real) inner product space.\cmmnt{ Then} $S_X$ will be
the unit
sphere $\{x:x\in X,\,\|x\|=1\}$.   Let $H_X$ be the isometry group of
$S_X$ with its
topology of pointwise convergence\cmmnt{ (441G)}.

A {\bf cap} in $S_X$ will be a set of the form
$\{x:x\in S_X,\,\innerprod{x}{e}\ge\alpha\}$ where $e\in S_X$ and
$-1\le\alpha\le 1$.

When $X$ is finite-dimensional, it is isomorphic\cmmnt{, as inner
product
space,} to $\BbbR^r$, where $r=\dim X$\cmmnt{ (4A4Je)}.   If $r\ge 1$,
$S_X$\cmmnt{ is non-empty and
compact, so} has a unique $H_X$-invariant Radon probability measure
$\nu_X$, which is strictly positive\cmmnt{ (476C)}.   If $r\ge 1$ is an
integer, \cmmnt{we know that} the $(r-1)$-dimensional Hausdorff
measure of the
sphere $S_{\Bbb R^r}$ is finite and non-zero\cmmnt{ (265F)}.
\cmmnt{Since Hausdorff
measures are invariant under isometries (264G, 471J), and are
quasi-Radon measures when totally finite (471Dh),} $(r-1)$-dimensional
Hausdorff
measure on $S_{\Bbb R^r}$ is a multiple of the normalized invariant
measure $\nu_{\Bbb R^r}$\cmmnt{, by 476C}.   The same is therefore true
in any
$r$-dimensional inner product space.

\leader{476J}{Lemma} Let $X$ be a real inner product space and
$f\in H_X$.
Then $\innerprod{f(x)}{f(y)}=\innerprod{x}{y}$ for all $x$, $y\in S_X$.
\cmmnt{Consequently} $f(\alpha x+\beta y)=\alpha f(x)+\beta f(y)$
whenever $x$, $y\in S_X$ and $\alpha$, $\beta\in\Bbb R$ are such that
$\alpha x+\beta y\in S_X$.

\proof{{\bf (a)} We have

\Centerline{$\rho(x,y)^2=\innerprod{x-y}{x-y}
=\innerprod{x}{x}-2\innerprod{x}{y}+\innerprod{y}{y}
=2-2\innerprod{x}{y}$,}

\noindent so

\Centerline{$\innerprod{x}{y}=1-\Bover12\rho(x,y)^2
=1-\Bover12\rho(f(x),f(y))^2=\innerprod{f(x)}{f(y)}$.}

\wheader{476J}{6}{2}{2}{72pt}

{\bf (b)}

$$\eqalign{\|f(\alpha x&+\beta y)-\alpha f(x)-\beta f(y)\|^2\cr
&=\innerprod{f(\alpha x+\beta y)-\alpha f(x)-\beta f(y)}
   {f(\alpha x+\beta y)-\alpha f(x)-\beta f(y)}\cr
&=1+\alpha^2+\beta^2-2\alpha\innerprod{f(\alpha x+\beta y)}{f(x)}
  -2\beta\innerprod{f(\alpha x+\beta y)}{f(y)}
  +2\alpha\beta\innerprod{f(x)}{f(y)}\cr
&=1+\alpha^2+\beta^2-2\alpha\innerprod{\alpha x+\beta y}{x}
  -2\beta\innerprod{\alpha x+\beta y}{y}
  +2\alpha\beta\innerprod{x}{y}\cr
&=\|(\alpha x+\beta y)-\alpha x-\beta y\|^2
=0.\cr}$$

}%end of proof of 476J

\leader{476K}{}\cmmnt{ I give a theorem on concentration of measure on
the sphere corresponding to 476F.

\medskip

\noindent}{\bf Theorem} Let $X$ be a finite-dimensional inner product
space of dimension at least $2$, $S_X$ its unit sphere and $\nu_X$ the
invariant Radon probability measure on $S_X$.   For a non-empty set
$A\subseteq S_X$ and $\epsilon>0$, write
$U(A,\epsilon)=\{x:\rho(x,A)<\epsilon\}$, where $\rho$ is
the usual metric of $X$.   Then there is a cap $C\subseteq S_X$ such
that $\nu_XC=\nu_X^*A$, and
$\nu_X(S_X\cap U(A,\epsilon))\ge\nu_X(S_X\cap U(C,\epsilon))$
for any such $C$.

\proof{ In order to apply the results of 476D-476E directly, and
simplify some of the formulae slightly, it will be helpful to write
$\nu$ for the Radon measure on $X$ defined by setting
$\nu E=\nu_X(E\cap S_X)$ whenever this is defined.   By 214Cd, $\nu^*$
agrees with $\nu_X^*$ on $\Cal PS_X$.

\medskip

{\bf (a)} The first step is to check that there is a cap $C$ of $S_X$
such that $\nu C=\nu^*A$.   \Prf\ Take any $e_0\in S_X$, and set
$C_{\alpha}=\{x:x\in S_X,\,\innerprod{x}{e_0}\ge\alpha\}$ for
$\alpha\in[-1,1]$.
$\nu C_{\alpha}$ is defined for every $\alpha\in\Bbb R$
because every $C_{\alpha}$ is closed and $\nu$ is a topological
measure.   Now examine the formulae of 265F.   We can identify $X$
with $\BbbR^{r+1}$ where $r+1=\dim X$;  do this in such a way that $e_0$
corresponds to the unit vector $(0,\ldots,0,1)$.   We have a
parametrization
$\phi_r:D_r\to S_X$, where $D_r$ is a Borel subset of $\BbbR^r$ with
interior $\ooint{-\pi,\pi}\times\ooint{0,\pi}^{r-1}$ and $\phi_r$ is
differentiable with continuous derivative.   Moreover, if
$x=(\xi_1,\ldots,\xi_r)\in D_r$, then
$\varinnerprod{\phi_r(x)}{e_0}=\cos\xi_r$, and the Jacobian $J_r$ of
$\phi_r$ is bounded by $1$ and never zero on $\interior D_r$.   Finally,
the boundary $\partial D_r$ is negligible.   What this means is that
$\nu_rC_{\alpha}=\int_{E_{\alpha}}J_rd\mu_r$, where $\mu_r$ is Lebesgue
measure on $\BbbR^r$, $\nu_r$ is normalized Hausdorff $r$-dimensional
measure on $\BbbR^{r+1}$, and
$E_{\alpha}=\{x:x\in D_r,\,\cos\xi_r\ge\alpha\}$.   So if
$-1\le\alpha\le\beta\le 1$ then

\Centerline{$\nu_rC_{\alpha}-\nu_rC_{\beta}
\le\mu_r(E_{\alpha}\setminus E_{\beta})
\le 2\pi^{r-1}(\arccos\alpha-\arccos\beta)$;}

\noindent because $\arccos$ is continuous, so is
$\alpha\mapsto\nu_rC_{\alpha}$.   Also, if $\alpha<\beta$, then
$E_{\alpha}\setminus E_{\beta}$ is non-negligible, so
$\int_{E_{\alpha}\setminus E_{\beta}}J_rd\mu_r\ne 0$ and
$\nu_rC_{\alpha}>\nu_rC_{\beta}$.

This shows that $\alpha\mapsto\nu_rC_{\alpha}$ is continuous and
strictly decreasing;  since $\nu_r$ is just a multiple of $\nu$ on
$S_X$, the same is true of $\alpha\mapsto\nu C_{\alpha}$.

Since $\nu C_{-1}=\nu S_X=1$ and $\nu C_1=\nu\{e_0\}=0$, the
Intermediate
Value Theorem tells us that there is a unique $\alpha$ such that
$\nu C_{\alpha}=\nu^*A$, and we can set $C=C_{\alpha}$.\ \Qed

\medskip

{\bf (b)} Now take any non-empty set $A\subseteq S_X$ and any
$\epsilon>0$, and set $\gamma=\nu^*A$, $\beta=\nu U(A,\epsilon)$.
Let $C$ be a cap such that $\nu^*A=\nu C$;  let $e_0$ be the centre of
$C$.   Consider the family

\Centerline{$\Cal F
=\{F:F\in\Cal C$, $F\subseteq S_X$,
$\nu F\ge\gamma$, $\nu U(F,\epsilon)\le\beta\}$,}

\noindent where $\Cal C$ is the family of closed subsets of $X$ with its
Fell topology.
Because $U(\overline{A},\epsilon)=U(A,\epsilon)$,
$\overline{A}\in\Cal F$ and $\Cal F$ is non-empty.   By
476A(b-i) and 476B, $\Cal F$ is closed in
$\Cal C$, therefore compact, by 4A2T(b-iii) as usual.   Next, the function

\Centerline{$F\mapsto\int_F\max(0,1+\innerprod{x}{e_0})\nu(dx):
\Cal C\to\coint{0,\infty}$}

\noindent is upper semi-continuous, by 476A(b-ii).
It therefore attains its
supremum on $\Cal F$ at some $F_0\in\Cal F$.   Let $F_1\subseteq F_0$ be
a self-supporting closed set with the same measure as $F_0$;  then
$F_1\in\Cal F$ and $\int_F(1+\innerprod{x}{e_0})\nu (dx)
\ge\int_{F_1}(1+\innerprod{x}{e_0})\nu(dx)$ for every $F\in\Cal F$.

\medskip

{\bf (c)} $F_1$ is a cap with centre $e_0$.   \Prf\Quer\ Otherwise,
there are $x_0\in S_X\setminus F_1$ and
$x_1\in F_1$ such that $\innerprod{x_0}{e_0}>\innerprod{x_1}{e_0}$.
Set $e=\Bover1{\|x_0-x_1\|}(x_0-x_1)$.   Then $e\in S_X$ and
$\innerprod{e}{e_0}>0$.   Set $R=R_{e0}$ and $\psi=\psi_{e0}$ as defined
in 476D;  write $F$ for $\psi(F_1)$.   Note that
$\innerprod{x_0+x_1}{x_0-x_1}=\|x_0\|^2-\|x_1\|^2=0$, so
$\innerprod{x_0+x_1}{e}=0$ and $R(x_0)=x_1$.   Also $R[S_X]=S_X$, so
$\nu$ is $R$-invariant, because $\nu$ is a multiple of Hausdorff
$(r-1)$-dimensional measure on $S_X$ and must be invariant under
isometries of $S_X$.

We have $\nu F=\nu F_1\ge\gamma$, by 476Eb, and

\Centerline{$\nu U(F,\epsilon)\le\nu\psi(U(F_1,\epsilon))
=\nu U(F_1,\epsilon)\le\nu U(F_0,\epsilon)\le\beta$}

\noindent by 476Da.   So $F\in\Cal F$.   But consider the standard
bijection $\phi=\phi_{F_1}:F_1\to F$ as defined in 476E.
We have

\Centerline{$\int_{F_1}(1+\innerprod{\phi(x)}{e_0})\nu(dx)
=\int_F(1+\innerprod{x}{e_0})\nu(dx)
\le\int_{F_1}(1+\innerprod{x}{e_0})\nu(dx)$.}

\noindent If we examine the definition of $\phi$, we see that
$\phi(x)\ne x$ only when $\innerprod{x}{e}<0$ and $\phi(x)=R(x)$, so
that in this case $\phi(x)-x$ is a positive multiple of $e$ and
$\innerprod{\phi(x)}{e_0}>\innerprod{x}{e_0}$.   So
$G=\{x:x\in F_1,\,\innerprod{x}{e}<0,\,R(x)\notin F_1\}$ must be
$\nu$-negligible.   But $G$ includes a relative neighbourhood of $x_1$
in $F_1$ and $F_1$ is supposed to be self-supporting for $\nu$, so this
is impossible.\ \Bang\Qed

\medskip

{\bf (e)} Now $\nu^*A=\gamma\le\nu F_1$, so $C\subseteq F_1$ and

\Centerline{$\nu U(A,\epsilon)=\beta
\ge\nu U(F_1,\epsilon)\ge\nu U(C,\epsilon)$,}

\noindent as claimed.
}%end of proof of 476K

\leader{476L}{Corollary} For any $\epsilon>0$, there is an $r_0\ge 1$
such that whenever $X$ is a finite-dimensional inner product space of
dimension at least $r_0$, $A_1$, $A_2\subseteq S_X$ and
$\min(\nu_X^*A_1,\nu_X^*A_2)\ge\epsilon$, then there are $x\in A_1$,
$y\in A_2$ such that $\|x-y\|\le\epsilon$.

\proof{ Take $r_0\ge 2$ such that $r_0\epsilon^3>2$.   Suppose that
$\dim X=r\ge r_0$.   Fix $e_0\in S_X$.   We need an estimate of
$\nu_XC_{\epsilon/2}$, where
$C_{\epsilon/2}=\{x:x\in S_X,\,\innerprod{x}{e_0}\ge\epsilon/2\}$ as in
476K.   To get this, let $e_1,\ldots,e_{r-1}$ be such that
$e_0,\ldots,e_{r-1}$ is an orthonormal basis of $X$ (4A4Kc).   For each
$i<r$, there is an $f\in H_X$ such that $f(e_i)=e_0$, so that
$\innerprod{x}{e_i}=\innerprod{f(x)}{e_0}$ for every $x$ (476J),
and

\Centerline{$\int\innerprod{x}{e_i}^2\nu_X(dx)
=\int\innerprod{f(x)}{e_0}^2\nu_X(dx)
=\int\innerprod{x}{e_0}^2\nu_X(dx)$,}

\noindent because $f:S_X\to S_X$ is \imp\ for $\nu_X$.

Accordingly

$$\eqalign{\nu_XC_{\epsilon/2}
&=\Bover12\nu_X\{x:x\in S_X,\,|\innerprod{x}{e_0}|\ge\epsilon/2\}
\le\Bover2{\epsilon^2}\int_{S_X}\innerprod{x}{e_0}^2\nu_X(dx)\cr
&<r\epsilon\int_{S_X}\innerprod{x}{e_0}^2\nu_X(dx)
=\epsilon\sum_{i=0}^{r-1}\int_{S_X}\innerprod{x}{e_i}^2\nu_X(dx)\cr
&=\epsilon\int_{S_X}\sum_{i=0}^{r-1}\innerprod{x}{e_i}^2\nu_X(dx)
=\epsilon
\le\nu_X^*A_1.\cr}$$

\noindent So, taking $C$ to be the cap of $S_X$ with centre $e_0$ and
measure $\nu^*A_1$, $C=C_{\alpha}$ where $\alpha<\bover12\epsilon$, and

\Centerline{$\nu_X(S_X\cap U(A_1,\bover12\epsilon))
\ge\nu_X(S_X\cap U(C_{\alpha},\bover12\epsilon))
\ge\nu_XC_{\alpha-\epsilon/2}>\Bover12$.}

\noindent Similarly, $\nu_X(S_X\cap U(A_2,\bover12\epsilon))>\bover12$
and there must be some
$z\in S_X\cap U(A_1,\bover12\epsilon)\cap U(A_2,\bover12\epsilon)$.
Take $x\in A_1$ and $y\in A_2$ such that $\|x-z\|<\bover12\epsilon$ and
$\|y-z\|<\bover12\epsilon$;  then $\|x-y\|\le\epsilon$, as required.
}%end of proof of 476L

\exercises{\leader{476X}{Basic exercises (a)}
%\spheader 476Xa
Let $X$ be a topological space, $\Cal C$ the set of
closed subsets of $X$, $\mu$ a topological measure on $X$ and $f$ a
$\mu$-integrable real-valued function;  set $\phi(F)=\int_Ffd\mu$ for
$F\in\Cal C$.   (i) Show that if {\it either} $\mu$ is inner regular with
respect to the closed sets and $\Cal C$ is given its Vietoris topology
{\it or} $\mu$ is tight and $\Cal C$ is given its Fell topology,
then $\phi$ is Borel measurable.   (ii) Show that if $X$ is
metrizable and $\Cal C\setminus\{\emptyset\}$ is given an appropriate
Hausdorff metric, then $\phi\restr\Cal C\setminus\{\emptyset\}$ is Borel
measurable.
%476A

\spheader 476Xb In the context of 476D, show that
$\diam\psi_{e\alpha}(A)\le\diam A$ for all $A$, $e$ and $\alpha$.
%476D

\sqheader 476Xc Find an argument along the lines of those in 476F and
476G to prove 264H.
%calculation of volume of ball
\Hint{476Xb.}
%476G

\sqheader 476Xd Let $X$ be an inner product space and $S_X$ its unit
sphere.   Show that every isometry $f:S_X\to S_X$ extends uniquely to an
isometry $T_f:X\to X$ which is a linear operator.   \Hint{first check
the cases in which $\dim X\le 2$.}   Show that $f$ is surjective iff
$T_f$ is, so that we have a natural isomorphism between the isometry
group of $S_X$ and the group of invertible isometric linear operators.
Show that this isomorphism is a homeomorphism for the topologies of
pointwise convergence.
%476J

\spheader 476Xe Let $X$ be a finite-dimensional inner product space,
$\nu_X$ the invariant Radon probability measure on the sphere $S_X$, and
$E\in\dom\nu_X$;  let $C\subseteq S_X$ be a cap with the same measure as
$E$, and let $\lambda$ be the product measure of $\nu_X$ with itself on
$S_X\times S_X$.    Show that
$\int_{C\times C}\|x-y\|\lambda(d(x,y))
\le\int_{E\times E}\|x-y\|\lambda(d(x,y))$.
%476G, 476K

\spheader 476Xf Let $X$ be a finite-dimensional inner product space and
$\nu_X$ the invariant Radon probability measure on the sphere $S_X$.
(i) Without appealing to the formulae in \S265, show that
$\nu_X(S_X\cap H)=0$ whenever $H\subseteq X$ is a proper affine
subspace.   \Hint{induce on
$\dim H$.}   (ii) Use this to prove that if $e\in S_X$ then
$\alpha\mapsto\nu_X\{x:\innerprod{x}{e}\ge\alpha\}$ is continuous.
%476K

\leader{476Y}{Further exercises (a)}
%\spheader 476Ya
Let $X$ be a compact metric space and $G$ its isometry
group.   Suppose that $H\subseteq G$ is a subgroup such that the action
of $H$ on $X$ is transitive.   Show that $X$ has a unique $H$-invariant
Radon probability measure which is also $G$-invariant.
%476C

\spheader 476Yb\dvAnew{2009.}
Let $r\ge 1$ be an integer, and $g\in C_0(\BbbR^r)$ a non-negative
$\gamma$-Lipschitz function, where $\gamma\ge 0$.   Let
$\phi:\BbbR^r\to\coint{0,\infty}$ be a convex function.   Let $F$ be the
set of non-negative $\gamma$-Lipschitz functions $f\in C(\BbbR^r)$ such
that $f$ has the same decreasing rearrangement as $g$ with respect to
Lebesgue measure $\mu$ on $\BbbR^r$ (\S373),
$\sup_{\|x\|\ge n}|f(x)|\le\sup_{\|x\|\ge n}|g(x)|$ for every $n\in\Bbb N$
and
$\int\phi(\grad f)d\mu\le\int\phi(\grad g)d\mu$.   (i) Show that $F$ is
compact for the topology of pointwise convergence.   \Hint{475Ye.}
(ii) Show that there
is a $g^*\in F$ such that $g^*(x)=g^*(y)$ whenever $\|x\|=\|y\|$.
\Hint{parts (a) and (b-i) of the proof of 479V.}
}%end of exercises

\endnotes{
\Notesheader{476}
The main theorems here (476F-476H, %476F 476G 476H
476K), like 264H, are all
`classical';  they go back to the roots of geometric measure theory, and
the contribution of the twentieth century was to extend the classes
of sets for which balls or caps provide the bounding examples.   It is
very striking that they can all be proved with the same
tools (see 476Xc).   Of course I should remark that the Compactness
Theorem (474T) lies at a much deeper level than the rest of the ideas
here.   (The proof of 474T relies on the distributional definition of
`perimeter' in 474D, while the arguments of 476Ee and 476H work with the
Hausdorff measures of essential boundaries;  so that we can join these
ideas together only after proving all the principal theorems of
\S\S472-475.)   So while `Steiner symmetrization' (264H) and
`concentration by partial reflection' (476D) are natural companions,
476H is essentially harder than the other results.

In all the theorems here, as in 264H, I have been content to show that a
ball or a cap is an optimum for whatever inequality is being considered.
I have not examined the question of whether, and in what sense, the optimum is unique.   It seems that this requires deeper analysis.
}%end of notes

\discrpage

%The formula $r_0\epsilon^3\ge 2$ in the proof of 476L can be greatly
%improved;  see {\smc Milman \& Schechtman 86}, 2.2.

