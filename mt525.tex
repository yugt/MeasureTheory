\frfilename{mt525.tex}
\versiondate{11.9.13}
\copyrightdate{2004}

\def\chaptername{Cardinal functions of measure theory}
\def\sectionname{Precalibers}

\newsection{525}

I continue the discussion of precalibers in \S516 with results applying
to measure algebras.   I start with connexions between measure spaces
and precalibers of their measure algebras (525B-525C). %525B 525C
The next step is to look at measure-precalibers.   Elementary facts are
in 525D-525G. %525D 525E 525F 525G
When we come to ask which cardinals are precalibers of which measure
algebras, there seem to be real difficulties;  partial answers, largely
based on infinitary combinatorics, are in
525I-525O.  %525I 525J 525K 525M 525N 525O
525P is a note on a particular pair of cardinals.   Finally, 525T deals
with precaliber triples $(\kappa,\kappa,k)$ where
$k$ is finite;  I approach it through a general result on correlations
in uniformly bounded families of random variables (525S).

\leader{525A}{Notation} If $(X,\Sigma,\mu)$ is a measure space,
$\Cal N(\mu)$ will be the null ideal of $\mu$.   For any set
$I$, $\nu_I$ will be the usual measure on $\{0,1\}^I$, $\Tau_I$ its
domain, $\Cal N_I=\Cal N(\nu_I)$ its null ideal and
$(\frak B_I,\bar\nu_I)$ its measure algebra.
In this context, set $e_i=\{x:x\in\{0,1\}^I,\,x(i)=1\}^{\ssbullet}$
in $\frak B_I$ for $i\in I$.
Then $\familyiI{e_i}$ is a stochastically independent
family of elements of measure $\bover12$ in $\frak B_I$, and
$\{e_i:i\in I\}\,\,\tau$-generates $\frak B_I$;  I will say that
$\familyiI{e_i}$ is the {\bf standard generating family} in $\frak B_I$.

\leader{525B}{Proposition} Let $(X,\frak T,\Sigma,\mu)$ be a quasi-Radon
measure space, and $\frak A$ its measure algebra.   Then the downwards
precaliber triples of the partially
ordered set $(\Sigma\setminus\Cal N(\mu),\subseteq)$ are just the
precaliber triples of the Boolean algebra $\frak A$.

\proof{ Put 521Dd and 516C together.}

\leader{525C}{Theorem} Let $(X,\frak T,\Sigma,\mu)$ be a Radon measure
space and $(\frak A,\bar\mu)$ its measure algebra.

(a) A pair $(\kappa,\lambda)$ of cardinals is a precaliber pair of
$\frak A$ iff whenever $\ofamily{\xi}{\kappa}{E_{\xi}}$ is a family in
$\Sigma\setminus\Cal N(\mu)$ there is an $x\in X$ such that
$\#(\{\xi:x\in E_{\xi}\})\ge\lambda$.

(b) A pair $(\kappa,\lambda)$ of cardinals is a measure-precaliber pair
of $(\frak A,\bar\mu)$ iff whenever $\ofamily{\xi}{\kappa}{E_{\xi}}$ is
a family in $\Sigma\setminus\Cal N(\mu)$ such that
$\inf_{\xi<\kappa}\mu E_{\xi}>0$ then there is an $x\in X$ such that
$\#(\{\xi:x\in E_{\xi}\})\ge\lambda$.

(c) Suppose that $\kappa\ge\sat(\frak A)$ is an infinite regular cardinal.
Then the following are equiveridical:

\quad (i) $\kappa$ is a precaliber of $\frak A$;

\quad (ii) $\mu_*(\bigcup_{\xi<\kappa}E_{\xi})=0$ whenever
$\ofamily{\xi}{\kappa}{E_{\xi}}$ is a non-decreasing family in
$\Cal N(\mu)$;

\quad (iii) whenever $\ofamily{\xi}{\kappa}{A_{\xi}}$ is a
non-decreasing family of sets such that $\bigcup_{\xi<\kappa}A_{\xi}=X$,
then there is some $\xi<\kappa$ such that $A_{\xi}$ has full outer
measure in $X$.

\proof{{\bf (a)(i)} Suppose that
$(\kappa,\lambda)$ is a precaliber pair of
$\frak A$ and $\ofamily{\xi}{\kappa}{E_{\xi}}$ is a family in
$\Sigma\setminus\Cal N(\mu)$.   For each $\xi<\kappa$, let
$K_{\xi}\subseteq E_{\xi}$ be a non-negligible compact set.   Then there
is a $\Gamma\in[\kappa]^{\lambda}$ such that
$\{K_{\xi}^{\ssbullet}:\xi\in\Gamma\}$ is centered in $\frak A$.   But
in this case $\{X\}\cup\{K_{\xi}:\xi\in\Gamma\}$ has the finite
intersection property, and must have non-empty intersection.   If $x$ is
any point of this intersection, $\{\xi:x\in E_{\xi}\}$ includes $\Gamma$
and has size at least $\lambda$.

\medskip

\quad{\bf (ii)} Suppose that whenever $\ofamily{\xi}{\kappa}{E_{\xi}}$
is a family in
$\Sigma\setminus\Cal N(\mu)$ there is an $x\in X$ such that
$\#(\{\xi:x\in E_{\xi}\})\ge\lambda$.
Because $\mu$ is complete and strictly localizable
(416B), it has a lifting $\psi:\frak A\to\Sigma$ (341K).   Let
$\ofamily{\xi}{\kappa}{a_{\xi}}$ be a family in $\frak A\setminus\{0\}$;
then there is an $x\in X$ such that $\Gamma=\{\xi:x\in\psi a_{\xi}\}$
has cardinal at least $\lambda$.   But now
$\{\psi a_{\xi}:\xi\in\Gamma\}$ is centered in $\Sigma$ so
$\{a_{\xi}:\xi\in\Gamma\}$ is centered in $\frak A$.   As
$\ofamily{\xi}{\kappa}{a_{\xi}}$ is arbitrary, $(\kappa,\lambda)$ is a
precaliber pair of $\frak A$.

\medskip

{\bf (b)} We can use exactly the same argument, provided that in part
(i) we make sure that $\mu K_{\xi}\ge\bover12\mu E_{\xi}$, so that
$\inf_{\xi<\kappa}\bar\mu K_{\xi}^{\ssbullet}>0$.

\medskip

{\bf (c)}{\bf (i)$\Rightarrow$(iii)} Suppose that $\kappa$ is a
precaliber of $\frak A$ and $\ofamily{\xi}{\kappa}{A_{\xi}}$ is a
non-decreasing family of sets with union $X$.   \Quer\ If no $A_{\xi}$
has full outer measure, then we can choose, for each $\xi<\kappa$, a
non-negligible compact set $K_{\xi}\subseteq X\setminus A_{\xi}$.
Because $\kappa$ is a precaliber of $\frak A$, there is a set
$\Gamma\in[\kappa]^{\kappa}$ such that
$\{K_{\xi}^{\ssbullet}:\xi\in\Gamma\}$ is centered.   Now
$\{K_{\xi}:\xi\in\Gamma\}$ has the finite intersection property and
there is some $x\in\bigcap_{\xi\in\Gamma}K_{\xi}$, in which case
$x\notin\bigcup_{\xi\in\Gamma}A_{\xi}$.   But since $\Gamma$ must be
cofinal with $\kappa$, $\bigcup_{\xi\in\Gamma}A_{\xi}=X$.\ \BanG\  As
$\ofamily{\xi}{\kappa}{A_{\xi}}$ is arbitrary, (iii) is true.

\medskip

\quad{\bf (iii)$\Rightarrow$(ii)} Suppose that (iii) is true, and that
$\ofamily{\xi}{\kappa}{E_{\xi}}$ is a non-decreasing family in
$\Cal N(\mu)$.   \Quer\ If $\bigcup_{\xi<\kappa}E_{\xi}$ has non-zero
inner measure, let $E\subseteq\bigcup_{\xi<\kappa}E_{\xi}$ be a
non-negligible measurable set.   Set $A_{\xi}=E_{\xi}\cup(X\setminus E)$
for each $\xi$;  then $\ofamily{\xi}{\kappa}{A_{\xi}}$ is a
non-decreasing family with union $X$, so there is some $\xi$ such that
$A_{\xi}$ has full outer measure.   But $E\setminus E_{\xi}$ is a
non-negligible measurable set disjoint from $A_{\xi}$.\ \BanG\
As $\ofamily{\xi}{\kappa}{E_{\xi}}$ is arbitrary, (ii) is true.

\medskip

\quad{\bf (ii)$\Rightarrow$(i)} Let $Z$ be the Stone space of
$\frak A$ and $\nu$ its usual
measure (411P).   Because $\mu$ has a lifting, there is an \imp\ function
$f:X\to Z$ (341P).

Let $\ofamily{\xi}{\kappa}{F_{\xi}}$ be a non-decreasing family of
nowhere dense subsets of $Z$.    Then they are all $\nu$-negligible
(411Pa),
so $\ofamily{\xi}{\kappa}{f^{-1}[F_{\xi}]}$ is a non-decreasing family
in $\Cal N(\mu)$ and $\mu_*(\bigcup_{\xi<\kappa}f^{-1}[F_{\xi}])=0$.
But this means that if $G=\interior(\bigcup_{\xi<\kappa}F_{\xi})$,
$\nu G=\mu f^{-1}[G]=0$ and $G$ is empty.   By 516Rb, $\kappa$ is a
precaliber of $\frak A$.
}%end of proof of 525C

\leader{525D}{Proposition} Let $(\frak A,\bar\mu)$ be a measure algebra.

(a) Any precaliber triple of $\frak A$ is a measure-precaliber triple of
$(\frak A,\bar\mu)$.

(b) If $\triplepc{\kappa}{\lambda}{\theta}$ is a measure-precaliber
triple of $(\frak A,\bar\mu)$ and $\kappa$ has uncountable cofinality,
then $\triplepc{\kappa}{\lambda}{\theta}$ is a precaliber triple of
$\frak A$.

(c) If $\kappa$ is a measure-precaliber of $(\frak A,\bar\mu)$, so is
$\cf\kappa$.

\proof{{\bf (a)} is immediate from the definitions in 511E.

\medskip

{\bf (b)} If $\ofamily{\xi}{\kappa}{a_{\xi}}$ is any family in
$\frak A^+$, then there is a $\delta>0$ such that
$\Gamma=\{\xi:\bar\mu a_{\xi}\ge\delta\}$ has cardinal $\kappa$;  and
now there is a $\Gamma'\in[\Gamma]^{\lambda}$ such that
$\{a_{\xi}:\xi\in I\}$ has a non-zero lower bound for every
$I\in[\Gamma']^{<\theta}$.

\medskip

{\bf (c)} The point is that $\kappa$ is a measure-precaliber of
$(\frak A,\bar\mu)$ iff it is a precaliber of the supported relation
$(A_{\delta},\Bsupseteqshort,\frak A^+)$ for every $\delta>0$, where
$A_{\delta}=\{a:a\in\frak A,\,\bar\mu a\ge\delta\}$;  so this is just a
special case of 516Bd.
}%end of proof of 525D

\leader{525E}{Proposition} (a) Let $(\frak A,\bar\mu)$ be a probability
algebra and $\kappa$ an infinite cardinal.   Then $\kappa$ is a
precaliber of $\frak A$ iff either $\frak A$ is finite or $\kappa$ is a
measure-precaliber of $(\frak A,\bar\mu)$ and $\cf\kappa>\omega$.

(b) An infinite cardinal $\kappa$ is a precaliber of every measurable
algebra iff it is a measure-precaliber of every probability algebra and
has uncountable cofinality.

\proof{{\bf (a)} If $\kappa$ is a precaliber of $\frak A$, of course
$\kappa$ is a measure-precaliber of $(\frak A,\bar\mu)$.   Also
$\cf\kappa$ is a precaliber of $\frak A$ (516Bd again), so
$\cf\kappa\ge\sat(\frak A)$ (516Ja);  and if $\frak A$ is infinite,
$\cf\kappa>\omega$.

If $\frak A$ is finite, then any infinite cardinal is a precaliber of
$\frak A$ (516Lc).   If $\kappa$ is a measure-precaliber of
$(\frak A,\bar\mu)$ and $\cf\kappa>\omega$, then $\kappa$ is a
precaliber of $\frak A$ by 525Db.

\medskip

{\bf (b)} Recall that an algebra $\frak A$ is `measurable' iff either
$\frak A=\{0\}$ or there is a functional $\bar\mu$ such that
$(\frak A,\bar\mu)$ is a probability algebra (391B).   So the result
follows directly from (a).
}%end of proof

\leader{525F}{Proposition} Let $(\frak A,\bar\mu)$ be a probability
algebra.

(a) $\omega$ is a measure-precaliber of $(\frak A,\bar\mu)$.

(b) If $\omega\le\kappa<\frak m(\frak A)$, then $\kappa$ is a
measure-precaliber of $(\frak A,\bar\mu)$.

\proof{{\bf (a)} Let $\sequencen{a_n}$ be a sequence in $\frak A$ such
that $\inf_{n\in\Bbb N}\bar\mu a_n=\delta>0$.   Set
$a=\inf_{n\in\Bbb N}\sup_{m\ge n}a_m$;  then
$\bar\mu a=\inf_{n\in\Bbb N}\bar\mu(\sup_{m\ge n}a_m)
\ge\delta>0$, so
$a\ne 0$.   If $0\ne b\Bsubseteq a$ and $n\in\Bbb N$, there is an
$m\ge n$ such that $b\Bcap a_m\ne 0$.   We can therefore choose
inductively a strictly increasing sequence $\sequence{i}{n_i}$ such that
$a\Bcap\inf_{j\le i}a_{n_j}\ne 0$ for every $i$, so that
$\sequence{i}{a_{n_i}}$ is centered.   As $\sequencen{a_n}$ is
arbitrary, $\omega$ is a measure-precaliber of $(\frak A,\bar\mu)$.

\medskip

{\bf (b)} If $\kappa=\omega$, this is (a).   Otherwise, let
$\ofamily{\xi}{\kappa}{a_{\xi}}$ be a family in $\frak A$ with
$\inf_{\xi<\kappa}\bar\mu a_{\xi}=\delta>0$.   Set

\Centerline{$c=\inf_{J\subseteq\kappa,\#(J)<\kappa}
\sup_{\xi\in\kappa\setminus J}a_{\xi}$;}

\noindent then

\Centerline{$\bar\mu c=\inf_{J\subseteq\kappa,\#(J)<\kappa}
\bar\mu(\sup_{\xi\in\kappa\setminus J}a_{\xi})\ge\delta$.}

\noindent Choose $\ofamily{\xi}{\kappa}{I_{\xi}}$ inductively so that,
for each $\xi<\kappa$, $I_{\xi}$ is a countable subset of
$\kappa\setminus\bigcup_{\eta<\xi}I_{\eta}$ and
$c\Bsubseteq\sup_{\eta\in I_{\xi}}a_{\eta}$.

For $\xi<\kappa$, set

\Centerline{$Q_{\xi}=\{b:0\ne b\Bsubseteq c$, $\Exists\eta\in I_{\xi}$,
$b\Bsubseteq a_{\eta}\}$.}

\noindent Then $Q_{\xi}$ is coinitial with $\frak A_c^+$.   Because
$\kappa<\frak m(\frak A)\le\frak m(\frak A_c)$, there is a centered
$R\subseteq\frak A_c^+$ meeting every $Q_{\xi}$.   Now

\Centerline{$\Gamma
=\{\eta:\eta<\kappa$, $\Exists b\in R$, $b\Bsubseteq a_{\eta}\}$}

\noindent meets every $I_{\xi}$ so has cardinal $\kappa$, and
$\{a_{\eta}:\eta\in\Gamma\}$ is centered.   As
$\ofamily{\xi}{\kappa}{a_{\xi}}$ is arbitrary, $\kappa$ is a
measure-precaliber of $(\frak A,\bar\mu)$.
}%end of proof of 525F

\leader{525G}{}\cmmnt{ As is surely to be expected, questions about
precalibers of measurable algebras can
generally be reduced to questions about
precalibers of the algebras $\frak B_{\kappa}$.   Some of these can be
quickly answered in terms of the cardinals examined earlier in this
chapter.

\medskip

\noindent}{\bf Proposition} (a) Let $(\frak A,\bar\mu)$ be a totally
finite measure algebra.   Let $K$ be the set of infinite cardinals
$\kappa'$ such that the Maharam-type-$\kappa'$ component of $\frak A$ is
non-zero\cmmnt{ (cf.\ 524M)}.   If $\kappa$, $\lambda$ and $\theta$
are cardinals, of which $\kappa$ is infinite, then
$\triplepc{\kappa}{\lambda}{\theta}$ is a measure-precaliber triple of
$(\frak A,\bar\mu)$ iff it is a measure-precaliber triple of
$(\frak B_{\kappa'},\bar\nu_{\kappa'})$ for every $\kappa'\in K$.

(b) Suppose that $\omega\le\kappa<\cov\Cal N_{\kappa'}$.   Then $\kappa$
is a measure-precaliber of $\frak B_{\kappa'}$.

(c) For any cardinal $\kappa'$, $\omega_1$ is a precaliber of
$\frak B_{\kappa'}$ iff $\cov\Cal N_{\kappa'}>\omega_1$.

(d) If $\kappa$, $\kappa'$ are cardinals such that
$\non\Cal N_{\kappa'}<\cf\kappa$, then $\kappa$ is a precaliber of
$\frak B_{\kappa'}$.

\proof{{\bf (a)(i)} Suppose that $\triplepc{\kappa}{\lambda}{\theta}$ is
a measure-precaliber triple of $(\frak A,\bar\mu)$, $\kappa'\in K$ and
$\ofamily{\xi}{\kappa}{b_{\xi}}$ is a family in $\frak B_{\kappa'}$ with
$\inf_{\xi<\kappa}\bar\nu_{\kappa'}b_{\xi}=\delta>0$.   Let $a\in\frak A$
be such that the principal ideal $\frak A_a$ is homogeneous with Maharam
type $\kappa'$, so that there is an isomorphism
$\pi:\frak B_{\kappa'}\to\frak A_a$ with
$\Bover1{\bar\mu a}\bar\mu(\pi b)=\bar\nu_{\kappa'}b$ for every
$b\in\frak B_{\kappa'}$ (331L).   Then
$\inf_{\xi<\kappa}\bar\mu(\pi b_{\xi})=\delta\bar\mu a>0$, so there is a
$\Gamma\in[\kappa]^{\lambda}$ such that
$\inf_{\xi\in I}\pi b_{\xi}$ and $\inf_{\xi\in I}b_{\xi}$ are non-zero
for every $I\in[\Gamma]^{<\theta}$.   As
$\ofamily{\xi}{\kappa}{b_{\xi}}$ is arbitrary,
$\triplepc{\kappa}{\lambda}{\theta}$ is a measure-precaliber triple of
$(\frak B_{\kappa'},\bar\nu_{\kappa'})$.

\medskip

\quad{\bf (ii)} Suppose that $\triplepc{\kappa}{\lambda}{\theta}$ is a
measure-precaliber triple of $(\frak B_{\kappa'},\bar\nu_{\kappa'})$ for
every $\kappa'\in K$ and $\ofamily{\xi}{\kappa}{a_{\xi}}$ is a family in
$\frak A$ with $\inf_{\xi<\kappa}\bar\mu a_{\xi}=\delta>0$.   Let
$D\subseteq\frak A\setminus\{0\}$ be a partition of unity in $\frak A$
such that all the principal ideals $\frak A_d$, for $d\in D$, are
homogeneous.   Let $C\subseteq D$ be a finite set such that
$\sum_{d\in D\setminus C}\bar\mu d\le\bover12\delta$.   Then for every
$\xi<\kappa$ there is a $c\in C$ such that
$\bar\mu(a_{\xi}\Bcap c)\ge\bover12\delta\bar\mu c$, so (because
$\kappa$ is infinite) there are $c\in C$ and
$\Gamma_0\in[\kappa]^{\kappa}$ such that
$\bar\mu(a_{\xi}\Bcap c)\ge\bover12\delta\bar\mu c$ for every
$\xi\in\Gamma_0$.   If $c$ is an atom then
$\inf_{\xi\in I}a_{\xi}\Bsupseteq c$ is
non-zero for every $I\subseteq\Gamma_0$.   Otherwise, the Maharam type
$\kappa'$ of $\frak A_c$ belongs to $K$.
Let $\pi:\frak B_{\kappa'}\to\frak A_c$ be an isomorphism with
$\bar\mu(\pi b)=\bar\mu c\cdot\bar\nu_{\kappa'}b$ for every
$b\in\frak B_{\kappa'}$.   Set
$b_{\xi}=\pi^{-1}(a_{\xi}\Bcap c)$;  then
$\bar\nu_{\kappa'}b_{\xi}\ge\bover12\delta$ for every $\xi\in\Gamma_0$.
There is therefore a $\Gamma\in[\Gamma_0]^{\lambda}$ such that
$\inf_{\xi\in I}b_{\xi}$ and $\inf_{\xi\in I}a_{\xi}$
are non-zero for every $I\in[\Gamma]^{<\theta}$.
As $\ofamily{\xi}{\kappa}{a_{\xi}}$ is arbitrary,
$\triplepc{\kappa}{\lambda}{\theta}$ is a measure-precaliber triple of
$(\frak A,\bar\mu)$.

\medskip

{\bf (b)} We have $\cov\Cal N_{\kappa'}=\frak m(\frak B_{\kappa'})$
(524Md), so we can use 525Fb.

\medskip

{\bf (c)} If $\cov\Cal N_{\kappa'}>\omega_1$ then (b) tells us that
$\omega_1$ is a precaliber of $\frak B_{\kappa'}$.   If
$\cov\Cal N_{\kappa'}=\omega_1$, let $\ofamily{\xi}{\omega_1}{E_{\xi}}$
be a cover of $\{0,1\}^{\kappa'}$ by negligible sets;  then
$\ofamily{\xi}{\omega_1}{\bigcup_{\eta<\xi}E_{\eta}}$ is a
non-decreasing
family in $\Cal N_{\kappa'}$ with union of non-zero inner measure, so
525Cc tells us that $\omega_1$ is not a precaliber of
$\frak B_{\kappa'}$.

\medskip

{\bf (d)} If $\kappa'$ is finite this is elementary.   Otherwise,
$d(\frak B_{\kappa'})=\non\Cal N_{\kappa'}$ (524Me).   By 516Lc,
$\kappa$ is a precaliber of $\frak B_{\kappa'}$.
}%end of proof of 525G

\leader{525H}{The structure of $\frak B_I$}\cmmnt{ Several of the
arguments below will depend on the following ideas.}   Let $I$ be any
set and $\familyiI{e_i}$ the standard generating family in $\frak B_I$.
If $a\in\frak B_I$,
there is a smallest countable set $J\subseteq I$ such that $a$ belongs
to the closed subalgebra $\frak C_J$ of $\frak B_I$ generated by
$\{e_i:i\in J\}$\cmmnt{ (254Rd, 325Mb)}.   \cmmnt{(Of course
$\frak C_J$ is canonically isomorphic to $\frak B_J$;  see 325Ma.)}

Now suppose that $\family{\xi}{\Gamma}{a_{\xi}}$ is a family in
$\frak B_I$, that for each $\xi\in\Gamma$ we are given a set
$I_{\xi}\subseteq I$ such that $a_{\xi}\in\frak C_{I_{\xi}}$, and that
$J\subseteq I$ is such that $I_{\xi}\cap I_{\eta}\subseteq J$ for all
distinct $\xi$, $\eta\in\Gamma$.   Then $\family{\xi}{\Gamma}{a_{\xi}}$
is relatively stochastically independent over
$\frak C_J$.   \prooflet{\Prf\
$\family{\xi}{\Gamma}{\frak C_{I_{\xi}\setminus J}}$ is stochastically
independent, because $\family{\xi}{\Gamma}{I_{\xi}\setminus J}$ is
disjoint;   moreover, $\frak C_J$ is independent of
$\frak C_{I\setminus J}
\supseteq\bigcup_{\xi\in\Gamma}\frak C_{I_{\xi}\setminus J}$, and
$\frak C_{I_{\xi}\cup J}$ is the closed subalgebra generated by
$\frak C_{I_{\xi}\setminus J}\cup\frak C_J$ for each $\xi$.
So 458Lg tells us that
$\family{\xi}{\Gamma}{\frak C_{I_{\xi}\cup J}}$ is relatively stochastically
independent over $\frak C_J$;  {\it a fortiori},
$\family{\xi}{\Gamma}{a_{\xi}}$ is relatively stochastically
independent over $\frak C_J$.\ \QeD}
It follows that if $\Delta\subseteq\Gamma$ is finite and
$\inf_{\xi\in\Delta}\upr(a_{\xi},\frak C_J)\ne 0$, then
$\inf_{\xi\in\Delta}a_{\xi}\ne 0$\cmmnt{ (458Lf\dvAformerly{4{}58Hb})};
\cmmnt{in particular,} if
$\family{\xi}{\Gamma}{\upr(a_{\xi},\frak C_J)}$ is centered, so is
$\family{\xi}{\Gamma}{a_{\xi}}$.

\leader{525I}{Theorem} (a)(i) If $\kappa>0$ and
$\triplepc{\kappa}{\lambda}{\theta}$
is a measure-precaliber triple of $(\frak B_{\kappa},\bar\nu_{\kappa})$,
then it is a measure-precaliber triple of every probability algebra.

\quad(ii) If $\kappa>0$ and $\triplepc{\kappa}{\lambda}{\theta}$
is a precaliber triple of $\frak B_{\kappa}$, then it is a precaliber
triple of every measurable algebra.

(b) Suppose that $\cf\kappa\ge\omega_2$.   If $(\kappa,\lambda)$
is a precaliber pair of
$\frak B_{\kappa'}$ for every $\kappa'<\kappa$, then it is a precaliber
pair of every measurable algebra.

(c) Suppose that $\triplepc{\kappa}{\lambda}{\theta}$ is a
measure-precaliber triple of
$(\frak B_{\omega},\bar\nu_{\omega})$ and that $\kappa'$ is such that
$\cff[\kappa']^{\le\omega}<\cf\kappa$.   Then
$\triplepc{\kappa}{\lambda}{\theta}$
is a measure-precaliber triple of
$(\frak B_{\kappa'},\bar\nu_{\kappa'})$.

\proof{{\bf (a)(i)} Let
$(\frak A,\bar\mu)$ be any probability algebra and
$\ofamily{\xi}{\kappa}{a_{\xi}}$
a family in $\frak A^+$ such that $\inf_{\xi<\kappa}\bar\mu a_{\xi}>0$.
Let $\frak B$ be the closed
subalgebra of $\frak A$ generated by $\{a_{\xi}:\xi<\kappa\}$.   Then
$(\frak B,\bar\mu\restrp\frak B)$ is a probability algebra with
Maharam type at most $\kappa$, so is isomorphic to a closed subalgebra
of $(\frak B_{\kappa},\bar\nu_{\kappa})$ (332N).   Since
$\triplepc{\kappa}{\lambda}{\theta}$ is a measure-precaliber triple of
$(\frak B_{\kappa},\bar\nu_{\kappa})$ it is a measure-precaliber triple
of $(\frak B,\bar\mu\restrp\frak B)$ (cf.\ 516Sb), and there
is a $\Gamma\in[\kappa]^{\lambda}$ such that $\{a_{\xi}:\xi\in I\}$ is
bounded below in $\frak B^+$ and therefore in $\frak A^+$ for every
$I\in[\Gamma]^{<\theta}$.   As $\ofamily{\xi}{\kappa}{a_{\xi}}$ is
arbitrary, $\triplepc{\kappa}{\lambda}{\theta}$ is a measure-precaliber
triple of $(\frak A,\bar\mu)$.

\medskip

\quad{\bf (ii)} The same argument applies, deleting the phrase
`$\inf_{\xi<\kappa}\bar\mu a_{\xi}>0$', since if $\frak A$ is a
measurable algebra other than $\{0\}$ there is a
functional $\bar\mu$ such that $(\frak A,\bar\mu)$ is a probability
algebra.

\medskip

{\bf (b)} By (a-ii), it is enough to prove that $(\kappa,\lambda)$ is a
precaliber pair
of $\frak B_{\kappa}$.   Let $\ofamily{\xi}{\kappa}{a_{\xi}}$ be a
family in $\frak B_{\kappa}^+$.   For each $I\subseteq\kappa$, let
$\frak C_I$ be the closed subalgebra of $\frak B_{\kappa}$ generated by
$\{e_i:i\in I\}$, as in 525H.
Then for each $\xi<\kappa$ we have a countable
set $I_{\xi}\subseteq\kappa$ such that $a_{\xi}\in\frak C_{I_{\xi}}$.
Because $\cf\kappa\ge\omega_2$,
there are a $\Gamma\in[\kappa]^{\kappa}$ and a
$J\in[\kappa]^{<\kappa}$ such that $I_{\xi}\cap I_{\eta}\subseteq J$ for
all distinct $\xi$, $\eta\in\Gamma$ (5A1I(a-i)).   Because $\#(J)<\kappa$,
$(\kappa,\lambda)$ is a precaliber pair of $\frak B_J\cong\frak C_J$, so
there is a $\Gamma'\in[\Gamma]^{\lambda}$ such that
$\family{\xi}{\Gamma'}{\upr(a_{\xi},\frak C_J)}$ is centered.   It
follows that $\family{\xi}{\Gamma'}{a_{\xi}}$ is
centered (525H).   As $\ofamily{\xi}{\kappa}{a_{\xi}}$ is arbitrary, we
have the result.

\medskip

{\bf (c)} Let $\ofamily{\xi}{\kappa}{a_{\xi}}$ be a family in
$\frak B_{\kappa'}$ such that $\bar\nu_{\kappa'}a_{\xi}\ge\delta>0$ for
every $\xi<\kappa$.   Fix a cofinal family $\Cal J$ in
$[\kappa']^{\le\omega}$ with cardinal less than $\cf\kappa$.
For each $\xi<\kappa$ let $J_{\xi}\in\Cal J$ be such that
$a_{\xi}\in\frak C_{J_{\xi}}$, where this time $\frak C_{J_{\xi}}$ is
interpreted as a subalgebra of $\frak B_{\kappa'}$.   Then there must be
some $J\in\Cal J$ such that $A=\{\xi:J_{\xi}=J\}$ has cardinal $\kappa$.
Now $(\frak C_J,\bar\nu_{\kappa'}\restrp\frak C_J)$ is isomorphic to a
subalgebra of $(\frak B_{\omega},\bar\nu_{\omega})$, so has
$\triplepc{\kappa}{\lambda}{\theta}$ as a measure-precaliber triple, and
there is a $\Gamma\in[A]^{\lambda}$ such that $\{a_{\xi}:\xi\in I\}$ has
a non-zero lower bound for every $I\in[\Gamma]^{<\theta}$.   As
$\ofamily{\xi}{\kappa}{a_{\xi}}$ is arbitrary,
$\triplepc{\kappa}{\lambda}{\theta}$ is a measure-precaliber triple of
$(\frak B_{\kappa'},\bar\nu_{\kappa'})$.
}%end of proof of 525I

\leader{525J}{Corollary} Suppose that $\kappa$ is an infinite cardinal
and $\kappa<\cov\Cal N_{\kappa}$.   Then $\kappa$ is a
measure-precaliber of every probability algebra.

\proof{ By 525Gb, $\kappa$ is a measure-precaliber of
$(\frak B_{\kappa},\bar\nu_{\kappa})$;  by 525Ia, it is a
measure-precaliber of every probability algebra.
}%end of proof of 525J

\leader{525K}{Proposition} Let $\kappa>\non\Cal N_{\omega}$ be a regular
cardinal such that $\cff[\lambda]^{\le\omega}<\kappa$ for every
$\lambda<\kappa$\cmmnt{ (e.g., $\kappa=\frak c^+$, $(\frak c^+)^+$,
etc.;  or $\kappa=\omega_2$ if $\non\Cal N_{\omega}=\omega_1$)}.
Then $\kappa$ is a precaliber of every measurable algebra.

\proof{ The point is that $\kappa$ is a precaliber of
$\frak B_{\lambda}$ for every $\lambda<\kappa$.   \Prf\ If $\lambda$ is
finite, this is trivial.   Otherwise,

\Centerline{$d(\frak B_{\lambda})=\non\Cal N_{\lambda}
\le\max(\non\Cal N_{\omega},\cff[\lambda]^{\le\omega})<\kappa=\cf\kappa$}

\noindent by 524Me and 523I(a-i);  it follows that $\kappa$ is a precaliber
of $\frak B_{\lambda}$ (516Lc once more).\ \Qed

By 525Ib, $\kappa$ is a precaliber of all measurable algebras.
}%end of proof of 525K

\leader{525L}{}\cmmnt{ If $\kappa>\frak c$ is not a strong limit
cardinal we can do a little better than 525K.

\medskip

\noindent}{\bf Proposition}\cmmnt{ ({\smc D\v{z}amonja \& Plebanek 04})}
Suppose that $\lambda$ and $\kappa$ are infinite cardinals such
that $\lambda^{\omega}<\cf\kappa\le\kappa\le 2^{\lambda}${, where
$\lambda^{\omega}$ is the cardinal power}.   Then
$\kappa$ is a precaliber of every measurable algebra.

\proof{ By 525Eb and
525I(a-ii), it is enough to show that $\kappa$ is a precaliber of
$\frak B_{\kappa}$.   Let $\ofamily{\xi}{\kappa}{a_{\xi}}$ be a family
in $\frak B_{\kappa}\setminus\{0\}$.   Let
$\theta:\frak B_{\kappa}\to\Tau_{\kappa}$ be a lifting, and for each
$\xi<\kappa$ let $K_{\xi}$ be a non-empty closed subset of
$\theta a_{\xi}$ which is determined by coordinates in a countable set
$I_{\xi}$.   We may suppose that each $I_{\xi}$ is infinite;  let
$h_{\xi}:\Bbb N\to I_{\xi}$ be a bijection, and set
$g_{\xi}(x)=xh_{\xi}$ for $x\in\{0,1\}^{\kappa}$, so that
$g_{\xi}:\{0,1\}^{\kappa}\to\{0,1\}^{\Bbb N}$ is continuous and
$K_{\xi}=g_{\xi}^{-1}[g_{\xi}[K_{\xi}]]$.   As $\frak c<\cf\kappa$,
there is an $L\subseteq\{0,1\}^{\Bbb N}$ such that
$\Gamma_0=\{\xi:\xi<\kappa$, $g_{\xi}[K_{\xi}]=L\}$ has cardinal
$\kappa$.

Because $\kappa\le 2^{\lambda}$,
there is an $f:\kappa\times\lambda^{\omega}\to\Bbb N$ such that
whenever $\sequencen{\xi_n}$ is a sequence of distinct elements of
$\kappa$ there is an $\eta<\lambda^{\omega}$ such that $f(\xi_n,\eta)=n$
for every $n$ (5A1Eg).   For each $\eta<\lambda^{\omega}$, set $A_{\eta}
=\{\xi:\xi<\kappa$, $f(h_{\xi}(n),\eta)=n$ for every $n\in\Bbb N\}$;
then $\bigcup_{\eta<\lambda^{\omega}}A_{\eta}=\kappa$, while
$\cf\kappa>\lambda^{\omega}$, so there is an $\eta^*<\lambda^{\omega}$
such that $\Gamma=\Gamma_0\cap A_{\eta^*}$ has $\kappa$ members.

Fix $z\in L$.   For $\xi$, $\eta\in\Gamma$ and $i$, $j\in\Bbb N$,

\Centerline{$h_{\xi}(i)=h_{\eta}(j)
\Longrightarrow i=f(h_{\xi}(i),\eta^*)=f(h_{\eta}(j),\eta^*)=j$.}

\noindent So we can find an $x\in\{0,1\}^{\kappa}$ such that
$x(h_{\xi}(i))=z(i)$ whenever $\xi\in\Gamma$ and $i\in\Bbb N$;  that is,
$g_{\xi}(x)=z$ for every $\xi\in\Gamma$.   But this means that
$x\in g_{\xi}^{-1}[L]=K_{\xi}$ for every $\xi\in\Gamma$.
It follows that whenever $I\in[\Gamma]^{<\omega}$ then
$\bigcap_{\xi\in I}\theta a_{\xi}\ne\emptyset$ and
$\inf_{\xi\in I}a_{\xi}\ne 0$;  that is, that $\{a_{\xi}:\xi\in\Gamma\}$
is centered.   As $\ofamily{\xi}{\kappa}{a_{\xi}}$ is arbitrary,
$\kappa$ is a precaliber of $\frak B_{\kappa}$.
}%end of proof of 525L

\leader{525M}{Proposition} Let $(\frak A,\bar\mu)$ be a probability
algebra and $\kappa$ an infinite cardinal such that $\cf\kappa$ is a
measure-precaliber of $(\frak A,\bar\mu)$ and
$\lambda^{\omega}<\kappa$ for every $\lambda<\kappa$.   Then $\kappa$ is
a measure-precaliber of $(\frak A,\bar\mu)$.

\proof{{\bf (a)} Suppose first that $\frak A=\frak B_I$ for some set
$I$;  let $\familyiI{e_i}$ be the standard generating family in
$\frak B_I$.   If $\kappa$ is regular, the result is
trivial.   Otherwise, let $\ofamily{\xi}{\kappa}{a_{\xi}}$ be a family
in $\frak A^+$ such that $\inf_{\xi<\kappa}\bar\nu_Ia_{\xi}=\delta>0$.
There is a strictly increasing family
$\ofamily{\alpha}{\cf\kappa}{\kappa_{\alpha}}$ of regular uncountable
cardinals with supremum $\kappa$ such that $\kappa_0>\cf\kappa$ and if
$\alpha<\cf\kappa$ and $\lambda<\kappa_{\alpha}$ then
$\lambda^{\omega}<\kappa_{\alpha}$.   \Prf\ All we need to know is that
if $\theta<\kappa$ there is a regular uncountable cardinal $\theta'$
such that $\theta\le\theta'<\kappa$ and $\lambda^{\omega}<\theta'$
whenever $\lambda<\theta'$;  and $\theta'=(\theta^{\omega})^+$ has this
property.\ \QeD\

For each $\xi<\kappa$, let $I_{\xi}\subseteq I$ be a countable set such
that $a_{\xi}$ belongs to the closed subalgebra of $\frak A$ generated
by $\{e_i:i\in I_{\xi}\}$.   By the $\Delta$-system Lemma (5A1I(a-ii)),
there is for each $\alpha<\cf\kappa$ a set
$\Gamma_{\alpha}\subseteq\kappa_{\alpha+1}\setminus\kappa_{\alpha}$ such
that $\#(\Gamma_{\alpha})=\kappa_{\alpha+1}$ and
$\family{\xi}{\Gamma_{\alpha}}{I_{\xi}}$ is a $\Delta$-system with
root $J_{\alpha}$ say.   Set $J=\bigcup_{\alpha<\cf\kappa}J_{\alpha}$,
so that $\#(J)\le\cf\kappa$, and

\Centerline{$\Gamma'_{\alpha}
=\{\xi:\xi\in\Gamma_{\alpha},\,
  (I_{\xi}\setminus J_{\alpha})
  \cap(J\cup\bigcup_{\eta<\kappa_{\alpha}}I_{\eta})
  =\emptyset\}$;}

\noindent then $\#(\Gamma'_{\alpha})=\kappa_{\alpha+1}$ for every
$\alpha<\cf\kappa$, and $I_{\xi}\cap I_{\eta}\subseteq J$ for all
distinct $\xi$,
$\eta\in\Gamma'=\bigcup_{\alpha<\cf\kappa}\Gamma'_{\alpha}$.   Let
$\frak C_J$ be the closed subalgebra of $\frak A$ generated by
$\{e_i:i\in J\}$.   For $\xi\in\Gamma'$, set
$b_{\xi}=\upr(a_{\xi},\frak C_J)$.   By 515Ma,

\Centerline{$\#(\frak C_J)\le\max(\omega,\#(J))^{\omega}
<\kappa_{\alpha+1}=\cf\kappa_{\alpha+1}$,}

\noindent there is for each $\alpha<\cf\kappa$ a
$c_{\alpha}\in\frak C_J$ such that
$\Gamma''_{\alpha}=\{\xi:\xi\in\Gamma'_{\alpha},\,b_{\xi}=c_{\alpha}\}$
has cardinal $\kappa_{\alpha+1}$.    Note that

\Centerline{$\bar\nu_Ic_{\alpha}=\bar\nu_Ib_{\xi}\ge\bar\nu_Ia_{\xi}
\ge\delta$}

\noindent whenever $\alpha<\cf\kappa$ and $\xi\in\Gamma''_{\alpha}$.

Now recall that we are supposing that $\cf\kappa$ is a
measure-precaliber of
$\frak A$.   So there is a $\Delta\in[\cf\kappa]^{\cf\kappa}$ such that
$\{c_{\alpha}:\alpha\in\Delta\}$ is centered in $\frak A$.   Now
$\Gamma''=\bigcup_{\alpha\in\Delta}\Gamma''_{\alpha}$ has cardinal
$\kappa$, and $\family{\xi}{\Gamma''}{b_{\xi}}$ is centered.   It
follows that $\family{\xi}{\Gamma''}{a_{\xi}}$ is
centered (525H).

As $\ofamily{\xi}{\kappa}{a_{\xi}}$ is arbitrary, $\kappa$ is a
measure-precaliber of $\frak A$.

\medskip

{\bf (b)} For the general case, observe that by Maharam's theorem (332B)
$\frak A$ is isomorphic to the simple product
$\prod_{k\in K}\frak A_{d_k}$ of a countable family of homogeneous
principal ideals, where $\family{k}{K}{d_k}$ is a partition of unity in
$\frak A$.   Let $\ofamily{\xi}{\kappa}{a_{\xi}}$ be a family in
$\frak A$ such that $\inf_{\xi<\kappa}\bar\mu a_{\xi}=\delta>0$.   Let
$L\subseteq K$ be a finite set such that
$\sum_{k\in K\setminus L}\bar\mu d_k=\delta'<\delta$.   Then there is
some $k\in L$ such that

\Centerline{$\Gamma_k=\{\xi:\xi<\kappa,\,\bar\mu(a_{\xi}\Bcap d_k)
\ge\Bover{\delta-\delta'}{\#(L)}\}$}

\noindent has cardinal $\kappa$.   Since $\cf\kappa$ is a
measure-precaliber of $\frak A$, it is also a measure-precaliber of
$\frak A_{d_k}$ (cf.\ 516Sc).   Since
$(\frak A_{d_k},\bar\mu\restrp\frak A_{d_k})$ is isomorphic, up to a
scalar multiple of the measure, to $(\frak B_I,\bar\nu_I)$ for some $I$,
(a) tells us that $\kappa$ is a measure-precaliber of $\frak A_{d_k}$.
There is therefore a set $\Gamma\in[\Gamma_k]^{\kappa}$ such that
$\family{\xi}{\Gamma}{a_{\xi}\Bcap d_k}$ and
$\family{\xi}{\Gamma}{a_{\xi}}$ are centered.   As
$\ofamily{\xi}{\kappa}{a_{\xi}}$ is arbitrary, $\kappa$ is a
measure-precaliber of $\frak A$.
}%end of proof of 525M

\leader{525N}{Proposition}\cmmnt{ ({\smc Argyros \& Tsarpalias 82})}
Let $\kappa$ be either $\omega$ or a strong limit cardinal of countable
cofinality, and suppose that $2^{\kappa}=\kappa^+$.   Then $\kappa^+$ is
not a precaliber of $\frak B_{\kappa}$.

\proof{ By 523Lb, $\non\Cal N_{\kappa}>\kappa$.   So if we enumerate
$\{0,1\}^{\kappa}$ as $\ofamily{\xi}{\kappa^+}{x_{\xi}}$ and set
$E_{\xi}=\{x_{\eta}:\eta<\xi\}$ for $\xi<\kappa^+$,
$\ofamily{\xi}{\kappa^+}{E_{\xi}}$ is an increasing family in
$\Cal N_{\kappa}$ with union $\{0,1\}^{\kappa}$.
By 525Cc, $\kappa^+$ is not a precaliber of $\frak B_{\kappa}$.
}%end of proof of 525N

\leader{525O}{}\cmmnt{ As in 523P, GCH decides the most important
questions.

\medskip

\noindent}{\bf Proposition} Suppose that the generalized continuum
hypothesis is true.

(a) An infinite cardinal $\kappa$ is a measure-precaliber of every
probability algebra iff $\cf\kappa$ is not the successor of a cardinal
of countable cofinality.

(b) An infinite cardinal $\kappa$ is a precaliber of every measurable
algebra iff $\cf\kappa$ is neither $\omega$ nor the successor of a
cardinal of countable cofinality.

\proof{{\bf (a)(i)} If $\kappa$ is a measure-precaliber of every
probability algebra, so is $\cf\kappa$ (525Dc).   By 525N, $\cf\kappa$
cannot be the successor of a cardinal of countable cofinality.

\medskip

\quad{\bf (ii)} Now suppose that $\cf\kappa$ is not the successor of a
cardinal of countable cofinality.   If $\kappa=\omega$, then
certainly $\kappa$ is a measure-precaliber of every probability algebra
(525Fa).   Otherwise,
$\kappa>\lambda^{\omega}$ for every $\lambda<\kappa$ and
$\cf\kappa>\lambda^{\omega}$ for every $\lambda<\cf\kappa$ (5A6Ac).   By
525K, $\cf\kappa$ is a measure-precaliber of every probability algebra;
by 525M, so is $\kappa$.

\medskip

{\bf (b)} Put (a) and 525Eb together.
}%end of proof of 525O

\leader{*525P}{}\cmmnt{ As in 522U, the Freese-Nation number of
$\Cal P\Bbb N$ is relevant to the questions here.

\medskip

\noindent}{\bf Proposition} $(\frakmctbl,\FN^*(\Cal P\Bbb N))$ is not a
precaliber pair of $\frak B_{\omega}$.

\proof{ By 518D(iv), the Freese-Nation number of the
topology of $\{0,1\}^{\omega}$ is $\FN(\Cal P\Bbb N)$;  the regular
Freese-Nation numbers are therefore also equal.   We know that
$\frakmctbl$ is the covering number of the meager ideal of $\Bbb R$
(522Sa), and therefore also of the meager ideal of $\{0,1\}^{\omega}$
(522Wb) and of the nowhere dense ideal of $\{0,1\}^{\omega}$.
By 518E, there
is a set $A\subseteq\{0,1\}^{\omega}$, with cardinal $\frakmctbl$, such
that $\#(A\cap F)<\FN^*(\Cal P\Bbb N)$ for every nowhere dense set
$F\subseteq\{0,1\}^{\omega}$.

Fix a nowhere dense compact set $K\subseteq\{0,1\}^{\omega}$ of
non-zero measure.   For each $x\in A$, set
$a_x=(K+x)^{\ssbullet}$ in $\frak B_{\omega}$, where $+$ here is the
usual group operation corresponding to the identification
$\{0,1\}^{\omega}\cong\Bbb Z_2^{\omega}$.   Then every $a_x$ is
non-zero.   If $B\subseteq A$ and $\{a_x:x\in B\}$ is centered, then
$\{K+x:x\in B\}$ has the finite intersection property, so there is a $y$
in its intersection;  now $B\subseteq A\cap(K+y)$, and $K+y$ is nowhere
dense, so $\#(B)<\FN^*(\Cal P\Bbb N)$.   Thus
$\family{x}{A}{a_x}$ has no centered subfamily of size
$\FN^*(\Cal P\Bbb N)$ and witnesses
that $(\frakmctbl,\FN^*(\Cal P\Bbb N))$ is not a precaliber pair of
$\frak B_{\omega}$.
}%end of proof of 525P

\leader{525Q}{}\cmmnt{ I turn now to some results which may be
interpreted as information on precaliber triples in which the third
cardinal is {\it finite}.

\medskip

\noindent}{\bf Lemma} Let $(\frak A,\bar\mu)$ be a semi-finite measure
algebra, $\sequencen{u_n}$ a $\|\,\|_2$-bounded sequence in
$L^2=L^2(\frak A,\bar\mu)^+$, and $\Cal F$ a non-principal
ultrafilter on $\Bbb N$.   Suppose that $p\in\coint{0,\infty}$ is such that
$\sup_{n\in\Bbb N}\|u_n^p\|_2$ is finite, and set
$v=\lim_{n\to\Cal F}u_n$, $w=\lim_{n\to\Cal F}u_n^p$, the limits being
taken for the weak topology in $L^2$.
Then $v^p\le w$.

\proof{ Of course the positive cone of $L^2$ is closed for the weak
topology so $v\ge 0$ and we can speak of $v^p$.   \Quer\ If
$v^p\not\le w$, there are
$\alpha$, $\beta\ge 0$ such that $\alpha^p>\beta$ and

\Centerline{$a=\Bvalue{v>\alpha}\Bsetminus\Bvalue{w>\beta}\ne 0$.}

\noindent Let $b\Bsubseteq a$ be such that $0<\bar\mu b<\infty$ and
consider $u=\Bover1{\bar\mu b}\chi b$.   Then, setting $q=p/(p-1)$ (of
course $p>1$),

$$\eqalignno{(\alpha\bar\mu b)^p
&\le(\int_bv)^p
=\lim_{n\to\Cal F}(\int u_n\times\chi b\times\chi b)^p
\le\lim_{n\to\Cal F}\bigl(\|u_n\times\chi b\|_p\|\chi b\|_q)^p\cr
\displaycause{by H\"older's inequality, 244Eb}
&=\lim_{n\to\Cal F}(\bar\mu b)^{p/q}\int_bu_n^p
=(\bar\mu b)^{p-1}\int_bw
\le\beta(\bar\mu b)^p
<\alpha^p(\bar\mu b)^p,\cr}$$

\noindent which is absurd.\ \BanG\  So $v^p\le w$.
}%end of proof of 525Q

\vleader{72pt}{525R}{Lemma}
Let $(\frak A,\bar\mu)$ be a probability algebra and
$\sequencen{u_n}$ a $\|\,\|_{\infty}$-bounded sequence in
$L^{\infty}(\frak A,\bar\mu)^+$
such that $\delta=\inf_{n\in\Bbb N}\int u_n>0$.   Let $k_0,\ldots,k_m$
be strictly positive integers with sum $k$.   Suppose that
$\gamma<\delta^k$.

(a) There are integers $n_0<n_1<\ldots<n_m$ such that
$\int\prod_{j=0}^mu_{n_j}^{k_j}\ge\gamma$.

(b) In fact, there is an infinite set $I\subseteq\Bbb N$ such that
$\int\prod_{j=0}^mu_{n_j}^{k_j}\ge\gamma$ whenever $n_0,\ldots,n_m$ belong
to $I$ and $n_0<n_1<\ldots<n_m$.

\proof{{\bf (a)} Let $\Cal F$ be a non-principal ultrafilter on
$\Bbb N$.   For each $j\le m$, let $v_j$ be the limit
$\lim_{n\to\Cal F}u_n^{k_j}$ for the weak topology on
$L^2(\frak A,\bar\mu)$;  let $v$ be the limit $\lim_{n\to\Cal F}u_n$.
By 525Q,

$$\eqalignno{\int\prod_{j=0}^mv_j
&\ge\int\prod_{j=0}^mv^{k_j}
=\int v^k
=(\|\chi 1\|_q\|v\|_k)^k\cr
\displaycause{where $q=\Bover{k}{k-1}$, or $\infty$ if $k=1$}
&\ge\bigl(\int v\times\chi 1\bigr)^k\cr
\displaycause{by H\"older's inequality again, if $k>1$}
&=\bigl(\int v\bigr)^k
=\lim_{n\to\Cal F}\bigl(\int u_n\bigr)^k
\ge\delta^k
>\gamma.\cr}$$

\noindent (Or use 244Xd to show more directly that
$\int v^k\ge(\int v)^k$.)   We can therefore choose $n_0,\ldots,n_m$
inductively so that

\Centerline{$\int\prod_{j=0}^su_{n_j}^{k_j}
  \times\prod_{j=s+1}^mv_j
>\gamma$}

\noindent for each $s\le m$ (interpreting the final product
$\prod_{j=m+1}^mv_j$ as $\chi 1$), since when we come to choose
$n_s$ we shall be able to use any member of

\Centerline{$\{n:n>n_j\text{ for }j<s,\,
\int u_n^{k_s}\times\prod_{j=0}^{s-1}u_{n_j}^{k_j}
  \times\prod_{j=s+1}^mv_j>\gamma\}$,}

\noindent which belongs to $\Cal F$ so is not empty.   At the end of the
induction we shall have a sequence $n_0<\ldots<n_m$  such that
$\int\prod_{j=0}^mu_{n_j}^{k_j}\ge\gamma$, as required.

\medskip

{\bf (b)} Let $\Cal J\subseteq[\Bbb N]^{m+1}$ be the family of all sets
of the form $\{n_0,\ldots,n_m\}$ where $n_0<\ldots<n_m$ and
$\int\prod_{j=0}^mu_{n_j}^{k_j}\ge\gamma$.   By (a), applied to
subsequences of $\sequencen{u_n}$, every infinite subset of $\Bbb N$
includes some member of $\Cal J$.   By Ramsey's theorem (4A1G), there is
an infinite $I\subseteq\Bbb N$ such that $[I]^{m+1}\subseteq\Cal J$,
which is what we need.
}%end of proof of 525R

\leader{525S}{Theorem}\cmmnt{ ({\smc Fremlin 88})} Let
$(\frak A,\bar\mu)$ be a probability algebra and $\kappa$ an infinite
cardinal.   Let $\ofamily{\xi}{\kappa}{u_{\xi}}$ be a
$\|\,\|_{\infty}$-bounded family in $L^{\infty}(\frak A)^+$.   Set
$\delta=\inf_{\xi<\kappa}\int u_{\xi}$.   Then for any $k\in\Bbb N$ and
$\gamma<\delta^{k+1}$ there is a $\Gamma\in[\kappa]^{\kappa}$ such that
$\int\prod_{i=0}^ku_{\xi_i}\ge\gamma$ for all
$\xi_0,\ldots,\xi_k\in\Gamma$.

\proof{{\bf (a)} It will be helpful to note straight away that it will be
enough to consider the case $(\frak A,\bar\mu)=(\frak B_I,\bar\nu_I)$ for
some set $I$.   \Prf\ There is always a $(\frak B_I,\bar\nu_I)$ in which
$(\frak A,\bar\mu)$ can be embedded.   In this case, $L^{\infty}(\frak A)$
can be identified, as $f$-algebra, with
a subspace of $L^{\infty}(\frak B_I)$,
and the embedding respects integrals.   So we can regard
$\ofamily{\xi}{\kappa}{u_{\xi}}$ as a family in
$L^{\infty}(\frak B_I)$ and perform all calculations there.\ \Qed

At the same time, the case $\delta=0$ is trivial, so let us suppose
henceforth that $\delta>0$.

\medskip

{\bf (b)} Next, having fixed on a suitable set $I$, let $\familyiI{e_i}$ be
the standard generating family in $\frak B_I$, and for $J\subseteq I$ let
$\frak C_J$ be the closed subalgebra of $\frak B_I$ generated by
$\{e_i:i\in J\}$;  following 325N, I will say that a member of $\frak C_J$
is `determined by coordinates in $J$'.   For $J\subseteq I$ let
$P_J:L^1(\frak B_I,\bar\nu_I)\to L^1(\frak C_J,\bar\nu_I\restr\frak C_J)$
be the conditional expectation operator.   Note that if $J$, $K\subseteq I$
then $P_JP_K=P_{J\cap K}$ (254Ra/458M(iii)).

It will be useful to start by looking at a particular
subset $W$ of $L^{\infty}(\frak B_I)$, being the set of linear
combinations $\sum_{i=0}^n\alpha_i\chi c_i$ where every $\alpha_i$ is
rational and every $c_i$ is determined by coordinates in a finite set.
Now $P_J[W]\subseteq W$ for every $J\subseteq I$.   \Prf\ If
$c\in\frak C_K$ where $K\subseteq I$ is finite, then

\Centerline{$P_J(\chi c)=P_JP_K(\chi c)=P_{J\cap K}(\chi c)
=\sum_{d\text{ is an atom of }\frak C_{J\cap K}}
   \Bover{\bar\nu_I(c\Bcap d)}{\bar\nu_Id}\chi d
\in W$.}

\noindent As $P_J$ is linear, this is enough.\ \QeD\  Observe also that if
$K\subseteq I$ is finite, then $P_K[W]$ is countable, being the set of
rational linear combinations of $\{\chi c:c\in\frak C_K\}$.

\medskip

{\bf (c)} Suppose, therefore, that we have a set $I$, a
$\|\,\|_{\infty}$-bounded family $\ofamily{\xi}{\kappa}{u_{\xi}}$
in $L^{\infty}(\frak B_I)^+$ with
$\inf_{\xi<\kappa}\int u_{\xi}\penalty-100=\delta>0$,  a $k\in\Bbb N$ and a
$\gamma<\delta^{k+1}$.   To begin with, let us suppose further that

\inset{($\alpha$) $u_{\xi}\le\chi 1$ for every $\xi<\kappa$,

($\beta$) $u_{\xi}\in W$, as described in (b),
for each $\xi<\kappa$;}

\noindent for each $\xi<\kappa$, let
$I_{\xi}\in[I]^{<\omega}$ be such that
$P_{I_{\xi}}u_{\xi}=u_{\xi}$.

\medskip

\quad{\bf (i)} Suppose that $\kappa=\omega$.   Because there are only
finitely many sequences $k_0,\ldots,k_m$ of strictly positive integers
with sum equal to $k+1$, we can use 525Rb a finite number of times to
find an infinite $\Gamma\subseteq\omega$ such that
$\int\prod_{j=0}^mu_{n_j}^{k_j}\ge\gamma$ whenever $\sum_{j=0}^mk_j=k+1$
and $n_0<\ldots<n_m$ belong to $\Gamma$.   But in this case we surely
have $\int\prod_{i=0}^ku_{n_i}\ge\gamma$ for all
$n_0,\ldots,n_k\in\Gamma$.

\medskip

\quad{\bf (ii)} Next, suppose that $\kappa>\omega$ is regular.   By the
$\Delta$-system Lemma (4A1Db) there is a $\Delta\in[\kappa]^{\kappa}$
such that $\family{\xi}{\Delta}{I_{\xi}}$ is a $\Delta$-system with root
$J$ say.
Since $P_J[W]$ is countable, there is a $v$ such that
$\Gamma=\{\xi:\xi\in\Delta,\,P_Ju_{\xi}=v\}$ has cardinal $\kappa$.
Of course

\Centerline{$\int v=\int u_{\xi}\ge\delta$}

\noindent for every $\xi\in\Gamma$.

By 458Lg again, $\family{\xi}{\Delta}{\frak C_{I_{\xi}}}$
is relatively independent over $\frak C_J$.
Now suppose that $\xi_0,\ldots,\xi_k$ belong to $\Gamma$.   Then

$$\eqalignno{\int\prod_{i=0}^ku_{\xi_i}
&\ge\int\prod_{i=0}^kP_Ju_{\xi_i}\cr
\displaycause{458Lh}
&=\int v^{k+1}
\ge\bigl(\int v\bigr)^{k+1}\cr
\displaycause{as in the proof of 525Ra}
&\ge\delta^{k+1}
\ge\gamma,\cr}$$

\noindent and this is what we need to know.

\medskip

\quad{\bf (iii)} Finally, suppose that $\kappa>\cf\kappa\ge\omega$.
Set $\lambda=\cf\kappa$ and let
$\ofamily{\zeta}{\lambda}{\kappa_{\zeta}}$ be a strictly increasing
family of regular cardinals greater than $\lambda$ and with supremum
$\kappa$.   For each $\zeta<\lambda$ let
$\Delta_{\zeta}\subseteq\kappa_{\zeta+1}\setminus\kappa_{\zeta}$ be a
set of size $\kappa_{\zeta+1}$ such that
$\family{\xi}{\Delta_{\zeta}}{I_{\xi}}$ is a $\Delta$-system with root
$J_{\zeta}$ say.   Set $J=\bigcup_{\zeta<\lambda}J_{\zeta}$;  note that
$\#(J)\le\lambda<\kappa_{\zeta+1}$ for every $\zeta$, so that

\Centerline{$\Delta'_{\zeta}=\{\xi:\xi\in\Delta_{\zeta},\,
(I_{\xi}\setminus
J_{\zeta})\cap(J\cup\bigcup_{\eta<\kappa_{\zeta}}I_{\eta})
  =\emptyset\}$}

\noindent still has cardinal $\kappa_{\zeta+1}$.

If $\zeta<\lambda$ and $\xi\in\Delta'_{\zeta}$,
$I_{\xi}\cap J$ is included in the
finite set $J_{\zeta}$;  so
$\{P_Ju_{\xi}:\xi\in\Delta'_{\zeta}\}$ is countable, and there is a
$v_{\zeta}$ such that
$\Delta''_{\zeta}=\{\xi:\xi\in\Delta'_{\zeta},\,P_Ju_{\xi}=v_{\zeta}\}$ has
cardinal $\kappa_{\zeta+1}$.   Note that

\Centerline{$\int v_{\zeta}=\int u_{\xi}\ge\delta$}

\noindent whenever $\zeta<\lambda$ and $\xi\in\Delta''_{\zeta}$.

Because $\lambda$ is regular, we can apply (i) or (ii) above to find an
$A\in[\lambda]^{\lambda}$ such that
$\int\prod_{i=0}^kv_{\zeta_i}\ge\gamma$ whenever
$\zeta_0,\ldots,\zeta_k\in A$.   Set
$\Gamma=\bigcup_{\zeta\in A}\Delta''_{\zeta}$;  because $A$ must be
cofinal with $\lambda$, $\#(\Gamma)=\kappa$.

If $\xi$, $\eta\in\Gamma$ are distinct, then
$I_{\xi}\cap I_{\eta}\subseteq J$.   So
$\family{\xi}{\Gamma}{\frak C_{I_{\xi}}}$ is
relatively independent over $\frak C_J$.   Take any
$\xi_0,\ldots,\xi_k\in\Gamma$;  for each $i\le k$, let $\zeta_i\in A$ be
such that $\xi_i\in\Delta''_{\zeta_i}$.   By 458Lh again,

\Centerline{$\int\prod_{i=0}^nu_{\xi_i}
\ge\int\prod_{i=0}^nP_Ju_{\xi_i}
=\int\prod_{i=0}^nv_{\zeta_i}\ge\gamma$,}

\noindent so we are done (provided ($\alpha$)-($\beta$) are true).

\medskip

{\bf (d)} Now let us unwind these conditions from the bottom.

\medskip

\quad{\bf (i)} If ($\alpha$) is true, but ($\beta$) is not,
take $\epsilon\in\ooint{0,\delta}$ such that
$(\delta-\epsilon)^{k+1}>\gamma+(k+1)\epsilon$.   For each $\xi<\kappa$,
let $u'_{\xi}\in W$ be such that $u'_{\xi}\le\chi 1$ and
$\int|u_{\xi}-u'_{\xi}|\le\epsilon$.   (Such a $u'_{\xi}$ exists because
$\bigcup\{\frak C_K:K\in[I]^{<\omega}\}$ is topologically dense in
$\frak B_I$ and $u\wedge\chi 1\in W$ for every $u\in W$.)   Then
$\int u'_{\xi}\ge\delta-\epsilon$ for each $\xi$, so we can apply
(c) to $\ofamily{\xi}{\kappa}{u'_{\xi}}$ to see that there is a
$\Gamma\in[\kappa]^{\kappa}$ such that
$\int\prod_{i=0}^ku'_{\xi_i}\ge\gamma+(k+1)\epsilon$ for all
$\xi_0,\ldots,\xi_k\in\Gamma$.   Now (because $u_{\xi}$ and $u'_{\xi}$
all lie between $0$ and $\chi 1$) we have

\Centerline{$\bigl|\prod_{i=0}^ku_{\xi_i}-\prod_{i=0}^ku'_{\xi_i}\bigr|
\le\sum_{i=0}^k|u_{\xi_i}-u'_{\xi_i}|$}

\noindent (see 285O), so that

\Centerline{$\int\prod_{i=0}^ku_{\xi_i}
\ge\int\prod_{i=0}^ku'_{\xi_i}
  -\sum_{i=0}^k\int|u_{\xi_i}-u'_{\xi_i}|
\ge\gamma$}

\noindent whenever $\xi_0,\ldots,\xi_n\in\Gamma$, and the theorem is
still true.

\medskip

\quad{\bf (ii)} Finally, for the general case, set
$M=1+\sup_{\xi<\kappa}\|u_{\xi}\|_{\infty}$, and
$u'_{\xi}=\Bover1Mu_{\xi}$ for $\xi<\kappa$.   Then every $u'_{\xi}$
belongs to $[0,\chi 1]$ and $\int u'_{\xi}\ge\Bover{\delta}M$.   By (i)
there is a $\Gamma\in[\kappa]^{\kappa}$ such that
$\int\prod_{i=0}^ku'_{\xi_i}\ge\Bover{\gamma}{M^{k+1}}$ for all
$\xi_0,\ldots,\xi_k\in\Gamma$;  in which case
$\int\prod_{i=0}^ku_{\xi_i}\ge\gamma$ for all
$\xi_0,\ldots,\xi_k\in\Gamma$.

This completes the proof.
}%end of proof of 525S

\leader{525T}{Corollary}\cmmnt{ ({\smc Argyros \& Kalamidas 82})} (a)
If $\kappa$ is an infinite cardinal and
$k\in\Bbb N$, $(\kappa,\kappa,k)$ is a measure-precaliber triple of
every probability algebra.

(b) If $\kappa$ is a cardinal of uncountable cofinality and
$k\in\Bbb N$, $(\kappa,\kappa,k)$ is a precaliber triple of every
measurable algebra.   In particular, every measurable algebra satisfies
Knaster's condition.

(c) If $\kappa$ is a cardinal of uncountable cofinality,
$(\frak A,\bar\mu)$ is a probability algebra, $k\ge 1$ and
$\ofamily{\xi}{\kappa}{a_{\xi}}$ is a family in $\frak A\setminus\{0\}$,
then there are a $\delta>0$ and a $\Gamma\in[\kappa]^{\kappa}$ such that
$\bar\mu(\inf_{\xi\in I}a_{\xi})\ge\delta$ for every $I\in[\Gamma]^k$.

(d)\discrversionA{\footnote{Revised 2009.}}{}
For any measurable algebra $\frak A$,
$\frak m(\frak A)\ge\frak m_{\text{K}}$;  and if
$\frak m(\frak A)>\omega_1$, then
$\frak m(\frak A)\ge\frak m_{\text{pc}\omega_1}$.
So if $\omega\le\kappa<\frak m_K$, $\kappa$ is a measure-precaliber of
every probability algebra.

\proof{ Really this is just the special case of 525S in which
every $u_{\xi}$ belongs to $\{\chi a:a\in\frak A\}$.

\medskip

{\bf (a)} If $(\frak A,\bar\mu)$
is a probability algebra and $\ofamily{\xi}{\kappa}{a_{\xi}}$ is a
family in $\frak A$ such that
$\inf_{\xi<\kappa}\bar\mu a_{\xi}=\delta>0$, take any
$\gamma\in\ooint{0,\delta^k}$.   Setting $u_{\xi}=\chi a_{\xi}$
for each $\xi$, $\int u_{\xi}\ge\delta$ for each $\xi$,
so there is a $\Gamma\in[\kappa]^{\kappa}$ such that
$\int\prod_{i=1}^ku_{\xi_i}\ge\gamma$ for every
$\xi_1,\ldots,\xi_k\in\Gamma$;  in which case
$\inf_{\xi\in J}a_{\xi}\ne 0$ for every $J\in[\Gamma]^k$.   As
$\ofamily{\xi}{\kappa}{a_{\xi}}$ is
arbitrary, $(\kappa,\kappa,k)$ is a measure-precaliber triple of
$(\frak A,\bar\mu)$.

\medskip

{\bf (b)} This now follows at once from 525Db, since any non-zero
measurable algebra can be given a probability measure.   Taking
$\kappa=\omega_1$ and $k=2$, we have Knaster's condition.

\medskip

{\bf (c)} For the quantitative version, we have only to note that there
must be some $\alpha>0$ such that $\#(\{\xi:\bar\mu a_{\xi}\ge\alpha\}$
has cardinal $\kappa$, and take $\delta<\alpha^k$.

\medskip

{\bf (d)} By (b), $\frak A$ satisfies
Knaster's condition;  it follows at once that
$\frak m(\frak A)\ge\frak m_{\text{K}}$, while $\sat(\frak A)\le\omega_1$.
If $\frak m(\frak A)>\omega_1$, then $\omega_1$ is a precaliber of
$\frak A$ (517Ig) so $\frak m(\frak A)\ge\frak m_{\text{pc}\omega_1}$.
By 525Fb, every infinite cardinal less than $\frak m_{\text{K}}$ is a
measure-precaliber of every probability algebra.
}%end of proof of 525T

\exercises{\leader{525X}{Basic exercises (a)}
%\spheader 525Xa
Let $(X,\Sigma,\mu)$ be any measure space and $\frak A$
its measure algebra.   (i) Show that
$(\frak A^+,\Bsupseteqshort)\prT(\Sigma\setminus\Cal N(\mu),\supseteq)$.
(ii) Show that a pair $(\kappa,\lambda)$ is a downwards precaliber pair
of $\Sigma\setminus\Cal N(\mu)$ iff it is a precaliber pair of
$\frak A$.
%525B 516Xb

\sqheader 525Xb Let $\frak A$ be a measurable algebra.   Show that
$\omega_1$ is a precaliber of $\frak A$ iff {\it either} $\frak A$ is
purely atomic {\it or} $\tau(\frak A)\le\omega$ and
$\cov\Cal N_{\omega}>\omega_1$ {\it or}
$\cov\Cal N_{\omega_1}>\omega_1$.    \Hint{525G, 523F.}
%525C

\sqheader 525Xc (i) Suppose that
$\add\Cal N_{\omega}=\cov\Cal N_{\omega}=\kappa$.   Show that $\kappa$
is not a precaliber of $\frak B_{\omega}$.
(ii)\dvAnew{2009}
Suppose that $\non\Cal N_{\omega}=\frak c$.   Show that $\frak c$ is not a
precaliber of $\frak B_{\omega}$.
%525C %525N

\spheader 525Xd Let $(X,\Sigma,\mu)$ be a complete strictly localizable
measure space and $\frak A$ its measure algebra.   Show that the
supported relation $(\Sigma\setminus\Cal N(\mu),\ni,X)$ has the same
precaliber pairs as the Boolean algebra $\frak A$.
%525C 525Xa

\spheader 525Xe Suppose that $(\kappa,\lambda)$ is a precaliber pair of
every measurable algebra, that $I$ is a set, and that
$X\subseteq\BbbR^I$ is a compact set such that
$\#(\{i:x(i)\ne 0\})<\lambda$ for every $x\in X$.   Show that
$\#(\{i:x(i)\ne 0\})<\kappa$ for every $x$ belonging to the closed
convex hull of $X$ in $\BbbR^I$.   \Hint{461I.}
%525Ca  Marciszewski n84 ?

\spheader 525Xf\dvAnew{2009}
Suppose that $\lambda\le\kappa$ are infinite cardinals,
$(\frak A,\bar\mu)$ is a homogeneous probability algebra,
and that $\gamma<1$ is such that whenever
$\ofamily{\xi}{\kappa}{a_{\xi}}$ is a family in
$\frak A$ and $\bar\mu a_{\xi}\ge\gamma$ for every $\xi<\kappa$,
there is a $\Gamma\in[\kappa]^{\lambda}$ such that
$\{a_{\xi}:\xi\in\Gamma\}$ is centered.   Show that $(\kappa,\lambda)$ is
a measure-precaliber pair of $(\frak A,\bar\mu)$.   \Hint{given that
$\inf_{\xi<\kappa}\bar\mu a_{\xi}>0$, take
$(\frak C,\bar\lambda)=\Tensorhat_m(\frak A,\bar\mu)\cong(\frak A,\bar\mu)$
to be the probability
algebra free product of a large finite number of copies of
$(\frak A,\bar\mu)$, and consider $c_{\xi}=\sup_{j<m}\varepsilon_ja_{\xi}$
for $\xi<\kappa$.}
%525D

\spheader 525Xg\dvAnew{2009}
Let $\frak A$ be a Boolean algebra, and $\lambda$, $\kappa$ cardinals such
that $(\kappa,\lambda)$ is a
measure-precaliber pair of every probability algebra.   Suppose that
$A\subseteq\frak A\setminus\{0\}$ has positive intersection number.
Show that if $\ofamily{\xi}{\kappa}{a_{\xi}}$ is a
family in $A$, then there is a $\Gamma\in[\kappa]^{\lambda}$ such that
$\{a_{\xi}:\xi\in\Gamma\}$ is centered.
%525D

\spheader 525Xh Let $\kappa$ be a cardinal such that ($\alpha$)
$\lambda^{\omega}<\kappa$ for every $\lambda<\kappa$ ($\beta$)
$\lambda^{\omega}<\cf\kappa$ for every $\lambda<\cf\kappa$.   Show that
$\kappa$ is a measure-precaliber of every probability algebra.
%525M 525K  D\v{z}amonja \& Plebanek 04, 4.1

%{\leader{525Y}{Further exercises (a)}

}%end of exercises

\leader{525Z}{Problem} Can we, in ZFC, find an infinite cardinal
$\kappa$ which is not a measure-precaliber of all probability algebras?
\cmmnt{From 525N we see that a negative answer will require a model of
set theory in which $2^{\kappa}>\kappa^+$ for all strong limit cardinals
$\kappa$ of countable cofinality;  for such models see
{\smc Foreman \& Woodin 91}, {\smc Cummings 92}.}

\leaveitout{(Haydon) If $\sequencen{\kappa_n}$ is a sequence of
precalibers of measurable algebras, is $\sup_n\kappa_n$ a
measure-precaliber of probability algebras?}

\endnotes{
\Notesheader{525}
There seem to be three methods of proving that a cardinal is a
precaliber of a measure algebra.   First, we have the counting arguments
of 516L;  since we know something about the centering numbers of measure
algebras (524Me), this gives us a start (see the proof of 525K).   Next,
we can try to use Martin numbers, as in 517Ig and 525F;  since we can
relate the
Martin number of a measure algebra to the cardinals of \S523 (524Md), we
get the formulation 525J.   In third place, we have arguments based on
the special
structure of measure algebras, using 525H to apply
$\Delta$-system theorems from infinitary combinatorics.   Subject to the
generalized continuum hypothesis, these ideas are enough to answer the
most natural questions (525O).   Without this simplification, they leave
conspicuous gaps.   The most important seems to be 525Z.   Even if we
know all the cardinals
$\add\Cal N_{\kappa}$, $\cov\Cal N_{\kappa}$, $\non\Cal N_{\kappa}$ and
$\cf\Cal N_{\kappa}$ of \S523, we may still not be able to determine
which cardinals are precalibers;   525Xb is an exceptional special case.

I have presented this section with a bias towards measure-precalibers
rather than precalibers.   When there is a difference, the former search
deeper.   `Cofinality $\omega_1$' has a rather special position in this
theory (525Ib), deriving from the combinatorial arguments of 5A1H.
}%end of notes

\discrpage

