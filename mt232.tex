\frfilename{mt232.tex}
\versiondate{5.6.02}
\copyrightdate{1994}

\def\chaptername{The Radon-Nikod\'ym theorem}
\def\sectionname{The Radon-Nikod\'ym theorem}

\newsection{232}

I come now to the chief theorem of this chapter, one of the central
results of measure theory, relating countably additive functionals to
indefinite integrals.   The objective is to give a complete description
of the functionals which can arise as indefinite integrals of integrable
functions (232E).   These can be characterized as the `truly continuous'
additive functionals (232Ab).   A more commonly used concept, and one
adequate in many cases, is that of `absolutely continuous' additive
functional (232Aa);  I spend the first few paragraphs (232B-232D) on
elementary facts about truly continuous and absolutely continuous
functionals.   I end the section with a discussion of the decomposition
of general countably additive functionals (232I).

\leader{232A}{Absolutely continuous functionals} Let $(X,\Sigma,\mu)$ be
a measure space and $\nu:\Sigma\to\Bbb R$ a finitely additive
functional.

\header{232Aa}{\bf (a)} $\nu$ is {\bf absolutely continuous} with
respect to $\mu$ (sometimes written `$\nu\ll\mu$') if
for every $\epsilon>0$ there is a $\delta>0$ such that
$|\nu E|\le\epsilon$ whenever  $E\in\Sigma$ and $\mu E\le\delta$.


\header{232Ab}{\bf (b)} $\nu$ is {\bf truly continuous} with respect to
$\mu$ if for
every $\epsilon>0$ there are $E\in\Sigma$, $\delta>0$ such that
$\mu E$ is finite and $|\nu F|\le\epsilon$ whenever $F\in\Sigma$ and
$\mu(E\cap F)\le\delta$.


\header{232Ac}{\bf (c)}\cmmnt{ For reference, I add another definition
here.}   If $\nu$ is countably additive, it is
{\bf singular} with respect to $\mu$
if there is a set $F\in\Sigma$ such that $\mu F=0$ and
$\nu E=0$ whenever $E\in\Sigma$ and $E\subseteq X\setminus F$.


\leader{232B}{Proposition} Let $(X,\Sigma,\mu)$ be a measure
space and $\nu:\Sigma\to\Bbb R$ a finitely additive functional.

(a) If $\nu$ is countably additive, it is absolutely continuous with
respect to $\mu$ iff $\nu E=0$
whenever $\mu E=0$.

(b) $\nu$ is truly continuous with respect to $\mu$ iff ($\alpha$) it is
countably additive ($\beta$) it is absolutely continuous with respect to
$\mu$ ($\gamma$) whenever $E\in\Sigma$ and $\nu E\ne 0$
there is an $F\in\Sigma$ such that $\mu F<\infty$ and
$\nu(E\cap F)\ne 0$.

(c) If $(X,\Sigma,\mu)$ is $\sigma$-finite, then $\nu$ is truly
continuous with respect
to $\mu$ iff it is countably additive and absolutely
continuous with respect to $\mu$.

(d) If $(X,\Sigma,\mu)$ is totally finite, then $\nu$ is truly
continuous with respect to $\mu$ iff it is absolutely continuous with
respect to $\mu$.

\proof{{\bf (a)(i)}  If $\nu$ is absolutely continuous with
respect to $\mu$ and $\mu E=0$, then $\mu E\le\delta$ for every
$\delta>0$, so $|\nu E|\le\epsilon$ for every $\epsilon>0$ and $\nu
E=0$.

\medskip

\quad{\bf (ii)}  \Quer\ Suppose, if possible, that $\nu E=0$ whenever
$\mu E=0$, but $\nu$ is not absolutely continuous.   Then there is an
$\epsilon>0$ such that for every $\delta>0$ there is an $E\in\Sigma$
such that $\mu E\le\delta$ but
$|\nu E|\ge\epsilon$.   For each $n\in\Bbb
N$ we may choose an $F_n\in\Sigma$ such that $\mu F_n\le 2^{-n}$ and
$|\nu F_n|\ge\epsilon$.   Consider $F=\bigcap_{n\in\Bbb N}\bigcup_{k\ge
n}F_k$.   Then we have

\Centerline{$\mu F\le\inf_{n\in\Bbb N}\mu(\bigcup_{k\ge n}F_k)
\le\inf_{n\in\Bbb N}\sum_{k=n}^{\infty}2^{-k}=0$,}

\noindent so $\mu F=0$.

Now recall that by 231Eb there is an $H\in\Sigma$ such that $\nu G\ge 0$
when $G\in\Sigma$ and $G\subseteq H$, and $\nu G\le 0$ when $G\in\Sigma$
and $G\cap H=\emptyset$.   As in 231F, set $\nu_1 G=\nu(G\cap H)$,
$\nu_2G=-\nu(G\setminus H)$ for $G\in\Sigma$, so that $\nu_1$ and
$\nu_2$ are totally finite measures, and $\nu_1F=\nu_2F=0$ because
$\mu(F\cap H)=\mu(F\setminus H)=0$.   Consequently

\Centerline{$0=\nu_iF=\lim_{n\to\infty}\nu_i(\bigcup_{m\ge n}F_m)
\ge\limsup_{n\to\infty}\nu_iF_n$}

\noindent for both $i$, and

\Centerline{$0=\lim_{n\to\infty}(\nu_1F_n+\nu_2F_n)
\ge\liminf_{n\to\infty}|\nu F_n|\ge\epsilon>0$,}

\noindent which is absurd.\ \Bang

\medskip

{\bf (b)(i)} Suppose that $\nu$ is truly continuous with respect to
$\mu$.   It is obvious from the definitions that $\nu$ is absolutely
continuous with respect to $\mu$.   If $\nu E\ne 0$, there
must be an $F$ of finite measure such that $|\nu G|<|\nu E|$ whenever
$G\cap F=\emptyset$, so that $|\nu(E\setminus F)|<|\nu E|$ and
$\nu(E\cap F)\ne 0$.   This deals with the conditions ($\beta$) and
($\gamma$).

To check that $\nu$ is countably additive, let $\sequencen{E_n}$ be a
disjoint sequence in $\Sigma$, with union $E$, and $\epsilon>0$.   Let
$\delta>0$, $F\in\Sigma$ be such $\mu F<\infty$ and $|\nu G|\le\epsilon$
whenever $G\in\Sigma$ and $\mu(F\cap G)\le\delta$.   Then

\Centerline{$\sum_{n=0}^{\infty}\mu(E_n\cap F)\le\mu F<\infty$,}

\noindent so there is an $n\in\Bbb N$ such that
$\sum_{i=n}^{\infty}\mu(E_i\cap F)\le\delta$.   Take any $m\ge n$ and
consider $E^*_m=\bigcup_{i\le m}E_i$.   We have

%$$\eqalign{|\nu E-\sum_{i=0}^m\nu E_i|
%&=|\nu E-\nu E^*_m|\cr
%&=|\nu(E\setminus E^*_m)|\cr
%&\le\epsilon\cr}$$

\Centerline{$|\nu E-\sum_{i=0}^m\nu E_i|
=|\nu E-\nu E^*_m|
=|\nu(E\setminus E^*_m)|
\le\epsilon$,}


\noindent because

\Centerline{$\mu(F\cap E\setminus E^*_m)
=\sum_{i=m+1}^{\infty}\mu(F\cap E_i)
\le\delta$.}

\noindent As $\epsilon$ is arbitrary,

\Centerline{$\nu E=\sum_{i=0}^{\infty}\nu E_i$;}

\noindent as $\sequencen{E_n}$ is arbitrary, $\nu$ is countably
additive.

\medskip

\quad{\bf (ii)} Now suppose that $\nu$ satisfies the three conditions.
By 231F, $\nu$ can be expressed as the difference of two non-negative
countably additive functionals $\nu_1$, $\nu_2$;  set
$\nuprime=\nu_1+\nu_2$, so that $\nuprime$ is a non-negative countably
additive
functional and $|\nu F|\le\nuprime F$ for every $F\in\Sigma$.    Set

\Centerline{$\gamma=\sup\{\nuprime F:F\in\Sigma,\,\mu
F<\infty\}\le\nuprime
X<\infty$,}

\noindent and
choose a sequence $\sequencen{F_n}$ of sets of finite measure such that
$\lim_{n\to\infty}\nuprime F_n=\gamma$;  set $F^*=\bigcup_{n\in\Bbb
N}F_n$.
If $G\in\Sigma$ and $G\cap F^*=\emptyset$ then $\nu G=0$.   \Prf\Quer\
Otherwise, by condition ($\gamma$), there is an $F\in\Sigma$ such that
$\mu F<\infty$ and $\nu(G\cap F)\ne 0$.   It follows that

\Centerline{$\nuprime(F\setminus F^*)\ge\nuprime(F\cap G)\ge|\nu(F\cap
G)|>0$,}

\noindent and there must be an $n\in\Bbb N$ such that

\Centerline{$\gamma<\nuprime F_n+\nuprime(F\setminus
F^*)=\nuprime(F_n\cup(F\setminus
F^*))\le\nuprime(F\cup F_n)\le\gamma$}

\noindent because $\mu(F\cup F_n)<\infty$;  but this is impossible.
\Bang\Qed

Setting $F_n^*=\bigcup_{k\le n}F_k$ for each $n$, we have
$\lim_{n\to\infty}\nuprime(F^*\setminus F^*_n)=0$.
Take any $\epsilon>0$, and (using condition ($\beta$)) let $\delta>0$
be such that $|\nu E|\le{1\over 2}\epsilon$ whenever $\mu E\le\delta$.
Let $n$ be such that
$\nuprime(F^*\setminus F_n^*)\le\bover12\epsilon$.   Now if $F\in\Sigma$
and $\mu(F\cap F^*_n)\le\delta$ then

$$\eqalign{|\nu F|
&\le|\nu(F\cap F^*_n)|
   +|\nu(F\cap F^*\setminus F^*_n)|
   +|\nu(F\setminus F^*)|\cr
&\le\Bover12\epsilon+\nuprime(F\cap F^*\setminus F^*_n)+0\cr
&\le\Bover12\epsilon+\nuprime(F^*\setminus F^*_n)
\le\Bover12\epsilon+\Bover12\epsilon
=\epsilon.\cr}$$

\noindent And $\mu F^*_n<\infty$.   As $\epsilon$ is arbitrary, $\nu$ is
truly continuous.

\medskip

{\bf (c)} Now suppose that $(X,\Sigma,\mu)$ is $\sigma$-finite and that
$\nu$ is countably additive and absolutely continuous with respect to
$\mu$.   Let $\sequencen{X_n}$ be a non-decreasing sequence of sets of finite measure covering $X$ (211D).   If $\nu E\ne 0$, then
$\lim_{n\to\infty}\nu(E\cap X_n)\ne 0$, so $\nu(E\cap X_n)\ne 0$ for
some $n$.   This shows that $\nu$ satisfies condition ($\gamma$) of (b),
so is truly continuous.

Of course the converse of this fact is already covered by (b).

\medskip

{\bf (d)} Finally, suppose that $\mu X<\infty$ and that $\nu$ is
absolutely continuous with respect to $\mu$.   Then it must be truly
continuous, because we can take $F=X$ in the definition 232Ab.
}%end of proof of 232B

\leader{232C}{Lemma} Let $(X,\Sigma,\mu)$ be a measure space and $\nu$,
$\nuprime$ two countably additive functionals on $\Sigma$ which are
truly continuous with respect to $\mu$.   Take $c\in\Bbb R$ and
$H\in\Sigma$,
and set $\nu_HE=\nu(E\cap H)$\cmmnt{, as in 231De.}   Then
$\nu+\nuprime$, $c\nu$ and $\nu_H$ are
all truly continuous with respect to $\mu$, and $\nu$ is expressible as
the difference of non-negative countably additive functionals which are
truly continuous with respect to $\mu$.

\proof{ Let $\epsilon>0$.   Set $\eta=\epsilon/(2+|c|)>0$.
Then there are $\delta$, $\delta'>0$ and $E$, $E'\in\Sigma$ such that
$\mu E<\infty$, $\mu E'<\infty$ and $|\nu F|\le\eta$ whenever $\mu(F\cap
E)\le\delta$, $|\nuprime F|\le\eta$ whenever $\mu(F\cap E)\le\delta'$.
Set
$\delta^*=\min(\delta,\delta')>0$, $E^*=E\cup E'\in\Sigma$;  then

\Centerline{$\mu
E^*\le\mu E+\mu E'<\infty$.}

\noindent   Suppose that $F\in\Sigma$ and $\mu(F\cap
E^*)\le\delta^*$;  then

\Centerline{$\mu(F\cap H\cap E)\le\mu(F\cap E)\le\delta^*\le\delta$,
\quad $\mu(F\cap E')\le\delta^*$}

\noindent so

\Centerline{$|(\nu+\nuprime)F|\le|\nu
F|+|\nuprime F|\le\eta+\eta\le\epsilon$,}

\Centerline{$|(c\nu)F|=|c||\nu F|\le|c|\eta\le\epsilon$,}

\Centerline{$|\nu_HF|=|\nu(F\cap H)|\le\eta\le\epsilon$.}

\noindent As $\epsilon$ is arbitrary, $\nu+\nuprime$, $c\nu$ and $\nu_H$
are
all truly continuous.

Now, taking $H$ from 231Eb, we see that $\nu_1=\nu_H$ and
$\nu_2=-\nu_{X\setminus H}$ are truly continuous and non-negative, and
$\nu=\nu_1-\nu_2$ is the difference of truly continuous measures.
}%end of proof of 232C

\leader{232D}{Proposition} Let $(X,\Sigma,\mu)$ be a measure space, and
$f$ a $\mu$-integrable real-valued function.   For $E\in\Sigma$ set
$\nu E=\int_Ef$.   Then $\nu:\Sigma\to\Bbb R$ is a countably additive
functional and is truly continuous with respect to $\mu$, therefore
absolutely continuous with respect to $\mu$.

\proof{ Recall that $\int_Ef=\int f\times\chi E$ is defined
for every $E\in\Sigma$ (131Fa).   So $\nu:\Sigma\to\Bbb R$ is
well-defined.   If $E$, $F\in\Sigma$ are disjoint then

$$\eqalign{\nu(E\cup F)
&=\int f\times\chi(E\cup F)
=\int (f\times\chi E)+(f\times\chi F)\cr
&=\int f\times\chi E+\int f\times\chi F
=\nu E+\nu F,\cr}$$

\noindent so $\nu$ is finitely additive.

Now 225A, without using the phrase `truly continuous', proved exactly
that $\nu$ is truly continuous with respect to $\mu$.   It follows from
232Bb that $\nu$ is countably additive and absolutely continuous.
}%end of proof of 232D

\medskip

\noindent{\bf Remark} The functional $E\mapsto\int_Ef$ is called the
{\bf indefinite integral} of $f$.


\leader{232E}{}\cmmnt{ We are now at last ready for the theorem.

\medskip

\noindent}{\bf The Radon-Nikod\'ym theorem}
Let $(X,\Sigma,\mu)$ be a measure space and $\nu:\Sigma\to\Bbb R$ a
function.   Then the following are equiveridical:

(i) there is a $\mu$-integrable function $f$
such that $\nu E=\int_Ef$ for every $E\in\Sigma$;

(ii) $\nu$ is finitely additive and truly continuous with respect to $\mu$.

\proof{{\bf (a)} If $f$ is a $\mu$-integrable real-valued
function and $\nu E=\int_Ef$ for
every $E\in\Sigma$, then 232D tells us that $\nu$ is finitely additive
and truly continuous.

\medskip

{\bf (b)} In the other direction, suppose that $\nu$ is
finitely additive and truly continuous;  note that (by 232B(a-b))
$\nu E=0$ whenever $\mu E=0$.   To begin with, suppose that $\nu$ is
non-negative and not zero.

In this case, there is a non-negative simple function $f$ such that
$\int f>0$ and $\int_Ef\le\nu E$ for every $E\in\Sigma$.   \Prf\ Let
$H\in\Sigma$ be such that $\nu H>0$;  set $\epsilon={1\over 3}\nu H>0$.
Let $E\in\Sigma$, $\delta>0$ be such that $\mu E<\infty$ and $\nu
F\le\epsilon$ whenever $F\in\Sigma$ and $\mu(F\cap E)\le\delta$;  then
$\nu(H\setminus E)\le\epsilon$ so $\nu E\ge\nu(H\cap E)\ge2\epsilon$ and
$\mu E\ge\mu(H\cap E)>0$.   Set
$\mu_EF=\mu(F\cap E)$ for every $F\in\Sigma$;  then $\mu_E$ is a
countably
additive functional on $\Sigma$.   Set $\nuprime=\nu-\alpha\mu_E$, where
$\alpha=\epsilon/\mu E$;  then $\nuprime$ is a countably additive
functional
and $\nuprime E>0$.   By 231Eb, as usual, there is a set $G\in\Sigma$
such that
$\nuprime F\ge 0$ if $F\in\Sigma$, $F\subseteq G$, but $\nuprime F\le 0$
if $F\in\Sigma$ and $F\cap G=\emptyset$.
As $\nuprime(E\setminus G)\le 0$,

\Centerline{$0<\nuprime E\le\nuprime(E\cap G)\le\nu(E\cap G)$}

\noindent and $\mu(E\cap G)>0$.   Set
$f={\alpha}\chi(E\cap G)$;  then $f$ is a non-negative simple function
and $\int f={\alpha}\mu(E\cap G)>0$.

If $F\in\Sigma$ then $\nuprime(F\cap G)\ge 0$, that is,

\Centerline{$\nu(F\cap G)\ge\alpha\mu_E(F\cap G)
=\alpha\mu(F\cap E\cap G)=\int_Ff$.}

\noindent So

\Centerline{$\nu F\ge\nu(F\cap G)\ge\int_Ff$,}

\noindent as required.\ \Qed

\medskip

{\bf (c)} Still supposing that $\nu$ is a non-negative, truly continuous
additive functional,   let $\Phi$ be the set of non-negative
simple functions $f:X\to\Bbb R$ such that
$\int_Ef\le\nu E$ for every $E\in\Sigma$;  then
the constant function $\tbf{0}$ belongs to $\Phi$, so $\Phi$ is not
empty.

If $f$, $g\in\Phi$ then $f\vee g\in\Phi$, where $(f\vee
g)(x)=\max(f(x),g(x))$ for $x\in X$.   \Prf\    Set
$H=\{x:(f-g)(x)\ge 0\}\in\Sigma$;   then $f\vee g=(f\times\chi
H)+(g\times\chi(X\setminus H))$ is a non-negative simple function, and
for any $E\in\Sigma$,

\Centerline{$\int_Ef\vee g=\int_{E\cap H}f+\int_{E\setminus H}g
\le\nu(E\cap H)+\nu(E\setminus H)=\nu E$. \Qed}

Set

\Centerline{$\gamma=\sup\{\int f:f\in\Phi\}\le\nu X<\infty$.}

\noindent    Choose a
sequence $\sequencen{f_n}$ in $\Phi$  such that $\lim_{n\to\infty}\int
f_n=\gamma$.   For each $n$, set $g_n=f_0\vee f_1\vee\ldots\vee f_n$;
then $g_n\in\Phi$ and $\int f_n\le\int g_n\le\gamma$ for each $n$, so
$\lim_{n\to\infty}\int g_n=\gamma$.   By B.Levi's theorem,
$f=\lim_{n\to\infty}g_n$ is integrable and $\int f=\gamma$.   Note that
if $E\in\Sigma$ then

\Centerline{$\int_Ef=\lim_{n\to\infty}\int_Ef_n\le\nu E$.}


\vthsp\Quer\ Suppose, if possible, that there is an $H\in\Sigma$ such
that $\int_Hf\ne\nu H$.   Set

\Centerline{$\nu_1F=\nu F-\int_Ff\ge 0$}

\noindent for every
$F\in\Sigma$;  then by (a) of this proof and 232C, $\nu_1$ is a truly
continuous finitely additive functional, and we are supposing that
$\nu_1\ne 0$.   By (b) of this proof, there
is a non-negative simple function $g$ such that $\int_Fg\le\nu_1F$ for
every $F\in\Sigma$ and $\int g>0$.   Take $n\in\Bbb N$ such that
$\int f_n+\int g>\gamma$.   Then $f_n+g$ is a non-negative simple
function and

\Centerline{$\int_F(f_n+g)=\int_Ff_n+\int_Fg\le\int_Ff+\int_Fg
=\nu F-\nu_1F+\int_Fg\le\nu F$}

\noindent for any $F\in\Sigma$, so $f_n+g\in\Phi$, and

\Centerline{$\gamma<\int f_n+\int g=\int f_n+g\le\gamma$,}

\noindent which is absurd.\ \BanG\ Thus we have $\int_Hf=\nu H$ for
every $H\in\Sigma$.

\medskip

{\bf (d)} This proves the theorem for non-negative $\nu$.   For general
$\nu$, we need only observe that $\nu$ is expressible as $\nu_1-\nu_2$,
where $\nu_1$ and $\nu_2$ are non-negative truly continuous countably
additive functionals, by 232C;  so that there are integrable functions
$f_1$, $f_2$ such that $\nu_iF=\int_Ff_i$ for both $i$ and every
$F\in\Sigma$.
Of course $f=f_1-f_2$ is integrable and $\nu F=\int_Ff$ for every
$F\in\Sigma$.   This completes the proof.
}%end of proof of 232E

\leader{232F}{Corollary} Let $(X,\Sigma,\mu)$ be a $\sigma$-finite
measure
space and $\nu:\Sigma\to\Bbb R$ a function.   Then there is a
$\mu$-integrable function $f$ such that $\nu E=\int_Ef$ for every
$E\in\Sigma$ iff $\nu$ is countably additive and absolutely continuous
with respect to $\mu$.

\proof{ Put 232Bc and 232E together.
}%end of proof of 232F

\leader{232G}{Corollary} Let $(X,\Sigma,\mu)$ be a  totally finite
measure
space and $\nu:\Sigma\to\Bbb R$ a function.   Then there is a
$\mu$-integrable function $f$ on $X$ such that $\nu E=\int_Ef$ for every
$E\in\Sigma$ iff $\nu$ is finitely additive and absolutely continuous
with respect to $\mu$.

\proof{ Put 232Bd and 232E together.
}%end of proof of 232G

\leader{232H}{Remarks}\cmmnt{ {\bf (a)} Most authors are satisfied with
232F as the
`Radon-Nikod\'ym theorem'.   In my view the problem of identifying
indefinite integrals is of sufficient importance to justify an analysis
which applies to all measure spaces, even if it requires a new concept
(the notion of `truly continuous' functional).

\header{232Hb}{\bf (b)} I ought to offer an example of an absolutely
continuous functional which is not truly continuous.   A simple one is
the following.   Let $X$ be any uncountable set.   Let $\Sigma$ be the
countable-cocountable $\sigma$-algebra of subsets of $X$ and $\nu$ the
countable-cocountable measure on $X$ (211R).   Let $\mu$ be the
restriction to $\Sigma$ of counting measure on $X$.   If $\mu E=0$ then
$E=\emptyset$ and $\nu E=0$, so
$\nu$ is absolutely continuous.   But for any $E$ of finite measure we
have $\nu(X\setminus E)=1$, so $\nu$ is not truly continuous.   See also
232Xf(i).

\header{232Hc}{\bf *(c)} The space $(X,\Sigma,\mu)$ of this example
is, in terms of
the classification developed in Chapter 21, somewhat irregular;  for
instance, it is neither locally determined nor localizable, and
therefore not strictly localizable, though it is complete and
semi-finite.   Can this phenomenon occur in
a strictly localizable measure space?   We are led here into a
fascinating question.   Suppose, in (b), I used the same idea, but with
$\Sigma=\Cal PX$.   No difficulty arises in constructing $\mu$;  but can
there now be a $\nu$ with the required properties, that is, a non-zero
countably additive functional from $\Cal PX$ to $\Bbb R$ which is zero
on all finite sets?   This is the `Banach-Ulam problem', on which I
have written extensively elsewhere ({\smc Fremlin 93}), and to which I
will return in Volume 5.   The present question is touched on again
in 363S in Volume 3.

\header{232Hd}{\bf (d)} Following the Radon-Nikod\'ym theorem, the
question immediately arises:  for a given $\nu$,
how much possible variation is there in the corresponding $f$?   The
answer is straightforward enough:  two integrable functions $f$ and $g$
give rise to the same indefinite integral iff they are equal almost
everywhere (131Hb).

\header{232He}{\bf (e)} I have stated the Radon-Nikod\'ym theorem in
terms of arbitrary integrable functions, meaning to interpret
`integrability'
in a wide sense, according to the conventions of Volume 1.   However,
given a truly continuous countably additive functional $\nu$, we can ask
whether there is in any sense a canonical integrable function
representing it.   The answer is no.   But we certainly do not need to
take arbitrary integrable functions of the type considered in Chapter
12.   If $f$ is any integrable function, there is a conegligible set $E$
such that $f\restr E$ is measurable, and now we can find a
conegligible measurable set $G\subseteq E\cap\dom f$;  if we set
$g(x)=f(x)$ for $x\in G$, $0$ for $x\in X\setminus G$, then $f\eae g$,
so $g$ has the same indefinite integral as $f$ (as noted in (d) just
above), while $g$ is measurable and has domain $X$.   Thus we can make a
trivial, but sometimes convenient, refinement to the theorem:  if
$(X,\Sigma,\mu)$ is a measure space, and $\nu:\Sigma\to\Bbb R$ is
finitely additive and truly continuous with respect to $\mu$, then there
is a $\Sigma$-measurable $\mu$-integrable function $g:X\to\Bbb R$ such
that $\int_Eg=\nu E$ for every $E\in\Sigma$.

\header{232Hf}{\bf (f)}
It is convenient to introduce now a general definition.} %end of comment
If $(X,\Sigma,\mu)$ is a measure space and $\nu$ is a
$[-\infty,\infty]$-valued
functional defined on a family of subsets of $X$, I will say that
a $[-\infty,\infty]$-valued function $f$ defined on a subset of $X$ is a
{\bf Radon-Nikod\'ym derivative} of $\nu$ with respect to $\mu$ if
$\int_Efd\mu$ is defined\cmmnt{ (in the sense of 214D)} and
equal to $\nu E$ for
every $E\in\dom\nu$.   \cmmnt{Thus the integrable functions called $f$ in
232E-232G %232E 232F 232G
are all `Radon-Nikod\'ym derivatives';  later on we shall have less
well-regulated examples.}

\cmmnt{When $\nu$ is a measure and $f$ is non-negative,
$f$ may be called a {\bf density function}.}

\cmmnt{\header{232Hg}{\bf (g)} Throughout the work above I have taken
it that $\nu$ is defined on the whole domain $\Sigma$ of $\mu$.   In
some of the
most important applications, however, $\nu$ is defined only on some
smaller $\sigma$-algebra $\Tau$.   In this case we commonly seek to
apply the same results with $\mu\restrp\Tau$ in place of $\mu$.
}%end of comment

\leader{232I}{}{\bf The Lebesgue decomposition of a countably additive
functional:  Proposition} (a) Let $(X,\Sigma,\mu)$ be a measure
space and $\nu:\Sigma\to\Bbb R$ a countably additive functional.
Then $\nu$ has unique expressions as

\Centerline{$\nu=\nu_s+\nu_{ac}=\nu_s+\nu_{tc}+\nu_e$,}

\noindent where $\nu_s$ is singular with respect to $\mu$, 
$\nu_{ac}$ is absolutely continuous with respect to $\mu$, $\nu_{tc}$ 
is truly continuous with respect to $\mu$,
and $\nu_e$ is absolutely continuous
with respect to $\mu$ and zero on every set of finite measure.

(b) If $X=\BbbR^r$, $\Sigma$ is the algebra of Borel sets in $\BbbR^r$
and $\mu$ is the restriction of Lebesgue measure to $\Sigma$, then $\nu$
is uniquely expressible as $\nu_p+\nu_{cs}+\nu_{ac}$ where $\nu_{ac}$ is
absolutely continuous with respect to $\mu$, $\nu_{cs}$ is singular with
respect to
$\mu$ and zero on singletons, and $\nu_pE=\sum_{x\in E}\nu_p\{x\}$ for
every $E\in\Sigma$.

\proof{{\bf (a)(i)} Suppose first that $\nu$ is non-negative.   In
this case, set

\Centerline{$\nu_sE=\sup\{\nu(E\cap F):F\in\Sigma,\,\mu F=0\}$,}

\Centerline{$\nu_tE=\sup\{\nu(E\cap F):F\in\Sigma,\,\mu F<\infty\}$.}

\noindent Then both $\nu_s$ and $\nu_t$ are countably additive.
\prooflet{\Prf\
Surely $\nu_s\emptyset=\nu_t\emptyset=0$.   Let $\sequencen{E_n}$ be a
disjoint sequence in $\Sigma$ with union $E$.   \grheada\   If
$F\in\Sigma$
and $\mu F=0$, then

\Centerline{$\nu(E\cap F)=\sum_{n=0}^{\infty}\nu(E_n\cap F)
\le\sum_{n=0}^{\infty}\nu_s(E_n)$;}

\noindent  as $F$ is arbitrary,

\Centerline{$\nu_sE\le\sum_{n=0}^{\infty}\nu_sE_n$.}


\noindent \grheadb\ If $F\in\Sigma$ and $\mu F<\infty$, then

\Centerline{$\nu(E\cap F)=\sum_{n=0}^{\infty}\nu(E_n\cap F)
\le\sum_{n=0}^{\infty}\nu_t(E_n)$;}

\noindent as $F$ is arbitrary,

\Centerline{$\nu_tE\le\sum_{n=0}^{\infty}\nu_tE_n$.}

\noindent  \grheadc\ If
$\epsilon>0$, then (because $\sum_{n=0}^{\infty}\nu E_n=\nu E<\infty$)
there is an $n\in\Bbb N$ such that $\sum_{k=n+1}^{\infty}\nu
E_k\le\epsilon$.   Now, for each $k\le n$, there is an $F_k\in\Sigma$
such that $\mu F_k=0$ and
$\nu(E_k\cap F_k)\ge\nu_sE_k-{{\epsilon}\over{n+1}}$.   In this case,
$F=\bigcup_{k\le n}F_k\in\Sigma$, $\mu F=0$ and

\Centerline{$\nu_sE\ge\nu(E\cap F)
\ge\sum_{k=0}^n\nu(E_k\cap F_k)\ge\sum_{k=0}^n\nu_sE_k-\epsilon
\ge\sum_{k=0}^{\infty}\nu_sE_k-2\epsilon$,}

\noindent because

\Centerline{$\sum_{k=n+1}^{\infty}\nu_sE_k
\le\sum_{k=n+1}^{\infty}\nu E_k\le\epsilon$.}

\noindent   As $\epsilon$ is arbitrary,

\Centerline{$\nu_sE\ge\sum_{k=0}^{\infty}\nu_sE_k$.}

\noindent
\grheadd\ Similarly, for each $k\le n$, there is an $F'_k\in\Sigma$
such that
$\mu F'_k<\infty$ and
$\nu(E_k\cap F'_k)\ge\nu_tE_k-{{\epsilon}\over{n+1}}$.   In this case,
$F'=\bigcup_{k\le n}F'_k\in\Sigma$, $\mu F'<\infty$ and

\Centerline{$\nu_tE\ge\nu(E\cap F')\ge\sum_{k=0}^n\nu(E_k\cap F'_k)
\ge\sum_{k=0}^n\nu_tE_k-\epsilon
\ge\sum_{k=0}^{\infty}\nu_tE_k-2\epsilon$,}

\noindent because

\Centerline{$\sum_{k=n+1}^{\infty}\nu_tE_k
\le\sum_{k=n+1}^{\infty}\nu E_k\le\epsilon$.}

\noindent   As $\epsilon$ is arbitrary,

\Centerline{$\nu_tE\ge\sum_{k=0}^{\infty}\nu_tE_k$.}

\noindent \grheade\ Putting these together,
$\nu_sE=\sum_{n=0}^{\infty}\nu_sE_n$ and
$\nu_tE=\sum_{n=0}^{\infty}\nu_tE_n$.   As $\sequencen{E_n}$ is
arbitrary, $\nu_s$ and $\nu_t$ are countably additive.\ \Qed}


\medskip

\quad{\bf (ii)}  Still supposing that $\nu$ is
non-negative, if we choose a sequence $\sequencen{F_n}$ in
$\Sigma$ such
that $\mu F_n=0$ for each $n$ and $\lim_{n\to\infty}\nu F_n=\nu_sX$,
then $F^*=\bigcup_{n\in\Bbb N}F_n$ has $\mu F^*=0$, $\nu F^*=\nu_sX$;
so that $\nu_s(X\setminus F^*)=0$, and $\nu_s$ is singular with respect
to $\mu$ in the sense of 232Ac.

Note that $\nu_sF=\nu F$ whenever $\mu F=0$.   So if we write
$\nu_{ac}=\nu-\nu_s$, then $\nu_{ac}$ is a countably additive functional
and $\nu_{ac}F=0$ whenever $\mu F=0$;  that is, $\nu_{ac}$ is absolutely
continuous with respect to $\mu$.

If we write $\nu_{tc}=\nu_t-\nu_s$, then $\nu_{tc}$ is a non-negative
countably additive functional;  $\nu_{tc}F=0$ whenever $\mu F=0$, and if
$\nu_{tc}E>0$ there is a set $F$ with $\mu F<\infty$ and
$\nu_{tc}(E\cap F)>0$.   So $\nu_{tc}$ is truly continuous with respect
to $\mu$, by 232Bb.   Set $\nu_e=\nu-\nu_t=\nu_{ac}-\nu_{tc}$.

Thus for any non-negative countably additive functional $\nu$, we have
expressions

\Centerline{$\nu=\nu_s+\nu_{ac}$,\quad $\nu_{ac}=\nu_{tc}+\nu_e$}

\noindent where $\nu_s$, $\nu_{ac}$, $\nu_{tc}$ and $\nu_e$ are all
non-negative
countably additive functionals, $\nu_s$ is singular with respect to
$\mu$, $\nu_{ac}$ and $\nu_e$ are absolutely continuous with respect to
$\mu$, $\nu_{tc}$ is truly continuous with respect to $\mu$, and
$\nu_eF=0$ whenever $\mu F<\infty$.


\medskip

\quad{\bf (iii)} For general countably additive functionals
$\nu:\Sigma\to\Bbb R$, we can express $\nu$ as
$\nuprime-\nu\mskip0.5mu''$,
where $\nuprime$ and $\nu\mskip0.5mu''$ are
non-negative countably additive functionals.   If we define
$\nuprime_s$,
$\nu_s'',\ldots,\nu_e''$ as in (i)-(ii), we get countably additive
functionals

\Centerline{$\nu_s=\nu_s'-\nu_s''$,
\quad $\nu_{ac}=\nu_{ac}'-\nu_{ac}''$,
\quad $\nu_{tc}=\nu_{tc}'-\nu_{tc}''$,
\quad $\nu_e=\nu_e'-\nu_e''$}

\noindent such that $\nu_s$ is singular with respect to $\mu$ (if $F'$,
$F''$ are such that

\Centerline{$\mu F=\mu F'=\nu_s'(X\setminus F)=\nu_s''(X\setminus
F)=0$,}

\noindent then $\mu(F'\cup F'')=0$ and $\nu_sE=0$ whenever $E\subseteq
X\setminus(F'\cup F'')$), $\nu_{ac}$ is absolutely continuous with
respect to $\mu$, $\nu_{tc}$ is truly continuous with respect to $\mu$,
and $\nu_eF=0$ whenever $\mu F<\infty$, while

\Centerline{$\nu=\nu_s+\nu_{ac}=\nu_s+\nu_{tc}+\nu_e$.}

\medskip

\quad{\bf (iv)} Moreover, these decompositions are unique.
\prooflet{\Prf\grheada\
If, for instance, $\nu=\tilde\nu_s+\tilde\nu_{ac}$, where $\tilde\nu_s$
is singular and $\tilde\nu_{ac}$ is absolutely continuous with respect
to $\mu$, let $F$, $\tilde F$ be such that $\mu F=\mu\tilde F=0$ and
$\tilde\nu_sE=0$ whenever $E\cap \tilde F=\emptyset$, $\nu_sE=0$
whenever $E\cap F=\emptyset$;  then we must have

\Centerline{$\nu_{ac}(E\cap(F\cup\tilde
F))=\tilde\nu_{ac}(E\cap(F\cup\tilde F))=0$}

\noindent
for every $E\in\Sigma$, so

\Centerline{$\nu_sE=\nu(E\cap(F\cup\tilde
F))=\tilde\nu_sE$}

\noindent for every $E\in\Sigma$.
Thus $\tilde\nu_s=\nu_s$ and $\tilde\nu_{ac}=\nu_{ac}$.
\medskip

\qquad\grheadb\  Similarly, if $\nu_{ac}=\tilde\nu_{tc}+\tilde\nu_e$
where $\tilde\nu_{tc}$ is truly continuous with respect to $\mu$ and
$\tilde\nu_eF=0$ whenever $\mu F<\infty$, then there are sequences
$\sequencen{F_n}$, $\sequencen{\tilde F_n}$ of sets of finite measure
such that $\nu_{tc}F=0$ whenever
$F\cap\bigcup_{n\in\Bbb N}F_n=\emptyset$ and
$\tilde\nu_{tc}F=0$ whenever
$F\cap\bigcup_{n\in\Bbb N}\tilde F_n=\emptyset$.   Write
$F^*=\bigcup_{n\in\Bbb N}(F_n\cup\tilde F_n)$;
then $\tilde\nu_eE=\nu_eE=0$ whenever $E\subseteq F^*$ and
$\tilde\nu_{tc}E=\nu_{tc}E=0$ whenever $E\cap F^*=\emptyset$, so
$\nu_eE=\nu_{ac}(E\setminus F^*)=\tilde\nu_eE$ for every $E\in\Sigma$,
and $\nu_e=\tilde\nu_e$, $\nu_{tc}=\tilde\nu_{tc}$.\ \Qed}

\medskip

{\bf (b)} In this case, $\mu$ is $\sigma$-finite (cf.\ 211P), so every
absolutely continuous countably additive functional is truly continuous
(232Bc), and we
shall always have $\nu_e=0$, $\nu_{ac}=\nu_{tc}$.   But in the other
direction we know that singleton sets, and therefore countable sets, are
all measurable.   We therefore have a
further decomposition $\nu_s=\nu_p+\nu_{cs}$, where there is a countable
set $K\subseteq\BbbR^r$ with $\nu_pE=0$ whenever $E\in\Sigma$,
$E\cap K=\emptyset$, and $\nu_{cs}$ is singular with respect to $\mu$
and zero on countable sets.   \prooflet{\Prf\ (i) If $\nu\ge 0$, set

\Centerline{$\nu_pE=\sup\{\nu(E\cap K):K\subseteq\BbbR^r$ is
countable$\}$;}

\noindent   just
as with $\nu_s$, dealt with in (a) above, $\nu_p$ is countably
additive and there is a countable $K\subseteq\BbbR^r$ such that
$\nu_pE=\nu(E\cap K)$ for every $E\in\Sigma$.    (ii) For general $\nu$,
we can express $\nu$ as $\nuprime-\nu\mskip0.5mu''$ where $\nuprime$ and
$\nu\mskip0.5mu''$ are
non-negative, and write $\nu_p=\nu_p'-\nu_p''$.   (iii) $\nu_p$ is
characterized by saying that there is a countable set $K$ such that
$\nu_pE=\nu(E\cap K)$ for every $E\in\Sigma$ and $\nu\{x\}=0$ for every
$x\in \BbbR^r\setminus K$.   (iv) So if we set $\nu_{cs}=\nu_s-\nu_p$,
$\nu_{cs}$ will be singular with respect to $\mu$ and zero on countable
sets.\ \Qed}

Now, for any $E\in\Sigma$,

\Centerline{$\nu_pE=\nu(E\cap K)=\sum_{x\in K\cap E}\nu\{x\}
=\sum_{x\in E}\nu\{x\}$.}

}%end of proof of 232I

\medskip

\noindent{\bf Remark} The expression $\nu=\nu_p+\nu_{cs}+\nu_{ac}$ of
(b) is the {\bf Lebesgue decomposition} of $\nu$.

\exercises{
\leader{232X}{Basic exercises (a)}
Let $(X,\Sigma,\mu)$ be a measure space and $\nu:\Sigma\to\Bbb
R$ a countably additive functional which is absolutely continuous with
respect to $\mu$.   Show that the following are equiveridical:  (i)
$\nu$
is truly continuous with respect to $\mu$;  (ii) there is a sequence
$\sequencen{E_n}$ in $\Sigma$ such that $\mu E_n<\infty$ for every
$n\in\Bbb N$ and $\nu F=0$ whenever $F\in\Sigma$ and
$F\cap\bigcup_{n\in\Bbb N}E_n=\emptyset$.

\sqheader 232Xb Let $g:\Bbb R\to\Bbb R$ be a bounded
non-decreasing function and
$\mu_g$ the associated Lebesgue-Stieltjes measure (114Xa).   Show
that $\mu_g$ is absolutely continuous (equivalently, truly continuous)
with respect to Lebesgue measure iff the restriction of $g$
to any closed bounded interval is absolutely continuous in the sense of
225B.

\header{232Xc}{\bf (c)} Let $X$ be a set and $\Sigma$ a $\sigma$-algebra
of subsets of
$X$;  let $\nu:\Sigma\to\Bbb R$ be a countably additive functional.
Let $\Cal I$ be an {\bf ideal} of $\Sigma$, that is, a subset of
$\Sigma$ such that ($\alpha$) $\emptyset\in\Cal I$ ($\beta$)
$E\cup F\in\Cal I$ for all $E$, $F\in\Cal I$ ($\gamma$) if $E\in\Sigma$,
$F\in\Cal I$ and $E\subseteq F$ then $E\in\Cal I$.   Show that $\nu$ has
a unique decomposition as $\nu=\nu_{\Cal I}+\nuprime_{\Cal I}$, where
$\nu_{\Cal I}$ and $\nuprime_{\Cal I}$ are countably additive
functionals,
$\nuprime_{\Cal I}E=0$ for every $E\in\Cal I$, and whenever
$E\in\Sigma$, $\nu_{\Cal I}E\ne 0$ there is an $F\in\Cal I$ such that
$\nu_{\Cal I}(E\cap F)\ne 0$.

\sqheader 232Xd Let $X$ be a non-empty set and $\Sigma$ a
$\sigma$-algebra of subsets of $X$.   Show that for any sequence
$\sequencen{\nu_n}$ of countably additive functionals on $\Sigma$ there
is a probability measure $\mu$ on $X$, with domain $\Sigma$, such that
every $\nu_n$ is absolutely continuous with respect to $\mu$.
\Hint{start with the case $\nu_n\ge 0$.}

\header{232Xe}{\bf (e)} Let $(X,\Sigma,\mu)$ be a measure space and
$(X,\hat\Sigma,\hat\mu)$ its completion (212C).   Let
$\nu:\Sigma\to\Bbb R$ be an additive functional such that $\nu E=0$
whenever $\mu E=0$.
Show that $\nu$ has a unique extension to an additive functional
$\hat\nu:\hat\Sigma\to\Bbb R$ such that $\hat\nu E=0$ whenever
$\hat\mu E=0$.

\spheader 232Xf Let $\Cal F$ be an ultrafilter on $\Bbb N$
including the filter $\{\Bbb N\setminus I:I\subseteq\Bbb N$ is
finite$\}$ (2A1O).    Define $\nu:\Cal P\Bbb N\to\{0,1\}$ by setting
$\nu E=1$ if $E\in\Cal F$, $0$ for $E\in\Cal P\Bbb N\setminus\Cal F$.
(i) Let $\mu_1$ be counting measure on $\Cal P\Bbb N$.   Show that $\nu$
is additive and absolutely continuous with respect to $\mu_1$, but is
not truly continuous.   (ii) Define $\mu_2:\Cal P\Bbb N\to[0,1]$ by
setting $\mu_2E=\sum_{n\in E}2^{-n-1}$.   Show that $\nu$ is zero on
$\mu_2$-negligible sets, but is not absolutely continuous with respect
to $\mu_2$.

\header{232Xg}{\bf (g)} Rewrite this section in terms of complex-valued
additive functionals.

\spheader 232Xh Let $(X,\Sigma,\mu)$ be a measure space, and $\nu$ and
$\lambda$ additive functionals on $\Sigma$ of which $\nu$ is positive
and countably additive, so that $(X,\Sigma,\nu)$ also is a measure
space.   (i) Show that if $\nu$ is absolutely continuous with respect to
$\mu$ and $\lambda$ is absolutely continuous with respect to $\nu$, then
$\lambda$ is absolutely continuous with respect to $\mu$.   (ii) Show
that if $\nu$ is truly continuous with respect to $\mu$ and $\lambda$ is
absolutely continuous with respect to $\nu$ then $\lambda$ is truly
continuous with respect to $\mu$.
%232B

\leader{232Y}{Further exercises (a)} Let $(X,\Sigma,\mu)$ be a measure
space and $\nu:\Sigma\to\Bbb R$ a finitely additive functional.   If
$E$, $F$, $H\in\Sigma$ and $\mu H<\infty$ set
$\rho_H(E,F)=\mu(H\cap(E\symmdiff F))$.   (i) Show that $\rho_H$ is a
pseudometric on $\Sigma$ (2A3Fa).   (ii)
Let $\frak T$ be the topology on $\Sigma$ generated by
$\{\rho_H:H\in\Sigma,\,\mu H<\infty\}$ (2A3Fc).   Show that $\nu$ is
continuous for $\frak T$
iff it is truly continuous in the sense of 232Ab.
($\frak T$ is the topology of {\bf convergence in
measure} on $\Sigma$.)

\header{232Yb}{\bf (b)} For a non-decreasing function
$F:[a,b]\to\Bbb R$, where $a<b$,
let $\nu_F$ be the corresponding Lebesgue-Stieltjes measure.   Show that
if we define
$(\nu_F)_{ac}$, etc., with regard to Lebesgue measure on $[a,b]$, as in
232I, then

\Centerline{$(\nu_F)_p=\nu_{F_p}$, \quad$(\nu_F)_{ac}=\nu_{F_{ac}}$,
\quad$(\nu_F)_{cs}=\nu_{F_{cs}}$,}

\noindent where $F_p$, $F_{cs}$ and $F_{ac}$ are defined as in 226C.

\header{232Yc}{\bf (c)} Extend the idea of (b) to general functions $F$
of bounded variation.

\header{232Yd}{\bf (d)} Extend the ideas of (b) and (c) to open,
half-open and unbounded intervals (cf.\ 226Yb).

\header{232Ye}{\bf (e)} Let $(X,\Sigma,\mu)$ be a measure space and
$(X,\tilde\Sigma,\tilde\mu)$ its c.l.d.\ version (213E).   Let
$\nu:\Sigma\to\Bbb R$ be an additive functional which is truly
continuous with respect to $\mu$.   Show that $\nu$ has a unique
extension to a functional $\tilde\nu:\tilde\Sigma\to\Bbb R$ which is
truly continuous with respect to $\tilde\mu$.

\header{232Yf}{\bf (f)} Let $(X,\Sigma,\mu)$ be a measure space and $f$
a $\mu$-integrable real-valued function.   Show that the indefinite
integral of $f$ is the unique countably additive functional
$\nu:\Sigma\to\Bbb R$ such that whenever $E\in\Sigma$ and $f(x)\in[a,b]$
for almost every $x\in E$, then $a\mu E\le\nu E\le b\mu E$.

\header{232Yg}{\bf (g)} Say that two bounded additive functionals
$\nu_1$, $\nu_2$ on an algebra $\Sigma$ of sets are {\bf mutually
singular} if for any $\epsilon>0$ there is an $H\in\Sigma$ such that

\Centerline{$\sup\{|\nu_1F|:F\in\Sigma,\,F\subseteq H\}\le\epsilon$,}

\Centerline{$\sup\{|\nu_2F|:F\in\Sigma,\,F\cap H=\emptyset\}
\le\epsilon$.}

\quad {(i)} Show that $\nu_1$ and $\nu_2$ are mutually singular iff, in
the language of 231Ya-231Yb, $|\nu_1|\wedge|\nu_2|=0$.

\quad{(ii)} Show that if $\Sigma$ is a $\sigma$-algebra and $\nu_1$ and
$\nu_2$ are countably additive, then they are mutually singular iff
there is an $H\in\Sigma$ such that $\nu_1F=0$ whenever $F\in\Sigma$ and
$F\subseteq H$, while $\nu_2F=0$ whenever $F\in\Sigma$ and
$F\cap H=\emptyset$.

\quad{(iii)} Show that if $\nu_s$, $\nu_{tc}$ and $\nu_e$ are defined
from $\nu$ and $\mu$ as in 232I, then each pair of the three are
mutually singular.

\spheader 232Yh Let $(X,\Sigma,\mu)$ be a measure space and $f$ a
non-negative real-valued function which is integrable over $X$;  let
$\nu$ be its indefinite integral.   Show that for any function
$g:X\to\Bbb R$, $\int g\,d\nu=\int f\times g\,d\mu$ in the sense that if
one of these is defined in $[-\infty,\infty]$ so is the other, and they
are then equal.   \Hint{start with simple functions $g$.}

\spheader 232Yi Let $(X,\Sigma,\mu)$ be a measure space, $f$ an
integrable function, and $\nu:\Sigma\to\Bbb R$ the indefinite integral
of $f$.   Show that $|\nu|$, as defined in 231Ya, is the
indefinite integral of $|f|$.
%232D

\spheader 232Yj Let $X$ be a set, $\Sigma$ a $\sigma$-algebra of subsets
of $X$, and $\nu:\Sigma\to\Bbb R$ a countably additive functional.
Show that $\nu$ has a Radon-Nikod\'ym
derivative with respect to $|\nu|$ as defined in 231Ya, and that any
such derivative has modulus equal to $1\,\,|\nu|$-a.e.
%232D

\spheader 232Yk (H.K\"onig)
Let $X$ be a set and $\mu$, $\nu$ two measures on $X$ with
the same domain $\Sigma$.   For $\alpha\ge 0$, $E\in\Sigma$ set
$(\alpha\mu\wedge\nu)(E)
=\inf\{\alpha\mu(E\cap F)+\nu(E\setminus F):F\in\Sigma\}$ (cf.\
112Ya\formerly{1{}12Yb}).
Show that the following are equiveridical:  (i) $\nu E=0$ whenever
$\mu E=0$;  (ii) $\sup_{\alpha\ge 0}(\alpha\mu\wedge\nu)(E)=\nu E$ for
every $E\in\Sigma$.
}%end of exercises

\endnotes{
\Notesheader{232} The Radon-Nikod\'ym theorem must be on any list of the
half-dozen most important theorems of measure theory, and not only the
theorem itself, but the techniques necessary to prove it, are at the
heart of the subject.   In my book {\smc Fremlin 74} I discussed a
variety of more or less abstract versions of the theorem and of the
method, to some of which I will return in \S\S327 and 365 of the
next volume.

As I have presented it here, the essence of the proof is split between
231E and 232E.   I think we can distinguish the following elements.
Let $\nu$ be a countably additive functional.

\inset{(i) $\nu$ is bounded (231Ea).}

\inset{(ii) $\nu$ is expressible as the difference of non-negative
functionals (231F).}

\noindent   (I gave this as a corollary of 231Eb, but it can also be
proved by simpler methods, as in 231Ya.)

\inset{(iii) If $\nu>0$, there is an integrable $f$ such that
$0<\nu_f\le\nu$,}

\noindent writing $\nu_f$ for the indefinite integral of $f$.   (This is
the point at which we really do need the Hahn decomposition 231Eb.)

\inset{(iv) The set $\Psi=\{f:\nu_f\le\nu\}$ is closed under countable
suprema, so there is an $f\in\Psi$ maximising $\int f$.}

\noindent (In part (b) of the proof of 232E, I spoke of simple
functions;  but this was solely to simplify the technical details, and
the same argument works if we apply it to $\Psi$ instead of $\Phi$.
Note the use here of B.Levi's theorem.)

\inset{(v) Take $f$ from (iv) and use (iii) to show that $\nu-\nu_f=0$.}

\noindent Each of the steps (i)-(iv) requires a non-trivial idea, and
the importance of the theorem lies not only in its remarkable direct
consequences in the rest of this chapter and elsewhere, but in the
versatility and power of these ideas.

I introduce the idea of `truly continuous' functional in order to
give a reasonably straightforward account of the status of the
Radon-Nikod\'ym theorem in non-$\sigma$-finite measure spaces.   Of
course the whole point is that a truly continuous functional, like an
indefinite integral, must be concentrated on a $\sigma$-finite part of
the space (232Xa), so that 232E, as stated, can be deduced easily from
the standard form 232F.   I dare to use the word `truly' in this
context because this kind of continuity does indeed correspond to a
topological notion (232Ya).

There is a possible trap in the definition I give of `absolutely
continuous' functional.   Many authors use the condition of 232Ba as a
definition, saying that $\nu$ is absolutely continuous with respect to
$\mu$ if $\nu E=0$ whenever $\mu E=0$.   For countably additive
functionals this coincides with the $\epsilon$-$\delta$ formulation in
232Aa;  but for other additive functionals this
need not be so (232Xf(ii)).   Mostly the distinction is insignificant,
but I
note that in 232Bd it is critical, since $\nu$ there is not assumed to
be countably additive.

In 232I I describe one of the many ways of decomposing a countably
additive functional into mutually singular parts with special
properties.   In 231Yf-231Yg I have already suggested a method of
decomposing an additive functional into the sum of a countably additive
part and a `purely finitely additive' part.   All these results have
natural expressions in terms of the ordered linear space of bounded
additive functionals on an algebra (231Yc).
}%end of notes

\discrpage


