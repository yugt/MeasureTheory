\frfilename{mt211.tex}
\versiondate{20.11.03}
     
\def\chaptername{Taxonomy of measure spaces}
\def\sectionname{Definitions}
     
\newsection{211}
     
I start with a list of definitions, corresponding to the concepts which
I have found to be of value in distinguishing different types of measure
space.   Necessarily, the significance of many of these ideas is likely
to be obscure until you have encountered some of the obstacles which
arise later on.   Nevertheless, you will I hope be
able to deal with these definitions on a formal, abstract basis, and to
follow the elementary arguments involved in establishing the
relationships between them (211L).
     
In 216C-216E %216C 216D 216E
below I will give three substantial examples to demonstrate the rich variety of objects which the definition of `measure space'
encompasses.   In the present section, therefore, I content myself with
very brief descriptions of sufficient cases to show at least that
each of the definitions here discriminates between different spaces
(211M-211R).
     
\leader{211A}{Definition} Let $(X,\Sigma,\mu)$ be a measure space.
Then $\mu$, or $(X,\Sigma,\mu)$, is ({\bf \Caratheodory}) {\bf complete} if whenever
$A\subseteq E\in\Sigma$ and $\mu E=0$ then $A\in\Sigma$\cmmnt{;  that is, if every negligible subset of $X$ is measurable}.
     
\leader{211B}{Definition} Let $(X,\Sigma,\mu)$ be a measure space.
Then $(X,\Sigma,\mu)$ is a {\bf probability space} if $\mu X=1$.
In this case $\mu$ is called a {\bf probability} or {\bf probability
measure}.
     
\leader{211C}{Definition} Let $(X,\Sigma,\mu)$ be a measure space.
Then $\mu$, or $(X,\Sigma,\mu)$, is {\bf totally finite} if $\mu
X<\infty$.
     
\leader{211D}{Definition} Let $(X,\Sigma,\mu)$ be a measure space.
Then $\mu$, or $(X,\Sigma,\mu)$, is {\bf $\sigma$-finite} if there is a
sequence $\sequencen{E_n}$ of measurable sets of finite measure such
that $X=\bigcup_{n\in\Bbb N}E_n$.
     
\cmmnt{\medskip
     
\noindent{\bf Remark} Note that in this case we can set
     
\Centerline{$F_n=E_n\setminus\bigcup_{i<n}E_i$,\quad $G_n=\bigcup_{i\le
n}E_i$}
     
\noindent for each $n$, to obtain a partition $\sequencen{F_n}$ of
$X$ (that is, a disjoint cover of $X$) into measurable sets of finite measure, and a non-decreasing sequence
$\sequencen{G_n}$ of sets of finite measure covering $X$.
}%end of comment
     
\leader{211E}{Definition} Let $(X,\Sigma,\mu)$ be a measure space.
Then $\mu$, or $(X,\Sigma,\mu)$, is {\bf strictly localizable} or {\bf
decomposable} if there is a partition
$\langle X_i\rangle_{i\in I}$ of $X$ into measurable sets of finite 
measure such that 
     
\Centerline{$\Sigma
=\{E:E\subseteq X,\,E\cap X_i\in\Sigma\Forall i\in I\}$,}
     
\Centerline{$\mu E=\sum_{i\in I}\mu(E\cap X_i)$ for every $E\in\Sigma$.}
     
\noindent I will call such a family $\langle X_i\rangle_{i\in I}$ a {\bf
decomposition} of $X$.
     
\cmmnt{\medskip
     
\noindent{\bf Remark} In this context, we can interpret the sum
$\sum_{i\in I}\mu(E\cap X_i)$ simply as
     
\Centerline{$\sup\{\sum_{i\in J}\mu(E\cap X_i):J$ is a finite subset of
$I\}$,}
     
\noindent taking $\sum_{i\in\emptyset}\mu(E\cap X_i)=0$, because we are
concerned only with sums of non-negative terms (cf.\ 112Bd).
}%end of comment
     
\leader{211F}{Definition} Let $(X,\Sigma,\mu)$ be a measure space.
Then $\mu$, or $(X,\Sigma,\mu)$, is {\bf semi-finite} if whenever
$E\in\Sigma$ and $\mu E=\infty$ there is an $F\subseteq E$ such that
$F\in\Sigma$ and $0<\mu F<\infty$.
     
\leader{211G}{Definition} Let $(X,\Sigma,\mu)$ be a measure space.
Then $\mu$, or $(X,\Sigma,\mu)$, is {\bf localizable} if it
is semi-finite and, for every $\Cal E\subseteq\Sigma$, there is an
$H\in\Sigma$ such that (i) $E\setminus H$ is negligible for every
$E\in\Cal E$ (ii) if $G\in\Sigma$ and $E\setminus G$ is negligible for
every $E\in\Cal E$, then $H\setminus G$ is negligible.
It will be
convenient to call such a set $H$  {\bf an essential supremum} of 
$\Cal E$ in $\Sigma$.
     
\cmmnt{\medskip
     
\noindent{\bf Remark} The definition  here is clumsy,
because really the concept applies to measure {\it algebras}
rather than to measure {\it spaces} (see 211Ya-211Yb).   However, the
present definition  can be made to work (see 213N, 241G, 243G below) and
enables us to proceed
without a formal introduction to the concept of `measure algebra'
before the time comes to do the job properly in Volume 3.
}
     
\leader{211H}{Definition} Let $(X,\Sigma,\mu)$ be a measure space.
Then $\mu$, or $(X,\Sigma,\mu)$, is {\bf locally determined} if it is
semi-finite and
     
\Centerline{$\Sigma=\{E:E\subseteq X,\,E\cap F\in\Sigma$ whenever
$F\in\Sigma$ and $\mu F<\infty\}$\dvro{.}{;}}
     
\cmmnt{\noindent that is to say, for any $E\in\Cal PX\setminus\Sigma$
there is
an $F\in\Sigma$ such that $\mu F<\infty$ and $E\cap F\notin\Sigma$.}
     
\leader{211I}{Definition} Let $(X,\Sigma,\mu)$ be a measure space.    A
set $E\in\Sigma$ is an {\bf atom} for $\mu$ if $\mu E>0$ and
whenever $F\in\Sigma$, $F\subseteq E$ one of $F$, $E\setminus F$ is
negligible.
     
\leader{211J}{Definition} Let $(X,\Sigma,\mu)$ be a measure space.
Then $\mu$, or $(X,\Sigma,\mu)$, is {\bf atomless} or {\bf diffused} if 
there are no atoms for $\mu$.
\cmmnt{(Note that this is {\it not} the same thing as saying that all 
finite sets are negligible;  see 211R below.   Some authors use the word 
{\bf continuous} in this context.)}
     
\leader{211K}{Definition} Let $(X,\Sigma,\mu)$ be a measure space.
Then $\mu$, or $(X,\Sigma,\mu)$, is {\bf purely atomic} if whenever
$E\in\Sigma$ and $E$ is not negligible there is an atom for $\mu$
included in $E$.
     
\medskip
     
\noindent{\bf Remark} \cmmnt{Recall that a measure $\mu$ on a set $X$ is 
{\bf point-supported} if $\mu$ measures every subset of $X$
and $\mu E=\sum_{x\in E}\mu\{x\}$ for every $E\subseteq X$ (112Bd).}   
Every point-supported measure is purely atomic\prooflet{, because 
$\{x\}$ must be an atom whenever
$\mu\{x\}>0$}, but not every purely atomic measure is
point-supported\cmmnt{ (211R)}.
     
% {\smc Ramachandran 02} uses `discrete' for `purely atomic'
     
\leader{211L}{}\cmmnt{ The relationships between the concepts above are in
a sense very straightforward;  all the direct implications in which one
property implies another are given in the next theorem.
     
\medskip
     
\noindent}{\bf Theorem} (a) A probability space is
totally finite.
     
(b) A totally finite measure space is $\sigma$-finite.
     
(c) A $\sigma$-finite measure space is strictly localizable.
     
(d) A strictly localizable measure space is localizable and locally
determined.
     
(e) A localizable measure space is semi-finite.
     
(f) A locally determined measure space is semi-finite.
     
\proof{ (a), (b), (e) and (f) are trivial.
     
\medskip
     
{\bf (c)} Let $(X,\Sigma,\mu)$ be a $\sigma$-finite measure space;  let
$\sequencen{F_n}$ be a disjoint sequence of measurable sets of finite
measure covering $X$ (see the remark in 211D).   If $E\in\Sigma$, then
of course $E\cap F_n\in\Sigma$ for every $n\in \Bbb N$, and
     
\Centerline{$\mu E=\sum_{n=0}^{\infty}\mu(E\cap
F_n)=\sum_{n\in \Bbb N}\mu(E\cap F_n)$.}
     
\noindent   If $E\subseteq X$ and $E\cap
F_n\in\Sigma$ for every $n\in \Bbb N$, then
     
\Centerline{$E=\bigcup_{n\in \Bbb N}E\cap F_n\in \Sigma$.}
     
\noindent So $\sequencen{F_n}$ is a decomposition of $X$
and $(X,\Sigma,\mu)$ is strictly localizable.
     
\medskip
     
{\bf (d)} Let $(X,\Sigma,\mu)$ be a strictly localizable measure space;
let $\langle X_i\rangle_{i\in I}$ be a decomposition of $X$.
     
\medskip
     
\quad{\bf (i)} Let $\Cal E$ be a family of measurable subsets of $X$.
Let $\Cal F$ be the family of measurable sets $F\subseteq X$ such that
$\mu(F\cap E)=0$ for every $E\in\Cal E$.   Note that
$\emptyset\in\Cal F$ and, if
$\sequencen{F_n}$ is any sequence in $\Cal F$, then
$\bigcup_{n\in\Bbb N}F_n\in\Cal F$.   For each $i\in I$, set $\gamma_i=\sup\{\mu(F\cap X_i):F\in\Cal F\}$ and choose a sequence $\sequencen{F_{in}}$ in $\Cal F$ such that $\lim_{n\to\infty}\mu(F_{in}\cap X_i)=\gamma_i$;  set
     
\Centerline{$F_i=\bigcup_{n\in\Bbb N}F_{in}\in\Cal F$.}
     
\noindent   Set
     
\Centerline{$F=\bigcup_{i\in I}F_i\cap X_i\subseteq X$}
     
\noindent and $H=X\setminus F$.
     
We see that $F\cap X_i=F_i\cap X_i$ for each $i\in I$ (because
$\langle X_i\rangle_{i\in I}$ is disjoint), so $F\in\Sigma$ and
$H\in \Sigma$.   For any $E\in\Cal E$,
     
\Centerline{$\mu(E\setminus H)=\mu(E\cap F)
=\sum_{i\in I}\mu(E\cap F\cap X_i)
=\sum_{i\in I}\mu(E\cap F_i\cap X_i)=0$}
     
\noindent because every $F_i$ belongs to $\Cal F$.   Thus $F\in\Cal F$.
If $G\in\Sigma$ and
$\mu(E\setminus G)=0$ for every $E\in\Cal E$, then $X\setminus G$ and
$F'=F\cup(X\setminus G)$ belong to $\Cal F$.   So
$\mu(F'\cap X_i)\le\gamma_i$ for each $i\in I$.   But also
$\mu(F\cap X_i)\ge\sup_{n\in\Bbb N}\mu(F_{in}\cap X_i)=\gamma_i$, so $\mu(F\cap X_i)=\mu(F'\cap X_i)$ for each $i$.   {\it Because $\mu X_i$ is finite,}
it follows that $\mu((F'\setminus F)\cap X_i)=0$, for each $i$.
Summing over $i$, $\mu(F'\setminus F)=0$, that is,
$\mu(H\setminus G)=0$.
     
Thus $H$ is an essential supremum for $\Cal E$ in
$\Sigma$.   As $\Cal E$ is arbitrary, $(X,\Sigma,\mu)$ is localizable.
     
\medskip
     
\quad{\bf (ii)}
If $E\in\Sigma$ and $\mu E=\infty$, then there is some $i\in I$ such
that
     
\Centerline{$0<\mu(E\cap X_i)\le\mu X_i<\infty$;}
     
\noindent  so $(X,\Sigma,\mu)$ is
semi-finite.    If $E\subseteq X$ and $E\cap F\in\Sigma$ whenever $\mu
F<\infty$, then, in particular, $E\cap X_i\in\Sigma$ for every $i\in I$,
so $E\in\Sigma$;  thus $(X,\Sigma,\mu)$ is locally determined.
}%end of proof of 211L
     
\leader{211M}{Example:  Lebesgue measure}\cmmnt{ Let us consider Lebesgue measure in the light of the concepts above.}   Write $\mu$ for Lebesgue measure on $\BbbR^r$\cmmnt{ and $\Sigma$ for its domain}.
     
\spheader 211Ma $\mu$ is complete\prooflet{, because it is
constructed by \Caratheodory's method;  if $A\subseteq E$ and $\mu
E=0$, then $\mu^*A=\mu^*E=0$ (writing $\mu^*$ for Lebesgue outer
measure), so, for any $B\subseteq\Bbb R$,
     
\Centerline{$\mu^*(B\cap A)+\mu^*(B\setminus A)\le 0+\mu^*B=\mu^*B$,}
     
\noindent and $A$ must be measurable}.
     
\spheader 211Mb $\mu$ is $\sigma$-finite\prooflet{, because
$\BbbR^r=\bigcup_{n\in\Bbb N}[-\tbf{n},\tbf{n}]$, writing $\tbf{n}$ for the vector
$(n,\ldots,n)$, and $\mu[-\tbf{n},\tbf{n}]=(2n)^r<\infty$ for every $n$.
Of course $\mu$ is neither totally finite nor a probability measure}.
     
\spheader 211Mc\cmmnt{ Because} $\mu$
is\cmmnt{ $\sigma$-finite, it is} strictly
localizable\cmmnt{ (211Lc)}, localizable\cmmnt{ (211Ld)}, locally
determined\cmmnt{ (211Ld)} and semi-finite\cmmnt{ (211Le or 211Lf)}.
     
\spheader 211Md $\mu$ is atomless.   \prooflet{\Prf\ Suppose that
$E\in\Sigma$.  Consider the function
     
\Centerline{$a\mapsto f(a)
=\mu(E\cap[-\tbf{a},\tbf{a}]):\coint{0,\infty}\to\Bbb R$}
     
\noindent We have
     
\Centerline{$f(a)
\le f(b)\le f(a)+\mu[-\tbf{b},\tbf{b}]-\mu[-\tbf{a},\tbf{a}]
=f(a)+(2b)^r-(2a)^r$}
     
\noindent whenever $a\le b$ in $\coint{0,\infty}$, so $f$ is
continuous.   Now $f(0)=0$ and $\lim_{n\to\infty}f(n)=\mu E>0$.   By the
Intermediate Value Theorem there is an $a\in\coint{0,\infty}$ such that
$0<f(a)<\mu E$.   So we have
     
\Centerline{$0<\mu(E\cap[-\tbf{a},\tbf{a}])<\mu E$.}
     
\noindent As $E$ is arbitrary, $\mu$ is atomless.\ \Qed}
     
\header{211Me}{\bf (e)} It is now a trivial observation that $\mu$
cannot be purely atomic\cmmnt{, because $\BbbR^r$ itself is a set of
positive measure not including any atom}.
     
\leader{211N}{Counting measure} Take $X$ to be any uncountable set\cmmnt{ (e.g., 
$X=\Bbb R$)}, and $\mu$ to be counting measure on $X$\cmmnt{ (112Bd)}.
     
\header{211Na}{\bf (a)} $\mu$ is
complete\ifwithproofs{, because if $A\subseteq E$ and
$\mu E=0$ then
     
\Centerline{$A=E=\emptyset\in\Sigma$.}
}\else{.}\fi
     
\header{211Nb}{\bf (b)} $\mu$ is not $\sigma$-finite\cmmnt{, because if $\sequencen{E_n}$ is any sequence of sets of finite measure then every $E_n$ is finite, therefore countable, and
$\bigcup_{n\in\Bbb N}E_n$ is countable (1A1F), so cannot be $X$}.   \cmmnt{{\it A fortiori,}} $\mu$ is not a
probability measure nor totally finite.
     
\spheader 211Nc $\mu$ is strictly localizable.   \prooflet{\Prf\ Set $X_x=\{x\}$ for every $x\in X$.  Then $\langle X_x\rangle_{x\in X}$ is a partition of $X$, and for any $E\subseteq X$
     
\Centerline{$\mu(E\cap X_x)=1$ if $x\in E$,\quad$0$ otherwise.}
     
\noindent By the definition of $\mu$,
     
\Centerline{$\mu E=\sum_{x\in X}\mu(E\cap X_x)$}
     
\noindent for every $E\subseteq X$, and $\langle X_x\rangle_{x\in X}$ is
a decomposition of $X$.\ \Qed
     
Consequently} $\mu$ is localizable, locally determined and semi-finite.
     
\header{211Nd}{\bf (d)} $\mu$ is purely atomic.   \prooflet{\Prf\ $\{x\}$ 
is an atom for every $x\in X$, and if $\mu E>0$ then surely $E$ includes 
$\{x\}$ for some $x$.\ \QeD}   \cmmnt{Obviously,} $\mu$ is not atomless.
     
\leader{211O}{A non-semi-finite space} Set $X=\{0\}$,
$\Sigma=\{\emptyset,X\}$, $\mu\emptyset=0$ and $\mu X=\infty$.   Then
$\mu$ is not semi-finite,\cmmnt{ as $\mu X=\infty$ but $X$ has no subset of non-zero finite measure.   It follows that $\mu$ cannot be} localizable,
locally determined, $\sigma$-finite, totally finite nor a probability
measure.   \cmmnt{Because $\Sigma=\Cal PX$,} $\mu$ is complete.
\cmmnt{$X$ is an atom
for $\mu$, so} $\mu$ is purely atomic (indeed, it is point-supported).
     
\leader{211P}{A non-complete space} Write $\Cal B$ for
the $\sigma$-algebra of Borel subsets of $\Bbb R$\cmmnt{ (111G)}, and
$\mu$ for the
restriction of Lebesgue measure to $\Cal B$\cmmnt{ (recall that by 114G 
every Borel subset of $\Bbb R$ is Lebesgue measurable)}.   
Then $(\Bbb R,\Cal B,\mu)$ is atomless, $\sigma$-finite and not complete.
     
\proof{{\bf (a)} To see that $\mu$ is not complete, recall that there is a
continuous, strictly increasing permutation
$g:[0,1]\to[0,1]$ such that $\mu g[C]>0$, where $C$ is the Cantor set,
so that there is a
set $A\subseteq g[C]$ which is not Lebesgue measurable (134Ib).   Now
$g^{-1}[A]\subseteq C$ cannot be a Borel set, since
$\chi A=\chi(g^{-1}[A])\smallcirc g^{-1}$ is not Lebesgue
measurable, therefore not Borel measurable, and the composition of 
two Borel measurable functions is Borel measurable (121Eg);  
so $g^{-1}[A]$ is a non-measurable subset of the negligible set $C$.
     
\medskip
     
{\bf (b)} The rest of the arguments of 211M apply to $\mu$ just as
well as to true Lebesgue measure, so $\mu$ is $\sigma$-finite and
atomless.
}%end of proof of 211P
     
\cmmnt{\medskip
     
\noindent{\bf *Remark} The argument offered in (a) could give rise to
a seriously false impression.   The set $A$ referred to there can be
constructed only with the use of a strong form of the axiom of choice.
No such device is necessary for the result here.   
There are many methods of
constructing non-Borel subsets of the Cantor set, all illuminating in
different ways.   In the absence of any form of the axiom of choice, there
are difficulties with the concept of `Borel set', and others with the
concept of `Lebesgue measure', which I will come to in
Chapter 56;  but countable choice is quite sufficient for the existence of
a non-Borel subset of $\Bbb R$.   For details of a possible approach
see 423L in Volume 4.
}
     
\leader{211Q}{Some probability spaces} Two\cmmnt{ obvious} constructions
of probability spaces\cmmnt{, restricting myself to the methods described in Volume 1,} are
     
(a) the subspace measure induced by Lebesgue measure on
$[0,1]$\cmmnt{ (131B)};
     
(b) the point-supported measure induced on a set $X$ by a function $h:X\to[0,1]$ such
that $\sum_{x\in X}h(x)=1$, writing $\mu E=\sum_{x\in E}h(x)$ for every
$E\subseteq X$;  for instance, if $X$ is a singleton
$\{x\}$ and $h(x)=1$, or if $X=\Bbb N$ and $h(n)=2^{-n-1}$.
     
Of these two, (a) gives an atomless probability measure and (b) gives a
purely atomic probability measure.
     
\leader{211R}{Countable-cocountable measure}\cmmnt{ The following is one of the basic constructions to keep in mind when considering abstract
measure spaces.
     
\medskip
     
}{\bf (a)} Let $X$ be any set.   
Let $\Sigma$ be the family of those sets $E\subseteq X$ 
such that either $E$ or $X\setminus E$ is
countable.   Then $\Sigma$ is a $\sigma$-algebra of subsets of $X$.
\prooflet{\Prf\  {(i)} $\emptyset$ is countable, so belongs to $\Sigma$.
{(ii)} The condition for $E$ to belong to $\Sigma$ is symmetric
between $E$ and $X\setminus E$, so $X\setminus E\in\Sigma$ for every
$E\in\Sigma$.   {(iii)} Let $\sequencen{E_n}$ be any sequence in
$\Sigma$,
and set $E=\bigcup_{n\in\Bbb N}E_n$.   If every $E_n$ is countable, then
$E$ is countable, so belongs to $\Sigma$.   Otherwise, there is
some $n$ such that $X\setminus E_n$ is countable, in which case
$X\setminus E\subseteq X\setminus E_n$ is countable, so again
$E\in\Sigma$.\ \QeD}   $\Sigma$ is called the
{\bf countable-cocountable $\sigma$-algebra} of $X$.
     
\header{211Rb}{\bf (b)} Now consider the function $\mu:\Sigma\to\{0,1\}$
defined by writing $\mu E=0$ if $E$ is countable, $\mu E=1$ if
$E\in\Sigma$ and $E$ is not countable.   Then $\mu$ is a measure.
\prooflet{\Prf\ {(i)} $\emptyset$ is countable so $\mu\emptyset=0$.   {(ii)}
Let $\sequencen{E_n}$ be a disjoint sequence in $\Sigma$, and $E$
its union.   {(${\alpha}$)} If every $E_m$ is countable, then so
is $E$, so
     
\Centerline{$\mu E=0=\sum_{n=0}^{\infty}\mu E_n$.}
     
\noindent{(${\beta}$)} If some $E_m$ is uncountable, then
$E\supseteq E_m$ also is uncountable, and $\mu E=\mu E_m=1$.   But in
this case, because $E_m\in\Sigma$, $X\setminus E_m$ is countable, so
$E_n$, being a subset of $X\setminus E_m$, is countable for every $n\ne
m$;  thus $\mu E_n=0$ for every $n\ne m$, and
     
\Centerline{$\mu E=1=\sum_{n=0}^{\infty}\mu E_n$.}
     
\noindent As $\sequencen{E_n}$ is arbitrary, $\mu$ is a measure.\ \Qed}
This is the {\bf countable-cocountable measure} on $X$.
     
\header{211Rc}{\bf (c)} If $X$ is any uncountable set and
$\mu$ is the countable-cocountable measure on $X$, then $\mu$ is a
complete, purely atomic probability measure, but is not point-supported.   \prooflet{\Prf\ {(i)} If
$A\subseteq E$ and $\mu E=0$, then $E$ is countable, so $A$ also is
countable and belongs to $\Sigma$.   Thus $\mu$ is complete.   {(ii)}
Because $X$ is uncountable, $\mu X=1$ and $\mu$ is a probability
measure.   {(iii)} If $\mu E>0$, then $\mu F=\mu E=1$ whenever $F$
is a non-negligible measurable subset of $E$, so $E$ is itself an atom;
thus $\mu$ is purely atomic.   (iv) $\mu X=1>0=\sum_{x\in X}\mu\{x\}$, so $\mu$ is not point-supported.\ \Qed}
     
\exercises{\leader{211X}{Basic exercises $\pmb{>}$(a)}
%\sqheader 211Xa 
Let $(X,\Sigma,\mu)$ be a measure space.   Show
that $\mu$ is $\sigma$-finite iff there is a totally finite measure
$\nu$ on $X$ with the same measurable sets and the same negligible sets
as $\mu$.
%211D
     
\sqheader 211Xb Let $g:\Bbb R\to\Bbb R$ be a non-decreasing
function and $\mu_g$ the associated Lebesgue-Stieltjes measure (114Xa).
Show that $\mu_g$ is complete and $\sigma$-finite.   Show that
     
\quad (i) $\mu_g$ is totally finite iff $g$ is bounded;
     
\quad (ii) $\mu_g$ is a probability measure iff $\lim_{x\to\infty}g(x)
-\lim_{x\to-\infty}g(x)=1$;
     
\quad (iii) $\mu_g$ is atomless iff $g$ is continuous;
     
\quad (iv) if $E$ is any atom for $\mu_g$, there is a point $x\in E$
such that $\mu_gE=\mu_g\{x\}$;
     
\quad (v) $\mu_g$ is purely atomic iff it is point-supported.
%211M
     
\sqheader 211Xc Let $\mu$ be counting measure on a set $X$.
Show that $\mu$ is always complete, strictly localizable and purely atomic, and
that it is $\sigma$-finite iff $X$ is countable, totally finite iff $X$
is finite, a probability measure iff $X$ is a singleton, and atomless
iff $X$ is empty.
%211N
     
\spheader 211Xd Show that a point-supported measure is always complete, 
and is strictly localizable iff it is semi-finite.
%211O
     
\spheader 211Xe Let $X$ be a set.   Show that for any $\sigma$-ideal
$\Cal I$ of subsets of $X$ (definition:  112Db), the set
     
\Centerline{$\Sigma=\Cal I\cup\{X\setminus A:A\in\Cal I\}$}
     
\noindent is a $\sigma$-algebra of subsets of $X$, and that there is a
measure $\mu:\Sigma\to\{0,1\}$ given by setting
     
\Centerline{$\mu E=0$ if $E\in\Cal I$,
\quad $\mu E=1$ if $E\in\Sigma\setminus\Cal I$.}
     
\noindent Show that $\Cal I$ is precisely the null ideal of $\mu$, that $\mu$ is complete, totally finite and purely atomic, and is a
probability measure iff $X\notin\Cal I$.
%211R
     
\spheader 211Xf Show that a point-supported measure is strictly localizable
iff it is semi-finite.
     
\spheader 211Xg Let $(X,\Sigma,\mu)$ be a measure space such that 
$\mu X>0$.    Show that the set of conegligible subsets of $X$ is a filter
on $X$.

\leader{211Y}{Further exercises (a)}
%\spheader 211Ya
Let $(X,\Sigma,\mu)$ be a measure space, and for
$E$, $F\in\Sigma$ write $E\sim F$ if $\mu(E\symmdiff F)=0$.   Show that
$\sim$ is an equivalence relation on $\Sigma$.   Let $\frak A$ be the
set of equivalence classes in $\Sigma$ for $\sim$;  for $E\in\Sigma$,
write $E^{\ssbullet}\in\frak A$ for its equivalence class.   Show that
there is a partial ordering $\Bsubseteq$ on $\frak A$ defined by saying
that, for $E$, $F\in\Sigma$,
     
\Centerline{$E^{\ssbullet}\Bsubseteq F^{\ssbullet}
\,\iff\,\mu(E\setminus F)=0$.}
     
\noindent Show that $\mu$ is localizable iff for every
$A\subseteq\frak A$ there is an $h\in\frak A$ such that (i)
$a\Bsubseteq h$ for every
$a\in A$ (ii) whenever $g\in\frak A$ is such that $a\Bsubseteq g$ for
every $a\in\frak A$, then $h\Bsubseteq g$.
%211G
     
\spheader 211Yb Let $(X,\Sigma,\mu)$ be a measure space, and
construct $\frak A$ as in 211Ya.   Show that there are operations
$\Bcup$, $\Bcap$, $\Bsetminus$ on $\frak A$ defined by saying that
     
\Centerline{$E^{\ssbullet}\Bcap F^{\ssbullet}=(E\cap F)^{\ssbullet}$,
\quad$E^{\ssbullet}\Bcup F^{\ssbullet}=(E\cup F)^{\ssbullet}$,
\quad$E^{\ssbullet}\Bsetminus F^{\ssbullet}=(E\setminus F)^{\ssbullet}$}
     
\noindent for all $E$, $F\in\Sigma$.
Show that if $A\subseteq\frak A$ is any {\it countable} set, then there
is an $h\in\frak A$ such that (i) $a\Bsubseteq h$ for every
$a\in A$ (ii) whenever $g\in\frak A$ is such that $a\Bsubseteq g$ for
every $a\in\frak A$, then $h\subseteq g$.   Show that there is a
functional $\bar\mu:\frak A\to[0,\infty]$ defined by saying that
$\bar\mu(E^{\ssbullet})=\mu E$ for every $E\in\Sigma$.
($(\frak A,\bar\mu)$ is called the {\bf measure algebra} of
$(X,\Sigma,\mu)$.)
%211Ya 211G
     
\spheader 211Yc Let $(X,\Sigma,\mu)$ be a semi-finite measure
space.   Show
that it is atomless iff whenever $\epsilon>0$, $E\in\Sigma$ and
$\mu E<\infty$, then there is a finite partition of $E$ into measurable sets of measure at most $\epsilon$.
%211J
     
\spheader 211Yd Let $(X,\Sigma,\mu)$ be a strictly localizable
measure space.   Show that it is atomless iff for every $\epsilon>0$
there is a decomposition of $X$ consisting of sets of measure at most
$\epsilon$.
%211Yc 211J
     
\spheader 211Ye Let $\Sigma$ be the countable-cocountable
$\sigma$-algebra of $\Bbb R$.   Show that
$\coint{0,\infty}\notin\Sigma$.   Let $\mu$ be the restriction of
counting measure to $\Sigma$.   Show that $(\Bbb R,\Sigma,\mu)$ is
complete, semi-finite and purely atomic, but not localizable nor locally
determined.
%211R
}%end of exercises
     
\endnotes{
\Notesheader{211} The list of definitions in 211A-211K
probably strikes you as quite long enough, even though I have omitted
many occasionally useful ideas.   The concepts here vary
widely in importance, and the importance of each varies widely with
context.   My own view is that it is absolutely necessary, when studying
any measure space, to know its classification under the eleven
discriminating features listed here, and to be able to describe any
atoms which are present.   Fortunately, for most `ordinary' measure
spaces, the classification is fairly quick, because if (for instance)
the space is $\sigma$-finite, and you know the measure of the whole
space, the only remaining questions concern completeness and atoms.
The distinctions between spaces which are, or are not,
strictly localizable, semi-finite, localizable and locally determined
are relevant only for spaces which are not $\sigma$-finite, and do not
arise in elementary applications.
     
I think it is also fair to say that the notions of `complete' and
`locally determined' measure space are technical;  I mean, that they do
not correspond to significant features of the essential structure of a
space, though there are some interesting problems concerning incomplete
measures.   One manifestation of this is the existence of canonical
constructions for rendering spaces complete or complete and locally
determined (212C, 213D-213E).   In addition,  measure spaces which
are not semi-finite do not really belong to measure theory, but rather
to the more general study of $\sigma$-algebras and $\sigma$-ideals.
The most important classifications, in terms of the behaviour of a
measure space, seem to me to be `$\sigma$-finite', `localizable'
and `strictly localizable';  these are the critical features which
cannot be forced by elementary constructions.
     
If you know anything about Borel subsets of the real line, the argument of part (a) of the proof of 211P must look very clumsy.   But `better' proofs
rely on ideas which we shall not need until Volume 4, and the proof here is based on a construction which we have to understand for other reasons.
}%end of notes
     
\discrpage
    
     
