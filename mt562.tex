\frfilename{mt562.tex}
\versiondate{20.10.13}
\copyrightdate{2006}

\def\chaptername{Choice and determinacy}
\def\sectionname{Borel codes}

\newsection{562}

The concept of `Borel set', either in
the real line or in general topological spaces, has been fundamental in
measure theory since before the modern subject existed.   It is at this
point that the character of the subject changes if we do not allow
ourselves even the countable axiom of choice.   I have already mentioned
the Feferman-L\'evy model in which
$\Bbb R$ is a countable union of countable sets;
immediately, every subset of $\Bbb R$ is a countable union of countable
sets and is `Borel' on the definition of 111G.   In these circumstances
that definition becomes unhelpful.

An alternative which leads to a non-trivial theory, coinciding
with the usual theory in the presence of AC, is the
algebra of `codable Borel sets' (562B).   This is not necessarily a
$\sigma$-algebra, but is closed under unions and intersections of `codable
sequences' (562K).   When we come to look for measurable functions, the
corresponding concept is that of `codable Borel function' (562L);  again,
we do not expect the limit of an arbitrary sequence of codable Borel
functions to be measurable in any useful sense, but the limit of a codable
sequence of codable Borel functions is again a codable Borel function
(562Ne).
The same ideas can be used to give a theory of `codable Baire sets' in
any topological space (562T).

\vleader{60pt}{562A}{Trees} \cmmnt{I review some ideas from \S421.

\medskip

}{\bf (a)} Set $S^*=\bigcup_{n\ge 1}\BbbN^n$.
For $\sigma$, $\tau\in S^*$ let $\sigma^{\smallfrown}\tau$ be their
concatenation, that is,

$$\eqalign{(\sigma^{\smallfrown}\tau)(n)
&=\sigma(n)\text{ if }n<\#(\sigma),\cr
&=\tau(n-\#(\sigma))\text{ if }\#(\sigma)\le n<\#(\sigma)+\#(\tau).\cr}$$

\noindent For $i\in\Bbb N$ I will write $\fraction{i}\in\BbbN^1$ for the
one-term sequence with value $i$.
For $\sigma\in S^*$ and $T\subseteq S^*$, write $T_{\sigma}$
for $\{\tau:\tau\in S^*$, $\sigma^{\smallfrown}\tau\in T\}$.

%maybe switch \Cal T and \Cal T_0 ?  see \S421
% but  \Cal T  is much commoner here!

\spheader 562Ab Let $\Cal T_0$ be the family of sets
$T\subseteq S^*$ such that
$\sigma\restr n\in T$ whenever $\sigma\in T$ and $n\ge 1$.
Recall from 421N\footnote{Early editions of Volume 4 
used a slightly different
definition of iterated derivations, so that the `rank' of a tree was not
quite the same.} that we have a derivation $\partial:\Cal T_0\to\Cal T_0$
defined by setting

\Centerline{$\partial T
=\{\sigma:\sigma\in S^*$, $T_{\sigma}\ne\emptyset\}$,}

\noindent with iterates $\partial^{\xi}$, for $\xi<\omega_1$, defined by
setting

\Centerline{$\partial^0T=T$,
\quad$\partial^{\xi}T=\bigcap_{\eta<\xi}\partial(\partial^{\eta}T)$ for
$\xi\ge 1$.}

\noindent Now for any $T\in\Cal T_0$ there is a $\xi<\omega_1$ such that
$\partial^{\xi}T=\partial^{\eta}T$ whenever $\xi\le\eta<\omega_1$.
\prooflet{\Prf\
The argument in 421Nd assumed that $\omega_1$ has uncountable cofinality,
but we can avoid this assumption, as follows.   Let
$\family{\sigma}{S^*}{\epsilon_{\sigma}}$ be a summable family of strictly
positive real numbers, and set
$\gamma_T(\xi)=\sum_{\sigma\in\partial^{\xi}T}\epsilon_{\sigma}$;  then
$\gamma_T:\omega_1\to\coint{0,\infty}$ is non-increasing, so 561A tells us
that there is a $\xi<\omega_1$ such that $\gamma_T(\xi+1)=\gamma_T(\xi)$,
that is, $\partial^{\xi+1}T=\partial^{\xi}T$.   Of course we now have
$\partial^{\eta}T=\partial^{\xi}T$ for every $\eta\ge\xi$.\ \Qed}

\spheader 562Ac We therefore still have a rank function
$r:\Cal T_0\to\omega_1$ defined by saying that $r(T)$ is the least ordinal
such that $\partial^{r(T)}T=\partial^{r(T)+1}T$.   Now $\partial^{r(T)}T$
is empty iff there is no $\alpha\in\NN$ such that $\alpha\restr n\in T$ for
every $n\ge 1$.   \prooflet{\Prf\ The argument in 421Nf used the word
`choose';  but we can avoid this by being more specific.   If
$\sigma\in\partial^{r(T)}T$, then we can define a sequence
$\sequencen{\sigma_n}$ by saying that $\sigma_0=\sigma$ and, given
$\sigma_n\in\partial^{r(T)}T$,
$\sigma_{n+1}={\sigma_n}^{\smallfrown}\fraction{i}$ for the least $i$ such
that ${\sigma_n}^{\smallfrown}\fraction{i}\in\partial^{r(T)}T$;
$\alpha=\bigcup_{n\in\Bbb N}\sigma_n$ will now have $\alpha\restr n\in T$
for every $n\ge 1$.   The argument in the other direction is unchanged.\
\Qed}

Let $\Cal T$ be the set of those $T\in\Cal T_0$
with no infinite branch, that is, such that $\partial^{r(T)}T=\emptyset$.
\cmmnt{Note that if $T\in\Cal T$ then $r(T)=0$ iff $T=\emptyset$, while
$r(T)=1$ iff there is a non-empty set $A\subseteq\Bbb N$ such that
$T=\{\fraction{i}:i\in A\}$.}

\spheader 562Ad For $T\in\Cal T$, set $A_T=\{i:\fraction{i}\in T\}$.
We need a fact not covered in \S421:  for any $T\in\Cal T$,
$r(T)=\sup\{r(T_{\fraction{i}})+1:i\in A_T\}$.   \prooflet{\Prf\ An easy
induction on $\xi$ shows that
$\partial^{\xi}(T_{\sigma})=(\partial^{\xi}T)_{\sigma}$ for any
$\xi<\omega_1$, $T\in\Cal T_0$ and $\sigma\in S^*$.   So, for $T\in\Cal T$
and $\xi<\omega_1$,

$$\eqalign{r(T)>\xi
&\Longrightarrow\partial^{\xi}T\ne\emptyset\cr
&\Longrightarrow\,\Exists i,\,\fraction{i}\in\partial^{\xi}T
  =\bigcap_{\eta<\xi}\partial^{\eta+1}T\cr
&\Longrightarrow\,\Exists i\in A_T,\,
  \partial^{\eta}(T_{\fraction{i}})
  =(\partial^{\eta}T)_{\fraction{i}}\ne\emptyset\Forall\eta<\xi\cr
&\Longrightarrow\,\Exists i\in A_T,\,
  r(T_{\fraction{i}})>\eta\Forall\eta<\xi\cr
&\Longrightarrow\,\Exists i\in A_T,\,
  r(T_{\fraction{i}})\ge\xi;\cr}$$

\noindent thus $r(T)\le\sup\{r(T_{\fraction{i}})+1:i\in A_T\}$.   In the
other direction, if $i\in A_T$ and $\eta<\xi=r(T_{\fraction{i}})$, then

\Centerline{$(\partial^{\eta}T)_{\fraction{i}}
=\partial^{\eta}(T_{\fraction{i}})\ne\emptyset$,}

\noindent so $\fraction{i}\in\partial^{\eta+1}T$;  as $\eta$ is arbitrary,
$\fraction{i}\in\partial^{\xi}T$ and $\xi<r(T)$;  as $i$ is arbitrary,
$r(T)\ge\sup\{r(T_{\fraction{i}})+1:i\in A_T\}$.\ \Qed}

\leader{562B}{Coding sets with trees (a)}
Let $X$ be a set and $\sequencen{E_n}$ a sequence of
subsets of $X$.   Define $\phi:\Cal T\to\Cal PX$ inductively by saying that

$$\eqalign{\phi(T)&=\bigcup_{i\in A_T}E_i\text{ if }r(T)\le 1,\cr
&=\bigcup_{i\in A_T}X\setminus\phi(T_{\fraction{i}})
   \text{ if }r(T)>1.\cr}$$

\cmmnt{\noindent By 562Ad, this definition is sound.}
I will call $\phi$ the {\bf interpretation of Borel codes} defined by
$X$ and $\sequencen{E_n}$.

\spheader 562Bb Of course $\phi(\emptyset)=\emptyset$.
If we set

\Centerline{$T^*
=\{\fraction{0},\fraction{0}^{\smallfrown}\fraction{0},
\fraction{0}^{\smallfrown}\fraction{0}^{\smallfrown}\fraction{0},
\fraction{1},\fraction{1}^{\smallfrown}\fraction{0}\}$}

\noindent and

\Centerline{$T
=\{\fraction{0}\}\cup\{\fraction{0}^{\smallfrown}\sigma:\sigma\in T^*\}$,}

\noindent then

\ifwithproofs
$$\eqalign{\phi(T^*)
&=(X\setminus\phi(T^*_{\fraction{0}}))
  \cup(X\setminus\phi(T^*_{\fraction{1}}))\cr
&=(X\setminus\phi(\{\fraction{0},\fraction{0}^{\smallfrown}\fraction{0}\}))
  \cup(X\setminus\phi(\{\fraction{0}\}))\cr
&=(X\setminus(X\setminus\phi(\{\fraction{0}\})))
  \cup(X\setminus\phi(\{\fraction{0}\}))
=X,\cr
\phi(T)
&=X\setminus\phi(T^*)
=\emptyset,\cr}$$
\else
\Centerline{$\phi(T^*)=X$,\quad$\phi(T)=\emptyset$,}
\fi

\noindent while $T\ne\emptyset$\cmmnt{, which it will be useful to know}.

\spheader 562Bc Now suppose that $X$ is a second-countable topological
space and that $\sequencen{U_n}$, $\sequencen{V_n}$ are two sequences
running over bases for the topology of $X$.   Let $\phi:\Cal T\to\Cal PX$
and $\phi':\Cal T\to\Cal PX$ be the interpretations of Borel codes defined
by $\sequencen{U_n}$, $\sequencen{V_n}$ respectively.   Then
there is a function $\Theta:\Cal T\to\Cal T\setminus\{\emptyset\}$
such that $\phi'\Theta=\phi$.
\prooflet{\Prf\ Define $\Theta$ inductively, as
follows.   If $r(T)\le 1$, then
$\phi(T)=\bigcup_{i\in A_T}U_i$ is open.   If $\phi(T)\ne\emptyset$,
set $\Theta(T)=\{\fraction{j}:j\in\Bbb N$, $V_j\subseteq\phi(T)\}$;  then

\Centerline{$\phi'(\Theta(T))
=\bigcup\{V_j:j\in\Bbb N$, $V_j\subseteq\phi(T)\}=\phi(T)$.}

\noindent If $\phi(T)=\emptyset$, take
$\Theta(T)$ to be any non-empty member of $\Cal T$ such that
$\phi'(\Theta(T))=\emptyset$;  e.g., that presented in (b) just above.

For the inductive step to $r(T)>1$, set

\Centerline{$\Theta(T)=\{\fraction{i}:i\in A_T\}
  \cup\{\fraction{i}^{\smallfrown}\sigma:
i\in A_T$, $\sigma\in\Theta(T_{\fraction{i}})\}$;}

\noindent then $r(\Theta(T))>1$ and

$$\eqalign{\phi'(\Theta(T))
&=\bigcup_{i\in A_{\Theta(T)}}X\setminus\phi'(\Theta(T)_{\fraction{i}})
=\bigcup_{i\in A_T}X\setminus\phi'(\Theta(T_{\fraction{i}}))\cr
&=\bigcup_{i\in A_T}X\setminus\phi(T_{\fraction{i}})
=\phi(T),\cr}$$

\noindent so the induction continues.\ \Qed}

\cmmnt{(There will be a substantial strengthening of this idea in
562Ma.)}

\spheader 562Bd Now say that a {\bf codable Borel set}
in $X$ is one expressible as
$\phi(T)$ for some $T\in\Cal T$, starting from some sequencen running over
a base for the topology of $X$\cmmnt{;  in view of (c),
we can restrict our calculations to a
fixed enumeration of a fixed base if we wish}.   I will write $\Cal B_c(X)$
for the family of codable Borel sets of $X$.

The definition of `interpretation of Borel codes'
makes it plain that any $\sigma$-algebra of
subsets of $X$ containing every open set will also contain every codable
Borel set;\cmmnt{ so} every codable Borel set is\cmmnt{ indeed} a 
`Borel set' on the
definition of 111G or 4A3A.

\cmmnt{As in the argument for (c) just above,
it will sometimes be useful to know that every element of $\Cal B_c(X)$ can
be coded by a non-empty member of $\Cal T$;  we
have only to check the case
of the empty set, which is dealt with in the formula in (b).}

\leader{562C}{}\cmmnt{
The point of these codings is that we can define explicit
functions on $\Cal T$ which will have appropriate reflections in the
coded sets.

\medskip

}{\bf (a)} For instance, there are functions
$\Theta_0:\Cal T\to\Cal T$, $\Theta_1:\Cal T\times\Cal T\to\Cal T$,
$\Theta_2:\Cal T\times\Cal T\to\Cal T$,
$\Theta_3:\Cal T\times\Cal T\to\Cal T$ such that, for any interpretation
$\phi$ of Borel codes,

\Centerline{$\phi(\Theta_0(T))=X\setminus\phi(T)$,
\quad$\phi(\Theta_1(T,T'))=\phi(T)\cup\phi(T')$,}

\Centerline{$\phi(\Theta_2(T,T'))=\phi(T)\cap\phi(T')$,
\quad$\phi(\Theta_3(T,T'))=\phi(T)\setminus\phi(T')$}

\noindent for all $T$, $T'\in\Cal T$.
\prooflet{\Prf\ Let $T^*$ be the tree described in 562Bb, so that
$\phi(T^*)=X$.   Set

$$\eqalign{\Theta_0(T)
&=T^*\text{ if }T=\emptyset,\cr
&=\{\fraction{0}\}\cup\{\fraction{0}^{\smallfrown}\sigma:\sigma\in T\}
\text{ otherwise};\cr}$$

\noindent then

$$\eqalign{\phi(\Theta_0(T))
&=\phi(T^*)=X=X\setminus\phi(T)\text{ if }T=\emptyset,\cr
&=X\setminus\phi(\Theta_0(T)_{\fraction{0}})
=X\setminus\phi(T)\text{ otherwise}.\cr}$$

\noindent Now set

\Centerline{$\Theta_1(T,T')
=\{\fraction{0},\fraction{1}\}
   \cup\{\fraction{0}^{\smallfrown}\sigma:\sigma\in\Theta_0(T)\}
   \cup\{\fraction{1}^{\smallfrown}\sigma:\sigma\in\Theta_0(T')\}$,}

\noindent so that

$$\eqalign{\phi(\Theta_1(T,T'))
&=(X\setminus\phi(\Theta_1(T,T'))_{\fraction{0}})
  \cup(X\setminus\phi(\Theta_1(T,T'))_{\fraction{1}})\cr
&=(X\setminus\phi(\Theta_0(T)))
  \cup(X\setminus\phi(\Theta_0(T')))\cr
&=(X\setminus(X\setminus\phi(T)))
  \cup(X\setminus(X\setminus\phi(T')))
=\phi(T)\cup\phi(T').\cr}$$

\noindent So we can take

\Centerline{$\Theta_2(T,T')
=\Theta_0(\Theta_1(\Theta_0(T),\Theta_0(T')))$,
\quad$\Theta_3(T,T')
=\Theta_2(T,\Theta_0(T'))$}

\noindent and get

\Centerline{$\phi(\Theta_2(T,T'))
=X\setminus((X\setminus\phi(T))\cup(X\setminus\phi(T')))
=\phi(T)\cap\phi(T')$,}

\Centerline{$\phi(\Theta_3(T,T'))
=\phi(T)\cap(X\setminus\phi(T'))
=\phi(T)\setminus\phi(T')$.  \Qed}}

\spheader 562Cb\cmmnt{ We can find codes for unions and intersections of
sequences, provided the sequences are presented in the right way;  I give
a general formulation of the process.}   For any countable set $K$ we have
functions $\tilde\Theta_1$,
$\tilde\Theta_2:\bigcup_{J\subseteq K}\Cal T^J\to\Cal T$ such that whenever
$X$ is a set, $\sequencen{E_n}$ is a sequence of subsets
of $X$ and $\phi$ is the corresponding interpretation of Borel codes, then
$\phi(\tilde\Theta_1(\tau))=\bigcup_{j\in J}\phi(\tau(j))$ and
$\phi(\tilde\Theta_2(\tau))=X\cap\bigcap_{j\in J}\phi(\tau(j))$ whenever
$J\subseteq K$ and $\tau\in\Cal T^J$.
\prooflet{\Prf\ Let $\sequencen{k_n}$ be a
sequence running over $K\cup\{\emptyset\}$.
For $J\subseteq K$ and $\tau\in\Cal T^J$, set
$A=\{n:k_n\in J$, $\tau(k_n)\ne\emptyset\}$ and

\Centerline{$\tilde\Theta_1(\tau)
=\{\fraction{n}:n\in A\}
\cup\{\fraction{n}^{\smallfrown}\sigma:
  n\in A,\,\sigma\in\Theta_0(\tau(k_n))\}$,}

\Centerline{$\tilde\Theta_2(\tau)
=\Theta_0(\tilde\Theta_1(\family{j}{J}{\Theta_0(\tau(j))}))$.}

\noindent Then

$$\eqalign{\phi(\tilde\Theta_1(\tau))
&=\emptyset=\bigcup_{j\in J}\phi(\tau(j))\text{ if }A=\emptyset,\cr
&=\bigcup_{n\in A}X\setminus\phi(\Theta_0(\tau(k_n)))
=\bigcup_{n\in A}\phi(\tau(k_n))
=\bigcup_{j\in J}\phi(\tau_j)\text{ otherwise},\cr
\phi(\tilde\Theta_2(\tau))
&=X\setminus\bigcup_{j\in J}(X\setminus\phi(\tau_j))
=X\cap\bigcap_{j\in J}\phi(\tau_j).  \text{\Qed}\cr}$$
}

\spheader 562Cc\cmmnt{ A
more sophisticated version of two of the codings in (a)
will be useful in \S564.}
%Fubini's theorem
Let $X$ be a regular second-countable space, $\sequencen{U_n}$ a sequence
running over a base for the topology of $X$ containing $\emptyset$, and
$\phi:\Cal T\to\Cal PX$ the associated interpretation of Borel codes.
Then there are functions
$\Theta'_1$, $\Theta'_2:\Cal T\times\Cal T\to\Cal T$ such that

\Centerline{$\phi(\Theta'_1(T,T'))=\phi(T)\cup\phi(T')$,
\quad$\phi(\Theta'_2(T,T'))=\phi(T)\cap\phi(T')$,}

\Centerline{$r(\Theta'_1(T,T'))=r(\Theta'_2(T,T'))=\max(r(T),r(T'))$}

\noindent for all $T$, $T'\in\Cal T$.   \prooflet{\Prf\ The point is
just that open sets in a regular second-countable space
are F$_{\sigma}$.   Because of the slightly awkward
form taken by the definition of $\phi$, we need to start with an
auxiliary function.   Define $T\mapsto\tilde T:\Cal T\to\Cal T$ by saying
that

$$\eqalign{\tilde T
&=\{\fraction{n}:\overline{U}_n\subseteq\phi(T)\}
\cup\{\fraction{n}^{\smallfrown}\fraction{i}:
  \overline{U}_n\subseteq\phi(T),\,U_i\cap U_n=\emptyset\}\cr
&\mskip300mu \text{ if }r(T)\le 1,\cr
&=T\text{ otherwise }.\cr}$$

\noindent Then $\phi(\tilde T)=\phi(T)$ and $r(\tilde T)=\max(2,r(T))$
for every $T$ (because if $r(T)\le 1$ there is some $n$ such that
$U_n=\emptyset$ and
$\fraction{n}^{\smallfrown}\fraction{n}\in\tilde T$).
Note that $\Tilde{\Tilde T}=\tilde T$.
We also need to fix a bijection $n\mapsto(i_n,j_n)$ between $\Bbb N$ and
$\Bbb N\times\Bbb N$.

Now define $\Theta'_1$ by saying that

\inset{----- if $\max(r(T),r(T'))\le 1$,
$\Theta'_1(T,T')=\Theta'_1(T)\cup\Theta'_1(T')$;

----- if $\max(r(T),r(T'))>1$, then

$$\eqalign{\Theta'_1(T,T')
&=\{\fraction{2n}:n\in A_{\tilde T}\}
\cup\{\fraction{2n}^{\smallfrown}\sigma:\sigma\in\tilde T_{\fraction{n}}\}
\cr
&\mskip100mu\cup\{\fraction{2n+1}:n\in A_{\tilde T'}\}
\cup\{\fraction{2n+1}^{\smallfrown}\sigma:
  \sigma\in\tilde T'_{\fraction{n}}\}.\cr}$$
}

\noindent For $\Theta'_2$ induce on $\max(r(T),r(T'))$:

\inset{----- if $T=T'=\emptyset$, $\Theta'_2(T,T')=\emptyset$;

----- if $\max(r(T),r(T'))=1$,

\Centerline{$\Theta'_2(T,T')
=\{\fraction{n}:U_n\subseteq\phi(T)\cap\phi(T')\}$;}

----- if $\max(r(T),r(T'))>1$, set $A=\{n:i_n\in A_{\tilde T}$,
$j_n\in A_{\tilde T'}\}$ and

\Centerline{$\Theta'_2(T,T')
=\{\fraction{n}:n\in A\}
\cup\{\fraction{n}^{\smallfrown}\sigma:n\in A$, $\sigma
  \in\Theta'_1(\tilde T_{\fraction{i_n}},\tilde T'_{\fraction{j_n}})\}$}

\noindent(interpreting $\tilde T'_{\fraction{j_n}}$ as
$((T')\sptilde)_{\fraction{j_n}}$).
}

\noindent These formulae work.   I run through the calculations for
$\Theta'_2(T,T')$ when $\max(r(T),r(T'))>1$.    We have
$r(\tilde T)\ge 2$ and $r(\tilde T')\ge 2$, so $A_{\tilde T}$,
$A_{\tilde T'}$ and $A$ are non-empty,

$$\eqalign{r(\Theta'_2(T,T'))
&=\sup_{n\in A}
   r(\Theta'_1(\tilde T_{\fraction{i_n}},\tilde T'_{\fraction{j_n}}))+1
=\sup_{i\in A_{\tilde T},j\in A_{\tilde T'}}
   r(\Theta'_1(\tilde T_{\fraction{i}},\tilde T'_{\fraction{j}}))+1\cr
&=\sup_{i\in A_{\tilde T},j\in A_{\tilde T'}}
   \max(r(\tilde T_{\fraction{i}}),r(\tilde T'_{\fraction{j}}))+1\cr
&=\max(\sup_{i\in A_{\tilde T}}r(\tilde T_{\fraction{i}})+1,
   \sup_{j\in A_{\tilde T'}}r(\tilde T'_{\fraction{j}})+1)\cr
&=\max(r(\tilde T),r(\tilde T'))
=\max(2,r(T),r(T'))
=\max(r(T),r(T'))\cr}$$

\noindent and

$$\eqalign{\phi(\Theta'_2(T,T'))
&=\bigcup_{n\in A}X\setminus
  \phi(\Theta'_1(\tilde T_{\fraction{i_n}},\tilde T'_{\fraction{j_n}}))\cr
&=\bigcup_{n\in A}X\setminus
  \phi(\Theta'_1(\Tilde{\Tilde T}_{\fraction{i_n}},
    \Tilde{\Tilde T}'_{\fraction{j_n}}))
=\bigcup_{n\in A}X\setminus
  (\phi(\tilde T_{\fraction{i_n}})\cup\phi(\tilde T'_{\fraction{j_n}}))\cr
&=\bigcup_{n\in A}(X\setminus\phi(\tilde T_{\fraction{i_n}}))
  \cap(X\setminus\phi(\tilde T'_{\fraction{j_n}}))\cr
&=\bigcup_{i\in A_{\tilde T},j\in A_{\tilde T'}}
   (X\setminus\phi(\tilde T_{\fraction{i_n}}))
     \cap(X\setminus\phi(\tilde T'_{\fraction{j_n}}))\cr
&=(\bigcup_{i\in A_{\tilde T}}X\setminus\phi(\tilde T_{\fraction{i_n}}))
   \cap(\bigcup_{j\in A_{\tilde T'}}
          X\setminus\phi(\tilde T'_{\fraction{j_n}}))\cr
&=\phi(\tilde T)\cap\phi(\tilde T')
=\phi(T)\cap\phi(T'). \text{ \Qed}\cr}$$
}

\leader{562D}{Proposition} (a) If $X$ is a second-countable space, then the
family of codable Borel subsets of $X$ is an algebra of subsets
of $X$ containing every G$_{\delta}$ set and every F$_{\sigma}$ set.

(b) [AC$(\omega)$] Every Borel set is a codable Borel set.

\proof{{\bf (a)} Let $\sequencen{U_n}$ be a sequence running over a base
for the topology of $X$, and
$\phi:\Cal T\to\Cal B_c(X)$ the corresponding surjection.
From 562Ca we see that $X\setminus E$ and $E\cup E'$ belong to
$\Cal B_c(X)$ for all $E$, $E'\in\Cal B_c(X)$;
since $\emptyset=\phi(\emptyset)$ belongs
to $\Cal B_c(X)$, $\Cal B_c(X)$ is an algebra of subsets of $X$.

If $E\subseteq X$ is an F$_{\sigma}$ set, there is a sequence
$\sequencen{F_n}$ of closed sets with union $E$.   Set

\Centerline{$T=\{\fraction{n}:n\in\Bbb N\}
  \cup\{\fraction{n}^{\smallfrown}\fraction{i}:
    n$, $i\in\Bbb N$, $U_i\subseteq X\setminus F_n\}$.}

\noindent Then $r(T)=2$, $\phi(T_{\fraction{n}})=X\setminus F_n$ for every
$n$ and $\phi(T)=E$.

Thus every F$_{\sigma}$ set belongs to $\Cal B_c(X)$;  it follows at once
that every G$_{\delta}$ set is also a codable Borel set.

\medskip

{\bf (b)} We can repeat the argument in (a), but this time in a more
general form.   If $\sequencen{E_n}$ is any sequence in $\Cal B_c(X)$, then
for each $n\in\Bbb N$ {\bf choose}
$T^{(n)}\in\Cal T\setminus\{\emptyset\}$ such that $\phi(T^{(n)})=E_n$;  set

\Centerline{$T=\{\fraction{n}:n\in\Bbb N\}
\cup\{\fraction{n}^{\smallfrown}\sigma:n\in\Bbb N$,
$\sigma\in T^{(n)}\}$;}

\noindent then $\bigcup_{n\in\Bbb N}X\setminus E_n=\phi(T)$ is a
codable Borel set.   Because $\Cal B_c(X)$ is an algebra, this is enough to
show that it is a $\sigma$-algebra and therefore equal to the
$\sigma$-algebra $\Cal B(X)$.
}%end of proof of 562D

\leader{562E}{Proposition} Let $X$ be a second-countable space and
$Y\subseteq X$ a subspace of $X$.   Then
$\Cal B_c(Y)=\{Y\cap E:E\in\Cal B_c(X)\}$.

\proof{ Let $\sequencen{U_n}$ be a sequence running over a base for the
topology of $X$, and set $V_n=Y\cap U_n$ for each
$n$;  let $\phi_X:\Cal T\to\Cal B_c(X)$ and $\phi_Y:\Cal T\to\Cal B_c(Y)$
be the interpretations of Borel codes corresponding to $\sequencen{U_n}$,
$\sequencen{V_n}$ respectively.   Then an easy induction on the rank of $T$
shows that $\phi_Y(T)=Y\cap\phi_X(T)$ for every $T\in\Cal T$.   So

\Centerline{$\Cal B_c(Y)=\phi_Y[\Cal T]=\{Y\cap\phi_X(T):T\in\Cal T\}
=\{Y\cap E:E\in\Cal B_c(X)\}$.}
}%end of proof of 562E

\leader{*562F}{}\cmmnt{ I do not expect to rely on the next
result, but it is
interesting that two of the basic facts of descriptive set theory have
versions in the new context.

\medskip

\noindent}{\bf Theorem}
(a) If $X$ is a Hausdorff second-countable space and $A$, $B$ are disjoint
analytic subsets of $X$, there is a codable Borel set $E\subseteq X$ such
that $A\subseteq E$ and $B\cap E=\emptyset$.

(b) Let $X$ be a Polish space.   Then a subset $E$ of $X$ is a codable
Borel set iff $E$ and $X\setminus E$ are analytic.

\proof{{\bf (a)(i)} If either $A$ or $B$ is empty, this is trivial, just
because $\emptyset$ and $X$ are codable Borel sets;  so suppose otherwise.
Let $f:\NN\to X$ and $g:\NN\to X$ be continuous functions such that
$f[\NN]=A$ and $g[\NN]=B$.   Fix an enumeration $\sequencen{(j_n,k_n)}$ of
$\Bbb N\times\Bbb N$, and a sequence $\sequencen{U_n}$ running over a base
for the topology of $X$;  let $\phi$ be the interpretation of Borel codes
defined by $\sequencen{U_n}$.   For
$\sigma\in S=\bigcup_{n\in\Bbb N}\BbbN^n$ set

\Centerline{$A_{\sigma}
  =\{f(\alpha):\alpha\in\NN$, $\alpha_i=j_{\sigma(i)}$
     for $i<\#(\sigma)\}$,}

\Centerline{$B_{\sigma}
  =\{g(\alpha):\alpha\in\NN$, $\alpha_i=k_{\sigma(i)}$
     for $i<\#(\sigma)\}$.}

\noindent Then $A_{\emptyset}=A$ and
$A_{\sigma}=\bigcup_{i\in\Bbb N}A_{\sigma^{\smallfrown}\fraction{i}}$ for
every $\sigma$, and similarly for $B$.

\medskip

\quad{\bf (ii)} Still in the setting-up stage, we need general union and
intersection operators on $\Cal T$.   As in 562Cb, let
$\tilde\Theta_1$,
$\tilde\Theta_2:\bigcup_{J\subseteq\Bbb N}\Cal T^J\to\Cal T$ be such that
$\phi(\tilde\Theta_1(\tau))=\bigcup_{j\in J}\phi(\tau(j))$ and
$\phi(\tilde\Theta_2(\tau))=X\cap\bigcap_{j\in J}\phi(\tau(j))$
whenever $J\subseteq\Bbb N$ and $\tau\in\Cal T^J$.

\medskip

\quad{\bf (iii)} Set

$$\eqalign{T
&=\{\sigma:\sigma\in S^*,\text{ there are no }i,\,n\in\Bbb N
\text{ such that }n<\#(\sigma)\cr
&\mskip200mu \text{and }A_{\sigma\restr n}\subseteq U_i
\subseteq X\setminus B_{\sigma\restr n}\}.\cr}$$

\noindent If $\sigma\in S$ and $n\in\Bbb N$ then
$A_{\sigma\restr n}\supseteq A_{\sigma}$ and
$B_{\sigma\restr n}\supseteq B_{\sigma}$, so $\sigma\restr m\in T$ whenever
$\sigma\in T$ and $m\ge 1$;
thus $T$ belongs to $\Cal T_0$ as defined in 562Ab.   In fact
$T\in\Cal T$.   \Prf\Quer\ Otherwise, by 562Ac, there is a $\gamma\in\NN$
such that $\gamma\restr n\in T$ for every $n\ge 1$.   Set
$\alpha=\sequencen{\gamma_{j_n}}$, $\beta=\sequencen{\gamma_{k_n}}$;
then $f(\alpha)\in A$ and $g(\beta)\in B$, so $f(\alpha)\ne g(\beta)$.
Because $X$ is Hausdorff, there are $i$, $j\in\Bbb N$ such that
$f(\alpha)\in U_i$, $g(\beta)\in U_j$ and $U_i\cap U_j=\emptyset$.
Because $f$ and $g$ are continuous, there is an $n\ge 1$ such that
$f(\alpha')\in U_i$ and $g(\beta')\in U_j$ whenever
$\alpha'$, $\beta'\in\NN$, $\alpha'\restr n=\alpha\restr n$ and
$\beta'\restr n=\beta\restr n$;  that is, such that
$A_{\gamma\restr n}\subseteq U_i$
and $B_{\gamma\restr n}\subseteq U_j$.   But this means that
$\gamma\restr n+1\notin T$.\ \Bang\Qed

\medskip

\quad{\bf (iv)} We know that $\langle\partial^{\xi}T\rangle_{\xi\le r(T)}$
is a non-increasing family in $\Cal T$ with last member $\emptyset$, and
moreover that $\partial^{\xi}T=\bigcap_{\eta<\xi}\partial^{\eta}T$ for
non-zero limit ordinals $\xi\le r(T)$.   So for $\sigma\in T$ there is
a unique $h(\sigma)<r(T)$ such that
$\sigma\in\partial^{h(\sigma)}T\setminus\partial(\partial^{h(\sigma)}T)$.
I seek to define $T^{(\sigma)}\in\Cal T$, for $\sigma\in T$, such that
$A_{\sigma}\subseteq\phi(T^{(\sigma)})\subseteq X\setminus B_{\sigma}$
for every $\sigma$.   I do this inductively.

\medskip

\quad{\bf (v)} If $h(\sigma)=0$, that is, $\sigma\in T\setminus\partial T$,
then $\sigma^{\smallfrown}0\notin T$.
So there is a first $i\in\Bbb N$ such that
$A_{\sigma}\subseteq U_i\subseteq X\setminus B_{\sigma}$.
Set $T^{(\sigma)}=\{\fraction{i}\}$,
so that $\phi(T^{(\sigma)})=U_i$ includes $A_{\sigma}$ and is
disjoint from $B_{\sigma}$.

\medskip

\quad{\bf (vi)} Now suppose that we have $\xi\le r(t)$ such that
$\xi\ge 1$ and
$T^{(\sigma)}$ has been defined for every $\sigma\in T$ with
$h(\sigma)<\xi$.
Take $\sigma\in T$ such that $h(\sigma)=\xi$;   then
$\sigma^{\smallfrown}\fraction{n}\in T$ for some, therefore every,
$n\in\Bbb N$, while
$h(\sigma^{\smallfrown}\fraction{n})<\xi$ and $T^{(\sigma^{\smallfrown}\fraction{n})}$ is defined
for every $n\in\Bbb N$.   Now, for each $n$, we have
$A_{\sigma^{\smallfrown}\fraction{n}}
\subseteq\phi(T^{(\sigma^{\smallfrown}\fraction{n})})$.   But of course
$A_{\sigma^{\smallfrown}\fraction{n}}=A_{\sigma^{\smallfrown}\fraction{m}}$
whenever $j_m=j_n$.   So we have
$A_{\sigma^{\smallfrown}\fraction{n}}
\subseteq\bigcap_{j_m=j_n}\phi(T^{(\sigma^{\smallfrown}\fraction{m})})$
for each $n$, and

\Centerline{$A_{\sigma}
\subseteq\bigcup_{n\in\Bbb N}\bigcap_{j_m=j_n}
   \phi(T^{(\sigma^{\smallfrown}\fraction{m})})
=\bigcup_{j\in\Bbb N}\bigcap_{j_m=j}
   \phi(T^{(\sigma^{\smallfrown}\fraction{m})})$.}

\noindent Similarly,
$B_{\sigma^{\smallfrown}\fraction{n}}=B_{\sigma^{\smallfrown}\fraction{m}}$
is disjoint from
$\phi(T^{(\sigma^{\smallfrown}\fraction{m})})$ whenever
$k_m=k_n$, so

\Centerline{$B_{\sigma}
=\bigcup_{k\in\Bbb N}\bigcap_{k_m=k}B_{\sigma^{\smallfrown}\fraction{m}}$}

\noindent is disjoint from
$\bigcap_{k\in\Bbb N}\bigcup_{k_m=k}
  \phi(T^{(\sigma^{\smallfrown}\fraction{m})})$.

On the other hand, for any $j$, $k\in\Bbb N$, there is a $p\in\Bbb N$ such
that $j_p=j$ and $k_p=k$, so that

\Centerline{$\bigcap_{j_m=j}
   \phi(T^{(\sigma^{\smallfrown}\fraction{m})})
\subseteq\phi(T^{(\sigma^{\smallfrown}\fraction{p})})
\subseteq\bigcup_{k_m=k}
  \phi(T^{(\sigma^{\smallfrown}\fraction{m})})$.}

\noindent But this means that
$\bigcup_{j\in\Bbb N}\bigcap_{j_m=j}
   \phi(T^{(\sigma^{\smallfrown}\fraction{m})})
\subseteq\bigcap_{k\in\Bbb N}\bigcup_{k_m=k}
  \phi(T^{(\sigma^{\smallfrown}\fraction{m})})$
is disjoint from $B_{\sigma}$.

If therefore we set

\Centerline{$T^{(\sigma)}
=\tilde\Theta_1(\sequence{j}{\tilde\Theta_2(
   \langle T^{(\sigma^{\smallfrown}\fraction{m})}\rangle_{j_m=j})})$}

\noindent we shall have
$A_{\sigma}\subseteq\phi(T^{(\sigma)})\subseteq X\setminus B_{\sigma}$, and
we have a formula defining a suitable tree
$T^{(\sigma)}$ whenever $\sigma\in T$ and
$h(\sigma)=\xi$, so we can continue the induction.

\medskip

\quad{\bf (vii)} This gives us a family
$\family{\sigma}{T}{T^{(\sigma)}}$ in $\Cal T$.
Of course what we are really
looking for is a tree $T^{(\emptyset)}$.   But if $T$ is empty, this is
because there is an $i\in\Bbb N$ such that
$A\subseteq U_i\subseteq X\setminus B$;  in which case $U_i$ is a
codable Borel set separating $A$ from $B$.   While if $T$ is not empty,
$\fraction{n}\in T$ for every $n\in\Bbb N$, and just as in (vi) we can set

\Centerline{$T^{(\emptyset)}
=\tilde\Theta_1(\sequence{j}
  {\tilde\Theta_2(\langle T^{(\fraction{m})}\rangle_{j_m=j})})$}

\noindent to obtain a codable Borel set $E=\phi(T^{\emptyset})$ such that
$A\subseteq E$ and $B\cap E=\emptyset$.

\medskip

{\bf (b)} If $X$ is empty, this is trivial;  suppose henceforth that
$X$ is not empty.

\medskip

\quad{\bf (i)} If $E$ and $X\setminus E$ are analytic, then (a) tells us
that there is a codable Borel set $F$ including $E$ and disjoint from
$X\setminus E$, so that $E=F$ is a codable Borel set.   So the rest of this
part of the proof will be devoted to the converse.

Let $\rho$ be a complete metric on $X$ inducing its
topology, $\sequencen{x_n}$ a sequence running over a dense subset of
$X$, and $\sequencen{U_n}$ a sequence running over a
base for the topology of $X$;  let $\phi$ be the interpretation of Borel
codes defined from $\sequencen{U_n}$.

\medskip

\quad{\bf (ii)} We need to fix on a continuous surjection from a
closed subset of $\NN$ onto $X$;  a convenient one is the following.   Set

\Centerline{$F
=\{\alpha:\alpha\in\NN$, $\rho(x_{\alpha(n+1)},x_{\alpha(n)})\le 2^{-n}$
for every $n\in\Bbb N$\};}

\noindent then $F\subseteq\NN$ is closed.   Define $f:F\to X$ by saying
that $f(\alpha)=\lim_{n\to\infty}x_{\alpha(n)}$ for every $\alpha\in\NN$.
If $\alpha$, $\beta\in\NN$ and $\alpha\restr n=\beta\restr n$ where
$n\ge 1$, then $\rho(f(\alpha),f(\beta))\le 2^{-n+2}$, so $f$ is
continuous.   If $x\in X$, we can define $\alpha\in\NN$ by saying that
$\alpha(n)$ is to be the least $i$ such that $\rho(x,x_i)\le 2^{-n-1}$;
then $\rho(x_{\alpha(n)},x_{\alpha(n+1)})\le 2^{-n-1}+2^{-n-2}\le 2^{-n}$
for every $n$, so $\alpha\in F$, and of course $f(\alpha)=x$.   So $f$ is
surjective.

The next thing we need is a choice function for the set $\Cal F$ of
non-empty closed subsets of $\NN$;  I described one in 561D.   Let
$g:\Cal F\to\NN$ be such that $g(F)\in F$ for every $F\in\Cal F$.

\medskip

\quad{\bf (iii)}
There is a family $\family{T}{\Cal T}{(F_T,f_T,F'_T,f'_T)}$
such that

\Centerline{$F_T$, $F'_T$ are closed subsets of $\NN$,}

\Centerline{$f_T:F_T\to\phi(T)$, $f'_T:F'_T\to X\setminus\phi(T)$ are
continuous surjections}

\noindent for each $T\in\Cal T$.   \Prf\ Start by fixing a homeomorphism
$\alpha\mapsto\sequence{i}{h_i(\alpha)}:\NN\to(\NN)^{\Bbb N}$.
Define the quadruples
$(F_T,f_T,F'_T,f'_T)$ inductively on the rank of $T$.

If $r(T)\le 1$ then $\phi(T)$ is open.   Set

\Centerline{$F_T
=\{\alpha:\alpha\in F$, $x_{\alpha(0)}\in\phi(T)$,
$\rho(x_{\alpha(n)},x_{\alpha(0)})
\le\Bover12\rho(x_{\alpha(0)},X\setminus\phi(T))$ for every $n\ge 1\}$}

\noindent (interpreting $\rho(x,\emptyset)$ as $\infty$ if necessary), and
$f_T=f\restr F_T$.   Then $F_T$ is a closed subset of $\NN$ and
$\rho(f(\alpha),x_{\alpha(0)})
\le\Bover12\rho(x_{\alpha(0)},X\setminus\phi(T))$, so
$f(\alpha)\in\phi(T)$, for every $\alpha\in F_T$.   If $x\in\phi(T)$ then
we can define $\alpha\in\NN$ by taking

\Centerline{$\alpha(n)
=\min\{i:\rho(x_i,x)\le\min(2^{-n-1},\Bover15\rho(x,X\setminus\phi(T)))\}$}

\noindent for every $n$, and now we find that $\alpha\in F_T$ and
$f_T(\alpha)=x$.
As for $F'_T$ and $f'_T$, just set
$F'_T=f^{-1}[X\setminus\phi(T)]$ and $f'_T=f\restr F'_T$.

For the inductive step to $r(T)>1$, set
$A_T=\{i:\fraction{i}\in T\}$, as in 562Ad.
We have $\phi(T)=\bigcup_{i\in A_T}X\setminus\phi(T_{\fraction{i}})$ and
$X\setminus\phi(T)=\bigcap_{i\in A_T}\phi(T_{\fraction{i}})$, while
$r(T_{\fraction{i}})<r(T)$ for every $i\in A_T$.   Set

\Centerline{$F_T=\bigcup_{i\in A_T}
\{\fraction{i}^{\smallfrown}\alpha:\alpha\in F'_{T_{\fraction{i}}}\}$,}

$$\eqalign{F'_T
&=\{\alpha:h_i(\alpha)\in F_{T_{\fraction{i}}}
  \text{ for every }i\in A_T,\cr
&\mskip100mu
f_{T_{\fraction{i}}}(h_i(\alpha))=f_{T_{\fraction{j}}}(h_j(\alpha))
  \text{ for all }i,\,j\in A_T\},\cr}$$

\Centerline{$f_T(\fraction{i}^{\smallfrown}\alpha)
=f'_{T_{\fraction{i}}}(\alpha)$ whenever
$i\in A_T$ and $\alpha\in F'_{T_{\fraction{i}}}$,}

\Centerline{$f'_T(\alpha)=f_{T_{\fraction{i}}}(h_i(\alpha))$ whenever
$i\in A_T$ and $\alpha\in F'_T$.}

\noindent It is straightforward to confirm that $F_T$ and $F'_T$ are
closed, $f_T:F_T\to\phi(T)$ and $f'_T:F'_T\to X\setminus\phi(T)$
are continuous and $f_T[F_T]=\phi(T)$.  To see that
$f'_T[F'_T]=X\setminus\phi(T)$, take any
$x\in X\setminus\phi(T)=\bigcap_{i\in A_T}\phi(T_{\fraction{i}})$.
Then $f_{T_{\fraction{i}}}^{-1}[\{x\}]$ is a non-empty closed subset
of $F_{T_{\fraction{i}}}$ for each $i\in A_T$.
Set $\alpha_i=g(f_{T_{\fraction{i}}}^{-1}[\{x\}])$, so that
$f_{T_{\fraction{i}}}(\alpha_i)=x$ for each $i\in A_T$.   For
$i\in\Bbb N\setminus A_T$, take $\alpha_i=\tbf{0}$.   Now
$\alpha=h^{-1}(\sequence{i}{\alpha_i})$ belongs to $F'_T$ and
$f'_T(\alpha)=x$.   Thus $f'_T[F'_T]=X\setminus\phi(T)$ and the induction
continues.\ \Qed

\medskip

\quad{\bf (iv)}
In particular, $\phi(T)=f_T[F_T]$ is a continuous image of a
closed subset of $\NN$ for every $T\in\Cal T$.

\medskip

\quad{\bf (v)} The definition of `analytic set' in 423A refers to
continuous images of $\NN$, so there is a final step to make.
If $E\subseteq X$ is a non-empty codable Borel set, it is a continuous
image of a closed subset $F_T$ of $\NN$;  but 561C tells us that $F_T$ is a
continuous image of $\NN$, so $E$ also is, and $E$ is analytic.
}%end of proof of 562F

\leader{562G}{Resolvable \dvrocolon{sets}}\cmmnt{ The essence of the
concept of `codable Borel set' is that it is not enough to know, in the
abstract, that a set is `Borel';  we need to know its pedigree.   For a
significant number of elementary sets, however,
starting with open sets and closed sets, we can determine codes from the
sets themselves.

\medskip

\noindent}{\bf Definition}\cmmnt{ (see {\smc Kuratowski 66}, \S12)}
I will say that a subset $E$ of a
topological space $X$ is {\bf resolvable} if there is no non-empty
set $A\subseteq X$ such that
$A\subseteq\overline{A\cap E}\cap\overline{A\setminus E}$.

\leader{562H}{Proposition} Let $X$ be a topological space, and $\Cal E$ the
set of resolvable subsets of $X$.   Then $\Cal E$ is an algebra of sets
containing every open subset of $X$.

\proof{{\bf (a)} If $G\subseteq X$ is open and $A\subseteq X$ is
non-empty, then either $A$ meets $G$ and
$A\not\subseteq\overline{A\setminus G}$, or $A\cap G=\emptyset$ then
$A\not\subseteq\overline{A\cap G}$.   So every open set is resolvable.

\medskip

{\bf (b)}
If $E$ is resolvable and $A\subseteq X$ is not empty, there is an open set
$G$ such that $A\cap G$ is not empty but one of $A\cap G\cap E$,
$A\cap G\setminus E$ is empty.   \Prf\ If
$A\not\subseteq\overline{A\cap E}$ take
$G=X\setminus\overline{A\cap E}$;  otherwise
take $G=X\setminus\overline{A\setminus E}$.\ \Qed

\medskip

{\bf (c)} If $E$, $E'\subseteq X$ are resolvable, so is $E\cup E'$.
\Prf\ Suppose that $A\subseteq X$ is non-empty.   Then there is an open set
$G$ such that $A\cap G$ is non-empty and disjoint from one of $E$,
$X\setminus E$.   Now there is an open set $H$ such that $A\cap G\cap H$ is
non-empty and disjoint from one of $E'$, $X\setminus E'$.   Consequently
one of $A\cap G\cap H\cap(E\cup E')$, $A\cap G\cap H\setminus(E\cup E')$ is
empty, and the open set $G\cap H$ is disjoint from
$\overline{A\cap(E\cup E')}\cap\overline{A\setminus(E\cup E')}$;  in which
case $A$ cannot be included in
$\overline{A\cap(E\cup E')}\cap\overline{A\setminus(E\cup E')}$.   As
$A$ is arbitrary, $E\cup E'$ is resolvable.\ \Qed

\medskip

{\bf (d)} Immediately from the definition in 562G we see that
the complement of a resolvable set is resolvable, so $\Cal E$ is
an algebra of subsets of $X$.
}

\leader{562I}{Theorem} Let $X$ be a second-countable space,
$\sequencen{U_n}$ a sequence running over a base for the topology of $X$,
and $\phi:\Cal T\to\Cal B_c(X)$ the associated
interpretation of Borel codes.
Let $\Cal E$ be the algebra of resolvable subsets of $X$.
Then there is a function $\psi:\Cal E\to\Cal T$ such that
$\phi(\psi(E))=E$ for every $E\in\Cal E$.

\proof{ We need to start by settling on functions

\Centerline{$\Theta'_1:\Cal T\times\Bbb N\to\Cal T$,
\quad$\Theta'_2:\Cal T\times\Cal T\times\Bbb N\to\Cal T$,
\quad$\tilde\Theta'_1:\Cal T^{\Bbb N}\to\Cal T$,
\quad$\tilde\Theta'_2:\Cal T^{\Bbb N}\to\Cal T$}

\noindent such that

\Centerline{$\phi(\Theta'_1(T,n))=\phi(T)\setminus U_n$,
\quad$\phi(\Theta'_2(T,T',n))=\phi(T)\cup(\phi(T')\cap U_n)$,}

\Centerline{$\phi(\tilde\Theta'_1(\tau))=\bigcup_{i\in\Bbb N}\phi(\tau(i))$,
\quad$\phi(\tilde\Theta'_2(\tau))=\bigcap_{i\in\Bbb N}\phi(\tau(i))$}

\noindent for $T\in\Cal T$, $n\in\Bbb N$ and $\tau\in\Cal T^{\Bbb N}$.
(We can take $\tilde\Theta_1$ and $\tilde\Theta_2$ directly from
562Cb, and

\Centerline{$\Theta'_1(T,n)=\Theta_3(T,\{\fraction{n}\}),
\Theta'_2(T,T',n)=\Theta_1(T,\Theta_2(T,\{\fraction{n}\}))$}

\noindent where $\Theta_1$, $\Theta_2$ and $\Theta_3$ are the functions of
562Ca.)

Now, given $E\in\Cal E$, define
$\ofamily{\xi}{\omega_1}
{(F_{\xi},\tilde T^{(\xi)},T^{(\xi)},n_{\xi})}$ inductively,
as follows.   The inductive
hypothesis will be that
$F_{\xi}\subseteq X$ is closed, $F_{\xi}\subseteq F_{\eta}$ for every
$\eta\le\xi$,
$\tilde T^{(\xi)}$, $T^{(\xi)}\in\Cal T$,
$\phi(\tilde T^{(\xi)})=F_{\xi}$ and
$\phi(T^{(\xi)})=E\setminus F_{\xi}$.
Start with $F_0=X$, $T^{(0)}=\emptyset$,
$\tilde T^{(0)}=\{\fraction{n}:n\in\Bbb N\}$.   For the
inductive step to $\xi+1$,

\inset{----- if $F_{\xi}=\emptyset$, set $n_{\xi}=0$ and
$(F_{\xi+1},\tilde T^{(\xi+1)},T^{(\xi+1)})
=(F_{\xi},\tilde T^{(\xi)},T^{(\xi)})$;

----- if there is an $n$ such that
$\emptyset\ne F_{\xi}\cap U_n\subseteq E$, let $n_{\xi}$ be the least
such, and set

\Centerline{$F_{\xi+1}=F_{\xi}\setminus U_{n_{\xi}}$,
\quad$\tilde T^{(\xi+1)}
=\Theta'_1(\tilde T^{(\xi)},n_{\xi})$,
\quad$T^{(\xi+1)}=\Theta'_2(T^{(\xi)},\tilde T^{(\xi)},n_{\xi})$;}

----- otherwise, $n_{\xi}$ is to be the least $n$ such that
$\emptyset\ne F_{\xi}\cap U_n\subseteq X\setminus E$, and

\Centerline{$F_{\xi+1}=F_{\xi}\setminus U_{n_{\xi}}$,
\quad$\tilde T^{(\xi+1)}
=\Theta'_1(\tilde T^{(\xi)},n_{\xi})$
\quad$T^{(\xi+1)}=T^{(\xi)}$.}
}%end of inset

\noindent (Because $E$ is resolvable, these three cases exhaust the
possibilities.)   It is easy to check that the inductive hypothesis
remains valid at level $\xi+1$.

For the inductive step to a non-zero
limit ordinal $\xi$, then if there is an $\eta<\xi$ such that
$F_{\eta}=\emptyset$, take the first such $\eta$ and set $n_{\xi}=0$ and
$(F_{\xi},T^{(\xi)},\tilde T^{(\xi)})
=(F_{\eta},T^{(\eta)},\tilde T^{(\eta)})$.   Otherwise,
we must have
$F_{\zeta}\subseteq F_{\eta}\setminus U_{n_{\eta}}\subset F_{\eta}$
whenever $\eta<\zeta<\xi$, so that $\eta\mapsto n_{\eta}:\xi\to\Bbb N$ is
injective.   Set

$$\eqalign{\tilde\tau(i)&=\tilde T^{(\eta)}
  \text{ if }\eta<\xi\text{ and }i=n_{\eta},\cr
&=\tilde T^{(0)}\text{ if there is no such }\eta,\cr
\tau(i)&=T^{(\eta)}\text{ if }\eta<\xi\text{ and }i=n_{\eta},\cr
&=\emptyset\text{ if there is no such }\eta;\cr}$$

\noindent now set

\Centerline{$F_{\xi}=\bigcap_{\eta<\xi}F_{\eta}$,
\quad$\tilde T^{(\xi)}=\tilde\Theta'_2(\tilde\tau)$,
\quad$T^{(\xi)}=\tilde\Theta'_1(\tau)$.}

\noindent Again, it is easy to check that the induction proceeds.

Now, with the family
$\ofamily{\xi}{\omega_1}
{(F_{\xi},\tilde T^{(\xi)},T^{(\xi)},n_{\xi})}$ complete, observe that
$\ofamily{\xi}{\omega_1}{n_{\xi}}$ cannot be injective.   There is
therefore a first $\xi=\xi_E$ for which $F_{\xi_E}$ is empty.   Set
$\psi(E)=T^{(\xi_E)}$;  then $\phi(\psi(E))=E\setminus F_{\xi_E}=E$, as
required.
}%end of proof of 562I

\leader{562J}{Codable families of sets}
Let $X$ be a second-countable space
and $\Cal B_c(X)$ the algebra of codable Borel subsets of $X$.   Let
$\sequencen{U_n}$, $\sequencen{V_n}$ be sequences running over bases for
the topology of $X$, and
$\phi:\Cal T\to\Cal B_c(X)$,
$\phi':\Cal T\to\Cal B_c(X)$ the corresponding interpretations of Borel
codes.   Let us say that
a family $\familyiI{E_i}$ is {\bf$\phi$-codable} if there is a family
$\familyiI{T^{(i)}}$ in $\Cal T$ such that $\phi(T^{(i)})=E_i$ for every
$i\in I$.   Then\cmmnt{ 562Bc tells us that} $\familyiI{E_i}$ is
$\phi$-codable iff it is $\phi'$-codable.

We may therefore say that a family $\familyiI{E_i}$ in $\Cal B_c(X)$ is
{\bf codable} if it is $\phi$-codable for some\cmmnt{, therefore any,}
interpretation of Borel codes defined by the procedure of 562B from a
sequence running over a base for the topology of $X$.

Note that any finite family in $\Cal B_c(X)$ is codable, and that any
family of resolvable sets is codable\cmmnt{, because
we can use 562I to provide codes};  also any subfamily of a codable family
is codable.   Slightly more generally, if
$\familyiI{E_i}$ is a codable family in $\Cal B_c(X)$, $J$ is a set, and
$f:J\to I$ is a function, then $\family{j}{J}{E_{f(j)}}$ is codable.
If $\familyiI{E_i}$ and
$\familyiI{F_i}$ are codable families in $\Cal B_c(X)$, then so are
$\familyiI{X\setminus E_i}$,
$\familyiI{E_i\cup F_i}$, $\familyiI{E_i\cap F_i}$ and
$\familyiI{E_i\setminus F_i}$\prooflet{,
since we have formulae to transform codes for $E$, $F$ into codes for
$X\setminus E$, $E\cup F$, $E\cap F$ and $E\setminus F$}.

\leader{562K}{Proposition} Let $X$ be a second-countable space and
$\sequencen{E_n}$ a codable sequence in $\Cal B_c(X)$.   Then

(a) $\bigcup_{n\in\Bbb N}E_n$, $\bigcap_{n\in\Bbb N}E_n$ belong to
$\Cal B_c(X)$;

(b) $\sequencen{\bigcup_{i<n}E_i}$ is a codable family in $\Cal B_c(X)$;

(c) $\sequencen{E_n\setminus\bigcup_{i<n}E_i}$ is a codable family in
$\Cal B_c(X)$.
%for 563B

\proof{ Let $\phi:\Cal T\to\Cal B_c(X)$ be an interpretation
of Borel codes defined from a sequence running over a base for the topology
of $X$;  then we have a sequence $\sequencen{T^{(n)}}$ in $\Cal T$
such that $\phi(T^{(n)})=E_n$ for every $n$, and using 562Bb we
can arrange that $T^{(n)}\ne\emptyset$ for every $n$.

\medskip

{\bf (a)} Setting

\Centerline{$T=\{\fraction{n}:n\in\Bbb N\}
\cup\{\fraction{n}^{\smallfrown}\fraction{0}:n\in\Bbb N\}
\cup\{\fraction{n}^{\smallfrown}\fraction{0}^{\smallfrown}\sigma:
n\in\Bbb N$, $\sigma\in T^{(n)}\}$,}

\Centerline{$T'=\{\fraction{0}\}
\cup\{\fraction{0}^{\smallfrown}\fraction{n}:n\in\Bbb N\}
\cup\{\fraction{0}^{\smallfrown}\fraction{n}^{\smallfrown}\sigma:
n\in\Bbb N$, $\sigma\in T^{(n)}\}$,}

\noindent we have $\phi(T)=\bigcup_{n\in\Bbb N}E_n$ and
$\phi(T')=\bigcap_{n\in\Bbb N}E_n$.

\medskip

{\bf (b)} Setting

\Centerline{$\hat T^{(n)}=\{\fraction{i}:i<n\}
\cup\{\fraction{i}^{\smallfrown}\fraction{0}:i<n\}
\cup\{\fraction{i}^{\smallfrown}\fraction{0}^{\smallfrown}\sigma:
i<n$, $\sigma\in T^{(i)}\}$,}

\noindent $\phi(\hat T^{(n)})=\bigcup_{i<n}E_i$ for every $n$.

\medskip

{\bf (c)} Setting

$$\eqalign{T''
&=\{\fraction{n}:n\in\Bbb N\}
  \cup\{\fraction{n}^{\smallfrown}\fraction{0}:n\in\Bbb N\}
\cup\{\fraction{n}^{\smallfrown}\fraction{0}^{\smallfrown}\sigma:
   \sigma\in T^{(n)}\}\cr
&\mskip100mu
  \cup\{\fraction{n}^{\smallfrown}\fraction{1}:n\in\Bbb N\}
  \cup\{\fraction{n}^{\smallfrown}\fraction{1}^{\smallfrown}\fraction{0}:
    n\in\Bbb N\}\cr
&\mskip100mu
\cup\{\fraction{n}^{\smallfrown}\fraction{1}^{\smallfrown}
  \fraction{0}^{\smallfrown}\sigma:
  \sigma\in \hat T^{(n)}\},\cr}$$

\noindent $\phi(T'')=\bigcup_{n\in\Bbb N}(E_n\setminus\bigcup_{i<n}E_i)$.
}%end of proof of 562K

\leader{562L}{Codable Borel functions} Let $X$ and
$Y$ be second-countable
spaces.   A function $f:X\to Y$ is a {\bf codable Borel function} if
$\langle f^{-1}[H]\rangle_{H\subseteq Y\text {is open}}$
is a codable family in $\Cal B_c(X)$.

\leader{562M}{Theorem} Let $X$ be a second-countable space,
$\sequencen{U_n}$ a sequence running over a base for the topology of $X$,
and $\phi:\Cal T\to\Cal B_c(X)$ the corresponding interpretation of Borel
codes.

(a) If $Y$ is another second-countable space, $\sequencen{V_n}$ a
sequence running over a base for the topology of $Y$ containing
$\emptyset$, $\phi_Y:\Cal T\to\Cal B_c(Y)$ the corresponding interpretation
of Borel codes, and
$f:X\to Y$ is a function, then the following are equiveridical:

\quad(i) $f$ is a codable Borel function;

\quad(ii) $\sequencen{f^{-1}[V_n]}$ is a codable sequence in $\Cal B_c(X)$;

\quad(iii) there is a function $\Theta:\Cal T\to\Cal T$ such that
$\phi(\Theta(T))=f^{-1}[\phi_Y(T)]$ for every $T\in\Cal T$.

(b) If $Y$ and $Z$ are second-countable spaces and $f:X\to Y$, $g:Y\to Z$
are codable Borel functions then $gf:X\to Z$ is a codable Borel function.

(c) If $Y$ and $Z$ are second-countable spaces and $f:X\to Y$, $g:X\to Z$
are codable Borel functions then $x\mapsto(f(x),g(x))$ is a codable Borel
function from $X$ to $Y\times Z$.

(d) If $Y$ is a second-countable space then any continuous function
from $X$ to $Y$ is a codable Borel function.

\proof{{\bf (a)(i)$\Rightarrow$(ii)} is trivial.

\medskip

\quad{\bf (ii)$\Rightarrow$(iii)} This is really a full-strength version
of 562Bc.    Because $\sequencen{f^{-1}[V_n]}$ is codable, we have a
sequence $\sequencen{T^{(n)}}$ in $\Cal T$ such that
$\phi(T^{(n)})=f^{-1}[V_n]$ for every $n$.
As in 562C, let $\Theta_0:\Cal T\to\Cal T$
and $\tilde\theta_1:\bigcup_{I\subseteq\Bbb N}\Cal T^I\to\Cal T$
be such that
$\phi(\Theta_0(T))=X\setminus\phi(T)$ for every $T\in\Cal T$ and
$\phi(\tilde\Theta_1(\tau))=\bigcup_{i\in I}\phi(\tau(i))$ whenever
$I\subseteq\Bbb N$ and $\tau\in\Cal T^I$.
Define $\Theta:\Cal T\to\Cal T$
inductively, as follows.   Given $T\in\Cal T$,
set $A_T=\{n:\fraction{n}\in T\}$.
If $r(T)=0$ set $\Theta(T)=T=\emptyset$.   If $r(T)=1$ set
$\Theta(T)=\tilde\Theta_1(\family{n}{A_T}{T^{(n)}})$, so that

\Centerline{$\phi(\Theta(T))=\bigcup_{n\in A_T}\phi(T^{(n)})
=\bigcup_{n\in A_T}f^{-1}[V_n]=f^{-1}[\phi_Y(T)]$.}

\noindent If $r(T)>1$ set

\Centerline{$\Theta(T)=\tilde\Theta_1(\family{n}{A_T}
   {\Theta_0(\Theta(T_{\fraction{n}}))})$}

\noindent so that

$$\eqalign{\phi(\Theta(T))
&=\bigcup_{n\in A_T}X\setminus\phi(\Theta(T_{\fraction{n}}))
=\bigcup_{n\in A_T}X\setminus f^{-1}[\phi_Y(T_{\fraction{n}})]\cr
&=f^{-1}[\bigcup_{n\in A_T}Y\setminus\phi_Y(T_{\fraction{n}})]
=f^{-1}[\phi_Y(T)]\cr}$$

\noindent and the induction continues.

\medskip

\quad{\bf (iii)$\Rightarrow$(i)} For open $H\subseteq Y$ set
$\psi_Y(H)=\{\fraction{n}:V_n\subseteq H\}$.   Taking $\Theta$ as above,

\Centerline{$\phi(\Theta(\psi_Y(H))=f^{-1}[\phi_Y(\psi_Y(H))]=f^{-1}[H]$}

\noindent for every $H$, so
$\langle\phi(\Theta(\psi_Y(H)))\rangle_{H\subseteq Y\text{ is open}}$ is a
family of codes for
$\langle f^{-1}[H]\rangle_{H\subseteq Y\text {is open}}$.

\medskip

{\bf (b)} Take $\sequencen{V_n}$, $\phi_Y$ and $\Theta:\Cal T\to\Cal T$
as in (a).   Write $\frak U$ for the
topology of $Z$;  then we have a function $\theta:\frak U\to\Cal T$ such
that $\phi_Y(\theta(H))=g^{-1}[H]$ for every $H\in\frak U$.
Now $\family{H}{\frak U}{\Theta(\theta(H))}$ is a coding for
$\family{H}{\frak U}{(gf)^{-1}[H]}$, so $gf$ is codable.

\medskip

{\bf (c)} Let $\sequencen{V_n}$, $\sequencen{W_n}$ be sequences running
over bases for the topologies of $Y$ and $Z$, and
$\sequencen{(i_n,j_n)}$ an enumeration of $\Bbb N\times\Bbb N$.
Set $H_n=V_{i_n}\times W_{j_n}$;  then $\sequencen{H_n}$ is a base for the
topology of $Y\times Z$.
Let $\Theta_2:\Cal T\times\Cal T\to\Cal T$ be such that
$\phi(\Theta_2(T,T'))=\phi(T)\cap\phi(T')$ for all $T$, $T'\in\Cal T$
(562Ca).   Let $\sequencen{T^{(n)}}$, $\sequencen{\hat T^{(n)}}$ be codings for
$\sequencen{f^{-1}[V_n]}$, $\sequencen{g^{-1}[W_n]}$.   Then
$\sequencen{\Theta_2(T^{(i_n)},\hat T^{(j_n)})}$ is a coding for
$\sequencen{h^{-1}[H_n]}$, where $h(x)=(f(x),g(x))$ for $x\in X$.
So $h$ is a codable Borel function.

\medskip

{\bf (d)} If $f:X\to Y$ is continuous, then
$\langle f^{-1}[H]\rangle_{H\subseteq Y\text{ is open}}$ is a family of
resolvable sets, therefore codable, as noted in 562J.
}%end of proof of 562M

\cmmnt{\medskip

\noindent{\bf Remark} Note in part
(a)(ii)$\Rightarrow$(iii) of the proof the function $\Theta$ is
constructed by a definite process from $\sequencen{T^{(n)}}$;  so we shall be
able to uniformize the process to define families $\familyiI{\Theta_i}$
from families $\familyiI{f_i}$, at least if we can reach a family
$\langle T^{(i,n)}\rangle_{i\in I,n\in\Bbb N}$ such that
$T^{(n,i)}$ codes $f_i^{-1}[V_n]$ for all $i\in I$ and $n\in\Bbb N$.
}

\leader{562N}{Proposition} Let $X$ be a second-countable space,
and $\phi:\Cal T\to\Cal B_c(X)$ the interpretation of Borel codes
associated with some sequence running over a base for the topology of $X$.

(a) If $f:X\to\Bbb R$ is a function, the following are equiveridical:

\quad(i) $f$ is a codable Borel function;

\quad(ii) the family
$\family{\alpha}{\Bbb R}{\{x:f(x)>\alpha\}}$ is codable;

\quad(iii)
$\family{q}{\Bbb Q}{\{x:f(x)>q\}}$ is codable.

(b) Write $\tilde{\Cal T}$ for the set of
functions $\tau:\Bbb R\to\Cal T$ such that

\Centerline{$\phi(\tau(\alpha))
=\bigcup_{\beta>\alpha}\phi(\tau(\beta))$ for every $\alpha\in\Bbb R$,}

\Centerline{$\bigcap_{n\in\Bbb N}\phi(\tau(n))=\emptyset$,
\quad$\bigcup_{n\in\Bbb N}\phi(\tau(-n))=X$.}

\noindent Then

\quad(i) for every $\tau\in\tilde{\Cal T}$ there is a unique codable Borel
function $\tilde\phi(\tau):X\to\Bbb R$ such that
$\phi(\tau(\alpha))=\{x:\tilde\phi(\tau)(x)>\alpha\}$ for every
$\alpha\in\Bbb R$;

\quad(ii) every codable Borel function from $X$ to $\Bbb R$ is expressible
as $\tilde\phi(\tau)$ for some $\tau\in\tilde{\Cal T}$.

(c) If $\sequencen{\tau_n}$ is a sequence in $\tilde{\Cal T}$ such
that $f(x)=\sup_{n\in\Bbb N}\tilde\phi(\tau_n)(x)$ is finite for every
$x\in X$, then $f$ is a codable Borel function.

(d) If $f$, $g:X\to\Bbb R$ are codable Borel functions and
$\gamma\in\Bbb R$, then
$f+g$, $\gamma f$, $|f|$ and $f\times g$ are codable Borel functions.

(e) If $\sequencen{\tau_n}$ is a sequence in $\tilde{\Cal T}$, then there
is a codable Borel function $f$ such
that $\liminf_{n\to\infty}\tilde\phi(\tau_n)(x)=f(x)$ whenever the lim inf
is finite.

(f) A subset $E$ of $X$ belongs to $\Cal B_c(X)$ iff $\chi E:X\to\Bbb R$ is
a codable Borel function.

\proof{{\bf (a)(i)$\Rightarrow$(ii)}
If $f:X\to\Bbb R$ is codable then of course
$\family{\alpha}{\Bbb R}{\{x:f(x)>\alpha\}}$ is codable, because it is
a subfamily of
$\langle f^{-1}[H]\rangle_{H\subseteq\Bbb R\text{ is open}}$.

\medskip

\quad{\bf (ii)$\Rightarrow$(iii)} Similarly, if
$\family{\alpha}{\Bbb R}{\{x:f(x)>\alpha\}}$ is codable, its subfamily
$\family{q}{\Bbb Q}{\{x:f(x)>q\}}$ is codable.

\medskip

\quad{\bf (iii)$\Rightarrow$(i)} If $\family{q}{\Bbb Q}{\{x:f(x)>q\}}$
is codable, we have a family
$\family{q}{\Bbb Q}{T^{(q)}}$ in $\Cal T$ coding it.
Let $\sequencen{(q_n,q'_n)}$ be an
enumeration of $\{(q,q'):q$, $q'\in\Bbb Q$, $q<q'\}$.
As in 562C, we have functions

\Centerline{$\Theta_3:\Cal T\times\Cal T\to\Cal T$,
\quad$\tilde\Theta_1:\bigcup_{I\subseteq\Bbb Q}\Cal T^I\to\Cal T$,
\quad$\tilde\Theta_2:\bigcup_{J\subseteq\Bbb N}\Cal T^J\to\Cal T$}

\noindent such that

\Centerline{$\phi(\Theta_3(T,T'))=\phi(T)\setminus\phi(T')$,
\quad$\phi(\tilde\Theta_1(\tau))=\bigcup_{q\in I}\phi(\tau(q)))$,}

\Centerline{$\phi(\tilde\Theta_2(\tau))
=X\cap\bigcap_{q\in I}\phi(\tau(j))$}

\noindent for $T$, $T'\in\Cal T$, $I\subseteq\Bbb Q$ and
$\tau\in\Cal T^I$.   Now for $n\in\Bbb N$ consider

\Centerline{$\hat T^{(n)}
=\tilde\Theta_1(\langle\Theta_3(T^{(q_n)},T^{(r)})\rangle_{r\in\Bbb Q,
  r<q'_n})$,}

\noindent so that
$\phi(\hat T^{(n)})
%=\bigcup_{q\in\Bbb Q,r<q'_n}\phi(T^{(q_n)},T^{(r)})
%=\bigcup_{q\in\Bbb Q,r<q'_n}\phi(T^{(q_n)})\setminus\phi(T^{(r)})
%=\bigcup_{q\in\Bbb Q,r<q'_n}\{x:q_n<f(x)\le r\}
%=\{x:q_n<f(x)<q'_n\}
=f^{-1}[\,\ooint{q_n,q'_n}\,]$ for every $n$, and
$\sequencen{f^{-1}[\,\ooint{q_n,q'_n}\,]}$ is codable;  by
562Ma, $f$ is a codable Borel function.

\medskip

{\bf (b)} This is elementary;  given $\tau\in\tilde{\Cal T}$ we can, and
must, set $\tilde\phi(\tau)(x)=\sup\{\alpha:x\in\phi(\tau(\alpha))\}$
for every $x\in X$;  and given $f$ we have a coding $\tau$ for
$\family{\alpha}{\Bbb R}{\{x:f(x)>\alpha\}}$ which must belong to
$\tilde{\Cal T}$ and be such that $\tilde\phi(\tau)=f$.

\medskip

{\bf (c)} Given $\sequencen{\tau_n}$ as described, and taking
$\tilde\Theta_1$ as in (a)(iii)$\Rightarrow$(i) above,

\Centerline{$\alpha\mapsto\tilde\Theta_1(\sequencen{\tau_n(\alpha)})$}

\noindent will be a Borel code for $f$.

\medskip

{\bf (d)} Use 562M(b)-(d).

\medskip

{\bf (e)} Let

\Centerline{$\Theta_0:\Cal T\to\Cal T$,
\quad$\Theta_1:\Cal T\times\Cal T\to\Cal T$}

\noindent be such that

\Centerline{$\phi(\Theta_0(T))=X\setminus\phi(T)$,
\quad$\phi(\Theta_1(T,T'))=\phi(T)\cup\phi(T')$}

\noindent for every $T$, $T'\in\Cal T$.
Now, given $\sequencen{\tau_n}$ as described, set

\Centerline{$\tau(\alpha)
=\tilde\Theta_1(\sequencen{\tilde\Theta_1(\langle\tilde\Theta_2(\langle
  \tau_m(q)\rangle_{m\ge n})\rangle_{q\in\Bbb Q,q>\alpha})})$}

\noindent for $\alpha\in\Bbb R$.   Then

\Centerline{$\phi(\tau(\alpha))
=\bigcup_{q>\alpha,n\in\Bbb N}\bigcap_{m\ge n}\{x:f_m(x)>q\}
=\{x:\liminf_{n\to\infty}f_n(x)>\alpha\}$}

\noindent for each $\alpha$.   We don't yet have a code for a
real-value function defined everywhere in $X$.   But if we set

\Centerline{$T=\tilde\Theta_1(\sequencen{\Theta_3(\tau(-n),\tau(n))})$,}

\noindent then

\Centerline{$\phi(T)
=\bigcup_{n\in\Bbb N}\phi(\tau(-n))\setminus\phi(\tau(n))
=\{x:\liminf_{n\to\infty}f_n(x)\text{ is finite}\}$.}

\noindent So take

$$\eqalign{\tau'(\alpha)
&=\Theta_3(\tau(\alpha),\Theta_0(T))\text{ if }\alpha\ge 0,\cr
&=\Theta_1(\tau(\alpha),\Theta_0(T))\text{ if }\alpha<0;\cr}$$

\noindent this will get $\tau'\in\tilde{\Cal T}$ such that

$$\eqalign{\tilde\phi(\tau')(x)
&=\liminf_{n\to\infty}f_n(x)\text{ if this is finite},\cr
&=0\text{ otherwise}.\cr}$$

\medskip

{\bf (f)} Elementary.
}%end of proof of 562N

\leader{562O}{Remarks (a)} For some purposes there are advantages
in coding real-valued functions by functions from $\Bbb Q$ to $\Cal T$
rather than by functions
from $\Bbb R$ to $\Cal T$\cmmnt{;  see 364Af and 556A}.

\spheader 562Ob As in 562C,
\cmmnt{it will be useful to observe that} the constructions here
are largely determinate.    For instance, the function $\Theta$ of
562M(a-iii) can be built by a definite rule from the sequence
$\sequencen{T^{(n)}}$ provided by the hypothesis (a-ii) there.
What this means is that if we have a family
$\familyiI{(Y_i,\sequencen{V_{in}},f_i)}$ such that $Y_i$ is a
second-countable space, $\sequencen{V_{in}}$ is a sequence running over a
base for the topology of $Y_i$, and $f_i:X\to Y_i$ is a continuous function
for each $i\in I$, then there will be a function
$\tilde\Theta:\Cal T\times I\to\Cal T$ such that
$\phi(\tilde\Theta(T,i))=f_i^{-1}[\phi_i(T)]$ for every $i\in I$ and
$T\in\Cal T$, where $\phi_i:\Cal T\to\Cal B_c(Y_i)$ is the interpretation
of Borel codes
corresponding to the sequence $\sequencen{V_{in}}$.   \cmmnt{(Start from

\Centerline{$T^{(i,n)}
=\{\fraction{j}:U_j\subseteq f_i^{-1}[V_{in}]\}$}

\noindent for $i\in I$ and $n\in\Bbb N$, and build $\tilde\Theta(T,i)$ as
562M.)}

\spheader 562Oc Similarly\cmmnt{, when we look at 562N(d)-(e),
we have something
better than just existence proofs for codes for $f+g$ and
$\liminf_{n\to\infty}f_n$.   For instance}, we have a function
$\tilde\Theta_1:\tilde{\Cal T}\times\tilde{\Cal T}\to\tilde{\Cal T}$
such that
$\tilde\phi(\tilde\Theta_1(\tau,\tau'))$ will always be
$\tilde\phi(\tau)-\tilde\phi(\tau')$
for $\tau$, $\tau'\in\tilde{\Cal T}$.
\prooflet{\Prf\ We need to have

\Centerline{$\phi(\tilde\Theta(\tau,\tau')(\alpha))
=\bigcup_{q\in\Bbb Q}\phi(\tau(q))\setminus\phi(\tau'(q-\alpha))$}

\noindent for every $\alpha$, and this is easy to build from a
set-difference operator, as in 562Ca,
and a general countable-union operator as built in 562Cb.\ \QeD}
Equally, we have a function
$\tilde\Theta_1^*:\tilde{\Cal T}^{\Bbb N}\to\tilde{\Cal T}^{\Bbb N}$ such
that

\Centerline{$\tilde\phi(\tilde\Theta_1^*(\sequencen{\tau_n})(m))
=\inf_{n\ge m}\tilde\phi(\tau_n)$}

\noindent for every $m$
whenever $\sequence{n}{\tau_n}$ is a sequence in $\tilde{\Cal T}$
such that $\inf_{n\in\Bbb N}\tilde\phi(\tau_n)$ is defined as a real-valued
function on $X$.   \prooflet{\Prf\ This time we need

\Centerline{$\phi(\tilde\Theta_1(\sequencen{\tau_n})(m)(\alpha))
=\bigcup_{q\in\Bbb Q,q>\alpha}
  \bigl(X\setminus\bigcup_{n\ge m}(X\setminus\phi(\tau_n(q)))\bigr)$}

\noindent for all $m$ and $\alpha$,
and once again a complementation operator and a general
countable-union operator will do the trick.
\Qed}

\leader{562P}{Codable Borel equivalence (a)} If $X$ is a set, we can say
that two second-countable topologies $\frak S$, $\frak T$ on $X$ are
{\bf codably Borel equivalent} if the identity functions
$(X,\frak S)\to(X,\frak T)$ and $(X,\frak T)\to(X,\frak S)$ are
codable Borel functions.   In this case, $\frak S$ and $\frak T$ give the
same families of codable Borel
functions and the same algebra $\Cal B_c(X)$\cmmnt{ (562Mb, 562Nf)}.

\spheader 562Pb If $(X,\frak T)$ is a second-countable space and
$\sequencen{E_n}$ is any codable sequence in $\Cal B_c(X)$, there is a
topology $\frak S$ on $X$, generated by a countable algebra of subsets of
$X$, such that
$\frak S$ and $\frak T$ are codably Borel equivalent and every $E_n$
belongs to $\frak S$.   \prooflet{\Prf\ Since there is certainly a codable
sequence running over a base for the topology of $X$, we can suppose that
such a sequence has been amalgamated with $\sequencen{E_n}$, so that
$\{E_n:n\in\Bbb N\}$ includes a base for $\frak T$.
Let $\Cal E$ be the algebra of subsets of $X$
generated by $\{E_n:n\in\Bbb N\}$ and
$\frak S$ the topology generated by $\Cal E$.
As $\Cal E$ is an algebra, $\frak S$ is zero-dimensional;  as
$\Cal E$ is countable, $\frak S$ is second-countable.

The identity map $(X,\frak S)\to(X,\frak T)$ is continuous, therefore a
codable Borel function (562Md).
In the reverse direction, we have a sequence
$\sequencen{T^{(n)}}$ of codes for $\sequencen{E_n}$.   From these we can
build, using our standard operations, codes $T_I$, for
$I\in[\Bbb N]^{<\omega}$, $T'_{IJ}$, for $I$, $J\in[\Bbb N]^{<\omega}$,
and $T''_{\Cal K}$, for
$\Cal K\in[[\Bbb N]^{<\omega}\times[\Bbb N]^{<\omega}]^{<\omega}$, such that

\inset{$T_I$ codes $\bigcup_{i\in I}E_i$,

$T'_{IJ}$ codes $\bigcup_{i\in I}E_i\setminus\bigcup_{i\in J}E_i$,

$T''_{\Cal K}$ codes
$\bigcup_{(I,J)\in\Cal K}
  (\bigcup_{i\in I}E_i\setminus\bigcup_{i\in J}E_i)$.}

\noindent But of course
$[[\Bbb N]^{<\omega}\times[\Bbb N]^{<\omega}]^{<\omega}$ is countable and
the $T''_{\Cal K}$ can be enumerated as a sequence $\sequencen{T^*_n}$
coding a sequence $\sequencen{V_n}$ running over $\Cal E$.
By 562Ma,
the identity map $(X,\frak T)\to(X,\frak S)$ is a codable Borel function.\
\Qed}

\cmmnt{Note that $\frak S$ here is necessarily regular;
this will be useful at
more than one point in the next couple of sections.}

\leader{562Q}{Resolvable functions} Let $X$ be a topological space.
I will say that a function $f:X\to[-\infty,\infty]$ is
{\bf resolvable} if whenever
$\alpha<\beta$ in $\Bbb R$ and $A\subseteq X$ is a non-empty set,
then\cmmnt{ at least} one of $\{x:x\in A$, $f(x)\le\alpha\}$,
$\{x:x\in A$, $f(x)\ge\beta\}$ is not dense in $A$.

\medskip

\noindent{\bf Examples (a)} Any semi-continuous
function from $X$ to $[-\infty,\infty]$ is resolvable.
\prooflet{\Prf\ If $f:X\to[-\infty,\infty]$ is
lower semi-continuous, $A\subseteq X$ is non-empty, and $\alpha<\beta$ in
$\Bbb R$, then $U=\{x:f(x)>\alpha\}$ is open;  if $A\cap U\ne\emptyset$
then $\{x:x\in A$, $f(x)\le\alpha\}$ is not dense in $A$;  otherwise
$\{x:x\in A$, $f(x)\ge\beta\}$ is not dense in $A$.\ \Qed}

\spheader 562Qb
If $f:X\to\Bbb R$ is such that $\{x:f(x)>\alpha\}$ is resolvable for
every $\alpha$, then $f$ is resolvable.   \prooflet{\Prf\ Suppose that
$A\subseteq X$ is non-empty and $\alpha<\beta$ in $\Bbb R$.   Set
$E=\{x:f(x)>\alpha\}$.   If $A\not\subseteq\overline{A\cap E}$, then
$\{x:x\in A$, $f(x)\ge\beta\}$ is not dense in $A$.
Otherwise $\{x:x\in A$, $f(x)\le\alpha\}=A\setminus E$ is not dense
in $A$.\ \Qed

}In particular, the indicator
function of a resolvable set is resolvable.

\spheader 562Qc
A function $f:\Bbb R\to\Bbb R$ which has bounded variation on every
bounded set is resolvable.  \prooflet{\Prf\ If $A\subseteq\Bbb R$ is
non-empty and $\alpha<\beta$ in $\Bbb R$, take $y\in A$.   If $y$ is
isolated in $A$, then we have an open set $U$ such that $U\cap A=\{y\}$,
so that one of $\{x:x\in A$, $f(x)\le\alpha\}$,
$\{x:x\in A$, $f(x)\ge\beta\}$ does not contain $y$ and is
not dense in $A$.   Otherwise, $y$ is in the
closure of one of $A\cap\ooint{y,\infty}$, $A\cap\ooint{-\infty,y}$;
suppose the former.   For each $n\in\Bbb N$ set
$I_n=[y+2^{-n-1},y+2^{-n}]$,
$\delta_n=\Var_{I_n}(f)$.   We have

\Centerline{$\infty>\Var_{\ocint{y,y+1}}(f)
=\sum_{n=0}^{\infty}\delta_n$,}

\noindent so there is an $n\in\Bbb N$ such that
$\delta_m\le\bover14(\beta-\alpha)$ for $m\ge n$.   Take $m>n$ such that
$I_m\cap A\ne\emptyset$, and consider
$U=\interior(I_{m-1}\cup I_m\cup I_{m+1})$.   Then
$\Var_U(f)\le\bover34(\beta-\alpha)$ so $U$ cannot meet both
$\{x:x\in A$, $f(x)\ge\beta\}$ and $\{x:x\in A$, $f(x)\ge\alpha\}$, and one
of these is not dense in $A$.\ \Qed}

\leader{562R}{Theorem} Let $X$ be a second-countable space,
$\sequencen{U_n}$ a sequence running over a base for the topology of $X$,
and $\phi:\Cal T\to\Cal B_c(X)$ the associated
interpretation of Borel codes.   Let $\Cal R$
be the family of resolvable real-valued functions on $X$.   Then
there is a function $\tilde\psi:\Cal R\to\Cal T^{\Bbb R}$ such that

\Centerline{$\phi(\tilde\psi(f)(\alpha))=\{x:f(x)>\alpha\}$}

\noindent for every $f\in\Cal R$ and $\alpha\in\Bbb R$.

\proof{{\bf (a)} Start by fixing a bijection

\Centerline{$k\mapsto(n_k,q_k,q'_k):
\Bbb N\to\Bbb N\times\{(q,q'):q$, $q'\in\Bbb Q$, $q<q'\}$.}

\noindent Next, fix a function $\Theta^*_1:\Cal T^3\times\Bbb N\to\Cal T$
such that

\Centerline{$\phi(\Theta^*_1(T,T',T'',n))
=\phi(T)\cup(U_n\setminus(\phi(T')\cup\phi(T'')))$}

\noindent for $T$, $T'$, $T''\in\Cal T$ and $n\in\Bbb N$, and a function
$\tilde\Theta^*_1:\bigcup_{J\subseteq\Bbb N}\Cal T^J\to\Cal T$ such that
$\phi(\tilde\Theta^*_1(\tau))=\bigcup_{i\in J}\phi(\tau(i))$ whenever
$J\subseteq\Bbb N$ and $\tau\in\Cal T^J$.   (See 562Ca.)

\medskip

{\bf (b)} Given $f\in\Cal R$, define $\zeta<\omega_1$ and a family
$\langle(\tau_{\xi},\tau'_{\xi},k_{\xi})\rangle_{\xi\le\zeta}$ in
$\Cal T^{\Bbb R}\times\Cal T^{\Bbb R}\times\Bbb N$
inductively, as follows.   The inductive hypothesis will be that
$k_{\eta}\ne k_{\xi}$ whenever $\eta<\xi<\zeta$.   Start with
$\tau_0(\alpha)=\tau_0'(\alpha)=\emptyset$ for every $\alpha\in\Bbb R$.

\medskip

{\it Inductive step to a successor ordinal $\xi+1$} Given $\tau_{\xi}$ and
$\tau'_{\xi}$ in $\Cal T^{\Bbb R}$, then for $q<q'$ in $\Bbb Q$
set
$F_{\xi}(q,q')=X\setminus(\phi(\tau_{\xi}(q))\cup\phi(\tau'_{\xi}(q')))$.
Now

\inset{----- if there is a $k\in\Bbb N\setminus\{k_{\eta}:\eta<\xi\}$
such that
$U_{n_k}\cap F_{\xi}(q_k,q'_k)\ne\emptyset$ and $f(x)\ge q_k$ for every
$x\in U_{n_k}\cap F_{\xi}(q_k,q'_k)$, take the first such $k$, and set

$$\eqalign{\tau_{\xi+1}(\alpha)&=\tau_{\xi}(\alpha)
   \text{ for every }\alpha\in\Bbb R,\cr
\tau'_{\xi+1}(\alpha)
&=\Theta^*_1(\tau'_{\xi}(\alpha),\tau_{\xi}(q_k),\tau'_{\xi}(q'_k),n_k)
  \text{ if }\alpha\le q_k,\cr
&=\tau'_{\xi}(\alpha)\text{ if }\alpha>q_k,\cr
k_{\xi}&=k;\cr}$$

----- if this is not so, but there is a
$k\in\Bbb N\setminus\{k_{\eta}:\eta<\xi\}$ such that
$U_{n_k}\cap F_{\xi}(q_k,q'_k)\ne\emptyset$ and $f(x)\le q'_k$ for every
$x\in U_{n_k}\cap F_{\xi}(q_k,q'_k)$, take the first such $k$, and set

$$\eqalign{\tau_{\xi+1}(\alpha)
&=\Theta^*_1(\tau_{\xi}(\alpha),\tau_{\xi}(q_k),\tau'_{\xi}(q'_k),n_k)
  \text{ if }\alpha\ge q'_k,\cr
&=\tau_{\xi}(\alpha)\text{ if }\alpha<q'_k,\cr
\tau'_{\xi+1}(\alpha)
&=\tau'_{\xi}(\alpha)\text{ for every }\alpha\in\Bbb R,\cr
k_{\xi}&=k;\cr}$$

----- and if that doesn't happen either, set $\zeta=\xi$
and stop.}

\medskip

{\it Inductive step to a countable limit ordinal $\xi$} Given
$\ofamily{\eta}{\xi}{(\tau_{\eta},\tau'_{\eta},k_{\eta})}$, set
$I=\{k_{\eta}:\eta<\xi\}$ and define $g:I\to\xi$ by setting
$g(i)=\eta$ whenever $\eta<\xi$ and $k_{\eta}=i$.
Now set

\Centerline{$\tau_{\xi}(\alpha)
=\tilde\Theta^*_1(\familyiI{\tau_{g(i)}(\alpha)})$,
\quad$\tau'_{\xi}(\alpha)
=\tilde\Theta^*_1(\familyiI{\tau'_{g(i)}(\alpha)})$}

\noindent for every $\alpha\in\Bbb R$.

\medskip

{\bf (c)} Now an induction on $\xi$ shows that

\Centerline{$\phi(\tau_{\eta}(\alpha))\subseteq\phi(\tau_{\xi}(\alpha))$,
\quad$\phi(\tau'_{\eta}(\alpha))\subseteq\phi(\tau'_{\xi}(\alpha))$,}

\Centerline{$\phi(\tau_{\xi}(\alpha))\subseteq\{x:f(x)\le\alpha\}$,
\quad$\phi(\tau'_{\xi}(\alpha))\subseteq\{x:f(x)\ge\alpha\}$}

\noindent whenever $\eta\le\xi$, $\alpha\in\Bbb R$ and the codes here are
defined.   Next, if $k_{\eta}=k$ is defined, we must have
$U_{n_k}\cap F_{\eta}(q_k,q'_k)\ne\emptyset$ and

\inset{----- either $f(x)\ge q_k$ for every
$x\in U_{n_k}\cap F_{\eta}(q_k,q'_k)$ and
$\phi(\tau'_{\eta+1}(q_k))
=\phi(\tau'_{\eta}(q_k))\cup(U_{n_k}\cap F_{\eta}(q_k,q_k'))$

----- or $f(x)\le q'_k$ for every
$x\in U_{n_k}\cap F_{\eta}(q_k,q'_k)$ and
$\phi(\tau_{\eta+1}(q_k))
=\phi(\tau_{\eta}(q_k))\cup(U_{n_k}\cap F_{\eta}(q_k,q_k'))$.}

\noindent In either case,
$U_{n_k}\cap F_{\eta}(q_k,q_k')$ must be disjoint from $F_{\xi}(q_k,q'_k)$
for every $\xi>\eta$ for which $F_{\xi}$ is defined;
consequently we cannot have $k_{\xi}=k$ for any
$\xi>\eta$.   The induction must therefore stop.

$F_{\zeta}(q,q')=\emptyset$ whenever $q$, $q'\in\Bbb Q$ and $q<q'$.
\Prf\Quer\ Otherwise, because $f$ is resolvable, there is an
$n\in\Bbb N$ such that $V=U_n\cap F_{\zeta}(q,q')$ is non-empty and
either $f(x)\ge q$ for every $x\in V$ or $f(x)\le q'$ for every $x\in V$.
Let $k\in\Bbb N$ be such that $n_k=n$, $q_k=q$ and $q'_k=q'$;  then
$U_{n_k}$ meets $F_{\zeta}(q_k,q'_k)$ so $k\ne k_{\eta}$ for any
$\eta<\zeta$.   But this means that we ought to have proceeded according to
one of the first two alternatives in the single-step inductive stage,
and ought not to have stopped at $\zeta$.\ \Bang\Qed

\medskip

{\bf (d)} Now set

\Centerline{$\tau(\alpha)
=\tilde\Theta^*_1(\langle\tau'_{\zeta}(q_n)\rangle_{n\in\Bbb N,q_n>\alpha})$}

\noindent for $\alpha\in\Bbb R$.   Then

$$\eqalign{\phi(\tau(\alpha))
&=\bigcup_{n\in\Bbb N,q_n>\alpha}\phi(\tau'_{\zeta}(q_n))\cr
&\subseteq\bigcup_{n\in\Bbb N,q_n>\alpha}\{x:f(x)\ge q_n\}
\subseteq\{x:f(x)>\alpha\}\cr}$$

\noindent for every $\alpha$.   \Quer\ If $\alpha$ is such that
$\phi(\tau(\alpha))\subset\{x:f(x)>\alpha\}$, let $x\in X$ and
$n\in\Bbb N$ be such that $f(x)>q'_n>q_n>\alpha$ and
$x\notin\phi(\tau(\alpha))$.   Then $x\notin\phi(\tau'_{\zeta}(q_n))$;
but also $f(y)\le q'_n$ for every $y\in\phi(\tau_{\zeta}(q'_n))$, so
$x\notin\phi(\tau_{\zeta}(q'_n))$ and $x\in F_{\zeta}(q_n,q'_n)$, which is
supposed to be impossible.\ \Bang

So we can set $\tilde\psi(f)=\tau$.
}%end of proof of 562R

\leader{562S}{Codable families of codable functions (a)} If $X$ and $Y$
are second-countable spaces, a family $\familyiI{f_i}$ of functions from
$X$ to $Y$ is a {\bf codable family of codable Borel functions} if
$\langle f_i^{-1}[H]\rangle_{i\in I,H\subseteq Y\text{ is open}}$ is a
codable family in $\Cal B_c(X)$.

\spheader 562Sb Uniformizing the arguments of 562N, it is easy to check
that a family $\familyiI{f_i}$ of real-valued functions on $X$ is a codable
family of codable Borel functions iff there is a family
$\familyiI{\tau_i}$ in $\tilde{\Cal T}$ such that\cmmnt{, in 
the language there,} $f_i=\tilde\phi(\tau_i)$ for every $i$.

\spheader 562Sc \cmmnt{In this language,} 562Ne can be rephrased as

\inset{if $\sequencen{f_n}$ is a codable sequence of real-valued
codable Borel functions on $X$, there
is a codable Borel function $f$ such
that $f(x)=\liminf_{n\to\infty}f_n(x)$ whenever the lim inf is finite,}

\noindent and 562R implies that

\inset{the family of resolvable real-valued functions on $X$ is a codable
family of codable Borel functions.}

\spheader 562Sd If $X$, $Y$ and $Z$ are second-countable spaces,
$\familyiI{f_i}$ is a codable family of codable Borel functions from
$X$ to $Y$, and $\familyiI{g_i}$ is a codable family of codable
Borel functions from $Y$ to $Z$, then $\familyiI{g_if_i}$ is a codable
family of codable functions from $X$ to $Z$\prooflet{;  this
is because the proof of 562Mb
gives a recipe for calculating a code for the composition of codable
functions, which can be performed simultaneously on the compositions
$g_if_i$ if we are given codes for the functions $g_i$ and $f_i$}.

\cmmnt{\spheader 562Se Extending the remarks in 562O(b)-(c), we see that 
(for instance) we
can define a sequence $\sequencen{\Phi_n}$ such that
$\Phi_n$ is a function from $\tilde{\Cal T}^{n+1}$ to $\tilde{\Cal T}$ 
for every $n$, and
$\tilde\phi(\Phi_n(\langle\tau_i\rangle_{i\le n}))
=\sum_{i=0}^n\tilde\phi(\tau_i)$ whenever 
$\tau_0,\ldots,\tau_n\in\tilde{\Cal T}$;  so that if
$\sequencen{f_n}$ is a codable sequence of codable Borel functions, then
$\sequencen{\sum_{i=0}^nf_i}$ is codable.}

\leader{562T}{Codable Baire sets}\cmmnt{ The
ideas here can be adapted to give a
theory of Baire algebras in general topological spaces.}
Start by settling on a sequence running over a base for the topology
of $\BbbR^{\Bbb N}$, with the associated interpretation
$\phi:\Cal T\to\Cal B_c(\BbbR^{\Bbb N})$  of Borel codes.
Let $X$ be a topological space.

\spheader 562Ta A subset $E$ of $X$ is a {\bf codable Baire set} if it
is of the form $f^{-1}[F]$ for some continuous $f:X\to\BbbR^{\Bbb N}$ and
$F\in\Cal B_c(\BbbR^{\Bbb N})$;  write $\CalBa_c(X)$ for the family of such
sets.   If $E\in\CalBa_c(X)$, then a {\bf code} for $E$ will be a pair
$(f,T)$ where $f:X\to\BbbR^{\Bbb N}$ is continuous, $T\in\Cal T$ and
$E=f^{-1}[\phi(T)]$.   A family $\familyiI{E_i}$ in $\CalBa_c(X)$ 
is\cmmnt{ now} a {\bf codable} family if there is a family
$\familyiI{(f_i,T^{(i)})}$ such that $(f_i,T^{(i)})$ codes $E_i$ for every $i$.

\spheader 562Tb{\bf (i)} Suppose that $\familyiI{f_i}$ is a countable
family of
continuous functions from $X$ to $\BbbR^{\Bbb N}$, and
$\familyiI{T^{(i)}}$ a
family in $\Cal T$.   Then there are a continuous function
$f:X\to\BbbR^{\Bbb N}$ and a sequence $\sequence{i}{\hat T^{(i)}}$ in $\Cal T$ such
that $(f,\hat T^{(i)})$ codes the same Baire set as $(f_i,T^{(i)})$ for every
$i\in I$.   \prooflet{\Prf\ If $I$ is empty, this is trivial.   Otherwise,
$I\times\Bbb N$ is countably infinite, so
$(\BbbR^{\Bbb N})^I\cong\BbbR^{I\times\Bbb N}$ is homeomorphic to
$\BbbR^{\Bbb N}$;  let
$h:\BbbR^{\Bbb N}\to(\BbbR^{\Bbb N})^I$ be a homeomorphism, and
set $f(x)=h^{-1}(\familyiI{f_i(x)})$ for each $x\in\Bbb N$.
Then $f_i=\pi_ihf$ for each $i$, where $\pi_i(z)=z(i)$ for
$z\in(\BbbR^{\Bbb N})^I$.   Now
$\langle(\pi_ih)^{-1}[V_n]\rangle_{i\in I,n\in\Bbb N}$
is a family of open sets
in $\BbbR^{\Bbb N}$, so is codable (562I, or otherwise);  let
$\langle T^{(i,n)}\rangle_{i\in I,n\in\Bbb N}$ be a family in $\Cal T$
such that $\phi(T^{(i,n)})=(\pi_ih)^{-1}[V_n]$ whenever $n\in\Bbb N$
and $i\in I$.
The construction of part (a)(ii)$\Rightarrow$(iii) in the proof of 562M
gives us a family $\familyiI{\Theta_i}$ of functions from $\Cal T$ to
$\Cal T$ such that $(\pi_ih)^{-1}[\phi(T)]=\phi(\Theta_i(T))$ whenever
$i\in I$ and $T\in\Cal T$.   So we can take $\hat T^{(i)}=\Theta_i(T^{(i)})$,
and we shall have

\Centerline{$f_i^{-1}[\phi(T^{(i)})]
=f^{-1}[(\pi_ih)^{-1}[\phi(T^{(i)})]]
=f^{-1}[\phi(\Theta_i(T^{(i)}))]
=f^{-1}[\phi(\hat T^{(i)})]$}

\noindent for every $i$, as required.\ \Qed}

\medskip

\quad{\bf (ii)} It follows that if $\sequence{i}{E_i}$ is a codable
sequence in $\CalBa_c(X)$ then $\bigcup_{i\in\Bbb N}E_i$ and
$\bigcap_{i\in\Bbb N}E_i$ belong to $\CalBa_c(X)$.   \prooflet{\Prf\
By (i), we have a continuous $f:X\to\BbbR^{\Bbb N}$ and a sequence
$\sequence{i}{\hat T^{(i)}}$ in $\Cal T$ such that $E_i=f^{-1}[\phi(\hat T^{(i)})]$ for
every $i\in\Bbb N$.   Now 562K tells us that
$F=\bigcup_{i\in\Bbb N}\phi(\hat T^{(i)})$ and
$F'=\bigcap_{i\in\Bbb N}\phi(\hat T^{(i)})$ are codable, so
$f^{-1}[F]=\bigcup_{i\in\Bbb N}E_i$ and
$f^{-1}[F']=\bigcap_{i\in\Bbb N}E_i$ belong to $\CalBa_c(X)$.\ \Qed}

\medskip

\quad{\bf (iii)}\cmmnt{ In particular,} $\CalBa_c(X)$ is\cmmnt{ closed
under finite
intersections; as it is certainly closed under complementation,
it is} an algebra of subsets of $X$.   Every zero set belongs to
$\CalBa_c(X)$.   \prooflet{\Prf\ If $g:X\to\Bbb R$ is continuous, set
$f(x)(i)=g(x)$ for $x\in X$, $i\in\Bbb N$;  then
$H=\{z:z\in\BbbR^{\Bbb N}$, $z(0)=0\}$ is closed, therefore a codable Borel
set, and $g^{-1}[\{0\}]=f^{-1}[H]$ is a codable Baire set.\ \Qed}

\medskip

\quad{\bf (iv)} If $Y$ is another topological space and
$g:X\to Y$ is continuous,
then $\familyiI{g^{-1}[F_i]}$ is a codable family
in $\CalBa_c(X)$ for every codable family $\familyiI{F_i}$ in
$\CalBa_c(Y)$.   \prooflet{\Prf\ If $\familyiI{(f_i,T^{(i)})}$ codes
$\familyiI{F_i}$, then $\familyiI{(f_ig,T^{(i)})}$ codes
$\familyiI{g^{-1}[F_i]}$.\ \Qed}

\spheader 562Tc{\bf (i)} 
A function $f:X\to\Bbb R$ is a {\bf codable Baire
function} if there are a continuous $g:X\to\BbbR^{\Bbb N}$ and a codable
Borel function $h:\BbbR^{\Bbb N}\to\Bbb R$ such that $f=hg$.
A family $\familyiI{f_i}$ of
codable Baire functions is a {\bf codable} family if there is a family
$\familyiI{(g_i,h_i)}$ such that $g_i:X\to\BbbR^{\Bbb N}$ is a continuous
function for every $i\in I$ and $\familyiI{h_i}$ is a codable family of
codable Borel functions from $\BbbR^{\Bbb N}$ to $\Bbb R$.

\medskip

\quad{\bf (ii)} Suppose that $\sequencen{f_n}$ is a codable sequence of
codable Baire functions from $X$ to $\Bbb R$.   Then there are a continuous
function $g:X\to\BbbR^{\Bbb N}$ and a codable sequence $\sequencen{h_n}$ of
codable Borel functions from $\BbbR^{\Bbb N}$ to $\Bbb R$ such that
$f_n=h_ng$ for every $n\in\Bbb N$.   \prooflet{\Prf\ Let
$\sequencen{(g_n,h'_n)}$ be such that $g_n:X\to\BbbR^{\Bbb N}$ is continuous
for every $n$, $\sequencen{h'_n}$ is a codable sequence of codable Borel
functions from $\BbbR^{\Bbb N}\to\Bbb R$, and $f_n=h'_ng_n$ for each $n$.
Now $\langle\{x:h'_n(x)>q\}\rangle_{n\in\Bbb N,q\in\Bbb Q}$ is a codable
family in $\Cal B_c(\BbbR^{\Bbb N})$;  let
$\langle T'_{nq}\rangle_{n\in\Bbb N,q\in\Bbb Q}$ be a family in $\Cal T$
coding it.   By (b-i) above, there are a continuous function
$g:X\to\BbbR^{\Bbb N}$ and a family
$\langle T^{(n,q)}\rangle_{n\in\Bbb N,q\in\Bbb N}$ in $\Cal T$ such that

\Centerline{$g^{-1}[\phi(T^{(n,q)})]=g_n^{-1}[\phi(T'_{nq})]
=\{x:f_n(x)>q\}$}

\noindent for every $n\in\Bbb N$ and $q\in\Bbb Q$.

To convert $\langle T^{(n,q)}\rangle_{n\in\Bbb N,q\in\Bbb Q}$ into a code for
a sequence of real-valued functions on $\BbbR^{\Bbb N}$, I copy ideas from
the proof of 562N.   Let

\Centerline{$\Theta_0:\Cal T\to\Cal T$,
\quad$\Theta_1:\Cal T\times\Cal T\to\Cal T$,}

\Centerline{$\Theta_3:\Cal T\times\Cal T\to\Cal T$,
\quad$\tilde\Theta_1:\bigcup_{I\subseteq\Bbb Q}\Cal T^I\to\Cal T$}

\noindent be such that

\Centerline{$\phi(\Theta_0(T))=X\setminus\phi(T)$,
\quad$\phi(\Theta_1(T,T'))=\phi(T)\cup\phi(T'),$}

\Centerline{$\phi(\Theta_3(T,T'))=\phi(T)\setminus\phi(T')$,
\quad$\phi(\tilde\Theta_1(\tau))=\bigcup_{q\in I}\phi(\tau(q)))$}

\noindent for $T$, $T'\in\Cal T$, $I\subseteq\Bbb Q$ and $\tau\in\Cal T^I$.
Now, for $\alpha\in\Bbb R$ and $n\in\Bbb N$, set

\Centerline{$\tau'_n(\alpha)
=\tilde\Theta_1(\langle T^{(n,q)}\rangle_{q\in\Bbb Q,q\ge\alpha})$,}

\Centerline{$T^{(n)}
=\tilde\Theta_1(\sequence{k}{\Theta_3(\tau'_n(-k),\tau'_n(k))})$,}

$$\eqalign{\tau_n(\alpha)
&=\Theta_3(\tau'_n(\alpha),\Theta_0(T^{(n)}))\text{ if }\alpha\ge 0,\cr
&=\Theta_1(\tau'_n(\alpha),\Theta_0(T^{(n)}))\text{ if }\alpha<0.\cr}$$

\noindent We now have a sequence $\sequencen{\tau_n}$ in $\tilde{\Cal T}$
(as defined in 562N) coding a sequence $\sequencen{h_n}$ of Borel functions
from $\BbbR^{\Bbb N}$ to $\Bbb R$
such that $f_n=h_ng$ for every $n$ (see 562Sb).\ \Qed}

\medskip

\quad{\bf (iii)}
If $\sequencen{f_n}$ is a codable sequence of codable Baire
functions, there is a codable Baire function $f$ such that
$f(x)=\liminf_{n\to\infty}f_n(x)$ whenever the lim inf is finite.
\prooflet{\Prf\ Take $g$ and $\sequencen{h_n}$ as in (i);  by
562Ne, there is a codable Borel function $h$ such that
$h(z)=\liminf_{n\to\infty}h_n(z)$ whenever $z\in\BbbR^{\Bbb N}$ is such
that the $\liminf$ is finite, and $f=hg:X\to\Bbb R$ will serve.\ \Qed}

\medskip

\quad{\bf (iv)}
The family of codable Baire functions is a Riesz subspace of
$\BbbR^X$ containing all continuous functions and closed under
multiplication.   \prooflet{(This time, use (i) and 562Nd.)}

\medskip

\quad{\bf (v)} The family of continuous real-valued
functions on $X$ is a codable family of codable Baire functions.
\prooflet{(For $f\in C(X)$, define $g_f\in C(X;\BbbR^{\Bbb N})$ by setting
$g_f(x)(n)=f(x)$ for every $x\in X$ and $n\in\Bbb N$;  setting
$\pi_0(z)=z(0)$ for $z\in\BbbR^{\Bbb N}$, $\family{f}{C(X)}{(g_f,\pi_0)}$
is a family of codes for $C(X)$.)}

\medskip

\quad{\bf (vi)}\dvAnew{2014} If $E\subseteq X$, then $E\in\CalBa_c(X)$ iff 
$\chi E:X\to\Bbb R$ is a codable Baire function.   \prooflet{\Prf\

$$\eqalignno{E\in\CalBa_c(X)
&\iff\text{ there are a continuous }g:X\to\BbbR^{\Bbb N}
\cr&\mskip100mu\text{ and an }
F\in\Cal B_c(\BbbR^{\Bbb N})\text{ such that }E=g^{-1}[F]
\cr&\iff\text{ there are a continuous }g:X\to\BbbR^{\Bbb N}
\cr&\mskip100mu\text{ and an }
F\in\Cal B_c(\BbbR^{\Bbb N})\text{ such that }\chi E=(\chi F)g
\cr&\iff\text{ there are a continuous }g:X\to\BbbR^{\Bbb N}
\cr&\mskip100mu\text{ and a codable
Borel function }h:\BbbR^{\Bbb N}\text{ such that }\chi E=hg\cr
\displaycause{562Nf, because if $\chi E=hg$ then $\chi E=(\chi F)g$ where
$F=\{y:h(y)>0\}$}
&\iff\chi E\text{ is a codable Baire function}.  \text{ \Qed}\cr}$$
}

\spheader 562Td\dvAnew{2014}
If $\sequencen{f_n}$ is a codable sequence of codable Baire
functions from $X$ to $\Bbb R$, then 
$\langle f_n^{-1}[H]\rangle_{n\in\Bbb N,H\subseteq\Bbb R\text{ is open}}$
is codable.   \prooflet{\Prf\ By (c-i), we have a continuous
$g:X\to\BbbR^{\Bbb N}$ and a codable sequence $\sequencen{h_n}$ of
codable Borel functions from $\BbbR^{\Bbb N}$ to $\Bbb R$ such that
$f_n=h_ng$ for every $n$.   Let $\psi:\Cal T\to\Cal B_c(\Bbb R)$ 
be an interpretation of Borel codes corresponding to some enumeration of a
base for the topology of $\Bbb R$.
By 562Md, $g$ is codable;  by 562M(a-iii), 
there is a function $\Theta:\Cal T\to\Cal T$ such that 
$g^{-1}[\psi(T)]=\phi(\Theta(T))$ for every $T\in\Cal T$.   Now
$\langle h_n^{-1}[H]\rangle_{n\in\Bbb N,H\subseteq\Bbb R\text{ is open}}$
is codable, by the definition in 562S;  that is, there is a family
$\langle T_{nH}\rangle_{n\in\Bbb N,H\subseteq\Bbb R\text{ is open}}$ in
$\Cal T$ such that $\phi(T_{nH})=h_n^{-1}[H]$ for all $n$ and $H$.   Now

\Centerline{$f_n^{-1}[H]=g^{-1}[h_n^{-1}[H]]
=g^{-1}[\phi(T_{nH})]=\psi(\Theta(T_{nH}))$}

\noindent for all $n$ and $H$, so we have a coding of 
$\langle f_n^{-1}[H]\rangle_{n\in\Bbb N,H\subseteq\Bbb R\text{ is open}}$.\
\Qed}

\leader{562U}{Proposition}
Let $(X,\frak T)$ be a second-countable space.   Then there is a
second-countable topology $\frak S$ on $X$, codably Borel equivalent to
$\frak T$, such that
$\Cal B_c(X)=\CalBa_c(X,\frak S)$ and the codable families in
$\Cal B_c(X)$ are exactly the codable families in $\CalBa_c(X,\frak S)$.

\proof{{\bf (a)} By 562Pb there is a topology $\frak S$ on $X$,
finer than $\frak T$, generated by a countable algebra $\Cal E$ of subsets
of $X$, which is codably Borel equivalent to $\frak T$.
Let $\sequencen{U_n}$ be a sequence running over $\Cal E$.
Define $g_0:X\to\BbbR^{\Bbb N}$ by setting
$g_0(x)=\sequencen{\chi U_n(x)}$ for each $x\in X$.
Then $g_0$ is continuous.
Set $W_n=\{z:z\in\BbbR^{\Bbb N}$, $z(n)>0\}$ for each $n$, so that
$W_n\subseteq\BbbR^{\Bbb N}$ is open and $U_n=g_0^{-1}[W_n]$;  let
$\sequencen{V_n}$ be a sequence running over a base for the topology of
$\BbbR^{\Bbb N}$ and such that $V_{2n}=W_n$ for
every $n$.   Let $\phi:\Cal T\to\Cal B_c(X)$,
$\phi':\Cal T\to\Cal B_c(\BbbR^{\Bbb N})$ be the interpretations of Borel
codes corresponding to $\sequencen{U_n}$, $\sequencen{V_n}$ respectively.

\medskip

{\bf (b)} We have a function $\Theta:\Cal T\to\Cal T\setminus\{\emptyset\}$
such that $\phi(T)=g^{-1}[\phi'(\Theta(T))]$ for every $T\in\Cal T$.
\Prf\ Induce on
$r(T)$.   As usual, set $A_T=\{n:\fraction{n}\in T\}$.
If $r(T)=0$, take $\Theta(T)\in\Cal T\setminus\{\emptyset\}$ such
that $\phi'(\Theta(T))=\emptyset$.   If $r(T)=1$,
set $\Theta(T)=\{\fraction{2n}:n\in A_T\}$;   then

\Centerline{$\phi'(\Theta(T))=\bigcup\{V_{2n}:n\in A_T\}$,
\quad$g_0^{-1}[\phi'(\Theta(T))]
=\bigcup\{U_n:n\in A_T\}=\phi(T)$.}

\noindent If $r(T)>1$, set

\Centerline{$\Theta(T)=\{\fraction{i}:i\in A_T\}
\cup\{\fraction{i}^{\smallfrown}\sigma:i\in A_T$,
  $\sigma\in\Theta(T_{\fraction{i}})\}$.  \Qed}

This means that if we have any codable family in $\Cal B_c(X)$, coded by a
family $\familyiI{T^{(i)}}$ in $\Cal T$, $\familyiI{(g_0,\Theta(T^{(i)}))}$ will
code the same family in $\CalBa_c(X,\frak S)$.

\medskip

{\bf (c)} Next, there is a function
$\Phi:C((X,\frak S);\BbbR^{\Bbb N})\times\Cal T
\to\Cal T\setminus\{\emptyset\}$ such that
$g^{-1}[\phi'(T))]=\phi(\Phi(g,T))$ for every $\frak S$-continuous
$g:X\to\BbbR^{\Bbb N}$ and $T\in\Cal T$.   \Prf\ If $r(T)\le 1$
and $g^{-1}[\phi'(T)]$ is
empty take $\Phi(g,T)\in\Cal T\setminus\{\emptyset\}$ such that
$\phi(\Phi(g,T))=\emptyset$.   If $r(T)=1$ and $g^{-1}[\phi'(T)]$ is
not empty set

\Centerline{$\Phi(g,T)
=\{\fraction{n}:U_n\subseteq g^{-1}[\phi'(T)]\}$.}

\noindent If $r(T)>1$ set

\Centerline{$\Phi(g,T)
=\{\fraction{i}:i\in A_T\}\cup\{\fraction{i}^{\smallfrown}\sigma:
  i\in A_T$, $\sigma\in\Phi(g,T_{\fraction{i}})\}$.   \Qed}

\noindent So given any codable family in $\CalBa_c(X,\frak S)$, coded by a
family $\familyiI{(g_i,T^{(i)})}$ in
$C((X,\frak S);\BbbR^{\Bbb N})\times\Cal T$, $\familyiI{\Phi(g_i,T^{(i)})}$
will code it in $\Cal B_c(X)$.
}%end of proof of 562U

\leader{562V}{}\cmmnt{ A
different use of Borel codes will appear when we come to re-examine
a result in Volume 3.   I will defer the application to 566O, but
the first part of the argument fits naturally into the ideas of this
section.

\medskip

\noindent}{\bf Theorem} (a) Let $\frak A$ be a Dedekind $\sigma$-complete
Boolean algebra, and $\sequencen{a_n}$ a sequence in $\frak A$.   Then we
have an interpretation $\phi:\Cal T\to\frak A$ of Borel codes such that

$$\eqalign{\phi(T)&=\sup_{i\in A_T}a_i\text{ if }r(T)\le 1,\cr
&=\sup_{i\in A_T}1\Bsetminus\phi(T_{\fraction{i}})\text{ if }r(T)>1,\cr}$$

\noindent where $A_T=\{i:\fraction{i}\in T\}$ as usual.

(b) For $n\in\Bbb N$, set $E_n=\{x:x\in\{0,1\}^{\Bbb N}$, $x(n)=1\}$.
Let $\frak A$ be a Dedekind $\sigma$-complete Boolean
algebra, and $\sequencen{a_n}$ a sequence in $\frak A$.   Let
$\phi:\Cal T\to\frak A$ and $\psi:\Cal T\to\Cal P(\{0,1\}^{\Bbb N})$
be the interpretations of Borel codes
corresponding to $\sequencen{a_n}$ and $\sequencen{E_n}$.
If $T$, $T'\in\Cal T$ are such that $\phi(T)\notBsubseteq\phi(T')$, then
$\psi(T)\not\subseteq\psi(T')$.

\proof{{\bf (a)} Define $\phi(T)$ inductively on the rank of $T$, as in
562Ba.

\medskip

{\bf (b)} Let $\sequencen{T^{(n)}}$ be a sequence running over
$\{T,T'\}\cup\{T_{\sigma}:\sigma\in S^*\}
\cup\{T'_{\sigma}:\sigma\in S^*\}$.
Define $\sequencen{c_n}$ inductively, as follows.
$c_0=\phi(T)\Bsetminus\phi(T')$.   Given that
$c_n\in\frak A\setminus\{0\}$, then

\inset{----- if $r(T^{(n)})\le 1$ and there
is an $i\in A_{T^{(n)}}$ such that $c_n\Bcap a_i\ne 0$, take the first such
$i$ and set $c_{n+1}=c_n\Bcap a_i$;

----- if $r(T^{(n)})>1$ and there is an $i\in A_{T^{(n)}}$ such that
$c_n\Bsetminus\phi(T^{(n)}_{\fraction{i}})\ne 0$, take the first such $i$
and set $c_{n+1}=c_n\Bsetminus\phi(T^{(n)}_{\fraction{i}})$;

----- otherwise, set $c_{n+1}=c_n$.}

\noindent Continue.

At the end of the induction, define $x\in\{0,1\}^{\Bbb N}$ by saying that
$x(i)=1$ iff there is an $n\in\Bbb N$ such that $c_n\Bsubseteq a_i$.
Now we find that, for every $m\in\Bbb N$,

\inset{----- if $x\in\psi(T^{(m)})$
there is an $n\in\Bbb N$ such that $c_n\Bsubseteq\phi(T^{(m)})$,

----- if $x\notin\psi(T^{(m)})$ there is an $n\in\Bbb N$ such that
$c_n\Bcap\phi(T^{(m)})=0$.}

\Prf\ Induce on $r(T^{(m)})$.   If $r(T^{(m)})\le 1$ then

$$\eqalign{x\in\psi(T^{(m)})
&\Longrightarrow\text{ there is an }i\in A_{T^{(m)}}\text{ such that }
  x\in E_i\cr
&\Longrightarrow\text{ there are }i\in A_{T^{(m)}},\,n\in\Bbb N
  \text{ such that }c_n\Bsubseteq a_i\cr
&\Longrightarrow\text{ there is an }n\in\Bbb N\text{ such that }
  c_n\Bsubseteq\phi(T^{(m)}),\cr}$$

$$\eqalign{x\notin\psi(T^{(m)})
&\Longrightarrow x\notin E_i\text{ for every }i\in A_{T^{(m)}}\cr
&\Longrightarrow c_{m+1}\notBsubseteq a_i
  \text{ for every }i\in A_{T^{(m)}}\cr
&\Longrightarrow c_m\Bcap a_i=0
  \text{ for every }i\in A_{T^{(m)}}
\Longrightarrow c_m\Bcap\phi(T^{(m)})=0.\cr}$$

\noindent If $r(T^{(m)})>1$ then

$$\eqalignno{x\in\psi(T^{(m)})
&\Longrightarrow\text{ there is an }i\in A_{T^{(m)}}\text{ such that }
  x\notin\psi(T^{(m)}_{\fraction{i}})\cr
&\Longrightarrow\text{ there are }i\in A_{T^{(m)}},\,n\in\Bbb N
  \text{ such that }c_n\Bcap\phi(T^{(m)}_{\fraction{i}})=0\cr
\displaycause{by the inductive hypothesis, because $T^{(m)}_{\fraction{i}}$
is always equal to $T^{(k)}$ for some $k$, and
$r(T^{(m)}_{\fraction{i}})<r(T^{(m)})$}
&\Longrightarrow\text{ there is an }n\in\Bbb N
  \text{ such that }c_n\Bsubseteq\phi(T^{(m)}),\cr}$$

$$\eqalign{x\notin\psi(T^{(m)})
&\Longrightarrow x\in\psi(T^{(m)}_{\fraction{i}})
  \text{ for every }i\in A_{T^{(m)}}\cr
&\Longrightarrow\text{ for every }i\in A_{T^{(m)}}
  \text{ there is an }n\in\Bbb N\text{ such that }
  c_n\Bsubseteq\phi(T^{(m)}_{\fraction{i}})\cr
&\Longrightarrow
  c_{m+1}\notBsubseteq 1\Bsetminus\phi(T^{(m)}_{\fraction{i}})
  \text{ for every }i\in A_{T^{(m)}}\cr
&\Longrightarrow c_m\Bsetminus\phi(T^{(m)}_{\fraction{i}})=0
  \text{ for every }i\in A_{T^{(m)}}\cr
&\Longrightarrow c_m\Bcap\phi(T^{(m)})=0.  \text{ \Qed}\cr}$$

In particular, since both $T$ and $T'$ appear in the list
$\sequence{m}{T^{(m)}}$, $c_n\Bcap\phi(T)\ne 0$ and $c_n\Bcap\phi(T')=0$
for every $n$, $x\in\psi(T)\setminus\psi(T')$ and
$\psi(T)\not\subseteq\psi(T')$.
}%end of proof of 562V

\exercises{\leader{562X}{Basic exercises (a)}
%\spheader 562Xa
Let $X$ be a regular second-countable space.   Show that a
resolvable subset of $X$ is F$_{\sigma}$.   \Hint{in the proof of 562I,
show that $\phi(T^{(\xi)})$ is always F$_{\sigma}$.}
%562I

\spheader 562Xb Let $X$ be a second-countable space and
$\langle E_{ni}\rangle_{n,i\in\Bbb N}$ a family of resolvable subsets of
$X$.   Show that $\bigcap_{n\in\Bbb N}\bigcup_{i\in\Bbb N}E_{ni}$ is
a codable Borel set.
%562K

\spheader 562Xc Let $X$ be a second-countable space and
$\familyiI{E_i}$ a codable family in $\Cal B_c(X)$.   (i) Show that
$\family{i}{J}{E_i}$ is codable for every $J\subseteq I$.
(ii) Show that if $I$ is countable and not empty then
$\bigcup_{i\in I}E_i$ and $\bigcap_{i\in I}E_i$ are codable Borel sets.
(iii) Show that if $h:J\times\Bbb N\to I$ is a function, where $J$ is any
other set, then $\family{j}{J}{\bigcup_{n\in\Bbb N}E_{h(j,n)}}$ is a
codable family.   (iv) Show that if $\familyiI{F_i}$ is another codable
family in $\Cal B_c(X)$ then $\familyiI{E_i\cap F_i}$ and
$\familyiI{E_i\symmdiff F_i}$ are codable families.
%562K

\spheader 562Xd Let $X$ and $Y$ be
second-countable spaces and $f:X\to Y$ a function.   Suppose that
$\{F:F\subseteq Y$, $f^{-1}[F]$ is resolvable$\}$ includes a countable
network for the topology of $Y$.   Show that $f$ is a codable Borel
function.
%562K

\spheader 562Xe Let $X$ be a second-countable space and $\familyiI{E_i}$
a family in $\Cal B_c(X)$.   (i) Show that $\{J:J\subseteq I$,
$\family{i}{J}{E_i}$ is codable$\}$ is an ideal of subsets of $I$.
(ii) Show that if every $E_i$ is resolvable then $\familyiI{E_i}$ is
codable.
%562K

\spheader 562Xf Let $X$ be a second-countable space and $f:X\to\Bbb R$ a
function.   Show that $f$ is a codable Borel function iff
$\{(x,\alpha):x\in X$, $\alpha<f(x)\}$ is a codable Borel subset of
$X\times\Bbb R$.
%562N

\spheader 562Xg Let $X$ be a topological space and $f$, $g:X\to\Bbb R$
resolvable functions.   (i) Show that $f\vee g$ and
$\alpha f$ are resolvable
for any $\alpha\in\Bbb R$.   (ii) Show that if $f$ is bounded
then $f+g$ is resolvable.   (iii) Show that if $f$ and $g$ are bounded,
$f\times g$ is resolvable.   (iv) Show that if $f$ and $g$
are non-negative, then $f+g$ is resolvable.   (v) Show that if
$h:\Bbb R\to\Bbb R$ is continuous and $h^{-1}[\{\alpha\}]$ is finite for
every $\alpha\in\Bbb R$, then $hf$ is resolvable.
%562Q

\spheader 562Xh Let $f:\Bbb R\to\Bbb R$ be such that
$\lim_{t\downarrow x}f(t)$ is defined in $[-\infty,\infty]$
for every $x\in\Bbb R$.   Show that $f$ is resolvable.
%562Q

\spheader 562Xi Let $X$ be a second-countable space and
$\sequencen{f_n}$ a sequence of resolvable real-valued functions on $X$.
Show that there is a codable Borel function $g$ such that
$g(x)=\lim_{n\to\infty}f_n(x)$ for any $x$ such that the limit is defined
in $\Bbb R$.
%562R

\spheader 562Xj Let $X$ be a second-countable space and $Y$ a subspace of
$X$.   Show that a family $\familyiI{g_i}$ in $\BbbR^Y$ is a codable family
of codable Borel functions iff there is a codable family $\familyiI{f_i}$
of codable real-valued Borel functions on $X$ such that $g_i=f_i\restr Y$
for every $i\in I$.
%562S 562E

\sqheader 562Xk Let $X$ be a second-countable space.   Show that every
codable family of codable Baire subsets of $X$ is a codable family of
codable Borel subsets of $X$.
%562T

\sqheader 562Xl Let $X$ be a regular second-countable space.
Show that every
codable family of codable Borel subsets of $X$ is a codable family of
codable Baire subsets of $X$.   \Hint{561Xk.}
%562T

\leader{562Y}{Further exercises (a)}
%\spheader 562Ya
Let $X$ be a second-countable space, $Y$ a T$_0$
second-countable space and $f:X\to Y$ a function with graph
$\Gamma\subseteq X\times Y$.   (i) Show that if $f$ is a codable Borel
function, then $\Gamma$ is a codable Borel subset of $X\times Y$.
(ii) Show that if $X$ and $Y$ are Polish and $\Gamma$ is a codable Borel
subset of $X\times Y$, then $f$ is a codable Borel function.
%562F  562L

\spheader 562Yb
Show that there is an analytic subset of $\NN$ which is not
a codable Borel set.   \Hint{423L.}
%*562F out of order query

\spheader 562Yc Show
that if $X$ is a Polish space then a subset of $X$ is resolvable iff it is
both F$_{\sigma}$ and G$_{\delta}$.
%562I

\spheader 562Yd Let $X$ be a Polish space.   Show that a function
$f:X\to\Bbb R$ is resolvable iff $\{x:\alpha<f(x)<\beta\}$ is
F$_{\sigma}$ for all $\alpha$, $\beta\in\Bbb R$.
%562Q

\spheader 562Ye Let $X$ be a topological space.   Let $\Phi$ be the set of
functions $f:X\to\Cal P\Bbb N$ such that $\{x:n\in f(x)\}$ is open for
every $n\in\Bbb N$.   Write $\Cal B'_c(X)$ for
$\{f^{-1}[F]:f\in\Phi$,  $F\in\Cal B_c(\Cal P\Bbb N)\}$;  say a family
$\familyiI{E_i}$ in $\Cal B'_c(X)$ is codable if there is a family
$\familyiI{(f_i,F_i)}$ in $\Phi\times\Cal B_c(\Cal P\Bbb N)$ such that
$\familyiI{F_i}$ is codable and $E_i=f_i^{-1}[F_i]$ for every $i$.   (i)
Show that if $X$ is second-countable then $\Cal B'_c(X)=\Cal B_c(X)$ and
the codable families on the definition here coincide with the codable
families of 562J.   (ii) Develop a theory of codable Borel sets and
functions corresponding to that in 562T.
%562T
}%end of exercises


\endnotes{
\Notesheader{562} The idea of `Borel code' is of great importance in
mathematical logic, for reasons quite separate from the questions
addressed here;
see {\smc Jech 78}, {\smc Jech 03} or {\smc Kunen 80}.   (Of course it is
not a coincidence that an approach which is effective in the absence of the
axiom of choice should also be relevant to absoluteness in the presence of
choice.)   Every author has his favoured formula corresponding to that in
562Ba.   The particular one
I have chosen is intended to be economical and direct, but is
slightly awkward at the initial stages, and some proofs demand an extra
moment's attention to the special case of trees of rank $1$.   The real
motivation for the calculations here will have to wait for \S565;  Lebesgue
measure can be defined in such a way that it is countably additive with
respect to {\it codable} sequences of Borel sets, and there are enough of
these to make the theory non-trivial.

Borel codes are wildly non-unique, which is why the concept of codable
family is worth defining.   But it is also important that certain sets,
starting with the open sets, are self-coding in the sense that from the set
we can pick out an appropriate code.   `Resolvable' sets and functions
(562G, 562Q) are
common enough to be very useful, and for these we can work with the objects
themselves, just as we always have, and leave the coding until we need it.

The Borel codes described here can be used only in second-countable
spaces.   It is easy enough to find variations of the concept
which can be applied
in more general contexts (562Ye), though it is not obvious that there are
useful
theorems to be got in such a way.   More relevant to the work of the next
few sections is the idea of `codable Baire set' (562T).   Because any
codable sequence of codable Baire sets can be factored through a single
continuous function to $\BbbR^{\Bbb N}$ (562T(b-i)),
we have easy paths to the elementary results given here.

}%end of notes

\discrpage

