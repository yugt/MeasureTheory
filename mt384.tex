\frfilename{mt384.tex}
\versiondate{29.8.03}
\copyrightdate{1994}
     
\def\chaptername{Automorphisms}
\def\sectionname{Outer automorphisms}
\def\cycleii#1#2#3{\cycle{#1\,_{#2}\,#3}}
     
\newsection{384}
     
Continuing with the investigation of the abstract group-theoretic nature
of the automorphism groups $\Aut\frak A$ and $\Aut_{\bar\mu}\frak A$, I
devote a section to some remarkable results concerning isomorphisms
between them.   Under any of a variety of conditions, any isomorphism between two groups $\Aut\frak A$ and $\Aut\frak B$ must correspond to an
isomorphism between the underlying Boolean algebras (384E, 384F, 384J,
384M);  consequently $\Aut\frak A$ has few, or no, outer automorphisms
(384G, 384K, 384O).   I organise the section around a single general
result (384D).
     
\leader{384A}{Lemma} %38{3}A
Let $\frak A$ be a Boolean algebra and $G$ a
subgroup of $\Aut\frak A$ which has many
involutions\cmmnt{ (definition:  382O)}.   Then for every non-zero
$a\in\frak A$ there is an automorphism $\psi\in G$, of order 4, which
is supported by $a$.
     
\proof{ Let $\pi\in G$ be an involution supported by $a$.   Let
$b\Bsubseteq a$ be such that $\pi b\ne b$.   Then at least one of
$b\Bsetminus\pi b$, $\pi b\Bsetminus b=\pi(b\Bsetminus\pi b)$ is
non-zero, so in fact both are.   Let $\phi$ be an involution supported
by $b\Bsetminus\pi b$.   Then $\pi\phi\pi=\pi\phi\pi^{-1}$ is an
involution supported by $\pi b\Bsetminus b$, so commutes with $\phi$,
and the product $\phi\pi\phi\pi$ is an involution.   But this means that
$\psi=\phi\pi$ has order 4, and of course it is supported by $a$ because
$\phi$ and $\pi$ both are.
}%end of proof of 384A
     
\leader{384B}{A note on supports}\cmmnt{ Since %38{3}B
in this section we
shall be looking at more than one automorphism group at a time, I shall
need to call on the following elementary extension of a fact in \S381.}
Let $\frak A$ and $\frak B$ be Boolean algebras, and
$\theta:\frak A\to\frak B$ a Boolean isomorphism.   If
$\pi\in\Aut\frak A$ is
supported by $a\in\frak A$, then $\theta\pi\theta^{-1}\in\Aut\frak B$ is
supported by $\theta a$.   \prooflet{(Use the same argument as in
381Ej.)}   Accordingly, if $a$ is the support of $\pi$ then $\theta a$
will be the support of $\theta\pi\theta^{-1}$\cmmnt{, as in 381Gd}.
     
\leader{384C}{Lemma} %38{3}C
Let $\frak A$ and $\frak B$ be two Boolean
algebras, and $G$ a subgroup of $\Aut\frak A$ with many involutions.
If $\theta_1$, $\theta_2:\frak A\to\frak B$ are distinct isomorphisms,
then there is a $\phi\in G$ such that
$\theta_1\phi\theta_1^{-1}\ne\theta_2\phi\theta_2^{-1}$.
     
\proof{ Because $\theta_1\ne\theta_2$,
$\theta=\theta_2^{-1}\theta_1$ is not the identity automorphism on
$\frak A$, and there is some non-zero $a\in\frak A$ such that
$\theta a\Bcap a=0$.
Let $\pi\in G$ be an involution supported by $a$;   then
$\theta\pi\theta^{-1}$ is supported by $\theta a$, so cannot be equal to
$\pi$, and $\theta_1\pi\theta_1^{-1}\ne\theta_2\pi\theta_2^{-1}$.
}%end of proof of 384C
     
\leader{384D}{Theorem} %38{3}D
Let $\frak A$ and $\frak B$ be Dedekind
complete Boolean algebras and $G$ and $H$ subgroups of $\Aut\frak A$,
$\Aut\frak B$ respectively, both having many involutions.    Let
$q:G\to H$ be an isomorphism.   Then there is a unique
Boolean isomorphism $\theta:\frak A\to\frak B$ such that
$q(\phi)=\theta\phi\theta^{-1}$ for every $\phi\in G$.
     
\proof{{\bf (a)}  The first half of the proof is devoted to setting up
some structures in the group $G$.   Let $\pi\in G$ be any involution.
Set
     
\Centerline{$C_{\pi}=\{\phi:\phi\in G,\,\phi\pi=\pi\phi\}$,}
     
\noindent the centralizer of $\pi$ in $G$;
     
\Centerline{$U_{\pi}=\{\phi:\phi\in C_{\pi}$, $\phi=\phi^{-1}$,
$\phi\psi\phi\psi^{-1}
=\psi\phi\psi^{-1}\phi$ for every $\psi\in C_{\pi}\}$,}
     
\noindent the set of involutions in $C_{\pi}$ commuting with all their
conjugates in $C_{\pi}$, together with the identity,
     
\Centerline{$V_{\pi}=\{\phi:\phi\in G,\,\phi\psi=\psi\phi$ for every
$\psi\in U_{\pi}\}$,}
     
\noindent the centralizer of $U_{\pi}$ in $G$,
     
\Centerline{$S_{\pi}=\{\phi^2:\phi\in V_{\pi}\}$,}
     
\Centerline{$W_{\pi}=\{\phi:\phi\in G,\,\phi\psi=\psi\phi$ for every
$\psi\in S_{\pi}\}$,}
     
\noindent the centralizer of $S_{\pi}$ in $G$.
     
\medskip
     
{\bf (b)} The point of this list is to provide a purely group-theoretic
construction corresponding to the support of $\pi$ in $\frak A$.
In the next few paragraphs of the proof (down to (f)), I set out to
describe the objects just introduced in terms of their action on
$\frak A$.   First, note that $\pi$ is an exchanging involution (382Fa);
express it as $\cycleii{a'}{\pi}{a''}$, so that the support of $\pi$ is
$a_{\pi}=a'\Bcup a''$.
     
\medskip
     
{\bf (c)} I start with two elementary properties of $C_{\pi}$:
     
\medskip
     
\quad{\bf (i)}
$\phi(a_{\pi})=a_{\pi}$ for every $\phi\in C_{\pi}$.   \Prf\ As remarked
in 381Gd, the support of $\pi=\phi\pi\phi^{-1}$ is $\phi(a_{\pi})$, so
this must be $a_{\pi}$.\ \Qed\
     
\medskip
     
\quad{\bf (ii)} If $\phi\in C_{\pi}$ and $\phi$ is not
supported by $a_{\pi}$, there is a non-zero
$d\Bsubseteq 1\Bsetminus a_{\pi}$ such that $d\Bcap\phi d=0$, by 381Ei.
     
\medskip
     
{\bf (d)} Now for the properties of $U_{\pi}$:
     
\medskip
     
\quad{\bf (i)} If $\phi\in U_{\pi}$, then $\phi$ is supported by
$a_{\pi}$.
     
\quad\Prf\grheada\Quer\ Suppose first that there is a 
$d\Bsubseteq 1\Bsetminus a_{\pi}$ such that 
$d\Bcap(\phi d\Bcup\phi^2d)=0$.   Let
$\psi\in G$ be an involution supported by $d$.   Then
$\supp\psi\Bcap\supp\pi=0$, so
$\psi\in C_{\pi}$.   There is a $c\Bsubseteq d$ such that $\psi c\ne c$,
so
     
\Centerline{$\psi\phi\psi^{-1}\phi c=\psi\phi^2c=\phi^2c$}
     
\noindent because $d\Bcap(\phi c\Bcup\phi^2c)=0$, while
     
\Centerline{$\phi\psi\phi\psi^{-1}c=\phi^2\psi^{-1}c$}
     
\noindent because $d\Bcap\phi\psi^{-1}c=0$;  but this means that
$\psi\phi\psi^{-1}\phi c\ne\phi\psi\phi\psi^{-1}c$, so $\phi$ and
$\psi\phi\psi^{-1}$ do not commute, and $\phi\notin U_{\pi}$.\ \Bang
     
\qquad\grheadb\Quer\ Suppose that $\phi^2$ is not supported by
$a_{\pi}$.   Then, as remarked in (c-ii), there is a non-zero $d\Bsubseteq 1\Bsetminus a_{\pi}$ such that $\phi^2d\Bcap d=0$.   Now $d\notBsubseteq\phi^2d$, so $d\notBsubseteq\phi d$;  set $d'=d\Bsetminus\phi d$.   Then
$d'\Bcap\phi d'=d'\Bcap\phi^2d'=0$ and 
$0\ne d'\Bsubseteq 1\Bsetminus a_{\pi}$;  but this is impossible, by ($\alpha$).\ \Bang
     
\qquad\grheadc\ Thus $\phi^2d=d$ for every 
$d\Bsubseteq 1\Bsetminus a_{\pi}$.   \Quer\ Suppose, if possible, that $\phi$ is not supported by
$a_{\pi}$.   Then there is a non-zero $d\Bsubseteq 1\Bsetminus a_{\pi}$
such that $\phi d\Bcap d=0$.   By 384A, there is a $\psi\in G$, of order
$4$, supported by $d$.   Because $d\Bcap a_{\pi}=0$, $\psi\in C_{\pi}$.
Because $\psi\ne\psi^{-1}$, there is a $c\Bsubseteq d$ such that $\psi
c\ne\psi^{-1}c$;  but now $\phi c\Bcap d=\phi\psi^{-1}c\Bcap d=0$, so
     
\Centerline{$\psi\phi\psi^{-1}\phi c=\psi\phi^2c=\psi c
\ne\psi^{-1}c
=\phi^2\psi^{-1}c
=\phi\psi\phi\psi^{-1}c$,}
     
\noindent and $\phi$ does not commute with its conjugate
$\psi\phi\psi^{-1}$, contradicting the assumption that $\phi\in
U_{\pi}$.\ \Bang
     
So $\phi$ is supported by $a_{\pi}$, as claimed.\ \Qed
     
\medskip
     
\quad{\bf (ii)} If $u\in\frak A$ and $\pi u=u$, then $\pi_u\in
U_{\pi}$, where
     
\Centerline{$\pi_u d=\pi d$ if $d\Bsubseteq u$,
\quad $\pi_ud=d$ if $d\Bcap u=0$,}
     
\noindent that is, $\pi_u=\cycleii{a'\Bcap u}{\pi}{a''\Bcap u}$.   \Prf\
For any $\psi\in\Aut\frak A$,
     
\Centerline{$\psi\pi_u\psi^{-1}
=\cycleii{\psi(a'\Bcap u)}{\psi\pi\psi^{-1}}{\psi(a''\Bcap u)}$}
     
\noindent (381Sb).   ($\alpha$)  Accordingly
     
\Centerline{$\pi\pi_u\pi^{-1}
=\cycleii{a''\Bcap u}{\pi}{a'\Bcap u}=\pi_u$}
     
\noindent and $\pi_u\in C_{\pi}$.   ($\beta$) If $\psi\in C_{\pi}$, then
     
\Centerline{$\pi
=\psi\pi\psi^{-1}
=\cycleii{\psi a'}{\psi\pi\psi^{-1}}{\psi a''}
=\cycleii{\psi a'}{\pi}{\psi a''}$.}
     
\noindent So
     
\Centerline{$\psi\pi_u\psi^{-1}
=\cycleii{\psi(a'\Bcap u)}{\psi\pi\psi^{-1}}{\psi(a''\Bcap u)}
=\cycleii{\psi a'\Bcap \psi u}{\pi}{\psi a''\Bcap\psi u}
=\pi_{\psi u}$.}
     
\noindent Now if $\pi v=v$ then $\pi_u\pi_v=\pi_{u\Bsymmdiff v}
=\pi_v\pi_u$;  in particular, $\pi_{\psi u}\pi_u=\pi_u\pi_{\psi u}$.
As
$\psi$ is arbitrary, $\pi_u\in U_{\pi}$.\ \Qed
     
In particular, of course, $\pi=\pi_1$ belongs to $U_{\pi}$.
     
\medskip
     
{\bf (e)} The two parts of (d) lead directly to the properties we need
of $V_{\pi}$.
     
\medskip
     
\quad{\bf (i)} $V_{\pi}\subseteq C_{\pi}$, because $\pi\in U_{\pi}$.
Consequently $\phi a_{\pi}=a_{\pi}$ for every $\phi\in V_{\pi}$.
     
\medskip
     
\quad{\bf (ii)} If $\phi\in V_{\pi}$ then $\phi d\Bsubseteq d\Bcup\pi d$
for every $d\Bsubseteq a_{\pi}$.   \Prf\Quer\ Suppose, if possible,
otherwise.   Set $u_0=d\Bcup\pi d$, so that $\pi u_0=u_0$, and
$u=\phi u_0\Bsetminus u_0\ne 0$;  also $u\Bsubseteq\phi
a_{\pi}=a_{\pi}$.   Since $\pi\phi u_0=\phi\pi u_0=\phi u_0$, $\pi u=u$.
Set $v=u\Bcap a'$, so that $u=v\Bcup\pi v$ and $v\ne\pi v$.   Because
$u\Bcap\phi v\Bsubseteq\phi(u_0\Bcap u)=0$,
     
\Centerline{$\pi_u\phi v=\phi v\ne\phi\pi v=\phi\pi_uv$,}
     
\noindent which is impossible.\ \Bang\Qed
     
\medskip
     
\quad{\bf (iii)} It follows that $\phi^2d=d$ whenever $\phi\in V_{\pi}$
and $d\Bsubseteq a_{\pi}$.   \Prf\ Let $e$ be the support of $\phi$.
Recall that $e=\sup\{c:c\Bcap\phi c=0\}$ (381Gb), so that
$d\Bcap e=\sup\{c:c\Bsubseteq d,\,c\Bcap\phi c=0\}$.   Now if
$c\Bsubseteq a_{\pi}$ and $c\Bcap\phi c=0$, we know that
$\phi c\Bsubseteq c\Bcup\pi c$, so in fact $\phi c\Bsubseteq\pi c$.
This shows that $\phi(d\Bcap e)\Bsubseteq\pi(d\Bcap e)$.   Also, because
$\pi\phi=\phi\pi$, by (i), we have
     
\Centerline{$\phi^2(d\Bcap e)\Bsubseteq\phi\pi(d\Bcap e)
=\pi\phi(d\Bcap e)\Bsubseteq\pi^2(d\Bcap e)=d\Bcap e$.}
     
\noindent Of course $\phi^2(d\Bsetminus e)=d\Bsetminus e$, so
$\phi^2d\Bsubseteq d$.   This is true for every $d\Bsubseteq a_{\pi}$.
But as also $\phi^2a_{\pi}=\phi a_{\pi}=a_{\pi}$, $\phi^2d=d$ for every
$d\Bsubseteq a_{\pi}$.\ \Qed
     
\medskip
     
\quad{\bf (iv)} The final thing we need to know about $V_{\pi}$ is that
$\phi\in V_{\pi}$ whenever $\phi\in G$ and $\supp\phi\Bcap a_{\pi}=0$;
this is immediate from (d-i) above.
     
\medskip
     
{\bf (f)} From (e-iii), we see that if $\phi\in S_{\pi}$ then
$\supp\phi\Bcap a_{\pi}=0$.   But we also see from (e-iv) that if $0\ne
c\Bsubseteq
1\Bsetminus a_{\pi}$ there is an involution in $S_{\pi}$ supported by
$c$;  for there is a member $\psi$ of $G$, of order 4, supported by $c$,
and now $\psi\in V_{\pi}$ so $\psi^2\in S_{\pi}$, while $\psi^2$ is an
involution.
     
\medskip
     
{\bf (g)} Consequently,
$W_{\pi}$ is just the set of members of $G$ supported by $a_{\pi}$.
\Prf\ (i) If $\supp\phi\Bsubseteq a_{\pi}$ and $\psi\in S_{\pi}$, then
$\supp\psi\Bcap a_{\pi}=0$, as noted in (e), so $\phi\psi=\psi\phi$;  as
$\psi$ is arbitrary, $\phi\in W_{\pi}$.   (ii) If
$\supp\phi\notBsubseteq a_{\pi}$, then take a non-zero $d\Bsubseteq
1\Bsetminus a_{\pi}$ such that $\phi d\Bcap d=0$.  Let $\psi\in S_{\pi}$
be an involution supported by $d$;  then if $c\Bsubseteq
d$ is such that $\psi c\ne c$,
     
\Centerline{$\phi\psi c\ne\phi c=\psi\phi c$,}
     
\noindent and $\phi\psi\ne\psi\phi$ so $\phi\notin W_{\pi}$.\ \Qed
     
\medskip
     
{\bf (h)} We can now return to consider the isomorphism $q:G\to H$.
If $\pi\in G$ is an involution, then $q(\pi)\in H$ is an involution, and
it is easy to check that
     
\Centerline{$q[C_{\pi}]=C_{q(\pi)}$,}
\Centerline{$q[U_{\pi}]=U_{q(\pi)}$,}
     
\Centerline{$q[V_{\pi}]=V_{q(\pi)}$,}
     
\Centerline{$q[S_{\pi}]=S_{q(\pi)}$,}
     
\Centerline{$q[W_{\pi}]=W_{q(\pi)}$,}
     
\noindent defining $C_{q(\pi)},\ldots,W_{q(\pi)}\subseteq H$ as in (a)
above.
So we see that, for any $\phi\in G$,
     
\Centerline{$\supp\phi\Bsubseteq\supp\pi\,\iff\,
\supp q(\phi)\Bsubseteq\supp q(\pi)$.}
     
\medskip
     
{\bf (i)} Define $\theta:\frak A\to\frak B$ by writing
     
\Centerline{$\theta a=\sup\{\supp q(\pi):\pi\in G$ is an involution and
$\supp\pi\Bsubseteq a\}$}
     
\noindent for every $a\in\frak A$.   Evidently $\theta$ is
order-preserving.   Now if $a\in\frak A$, $\pi\in G$ is an involution
and $\supp\pi\notBsubseteq a$, $\supp q(\pi)\notBsubseteq\theta a$.
\Prf\    There is a  $\phi\in G$, of order 4, supported by
$\supp\pi\Bsetminus a$.  Now $\phi^2$ is an involution supported by
$\supp\pi$, so $\supp q(\phi^2)\Bsubseteq\discretionary{}{}{}\supp q(\pi)$.
On the other
hand, if $\pi'\in G$ is an involution supported by $a$, then 
$\phi\in V_{\pi'}$ and $\phi^2\in S_{\pi'}$, so 
$q(\phi^2)\in S_{q(\pi')}$ and
$\supp q(\phi^2)\Bcap\supp q(\pi')=0$.   As $\pi'$ is arbitrary, $\supp
q(\phi^2)\Bcap\theta a=0$;  so
     
\Centerline{$\supp q(\pi)\Bsetminus\theta a\Bsupseteq\supp q(\phi^2)\ne
0$.   \Qed}
     
\wheader{384D}{4}{2}{2}{24pt}
     
{\bf (j)} In the same way, we can define $\theta^*:\frak B\to\frak A$ by
setting
     
\Centerline{$\theta^*b=\sup\{\supp q^{-1}(\pi):\pi\in H$ is an
involution and $\supp\pi\Bsubseteq b\}$}
     
\noindent for every $b\in\frak B$.   Now $\theta^*\theta a=a$ for every
$a\in\frak A$.   \Prf\ ($\alpha$) If $0\ne u\Bsubseteq
a$, there is an involution $\pi\in G$ supported by $u$.   Now $q(\pi)$
is an involution in $H$ supported by $\theta a$, so
     
\Centerline{$u\Bcap\theta^*\theta a
\Bsupseteq u\Bcap\supp q^{-1}q(\pi)=\supp \pi\ne 0$.}
     
\noindent As $u$ is arbitrary, $a\Bsubseteq\theta^*\theta a$.
($\beta$) If
$\pi\in H$ is an involution supported by $\theta a$, then
$\phi= q^{-1}(\pi)$ is an involution in $G$ with $\supp
q(\phi)=\supp\pi\Bsubseteq\theta a$, so $\supp\phi\Bsubseteq a$, by
(i) above;  as $\pi$ is arbitrary, $\theta^*\theta a\Bsubseteq a$.\
\Qed\
     
Similarly, $\theta\theta^*b=b$ for every $b\in\frak B$.   But this means
that $\theta$ and $\theta^*$ are the two halves of an
order-isomorphism between $\frak A$ and $\frak B$.   By 312M, both are
Boolean homomorphisms.
     
\medskip
     
{\bf (k)} If $\pi\in G$ is an involution, then 
$\theta(\supp\pi)=\supp q(\pi)$.   \Prf\ By the definition of $\theta$, $\supp q(\pi)\Bsubseteq\theta(\supp\pi)$.   On the other hand,
     
\Centerline{$\supp q(\pi)
=\theta\theta^*(\supp q(\pi))
\Bsupseteq\theta(\supp q^{-1} q(\pi))
=\theta(\supp\pi)$.  \Qed}
     
Similarly, if $\pi\in H$ is an involution, 
$\theta^{-1}(\supp \pi)=\theta^*(\supp\pi)=\supp q^{-1}(\pi)$.
     
\medskip
     
{\bf (l)}  We are nearly home.  Let us confirm that
$q(\phi)=\theta\phi\theta^{-1}$ for every $\phi\in G$.   \Prf\Quer\
Otherwise, $\psi=q(\phi)^{-1}\theta\phi\theta^{-1}$ is
not the identity automorphism on $\frak B$, and there is a non-zero
$b\in\frak B$ such that $\psi b\Bcap b=0$, that is,
$\theta\phi\theta^{-1}b\Bcap q(\phi)b=0$.   Let $\pi\in H$ be an
involution supported by $b$.   Then $q^{-1}(\pi)$ is supported by
$\theta^{-1}b$, by (j), so $\phi\theta^{-1}b$ supports
$\phi q^{-1}(\pi)\phi^{-1}$ and $\theta\phi\theta^{-1}b$ supports
$q(\phi q^{-1}(\pi)\phi^{-1})=q(\phi)\pi q(\phi)^{-1}$.   On the other
hand, $q(\phi)b$ also supports $q(\phi)\pi q(\phi)^{-1}$, which is not
the identity automorphism;  so these two elements of $\frak B$ cannot be
disjoint.\ \Bang\Qed
     
\medskip
     
{\bf (m)} Finally, $\theta$ is unique by 384C.
}%end of proof of 384D
     
\cmmnt{\medskip
     
\noindent{\bf Remark} The ideas of the proof here are taken from
{\smc Eigen 82}.
}
     
\leader{384E}{}\cmmnt{ The %38{3}E
rest of this section may be regarded as a series of
corollaries of this theorem.   But I think it will be apparent that they
are very substantial results.
     
\medskip
     
\noindent}{\bf Theorem} Let $\frak A$ and $\frak B$ be atomless
homogeneous Boolean algebras, and $q:\Aut\frak A\to\Aut\frak B$ an
isomorphism.   Then there is a unique Boolean isomorphism
$\theta:\frak A\to\frak B$ such that $q(\phi)=\theta\phi\theta^{-1}$ for
every $\phi\in\Aut\frak A$.
     
\proof{{\bf (a)} Let $\widehat{\frak A}$ be the Dedekind completion of
$\frak A$ (314U).   Then every $\phi\in\Aut\frak A$ has a unique
extension to a Boolean homomorphism
$\hat\phi:\widehat{\frak A}\to\widehat{\frak A}$ (314Tb).   Because the
extension is unique, we
must have $(\phi\psi)\sphat=\hat\phi\hat\psi$ for all $\phi$,
$\psi\in\Aut\frak A$;  consequently, $\hat\phi$ and
$\widehat{\phi^{-1}}$
are inverses of each other, and $\hat\phi\in\Aut\widehat{\frak A}$ for
each $\phi\in\Aut\frak A$;  moreover, $\phi\mapsto\hat\phi$ is a group
homomorphism.   Of course it is injective, so we have a subgroup
$G=\{\hat\phi:\phi\in\Aut\frak A\}$ of $\Aut\widehat{\frak A}$ which is
isomorphic to $\Aut\frak A$.   Clearly
     
\Centerline{$G=\{\phi:\phi\in\Aut\widehat{\frak A},\,\phi u\in\frak A$
for every $u\in\frak A\}$.}
     
\noindent If $a\in\widehat{\frak A}$ is non-zero, then there is a
non-zero $u\Bsubseteq a$ belonging to $\frak A$.   Because $\frak A$ is
atomless and homogeneous, there is an involution $\pi\in\Aut\frak A$
supported by $u$ (382P);  now $\hat\pi\in G$ is an involution supported by $a$.   As $a$ is arbitrary, $G$ has many involutions.
     
Similarly, writing $\widehat{\frak B}$ for the Dedekind completion of
$\frak B$, we have a subgroup $H=\{\hat\psi:\psi\in\Aut\frak B\}$ of
$\Aut\widehat{\frak B}$ isomorphic to $\Aut\frak B$, and with many
involutions.   Let $\hat q:G\to H$ be the corresponding isomorphism, so
that $\hat q(\hat\phi)=\widehat{q(\phi)}$ for every
$\phi\in\Aut\frak A$.
     
By 384D, there is a Boolean isomorphism
$\hat\theta:\widehat{\frak A}\to\widehat{\frak B}$ such that 
$\hat q(\phi)=\hat\theta\phi\hat\theta^{-1}$ for every $\phi\in G$.
Note that 

\Centerline{$\hat\theta(\supp\phi)
=\supp(\hat\theta\phi\hat\theta^{-1})=\supp\hat q(\phi)$}

\noindent for every $\phi\in G$, so that 
$\hat\theta(\supp\hat q^{-1}(\pi))=\supp\pi$ for every $\pi\in H$.      

\medskip
     
{\bf (b)} If $u\in\frak A$, then $\hat\theta u\in\frak B$.   \Prf\ It is
enough to consider the case $u\notin\{0,1\}$, since surely
$\hat\theta 0=0$ and $\hat\theta 1=1$.   
Take any $w\in\frak B$ which is neither $0$
nor $1$;  then there is an involution in $\Aut\frak B$ with support $w$
(382P again);  the corresponding member $\pi$ of $H$ is still an involution with support $w$.   Its
image $\hat q^{-1}(\pi)$ in $G$ is an involution with support
$a=\hat\theta^{-1}w\in\widehat{\frak A}$;
of course $0\ne a\ne 1$.   Take non-zero $u_1$, $u_3\in\frak A$ such
that $u_1\Bsubset a$ and $u_3\Bsubseteq 1\Bsetminus a$;  set
$u_2=1\Bsetminus(u_1\Bcup u_3)$.   Because $\frak A$ is homogeneous,
there are $\phi$, $\psi\in G$ such that $\phi u_1=u$, $\psi u_1=u_1$,
$\psi u_2=u_3$;  set $\phi_2=\phi\psi$.   Then we have
     
\Centerline{$u=\phi u_1\Bsubseteq \phi(\supp\hat q^{-1}(\pi))
=\supp(\phi\hat q^{-1}(\pi)\phi^{-1})
\Bsubseteq \phi(u_1\Bcup u_2)=u\Bcup\phi u_2$,}
     
\Centerline{$u=\phi_2u_1\Bsubseteq \phi_2(\supp\hat q^{-1}(\pi))
=\supp(\phi_2\hat q^{-1}(\pi)\phi_2^{-1})
\Bsubseteq u\Bcup\phi_2u_2= u\Bcup\phi u_3$,}
     
\noindent so
     
\Centerline{$\phi(\supp\hat q^{-1}(\pi))
\Bcap\phi_2(\supp\hat q^{-1}(\pi))=u$,}
     
\noindent and
     
$$\eqalignno{\hat\theta u
&=\hat\theta(\phi(\supp\hat q^{-1}(\pi)))
  \Bcap\hat\theta(\phi_2(\supp\hat q^{-1}(\pi)))\cr
&=\hat\theta(\supp\phi\hat q^{-1}(\pi)\phi^{-1})
  \Bcap\hat\theta(\supp\phi_2\hat q^{-1}(\pi)\phi_2^{-1})\cr
&=\hat\theta(\supp\hat q^{-1}(\hat q(\phi)\pi\hat q(\phi)^{-1}))
  \Bcap\hat\theta(\supp\hat q^{-1}(\hat q(\phi_2)\pi
  \hat q(\phi_2)^{-1}))\cr
&=\supp(\hat q(\phi)\pi\hat q(\phi)^{-1})
  \Bcap\supp(\hat q(\phi_2)\pi\hat q(\phi_2)^{-1})\cr
\displaycause{see the last sentence of (a) above}
&=\hat q(\phi)(\supp\pi)
  \Bcap\hat q(\phi_2)(\supp\pi)
=\hat q(\phi)w\Bcap\hat q(\phi_2)w
\in\frak B\cr}$$
     
\noindent because both $\hat q(\phi)$ and $\hat q(\phi_2)$ belong to
$H$.\ \Qed
     
Similarly, $\hat\theta^{-1}v\in\frak A$ for every $v\in\frak B$, and
$\theta=\hat\theta\restrp\frak{A}$ is an isomorphism between
$\frak A$ and $\frak B$.
     
We now have
     
\Centerline{$q(\phi)=\hat q(\hat\phi)\restrp\frak{B}
=(\hat\theta\hat\phi\hat\theta^{-1})\restrp\frak{B}
=\theta\phi\theta^{-1}$}
     
\noindent for every $\phi\in\Aut\frak A$.   Finally, $\theta$ is unique
by 384C, as before.
}%end of proof of 384E
     
\leader{384F}{Corollary} %38{3}F
If $\frak A$ and $\frak B$ are atomless homogeneous
Boolean algebras with isomorphic automorphism groups, they are
isomorphic as Boolean algebras.
     
\cmmnt{\medskip
     
\noindent{\bf Remark} Of course a one-element Boolean algebra $\{0\}$
and a two-element Boolean algebra $\{0,1\}$ have isomorphic automorphism
groups without being isomorphic.
}

\leaveitout{384E}
     
\leader{384G}{Corollary} %38{3}G
If $\frak A$ is a homogeneous Boolean algebra, 
then $\Aut\frak A$ has no outer automorphisms.
     
\proof{If $\frak A=\{0,1\}$ this is trivial.   Otherwise, $\frak A$ is
atomless, so if $q$ is any automorphism of $\Aut\frak A$, there is a
Boolean isomorphism $\theta:\frak A\to\frak A$ such that
$q(\phi)=\theta\phi\theta^{-1}$ for every $\phi\in\Aut\frak A$, and $q$
is an inner automorphism.
}%end of proof of 384G
     
\leader{384H}{Definitions}\cmmnt{ Complementary %38{3}H
to the notion of `many involutions' is the following concept.
     
\medskip
     
}{\bf (a)} A Boolean algebra $\frak A$ is {\bf rigid} if
the only automorphism of $\frak A$ is the identity automorphism.
     
\medskip
     
{\bf (b)} A Boolean algebra $\frak A$ is {\bf nowhere rigid} if no
non-trivial principal ideal of $\frak A$ is rigid.
     
\leader{384I}{Lemma} %38{3}I
Let $\frak A$ be a Boolean algebra.    Then the
following are equiveridical:
     
(i) $\frak A$ is nowhere rigid;
     
(ii) for every $a\in\frak A\setminus\{0\}$ there is a 
$\phi\in\Aut\frak A$, not the identity, supported by $a$;
     
(iii) for every $a\in\frak A\setminus\{0\}$ there are distinct 
$b$, $c\Bsubseteq a$ such that the principal ideals $\frak A_b$, 
$\frak A_c$ they generate are isomorphic;
     
(iv) the automorphism group $\Aut\frak A$ has many involutions.
     
\proof{{\bf (a)(ii)$\Rightarrow$(i)} If $a\in\frak A\setminus\{0\}$, let
$\phi\in\Aut\frak A$ be a non-trivial automophism supported by $a$;
then $\phi\restrp\frak{A}_a$ is a
non-trivial automorphism of the principal ideal $\frak A_a$, so $\frak
A_a$ is not rigid.
     
\medskip
     
{\bf (b)(i)$\Rightarrow$(iii)} There is a non-trivial automorphism
$\psi$ of $\frak A_a$;  now if $b\in\frak A_a$ is such that 
$\psi b=c\ne b$, $\frak A_b$ is isomorphic to 
$\psi[\frak A_b]=\frak A_c$.
     
\medskip
     
{\bf (c)(iii)$\Rightarrow$(iv)}  Take any non-zero $a\in\frak A$.   By
(iii), there are distinct $b$, $c\Bsubseteq a$ such that $\frak A_b$,
$\frak A_c$ are isomorphic.   At least one of $b\Bsetminus c$,
$c\Bsetminus b$ is non-zero;  suppose the former.   Let $\psi:\frak
A_b\to\frak A_c$ be an isomorphism, and set $d=b\Bsetminus c$,
$d'=\psi(b\Bsetminus c)$;  then $d'\Bsubseteq c$, so $d'\Bcap d=0$, and
$\phi=\cycleii{d}{\psi}{d'}$ is an involution supported by $a$.
     
\medskip
     
{\bf (d)(iv)$\Rightarrow$(ii)} is trivial.
}%end of proof of 384I
     
\leader{384J}{Theorem} %38{3}J
Let $\frak A$ and $\frak B$ be nowhere rigid Dedekind
complete Boolean algebras and $q:\Aut\frak A\to\Aut\frak B$ an
isomorphism.   Then there is a unique Boolean isomorphism 
$\theta:\frak A\to\frak B$ such that $q(\phi)=\theta\phi\theta^{-1}$ for every $\phi\in\Aut\frak A$.
     
\proof{ Put 384I(i)$\Rightarrow$(iv) and 384D together.
}%end of proof of 384J
     
\leader{384K}{Corollary} %38{3}K
Let $\frak A$ be a nowhere rigid Dedekind complete
Boolean algebra.   Then $\Aut\frak A$ has no outer automorphisms.
     
\leaveitout{384J}
     
\leader{384L}{Examples}\cmmnt{ I %38{3}L
note the following examples of nowhere rigid algebras.
     
\medskip
     
}(a) A non-trivial homogeneous Boolean algebra is nowhere rigid.
     
(b) Any principal ideal of a nowhere rigid Boolean algebra is nowhere
rigid.
     
(c) A simple product of nowhere rigid Boolean algebras is nowhere rigid.
     
(d) Any atomless semi-finite measure algebra is nowhere rigid.
     
(e) A free product of nowhere rigid Boolean algebras is nowhere rigid.
     
(f) The Dedekind completion of a nowhere rigid Boolean algebra is
nowhere rigid.
     
\cmmnt{\medskip
     
\noindent Indeed, the difficulty is to find an atomless Boolean algebra which is
not nowhere rigid;  for a variety of constructions of rigid algebras,
see {\smc Bekkali \& Bonnet 89}.
}
     
\leader{384M}{Theorem} %38{3}M
Let $(\frak A,\bar\mu)$ and $(\frak B,\bar\nu)$ be atomless
localizable measure algebras, and $\Aut_{\bar\mu}\frak A$,
$\Aut_{\bar\nu}\frak B$ the corresponding groups of measure-preserving 
automorphisms.
Let $q:\Aut_{\bar\mu}\frak A\to\Aut_{\bar\nu}\frak B$ be an isomorphism.
Then there is a unique Boolean isomorphism $\theta:\frak A\to\frak B$ such that $q(\phi)=\theta\phi\theta^{-1}$ for every
$\phi\in\Aut_{\bar\mu}\frak A$.
     
\proof{ The point is just that $\Aut_{\bar\mu}\frak A$ has many
involutions.   \Prf\ Let $a\in\frak A\setminus\{0\}$.   Then
there is a non-zero $b\Bsubseteq a$ such that the principal ideal
$\frak A_b$ is \Mth.   Take $c\Bsubseteq b$ and
$d\Bsubseteq b\Bsetminus c$ such that
$\bar\mu c=\bar\mu d=\min(1,\bover12\bar\mu b)$ (331C).   The principal
ideals $\frak A_c$,
$\frak A_d$ are now isomorphic as measure algebras (331I);  let
$\psi:\frak A_c\to\frak A_d$ be a measure-preserving isomorphism.   Then
$\cycleii{c}{\psi}{d}\in\Aut_{\bar\mu}\frak A$ is an involution
supported by $a$.\ \Qed
     
Similarly, $\Aut_{\bar\nu}\frak B$ has many involutions, and the result
follows at once from 384D.
}%end of proof of 384M
     
\leader{384N}{}\cmmnt{ To %38{3}N
make proper use of the last theorem we need the following result.
     
\medskip
     
\noindent}{\bf Proposition} Let $(\frak A,\bar\mu)$ and
$(\frak B,\bar\nu)$ be
localizable measure algebras and $\theta:\frak A\to\frak B$ a Boolean
isomorphism.   For each infinite cardinal $\kappa$ let $e_{\kappa}$ be
the Maharam-type-$\kappa$ component of $\frak A$\cmmnt{ (332Gb)} and
for each $\gamma\in\ooint{0,\infty}$ let $A_{\gamma}$ be the set of atoms 
of $\frak A$ of measure $\gamma$.   Then the following are equiveridical:
     
(i) for every $\phi\in\Aut_{\bar\mu}\frak A$,
$\theta\phi\theta^{-1}\in\Aut_{\bar\nu}\frak B$;
     
(ii)($\alpha$) for every infinite cardinal $\kappa$ there is an
$\alpha_{\kappa}>0$ such that
$\bar\nu(\theta a)=\alpha_{\kappa}\bar\mu a$ for
every $a\Bsubseteq e_{\kappa}$, 

\quad($\beta$) for every
$\gamma\in\ooint{0,\infty}$ there is an $\alpha_{\gamma}>0$ such that
$\bar\nu(\theta a)=\alpha_{\gamma}\bar\mu a$ for every
$a\in A_{\gamma}$.
     
\proof{{\bf (a)(i)$\Rightarrow$(ii)}\grheada\ Let $\kappa$ be an
infinite cardinal.   The point is that if $a$, $a'\Bsubseteq e_{\kappa}$
and $\bar\mu a=\bar\mu a'<\infty$ then
$\bar\nu(\theta a)=\bar\nu(\theta a')$.   \Prf\ The principal ideals
$\frak A_a$, $\frak A_{a'}$ are
isomorphic as measure algebras;  moreover, by 332J, the principal
ideals $\frak A_{e_{\kappa}\Bsetminus a}$,
$\frak A_{e_{\kappa}\Bsetminus a'}$ are isomorphic.   We therefore have
a
$\phi\in\Aut_{\bar\mu}\frak A$ such that $\phi a=a'$.   Consequently
$\psi\theta a=\theta a'$, where
$\psi=\theta\phi\theta^{-1}\in\Aut_{\bar\nu}\frak B$, and
$\bar\nu(\theta a)=\bar\nu(\theta a')$.\ \Qed
     
If $e_{\kappa}=0$ we can take $\alpha_{\kappa}=1$.   Otherwise fix on
some $c_0\Bsubseteq e_{\kappa}$ such that $0<\bar\mu c_0<\infty$;   take
$b\Bsubseteq\theta c_0$ such that $0<\bar\nu b<\infty$, and set
$c=\theta^{-1}b$, $\alpha_{\kappa}=\bar\nu b/\bar\mu c$.
Then we shall have
$\bar\nu(\theta a)=\bar\nu(\theta c)=\alpha_{\kappa}\bar\mu a$ whenever
$a\Bsubseteq e_{\kappa}$ and $\bar\mu a=\bar\mu c$.   But we can
find for
any $n\ge 1$ a partition $c_{n1},\ldots,c_{nn}$ of $c$ into elements of
measure $\bover1n\bar\mu c$;  since $\bar\nu(\theta
c_{ni})=\bar\nu(\theta c_{nj})$ for
all $i$, $j\le n$, we must have
$\bar\nu(\theta c_{ni})=\bover1n\bar\nu(\theta c)=\alpha_{\kappa}\bar\mu
c_{ni}$ for all $i$.  So if $a\Bsubseteq
e_{\kappa}$ and $\bar\mu a=\bover1n\bar\mu c$,
$\bar\nu(\theta a)=\bar\nu(\theta c_{n1})=\alpha_{\kappa}\bar\mu a$.
Now suppose that $a\Bsubseteq e_{\kappa}$ and 
$\bar\mu a=\bover{k}{n}\bar\mu c$ for some $k$, $n\ge 1$;  then
$a$ can be partitioned into $k$ elements of measure $\bover1n\bar\mu c$, 
so in this case also $\bar\nu(\theta a)=\alpha_{\kappa}\bar\mu a$.
Finally, for any $a\Bsubseteq e_{\kappa}$, set
     
\Centerline{$D=\{d:d\Bsubseteq a,\,\bar\mu d$ is a rational multiple of
$\bar\mu c\}$,}
     
\noindent and let $D'\subseteq D$ be a maximal upwards-directed set.
Then $\sup D'=a$, so $\theta[D']$
is an upwards-directed set with supremum $\theta a$, and
     
\Centerline{$\bar\nu(\theta a)=\sup_{d\in D'}\bar\nu(\theta d)
=\sup_{d\in D'}\alpha_{\kappa}\bar\mu d=\alpha_{\kappa}\bar\mu a$.}
     
\medskip
     
\quad\grheadb\ Let $\gamma\in\ooint{0,\infty}$.   If
$A_{\gamma}=\emptyset$ take $\alpha_{\gamma}=1$.   Otherwise, fix on any
$c\in A_{\gamma}$ and set $\alpha_{\gamma}=\bar\nu(\theta c)/\gamma$.
If $a\in A_{\gamma}$ then there is a $\phi\in\Aut_{\bar\mu}\frak A$
exchanging the atoms $a$ and $c$, so that
$\theta\phi\theta^{-1}\in\Aut_{\bar\nu}\frak B$
exchanges the atoms $\theta a$ and $\theta c$, and
     
\Centerline{$\bar\nu(\theta a)=\bar\nu(\theta c)
=\alpha_{\gamma}\bar\mu a$.}
     
\medskip
     
{\bf (b)(ii)$\Rightarrow$(i)} Now suppose that the conditions ($\alpha$)
and ($\beta$) are satisfied, that $\phi\in\Aut_{\bar\mu}\frak A$
and that $a\in\frak A$.   For each infinite cardinal $\kappa$, we have
$\phi e_{\kappa}=e_{\kappa}$, so
     
$$\bar\nu(\theta\phi(e_{\kappa}\Bcap a))
=\alpha_{\kappa}\bar\mu(\phi(e_{\kappa}\Bcap a))
=\alpha_{\kappa}\bar\mu(e_{\kappa}\Bcap a)
=\bar\nu(\theta(e_{\kappa}\Bcap a)).$$
     
\noindent Similarly, if we write $a_{\gamma}=\sup A_{\gamma}$, then for
each $\gamma\in\ooint{0,\infty}$ we have $\phi[A_{\gamma}]=A_{\gamma}$ and $\phi a_{\gamma}=a_{\gamma}$, and for $c\Bsubseteq a_{\gamma}$ we have
     
\Centerline{$\bar\mu c
=\gamma\#(\{e:e\in A_{\gamma},\,e\Bsubseteq c\})$;}
     
\noindent so
     
$$\eqalign{\bar\nu(\theta\phi(a_{\gamma}\Bcap a))
&=\alpha_{\gamma}\gamma\#(\{e:e\in A_{\gamma},\,
e\Bsubseteq\phi a\})\cr
&=\alpha_{\gamma}\gamma\#(\{e:e\in A_{\gamma},\,
e\Bsubseteq a\})\cr
&=\sum_{e\in A_{\gamma},e\Bsubseteq a}\bar\nu(\theta e)
=\bar\nu(\theta(a_{\gamma}\Bcap a)).\cr}$$
     
\noindent Putting these together,

$$\eqalign{\bar\nu(\theta\phi a)
&=\sum_{\kappa\text{ is an infinite cardinal}}
  \bar\nu(\theta\phi(e_{\kappa}\Bcap a))
+\sum_{\gamma\in\ooint{0,\infty}}
  \bar\nu(\theta\phi(a_{\gamma}\Bcap a))\cr
&=\sum_{\kappa\text{ is an infinite cardinal}}
  \bar\nu(\theta(e_{\kappa}\Bcap a))
+\sum_{\gamma\in\ooint{0,\infty}}
  \bar\nu(\theta(a_{\gamma}\Bcap a))
=\bar\nu(\theta a).\cr}$$
     
\noindent But this means that
     
\Centerline{$\bar\nu(\theta\phi\theta^{-1}b)
=\bar\nu(\theta\theta^{-1}b)=\bar\nu b$}
     
\noindent for every $b\in\frak B$, and $\theta\phi\theta^{-1}$ is
measure-preserving, as required by (i).
}%end of proof of 384N
     
\leader{384O}{Corollary} %38{3}O
If $(\frak A,\bar\mu)$ is an atomless totally
finite measure algebra, $\Aut_{\bar\mu}\frak A$ has no outer
automorphisms.
     
\proof{ Let $q:\Aut_{\bar\mu}\frak A\to \Aut_{\bar\mu}\frak A$ be any
automorphism.   By 384M, there is a corresponding $\theta\in\Aut\frak A$
such that $q(\phi)=\theta\phi\theta^{-1}$ for every
$\phi\in\Aut_{\bar\mu}\frak A$.   By 384N, there is for each infinite
cardinal $\kappa$ an $\alpha_{\kappa}>0$ such that $\bar\mu(\theta
a)=\alpha_{\kappa}\bar\mu a$ whenever $a\Bsubseteq e_{\kappa}$, the
Maharam-type-$\kappa$ component of $\frak A$.   But since $\theta
e_{\kappa}=e_{\kappa}$ and $\bar\mu e_{\kappa}<\infty$ for every
$\kappa$,
we must have $\alpha_{\kappa}=1$ whenever $e_{\kappa}\ne 0$;  as $\frak
A$ is atomless,
     
$$\eqalign{\bar\mu(\theta a)
&=\sum_{\kappa\text{ is an infinite cardinal}}\bar\mu(\theta(a\Bcap
e_{\kappa}))\cr
&=\sum_{\kappa\text{ is an infinite
cardinal}}\alpha_{\kappa}\bar\mu(a\Bcap
e_{\kappa})\cr
&=\sum_{\kappa\text{ is an infinite cardinal}}\bar\mu(a\Bcap
e_{\kappa})
=\bar\mu a\cr}$$
     
\noindent for every $a\in\frak A$.   Thus $\theta\in\Aut_{\bar\mu}\frak
A$
and $q$ is an inner automorphism.
}%end of proof of 384O
     
\leader{384P}{}\cmmnt{ The %38{3}P
results above are satisfying and complete in their
own terms, but leave open a number of obvious questions concerning
whether some of the hypotheses can be relaxed.   Atoms can produce a
variety of complications (see 384Ya-384Yd below).   To show that we
really do need to assume that our algebras are Dedekind complete or
localizable, I offer the following.
     
\medskip
     
\noindent}{\bf Example (a)} There are an atomless localizable measure
algebra $(\frak A,\bar\mu)$ and an atomless semi-finite measure algebra
$(\frak B,\bar\nu)$ such that $\Aut\frak A\cong\Aut\frak B$,
$\Aut_{\bar\mu}\frak
A\cong\Aut_{\bar\nu}\frak B$ but $\frak A$ and $\frak B$ are not
isomorphic.
     
\proof{ Let $(\frak A_0,\bar\mu_0)$ be an atomless
homogeneous probability algebra;  for instance, the measure algebra of
Lebesgue measure on the unit interval.   Let $(\frak A,\bar\mu)$ be 
the simple product measure algebra $(\frak A_0,\bar\mu_0)^{\omega_1}$ 
(322L);  then $(\frak A,\bar\mu)$ is an atomless
localizable measure algebra.   In $\frak A$ let $I$ be the set
     
\Centerline{$\{a:a\in\frak A$ and the principal ideal $\frak A_{a}$ is
ccc$\}$;}
     
\noindent then $I$ is an ideal of $\frak A$, the $\sigma$-ideal
generated by the elements of finite measure (cf.\ 322G).   Set
     
\Centerline{$\frak B=\{a:a\in\frak A$, either $a\in I$ or 
$1\Bsetminus a\in I\}$.}
     
\noindent Then $\frak B$ is a $\sigma$-subalgebra of $\frak A$, so if we
set $\bar\nu=\bar\mu\restrp\frak{B}$ then $(\frak B,\bar\nu)$ is
a measure algebra in its own right.
     
The definition of $I$ makes it plain that it is invariant under all
Boolean automorphisms of $\frak A$;  so $\frak B$ also is invariant
under all automorphisms, and we have a homomorphism 
$\phi\mapsto q(\phi)=\phi\restrp\frak{B}:\Aut\frak A\to\Aut\frak B$.   On the other
hand, because $\frak B$ is order-dense in $\frak A$, and $\frak A$ is
Dedekind complete, every automorphism of $\frak B$ can be extended to an
automorphism of $\frak A$ (see part (a) of the proof of 384E).   So $q$
is actually an isomorphism between $\Aut\frak A$ and $\Aut\frak B$.
Moreover, still because $\frak B$ is order-dense, $q(\phi)$ is
measure-preserving iff $\phi$ is measure-preserving, so
$\Aut_{\bar\mu}\frak
A$ is isomorphic to $\Aut_{\bar\nu}\frak B$.   But of course there is no
Boolean isomorphism, let alone a measure algebra isomorphism, between
$\frak A$ and $\frak B$, because $\frak A$ is Dedekind complete while
$\frak B$ is not.
}%end of proof of 384Oa
     
\cmmnt{\medskip
     
\noindent{\bf Remark} Thus the hypothesis `Dedekind complete' in 384D
and 384J (and
`localizable' in 384M), and the hypothesis `homogeneous' in
384E-384F,
are essential.}
     
     
\medskip
     
{\bf (b)} There is an atomless semi-finite measure algebra $(\frak
C,\bar\lambda)$
such that $\Aut\frak C$ has an outer automorphism.
     
\proof{ In fact we can take $\frak C$ to be the simple product of $\frak
A$ and $\frak B$ above.   I claim that the isomorphism between
$\Aut\frak A$ and $\Aut\frak B$ gives rise to an outer automorphism of
$\Aut\frak C$;  this seems very natural, but I think there is a fair bit
to check, so I take the argument in easy stages.
     
\medskip
     
\quad{\bf (i)} We may identify the Dedekind completion of
$\frak C=\frak A\times\frak B$ with
$\frak A\times\frak A$.   For $\phi\in\Aut\frak C$, we have a
corresponding $\hat\phi\in\Aut(\frak A\times\frak A)$.   Now
$\frak B\times\frak A$ is invariant under $\hat\phi$.   \Prf\
Consider first $\phi(0,1)=(a_1,b_1)\in\frak C$.  The corresponding
principal ideal
$\frak C_{(a_1,b_1)}\cong\frak A_{a_1}\times\frak B_{b_1}$ of $\frak C$ must be isomorphic to the principal ideal $\frak
C_{(0,1)}\cong\frak B$;  so that if $(a,b)\in\frak C$ and
$(a,b)\Bsubseteq(a_1,b_1)$, then just one of the principal ideals
$\frak C_{(a,b)}\cong \frak A_a\times\frak B_b$,
$\frak C_{(a_1\Bsetminus a,b_1\Bsetminus b)}
\cong \frak A_{a_1\Bsetminus a}\times\frak B_{b_1\Bsetminus b}$ is ccc.   But this can only happen if $\frak A_{a_1}$ is ccc and $\frak B_{b_1}$ is not;  that is, if $a_1$ and
$1\Bsetminus b_1$ belong to $I$.   Consequently
$\hat\phi(0,a)\Bsubseteq (a_1,b_1)$ belongs to $\frak B\times\frak A$ for every $a\in\frak A$.   We also find that
     
\Centerline{$\phi(1,0)=(1,1)\Bsetminus\phi(0,1)
=(1\Bsetminus a_1,1\Bsetminus b_1)\in\frak B\times\frak A$.}
     
\noindent Now if $b\in I$, then
     
\Centerline{$\frak C_{\phi(b,0)}\cong\frak C_{(b,0)}\cong\frak A_b$}
     
\noindent is ccc and
     
\Centerline{$\phi(b,0)\in I\times I\Bsubseteq\frak B\times\frak A$; }
     
\noindent while
     
\Centerline{$\phi(1\Bsetminus b,0)=(1\Bsetminus a_1,1\Bsetminus b_1)
\Bsetminus\phi(b,0)\in\frak B\times\frak A$.}
     
\noindent This shows that $\phi(b,0)\in\frak B\times\frak A$ for every
$b\in\frak B$.   So
     
\Centerline{$\hat\phi(b,a)=\hat\phi(b,0)\Bcup\hat\phi(0,a)\in\frak
B\times\frak A$}
     
\noindent for every $b\in\frak B$ and $a\in\frak A$.\ \Qed
     
\medskip
     
\quad{\bf (ii)} Let $\theta:\frak A\times\frak A\to\frak A\times\frak A$
be the involution defined by setting $\theta(a,b)=(b,a)$ for all $a$,
$b\in\frak A$.   Take $\phi\in\Aut\frak C$ and consider
$\psi=\theta\hat\phi\theta^{-1}\in\Aut(\frak A\times\frak A)$.   If
$c=(a,b)\in\frak C$, then  $\theta^{-1}c=(b,a)\in\frak B\times\frak A$,
so $\hat\phi\theta^{-1}c\in\frak B\times\frak A$, by (i), and $\psi
c\in\frak A\times\frak B=\frak C$.   This shows that
$\psi\restrp\frak{C}$
is a homomorphism from $\frak C$ to itself.   Of course
$\psi^{-1}=\theta\hat\phi^{-1}\theta^{-1}$ has the same property.   So
we have a map $q:\Aut\frak C\to\Aut\frak C$ given by setting
     
\Centerline{$q(\phi)=\theta\hat\phi\theta^{-1}\restrp\frak{C}$}
     
\noindent for $\phi\in\Aut\frak C$.   Evidently $q$ is an automorphism.
     
\medskip
     
\quad{\bf (iii)} \Quer\ Suppose, if possible, that $q$ were an inner
automorphism.   Let $\chi\in\Aut\frak C$ be such that
$q(\phi)=\chi\phi\chi^{-1}$ for every $\phi\in\Aut\frak C$.   Then
     
\Centerline{$\hat\chi\hat\phi\hat\chi^{-1}
=\widehat{q(\phi)}
=\theta\hat\phi\theta^{-1}$}
     
\noindent for every $\phi\in\Aut\frak C$.   Since
$G=\{\hat\phi:\phi\in\Aut\frak C\}$ is a subgroup of
$\Aut(\frak A\times\frak A)$ with many involutions, the `uniqueness' assertion of 384D tells us that $\hat\chi=\theta$.   But
     
\Centerline{$\theta[\frak C]
=\frak B\times\frak A\ne\frak C=\chi[\frak C]=\hat\chi[\frak C]$,}
     
\noindent so this cannot be.\ \Bang
     
Thus $q$ is the required outer automorphism of $\Aut\frak C$.
}%end of proof of 384Pb
     
\cmmnt{\medskip
     
\noindent{\bf Remark} Thus the hypothesis `homogeneous' in 384E, and the hypothesis `Dedekind complete' in 384J, are necessary.}
     
\leader{384Q}{Example} %38{3}Q
Let $\mu$ be Lebesgue measure on $\Bbb R$, and
$(\frak A,\bar\mu)$ its measure algebra.   Then $\Aut_{\bar\mu}\frak A$
has an outer automorphism.   \prooflet{\Prf\    Set $f(x)=2x$ for
$x\in\Bbb R$.
Then $E\mapsto f^{-1}[E]=\bover12E$ is a Boolean automorphism of the
domain $\Sigma$ of $\mu$, and $\mu(\bover12E)=\bover12\mu E$ for every
$E\in\Sigma$ (263A, or otherwise).   So we have a corresponding
$\theta\in\Aut\frak A$
defined by setting $\theta E^{\ssbullet}=(\bover12E)^{\ssbullet}$ for
every $E\in\Sigma$, and $\bar\mu(\theta a)=\bover12\bar\mu a$ for every
$a\in\frak A$.   By 384N, we have an automorphism $q$ of
$\Aut_{\bar\mu}\frak A$
defined by setting $q(\phi)=\theta\phi\theta^{-1}$ for every
measure-preserving automorphism $\phi$.   But $q$ is now an outer
automorphism of $\Aut_{\bar\mu}\frak A$, because (by 384D) the only
possible automorphism of $\frak A$ corresponding to $q$ is $\theta$, and
$\theta$ is not measure-preserving.\ \Qed}
     
\cmmnt{Thus the hypothesis `totally finite' in 384O cannot be
omitted.}
     
\exercises{
\leader{384X}{Basic exercises (a)} %38{3}Xa
Let $\frak A$ be a Boolean algebra.
Show that the following are equiveridical:  (i) $\frak A$ is nowhere rigid;
(ii) for every $a\in\frak A\setminus\{0\}$ and $n\in\Bbb N$ there are
disjoint non-zero $b_0,\ldots,b_n\Bsubseteq a$ such that the principal
ideals $\frak A_{b_i}$ they generate are all isomorphic;
(iii) for every $a\in\frak A\setminus\{0\}$ and $n\ge 1$ there is a
$\phi\in\Aut\frak A$, of order $n$, supported by $a$.
%384I
     
\spheader 384Xb %38{3}Xb
Let $\frak A$ be an atomless homogeneous Boolean
algebra and $\frak B$ a nowhere rigid Boolean algebra, and suppose that
$\Aut\frak A$ is isomorphic to $\Aut\frak B$.   Show that there is an
invariant order-dense subalgebra of $\frak B$ which is isomorphic to
$\frak A$.
%384E, 384I
     
\spheader 384Xc %38{3}Xc
Let $\frak A$ and $\frak B$ be nowhere rigid
Boolean algebras.   Show that if $\Aut\frak A$ and $\Aut\frak B$ are
isomorphic, then the Dedekind completions $\widehat{\frak A}$ and
$\widehat{\frak B}$ are isomorphic.
%384L
     
\spheader 384Xd %38{3}Xd
Find two non-isomorphic atomless totally finite measure
algebras $(\frak A,\bar\mu)$, $(\frak B,\bar\nu)$ such that
$\Aut_{\bar\mu}\frak A$ and $\Aut_{\bar\nu}\frak B$ are isomorphic.
(This is easy.)
%384F, 384M
     
\spheader 384Xe %38{3}Xe
Let $(\frak A,\bar\mu)$ and $(\frak B,\bar\nu)$ be
semi-finite measure algebras and $\theta:\frak A\to\frak B$ a Boolean
isomorphism.   Show that the following are equiveridical:
(i) for every $\phi\in\Aut_{\bar\mu}\frak A$,
$\theta\phi\theta^{-1}\in\Aut_{\bar\nu}\frak B$;
(ii)($\alpha$) for every infinite cardinal $\kappa$ there is an
$\alpha_{\kappa}>0$ such that
$\bar\nu(\theta a)=\alpha_{\kappa}\bar\mu a$ whenever $a\in\frak A$ and
the principal ideal $\frak A_a$ is \Mth\ with Maharam type $\kappa$;
($\beta$) for every $\gamma\in\ooint{0,\infty}$ there is an
$\alpha_{\gamma}>0$ such that
$\bar\nu(\theta a)=\alpha_{\gamma}\bar\mu a$ whenever $a\in\frak A$ is
an atom of measure $\gamma$.
%384N
     
\spheader 384Xf %38{3}Xf
Let $q:\Aut\frak C\to\Aut\frak C$ be the
automorphism of 384Pb.   Show that $q(\phi)$ is measure-preserving
whenever $\phi$ is measure-preserving, so that
$q\restr\Aut_{\bar\lambda}\frak C$ is an outer automorphism of
$\Aut_{\bar\lambda}\frak C$.
%384P
     
\leader{384Y}{Further exercises (a)} %38{3}Ya
Let $(\frak A,\bar\mu)$ and
$(\frak B,\bar\nu)$ be localizable measure algebras such that
$\Aut\frak A\cong\Aut\frak B$.   Show that {\it either}
$\frak A\cong\frak B$ {\it
or} one of $\frak A$, $\frak B$ has just one atom and the other is
atomless.
%384M
     
\header{384Yb}{\bf (b)} %38{3}Yb
Let $(\frak A,\bar\mu)$, $(\frak B,\bar\nu)$ be
localizable measure algebras such that
$\Aut_{\bar\mu}\frak A\cong\Aut_{\bar\nu}\frak B$.   Show that {\it
either} $(\frak A,\bar\mu)\cong(\frak B,\bar\nu)$ {\it or} there is some
$\gamma\in\ooint{0,\infty}$ such that one of $\frak A$, $\frak B$ has
just one atom of measure $\gamma$ and the other has none {\it or} there
are $\gamma$, $\gamma'\in\ooint{0,\infty}$ such that the number of atoms
of $\frak A$ of measure $\gamma$ is equal to the number of atoms of
$\frak B$ of measure $\gamma'$, but not to the number of atoms of
$\frak A$ of measure $\gamma'$.
%384M
     
\header{384Yc}{\bf (c)} %38{3}Yc
Let $(\frak A,\bar\mu)$ be a localizable measure
algebra.    Show that there is an outer automorphism of $\Aut\frak A$
iff $\frak A$ has exactly six atoms.
%384O
     
\header{384Yd}{\bf (d)} %38{3}Yd
Let $(\frak A,\bar\mu)$ be a localizable measure
algebra.   For each infinite cardinal $\kappa$ let $e_{\kappa}$ be the
Maharam-type-$\kappa$ component of $\frak A$ and for each
$\gamma\in\ooint{0,\infty}$ let $A_{\gamma}$ be the set of atoms of
$\frak A$ of measure $\gamma$.
Show that there is an outer automorphism of $\Aut_{\bar\mu}\frak A$ iff
     
\quad{\it either} there is an infinite cardinal $\kappa$ such that
$\bar\mu e_{\kappa}=\infty$
     
\quad{\it or} there are distinct $\gamma$, $\delta\in\ooint{0,\infty}$
such that $\#(A_{\gamma})=\#(A_{\delta})\ge 2$
     
\quad{\it or} there is a $\gamma\in\ooint{0,\infty}$ such that
$\#(A_{\gamma})=6$
     
\quad{\it or} there are $\gamma$, $\delta\in\ooint{0,\infty}$ such that
$\#(A_{\gamma})=2<\#(A_{\delta})<\omega$.
%384O, 384Yc
}%end of exercises
     
\cmmnt{
\Notesheader{384} Let me recapitulate the results above.   If
$\frak A$ and $\frak B$ are Boolean algebras, any isomorphism between
$\Aut\frak A$ and $\Aut\frak B$ corresponds to an isomorphism between
$\frak A$ and
$\frak B$ if {\it either} $\frak A$ and $\frak B$ are atomless and
homogeneous (384E) {\it or} they are nowhere rigid and Dedekind complete
(384J).   If $(\frak A,\bar\mu)$ and $(\frak B,\bar\nu)$ are atomless
localizable measure algebras, then any automorphism between
$\Aut_{\bar\mu}\frak A$ and $\Aut_{\bar\nu}\frak B$ corresponds to an
isomorphism between $\frak A$ and $\frak B$ (384M) which if
$\bar\mu=\bar\nu$ is totally finite will be measure-preserving (384O).
     
These results may appear a little less
surprising if I remark that the elementary Boolean algebras $\Cal PX$
give rise to some of the same phenomena.   The automorphism group of
$\Cal PX$ can be identified with the group $S_X$ of all permutations of
$X$, and this has no outer automorphisms unless $X$ has just six
elements.   Some of the ideas of the fundamental theorem 384D
can be traced through in the purely atomic case also, though of course
there are significant changes to be made, and some serious complications
arise, of which the most striking surround the remarkable fact that
$S_6$ {\it does} have an outer automorphism ({\smc Burnside 1911},
\S162;  {\smc Rotman 84}, Theorem 7.8).   I have not attempted to
incorporate these into the main results.   For localizable measure
algebras, where the only rigid parts are atoms, the complications are
superable, and I think I have listed them all
(384Ya-384Yd).  %384Ya 384Yb 384Yc 384Yd
}
     
\discrpage

