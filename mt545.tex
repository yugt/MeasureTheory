\frfilename{mt545.tex}
\versiondate{10.2.14}
\copyrightdate{2005}

\def\chaptername{Real-valued-measurable cardinals}
\def\sectionname{PMEA and NMA}

\newsection{545}

One of the reasons for supposing that it is consistent to assume that there
are measurable cardinals is that very much stronger axioms have been
studied at length without any contradiction appearing.   Here I mention two
such axioms which have obvious consequences in measure theory.

\leader{545A}{Theorem} The following are equiveridical:

(i) for every cardinal $\lambda$, there is a probability space
$(X,\Cal PX,\mu)$ with $\tau(\mu)\ge\lambda$ and
$\add\mu\ge\frak c$;

(ii) for every cardinal $\lambda$, there is an extension of the usual
measure $\nu_{\lambda}$ on $\{0,1\}^{\lambda}$ to a $\frak c$-additive
probability measure with domain $\Cal P(\{0,1\}^{\lambda})$;

(iii) for every semi-finite locally compact measure space
$(X,\Sigma,\mu)$\cmmnt{ (definition:  342Ad)},
there is an extension of $\mu$ to a
$\frak c$-additive measure with domain $\Cal PX$.

\proof{{\bf (i)$\Rightarrow$(ii)} Assume (i).   Let $\lambda$
be a cardinal;  of course (ii) is surely true for finite $\lambda$,
so we may take it that $\lambda\ge\omega$.   Let $(X,\Cal PX,\mu)$
be a probability space with Maharam type at least $\lambda^+$
and with
$\add\mu\ge\frak c$.   Taking $\frak A$ to be the measure algebra of $\mu$,
there is an $a\in\frak A$ such that the principal ideal $\frak A_a$ it
generates is homogeneous with Maharam type at least $\lambda$
(332S).   Let $E\in\Cal PX$ be
such that $E^{\ssbullet}=a$, so that the subspace measure $\mu_E$ is
\Mth\ with Maharam type at least $\lambda$.   Setting $\mu'A=\mu A/\mu E$
for $A\subseteq E$, $(E,\Cal PE,\mu')$ is a \Mth\
probability space with Maharam type at least $\lambda$, and
$\add\mu'\ge\add\mu\ge\frak c$.   By 343Ca, there is a function
$f:E\to\{0,1\}^{\lambda}$ which is \imp\ for $\mu'$ and $\nu_{\lambda}$.
Now the image measure $\nu=\mu'f^{-1}$ is a $\frak c$-additive
extension of $\nu_{\lambda}$ to $\Cal P(\{0,1\}^{\lambda})$.

\wheader{545A}{4}{2}{2}{60pt}

{\bf(ii)$\Rightarrow$(iii)} Assume (ii).

\medskip

\quad\grheada\ Suppose that $(X,\Sigma,\mu)$ is a compact probability
space.   Set $\lambda=\max(\omega,\tau(\mu))$.   Then there is
an \imp\ function $f:\{0,1\}^{\lambda}\to X$
(343Cd).  If $\nu$ is a $\frak c$-additive
extension of $\nu_{\lambda}$ to $\Cal P(\{0,1\}^{\lambda})$, then
$\nu f^{-1}$ is a $\frak c$-additive extension of $\mu$ to $\Cal PX$.

\medskip

\quad\grheadb\ Let $(X,\Sigma,\mu)$ be any semi-finite locally compact
measure space.   Set $\Sigma^{f+}=\{E:E\in\Sigma$, $0<\mu E<\infty\}$ and
let $\Cal E\subseteq\Sigma^{f+}$ be maximal subject to $E\cap F$ being
negligible for all distinct $E$, $F\in\Sigma$.   If
$H\in\Sigma$ and $\mu H<\infty$, then
$\Cal H=\{E:E\in\Cal E$, $\mu(E\cap H)>0\}$ is countable and
$E\setminus\bigcup\Cal H$ is negligible, so
$\mu H=\sum_{E\in\Cal E}\mu(E\cap H)$;  because $\mu$ is semi-finite,
$\mu H=\sum_{E\in\Cal E}\mu(E\cap H)$ for every $H\in\Sigma$.

For each $E\in\Cal E$, the subspace measure $\mu_E$ is compact;  applying
($\alpha$) to a normalization of $\mu_E$, we have an
extension $\mu'_E$ of $\mu_E$ to a  $\frak c$-additive measure with domain
$\Cal PE$.   Set $\mu'A=\sum_{E\in\Cal E}\mu'_E(A\cap E)$ for
$A\subseteq X$;  then $\mu':\Cal PX\to[0,\infty]$ is a $\frak c$-additive
measure extending $\mu$.

\medskip

{\bf(iii)$\Rightarrow$(ii)} and {\bf (ii)$\Rightarrow$(i)} are trivial.
}%end of proof of 545A

\leader{545B}{Definition} PMEA (the `{\bf product measure extension axiom}')
is the assertion that the statements (i)-(iii) of 545A are true.

\leader{545C}{Proposition} If PMEA is true, then $\frak c$ is
atomlessly-measurable.

\proof{ By 545A(ii) we have an extension of the usual measure on
$\{0,1\}^{\omega}$ to a $\frak c$-additive measure $\mu$ with domain
$\Cal P(\{0,1\}^{\omega})$.   Since $\mu$ is zero on singletons,
$\add\mu=\frak c$ exactly, so 543Ba and 543Bc tell
us that $\frak c$ is real-valued-measurable, therefore
atomlessly-measurable.
}

\leader{545D}{Definition} NMA (the `{\bf normal measure axiom}') is the
statement

\inset{\noindent For every set $I$ there is a $\frak c$-additive
probability measure $\nu$ on $S=[I]^{<\frak c}$, with domain $\Cal PS$,
such that

($\alpha$) $\nu\{s:i\in s\in S\}=1$ for every $i\in I$,

($\beta$) if $A\subseteq S$, $\nu A>0$ and $f:A\to I$ is such that
$f(s)\in s$ for every $s\in A$, then there is an $i\in I$ such that
$\nu\{s:s\in A$, $f(s)=i\}>0$.
}

\leader{545E}{Proposition} NMA implies PMEA.

\proof{ Assume NMA.   Let $\lambda$ be any cardinal.   Let $\kappa$ be a
regular
infinite cardinal greater than the cardinal power $\lambda^{\omega}$, and
$\nu$ a $\frak c$-additive
probability on $[\kappa]^{<\frak c}$ as in 545D.   For $\xi<\kappa$
define $f_{\xi}:[\kappa]^{<\frak c}\to\frak c$ by setting
$f_{\xi}(s)=\otp(s\cap\xi)$ for every $s\in[\kappa]^{<\frak c}$.   Then if
$\xi<\eta<\kappa$ we have $f_{\xi}(s)<f_{\eta}(s)$ whenever $\xi\in s$,
that is, for $\nu$-almost every $s$.

Let $g:\frak c\to\Cal P\Bbb N$ be any injection.   For $\xi<\kappa$ and
$n\in\Bbb N$ let $a_{\xi n}$ be the equivalence class
$\{s:n\in g(f_{\xi}(s))\}^{\ssbullet}$ in the measure algebra $\frak A$ of
$\nu$.   If $\xi<\eta<\kappa$ then $g(f_{\xi}(s))\ne g(f_{\eta}(s))$ for
$\nu$-almost every $s$, so
$\sup_{n\in\Bbb N}a_{\xi n}\Bsymmdiff a_{\eta n}=1$ in $\frak A$ and there
is an $n\in\Bbb N$ such that $a_{\xi n}\ne a_{\eta n}$.   Accordingly
$\#(\frak A)^{\omega}\ge\kappa>\lambda^{\omega}$ and
$\#(\frak A)>\lambda^{\omega}$.   As
$\#(\frak A)\le\max(4,\tau(\frak A)^{\omega})$ (4A1O/514De),
$\tau(\frak A)>\lambda$.   So $\nu$ witnesses that 545A(i) is true of
$\lambda$.
}%end of proof of 545E

\leader{545F}{Proposition} Suppose that NMA is true.   Let $\frak A$ be a
Boolean algebra such that whenever $s\in[\frak A]^{<\frak c}$ there is
a subalgebra $\frak B\subseteq\frak A$, including $s$, with a strictly
positive countably additive functional.   Then there is a strictly
positive countably additive functional on $\frak A$.

\cmmnt{\medskip

\noindent{\bf Remark} For the definition and elementary properties of
countably additive functionals on arbitrary Boolean algebras, see \S326.
}

\proof{ Of course we can suppose that $\frak A\ne\{0\}$.
Let $\nu$ be a $\frak c$-additive probability on
$S=[\frak A]^{<\frak c}$ as in 545D.   For each $s\in S$, let $\frak B_s$
be a subalgebra of $\frak A$ including $s$ with a strictly
positive countably additive functional $\mu_s$.   Normalizing
$\mu_s$ if necessary, we may suppose that $\mu_s1=1$.   Now, for
$a\in\frak A$, set $\mu(a)=\int\mu_s(a)\nu(ds)$;  because
$a\in s\subseteq\frak B_s=\dom\mu_s$ for $\nu$-almost every $s$, the
integral is well-defined.   Because every $\mu_s$ is additive, so is $\mu$;
because every $\mu_s$ is strictly positive, so is $\mu$.   If
$\sequencen{a_n}$ is a non-increasing sequence in $\frak A$ with infimum
$0$, then $\lim_{n\to\infty}\mu_s(a_n)=0$ whenever
$s\in[\frak A]^{<\frak c}$ contains every $a_n$, that is, for
$\nu$-almost every $s$;  so $\lim_{n\to\infty}\mu a_n=0$.   As
$\sequencen{a_n}$ is arbitrary, $\mu$ is countably additive
(326Ka).
}%end of proof of 545F

\leader{545G}{Corollary} Suppose that NMA is true.   Let $\frak A$ be a
Boolean algebra such that every $s\in[\frak A]^{<\frak c}$ is included in a
subalgebra of $\frak A$ which is, in itself, a measurable algebra.
Then $\frak A$ is a measurable algebra.

\proof{{\bf (a)} Because $\frak c$ is atomlessly-measurable,
it is surely greater
than $\omega_1$ (419G/438Cd/542C).   So a family in $\frak A\setminus\{0\}$
with cardinal $\omega_1$ lies within some measurable subalgebra of $\frak A$
and cannot be disjoint.   Thus $\frak A$ is ccc.

\medskip

{\bf (b)} If $A\subseteq\frak A$, set

\Centerline{$D_1=\{d:d\in\frak A$, $d\Bsubseteq a$ for some $a\in A\}$,}

\Centerline{$D_2=\{d:d\in\frak A$, $d\Bcap a=0$ for every $a\in A\}$.}

\noindent Then $D_1\cup D_2$ is order-dense in $\frak A$ so includes a
partition $D$ of unity in $\frak A$.   By (a), $D$ is countable, so lies
within a measurable subalgebra $\frak B$ of $\frak A$.   Now $D\cap D_1$
has a supremum $b$ in $\frak B$ which is disjoint from every member of
$D\cap D_2$.   But this means that $b$ is the supremum of $A$ in $\frak A$.
As $A$ is arbitrary, $\frak A$ is Dedekind complete.

\medskip

{\bf (c)} By 545F, $\frak A$ has a strictly positive countably additive
functional $\mu$;  but now $(\frak A,\mu)$ is a totally finite measure
algebra.
}%end of proof of 545G

%\query:  do we really need NMA here?  quite a lot works if we just
%say that every set of size \le\omega_1  is in a measurable subalgebra

\exercises{\leader{545X}{Basic exercises (a)}
%\spheader 545Xa
Suppose that $I$ is a set, and that $\nu$
is a $\frak c$-additive probability measure with domain
$\Cal P([I]^{<\frak c})$ satisfying the conditions of 545D.   Suppose that
$A\subseteq[I]^{<\frak c}$ and $f:A\to I$ are such
that $f(s)\in s$ for every $s\in A$.   Show that there is a countable set
$D\subseteq I$ such that $f(s)\in D$ for $\nu$-almost every $s\in A$.
%545D

\leader{545Y}{Further exercises (a)}
%\spheader 545Ya
Suppose that $I$ is a set, and that $\nu$ is a
$\frak c$-additive probability measure with domain
$\Cal PS$, where $S=[I]^{<\frak c}$, satisfying the conditions of 545D.
Suppose that $f:[I]^{<\omega}\to S$ is any function.   Show that
$\nu\{s:s\in S$, $f(J)\subseteq s$ for every $J\in[s]^{<\omega}\}=1$.
%545D

\spheader 545Yb Suppose that NMA is true.   Show that
$\square_{\lambda}$ is false for every
$\lambda\ge\frak c$.   (Cf.\ 555Yf below.)
%545D 555Yf 55bits
}%end of exercises

\endnotes{
\Notesheader{545} I have given the sketchiest of accounts here.   The
main interest of PMEA and NMA has so far been
in their remarkable consequences in
general topology and (for NMA) its associated reflection principles;
see {\smc Fremlin 93} and the references there.   545G is such a reflection
principle.   Note that the measurable subalgebras declared to exist need
not be regularly embedded in the given algebra.
For K.Prikry's theorem that
it is consistent to assume NMA if it is consistent to suppose that there is
a supercompact cardinal, see 555N below.
}%end of notes

\discrpage



