\frfilename{mt517.tex}
\versiondate{14.11.14}
\copyrightdate{2003}


\def\chaptername{Cardinal functions}
\def\sectionname{Martin numbers}

\newsection{517}

I devote a section to the study of `Martin numbers' of partially ordered
sets and Boolean algebras.   Like additivity and cofinality they
enable us to frame as theorems of ZFC some important arguments which
were first used in special models of set theory, and to pose 
challenging questions on the relationships between classical structures
in analysis.   I begin with some general remarks on the Martin numbers
of partially ordered sets (517A-517E);  %517A 517B 517C 517D 517E
most of these are perfectly elementary but the equivalent conditions of
517B, in particular, are useful and not all obvious.   Much of the
importance of Martin numbers comes from their effect on precalibers
(517F, 517H) and hence on saturation of products (517G).   The same
ideas can be expressed in terms of Boolean algebras, with no surprises
(517I).   I have not set out a definition of `Martin number' for a
topological space, but the Nov\'ak number of a locally compact Hausdorff
space is closely related to the Martin numbers of its regular open
algebra and its algebra of open-and-closed sets (517J-517K).
Consequently we have connexions between the Martin number and the weak
distributivity of a Boolean algebra (517L).   A striking fact, which
will have a prominent role in the next chapter, is that
non-trivial countable partially ordered sets all have the same Martin
number $\frakmctbl$ (517P).

\leader{517A}{Proposition} For any partially ordered set $P$,
$\frak m^{\uparrow}(P)\ge\omega_1$.

\proof{ If $\Cal Q$ is a countable family of cofinal subsets of $P$ and
$p_0\in P$, let $\sequencen{Q_n}$ be a sequence running over
$\Cal Q\cup\{P\}$, and choose $\langle p_n\rangle_{n\ge 1}$ inductively
such that $p_{n+1}\ge p_n$ and $p_{n+1}\in Q_n$ for every $n\in\Bbb N$.
Then $\{p_n:n\in\Bbb N\}$ is an upwards-linked subset of $P$ meeting
every member of $\Cal Q$.}

\vleader{72pt}{517B}{Lemma} 
Let $P$ be a partially ordered set, and $\kappa$ a
cardinal.   Then the following are equiveridical:

(i) $\kappa<\frak m^{\uparrow}(P)$;

(ii) whenever $p_0\in P$ and $\Cal Q$ is a family of up-open cofinal
subsets of $P$ with $\#(\Cal Q)\le\kappa$, there is an upwards-linked
subset of $P$ which contains $p_0$ and meets every member of $\Cal Q$;

(iii) whenever $p_0\in P$ and $\Cal A$ is a family of maximal
up-antichains in $P$ with $\#(\Cal A)\le\kappa$, there is an
upwards-linked subset of $P$ which contains $p_0$ and meets every member
of $\Cal A$;

(iv) whenever $p_0\in P$ and $\Cal Q$ is a family of cofinal subsets of
$P$ with $\#(\Cal Q)\le\kappa$, there is an upwards-directed subset of
$P$ which contains $p_0$ and meets every member of $\Cal Q$;

(v) whenever $p_0\in P$ and $\Cal Q$ is a family of up-open cofinal
subsets of $P$ with $\#(\Cal Q)\le\kappa$, there is an upwards-directed
subset of $P$ which contains $p_0$ and meets every member of $\Cal Q$;

(vi) whenever $p_0\in P$ and $\Cal A$ is a family of maximal
up-antichains in $P$ with $\#(\Cal A)\le\kappa$, there is an
upwards-directed subset of $P$ which contains $p_0$ and meets every
member of $\Cal A$.

\proof{{\bf (a)} Most of the circuit is elementary.

\medskip

\quad{\bf(vi)$\Rightarrow$(iv)} 
because every cofinal subset of $P$ includes a
maximal up-antichain (513Aa).

\medskip

\quad{\bf(iv)$\Rightarrow$(v)$\Rightarrow$(ii)} are trivial.

\medskip

\quad{\bf(ii)$\Rightarrow$(i)} Assuming (ii), let $\Cal Q$ be a family of
cofinal subsets of $P$ with $\#(\Cal Q)\le\kappa$.   For each $Q\in\Cal Q$,
$U_Q=\bigcup_{q\in Q}\coint{q,\infty}$ is up-open and cofinal with $P$
(513Ab).
If $p_0\in P$, (ii) tells us that there is an upwards-linked subset $R_0$ of
$P$ containing $p_0$ and meeting $U_Q$ for every $Q\in\Cal Q$.
Set $R=\bigcup_{p\in R_0}\ocint{-\infty,p}$;  then $R$ is an upwards-linked
subset of $P$ containing $p_0$ and meeting every member of $\Cal Q$.   As
$\Cal Q$ and $p_0$ are arbitrary, (i) is true.

\medskip

\quad{\bf(i)$\Rightarrow$(iii)} Assuming (i), let $\Cal A$ be a family of
maximal up-antichains in $P$ with $\#(\Cal A)\le\kappa$.   For each
$A\in\Cal A$, $U_A=\bigcup_{p\in A}\coint{p,\infty}$ is cofinal with $P$.
So (i)
tells us that if $p_0\in P$ there is an upwards-linked subset $R_0$ of $P$
containing $p_0$ and meeting $U_A$ for every $A\in\Cal A$.   As just
above, set
$R=\bigcup_{p\in R_0}\ocint{-\infty,p}$;  then $R$ is upwards-linked, contains
$p_0$ and meets every member of $\Cal A$.   As $\Cal A$ and $p_0$ are arbitrary,
(iii) is true.

\medskip

{\bf (b)} So we are left with (iii)$\Rightarrow$(vi).
Assume (iii), and take $p_0\in P$
and a family $\Cal A$ of maximal up-antichains in $P$ with
$\#(\Cal A)\le\kappa$.   Let $\Cal C$ be the set of all maximal up-antichains in
$P$.   For $A\in\Cal C$, set
$U_A=\bigcup_{q\in A}\coint{q,\infty}$.   Then $U_A$ is cofinal with
$P$.   Consequently $U_A\cap U_B$ is cofinal with
$P$ for any $A$, $B\in\Cal C$, because if $p\in P$ there are $q\in U_A$
and $r\in U_B$ such that $p\le q\le r$, and now $r\in U_A\cap U_B$.   It
follows that $U_A\cap U_B$ includes a maximal up-antichain $D(A,B)$.

Take any $A_0\in\Cal C$ such that $p_0\in A_0$.
Let $\Cal A^*\subseteq\Cal C$ be such that
$\{A_0\}\cup\Cal A\subseteq\Cal A^*$,
$D(A,B)\in\Cal A^*$ for all $A$, $B\in\Cal A^*$, and
$\#(\Cal A^*)\le\max(\omega,\kappa)$ (5A1Fb).
Then there is an upwards-linked subset
$R_0$ of $P$ containing $p_0$ and meeting every member of $\Cal A^*$.   \Prf\ If
$\#(\Cal A^*)\le\kappa$, this is immediate from (iii);  if
$\#(\Cal A^*)\le\omega$,
it is because $\omega<\frak m^{\uparrow}(P)$, by 517A, and
(i)$\Rightarrow$(iii).\ \Qed

Set $R=R_0\cap\bigcup\Cal A^*$.   Then $R$ contains $p_0$ (because
$p_0\in R_0\cap A_0$) and $R$ meets every member of $\Cal A$.   Also $R$ is
upwards-directed.   \Prf\ If $p$, $q\in R$, take $A$, $B\in\Cal A^*$ such that
$p\in A$ and $q\in B$.   Then $D(A,B)\in\Cal A^*$, so
there is an $r\in R_0\cap D(A,B)$, and $r\in R$.
As $r\in U_A\cap U_B$, there must be $p'\in A$ and $q'\in B$ such that
$p'\le r$ and $q'\le r$.   But $R_0$ is upwards-linked, so

\Centerline{$\emptyset\ne\coint{p,\infty}\cap\coint{r,\infty}\subseteq
\coint{p,\infty}\cap\coint{p',\infty}$;}

\noindent as $A$ is an up-antichain, $p=p'$.   Similarly, $q=q'$ and
$r\in R$ is an upper bound of $\{p,q\}$.   As $p$ and $q$ are arbitrary,
$R$ is upwards-directed.\ \Qed

So we have a set $R$ of the kind required by (vi).
}%end of proof of 517B

\leaveitout{
elementary-submodel version of (iii)=>(vi):

Let $M$ be a suitable elementary submodel with cardinal $\max(\omega,\kappa)$.
Let $R_1$ be an upwards-linked subset of $P$ containing $p_0$ and meeting every
maximal antichain belonging to $M$;  let $R$ be the set of elements of $R_1$
which belong to maximal antichains in $M$.   There is a maximal antichain in $M$
containing $p_0$, so $p_0\in R$.   If $p$, $q\in R$, let $A$, $B$ be maximal
antichains in $M$ containing $p$, $q$.   Then there is a maximal antichain
$C$ in
$M$ refining both $A$ and $B$;  let $r$ be the member of $R_1\cap C$;  then $r$
dominates both $p$ and $q$ and belongs to $R$, so $R$ is upwards-directed.
}%end of leaveitout

\leader{517C}{Lemma} Let $P_0$ and $P_1$ be partially ordered sets, and
suppose that there is a relation $S\subseteq P_0\times P_1$ such that
$S[P_0]$ is cofinal with $P_1$, $S^{-1}[Q]$ is cofinal with $P_0$ for
every cofinal $Q\subseteq P_1$, and $S[R]$ is upwards-linked in $P_1$
for every upwards-linked $R\subseteq P_0$.   Then
$\frak m^{\uparrow}(P_1)\ge\frak m^{\uparrow}(P_0)$.

\proof{ Suppose that $p_1\in P_1$ and that $\Cal Q$ is a family of
cofinal subsets of $P_1$ with $\#(\Cal Q)<\frak m^{\uparrow}(P_0)$.
Then there is a pair $(p_0,p'_1)\in S$ such that $p'_1\ge p_1$.   Now
$S^{-1}[Q]$ is cofinal with $P_0$ for every $Q\in\Cal Q$, so there is an
upwards-linked $R\subseteq P_0$ containing $p_0$ and meeting $S^{-1}[Q]$
for every $Q\in\Cal Q$.   In this case $p'_1\in S[R]$ and
$S[R]$ is upwards-linked, so $\{p_1\}\cup S[R]$ is an
upwards-linked subset of $P_1$ containing $p_1$ and meeting every member
of $\Cal Q$.   As $p_1$ and $\Cal Q$ are arbitrary,
$\frak m^{\uparrow}(P_1)\ge\frak m^{\uparrow}(P_0)$.
}%end of proof of 517C

\vleader{48pt}{517D}{Proposition} (a) 
If $P$ is a partially ordered set and $Q$ is a cofinal subset of $P$, then
$\frak m^{\uparrow}(P)=\frak m^{\uparrow}(Q)$.

(b) If $P$ is any partially ordered set and $\RO^{\uparrow}(P)$ is its
regular open algebra when it is given its up-topology, then
$\frak m^{\uparrow}(P)=\frak m(\RO^{\uparrow}(P))$.

(c) If $P$ is a partially ordered set and $p_0\in P$, then
$\frak m^{\uparrow}(\coint{p_0,\infty})\ge\frak m^{\uparrow}(P)$.

\proof{{\bf (a)} Let $P_0$, $P_1$ be cofinal subsets of $P$, and set
$S=\{(p_0,p_1):p_0\in P_0$, $p_1\in P_1$, $p_0\ge p_1\}$.   Then $S$
satisfies the conditions of 517C, so
$\frak m^{\uparrow}(P_1)\ge\frak m^{\uparrow}(P_0)$.   It follows at
once that all cofinal subsets of $P$, including $P$ itself,
have the same Martin number.

\medskip

{\bf (b)(i)} Setting $S=\{(p,G):p\in G\in\RO^{\uparrow}(P)\}$, $S$
satisfies the conditions of 517C with $P_0=(P,\le)$ and
$P_1=(\RO^{\uparrow}(P)^+,\supseteq)$, so
$\frak m(\RO^{\uparrow}(P))\ge\frak m^{\uparrow}(P)$.

\medskip

\quad{\bf (ii)} Setting
$S'=\{(G,p):G\in\RO^{\uparrow}(P)^+,\,
p\in P,\,G\subseteq\overline{\coint{p,\infty}}\}$,
$S'$ satisfies the conditions of 517C with
$P_0=(\RO^{\uparrow}(P)^+,\supseteq)$ and $P_1=(P,\le)$, so
$\frak m^{\uparrow}(P)\ge\frak m(\RO^{\uparrow}(P))$.

\medskip

{\bf (c)} Let $\Cal Q$ be a family of upwards-cofinal subsets of
$\coint{p_0,\infty}$ with $\#(\Cal Q)<\frak m^{\uparrow}(P)$, and
$p_1\in\coint{p_0,\infty}$.   For each $Q\in\Cal Q$, set
$Q'=Q\cup\{p:p\in P,\,
\coint{p,\infty}\cap\coint{p_0,\infty}=\emptyset\}$.   Then every $Q'$
is cofinal with $P$.   So there is an upwards-linked set $R\subseteq P$
containing $p_1$ and meeting $Q'$ for every $Q\in\Cal Q$.   If
$Q\in\Cal Q$ and $r\in R\cap Q'$, then
$\coint{r,\infty}\cap\coint{p_0,\infty}
\supseteq\coint{r,\infty}\cap\coint{p_1,\infty}$ is non-empty, so
$r\in Q$.   Thus $R\cap\coint{p_0,\infty}$ is an upwards-linked subset
of $\coint{p_0,\infty}$ containing $p_1$ and meeting every member of
$\Cal Q$.  As $\Cal Q$ and $p_1$ are arbitrary,
$\frak m^{\uparrow}(\coint{p_0,\infty})\ge\frak m^{\uparrow}(P)$.
}%end of proof of 517D

\leader{517E}{Corollary} Let $P$ be a partially ordered set such that
$\frak m^{\uparrow}(P)$ is not $\infty$.   Then
$\frak m^{\uparrow}(P)\le 2^{\cf P}$.

\proof{ Let $Q_0$ be a
cofinal subset of $P$ with $\#(Q_0)=\cf P$.   Then
$\frak m^{\uparrow}(Q_0)=\frak m^{\uparrow}(P)<\infty$.   So there are
$q_0\in Q_0$ and a family
$\Cal Q$ of cofinal subsets of $Q_0$ such that no upwards-linked subset
of $Q_0$ containing $p_0$ can meet every member of $\Cal Q$.   Now

\Centerline{$\frak m^{\uparrow}(P)
=\frak m^{\uparrow}(Q_0)\le\#(\Cal Q)\le 2^{\#(Q_0)}=2^{\cf P}$.}
}%end of proof of 517E

\leader{517F}{Proposition} Let $P$ be a non-empty partially ordered set.

(a) Suppose that $\kappa$ and $\lambda$ are cardinals such that
$\sat^{\uparrow}(P)\le\cf\kappa$, $\lambda\le\kappa$ and
$\lambda<\frak m^{\uparrow}(P)$.   Then $(\kappa,\lambda)$ is an upwards
precaliber pair of $P$.

(b) In particular, if
$\sat^{\uparrow}(P)\le\cf\kappa\le\kappa<\frak m^{\uparrow}(P)$ then
$\kappa$ is an up-precaliber of $P$.

\proof{{\bf (a)} Since $P$ is not empty, $\sat^{\uparrow}(P)\ge 2$ and
$\kappa$ is infinite.
Write $\theta$ for $\sat^{\uparrow}(P)$.   Let
$\ofamily{\xi}{\kappa}{p_{\xi}}$ be a family in $P$.   For
$I\subseteq\kappa$, set

\Centerline{$U_I=\bigcup_{\xi\in I}\coint{p_{\xi},\infty}$,
\quad$V_I=\{q:q\in P,\,
  \coint{q,\infty}\cap\coint{p_{\xi},\infty}=\emptyset$ for
every $\xi\in I\}$.}

\noindent Then for every $J\subseteq\kappa$ there is an
$I\in[J]^{<\theta}$ such that $V_J\cup U_I$ is cofinal with $P$.   \Prf\
$V_J\cup U_J$ is cofinal with $P$, so there is a maximal up-antichain
$A\subseteq V_J\cup U_J$.   Now
$\#(A\cap U_J)<\sat^{\uparrow}(P)=\theta$, so there is a set
$I\in[J]^{<\theta}$ such that $A\cap U_J\subseteq U_I$, and
$A\subseteq V_J\cup U_I$.   Now
$V_J\cup U_I\supseteq\bigcup_{q\in A}\coint{q,\infty}$, so is cofinal
with $P$.\ \Qed

Next, $Q=\bigcup\{V_{\kappa\setminus I}:I\in[\kappa]^{<\kappa}\}$ is not
cofinal with $P$.   \Prf\Quer\ If it were, there would be a maximal
up-antichain $A\subseteq Q$.   For each $q\in A$, let
$I_q\in[\kappa]^{<\kappa}$ be such that $q\in V_{\kappa\setminus I_q}$.
Because $\#(A)<\theta\le\cf\kappa$,
$\bigcup_{q\in A}I_q\ne\kappa$, and there is a
$\xi\in\kappa\setminus\bigcup_{q\in A}I_q$.   But now
$\coint{q,\infty}\cap\coint{p_{\xi},\infty}=\emptyset$ for every
$q\in A$, and $A$ is not a maximal antichain.\ \Bang\Qed

Let $q_0\in P$ be such that $Q\cap\coint{q_0,\infty}=\emptyset$.
Choose
$\ofamily{\xi}{\lambda}{I_{\xi}}$ inductively in such a way that,
writing $J_{\xi}=\kappa\setminus\bigcup_{\eta<\xi}I_{\eta}$,
$I_{\xi}\in[J_{\xi}]^{<\theta}$ and
$Q_{\xi}=V_{J_{\xi}}\cup U_{I_{\xi}}$ is cofinal with $P$ for every
$\xi<\lambda$.   Because $\lambda<\frak m^{\uparrow}(P)$, there is an
upwards-directed set $R\subseteq P$ containing $q_0$ and meeting every
$Q_{\xi}$.   Set $\Gamma
=\{\eta:\eta<\kappa,\,R\cap\coint{p_{\eta},\infty}\ne\emptyset\}$;  then
$\{p_{\eta}:\eta\in\Gamma\}$ is upwards-centered.   Next,
$\kappa\setminus J_{\xi}=\bigcup_{\eta<\xi}I_{\eta}$ has cardinal less
than $\kappa$ for every $\xi<\lambda$.   \Prf\ If $\theta=\kappa$ or
$\kappa=\omega$, this
is because $\#(\xi)<\kappa$ and $\#(I_{\eta})<\kappa$ for every
$\eta<\xi$ and $\kappa$ is regular (use 513Bb).   Otherwise it's
because $\max(\omega,\theta,\#(\xi))<\kappa$.\ \Qed

This means that $V_{J_{\xi}}\cap\coint{q_0,\infty}=\emptyset$ and
$R\cap V_{J_{\xi}}$
must be empty, for every $\xi<\lambda$.   We must therefore have
$R\cap U_{I_{\xi}}\ne\emptyset$ for each $\xi<\lambda$, so that
$\Gamma\cap I_{\xi}\ne\emptyset$;  as $\ofamily{\xi}{\lambda}{I_{\xi}}$
is disjoint, $\#(\Gamma)\ge\lambda$.

As $\ofamily{\xi}{\kappa}{p_{\xi}}$ is arbitrary, $(\kappa,\lambda)$ is
an upwards precaliber pair of $P$.

\medskip

{\bf (b)} This follows at once, setting $\lambda=\kappa$.
}%end of proof of 517F

\vleader{48pt}{517G}{Corollary} (a) 
If $P$ and $Q$ are partially ordered sets
and $\sat^{\uparrow}(Q)<\frak m^{\uparrow}(P)$, then
$\sat^{\uparrow}(P\times Q)$ is at most
$\max(\omega,\sat^{\uparrow}(P),\sat^{\uparrow}(Q))$.

(b) Let $\familyiI{P_i}$ be a family of non-empty partially ordered sets
with upwards finite-support product $P$.   Let $\kappa$ be a regular
uncountable cardinal such that
$\sat^{\uparrow}(P_i)\le\kappa<\frak m^{\uparrow}(P_i)$
for every $i\in I$.   Then $\sat^{\uparrow}(P)\le\kappa$.

\proof{{\bf (a)} Set $\lambda=\sat^{\uparrow}(Q)$,
$\kappa=\max(\omega,\sat^{\uparrow}(P),\sat^{\uparrow}(Q))$.   Then
$\kappa$ is regular (513Bb again), $\lambda\le\kappa$ and
$\lambda<\frak m^{\uparrow}(P)$, so $(\kappa,\lambda)$ is an upwards
precaliber pair of $P$ and $(\kappa,\lambda,2)$ is an upwards
precaliber triple of $P$.   By 516Ta,
$\sat^{\uparrow}(P\times Q)\le\kappa$.

\medskip

{\bf (b)} By 517Fb, $\kappa$ is an up-precaliber of $P_i$ for every $i$,
so $(\kappa,\kappa,2)$ is an upwards precaliber triple of every $P_i$, and
we can use 516Tb.
}%end of proof of 517G

\leader{517H}{Proposition} Let $P$ be a non-empty partially ordered set,
and let $P^*$ be the upwards finite-support product of the family
$\sequencen{P_n}$ where $P_n=P$ for every $n$.   Suppose that
$\kappa<\frak m^{\uparrow}(P^*)$.

(a) Every subset of $P$ with $\kappa$ or fewer members can be covered by
a sequence of upwards-directed sets.

(b) In particular, if $\kappa$ is uncountable then $(\kappa,\lambda)$ is
an upwards precaliber pair of $P$ for every $\lambda<\kappa$, and if
$\kappa$ has uncountable cofinality then $\kappa$ is an up-precaliber of
$P$.

\proof{{\bf (a)} Let $A\in[P]^{<\kappa}$.   For each $p\in A$, set
$Q_p=\{q:q\in P^*,\Exists\ n\in\dom q,\,q(n)=p\}$;  then $Q_p$ is
cofinal with $P^*$.   So there is an upwards-directed set
$R\subseteq P^*$ such that
$R\cap Q_p\ne\emptyset$ for every $p\in A$.   For each $n\in\Bbb N$, set
$R_n=\{q(n):q\in R,\,n\in\dom q\}$.   Then
$A\subseteq\bigcup_{n\in\Bbb N}R_n$.   If $n\in\Bbb N$ and $r$,
$r'\in R_n$, there are $q$, $q'\in R$ such that $q(n)=r$ and $q'(n)=r'$.
Now there is a $q''\in R$ such that $q''\ge q$ and $q''\ge q'$, in which
case $q''(n)$ belongs to $R_n\cap\coint{r,\infty}\cap\coint{r',\infty}$.
As $r$ and $r'$ are arbitrary, $R_n$ is upwards-directed.   Thus
$\sequencen{R_n}$ is an appropriate sequence.

\medskip

{\bf (b)} Suppose that $\kappa$ is uncountable and that either
$\lambda<\kappa$ or $\cf\kappa>\omega$ and $\lambda=\kappa$.   Let
$\ofamily{\xi}{\kappa}{p_{\xi}}$ be any family in $P$.   Let
$\sequencen{R_n}$ be a sequence of upwards-directed sets covering
$\{p_{\xi}:\xi<\kappa\}$, and for each $n\in\Bbb N$ set
$\Gamma_n=\{\xi:p_{\xi}\in R_n\}$.   There must be some $n$ such that
$\#(\Gamma_n)\ge\lambda$, and $\{p_{\xi}:\xi\in\Gamma_n\}$ is
upwards-centered.
}%end of proof of 517H

\leader{517I}{Proposition} Let $\frak A$ be a Boolean algebra.

(a) If $\frak B$ is a regularly embedded subalgebra of $\frak A$, then
$\frak m(\frak B)\ge\frak m(\frak A)$.

(b) If $\frak B$ is a principal ideal of $\frak A$, then
$\frak m(\frak B)\ge\frak m(\frak A)$.

(c) If $\frak B$ is an order-dense subalgebra of $\frak A$, then
$\frak m(\frak B)=\frak m(\frak A)$.

(d) If $\widehat{\frak A}$ is the Dedekind completion of $\frak A$, then
$\frak m(\widehat{\frak A})=\frak m(\frak A)$.

(e) If $D\subseteq\frak A$ is non-empty and $\sup D=1$, then
$\frak m(\frak A)=\min_{d\in D}\frak m(\frak A_d)$, where $\frak A_d$ is
the principal ideal generated by $d$.

(f) If $\frak A$ is the simple product of a non-empty family
$\familyiI{\frak A_i}$ of Boolean algebras, then
$\frak m(\frak A)=\min_{i\in I}\frak m(\frak A_i)$.

(g) Suppose that $\kappa$ and $\lambda$ are infinite cardinals such that
$\sat(\frak A)\le\cf\kappa$, $\lambda\le\kappa$ and
$\lambda<\frak m(\frak A)$.   Then $(\kappa,\lambda)$ is a
precaliber pair of $\frak A$.

\proof{{\bf (a)} Setting
$S=\{(a,b):a\in\frak A^+,\,a\Bsubseteq b\in\frak B\}$,
$S$ satisfies the conditions of 517C for
$P_0=(\frak A^+,\Bsupseteqshort)$
and $P_1=(\frak B^+,\Bsupseteqshort)$.  \Prf\ The
only non-trivial part is the check that if $Q$ is coinitial with $\frak B^+$
then $S^{-1}[Q]$ is coinitial with $\frak A^+$.   But $\sup Q=1$ in $\frak B$;
as $\frak B$ is regularly embedded in $\frak A$, $\sup Q=1$ in $\frak A$.  So if
$a\in\frak A^+$, there is a $b\in Q$ such that $a\Bcap b\ne 0$, and now
$a\Bcap b\in S^{-1}[Q]$ and $a\Bcap b\Bsubseteq a$.   As $a$ is arbitrary,
$S^{-1}[Q]$ is coinitial with $\frak A^+$.\ \QeD\  So
$\frak m(\frak B)\ge\frak m(\frak A)$.

\medskip

{\bf (b)} If $\frak A_a$ is the principal ideal generated by
$a\in\frak A^+=\frak A\setminus\{0\}$, we have

$$\eqalignno{\frak m(\frak A)
&=\frak m^{\downarrow}(\frak A^+)\le\frak m^{\downarrow}(\ocint{0,a})\cr
\displaycause{by 517Dc, inverted}
&=\frak m(\frak A_a).\cr}$$

\noindent On my definitions the trivial ideal $\{0\}$ also is a
principal ideal, but of course
$\frak m(\{0\})=\infty\ge\frak m(\frak A)$.

\medskip

{\bf (c)} Apply 517Da (inverted) to $\frak A^+$ and $\frak B^+$.

\medskip

{\bf (d)} This follows from (c), because $\frak A$ is order-dense in
$\widehat{\frak A}$.

\medskip

{\bf (e)} By (b), $\frak m(\frak A)\le\frak m(\frak A_d)$ for every $d$.
In the other direction, let $\Cal Q$ be a family of coinitial subsets of
$\frak A^+$ such that $\#(\Cal Q)<\min_{d\in D}\frak m(\frak A_d)$, and
take any $c\in\frak A^+$.   Then there is a $d\in D$ such that
$c\Bcap d\ne 0$.   For $Q\in\Cal Q$ set
$Q'=\{a\Bcap d:a\in Q\}\setminus\{0\}$;  then $Q'$ is coinitial with
$\frak A_d^+$.   Since $\#(\{Q':Q\in\Cal Q\})<\frak m(\frak A_d)$, there
is a downwards-linked set $R'\subseteq\frak A_d^+$ meeting every $Q'$
and containing $c\Bcap d$.   Set $R=\{a:a\in\frak A,\,a\Bcap d\in R'\}$;
then $R$ is a downwards-linked subset of $\frak A^+$ meeting every
member of $\Cal Q$ and containing $c$.   As $c$ and $\Cal Q$ are
arbitrary, $\frak m(\frak A)\ge\min_{d\in D}\frak m(\frak A_d)$.

\medskip

{\bf (f)} This is, in effect, a special case of (e), since we can
identify the $\frak A_i$ with principal ideals of $\frak A$ (315E).

\medskip

{\bf (g)} Apply 517Fa (inverted) to $\frak A^+$.
}%end of proof of 517I

\leader{517J}{Proposition} Let $X$ be a locally compact Hausdorff space,
and $\kappa$ a cardinal.   Then the following are equiveridical:

(i) $\kappa<\frak m(\RO(X))$, where $\RO(X)$ is the regular open algebra
of $X$;

(ii) $X\cap\bigcap\Cal G$ is dense in $X$ whenever $\Cal G$ is a family
of dense open subsets of $X$ and $\#(\Cal G)\le\kappa$;

(iii) $\kappa<n(H)$ for every non-empty open set $H\subseteq X$.

\proof{{\bf (i)$\Rightarrow$(iii)} Suppose that
$\kappa<\frak m(\RO(X))$.   Let $H\subseteq X$ be a non-empty open set
and $\Cal E$ a family of nowhere dense subsets of $H$ with
$\#(\Cal E)\le\kappa$.   Note that every member of $\Cal E$ is nowhere
dense in $X$.   Because $X$ is locally compact and regular, we have a
non-empty regular open set $H_0$ such that $K=\overline{H}_0$ is compact
and included in $H$.   For each $E\in\Cal E$, set
$\Cal G_E=\{G:G\in\RO(X)^+,\,\overline{G}\cap E=\emptyset\}$;  then
$\Cal G_E$ is coinitial with $\RO(X)^+$.   Because
$\kappa<\frak m^{\downarrow}(\RO(X)^+)$, there is a
centered $\Cal G\subseteq\RO(X)^+$ containing $H_0$ and meeting every
$\Cal G_E$.   But in this case
$\{K\}\cup\{\overline{G}:G\in\Cal G\}$ is a
family of closed sets in $X$ containing the compact set $K$ and with the
finite intersection property, so has non-empty intersection $F$, which
is included in $H\setminus\bigcup\Cal E$.   As $H$ and $\Cal E$ are
arbitrary, (iii) is true.

\medskip

{\bf (iii)$\Rightarrow$(ii)} This is easy.   If (iii) is true, $\Cal G$
is a family of dense open subsets of $X$ with $\#(\Cal G)\le\kappa$, and
$H\subseteq X$ is a non-empty open set, then
$\Cal E=\{H\setminus G:G\in\Cal G\}$ is a family of nowhere dense
subsets of $H$, so cannot cover $H$, and
$H\cap\bigcap\Cal G\ne\emptyset$.   As $\Cal G$ and $H$ are arbitrary,
(ii) is true.

\medskip

{\bf (ii)$\Rightarrow$(i)} Suppose that (ii) is true.   Take
$H\in\RO(X)^+$ and a family $\frak G$ of coinitial subsets of $\RO(X)^+$
with $\#(\frak G)\le\kappa$.   For each $\Cal G\in\frak G$,
$\bigcup\Cal G$ is a dense open subset of $X$.   Accordingly there is a
point $x\in H\cap\bigcap_{\Cal G\in\frak G}\bigcup\Cal G$.   Set
$R=\{G:G\in\RO(X),\,x\in G\}$.   Then $R$ is a
downwards-linked subset of $\RO(X)^+$ containing $H$ and meeting every
member of $\frak G$.   As $H$ and $\frak G$ are arbitrary,
$\kappa<\frak m(\RO(X))$.
}%end of proof of 517J

\leader{517K}{Corollary} Let $\frak A$ be a Boolean algebra with Stone
space $Z$.

(a) $\frak m(\frak A)=\frak m(\RO(Z))$.

(b) For any cardinal $\kappa$, the following are equiveridical:

\inset{(i) $\kappa<\frak m(\frak A)$;

(ii) $Z\cap\bigcap\Cal G$ is dense in $Z$ whenever $\Cal G$ is a family
of dense open subsets of $Z$ and $\#(\Cal G)\le\kappa$;

(iii) $\kappa<n(H)$ for every non-empty open set $H\subseteq Z$.}

\proof{{\bf (a)} $\frak A$ is isomorphic to the algebra of
open-and-closed subsets of $Z$, which is an order-dense subalgebra of
$\RO(Z)$ (314Ta).   So $\frak m(\frak A)=\frak m(\RO(Z))$ by 517Ic.

\medskip

{\bf (b)} now follows from 517J.
}%end of proof of 517K

\leader{517L}{}\cmmnt{ These identifications make the following
results easy.

\medskip

\noindent}{\bf Proposition} Let $\frak A$ be a Boolean algebra.

(a) $\wdistr(\frak A)\le\frak m(\frak A)$.

(b) If $\wdistr(\frak A)$ is a precaliber of $\frak A$ then
$\wdistr(\frak A)<\frak m(\frak A)$.

\proof{{\bf (a)} Let $Z$ be the Stone space of $\frak A$ and $\CalNwd$
the ideal of nowhere dense subsets of $Z$.   Then
$\wdistr(\frak A)=\add\CalNwd$ (514Be), while $\frak m(\frak A)$ is the
least cardinal of any subset of $\CalNwd$ covering a non-empty open
subset of $Z$, if there is one (517Kb).   Since no non-empty open
subset of $Z$ can belong to $\CalNwd$,
$\wdistr(\frak A)\le\frak m(\frak A)$.

\medskip

{\bf (b)} Because $\wdistr(\frak A)=\add\CalNwd\ge\omega$, it is a
regular infinite cardinal (513C(a-i)).   If
$\ofamily{\xi}{\wdistr(\frak A)}{G_{\xi}}$ is a family of dense open
subsets of $Z$, and $H\subseteq Z$ is open and not empty, then
$H_{\xi}=H\cap\interior(\bigcap_{\eta<\xi}G_{\eta})$ is
non-empty for every $\xi<\wdistr(\frak A)$.
So if also $\wdistr(\frak A)$ is a
precaliber of $\frak A$ and therefore of $Z$ (516Ha), 
there is a point $z$ of $Z$ such that
$\{\xi:z\in H_{\xi}\}$ has cardinal $\wdistr(\frak A)$ (516Qb) and is
therefore cofinal with $\wdistr(\frak A)$;  which means that
$z\in H\cap\bigcap_{\xi<\wdistr(\frak A)}G_{\xi}$.   Thus
$n(H)>\wdistr(\frak A)$;  as $H$ is arbitrary,
$\frak m(\frak A)>\wdistr(\frak A)$, by 517Kb again.
}%end of proof of 517L

\leader{517M}{}\cmmnt{ It is worth extracting an idea from the proofs
just above as a general result.

\medskip

\noindent}{\bf Proposition} Let $X$ be any topological space.   Then the
Nov\'ak number $n(X)$ of $X$\cmmnt{ (5A4Af)} is at most
$\sup\{\frak m(\RO(G)):G\subseteq X$ is open and not empty$\}$, where
$\RO(G)$ is the regular open algebra of $G$.

\proof{{\bf (a)} If there is a non-empty open subset $G$ of $X$ such
that $\frak m(\RO(G))=\infty$, the result is trivial;  suppose
otherwise.   Set
$\kappa=\sup\{\frak m(\RO(G)):G\subseteq X$ is open and not empty$\}$.
Then for any non-empty open set $G\subseteq X$ there is a family
$\ofamily{\xi}{\kappa}{E_{\xi}}$ of nowhere dense sets such that
$\#(\Cal E)\le\kappa$ and
$G\cap\interior(\bigcup_{\xi<\kappa}E_{\xi})\ne\emptyset$.   \Prf\ We
have a family $\ofamily{\xi}{\kappa}{\Cal Q_{\xi}}$ of order-dense
subsets of $\RO(G)^+$ and an $H\in\RO(G)^+$ such that there is no
downwards-directed family in $\RO(G)^+$ containing $H$ and meeting every
$\Cal Q_{\xi}$.   Set
$E_{\xi}=G\setminus\bigcup\Cal Q_{\xi}$ for each $\xi$;
then $E_{\xi}$ must be nowhere dense in the topological sense because
any open set meeting $G$ at all must meet some member of $\Cal Q_{\xi}$.
If $x\in H$, then $\Cal R=\{U:U\in\RO(G)$, $x\in U\}$ is a
downwards-directed family in $\RO(G)^+$ containing $H$, so does not meet
every $\Cal Q_{\xi}$, and there must be a $\xi<\kappa$ such that
$x\notin\bigcup\Cal Q_{\xi}$, that is, $x\in E_{\xi}$.   As $x$ is
arbitrary, $G\cap\interior(\bigcup_{\xi<\kappa}E_{\xi})\supseteq H$ is
not empty.\ \Qed

\medskip

{\bf (b)} Let $\familyiI{H_i}$ be a maximal disjoint family of non-empty
open sets in $X$ such that every $H_i$ can be covered by a family of at
most $\kappa$ nowhere dense sets.   By (a), $\bigcup_{i\in I}H_i$ is
dense.   For each $i\in I$, let $\ofamily{\xi}{\kappa}{E_{i\xi}}$ be a
family of nowhere dense sets covering $H_i$.   Set
$E_{\xi}=\bigcup_{i\in I}H_i\cap E_{i\xi}$ for each $\xi<\kappa$;  then
$E_{\xi}$ is nowhere dense (5A4Ea).   Also
$\bigcup_{\xi<\kappa}E_{\xi}=\bigcup_{i\in I}H_i$ is a dense open set,
so that $\{E_{\xi}:\xi<\kappa\}\cup(X\setminus\bigcup_{i\in I}H_i)$ is a
cover of $X$ by nowhere dense sets, and $n(X)\le\kappa$.   (Of course
$\kappa$ is infinite, by 517A, except in the trivial case $X=\emptyset$.)
}%end of proof of 517M

\leader{517N}{Corollary} If $\frak A$ is a Martin-number-homogeneous
Boolean algebra with Stone space $Z$, then $\frak m(\frak A)=n(Z)$.

\proof{ By 517Kb(i)$\Rightarrow$(iii), $\frak m(\frak A)\le n(Z)$.
In the other direction, given $a\in\frak A$, write $\widehat{a}$ for the
open-and-closed subset of $Z$ corresponding to $a$, and $\frak A_a$ for
the principal ideal generated by $a$.   If $G\subseteq Z$ is a non-empty
regular open set, let $a\in\frak A\setminus\{0\}$ be such that
$\widehat{a}\subseteq G$.   Then

$$\eqalignno{\frak m(\RO(G))
&\le\frak m(\RO(\widehat{a}))\cr
\displaycause{by 517Ib, because $\RO(\widehat{a})$ can be regarded as a
principal ideal of $\RO(G)$}
&=\frak m(\frak A_a)\cr
\displaycause{because we can identify $\widehat{a}$ with the Stone space
of $\frak A_a$, by 312T, and use 517Ka}
&=\frak m(\frak A).\cr}$$

\noindent By 517M, $n(Z)\le\frak m(\frak A)$.
}%end of proof of 517N

\leader{517O}{Martin cardinals (a)} For any class $\Cal P$ of partially
ordered sets, we have an associated cardinal

\Centerline{$\frak m^{\uparrow}_{\Cal P}
=\min\{\frak m^{\uparrow}(P):P\in\Cal P\}$.}

\noindent Much the most important of these is the cardinal

\Centerline{$\frak m=\min\{\frak m^{\uparrow}(P):P$ is upwards-ccc$\}$.}

\noindent Others of great interest are

\Centerline{$\frak p
=\min\{\frak m^{\uparrow}(P):P$ is $\sigma$-centered upwards$\}$,}

\Centerline{$\frak m_{\text{K}}
=\min\{\frak m^{\uparrow}(P):P$ satisfies Knaster's condition
upwards$\}$,}

\Centerline{$\frakmctbl=\min\{\frak m^{\uparrow}(P):P$ is a countable
partially ordered set$\}$.}

\noindent Two more which are worth examining are

\Centerline{$\frak m_{\sigma\text{-linked}}
=\min\{\frak m^{\uparrow}(P):P$ is $\sigma$-linked upwards$\}$,}

\Centerline{$\frak m_{\text{pc}\omega_1}
=\min\{\frak m^{\uparrow}(P):\omega_1$ is an up-precaliber of $P\}$.}

\wheader{517O}{0}{0}{0}{72pt}
\spheader 517Ob These cardinals are related as follows:

\def\tmphrule{\hskip0.4em\raise
2.5pt\hbox{\leaders\hrule\hskip1.7em\hfil}\hskip0.4em}
\def\tmpstrut{\vrule height10.5pt depth5.5pt width0pt}

$$\vbox{\offinterlineskip
\halign{\hfil#\hfil&\hfil#\hfil&\hfil#\hfil&\hfil#\hfil
  &\hfil#\hfil&\hfil#\hfil&\hfil#\hfil
  &\hfil#\hfil&\hfil#\hfil&\hfil#\hfil&\hfil#\hfil\cr
&\tmpstrut&&&$\frak m_{\sigma\text{-linked}}$&\tmphrule
  &$\frak p$&\tmphrule&$\frakmctbl$&\tmphrule&$\frak c$\cr
&\tmpstrut&&&\vrule&&\vrule\cr
$\tmpstrut\omega_1$&\tmphrule&$\frak m$&\tmphrule&$\frak m_{\text{K}}$
&\tmphrule&$\frak m_{\text{pc}\omega_1}$\cr
}}$$

\noindent The numbers here increase from bottom left to
top right;  that is,

\Centerline{$\omega_1\le\frak m\le\frak m_{\text{K}}
\le\frak m_{\text{pc}\omega_1}\le\frak p\le\frakmctbl\le\frak c$,}

\Centerline{$\frak m_{\text{K}}
\le\frak m_{\sigma\text{-linked}}\le\frak p$.}

\prooflet{
\noindent From 517A we see that $\omega_1\le\frak m$.   For the proof
that $\frakmctbl\le\frak c$, see 517P below.   As for the intermediate
inequalities involving Martin cardinals, they follow directly from
inclusions between the corresponding classes of partially ordered set.
These are all immediate from the definitions;  I give references to the
general results of this chapter which cover the relevant facts, as
follows.

\medskip

\quad{\bf (i)} Every partially ordered set satisfying Knaster's
condition upwards is ccc.   \prooflet{(If $(\omega_1,2)$ is an upwards
precaliber pair of $P$, then $\sat^{\uparrow}(P)\le\omega_1$ (516Ka).)}

\medskip

\quad{\bf (ii)} If $\omega_1$ is an up-precaliber of $P$, then $P$
satisfies Knaster's condition upwards.   \prooflet{(If
$\triplepc{\omega_1}{\omega_1}{\omega}$ is a triple precaliber of $P$,
so is $\triplepc{\omega_1}{2}{\omega}$, by 516Ba.)}

\medskip

\quad{\bf (iii)} If $P$ is $\sigma$-linked upwards, it satisfies
Knaster's condition upwards.  \prooflet{(As
$\omega_1>\max(\omega,\omega,\link^{\uparrow}(P))$,
$\triplepc{\omega_1}{\omega_1}{3}$ is an upwards precaliber triple of
$P$ (516Kb), so $\triplepc{\omega_1}{2}{3}$ and
$\triplepc{\omega_1}{2}{\omega}$ also are, by 516Ba again.)}

\medskip

\quad{\bf (iv)} If $P$ is $\sigma$-centered upwards, it is
$\sigma$-linked upwards.
\prooflet{($\link(P)\le\link_{<\omega}(P)$, by 511Hb.)}

\medskip

\quad{\bf (v)} If $P$ is $\sigma$-centered upwards, $\omega_1$ is an
up-precaliber of $P$.   \prooflet{(As
$\omega_1>\max(\omega,\omega,\link_{<\omega}^{\uparrow}(P))$,
$(\omega_1,\omega_1)$ is an upwards precaliber pair
of $P$, by 516Kb again.)}

\medskip

\quad{\bf (vi)} If $P$ is countable, it is $\sigma$-centered upwards.
\prooflet{(Singleton subsets are centered.)}
}%end of prooflet

\spheader 517Oc\cmmnt{ I should note a special feature of the bottom
row of this diagram.}   In the chain
$\omega_1\le\frak m\le\frak m_{\text{K}}\le\frak m_{\text{pc}\omega_1}$,
at most one of the inequalities can be strict.   \prooflet{\Prf\ Suppose
that $P$ is upwards-ccc and
$\frak m^{\uparrow}(P)>\omega_1$.   Then $\omega_1$ is an up-precaliber
of $P$ (517Fb), so
$\frak m^{\uparrow}(P)\ge\frak m_{\text{pc}\omega_1}$.   So if, for
instance, $\frak m_{\text{K}}>\omega_1$ and $P$ satisfies Knaster's
condition upwards, $\frak m^{\uparrow}(P)>\omega_1$ and
$\frak m^{\uparrow}(P)\ge\frak m_{\text{pc}\omega_1}$;  as $P$ is
arbitrary, $\frak m_{\text{K}}\ge\frak m_{\text{pc}\omega_1}$.
Similarly, if $\frak m>\omega_1$ then
$\frak m=\frak m_{\text{pc}\omega_1}$.\ \Qed}

\spheader 517Od Now {\bf Martin's Axiom} is the assertion

\Centerline{`$\frak m=\frak c$'.}

\noindent From the diagram above, we see that this is a consequence of
the continuum hypothesis (`$\omega_1=\frak c$'), and fixes all the
intermediate cardinals.

\cmmnt{\spheader 517Oe All the partially ordered sets considered in
(b) are ccc, which is why $\frak m$ appears at bottom left.   The same
idea can be applied to larger classes, e.g.\ `proper' or
`stationary-set-preserving' partial orders.   For the moment I will not
even define these classes;  I mention them only for the sake of readers
who are already familiar with them and may be expecting a reference
here.   There is an important difference, however, in that if the
cardinal which we might call

\Centerline{$\frak m_{\text{proper}}
=\min\{\frak m^{\uparrow}(P):P$ is upwards-proper$\}$}

\noindent is greater than $\omega_1$, then
$\frak c=\frak m_{\text{proper}}=\omega_2$ ({\smc Veli\v{c}kovi\'c 92}, or
{\smc Moore 05});  so that we have
only to say whether the Proper Forcing Axiom
(`$\frak m_{\text{proper}}>\omega_1$') is true or false to 
determine the value of $\frak m_{\text{proper}}$.
}%end of comment

\leader{517P}{}\cmmnt{ All the cardinals here have special features,
but the ones I will concentrate on just now are the two largest,
$\frakmctbl$ and $\frak p$.

\medskip

\noindent}{\bf Proposition} (a) $\omega_1\le\frakmctbl\le\frak c$.

(b) Let $\frak A$ be a Boolean algebra with countable $\pi$-weight.
If $\frak A$ is purely atomic, then $\frak m(\frak A)=\infty$;
otherwise, $\frak m(\frak A)=\frakmctbl$.

(c) If $P$ is a partially ordered set of countable cofinality and
$\frak m^{\uparrow}(P)$ is not $\infty$, then
$\frak m^{\uparrow}(P)=\frakmctbl$.

(d)(i) Let $X$ be a topological space such that its category algebra is
atomless and has countable $\pi$-weight.   Then $n(X)\le\frakmctbl$.

\quad (ii) If $X$ is a non-empty locally compact Hausdorff space with
countable $\pi$-weight and no isolated points, then $n(X)=\frakmctbl$.

\quad (iii) If $X$ is a non-empty Polish space with no isolated points,
then $n(X)=\frakmctbl$.

\proof{ Let $\frak B$ be the algebra of open-and-closed subsets of
$\{0,1\}^{\Bbb N}$.   The argument will go more smoothly if I prove
(a)-(c) with $\frak m(\frak B)$ in place of $\frakmctbl$, and at an
appropriate moment point out that I have shown that the two are equal.

\medskip

{\bf (a)} $\frak m(\frak B)=\frak m^{\downarrow}(\frak B^+)$ is
uncountable, by 517A.    To see that
$\frak m(\frak B)\le\frak c$, let $\Cal Q$ be the set of all
coinitial subsets of $\frak B^+$;  then $\#(\Cal Q)\le\frak c$ because
$\frak B$ is countable.   \Quer\ If $\frak m(\frak B)>\frak c$, there
must be a linked set $R\subseteq\frak B^+$ meeting every member of
$\Cal Q$.   But now consider $Q=\frak B^+\setminus R$.   If
$a\in\frak B^+$, there are disjoint non-zero $a'$, $a''\Bsubseteq a$
which cannot both belong to $R$, so at least one belongs to $Q$.   But
this means that $Q$ is order-dense in $\frak B$ and ought to meet $R$.\
\BanG\  (Compare 517E.)

\medskip

{\bf (b)(i)} If $\frak A$ is purely atomic, $\frak m(\frak A)=\infty$,
by 511If.

\medskip

\quad{\bf (ii)} Suppose that $\frak A$ is not purely atomic.   Because
$\pi(\frak A)$ is countable, there is a countable order-dense set
$C\subseteq\frak A$.   Let $\frak C$ be the subalgebra of $\frak A$
generated by $C$, so that $\frak C$ is a countable order-dense
subalgebra of $\frak A$, and is not purely atomic.
Consider the free product $\frak C\otimes\frak B$ (315N).
This is a countable atomless Boolean algebra (use 315O), 
so is isomorphic to
$\frak B$ (316M).   Also we have an injective
order-continuous Boolean homomorphism from $\frak C$ to
$\frak C\otimes\frak B$ (315K), so that $\frak C$ is isomorphic to a
regularly embedded subalgebra of $\frak B$ and
$\frak m(\frak C)\ge\frak m(\frak B)$ (517Ia).

Next, $\frak C$ has a non-trivial atomless principal ideal $\frak C_a$
say.   Because $\frak C_a$ is still countable, it is itself isomorphic
to $\frak B$.   So 517Ib tells us that
$\frak m(\frak C)\le\frak m(\frak C_a)=\frak m(\frak B)$, and
$\frak m(\frak C)=\frak m(\frak B)$.

Finally, $\frak m(\frak A)=\frak m(\frak C)$ by 517Ic.

\medskip

{\bf (c)} We know that
$\frak m^{\uparrow}(P)=\frak m(\RO^{\uparrow}(P))$  (517Db) and that
$\pi(\RO^{\uparrow}(P))\le\cf P$ (514Nb) is countable.   Let
$D\subseteq\RO^{\uparrow}(P)^+$ be a countable order-dense set, and
$\frak A$ the subalgebra of $\RO^{\uparrow}(P)$ generated by $D$.   Then
$\frak m(\RO^{\uparrow}(P))=\frak m(\frak A)$ by 517Ic, and $\frak A$ is
countable.   By (b), $\frak m^{\uparrow}(P)=\frak m(\frak A)$ is either
$\infty$ or $\frak m(\frak B)$;  since the former is ruled out by
hypothesis, we are left with the latter.

What this shows, however, is that

$$\eqalignno{\frak m(\frak B)
&\le\min\{\frak m^{\uparrow}(P):\cf P\le\omega\}
\le\min\{\frak m^{\uparrow}(P):\#(P)\le\omega\}\cr
&=\frakmctbl
\le\frak m^{\downarrow}(\frak B^+)
=\frak m(\frak B)\cr}$$

\noindent so that $\frakmctbl=\frak m(\frak B)$ and we can rewrite the
results so far in the forms given in the statement of the proposition.

\medskip

{\bf (d)(i)} Consider first the case in which $X$ is a non-empty Baire
space, so that its category algebra is isomorphic to $\RO(X)$ (514If).
Since $\RO(X)$ is atomless and not $\{\emptyset\}$, and in particular is
not purely atomic, but has countable $\pi$-weight,
$\frak m(\RO(X))=\frakmctbl$, by (b).    The same applies to any
non-empty open subset $G$ of $X$, recalling that the category algebra of
$G$ can be identified with a principal ideal of the category algebra of
$X$ (514Id).   So $n(X)\le\frakmctbl$ by 517M.

If $X$ is not a Baire space, then it has a smallest comeager regular
open set $H$, which is itself a Baire space (4A3Ra), and $X$ and $H$
have isomorphic category algebras (514Ic), so we see from the argument
just above that $n(H)\le\frakmctbl$.   But $X\setminus H$ is a countable
union of nowhere dense subsets of $X$, and every subset of $H$ which is
nowhere dense in $H$ is also nowhere dense in $X$, so
$n(X)\le\max(\omega,n(H))\le\frakmctbl$.

\medskip

\quad{\bf (ii)} Because $X$ is Hausdorff and has no isolated points,
$\RO(X)$ is atomless.   Next, $\pi(\RO(X))\le\pi(X)$ is countable
(514H(b-i)), and $\RO(X)$ is isomorphic to the category algebra of $X$, by
Baire's theorem.   So the first part of the proof of (i) tells us that
$n(X)\le\frakmctbl=\frak m(\RO(X))$.   From 517J we now see that

\Centerline{$\frak m(\RO(X))
=\min\{n(H):H\subseteq X$ is a non-empty open set$\}
\le n(X)$,}

\noindent so $n(X)=\frakmctbl$ exactly.

\medskip

\quad{\bf (iii)} Now suppose that $X$ is a non-empty Polish space
without isolated points.   As in (ii), the category algebra of $X$ is
atomless and has countable $\pi$-weight, so $n(X)\le\frakmctbl$.   In
the other direction, suppose that $\kappa<\frakmctbl$ and that
$\ofamily{\xi}{\kappa}{E_{\xi}}$ is a family of nowhere dense subsets of
$X$.   Let $\rho$ be a metric defining the topology of $X$ under which
$X$ is complete, and $\Cal U$ a countable base for the topology of $X$,
not containing $\emptyset$.   For $\xi<\kappa$, set
$\Cal Q_{\xi}=\{U:U\in\Cal U$, $\overline{U}\cap E_{\xi}=\emptyset\}$;
for $n\in\Bbb N$ set $\Cal Q'_n=\{U:U\in\Cal U$, $\diam U\le 2^{-n}\}$.
Then every $\Cal Q_{\xi}$ and every $\Cal Q'_n$ is coinitial with
$\Cal U$.   By (c) above,

\Centerline{$\frak m^{\downarrow}(\Cal U)\ge\frakmctbl
>\max(\kappa,\omega)$,}

\noindent so there is a downwards-directed $\Cal V\subseteq\Cal U$
meeting every $\Cal Q_{\xi}$ and every $\Cal Q_n$.   Now
$\{\overline{V}:V\in\Cal V\}$ is a downwards-directed set containing
sets of arbitrarily small diameter, so generates a Cauchy filter and
(because $(X,\rho)$ is complete) has non-empty intersection.   Take any
$x\in\bigcap_{V\in\Cal V}\overline{V}$.   Because $\Cal V$ meets every
$\Cal Q_{\xi}$, $x\notin\bigcup_{\xi<\kappa}E_{\xi}$ and
$\ofamily{\xi}{\kappa}{E_{\xi}}$ does not cover $X$.  As
$\ofamily{\xi}{\kappa}{E_{\xi}}$ is arbitrary, $n(X)\ge\frakmctbl$ and
the two are equal.
}%end of proof of 517P

\leader{517Q}{Lemma} If $P$ is any partially ordered set,
$\frak m^{\uparrow}(P)\ge\min(\add_{\omega}P,\frakmctbl)$.

\proof{ (For the definition of $\add_{\omega}P$, see 513H.)   Take
$\kappa<\min(\add_{\omega}P,\frakmctbl)$,
$p_0\in P$ and a family $\ofamily{\xi}{\kappa}{Q_{\xi}}$ of cofinal
subsets of $P$.   Choose $\sequencen{R_n}$ and
$\langle Q_{n\xi}\rangle_{n\in\Bbb N,\xi<\kappa}$ as follows.
$R_0=\{p_0\}$.   Given that $R_n\subseteq P$ is countable, then for each
$\xi<\kappa$ choose a countable set $Q_{n\xi}\subseteq Q_{\xi}$ such
that for every $p\in R_n$ there is a $q\in Q_{n\xi}$ such that $p\le q$.
Now, because $\add_{\omega}P>\kappa$ (and, of course,
$\add_{\omega}P>\omega$, as noted in 513Ib), we can find a countable set
$R_{n+1}\subseteq P$ such that whenever
$q\in\bigcup_{\xi<\kappa}Q_{n\xi}$ there is an $r\in R_{n+1}$ such that
$q\le r$.   This will ensure that whenever $p\in R_n$ and $\xi<\kappa$
there are $q\in Q_{\xi}$ and $p'\in R_{n+1}$ such that
$p\le q\le p'$.

At the end of the induction, consider the countable partially ordered
set $R=\bigcup_{n\in\Bbb N}R_n$.   For $\xi<\kappa$ set

\Centerline{$Q'_{\xi}=\{r:r\in R,\Exists q\in Q_{\xi},\,q\le r\}$;}

\noindent then $Q'_{\xi}$ is cofinal with $R$.   Because
$\kappa<\frakmctbl$, there is an upwards-linked subset
$S$ of $R$ meeting every $Q'_{\xi}$ and containing $p_0$.   But now
$\{p:p\in P,\Exists s\in S,\,p\le s\}$ is an upwards-linked subset of
$p$ containing $p_0$ and meeting every $Q_{\xi}$.   As $p_0$ and
$\ofamily{\xi}{\kappa}{Q_{\xi}}$ are arbitrary,
$\frak m^{\uparrow}(P)\ge\min(\add_{\omega}P,\frakmctbl)$.
}%end of proof of 517Q

\leader{517R}{Proposition} (a)\cmmnt{ (`Booth's Lemma';  see
{\smc Booth 70}}) Suppose that $\Cal A$ is a
family of subsets of $\Bbb N$ such that $\#(\Cal A)<\frak p$ and
$\bigcap\Cal J$ is infinite for every finite $\Cal J\subseteq\Cal A$.
Then there is an infinite $I\subseteq\Bbb N$ such that $I\setminus A$ is
finite for every $A\in\Cal A$.

(b) $2^{\kappa}\le\frak c$ for every $\kappa<\frak p$.

(c)\dvAformerly{5{}55K} Suppose that $X$ is a set and $\#(X)<\frak p$.
Then there is a
countable set $\Cal A\subseteq\Cal PX$ such that $\Cal PX$ is the
$\sigma$-algebra generated by $\Cal A$.

\proof{{\bf (a)} Let $P$ be $[\Bbb N]^{<\omega}\times[\Cal A]^{<\omega}$,
ordered by saying that $(K,\Cal J)\le(K',\Cal J')$ if
$K\subseteq K'\subseteq K\cup\bigcap\Cal J$ and
$\Cal J\subseteq\Cal J'$.   If $(K,\Cal J)\le(K',\Cal J')\le(K'',\Cal J'')$
then of course $K\subseteq K''$ and $\Cal J\subseteq\Cal J''$;  also

\Centerline{$K''\subseteq K'\cup\bigcap\Cal J'
\subseteq K\cup\bigcap\Cal J\cup\bigcap\Cal J'
\subseteq K\cup\bigcap\Cal J$.}

\noindent So $\le$ is a partial ordering of $P$.   For any
$K\in[\Bbb N]^{<\omega}$, $\{(K,\Cal J):J\in[\Cal A]^{<\omega}\}$ is
upwards-centered;  so $P$ is $\sigma$-centered upwards.

For each $A\in\Cal A$, set
$Q_A=\{(K,\Cal J):(K,\Cal J)\in P$, $A\in\Cal J\}$;  since
$(K,\Cal J)\le(K,\Cal J\cup\{A\})$ whenever $(K,\Cal J)\in P$, $Q_A$ is
cofinal with $P$.   For $n\in\Bbb N$, set
$Q'_n=\{(K,\Cal J):(K,\Cal J)\in P$, $K\not\subseteq n\}$.   If
$(K,\Cal J)\in P$, $\bigcap\Cal J$ must be infinite, and
there is an $m\in\Bbb N\cap\bigcap\Cal J\setminus n$;  now
$(K,\Cal J)\le(K\cup\{m\},\Cal J)\in Q'_n$.   So $Q'_n$ is cofinal with
$P$.

Because $\max(\omega,\#(\Cal A))<\frak p$ (517Ob),
there is an upwards-linked
$R\subseteq P$ meeting every $Q_A$ and every $Q'_n$.   Set
$I=\bigcup\{K:(K,\Cal J)\in R\}$.   If $n\in\Bbb N$, there is a
$(K,\Cal J)\in R\cap Q'_n$;  now $K\not\subseteq n$ and $K\subseteq I$, so
$I\not\subseteq n$;  as $n$ is arbitrary, $I$ is infinite.   If
$A\in\Cal A$, there is $(K_0,\Cal J_0)\in R\cap Q_A$.   \Quer\ If
$I\not\subseteq K_0\cup A$, there is a $(K,\Cal J)\in R$ such that
$K\not\subseteq K_0\cup A$.   Now there is a $(K',\Cal J')\in P$ such
that $(K,\Cal J)\le(K',\Cal J')$ and $(K_0,\Cal J_0)\le(K',\Cal J')$.   But
in this case

\Centerline{$K\subseteq K'\subseteq K_0\cup\bigcap\Cal J_0
\subseteq K_0\cup A$.\ \Bang}

\noindent So $I\setminus A\subseteq K_0$ is finite.   As $A$ is arbitrary,
we have a suitable $I$.

\medskip

{\bf (b)} We may suppose that $\kappa$ is infinite.   By 515H, or
otherwise, there is a Boolean-independent family
$\ofamily{\xi}{\kappa}{J_{\xi}}$ in $\Cal P\Bbb N$.
Note that
$I=\bigcap_{\xi\in K}J_{\xi}\setminus\bigcup_{\xi\in L}J_{\xi}$ must be
infinite whenever $K$, $L\subseteq\kappa$ are disjoint finite sets, because
$\family{\xi}{\kappa\setminus(K\cup L)}{I\cap J_{\xi}}$ is
Boolean-independent.   For $C\subseteq\kappa$ set

\Centerline{$\Cal A_C
=\{J_{\xi}:\xi\in C\}\cup\{\Bbb N\setminus J_{\xi}:
  \xi\in\kappa\setminus C\}$.}

\noindent By (a), there is an infinite $I_C\subseteq\Bbb N$ such that
$I_C\setminus A$ is finite for every $A\in\Cal A_C$.   If
$C$, $D\subseteq\kappa$ and $\xi\in C\setminus D$, then
$I_C\setminus J_{\xi}$ and $I_D\cap J_{\xi}$ are
finite, so $I_C\cap I_D$ is finite and $I_C\ne I_D$.   Thus
$C\mapsto I_C$ is injective and $2^{\kappa}\le\frak c$.

\medskip

{\bf (c)} Let
$\family{x}{X}{I_x}$ be a family of infinite subsets of $\Bbb N$ such that
$I_x\cap I_y$ is finite for all distinct $x$, $y\in X$ (5A1Fa).   Set
$A_n=\{x:n\in I_x\}$ for $n\in\Bbb N$.

Take any $A\subseteq X$ and set
$P_A=\Fn_{<\omega}(\Bbb N;\{0,1\})\times[X\setminus A]^{<\omega}$,
partially ordered by saying that

\doubleinset{
$(f,J)\le(f',J')$ if $f'$ extends $f$, $J'\supseteq J$ and
whenever $x\in J$ and
$i\in I_x\cap\dom f'\setminus\dom f$, then $f'(i)=0$.}

\noindent Then $P_A$ is $\sigma$-centered upwards because
$\{(f,J):J\in[X\setminus A]^{<\omega}\}$ is upwards-centered for every
$f\in\Fn_{<\omega}(\Bbb N;\{0,1\})$.   For $x\in A$ and $m\in\Bbb N$ set

\Centerline{$Q_{xm}=\{(f,J):(f,J)\in P_A$, $f(i)=1$ for some
$i\in I_x\setminus m\}$;}

\noindent for $x\in X\setminus A$ set

\Centerline{$Q'_x=\{(f,J):(f,J)\in P_A$, $x\in J\}$.}

\noindent Then every $Q_{xm}$ and every $Q'_x$ is cofinal with $P_A$.
Because $\#(X)<\frak p$, there is an upwards-directed $R\subseteq P_A$
meeting every $Q_{xm}$ and every $Q'_x$.   Set
$L=\bigcup_{(f,J)\in R}\{i:f(i)=1\}$.   Now

\inset{----- if $x\in A$ and $m\in\Bbb N$ then $L\cap I_x\setminus m$ is
non-empty, so $L\cap I_x$ is infinite,

----- if $x\in X\setminus A$, there is a pair $(f_0,J_0)\in R$ such that
$x\in J_0$;  now $f(i)=0$ whenever $(f,J)\in R$ and 
$i\in\dom f\setminus\dom f_0$, so $L\cap I_x\subseteq\dom f_0$.}

\noindent Accordingly

\Centerline{$A=\{x:x\in X$, $I_x\cap L$ is infinite$\}
=\bigcap_{n\in\Bbb N}\bigcup_{m\in L\setminus n}A_m$}

\noindent belongs to the $\sigma$-algebra generated by
$\Cal A=\{A_n:n\in\Bbb N\}$, and we have a suitable family.
}%end of proof of 517R

\leader{517S}{Proposition} Let $P$ be a partially ordered set which
satisfies Knaster's condition upwards.   If $A\subseteq P$ and
$\#(A)<\frak m_{\text{K}}$, then $A$ can be covered by a sequence of
upwards-directed subsets of $P$.

\proof{ By 516P, the upwards finite-support product $P^*$ of countably
many copies of $P$ also satisfies Knaster's condition upwards.   So we
can use 517Ha.
}%end of proof of 517S

\exercises{\leader{517X}{Basic exercises (a)}
%\spheader 517Xa
Let $P$ be a partially ordered set and $\kappa$ a
cardinal.   Show that the following are equiveridical:  (i)
$\kappa<\frak m^{\uparrow}(P)$;  (ii) whenever $p_0\in P$ and $\Cal Q$
is a family of cofinal subsets of $P$ with $\#(\Cal Q)\le\kappa$, there
is an upwards-centered subset of $P$ which contains $p_0$ and meets
every member of $\Cal Q$;
(iii) whenever $p_0\in P$ and $\Cal Q$ is a family of up-open cofinal
subsets of $P$ with $\#(\Cal Q)\le\kappa$, there is an upwards-centered
subset of $P$ which contains $p_0$ and meets every member of $\Cal Q$;
(iv) whenever $p_0\in P$ and $\Cal A$ is a family of maximal
up-antichains in $P$ with $\#(\Cal A)\le\kappa$, there is an
upwards-centered subset of $P$ which contains $p_0$ and meets every
member of $\Cal A$.
%517B

\spheader 517Xb Let $P$ be a partially ordered set and $A$ a maximal
up-antichain in $P$.   Show that

\Centerline{$\frak m^{\uparrow}(P)
=\min_{p\in A}\frak m^{\uparrow}(\coint{p,\infty})$.}
%517D

\spheader 517Xc(i) Let $\frak A$ be a Boolean algebra, not $\{0\}$.
For $a\in\frak A$ let $\frak A_a$ be the corresponding principal ideal.
Show that there is an $a\in\frak A^+$ such that
$\frak m(\frak A_b)=\frak m(\frak A_a)$ whenever $0\ne b\Bsubseteq a$.
(ii) Show that any Dedekind complete Boolean algebra is isomorphic to a
simple product of Martin-number-homogeneous Boolean algebras.
%517Xb 517D

\sqheader 517Xd Let $P$ be a partially ordered set.   Show that
$\frak m^{\uparrow}(P)=\infty$ iff
$\{p:\coint{p,\infty}$ is upwards-linked$\}$ is cofinal with $P$.
%517D

\spheader 517Xe Let $\familyiI{\frak A_i}$ be a family of Boolean
algebras and $\frak A$ its free product;  suppose that $\kappa$ is a
regular uncountable cardinal such that
$\sat(\frak A_i)\le\kappa<\frak m(\frak A_i)$ for every $i\in I$.   Show
that $\sat(\frak A)\le\kappa$.
%517I 517G

\spheader 517Xf Let $\frak A$ be a Boolean algebra and $\frak C$ the
free product of a sequence of copies of $\frak A$.   Suppose that
$\kappa<\frak m(\frak C)$.   (i) Show if $A\in[\frak A^+]^{\le\kappa}$
then $A$ can be covered by a sequence of centered subsets of
$\frak A^+$.   (ii) Show that if $\cf\kappa\ge\omega_1$ then $\kappa$ is
a precaliber of $\frak A$.
%517I 517H

\spheader 517Xg Let $\frak A$ be a Boolean algebra and $Z$ its Stone
space.   Show that $\frak m(\frak A)
=\min\{n(Y):Y\subseteq Z$ is a non-meager set with the
Baire property$\}$.
%517K

\sqheader 517Xh Let $P$ be a non-empty partially ordered set and $P^*$
the upwards finite-support product of a sequence of copies of $P$.
Show that if $\frak m^{\uparrow}(P^*)>\omega_1$ then $P$ must be
upwards-ccc.
%517H

\spheader 517Xi Let $X$ be any topological space.   Show that
$\frak m(\RO(X))\ge\min\{n(G):G\subseteq X$ is a non-empty open set$\}$.
%517J

\spheader 517Xj Let $X$ be a locally compact Hausdorff space such that
$\RO(X)$ is Martin-number-homogeneous.   Show that
$\frak m(\RO(X))=n(X)$.
%517N 517J

\spheader 517Xk(i) Let $P$ be a partially ordered set which is
$\sigma$-linked upwards.   Show that if $A\subseteq P$ and
$\#(A)<\frak m_{\sigma\text{-linked}}$,
then $A$ can be covered by a sequence of
upwards-directed subsets of $P$.
(ii) Let $P$ be a partially ordered set such that $\omega_1$ is an
up-precaliber of $P$.   Show that if $A\subseteq P$ and
$\#(A)<\frak m_{\text{pc}\omega_1}$,
then $A$ can be covered by a sequence of
upwards-directed subsets of $P$.
(iii) Let $P$ be a partially ordered set which is
$\sigma$-centered upwards.   Show that if $A\subseteq P$ and
$\#(A)<\frak p$, then $A$ can be covered by a sequence of
upwards-directed subsets of $P$.
%517S

\leader{517Y}{Further exercises (a)}
%\spheader 517Ya
For a partially ordered set $P$,
write $A_P$ for the family of upwards-linked subsets of $P$, $B_P$ for
the family of cofinal subsets of $P$, and $T_P$ for
$\{(R,Q):R\in A_P,\,Q\in B_P,\,R\cap Q=\emptyset\}$.   (i) Show that
$\frak m^{\uparrow}(P)=\min_{p\in P}
\cov(A_{\coint{p,\infty}},T_{\coint{p,\infty}},B_{\coint{p,\infty}})$.
(ii) Show that if $Q$ is another partially ordered set and there is a
relation $S\subseteq P\times Q$ with the properties described in 517C,
then for every $q\in Q$ there is a $p\in P$ such that
$(A_{\coint{p,\infty}},T_{\coint{p,\infty}},B_{\coint{p,\infty}})
\prGT(A_{\coint{q,\infty}},T_{\coint{q,\infty}},B_{\coint{q,\infty}})$.
%517C

\spheader 517Yb Show that for every infinite regular
cardinal $\kappa$ there is a partially ordered set with Martin number
$\kappa^+$.
% \bigcup_{\xi<\kappa}(\kappa^+)^{\xi}

\spheader 517Yc Show that
$\frak m(\Cal P\Bbb N/[\Bbb N]^{<\omega})\ge\frak p$.
% order-dense sets lead to mad families
}%end of exercises

%Note:  \cov\Cal N  is a Martin number

\endnotes{
\Notesheader{517} The study of `Martin numbers' is a natural extension
of investigations into consequences of Martin's axiom.   Most of the
results here are straightforward
expressions of techniques developed for deducing consequences from
$\frak m=\frak c$ or
$\frak m>\omega_1$.   In particular, 517F, 517G and 517H correspond to
the now-classical theorems that if $\frak m>\omega_1$ then $\omega_1$ is
a precaliber of every ccc partially ordered set, the product of any
family of ccc topological spaces is ccc, and a ccc partially ordered set
with cardinal $\omega_1$ is a countable union of directed sets (see
{\smc Fremlin 84a}, \S41).
For those familiar with the use of Martin's axiom there are no surprises
here, though some refinements in the arguments are necessary.   The
cardinal $\frakmctbl$ is probably most commonly known as the Nov\'ak
number of $\Bbb R$ (517Pd), the covering number of the ideal of meager
subsets of $\Bbb R$.   In countable partially ordered sets, most of the
arguments above short-circuit to some degree;  precalibers become
trivial, finite-support products are automatically ccc, and directed
sets have cofinal totally ordered subsets, so that the ideas take on new
colours.

In {\smc Fremlin 84a} I found that focusing on the cardinals
$\frak p$, $\frak m_{\text{K}}$ and $\frak m$
broke the arguments up into reasonably balanced chapters.   Within the
chapter on $\frak m_{\text{K}}$, however, there is a natural division
between
arguments applying to $\frak m_{\text{pc}\omega_1}$ and those applying
to $\frak m_{\sigma\text{-linked}}$, which in the present book I intend
to make explicit.   The notation $\frak p$ is the standard name for the
cardinal $\frak m_{\sigma\text{-centered}}$;  
its special position comes in part from the fact that it had been
studied under a different, combinatorial, definition for a decade before
M.G.Bell showed that it could also be described by the definition here
({\smc Bell 81}, or {\smc Fremlin 84a}, 14C).

}%end of notes

\discrpage

