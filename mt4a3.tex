\frfilename{mt4a3.tex}
\versiondate{11.10.07}
\copyrightdate{2007}

\def\chaptername{Appendix}
\def\sectionname{Topological $\sigma$-algebras}

\def\tildeClll{\hbox{$\tilde C$
  \hskip-4.5pt\raise4pt\hbox{\char'23
  \hskip-1.7pt\vrule height6.3pt width.5pt depth0pt
  \hskip.7pt\vrule height6.3pt width.5pt depth0pt
  \hskip-2.5pt\lower5pt\hbox{\char'23}}}}

\newsection{4A3}

I devote a section to some $\sigma$-algebras which can be defined on
topological spaces.   While `measures' will not be mentioned here, the
manipulation of these $\sigma$-algebras is an essential part of the
technique of measure theory, and I will give proofs and exercises as if
this were part of the main work.   I look at Borel $\sigma$-algebras
(4A3A-4A3J), % 4A3A 4A3B 4A3C 4A3D 4A3E 4A3F 4A3G 4A3H 4A3I 4A3J
Baire $\sigma$-algebras (4A3K-4A3P), % 4A3K 4A3L 4A3M 4A3N 4A3O 4A3P
Baire-property algebras (4A3Q, 4A3R), cylindrical $\sigma$-algebras
on linear spaces (4A3T-4A3V) % 4A3T 4A3U 4A3V
and spaces of \cadlag\ functions (4A3W).

\leader{4A3A}{Borel sets} If $(X,\frak T)$ is a topological space, the
{\bf Borel $\sigma$-algebra} of $X$ is the $\sigma$-algebra $\Cal B(X)$
of subsets
of $X$ generated by $\frak T$.   Its elements are the {\bf Borel sets}
of $X$.   If $(Y,\frak S)$ is another topological space with Borel
$\sigma$-algebra $\Cal B(Y)$, a function $f:X\to Y$ is
{\bf Borel measurable} if
$f^{-1}[H]\in\Cal B(X)$ for every $H\in\frak S$, and is a
{\bf Borel isomorphism} if it is a bijection and
$\Cal B(Y)=\{F:F\subseteq Y,\,f^{-1}[F]\in\Cal B(X)\}$\cmmnt{, that
is, $f$ is an isomorphism between the structures $(X,\Cal B(X))$ and
$(Y,\Cal B(Y))$}.
%4@11 %4@33
%Engelking 1.3

\leader{4A3B}{$(\Sigma,\text{\rm{T}})$-measurable
functions }\cmmnt{It is time I put the following idea into bold type.

\medskip

}{\bf (a)} Let $X$ and $Y$ be sets, with $\sigma$-algebras
$\Sigma\subseteq\Cal PX$ and
$\Tau\subseteq\Cal PY$.   A function $f:X\to Y$ is
$(\Sigma,\Tau)${\bf-measurable} if $f^{-1}[F]\in\Sigma$ for every
$F\in\Tau$.

\spheader 4A3Bb If $\Sigma$, $\Tau$ and $\Upsilon$ are $\sigma$-algebras
of subsets of $X$, $Y$ and $Z$ respectively, and $f:X\to Y$ is
$(\Sigma,\Tau)$-measurable while $g:Y\to Z$ is
$(\Tau,\Upsilon)$-measurable, then $gf:X\to Z$ is
$(\Sigma,\Upsilon)$-measurable.  \prooflet{ (If $H\in\Upsilon$,
$g^{-1}[H]\in\Tau$ so $(gf)^{-1}[H]=f^{-1}[g^{-1}[H]]\in\Sigma$.)}

\spheader 4A3Bc Let $\familyiI{X_i}$ be a family of sets with product
$X$, $Y$ another set, and $f:X\to Y$ a function.   If
$\Tau\subseteq\Cal PY$, $\Sigma_i\subseteq\Cal PX_i$ are
$\sigma$-algebras for each $i$, then $f$ is
$(\Tau,\Tensorhat_{i\in I}\Sigma_i)$-measurable iff $\pi_if:Y\to X_i$ is
$(\Tau,\Sigma_i)$-measurable for every $i$, where $\pi_i:X\to X_i$ is
the coordinate map.   \prooflet{\Prf\ $\pi_i$ is
$(\Tensorhat_{j\in I}\Sigma_j,\Sigma_i)$-measurable, so if $f$ is
$(\Tau,\Tensorhat_{j\in I}\Sigma_j)$-measurable then $\pi_if$ must be
$(\Tau,\Sigma_i)$-measurable.   In the other direction, if every
$\pi_if$ is measurable, then $\{H:H\subseteq X$, $f^{-1}[E]\in\Tau\}$ is
a $\sigma$-algebra of subsets of $X$ containing $\pi_i^{-1}[E]$ whenever
$i\in I$ and $E\in\Sigma_i$, so includes $\Tensorhat_{i\in I}\Sigma_i$,
and $f$ is measurable.\ \Qed}

\leader{4A3C}{Elementary facts (a)} If $X$ is a topological
space and $Y$ is a subspace of $X$, then $\Cal B(Y)$ is just the
subspace $\sigma$-algebra $\{E\cap Y:E\in\Cal B(X)\}$.
\prooflet{\Prf\ $\{E:E\subseteq X,\,E\cap Y\in\Cal B(Y)\}$ and
$\{E\cap Y:E\in\Cal B(X)\}$
are $\sigma$-algebras containing all open sets, so include
$\Cal B(X)$, $\Cal B(Y)$ respectively.\ \Qed}
%4@12 %4@27 %4@33 %4@36

\spheader 4A3Cb If $X$ is a set, $\Sigma$ is a $\sigma$-algebra of
subsets of $X$, $(Y,\frak S)$ is a topological space and $f:X\to Y$ is a
function, then $f$ is $(\Sigma,\Cal B(Y))$-measurable iff
$f^{-1}[H]\in\Sigma$ for every $H\in\frak S$.   \prooflet{\Prf\ If $f$
is $(\Sigma,\Cal B(Y))$-measurable then $f^{-1}[H]\in\Sigma$ for every
$H\in\frak S$ just because $\frak S\subseteq\Cal B(Y)$.   If
$f^{-1}[H]\in\Sigma$ for every $H\in\frak S$, then
$\{F:F\subseteq Y,\,f^{-1}[F]\in\Sigma\}$ is a
$\sigma$-algebra of subsets of $Y$ (111Xc;  cf.\ 234C\formerly{1{}12E})
including $\frak S$, so
includes $\Cal B(Y)$, and $f$ is $(\Sigma,\Cal B(Y))$-measurable.\ \Qed}

\spheader 4A3Cc If $X$ and $Y$ are topological spaces, and $f:X\to Y$ is
a function, then $f$ is Borel measurable iff it is
$(\Cal B(X),\Cal B(Y))$-measurable.
\prooflet{(Apply (b) with $\Sigma=\Cal B(X)$.)}
So if $X$, $Y$ and $Z$ are topological spaces and $f:X\to Y$,
$g:Y\to Z$ are Borel measurable functions, then $gf:X\to Z$ is Borel
measurable.
\prooflet{(Use 4A3Bb.)}
%4@39

\spheader 4A3Cd If $X$ and $Y$ are topological spaces and $f:X\to Y$
is continuous, it is Borel measurable.   \prooflet{(Immediate from the
definitions in 4A3A.)}
%4@13, %4@16, %4@17, %4@21

\spheader 4A3Ce If $X$ is a topological space and
$f:X\to[-\infty,\infty]$ is lower semi-continuous, then it is Borel
measurable.
\prooflet{(The inverse image of a half-open interval
$\ocint{\alpha,\beta}$ is a difference of open sets, so is a Borel set,
and every open subset of $[-\infty,\infty]$ is a countable union of such
half-open intervals.)}
%4@45

\spheader 4A3Cf If $\familyiI{X_i}$ is a family of topological spaces
with product $X$, then
$\Cal B(X)\supseteq\Tensorhat_{i\in I}\Cal B(X_i)$.
\prooflet{(Put (d) and 4A3Bc together.)}
%4@25

\spheader 4A3Cg Let $X$ be a topological space.

\medskip

\quad (i) The algebra $\frak A$ of
subsets generated by the open sets is precisely the family of sets
expressible as a disjoint union $\bigcup_{i\le n}G_i\cap F_i$ where
every $G_i$ is open and every $F_i$ is closed.
\prooflet{\Prf\ Write $\Cal A$ for the family of sets expressible in
this form.   Of course $\Cal A\subseteq\frak A$.   In the other
direction, observe that

\inset{$X\in\Cal A$,

if $E$, $E'\in\Cal A$ then $E\cap E'\in\Cal A$,

if $E\in\Cal A$ then $X\setminus E\in\Cal A$}

\noindent because if $G_i$ is open and $F_i$ is closed for $i\le n$,
then (identifying $\{0,\ldots,n\}$ with $n+1$)

$$\eqalign{X\setminus\bigcup_{i\le n}(G_i\cap F_i)
&=\bigcap_{i\le n}(X\setminus G_i)\cup(G_i\setminus F_i)\cr
&=\bigcup_{I\subseteq n+1}\bigl(\bigcap_{i\in I}(G_i\setminus F_i)
  \cap\bigcap_{i\in(n+1)\setminus I}(X\setminus G_i)\bigr)\cr}$$

\noindent belongs to $\Cal A$.   So $\Cal A$ is an algebra of sets and
must be equal to $\frak A$.\ \Qed}

\medskip

\quad (ii) $\Cal B(X)$ is the smallest
family $\Cal E\supseteq\frak A$ such that
$\bigcup_{n\in\Bbb N}E_n\in\Cal E$ for every non-decreasing sequence
$\sequencen{E_n}$ in
$\Cal E$ and $\bigcap_{n\in\Bbb N}E_n\in\Cal E$ for every non-increasing
sequence $\sequencen{E_n}$ in $\Cal E$.
\prooflet{(136G.)}
%4@44

\leader{4A3D}{Hereditarily Lindel\"of spaces (a)} Suppose that $X$ is a
hereditarily Lindel\"of space and $\Cal U$ is a subbase for the topology
of $X$.   Then $\Cal B(X)$ is the $\sigma$-algebra of subsets of $X$
generated by $\Cal U$.   \prooflet{\Prf\ Write $\Sigma$ for the
$\sigma$-algebra generated by $\Cal U$.   Of course
$\Sigma\subseteq\Cal B(X)$ just because every member of $\Cal U$ is
open.   In the other direction, set

\Centerline{$\Cal V
=\{X\}\cup\{U_0\cap U_1\cap\ldots\cap U_n:U_0,\ldots,U_n\in\Cal U\}$;}

\noindent then $\Cal V\subseteq\Sigma$ and $\Cal V$ is a base for the
topology of $X$ (4A2B(a-i)).   If $G\subseteq X$ is open, set
$\Cal V_1=\{V:V\in\Cal V$, $V\subseteq G\}$;  then $G=\bigcup\Cal V_1$.
Because $X$ is hereditarily Lindel\"of, there is a countable set
$\Cal V_0\subseteq\Cal V_1$ such that $G=\bigcup\Cal V_0$ (4A2H(c-i)),
so that $G\in\Sigma$.   Thus every open set belongs to $\Sigma$ and
$\Cal B(X)\subseteq\Sigma$.\ \Qed}

\spheader 4A3Db Let $X$ be a set, $\Sigma$ a $\sigma$-algebra of subsets
of $X$, $Y$ a hereditarily Lindel\"of space, $\Cal U$ a subbase for the
topology of $Y$, and $f:X\to Y$ a function.   If $f^{-1}[U]\in\Sigma$
for every $U\in\Cal U$, then $f$ is $(\Sigma,\Cal B(Y))$-measurable.
\prooflet{\Prf\ $\{F:F\subseteq Y$, $f^{-1}[F]\in\Sigma\}$ is a
$\sigma$-algebra of subsets of $Y$ including $\Cal U$, so contains every
open set, by (a).\ \Qed}

\spheader 4A3Dc Let $\familyiI{X_i}$ be a family of topological spaces
with product $X$.   Suppose that $X$ is hereditarily Lindel\"of.

\medskip

\quad(i) $\Cal B(X)=\Tensorhat_{i\in I}\Cal B(X_i)$.   \prooflet{\Prf\
By 4A3Cf, $\Cal B(X)\supseteq\Tensorhat_{i\in I}\Cal B(X_i)$.   On the
other hand, $\Tensorhat_{i\in I}\Cal B(X_i)$ is a $\sigma$-algebra
including

\Centerline{$\Cal U
=\{\pi_i^{-1}[G]:i\in I$, $G\subseteq X_i$ is open$\}$,}

\noindent where $\pi_i(x)=x(i)$ for $i\in I$ and $x\in X$;  since
$\Cal U$ is a subbase for the topology of $X$, (a) tells us that
$\Tensorhat_{i\in I}\Cal B(X_i)$ includes $\Cal B(X)$.\ \Qed}

\medskip

\quad(ii) If $Y$ is another topological space, then a function
$f:Y\to X$ is Borel measurable iff $\pi_if:Y\to X_i$ is Borel measurable
for every $i\in I$, where $\pi_i:X\to X_i$ is the canonical map.
\prooflet{(Use 4A3Bc.)}
%4@A2O

\cmmnt{
\leader{4A3E}{Applications} Recall that any topological space with a
countable network (in particular, any second-countable space, any
separable metrizable space) is hereditarily
Lindel\"of (4A2Nb, 4A2Oc, 4A2P(a-iii));  and so is any ccc
totally ordered space (4A2Rn).   So 4A3Da-4A3Db will be
applicable to these.   As for product
spaces, the product of a countable family of spaces with countable
networks again has a countable network (4A2Ne), so for such
spaces we shall be able to use 4A3Dc.
For instance, $\Cal B(\{0,1\}^{\Bbb N})$ is the $\sigma$-algebra of
subsets of $\{0,1\}^{\Bbb N}$
generated by the sets $\{x:x(n)=1\}$ for $n\in\Bbb N$.
%4@33
}%end of comment

\leader{4A3F}{Spaces with countable networks (a)} Let $X$ be a
topological space with a countable network.   Then
$\#(\Cal B(X))\le\frak c$.
\prooflet{\Prf\ Let $\Cal E$ be a countable network for the topology of
$X$ and $\Sigma$ the $\sigma$-algebra of subsets of $X$ generated by
$\Cal E$.   Then $\#(\Sigma)\le\frak c$ (4A1O).   If $G\subseteq X$ is
open, there is a subset $\Cal E'$ of $\Cal E$ such that
$\bigcup\Cal E'=G$;  but $\Cal E'$ is necessarily countable, so
$G\in\Sigma$.   It follows that $\Cal B(X)\subseteq\Sigma$ and
$\#(\Cal B(X))\le\frak c$.\ \Qed}

\spheader 4A3Fb $\#(\Cal B(\BbbN^{\Bbb N}))=\frak c$.
\prooflet{\Prf\ $\BbbN^{\Bbb N}$ has a countable network (4A2Ub), so
$\#(\Cal B(\BbbN^{\Bbb N}))\le\frak c$.   On the other hand,
$\Cal B(\BbbN^{\Bbb N})$ contains all singletons, so

\Centerline{$\#(\Cal B(\BbbN^{\Bbb N}))\ge\#(\BbbN^{\Bbb N})
\ge\#(\{0,1\}^{\Bbb N})=\frak c$. \Qed}}
%4@24

\leader{4A3G}{Second-countable spaces (a)} Suppose that $X$ is a
second-countable space and $Y$ is any topological space.   Then
$\Cal B(X\times Y)=\Cal B(X)\tensorhat\Cal B(Y)$.
\prooflet{\Prf\ By 4A3Cf,
$\Cal B(X\times Y)\supseteq\Cal B(X)\tensorhat\Cal B(Y)$.   On the other
hand, let $\Cal U$ be a countable base for the topology of $X$.   If
$W\subseteq X\times Y$ is open, set

\Centerline{$V_U
=\bigcup\{H:H\subseteq Y$ is open, $U\times H\subseteq W\}$}

\noindent for $U\in\Cal U$.   Then $W=\bigcup_{U\in\Cal U}U\times V_U$
belongs to $\Cal B(X)\tensorhat\Cal B(Y)$.   As $W$ is arbitrary,
$\Cal B(X\times Y)\subseteq\Cal B(X)\tensorhat\Cal B(Y)$.\ \Qed}
%countable network for X will do if we have a network consisting of
%Borel sets

\spheader 4A3Gb\dvAnew{2010}
If $X$ is any topological space, $Y$ is a T$_0$
second-countable space, and $f:X\to Y$ is Borel measurable, then
(the graph of) $f$ is a Borel set in $X\times Y$.   \prooflet{\Prf\
Let $\Cal U$ be a countable base for the topology of $Y$.   Because $Y$ is
T$_0$,

\Centerline{$f=\bigcap_{U\in\Cal U}
 (\{(x,y):x\in f^{-1}[U]$, $y\in U\}
 \cup\{(x,y):x\in X\setminus f^{-1}[U]$, $y\in Y\setminus U\})$}

\noindent which is a Borel subset of $X\times Y$ by 4A3Cc and 4A3Cf.\ \Qed}

\leader{4A3H}{Borel sets in Polish spaces:  Proposition} Let
$(X,\frak T)$ be a Polish space and $E\subseteq X$ a
Borel set.   Then there is a Polish topology $\frak S$ on $X$, including
$\frak T$, for which $E$ is open.

\proof{ Let $\Cal E$ be the union of all the Polish topologies on $X$
including $\frak T$.   Of course $X\in\Cal E$.   If $E\in\Cal E$ then
$X\setminus E\in\Cal E$.   \Prf\ There is a Polish topology
$\frak S\supseteq\frak T$ such that $E\in\frak S$.   As both $E$ and
$X\setminus E$ are Polish in the subspace topologies $\frak S_E$,
$\frak S_{X\setminus E}$ induced by $\frak S$ (4A2Qd), the disjoint
union topology $\frak S'$ of $\frak S_E$ and $\frak S_{X\setminus E}$ is
also Polish (4A2Qe).   Now $\frak S'\supseteq\frak S\supseteq\frak T$
and $X\setminus E\in\frak S'$, so $X\setminus E\in\Cal E$.\ \QeD\
Moreover, the union of any sequence $\sequencen{E_n}$ in $\Cal E$
belongs to $\Cal E$.   \Prf\ For each $n\in\Bbb N$ let
$\frak S_n\supseteq\frak T$ be a Polish topology containing $E_n$.   If
$m$, $n\in\Bbb N$ then $\frak S_m\cap\frak S_n$ includes $\frak T$, so
is Hausdorff.   By 4A2Qf, the topology $\frak S$ generated by
$\bigcup_{n\in\Bbb N}\frak S_n$ is Polish.   Of course
$\frak S\supseteq\frak T$, and $\bigcup_{n\in\Bbb N}E_n\in\frak S$, so
$\bigcup_{n\in\Bbb N}E_n\in\Cal E$.\ \Qed

Thus $\Cal E$ is a $\sigma$-algebra.   Since it surely includes
$\frak T$, it includes $\Cal B(X,\frak T)$, as claimed.
}%end of proof of 4A3H
%4@23

\leader{4A3I}{Corollary} If $(X,\frak T)$ is a Polish space and
$\sequencen{E_n}$ is a
sequence of Borel subsets of $X$, then there is a 
zero-dimensional Polish topology
$\frak S$ on $X$, including $\frak T$, for which every $E_n$ is
open-and-closed.

\proof{{\bf (a)} For each $n\in\Bbb N$ we can find a Polish topology
$\frak T_n\supseteq\frak T$ containing $E_k$ (if $n=2k$ is even) or
$X\setminus E_k$ (if $n=2k+1$ is odd);  now the topology
$\frak T'$ generated by $\bigcup_{n\in\Bbb N}\frak T_n$ is
Polish, by 4A2Qf, and every $E_n$ is open-and-closed for $\frak T'$.

\medskip

{\bf (b)} Now choose $\sequencen{\Cal V_n}$ and $\sequencen{\frak T'_n}$
inductively such that

\inset{$\frak T'_0=\frak T'$,

given that $\frak T'_n$ is a Polish topology on $X$, then $\Cal V_n$ is a
countable base for $\frak T'_n$ and
$\frak T'_{n+1}$ is a Polish topology on $X$ including 
$\frak T'_n\cup\{X\setminus V:V\in\Cal V_n\}$}

\noindent (using (a) for the inductive step).   Now the topology
$\frak S$ generated by $\bigcup_{n\in\Bbb N}\frak T'_n$ is Polish,
includes $\frak T$, contains every $E_n$ and has a base 
$\bigcup_{n\in\Bbb N}\Cal V_n$ consisting of open-and-closed sets.
}%end of proof of 4A3I

\leader{4A3J}{Borel sets in $\omega_1$:  Proposition} A set
$E\subseteq\omega_1$ is a Borel set iff either $E$ or its complement
includes a closed cofinal set.

\proof{{\bf (a)} Let $\Sigma$ be the family of all those sets
$E\subseteq\omega_1$ such that either $E$ or $\omega_1\setminus E$
includes a closed cofinal set.   Of course $\Sigma$ is closed under
complements.
Because the intersection of a sequence of closed cofinal sets is a
closed cofinal set (4A1Bd), the
union of any sequence in $\Sigma$ belongs to $\Sigma$;  so $\Sigma$ is a
$\sigma$-algebra.   If $E$ is closed, then either it is cofinal with
$\omega_1$, and is a closed cofinal set, or there is a $\xi<\omega_1$
such that
$E\subseteq\xi$, in which case $\omega_1\setminus E$ includes the
closed cofinal set
$\omega_1\setminus\xi$;  in either case, $E\in\Sigma$.   Thus every open
set belongs to $\Sigma$, and $\Sigma$ includes $\Cal B(\omega_1)$.

\medskip

{\bf (b)} Now suppose that $E\subseteq\omega_1$ is such that there is a
closed cofinal set $F\subseteq\omega_1\setminus E$.   For each
$\xi<\omega_1$ let
$f_{\xi}:\xi\to\Bbb N$ be an injective function.   Define $g:E\to\Bbb N$
by setting $g(\eta)=f_{\alpha(\eta)}(\eta)$, where
$\alpha(\eta)=\min(F\setminus\eta)$ for $\eta\in E$.   Set
$A_n=g^{-1}[\{n\}]$ for $n\in\Bbb N$, so that
$E=\bigcup_{n\in\Bbb N}A_n$.   If $\xi\in\overline{A}_n\setminus A_n$
and $\xi'<\xi$, there must be $\eta$, $\eta'\in A_n$ such that
$\xi'\le\eta<\eta'<\xi$.   But now, because $f_{\alpha(\eta')}$ is
injective, while $g(\eta)=g(\eta')=n$, $\alpha(\eta)\ne\alpha(\eta')$, so
$\alpha(\eta)\in F\cap\ooint{\xi',\xi}$.   As $\xi'$ is arbitrary,
$\xi\in\overline{F}=F$.   This shows that
$\overline{A}_n\subseteq A_n\cup F$ and $A_n=\overline{A}_n\setminus F$
is a Borel set.   This is true for every $n\in\Bbb N$, so
$E=\bigcup_{n\in\Bbb N}A_n$ is a Borel set.

\medskip

{\bf (c)} If $E\subseteq\omega_1$ includes a closed cofinal set, then
(b) tells us that $\omega_1\setminus E$ and $E$ are Borel sets.
Thus $\Sigma\subseteq\Cal B(\omega_1)$ and $\Sigma=\Cal B(\omega_1)$, as
claimed.
}%end of proof of 4A3J

\leader{4A3K}{Baire sets }\cmmnt{When we come to study
measures in terms of the integrals of
continuous functions (\S436), we find that it is sometimes inconvenient
or even impossible to apply them to arbitrary Borel sets, and we need to
use a smaller $\sigma$-algebra, as follows.

\medskip

}{\bf (a) Definition} Let $X$ be a topological space.
The {\bf Baire $\sigma$-algebra} $\CalBa(X)$ of $X$ is the
$\sigma$-algebra generated
by the zero sets.   Members of $\CalBa(X)$ are called
{\bf Baire} sets.   \cmmnt{({\bf Warning!}
Do not confuse `Baire sets' in this sense
with `sets with the Baire property' in the sense of 4A3Q, nor
with `sets which are Baire spaces in their subspace topologies'.)}

\spheader 4A3Kb For any topological space $X$,
$\CalBa(X)\subseteq\Cal B(X)$\cmmnt{ (because every zero set is
closed, therefore Borel)}.   If $\frak T$ is perfectly
normal\cmmnt{ -- for instance, if it is metrizable (4A2Lc), or is
regular and hereditarily Lindel\"of (4A2H(c-ii)) --} then
$\CalBa(X)=\Cal B(X)$\cmmnt{ (because
every closed set is a zero set, by 4A2Fi, so every open set belongs
to $\CalBa(X)$)}.

\spheader 4A3Kc Let $X$ and $Y$ be topological
spaces, with Baire $\sigma$-algebras $\CalBa(X)$, $\CalBa(Y)$
respectively.   If $f:X\to Y$
is continuous, it is $(\CalBa(X),\CalBa(Y))$-measurable.
\prooflet{\Prf\ Let $\Tau$ be the $\sigma$-algebra
$\{F:F\subseteq Y,\,f^{-1}[F]\in\CalBa(X)\}$.   If
$g:Y\to\Bbb R$ is continuous, then $gf:X\to\Bbb R$
is continuous, so

\Centerline{$f^{-1}[\{y:g(y)=0\}]
=\{x:gf(x)=0\}\in\CalBa(X)$,}

\noindent and $\{y:g(y)=0\}\in\Tau$.   Thus every zero set belongs to
$\Tau$, and $\Tau\supseteq\CalBa(Y)$.\ \Qed}

\spheader 4A3Kd In particular, if $X$ is a subspace of $Y$, then
$E\cap X\in\CalBa(X)$ whenever
$E\in\CalBa(Y)$.   \cmmnt{More fundamentally,} $F\cap X$ is a
zero set in $X$ for every zero set $F\subseteq Y$\cmmnt{, just because
$g\restrp X$ is continuous for any continuous $g:Y\to\Bbb R$}.

\spheader 4A3Ke If $X$ is a topological space and $Y$ is a separable
metrizable space, a function $f:X\to Y$ is {\bf Baire measurable} if
$f^{-1}[H]\in\CalBa(X)$ for every open $H\subseteq Y$.   Observe that
$f$ is Baire measurable in this sense iff it is
$(\CalBa(X),\Cal B(Y))$-measurable iff it is
$(\CalBa(X),\CalBa(Y))$-measurable.

\leader{4A3L}{Lemma} Let $(X,\frak T)$ be a topological space.   Then
$\CalBa(X)$ is just the smallest $\sigma$-algebra of subsets of $X$ with
respect to
which every continuous real-valued function on $X$ is measurable.

\proof{{\bf (a)} Let $f:X\to\Bbb R$ be a continuous function and
$\alpha\in\Bbb R$.   Set $g(x)=\max(0,f(x)-\alpha)$ for $x\in X$;  then
$g$ is continuous, so

\Centerline{$\{x:f(x)\le\alpha\}=\{x:g(x)=0\}$}

\noindent is a zero set and belongs to $\CalBa(X)$.   As $\alpha$ is
arbitrary, $f$ is $\CalBa(X)$-measurable.

\medskip

{\bf (b)} On the other hand, if $\Sigma$ is any $\sigma$-algebra of
subsets of $X$ such that every continuous real-valued function on $X$ is
$\Sigma$-measurable, and $F\subseteq X$ is a zero set, then there is a
continuous $g$ such that $F=g^{-1}[\{0\}]$, so that $F\in\Sigma$;  as
$F$ is arbitrary, $\Sigma\supseteq\CalBa(X)$.
}%end of proof of 4A3L

\leader{4A3M}{Product spaces} Let $\familyiI{X_i}$ be a family of
topological spaces with product $X$.

\spheader 4A3Ma $\CalBa(X)\supseteq\Tensorhat_{i\in I}\CalBa(X_i)$.
\prooflet{(Apply 4A3Bc to the identity map from $X$ to itself;
compare 4A3Cf.)}

\spheader 4A3Mb Suppose that $X$ is ccc.   Then every Baire subset of
$X$ is determined by coordinates in a countable set.
\prooflet{\Prf\ By 254Mb, the family $\Cal W$ of sets determined by
coordinates in countable sets is a $\sigma$-algebra of subsets of $X$.
By 4A2E(b-ii), every continuous real-valued function is
$\Cal W$-measurable, so $\Cal W$ contains every zero set and every Baire
set.\ \Qed}

\leader{4A3N}{Products of separable metrizable spaces:  Proposition} Let
$\familyiI{X_i}$ be a family of separable metrizable spaces, with
product $X$.

(a) $\CalBa(X)=\Tensorhat_{i\in I}\CalBa(X_i)
=\Tensorhat_{i\in I}\Cal B(X_i)$.

(b) $\CalBa(X)$ is the family of Borel subsets of $X$ which are
determined by coordinates in countable sets.

(c) A set $Z\subseteq X$ is a zero set iff it is closed and
determined by coordinates in a countable set.

(d) If $Y$ is a dense subset of $X$, then the Baire
$\sigma$-algebra $\CalBa(Y)$ of $Y$ is just the subspace
$\sigma$-algebra $\CalBa(X)_Y$ induced by $\CalBa(X)$.

(e) If $Y$ is a set, $\Tau$ is a $\sigma$-algebra of subsets of $Y$, and
$f:Y\to X$ is a function, then $f$ is $(\Tau,\CalBa(X))$-measurable iff
$\pi_if:Y\to X_i$ is $(\Tau,\Cal B(X_i))$-measurable for every $i\in I$,
where $\pi_i(x)=x(i)$ for $x\in X$ and $i\in I$.

\proof{{\bf (a)} $X$ is ccc (4A2E(a-iii)), so if $f:X\to\Bbb R$ is
continuous,
there are a countable set $J\subseteq I$ and a continuous function
$g:X_J\to\Bbb R$ such that $f=g\tilde\pi_J$, where
$X_J=\prod_{i\in J}X_i$ and
$\tilde\pi_J:X\to X_J$ is the canonical map (4A2E(b-ii)).   Now $X_J$ is
separable and metrizable (4A2P(a-v)), therefore hereditarily Lindel\"of
(4A2P(a-iii)),
so $\Cal B(X_J)=\Tensorhat_{i\in J}\Cal B(X_i)$, by
4A3D(c-i).   By 4A3Bc, $\tilde\pi_J$ is
$(\Tensorhat_{i\in I}\Cal B(X_i),
\Tensorhat_{j\in J}\Cal B(X_j))$-measurable, so $f$ is
$\Tensorhat_{i\in I}\Cal B(X_i)$-measurable.   As $f$ is arbitrary,
$\CalBa(X)\subseteq\Tensorhat_{i\in I}\Cal B(X_i)$ (4A3L).
Also $\Cal B(X_i)=\CalBa(X_i)$ for every $i$ (4A3Kb), so
$\Tensorhat_{i\in I}\Cal B(X_i)=\Tensorhat_{i\in I}\CalBa(X_i)$.
With 4A3Ma, this proves the result.

\medskip

{\bf (b)}(i) If $W\in\CalBa(X)$ it is certainly a Borel
set (4A3Kb), and by 4A3Mb it is determined by coordinates in a countable
set.

\medskip

\quad(ii) If $W$ is a Borel subset of $X$ determined by
coordinates in a countable subset $J$ of $X$, then write
$X_J=\prod_{i\in J}X_i$ and
$X_{I\setminus J}=\prod_{i\in I\setminus J}X_i$;  let
$\tilde\pi_J:X\to X_J$ be the canonical map.   We can identify $W$
with $W'\times X_{I\setminus J}$, where $W'$ is some subset of $X_J$.
Now if
$z\in X_{I\setminus J}$, $W'=\{w:w\in X_J,\,(w,z)\in W\}$ is a Borel
subset of $X_J$, because $w\mapsto(w,z):X_J\to X$ is continuous.   (I am
passing over the trivial case $X=\emptyset$.)   Since $X_J$ is
metrizable (4A2P(a-v)), $W'\in\CalBa(X_J)$ (4A3Kb) and
$W=\tilde\pi_J^{-1}[W']$ is a Baire set (4A3Kc).

\medskip

{\bf (c)}(i) If $Z$ is a zero set, it is surely closed;
and it is determined by coordinates in a
countable set by (b) above, or directly from 4A2E(b-ii).

\medskip

\quad(ii) If $Z$ is closed and determined by
coordinates in a countable set $J$, then (in the language of (b) above)
it can be identified with
$Z'\times X_{I\setminus J}$ for some $Z'\subseteq X_J$.   As in the
proof of (b), $Z'$ is closed (at least, if
$X_{I\setminus J}\ne\emptyset$), so is a zero set (4A2Lc), and
$Z=\tilde\pi_J^{-1}[Z']$ is a zero set (4A2C(b-iv)).

\medskip

{\bf (d)}(i) $\CalBa(Y)\supseteq\CalBa(X)_Y$ by 4A3Kd.

\medskip

\quad(ii) Let $f:Y\to\Bbb R$ be any continuous function.   For each
$n\in\Bbb N$, there is an open set $G_n\subseteq X$ such that
$G_n\cap Y=\{y:f(y)>2^{-n}\}$.   Now $\overline{G}_n$ is determined by
coordinates in a countable set (4A2E(b-i)), so is a zero set,
by (c) here.   Because $Y$ is dense in $X$,
$\overline{G}_n=\overline{G_n\cap Y}$ does not meet $\{y:f(y)=0\}$, and
$\{y:f(y)>0\}=Y\cap\bigcup_{n\in\Bbb N}\overline{G}_n$ belongs to
$\CalBa(X)_Y$.   Thus $\CalBa(X)_Y$ contains every cozero subset of
$Y$ and includes $\CalBa(Y)$.

\medskip

{\bf (e)} Put (a) and 4A3Bc together.
}%end of proof of 4A3N

\leader{4A3O}{Compact spaces (a)} Let $(X,\frak T)$ be a topological
space, $\Cal U$ a subbase for
$\frak T$, and $\frak A$ the algebra of subsets of $X$ generated by
$\Cal U$.   If $H\subseteq X$ is open and $K\subseteq H$ is compact,
there is an $E\in\frak A$ such that $K\subseteq E\subseteq H$.
\prooflet{\Prf\ Set $\Cal V
=\{X\}\cup\{U_0\cap U_1\cap\ldots\cap U_n:U_0,\ldots,U_n\in\Cal U\}$, so
that $\Cal V$ is a base for $\frak T$ and
$\Cal V\subseteq\frak A$.   $\{U:U\in\Cal V,\,U\subseteq H\}$ is an
open cover of the compact set $K$, so there is a finite set
$\Cal U_0\subseteq\Cal V$ such that $E=\bigcup\Cal U_0$ includes $K$
and is included in $H$;  now $E\in\frak A$.\ \Qed}

\spheader 4A3Ob Let $(X,\frak T)$ be a compact space and $\Cal U$ a
subbase for $\frak T$.   Then every open-and-closed subset of $X$
belongs to the algebra of subsets of $X$ generated by $\Cal U$.
\prooflet{(If $F\subseteq X$ is open-and-closed, it is also compact;
apply (a) here with $K=H=F$.)}
%4@54

\spheader 4A3Oc Let $(X,\frak T)$ be a compact space and $\Cal U$ a
subbase for $\frak T$.   Then $\CalBa(X)$ is included in the
$\sigma$-algebra of subsets of $X$ generated by $\Cal U$.
\prooflet{\Prf\ Let $\Sigma$ be the $\sigma$-algebra generated by
$\Cal U$.   If $Z\subseteq X$ is a zero set, there is a sequence
$\sequencen{H_n}$ of open sets with intersection $Z$ (4A2C(b-vi)); now
we can find a sequence $\sequencen{E_n}$ in $\Sigma$ such that
$Z\subseteq E_n\subseteq H_n$ for every $n$, by (a), so that
$Z=\bigcap_{n\in\Bbb N}E_n\in\Sigma$.   This shows that every zero set
belongs to $\Sigma$, so $\Sigma$ must include $\CalBa(X)$.\ \Qed}

\spheader 4A3Od Let $\familyiI{X_i}$ be a family of compact Hausdorff
spaces with product $X$.
Then $\CalBa(X)=\Tensorhat_{i\in I}\CalBa(X_i)$.
\prooflet{\Prf\ By 4A3Ma,
$\CalBa(X)\supseteq\Tensorhat_{i\in I}\CalBa(X_i)$.   On the
other hand, let $\Cal U_i$ be the family of cozero sets in $X_i$ for
each $i$.
Because $X_i$ is completely regular (3A3Bb), $\Cal U_i$ is a base for
its topology (4A2Fc).   Set

\Centerline{$\Cal W
=\{\prod_{i\in I}U_i:U_i\in\Cal U_i$ for every $i\in I$,
  $\{i:U_i\ne X_i\}$ is finite$\}$,}

\noindent so that $\Cal W\subseteq\Tensorhat_{i\in I}\CalBa(X_i)$ is a
base for the topology of $X$.   By (c) above, $\CalBa(X)$ is the
$\sigma$-algebra generated by $\Cal W$, and
$\CalBa(X)\subseteq\Tensorhat_{i\in I}\CalBa(X_i)$.\ \Qed}
%4@63

\spheader 4A3Oe In a compact Hausdorff zero-dimensional space the Baire
$\sigma$-algebra is the $\sigma$-algebra generated by the
open-and-closed sets.
\prooflet{(Apply (c) with $\Cal U$ the family of open-and-closed
sets.)}
%4@52 %4@39?

\spheader 4A3Of In particular, for any set $I$, $\CalBa(\{0,1\}^I)$ is
the $\sigma$-algebra generated by sets of the form $\{x:x(i)=1\}$ as $i$
runs over $I$.
%for 4@53

\leader{4A3P}{Proposition} The Baire $\sigma$-algebra $\CalBa(\omega_1)$
of $\omega_1$ is\cmmnt{ just} the
countable-cocountable algebra\cmmnt{ (211R)}.

\proof{ We
see from 4A2S(b-iii) that every continuous function is measurable with
respect to
the countable-cocountable algebra, so $\CalBa(\omega_1)$ is included in
the countable-cocountable algebra.   On the other hand,

\Centerline{$[0,\xi]=\{\eta:\eta\le\xi\}=\coint{0,\xi+1}
=\omega_1\setminus\ooint{\xi,\omega_1}$}

\noindent is an open-and-closed set
(4A2S(a-i)), therefore a zero set, therefore belongs to
$\CalBa(\omega_1)$, for
every $\xi<\omega_1$.   Now if $\xi<\omega_1$, it is itself a countable
set, so

\Centerline{$\coint{0,\xi}=\bigcup_{\eta<\xi}[0,\eta]
\in\CalBa(\omega_1)$,
\quad$\{\xi\}=[0,\xi]\setminus\coint{0,\xi}\in\CalBa(\omega_1)$.}

\noindent It follows that every countable set belongs to
$\CalBa(\omega_1)$ and
the countable-cocountable algebra is included in $\CalBa(\omega_1)$.
}%end of proof of 4A3P

\leader{4A3Q}{Baire property} Let $X$ be a topological space, and $\Cal M$
the ideal of meager subsets of $X$.   A subset
$X$ has the {\bf Baire property} if it is expressible in the form
$G\symmdiff M$
where $G\subseteq X$ is open and $M\in\Cal M$;
\cmmnt{that is,}
$A\subseteq X$ has the Baire property if there is an open set
$G\subseteq X$ such that $G\symmdiff A$ is meager.   \prooflet{(For
$A=G\symmdiff M$ iff $M=G\symmdiff A$.)}   The family
$\widehat{\Cal B}(X)$ of all such sets is the {\bf Baire-property
algebra} of $X$.
\cmmnt{(See 4A3R.)}   \cmmnt{({\bf Warning!} do not confuse the
`Baire-property algebra' $\widehat{\Cal B}$ with the
`Baire $\sigma$-algebra' $\CalBa$ as defined in 4A3K.)}
The quotient algebra $\widehat{\Cal B}(X)/\Cal M$ is the {\bf category
algebra} of $X$.\dvAnew{2009}

\leader{4A3R}{Proposition} Let $X$ be a topological space.

(a) Let $A\subseteq X$ be any set.
%this could be made more useful;  see 5A4Ea, 5A4Eb.  \query

\quad(i) There is a smallest regular open set $H\subseteq X$ such that
$A\setminus H$ is meager.

\quad(ii) $H\cap G$ is empty whenever $G\subseteq X$ is open and
$A\cap G$ is meager;  in particular, $H\subseteq\overline{A}$.

\quad(iii) $H$ is in itself a Baire space.

\quad(iv) If $A\in\widehat{\Cal B}(X)$, $H\symmdiff A$ is meager.

\quad(v) If $X$ is a Baire space and $A\in\widehat{\Cal B}(X)$, then
$H$ is the largest open subset of $X$ such that $H\setminus A$ is meager.

(b)(i) $\widehat{\Cal B}(X)$ is a $\sigma$-algebra of subsets of $X$
including $\Cal B(X)$.

\quad(ii)\dvAnew{2009} $\widehat{\Cal B}(X)=\{G\symmdiff M:G\subseteq X$ is
a regular open set, $M\in\Cal M\}$.

(c) If $X$ has a
countable network, its category algebra has a countable
order-dense set\cmmnt{ (definition:  313J)}.

\proof{{\bf (a)} (See {\smc Kechris 95}, 8.29.)

\medskip

\quad{\bf (i)} Set
$\Cal G=\{G:G\subseteq X$ is open, $A\cap G$ is meager$\}$.   Let
$\Cal G_0\subseteq\bigcup\Cal G$ be a maximal disjoint set, and
$G_0=\bigcup\Cal G_0$.   Then $A\cap G_0$ is meager.   \Prf\ For each
$G\in\Cal G_0$, let $\sequencen{F_{Gn}}$ be a sequence of nowhere dense
closed sets covering $A\cap G$.   Set
$A_n=\bigcup_{G\in\Cal G_0}G\cap F_{Gn}$.   If $U\subseteq X$ is any
non-empty open set, either $U\cap A_n$ is empty or there is a
$G\in\Cal G_0$ such that $U\cap G\ne\emptyset$, in which case
$U\cap G\setminus F_{Gn}$ is a non-empty open subset of $U$ not meeting
$A_n$.   Thus $A_n$ is nowhere dense.   This is true for every $n$, so
$A\cap G_0\subseteq\bigcup_{n\in\Bbb N}A_n$ is meager.\ \Qed

Set $H=\interior(X\setminus G_0)$, so that $H$ is a regular open set
(definition:  314O).
$A\setminus H\subseteq(\overline{G}_0\setminus G_0)\cup(A\cap G_0)$ is
meager.   If $H'$ is another regular open set such that $A\setminus H'$ is
meager, then $H\setminus\overline{H'}\in\Cal G$;  as
$H\setminus\overline{H'}$ does not meet $G_0$, it must be empty, by the
maximality of $\Cal G_0$.   So $H\subseteq\interior\overline{H'}=H'$.
Thus $H$ is the smallest regular open set such that $A\setminus H$ is
meager.

\medskip

\quad{\bf (ii)}
If $G\in\Cal G$ then $G\cap H=G\setminus\overline{G}_0$
belongs to $\Cal G$ and is disjoint from every member of $\Cal G_0$;  by
the maximality of $\Cal G_0$, it is empty.

In particular, $X\setminus\overline{A}$ does not meet $H$, and
$H\subseteq\overline{A}$.

\medskip

\quad{\bf (iii)} If now $\sequencen{H_n}$ is a sequence of open subsets of
$H$ which are dense in $H$, and $H'\subseteq H$ is any non-empty open set,
$H'\setminus H_n$ is nowhere dense for every $n$, so
$H'\setminus\bigcap_{n\in\Bbb N}H_n$ is meager.   On the other hand,
$A\cap H'$ is non-meager so $H'$ also is, and $H'$ must meet
$\bigcap_{n\in\Bbb N}H_n$.   As $H'$ is arbitrary,
$\bigcap_{n\in\Bbb N}H_n$ is dense in
$H$;  as $\sequencen{H_n}$ is arbitrary, $H$ is a Baire space in its
subspace topology.

\medskip

\quad{\bf (iv)} If $A$ has the Baire property, there is an open set $G$
such that $A\symmdiff G$ is meager.   In this case,
$A\cap H\setminus\overline{G}$ must be meager, so
$H\subseteq\overline{G}$ and
$H\setminus A\subseteq(G\setminus A)\cup(\overline{G}\setminus G)$ is
meager and $H\symmdiff A=(A\setminus H)\cup(H\setminus A)$ is meager.

\medskip

\quad{\bf (v)} By (iv), $H\setminus A$ is meager.   If $G\subseteq X$ is
open and $G\setminus A$ is meager, set $G'=G\setminus\overline{H}$.
Then $G'\setminus A$ and $A\setminus G'$ are both meager, so $G'$ is
meager, and must be empty, since $X$ is a Baire space.   Thus
$G\subseteq\overline{H}$ and $G\subseteq H$, because $H$ is a regular open
set.

\medskip

{\bf (b)(i)} (See {\smc \v{C}ech 66}, \S22C;  {\smc Kuratowski 66},
\S11.III;  {\smc Kechris 95}, 8.22.)
Of course $X=X\symmdiff\emptyset$ belongs to $\widehat{\Cal B}(X)$.   If
$E\in\widehat{\Cal B}(X)$, let $G\subseteq X$ be an open set such that
$E\symmdiff G$ is meager.   Then

\Centerline{$(X\setminus\overline{G})\symmdiff(X\setminus E)
\subseteq(\overline{G}\setminus G)\cup(G\symmdiff E)$}

\noindent is meager, so $X\setminus E\in\widehat{\Cal B}(X)$.   If
$\sequencen{E_n}$ is a sequence in $\widehat{\Cal B}(X)$, then for each
$n\in\Bbb N$ we can find an open set $G_n$ such that $G_n\symmdiff E_n$
is meager, and now

\Centerline{$(\bigcup_{n\in\Bbb N}G_n)\symmdiff(\bigcup_{n\in\Bbb N}E_n)
\subseteq\bigcup_{n\in\Bbb N}G_n\symmdiff E_n$}

\noindent is meager, so $\bigcup_{n\in\Bbb N}E_n\in\widehat{\Cal B}(X)$.

This shows that $\widehat{\Cal B}(X)$ is a $\sigma$-algebra of subsets
of $X$.   Since it contains every open set, it must include the Borel
$\sigma$-algebra.

\medskip

\quad{\bf (ii)} All we have to observe is that
$\overline{G}\setminus G$ and $F\setminus\interior F$ are meager for all
open sets $G$ and closed sets $F$, so that if $G$ is open then there is a
regular open set $H=\interior\overline{G}$ such that
$G\symmdiff H\in\Cal M$.

\medskip

{\bf (c)} Suppose that $X$ has a countable network $\Cal A$.   For
$A\in\Cal A$, set $d_A=\overline{A}^{\ssbullet}$, the equivalence class
of $\overline{A}\in\widehat{\Cal B}$
in $\widehat{\Cal B}/\Cal M$.   If $b\in\widehat{\Cal B}/\Cal M$ is
non-zero, take $E\in\widehat{\Cal B}$ such that $E^{\ssbullet}=b$, and an open set $G\subseteq X$ such that
$E\symmdiff G\in\Cal M$.   Then $G$ is not meager.   Set $\Cal
A_1=\{A:A\in\Cal A$, $A\subseteq G\}$;  then $\Cal A_1$ is countable and
$G=\bigcup\Cal A_1$, so there is a non-meager $A\in\Cal A_1$.   In this
case $\overline{A}$ is not meager, so $d_A\ne 0$, while
$\overline{A}\setminus G\subseteq\overline{G}\setminus G$ is nowhere
dense, so $d_A\Bsubseteq b$.   As $b$ is arbitrary,
$\{d_A:A\in\Cal A\}$ is order-dense in $\widehat{\Cal B}/\Cal M$, and is
countable because $\Cal A$ is.
}%end of proof of 4A3R

\leader{*4A3S}{}\cmmnt{The following result will be useful in \S424
and in Volume 5.

\medskip

\noindent}{\bf Lemma} Let $X$ and $Y$ be sets, $\Sigma$ a
$\sigma$-algebra
of subsets of $X$, $\Tau$ a $\sigma$-algebra of subsets of $Y$ and
$\Cal J$ a
$\sigma$-ideal of $\Tau$.   Suppose that the quotient Boolean algebra
$\Tau/\Cal J$ has a countable order-dense set.

(a) $\{x:x\in X,\,W[\{x\}]\cap A\in\Cal J\}$ belongs to $\Sigma$ for any
$W\in\Sigma\tensorhat\Tau$ and $A\subseteq Y$.

(b) For every $W\in\Sigma\tensorhat\Tau$ there are sequences
$\sequencen{E_n}$ in $\Sigma$, $\sequencen{V_n}$ in $\Tau$ such that
$(W\symmdiff W_1)[\{x\}]\in\Cal J$ for every
$x\in X$, where $W_1=\bigcup_{n\in\Bbb N}E_n\times V_n$.
%not sure if (b) needed, but see 527J

\woddheader{*4A3S}{0}{0}{0}{60pt}

\proof{ (Compare {\smc Kechris 95}, 16.1.)

\medskip

{\bf (a)} Let $D\subseteq\Tau/\Cal J$ be a countable order-dense set.
Let $\Cal V\subseteq\Tau$ be a countable set such that
$D=\{V^{\ssbullet}:V\in\Cal V\}$.
Let $\Lambda$ be the family of subsets $W$ of $X\times Y$  such
that

\Centerline{$W[\{x\}]\in\Tau$ for every $x\in X$,}

\Centerline{$\{x:x\in X,\,W[\{x\}]\cap A\in\Cal J\}\in\Sigma$
for every $A\subseteq Y$.}

\medskip

\quad{\bf (i)} Of course $\emptyset\in\Lambda$, because if $W=\emptyset$
then $W[\{x\}]=\emptyset$ for every $x\in X$, and
$\{x:W[\{x\}]\cap A\in\Cal J\}=X$ for every $A\subseteq Y$.

\medskip

\quad{\bf (ii)} Suppose that $W\in\Lambda$, and set
$W'=(X\times Y)\setminus W$.   Then

\Centerline{$W'[\{x\}]=Y\setminus W[\{x\}]\in\Tau$}

\noindent for every $x\in X$.   Now suppose that $A\subseteq Y$.   Set
$\Cal V^*=\{V:V\in\Cal V,\,A\cap V\notin\Cal J\}$,

\Centerline{$E=\{x:W'[\{x\}]\cap A\in\Cal J\}$,}

\Centerline{$E'
=\{x:V\cap W[\{x\}]\notin\Cal J\text{ for every }V\in\Cal V^*\}$.}

\noindent Then $E=E'$.   \Prf\ ($\alpha$) If $x\in E$ and
$V\in\Cal V^*$, then $A\cap W'[\{x\}]=A\setminus W[\{x\}]$ belongs to
$\Cal J$, so
$V\cap A\setminus W[\{x\}]\in\Cal J$ and
$V\cap W[\{x\}]\supseteq V\cap A\cap W[\{x\}]$ is not in $\Cal J$.   As
$V$
is arbitrary, $x\in E'$;  as $x$ is
arbitrary, $E\subseteq E'$.   ($\beta$) If $x\notin E$, then
$W'[\{x\}]\cap A\notin\Cal J$.   Set $\Cal V_1=\{V:V\in\Cal V$,
$V\setminus W'[\{x\}]\in\Cal J\}$, $D_1=\{V^{\ssbullet}:V\in\Cal V_1\}$.
Then $D_1=\{d:d\in D$, $d\Bsubseteq W'[\{x\}]^{\ssbullet}\}$, so
$W'[\{x\}]^{\ssbullet}=\sup D_1$ in $\Tau/\Cal J$ (313K).   Because
$\Cal J$ is
a $\sigma$-ideal, the map $F\mapsto F^{\ssbullet}:\Tau\to\Tau/\Cal J$ is
sequentially order-continuous (313Qb), and
$(\bigcup\Cal V_1)^{\ssbullet}=\sup D_1$
(313Lc), that is, $W'[\{x\}]\symmdiff\bigcup\Cal V_1\in\Cal J$.   There
must therefore be a $V\in\Cal V_1$ such that $V\cap A\notin\Cal J$, that
is, $V\in\Cal V^*$.   At the same time,
$V\cap W[\{x\}]=V\setminus W'[\{x\}]$ belongs to $\Cal J$, so $V$
witnesses that $x\notin E'$.   As $x$ is arbitrary, $E'\subseteq E$.\
\Qed

Since $W\in\Lambda$,

\Centerline{$E=E'
=X\setminus\bigcup_{V\in\Cal V^*}\{x:V\cap W[\{x\}]\in\Cal J\}$}

\noindent belongs to $\Tau$.   As $A$ is arbitrary,
$W'\in\Lambda$.   Thus the complement of any member of $\Lambda$ belongs
to $\Lambda$.

\medskip

\quad{\bf (iii)} If $\sequencen{W_n}$ is any sequence in $\Lambda$ with
union $W$, then

\Centerline{$W[\{x\}]
=\bigcup_{n\in\Bbb N}W_n[\{x\}]\in\Tau$}

\noindent for every $x$, while

\Centerline{$\{x:W[\{x\}]\cap A\in\Cal J\}
=\bigcap_{n\in\Bbb N}\{x:W_n[\{x\}]\cap A\in\Cal J\}\in\Sigma$}

\noindent for every $A\subseteq Y$.   So $W\in\Lambda$.

\medskip

\quad{\bf (iv)} What this shows is that $\Lambda$ is a $\sigma$-algebra
of
subsets of $X\times Y$.   But if $W=E\times F$, where $E\in\Sigma$ and
$F\in\Tau$, then

\Centerline{$W[\{x\}]\in\{\emptyset,F\}\subseteq\Tau$ for
every $x\in X$,}

\Centerline{$\{x:W[\{x\}]\cap A\in\Cal J\}\in\{X\setminus
E,X\}\subseteq\Sigma$}

\noindent for every $A\subseteq Y$;  so $W\in\Lambda$.   Accordingly
$\Lambda$ must include $\Sigma\tensorhat\Tau$, as
claimed.

\medskip

{\bf (b)} Let $\sequencen{V_n}$ be a sequence running over
$\Cal V\cup\{\emptyset\}$.   For each $n\in\Bbb N$, set

\Centerline{$E_n=\{x:V_n\setminus W[\{x\}]\in\Cal J\}$;}

\noindent by (a), $E_n\in\Sigma$.   Set
$W_1=\bigcup_{n\in\Bbb N}E_n\times V_n$.   Take any $x\in X$.
Then $W_1[\{x\}]=\bigcup_{n\in I}V_n$ where $I=\{n:x\in E_n\}$.
Since $V_n\setminus W[\{x\}]\in\Cal J$ for every $n\in I$,
$W_1[\{x\}]\setminus W[\{x\}]\in\Cal J$.   \Quer\ If
$W[\{x\}]\setminus W_1[\{x\}]\notin\Cal J$, there is a
$d\in D$ such that
$0\ne d\Bsubseteq W[\{x\}]^{\ssbullet}\Bsetminus W_1[\{x\}]^{\ssbullet}$.   Let $n\in\Bbb N$ be such that
$V_n^{\ssbullet}=d$;  then $V_n\setminus W[\{x\}]\in\Cal J$, so $n\in I$
and $V_n\subseteq W_1[\{x\}]$ and $d\Bsubseteq W_1[\{x\}]^{\ssbullet}$,
which is impossible.\ \Bang

As $x$ is arbitrary, $W_1$ has the required properties.
}%end of proof of 4A3S
%4@24

\leaveitout{\spheader 4{}A2Oc Let $X$ be any topological space and
$\widehat{\Cal B}(X)$ its Baire-property algebra.   If $Y$ is a
second-countable space and $f:X\to Y$ is
$\widehat{\Cal B}(X)$-measurable,
then there is a Borel measurable function $g:X\to Y$ such that
$\{x:x\in X$, $g(x)\ne f(x)\}$ is meager.
\prooflet{\Prf\ Let $\Cal U$ be a countable base for the topology of
$Y$.   For each $U\in\Cal U$, let $G_U\subseteq X$ be an open set such
that $f^{-1}[U]\symmdiff G_U$ is meager;  let
$F_U\supseteq f^{-1}[U]\symmdiff G_U$ be a meager F$_{\sigma}$ set.
Then $Q=X\setminus\bigcup_{U\in\Cal U}F_U$ is a comeager G$_{\delta}$
set.   Take any $y_0\in Y$ (I am passing over the trivial case
$Y=\emptyset$), and set $g(x)=f(x)$ for $x\in Q$, $g(x)=y_0$ for
$x\in X\setminus Q$.   Then $\{x:g(x)=f(x)\}\supseteq Q$ is comeager.
If $U\in\Cal U$,   $g^{-1}[U]$ is either $Q\cap f^{-1}[U]=Q\cap G_U$ or
$(Q\cap G_U)\cup(X\setminus Q)$, and in either case is Borel.   So $g$
is Borel measurable, by 4A3?\ \Qed}
}%not needed I think, until vol 5 anyway

\leader{4A3T}{Cylindrical \dvrocolon{$\sigma$-algebras}}\cmmnt{ I 
offer a note on a particular type of Baire $\sigma$-algebra.

\medskip

\noindent}{\bf Definition} Let $X$ be a linear topological
space.   Then the {\bf cylindrical $\sigma$-algebra} of $X$ is the
smallest $\sigma$-algebra $\Sigma$ of subsets of $X$ such that every
continuous linear functional on $X$ is $\Sigma$-measurable.

\leader{4A3U}{Proposition} Let $X$ be a linear
topological space and $\frak T_s=\frak T_s(X,X^*)$ its weak topology.
Then the cylindrical $\sigma$-algebra of $X$ is just the Baire
$\sigma$-algebra of $(X,\frak T_s)$.

\proof{{\bf (a)} Let $\familyiI{f_i}$ be a Hamel basis for $X^*$
(4A4Ab).   For $x\in X$, set $Tx=\familyiI{f_i(x)}$;  then
$T:X\to\BbbR^I$ is a linear operator.   Now
$Y=T[X]$ is dense in $\BbbR^I$.   \Prf\ $Y$ is a linear
subspace of $\BbbR^I$, so its closure $\overline{Y}$ also is (2A5Ec).
If $\phi\in(\BbbR^I)^*$ is such that $\phi(Tx)=0$ for every
$x\in X$, there are a finite set $J\subseteq I$ and a family
$\family{i}{J}{\alpha_i}\in\BbbR^J$ such that
$\phi(y)=\sum_{i\in J}\alpha_iy(i)$ for every $y\in\BbbR^I$ (4A4Be).
In this case, $\sum_{i\in J}\alpha_if_i(x)=\phi(Tx)=0$ for every
$x\in X$;  but $\familyiI{f_i}$ is linearly independent, so $\alpha_i=0$
for every $i\in J$ and $\phi=0$.   By 4A4Eb, $\overline{Y}$ must be the
whole of $\BbbR^I$.\ \Qed

\medskip

{\bf (b)(i)} Set $\frak T'_s=\{T^{-1}[H]:H\subseteq Y$ is open$\}$.
Then $\frak T'_s=\frak T_s$.   \Prf\ Because every $f_i$ is continuous,
$T$ is continuous, so $\frak T'_s\subseteq\frak T_s$.   On the other
hand, any $f\in X^*$ is a linear combination of the $f_i$, so is
$\frak T'_s$-continuous, and $\frak T_s\subseteq\frak T'_s$.\ \Qed

\medskip

\quad{\bf (ii)} If $g:X\to\Bbb R$ is $\frak T_s$-continuous, there is a
continuous $g_1:Y\to\Bbb R$ such that $g=g_1T$.   \Prf\ If $x$,
$x'\in X$ are such that $Tx=Tx'$, then every $\frak T_s$-open set
containing one must contain the other, so $g(x)=g(x')$.   This means
that there is a function $g_1:Y\to\Bbb R$ such that $g=g_1T$.
Next, if $U\subseteq\Bbb R$ is open, $T^{-1}[g_1^{-1}[U]]=g^{-1}[U]$
belongs to $\frak T_s=\frak T'_s$, so $g_1^{-1}[U]$ must be open in $Y$;
as $U$ is arbitrary, $g_1$ is continuous.\ \Qed

\medskip

{\bf (c)} Now let $\Sigma$ be the cylindrical $\sigma$-algebra of $X$.
Then every $f_i:X\to\Bbb R$ is $\Sigma$-measurable, so $T:X\to\Bbb R^I$
is $(\Sigma,\CalBa(\BbbR^I))$-measurable, by 4A3Ne.
Because $Y$ is dense in $\BbbR^I$, $\CalBa(Y)$ is the subspace
$\sigma$-algebra induced by
$\CalBa(\BbbR^I)$ (4A3Nd).   So $T:X\to Y$ is
$(\Sigma,\CalBa(Y))$-measurable.   Now if $g:X\to\Bbb R$ is a continuous
function, there is a continuous function $g_1:Y\to\Bbb R$ such that
$g=g_1T$;  $g_1$ is $\CalBa(Y)$-measurable, so $g$ is
$\Sigma$-measurable.   As this is true for every $\frak T_s$-continuous
$g:X\to\Bbb R$, $\CalBa(X)\subseteq\Sigma$.

\medskip

{\bf (d)} On the other hand, $\Sigma\subseteq\CalBa(X)$ just because
every member of $X^*$ is $\frak T_s$-continuous.   So
$\Sigma=\CalBa(X)$, as claimed.
}%end of proof of 4A3U

\leader{4A3V}{Proposition} Let $(X,\frak T)$ be a separable metrizable
locally convex linear topological space, and
$\frak T_s=\frak T_s(X,X^*)$ its weak topology.   Then the cylindrical
$\sigma$-algebra of $X$ is also both the Baire $\sigma$-algebra and the
Borel $\sigma$-algebra for both $\frak T$ and $\frak T_s$.

\proof{{\bf (a)} For any linear topological space, we have

\Centerline{$\Sigma\subseteq\CalBa(X,\frak T_s)
\subseteq\CalBa(X,\frak T)\subseteq\Cal B(X,\frak T)$,
\quad$\CalBa(X,\frak T_s)\subseteq\Cal B(X,\frak T_s)
\subseteq\Cal B(X,\frak T)$,}

\noindent writing $\Sigma$ for the cylindrical $\sigma$-algebra and
$\CalBa(X,\frak T_s)$, $\CalBa(X,\frak T)$, $\Cal B(X,\frak T_s)$
and $\Cal B(X,\frak T)$ for the Baire and Borel $\sigma$-algebras of the
two topologies.   So all I have to do is to show that
$\Cal B(X,\frak T)\subseteq\Sigma$.

\medskip

{\bf (b)} Let $\sequencen{\tau_n}$ be a sequence of seminorms defining the
topology of $X$ (4A4Cf), and $D\subseteq X$ a countable dense set;  for
$n\in\Bbb N$, $\delta>0$ and $x\in X$, set
$U_n(x,\delta)=\{y:\tau_i(y-x)<\delta$ for every $i\le n\}$.   Then
every $U_n(x,\delta)$ is a convex open set.
Set

\Centerline{$\Cal U=\{U_n(z,2^{-m}):z\in D,\,m,\,n\in\Bbb N\}$.}

\medskip

{\bf (c)} If $G\subseteq X$ is any convex open set,
$\overline{G}\in\Sigma$.
\Prf\ Set $\Cal U_G=\{U:U\in\Cal U,\,U\cap G=\emptyset\}$.
For each $U\in\Cal U_G$, there are $f_U\in X^*$, $\alpha_U\in\Bbb R$
such that $f_U(x)<\alpha_U$ for every $x\in G$ and $f_U(x)>\alpha_U$ for
every $x\in U$ (4A4Db).   So $F=\{x:f_U(x)\le\alpha_U$ for every
$U\in\Cal U_G\}$ belongs to $\Sigma$.   Of course
$F\supseteq\overline{G}$.   On the other hand, if $x\notin\overline{G}$,
there are $m$, $n\in\Bbb N$ such that
$G\cap U_n(x,2^{-m})=\emptyset$;  if we take
$z\in D\cap U_n(x,2^{-m-1})$, $U=U_n(z,2^{-m-1})$ then
$x\in U\in\Cal U_G$, so $f_U(x)>\alpha_U$ and $x\notin F$.   Thus
$\overline{G}=F$ belongs to $\Sigma$.\ \Qed

\medskip

{\bf (d)} In particular, $\Cal V=\{\overline{U}:U\in\Cal U\}$ is
included in $\Sigma$.   But $\Cal V$ is a countable network for
$\frak T$.   So every member of $\frak T$ is a union of countably many
members of $\Sigma$ and belongs to $\Sigma$.   It follows at once that
$\Cal B(X,\frak T)\subseteq\Sigma$, as required.
}%end of proof of 4A3V

\leader{4A3W}{C\`adl\`ag functions}\dvAnew{2009, with revision 2012} 
Let $X$ be a Polish space, and $\Cdlg$ the set of \cadlag\
functions\cmmnt{ (definition:  4A2A)} from $\coint{0,\infty}$ to $X$,
with its topology of
pointwise convergence inherited from $X^{\coint{0,\infty}}$.

(a) $\CalBa(\Cdlg)$ is the subspace $\sigma$-algebra induced by
$\CalBa(X^{\coint{0,\infty}})$.

(b) $(\Cdlg,\CalBa(\Cdlg))$ is a standard Borel space.

(c)(i) For any $t\ge 0$, let $\CalBa_t(\Cdlg)$ be the $\sigma$-algebra
of subsets of $\Cdlg$ generated by the functions $\omega\mapsto\omega(s)$
for $s\le t$.   Then
$(s,\omega)\mapsto\omega(s):\Cdlg\times[0,t]\to X$ is
$\Cal B([0,t])\tensorhat\CalBa_t(\Cdlg)$-measurable.

\quad(ii) $(\omega,t)\mapsto\omega(t):\Cdlg\times\coint{0,\infty}\to X$ is
$\Cal B(\coint{0,\infty})\tensorhat\CalBa(\Cdlg)$-measurable.

(d) The set $C(\coint{0,\infty};X)$ of continuous functions from
$\coint{0,\infty}$ to $X$ belongs to $\CalBa(\Cdlg)$.

\proof{{\bf (a)} Use 4A3Nd.

\medskip

{\bf (b)(i)}
Fix a complete metric $\rho$ on $X$ defining its topology.
For $A\subseteq B\subseteq\Bbb R$,
$f\in X^B$ and $\epsilon>0$, set

$$\eqalign{\jump_A(f,\epsilon)
&=\sup\{n:\text{ there is an }I\in[A]^n
\text{ such that }\rho(f(s),f(t))>\epsilon\cr
&\mskip180mu
\text{ whenever }s<t\text{ are successive elements of }I\}\cr}$$

\noindent (see 438P).   Set $D=\Bbb Q\cap\coint{0,\infty}$.
Then any set of the form
$\{f:\jump_A(f,\epsilon)>m\}$ is open, so

$$\eqalign{E
&=\bigcap_{n\in\Bbb N}\bigcup_{m\in\Bbb N}
  \{f:f\in X^D,\,\jump_{D\cap[0,n]}(f,2^{-n})\le m\}\cr
&\mskip150mu  \cap\bigcap_{\Atop{q'\in D}{n\in\Bbb N}}
   \bigcup_{m\ge n}
   \{f:f\in X^D,\,\rho(f(q'+2^{-m}),f(q'))\le 2^{-n}\}\cr}$$

\noindent is Borel in the Polish space $X^D$.

\medskip

\quad{\bf (ii)} If $\omega\in\Cdlg$, then $\jump_{[0,n]}(\omega,2^{-n})$
is finite for every $n\in\Bbb N$.   (Apply 438Pa to an extension of
$\omega$ to a member of $X^{\Bbb R}$ which is constant on
$\ooint{-\infty,0}$;  or make the trifling required changes to the
argument of 438Pa.)   Since $\lim_{n\to\infty}\omega(q'+2^{-n})=\omega(q')$
for every $q'\in D$, $\omega\restr D\in E$.

\medskip

\quad{\bf (iii)} Conversely, given $f\in E$, $\jump_{[0,n]}(f,\epsilon)$
is finite for every $n\in\Bbb N$ and $\epsilon>0$, so
$\lim_{q\in D,q\downarrow t}f(q)$ is defined in $X$ for every $t\ge 0$,
and $\lim_{q\in D,q\uparrow t}f(q)$ is defined in $X$ for every $t>0$.
(Apply the argument of (a-ii) of the proof of 438P.)   Set
$\omega_f(t)=\lim_{q\in D,q\downarrow t}f(q)$ for $t\ge 0$.
Because $f(q)$ is a cluster point of $\sequencen{f(q+2^{-n})}$ for every
$q\in D$, $\omega_f$ extends $f$.   It is easy to see that
$\jump_{\coint{0,n}}(\omega_f,\epsilon)=\jump_{\coint{0,n}}(f,\epsilon)$
is finite for every $n\in\Bbb N$ and $\epsilon>0$, so
$\lim_{s\downarrow t}\omega_f(s)$ is defined for every $t\ge 0$ and
$\lim_{s\uparrow t}\omega_f(s)$ is defined for every $t>0$.
Also, for $t\ge 0$,

\Centerline{$\lim_{s\downarrow t}\omega_f(s)
=\lim_{q\in D,q\downarrow t}\omega_f(q)
=\lim_{q\in D,q\downarrow t}f(q)=\omega_f(t)$.}

\noindent Thus $\omega_f\in\Cdlg$.   Clearly a member of $\Cdlg$ is
uniquely determined by its values on $D$, so $f\mapsto\omega_f$ and
$\omega\mapsto\omega\restr D$ are the two halves of a
bijection between $E$ and $\Cdlg$.

\medskip

\quad{\bf (iv)} By 424G, $E$, with its Borel (or Baire)
$\sigma$-algebra $\Cal B(E)$, is a standard Borel space.
Of course the map $\omega\mapsto\omega\restr D$ is
$(\CalBa(\Cdlg),\Cal B(E))$-measurable.   But also the map
$f\mapsto\omega_f(t):E\to X$ is $\Cal B(E)$-measurable for every
$t\ge 0$.   \Prf\ If $\sequencen{q_n}$ is a sequence in $D$
decreasing to $t$, $\omega_f(t)=\lim_{n\to\infty}f(q_n)$ for every
$f\in E$, and we can use 418Ba.\ \QeD\  Since $\CalBa(\Cdlg)$ is the
$\sigma$-algebra induced by $\CalBa(X^{\coint{0,\infty}})$ (4A3Nd), and
$\CalBa(X^{\coint{0,\infty}})
=\Tensorhat_{\coint{0,\infty}}\Cal B(X)$ (4A3Na),
this is enough to show that
$f\mapsto\omega_f$ is $(\Cal B(E),\CalBa(\Cdlg))$-measurable.

Thus $(\Cdlg,\CalBa(\Cdlg))\cong(E,\Cal B(E))$ is a standard Borel
space.

\medskip

{\bf (c)(i)} For $n\in\Bbb N$, $\omega\in\Cdlg$ and $s\in[0,t]$, set
$h_n(s,\omega)=\omega(\min(t,2^{-n}i))$ if $i\in\Bbb N$ and
$2^{-n}(i-1)<s\le 2^{-n}i$.   If $G\subseteq X$ is open then

$$\eqalign{\{(s,\omega):s\le t,\,h_n(s,\omega)\in G\}
&=\bigcup_{i\in\Bbb N}([0,t]\cap\ocint{2^{-n}(i-1),2^{-n}i})\cr
&\mskip100mu
\times\{\omega:\omega(\min(t,2^{-n}i))\in G\}\cr
&\in\Cal B([0,t])\tensorhat\CalBa_t(\Cdlg),\cr}$$

\noindent so
$h_n$ is $\Cal B([0,t])\tensorhat\CalBa_t(\Cdlg)$-measurable.
Now $\omega(s)=\lim_{n\to\infty}h_n(s,\omega)$ for every
$\omega\in\Cdlg$ and $s\in[0,t]$, so $(s,\omega)\mapsto\omega(s)$ is
measurable in the same sense, by 418Ba again.

\medskip

\quad{\bf (ii)} If $G\subseteq X$ is open then

\Centerline{$\{(t,\omega):t\ge 0,\,\omega(t)\in G\}
=\bigcup_{n\in\Bbb N}\{(t,\omega):t\in[0,n],\,\omega(t)\in G\}
\in\Cal B(\coint{0,\infty})\tensorhat\CalBa(\Cdlg)$.}

\medskip

{\bf (d)}

$$\eqalign{C(\coint{0,\infty};X)
&=\bigcap_{n\in\Bbb N}\bigcup_{m\in\Bbb N}
  \{\omega:\omega\in\Cdlg,\,\rho(\omega(q),\omega(q'))\le 2^{-n}\cr
&\mskip100mu
  \text{ whenever }q,\,q'\in\Bbb Q\cap[0,n]\text{ and }
    |q-q'|\le 2^{-m}\}.\cr}$$
}%end of proof of 4A3W

\exercises{\leader{4A3X}{Basic exercises (a)}
%\spheader 4A3Xa
(i) Let $X$ be a regular space with a countable network and
$Y$ any topological space.   Show that
$\Cal B(X\times Y)=\Cal B(X)\tensorhat\Cal B(Y)$.  \Hint{4A2Ng.}
(ii) Set $X=[0,1]$ with the topology generated by
$\{X\setminus\{x\}:x\in X\}$ and $Y=[0,1]$ with its discrete topology.
Show that $X$ has a countable network and that $\{(x,x):x\in[0,1]\}$
is a closed subset of $X\times Y$ not belonging to
$\Cal B(X)\times\Cal B(Y)$.
%4A3G

\spheader 4A3Xb Let $X$ be a topological space, and $E\subseteq X$.
Show that the following are equiveridical:   (i) $E\in\CalBa(X)$;  (ii)
there are a continuous function
$f:X\to[0,1]^{\Bbb N}$ and $F\in\Cal B([0,1]^{\Bbb N})$ such
that $E=f^{-1}[F]$.
%4A3L

\sqheader 4A3Xc Let $X$ be a topological space, and $K\subseteq X$ a
compact set such that $K\in\CalBa(X)$.   Show that $K$ is a zero set.
\Hint{4A3Xb.}
%4A3Xb 4A3L

\spheader 4A3Xd Let $X$ be a compact Hausdorff space such that
$\CalBa(X)=\Cal B(X)$.   Show that $X$ is perfectly normal.
%4A3L 4A3Xc

\spheader 4A3Xe Let $\frak S$ be the topology on $\Bbb R$ generated by
the usual topology and $\{\{x\}:x\in\Bbb R\setminus\Bbb Q\}$.   Show
that $\frak S$ is completely regular and Hausdorff and that $\Bbb Q$ is
a closed Baire set which is not a zero set.
%4A3Xc 4A3L
%what if we have local compactness?

\sqheader 4A3Xf Let $X$ be a ccc completely regular topological space.
Show that any nowhere dense set is included in a nowhere dense zero set.
%4A3M repeated in 5A4E(d-i)

\spheader 4A3Xg Let $\familyiI{X_i}$ be a family of spaces with
countable networks, and $X$ their product.   Show that
$\CalBa(X)=\Tensorhat_{i\in I}\CalBa(X_i)$.
%4A3N

\sqheader 4A3Xh(i) Let $I$ be an uncountable set with its discrete
topology, and $X$ the one-point compactification of $I$.   Show that
$\CalBa(I)$ is not the subspace $\sigma$-algebra generated by
$\CalBa(X)$.  (ii) Show that $\omega_1$, with its order topology,
has a subset $I$ such that
$\CalBa(I)$ is not the subspace $\sigma$-algebra induced by
$\CalBa(\omega_1)$.
%4A3Kb 4A3P

\leader{4A3Y}{Further exercises (a)}
%\spheader 4A3Ya
Let $X$ be a \v{C}ech-complete space and $\sequencen{E_n}$ a sequence of
Borel sets in $X$.   Show that there is a \v{C}ech-complete topology on
$X$, finer than the given topology, with the same weight, and containing
every $E_n$.
%4A3I

\spheader 4A3Yb Let $\familyiI{X_i}$ be a family of regular spaces with
countable networks, with product $X$.   Show that (i) $\CalBa(X)$ is the
family of Borel subsets of $X$ which are
determined by coordinates in countable sets;  (ii) a set $Z\subseteq X$
is a zero set iff it is closed and
determined by coordinates in a countable set;  (iii) if $Y\subseteq X$
is dense, then $\CalBa(Y)$ is just the subspace
$\sigma$-algebra $\CalBa(X)_Y$ induced by $\CalBa(X)$.
%4A3N

\spheader 4A3Yc Let $X$ be a normal topological space and $Y$ a closed
subset of $X$.
Show that every Baire subset of $Y$ is the intersection of $Y$ with a
Baire subset of $X$.   \Hint{use Tietze's theorem.}
%4A3L 4{}12P 4A3N

%Skorokhod metric?
}%end of exercises

\endnotes{
\Notesheader{4A3} Much of this section consists of easy technicalities.
It is however not always easy to guess at the exact results obtainable
by these methods.
It is important to notice that Baire $\sigma$-algebras on subspaces can
give difficulties which do not arise with Borel $\sigma$-algebras
(4A3Xh, 4A3Ca).   I have expressed 4A3D in terms of `hereditarily
Lindel\"of' spaces.   Of course the separable metrizable spaces form by
far the most important class of these, but there are others (the split
interval, for instance) which are of great interest in measure theory.
Similarly, there are important products of non-metrizable spaces which
are ccc (e.g., 417Xt(vii)), so that 4A3Mb has something to say.

4A3H-4A3I are a most useful tool in studying Borel subsets of Polish
spaces, especially in conjunction with the First Separation Theorem
(422I);  see 424G and 424H.
I include 4A3Ya and 4A3Yb to show that some more of the arguments here
can be adapted to non-separable or non-metrizable spaces.

You will note my caution in the definition of `Baire measurable'
function (4A3Ke).   This is supposed to cover the case of functions
taking values in $[-\infty,\infty]$ without taking a position on
functions between general topological spaces.

It is relatively easy to show that spaces of \cadlag\ functions have
standard Borel structures (4A3Wb).   To exhibit usable complete metrics
generating these is another matter;  see chap.\ 3 of {\smc Billingsley 99}.
}%end of notes

\discrpage

