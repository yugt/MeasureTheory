\frfilename{mt475.tex}
\versiondate{24.1.13}
\copyrightdate{2000}

\def\chaptername{Geometric measure theory}
\def\sectionname{The essential boundary}

\newsection{475}

The principal aim of this section is to translate Theorem 474E into
geometric terms.   We have already identified the Federer exterior
normal as an outward-normal function\cmmnt{ (474R)}, so we need to
find a description of perimeter measures.   Most remarkably, these turn
out, in every case considered in 474E,
to be just normalized Hausdorff measures (475G).   This
description needs the
concept of `essential boundary' (475B).   In order to complete the
programme, we need to be able to determine which sets have `locally
finite perimeter';  there is a natural criterion in the same language
(475L).   We now have all the machinery for a direct statement of the
Divergence Theorem (for Lipschitz functions) which depends on nothing
more advanced than the definition of Hausdorff measure (475N).   (The
definitions, at least, of `Federer exterior normal' and `essential
boundary' are elementary.)

This concludes the main work of the first part of this chapter.   But
since we are now
within reach of a reasonably direct proof of a fundamental fact about
the $(r-1)$-dimensional Hausdorff measure of the boundaries of subsets
of
$\BbbR^r$ (475Q), I continue up to Cauchy's Perimeter Theorem and the
Isoperimetric Theorem for convex sets (475S, 475T).


\cmmnt{\leader{475A}{Notation} As in the last two sections, $r$ will
be an integer (greater than or equal to $2$, unless explicitly permitted
to take the value $1$).   $\mu$ will be Lebesgue measure on $\BbbR^r$;
I will sometimes write
$\mu_{r-1}$ for Lebesgue measure on $\BbbR^{r-1}$ and $\mu_1$ for
Lebesgue measure on $\Bbb R$.   $\beta_r=\mu B(\tbf{0},1)$ will be the
measure of the unit ball in $\BbbR^r$, and
$S_{r-1}=\partial B(\tbf{0},1)$ will be the unit sphere.   $\nu$ will be
normalized
$(r-1)$-dimensional Hausdorff measure on $\BbbR^r$ (265A), that is,
$\nu=2^{-r+1}\beta_{r-1}\mu_{H,r-1}$, where $\mu_{H,r-1}$ is
$(r-1)$-dimensional Hausdorff measure on $\BbbR^r$.   Recall that
$\nu S_{r-1}=r\beta_r$ (265H).
I will take it for granted that $x\in\BbbR^r$ has coordinates
$(\xi_1,\ldots,\xi_r)$.

If $E\subseteq\BbbR^r$ has locally finite perimeter (474D),
$\lambda^{\partial}_E$ will be its perimeter measure (474F),
$\partial^{\$}E$ its reduced boundary (474G) and $\psi_E$ its
canonical outward-normal function (474G).
}%end of comment

\vleader{40pt}{475B}{The essential boundary}\cmmnt{ (In this paragraph I
allow $r=1$.)}   Let $A\subseteq\BbbR^r$ be any
set.   The {\bf essential closure} of $A$ is the set

\Centerline{$\clstar A=\{x:\limsup_{\delta\downarrow 0}
\Bover{\mu^*(B(x,\delta)\cap A)}{\mu B(x,\delta)}>0\}$}

\noindent\cmmnt{(see 266B).}
Similarly, the {\bf essential interior} of $A$ is the set

\Centerline{$\intstar A=\{x:\liminf_{\delta\downarrow 0}
\Bover{\mu_*(B(x,\delta)\cap A)}{\mu B(x,\delta)}=1\}$.}

\noindent\cmmnt{(If $A$ is Lebesgue measurable, this is the lower
Lebesgue density of $A$,
as defined in 341E;  see also 223Yf.)  }Finally,
the {\bf essential boundary} $\partstar A$ of $A$
is\cmmnt{ the difference} $\clstar A\setminus\intstar A$.

\cmmnt{Note that if $E\subseteq\BbbR^r$ is Lebesgue measurable
then $\BbbR^r\setminus\partstar E$ is the Lebesgue set of the function
$\chi E$, as defined in 261E.}

\leader{475C}{Lemma}\cmmnt{ (In this lemma I allow $r=1$.)} Let $A$,
$A'\subseteq\BbbR^r$.

(a)
\Centerline{$\interior A\subseteq\intstar A
\subseteq\clstar A\subseteq\overline{A}$,
\quad$\partstar A\subseteq\partial A$,}

\Centerline{$\clstar A=\BbbR^r\setminus\intstar(\BbbR^r\setminus A)$,
\quad$\partstar(\BbbR^r\setminus A)=\partstar A$.}

(b) If $A\setminus A'$ is negligible, then
$\clstar A\subseteq\clstar A'$ and $\intstar A\subseteq\intstar A'$;
\cmmnt{in particular,} if $A$ itself is negligible, $\clstar A$,
$\intstar A$ and $\partstar A$ are all empty.

(c) $\intstar A$, $\clstar A$ and $\partstar A$ are Borel sets.

(d) $\clstar(A\cup A')=\clstar A\cup\clstar A'$ and
$\intstar(A\cap A')=\intstar A\cap\intstar A'$, so
$\partstar(A\cup A')$, $\partstar(A\cap A')$ and
$\partstar(A\symmdiff A')$ are all included in
$\partstar A\cup\partstar A'$.

(e) $\clstar A\cap\intstar A'\subseteq\clstar(A\cap A')$,
$\partstar A\cap\intstar A'\subseteq\partstar(A\cap A')$ and
$\partstar A\setminus\clstar A'\subseteq\partstar(A\cup A')$.

(f) $\partstar(A\cap A')
  \subseteq(\clstar A'\cap\partial A)\cup(\partstar A'\cap\interior A)$.

(g) If $E\subseteq\BbbR^r$ is Lebesgue measurable, then
$E\symmdiff\intstar E$, $E\symmdiff\clstar E$ and $\partstar E$ are
Lebesgue negligible.

(h) $A$ is Lebesgue measurable iff $\partstar A$ is Lebesgue negligible.

\proof{{\bf (a)} It is obvious that

\Centerline{$\interior A\subseteq\intstar A
\subseteq\clstar A\subseteq\overline{A}$,}

\noindent so that $\partstar A\subseteq\partial A$.   Since

\Centerline{$\Bover{\mu^*(B(x,\delta)\cap A)}{\mu B(x,\delta)}
+\Bover{\mu_*(B(x,\delta)\setminus A)}{\mu B(x,\delta)}
=1$}

\noindent for every $x\in\BbbR^r$ and every $\delta>0$ (413Ec),
$\BbbR^r\setminus\intstar A=\clstar(\BbbR^r\setminus A)$.   It follows
that

$$\eqalign{\partstar(\BbbR^r\setminus A)
&=\clstar(\BbbR^r\setminus A)
  \symmdiff\intstar(\BbbR^r\setminus A)\cr
&=(\BbbR^r\setminus\intstar A)\symmdiff(\BbbR^r\setminus\clstar A)
=\intstar A\symmdiff\clstar A
=\partstar A.\cr}$$

\medskip

{\bf (b)} If $A\setminus A'$ is negligible, then

\Centerline{$\mu_*(B(x,\delta)\cap A)\le\mu_*(B(x,\delta)\cap A')$,
\quad$\mu^*(B(x,\delta)\cap A)\le\mu^*(B(x,\delta)\cap A')$}

\noindent for all $x$ and $\delta$, so $\intstar A\subseteq\intstar A'$
and $\clstar A\subseteq\clstar A'$.

\medskip

{\bf (c)} The point is just that
$(x,\delta)\mapsto\mu^*(A\cap B(x,\delta))$ is continuous.   \Prf\ For
any $x$, $y\in\BbbR^r$ and $\delta$, $\eta>0$ we have

$$\eqalign{|\mu^*(A\cap B(y,\eta))-\mu^*(A\cap B(x,\delta))|
&\le\mu(B(y,\eta)\symmdiff B(x,\delta))\cr
&=2\mu(B(x,\delta)\cup B(y,\eta))-\mu B(x,\delta)-\mu B(y,\eta)\cr
&\le\beta_r\bigl(2(\max(\delta,\eta)+\|x-y\|)^r-\delta^r-\eta^r\bigr)
\to 0\cr}$$


\noindent as $(y,\eta)\to(x,\delta)$.\ \QeD\  So

\Centerline{$x\mapsto
\limsup_{\delta\downarrow 0}
  \Bover{\mu^*(A\cap B(x,\delta))}{\mu B(x,\delta)}
=\inf_{\alpha\in\Bbb Q,\alpha>0}\sup_{\gamma\in\Bbb Q,0<\gamma\le\alpha}
  \Bover1{\beta_r\gamma^r}\mu^*(A\cap B(x,\gamma))$}

\noindent is Borel measurable, and

\Centerline{$\clstar A=\{x:\limsup_{\delta\downarrow 0}
  \Bover{\mu^*(A\cap B(x,\delta))}{\mu B(x,\delta)}>0\}$}

\noindent is a Borel set.

Accordingly $\intstar A=\BbbR^r\setminus\clstar(\BbbR^r\setminus A)$ and
$\partstar A$ are also Borel sets.

\medskip

{\bf (d)} For any $x\in\BbbR^r$,

$$\eqalign{\limsup_{\delta\downarrow 0}
  \Bover{\mu^*((A\cup A')\cap B(x,\delta))}{\mu B(x,\delta)}
&\le\limsup_{\delta\downarrow 0}
  \Bover{\mu^*(A\cap B(x,\delta))}{\mu B(x,\delta)}
   +\Bover{\mu^*(A'\cap B(x,\delta))}{\mu B(x,\delta)}\cr
&\le\limsup_{\delta\downarrow 0}
  \Bover{\mu(A\cap B(x,\delta))}{\mu B(x,\delta)}
   +\limsup_{\delta\downarrow 0}
  \Bover{\mu(A'\cap B(x,\delta))}{\mu B(x,\delta)},\cr}$$

\noindent so $\clstar(A\cup A')\subseteq\clstar A\cup\clstar A'$.
By (b), $\clstar A\cup\clstar A'\subseteq\clstar(A\cup A')$, so we have
equality.   Accordingly

\Centerline{$\intstar(A\cap A')
=\BbbR^r\setminus\clstar((\BbbR^r\setminus A)\cup(\BbbR^r\setminus A'))
=\intstar A\cap\intstar A'$.}

\noindent Since
$\intstar(A\cup A')\supseteq\intstar A\cup\intstar A'$,
$\partstar(A\cup A')\subseteq\partstar A\cup\partstar A'$.   Now

\Centerline{$\partstar(A\cap A')
=\partstar(\BbbR^r\setminus(A\cap A'))
\subseteq\partstar(\BbbR^r\setminus A)
  \cup\partstar(\BbbR^r\setminus A'))
=\partstar A\cup\partstar A'$}

\noindent and

\Centerline{$\partstar(A\symmdiff A')
\subseteq\partstar(A\cap(\BbbR^r\setminus A'))
  \cup\partstar(A'\cap(\BbbR^r\setminus A))
\subseteq\partstar A\cup\partstar A'$.}

\medskip

{\bf (e)} If $x\in\clstar A\cap\intstar A'$, then
$x\notin\clstar(\BbbR^r\setminus A')$ so
$x\notin\clstar(A\setminus A')$.   But from (d) we know that
$x\in\clstar(A\cap A')\cup\clstar(A\setminus A')$, so
$x\in\clstar(A\cap A')$.

Now

$$\eqalign{\partstar A\cap\intstar A'
&=(\clstar A\cap\intstar A')\setminus\intstar A\cr
&\subseteq\clstar(A\cap A')\setminus\intstar(A\cap A')
=\partstar(A\cap A'),\cr
\cr
\partstar A\setminus\clstar A'
&=\partstar(\BbbR^r\setminus A)\cap\intstar(\BbbR^r\setminus A')\cr
&\subseteq\partstar((\BbbR^r\setminus A)\cap(\BbbR^r\setminus A'))\cr
&=\partstar(\BbbR^r\setminus(A\cup A'))
=\partstar(A\cup A').\cr}$$

\medskip

{\bf (f)} If $x\in\partstar(A\cap A')\cap\partial A$, then of course
$x\in\clstar(A\cap A')\subseteq\clstar A$, so
$x\in\clstar A\cap\partial A$.   If
$x\in\partstar(A\cap A')\setminus\partial A$, then surely
$x\in\overline{A}$, so $x\in\interior A$.   But this means that

\Centerline{$\mu_*(B(x,\delta)\cap A')=\mu_*(B(x,\delta)\cap A\cap A')$,
\quad$\mu^*(B(x,\delta)\cap A')=\mu^*(B(x,\delta)\cap A\cap A')$}

\noindent for all $\delta$ small enough, so $x\in\partstar A'$ and
$x\in\partstar A'\cap\interior A$.

\medskip

{\bf (g)} Applying 472Da to $\mu$ and $\chi E$, we see that

$$\eqalign{\partstar E
&\subseteq(E\symmdiff\clstar E)\cup(E\symmdiff\intstar E)\cr
&\subseteq\{x:\chi E(x)\ne\lim_{\delta\downarrow 0}
  \Bover{\mu(E\cap B(x,\delta))}{\mu B(x,\delta)}\}\cr}$$

\noindent are all $\mu$-negligible.

\medskip

{\bf (h)} If $A$ is Lebesgue measurable then (g) tells us that
$\partstar A$ is negligible.   If $\partstar A$ is negligible, let $E$
be a measurable envelope
of $A$.   Then $\mu(E\cap B(x,\delta))=\mu^*(A\cap B(x,\delta))$ for all
$x$ and $\delta$, so $\clstar E=\clstar A$.   Similarly, if $F$ is a
measurable envelope
of $\BbbR^r\setminus A$, then $\clstar F=\clstar(\BbbR^r\setminus
A)=\BbbR^r\setminus\intstar A$ (using (a)).   Now (g) tells us that

\Centerline{$\mu(E\cap F)=\mu(\clstar E\cap\clstar F)=\mu(\clstar
A\setminus\intstar A)=0$.}

\noindent But now $A\setminus E$ and $E\setminus A\subseteq(E\cap
F)\cup((\BbbR^r\setminus A)\setminus F)$ are Lebesgue negligible, so $A$
is Lebesgue measurable.
}%end of proof of 475C

\leader{475D}{Lemma} Let $E\subseteq\BbbR^r$ be a set with locally
finite
perimeter\cmmnt{, and $\partial^{\$}E$ its reduced boundary}.   Then
$\partial^{\$}E\subseteq\partstar E$ and
$\nu(\partstar E\setminus\partial^{\$}E)=0$.

\proof{{\bf (a)} By 474N(i), $\partial^{\$}E\subseteq\clstar E$;  by
474N(ii), $\partial^{\$}E\cap\intstar E=\emptyset$;  so
$\partial^{\$}E\subseteq\partstar E$.

\medskip

{\bf (b)} For any $y\in\partstar E$,

\Centerline{$\limsup_{\delta\downarrow 0}\Bover1{\delta^{r-1}}
  \lambda^{\partial}_EB(y,\delta)>0$.}

\noindent\Prf\ We have an $\epsilon\in\ocint{0,\bover12}$ such that

\Centerline{$\liminf_{\delta\downarrow 0}
  \Bover{\mu(E\cap B(y,\delta))}{\mu B(y,\delta)}<1-\epsilon$,
\quad$\limsup_{\delta\downarrow 0}
  \Bover{\mu(E\cap B(y,\delta))}{\mu B(y,\delta)}>\epsilon$.}

\allowmorestretch{468}{
\noindent Since the function
$\delta\mapsto\Bover{\mu(E\cap B(y,\delta))}{\mu B(y,\delta)}$ is
continuous, there is a sequence $\sequencen{\delta_n}$ in
$\ooint{0,\infty}$ such that $\lim_{n\to\infty}\delta_n=0$ and
}

\Centerline{$\epsilon\mu B(y,\delta_n)\le\mu(E\cap B(y,\delta_n))
\le(1-\epsilon)\mu B(y,\delta_n)$}

\noindent for every $n$.   Now from 474Lb we have

$$\eqalign{(\epsilon\beta_r)^{(r-1)/r}\delta_n^{r-1}
&=(\epsilon\mu B(y,\delta_n))^{(r-1)/r}\cr
&\le\min(\mu(B(y,\delta_n)\cap E),
   \mu(B(y,\delta_n)\setminus E))^{(r-1)/r}
\le 2c\lambda^{\partial}_EB(y,\delta_n)\cr}$$

\noindent for every $n$, where $c>0$ is the constant there.
But this means that

\Centerline{$\limsup_{\delta\downarrow 0}\Bover1{\delta^{r-1}}
  \lambda^{\partial}_EB(y,\delta)
\ge\limsup_{n\to\infty}\Bover1{\delta_n^{r-1}}
  \lambda^{\partial}_EB(y,\delta_n)
\ge\Bover1{2c}(\epsilon\beta_r)^{(r-1)/r}
>0$.  \Qed}

\medskip

{\bf (c)} Let $\epsilon>0$.   Set

\Centerline{$F_{\epsilon}
=\{y:y\in\BbbR^r\setminus\partial^{\$}E,\,
\limsup_{\delta\downarrow 0}\Bover1{\delta^{r-1}}
\lambda^{\partial}_EB(y,\delta)>\epsilon\}$.}

\noindent Because $\partial^{\$}E$ is
$\lambda^{\partial}_E$-conegligible (474G),
$\lambda^{\partial}_EF_{\epsilon}=0$.   So there is an open set
$G\supseteq F_{\epsilon}$ such that $\lambda^{\partial}_EG\le\epsilon^2$
(256Bb/412Wb).   Let $\delta>0$.   Let $\Cal I$ be the family of all
those
non-singleton closed balls $B\subseteq G$ such that $\diam B\le\delta$
and $\lambda^{\partial}_EB\ge 2^{-r+1}\epsilon(\diam B)^{r-1}$.   Then
every point of $F_{\epsilon}$ is the centre of arbitrarily small members
of $\Cal I$.   By Besicovitch's Covering Lemma (472B), there is a family
$\ofamily{k}{5^r}{\Cal I_k}$ of disjoint countable subsets of $\Cal I$
such that $\Cal I^*=\bigcup_{k<5^r}\bigcup\Cal I_k$ covers
$F_{\epsilon}$.   Now

\Centerline{$\sum_{B\in\Cal I^*}(\diam B)^{r-1}
\le\sum_{k<5^r}\sum_{B\in\Cal I_k}\Bover{2^{r-1}}{\epsilon}
  \lambda^{\partial}_EB
\le\Bover{5^r2^{r-1}}{\epsilon}\lambda^{\partial}_EG
\le 5^r2^{r-1}\epsilon$.}

\noindent As $\delta$ is arbitrary, $\mu_{H,r-1}^*F_{\epsilon}$ is at
most $5^r2^{r-1}\epsilon$ (264Fb/471Dc) and
$\nu^*F_{\epsilon}\le 5^r\beta_{r-1}\epsilon$.   As $\epsilon$ is
arbitrary,

\Centerline{$\partstar E\setminus\partial^{\$}E
\subseteq\{y:y\in\BbbR^r\setminus\partial^{\$}E,\,
\limsup_{\delta\downarrow 0}\Bover1{\delta^{r-1}}
\lambda^{\partial}_EB(y,\delta)>0\}$}

\noindent is $\nu$-negligible, as claimed.
}%end of proof of 475D

\leader{475E}{Lemma} Let $E\subseteq\BbbR^r$ be a set with locally
finite perimeter.

(a) If $A\subseteq\partial^{\$}E$, then
$\nu^*A\le(\lambda^{\partial}_E)^*A$.

(b) If $A\subseteq\BbbR^r$ and $\nu A=0$, then
$\lambda^{\partial}_EA=0$.

\proof{{\bf (a)} Given $\epsilon$, $\delta>0$ let $\Cal I$ be the family
of non-trivial closed balls $B\subseteq\BbbR^r$ of diameter at most
$\delta$ such that $\beta_{r-1}(\Bover12\diam B)^{r-1}
\le(1+\epsilon)\lambda^{\partial}_EB$.   By 474S, every point of $A$ is
the centre of arbitrarily small members of $\Cal I$.   By 472Cb, there is
a countable family $\Cal I_1\subseteq I$ such that
$A\subseteq\bigcup\Cal I_1$ and
$\sum_{B\in\Cal I_1}\lambda^{\partial}_EB
\le(\lambda^{\partial}_E)^*A+\epsilon$.   But this means that

\Centerline{$\sum_{B\in\Cal I_1}(\diam B)^{r-1}
\le(1+\epsilon)\Bover{2^{r-1}}{\beta_{r-1}}
  ((\lambda^{\partial}_E)^*A+\epsilon)$.}

\noindent As $\delta$ is arbitrary,

\Centerline{$\nu^*A
=\Bover{\beta_{r-1}}{2^{r-1}}\mu_{H,r-1}^*A
\le(1+\epsilon)((\lambda^{\partial}_E)^*A+\epsilon)$.}

\noindent As $\epsilon$ is arbitrary, we have the result.

\medskip

{\bf (b)} For $n\in\Bbb N$, set

\Centerline{$A_n=\{x:x\in A$, $\lambda^{\partial}_EB(x,\delta)
\le 2\beta_{r-1}\delta^{r-1}$ whenever $0<\delta\le 2^{-n}\}$.}

\noindent Now, given $\epsilon>0$, there is a sequence
$\sequence{i}{D_i}$ of sets covering $A_n$ such that
$\diam D_i\le 2^{-n}$ for every $i$ and
$\sum_{i=0}^{\infty}(\diam D_i)^{r-1}\le\epsilon$.
Passing over the trivial case $A_n=\emptyset$, we may suppose that for
each $i\in\Bbb N$ there is an $x_i\in A_n\cap D_i$, so that
$D_i\subseteq B(x_i,\diam D_i)$ and

$$\eqalign{(\lambda^{\partial}_E)^*A_n
&\le\sum_{i=0}^{\infty}(\lambda^{\partial}_E)^*D_i
\le\sum_{i=0}^{\infty}\lambda^{\partial}_EB(x_i,\diam D_i)\cr
&\le\sum_{i=0}^{\infty}2\beta_{r-1}(\diam D_i)^{r-1}
\le 2\beta_{r-1}\epsilon.\cr}$$

\noindent As $\epsilon$ is arbitrary, $\lambda^{\partial}_EA_n=0$.
And this is true for every $n$.   As
$\bigcup_{n\in\Bbb N}A_n\supseteq A\cap\partial^{\$}E$ (474S again),
$A\setminus\bigcup_{n\in\Bbb N}A_n$
is $\lambda^{\partial}_E$-negligible (474G), and so is $A$.
}%end of proof of 475E

\leader{475F}{Lemma} Let $E\subseteq\BbbR^r$ be a set with locally
finite perimeter, and $\epsilon>0$.   Then $\lambda^{\partial}_E$ is
inner regular with respect to the family
$\Cal E=\{F:F\subseteq\BbbR^r$ is Borel,
$\lambda^{\partial}_EF\le(1+\epsilon)\nu F\}$.

\proof{{\bf (a)} We need some elementary bits of geometry.

\medskip

\quad{\bf (i)} If $x\in\BbbR^r$, $\delta>0$, $\alpha\ge 0$ and
$v\in S_{r-1}$, then

\Centerline{$\mu\{z:z\in B(x,\delta)$,
$|\varinnerprod{(z-x)}{v}|\le\alpha\}
\le 2\alpha\beta_{r-1}\delta^{r-1}$.}

\noindent\Prf\ Translating and rotating, if necessary, we can reduce to
the case $x=\tbf{0}$, $v=(0,\ldots,1)$.   In this case we are looking at

\Centerline{$\{z:\|z\|\le\delta,\,|\zeta_r|\le\alpha\}
\subseteq\{u:u\in\BbbR^{r-1},\,\|u\|\le\delta\}\times[-\alpha,\alpha]$}

\noindent which has measure $2\alpha\beta_{r-1}\delta^{r-1}$.\ \Qed

\medskip

\quad{\bf (ii)} If $x\in\BbbR^r$, $\delta>0$, $0<\eta\le\bover12$,
$v\in S_{r-1}$, $H=\{z:\varinnerprod{z}{v}\le\alpha\}$ and

\Centerline{$|\mu(H\cap B(x,\delta))-\Bover12\mu B(x,\delta)|
<2^{-r+1}\beta_{r-1}\eta\delta^r$,}

\noindent then $|\varinnerprod{x}{v}-\alpha|\le\eta\delta$.   \Prf\
Again translating and rotating if necessary, we may suppose that
$x=\tbf{0}$ and $v=(0,\ldots,1)$.   Set $H_0=\{z:\zeta_r\le 0\}$.
\Quer\ If $\alpha>\eta\delta$, then $H\cap B(\tbf{0},\delta)$ includes

\Centerline{$(H_0\cap B(\tbf{0},\delta))
\cup(\{u:u\in\BbbR^{r-1},\,\|u\|\le\Bover12\delta\}\times[0,\alpha'])$}

\noindent where $\alpha'=\min(|\alpha|,\Bover{\sqrt3}2\delta)>\eta\delta)$,
so

$$\eqalign{\mu(H\cap B(\tbf{0},\delta))-\Bover12\mu B(\tbf{0},\delta)
&=\mu(H\cap B(\tbf{0},\delta))-\mu(H_0\cap B(\tbf{0},\delta))\cr
&\ge 2^{-r+1}\beta_{r-1}\delta^{r-1}\alpha'
>2^{-r+1}\beta_{r-1}\delta^r\eta,\cr}$$

\noindent contrary to hypothesis.\ \BanG\  Similarly, \Quer\ if
$\alpha<-\eta\delta$, then $H\cap B(\tbf{0},\delta)$ is included in

\Centerline{$H_0\cap B(\tbf{0},\delta)
\setminus(\{u:u\in\BbbR^{r-1},\,
\|u\|\le\Bover12\delta\}\times\ocint{\alpha,0})$,}

\noindent so

$$\eqalign{\Bover12\mu B(\tbf{0},\delta)-\mu(H\cap B(\tbf{0},\delta))
&=\mu(H_0\cap B(\tbf{0},\delta))-\mu(H\cap B(\tbf{0},\delta))\cr
&\ge 2^{-r+1}\beta_{r-1}\delta^{r-1}\alpha'
>2^{-r+1}\beta_{r-1}\delta^r\eta,\cr}$$

\noindent which is equally impossible.\ \BanG\   So
$|\alpha|\le\eta\delta$.\ \Qed

\medskip

{\bf (b)} Let $F$ be such that $\lambda^{\partial}_EF>0$.   Let $\eta$,
$\zeta>0$ be such that

\Centerline{$\eta<1$,
\quad$\Bover{(1+\eta)^2}{(1-\eta)^{r-1}}\le 1+\epsilon$,
\quad$2(1+2^r)\beta_r\zeta<2^{-r}\beta_{r-1}\eta$.}

\noindent Because $\partial^{\$}E$ is
$\lambda^{\partial}_E$-conegligible (474G again),
$\lambda^{\partial}_E(F\cap\partial^{\$}E)>0$.   Because
$\lambda^{\partial}_E$ is a Radon measure (474E) and
$\psi_E:\partial^{\$}E\to S_{r-1}$ is
$\dom(\lambda^{\partial}_E)$-measurable (474E(i), 474G), there is a
compact set
$K_1\subseteq F\cap\partial^{\$}E$ such that $\lambda^{\partial}_EK_1>0$
and $\psi_E\restr K_1$ is continuous, by Lusin's theorem (418J).   For
$x\in\partial^{\$}E$, set
$H_x=\{z:\varinnerprod{(z-x)}{\psi_E(x)}\le 0\}$.   The function

\Centerline{$(x,\delta)\mapsto\mu((E\symmdiff H_x)\cap B(x,\delta)):
K_1\times\ooint{0,\infty}\to\Bbb R$}

\noindent is Borel measurable.   \Prf\ Take a Borel set $E'$ such that
$\mu(E\symmdiff E')=0$.   Then

\Centerline{$\{(x,\delta,z):x\in K_1$,
$z\in(E'\symmdiff H_x)\cap B(x,\delta)\}$}

\noindent is a Borel set in
$\BbbR^r\times\ooint{0,\infty}\times\BbbR^r$, so its sectional measure
is a Borel measurable function, by 252P.\ \Qed

For each $x\in K_1$,

\Centerline{$\lim_{n\to\infty}
  \Bover{\mu((E\symmdiff H_x)\cap B(x,2^{-n}))}{\mu B(x,2^{-n})}=0$}

\noindent (474R).   So there is an $n_0\in\Bbb N$ such that
$\lambda^{\partial}_EF_1>0$, where $F_1$ is the Borel set

\Centerline{$\{x:x\in K_1$, $\mu((E\symmdiff H_x)\cap B(x,2^{-n}))
\le\zeta\mu B(x,2^{-n})$ for every $n\ge n_0\}$.}

\noindent Let $K_2\subseteq F_1$ be a compact set such that
$\lambda^{\partial}_EK_2>0$.

For each $n\in\Bbb N$, the function

\Centerline{$x\mapsto\lambda^{\partial}_EB(x,2^{-n}))
=\lambda^{\partial}_E(x+B(\tbf{0},2^{-n}))$}

\noindent is Borel measurable (444Fe).   Let $y\in K_2$ be such that
$\lambda^{\partial}_E(K_2\cap B(y,\delta))>0$ for every $\delta>0$
(cf.\ 411Nd).   Set $v=\psi_E(y)$.   Let $n>n_0$ be so large that
$2\beta_{r-1}\|\psi_E(x)-v\|\le\beta_r\zeta$ whenever $x\in K_1$ and
$\|x-y\|\le 2^{-n}$.   Set $K_3=K_2\cap B(y,2^{-n-1})$, so that
$\lambda^{\partial}_EK_3>0$.

\medskip

{\bf (c)} We have $|\varinnerprod{(x-z)}{v}|\le\eta\|x-z\|$ whenever
$x$, $z\in K_3$.   \Prf\ If $x=z$ this is trivial.
Otherwise, let $k\ge n$ be such that $2^{-k-1}\le\|x-z\|\le 2^{-k}$, and
set $\delta=2^{-k}$.   Set

\Centerline{$H'_x=\{w:\varinnerprod{(w-x)}{v}\le 0\}$,
\quad$H'_z=\{w:\varinnerprod{(w-z)}{v}\le 0\}$.}

\noindent Since
$|\varinnerprod{(w-x)}{v}-\varinnerprod{(w-x)}{\psi_E(x)}|
\le 2\delta\|\psi_E(x)-v\|$ whenever $w\in B(x,2\delta)$,

\Centerline{$(H_x\symmdiff H'_x)\cap B(x,2\delta)
\subseteq\{w:w\in B(x,2\delta),
  \,|\varinnerprod{(w-x)}{v}|\le 2\delta\|\psi_E(x)-v\|\}$}

\noindent has measure at most

\Centerline{$4\delta\|\psi_E(x)-v\|\beta_{r-1}(2\delta)^{r-1}
\le 2\delta\beta_r\zeta(2\delta)^{r-1}
=\zeta\mu B(x,2\delta)$,}

\noindent using (a-i) for the first inequality.   So

$$\eqalign{\mu((E\symmdiff H'_x)\cap B(x,2\delta))
&\le\mu((E\symmdiff H_x)\cap B(x,2\delta))
+\mu((H_x\symmdiff H'_x)\cap B(x,2\delta))\cr
&\le 2\zeta\mu B(x,2\delta)\cr}$$

\noindent because $k>n_0$ and $x\in F_1$.   Similarly,
$\mu((E\symmdiff H'_z)\cap B(z,\delta))\le 2\zeta\mu B(z,\delta)$.   Now
observe that because $\|x-z\|\le\delta$,
$B(z,\delta)\subseteq B(x,2\delta)$,

\Centerline{$\mu((E\symmdiff H'_x)\cap B(z,\delta))
\le 2\zeta\mu B(x,2\delta)=2^{r+1}\zeta\mu B(z,\delta)$,}

\noindent and

$$\eqalign{\mu((H'_x\symmdiff H'_z)\cap B(z,\delta))
&\le\mu((E\symmdiff H'_x)\cap B(z,\delta))
  +\mu((E\symmdiff H'_z)\cap B(z,\delta))\cr
&\le(2+2^{r+1})\zeta\mu B(z,\delta).\cr}$$

\noindent Since $\mu(H'_z\cap B(z,\delta))=\Bover12\mu B(z,\delta)$,

\Centerline{$|\mu(H'_x\cap B(z,\delta))-\Bover12\mu B(z,\delta)|
\le 2(1+2^r)\zeta\mu B(z,\delta)<2^{-r}\beta_{r-1}\eta\delta^r$,}

\noindent and (using (a-ii) above)

\Centerline{$|\varinnerprod{(x-z)}{v}|\le\Bover12\eta\delta
\le\eta\|x-v\|$.  \Qed}

\medskip

{\bf (d)} Let $V$ be the hyperplane $\{w:\varinnerprod{w}{v}=0\}$, and
let $T:K_3\to V$ be the orthogonal projection, that is,
$Tx=x-(\varinnerprod{x}{v})v$ for every $x\in K_3$.   Then (c)
tells us that if $x$, $z\in K_3$,

\Centerline{$\|Tx-Tz\|\ge\|x-z\|-|\varinnerprod{(x-z)}{v}|
\ge(1-\eta)\|x-z\|$.}

\noindent Because $\eta<1$, $T$ is injective.   Consider the compact set
$T[K_3]$.   The inverse $T^{-1}$ of $T$ is $\Bover1{1-\eta}$-Lipschitz,
and $\nu K_3>0$ (by 475Eb), so
$\nu(T[K_3])\ge(1-\eta)^{r-1}\nu K_3>0$ (264G/471J).
Let $G\supseteq T[K_3]$ be an open set such that
$\nu(G\cap V)\le(1+\eta)\nu(T[K_3])$.   (I am using the fact that the
subspace measure $\nu_V$ induced by $\nu$ on $V$ is a copy of Lebesgue
measure on $\BbbR^{r-1}$, so is a Radon measure.)   Let $\Cal I$ be the
family of non-trivial closed balls $B\subseteq G$ such that
$(1-\eta)^{r-1}\lambda^{\partial}_ET^{-1}[B]
\le(1+\eta)\nu(B\cap V)$.   Then every point of $T[K_3]$ is the centre
of arbitrarily small members of
$\Cal I$.   \Prf\ If $x\in K_3$ and $\delta_0>0$, there is a
$\delta\in\ocint{0,\delta_0}$ such that $B(Tx,\delta)\subseteq G$ and
$\lambda^{\partial}_EB(x,\delta)\le(1+\eta)\beta_{r-1}\delta^{r-1}$
(474S once more).   Now consider $B=B(Tx,(1-\eta)\delta)$.  Then
$T^{-1}[B]\subseteq B(x,\delta)$, so

\Centerline{$\lambda^{\partial}_ET^{-1}[B]
\le\lambda^{\partial}_EB(x,\delta)
\le(1+\eta)\beta_{r-1}\delta^{r-1}
=\Bover{1+\eta}{(1-\eta)^{r-1}}\nu(B\cap V)$.  \Qed}

\noindent By 261B/472Ca, applied in $V\cong\BbbR^{r-1}$, there is a
countable disjoint family $\Cal I_0\subseteq\Cal I$ such that
$\nu(T[K_3]\setminus\bigcup\Cal I_0)=0$.

Now $\nu(K_3\setminus\bigcup_{B\in\Cal I_0}T^{-1}[B])=0$, because
$T^{-1}$ is Lipschitz, so
$\lambda^{\partial}_E(K_3\setminus\bigcup_{B\in\Cal I_0}T^{-1}[B])=0$
(475Eb again), and

$$\eqalignno{\lambda^{\partial}_EK_3
&\le\sum_{B\in\Cal I_0}\lambda^{\partial}_ET^{-1}[B]
\le\Bover{1+\eta}{(1-\eta)^{r-1}}\sum_{B\in\Cal I_0}\nu(B\cap V)\cr
&\le\Bover{1+\eta}{(1-\eta)^{r-1}}\nu(G\cap V)
\le\Bover{(1+\eta)^2}{(1-\eta)^{r-1}}\nu(T[K_3])
\le\Bover{(1+\eta)^2}{(1-\eta)^{r-1}}\nu K_3\cr
\displaycause{because $T$ is $1$-Lipschitz}
&\le(1+\epsilon)\nu K_3\cr}$$

\noindent by the choice of $\eta$.   Thus $K_3\in\Cal E$.

\medskip

{\bf (e)} This shows that every $\lambda^{\partial}_E$-non-negligible set
measured by
$\lambda^{\partial}_E$ includes a $\lambda^{\partial}_E$-non-negligible
member of $\Cal E$.
As $\Cal E$ is closed under disjoint unions, $\lambda^{\partial}_E$ is
inner regular with respect to $\Cal E$ (412Aa).
}%end of proof of 475F

\leader{475G}{Theorem} Let $E\subseteq\BbbR^r$ be a set with locally
finite perimeter.   Then
$\lambda^{\partial}_E=\nu\LLcorner\partstar E$\cmmnt{, that is, for
$F\subseteq\BbbR^r$,
$\lambda^{\partial}_EF=\nu(F\cap\partstar E)$ whenever either is
defined}.

\proof{{\bf (a)} Suppose first that $F$ is a Borel set included in the
reduced boundary $\partial^{\$}E$ of $E$.   Then
$\nu F\le\lambda^{\partial}_EF$, by 475Ea.   On the other hand, for any
$\epsilon>0$ and $\gamma<\lambda^{\partial}_EF$, there is an
$F_1\subseteq F$ such that

\Centerline{$\gamma\le\lambda^{\partial}_EF_1\le(1+\epsilon)\nu F_1
\le(1+\epsilon)\nu F$,}

\noindent by 475F;  so we must have $\lambda^{\partial}_EF=\nu F$.

\medskip

{\bf (b)} Now suppose that $F$ is measured by $\lambda^{\partial}_E$.
Because $\partial^{\$}E$ is $\lambda^{\partial}_E$-conegligible, and
$\lambda^{\partial}_E$ is a $\sigma$-finite Radon measure, there is a
Borel set $F'\subseteq F\cap\partial^{\$}E$ such that
$\lambda^{\partial}_E(F\setminus F')=0$.   Now
$\nu F'=\lambda^{\partial}_EF'$, by (a), and
$\nu(F\cap\partial^{\$}E\setminus F')=0$, by 475Ea, and
$\nu(\partstar E\setminus\partial^{\$}E)=0$, by 475D;  so
$\nu(F\cap\partstar E)$ is defined and equal to
$\lambda^{\partial}_EF'=\lambda^{\partial}_EF$.

\medskip

{\bf (c)} Let $\sequencen{K_n}$ be a non-decreasing sequence of compact
subsets of $\partial^{\$}E$ such that $\bigcup_{n\in\Bbb N}K_n$ is
$\lambda^{\partial}_E$-conegligible.
By (b), $\nu(\partstar E\setminus\bigcup_{n\in\Bbb N}K_n)=0$, while
$\nu K_n=\lambda^{\partial}_EK_n$ is finite for each $n$.
For each $n$, the subspace measure $\nu_{K_n}$ on $K_n$ is a multiple of
Hausdorff $(r-1)$-dimensional measure on $K_n$ (471E), so is a Radon
measure (471Dh, 471F),
as is $(\lambda^{\partial}_E)_{K_n}$;  since, by (b), $\nu_{K_n}$
and $(\lambda^{\partial}_E)_{K_n}$ agree on the Borel subsets of $K_n$,
they are actually identical.   So if $F\subseteq\BbbR^r$ is such that
$\nu$ measures $F\cap\partstar E$, $\lambda^{\partial}_E$ will measure
$F\cap K_n$ for every $n$, and therefore will measure $F$;  so that in
this case also $\lambda^{\partial}_EF=\nu(F\cap\partstar E)$.
}%end of proof of 475G

\leader{475H}{Proposition} Let $V\subseteq\BbbR^r$ be a hyperplane, and
$T:\BbbR^r\to V$ the orthogonal projection.   Suppose that
$A\subseteq\BbbR^r$ is such that $\nu A$ is defined and finite,
and for $u\in V$ set

$$\eqalign{f(u)&=\#(A\cap T^{-1}[\{u\}])\text{ if this is finite},\cr
&=\infty\text{ otherwise}.\cr}$$

\noindent Then $\int_Vf(u)\nu(du)$ is defined and at most $\nu A$.

\proof{{\bf (a)} Because $\nu$ is invariant under isometries, we can
suppose that $V=\{x:\xi_r=0\}$, so that
$Tx=(\xi_1,\ldots,\xi_{r-1},0)$ for $x=(\xi_1,\ldots,\xi_r)$.   For
$m$, $n\in\Bbb N$ and $u\in V$ set

\Centerline{$f_{mn}(u)
=\#(\{k:k\in\Bbb Z$, $|k|\le 4^m$,
$A\cap(\{u\}\times\coint{2^{-m}k,2^{-m}(k+1-2^{-n})})\ne\emptyset\})$;}

\noindent so that $f(u)=\lim_{m\to\infty}\lim_{n\to\infty}f_{mn}(u)$ for
every $u\in V$.

\medskip

{\bf (b)} Suppose for the moment that $A$ is actually a Borel set.
Then $T[A\cap(\BbbR^r\times\coint{\alpha,\beta})]$ is always analytic
(423Eb, 423Bb),
therefore measured by $\nu$ (432A), and every $f_{mn}$ is measurable.
Next, given $\epsilon>0$ and $m$, $n\in\Bbb N$, there is a sequence
$\sequence{i}{F_i}$ of closed sets of diameter at most $2^{-m-n}$,
covering $A$, such that
$\sum_{i=0}^{\infty}2^{-r+1}\beta_{r-1}(\diam F_i)^{r-1}
\le\nu A+\epsilon$.   Now each $T[F_i]$ is a compact set of diameter at
most $\diam F_i$, so
$\nu(T[F_i])\le 2^{-r+1}\beta_{r-1}(\diam F_i)^{r-1}$ (264H);  and if we
set $g=\sum_{i=0}^{\infty}\chi T[F_i]$,
$f_{mn}\le g$ everywhere on $V$, so
$\int f_{mn}d\nu\le\int g\,d\nu\le\nu A+\epsilon$.   As $\epsilon$ is
arbitary, $\int f_{mn}d\nu\le\nu A$.  Accordingly

$$\eqalignno{\int fd\nu
&=\lim_{m\to\infty}\lim_{n\to\infty}\int f_{mn}d\nu\cr
\displaycause{because the limits are monotonic}
&\le\nu A+\epsilon.\cr}$$

\noindent As $\epsilon$ is arbitrary, $\int fd\nu\le\nu A$.

\medskip

{\bf (c)} In general, there are Borel sets $E$, $F$ such that
$E\setminus F\subseteq A\subseteq E$ and $\nu F=0$, by 264Fc/471Db.
By (b),

\Centerline{$\int\#(E\cap T^{-1}[\{u\}])\nu(du)\le\nu E$,
\quad$\int\#(F\cap T^{-1}[\{u\}])\nu(du)\le\nu F=0$,}

\noindent so
$\int\#(A\cap T^{-1}[\{u\}])\nu(du)$ is defined and equal to
$\int\#(E\cap T^{-1}[\{u\}])\nu(du)\le\nu A$.
}%end of proof of 475H

\leader{475I}{Lemma}\cmmnt{ (In this lemma I allow $r=1$.)}   Let
$\Cal K$ be the family of compact sets $K\subseteq\BbbR^r$ such that
$K=\clstar K$.   Then $\mu$ is inner regular with respect to $\Cal K$.

\proof{
%\allowmorestretch{468}{
{\bf (a)} Write $\Cal D$ for the set of dyadic (half-open) cubes in
$\BbbR^r$, that is, sets expressible in the form
$\prod_{i<r}[2^{-m}k_i$,\discretionary{}{}{}$2^{-m}(k_i+1)[$
where $m$,
$k_0,\ldots,k_{r-1}\in\Bbb Z$.   For $m\in\Bbb N$ and $x\in\BbbR^r$
write $C(x,m)$ for the dyadic cube with side of length $2^{-m}$ which
contains $x$.   Then, for any $A\subseteq\BbbR^r$,
%}

\Centerline{$\lim_{m\to\infty}2^{mr}\mu_*(A\cap C(x,m))=1$}

\noindent for every $x\in\intstar A$.   \Prf\
$C(x,m)\subseteq B(x,2^{-m}\sqrt r)$ for each $m$, so

$$2^{mr}\mu^*(C(x,m)\setminus A)
\le\beta_rr^{r/2}\Bover{\mu^*(B(x,2^{-m}\sqrt r)\setminus A)}
  {\mu B(x,2^{-m}\sqrt r)}
\to 0$$

\noindent as $m\to\infty$.\ \Qed

\medskip

{\bf (b)} Now, given a Lebesgue measurable set $E\subseteq\BbbR^r$ and
$\gamma<\mu E$, choose $\sequencen{E_n}$,
$\sequencen{\gamma_n}$, $\sequencen{m_n}$ and $\sequencen{K_n}$
inductively, as follows.   Start with $E_0=E$ and $\gamma_0=\gamma$.
Given that $\gamma_n<\mu E_n$, let $K_n\subseteq E_n$ be a compact set
of measure greater than $\gamma_n$.   Now, by (a), there is an
$m_n\ge n$ such that $\mu^*E_{n+1}>\gamma_n$, where

\Centerline{$E_{n+1}=\{x:x\in K_n,\,
\mu(K_n\cap C(x,m_n))\ge\Bover23\mu C(x,m_n)\}$;}

\noindent note that $E_{n+1}$ is of the form $K_n\cap\bigcup\Cal D_0$ for
some set $\Cal D_0$ of half-open cubes of side $2^{-m_n}$, so that
$E_{n+1}$ is measurable and

\Centerline{$\mu(E_{n+1}\cap C(x,m_n))
=\mu(K_n\cap C(x,m_n))\ge\Bover23\mu C(x,m_n)$}

\noindent for every $x\in E_{n+1}$.   Now set

\Centerline{$\gamma_{n+1}
=\max(\gamma_n,\mu E_{n+1}-\Bover13\cdot 2^{-m_nr})$,}

\noindent and continue.

At the end of the induction, set
$K=\bigcap_{n\in\Bbb N}K_n=\bigcap_{n\in\Bbb N}E_n$.   Then $K$ is
compact, $K\subseteq E$ and

\Centerline{$\mu K=\lim_{n\to\infty}\mu K_n
=\lim_{n\to\infty}\gamma_n\ge\gamma$.}

\noindent If $x\in K$ and $n\in\Bbb N$, then

$$\eqalign{\mu(K\cap B(x,2^{-m_n}\sqrt r))
&\ge\mu(K\cap C(x,m_n))
\ge\mu(E_{n+1}\cap C(x,m_n))-\mu E_{n+1}+\mu K\cr
&\ge\Bover23\mu C(x,m_n)-\mu E_{n+1}+\gamma_{n+1}\cr
&\ge\Bover13\cdot 2^{-m_nr}
=\Bover1{3\beta_rr^{r/2}}\mu B(x,2^{-m_n}\sqrt r),\cr}$$

\noindent so

\Centerline{$\limsup_{\delta\downarrow 0}
  \Bover{\mu(K\cap B(x,\delta))}{\mu B(x,\delta)}
\ge\Bover1{3\beta_rr^{r/2}}>0$}

\noindent and $x\in\clstar K$.

\medskip

{\bf (c)} Thus $K\subseteq\clstar K$.   Since certainly 
$\clstar K\subseteq\overline{K}=K$, we have $K=\clstar K$.
}%end of proof of 475I

\vleader{60pt}{475J}{Lemma} Let $E$ be a Lebesgue measurable subset of
$\BbbR^r$, identified with $\BbbR^{r-1}\times\Bbb R$.
For $u\in\BbbR^{r-1}$, set $G_u=\{t:(u,t)\in\intstar E\}$,
$H_u=\{t:(u,t)\in\intstar(\BbbR^r\setminus E)\}$ and
$D_u=\{t:(u,t)\in\partstar E\}$, so that $G_u$, $H_u$ and $D_u$ are
disjoint and cover $\Bbb R$.

(a) There is a
$\mu_{r-1}$-conegligible set $Z\subseteq\BbbR^{r-1}$ such that whenever
$u\in Z$, $t<t'$ in $\Bbb R$, $t\in G_u$ and
$t'\in H_u$, there is an $s\in D_u\cap\ooint{t,t'}$.

(b) There is a
$\mu_{r-1}$-conegligible set $Z_1\subseteq\BbbR^{r-1}$ such that whenever
$u\in Z_1$, $t$, $t'\in\Bbb R$, $t\in G_u$ and
$t'\in H_u$, there is a member of $D_u$ between $t$ and $t'$.

(c) If $E$ has locally finite perimeter, there is a conegligible set
$Z_2\subseteq Z_1$ such that,
for every $u\in Z_2$, $D_u\cap[-n,n]$ is finite for every $n\in\Bbb N$,
$G_u$ and $H_u$ are open, and $D_u=\partial G_u=\partial H_u$, so that
the constituent intervals of $\Bbb R\setminus D_u$ lie alternately in
$G_u$ and $H_u$.

\proof{{\bf (a)(i)} For $q\in\Bbb Q$, set
$f_q(u)=\sup(G_u\cap\ooint{-\infty,q})$, taking $-\infty$ for
$\sup\emptyset$.   Then
$f_q:\BbbR^{r-1}\to[-\infty,q]$ is Lebesgue measurable.   \Prf\ For
$\alpha<q$,

\Centerline{$\{u:f_q(u)>\alpha\}=\{u:$ there is some
$t\in\ooint{\alpha,q}$ such that $(u,t)\in\intstar E\}$.}

\noindent Because $\intstar E$ is a Borel set (475Cc),
$\{u:f_q(u)>\alpha\}$ is
analytic (423Eb, 423Bc), therefore measurable (432A again).\ \Qed

\medskip

\quad{\bf (ii)} For any $q\in\Bbb Q$,
$W_q=\{u:f_q(u)<q,\,f_q(u)\in G_u\}$ is negligible.
\Prf\Quer\ Suppose, if possible, otherwise.   Let $n\in\Bbb N$ be such
that $\{u:u\in W_q,\,f_q(u)>-n\}$ is not negligible.
If we think of $[-\infty,q]$ as a compact metrizable space, 418J again
tells us that there is a Borel set
$F\subseteq\BbbR^{r-1}$ such that $f_q\restr F$ is continuous and
$F_1=\{u:u\in F\cap W_q$, $f_q(u)>-n\}$ is not
negligible.   Note that $F_1$ is measurable, being the projection of the
Borel set $\{(u,f_q(u)):u\in F,\,-n<f_q(u)<q\}\cap\intstar E$.   By 475I,
there is a non-negligible compact set $K\subseteq F_1$ such that
$K\subseteq\clstar K$, interpreting $\clstar K$ in $\BbbR^{r-1}$.
Because $g\restr K$ is continuous, it attains its maximum at $u\in K$
say.

Set $x=(u,f_q(u))$.   Then, whenever
$0<\delta\le\min(n+f_q(u),q-f_q(u))$,

$$\eqalignno{B(x,2\delta)\setminus\intstar E
&\supseteq\{(w,t):w\in K\cap V(u,\delta),\,|t-f_q(u)|\le\delta,\,
                   f_q(w)<t<q\}\cr
\displaycause{writing $V(u,\delta)$ for
   $\{w:w\in\BbbR^{r-1},\,\|w-u\|\le\delta\}$}
&\supseteq\{(w,t):w\in K\cap V(u,\delta),\,
  f_q(u)<t<f_q(u)+\delta\}\cr}$$

\noindent because $f_q(w)\le f_q(u)$ for every $w\in K$.
So, for such $\delta$,

$$\eqalign{\mu(B(x,2\delta)\setminus\intstar E)
&\ge\delta\mu_{r-1}(K\cap V(u,\delta))\cr
&=\Bover{\beta_{r-1}}{2^r\beta_r}
  \cdot\Bover{\mu_{r-1}(K\cap V(u,\delta))}{\mu_{r-1}V(u,\delta)}
  \cdot\mu B(x,2\delta),\cr}$$

\noindent and

$$\eqalign{\limsup_{\delta\downarrow 0}
  \Bover{\mu(B(x,\delta)\setminus E)}{\mu B(x,\delta)}
&=\limsup_{\delta\downarrow 0}
  \Bover{\mu(B(x,2\delta)\setminus\intstar E)}{\mu B(x,2\delta)}\cr
&\ge\Bover{\beta_{r-1}}{2^r\beta_r}\limsup_{\delta\downarrow 0}
  \Bover{\mu_{r-1}(K\cap V(u,\delta))}{\mu_{r-1}V(u,\delta)}
>0\cr}$$

\noindent because $u\in\clstar K$;  but $x=(u,f_q(u))$ is supposed to
belong to $\intstar E$.\ \Bang\Qed

\medskip

\quad{\bf (iii)} Similarly, setting

\Centerline{$f'_q(u)
=\inf(H_u\cap\ooint{u,\infty})$,
\quad$W'_q=\{u:q<f'_q(u)<\infty,\,f'_q(u)\in H_u\}$,}

\noindent every $W'_q$ is negligible.   So
$Z=\BbbR^{r-1}\setminus\bigcup_{q\in\Bbb Q}(W_q\cup W'_q)$ is
$\mu_{r-1}$-conegligible.

Now if $u\in Z$, $t\in G_u$ and
$t'\in H_u$, where $t<t'$, there is some
$s\in\ooint{t,t'}\cap D_u$.   \Prf\Quer\
Suppose, if possible, otherwise.   Since, by hypothesis, neither $t$
nor $t'$ belongs to $D_u$, $D_u\cap[t,t']=\emptyset$.   Note that,
because $u\in Z$,

\Centerline{$s=\inf(G_u\cap\ooint{s,\infty})$ for every $s\in G_u$,
\quad$s=\sup(H_u\cap\ooint{-\infty,s})$ for every $s\in H_u$.}

\noindent Choose $\sequencen{s_n}$, $\sequencen{s'_n}$ and
$\sequencen{\delta_n}$ inductively, as follows.   Set $s_0=t$,
$s'_0=t'$.   Given that $t\le s_n<s'_n\le t'$ and $s_n\in G_u$ and
$s'_n\in H_u$, where $n$ is even,
set $s'_{n+1}=\sup(G_u\cap[s_n,s'_n])$.   Then either $s'_{n+1}=s'_n$,
so $s'_{n+1}\in H_u$, or $s'_{n+1}<s'_n$ and
$\ocint{s'_{n+1},s'_n}\cap G_u=\emptyset$, so $s'_{n+1}\notin G_u$ and
again $s'_{n+1}\in H_u$.   Let $\delta_n>0$ be such that
$\delta_n\le 2^{-n}$ and
\penalty-1000
$\mu(E\cap B((u,s'_{n+1}),\delta_n))<\bover12\beta_r\delta_n^r$.
Because the function $s\mapsto\mu(E\cap B((u,s),\delta_n))$ is
continuous (443C, or otherwise), there is an
$s_{n+1}\in G_u\cap\coint{s_n,s'_{n+1}}$ such that
$\mu(E\cap B((u,s),\delta_n))\le\bover12\beta_r\delta_n^r$ whenever
$s\in[s_{n+1},s'_{n+1}]$.

This is the inductive step from even $n$.   If $n$ is odd and
$s_n\in G_u$, $s'_n\in H_u$ and $s_n<s'_n$, set
$s_{n+1}=\inf(H_u\cap[s_n,s'_n])$.   This time we find that $s_{n+1}$
must belong to $G_u$.   Let $\delta_n\in\ocint{0,2^{-n}}$ be such that
$\mu(E\cap B((u,s_{n+1}),\delta_n))>\bover12\beta_r\delta_n^r$, and let
$s'_{n+1}\in H_u\cap\ocint{s_{n+1},s'_n}$ be such that
$\mu(E\cap B((u,s),\delta_n))\ge\bover12\beta_r\delta_n^r$ whenever
$s\in[s_{n+1},s'_{n+1}]$.

The construction provides us with a non-increasing sequence
$\sequencen{[s_n,s'_n]}$ of closed intervals, so there must be some $s$
in their intersection.   In this case

\Centerline{$\mu(E\cap B((u,s),\delta_n)\le\bover12\beta_r\delta_n^r$ if
$n$ is even,}

\Centerline{$\mu(E\cap B((u,s),\delta_n)\ge\bover12\beta_r\delta_n^r$ if
$n$ is odd.}

\noindent Since $\lim_{n\to\infty}\delta_n=0$,

\Centerline{$\liminf_{\delta\downarrow 0}
  \Bover{\mu(E\cap B((u,s),\delta))}{\mu B((u,s),\delta)}\le\Bover12$,
\quad$\limsup_{\delta\downarrow 0}
  \Bover{\mu(E\cap B((u,s),\delta))}{\mu B((u,s),\delta)}\ge\Bover12$,}

\noindent and $s\in D_u$, while of course $t\le s\le t'$.\ \Bang\Qed

\medskip

\quad{\bf (iv)} Thus the conegligible set $Z$ has the property required
by (a).

\medskip

{\bf (b)} Applying
(a) to $\BbbR^r\setminus E$, there is a conegligible set
$Z'\subseteq\Bbb R^{r-1}$ such that if $u\in Z'$, $t\in H_u$,
$t'\in G_u$ and $t<t'$, then $D_u$ meets $\ooint{t,t'}$.
So we can use $Z\cap Z'$.

\medskip

{\bf (c)} Now suppose that $E$ has locally finite perimeter.
We know that
$\nu(\partstar E\cap B(\tbf{0},n))=\lambda^{\partial}_EB(\tbf{0,}n)$ is
finite for every $n\in\Bbb N$ (475G).   By 475H,

\Centerline{$\int_{\|u\|\le n}\#(D_u\cap[-n,n])\mu_{r-1}(du)
\le\nu(\partstar E\cap B(\tbf{0},2n))<\infty$}

\noindent for every $n\in\Bbb N$;  but this means that $D_u\cap[-n,n]$
must be finite for almost every $u$ such that $\|u\|\le n$, for every
$n$, and therefore that

\Centerline{$Z_1'=\{u:u\in Z_1,\,
  D_u\cap[-n,n]$ is finite for every $n\}$}

\noindent is conegligible.   For $u\in Z'_1$, $\Bbb R\setminus D_u$ is an
open set, so is made up of a disjoint sequence of intervals with
endpoints in $D_u\cup\{-\infty,\infty\}$ (see 2A2I);  and because
$u\in Z_1$, each of these intervals is included in either $G_u$ or
$H_u$.   Now

\Centerline{$A=\{u:$ there are successive constituent intervals of
$\Bbb R\setminus D_u$ both included in $G_u\}$}

\noindent is negligible.   \Prf\Quer\ Otherwise, there are rationals
$q<q'$ such that

\Centerline{$F
=\{u:\ooint{q,q'}\setminus G_u$ contains exactly one point$\}$}

\noindent is not negligible.   Note that, writing
$T:\BbbR^r\to\BbbR^{r-1}$ for the orthogonal projection,

$$\eqalign{F&=T[(\BbbR^{r-1}\times\ooint{q,q'})\setminus\intstar E]\cr
&\qquad\setminus\bigcup_{q''\in\Bbb Q,q<q''<q'}
  T[(\BbbR^{r-1}\times\ooint{q,q''})\setminus\intstar E]
  \cap T[(\BbbR^{r-1}\times\ooint{q'',q'})\setminus\intstar E]\cr}$$

\noindent is measurable.   Take any $u\in F\cap\intstar F$, and let $t$
be the unique point in $\ooint{q,q'}\setminus G_u$.   Then whenever
$0<\delta\le\min(t-q,q'-t)$ we shall have

\Centerline{$\mu(B((u,t),\delta)\setminus E)
=\mu(B((u,t),\delta)\setminus\intstar E)
\le 2\delta\mu_{r-1}(V(u,\delta)\setminus F)$,}

\noindent because if $w\in F$ then $(w,s)\in\intstar E$ for almost
every $s\in[t-\delta,t+\delta]$.   So

\Centerline{$\limsup_{\delta\downarrow 0}
  \Bover{\mu(B((u,t),\delta)\setminus E)}{\mu B((u,t),\delta)}
\le\Bover{2\beta_{r-1}}{\beta_r}\limsup_{\delta\downarrow 0}
   \Bover{\mu_{r-1}(V(u,\delta)\setminus F)}{\mu_{r-1}V(u,\delta)}
=0$,}

\noindent and $(u,t)\in\intstar E$.\ \Bang\Qed

Similarly,

\Centerline{$A'=\{u:$ there are successive constituent intervals of
$\Bbb R\setminus D_u$ both included in $H_u\}$}

\noindent is negligible.   So $Z_2=Z'_1\setminus(A\cup A')$
is a conegligible set of the kind we need.
}%end of proof of 475J

\leader{475K}{Lemma} Suppose that $h:\BbbR^r\to[-1,1]$ is a Lipschitz
function with compact support;  let $n\in\Bbb N$ be such that $h(x)=0$
for $\|x\|\ge n$.  Suppose that $E\subseteq\BbbR^r$ is a Lebesgue
measurable set.   Then

\Centerline{$|\int_E\Pd{h}{\xi_r}d\mu|
\le 2\bigl(\beta_{r-1}n^{r-1}
  +\nu(\partstar E\cap B(\tbf{0},n))\bigr)$.}

\proof{ By Rademacher's theorem (262Q), $\Pd{h}{\xi_r}$ is defined
almost everywhere;  as it is
measurable and bounded, and is zero outside $B(\tbf{0},n)$, the
integral is well-defined.   If $\nu(\partstar E\cap B(\tbf{0},n))$ is
infinite, the result is trivial;  so henceforth let us suppose that
$\nu(\partstar E\cap B(\tbf{0},n))<\infty$.   Identify $\BbbR^r$ with
$\BbbR^{r-1}\times\Bbb R$.   For $u\in\BbbR^{r-1}$, set

$$\eqalign{f(u)&=\#(\{t:(u,t)\in\partstar E\cap B(\tbf{0},n)\})
  \text{ if this is finite},\cr
&=\infty\text{ otherwise}.\cr}$$

\noindent By 475H,
$\int fd\mu_{r-1}\le\nu(\partstar E\cap B(\tbf{0},n))$.   By 475Jb,
there is a
$\mu_{r-1}$-conegligible set $Z\subseteq\BbbR^{r-1}$ such that
$f(u)$ is finite for every $u\in Z$ and

\inset{\noindent whenever $u\in Z$ and $(u,t)\in\intstar E$ and
$(u,t')\in\BbbR^r\setminus\clstar E$, there is an $s$ lying between $t$
and $t'$ such that $(u,s)\in\partstar E$.}

\noindent For $u\in Z$, set
$D'_u=\{t:(u,t)\in B(\tbf{0},n)\setminus\partstar E\}$,
and define $g_u:D'_u\to\{0,1\}$ by setting

$$\eqalign{g_u(t)
&=1\text{ if }(u,t)\in B(\tbf{0},n)\cap\intstar E,\cr
&=0\text{ if }(u,t)\in B(\tbf{0},n)\setminus\clstar E.\cr}$$

\noindent Now if $t$, $t'\in D'_u$ and $g(t)\ne g(t')$, there is a point
$s$ between $t$ and $t'$ such that $(u,s)\in\partstar E$;  so the
variation $\Var_{D'_u}g_u$ of $g_u$ (224A) is at most $f(u)$.   Setting
$h_u(t)=h(u,t)$ for $u\in\BbbR^{r-1}$ and $t\in\Bbb R$, we now have

$$\eqalignno{\int_{-\infty}^{\infty}\Pd{h}{\xi_r}(u,t)
  \chi(B(\tbf{0},n)\cap\intstar E)(u,t)dt
&=\int_{D'_u}h'_u(t)g_u(t)dt\cr
\displaycause{because $h'_u(t)=0$ if $(u,t)\notin B(\tbf{0},n)$}
&\le(1+\Var_{D'_u}g_u)\sup_{a\le b}|\int_a^bh'_u(t)|dt\cr
\displaycause{by 224J, recalling that $D'_u$ is either empty or a
bounded interval with finitely many points deleted}
&\le (1+f(u))\sup_{a\le b}|h_u(b)-h_u(a)|\cr
\displaycause{because $h_u$ is Lipschitz, therefore absolutely
continuous on any bounded interval}
&\le 2(1+f(u)).\cr}$$

\noindent Integrating over $u$, we now have

$$\eqalignno{|\int_E\Pd{h}{\xi_r}d\mu|
&=|\int_{B(\tbf{0},n)\cap\intstar E}\Pd{h}{\xi_r}d\mu|\cr
\displaycause{475Cg}
&=\int_{V(\tbf{0},n)}\int_{-\infty}^{\infty}\Pd{h}{\xi_r}(u,t)
  \chi(B(\tbf{0},n)\cap\intstar E)(u,t)dt\,\mu_{r-1}(du)\cr
\displaycause{where $V(\tbf{0},n)=\{u:u\in\BbbR^{r-1},\,\|u\|\le n\}$}
&\le\int_{V(\tbf{0},n)}2(1+f(u))\mu_{r-1}(du)\cr
&\le 2\bigl(\beta_{r-1}n^{r-1}
  +\nu(\partstar E\cap B(\tbf{0},n))\bigr).\cr}$$
}%end of proof of 475K

\leader{475L}{Theorem} Suppose that $E\subseteq\BbbR^r$.   Then $E$ has
locally finite perimeter iff $\nu(\partstar E\cap B(\tbf{0},n))$ is
finite for every $n\in\Bbb N$.

\proof{ If $E$ has locally finite perimeter, then
$\nu(\partstar E\cap B(\tbf{0},n))=\lambda^{\partial}_EB(\tbf{0},n)$ is
finite for every $n$, by 475G.   So let us suppose that
$\nu(\partstar E\cap B(\tbf{0},n))$ is finite for every
$n\in\Bbb N$.   Then $\mu(\partstar E\cap B(\tbf{0},n))=0$ for every
$n$ (471L), $\mu(\partstar E)=0$ and $E$ is Lebesgue measurable (475Ch).

If $n\in\Bbb N$, then

$$\eqalign{\sup\{|\int_E\diverg\phi\,d\mu|:
  \phi:\BbbR^r\to\BbbR^r&\text{ is Lipschitz},\,
  \|\phi\|\le\chi B(\tbf{0},n)\}\cr
&\le 2r\bigl(\beta_{r-1}n^{r-1}
   +\nu(\partstar E\cap B(\tbf{0},n))\bigr)\cr}$$

\noindent is finite.   \Prf\ If
$\phi=(\phi_1,\ldots,\phi_r):\BbbR^r\to\BbbR^r$ is a Lipschitz
function and $\|\phi\|\le\chi B(\tbf{0},n)$, then 475K tells us that

\Centerline{$|\int_E\Pd{\phi_r}{\xi_r}d\mu|
\le 2\bigl(\beta_{r-1}n^{r-1}
  +\nu(\partstar E\cap B(\tbf{0},n))\bigr)$.}

\noindent But of course it is equally true that

\Centerline{$|\int_E\Pd{\phi_j}{\xi_j}d\mu|
\le 2\bigl(\beta_{r-1}n^{r-1}
  +\nu(\partstar E\cap B(\tbf{0},n))\bigr)$}

\noindent for every other $j\le r$;  adding, we have the result.\ \Qed

Since $n$ is arbitrary, $E$ has locally finite perimeter.
}%end of proof of 475L

\leader{475M}{Corollary} (a) The family of sets with locally finite
perimeter is a subalgebra of the algebra of Lebesgue measurable subsets
of $\BbbR^r$.

(b) A set $E\subseteq\BbbR^r$ is Lebesgue measurable and
has finite perimeter iff
$\nu(\partstar E)<\infty$, and in this case $\nu(\partstar E)$ is the
perimeter of $E$.

(c) If $E\subseteq\BbbR^r$ has finite measure, then
$\per E=\liminf_{\alpha\to\infty}\per(E\cap B(\tbf{0},\alpha))$.

%?? shouldn't it be lim_{\alpha\to\infty} ? yes, see 475Xk

\proof{{\bf (a)} Recall that the definition in 474D insists that sets
with locally finite perimeter should be Lebesgue measurable.
If $E\subseteq\BbbR^r$ has locally finite perimeter, then so
has $\BbbR^r\setminus E$, by 474J.   If $E$, $F\subseteq\BbbR^r$ have
locally finite perimeter, then

\Centerline{$\nu(\partstar(E\cup F)\cap B(\tbf{0},n))
\le\nu(\partstar E\cap B(\tbf{0},n))
  +\nu(\partstar F\cap B(\tbf{0},n))$}

\noindent is finite for every $n\in\Bbb N$, by 475L and 475Cd;  by 475L
in the other direction, $E\cup F$ has locally finite perimeter.

\medskip

{\bf (b)} If $E$ is Lebesgue measurable and has
finite perimeter (on the definition in 474D), then
$\nu(\partstar E)=\lambda^{\partial}_E\BbbR^r$ is the perimeter of $E$,
by 475G.
If $\nu(\partstar E)<\infty$, then $\mu(\partstar E)=0$ and
$E$ is measurable (471L and 475Ch);  now
$E$ has locally finite perimeter, by 475L, and 475G again tells us that
$\nu(\partstar E)=\lambda^{\partial}_E\BbbR^r$ is the perimeter of $E$.

\medskip

{\bf (c)} Now suppose that $\mu E<\infty$.   For any $\alpha\ge 0$,

\Centerline{$\partstar(E\cap B(\tbf{0},\alpha))
\subseteq(\partstar E\cap B(\tbf{0},\alpha))
\cup(\clstar E\cap\partial B(\tbf{0},\alpha))
\subseteq\partstar E\cup(\clstar E\cap\partial B(\tbf{0},\alpha))$}

\noindent by 475Cf.    Now we know also that

\Centerline{$\int_0^{\infty}
   \nu(\clstar E\cap\partial B(\tbf{0},\alpha))d\alpha
=\mu(\clstar E)=\mu E<\infty$}

\noindent (265G), so
$\liminf_{\alpha\to\infty}\nu(\clstar E\cap\partial B(\tbf{0},\alpha))
=0$.   This means that

$$\eqalign{\liminf_{\alpha\to\infty}\per(E\cap B(\tbf{0},\alpha))
&=\liminf_{\alpha\to\infty}\nu(\partstar(E\cap B(\tbf{0},\alpha)))\cr
&\le\liminf_{\alpha\to\infty}\nu(\partstar E)
  +\nu(\clstar E\cap\partial B(\tbf{0},\alpha))
=\nu(\partstar E).\cr}$$

\noindent In the other direction,

\Centerline{$\partstar E\cap\interior B(\tbf{0},\alpha)
=\partstar E\cap\intstar B(\tbf{0},\alpha)
\subseteq\partstar(E\cap B(\tbf{0},\alpha))$}

\noindent for every $\alpha$, by 475Ce, so

$$\eqalign{\per E
&=\nu(\partstar E)
=\lim_{\alpha\to\infty}
  \nu(\partstar E\cap\interior B(\tbf{0},\alpha))\cr
&\le\liminf_{\alpha\to\infty}\nu(\partstar(E\cap B(\tbf{0},\alpha)))
=\liminf_{\alpha\to\infty}\per(E\cap B(\tbf{0},\alpha))\cr}$$

\noindent and we must have equality.
}%end of proof of 475M

\cmmnt{\medskip

\noindent{\bf Remark} See 475Xk.
}

\leader{475N}{The Divergence Theorem}
%de Giorgi-Federer
Let $E\subseteq\BbbR^r$ be such
that $\nu(\partstar E\cap B(\tbf{0},n))$ is finite for every
$n\in\Bbb N$.

(a) $E$ is Lebesgue measurable.

(b) For $\nu$-almost every $x\in\partstar E$, there is a Federer
exterior normal $v_x$ of $E$ at $x$.

(c) For every Lipschitz function $\phi:\BbbR^r\to\BbbR^r$ with compact
support,

\Centerline{$\int_E\diverg\phi\,d\mu
=\int_{\partstar E}\varinnerprod{\phi(x)}{v_x}\,\nu(dx)$.}

\proof{ By 475L, $E$ has locally finite perimeter, and in particular is
Lebesgue measurable.   By 474R, there is a
Federer exterior normal $v_x=\psi_E(x)$ of $E$ at $x$ for every
$x\in\partial^{\$}E$;  by 475D, $\nu$-almost every point in
$\partstar E$ is of this kind.   By 474E-474F,

\Centerline{$\int_E\diverg\phi\,d\mu
=\int\varinnerprod{\phi}{\psi_E}\,d\lambda^{\partial}_E
=\int_{\partial^{\$}E}
   \varinnerprod{\phi}{\psi_E}\,d\lambda^{\partial}_E$,}

\noindent and this is also equal to
$\int_{\partstar E}
   \varinnerprod{\phi}{\psi_E}\,d\lambda^{\partial}_E$, because
$\partial^{\$}E\subseteq\partstar E$ and $\partial^{\$}E$ is
$\lambda^{\partial}_E$-conegligible.   But $\lambda^{\partial}_E$ and
$\nu$ induce the same subspace measures on $\partstar E$, by 475G, so

\Centerline{$\int_E\diverg\phi\,d\mu
=\int_{\partstar E}\varinnerprod{\phi}{\psi_E}\,d\nu
=\int_{\partstar E}\varinnerprod{\phi(x)}{v_x}\,\nu(dx)$,}

\noindent as claimed.
}%end of proof of 475N

\leader{475O}{}\cmmnt{ At the price of some technicalities which are
themselves instructive, we can now proceed to some basic properties of
the essential boundary.

\medskip

\noindent}{\bf Lemma} Let $E\subseteq\BbbR^r$ be a set with locally
finite perimeter, and $\psi_E$ its canonical outward-normal function.
Let $v$ be the unit vector $(0,\ldots,0,1)$.   Identify $\BbbR^r$ with
$\BbbR^{r-1}\times\Bbb R$.   Then we have sequences $\sequencen{F_n}$,
$\sequencen{g_n}$ and $\sequencen{g'_n}$ such that

(i) for each $n\in\Bbb N$, $F_n$ is a Lebesgue measurable subset of
$\BbbR^{r-1}$, and $g_n$, $g'_n:F_n\to[-\infty,\infty]$ are Lebesgue
measurable functions such that $g_n(u)<g'_n(u)$ for every $u\in F_n$;

(ii) if $m$, $n\in\Bbb N$ are distinct and $u\in F_m\cap F_n$, then
$[g_m(u),g'_m(u)]\cap[g_n(u),g'_n(u)]=\emptyset$;

(iii) $\sum_{n=0}^{\infty}\int_{F_n}g'_n-g_nd\mu_{r-1}=\mu E$;

(iv) for any continuous function $h:\BbbR^r\to\Bbb R$ with compact
support,

\Centerline{$\int_{\partstar E}
  h(x)\varinnerprod{v}{\psi_E(x)}\,\nu(dx)
=$$\sum_{n=0}^{\infty}$$\int_{F_n}h(u,g'_n(u))-h(u,g_n(u))
  \mu_{r-1}(du)$,}

\noindent where we interpret $h(u,\infty)$ and $h(u,-\infty)$ as $0$ if
necessary;

(v) for $\mu_{r-1}$-almost every $u\in\BbbR^{r-1}$,

$$\eqalign{\{t:(u,t)\in\partstar E\}
&=\{g_n(u):n\in\Bbb N,\,u\in F_n,\,g_n(u)\ne-\infty\}\cr
&\hskip10em
  \cup\{g'_n(u):n\in\Bbb N,\,u\in F_n,\,g'_n(u)\ne\infty\}.\cr}$$

\proof{{\bf (a)} Take a conegligible set $Z_2\subseteq\BbbR^{r-1}$ as in
475Jc.   Let $Z\subseteq Z_2$ be a conegligible Borel set.
For $u\in Z$ set $D_u=\{t:(u,t)\in\partstar E\}$.

\medskip

{\bf (b)} For each $q\in\Bbb Q$, set
$F'_q=\{u:u\in Z,\,(u,q)\in\intstar E\}$, so that $F'_q$ is a Borel set,
and for $u\in F'_q$ set

\Centerline{$f_q(u)=\sup(D_u\cap\ooint{-\infty,q})$,
\quad$f'_q(u)=\inf(D_u\cap\ooint{-\infty,q})$,}

\noindent allowing $-\infty$ as $\sup\emptyset$ and $\infty$ as
$\inf\emptyset$.   Observe that (because $D_u\cap[q-1,q+1]$ is finite)
$f_q(u)<q<f'_q(u)$.   Now $f_q$ and $f'_q$ are measurable, by the
argument of part (a-i) of the proof of 475J.   Enumerate $\Bbb Q$ as
$\sequencen{q_n}$, and set

\Centerline{$F_n
=F'_{q_n}\setminus\bigcup_{m<n}\{u:u\in F'_{q_m},
  \,f_{q_m}(u)<q_n<f'_{q_m}(u)\}$}

\noindent for $n\in\Bbb N$.   Set $g_n(u)=f_{q_n}(u)$,
$g'_n(u)=f'_{q_n}(u)$ for $u\in F_n$.

The effect of this construction is to ensure that, for any $u\in Z$,
$u\in F_n$ iff $q_n$ is the first rational lying in one of those
constituent intervals $I$ of $\Bbb R\setminus D_u$ such that
$\{u\}\times I\subseteq\intstar E$, and that now $g_n(u)$ and $g'_n(u)$
are the endpoints of that interval, allowing $\pm\infty$ as endpoints.

\medskip

{\bf (c)} Now let us look at the items (i)-(v) of the statement of this
lemma.    We have already achieved (i).   If $u\in F_m\cap F_n$, then,
in the language of 475J,
$\ooint{g_m(u),g'_m(u)}$ is one of the constituent intervals of $G_u$,
and $\ooint{g_n(u),g'_n(u)}$ is another;  since these must be separated by
one of the constituent intervals of $H_u$, their closures are disjoint.
Thus (ii) is true.   For any $u\in Z$,

\Centerline{$\sum_{n\in\Bbb N,u\in F_n}g'_n(u)-g_n(u)
=\mu_1\{t:(u,t)\in\intstar E\}$,}

\noindent so

\Centerline{$\sum_{n=0}^{\infty}$$\int_{F_n}g'_n-g_nd\mu_{r-1}
=\int_Z\mu_1\{t:(u,t)\in\intstar E\}\mu_{r-1}(du)
=\mu(\intstar E)
=\mu E$.}

\noindent So (iii) is true.   Also, for $u\in Z$,

$$\eqalign{\{t:(u,t)\in\partstar E\}
&=D_u\cr
&=\{g_n(u):n\in\Bbb N,\,u\in F_n,\,g_n(u)\ne-\infty\}\cr
&\hskip10em
  \cup\{g'_n(u):n\in\Bbb N,\,u\in F_n,\,g'_n(u)\ne\infty\},\cr}$$

\noindent so (v) is true.

\medskip

{\bf (d)} As for (iv), suppose first that $h:\BbbR^r\to\Bbb R$ is a
Lipschitz function with compact support.   Set $\phi(x)=(0,\ldots,h(x))$
for $x\in\BbbR^r$.   Then

$$\eqalignno{\int_{\partstar E}
  h(x)\varinnerprod{v}{\psi_E(x)}\,\nu(dx)
&=\int h(x)\varinnerprod{v}{\psi_E(x)}\,\lambda^{\partial}_E(dx)\cr
\displaycause{475G}
&=\int\varinnerprod{\phi}{\psi_E}\,d\lambda^{\partial}_E
=\int_E\diverg\phi\,d\mu\cr
\displaycause{474E}
&=\int_E\Pd{h}{\xi_r}\,d\mu
=\int_{\intstar E}\Pd{h}{\xi_r}\,d\mu\cr
&=\int\sum_{n\in\Bbb N,u\in F_n}
  \int_{g_n(u)}^{g'_n(u)}\Pd{h}{\xi_r}(u,t)dt\,\mu_{r-1}(du)\cr
&=\int\sum_{n\in\Bbb N,u\in F_n}
  h(u,g'_n(u))-h(u,g_n(u))\mu_{r-1}(du)\cr
\displaycause{with the convention that
$h(u,\pm\infty)=\lim_{t\to\pm\infty}h(u,t)=0$}
&=\sum_{n=0}^{\infty}\int_{F_n}
  h(u,g'_n(u))-h(u,g_n(u))\mu_{r-1}(du)\cr}$$

\noindent because if $|h|\le m\chi B(\tbf{0},m)$ then

$$\eqalign{\int\sum_{n\in\Bbb N,u\in F_n}
  |h(u,g'_n(u))&-h(u,g_n(u))|\mu_{r-1}(du)\cr
&\le 2m\int
   \#(\{t:(u,t)\in B(\tbf{0},m)\cap\partstar E\})\mu_{r-1}(du)\cr
&\le 2m\mskip2mu\nu(B(\tbf{0},m)\cap\partstar E)\cr}$$

\noindent (475H) is finite.

For a general continuous function $h$ of compact support, consider the
convolutions $h_k=h*\tilde h_k$ for large $k$, where $\tilde h_k$ is
defined in 473E.   If $|h|\le m\chi B(\tbf{0},m)$ then
$|h_k|\le m\chi B(\tbf{0},m+1)$ for every $k$, so that

\Centerline{$\sum_{n\in\Bbb N,u\in F_n}|h_k(g'_n(u))-h_k(g_n(u))|
\le 2m\#(\{t:(u,t)\in B(\tbf{0},m+1)\cap\partstar E\})$}

\noindent for every $u\in\Bbb R^r$, $k\in\Bbb N$.   Since $h_k\to h$
uniformly (473Ed),

$$\eqalign{\int_{\partstar E}
  h(x)\varinnerprod{v}{\psi_E(x)}\,\nu(dx)
&=\lim_{k\to\infty}\int_{\partstar E}
  h_k(x)\varinnerprod{v}{\psi_E(x)}\,\nu(dx)\cr
&=\lim_{k\to\infty}\int\sum_{n\in\Bbb N,u\in F_n}
  h_k(u,g'_n(u))-h_k(u,g_n(u))\mu_{r-1}(du)\cr
&=\int\sum_{n\in\Bbb N,u\in F_n}h(u,g'_n(u))-h(u,g_n(u))
  \mu_{r-1}(du)\cr
&=\sum_{n=0}^{\infty}\int_{F_n}h(u,g'_n(u))-h(u,g_n(u))
  \mu_{r-1}(du),\cr}$$

\noindent as required.
}%end of proof of 475O

\leader{475P}{Lemma} Let $v\in S_{r-1}$ be any unit vector, and
$V\subseteq\BbbR^r$ the hyperplane $\{x:\varinnerprod{x}{v}=0\}$.   Let
$T:\BbbR^r\to V$ be the orthogonal projection.   If $E\subseteq\BbbR^r$
is any set with locally finite perimeter and canonical outward-normal
function $\psi_E$, then

\Centerline{$\int_{\partstar E}|\varinnerprod{v}{\psi_E}|d\nu
=\int_V\#(\partstar E\cap T^{-1}[\{u\}])\nu(du)$,}

\noindent interpreting $\#(\partstar E\cap T^{-1}[\{u\}])$ as $\infty$
if $\partstar E\cap T^{-1}[\{u\}]$ is infinite.

\proof{ As usual, we may suppose that the structure $(E,v)$ is rotated
until $v=(0,\ldots,1)$, so that we can identify $T(\xi_1,\ldots,\xi_r)$
with $(\xi_1,\ldots,\xi_{r-1})\in\BbbR^{r-1}$, and
$\int_V\#(\partstar E\cap T^{-1}[\{u\}])\nu(du)$ with
$\int_{\Bbb R^{r-1}}\#(D_u)\mu_{r-1}(du)$, where
$D_u=\{t:(u,t)\in\partstar E\}$.   For each $m\in\Bbb N$ and
$u\in\BbbR^{r-1}$, set
$D^{(m)}_u=\{t:(u,t)\in\partstar E,\,\|(u,t)\|<m\}$;  note that
$\int\#(D^{(m)}_u)\mu_{r-1}(du)\le\nu(\partstar E\cap B(\tbf{0},m))$ is
defined and finite (475H).   It follows that the integral
$\int_V\#(\partstar E\cap T^{-1}[\{u\}])\nu(du)$ is defined in
$[0,\infty]$.

\woddheader{475P}{4}{2}{2}{30pt}

{\bf (a)} Write $\Phi$ for the set of continuous functions
$h:\BbbR^r\to[-1,1]$ with compact support.   Then

\Centerline{$\int_{\partstar E}|\varinnerprod{v}{\psi_E}|d\nu
=\sup_{h\in\Phi}\int_{\partstar E}
   h(x)\varinnerprod{v}{\psi_E(x)}\,\nu(dx)$.}

\noindent\Prf\ Of course

\Centerline{$\int_{\partstar E}|\varinnerprod{v}{\psi_E}|d\nu
\ge\int_{\partstar E}
   h(x)\varinnerprod{v}{\psi_E(x)}\,\nu(dx)$}

\noindent for any $h\in\Phi$.   On the other hand, if

\Centerline{$\gamma
<\int_{\partstar E}|\varinnerprod{v}{\psi_E}|d\nu
=\int|\varinnerprod{v}{\psi_E}|d\lambda^{\partial}_E$}

\noindent (475G), then, because $\lambda^{\partial}_E$ is a Radon
measure, there is an $n\in\Bbb N$ such that

\Centerline{$\gamma<\int_{B(\tbf{0},n)}|\varinnerprod{v}{\psi_E}|
  d\lambda^{\partial}_E
=\int h_0(x)\varinnerprod{v}{\psi_E(x)}\,\lambda^{\partial}_E(dx)$}

\noindent where

$$\eqalign{h_0(x)
&=\bover{\varinnerprod{v}{\psi_E(x)}}{|\varinnerprod{v}{\psi_E(x)}|}
\text{ if }x\in B(\tbf{0},n)
  \text{ and }\varinnerprod{v}{\psi_E(x)}\ne 0,\cr
&=0\text{ otherwise}.\cr}$$

\noindent Again because
$\lambda^{\partial}_E$ is a Radon measure, there is a
continuous function $h_1:\BbbR^r\to\Bbb R$ with compact support such
that

\Centerline{$\int|h_1-h_0|d\lambda^{\partial}_E
\le\int_{B(\tbf{0},n)}|\varinnerprod{v}{\psi_E}|
  d\lambda^{\partial}_E-\gamma$}

\noindent (416I);  of course we may suppose that $h_1$, like $h$, takes
values in $[-1,1]$, so that $h_1\in\Phi$.   Now

\Centerline{$\int_{\partstar E}h_1(x)
  \varinnerprod{v}{\psi_E(x)}\nu(dx)
=\int h_1(x)\varinnerprod{v}{\psi_E(x)}\,\lambda^{\partial}_E(dx)
\ge\gamma$.}

\noindent As $\gamma$ is arbitrary,

\Centerline{$\sup_{h\in\Phi}
  \int h(x)\varinnerprod{v}{\psi_E(x)}\,\nu(dx)
=\int|\varinnerprod{v}{\psi_E}|d\nu$}

\noindent as required.\ \Qed

\medskip

{\bf (b)} Now take $\sequencen{F_n}$, $\sequencen{g_n}$ and
$\sequencen{g'_n}$ as in 475O.   Then

\Centerline{$\int\#(D_u)\mu_{r-1}(du)
=\sup_{h\in\Phi}$$\sum_{n=0}^{\infty}$$\int_{F_n}
  h(u,g'_n(u))-h(u,g_n(u))\mu_{r-1}(du)$.}

\noindent (As in 475O(iv), interpret $h(u,\pm\infty)$ as $0$ if
necessary.)   \Prf\ By 475O(ii), $g'_m(u)\ne g_n(u)$ whenever
$m$, $n\in\Bbb N$ and $u\in F_m\cap F_n$,
while for almost all $u\in\BbbR^{r-1}$

\Centerline{$D_u
=(\{g_n(u):n\in\Bbb N,\,u\in F_n\}\setminus\{-\infty\})
  \cup(\{g'_n(u):n\in\Bbb N,\,u\in F_n\}\setminus\{\infty\})$.}

\noindent So if $h\in\Phi$,

$$\eqalign{\sum_{n\in\Bbb N,u\in F_n}h(u,g'_n(u))-h(u,g_n(u))
&\le\#(\{g_n(u):u\in F_n,\,g_n(u)\ne-\infty\})\cr
&\hskip5em +\#(\{g'_n(u):u\in F_n,\,g'_n(u)\ne\infty\})\cr
&=\#(D_u)\cr}$$

\noindent for almost every $u\in\BbbR^{r-1}$, and

\Centerline{$\sum_{n=0}^{\infty}$$\int_{F_n}
   h(u,g'_n(u))-h(u,g_n(u))\mu_{r-1}(du)
\le\int\#(D_u)\mu_{r-1}(du)$.}

\allowmorestretch{468}{
In the other direction, given $\gamma<\int\#(D_u)\mu_{r-1}(du)$,
there is an $m\in\Bbb N$ such that
$\gamma<\int\#(D^{(m)}_u)\mu_{r-1}(du)$.   Setting
}

\Centerline{$H_n=\{u:u\in F_n,\,g_n(u)\in D^{(m)}_u\}$,
\quad$H'_n=\{u:u\in F_n,\,g'_n(u)\in D^{(m)}_u\}$}

\noindent for $n\in\Bbb N$, we have

\Centerline{$\gamma
<\sum_{n=0}^{\infty}\mu_{r-1}H_n+\sum_{n=0}^{\infty}\mu_{r-1}H'_n
\le\int\#(D^{(m)}_u\mu_{r-1}(du)
<\infty$.}

\noindent By 418J once more, we can find $n\in\Bbb N$ and
compact sets
$K_i\subseteq H_i$, $K'_i\subseteq H'_i$ such that $g_i\restr K_i$ and
$g'_i\restr K'_i$ are continuous for every $i$ and

$$\eqalign{\gamma
&\le\sum_{i=0}^n(\mu_{r-1}K_i+\mu_{r-1}K'_i)
   -\sum_{i=n+1}^{\infty}(\mu_{r-1}H_i+\mu_{r-1}H'_i)\cr
&\qquad\qquad-\sum_{i=0}^n(\mu_{r-1}(H_i\setminus K_i)
       +\mu_{r-1}(H'_i\setminus K'_i)).\cr}$$

\noindent Set

\Centerline{$K=\bigcup_{i\le n}\{(u,g_i(u)):u\in K_i\}$,
\quad$K'=\bigcup_{i\le n}\{(u,g'_i(u)):u\in K'_i\}$,}

\noindent so that $K$ and $K'$ are disjoint compact subsets of
$\interior B(\tbf{0},m)$.   Let $h:\BbbR^r\to\Bbb R$ be a continuous
function such that

$$\eqalign{h(x)
&=1\text{ for }x\in K',\cr
&=-1\text{ for }x\in K,\cr
&=0\text{ for }x\in\BbbR^r\setminus\interior B(\tbf{0},m)\cr}$$

\noindent (4A2F(d-ix));  we can suppose that $-1\le h(x)\le 1$ for
every $x$.   Then $h\in\Phi$, and

$$\eqalignno{\sum_{i=0}^{\infty}\int_{F_i}h(u,g'_i(u))&-h(u,g_i(u))
  \mu_{r-1}(du)\cr
&=\sum_{i=0}^{\infty}\int_{H'_i}h(u,g'_i(u))\mu_{r-1}(du)
  -\sum_{i=0}^{\infty}\int_{H_i}h(u,g_i(u))\mu_{r-1}(du)\cr
\displaycause{because $h$ is zero outside $\interior B(\tbf{0},m)$}
&\ge\sum_{i=0}^n\int_{K'_i}h(u,g'_i(u))\mu_{r-1}(du)
  -\sum_{i=0}^n\int_{K_i}h(u,g_i(u))\mu_{r-1}(du)\cr
&\qquad\qquad-\sum_{i=0}^n(\mu(H'_i\setminus K'_i)+\mu(H_i\setminus
K_i))
  -\sum_{i=n+1}^{\infty}(\mu H'_i+\mu H_i)\cr
&\ge\gamma.\cr}$$

\noindent This shows that

\Centerline{$\int\#(D_u)\mu_{r-1}(du)
\le\sup_{h\in\Phi}$$\sum_{n=0}^{\infty}$$\int_{F_n}
   h(u,g'_n(u))-h(u,g_n(u))\mu_{r-1}(du)$,}

\noindent and we have equality.\ \Qed

\medskip

{\bf (c)} Putting (a) and (b) together with 475O(iv), we have the
result.
}%end of proof of 475P

\leader{475Q}{Theorem}\discrversionA{\footnote{New in 2009:
the final formula in (a)
and part (b).}}{} (a) Let $E\subseteq\BbbR^r$ be a set with finite
perimeter.   For $v\in S_{r-1}$ write $V_v$ for
$\{x:\varinnerprod{x}{v}=0\}$, and let $T_v:\BbbR^r\to V_v$ be the
orthogonal projection.   Then

$$\eqalignno{\per E
&=\nu(\partstar E)
=\Bover1{2\beta_{r-1}}\int_{S_{r-1}}\int_{V_v}
  \#(\partstar E\cap T_v^{-1}[\{u\}])\nu(du)\nu(dv)\cr
&=\lim_{\delta\downarrow 0}\Bover1{2\beta_{r-1}\delta}
  \int_{S_{r-1}}\mu(E\symmdiff(E+\delta v))\nu(dv).\cr}$$

(b) Suppose that $E\subseteq\BbbR^r$ is Lebesgue measurable.   Set

\Centerline{$\gamma
=\sup_{x\in\BbbR^r\setminus\{0\}}\Bover1{\|x\|}\mu(E\symmdiff(E+x))$.}

\noindent Then $\gamma\le\per E\le\Bover{r\beta_r\gamma}{2\beta_{r-1}}$.

\proof{{\bf (a)(i)} I start with an elementary fact:  there is a constant
$c$ such that
$\int|\varinnerprod{w}{v}|\lambda^{\partial}_{S_{r-1}}(dv)=c$ for every
$w\in S_{r-1}$;
this is because whenever $w$, $w'\in S_{r-1}$ there is an orthogonal
linear transformation taking $w$ to $w'$, and this transformation is an
automorphism of the structure
$(\BbbR^r,\nu,S_{r-1},\lambda^{\partial}_{S_{r-1}})$ (474H).   (In (iii)
below I will come to the calculation of $c$.)

\medskip

\quad{\bf (ii)} Now, writing $\psi_E$ for the canonical outward-normal
function of $E$, we have

$$\eqalignno{\int_{S_{r-1}}\int_{V_v}
  \#(\partstar E\cap T_v^{-1}[\{u\}])\nu(du)\nu(dv)
&=\int_{S_{r-1}}\int_{\partstar E}
   |\varinnerprod{\psi_E(x)}{v}|\nu(dx)\nu(dv)\cr
\displaycause{475P}
&=\iint|\varinnerprod{\psi_E(x)}{v}|\lambda^{\partial}_E(dx)
  \lambda^{\partial}_{S_{r-1}}(dv)\cr
\displaycause{by 235K, recalling that $\lambda^{\partial}_{S_{r-1}}$ and
$\lambda^{\partial}_E$ are just
indefinite-integral measures over $\nu$, while $S_{r-1}$ and
$\partstar E$ are Borel sets}
&=\iint|\varinnerprod{\psi_E(x)}{v}|\lambda^{\partial}_{S_{r-1}}(dv)
  \lambda^{\partial}_E(dx)\cr
\displaycause{because $\varinnerprod{}{}$ is continuous, so
$(x,v)\mapsto\varinnerprod{\psi_E(x)}{v}$ is measurable, while
$\lambda^{\partial}_{S_{r-1}}$
and $\lambda^{\partial}_E$ are totally finite, so we can use Fubini's
theorem}
&=c\per E\cr
\displaycause{by (i) above}
&=c\nu(\partstar E).\cr}$$

\medskip

\quad{\bf (iii)}
We have still to identify the constant $c$.   But observe that
the argument above applies whenever $\nu(\partstar E)$ is finite, and in
particular applies to $E=B(\tbf{0},1)$.   In this case,
$\partstar B(\tbf{0},1)=S_{r-1}$, and for any $v\in S_{r-1}$, $u\in V_v$
we have

$$\eqalign{\#(\partstar B(\tbf{0},1)\cap T^{-1}[\{u\}])
&=2\text{ if }\|u\|<1,\cr
&=1\text{ if }\|u\|=1,\cr
&=0\text{ if }\|u\|>1.\cr}$$

\noindent Since we can identify $\nu$ on $V_v$ as a copy of Lebesgue
measure $\mu_{r-1}$,

\Centerline{$\int_{V_v}
  \#(\partstar B(\tbf{0},1)\cap T^{-1}[\{u\}])\nu(du)
=2\nu\{u:u\in V_v,\,\|u\|<1\}
=2\beta_{r-1}$.}

\noindent This is true for every $v\in V_u$, so from (ii) we get

\Centerline{$2\beta_{r-1}\nu S_{r-1}
=\int_{S_{r-1}}\int_{V_v}
  \#(\partstar E\cap T_v^{-1}[\{u\}])\nu(du)\nu(dv)
=c\nu S_{r-1}$,}

\noindent and $c=2\beta_{r-1}$.   Substituting this into the result of
(ii), we get

\Centerline{$\per E=\nu(\partstar E)
=\Bover1{2\beta_{r-1}}\int_{S_{r-1}}\int_{V_v}
  \#(\partstar E\cap T_v^{-1}[\{u\}])\nu(du)\nu(dv)$.}

\medskip

\quad{\bf (iv)} Before continuing with the main argument, it will help
to set out another elementary fact, this time about
translates of certain
simple subsets of $\Bbb R$.   Suppose that $(G,H,D)$ is a
partition of $\Bbb R$ such that $G$ and $H$ are open,
$D=\partial G=\partial H$ is the common boundary of $G$ and $H$,
and $D$ is locally finite, that is,
$D\cap[-n,n]$ is finite for every $n\in\Bbb N$.   Then

\Centerline{$\sup_{\delta>0}\Bover1{\delta}\mu_1(G\symmdiff(G+\delta))
=\lim_{\delta\downarrow 0}\Bover1{\delta}\mu_1(G\symmdiff(G+\delta))
=\#(D)$}

\noindent if you will allow me to identify `$\infty$' with the cardinal
$\omega$.   \Prf\ If we look at the components of $G$, these are intervals
with endpoints in $D$;  and because $\partial G=\partial H$, distinct
components of $G$ have disjoint closures.   Set
$f(\delta)=\mu_1(G\symmdiff(G+\delta))$ for $\delta>0$.   If
$D$ is infinite, then for any $n\in\Bbb N$ we have disjoint
bounded components
$I_0,\ldots,I_n$ of $G$;  for any $\delta$ small enough,
$G\cap(I_j+\delta)\subseteq I_j$ for every $j\le n$ (because $D$ is locally
finite);  so that

\Centerline{$f(\delta)\ge\sum_{j=0}^n\mu(I_j\symmdiff(I_j+\delta))
=2(n+1)\delta$,}

\noindent and
$\liminf_{\delta\downarrow 0}\Bover1{\delta}f(\delta)\ge 2(n+1)$.
As $n$ is arbitrary,

\Centerline{$\sup_{\delta>0}\Bover1{\delta}f(\delta)
=\lim_{\delta\downarrow 0}\Bover1{\delta}f(\delta)
=\infty$.}

If $D$ is empty, then $G$ is either empty or $\Bbb R$, and

\Centerline{$\sup_{\delta>0}\Bover1{\delta}f(\delta)
=\lim_{\delta\downarrow 0}\Bover1{\delta}f(\delta)
=0$.}

\noindent If $D$ is finite and not empty, let $I_0,\ldots,I_n$ be the
components of $G$.   Then, for all small $\delta>0$, we have

\Centerline{$f(\delta)
=\sum_{j=0}^n\mu(I_j\symmdiff(I_j+\delta))=2(n+1)\delta=\delta\#(D)$,}

\noindent so again

\Centerline{$\sup_{\delta>0}\Bover1{\delta}f(\delta)
=\lim_{\delta\downarrow 0}\Bover1{\delta}f(\delta)
=\#(D)$.  \Qed}

\medskip

\quad{\bf (v)} Returning to the proof in hand, we find that if
$v\in S_{r-1}$ is such that
$\int_{V_v}\#(\partstar E\cap T_v^{-1}[\{u\}])\nu(du)$ is finite, then
the integral is equal to

\Centerline{$\lim_{\delta\downarrow 0}\Bover1{\delta}
  \mu(E\symmdiff(E+\delta v))
=\sup_{\delta>0}\Bover1{\delta}\mu(E\symmdiff(E+\delta v))$.}

\Prf\ It is enough to consider the case in which $v$ is the unit
vector $(0,\ldots,0,1)$, so that we can identify $V_v$ with $\BbbR^{r-1}$
and $\BbbR^r$ with $\BbbR^{r-1}\times\Bbb R$, as in 475J;  in this case,
$E\cap T_v^{-1}[\{u\}]$ turns into $E[\{u\}]$.   Let
$Z_2\subseteq\BbbR^{r-1}$ and $G_u$, $H_u$ and $D_u$, for $u\in\BbbR^{r-1}$,
be as in 475Jc.   In this case, for any $\delta>0$,

$$\eqalign{\mu(E\symmdiff(E+\delta v))
&=\mu(\intstar E\symmdiff(\intstar E+\delta v))\cr
&=\int_{\BbbR^{r-1}}
   \mu_1((\intstar E\symmdiff(\intstar E+\delta v))[\{u\}])
   \mu_{r-1}(du)\cr
&=\int_{\BbbR^{r-1}}\mu_1(G_u\symmdiff(G_u+\delta))\mu_{r-1}(du)\cr
&=\int_{Z_2}\mu_1(G_u\symmdiff(G_u+\delta))\mu_{r-1}(du)\cr}$$

\noindent because $Z_2$ is conegligible.   By (iv),

\Centerline{$\lim_{\delta\downarrow 0}
   \Bover1{\delta}\mu_1(G_u\symmdiff(G_u+\delta))
=\sup_{\delta>0}\Bover1{\delta}\mu_1(G_u\symmdiff(G_u+\delta))
=\#(D_u)$}

\noindent for any $u\in Z_2$.   Applying Lebesgue's Dominated
Convergence Theorem to arbitrary sequences
$\sequencen{\delta_n}\downarrow 0$, we see that

\Centerline{$\lim_{\delta\downarrow 0}
   \Bover1{\delta}\mu(E\symmdiff(E+\delta v))
=\int_{Z_2}\#(D_u)\nu(du)
=\int_{\BbbR^{r-1}}\#(\partstar E\cap T_v^{-1}[\{u\}])\nu(du)$,}

\noindent as required.   To see that

\Centerline{$\lim_{\delta\downarrow 0}\Bover1{\delta}
  \mu(E\symmdiff(E+\delta v))
=\sup_{\delta>0}\Bover1{\delta}\mu(E\symmdiff(E+\delta v))$,}

\noindent set $g(\delta)=\mu(E\symmdiff(E+\delta v))$ for $\delta>0$.
Then for $\delta$, $\delta'>0$ we have

$$\eqalign{g(\delta+\delta')
&=\mu(E\symmdiff(E+(\delta+\delta')v))
\le\mu(E\symmdiff(E+\delta v))
   +\mu((E+\delta v)\symmdiff(E+(\delta+\delta'v)))\cr
&=g(\delta)
   +\mu(\delta v+(E\symmdiff(E+\delta'v)))
=g(\delta)+g(\delta').\cr}$$

\noindent Consequently $g(\delta)\le ng(\bover1n\delta)$ whenever
$\delta>0$ and $n\ge 1$, so

\Centerline{$\Bover1{\delta}g(\delta)
\le\liminf_{n\to\infty}\Bover{n}{\delta}g(\bover1n\delta)
=\lim_{\delta'\downarrow 0}\Bover1{\delta'}g(\delta')$}

\noindent for every $\delta>0$.\ \Qed

\medskip

\quad{\bf (vi)} Putting (v) together with (i)-(iii) above,

\Centerline{$\per E
=\Bover1{2\beta_{r-1}}\int_{S_{r-1}}
  \lim_{\delta\downarrow 0}\Bover1{\delta}
  \mu(E\symmdiff(E+\delta v))\nu(dv)$.}

\noindent To see that we can exchange the limit and the integral,
observe that we can again use the dominated convergence theorem, because

\Centerline{$\int_{S_{r-1}}\sup_{\delta>0}\Bover1{\delta}
  \mu(E\symmdiff(E+\delta v))=2\beta_{r-1}\per E$}

\noindent is finite.   So

\Centerline{$\per E
=\lim_{\delta\downarrow 0}\Bover1{2\beta_{r-1}\delta}
  \int_{S_{r-1}}\mu(E\symmdiff(E+\delta v))\nu(dv)$.}

\medskip

{\bf (b)(i)} $\gamma\le\per E$.   \Prf\
We can suppose that $E$ has finite perimeter.

\medskip

\qquad\grheada\ To begin with, suppose that $x=(0,\ldots,0,\delta)$ where
$\delta>0$.
As in part (a-v) of this proof, set $v=(0,\ldots,0,1)$, identify
$\BbbR^r$ with $\BbbR^{r-1}\times\Bbb R$, and define $G_u$,
$D_u\subseteq\Bbb R$, for $u\in\BbbR^{r-1}$, and
$Z_2\subseteq\BbbR^{r-1}$ as in 475Jc.   Suppose that $u\in Z_2$.
Then $G_u=(\intstar E)[\{u\}]$ is an open set and its
constituent intervals have endpoints in $D_u=(\partstar E)[\{u\}]$.
It follows that for any $t$ in

\Centerline{$G_u\symmdiff(G_u+\delta)
=(\intstar E\symmdiff(\intstar E+\delta v))[\{u\}]$,}

\noindent there must be an
$s\in D_u\cap[t-\delta,\delta]$, and $t\in D_u+[0,\delta]$.
Accordingly

\Centerline{$(\intstar E\symmdiff(\intstar E+\delta v))
     \cap(Z_2\times\Bbb R)
  \subseteq\partstar E+[0,\delta v]$,}

\noindent writing $[0,\delta v]$ for $\{tv:t\in[0,\delta]\}$.   So

\Centerline{$\mu(E\symmdiff(E+\delta v))
=\mu(\intstar E\symmdiff(\intstar E+\delta v))
\le\mu^*(\partstar E+[0,\delta v])$.}

Take any $\epsilon>0$.   We have a sequence $\sequencen{A_n}$ of sets,
all of
diameter at most $\epsilon$, covering $\partstar E$, and such that
$2^{-r+1}\beta_{r-1}\sum_{n=0}^{\infty}(\diam A_n)^{r-1}
\le\epsilon+\nu(\partstar E)$;
we can suppose that every $A_n$ is closed.
Taking $T:\BbbR^r\to\BbbR^{r-1}$ to be the natural projection,
$T[A_n]$ is compact and has
diameter at most $\diam A_n$, so that
$\mu_{r-1}T[A_n]\le 2^{-r+1}\beta_{r-1}(\diam A_n)^{r-1}$ (264H again).
For each $n\in\Bbb N$, the vertical sections of
$A_n+[0,\delta v]$ have diameter at most $\epsilon+\delta$.   So

\Centerline{$\mu(A_n+[0,\delta v])
\le 2^{-r+1}\beta_{r-1}(\diam A_n)^{r-1}(\epsilon+\delta)$.}

\noindent Consequently,

$$\eqalign{\mu(E\symmdiff(E+\delta v))
&\le\mu^*(\partstar E+[0,\delta v])
\le\sum_{n=0}^{\infty}\mu(A_n+[0,\delta v])\cr
&\le\sum_{n=0}^{\infty}
   2^{-r+1}\beta_{r-1}(\diam A_n)^{r-1}(\epsilon+\delta)
\le(\epsilon+\delta)(\epsilon+\nu(\partstar E)).\cr}$$

\noindent As $\epsilon$ is arbitrary,

\Centerline{$\mu(E\symmdiff(E+x))
=\mu(E\symmdiff(E+\delta v))\le\delta\nu(\partstar E)
=\|x\|\nu(\partstar E)$.}

\medskip

\qquad\grheadb\ Of course the same must be true for all other non-zero
$x\in\BbbR^r$, so $\gamma\le\nu(\partstar E)=\per E$.\ \Qed

\medskip

\quad{\bf (ii)} For the other inequality, we need look only at the case
in which $\gamma$ is finite.

\medskip

\qquad\grheada\ In this case, $E$ has finite perimeter.   \Prf\
Let $\phi:\BbbR^r\to B(\tbf{0},1)$ be a Lipschitz function with compact
support.   Take $i$ such that $1\le i\le r$, and consider

$$\eqalignno{\bigl|\int_E\Pd{\phi_i}{\xi_i}d\mu\bigr|
&=\bigl|\int_E\lim_{n\to\infty}2^n(\phi_i(x+2^{-n}e_i)-\phi_i(x))\mu(dx)
   \bigr|\cr
\displaycause{where $\phi=(\phi_1,\ldots,\phi_r)$}
&=\bigl|\lim_{n\to\infty}\int_E2^n(\phi_i(x+2^{-n}e_i)-\phi_i(x))\mu(dx)
   \bigr|\cr
\displaycause{by Lebesgue's Dominated Convergence Theorem, because
$\phi_i$ is Lipschitz and has bounded support}
&=\lim_{n\to\infty}2^n\bigl|\int_E(\phi_i(x+2^{-n}e_i)-\phi_i(x))\mu(dx)
   \bigr|\cr
&=\lim_{n\to\infty}2^n\bigl|\int_{E+2^{-n}e_i}\phi_id\mu
  -\int_E\phi_id\mu\bigr|\cr
&\le\limsup_{n\to\infty}2^n\mu((E+2^{-n}e_i)\symmdiff E)\cr
\displaycause{because $\|\phi_i\|_{\infty}\le 1$}
&\le\gamma.\cr}$$

\noindent Summing over $i$,
$|\int_E\diverg\phi\,d\mu|\le r\gamma$.   As $\phi$ is arbitrary,
$\per E\le r\gamma$ is finite.\ \Qed

\medskip

\qquad\grheadb\ By (a), we have

$$\eqalign{\per E
&=\lim_{\delta\downarrow 0}\Bover1{2\beta_{r-1}\delta}\int_{S_{r-1}}
   \mu(E\symmdiff(E+\delta v))\nu(dv)\cr
&\le\Bover1{2\beta_{r-1}}\gamma\nu S_{r-1}
=\Bover{r\beta_r\gamma}{2\beta_{r-1}},\cr}$$

\noindent as required.
}%end of proof of 475Q

\leader{475R}{Convex sets in \dvrocolon{$\BbbR^r$}}\cmmnt{ For the
next result it will help to have some elementary facts about convex
sets in finite-dimensional spaces out in the open.

\medskip

\noindent}{\bf Lemma}\cmmnt{ (In this lemma I allow $r=1$.)} Let
$C\subseteq\BbbR^r$ be a convex set.

(a) If $x\in C$ and $y\in\interior C$, then $ty+(1-t)x\in\interior C$
for every $t\in\ocint{0,1}$.

(b) $\overline{C}$ and $\interior C$ are convex.

(c) If $\interior C\ne\emptyset$ then
$\overline{C}=\overline{\interior C}$.

(d) If $\interior C=\emptyset$ then $C$ lies within some hyperplane.

(e) $\interior\overline{C}=\interior C$.

\proof{{\bf (a)} Setting $\phi(z)=x+t(z-x)$ for $z\in\BbbR^r$,
$\phi:\BbbR^r\to\BbbR^r$ is a homeomorphism and $\phi[C]\subseteq C$, so

\Centerline{$\phi(y)\in\phi[\interior C]
=\interior\phi[C]\subseteq\interior C$.}

\medskip

{\bf (b)} It follows at once from (a) that $\interior C$ is convex;
$\overline{C}$ is convex because $(x,y)\mapsto tx+(1-t)y$ is continuous
for every $t\in[0,1]$.

\medskip

{\bf (c)} From (a) we see also that if $\interior C\ne\emptyset$ then
$C\subseteq\overline{\interior C}$, so that
$\overline{C}\subseteq\overline{\interior C}$ and
$\overline{C}=\overline{\interior C}$.

\medskip

{\bf (d)} It is enough to consider the case in which $\tbf{0}\in C$,
since if $C=\emptyset$ the result is trivial.   \Quer\ If
$x_1,\ldots,x_r$ are linearly independent elements of $C$, set
$x=\bover1{r+1}\sum_{i=1}^rx_i$;  then

\Centerline{$x+\sum_{i=1}^r\alpha_ix_i
=\sum_{i=1}^r(\alpha_i+\Bover1{r+1})x_i\in C$}

\noindent whenever $\sum_{i=1}^r|\alpha_i|\le\Bover1{r+1}$.   Also,
writing $e_1,\ldots,e_r$ for the standard orthonormal basis of
$\BbbR^r$, we can express $e_i$ as $\sum_{j=1}^r\alpha_{ij}x_j$ for each
$j$;  setting $M=(r+1)\max_{i\le r}\sum_{j=1}^r|\alpha_{ij}|$, we have
$x\pm\Bover1Me_i\in C$ for every $i\le r$, so that $x+y\in C$ whenever
$\|y\|\le\Bover1{M\sqrt r}$, and $x\in\interior C$.\ \Bang

So the linear subspace of $\BbbR^r$ spanned by $C$ has dimension at most
$r-1$.

\medskip

{\bf (e)} If $\interior C=\BbbR^r$ the result is trivial.   If
$\interior C$ is empty, then (d) shows that $C$ is included
in a hyperplane, so that $\interior\overline{C}$ is empty.   Otherwise,
if $x\in\BbbR^r\setminus\interior C$, there is a non-zero $e\in\BbbR^r$
such that $\varinnerprod{e}{y}\le\varinnerprod{e}{x}$ for every
$y\in\interior C$ (4A4Db, or otherwise).   Now, by (c),
$\varinnerprod{e}{y}\le\varinnerprod{e}{x}$ for every
$y\in\overline{C}$, so $x\notin\interior\overline{C}$.   This shows that
$\interior\overline{C}\subseteq\interior C$, so that the two are equal.
}%end of proof of 475R

\leader{475S}{Corollary:  Cauchy's Perimeter Theorem} Let
$C\subseteq\BbbR^r$ be a bounded convex set with non-empty interior.
For $v\in S_{r-1}$ write $V_v$ for $\{x:\varinnerprod{x}{v}=0\}$, and
let $T_v:\BbbR^r\to V_v$ be the orthogonal projection.   Then

\Centerline{$\nu(\partial C)=\Bover1{\beta_{r-1}}\int_{S_{r-1}}
  \nu(T_v[C])\nu(dv)$.}

\proof{{\bf (a)} The first thing to note is that
$\partstar C=\partial C$.   \Prf\ Of course
$\partstar C\subseteq\partial C$ (475Ca).   If
$x\in\partial C$, there is a half-space $V$ containing $x$ and disjoint
from $\interior C$ (4A4Db again, because $\interior C$ is convex), so that

\Centerline{$\limsup_{\delta\downarrow 0}
  \Bover{\mu(B(x,\delta)\setminus C)}{\mu B(x,\delta)}
\ge\limsup_{\delta\downarrow 0}
  \Bover{\mu(B(x,\delta)\setminus\interior V)}{\mu B(x,\delta)}
=\Bover12$,}

\noindent and $x\notin\intstar C$.   On the other hand, if
$x_0\in\interior C$ and $\eta>0$ are
such that $B(x_0,\eta)\subseteq\interior C$, then for any
$\delta\in\ocint{0,1}$ we
can set $t=\Bover{\delta}{\eta+\|x_0-x\|}$, and then

\Centerline{$B(x+t(x_0-x),t\eta)
\subseteq B(x,\delta)\cap((1-t)x+tB(x_0,\eta))
\subseteq B(x,\delta)\cap C$,}

\noindent so that

\Centerline{$\mu(B(x,\delta)\cap C)
\ge\beta_rt^r\eta^r
=\bigl(\Bover{\eta}{\|x_0-x\|+\eta}\bigr)^r\mu B(x,\delta)$;}

\noindent as $\delta$ is arbitrary, $x\in\clstar C$.   This shows that
$\partial C\subseteq\partstar C$ so that $\partstar C=\partial C$.\ \Qed

\medskip

{\bf (b)} We have a function
$\phi:\BbbR^r\to\overline{C}$ defined by taking $\phi(x)$ to be the
unique point
of $C$ closest to $x$, for every $x\in\BbbR^r$ (3A5Md).   This function
is $1$-Lipschitz.
\Prf\ Take any $x$, $y\in\BbbR^r$ and set $e=\phi(x)-\phi(y)$.   We know
that $\phi(x)-\epsilon e\in\overline{C}$, so that
$\|x-\phi(x)-\epsilon e\|\ge\|x+\phi(x)\|$,  for $0\le\epsilon\le 1$;
it follows that $\varinnerprod{(x-\phi(x))}{e}\ge 0$.   Similarly,
$\varinnerprod{(y-\phi(y))}{(-e)}\ge 0$.   Accordingly
$\varinnerprod{(x-y)}{e}\ge\varinnerprod{e}{e}$ and $\|x-y\|\ge\|e\|$.
As $x$ and $y$ are arbitrary, $\phi$ is $1$-Lipschitz.
\Qed

Now suppose that $C'\supseteq C$ is a closed bounded convex set.
Then $\nu(\partial C')\ge\nu(\partial C)$.   \Prf\ Let $\phi$ be the
function defined just above.   By 264G/471J again,
$\nu^*(\phi[\partial C'])\le\nu(\partial C')$.   But if
$x\in\partial C$, there is an $e\in\BbbR^r\setminus\{0\}$ such that
$\varinnerprod{x}{e}\ge\varinnerprod{y}{e}$ for every $y\in C$.   Then
$\phi(x+\alpha e)=x$ for every $\alpha\ge 0$.   Because $C'$ is closed
and bounded, and $x\in C\subseteq C'$, there is a greatest $\alpha\ge 0$
such that $x+\alpha e\in C'$, and in this case
$x+\alpha e\in\partial C'$;
since $\phi(x+\alpha e)=x$, $x\in\phi[\partial C']$.   As $x$ is
arbitrary, $\partial C\subseteq\phi[\partial C']$, and

\Centerline{$\nu(\partial C)
\le\nu^*(\phi[\partial C'])\le\nu(\partial C')$.   \Qed}

Since we can certainly find a closed convex set $C'\supseteq C$ such
that $\nu(\partial C')$ is finite (e.g., any sufficiently large ball or
cube), $\nu(\partial C)<\infty$.   It follows at once that
$\mu(\partial C)=0$ (471L once more).

\medskip

{\bf (c)} The argument so far applies, of course, to every $r\ge 1$ and
every bounded convex set with non-empty interior in $\BbbR^r$.   Moving
to the intended case $r\ge 2$, and fixing $v\in S_{r-1}$ for the moment,
we see that, because $T_v$ is an open map (if we give $V_v$ its subspace
topology), $T_v[C]$ is again a bounded convex set with non-empty
(relative) interior.   Since the subspace measure induced by $\nu$ on
$V_v$ is just a copy of Lebesgue measure, (b) tells us that
$\nu T_v[C]=\nu(\interior_{V_v}T_v[C])$, where here I write
$\interior_{V_v}\mskip-3muT_v[C]$ for the interior of $T_v[C]$ in the subspace topology
of $V_v$.   Now the point is that
$\interior_{V_v}\mskip-3muT_v[C]\subseteq T_v[\interior C]$.
\Prf\ $\interior C$ (taken in $\BbbR^r$) is
dense in $C$ (475Rc), so $W=T_v[\interior C]$ is a relatively
open convex set
which is dense in $T_v[C]$;  now $W=\interior_{V_v}\overline{W}$ (475Re,
applied in $V_v\cong\BbbR^{r-1}$),
so $W\supseteq\interior_{V_v}\mskip-3muT_v[C]$.\ \Qed

It follows that $\#(\partial C\cap T_v^{-1}[\{u\}])=2$ for every
$u\in\interior_{V_v}\mskip-3muT_v[C]$.   \Prf\ $T_v^{-1}[\{u\}]$ is a straight line
meeting $\interior C$ in $y_0$ say.   Because $\overline{C}$ is a
bounded convex set, it meets $T_v^{-1}[\{u\}]$ in a bounded convex set,
which must be a non-trivial closed line segment with endpoints $y_1$,
$y_2$ say.   Now certainly neither $y_1$ nor $y_2$ can be in the
interior of $C$.   Moreover, the open line segments between $y_1$ and
$y_0$, and between $y_2$ and $y_0$, are covered by $\interior C$, by
475Ra;  so
$T_v^{-1}[\{u\}]\cap\partial C=\{y_1,y_2\}$ has just two members.\ \Qed

\medskip

{\bf (d)} This is true for every $v\in S_{r-1}$.   But this means that
we can apply 475Q to see that

$$\eqalign{\nu(\partial C)
&=\nu(\partstar C)
=\Bover1{2\beta_{r-1}}\int_{S_{r-1}}\int_{V_v}
  \#(\partstar C\cap T_v^{-1}[\{u\}])\nu(du)\nu(dv)\cr
&=\Bover1{2\beta_{r-1}}\int_{S_{r-1}}\int_{T_v[C]}
  \#(\partstar C\cap T_v^{-1}[\{u\}])\nu(du)\nu(dv)\cr
&=\Bover1{2\beta_{r-1}}\int_{S_{r-1}}\int_{\interior_{V_v}\mskip-3muT_v[C]}
  \#(\partstar C\cap T_v^{-1}[\{u\}])\nu(du)\nu(dv)\cr
&=\Bover1{\beta_{r-1}}\int_{S_{r-1}}
   \nu(\interior_{V_v}\mskip-3muT_v[C])\nu(dv)
=\Bover1{\beta_{r-1}}\int_{S_{r-1}}\nu(T_v[C])\nu(dv),\cr}$$

\noindent as required.
}%end of proof of 475S

\leader{475T}{Corollary:  the Convex Isoperimetric Theorem} If
$C\subseteq\BbbR^r$ is a bounded convex set, then
$\nu(\partial C)\le r\beta_r(\bover12\diam C)^{r-1}$.

\proof{{\bf (a)} If $C$ is included in some $(r-1)$-dimensional affine
subspace, then

\Centerline{$\nu(\partial C)=\nu\overline{C}
\le\beta_{r-1}(\Bover12\diam C)^{r-1}$}

\noindent by 264H once more.   For completeness, I should check that
$\beta_{r-1}\le r\beta_r$.   \Prf\ Comparing 265F with 265H, or working
from the formulae in 252Q, we have $r\beta_r=2\pi\beta_{r-2}$.   On the
other hand, by the argument of 252Q,

\Centerline{$\beta_{r-1}
=\beta_{r-2}\int_{-\pi/2}^{\pi/2}\cos^{r-1}t\,dt
\le\pi\beta_{r-2}$,}

\noindent so (not coincidentally) we have a factor of two to
spare.\ \Qed

\medskip

{\bf (b)} Otherwise, $C$ has non-empty interior (475Rd), and for any
orthogonal projection $T$ of $\BbbR^r$ onto an $(r-1)$-dimensional
linear
subspace, $\diam T[C]\le\diam C$, so
$\nu(T[C])\le\beta_{r-1}(\Bover12\diam C)^{r-1}$.    Now 475S tells us
that

\Centerline{$\nu(\partial C)\le(\Bover12\diam C)^{r-1}\nu S_{r-1}
=r\beta_r(\Bover12\diam C)^{r-1}$.}
}%end of proof of 475T

\cmmnt{\medskip

\noindent{\bf Remark} Compare 476H below.
}

\exercises{\leader{475X}{Basic exercises (a)}
%\spheader 475Xa
Show that if $C\subseteq\BbbR^r$ is convex, then either
$\mu C=0$ and $\partstar C=\emptyset$, or $\partstar C=\partial C$.
%475B

\spheader 475Xb
Let $A$, $A'\subseteq\BbbR^r$ be any sets.   Show that

\Centerline{$(\partstar A\cap\intstar A')
   \cup(\partstar A'\cap\intstar A)
\subseteq\partstar(A\cap A')
\subseteq(\partstar A\cap\clstar A')\cup(\partstar A'\cap\clstar A)$.}
%475C

\spheader 475Xc Let $A\subseteq\BbbR^r$ be any set, and $B$ a
non-trivial closed ball.   Show that

\Centerline{$\partstar(A\cap B)
  \symmdiff((B\cap\partstar A)\cup(A\cap\partial B))
\subseteq A\cap\partial B\setminus\partstar A$.}
%475C

\sqheader 475Xd Let $E$, $F\subseteq\BbbR^r$ be measurable sets, and $v$
the Federer exterior normal to $E$ at $x\in\intstar F$.   Show that $v$ is
the Federer exterior normal to $E\cap F$ at $x$.
%475C

\spheader 475Xe Let $\frak T$ be the density topology on $\BbbR^r$
(414P) defined from lower Lebesgue density (341E).   Show that, for any
$A\subseteq\BbbR^r$, $A\cup\clstar A$ is the $\frak T$-closure of $A$
and $\intstar A$ is the $\frak T$-interior of the $\frak T$-closure of
$A$.
%475C

\spheader 475Xf Let $A\subseteq\BbbR^r$ be any set.   Show that
$A\setminus\clstar A$ and $\intstar A\setminus A$ are Lebesgue
negligible.
%475C

\sqheader 475Xg Let $E\subseteq\BbbR^r$ be such that $\nu(\partstar E)$
and $\mu E$ are both finite.   Show that, taking $v_x$ to be the Federer
exterior normal to $E$ at any point $x$ where this is defined,

\Centerline{$\int_E\diverg\phi\,d\mu
=\int_{\partstar E}\varinnerprod{\phi(x)}{v_x}\,\nu(dx)$}

\noindent for every bounded Lipschitz function $\phi:\BbbR^r\to\BbbR^r$.
%475L

\sqheader 475Xh Let $\sequencen{E_n}$ be a sequence of measurable
subsets of $\BbbR^r$ such that (i) there is a measurable set $E$ such
that $\lim_{n\to\infty}\mu((E_n\symmdiff E)\cap B(\tbf{0},m))=0$ for
every $m\in\Bbb N$ (ii)
$\sup_{n\in\Bbb N}\nu(\partstar E_n\cap B(\tbf{0},m))$ is finite for
every $m\in\Bbb N$.   Show that $E$ has locally finite
perimeter.
\Hint{$\int_E\diverg\phi\,d\mu
=\lim_{n\to\infty}\int_{E_n}\diverg\phi\,d\mu$
for every Lipschitz function $\phi$ with compact support.}
%475L

\spheader 475Xi Give an example of bounded convex sets $E$ and $F$ such
that
$\partial^{\$}(E\cup F)\not\subseteq\partial^{\$}E\cup\partial^{\$}F$.
%475M

\spheader 475Xj(i) Show that if $A$, $B\subseteq\Bbb R^r$ then
$\partstar(A\cap B)\cap\partstar(A\cup B)
\subseteq\partstar A\cap\partstar B$.    (ii) Show that if $E$,
$F\subseteq\Bbb R$ are Lebesgue measurable, then
$\per(E\cap F)+\per(E\cup F)\le\per E+\per F$.
%475M

\sqheader 475Xk Let $E\subseteq\BbbR^r$ be a set with finite Lebesgue
measure and finite perimeter.   (i) Show that if $H\subseteq\BbbR^r$ is
a half-space, then $\per(E\cap H)\le\per E$.   \Hint{475Ja.}   (ii) Show
that if $C\subseteq\BbbR^r$ is convex, then $\per(E\cap C)\le\per E$.
\Hint{by the Hahn-Banach theorem, $C$ is a limit of polytopes;  use
474Ta.}   (iii) Show that in 475Mc we have
$\per E=\lim_{\alpha\to\infty}\per(E\cap B(\tbf{0},\alpha))$.
%475M

\spheader 475Xl\dvAnew{2009}
Let $E\subseteq\BbbR^r$ be a set with finite measure and finite
perimeter, and
$f:\BbbR^r\to\Bbb R$ a Lipschitz function.   Show that for any unit vector
$v\in\BbbR^r$,
$|\int_E\varinnerprod{v}{\grad f}\,d\mu|\le\|f\|_{\infty}\per E$.
%475M out of order query

\spheader 475Xm\dvAnew{2009} For measurable $E\subseteq\BbbR^r$ set
$p(E)
=\sup_{x\in\BbbR^r\setminus\{0\}}\Bover1{\|x\|}\mu(E\symmdiff(E+x))$.
(i) Show that for any measurable $E$,
$p(E)=\limsup_{x\to\tbf{0}}\Bover1{\|x\|}\mu(E\symmdiff(E+x))$.
(ii) Show that for every $\epsilon>0$ there is an
$E\subseteq\BbbR^r$ such that $\per E=1$ and $p(E)\ge 1-\epsilon$.
(iii) Show that if $E\subseteq\BbbR^r$ is a non-trivial
ball then $\per E=\Bover{r\beta_r}{2\beta_{r-1}}p(E)$.
(iv) Show that if $E\subseteq\BbbR^r$ is a cube then
$\per E=\sqrt{r}p(E)$.   \Hint{475Yf.}
%475Q

\spheader 475Xn\dvAnew{2013} Suppose that $E\subseteq\BbbR^r$ is a bounded
set with finite perimeter, and $\phi$, $\psi:\BbbR^r\to\Bbb R$ two
differentiable functions such that $\grad\phi$ and $\grad\psi$ are
Lipschitz.   Show that

\Centerline{$\int_E\phi\times\nabla^2\psi-\psi\times\nabla^2\phi\,d\mu
=\int_{\partstar E}
 \varinnerprod{(\phi\times\grad\psi-\psi\times\grad\phi)}{v_x}\,\nu(dx)$}

\noindent where $v_x$ is the Federer exterior normal to $E$ at $x$ when
this is defined.   (This is {\bf Green's second identity}.)

\leader{475Y}{Further exercises (a)}
%\spheader 475Ya
Show that if $A\subseteq\BbbR^r$ is Lebesgue
negligible, then there is a Borel set $E\subseteq\BbbR^r$ such that
$A\subseteq\partstar E$.
%475C

\spheader 475Yb\dvAnew{2013} Let $(X,\rho)$ be a metric space and $\mu$ a
strictly positive locally finite topological measure on $X$.   Show that we
can define operations $\clstar$, $\intstar$ and $\partstar$ on $\Cal PX$
for which parts (a)-(f) of 475C will be true.
%475C (easy, but not much use for anything that I know of)

\spheader 475Yc Let $B$ be a ball in $\BbbR^r$ with centre
$y$, and $v$, $v'$ two unit vectors in $\BbbR^r$.   Set

\Centerline{$H=\{x:x\in\BbbR^r,\,\varinnerprod{(x-y)}{v}\le 0\}$,
\quad$H'=\{x:x\in\BbbR^r,\,\varinnerprod{(x-y)}{v'}\le 0\}$.}

\noindent Show that $\mu((H\symmdiff H')\cap B)
=\Bover1{\pi}\arccos(\varinnerprod{v}{v'})\mu B$.
%475F

\allowmorestretch{468}{
\spheader 475Yd Show that $\mu$ is inner regular with respect to the
family of compact sets $K\subseteq\BbbR^r$ such that
$\limsup_{\delta\downarrow 0}
\Bover{\mu(B(x,\delta)\cap K)}{\mu B(x,\delta)}\ge\Bover12$ for every
$x\in K$.
}
%475I 47bits

\spheader 475Ye Let $\sequencen{f_n}$ be a
sequence of functions from $\BbbR^r$ to $\Bbb R$ which is uniformly
bounded and {\bf uniformly Lipschitz}
in the sense that there is some $\gamma\ge 0$
such that every $f_n$ is $\gamma$-Lipschitz.   Suppose that
$f=\lim_{n\to\infty}f_n$ is defined everywhere in $\BbbR^r$.   (i) Show
that if $E\subseteq\BbbR^r$ has finite measure, then
$\int_E\varinnerprod{z}{\grad f}d\mu
=\lim_{n\to\infty}\int_E\varinnerprod{z}{\grad f_n}d\mu$ for every
$z\in\BbbR^r$.   \Hint{look at $E$ of finite perimeter first.}   
(ii) Show that
for any convex function $\phi:\BbbR^r\to\coint{0,\infty}$,
$\int\phi(\grad f)d\mu\le\liminf_{n\to\infty}\int\phi(\grad f_n)d\mu$.
%233Yg %461C %475N out of order query

\spheader 475Yf Let $E\subseteq\BbbR^r$ be a set with locally
finite perimeter.   Show that

\Centerline{$\sup_{x\in\BbbR^r\setminus\{0\}}
  \Bover1{\|x\|}\mu(E\symmdiff(E+x))
=\sup_{\|v\|=1}\int_{\partstar E}|\varinnerprod{v}{v_x}|\nu(dx)$,}

\noindent where $v_x$ is the Federer exterior normal of $E$ at $x$ when
this is defined.
%475Q 475Xm

\spheader 475Yg\dvAnew{2011}
Let $E\subseteq\BbbR^r$ be Lebesgue measurable.
(i) Show that $\intstar E$ is an F$_{\sigma\delta}$
(= $\pmb{\Pi}^0_3$) set, that is, is expressible as the intersection of a
sequence of F$_{\sigma}$ sets.   (ii)(cf.\ {\smc Andretta \& Camerlo 13})
Show that if $E$ is not negligible and $\clstar E$ has empty interior,
then $\intstar E$ is not G$_{\delta\sigma}$
(= $\pmb{\Sigma}^0_3$), that is, cannot be expressed as the union of
sequence of G$_{\delta}$ sets.
%475B out of order query
%n11516

\leaveitout{\query\spheader 475Y? Let $F\subseteq[0,1]$ be the `Cantor
$\bover14$-set', that is, the set
$\{\sum_{i=1}^{\infty}4^{-i}\epsilon_i:\epsilon_i\in\{0,3\}$ for every
$i\}$, so that $F=\bover14F\cup(\bover34+\bover14F)$, and for each
$k\in\Bbb N\,\,F$ is the union of $2^k$ copies of $4^{-k}F$.   Set
$K=F^2\subseteq\BbbR^2$.   Show that the one-dimensional Hausdorff
measure of $K$ is $\sqrt 2$.   (Compare 264J.)   Show that, in the
language of 475Q, $\nu(T_v[K])=0$ for almost every $v\in S_1$.
}%end of leaveitout
}%end of exercises

\endnotes{
\Notesheader{475} The successful identification of the
distributionally-defined notion of `perimeter', as described in \S474,
with the geometrically accessible concept of Hausdorff measure of an
appropriate boundary, is of course the key to any proper understanding
of the results of the last section as well as this one.   The very word
`perimeter' would be unfair if the perimeter of $E\cup F$ were unrelated
to the perimeters of $E$ and $F$;  and from this point of view the
reduced boundary is less suitable than the essential boundary (475Cd,
475Xi).   If we re-examine 474M, we see that it is saying, in effect,
that for many balls $B$ the boundary $\partstar(E\cap B)$ is nearly
$(B\cap\partstar E)\cup(E\cap\partial B)$, and that an outward-normal
function for $E\cap B$ can be assembled from outward-normal functions
for $E$ and $B$.   But looking at 475Xc-475Xd we see that this is
entirely natural;  we need only ensure that $\nu(F\cap\partial B)=0$ for
a $\mu$-negligible set $F$ defined from $E$;  and the `almost every
$\delta$' in the statement of 474M is fully enough to arrange this.
On the other hand, 475Xg seems to be very hard to prove without using
the identification between $\nu(\partstar E)$ and $\per E$.

Concerning 475Q, I ought to emphasize that it is {\it not} generally
true that

\Centerline{$\nu F
=\Bover1{2\beta_{r-1}}\int_{S_{r-1}}\int_{V_v}
  \#(F\cap T_v^{-1}[\{u\}])\nu(du)\nu(dv)$}

\noindent even for $r=2$ and compact sets $F$ with $\nu F<\infty$.
%(See 474Xb\query.)
We are here approaching one of the many fundamental
concepts of geometric measure theory which I am ignoring.   The key word
is `rectifiability';
for `rectifiable' sets a wide variety of concepts of $k$-dimensional
measure coincide, including the integral-geometric form above, and
$\partstar E$ is rectifiable whenever $E$ has locally finite perimeter
({\smc Evans \& Gariepy 92}, 5.7.3).
For the general theory of rectifiable sets, see the last quarter of
{\smc Mattila 95}, or Chapter 3 of {\smc Federer 69}.

I have already noted that the largest volumes for sets of given diameter
or perimeter are provided by balls (see 264H and the notes to \S474).
The isoperimetric theorem for convex sets (475T) is of the same form:
once again, the best constant (here, the largest perimeter for a convex
set of given diameter, or the smallest diameter for a convex set of
given perimeter) is provided by balls.

475Qb gives an alternative characterization of `set of finite perimeter',
with bounds on the perimeter which are sometimes useful.
}%end of notes

\discrpage

