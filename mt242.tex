\frfilename{mt242.tex}
\versiondate{19.11.03}
\copyrightdate{1997}

\def\chaptername{Function spaces}
\def\sectionname{$L^1$}

\newsection{242}

While the space $L^0$ treated in the previous section is of very great
intrinsic interest, its chief use in the elementary theory is as a
space
in which some of the most important spaces of functional analysis are
embedded.   In the next few sections I introduce these one at a time.

The first is the space $L^1$ of equivalence classes of integrable
functions.   The importance of this space is not only that it offers a
language in which to express those many theorems about integrable
functions which do not depend on the differences between two functions
which are equal almost everywhere.   It can also appear as the natural
space in which to seek solutions to a wide variety of integral
equations, and as the completion of a space of continuous functions.

\vleader{72pt}{242A}{The space $L^1$}  Let $(X,\Sigma,\mu)$ be any
measure space.

\header{242Aa}{\bf (a)} Let $\eusm L^1=\eusm L^1(\mu)$ be the set of
real-valued functions, defined on subsets of $X$, which are integrable
over $X$.   Then
$\eusm L^1\subseteq\eusm L^0=\eusm L^0(\mu)$\cmmnt{, as
defined in \S241}, and, for $f\in\eusm L^0$, we
have $f\in\eusm L^1$ iff there is a $g\in\eusm L^1$ such that
$|f|\leae g$;   if $f\in\eusm L^1$, $g\in\eusm L^0$ and $f\eae g$,
then
$g\in\eusm L^1$.   \prooflet{(See 122P-122R.)}


\header{242Ab}{\bf (b)} Let $L^1=L^1(\mu)\subseteq L^0=L^0(\mu)$ be
the
set of equivalence
classes of members of $\eusm L^1$.   If $f$, $g\in \eusm L^1$ and
$f\eae g$ then $\int f=\int g$\cmmnt{ (122Rb)}.   Accordingly we may
define a
functional $\int$ on $L^1$ by writing $\int f^{\ssbullet}=\int f$ for
every $f\in\eusm L^1$.

\header{242Ac}{\bf (c)}\cmmnt{ It will be convenient to be able to
write} $\int_Au$ for $u\in L^1$, $A\subseteq X$\dvro{ is}{;  this may
be} defined by saying that
$\int_Af^{\ssbullet}=\int_Af$ for every $f\in\eusm L^1$\cmmnt{,
where
the integral is defined in 214D}.   \prooflet{\Prf\ I
have only to check that if $f\eae g$ then $\int_Af=\int_Ag$;  and this
is because $f\restr A=g\restr A$ almost everywhere in $A$.\ \Qed}

If $E\in\Sigma$ and $u\in L^1$ then
$\int_Eu=\int u\times(\chi E)^{\ssbullet}$\prooflet{;  this is because
$\int_Ef=\int f\times\chi E$ for every integrable function $f$
(131Fa)}.

\spheader 242Ad If $u\in L^1$, there is a $\Sigma$-measurable,
$\mu$-integrable function $f:X\to\Bbb R$ such that $f^{\ssbullet}=u$.
\prooflet{\Prf\ As noted in 241Bk, there is a measurable $f:X\to\Bbb
R$
such that $f^{\ssbullet}=u$;  but of course $f$ is integrable because
it
is equal almost everywhere to some integrable function.\ \Qed}

\leader{242B}{Theorem} Let $(X,\Sigma,\mu)$ be any measure space.
Then
$L^1(\mu)$ is a linear subspace of $L^0(\mu)$ and $\int:L^1\to\Bbb R$
is
a linear functional.

\proof{ If $u$, $v\in L^1=L^1(\mu)$ and $c\in\Bbb R$ let
$f$, $g$ be integrable functions such that $u=f^{\ssbullet}$ and
$v=g^{\ssbullet}$;  then $f+g$ and $cf$ are integrable, so
$u+v=(f+g)^{\ssbullet}$ and $cu=(cf)^{\ssbullet}$ belong to $L^1$.
Also

\Centerline{$\int u+v=\int f+g=\int f+\int g=\int u+\int v$}

\noindent and

\Centerline{$\int cu=\int cf=c\int f=c\int u$.}
}

\leader{242C}{The order structure of $L^1$} Let $(X,\Sigma,\mu)$ be
any
measure space.

\header{242Ca}{\bf (a)} $L^1=L^1(\mu)$ has an order structure derived
from that of $L^0=L^0(\mu)$\cmmnt{ (241E);  that is,
$f^{\ssbullet}\le g^{\ssbullet}$ iff $f\le g$ a.e}.   \cmmnt{Being a
linear subspace of
$L^0$,} $L^1$ \dvro{is}{must be} a partially ordered linear
space\cmmnt{;  the two
conditions of 241Ec are obviously inherited by linear subspaces}.

\dvro{If}{Note also that if} $u$,
$v\in L^1$ and $u\le v$ then $\int u\le\int v$\prooflet{, because if
$f$, $g$ are
integrable functions and $f\leae g$ then $\int f\le\int g$ (122Od)}.

\header{242Cb}{\bf (b)} If $u\in L^0$, $v\in L^1$ and $|u|\le |v|$
then
$u\in L^1$.  \prooflet{\Prf\ Let $f\in\eusm L^0=\eusm L^0(\mu)$,
$g\in\eusm L^1=\eusm L^1(\mu)$ be such that
$u=f^{\ssbullet}$ and $v=g^{\ssbullet}$;   then $g$ is integrable and
$|f|\leae |g|$, so $f$ is integrable and $u\in L^1$.\ \Qed}

\header{242Cc}{\bf (c)} In particular, $|u|\in L^1$ whenever $u\in
L^1$,
and

\Centerline{$|\int u|=\max(\int u,\int(-u))\le\int|u|$\dvro{.}{,}}

\cmmnt{\noindent because $u$, $-u\le|u|$.}

\header{242Cd}{\bf (d)} \cmmnt{Because $|u|\in L^1$ for every $u\in
L^1$,

\Centerline{$u\vee v=\Bover12(u+v+|u-v|)$,
\quad$u\wedge v=\Bover12(u+v-|u-v|)$}

\noindent belong to $L^1$ for all $u$, $v\in L^1$.   But if $w\in L^1$
we surely have

\Centerline{$w\le u\ \&\ w\le v\iff w\le u\wedge v$,}

\Centerline{$w\ge u\ \&\ w\ge v\iff w\ge u\vee v$}

\noindent because these are true for all $w\in L^0$, so
$u\vee v=\sup\{u,v\}$ and $u\wedge v=\inf\{u,v\}$ in $L^1$.
Thus }$L^1$ is\cmmnt{, in itself,} a Riesz space.

\header{242Ce}{\bf (e)} Note that if $u\in L^1$,
then $u\ge 0$ iff $\int_Eu\ge 0$ for every $E\in\Sigma$\cmmnt{;
this
is because if $f$ is an integrable function on $X$ and $\int_Ef\ge 0$
for every $E\in\Sigma$,
then $f\ge 0$ a.e.\ (131Fb)}.   \dvro{If}{More generally, if} $u$, $v\in
L^1$ and $\int_Eu\le\int_Ev$ for every $E\in\Sigma$, then $u\le v$.
\dvro{If}{It follows at once that
if} $u$, $v\in L^1$ and $\int_Eu=\int_Ev$ for every $E\in\Sigma$, then
$u=v$\cmmnt{ (cf.\ 131Fc)}.

\spheader 242Cf If $u\ge 0$ in $L^1$, there is a non-negative
$f\in\eusm  L^1$ such that $f^{\ssbullet}=u$\cmmnt{ (compare
241Eg)}.

\leader{242D}{The norm of $L^1$} Let $(X,\Sigma,\mu)$ be any measure
space.

\header{242Da}{\bf (a)} For $f\in\eusm L^1=\eusm L^1(\mu)$ I write
$\|f\|_1=\int|f|\in\coint{0,\infty}$.
For $u\in L^1=L^1(\mu)$ set $\|u\|_1=\int|u|$, so that
$\|f^{\ssbullet}\|_1=\|f\|_1$ for every $f\in\eusm L^1$.
Then $\|\,\|_1$ is a norm on $L^1$.   \prooflet{\Prf\ (i) If $u$,
$v\in L^1$ then $|u+v|\le|u|+|v|$, by 241Ee, so

\Centerline{$\|u+v\|_1=\int|u+v|
\le\int|u|+|v|=\int|u|+\int|v|=\|u\|_1+\|v\|_1$.}

\noindent (ii) If $u\in L^1$ and $c\in\Bbb R$ then

\Centerline{$\|cu\|_1=\int|cu|=\int|c||u|=|c|\int|u|=|c|\|u\|_1$.}

\noindent (iii) If $u\in L^1$ and
$\|u\|_1=0$, express $u$ as $f^{\ssbullet}$, where $f\in\eusm L^1$;
then $\int|f|=\int|u|=0$.   Because $|f|$ is non-negative, it must be
zero almost everywhere (122Rc), so $f=0$ a.e.\ and $u=0$ in $L^1$.\
\Qed}

\header{242Db}{\bf (b)}\cmmnt{ Thus} $L^1$, with $\|\,\|_1$, is a
normed space and $\int:L^1\to\Bbb R$ is a linear operator;
\cmmnt{observe that} $\|\int\|\le 1$\prooflet{, because

\Centerline{$|\int u|\le\int|u|=\|u\|_1$}

\noindent for every $u\in L^1$}.

\header{242Dc}{\bf (c)} If $u$, $v\in L^1$ and $|u|\le|v|$, then

\Centerline{$\|u\|_1=\int|u|\le\int|v|=\|v\|_1$.}

\noindent In particular, $\|u\|_1=\||u|\|_1$ for every $u\in L^1$.

\header{242Dd}{\bf (d)} \dvro{If}{Note the following property of the
normed Riesz space $L^1$:  if} $u$, $v\in L^1$ and $u$, $v\ge 0$, then

\Centerline{$\|u+v\|_1\prooflet{\mskip5mu=\intop
u+v
=\intop u+\intop v}=\|u\|_1+\|v\|_1$.}

\spheader 242De The set $(L^1)^+=\{u:u\ge 0\}$ is closed in $L^1$.
\prooflet{\Prf\ If $v\in L^1$, $u\in(L^1)^+$ then
$\|u-v\|_1\ge\|v\wedge 0\|_1$;  this is because if $f$,
$g\in\eusm L^1$ and $f\ge 0$ a.e.,
$|f(x)-g(x)|\ge|\min(g(x),0)|$ whenever $f(x)$ and $g(x)$ are both
defined and $f(x)\ge 0$, which is almost everywhere, so

\Centerline{$\|u-v\|_1=\int|f-g|\ge\int|g\wedge\tbf{0}|
=\|v\wedge 0\|_1$.}

\noindent  Now this means that if $v\in L^1$ and $v\not\ge 0$, the
ball
$\{w:\|w-v\|_1<\delta\}$ does not meet $(L^1)^+$, where
$\delta=\|v\wedge 0\|_1>0$ because $v\wedge 0\ne 0$.   Thus
$L^1\setminus(L^1)^+$ is open and $(L^1)^+$ is closed.\ \Qed}

\leader{242E}{}\cmmnt{ For the next result we need a variant of
B.Levi's theorem.

\medskip

\noindent}{\bf Lemma} Let $(X,\Sigma,\mu)$ be a measure space and
$\sequencen{f_n}$ a sequence of $\mu$-integrable real-valued functions
such that $\sum_{n=0}^{\infty}\int|f_n|<\infty$.   Then
$f=\sum_{n=0}^{\infty}f_n$ is integrable and

\Centerline{$\int f=\sum_{n=0}^{\infty}\int f_n$,
\quad$\int|f|\le\sum_{n=0}^{\infty}\int|f_n|$.}

\proof{{\bf (a)} Suppose first that every $f_n$ is non-negative.
Set $g_n=\sum_{k=0}^nf_k$ for each $n$;  then $\sequencen{g_n}$ is
increasing a.e.\ and

\Centerline{$\lim_{n\to\infty}\int g_n
=\sum_{k=0}^{\infty}\int f_k$}

\noindent is finite, so by B.Levi's theorem (123A)
$f=\lim_{n\to\infty}g_n$ is integrable and

\Centerline{$\int f=\lim_{n\to\infty}\int g_n
=\sum_{k=0}^{\infty}\int  f_k$.}

\noindent In this case, of course,

\Centerline{$\int|f|=\int f
=\sum_{n=0}^{\infty}\int f_n
=\sum_{n=0}^{\infty}\int|f_n|$.}

\medskip

{\bf (b)}  For the general case, set $f_n^+={1\over 2}(|f_n|+f_n)$,
$f_n^-={1\over2}(|f_n|-f_n)$, as in 241Ef;  then $f_n^+$ and
$f_n^-$ are non-negative integrable functions, and

\Centerline{$\sum_{n=0}^{\infty}\int f_n^+
  +\sum_{n=0}^{\infty}\int f_n^-
=\sum_{n=0}^{\infty}\int|f_n|<\infty$.}

\noindent So $h_1=\sum_{n=0}^{\infty}f_n^+$ and
$h_2=\sum_{n=0}^{\infty}f_n^-$ are both integrable.   Now
$f\eae h_1-h_2$, so

\Centerline{$\int f=\int h_1-\int h_2
=\sum_{n=0}^{\infty}\int f_n^+
  -\sum_{n=0}^{\infty}\int f_n^-
=\sum_{n=0}^{\infty}\int f_n$.}

\noindent Finally

\Centerline{$\int|f|\le\int|h_1|+\int|h_2|
=\sum_{n=0}^{\infty}\int f_n^+
  +\sum_{n=0}^{\infty}\int f_n^-
=\sum_{n=0}^{\infty}\int|f_n|$.}
}

\leader{242F}{Theorem} For any measure space $(X,\Sigma,\mu)$,
$L^1(\mu)$ is
complete under its norm $\|\,\|_1$.

\proof{ Let $\sequencen{u_n}$ be a sequence in
$L^1$ such that $\|u_{n+1}-u_n\|_1\le 4^{-n}$ for every $n\in\Bbb N$.
Choose integrable functions $f_n$
such that $f_0^{\ssbullet}=u_0$,
$f_{n+1}^{\ssbullet}=u_{n+1}-u_n$ for each $n\in\Bbb N$.   Then

\Centerline{$\sum_{n=0}^{\infty}\int|f_n|
=\|u_0\|_1+\sum_{n=0}^{\infty}\|u_{n+1}-u_n\|_1<\infty$.}

\noindent So $f=\sum_{n=0}^{\infty}f_n$ is integrable, by 242E, and
$u=f^{\ssbullet}\in L^1$.   Set $g_n=\sum_{j=0}^nf_j$ for each $n$;
then $g_n^{\ssbullet}=u_n$, so

\Centerline{$\|u-u_n\|_1
=\int|f-g_n|
\le\int\sum_{j=n+1}^{\infty}|f_j|
\le\sum_{j=n+1}^{\infty}4^{-j}
=4^{-n}/3$}

\noindent for each $n$.      Thus $u=\lim_{n\to\infty}u_n$ in $L^1$.
As $\sequencen{u_n}$ is arbitrary, $L^1$ is complete (2A4E).
}

\leader{242G}{Definition}\cmmnt{ It will be convenient, for later
reference, to
introduce the following phrase.}   A {\bf Banach lattice} is a Riesz
space $U$ together with a norm $\|\,\|$ on $U$ such that (i)
$\|u\|\le\|v\|$ whenever $u$, $v\in U$ and $|u|\le|v|$\cmmnt{,
writing
$|u|$ for $u\vee(-u)$, as in 241Ee} (ii) $U$ is
complete under $\|\,\|$.   \cmmnt{Thus 242Dc and 242F amount to
saying
that the
normed Riesz space $(L^1,\|\,\|_1)$ is a Banach lattice.}

\leader{242H}{$L^1$ as a Riesz \dvrocolon{space}}\cmmnt{ We can
discuss the ordered linear space $L^1$ in the language already used in
241E-241G %241E 241F 241G
for $L^0$.

\medskip

\noindent}{\bf Theorem} Let $(X,\Sigma,\mu)$ be any measure space.
Then $L^1=L^1(\mu)$ is Dedekind complete.
\proof{{\bf (a)}  Let $A\subseteq L^1$ be any non-empty set which is
bounded above in $L^1$.   Set

\Centerline{$A'=\{u_0\vee\ldots\vee u_n:u_0,\ldots, u_n\in A\}$.}

\noindent Then $A\subseteq A'$, $A'$ has the same upper bounds as $A$
and $u\vee v\in A'$ for all $u$, $v\in A'$.   Taking $w_0$ to be any
upper bound of $A$ and $A'$, we have $\int u\le\int w_0$ for every
$u\in A'$, so $\gamma=\sup_{u\in A'}\int u$ is defined in $\Bbb R$.
For each
$n\in\Bbb N$, choose $u_n\in A'$ such that $\int u_n\ge\gamma-2^{-n}$.
Because $L^0=L^0(\mu)$ is Dedekind $\sigma$-complete (241Ga),
$u^*=\sup_{n\in\Bbb N}u_n$ is defined in $L^0$, and $u_0\le u^*\le
w_0$
in $L^0$.   Consequently

\Centerline{$0\le u^*-u_0\le w_0-u_0$}

\noindent in $L^0$.   But $w_0-u_0\in L^1$, so $u^*-u_0\in L^1$
(242Cb)
and $u^*\in L^1$.

\medskip

{\bf (b)} The point is that $u^*$ is an upper bound for $A$.
\Prf\ If $u\in A$, then $u\vee u_n\in A'$ for every $n$, so

$$\eqalignno{\|u-u\wedge u^*\|_1
&=\int u-u\wedge u^*
\le \int u-u\wedge u_n\cr
\noalign{\noindent (because $u\wedge u_n\le u_n\le u^*$, so $u\wedge
u_n\le u\wedge u^*$)}
&=\int u\vee u_n-u_n\cr
\noalign{\noindent (because $u\vee u_n+u\wedge u_n=u+u_n$ -- see the
formulae in 242Cd)}
&=\int u\vee u_n-\int u_n
\le\gamma-(\gamma-2^{-n})
=2^{-n}\cr}$$

\noindent for every $n$;  so $\|u-u\wedge u^*\|_1=0$.   But this
means that $u=u\wedge u^*$, that is, that $u\le u^*$.   As $u$ is
arbitrary, $u^*$ is an upper bound for $A$.\ \Qed

\medskip

{\bf (c)} On the other hand, any upper bound for $A$ is surely an
upper
bound for $\{u_n:n\in\Bbb N\}$, so is greater than or equal to $u^*$.
Thus $u^*=\sup A$ in $L^1$.   As $A$ is arbitrary, $L^1$ is Dedekind
complete.
}%end of proof of 242H

\cmmnt{
\medskip

\noindent{\bf Remark} Note that the order-completeness of $L^1$,
unlike
that of $L^0$, does not depend on any particular property of the
measure
space $(X,\Sigma,\mu)$.
}%end of comment

\leader{242I}{The Radon-Nikod\'ym \dvro{Theorem}{theorem}}\cmmnt{ I
think it is worth re-writing the Radon-Nikod\'ym theorem (232E) in the
language of this chapter.

\medskip

\noindent{\bf Theorem}} Let $(X,\Sigma,\mu)$ be a measure space.
Then
there is a canonical bijection between $L^1=L^1(\mu)$ and the set of
truly continuous additive functionals $\nu:\Sigma\to\Bbb R$, given by
the formula

\Centerline{$\nu F=\int_Fu$ for $F\in\Sigma$, $u\in L^1$.}

\cmmnt{\medskip

\noindent{\bf Remark} Recall that if $\mu$ is $\sigma$-finite, then
the
truly continuous additive functionals are just the absolutely
continuous
countably additive functionals;  and that if $\mu$ is totally finite,
then all absolutely continuous (finitely) additive functionals are
truly
continuous (232Bd).
}%end of comment

\proof{ For $u\in L^1$, $F\in\Sigma$ set $\nu_uF=\int_Fu$.
If $u\in L^1$, there is an integrable function $f$ such that
$f^{\ssbullet}=u$, in which case

\Centerline{$F\mapsto\nu_uF=\int_Ff:\Sigma\to\Bbb R$}

\noindent is additive and truly continuous, by 232D.   If
$\nu:\Sigma\to\Bbb R$ is additive and truly continuous, then by 232E
there is an integrable function $f$ such that $\nu F=\int_Ff$ for
every
$F\in\Sigma$;  setting $u=f^{\ssbullet}$ in $L^1$, $\nu=\nu_u$.
Finally, if $u$, $v$ are distinct members of $L^1$, there is an
$F\in\Sigma$ such
that $\int_Fu\ne\int_Fv$ (242Ce), so that $\nu_u\ne\nu_v$;  thus
$u\mapsto\nu_u$ is injective as well as surjective.
}%end of proof of 242I

\leader{242J}{Conditional expectations revisited}\cmmnt{ We now have
the machinery necessary for a new interpretation of some of the ideas
of \S233.

\medskip

}{\bf (a)} Let $(X,\Sigma,\mu)$ be
a measure space, and $\Tau$ a $\sigma$-subalgebra of
$\Sigma$\cmmnt{,
as in 233A}.   Then $(X,\Tau,\mu\restrp\Tau)$ is a measure space, and
$\eusm L^0(\mu\restrp\Tau)\subseteq\eusm L^0(\mu)$;
\cmmnt{moreover,}
if $f$, $g\in\eusm L^0(\mu\restrp\Tau)$, then
$f=g\,\,(\mu\restrp\Tau)$-a.e.\ iff
$f=g\,\,\mu$-a.e.   \prooflet{\Prf\ There are
$\mu\restrp\Tau$-conegligible sets
$F$, $G\in\Tau$ such that $f\restr F$ and $g\restr G$ are
$\Tau$-measurable;  set

\Centerline{$E=\{x:x\in F\cap G,\,f(x)\ne g(x)\}\in\Tau$;}

\noindent then

\Centerline{$f=g\,\,(\mu\restrp\Tau)$-a.e.\ $\iff
(\mu\restrp\Tau)(E)=0
\iff \mu E=0 \iff f=g\,\,\mu$-a.e.  \Qed}
}

Accordingly we have a canonical map $S:L^0(\mu\restrp\Tau)\to L^0(\mu)$
defined by saying that if $u\in L^0(\mu\restrp\Tau)$ is the
equivalence
class of $f\in\eusm L^0(\mu\restrp\Tau)$, then $Su$ is the equivalence
class of $f$ in $L^0(\mu)$.   \cmmnt{It is easy to check, working
through the
operations described in 241D, 241E and 241H, that} $S$ is linear,
injective and order-preserving, and\cmmnt{ that} $|Su|=S|u|$,
$S(u\vee v)=Su\vee Sv$ and $S(u\times v)=Su\times Sv$ for $u$,
$v\in L^0(\mu\restrp\Tau)$.

\header{242Jb}{\bf (b)} Next, if $f\in\eusm L^1(\mu\restrp\Tau)$, then
$f\in\eusm L^1(\mu)$ and
$\int fd\mu=\int fd(\mu\restrp\Tau)$\prooflet{ (233B)};  so
$Su\in L^1(\mu)$ and $\|Su\|_1=\|u\|_1$ for every
$u\in L^1(\mu\restrp\Tau)$.

Observe also that every member of $L^1(\mu)\cap
S[L^0(\mu\restrp\Tau)]$
is actually in $S[L^1(\mu\restrp\Tau)]$.    \prooflet{\Prf\ Take
$u\in L^1(\mu)\cap S[L^0(\mu\restrp\Tau)]$.   Then $u$ is expressible
both as
$f^{\ssbullet}$ where $f\in\eusm L^1(\mu)$, and as $g^{\ssbullet}$
where
$g\in \eusm L^0(\mu\restrp\Tau)$.   So $g\eae f$, and $g$ is
$\mu$-integrable, therefore $(\mu\restrp\Tau)$-integrable (233B
again).\
\Qed}

This means that
$S:L^1(\mu\restrp\Tau)\to L^1(\mu)\cap S[L^0(\mu\restrp\Tau)]$ is a
bijection.

\header{242Jc}{\bf (c)} Now suppose that $\mu X=1$, so that
$(X,\Sigma,\mu)$ is a
probability space.   \cmmnt{Recall that $g$ is a conditional
expectation of $f$
on $\Tau$ if $g$ is $\mu\restrp\Tau$-integrable, $f$ is
$\mu$-integrable
and $\int_Fg=\int_Ff$ for every $F\in\Tau$;  and that every
$\mu$-integrable function has such a conditional expectation
(233D).}   If
$g$ is a conditional expectation of $f$ and $f_1=f\,\,\mu$-a.e.\ then
$g$ is a conditional expectation of $f_1$\cmmnt{, because
$\int_Ff_1=\int_Ff$
for every $F$};  and \cmmnt{I have already remarked in 233Dc that}
if
$g$, $g_1$ are conditional expectations of $f$ on $\Tau$ then
$g=g_1\,\,\mu\restrp\Tau$-a.e.

\header{242Jd}{\bf (d)} This means that we have an operator
$P:L^1(\mu)\to L^1(\mu\restrp\Tau)$
defined by saying that $P(f^{\ssbullet})=g^{\ssbullet}$ whenever
$g\in\eusm L^1(\mu\restrp\Tau)$ is a conditional expectation of
$f\in\eusm L^1(\mu)$
on $\Tau$\cmmnt{;  that is, that $\int_FPu=\int_Fu$ whenever
$u\in L^1(\mu)$ and $F\in\Tau$}.   \cmmnt{If we identify $L^1(\mu)$,
$L^1(\mu\restrp\Tau)$ with the sets of absolutely continuous additive
functionals defined on $\Sigma$ and $\Tau$, as in 242I, then $P$
corresponds to the operation $\nu\mapsto\nu\restrp\Tau$.}

\header{242Je}{\bf (e)} \dvro{$P$ is linear and
order-preserving.}{Because $Pu$ is uniquely defined in
$L^1(\mu\restrp\Tau)$ by the requirement $\int_FPu=\int_Fu$ for every
$F\in\Tau$ (242Ce), we see
that $P$ must be linear.}   \prooflet{\Prf\ If $u$, $v\in L^1(\mu)$
and
$c\in\Bbb R$, then

\Centerline{$\int_FPu+Pv=\int_FPu+\int_FPv=\int_Fu+\int_Fv
=\int_Fu+v=\int_FP(u+v)$,}

\Centerline{$\int_FP(cu)=\int_Fcu=c\int_Fu=c\int_FPu=\int_FcPu$}

\noindent for every $F\in\Tau$.\ \QeD\    Also, if $u\ge 0$, then
$\int_FPu=\int_Fu\ge 0$ for every $F\in\Tau$, so $Pu\ge 0$ (242Ce
again).

It follows at once that $P$ is order-preserving, that is, that
$Pu\le Pv$ whenever $u\le v$.}   Consequently

\Centerline{$|Pu|=Pu\vee(-Pu)=Pu\vee P(-u)\le P|u|$}

\noindent for every $u\in L^1(\mu)$\prooflet{, because
$u\le|u|$ and $-u\le|u|$}.   Finally, $P$ is a bounded linear operator,
with norm $1$.   \prooflet{\Prf\ The last formula tells us that

\Centerline{$\|Pu\|_1\le\|P|u|\|_1=\int P|u|=\int |u|=\|u\|_1$}

\noindent for every $u\in L^1(\mu)$, so $\|P\|\le 1$.   On the other hand,
$P(\chi X^{\ssbullet})=\chi X^{\ssbullet}\ne 0$, so $\|P\|=1$.\ \Qed}

\header{242Jf}{\bf (f)} We may legitimately regard
$Pu\in L^1(\mu\restrp\Tau)$ as
`the' conditional expectation of $u\in L^1(\mu)$ on $\Tau$;  $P$ is
the {\bf conditional expectation operator}.

\header{242Jg}{\bf (g)} If $u\in L^1(\mu\restrp\Tau)$, then\cmmnt{
we have a
corresponding} $Su\in L^1(\mu)$, as in (b);  now $PSu=u$.
\prooflet{\Prf\
$\int_FPSu=\int_FSu=\int_Fu$ for every $F\in\Tau$.\ \QeD}
Consequently $SPSP=SP:L^1(\mu)\to L^1(\mu)$.

\cmmnt{\header{242Jh}{\bf (h)} The distinction drawn above between
$u=f^{\ssbullet}\in L^0(\mu\restrp\Tau)$ and $Su=f^{\ssbullet}\in L^0(\mu)$
is of course pedantic.   I believe it is necessary to be aware of such
distinctions, even though for nearly all purposes it is safe as well
as convenient to regard $L^0(\mu\restrp\Tau)$ as actually a subset of
$L^0(\mu)$.   If we do so, then (b) tells us that we can identify
$L^1(\mu\restrp\Tau)$ with $L^1(\mu)\cap L^0(\mu\restrp\Tau)$, while
(g) becomes `$P^2=P$'.
}

\leader{242K}{}\cmmnt{ The language just introduced allows the
following re-formulations of 233J-233K.

\medskip

\noindent}{\bf Theorem} Let $(X,\Sigma,\mu)$ be a probability space
and
$\Tau$ a $\sigma$-subalgebra of $\Sigma$.   Let $\phi:\Bbb R\to\Bbb R$
be a convex function and $\bar\phi:L^0(\mu)\to L^0(\mu)$ the
corresponding operator defined by setting
$\bar\phi(f^{\ssbullet})=(\phi f)^{\ssbullet}$\cmmnt{ (241I)}.   If
$P:L^1(\mu)\to L^1(\mu\restrp\Tau)$ is the conditional expectation
operator, then $\bar\phi(Pu)\le P(\bar\phi u)$ whenever
$u\in L^1(\mu)$ is such that $\bar\phi(u)\in L^1(\mu)$.

\proof{ This is just a restatement of 233J.}

\leader{242L}{Proposition} Let $(X,\Sigma,\mu)$ be a probability
space,
and $\Tau$ a $\sigma$-subalgebra of $\Sigma$.   Let
$P:L^1(\mu)\to L^1(\mu\restrp\Tau)$
be the corresponding conditional expectation
operator.   If $u\in L^1=L^1(\mu)$ and $v\in L^0(\mu\restrp\Tau)$,
then
$u\times v\in L^1$ iff $P|u|\times v\in L^1$, and in this case
$P(u\times v)=Pu\times v$;  in particular, $\int u\times v=\int
Pu\times
v$.

\proof{ (I am here using the identification of $L^0(\mu\restrp\Tau)$
as
a subspace of $L^0(\mu)$, as suggested in 242Jh.)
Express $u$ as $f^{\ssbullet}$ and $v$ as $h^{\ssbullet}$, where
$f\in\eusm L^1=\eusm L^1(\mu)$ and $h\in\eusm L^0(\mu\restrp\Tau)$.
Let $g$, $g_0\in\eusm L^1(\mu\restrp\Tau)$ be conditional expectations
of $f$, $|f|$ respectively, so that $Pu=g^{\ssbullet}$ and
$P|u|=g_0^{\ssbullet}$.   Then, using 233K,

\Centerline{$u\times v\in L^1\iff f\times h\in\eusm L^1
\iff g_0\times h\in\eusm L^1\iff P|u|\times v\in L^1$,}

\noindent and in this case $g\times h$ is a conditional expectation of
$f\times h$, that is, $Pu\times v=P(u\times v)$.
}%end of proof of 242L

\leader{242M}{$L^1$ as a \dvro{completion:}{completion}}\cmmnt{ I
mentioned in the introduction to
this section that $L^1$ appears in functional analysis as a completion
of some important spaces;  put another way, some dense subspaces of
$L^1$ are significant.   The first is elementary.

\medskip

\noindent}{\bf Proposition} Let $(X,\Sigma,\mu)$ be any measure space,
and write $\eusm S$ for the space of $\mu$-simple functions on $X$.
Then

(a) whenever $f$ is a $\mu$-integrable real-valued function and
$\epsilon>0$, there is an $h\in\eusm S$ such that
$\int|f-h|\le\epsilon$;

(b) $S=\{f^{\ssbullet}:f\in\eusm S\}$ is a dense linear subspace of
$L^1=L^1(\mu)$.

\proof{{\bf (a)}(i) If $f$ is non-negative, then there is a simple
function $h$ such that $h\leae f$ and
$\int h\ge\int f-\bover12\epsilon$ (122K), in which case

\Centerline{$\int|f-h|=\int f-h
=\int f-\int h\le\Bover12\epsilon$.}

\noindent  (ii) In the general case, $f$ is expressible as a
difference $f_1-f_2$ of non-negative integrable functions.   Now there
are $h_1$, $h_2\in \eusm S$ such that
$\int|f_j-h_j|\le\bover12\epsilon$
for both $j$ and

\Centerline{$\int|f-h|\le\int|f_1-h_1|+\int|f_2-h_2|\le\epsilon$.}

\medskip

{\bf (b)}  Because $\eusm S$ is a
linear subspace of $\Bbb R^X$ included in $\eusm L^1=\eusm L^1(\mu)$,
$S$ is a linear subspace of $L^1$.   If
$u\in L^1$ and $\epsilon>0$, there are an $f\in\eusm L^1$ such that
$f^{\ssbullet}=u$ and an $h\in\eusm S$ such that
$\int|f-h|\le\epsilon$;  now $v=h^{\ssbullet}\in S$ and

\Centerline{$\|u-v\|_1=\int|f-h|\le\epsilon$.}

\noindent As $u$ and $\epsilon$ are arbitrary, $S$ is dense in $L^1$.
}

\leader{242N}{}\cmmnt{ As always, Lebesgue measure on $\BbbR^r$ and
its subsets is by far the most important example;  and in this case we
have further classes of dense subspace of $L^1$.   If you have reached
this point without yet troubling to master multi-dimensional Lebesgue
measure, just take $r=1$.   If you feel uncomfortable with general
subspace measures, take $X$ to be $\BbbR^r$ or $[0,1]\subseteq\Bbb R$
or some other particular subset which you find interesting.   The
following term will be useful.

\medskip

\noindent}{\bf Definition} If $f$ is a real- or complex-valued
function
defined on a subset of $\BbbR^r$, say that the {\bf support} of $f$ is
$\overline{\{x:x\in\dom f,\,f(x)\ne 0\}}$.

\leader{242O}{Theorem} Let $X$ be any subset of $\BbbR^r$, where
$r\ge 1$, and let $\mu$ be Lebesgue measure on $X$\cmmnt{, that is,
the subspace measure on $X$ induced by Lebesgue measure on $\BbbR^r$}.
Write $C_k$ for the space of bounded continuous functions
$f:\BbbR^r\to\Bbb R$ which have bounded support, and $S_0$ for
the space of linear combinations of functions of the form $\chi I$
where
$I\subseteq\BbbR^r$ is a bounded half-open interval.   Then

(a) whenever $f\in\eusm L^1=\eusm L^1(\mu)$ and
$\epsilon>0$, there are $g\in C_k$, $h\in S_0$ such that
$\int_X|f-g|\le\epsilon$ and $\int_X|f-h|\le\epsilon$;

(b) $\{(g\restr X)^{\ssbullet}:g\in C_k\}$ and
$\{(h\restr X)^{\ssbullet}:h\in S_0\}$ are dense
linear subspaces of $L^1=L^1(\mu)$.

\cmmnt{\medskip

\noindent{\bf Remark} Of course there is a redundant `bounded' in the
description of $C_k$;  see 242Xh.}

\proof{{\bf (a)} I argue in turn that the result is valid for each of
an
increasing number of members $f$ of $\eusm L^1=\eusm L^1(\mu)$.
Write
$\mu_r$ for Lebesgue measure on $\BbbR^r$, so
that $\mu$ is the subspace measure $(\mu_r)_X$.

\medskip

\quad{\bf (i)} Suppose first that $f=\chi I\restr X$ where
$I\subseteq\BbbR^r$ is a bounded half-open interval.
Of course $\chi I$ is already in $S_0$, so I have only to show
that it is approximated by members of $C_k$.   If $I=\emptyset$
the result is trivial;  we can take $g=0$.
Otherwise, express $I$ as $\coint{a-b,a+b}$ where
$a=(\alpha_1,\ldots,\alpha_r)$, $b=(\beta_1,\ldots,\beta_r)$ and
$\beta_j>0$ for each $j$.   Let $\delta>0$ be such that

\Centerline{$2^r\prod_{j=1}^r(\beta_j+\delta)
\le\epsilon+2^r\prod_{j=1}^r\beta_j$.}

\noindent For $\xi\in\Bbb R$ set

$$\eqalign{g_j(\xi)
&=1\text{ if }|\xi-\alpha_j|\le\beta_j,\cr
&=(\beta_j+\delta-|\xi-\alpha_j|)/\delta
  \text{ if }\beta_j\le|\xi-\alpha_j|\le\beta_j+\delta,\cr
&=0\text{ if }|\xi-\alpha_j|\ge\beta_j+\delta.\cr}$$

\def\Caption{The function $g_j$}
\picture{mt242m}{100pt}

\noindent For $x=(\xi_1,\ldots,\xi_r)\in \BbbR^r$ set

\Centerline{$g(x)=\prod_{j=1}^rg_j(\xi_j)$.}

\noindent Then $g\in C_k$ and $\chi I\le g\le\chi J$, where
$J=[a-b-\delta \tbf{1},a+b+\delta\tbf{1}]$ (writing
$\tbf{1}=(1,\ldots,1)$), so that (by the choice of $\delta$) $\mu_r
J\le\mu_r I+\epsilon$, and

$$\eqalign{\int_X|g-f|
&\le\int(\chi(J\cap X)-\chi(I\cap X))d\mu
=\mu((J\setminus I)\cap X)\cr
&\le\mu_r(J\setminus I)
=\mu_rJ-\mu_rI
\le\epsilon,\cr}$$

\noindent as required.

\medskip

\quad{\bf (ii)} Now suppose that $f=\chi(X\cap E)$ where $E\subseteq
\BbbR^r$ is a
set of finite measure.   Then there is a disjoint family
$I_0,\ldots,I_n$ of half-open intervals such that
$\mu_r(E\symmdiff\bigcup_{j\le n}I_j)\le\bover12\epsilon$.   \Prf\
There
is an open set $G\supseteq E$ such that
$\mu_r(G\setminus E)\le\bover14\epsilon$ (134Fa).   For each
$m\in\Bbb N$, let $\Cal I_m$
be the family of half-open intervals in $\BbbR^r$ of the form
$\coint{a,b}$ where $a=(2^{-m}k_1,\ldots,2^{-m}k_r)$, $k_1,\ldots,k_r$
being integers, and $b=a+2^{-m}\tbf{1}$;  then $\Cal I_m$ is a
disjoint
family.   Set $H_m=\bigcup\{I:I\in\Cal I_m,\,I\subseteq G\}$;  then
$\sequence{m}{H_m}$ is a non-decreasing family with union $G$, so that
there is an $m$ such that $\mu_r(G\setminus H_m)\le\bover14\epsilon$
and
$\mu_r(E\symmdiff H_m)\le\bover12\epsilon$.   But now $H_m$ is
expressible as a disjoint union $\bigcup_{j\le n}I_j$ where
$I_0,\ldots,I_n$ enumerate the members of $\Cal I_m$ included in
$H_m$.
(The last sentence derails if $H_m$ is empty.   But if $H_m=\emptyset$
then
we can take $n=0$, $I_0=\emptyset$.)   \Qed

Accordingly $h=\sum_{j=0}^n\chi I_j\in S_0$ and

\Centerline{$\int_X|f-h|
=\mu(X\cap(E\symmdiff\bigcup_{j\le n}I_j))\le\Bover12\epsilon$.}

\noindent As for $C_k$, (i) tells us that there is for each
$j\le n$ a $g_j\in C_k$ such that
$\int_X|g_j-\chi I_j|\le\epsilon/2(n+1)$, so that
$g=\sum_{j=0}^ng_j\in C_k$ and

\Centerline{$\int_X|f-g|\le\int_X|f-h|+\int_X|h-g|
\le\Bover{\epsilon}2+\sum_{j=0}^n\int_X|g_j-\chi I_j|\le\epsilon$.}

\medskip

\quad{\bf (iii)} If $f$ is a simple function, express $f$ as
$\sum_{k=0}^na_k\chi E_k$ where each $E_k$ is of finite measure for
$\mu$.
Each $E_k$ is expressible as $X\cap F_k$ where $\mu_rF_k=\mu E_k$
(214Ca).   By (ii), we can find $g_k\in C_k$, $h_k\in S_0$ such that

\Centerline{$|a_k|\int_X|g_k-\chi F_k|\le\Bover{\epsilon}{n+1}$,
\quad$|a_k|\int_X|h_k-\chi F_k|\le\Bover{\epsilon}{n+1}$}

\noindent for each $k$.   Set
$g=\sum_{k=0}^na_kg_k$ and $h=\sum_{k=0}^na_kh_k$;  then $g\in C_k$,
$h\in S_0$ and

\Centerline{$\int_X|f-g|\le\int_X\sum_{k=0}^n|a_k||\chi F_k-g_k|
=\sum_{k=0}^n|a_k|\int_X|\chi F_k-g_k|\le\epsilon$,}

\Centerline{$\int_X|f-h|
\le\sum_{k=0}^n|a_k|\int_X|\chi F_k-h_k|\le\epsilon$,}

\noindent as required.

\medskip

\quad{\bf (iv)} If $f$ is any integrable function on $X$, then by
242Ma
we can find a simple function $f_0$ such that
$\int|f-f_0|\le\bover12\epsilon$, and now by (iii) there are
$g\in C_k$, $h\in S_0$ such that
$\int_X|f_0-g|\le\bover12\epsilon$,
$\int_X|f_0-h|\le\bover12\epsilon$;  so that

\Centerline{$\int_X|f-g|
\le\int_X|f-f_0|+\int_X|f_0-g|\le\epsilon$,}

\Centerline{$\int_X|f-h|
\le\int_X|f-f_0|+\int_X|f_0-h|\le\epsilon$.}


\medskip

{\bf (b)(i)} We must check first that if $g\in C_k$
then $g\restr X$ is
actually $\mu$-integrable.   The point here is that if $g\in C_k$ and
$a\in\Bbb R$ then

\Centerline{$\{x:x\in X,\,g(x)>a\}$}

\noindent is the intersection of $X$ with an open subset of $\BbbR^r$,
and is therefore measured by $\mu$, because all open sets are measured
by $\mu_r$ (115G).   Next, $g$ is bounded and the set
$E=\{x:x\in X,\,g(x)\ne 0\}$ is bounded in $\BbbR^r$, therefore of
finite outer
measure for $\mu_r$ and of finite measure for $\mu$.   Thus there is
an
$M\ge 0$ such that $|g|\le M\chi E$, which is $\mu$-integrable.
Accordingly $g$ is $\mu$-integrable.

Of course $h\restr X$ is $\mu$-integrable for every $h\in S_0$
because (by the
definition of subspace measure) $\mu(I\cap X)$ is defined and finite
for
every bounded half-open interval $I$.
\medskip

\quad{\bf (ii)} Now the rest follows by just the same arguments as in
242Mb.   Because $\{g\restr X:g\in C_k\}$ and
$\{h\restr X:h\in S_0\}$ are linear subspaces of $\Bbb R^X$
included in $\eusm L^1(\mu)$, their images $C_k^{\#}$ and $S_0^{\#}$
are
linear subspaces of $L^1$.
If $u\in L^1$ and $\epsilon>0$, there are an $f\in\eusm L^1$ such that
$f^{\ssbullet}=u$, and  $g\in C_k$, $h\in S_0$ such that
$\int_X|f-g|$, $\int_X|f-h|\le\epsilon$;  now
$v=(g\restr X)^{\ssbullet}\in C_k^{\#}$
and $w=(h\restr X)^{\ssbullet}\in S_0^{\#}$ and

\Centerline{$\|u-v\|_1=\int_X|f-g|\le\epsilon$,
\quad$\|u-w\|_1=\int_X|f-h|\le\epsilon$.}

\noindent As $u$ and $\epsilon$ are arbitrary, $C_k^{\#}$ and
$S_0^{\#}$ are dense in $L^1$.
}%end of proof of 242O

\leader{242P}{Complex $L^1$}\cmmnt{ As you would, I hope, expect, we
can repeat the work above with
$\eusm L^1_{\Bbb C}$, the space of complex-valued integrable
functions,
in place of $\eusm L^1$, to construct a complex Banach space
$L^1_{\Bbb C}$.   The required changes, based on the ideas of 241J,
are
minor.

\medskip

\noindent}{\bf (a)}\dvro{ For}{ In 242Aa, it is perhaps helpful to
remark that, for} $f\in\eusm L^0_{\Bbb C}$,

\Centerline{$f\in\eusm L^1_{\Bbb C}\iff |f|\in\eusm L^1\iff
\Real(f),\,\Imag(f)\in\eusm L^1$.}

\noindent Consequently, for $u\in L^0_{\Bbb C}$,

\Centerline{$u\in L^1_{\Bbb C}\iff |u|\in L^1\iff
\Real(u),\,\Imag(u)\in
L^1$.}

\header{242Pb}{\bf (b)}\cmmnt{ To prove a complex version of 242E,
observe that if $\sequencen{f_n}$ is a sequence in $\eusm L^1_{\Bbb
C}$
such that
$\sum_{n=0}^{\infty}\int|f_n|<\infty$, then
$\sum_{n=0}^{\infty}\int|\Real(f_n)|$ and
$\sum_{n=0}^{\infty}\int|\Imag(f_n)|$ are both finite, so we may apply
242E twice and see that

\Centerline{$\int(\sum_{n=0}^{\infty}f_n)
=\int(\sum_{n=0}^{\infty}\Real(f_n))
  +\int(\sum_{n=0}^{\infty}\Imag(f_n))
=\sum_{n=0}^{\infty}\int f_n$.}

\noindent Accordingly we can prove that} $L^1_{\Bbb C}$ is complete
under $\|\,\|_1$\cmmnt{ by the argument of 242F}.

\cmmnt{\header{242Pc}{\bf (c)} Similarly, little change is needed to
adapt 242J to give a description of a
conditional expectation operator $P:L^1_{\Bbb C}(\mu)\to L^1_{\Bbb
C}(\mu\restrp\Tau)$ when $(X,\Sigma,\mu)$ is a probability space and
$\Tau$ is a $\sigma$-subalgebra of $\Sigma$.
In the formula

\Centerline{$|Pu|\le P|u|$}

\noindent of 242Je, we need to know that

\Centerline{$|Pu|=\sup_{|\zeta|=1}\Real(\zeta Pu)$}

\noindent in $L^0(\mu\restrp\Tau)$ (241Jc), while

\Centerline{$\Real(\zeta Pu)=\Real(P(\zeta u))
=P(\Real(\zeta u))\le P|u|$}

\noindent whenever $|\zeta|=1$.
}

\cmmnt{
\header{242Pd}{\bf (d)} In 242M, we need to replace $\eusm S$ by
$\eusm S_{\Bbb C}$, the space
of `complex-valued simple functions' of the form
$\sum_{k=0}^na_k\chi E_k$ where each $a_k$ is a complex number and
each $E_k$ is a measurable set
of finite measure;  then we get a dense linear subspace $S_{\Bbb
C}=\{f^{\ssbullet}:f\in\eusm S_{\Bbb C}\}$ of $L^1_{\Bbb C}$.   In
242O, we must replace $C_k$ by $C_k(\BbbR^r;\Bbb C)$, the space of
bounded continuous complex-valued functions of bounded support, and
$S_0$ by the linear span over
$\Bbb C$ of $\{\chi I:I$ is a bounded half-open interval$\}$.
}

\exercises{
\leader{242X}{Basic exercises $\pmb{>}$(a)} Let $X$ be a set, and let
$\mu$ be
counting measure on $X$.   Show that $L^1(\mu)$ can be identified with
the space $\ell^1(X)$ of absolutely summable real-valued functions on
$X$ (see 226A).   In particular, the space $\ell^1=\ell^1(\Bbb N)$ of
absolutely summable real-valued sequences is an $L^1$ space.
Write out proofs of 242F adapted to these special cases.
%242F, but put at start

\sqheader 242Xb Let $(X,\Sigma,\mu)$ be any measure space, and
$\hat\mu$ the completion of $\mu$.   Show that
$\eusm L^1(\hat\mu)=\eusm L^1(\mu)$ and $L^1(\hat\mu)=L^1(\mu)$ (cf.\
241Xb).
%242A

\spheader 242Xc Let
$\langle(X_i,\Sigma_i,\mu_i)\rangle_{i\in I}$ be a family of measure
spaces, and $(X,\Sigma,\mu)$ their direct
sum.   Show that the isomorphism between $L^0(\mu)$ and
$\prod_{i\in I}L^0(\mu_i)$ (241Xd) induces an identification between
$L^1(\mu)$ and

\Centerline{$\{u:u\in\prod_{i\in I}L^1(\mu_i),\,
  \|u\|=\sum_{i\in I}\|u(i)\|_1<\infty\}
\subseteq\prod_{i\in I}L^1(\mu_i)$.}
%242D

\spheader 242Xd Let $(X,\Sigma,\mu)$ and $(Y,\Tau,\nu)$ be measure
spaces, and $\phi:X\to Y$ an \imp\ function.   Show that
$g\mapsto g\phi:\eusm L^1(\nu)\to\eusm L^1(\mu)$ (235G) induces a
linear operator
$T:L^1(\nu)\to L^1(\mu)$ such that $\|Tv\|_1=\|v\|_1$ for every
$v\in L^1(\nu)$.
%242D

\spheader 242Xe Let $U$ be a Riesz space (definition:  241Ed).   A
{\bf Riesz norm} on $U$ is a norm $\|\,\|$ such that $\|u\|\le\|v\|$
whenever $|u|\le|v|$.    Show that if $U$ is given its norm topology 
(2A4Bb) for such a norm, then (i)
$u\mapsto|u|:U\to U$, $(u,v)\mapsto u\vee v:U\times U\to U$ are
continuous (ii) $\{u:u\ge 0\}$ is closed.
%242D

\spheader 242Xf Show that any Banach lattice must be an
Archimedean Riesz space (241Fa).
%242G

\spheader 242Xg Let $(X,\Sigma,\mu)$ be a probability space, and
$\Tau$ a $\sigma$-subalgebra of $\Sigma$, $\Upsilon$ a
$\sigma$-subalgebra of $\Tau$.   Let
$P_1:L^1(\mu)\to L^1(\mu\restrp\Tau)$,
$P_2:L_1(\mu\restrp\Tau)\to L^1(\mu\restrp\Upsilon)$ and
$P:L^1(\mu)\to L^1(\mu\restrp\Upsilon)$ be the corresponding
conditional
expectation operators.   Show that $P=P_2P_1$.
%242J

\spheader 242Xh Show that if $g:\BbbR^r\to\Bbb R$ is continuous and
has bounded support it is bounded and attains its bounds.   
\Hint{2A2F-2A2G.}
%242O

\spheader 242Xi Let $\mu$ be Lebesgue measure on $\Bbb R$.   (i) Take
$\delta>0$.   Show
that if $\phi_{\delta}(x)=\exp(-\Bover{1}{\delta^2-x^2})$ for
$|x|<\delta$, $0$ for $|x|\ge\delta$ then $\phi$ is {\bf smooth}, that
is, differentiable arbitrarily often.   (ii) Show that if
$F_{\delta}(x)=\int_{-\infty}^x\phi_{\delta}d\mu$ for $x\in\Bbb R$
then
$F_{\delta}$ is smooth.   (iii) Show that if $a<b<c<d$ in $\Bbb R$
there is a smooth function $h$ such that $\chi[b,c]\le
h\le\chi[a,d]$.   (iv) Write $\eusm D$ for the space of smooth
functions $h:\Bbb R\to\Bbb R$ such that $\{x:h(x)\ne 0\}$ is bounded.
  Show that
$\{h^{\ssbullet}:h\in\eusm D\}$ is dense in $L^1(\mu)$.
(v) Let $f$ be a real-valued function which is integrable over every
bounded subset of $\Bbb R$.   Show that $f\times h$ is integrable for
every $h\in\eusm D$, and that if $\int f\times h=0$ for every
$h\in\eusm D$ then $f=0$ a.e.   \Hint{222D.}
%242O

\spheader 242Xj Let $(X,\Sigma,\mu)$ be a probability space,
$\Tau$ a $\sigma$-subalgebra of $\Sigma$ and
$P:L^1(\mu)\to L^1(\mu\restrp\Tau)\subseteq L^1(\mu)$
the corresponding conditional expectation operator.   Show that if
$u$, $v\in L^1(\mu)$ are such that $P|u|\times P|v|\in L^1(\mu)$, then
$\int Pu\times v=\int Pu\times Pv=\int u\times Pv$.
%242L

\leader{242Y}{Further exercises (a)}
%\spheader 242Ya
Let $(X,\Sigma,\mu)$ be a measure space.   Let $A\subseteq
L^1=L^1(\mu)$
be a non-empty downwards-directed set, and
suppose that $\inf A=0$ in $L^1$.   (i) Show that
$\inf_{u\in A}\|u\|_1=0$.
\Hint{set $\gamma=\inf_{u\in A}\|u\|_1$;  find a
non-increasing sequence $\sequencen{u_n}$ in $A$ such that
$\lim_{n\to\infty}\|u_n\|_1=\gamma$;  set $v=\inf_{n\in\Bbb N}u_n$ and
show that $u\wedge v=v$ for every $u\in A$, so that $v=0$.}   (ii)
Show
that if $U$ is any open set containing $0$, there is a $u\in A$ such
that $v\in U$ whenever $0\le v\le u$.
%242D

\spheader 242Yb Let $(X,\Sigma,\mu)$ be a measure space and $Y$
any subset of $X$;  let $\mu_Y$ be the subspace measure on $Y$ and
$T:L^0(\mu)\to L^0(\mu_Y)$ the canonical map described in 241Yg.   (i)
Show that $Tu\in L^1(\mu_Y)$ and $\|Tu\|_1\le\|u\|_1$ for every
$u\in L^1(\mu)$.   (ii) Show that if $u\in L^1(\mu)$ then
$\|Tu\|_1=\|u\|_1$
iff $\int_Eu=\int_{Y\cap E}Tu$ for every $E\in\Sigma$.  (iii) Show
that $T$ is surjective and that
$\|v\|_1=\min\{\|u\|_1:u\in L^1(\mu),\,Tu=v\}$ for every
$v\in L^1(\mu_Y)$.   \Hint{214Eb.}  (See also 244Yd below.)
%242D

\spheader 242Yc Let $(X,\Sigma,\mu)$ be a measure space.   Write
$\eusm L^1_{\Sigma}$ for the space of all integrable
$\Sigma$-measurable functions
from $X$ to $\Bbb R$, and  $\eusm N$ for the subspace of
$\eusm L^1_{\Sigma}$ consisting of measurable functions which
are
zero almost everywhere.   (i) Show that $\eusm L^1_{\Sigma}$
is a
Dedekind $\sigma$-complete Riesz space.   (ii) Show that $L^1(\mu)$
can
be identified, as ordered linear space, with the quotient
$\eusm L^1_{\Sigma}/\eusm N$ as defined in 241Yb.   (iii) Show
that $\|\,\|_1$ is a seminorm on $\eusm L^1_{\Sigma}$.
(iv) Show that
$f\mapsto |f|:\eusm L^1_{\Sigma}\to\eusm L^1_{\Sigma}$
is continuous if $\eusm L^1_{\Sigma}$ is
given the topology defined from $\|\,\|_1$.   (v) Show that
$\{f:f=0$ a.e.$\}$ is closed in $\eusm L^1_{\Sigma}$, but that
$\{f:f\ge 0\}$ need not be.
%242D

\spheader 242Yd Let $(X,\Sigma,\mu)$ be a measure space, and
$\tilde\mu$ the c.l.d.\ version of $\mu$ (213E).   Show that the
inclusion $\eusm L^1(\mu)\subseteq\eusm L^1(\tilde\mu)$ induces an
isomorphism, as ordered normed linear spaces, between $L^1(\tilde\mu)$
and $L^1(\mu)$.
%242D

\spheader 242Ye Let $(X,\Sigma,\mu)$ be a measure space and
$u_0,\ldots,u_n\in L^1(\mu)$.   (i) Suppose $k_0,\ldots,k_n\in\Bbb Z$
are such that $\sum_{i=0}^nk_i=1$.   Show that
$\sum_{i=0}^n\sum_{j=0}^nk_ik_j\|u_i-u_j\|_1\le 0$.   \Hint{
$\sum_{i=0}^n\sum_{j=0}^n
\ifdim\pagewidth=390pt\penalty-100\fi
k_ik_j|\alpha_i-\alpha_j|\le 0$ for all
$\alpha_0,\ldots,\alpha_n\in\Bbb R$.}   (ii) Suppose
$\gamma_0,\ldots,\gamma_n\in\Bbb R$ are such that
$\sum_{i=0}^n\gamma_i=0$.   Show that
$\sum_{i=0}^n\sum_{j=0}^n\gamma_i\gamma_j\|u_i-u_j\|_1\le 0$.
%242D

\spheader 242Yf Let $(X,\Sigma,\mu)$ be a measure space, and
$A\subseteq L^1=L^1(\mu)$ a non-empty upwards-directed set.   Suppose
that $A$ is bounded for the norm $\|\,\|_1$.   (i) Show that there is
a
non-decreasing sequence $\sequencen{u_n}$ in $A$ such that
$\lim_{n\to\infty}\int u_n=\sup_{u\in A}\int u$, and that
$\sequencen{u_n}$ is Cauchy.   (ii) Show that $w=\sup A$ is defined in
$L^1$ and belongs to the norm-closure of $A$ in $L^1$, so that, in
particular, $\|w\|_1\le\sup_{u\in A}\|u\|_1$.
%242F

\spheader 242Yg A Riesz norm (definition:  242Xe) on a Riesz space
$U$ is {\bf order-continuous} if $\inf_{u\in A}\|u\|=0$ whenever
$A\subseteq U$
is a non-empty downwards-directed set with infimum $0$.   (Thus 242Ya
tells us that the norms $\|\,\|_1$ are all order-continuous.)
Show that in this case (i) any non-decreasing sequence in $U$
which has an upper bound in $U$ must be Cauchy (ii) if $U$ is a
Banach lattice, it is $U$ is Dedekind
complete.   ({\it Hint for (i)\/}: if $\sequencen{u_n}$ is a
non-decreasing sequence with an upper bound in $U$, let $B$ be the set
of upper bounds of $\{u_n:n\in\Bbb N\}$ and show that
$A=\{v-u_n:v\in B,\,n\in\Bbb N\}$ has infimum $0$ because $U$ is
Archimedean.)
%242Yf 242G

\spheader 242Yh Let $(X,\Sigma,\mu)$ be any measure space.
Show that $L^1(\mu)$ has the countable sup property (241Ye).
%242Yg 242Yf 242G

\spheader 242Yi More generally, show that any Riesz space
with an order-continuous Riesz norm has the countable sup property.
%242Yh 242Yg 242Yf 242G

\spheader 242Yj Let $(X,\Sigma,\mu)$ and $(Y,\Tau,\nu)$ be measure
spaces and $U\subseteq L^0(\mu)$ a linear subspace.   Let
$T:U\to L^0(\nu)$ be a linear operator such that $Tu\ge 0$ in
$L^0(\nu)$
whenever $u\in U$ and $u\ge 0$ in $L^0(\mu)$.   Suppose that $w\in U$ is
such that $w\ge 0$ and $Tw=(\chi Y)^{\ssbullet}$.   Show that whenever
$\phi:\Bbb R\to\Bbb R$ is a convex function and $u\in L^0(\mu)$ is
such that $w\times u$ and $w\times\bar\phi(u)\in U$, defining
$\bar\phi:L^0(\mu)\to L^0(\mu)$ as in 241I, then
$\bar\phi T(w\times u)\le T(w\times\bar\phi u)$.   Explain how this
result may be regarded
as a common generalization of Jensen's inequality, as stated in 233I,
and 242K above.   See also 244M below.
%242K

\spheader 242Yk(i) A function $\phi:\Bbb C\to\Bbb R$ is {\bf convex}
if
$\phi(ab+(1-a)c)\le a\phi(b)+(1-a)\phi(c)$ for all $b$, $c\in\Bbb C$
and
$a\in [0,1]$.   (ii) Show that such a function must be bounded on any
bounded subset of $\Bbb C$.   (iii) If $\phi:\Bbb C\to\Bbb R$ is
convex
and $c\in\Bbb C$, show that there is a $b\in\Bbb C$ such that
$\phi(x)\ge\phi(c)+\Real(b(x-c))$ for every $x\in\Bbb C$.   (iv) If
$\langle b_c\rangle_{c\in\Bbb C}$ is such that
$\phi(x)\ge\phi_c(x)=\phi(c)+\Real(b_c(x-c))$ for all $x$, $c\in\Bbb
C$,
show that $\{b_c:c\in I\}$ is bounded for any bounded
$I\subseteq\Bbb C$.
(v) Show that if $D\subseteq\Bbb C$ is any dense set,
$\phi(x)=\sup_{c\in D}\phi_c(x)$ for every $x\in\Bbb C$.
%242P

\spheader 242Yl Let $(X,\Sigma,\mu)$ be a probability space and
$\Tau$ a
$\sigma$-subalgebra of $\Sigma$.   Let
$P:L^1_{\Bbb C}(\mu)\to L^1_{\Bbb C}(\mu\restrp\Tau)$ be the
conditional
expectation operator.   Show that
if $\phi:\Bbb C\to\Bbb R$ is any convex function, and we define
$\bar\phi(f^{\ssbullet})=(\phi f)^{\ssbullet}$ for every
$f\in\eusm L^0_{\Bbb C}(\mu)$, then $\bar\phi(Pu)\le P(\bar\phi(u))$
whenever $u\in L^1_{\Bbb C}(\mu)$ is such that
$\bar\phi(u)\in L^1(\mu)$.
%242P
}%end of exercises

\endnotes{
\Notesheader{242} Of course $L^1$-spaces compose one of the most
important classes of Riesz space, and accordingly their properties
have
great prominence in the general theory;  242Xe, 242Xf, 242Ya and
242Yf-242Yi %242Yf 242Yg 242Yh 242Yi
outline some of the
interrelations between these properties.   I will return to these
questions in Chapter 35 in the next volume.   I have mentioned in
passing (242Dd) the additivity of the norm of $L^1$ on the positive
elements. This elementary fact actually characterizes $L^1$ spaces
among Banach lattices;  see 369E in the next volume.

Just as $L^0(\mu)$ can be regarded as a quotient of a linear space
$\eusm L^0_{\Sigma}$, so can $L^1(\mu)$ be regarded as a
quotient
of a linear space $\eusm L^1_{\Sigma}$ (242Yc).   I have
discussed this question in the notes to \S241;  all I try to do here
is to be consistent.

We now have a language in which we can speak of `the' conditional
expectation of a function $f$, the equivalence class in
$L^1(\mu\restrp\Tau)$ consisting precisely of all the conditional
expections of $f$ on $\Tau$.   If we think of $L^1(\mu\restrp\Tau)$ as
identified with its image in $L^1(\mu)$, then the conditional
expectation
operator $P:L^1(\mu)\to L^1(\mu\restrp\Tau)$ becomes a projection
(242Jh).   We therefore have re-statements of 233J-233K, as in 242K,
242L and 242Yj.

I give 242O in a fairly general form;  but its importance already
appears if we take $X$ to be $[0,1]$ with one-dimensional Lebesgue
measure.   In this case, we have a natural norm on
$C([0,1])$, the space of all continuous
real-valued functions on $[0,1]$, given by setting

\Centerline{$\|f\|_1=\int_0^1|f(x)|dx$}

\noindent for every $f\in C([0,1])$.   The integral here can, of course,
be taken to be the Riemann integral;  we do not need the Lebesgue theory
to show that $\|\,\|_1$ is a norm on $C([0,1])$.   It is easy to check
that $C([0,1])$ is not complete for this norm (if we set
$f_n(x)=\min(1,2^nx^n)$ for $x\in[0,1]$, then $\sequencen{f_n}$ is a
$\|\,\|_1$-Cauchy sequence with no $\|\,\|_1$-limit in $C([0,1])$).
We can use the abstract theory of normed spaces to construct a completion
of $C([0,1])$;  but it is much more satisfactory if this completion can
be given a relatively concrete form, and this is what the identification
of $L^1$ with the completion of $C([0,1])$ can do.   (Note that the
remark that $\|\,\|_1$ is a norm on $C([0,1])$, that is, that
$\|f\|_1\ne 0$ for every non-zero $f\in C([0,1])$, means just that the
map $f\mapsto f^{\ssbullet}:C([0,1])\to L^1$ is injective, so that
$C([0,1])$ can be
identified, as ordered normed space, with its image in $L^1$.)    It
would be even better if we could find a realization of the completion of
$C([0,1])$ as a space of functions on some set $Z$, rather than as a
space of equivalence classes of functions on $[0,1]$.   Unfortunately
this is not practical;  such realizations do exist, but necessarily
involve either a thoroughly unfamiliar base set $Z$, or an intolerably
arbitrary embedding map from $C([0,1])$ into $\BbbR^Z$.

You can get an idea of the obstacle to realizing the completion of
$C([0,1])$ as a space of functions on $[0,1]$ itself by considering
$f_n(x)=\bover1nx^n$ for $n\ge 1$.   An easy calculation shows that
$\sum_{n=1}^{\infty}\|f_n\|_1<\infty$, so that $\sum_{n=1}^{\infty}f_n$
must exist in the completion of $C([0,1])$;  but there is no natural
value to assign to it at the point $1$.   Adaptations of this idea can
give rise to indefinitely complicated phenomena -- indeed, 242O shows
that every integrable function is associated with some appropriate
sequence from $C([0,1])$.   In \S245 I shall have more to say about
what $\|\,\|_1$-convergent sequences look like.

From the point of view of measure theory, narrowly conceived, most of
the interesting ideas appear most clearly with real functions and real
linearspaces.   But some of the most important applications of measure
theory -- important not only as mathematics in general, but also for
the measure-theoretic questions they inspire -- deal with complex
functions
and complex linear spaces.   I therefore continue to offer sketches of
the complex theory, as in 242P.   I note that at irregular intervals
we need
ideas not already spelt out in the real theory, as in 242Pb and 242Yl.
}%end of notes


\discrpage

