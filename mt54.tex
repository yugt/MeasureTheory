\frfilename{mt54.tex} 
\versiondate{23.10.14} 
\copyrightdate{2005} 
 
\def\chaptername{Real-valued-measurable cardinals} 
 
\def\pssqa{power set $\sigma$-quotient algebra} 
 
\newchapter{54} 
 
Of the many questions in measure theory which involve non-trivial set 
theory, perhaps the first to have been  
recognised as fundamental is what I call the `Banach-Ulam problem':   
is there a non-trivial measure space in which every set is measurable? 
In various forms, this question has arisen repeatedly in earlier volumes 
of this treatise (232Hc, 363S, 438A).   The time has now come for an 
account of the developments of the last fifty years. 
 
The measure theory of this chapter will begin in \S543;  the first two 
sections deal with generalizations to wider contexts.   If $\nu$ is a 
probability measure with domain $\Cal PX$, its null ideal is 
$\omega_1$-additive and $\omega_1$-saturated 
in $\Cal PX$.   In \S541 I look at ideals $\Cal I\normalsubgroup\Cal PX$ 
such that $\Cal I$ is simultaneously
$\kappa$-additive and $\kappa$-saturated for some 
$\kappa$;  this is already enough to lead us to  
a version of the Keisler-Tarski theorem on 
normal ideals (541J), a great strengthening of Ulam's theorem on 
inaccessibility of real-valued-measurable cardinals (541Lc), a form of 
Ulam's dichotomy (541P), and some very striking infinitary combinatorics 
(541Q-541S).   In \S542 I specialize to the case $\kappa=\omega_1$, still 
without calling on the special properties of null ideals, with more 
combinatorics (542E, 542I). 
 
I have said many times in the course of this treatise that almost the first 
thing to ask about any measure is, what does its measure algebra look like? 
For an atomless probability measure with domain $\Cal PX$, the Gitik-Shelah 
theorem (543E-543F) gives a great deal of information, associated with a
tantalizing problem (543Z).   \S544 is devoted to the measure-theoretic 
consequences of assuming that there is some \am\ cardinal, with results on 
repeated integration (544C, 544I, 544J),  
the null ideal of a normal witnessing 
probability (544E-544F) and regressive functions (544M). 
 
I do not discuss consistency questions in this chapter (I will touch on 
some of them in Chapter 55).    
The ideas of \S\S541-544 would be in danger of becoming irrelevant if it 
turned out that there can be no \2vm\ cardinal.   I have no real qualms 
about this.   One of my reasons for 
confidence is the fact that very much stronger assumptions have been 
investigated without any hint of catastrophe.   Two of these, the `product 
measure extension axiom' and the `normal measure axiom' are mentioned in 
\S545. 
 
One way of looking at the Gitik-Shelah theorem is to say that if $X$ is a 
set and $\Cal I$ is a proper $\sigma$-ideal of subsets of $X$, then  
$\Cal PX/\Cal I$ cannot be an atomless measurable algebra of small Maharam 
type.   We can ask whether there are further theorems of this type provable 
in ZFC.   Two such results are in \S546:  the `Gitik-Shelah theorem for 
category' (546G, 546I), showing that $\Cal PX/\Cal I$ cannot be isomorphic 
to $\RO(\Bbb R)$, and 546P-546Q, 
showing that some algebras of a type considered in \S527 also cannot appear 
as \pssqa{s}.   Remarkably, this leads us to a striking fact about disjoint 
refinements of sequences of sets (547F). 
 
\discrpage 
 
