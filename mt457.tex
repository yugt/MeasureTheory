\frfilename{mt457.tex}
\versiondate{18.1.13}
\copyrightdate{2010}

\def\chaptername{Perfect measures, disintegrations and processes}
\def\sectionname{Simultaneous extension of measures}

\newsection{457}

The questions addressed in \S\S451, 454 and 455 can all be regarded as
special cases of a general class of problems:  given a set $X$ and a
family $\familyiI{\nu_i}$ of (probability) measures on $X$, when can we
expect to find a measure on $X$ extending every $\nu_i$?   An
alternative formulation, superficially more general, is to ask:  given a
set $X$, a family $\familyiI{(Y_i,\Tau_i,\nu_i)}$ of probability spaces,
and functions $\phi_i:X\to Y_i$ for each $i$, when can we find a measure
on $X$ for which every $\phi_i$ is \imp?   Even the simplest non-trivial
case, when $X=\prod_{i\in I}Y_i$ and every $\phi_i$ is the coordinate
map, demands a significant construction (the product measures of Chapter
25).   In this section I bring together a handful of important further
cases which are accessible by the methods of this chapter.   I begin
with a discussion of extensions of finitely additive measures
(457A-457D), which are much easier, before considering the problems
associated with countably additive measures (457E-457G), with examples
(457H-457J).   In 457K-457M %457L
I look at a pair of optimisation problems.

\leader{457A}{}\cmmnt{ It is helpful to start with a widely applicable
result on common extensions of finitely additive measures.

\medskip

\noindent}{\bf Lemma} Let $\frak A$ be a Boolean algebra and
$\familyiI{\frak B_i}$ a non-empty family of subalgebras of $\frak A$.
For each $i\in I$, we may identify $L^{\infty}(\frak B_i)$ with the
closed linear subspace of $L^{\infty}(\frak A)$ generated by
$\{\chi b:b\in\frak B_i\}$\cmmnt{ (363Ga)}.
Suppose that for each $i\in I$ we are given a finitely additive
functional $\nu_i:\frak B_i\to[0,1]$ such that $\nu_i1=1$;  write
$\dashint\ldots d\nu_i$ for the
corresponding positive linear functional on
$L^{\infty}(\frak B_i)$\cmmnt{ (363L)}.   Then the following
are equiveridical:

\quad(i) there is an additive functional $\mu:\frak A\to[0,1]$ extending
every $\nu_i$;

\quad(ii) whenever $i_0,\ldots,i_n\in I$, $a_k\in\frak B_{i_k}$ for
$k\le n$, and $\sum_{k=0}^n\chi a_k\ge m\chi 1$ in $S(\frak A)$, where
$m\in\Bbb N$, then $\sum_{k=0}^n\nu_{i_k}a_k\ge m$;

\quad(iii) whenever $i_0,\ldots,i_n\in I$, $a_k\in\frak B_{i_k}$ for
$k\le n$, and $\sum_{k=0}^n\chi a_k\le m\chi 1$, where $m\in\Bbb N$,
then $\sum_{k=0}^n\nu_{i_k}a_k\le m$;

\ifdim\pagewidth>467pt\fontdimen3\tenrm=2pt\fi
\ifdim\pagewidth>467pt\fontdimen4\tenrm=1.67pt\fi
\quad(iv)\dvArevised{2013} whenever $i_0,\ldots,i_n\in I$ are distinct,
$u_k\in L^{\infty}(\frak B_{i_k})$ for every $k\le n$, and
$\sum_{k=0}^nu_k\ge\chi 1$, then
$\sum_{i=0}^n\dashint u_kd\nu_{i_k}\ge 1$;
\fontdimen3\tenrm=1.67pt
\fontdimen4\tenrm=1.11pt

\quad(v)\dvArevised{2013} whenever $i_0,\ldots,i_n\in I$ are distinct,
$u_k\in L^{\infty}(\frak B_{i_k})$ for every $k\le n$, and
$\sum_{k=0}^nu_k\le\chi 1$, then
$\sum_{i=0}^n\dashint u_kd\nu_{i_k}\le 1$.

%maybe we could use versions with non-negative u_k?
%or with i_k declared distinct?

\proof{{\bf (a)} It is elementary to check that if (i) is true then
(ii)-(v) are all true, simply because we have a positive linear
functional $\dashint\,d\mu$ extending all the functionals
$\dashint\,d\nu_i$.

\wheader{457A}{6}{2}{2}{24pt}

{\bf (b)(ii)$\Rightarrow$(iii)} Given that $a_k\in\frak B_{i_k}$ and
$\sum_{k=0}^n\chi a_k\le m\chi 1$, then

\Centerline{$\sum_{k=0}^n\chi(1\Bsetminus a_k)
=(n+1)\chi 1-\sum_{k=0}^n\chi a_k
\ge(n+1-m)\chi 1$,}

\noindent so

\Centerline{$\sum_{k=0}^n\nu_{i_k}a_k
=n+1-\sum_{k=0}^n\nu_{i_k}(1\Bsetminus a_k)
\le n+1-(n+1-m)=m$,}

\noindent as required by (iii).

\medskip

{\bf (c)(iii)$\Rightarrow$(i)} Assume (iii).   Set
$\psi a=\sup\{\nu_ia:i\in I,\,a\in\frak B_i\}$ for $a\in\frak A$
(interpreting $\sup\emptyset$ as $0$, as usual in such contexts).   Then
$\psi$ satisfies the condition (ii) of 391F.
\Prf\Quer\ Otherwise,
there is a finite indexed family $\family{k}{K}{a_k}$ in
$\frak A$ such that $\inf_{k\in J}a_k=0$ whenever $J\subseteq K$ and
$\#(J)\ge\sum_{k\in K}\psi a_i$.   The general hypothesis of the lemma
implies that $\frak A\ne\{0\}$, so $\inf\emptyset=1\ne 0$ and $K$ is
non-empty.   Taking $K$ to be of minimal size, we get
an example in which $\psi a_k>0$ for every $k\in K$.
Set
$m=\|\sum_{k\in K}\chi a_k\|_{\infty}$;  then $m\in\Bbb N$ and
$m<\sum_{k\in K}\psi a_k$, so we can find for each $k\in K$ an $i_k\in I$
such that $a_k\in\frak B_{i_k}$ and $m<\sum_{k\in K}\nu_{i_k}a_k$.   But
this contradicts our hypothesis (iii).\ \Bang\Qed

By 391F, there is
a non-negative finitely additive functional $\mu$ such that $\mu 1=1$
and $\mu a\ge\psi a$ for every $a\in\frak A$, that is, $\mu b\ge\nu_ib$
whenever $i\in I$ and $b\in\frak B_i$.   But observe now
that, because $\mu 1=\nu_i1$ and
$\mu(1\Bsetminus b)\ge\nu_i(1\Bsetminus b)$, we actually have
$\mu b=\nu_ib$ for every $b\in\frak B_i$, so that $\mu$ extends $\nu_i$,
for every $i\in I$.

\medskip

{\bf (d)(iv)$\Rightarrow$(ii)} Suppose that (iv) is true, and that
$i_0,\ldots,i_n\in I$, $a_k\in\frak B_{i_k}$ for
$k\le n$, and $\sum_{k=0}^n\chi a_k\ge m\chi 1$ in $S(\frak A)$, where
$m\in\Bbb N$.   If $m=0$ then of course $\sum_{k=0}^n\nu_{i_k}a_k\ge m$.
Otherwise, set $J=\{i_k:k\le n\}$ and enumerate $J$ as
$\langle j_l\rangle_{l\le r}$.   For $l\le r$ set 
$u_l=\bover1m\sum_{k\le n,i_k=j_l}\chi a_k$.   Then 
$u_l\in S(\frak B_{j_l})$ for each $l$, and

\Centerline{$\sum_{l=0}^ru_l
=\Bover1m\sum_{l=0}^r\sum_{k\le n,i_k=j_l}\chi a_k
=\Bover1m\sum_{k=0}^n\chi a_k\ge\chi 1$.}

\noindent As $j_0,\ldots,j_l$ are distinct, 

\Centerline{$\sum_{l=0}^r\dashint u_ld\nu_{j_l}
=\Bover1m\sum_{k=0}^n\nu_{i_k}a_k\ge 1$.}

\noindent So (ii) is true.

\medskip

{\bf (e)(v)$\Rightarrow$(iii)} Use the same argument as in (d) above.
}%end of proof of 457A

\leader{457B}{Corollary} Let $X$ be a set and $\familyiI{Y_i}$ a family
of sets.   Suppose that for each $i\in I$ we have an algebra $\Cal E_i$
of subsets of $Y_i$, an additive functional $\nu_i:\Cal E_i\to[0,1]$
such that $\nu_iY_i=1$, and a function $f_i:X\to Y_i$.   Then the
following are equiveridical:

\quad(i) there is an additive functional $\mu:\Cal PX\to[0,1]$ such that
$\mu f_i^{-1}[E]=\nu_iE$ whenever $i\in I$ and $E\in\Cal E_i$;

\quad(ii) whenever $i_0,\ldots,i_n\in I$ and
$E_k\in\Cal E_{i_k}$ for $k\le n$, then there is an $x\in X$ such that
$\sum_{k=0}^n\nu_{i_k}E_k\le\#(\{k:k\le n,\,f_{i_k}(x)\in E_k\})$.

\proof{{\bf (i)$\Rightarrow$(ii)} is elementary;  if
$m=\lceil\sum_{k=0}^n\nu_{i_k}E_k\rceil-1$, then
$\sum_{k=0}^n\mu f_{i_k}^{-1}[E_k]>m\mu X$, so
$\sum_{k=0}^n\chi f_{i_k}^{-1}E_k\not\le m\chi X$, that is, there is an
$x\in X$ such that

\Centerline{$\#(\{k:f_{i_k}(x)\in E_k\})
=\sum_{k=0}^n(\chi f_{i_k}^{-1}[E_k])(x)
\ge m+1\ge\sum_{k=0}^n\nu_{i_k}E_k$.}

\medskip

{\bf (ii)$\Rightarrow$(i)} Now suppose that (ii) is true.   For $i\in I$
set $\frak B_i=\{f_i^{-1}[E]:E\in\Cal E_i\}$.   Note that if
$E\in\Cal E_i$ and $\nu_iE>0$, then (applying (ii) with $n=0$, $i_0=i$
and $E_0=E$) $f_i^{-1}[E]$ cannot be empty;  accordingly we have an
additive functional $\nuprime_i:\frak B_i\to[0,1]$ defined by setting
$\nuprime_if^{-1}[E]=\nu_iE$ for every $E\in\Cal E_i$, and
$\nuprime_iX=1$.
If $i_0,\ldots,i_n\in I$,
$H_0\in\frak B_{i_0},\ldots,H_n\in\frak B_{i_n}$ and $m\in\Bbb N$ are
such that $\sum_{k=0}^n\chi H_k\le m\chi X$, express each $H_k$ as
$f_{i_k}^{-1}[E_k]$, where $E_k\in\Cal E_{i_k}$;  then there is an
$x\in X$ such that

\Centerline{$\sum_{k=0}^n\nuprime_{i_k}H_k=\sum_{k=0}^n\nu_{i_k}E_k
\le\#(\{k:f_k(x)\in E_k\})=\sum_{k=0}^m\chi H_k(x)\le m$.}

\noindent But this means that the condition of 457A(iii) is satisfied,
with $\frak A=\Cal PX$, so 457A(i) and (i) here are also true.
}%end of proof of 457B

\leader{457C}{Corollary} (a) Let $\frak A$ be a Boolean algebra and
$\frak B_1$,
$\frak B_2$ two subalgebras of $\frak A$ with finitely additive
functionals $\nu_i:\frak B_i\to[0,1]$ such that $\nu_11=\nu_21=1$.
Then the following are equiveridical:

\quad(i) there is an additive functional
$\mu:\frak A\to[0,1]$ extending both the $\nu_i$;

\quad(ii) whenever $b_1\in\frak B_1$, $b_2\in\frak B_2$ and
$b_1\Bcup b_2=1$, then $\nu_1b_1+\nu_2b_2\ge 1$;

\quad(iii) whenever $b_1\in\frak B_1$, $b_2\in\frak B_2$ and
$b_1\Bcap b_2=0$, then $\nu_1b_1+\nu_2b_2\le 1$.

(b) Let $X$, $Y_1$, $Y_2$ be sets, and for $i\in\{1,2\}$ let $\Cal E_i$
be an algebra of subsets of $Y_i$, $\nu_i:\Cal E_i\to[0,1]$ an additive
functional such that $\nu_iY_i=1$, and $f_i:X\to Y_i$ a function.   Then
the following are equiveridical:

\quad(i) there is an additive functional
$\mu:\Cal PX\to[0,1]$ such that $\mu f_i^{-1}[E]=\nu_iE$ whenever
$i\in\{1,2\}$ and $E\in\Cal E_i$;

\quad(ii) $f_1^{-1}[E_1]\cap f_2^{-1}[E_2]\ne\emptyset$ whenever
$E_1\in\Cal E_1$, $E_2\in\Cal E_2$ and $\nu_1E_1+\nu_2E_2>1$;

\quad(iii) $\nu_1E_1\le\nu_2E_2$ whenever $E_1\in\Cal E_1$,
$E_2\in\Cal E_2$ and $f_1^{-1}[E_1]\subseteq f_2^{-1}[E_2]$.

\proof{{\bf (a)(i)$\Rightarrow$(iii)} is elementary (and is a special
case of 457A(i)$\Rightarrow$457A(iii)).

\medskip

\quad{\bf (iii)$\Rightarrow$(ii)} If (iii) is true, and
$b_1\in\frak B_1$, $b_2\in\frak B_2$ are such that $b_1\Bcup b_2=1$,
then $(1\Bsetminus b_1)\Bcap(1\Bsetminus b_2)=0$, so

\Centerline{$\nu_1b_1+\nu_2b_2
=2-\nu_1(1\Bsetminus b_1)-\nu_2(1\Bsetminus b_2)\ge 1$.}

\medskip

\quad{\bf (ii)$\Rightarrow$(i)} The point is that (ii) here implies (ii)
of 457A.   \Prf\
Suppose that $i_0,\ldots,i_n\in\{1,2\}$, $a_k\in\frak B_{i_k}$ for
$k\le n$ and $\sum_{k=0}^n\chi a_k\ge m\chi 1$ in $S(\frak A)$, where
$m\in\Bbb N$.   Set $K_j=\{k:k\le n,\,i_k=j\}$ for each $j$,
$u=\sum_{k\in K_1}\chi a_k\in S(\frak B_1)$,
$v=\sum_{k\in K_2}\chi a_k\in S(\frak B_2)$.   Then we can express $u$
as $\sum_{j=0}^{m_1}\chi c_j$ where $c_j\in\frak B_1$ for each
$j\le m_1$ and $c_0\Bsupseteq c_1\Bsupseteq\ldots\Bsupseteq c_{m_1}$
(see the proof of 361Ec).   Taking $c_j=0$ for $m_1<j\le m$ if
necessary, we may suppose that $m_1\ge m$.   Similarly,
$v=\sum_{j=0}^{m_2}\chi d_j$ where $m_2\ge m$, $d_j\in\frak B_2$ for
each $j\le m_2$ and $d_0\Bsupseteq\ldots\Bsupseteq d_{m_2}$.

For $j<m$, set $b_j=1\Bsetminus(c_j\Bcup d_{m-j-1})$.
Then, because $b_j\Bcap c_j=0$,

\Centerline{$u\times\chi b_j
=\sum_{r=0}^{m_1}\chi(c_r\Bcap b_j)
=\sum_{r=0}^{j-1}\chi(c_r\Bcap b_j)
\le j\chi b_j$,}

\noindent and similarly $v\times\chi b_j\le(m-j-1)\chi b_j$, so

\Centerline{$m\chi b_j\le(u+v)\times\chi b_j
=u\times\chi b_j+v\times\chi b_j\le(m-1)\chi b_j$,}

\noindent and $b_j$ must be $0$.

Thus $c_j\Bcup d_{m-j-1}=1$ for every $j<m$.   But this means that
$\nu_1c_j+\nu_2d_{m-j-1}\ge 1$ for every $j<m$, so that

$$\eqalign{\sum_{k=0}^n\nu_{i_k}a_k
&=\sum_{k\in K_1}\nu_1a_k+\sum_{k\in K_2}\nu_2a_k
=\dashint u\,d\nu_1+\dashint v\,d\nu_2\cr
&=\sum_{j=0}^{m_1}\nu_1c_j+\sum_{j=0}^{m_2}\nu_2d_j
\ge\sum_{j=0}^{m-1}\nu_1c_j+\nu_2d_{m-1-j}
\ge m,\cr}$$

\noindent as required.\ \Qed

Because 457A(ii) implies 457A(i), we have the result.

\medskip

{\bf (b)} We can convert (i) and (ii) here into (a-i) and (a-iii) just
above by the same translation as in 457B.   So (i) and (ii) are
equiveridical.   As for (iii), this corresponds exactly to replacing
$E_2$ by $Y_2\setminus E_2$ in (ii).
}%end of proof of 457C

\leader{*457D}{}\cmmnt{ The proof of 457A is based, at some remove,
on the Hahn-Banach theorem, as applied in the proof of 391E-391F.   An
alternative proof uses the max-flow min-cut theorem of graph theory.
To show the power of this method I apply it to an elaboration of 457C,
as follows.

\woddheader{457D}{4}{2}{2}{72pt}

\noindent}{\bf Proposition}\cmmnt{ ({\smc Strassen 65})} Let $\frak A$
be a Boolean algebra and $\frak B_1$, $\frak B_2$ two subalgebras of
$\frak A$.   Suppose that $\nu_i:\frak B_i\to[0,1]$ are finitely
additive functionals such that $\nu_11=\nu_21=1$, and
$\theta:\frak A\to\coint{0,\infty}$ another additive functional.
Then the following are equiveridical:

(i) there is an additive functional $\mu:\frak A\to\coint{0,\infty}$
extending both the $\nu_i$, and such that $\mu a\le\theta a$ for every
$a\in\frak A$;

(ii) $\nu_1b_1+\nu_2b_2\le 1+\theta(b_1\Bcap b_2)$ whenever
$b_1\in\frak B_1$ and $b_2\in\frak B_2$.

\proof{{\bf (a)} As usual in this context, (i)$\Rightarrow$(ii) is
elementary;  if $\mu\le\theta$ extends both $\nu_j$, and $b_j\in\frak
B_j$ for both $j$, then

\Centerline{$\nu_1b_1+\nu_2b_2
=\mu b_1+\mu b_2
=\mu(b_1\Bcup b_2)+\mu(b_1\Bcap b_2)
\le 1+\theta(b_1\Bcap b_2)$.}

\medskip

{\bf (b)} For the reverse implication, suppose to begin with (down to
the end of (d) below) that $\frak A$ is finite.   Let $I$, $J$ and $K$
be the sets of atoms of $\frak B_1$, $\frak B_2$ and $\frak A$
respectively.   Consider the transportation network $(V,E,\gamma)$ where

\Centerline{$V=\{(0,0)\}\cup\{(b,1):b\in I\}\cup\{(d,2):d\in
K\}\cup\{(c,3):c\in J\}\cup\{(1,4)\},$}

\Centerline{$E=\{e^0_b:b\in I\}\cup\{e^1_d:d\in K\}\cup\{e^2_d:d\in
K\}\cup\{e^3_c:c\in J\}$,}

\noindent where

\inset{for $b\in I$, $e^0_b$ runs from $(0,0)$ to $(b,1)$,

for $d\in K$, $e^1_d$ runs from $(b,1)$ to $(d,2)$, where $b$ is the
member of $I$ including $d$,

for $d\in K$, $e^2_d$ runs from $(d,2)$ to $(c,3)$, where $c$ is the
member of $J$ including $d$,

for $c\in J$, $e^3_c$ runs from $(c,3)$ to $(1,4)$.}

\noindent Define the capacity $\gamma(e)$ of each link by setting

\Centerline{$\gamma(e^0_b)=\nu_1b$ for $b\in I$,}

\Centerline{$\gamma(e^1_d)=\gamma(e^2_d)=\theta d$ for $d\in K$,}

\Centerline{$\gamma(e^3_c)=\nu_2c$ for $c\in J$.}

By the max-flow min-cut theorem (4A4N), there are a flow $\phi$ and a
cut $X$ of the same value;  that is, we have a function
$\phi:E\to\coint{0,\infty}$ and a set $X\subseteq E$ such that

\Centerline{$\sum_{e\text{ starts from }v}\phi(e)
=\sum_{e\text{ ends at }v}\phi(e)$}

\noindent for every $v\in V\setminus\{(0,0),(1,4)\}$,

\Centerline{$\phi(e)\le\gamma(e)$}

\noindent for every $e\in E$,

\Centerline{$\sum_{e\text{ starts from }(0,0)}\phi e
=\sum_{e\text{ ends at }(1,4)}\phi e
=\sum_{e\in X}\gamma(e)$,}

\noindent and there is no path from $(0,0)$ to $(1,4)$ using only links
in $E\setminus X$.

Now, for any $d\in K$, there is exactly one link $e^1_d$ ending at $d$
and exactly one link $e^2_d$ starting from $d$.   So
$\phi(e^1_d)=\phi(e^2_d)$, and we may define an additive functional
$\mu$ on $\frak A$ by setting

\Centerline{$\mu a=\sum_{d\in K,d\Bsubseteq a}\phi(e^1_d)
=\sum_{d\in K,d\Bsubseteq a}\phi(e^2_d)$}

\noindent for every $a\in\frak A$.

\medskip

{\bf (c)(i)} $\mu b\le\nu_1b$ for every $b\in\frak B_1$.   \Prf\ Because
$I$ is the set of atoms of the finite Boolean algebra $\frak B_1$, it is
enough to show that $\mu b\le\nu_1b$ for every $b\in I$.   Now, for such
$b$,

$$\eqalign{\mu b
&=\sum_{d\in K,d\Bsubseteq b}\phi(e^1_d)
=\sum_{e\text{ starts from }(b,1)}\phi(e)\cr
&=\sum_{e\text{ ends at }(b,1)}\phi(e)
=\phi(e^0_b)
\le\gamma(e^0_b)
=\nu_1b,\cr}$$

\noindent because the only link ending at $(b,1)$ is $e^0_b$.\ \Qed

\medskip

\quad{\bf (ii)} Similarly, because the only link starting at $(c,3)$ has
capacity $\nu_2c$, $\mu c\le\nu_2c$ for every $c\in J$.   But this means
that $\mu c\le\nu_2c$ for every $c\in\frak B_2$.

\medskip

\quad{\bf (iii)} In third place, because

\Centerline{$\mu d=\phi(e^1_d)\le\gamma(e^1_d)
=\theta d$}

\noindent for every $d\in K$, $\mu a\le\theta a$ for every
$a\in\frak A$.

\medskip

{\bf (d)} (The key.)  $\mu 1\ge 1$.  \Prf\ We have

$$\eqalign{\mu 1
&=\sum_{d\in K}\mu d
=\sum_{d\in K}\phi(e^1_d)\cr
&=\sum_{b\in I}\sum_{d\in K,d\Bsubseteq b}\phi(e^1_d)
=\sum_{b\in I}\sum_{e\text{ starts from }(b,1)}\phi(e)\cr
&=\sum_{b\in I}\sum_{e\text{ ends at }(b,1)}\phi(e)
=\sum_{e\text{ starts from }(0,0)}\phi(e)
=\sum_{e\in X}\gamma(e).\cr}$$

\noindent Set

$$\eqalign{b^*&=\sup\{b:b\in I,\,e^0_b\in X\}\in\frak B_1,\cr
a_1^*&=\sup\{d:d\in K,\,e^1_d\in X\},\cr
a_2^*&=\sup\{d:d\in K,\,e^2_d\in X\},\cr
c^*&=\sup\{c:c\in J,\,e^3_c\in X\}\in\frak B_2.\cr}$$

\noindent For any $d\in K$, we have a four-link path
$e^0_b,e^1_d,e^2_d,e^3_c$ from $(0,0)$ to $(1,4)$, where $b\in I$,
$c\in J$ are the atoms of $\frak B_1$, $\frak B_2$ including $d$.   At
least one of the links in this path must belong to $X$, so that $d$ is
included in $b^*\Bcup a_1^*\Bcup a_2^*\Bcup c^*$.   Thus, writing
$a=(1\Bsetminus b^*)\Bcap(1\Bsetminus c^*)$,
$a\Bsubseteq a_1^*\Bcup a_2^*$ and
$\theta a\le\theta a_1^*+\theta a_2^*$.
But this means that

$$\eqalignno{\mu 1
&=\sum_{e\in X}\gamma(e)\cr
&=\sum_{b\in I,e^0_b\in X}\gamma(e^0_b)
  +\sum_{d\in K,e^1_d\in X}\gamma(e^1_d)
  +\sum_{d\in K,e^2_d\in X}\gamma(e^2_d)
  +\sum_{c\in J,e^3_c\in X}\gamma(e^3_c)\cr
&=\sum_{b\in I,e^0_b\in X}\nu_1b
  +\sum_{d\in K,e^1_d\in X}\theta d
  +\sum_{d\in K,e^2_d\in X}\theta d
  +\sum_{c\in J,e^3_c\in X}\nu_2c\cr
&=\nu_1b^*+\theta a_1^*+\theta a_2^*+\nu_2c^*\cr
\displaycause{remember that $\theta$ is additive}
&\ge\nu_1b^*+\theta((1\Bsetminus b^*)\Bcap(1\Bsetminus c^*))+\nu_2c^*
\ge\nu_1b^*+\nu_1(1\Bsetminus b^*)+\nu_2(1\Bsetminus c^*)-1+\nu_2c^*\cr
\displaycause{applying the hypothesis (ii)}
&=1,\cr}$$

\noindent as claimed.\ \Qed

Since we already know that $\nu_11=1$ and that $\mu b\le\nu_1b$ for
every $b\in\frak B_1$, we must have $\mu 1=1$ and $\mu b=\nu_1b$ for
every $b\in\frak B$, so that $\mu$ extends $\nu_1$.   Similarly, $\mu$
extends $\nu_2$.

\medskip

{\bf (e)} Thus the proposition is proved in the case in which $\frak A$
is finite.   In the general case, for each finite subset $K$ of $\frak
A$ write $\frak A_K$ for the subalgebra of $\frak A$ generated by $K$.
Then (b)-(d) tell us that there is a non-negative additive functional
$\mu_K$ on $\frak A_K$, dominated by $\theta$ on $\frak A_K$, agreeing
with $\nu_1$ on $\frak A_K\cap\frak B_1$ and agreeing with $\nu_2$ on
$\frak A_K\cap\frak B_2$.   Let $\mu$ be any cluster point of the
$\mu_K$ in $[0,1]^{\frak A}$ as $K$ increases through the finite subsets
of $\frak A$;  then $\mu$ will be a non-negative additive functional on
$\frak A$, dominated by $\theta$, and extending $\nu_1$ and $\nu_2$.

This proves the result.
}%end of proof of 457D

\leader{457E}{Proposition} Let $X$ be a non-empty set and
$\familyiI{\nu_i}$ a family of probability measures on $X$ satisfying
the conditions of Lemma 457A, taking $\frak A=\Cal PX$ and
$\frak B_i=\dom\nu_i$ for each $i$.   Suppose that there is a countably
compact class $\Cal K\subseteq\Cal PX$ such that every $\nu_i$ is inner
regular with respect to $\Cal K$.   Then there is a probability measure
$\mu$ on $X$ extending every $\nu_i$.

\proof{ If $I=\emptyset$ this is trivial.   Otherwise, by 457A, there is
a finitely additive functional $\nu$ on $\Cal PX$ extending every
$\nu_i$.   Now 413Sa tells us that there is a complete measure $\mu$ on
$X$ such that $\mu X\le\nu X$ and $\mu K\ge\nu K$ for every
$K\in\Cal K$.   In this case, for any $i\in I$ and
$E\in\Tau_i=\dom\nu_i$, we must have

$$\eqalign{\mu_*E
&\ge\sup_{K\in\Cal K,K\subseteq E}\mu K
\ge\sup_{K\in\Cal K\cap\dom\nu_i,K\subseteq E}\mu K\cr
&\ge\sup_{K\in\Cal K\cap\dom\nu_i,K\subseteq E}\nu K
=\sup_{K\in\Cal K\cap\dom\nu_i,K\subseteq E}\nu_iK
=\nu_iE.\cr}$$

\noindent In particular, $\mu X\ge\nu_iX=1$.   Also $\mu X\le\nu X=1$, so

\Centerline{$\mu^*E=1-\mu_*(X\setminus E)
\le 1-\nu_i(X\setminus E)=\nu_iE$}

\noindent for any $E\in\Tau_i$;  as $\mu$ is complete, $\mu E$ is
defined and equal to $\nu_iE$ for every $E\in\Tau_i$, and $\mu$ extends
$\nu_i$, as required.
}%end of proof of 457E

\leader{457F}{Proposition} (a) Let $(X,\Sigma,\mu)$ be a perfect
probability space and $(Y,\Tau,\nu)$ any probability space.   Write
$\Sigma\otimes\Tau$ for the algebra of subsets of $X\times Y$ generated
by $\{E\times F:E\in\Sigma,\,F\in\Tau\}$.   Suppose that
$Z\subseteq X\times Y$ is such that

\inset{(i) $Z$ is expressible as the intersection of a sequence in
$\Sigma\otimes\Tau$,

(ii) $Z\cap(E\times F)\ne\emptyset$ whenever $E\in\Sigma$, $F\in\Tau$
are such that $\mu E+\nu F>1$.}

\noindent Then there is a probability measure $\lambda$ on $Z$ such that
the maps $(x,y)\mapsto x:Z\to X$ and $(x,y)\mapsto y:Z\to Y$ are both
\imp.

(b) Let $\familyiI{(X_i,\Sigma_i,\mu_i)}$ be a family of perfect
probability spaces.   Write $\bigotimes_{i\in I}\Sigma_i$ for the
algebra of subsets of $X=\prod_{i\in I}X_i$ generated by
$\{\{x:x\in X,\,x(i)\in E\}: i\in I,\,E\in\Sigma_i\}$.
Suppose that $Z\subseteq X$ is such that

\inset{(i) $Z$ is expressible as the intersection of a sequence in
$\bigotimes_{i\in I}\Sigma_i$,

(ii) whenever $i_0,\ldots,i_n\in I$ and $E_k\in\Sigma_{i_k}$ for
$k\le n$, there is a $z\in Z$ such that
$\#(\{k:k\le n, z(i_k)\in E_k\})\ge\sum_{k=0}^n\mu_{i_k}E_k$.}

\noindent Then there is a perfect probability measure $\lambda$ on $Z$
such that $z\mapsto z(i):Z\to X_i$ is \imp\ for every $i\in I$.

\proof{{\bf (a)} Apply 457Cb to the coordinate maps $f_1:Z\to X$ and
$f_2:Z\to Y$.   The condition (ii) here shows that 457C(b-ii) is
satisfied, so
there is an additive functional $\theta:\Cal PZ\to[0,1]$ such that
$\theta f_1^{-1}[E]=\mu E$ for every $E\in\Sigma$ and
$\theta f_2^{-1}[F]=\nu F$ for every $F\in\Tau$.

Define $\theta':\Sigma\otimes\Tau\to[0,1]$ by setting
$\theta'W=\theta(Z\cap W)$ for every $W\in\Sigma\otimes\Tau$.
Then $\theta'(E\times Y)=\mu E$ for every $E\in\Sigma$ and
$\theta'(X\times F)=\nu F$ for every $F\in\Tau$.   Because $\mu$ is
perfect, $\theta'$ has an extension to a measure $\tilde\lambda$ defined
on $\Sigma\tensorhat\Tau$ (454C).   Now $Z$ is supposed to be
expressible as $\bigcap_{n\in\Bbb N}W_n$ where $W_n\in\Sigma\otimes\Tau$
for every $n$;  since

\Centerline{$\tilde\lambda W_n=\theta'W_n
=\theta(Z\cap W_n)=\theta Z=1$}

\noindent for every $n$, $\tilde\lambda Z=1$.   So if we take $\lambda$
to be the subspace measure on $Z$ induced by $\tilde\lambda$, $\lambda$
will be a probability measure on $Z$.   If $E\in\Sigma$, then

$$\eqalign{\lambda(Z\cap(E\times Y))
&=\tilde\lambda(Z\cap(E\times Y))
=\tilde\lambda(E\times Y)\cr
&=\theta'(E\times Y)
=\theta(Z\cap(E\times Y))
=\mu E.\cr}$$

\noindent So $f_1:Z\to X$ is \imp\ for $\lambda$ and $\mu$.
Similarly, $f_2:Z\to Y$ is \imp\ for $\lambda$ and $\nu$.

\medskip

{\bf (b)} We use the same ideas, but appealing to 457B and 454D instead
of 457Cb and 454C.   Taking $f_i:X\to X_i$ to be the coordinate map for
each $i\in I$, (ii) here, with 457B, tells us that there is an additive
functional $\theta:\Cal PZ\to[0,1]$ such that
$\theta f_i^{-1}[E]=\mu_iE$ whenever $i\in I$ and $E\in\Sigma_i$.

Define $\theta':\bigotimes_{i\in I}\Sigma_i\to[0,1]$ by setting
$\theta'W=\theta(Z\cap W)$ for every $W\in\bigotimes_{i\in I}\Sigma_i$.
Then

\Centerline{$\theta'\{x:x\in X,\,x(i)\in E\}
=\theta\{z:z\in Z,\,z(i)\in E\}=\mu_iE$}

\noindent whenever $i\in I$ and $E\in\Sigma_i$.   Because every $\mu_i$
is perfect, $\theta'$ has an extension to a perfect measure
$\tilde\lambda$ defined on $\Tensorhat_{i\in I}\Sigma_i$ (454D).   Now
$Z$ is supposed to be expressible as $\bigcap_{n\in\Bbb N}W_n$ where
$W_n\in\bigotimes_{i\in I}\Sigma_i$ for every $n$;  since

\Centerline{$\tilde\lambda W_n=\theta'W_n
=\theta(Z\cap W_n)=\theta Z=1$}

\noindent for every $n$, $\tilde\lambda Z=1$.   So if we take $\lambda$
to be the subspace measure on $Z$ induced by $\tilde\lambda$, $\lambda$
will be a probability measure on $Z$;  by 451Dc, $\lambda$ is perfect.
If $i\in I$ and $E\in\Sigma_i$, then

$$\eqalign{\lambda\{z:z\in Z,\,z(i)\in E\}
&=\tilde\lambda\{x:x\in X,\,x(i)\in E\}
=\theta'\{x:x\in X,\,x(i)\in E\}\cr
&=\theta\{z:z\in Z,\,z(i)\in E\}
=\mu_iE.\cr}$$

\noindent So $z\mapsto z(i):Z\to X_i$ is \imp\ for $\lambda$ and $\mu_i$
for every $i\in I$, as required.
}%end of proof of 457F

\leader{457G}{Theorem} Let $X$ be a set and $\familyiI{\mu_i}$ a
family of probability measures on $X$ which is upwards-directed in the
sense that for any $i$, $j\in I$ there is a $k\in I$ such that $\mu_k$
extends both $\mu_i$ and $\mu_j$.   Suppose that for any countable
$J\subseteq I$ there is a measure on $X$ extending $\mu_i$ for every
$i\in J$.   Then there is a measure on $X$ extending $\mu_i$ for every
$i\in I$.

\proof{ Set $\Sigma_i=\dom\mu_i$ for each $i\in I$.   Because
$\familyiI{\mu_i}$ is upwards-directed,
$\Tau=\bigcup_{i\in I}\Sigma_i$ is an algebra of subsets of $X$, and we
have a finitely additive functional $\nu:\Tau\to[0,1]$ defined by saying
that $\nu E=\mu_iE$ whenever $i\in I$ and $E\in\Sigma_i$.   Now if
$\sequencen{E_n}$ is any non-increasing sequence in $\Tau$ with empty
intersection, there is a countable set $J\subseteq I$ such that
$E_n\in\bigcup_{i\in J}\Sigma_i$ for every $n\in\Bbb N$.   We are
told that there is a measure $\lambda$ on $X$ extending $\mu_i$ for
every $i\in J$;  now $\nu E_n=\lambda E_n$ for every $n\in\Bbb N$, so
$\lim_{n\to\infty}\nu E_n=0$.   By 413Kb, $\nu$ has an extension to a
measure on $X$, which of course extends every $\mu_i$.
}%end of proof of 457G

\leader{457H}{Example} Set $X=\{(x,y):0\le x<y\le 1\}\subseteq[0,1]^2$.
Write $\pi_1$, $\pi_2:X\to\Bbb R$ for the coordinate maps, and $\mu_L$
for Lebesgue measure on $[0,1]$, with $\Sigma_L$ its domain.

\spheader 457Ha There is a finitely additive functional
$\nu:\Cal PX\to[0,1]$ such that $\nu\pi_i^{-1}[E]=\mu_LE$ whenever
$i\in\{1,2\}$ and $E\in\Sigma_L$.   \prooflet{\Prf\ If $E_1$,
$E_2\in\Sigma_L$ and $\mu_LE_1+\mu_LE_2>1$, then neither is empty and
$\inf E_1<\sup E_2$, so there are $x\in E_1$, $y\in E_2$ such that
$x<y$, and
$(x,y)\in\pi_1^{-1}[E_1]\cap\pi_2^{-1}[E_2]$.   So the result follows by
457Cb.\ \Qed}

\spheader 457Hb However, there is no measure $\mu$ on $X$ for which both
$\pi_1$ and $\pi_2$ are \imp.
\prooflet{\Prf\Quer\ If there were,

\Centerline{$\int\pi_1(x,y)\mu(d(x,y))
=\int x\mu_L(dx)
=\int y\mu_L(dy)
=\int\pi_2(x,y)\mu(d(x,y))$}

\noindent by 235G;  but $\pi_1(x,y)<\pi_2(x,y)$ for every $(x,y)\in X$,
so this is impossible.\ \Bang\Qed}

\cmmnt{\spheader 457Hc If we write
$\Tau_i=\{\pi_i^{-1}[E]:E\subseteq[0,1]$ is Borel$\}$ for each $i$, then
we have a measure $\nu_i$ with domain $\Tau_i$ defined by setting
$\nu_i\pi_i^{-1}[E]=\mu_LE$ for each Borel set $E\subseteq[0,1]$.   Now
$\nu_1$ and $\nu_2$ have no common extension to a Borel measure on $X$,
even though $X$ is a Polish space and each
$\nu_i$ is a compact measure, being inner regular with respect to the
compact class $\Cal K_i=\{\pi_i^{-1}[K]:K\subseteq\ooint{0,1}$ is
compact$\}$.   (The trouble is that $\Cal K_1\cup\Cal K_2$ is {\it not}
compact, so we cannot apply 457E.)}

\leader{457I}{Example} Let $\mu_L$ be Lebesgue measure on $[0,1]$ and
$\Sigma_L$ its domain.   Set

\Centerline{$X=\{(\xi_1,\xi_2,\xi_3):0\le\xi_i\le 1$ for each $i$,
$\sum_{i=1}^3\xi_i\le\Bover32$, $\sum_{i=1}^3\xi_i^2\le 1\}$.}

\noindent For $1\le i\le 3$ set $\pi_i(x)=\xi_i$ for
$x=(\xi_1,\xi_2,\xi_3)\in X$.

\spheader 457Ia If $E_i\in\Sigma_L$ for $i\le 3$, then there is an
$x\in X$ such that $\#(\{i:\pi_i(x)\in E_i\})\ge\sum_{i=1}^3\mu_LE_i$.
\prooflet{\Prf\ Set $\alpha_i=\inf(E_i\cup\{1\})$ for each $i$, and set

\Centerline{$m=\lceil\sum_{i=1}^3\mu_LE_i\rceil
\le\lceil\sum_{i=1}^31-\alpha_i\rceil
=3-\lfloor\sum_{i=1}^3\alpha_i\rfloor$,}

\noindent so that $\sum_{i=1}^3\alpha_i<4-m$.   Take
$\xi_i\in E_i\cup\{1\}$ such that $\sum_{i=1}^3\xi_i<4-m$.
It will be enough to consider the case in which $\xi_1\le\xi_2\le\xi_3$.

\medskip

\quad{\bf (i)} If $m=1$, then $\sum_{i=1}^3\xi_i<3$ so $\xi_1<1$ and
$\xi_1\in E_1$.   Set $x=(\xi_1,0,0)$;  then $x\in X$ and

\Centerline{$\#(\{i:\pi_i(x)\in E_i\})\ge 1\ge\sum_{i=1}^3\mu_LE_i$.}

\medskip

\quad{\bf (ii)} If $m=2$, then $\sum_{i=1}^3\xi_i<2$ so $\xi_2<1$ and
$\xi_1\in E_1$, $\xi_2\in E_2$.   Set $x=(\xi_1,\xi_2,0)$.   We have
$\xi_1+\xi_2\le\bover43\le\bover32$.   Also

\Centerline{$\xi_2\le\Bover12(\xi_2+\xi_3)\le 1-\Bover12\xi_1$,}

\noindent so

\Centerline{$\xi_1^2+\xi_2^2\le\xi_1^2+(1-\Bover12\xi_1)^2
=1-\xi_1+\Bover54\xi_1^2\le 1$}

\noindent because $\xi_1\le\bover23\le\bover45$.   So $x\in X$ and

\Centerline{$\#(\{i:\pi_i(x)\in E_i\})\ge 2\ge\sum_{i=1}^3\mu_LE_i$.}

\medskip

\quad{\bf (iii)} If $m=3$ then $\sum_{i=1}^3\xi_i<1$ so $\xi_i\in E_i$
for every $i$;  set $x=(\xi_1,\xi_2,\xi_3)$.   Since
$\sum_{i=1}^3\xi_i^2\le\sum_{i=1}^3\xi_i\le 1$, $x\in X$ and

\Centerline{$\#(\{i:\pi_i(x)\in E_i\})=3\ge\sum_{i=1}^3\mu_LE_i$.}

Putting these together, we have the result.\ \Qed}

\spheader 457Ib There is no finitely additive functional $\nu$ on $X$
such that $\nu\pi_i^{-1}[E]=\mu_LE$ for each $i$ and every
$E\in\Sigma_L$.
\prooflet{\Prf\Quer\ Suppose there were.   Set
$\Tau_i=\{\pi_i^{-1}[E]:E\in\Sigma_L\}$ and $\nu_i=\nu\restrp\Tau_i$ for
each $i$.   Then $\nu_i$ is a probability measure on $X$;  moreover,
because $X$ is compact, $\pi_i^{-1}[K]$ is compact for every compact
$K\subseteq[0,1]$, so $\nu_i$ is inner regular with respect to the
compact subsets of $X$.   By 457E, the $\nu_i$ have a common extension
to a countably additive measure $\mu$.   Now

\Centerline{$\int_X\xi_1+\xi_2+\xi_3\,\mu(dx)=3\int_0^1t\,dt
=\Bover32$,}

\noindent so we must have $\xi_1+\xi_2+\xi_3=\Bover32$ for $\mu$-almost
every $x$;  similarly,

\Centerline{$\int_X\xi_1^2+\xi_2^2+\xi_3^2\,\mu(dx)
=3\int_0^1t^2\,dt=1$,}

\noindent so we must have $\xi_1^2+\xi_2^2+\xi_3^2=1$ for $\mu$-almost
every $x$.   Since

\Centerline{$(\Bover32-\xi_3)^2
=(\xi_1+\xi_2)^2\le 2(\xi_1^2+\xi_2^2)\le 2(1-\xi_3^2)$}

\noindent for almost every $x$, $\xi_3-\xi_3^2\ge\bover1{12}$ for almost
every $x$, which is impossible, since
$\mu\{x:\xi_3\le\Bover12-\Bover1{\sqrt 6}\}>0$.\
\Bang\Qed}%end of prooflet

\leader{457J}{Example} There are a set $X$ and a family
$\familyiI{\mu_i}$ of probability measures on $X$ such that (i) for
every countable set $J\subseteq I$ there is a measure on $X$ extending
$\mu_i$ for every $i\in J$ (ii) there is no measure on $X$ extending
$\mu_i$ for every $i\in I$.

\proof{ By 439Fc, there is an uncountable universally negligible subset
of $[0,1]$.   Because $[0,1]$ and $\Cal P\Bbb N$ are uncountable Polish
spaces, they have isomorphic Borel structures (424Cb), so there is an
uncountable universally negligible set
$X_0\subseteq\Cal P\Bbb N$.   The map $a\mapsto\Bbb N\setminus a$ is an
autohomeomorphism of $\Cal P\Bbb N$, so
$X_1=\{\Bbb N\setminus a:a\in X_0\}$ is universally negligible, and
$X=X_0\cup X_1$ is universally negligible (439Cb).

For $n\in\Bbb N$, set $E_n=\{a:n\in a\in X\}$ and
$\Sigma_n=\{\emptyset,E_n,X\setminus E_n,X\}$;  note that, because $X$
is closed under complementation, neither $E_n$ nor $X\setminus E_n$ is
empty, and we have a probability measure $\mu_n$ with domain $\Sigma_n$
defined by setting $\mu_nE_n=\mu_n(X\setminus E_n)=\bover12$.   Next,
for $a\in X$, set $\Sigma'_a=\{\emptyset,\{a\},X\setminus\{a\},X)$, and
let $\mu'_a$ be the probability measure with domain $\Sigma'_a$ defined
by setting $\mu'_a\{a\}=0$.

If $J\subseteq X$ is countable, then there is a probability measure on
$X$ extending $\mu_n$ for every $n\in\Bbb N$ and $\mu'_a$ for every
$a\in J$.   \Prf\ Because $X_0$ is uncountable, there is a $b\in X_0$
such that neither $b$ nor $b'=\Cal P\Bbb N\setminus b$ belongs to $J$.
Let $\mu$ be the probability measure with domain $\Cal PX$ defined by
setting $\mu\{b\}=\mu\{b'\}=\bover12$;  this extends all the $\mu_n$ and
all the $\mu'_a$ for $a\in J$.\ \Qed

\Quer\ Suppose, if possible, that $\mu$ is a measure on $X$ extending
every $\mu_n$ and every $\mu'_a$.   In this case, because $\mu$ extends
every $\mu_n$, its domain includes the Borel $\sigma$-algebra $\Cal B$
of $X$, and $\mu\restr\Cal B$ is a Borel probability measure on $X$.
Since $X$ is universally negligible, there is a point $a\in X$ such that
$\mu\{a\}>0$;  in which case $\mu$ cannot extend $\mu'_a$.\ \Bang

Thus the $\mu_n$, $\mu'_a$ constitute a family of the kind required.
}%end of proof of 457J

\leader{457K}{}\cmmnt{ In addition to existence, we can ask for
solutions to simultaneous-extension problems
which are optimal in some sense;
some transportation problems can be interpreted as questions of this kind.
In this direction I give just one result, which is also connected to
the ideas of \S437.\footnote{I am indebted to J.Pachl for leading me to
this material.}

\medskip

\noindent}{\bf Definition}\dvAnew{2009}\cmmnt{ ({\smc Bogachev 07},
\S8.10(viii))}
Let $(X,\rho)$ be a metric space.   For
quasi-Radon probability measures $\mu$, $\nu$ on $X$, set

\Centerline{$\rho_{\text{W}}(\mu,\nu)
=\sup\{|\int u\,d\mu-\int u\,d\nu|:
u:X\to\Bbb R$ is bounded and $1$-Lipschitz$\}$.}

\cmmnt{\noindent (Compare the metric $\rhoKR$ of 437Qb.  $\rho_{\text{W}}$
is sometimes called the `Wasserstein metric'.)}
%see "Wasserstein metric" or "Monge-Wasserstein metric"
% or"Vasershtein metric" ({\smc Dudley 76},
%Wikipedia
%R.L. Dobrushin,
%"Prescribing a system of random variables by conditional distributions"
%Theor. Probab. Appl. , 15  (1970)  pp. 458-486)
%Vasershtein 69
%Hutchinson 81, \S4.3 calls  \rho_{\text{W}}  the  "L-metric"
%thanks to J.Pachl for the lead

\leader{457L}{Theorem}\dvAnew{2009} Let $(X,\rho)$ be a metric space and
$P_{\text{qR}}$ the set of
quasi-Radon probability measures on $X$;  define $\rho_{\text{W}}$ as in
457K.

(a) For all $\mu$, $\nu$ and $\lambda$ in $P_{\text{qR}}$,

\Centerline{$\rho_{\text{W}}(\mu,\nu)=\rho_{\text{W}}(\nu,\mu)$,
\quad$\rho_{\text{W}}(\mu,\lambda)
  \le\rho_{\text{W}}(\mu,\nu)+\rho_{\text{W}}(\nu,\lambda)$,}

\Centerline{$\rho_{\text{W}}(\mu,\nu)=0$ iff $\mu=\nu$.}

(b)\cmmnt{ (cf.\ {\smc Vasershtein 69})} If $\mu$,
$\nu\in P_{\text{qR}}$, then $\rho_{\text{W}}(\mu,\nu)
=\inf_{\lambda\in Q(\mu,\nu)}\int\rho(x,y)\lambda(d(x,y))$, where
$Q(\mu,\nu)$ is the set of quasi-Radon probability measures on $X\times X$
with marginal measures $\mu$ and $\nu$.
%"Kantorovich-Rubinstein theorem" ({\smc Dudley 76}, 11.8.2)

(c) In (b), if $\mu$ and $\nu$ are Radon measures, $Q(\mu,\nu)$ is included
in $P_{\text{R}}(X\times X)$, the space of Radon probability measures on
$X\times X$, and is compact for the narrow topology on
$P_{\text{R}}(X\times X)$;  and there
is a $\lambda\in Q(\mu,\nu)$ such that
$\rho_{\text{W}}(\mu,\nu)=\int\rho(x,y)\lambda(d(x,y))$.

(d) If $\rho$ is bounded, then  $\rho_{\text{W}}$ is a metric on
$P_{\text{qR}}$ inducing
the narrow topology\cmmnt{ (definition:  437Jd)}.

\proof{{\bf (a)} The first two clauses are immediate from the definition.
For the third, observe that if $\mu\ne\nu$ then
$\rho_{\text{W}}(\mu,\nu)\ge\rhoKR(\mu,\nu)>0$ by 437R.

\medskip

{\bf (b)} Write $\zeta\in[0,\infty]$ for $\rho_{\text{W}}(\mu,\nu)$,
$\eusm L^{\infty}_{\dom\mu}$ for the space of bounded
$\dom\mu$-measurable functions from $X$ to $\Bbb R$ and
$\eusm L^{\infty}_{\dom\nu}$ for the space of bounded
$\dom\nu$-measurable functions from $X$ to $\Bbb R$.

\medskip

\quad{\bf (i)} We have

$$\eqalign{\zeta
&=\sup\{\int u\,d\mu+\int v\,d\nu:
   u\in\eusm L^{\infty}_{\dom\mu},\,v\in\eusm L^{\infty}_{\dom\nu},\cr
&\mskip150mu u(x)+v(y)\le\rho(x,y)\text{ for all }x,\,y\in X\}.\cr}$$

\noindent\Prf\grheada\ Suppose that $u\in\eusm L^{\infty}_{\dom\mu}$,
$v\in\eusm L^{\infty}_{\dom\nu}$ and
$u(x)+v(y)\le\rho(x,y)$ for all $x$, $y\in X$.   Set

\Centerline{$w(x)=\inf_{y\in X}\rho(x,y)-v(y)$}

\noindent for $x\in X$.   Then $u(x)\le w(x)$ and $w(x)+v(x)\le 0$ for
every $x$, so $u\le w\le -v$ and $w$ is bounded;  also $w$ is
$1$-Lipschitz, because if $x$, $x'\in X$ then

\Centerline{$w(x)-\rho(x,x')=\inf_{y\in X}\rho(x,y)-v(y)-\rho(x,x')
\le\inf_{y\in X}\rho(x',y)-v(y)=w(x')$.}

\noindent Accordingly

\Centerline{$\int u\,d\mu+\int v\,d\nu
\le\int w\,d\mu-\int w\,d\nu\le\zeta$.}

\noindent\grheadb\ In the other direction, given $\gamma<\zeta$,
there is a
bounded $1$-Lipschitz function $u:X\to\Bbb R$ such that
$|\int u\,d\mu-\int u\,d\nu|\ge\gamma$.   Replacing $u$ by $-u$ if
necessary, we can arrange that $\int u\,d\mu-\int u\,d\nu\ge\gamma$.
Now set $v=-u$;  then
$u(x)+v(y)\le\rho(x,y)$ for all $x$, $y$, and
$\int u\,d\mu+\int v\,d\nu\ge\gamma$.\ \Qed

It follows that if $u\in\eusm L^{\infty}_{\dom\mu}$,
$v\in\eusm L^{\infty}_{\dom\nu}$ and
$u(x)+v(y)\le\beta\rho(x,y)$ for
all $x$, $y\in X$, where $\beta>0$, then

\Centerline{$\int u\,d\mu+\int v\,d\nu
=\beta(\int\Bover1{\beta}u\,d\mu+\int\Bover1{\beta}v\,d\nu)\le\beta\zeta$.}

\medskip

\quad{\bf (ii)} $\int\rho\,d\lambda\ge\zeta$ for every
$\lambda\in Q(\mu,\nu)$.   \Prf\ If
$u\in\eusm L^{\infty}_{\dom\mu}$,
$v\in\eusm L^{\infty}_{\dom\nu}$
and $u(x)+v(y)\le\rho(x,y)$ for all $x$, $y\in X$, then

$$\eqalignno{\int u\,d\mu+\int v\,d\nu
&=\int u(x)\lambda(d(x,y))+\int v(y)\lambda(d(x,y))\cr
\displaycause{235G}
&\le\int\rho\,d\lambda}$$

\noindent so (i) gives us the result.\ \Qed

If $\zeta=\infty$, we can stop;  so henceforth suppose that $\zeta$ is
finite.

\medskip

\quad{\bf (iii)} Define $p:\ell^{\infty}(X\times X)\to\coint{0,\infty}$ by
setting

\Centerline{$p(w)=\inf\{\alpha+\beta\zeta:\alpha$, $\beta>0$,
$w(x,y)\le\alpha+\beta\rho(x,y)$ for all $x$, $y\in X\}$.}

\noindent Then $p(w+w')\le p(w)+p(w')$ and $p(\alpha w)=\alpha p(w)$
whenever $w$, $w'\in\ell^{\infty}(X\times X)$ and
$\alpha\in\coint{0,\infty}$.   For $u$, $v\in\BbbR^X$ define
$u\otimes v\in\BbbR^{X\times X}$ by setting $(u\otimes v)(x,y)=u(x)v(y)$
for all $x$, $y\in X$ (cf.\ 253B);  set

\Centerline{$V=\{(u\otimes\chi X)+(\chi X\otimes v):
u\in\eusm L^{\infty}_{\dom\mu}$,
$v\in\eusm L^{\infty}_{\dom\nu}\}$.}

\noindent Let $\mu\times\nu$ be the quasi-Radon product measure on
$X\times X$ (417R).
Then we have a linear functional $h_0:V\to\Bbb R$ defined by saying
that $h_0(w)=\int w\,d(\mu\times\nu)$ for $w\in V$.   The point is that
$h_0(w)\le p(w)$ for every $w\in V$.   \Prf\ We have
$u\in\eusm L^{\infty}_{\dom\mu}$,
$v\in\eusm L^{\infty}_{\dom\nu}$ such
that $w(x,y)=u(x)+v(y)$ for all $x$, $y\in X$.   If $\alpha$, $\beta>0$
are such that $w(x,y)\le\alpha+\beta\rho(x,y)$ for all $x$, $y\in X$, set
$u_0(x)=u(x)-\alpha$ for every $x$;  then $u_0(x)+v(y)\le\beta\rho(x,y)$
for all $x$ and $y$, so

$$\eqalign{h_0(w)
&=\int u\otimes\chi X\,d(\mu\times\nu)
  +\int\chi X\otimes v\,d(\mu\times\nu)
=\int u\,d\mu+\int v\,d\nu\cr
&=\alpha+\int u_0\,d\mu+\int v\,d\nu
\le\alpha+\beta\zeta\cr}$$

\noindent by the last remark in (i).
As $\alpha$ and $\beta$ are arbitrary, $h_0(w)\le p(w)$.\ \Qed

\medskip

\quad{\bf (iv)} By the Hahn-Banach theorem (3A5Aa), there is a linear
functional $h:\ell^{\infty}(X\times X)\to\Bbb R$, extending $h_0$, such
that $h(w)\le p(w)$ for every $w\in\ell^{\infty}(X\times X)$.
In this case, $h$ must be a positive linear functional, because if $w\ge 0$
then $p(-w)=0$, so $h(-w)\le 0$.   Since also

\Centerline{$h(\chi(X\times X))
=h_0(\chi(X\times X))=(\mu\times\nu)(X\times X)=1$,}

\noindent $\|h\|=1$ in $\ell^{\infty}(X\times X)^*$.   If $u$,
$v\in C_b(X)$ then

\Centerline{$h(u\otimes\chi X)=h_0(u\otimes\chi X)=\int u\,d\mu$,
\quad$h(\chi X\otimes v)=h_0(\chi X\otimes v)=\int v\,d\nu$.}

\noindent
%If $\gamma\ge 0$, then
%$h(\rho\wedge\gamma\chi(X\times X))\le p(\rho)\le 1$.
Let $\theta:\Cal P(X\times X)\to[0,1]$ be the additive functional
defined by setting $\theta W=h(\chi W)$ for $W\subseteq X\times X$.
Observe that $\theta(E\times X)=\mu E$ for every $E\in\dom\mu$
and $\theta(X\times E)=\nu E$ for every $E\in\dom\nu$.

\medskip

\quad{\bf (v)} Because both $\mu$ and $\nu$ are inner regular with
respect to the totally bounded sets (434L), there is a separable subset $Y$
of $X$ such that $\mu Y=\nu Y=1$, and we can take $Y$ to be a Borel set.
Now let $\epsilon>0$.   Then we have a
countable partition $\familyiI{E_i}$ of $Y$ into non-empty Borel sets of
diameter at most $\epsilon$.   For $i$, $j\in I$, set

$$\eqalign{\alpha_{ij}
&=\Bover{\theta(E_i\times E_j)}{\mu E_i\,\nu E_j}
\text{ if }\mu E_i\cdot\nu E_j>0,\cr
&=0\text{ otherwise}.\cr}$$

\noindent Since $\theta(E_i\times E_j)\le\min(\mu E_i,\nu E_j)$,
$\theta(E_i\times E_j)=\alpha_{ij}\mu E_i\nu E_j$.   If $i\in I$ is
such that $\mu E_i>0$, then $\sum_{j\in I}\alpha_{ij}\nu E_j=1$.   \Prf\
For any $\eta>0$ there is a finite $K_0\subseteq I$ such that
$\nu(X\setminus\bigcup_{j\in K_0}E_j)\le\eta$.   Now

$$\eqalign{|1-\sum_{j\in K}\alpha_{ij}\nu E_j|\mu E_i
&=|\mu E_i-\sum_{j\in K}\theta(E_i\times E_j)|
=|\theta(E_i\times X)-\theta(E_i\times\bigcup_{j\in K}E_j)|\cr
&=\theta(E_i\times(X\setminus\bigcup_{j\in K}E_j))
\le\theta(X\times(X\setminus\bigcup_{j\in K}E_j))\cr
&=\nu(X\setminus\bigcup_{j\in K}E_j)
\le\eta\cr}$$

\noindent whenever $K$ is a finite subset of $I$ including $K_0$;
as $\eta$ is arbitrary,
$\mu E_i\cdot\sum_{j\in I}\alpha_{ij}\nu E_j=\mu E_i$
and $\sum_{j\in I}\alpha_{ij}\nu E_j=1$.\ \QeD\  Similarly,
$\sum_{i\in I}\alpha_{ij}\mu E_i=1$ whenever $\nu E_j>0$.

\medskip

\quad{\bf (vi)} Define a Borel measurable function
$w_0:X\times X\to\coint{0,\infty}$ by setting

$$\eqalign{w_0(x,y)
&=\alpha_{ij}\text{ if }i,\,j\in I,\,x\in E_i\text{ and }y\in E_j,\cr
&=0\text{ if }(x,y)\in(X\times X)\setminus(Y\times Y).\cr}$$

\noindent Let $\lambda$ be the indefinite-integral measure
over $\mu\times\nu$ defined by $w_0$;  then $\lambda$ is a quasi-Radon
probability
measure with marginals $\mu$, $\nu$.   \Prf\ If $E\in\dom\mu$, then

$$\eqalignno{\lambda(E\times X)
&=\int_{E\times X}w_0d(\mu\times\nu)
=\sum_{i,j\in I}\int_{(E\cap E_i)\times E_j}w_0d(\mu\times\nu)\cr
&=\sum_{i,j\in I}\alpha_{ij}\mu(E\cap E_i)\cdot\nu E_j
=\sum_{i\in I}\mu(E\cap E_i)\cr
\displaycause{because $\sum_{j\in I}\alpha_{ij}\nu E_j=1$ whenever
$\mu E_i>0$}
&=\mu E.\cr}$$

\noindent In particular, $\lambda(X\times X)=1$, so $\lambda$ is a
probability measure, and is quasi-Radon by 415Ob;  and the coordinate
projection $(x,y)\mapsto x$ is \imp\ for $\lambda$ and $\mu$.   To see that
$\mu$ is exactly the image measure, observe that if $E\subseteq X$ is such
that $\lambda(E\times X)$ is defined, then $(E\cap E_i)\times E_j$ must
be measured by $\mu\times\nu$ whenever $\alpha_{ij}>0$.   For any $i\in I$
such that $\mu E_i>0$, there is surely some $j$ such that $\alpha_{ij}>0$,
in which case $E\cap E_i\in\dom\mu$;  since $\bigcup_{i\in I}E_i$ is
$\mu$-conegligible (and $\mu$ is complete and $I$ is countable),
$E\in\dom\mu$.   Thus $\mu$ is the marginal of $\lambda$ on the first
coordinate.   Similarly, $\nu$ is the marginal of $\lambda$ on the second
coordinate.\ \Qed

For $i$, $j\in I$ we have

\Centerline{$\lambda(E_i\times E_j)
=\alpha_{ij}\mu E_i\cdot\nu E_j=\theta(E_i\times E_j)$.}

\medskip

\quad{\bf (vii)} $\int\rho\,d\lambda\le\zeta+2\epsilon$.   \Prf\
For $i$, $j\in I$, set

\Centerline{$\beta_{ij}=\inf_{x\in E_i,y\in E_j}\rho(x,y)$;}

\noindent set

$$\eqalign{w(x,y)
&=\beta_{ij}\text{ if }i,\,j\in I,\,x\in E_i\text{ and }y\in E_j,\cr
&=0\text{ if }(x,y)\in(X\times X)\setminus(Y\times Y).\cr}$$

\noindent Then

\Centerline{$w\le\rho\times\chi(Y\times Y)
\le w+2\epsilon\chi(X\times X)$,}

\noindent so

$$\eqalign{\int\rho\,d\lambda
&=\int_{Y\times Y}\rho\,d\lambda
\le 2\epsilon+\int w\,d\lambda\cr
&=2\epsilon+\sum_{i,j\in I}\beta_{ij}\lambda(E_i\times E_j)
=2\epsilon+\sum_{i,j\in I}\beta_{ij}\theta(E_i\times E_j).\cr}$$

\noindent Now, for any finite $K\subseteq I$,

\Centerline{$\sum_{i,j\in K}\beta_{ij}\theta(E_i\times E_j)
=h(w\times\chi(\bigcup_{i,j\in K}E_i\times E_j))
\le h(\rho)
\le p(\rho)\le\zeta$}

\noindent by the definition of $p$.
So $\int\rho\,d\lambda\le 2\epsilon+\zeta$, as claimed.\ \Qed

\medskip

\quad{\bf (viii)} As $\epsilon$ is arbitrary,

\Centerline{$\inf_{\lambda\in Q(\mu,\nu)}\int\rho\,d\lambda\le\zeta$.}

\noindent With (ii), this completes the proof of (b).

\medskip

{\bf (c)} For every $\epsilon>0$, there is a compact set $K\subseteq X$
such that $\mu(X\setminus K)+\nu(X\setminus K)\le\epsilon$.   In this case
$\lambda((X\times X)\setminus(K\times K))\le\epsilon$ for every
$\lambda\in Q(\mu,\nu)$.   In the first place, this shows that if
$\lambda\in Q(\mu,\nu)$, then
$\lambda W=\sup_{L\subseteq X\times X\text{ is compact}}\lambda(W\cap L)$
for every $W\in\dom\lambda$;  by 416F, $\lambda$
is a Radon measure.   Thus $Q(\mu,\nu)\subseteq P_{\text{R}}(X\times X)$.
Next, we
see also that $Q(\mu,\nu)$ is uniformly tight (437O), therefore relatively
compact in the space $M^+_{\text{R}}(X\times X)$ of totally finite Radon
measures on $X\times X$ (437P).

Writing $\pi_1$, $\pi_2$ for the coordinate projections from $X\times X$ to
$X$, we see that

\Centerline{$Q(\mu,\nu)
=\{\lambda:\lambda\in M_{\text{R}}^+(X\times X)$, $\lambda\pi_1^{-1}=\mu$
and $\lambda\pi_2^{-1}=\nu\}$.}

\noindent Since the functions $\lambda\mapsto\lambda\pi_1^{-1}$ and
$\lambda\mapsto\lambda\pi_2^{-1}$ from $M^+_{\text{R}}(X\times X)$ to
$M^+_{\text{R}}(X)$ are continuous (437N), and $M^+_{\text{R}}(X)$ is
Hausdorff in its narrow topology (437R(a-ii)), $Q(\mu,\nu)$ is closed in
$M^+_{\text{R}}(X\times X)$, therefore compact.

Finally, the function $\lambda\mapsto\int\rho\,d\lambda$
from $M^+_{\text{R}}(X\times X)$ to $[0,\infty]$ is lower
semi-continuous (437Jg), and must
attain its infimum on the compact set $Q(\mu,\nu)$ (4A2B(d-viii)).
But (b) tells us that this infimum is just $\rho_{\text{W}}(\mu,\nu)$.

\medskip

{\bf (d)(i)} Suppose first that $\rho(x,y)\le 2$ for all $x$, $y\in X$.
Then $\rho_{\text{W}}
=\rhoKR\restr P_{\text{qR}}\times P_{\text{qR}}$.
\Prf\ As already noted in (a),
$\rho_{\text{W}}(\mu,\nu)\ge\rhoKR(\mu,\nu)$ for all $\mu$, $\nu\in P_{\text{qR}}$.
In the other direction, if $\mu$, $\nu\in P_{\text{qR}}$ and $u:X\to\Bbb R$ is
$1$-Lipschitz, then $|u(x)-u(y)|\le 2$ for all $x$, $y\in X$, so there is
an $\alpha\in\Bbb R$ such that $|u(x)-\alpha|\le 1$ for all $x\in X$.
Set $v(x)=u(x)-\alpha$ for every $x$;  then $v:X\to[-1,1]$ is
$1$-Lipschitz, so

$$\eqalignno{|\int u\,d\mu-\int u\,d\nu|
&=|\int v\,d\mu-\int v\,d\nu|\cr
\displaycause{because $\mu X=\nu X$}
&\le\rhoKR(\mu,\nu).\cr}$$

\noindent As $u$ is arbitrary,
$\rho_{\text{W}}(\mu,\nu)\le\rhoKR(\mu,\nu)$ and
the two metrics are equal.\ \Qed

\medskip

\quad{\bf (ii)} In general, take $\gamma>0$ such that
$\rho(x,y)\le 2\gamma$ for all
$x$, $y\in X$.   Set $\sigma=\Bover1{\gamma}\rho$, so that $\sigma$
is a metric on $X$ equivalent to $\rho$.   Now $\sigma_{\text{KR}}$
defines the narrow topology on $P_{\text{qR}}$, by 437R(g-i), so
$\rho_{\text{W}}=\gamma\sigma_{\text{W}}
=\gamma\sigma_{\text{KR}}\restr P_{\text{qR}}\times P_{\text{qR}}$
also does.
}%end of proof of 457L

\leader{457M}{}\cmmnt{ If we relax our demands, and look for measures
dominated by each measure in a family rather than extending them,
similar methods give further results.

\medskip

\noindent}{\bf Theorem}\dvAnew{2010}\cmmnt{ (see {\smc Kellerer 84})}
Let $X$ be a Hausdorff space and $\familyiI{\nu_i}$ a non-empty
finite family of locally finite
measures on $X$ all inner regular with respect to the closed sets.

(a) For $A\subseteq X\times\coint{0,\infty}$, set

$$\eqalign{c(A)
&=\inf\{\sum_{i\in I}\int h_id\nu_i:
h_i:X\to[0,\infty]\text{ is }\dom\nu_i\text{-measurable
for each }i\in I,\cr
&\mskip250mu \alpha\le\sum_{i\in I}h_i(x)\text{ whenever }
  (x,\alpha)\in A\}.\cr}$$

\quad(i) $c$ is a Choquet capacity\cmmnt{ (definition:  432J)}.

\quad(ii)
For every $A\subseteq X\times\coint{0,\infty}$, the infimum in the
definition of $c(A)$ is attained.

(b) Let $f:X\to\coint{0,\infty}$ be a function such that
$\{x:f(x)\ge\alpha\}$ is K-analytic for every $\alpha>0$.   Then

$$\eqalignno{&\inf\{\sum_{i\in I}\int h_id\nu_i:
h_i:X\to[0,\infty]\text{ is }\dom\nu_i\text{-measurable
for each }i\in I,\,f\le\sum_{i\in I}h_i\}\cr
&\mskip75mu=\sup\{\int f\,d\mu:\mu\text{ is a Radon measure on }X
  \text{ and }\mu\le\nu_i\text{ for every }i\in I\}\dvro{.}{,}\cr}$$

\cmmnt{\noindent where `$\mu\le\nu_i$' here is to be interpreted in
the sense of 234P.}

\proof{{\bf (a)(i)}\grheada\ For $f:X\to[0,\infty]$ set

\Centerline{$\Omega_f=\{(x,\alpha):x\in X$, $\alpha\le f(x)\}$,
\quad$\Omega'_f=\{(x,\alpha):x\in X$, $\alpha<f(x)\}$}

\noindent as in 252N.

It will be convenient to amalgamate the $\nu_i$ into a
single measure, as follows.   Let $(Y,\Tau,\nu)$ be the direct sum of the
family
$\familyiI{(X_i,\nu_i)}$ in the sense of 214L, so that
$Y=X\times I$ and $\nu E=\sum_{i\in I}\nu_i\{x:(x,i)\in E\}$ for those
$E\subseteq Y$ for which the sum is defined.   Give $Y$ its disjoint-union
topology, that is, the product topology if $I$ is given the discrete
topology;  then it is easy to check that
$\nu$ is locally finite (see 411Xh) and inner regular with respect to
the closed sets (see 412Xm).   For $h\in[0,\infty]^Y$ and $x\in X$ set
$(Th)(x)=\sum_{i\in I}h(x,i)$;  observe that $T(h+h')=Th+Th'$ and
$T(\alpha h)=\alpha Th$ for all $h$, $h':Y\to[0,\infty]$ and $\alpha\ge 0$.
Now, for any $A\subseteq X\times\coint{0,\infty}$, we have

$$\eqalignno{c(A)
&=\inf\{\int h\,d\nu:
  h:Y\to[0,\infty]\text{ is }\Tau\text{-measurable},\cr
&\mskip200mu \alpha\le Th(x)\text{ whenever }(x,\alpha)\in A\}\cr
\displaycause{because $\int h\,d\nu=\sum_{i\in I}\int h(x,i)\nu_i(dx)$
for non-negative $h$, by 214M}
&=\inf\{\int h\,d\nu:
  h:Y\to[0,\infty]\text{ is }\Tau\text{-measurable},\,
  A\subseteq\Omega_{Th}\}.\cr}$$

\medskip

\qquad\grheadb\ Of course $c:\Cal P(X\times\coint{0,\infty})\to[0,\infty]$
is non-decreasing.   To see that it is sequentially order-continuous on the
left, I show in fact that if $\sequencen{A_n}$ is a
non-decreasing sequence of subsets of $X\times[0,\infty]$ with union
$A$, and $\gamma=\sup_{n\in\Bbb N}c(A_n)$ is finite, then there is
a $\Tau$-measurable $h:Y\to[0,\infty]$ such that $\alpha\le Th(x)$ whenever
$(x,\alpha)\in A$ and $\int h\,d\nu=\gamma$.   \Prf\
Surely $c(A)\ge\gamma$.   For each $n\in\Bbb N$
we have a $\Tau$-measurable $h_n:Y\to[0,\infty]$ such that
$\int h_nd\nu\le\gamma+2^{-n}$ and $A_n\subseteq\Omega_{Th_n}$.   By
Koml\'os' theorem (276H), there is a strictly increasing sequence
$\sequence{k}{n(k)}$ in $\Bbb N$ such that
$\lim_{m\to\infty}\Bover1{m+1}\sum_{k=0}^mh_{n(k)}$ is defined $\nu$-a.e.;
set $h=\limsup_{m\to\infty}\Bover1{m+1}\sum_{k=0}^mh_{n(k)}$.   Then
$h:Y\to[0,\infty]$ is $\Tau$-measurable, and
$h\eae\liminf_{m\to\infty}\Bover1{m+1}\sum_{k=0}^mh_{n(k)}$.
By Fatou's Lemma,

\Centerline{$\int h\,d\nu
\le\liminf_{m\to\infty}\Bover1{m+1}\sum_{k=0}^m\int h_{n(k)}d\nu
\le\gamma$,}

\noindent while if $j\in\Bbb N$ and $(x,\alpha)\in A_j$.
$\alpha\le Th_{n(k)}(x)$ for every $k\ge j$, so

$$\eqalignno{\alpha
&\le\liminf_{m\to\infty}\Bover1{m+1}\sum_{k=0}^mTh_{n(k)}(x)
\le\limsup_{m\to\infty}\sum_{i\in I}\Bover1{m+1}\sum_{k=0}^mh_{n(k)}(x,i)\cr
&\le\sum_{i\in I}\limsup_{m\to\infty}
  \Bover1{m+1}\sum_{k=0}^mh_{n(k)}(x,i)\cr
\displaycause{because $I$ is finite}
&=\sum_{i\in I}h(x,i)
=Th(x).\cr}$$

\noindent Thus $A\subseteq\Omega_{Th}$, so

\Centerline{$c(A)\le\int h\,d\nu\le\gamma\le c(A)$}

\noindent and we have equality.\ \Qed

\medskip

\qquad\grheadc\ Now suppose that $K\subseteq X\times\coint{0,\infty}$ is
compact, and $\epsilon>0$.   Set $L=\pi_1[K]$, where
$\pi_1:X\times\coint{0,\infty}\to X$ is the canonical map;  then
$L\subseteq X$ and $L\times I\subseteq Y$ are compact.   Because $\nu$ is
locally finite, there is an open set $H\subseteq Y$ such that
$L\times I\subseteq H\in\Tau$ and $\nu H$ is finite (see 411Ga).
Let $\nu_H$ be the subspace measure induced by $\nu$ on $H$, and $\Tau_H$
its domain;  then $\nu_H$
is totally finite and inner regular with respect to the closed sets
(412Pc), therefore outer regular with respect to the open sets (411D).
Let $h:Y\to[0,\infty]$ be a $\Tau$-measurable function such that
$A\subseteq\Omega_{Th}$ and $\int h\,d\nu\le c(K)+\epsilon$.   Set
$h_1(y)=h(y)+\Bover{\epsilon}{\nu H}$ for $y\in H$;  then
$\int_Hh_1d\nu_H\le c(K)+2\epsilon$.   By 412Wa, there is a lower
semi-continuous $\Tau_H$-measurable $g_1:H\to[0,\infty]$ such that
$h_1\le g_1$ and $\int_Hg_1d\nu_H\le c(K)+3\epsilon$.   Extend $g_1$ to a
function $g:Y\to[0,\infty]$ by setting $g(y)=0$ for $y\in Y\setminus H$;
then $g$ is $\Tau$-measurable and lower semi-continuous and
$\int g\,d\nu\le c(K)+3\epsilon$.   Moreover, if $(x,\alpha)\in K$, then

$$\eqalignno{Tg(x)
&>T(h\times\chi H)(x)
=Th(x)\cr
\displaycause{because $\{x\}\times I\subseteq H$}
&\ge\alpha,\cr}$$

\noindent so $K\subseteq\Omega'_{Tg}$.

The point is that $\Omega'_{Tg}$ is open in $X\times\coint{0,\infty}$.
\Prf\ If $x\in X$ and
$0\le\alpha<Tg(x)=\sum_{i\in I}g(x,i)$, let $\familyiI{\alpha_i}$ be such
that $0\le\alpha_i<g(x,i)$ for each $i\in I$ and
$\sum_{i\in I}\alpha_i=\alpha'>\alpha$.
Set $G=\bigcap_{i\in I}\{z:z\in X$, $g(z,i)>\alpha_i\}$;  then $G$ is an
open subset of $X$, and
$(x,\alpha)\in G\times\coint{0,\alpha'}\subseteq\Omega'_{Tg}$.   Thus
$(x,\alpha)\in\interior\Omega'_g$;  as $(x,\alpha)$ is arbitrary,
$\Omega'_{Tg}$ is open.\ \Qed

Since $c(\Omega'_{Tg})$ is surely less than or equal to $\int g\,d\nu$,
and $\epsilon$ is arbitrary, we have

\Centerline{$c(K)=\inf\{c(U):U\subseteq X\times\coint{0,\infty}$ is open
and $K\subseteq U\}$.}

\noindent Thus all the conditions of 432Ja are satisfied, and $c$ is a
Choquet capacity.

\medskip

\quad{\bf (ii)} We need consider only the case $c(A)<\infty$,
which is dealt with in (i-$\beta$) above, if we
take $A_n=A$ for every $n$.

\medskip

{\bf (b)(i)} For $g:X\to[-\infty,\infty]$, set

$$\eqalignno{p(g)
&=\inf\{\sum_{i\in I}\int h_id\nu_i:
h_i\in[0,\infty]^X\text{ is }\dom\nu_i\text{-measurable
for each }i\in I,\cr
&\mskip400mu |g|\le\sum_{i\in I}h_i\}\cr
&=\inf\{\int h\,d\nu:h:Y\to[0,\infty]\text{ is }\Tau\text{-measurable},\,
  \Omega_{|g|}\subseteq\Omega_{Th}\}
%&=\inf\{\int h\,d\nu:h:Y\to[0,\infty]\text{ is }\Tau\text{-measurable},\,
%  \Omega'_{|g|}\subseteq\Omega_{Th}\}\cr
=c(\Omega_{|g|})%
%=c(\Omega'_{|g|})
.\cr}$$

\noindent Then $p(\alpha g)=|\alpha|p(g)$
whenever $g\in[-\infty,\infty]^X$ and $\alpha\in\Bbb R$,
$p(g_1)\le p(g_2)$ whenever $|g_1|\le|g_2|$,
and $p(g_1+g_2)\le p(g_1)+p(g_2)$ for
all $g_1$, $g_2:X\to[-\infty,\infty]$;  so if we set
$V=\{g:g\in\BbbR^X$, $p(g)<\infty\}$, $V$ is a
solid linear subspace of $\BbbR^X$
and $p\restr V$ is a seminorm.

Suppose that $\mu$ is a Radon measure on $X$ and $\mu\le\nu_i$
for every $i\in I$.   Then $\int fd\mu\le p(f)$.
\Prf\ Because $\mu$ measures every K-analytic set (432A),
$\int fd\mu$ is defined.   If $p(f)=\infty$ then of course
$\int fd\mu\le p(f)$.
Otherwise, for any $\gamma>p(f)$, we have $\dom\nu_i$-measurable functions
$h_i:X\to[0,\infty]$ such that $f\le\sum_{i\in I}h_i$ and
$\sum_{i\in I}\int h_id\nu_i\le\gamma$.   But now $\int h_id\mu$ is defined
and less than or equal to $\int h_id\nu_i$ for each $i$
(234Qc), so
$\int fd\mu\le\sum_{i\in I}\int h_id\mu\le\gamma$.   As $\gamma$ is
arbitrary, $\int fd\mu\le p(f)$.\ \Qed

\medskip

\quad{\bf (ii)}\grheada\
In the other direction, suppose that $\gamma<p(f)$, and
set $A=\{(x,\alpha):0<\alpha<f(x)\}$;   then

\Centerline{$A=\bigcup_{q\in\Bbb Q}\{(x,\alpha):f(x)\ge q>\alpha>0\}$}

\noindent is K-analytic (422Ge,
%countable prod of K-analytic is K-analytic
422Hc,
%K-analytics closed under Souslin opn
423Ba,
%cts image of analytic is analytic
423C).
%analytic iff K-analytic + ctble network
On the other hand,
for any $h:Y\to[0,\infty]$, $A\subseteq\Omega_{Th}$ iff
$\Omega_f\subseteq\Omega_{Th}$.   So

\Centerline{$c(A)=c(\Omega_f)=p(f)>\gamma$.}

\noindent By Choquet's theorem 432K, there is a compact set $K\subseteq A$
such that $c(K)>\gamma$.   Set

\Centerline{$f_1(x)=\sup(\{0\}\cup K[\{x\}])$}

\noindent for $x\in X$.   As in
(a-i-$\gamma$) above, we have for any $i\in I$ an open set $G$ including
$L_0=\pi_1[K]$ such that $\nu_iG$ is
defined and finite, so $\chi L_0$ and $f_1$ belong to $V$.   By the
Hahn-Banach theorem (4A4Da), there is a linear functional
$\theta:V\to\Bbb R$ such that $|\theta(g)|\le p(g)$ for every $g\in V$ and
$\theta(f_1)=p(f_1)$.   Since $|\theta(g)|\le p(g_0)$ whenever
$|g|\le g_0$, $\theta$ is
order-bounded, and if $\theta^+$ is its positive part (355Eb),
we shall still have $\theta^+(g)\le p(g)$ for every $g\in V$ and
$\theta^+(f_1)=p(f_1)$.

\medskip

\qquad\grheadb\ Set $\mu_0C=\theta^+(\chi(C\cap L_0))$ for $C\subseteq X$.   Then
$\mu_0:\Cal PX\to\coint{0,\infty}$ is additive.   By 416K, there is a Radon
measure $\mu$ on $X$ such that $\mu L\ge\mu_0L$ for every compact
$L\subseteq X$ and $\mu G\le\mu_0G$ for every open $G\subseteq X$.
Now $\dom\nu_i\subseteq\dom\mu$ for every $i\in I$.
\Prf\ Suppose that $E\in\dom\nu_i$.    Let $L\subseteq X$ be
compact.   Then there is an open set $G_0\supseteq L$ such that $\nu_iG_0$
is defined and finite.   Take any $\delta>0$.
Because the subspace measure induced by $\nu_i$ on $G_0$ is
totally finite and inner regular with respect to the closed sets,
there are a closed set $F$ and an open set $G$, both
measured by $\nu_i$, such that $F\subseteq E\cap G_0\subseteq G$ and
$\nu_i(G\setminus F)\le\delta$.  In this case

\Centerline{$\mu(G\setminus F)\le\mu_0(G\setminus F)
=\theta^+(\chi(L_0\cap G\setminus F))
\le p(\chi(G\setminus F))
\le\int\chi(G\setminus F)d\nu_i\le\delta$.}

\noindent So

\Centerline{$\mu^*(E\cap G_0)
\le\mu G
\le\mu F+\delta
\le\mu_*(E\cap G_0)+\delta$;}

\noindent as $\delta$ is arbitrary, $\mu^*(E\cap G_0)=\mu_*(E\cap G_0)$
and $\mu$ measures $E\cap G_0$ (413Ef), and therefore also measures
$E\cap L=E\cap G_0\cap L$.   As
$L$ is arbitrary, $\mu$ measures $E$ (412Ja).\ \Qed

In fact, $\mu\le\nu_i$.   \Prf\ If $\nu_i$ measures $E$ and $L\subseteq X$
is compact, the arguments just above show that for any $\delta>0$
there is an open set
$G\supseteq E\cap L$ such that $\nu_iG\le\nu_iE+\delta$, so
that

\Centerline{$\mu(E\cap L)\le\mu G\le\mu_0G=\theta^+(\chi(G\cap L_0))
\le p(\chi(G\cap L_0))\le\nu_iG\le\nu_iE+\delta$.}

\noindent As $L$ and $\delta$ are arbitrary, $\mu E\le\nu_iE$.\ \Qed

\medskip

\qquad\grheadc\ To estimate $\int fd\mu$, recall that
$\theta^+(f_1)>\gamma$, while $\theta^+(\chi L_0)$ is finite.
There is therefore an $\eta>0$ such that
$\eta\theta^+(\chi L_0)\le\theta^+(f_1)-\gamma$ and
$\theta^+(f_2)\ge\gamma$, where
$f_2=(f_1-\eta\chi L_0)^+=(f_1-\eta\chi X)^+$.
For $k\in\Bbb N$
set $F_k=\pi_1[K\cap\coint{(k+1)\eta,\infty}\,]$, so that each
$F_k$ is a compact subset of $L_0$ and
$f_2\le\sum_{k=0}^m\eta\chi F_k\le f$,
where $m\in\Bbb N$ is such that
$K\subseteq X\times[0,m\eta]$.   Now

$$\eqalign{\gamma
&\le\theta^+(f_2)
\le\theta^+(\sum_{k=0}^m\eta\chi F_k)
=\eta\sum_{k=0}^m\theta^+(\chi F_k)\cr
&=\eta\sum_{k=0}^m\mu_0F_k
\le\eta\sum_{k=0}^m\mu F_k
\le\int fd\mu.\cr}$$

\qquad\grheadd\ As $\gamma$ is arbitrary,

\Centerline{$\sup\{\int fd\mu:
\mu\text{ is a Radon measure on }X,\,\mu\le\nu_i
\text{ for every }i\in I\}
\ge p(f)$}

\noindent and we must have equality.   This completes the proof.
}%end of proof of 457M

\cmmnt{
\leader{457N}{Remarks} It may not be quite obvious how close the domination
requirement `$\mu\le\nu_i$ for every $i\in I$' is to the marginal
requirement `$\nu_i=\mu\pi_i^{-1}$ for every $i\in I$', so I spell out the
correspondence.   Let
$\familyiI{(X_i,\Sigma_i,\mu_i)}$ be a family of probability
spaces, $X=\prod_{i\in I}X_i$, and $\pi_i:X\to X_i$ the canonical map
for each $i$.

\spheader 457Na For each $i\in I$ we have a (unique)
pull-back probability measure $\nu_i$ on $X$ with domain
$\{\pi_i^{-1}[E]:E\in\Sigma_i\}$ such that the image measure
$\nu_i\pi_i^{-1}$ is $\mu_i$ (see
234F).   Now it is elementary to check that, for a measure
$\mu$ on $X$, $\mu\le\nu_i$ iff $\mu\pi_i^{-1}\le\mu_i$;  and if $\mu$ is
required to be a probability measure, then $\mu\le\nu_i$ iff $\mu$ extends
$\nu_i$ iff $\mu\pi_i^{-1}$ extends $\mu_i$.

\spheader 457Nb We find also that if $\mu\le\nu_i$ for
every $i$, then there is a probability measure $\mu'$ on $X$ such that
$\mu\le\mu'$ and $\mu'$ extends $\nu_i$ for every $i$.
\Prf\ Set $\gamma=\mu X$.   If
$\gamma=1$, set $\mu'=\mu$.   Otherwise, for each
$i\in I$, set $\lambda_iE=\Bover1{1-\gamma}(\mu_iE-\mu\pi_i^{-1}[E])$ for
$E\in\Sigma_i$.   Then $\lambda_i$ is a probability measure on $X_i$;  let
$\lambda=\prod_{i\in I}\lambda_i$ be the product measure, and set
$\mu'=\mu+(1-\gamma)\lambda$.   Then

\Centerline{$\mu'\pi_i^{-1}
=\mu\pi_i^{-1}+(1-\gamma)\lambda\pi_i^{-1}
=\mu\pi_i^{-1}+(1-\gamma)\lambda_i=\mu_i$}

\noindent and $\mu'$ extends $\nu_i$ for each $i$.\ \Qed

\spheader 457Nc In the simplest intended applications, therefore, in which
we have two Radon probability spaces $(X_1,\Sigma_1,\mu_1)$ and
$(X_2,\Sigma_2,\mu_2)$ and a profit function
$f:X\to\coint{0,\infty}$, and we are looking for a Radon
probability measure $\mu$ on $X=X_1\times X_2$, with marginals $\mu_1$ and
$\mu_2$, maximising $\int fd\mu$, then we can seek to apply 457Mb with the
pull-back measures $\nu_1$ and $\nu_2$ of (a) here to see that the optimum
is

\Centerline{$\inf\{\int h_1d\mu_1+\int h_2d\mu_2:
f(x_1,x_2)\le h_1(x_1)+h_2(x_2)\Forall x_1\in X_1,\,x_2\in X_2\}$.}

\noindent If the process of part (b-ii) of the proof of 457M leads to a
more or less optimal
measure $\mu$ which is not itself a probability measure, we can increase
it to $\mu'$ with $\mu'\pi_i^{-1}$ extending $\mu_i$ for each $i$;  and in
this case we shall have $\mu'\pi_i^{-1}=\mu_i$ for each $i$, by 418I and
416E, as usual.   Of course we shall need to confirm that $\int fd\mu'$ is
defined, but in the context of 457Mb, this will automatically be so.

\spheader 457Nd There is an obvious parallel between the formulae of 457M
and that in part (b-i) of the proof of 457L.   Allowing for the change of
direction, where an infimum in 457L corresponds to a supremum in 457M,
the pattern of the duality is the same in both cases, and there is some
overlap (457Xq).   But the arguments of the two
theorems -- in particular, the proofs that we can get countably
additive measures from the finitely additive measures provided by the
Hahn-Banach theorem -- are rather different.
}%end of comment

\exercises{\leader{457X}{Basic exercises (a)}
%\spheader 457Xa
Let $X$ be a non-empty set and $\familyiI{\nu_i}$ a family of
probability measures on $X$ satisfying the conditions of Lemma 457A,
taking $\frak A=\Cal PX$ and $\frak B_i=\dom\nu_i$ for each $i$.
Suppose that there is a totally finite measure $\theta$ on $X$ such that
$\theta E$ is defined and greater than or equal to $\nu_iE$ whenever
$i\in I$ and $\nu_i$ measures $E$.   Show that there is a measure on $X$
extending every $\nu_i$.   \Hint{391E.}
%457A

\spheader 457Xb Find a set $X$ and non-negative additive functionals
$\mu_1$, $\mu_2$ defined on subalgebras of $\Cal PX$ which agree on
$\dom\mu_1\cap\dom\mu_2$ but have
no common extension to a non-negative additive functional.   \Hint{take
$\#(X)=3$.}
%457C

\spheader 457Xc Let $\frak A$ be a Boolean algebra and
$\familyiI{\nu_i}$ a family of non-negative finitely additive
functionals, each $\nu_i$ being defined on a subalgebra $\frak B_i$ of
$\frak A$.   Show that if any finite number of the $\nu_i$ have a common
extension to an additive functional on a subalgebra of $\frak A$, then
the whole family has a common extension to an additive functional on the
whole algebra $\frak A$.
%457Xb 457C

\spheader 457Xd Set $X=\{0,1,2\}$ and in the algebra $\Cal PX$ let
$\frak B_i$ be the subalgebra $\{\emptyset,\{i\},X\setminus\{i\},X\}$ for
each $i$.   Let $\nu_i:\frak B_i\to[0,1]$ be the additive functional such
that $\nu_i\{i\}=\bover12$, $\nu_iX=1$.   Show that any pair of $\nu_0$,
$\nu_1$, $\nu_2$ have a common extension to an additive functional on
$\Cal PX$, but that the three together have no such extension.
%457C

\spheader 457Xe Let $\frak A$ be a Boolean algebra, $\frak B$ a
subalgebra of $\frak A$, and $\nu:\frak B\to\coint{0,\infty}$,
$\theta:\frak A\to\coint{0,\infty}$ additive functionals such that
$\nu b\le\theta b$ for every $b\in\frak B$.   Show directly, without
using either 457D or 391F, that there is an additive functional
$\mu:\frak A\to\coint{0,\infty}$, extending $\nu$, such that
$\mu a\le\theta a$ for every $a\in\frak A$.   \Hint{first consider the
case in which $\frak A$ is the algebra generated by $\frak B\cup\{c\}$.}
%457D

\sqheader 457Xf Let $(Y_1,\frak S_1,\Tau_1,\nu_1)$ and
$(Y_2,\frak S_2,\Tau_2,\nu_2)$ be Radon probability spaces and
$X\subseteq Y_1\times Y_2$ a closed set.   Show that the following are
equiveridical:  (i) there is a measure on $X$ such that the coordinate
map from $X$ to $Y_i$ is \imp\ for both $i$;  (ii) there is a Radon
measure on $X$ such that the coordinate map from $X$ to $Y_i$ is \imp\
for both $i$;  (iii) for every compact $K\subseteq Y_1$,
$\nu_1K\le\nu^*_2(X[K])$.   \Hint{for (iii)$\Rightarrow$(ii), use 457C
to show that there is a finitely additive functional $\nu$ on $\Cal PX$
of the required type;  now observe that $\nu$ must give large mass to
compact subsets of $X$, and apply 413S.}
%457E

\sqheader 457Xg Suppose that $\frak A$ is a Boolean algebra, $\frak B$
is a subalgebra of $\frak A$ and $I\subseteq\frak A$ a finite set;  let
$\frak C$ be the subalgebra of $\frak A$ generated by $I\cup\frak B$ and
$\nu:\frak C\to\coint{0,\infty}$ a finitely additive functional.   (i)
Show that if
$\nu\restr\frak B$ is completely additive then $\nu$ is completely
additive.
(ii) Show that if $\frak A$ is Dedekind $\sigma$-complete,
$\frak B$ is a $\sigma$-subalgebra and
$\nu\restr\frak B$ is countably additive then $\nu$ is countably additive.
%457E

\spheader 457Xh Let $(X,\Sigma,\mu)$ be a probability space, $\Cal A$ a
finite family of subsets of $X$ and $\Tau$ the subalgebra of $\Cal PX$
generated by $\Sigma\cup\Cal A$.   Show that if $\nu:\Tau\to[0,1]$ is a
finitely additive functional extending $\mu$, then $\nu$ is countably
additive.
%457Xg 457E

\spheader 457Xi Let $(X,\Sigma,\mu)$ be a probability space,
$\familyiI{A_i}$ a partition of $X$ and $\familyiI{\alpha_i}$ a
family in $[0,1]$ summing to $1$.   Show that the following are
equiveridical:  (i) there is a measure $\nu$ on $X$, extending $\mu$, such
that $\nu A_i=\alpha_i$ for every $i\in I$;
(ii) there is a finitely additive functional $\nu:\Cal PX\to[0,1]$,
extending $\mu$, such that $\nu A_i=\alpha_i$ for every $i\in I$;
(iii) $\mu_*(\bigcup_{i\in J}A_i)\le\sum_{i\in J}\alpha_i$ for every
$J\subseteq I$;  (iv) $\mu^*(\bigcup_{i\in J}A_i)\ge\sum_{i\in J}\alpha_i$
for every finite $J\subseteq I$.   \Hint{for (ii)$\Rightarrow$(i) use
457Xh.}
%457Xhm 457E

\spheader 457Xj Let $X\subseteq[0,1]^2$ be a Lebesgue measurable set
such that $X\cap(E\times F)$ is not negligible for any non-negligible
sets $E$, $F\subseteq[0,1]$.   (For the construction of such sets, see
the notes to \S325.)   Show that there is a Radon measure on $X$ such
that both the coordinate projections from $X$ to $[0,1]$ are \imp, where
$[0,1]$ is given Lebesgue measure.   \Hint{show that there is a
measure-preserving bijection $\phi$ between conegligible subsets of
$[0,1]$ which is covered by $X$;   $\phi$ can be taken to be of the form
$\phi(x)=x-\alpha_n$ for $x\in E_n$.}
%457H

\spheader 457Xk Set
$X=\{(t,2t):0\le t\le\bover12\}\cup\{(t,2t-1):\bover12\le t\le 1\}$.
Show that there is a Radon measure on $X$ for which both the coordinate
maps onto $[0,1]$ are \imp, but that $X$ does not include the graph of
any measure-preserving bijection between conegligible subsets of
$[0,1]$.
%457H

\spheader 457Xl Let $X$ be the eighth-sphere
$\{x:x\in[0,1]^3,\,\|x\|=1\}$.   Show that there is a measure on $X$
such that all three coordinate maps from $X$ onto $[0,1]$ are \imp.
\Hint{265Xe.}
%457I

\spheader 457Xm Set $X=\{x:x\in[0,1]^3,\,\xi_1+\xi_2+\xi_3=\bover32\}$.
Show that there is a measure on $X$ such that all the coordinate maps
from $X$ onto $[0,1]$ are \imp.   \Hint{note that $X$ is a regular
hexagon;  try one-dimensional Hausdorff measure on its boundary.}
%457I

\spheader 457Xn Explain how to adapt the example in 457J to provide a
family $\familyiI{\mu_i}$ of probability measures on a set $X$ such that
(i) $\familyiI{\mu_i}$ is upwards-directed, in the sense of 457G
(iii) there is no measure on
$X$ extending $\mu_i$ for every $i\in I$.
%457J

\spheader 457Xo\dvAnew{2009} Let $X$ be a topological space and
$P_{\text{qR}}$ the set of quasi-Radon probability measures on $X$.
For $\mu$,
$\nu\in P_{\text{qR}}$, write $Q(\mu,\nu)$ for the set of quasi-Radon
probability measures on $X\times X$ which have marginal measures
$\mu$ on the first copy of $X$, $\nu$ on the second.   (i) For
a bounded continuous pseudometric $\rho$ on $X$, set
$\rho_{\text{W}}(\mu,\nu)
=\inf\{\int\rho(x,y)\lambda(d(x,y)):\lambda\in Q(\mu,\nu)\}$.   Show that
$\rho_{\text{W}}$ is a pseudometric on $P_{\text{qR}}$.
(ii) Show that if $X$ is completely regular and $\Rho$ is a family of
bounded pseudometrics defining
the topology of $X$, then $\{\rho_{\text{W}}:\rho\in\Rho\}$ defines the
narrow topology of $P_{\text{qR}}$.
%457L

\spheader 457Xp\dvAnew{2010} Suppose that $X$, $\familyiI{\nu_i}$ and
$c:\Cal P(X\times\coint{0,\infty})\to[0,\infty]$ are as in 457M.
(i) Show that $c$ is a
submeasure.   (ii) Show that if every $\nu_i$ is outer
regular with respect to the open sets, then $c$ is an outer regular Choquet
capacity.
%457M

\spheader 457Xq\dvAnew{2010} Show that if the metric $\rho$ is bounded,
then 457Lc can be deduced from 457Mb and part (b-i) of the proof of 457L.
%457N

\spheader 457Xr\dvAnew{2010} Let
$\langle(X_i,\frak T_i,\Sigma_i,\mu_i)\rangle_{i\le n}$ be a finite
family of Radon probability spaces,
$X=\prod_{i\in I}X_i$, and $f:X\to\Bbb R$ a bounded Baire measurable
function.   Show that

$$\eqalign{&\inf\{\int fd\mu:\mu\text{ is a Radon measure on }X
\text{ with marginal measure }\mu_i\text{ on each }X_i\}\cr
&\mskip50mu
=\sup\{\sum_{i=0}^n\int h_id\mu_i:h_i\in\ell^{\infty}(X_i)\text{ is
}\Sigma_i\text{-measurable for each }i,\cr
&\mskip200mu
\sum_{i=0}^nh_i(\xi_i)\le f(x)\text{ whenever }
x=(\xi_0,\ldots,\xi_n)\in X\}.\cr}$$

\noindent\Hint{reduce to the case in which every $X_i$ is K$_{\sigma}$.}
%457N

\leader{457Y}{Further exercises (a)}
%\spheader 457Ya
Show that for any $n\ge 2$ there are a finite set $X$
and a family $\langle\mu_i\rangle_{i\le n}$ of measures on $X$ such that
$\{\mu_i:i\le n,\,i\ne j\}$ have a common extension to a measure on $X$
for every $j\le n$, but the whole family $\{\mu_i:i\le n\}$ has no such
extension.
%457Xb 457C

\spheader 457Yb Show that the example in 457H has the property:  if
$f_i$ is a $\nu_i$-integrable real-valued function for each $i$, and
$\int f_1d\nu_1+\int f_2d\nu_2<1$, then there is an
$(x,y)\in\dom f_1\cap\dom f_2$ such that $f_1(x,y)+f_2(x,y)<1$.
%457H

\spheader 457Yc Suppose we replace the set $X$ in 457H with
$X'=X\cup\{(x,x):x\in[0,\bover12]\}$, and write $\nuprime_i$ for the
measures on $X'$ defined by the coordinate projections.   Show that
(i) if $f_i$ is a $\nuprime_i$-integrable real-valued function on $X'$
for
each $i$, and $\int f_1d\nuprime_1+\int f_2d\nuprime_2\le 1$, then there
is an
$(x,y)\in\dom f_1\cap\dom f_2$ such that $f_1(x,y)+f_2(x,y)\le 1$
(ii) there is no measure on $X'$ extending both $\nuprime_i$.
%457H

\spheader 457Yd In 457Xm, show that there are many Radon measures on
$X$ such that all the coordinate maps from $X$ onto $[0,1]$ are \imp.
%457Xm, 457I

\spheader 457Ye\dvAnew{2010}
Give an example of a compact Hausdorff space $X$, a
sequence $\sequencen{\nu_n}$ of tight probability measures on $X$,
and a K$_{\sigma}$ set $E\subseteq X$ such that

\Centerline{$\inf\{\sum_{n=0}^{\infty}\int h_nd\nu_n:
\chi E\le\sum_{n=0}^{\infty}h_n\}=1$,}

\Centerline{$\sup\{\mu E:\mu\text{ is a Radon measure on }X
  \text{ and }\mu\le\nu_n\text{ for every }n\in\Bbb N\}\le\Bover12$.}
%mt45bits 457M
}%end of exercises

\leaveitout{if $(Z,\nu)$ is the Stone space of the measure algebra
of Lebesgue measure, is the product measure on $Z\times Z$ inner regular
with respect to $\Cal PZ\tensorhat\Cal PZ$?}

\leader{457Z}{Problems} Give $[0,1]$ Lebesgue measure.

\spheader 457Za Characterize the sets $X\subseteq[0,1]^2$
for which there is a measure on $X$ such that both the projections from
$X$ to $[0,1]$ are \imp.

\spheader 457Zb Set $X=\{x:x\in[0,1]^3,\,\|x\|=1\}$.   Is there more
than one Radon measure on $X$ for which all three coordinate maps from $X$
onto $[0,1]$ are \imp?   \cmmnt{(See 457Xl, 457Yd.)}

\endnotes{
\Notesheader{457}
In the context of this section, as elsewhere (compare
391E-391G %391E 391F 391G
and 391J),
finitely additive extensions, as in 457A-457D, generally present
easier problems than countably additive extensions.   So techniques for
turning additive functionals into measures (391D, 413K, 413S, 416K,
454C, 454D, 457E, 457G, 457Lb, 457Mb, 457Xi) are very valuable.
Note that 457D offers possibilities in this
direction:  if $\theta$ there is countably additive, $\mu$ also will be
(457Xa).

457H and 457J demonstrate obstacles which can arise when seeking countably
additive extensions even when finitely additive extensions give no
difficulty.   For finitely additive extensions a problem can arise at
any finite number of measures (see 457Ya), but there is no further
obstruction with infinite families (457Xc).   For countably additive
measures we have a positive result (457G) only under very restricted
circumstances;  relaxing any of the hypotheses can lead to failure
(457J, 457Xn).   Even in the apparently concrete case in which we have
an open or closed set $X\subseteq[0,1]^2$ and we are seeking a measure
on $X$ with
prescribed image measures on each coordinate, there can be surprises
(457H, 457Xj, 457Xk), and I know of no useful description of the sets
for which such a measure can be found (457Za).

The two-dimensional case has a special feature:  when verifying the
conditions (ii) or (iii) in 457A, or the condition (ii) of 457B, it is
enough to consider only one set associated with each coordinate (457C).
Put another way, in conditions (iv) and (v) of 457A it is enough to
examine indicator functions.   This is not the case as soon as we
have three coordinates (457I).   Compare 457A(ii)-(iii) with 
the definition of `intersection number'
of an indexed family in a Boolean algebra (391H), where we had to allow
repetitions for essentially the same reason.

In 457K-457L, we can of course work with
$\tau$-additive Borel measures in place of quasi-Radon measures, as in
437M.   The essential content of 457L
is already displayed in the case of separable $X$, in which case all Borel
measures are $\tau$-additive, and we can fractionally simplify our
hypotheses;  indeed this is true whenever $X$ has measure-free weight
(438J).

The functional $\rho_{\text{W}}$ of 457K-457L
is a kind of $[0,\infty]$-valued metric;  see
4A2T for another occasion on which it would have saved explanation if the
definition of `metric' allowed infinite distances.   In 457Lb we think of
the metric $\rho$ as representing a cost to be minimised, and in 457Mb we
think of $f$ as a profit to be maximised;  since both arguments rely on the
functions being non-negative, they cannot be simply
inverted unless $\rho$ or $f$ is bounded above (as in 457Xq), and there
is a further complication from the asymmetric nature of the condition
`$\{x:f(x)\ge\alpha\}$ is K-analytic' in 457M.   However, for the primary
applications, as in 457Xr, this is not a problem.   Observe that the same
pattern has already appeared in 457A(iv)-(v).
}%end of notes

\discrpage

