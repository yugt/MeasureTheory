\frfilename{mt34.tex}
\versiondate{28.5.07}
\copyrightdate{1996}

\def\chaptername{The lifting theorem}

\newchapter{34}

Whenever we have a surjective homomorphism $\phi:P\to Q$, where $P$ and
$Q$ are mathematical structures, we can ask whether there is a right
inverse of $\phi$, a homomorphism $\psi:Q\to P$ such that $\phi\psi$ is
the identity on $Q$.   As a general rule, we expect a negative answer;
those categories in which epimorphisms always have right inverses (e.g.,
the category of linear spaces) are rather special, and elsewhere the
phenomenon is relatively rare and almost always important.   So it is
notable that we have a case of this at the very heart of the theory of
measure algebras:  for any complete probability space $(X,\Sigma,\mu)$
(in fact, for any complete strictly localizable space of non-zero
measure) the canonical homomorphism from $\Sigma$ to the measure algebra
of $\mu$ has a right inverse (341K).   This is the von Neumann-Maharam
lifting theorem.   Its proof, together with some essentially elementary
remarks, takes up the whole of of \S341.

As a first application of the theorem (there will be others in Volume
4) I apply it to one of the central problems of measure theory:  under
what circumstances will a homomorphism between measure algebras be
representable by a function between measure spaces?   Variations on this
question are addressed in \S343.   For a reasonably large proportion of
the measure spaces arising naturally in analysis, homomorphisms are
representable (343B).   New difficulties arise if we ask for
isomorphisms of measure algebras to be representable by isomorphisms of
measure spaces, and here we have to work rather hard for rather narrowly
applicable results;  but in the case of Lebesgue measure and its
closest relatives, a good deal can be done, as in 
344I-344K.  %344I 344J 344K

Returning to liftings, there are many difficult questions concerning the
extent to which liftings can be required to have special properties,
reflecting the natural symmetries of the standard measure spaces.   For
instance, Lebesgue measure is translation-invariant;  if liftings were
in any sense canonical, they could be expected to be automatically
translation-invariant in some sense.   It seems sure that there is no
canonical lifting for Lebesgue measure -- all constructions of liftings
involve radical use of the axiom of choice -- but even so we do have
many translation-invariant liftings (\S345).   We have less luck with
product spaces;  here the construction of liftings which respect the
product structure is fraught with difficulties.   I give the currently
known results in \S346.


\discrpage


