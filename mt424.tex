\frfilename{mt424.tex}
\versiondate{21.3.08}
\copyrightdate{1998}

\def\NN{\BbbN^{\Bbb N}}

\def\chaptername{Descriptive set theory}
\def\sectionname{Standard Borel spaces}

\newsection{424}

This volume is concerned with topological measure spaces, and it will
come as no surprise that the topological properties of
Polish spaces are central to the theory.   But even from the
point of view of unadorned measure theory, not looking for
topological structures on the underlying spaces, it turns
out that the Borel algebras of Polish spaces have a very
special position.   It will be useful later on to be able to
refer to some fundamental facts concerning them.

\leader{424A}{Definition} Let $X$ be a set and $\Sigma$ a
$\sigma$-algebra of subsets of $X$.   We say that $(X,\Sigma)$ is a {\bf
standard Borel space} if there is a Polish topology on $X$ for which
$\Sigma$ is the algebra of Borel sets.

\cmmnt{{\bf Warning!}  Many authors reserve the phrase `standard Borel
space' for the case in which $X$ is uncountable.   I have seen the
phrase `Borel space' used for what I call a `standard Borel space'.}

\leader{424B}{Proposition} (a) If $(X,\Sigma)$ is a standard Borel
space, then $\Sigma$ is countably generated as $\sigma$-algebra of sets.

(b) If $\familyiI{(X_i,\Sigma_i)}$ is a countable family of standard
Borel spaces, then
$(\prod_{i\in I}X_i,\Tensorhat_{i\in I}\Sigma_i)$\cmmnt{ (definition:
254E)} is a standard Borel space.

(c) Let $(X,\Sigma)$ and $(Y,\Tau)$ be standard Borel spaces and
$f:X\to Y$ a $(\Sigma,\Tau)$-measurable surjection.   Then

\quad(i) if $E\in\Sigma$ is such that
$f[E]\cap f[X\setminus E]=\emptyset$, then $f[E]\in\Tau$;

\quad(ii) $\Tau=\{F:F\subseteq Y,\,f^{-1}[F]\in\Sigma\}$;

\quad(iii) if $f$ is a bijection it is an isomorphism.

(d) Let $(X,\Sigma)$ and $(Y,\Tau)$ be standard Borel spaces and
$f:X\to Y$ a $(\Sigma,\Tau)$-measurable injection.   Then
$Z=f[X]\in\Tau$ and
$f$ is an isomorphism between $(X,\Sigma)$ and $(Z,\Tau_Z)$, where
$\Tau_Z$ is the subspace $\sigma$-algebra.

\proof{{\bf (a)} Let $\frak T$ be a Polish topology on $X$ such that
$\Sigma$ is the algebra of Borel sets.   Then $\frak T$ has a countable
base $\Cal U$, which generates $\Sigma$ (4A3Da/4A3E).

\medskip

{\bf (b)} For each $i\in I$ let $\frak T_i$ be a Polish topology on
$X_i$ such that $\Sigma_i$ is the algebra of $\frak T_i$-Borel sets.
Then $X=\prod_{i\in I}\frak T_i$, with the product topology
$\frak T$, is Polish (4A2Qc).   By 4A3Dc/4A3E,
$\Sigma=\Tensorhat_{i\in I}\Sigma_i$ is just
the Borel $\sigma$-algebra of $X$, so $(X,\Sigma)$ is a standard Borel space.

\medskip

{\bf (c)} Let $\frak T$, $\frak S$ be Polish topologies on $X$, $Y$
respectively for which $\Sigma$ and $\Tau$ are the Borel
$\sigma$-algebras.   Then $f$ is Borel measurable.

\medskip

\quad{\bf (i)} By 423Eb and 423Gb, $f[E]$ and $f[X\setminus E]$ are
analytic subsets of $Y$.   But they are complementary, so they are Borel
sets, by 423Fa.

\medskip

\quad{\bf (ii)} $f^{-1}[F]\in\Sigma$ for every $F\in\Tau$, just
because $f$ is measurable.   On the other hand, if $F\subseteq Y$ and
$E=f^{-1}[F]\in\Sigma$, then $F=f[E]\in\Tau$ by (i).

\medskip

\quad{\bf (iii)} follows at once.

\medskip

{\bf (d)} Give $X$ and $Y$ Polish topologies for which $\Sigma$, $\Tau$
are the Borel $\sigma$-algebras.   By 423Ib, $f[E]\in\Tau$ for every
$E\in\Sigma$;
in particular, $Z=f[X]$ belongs to $\Tau$.   Also $f^{-1}[F]\in\Sigma$
for every $F\in\Tau_Z$, so $f$ is an isomorphism between $(X,\Sigma)$
and $(Z,\Tau_Z)$.
}%end of proof of 424B

\leader{424C}{Theorem} Let $(X,\Sigma)$ be a standard Borel space.

(a) If $X$ is countable then $\Sigma=\Cal PX$.

(b) If $X$ is uncountable then $(X,\Sigma)$ is isomorphic to
$(\NN,\Cal B(\NN))$, where $\Cal B(\NN)$ is the algebra of Borel subsets of
$\NN$.

\proof{ Let $\frak T$ be a Polish topology on $X$ such that $\Sigma$ is
its Borel $\sigma$-algebra.

\medskip

{\bf (a)} Every singleton subset of $X$ is closed, so must belong to
$\Sigma$.   If $X$ is countable, every subset of $X$ is a countable
union of singletons, so belongs to $\Sigma$.

\medskip

{\bf (b)} ({\smc Rao \& Srivastava 94}) The strategy of the proof is to
find Borel sets $Z\subseteq
X$, $W\subseteq\NN$ such that $(Z,\Sigma_Z)\cong(\NN,\Cal B(\NN))$ and
$(W,\Cal B(W))\cong(X,\Sigma)$ (writing
$\Sigma_Z$, $\Cal B(W)$ for the subspace $\sigma$-algebras), and use a
form of the Schr\"oder-Bernstein theorem.

\medskip

\quad{\bf (i)} By 423J, $X$ has a subset $Z$ homeomorphic to $\NN$;  let
$h:\NN\to Z$ be a homeomorphism.   By 424Bd, $h$ is an isomorphism
between $(\NN,\Cal B(\NN))$ and $(Z,\Sigma_Z)$.

\medskip

\quad{\bf (ii)} Let $\sequencen{U_n}$ run over a base for the topology
of $X$.   Define $g:X\to\{0,1\}^{\Bbb N}\subseteq\NN$ by setting
$g(x)=\sequencen{\chi U_n(x)}$ for every $x\in\Bbb N$.   Then $g$ is
injective, because $X$ is Hausdorff.   Also $g$ is Borel measurable,
by 4A3D(c-ii).   By 424Bd, $g$ is an isomorphism between $(X,\Sigma)$ and
$(W,\Cal B(W))$, where $W=g[X]$ belongs to $\Cal B(\NN)$.

\medskip

\quad{\bf (iii)} We have $Z\in\Sigma$,
$W\in\Cal B(\NN)$ such that $(Z,\Sigma_Z)\cong(\NN,\Cal B(\NN))$ and
$(W,\Cal B(W))\cong(X,\Sigma)$.   By 344D,
$(X,\Sigma)\cong(\NN,\Cal B(\NN))$, as claimed.
}%end of proof of 424C

\leader{424D}{Corollary} (a) If $(X,\Sigma)$ and $(Y,\Tau)$ are standard
Borel spaces and $\#(X)=\#(Y)$, then $(X,\Sigma)$ and $(Y,\Tau)$ are
isomorphic.

(b) If $(X,\Sigma)$ is an uncountable standard Borel space then
$\#(X)=\#(\Sigma)=\frak c$.

\proof{ These follow immediately from 424C, if we recall that
$\#(\Cal B(\NN))=\frak c$ (4A3Fb).
}%end of proof of 424D

\leader{424E}{Proposition} Let $X$ be a set and $\Sigma$ a
$\sigma$-algebra of subsets of $X$;  suppose that $(X,\Sigma)$ is
countably separated in the sense that there is a countable set
$\Cal E\subseteq\Sigma$ separating the points of $X$.   If
$A\subseteq X$ is
such that $(A,\Sigma_A)$ is a standard Borel space, where $\Sigma_A$ is
the subspace $\sigma$-algebra, then $A\in\Sigma$.

\proof{ Give $A$ a Polish topology $\frak T$ such that $\Sigma_A$ is the
Borel $\sigma$-algebra of $A$, and let $\frak S$ be the topology on $X$
generated
by $\Cal E\cup\{X\setminus E:E\in\Cal E\}$.   Then $\frak S$ is
second-countable (4A2Oa), so has a countable network (4A2Oc),
and is Hausdorff because $\Cal E$ separates the points of $X$.   The
identity map from $A$ to $X$ is Borel measurable for $\frak T$ and
$\frak S$, so
423Ib tells us that $A$ is $\frak S$-Borel;  but of course the
$\frak S$-Borel $\sigma$-algebra is just the $\sigma$-algebra generated
by $\Cal E$ (4A3Da), so is included in $\Sigma$.
}%end of proof of 424E

\leader{424F}{Corollary} Let $X$ be a Polish space and $A\subseteq X$
any set which is not Borel.   Let $\Cal B(A)$ be the Borel
$\sigma$-algebra of
$A$.   Then $(A,\Cal B(A))$ is not a standard Borel space.

\leader{424G}{Proposition} Let $(X,\Sigma)$ be a standard Borel space.
Then $(E,\Sigma_E)$ is a standard Borel space for every
$E\in\Sigma$\cmmnt{, writing $\Sigma_E$ for the subspace
$\sigma$-algebra}.

\proof{ Let $\frak T$ be a Polish topology on $X$ for which $\Sigma$ is
the Borel $\sigma$-algebra.   Then there is a Polish topology
$\frak T'\supseteq\frak T$ for which $E$ is closed (4A3I), therefore
itself Polish in the subspace topology $\frak T'_E$ (4A2Qd).   But
$\frak T'$ and $\frak T$ have the same Borel sets (423Fb), so $\Sigma_E$
is just the Borel $\sigma$-algebra of $E$ for $\frak T'_E$, and
$(E,\Sigma_E)$ is a standard Borel space.
}%end of proof of 424G

\leader{*424H}{}\cmmnt{ For the full strength of a theorem in \S448
%448P
we need a remarkable result concerning group actions on Polish spaces.

\medskip

\noindent}{\bf Theorem}\cmmnt{ ({\smc Becker \& Kechris 96})} Let $G$
be a Polish group, $(X,\frak T)$ a Polish space and $\action$ a Borel
measurable action of $G$ on $X$.   Then there is a Polish topology
$\frak T'$ on $X$, yielding the same Borel sets as $\frak T$, such that
the action is continuous for $\frak T'$ and the given topology of $G$.

\proof{{\bf (a)} Fix on a right-translation-invariant metric $\rho$ on
$G$ defining the topology of $G$ (4A5Q), and let $D$ be a countable
dense subset of $G$;  write $e$ for the identity of $G$.   Let
$Z$ be the set of $1$-Lipschitz functions from $G$ to $[0,1]$, that is,
functions $f:G\to[0,1]$ such that $|f(g)-f(h)|\le\rho(g,h)$ for all $g$,
$h\in G$.   Then $Z$, with the topology of pointwise convergence
inherited from the
product topology of $[0,1]^G$, is a compact metrizable space.   \Prf\ It
is a closed subset of $[0,1]^G$, so is a compact Hausdorff space.
Writing $q(f)=f\restr D$ for $f\in Z$, $q:Z\to[0,1]^D$ is injective,
because $D$ is dense and every member of $Z$ is continuous;   but this
means that $Z$ is homeomorphic to $q[Z]$, which is metrizable, by
4A2Pc.\ \Qed

We see also that the Borel $\sigma$-algebra of $Z$ is the
$\sigma$-algebra
generated by sets of the form $W_{g\alpha}=\{f:f(g)<\alpha\}$ where
$g\in G$ and $\alpha\in[0,1]$.   \Prf\ This $\sigma$-algebra contains
every set of the form $\{f:f\in Z,\,\alpha<f(g)<\beta\}$, where $g\in G$
and $\alpha$, $\beta\in\Bbb R$;  since these sets
generate the topology of $Z$, the $\sigma$-algebra they generate is the
Borel $\sigma$-algebra of $Z$, by 4A3Da.\ \Qed

\medskip

{\bf (b)} There is a continuous action of $G$ on $Z$ defined by setting

\Centerline{$(g\action_rf)(h)=f(hg)$}

\noindent for $f\in Z$ and $g$, $h\in G$.   \Prf\ 
(i) If $f\in Z$ and $g$, $h_1$, $h_2\in G$, then

\Centerline{$|(g\action_r f)(h_1)-(g\action_r f)(h_2)|
=|f(h_1g)-f(h_2g)|
\le\rho(h_1g,h_2g)=\rho(h_1,h_2)$}

\noindent because $\rho$ is right-translation-invariant.   So 
$g\action_rf\in Z$ for every $f\in Z$, $g\in G$.   (ii)
As in 4A5Cc-4A5Cd, $\action_r$ is an action of $G$ on $Z$.
(iii) Suppose that
$g_0\in G$, $f_0\in Z$, $h\in G$ and $\epsilon>0$.   Set

\Centerline{$V=\{g:g\in G,\,\rho(hg,hg_0)<\bover12\epsilon\}$,}

\Centerline{$W=\{f:f\in Z,\,|f(hg_0)-f_0(hg_0)|<\bover12\epsilon\}$.}

\noindent Then $V$ is an open set in $G$ containing $g_0$ (because
$g\mapsto\rho(hg,hg_0)$ is continuous) and $W$ is an open set in $Z$
containing $f_0$.   If $g\in V$ and $f\in W$,

$$\eqalign{|(g\action_r f)(h)-(g_0\action_r f_0)(h)|
&=|f(hg)-f_0(hg_0)|\cr
&\le|f(hg)-f(hg_0)|+|f(hg_0)-f_0(hg_0)|\cr
&\le\rho(hg,hg_0)+\Bover12\epsilon
\le\epsilon.\cr}$$

\noindent As $f_0$, $g_0$, $\epsilon$ are arbitrary, the map
$(g,f)\mapsto (g\action_r f)(h)$ is continuous;  as $h$ is arbitrary, the
map $(g,f)\mapsto g\action_r f$ is continuous.\ \Qed

\medskip

{\bf (c)} Let $\Cal B(X)$ be the Borel $\sigma$-algebra of $X$.   For
$x\in X$ and $B\in\Cal B(X)$, set

\Centerline{$P_B(x)=\{g:g\in G,\,g\action x\in B\}$,}

\Centerline{$Q_B(x)=\bigcup\{V:V\subseteq G$ is open,
$V\setminus P_B(x)$ is meager$\}$,}

\Centerline{$f_B(x)(g)
=\inf(\{1\}\cup\{\rho(g,h):h\in G\setminus Q_B(x)\})$}

\noindent for $g\in G$.   It is easy to check that

\Centerline{$f_B(x)(g')\le\rho(g,g')+f_B(x)(g')$}

\noindent for all $g$, $g'\in G$, so that every $f_B(x)$ belongs to $Z$.

Every $P_B(x)$ is a Borel set, because $\action$ is Borel measurable, so
has the Baire property in $X$ (4A3Rb).
Because $X$ is a Baire space (4A2Ma), $Q_B(x)\subseteq\overline{P_B(x)}$
(4A3Ra).

Let $\Cal V$ be a countable base for the topology of $G$ containing $G$.

\medskip

{\bf (d)} For each $B\in\Cal B(X)$, the map $f_B:X\to Z$ is Borel
measurable.   \Prf\ Because the Borel $\sigma$-algebra of $Z$ is
generated by the
sets $W_{g\alpha}$ of (a) above, it is enough to show that

\Centerline{$\{x:f_B(x)\in W_{g\alpha}\}=\{x:f_B(x)(g)<\alpha\}$}

\noindent always belongs to $\Cal B(X)$, because $\{W:f_B^{-1}[W]\in\Cal
B\}$ is surely a $\sigma$-algebra of subsets of $Z$.   But if $\alpha>1$
this set is $X$, while if $\alpha\le 1$ it is

$$\eqalignno{\{x:\text{ there is some }&h\in X\setminus Q_B(x)
  \text{ such that }\rho(g,h)<\alpha\}\cr
&=\{x:\text{ there is some }V\in\Cal V
  \text{ such that }V\setminus Q_B(x)\ne\emptyset\cr
&\qquad\qquad\qquad\text{ and }\rho(g,h)<\alpha\text{ for every }h\in
V\}\cr
&=\bigcup_{V\in\Cal V'}\{x:V\not\subseteq Q_B(x)\}\cr
\noalign{\noindent (where $\Cal V'=\{V:V\in\Cal V,\,\rho(g,h)<\alpha$
for every $h\in V\}$)}
&=\bigcup_{V\in\Cal V'}\{x:V\setminus P_B(x)\text{ is not
meager}\}.\cr}$$

\noindent But for any fixed $V\in\Cal V$,

\Centerline{$\{x:V\setminus P_B(x)$ is not meager$\}
=X\setminus\{x:W[\{x\}]$ is meager$\}$}

\noindent where

\Centerline{$W=\{(y,g):y\in X,\,g\in V,\,g\action y\in X\setminus B\}$}

\noindent is a Borel subset of $X\times G$, because $\action$ is
supposed to be Borel measurable;  and therefore
$W\in\Cal B(X)\tensorhat\Cal B(G)$, writing $\Cal B(G)$ for the Borel
$\sigma$-algebra of $G$ (4A3G).
Now $\Cal B(G)\subseteq\widehat{\Cal B}(G)$, the Baire-property algebra
of $G$ (4A3Rb), so $W\in\Cal B(X)\tensorhat\widehat{\Cal B}(G)$.
By 4A3Rc, the quotient
algebra $\widehat{\Cal B}/\Cal M$ has a countable order-dense set,
where $\Cal M$ is the $\sigma$-ideal of meager sets, so
4A3Sa tells us that $\{x:W[\{x\}]$ is meager$\}$ is a Borel subset of
$X$.   Accordingly
$\{x:V\setminus P_B(x)$ is not meager$\}$ is Borel for every $V$, and
$\{x:f_B(x)(g)<\alpha\}$ is Borel.\ \Qed

\medskip

{\bf (e)} If $g\in G$, $B\in\Cal B(X)$ and $x\in X$, then
$g\action_rf_B(x)=f_B(g\action x)$.   \Prf\

\Centerline{$P_B(g\action x)=\{h:h\action(g\action x)\in B\}
=\{h:hg\in P_B(x)\}
=P_B(x)g^{-1}$.}

\noindent Because the map $h\mapsto hg^{-1}:G\to G$ is a homeomorphism,
$Q_B(g\action x)=Q_B(x)g^{-1}$;  because it is an isometry,

$$\eqalign{f_B(g\action x)(h)
&=\min(1,\rho(h,X\setminus Q_B(g\action x)))
=\min(1,\rho(h,X\setminus Q_B(x)g^{-1}))\cr
&=\min(1,\rho(hg,X\setminus Q_B(x)))
=f_B(x)(hg)
=(g\action_rf_B(x))(h)\cr}$$

\noindent for every $h\in G$.\ \Qed

\medskip

{\bf (f)} Let $\sequence{m}{B_{0m}}$ run over a base for
$\frak T$ containing $X$.   We can now find a countable set
$\Cal E\subseteq\Cal B(X)$ such
that (i) the topology $\frak T^*$ generated by $\Cal E$ is a Polish
topology finer than $\frak T$ (ii) $f_E$ is $\frak T^*$-continuous for
every $E\in\Cal E$ (iii) $X\in\Cal E$.   \Prf\ Let $\Cal W$ be a
countable base for the topology of $Z$.   Enumerate
$\Bbb N\times\Bbb N\times\Cal W$ as $\sequencen{(k_n,m_n,W_n)}$ in such
a way that $k_n\le n$ for every $n$.   Having chosen Borel sets
$B_{ij}\subseteq X$
for $i\le n$, $j\in\Bbb N$ in such a way that the topology $\frak S_n$
generated by $\{B_{ij}:i\le n,\,j\in\Bbb N\}$ is a Polish topology finer
than $\frak T$, consider the set

\Centerline{$C_n=\{x:f_{B_{k_n,m_n}}(x)\in W_n\}$.}

\noindent This is $\frak T$-Borel, therefore $\frak S_n$-Borel, so by
4A3H there is a Polish topology $\frak S_{n+1}\supseteq\frak S_n$ such
that $C_n\in\frak S_{n+1}$.   Let $\sequence{m}{B_{n+1,m}}$ run over a
base for $\frak S_{n+1}$;  by 423Fb, every $B_{n+1,m}$ belongs to
$\Cal B(X)$.   Continue.

Let $\frak T^*$ be the topology generated by
$\Cal E=\{B_{ij}:i,\,j\in\Bbb N\}$.
By 4A2Qf, $\frak T^*$ is Polish, because it is the topology
generated by $\bigcup_{n\in\Bbb N}\frak S_n$.   If $W\in\Cal W$ and
$E\in\Cal E$ there are $i$, $j\in\Bbb N$ such that $B=B_{ij}$ and an
$n\in\Bbb N$ such that $(i,j,W)=(k_n,m_n,W_n)$;  now

\Centerline{$f_E^{-1}[W]=C_n\in\frak S_{n+1}\subseteq\frak T^*$.}

\noindent As $W$ is arbitrary, $f_E$ is $\frak T^*$-continuous.   Also

\Centerline{$X\in\{B_{0m}:m\in\Bbb N\}\subseteq\Cal E$,}

\noindent as required.\ \Qed

\medskip

{\bf (g)} Define $\theta:X\to Z^{\Cal E}$ by setting
$\theta(x)(E)=f_E(x)$ for $x\in X$ and $E\in\Cal E$.   Then $\theta$ is
injective.   \Prf\ Suppose that $x$, $y\in X$ and that $x\ne y$.   For
every $g\in G$,

\Centerline{$g^{-1}\action(g\action x)=x\ne y=g^{-1}\action(g\action
y)$,}

\noindent so $g\action x\ne g\action y$ and there is some $m\in\Bbb N$
such that $g\action x\in B_{0m}$ while $g\action y\notin B_{0m}$, that
is, $g\in P_{B_{0m}}(x)\setminus P_{B_{0m}}(y)$.   Thus
$G=\bigcup_{m\in\Bbb N}P_{B_{0m}}(x)\setminus P_{B_{0m}}(y)$;  by
Baire's theorem, there is some $m\in\Bbb N$ such that
$P_{B_{0m}}(x)\setminus P_{B_{0m}}(y)$ is non-meager.    Because
$Q_{B_{0m}}(x)\symmdiff P_{B_{0m}}(x)$ and
$Q_{B_{0m}}(y)\symmdiff P_{B_{0m}}(y)$ are both meager,

\Centerline{$\{g:f_{B_{0m}}(x)(g)>0\}=Q_{B_{0m}}(x)\ne Q_{B_{0m}}(y)
=\{g:f_{B_{0m}}(y)(g)>0\}$,}

\noindent and

\Centerline{$\theta(x)(B_{0m})=f_{B_{0m}}(x)\ne f_{B_{0m}}(y)
=\theta(y)(B_{0m})$.}

\noindent So $\theta(x)\ne\theta(y)$.\ \QeD\  Because $f_E$ is $\frak
T^*$-continuous for every $E\in\Cal E$, $\theta$ is
$\frak T^*$-continuous.

\medskip

{\bf (h)} Let $\frak T'$ be the topology on $X$ induced by $\theta$;
that is, the topology which renders $\theta$ a homeomorphism between $X$
and $\theta[X]$.   Because $\theta[X]\subseteq Z^{\Cal E}$ is separable
and metrizable, $\frak T'$ is separable and metrizable.   Because
$\theta$ is
$\frak T^*$-continuous, $\frak T'\subseteq\frak T^*$.

\medskip

{\bf (i)} The action of $G$ on $X$ is continuous for the given topology
$\frak S$ on $G$ and $\frak T'$ on $X$.   \Prf\ For any $E\in\Cal E$,

\Centerline{$(g,x)\mapsto\theta(g\action x)(E)
=f_E(g\action x)=g\action_rf_E(x)$}

\noindent ((e) above) is $\frak S\times\frak T'$-continuous because the
action of $G$ on $Z$ is continuous ((b) above) and $f_E:X\to Z$ is
$\frak T'$-continuous (by the definition of $\frak T'$).   But this
means that $(g,x)\mapsto\theta(g\action x)$ is
$\frak S\times\frak T'$-continuous, so that $(g,x)\mapsto g\action x$ is
$(\frak S\times\frak T',\frak T')$-continuous.\ \Qed

\medskip

{\bf (j)} Let $\sigma$ be a complete metric on $G$ defining the topology
$\frak S$, and $\tau$ a complete metric on $X$ defining the topology
$\frak T^*$.   (We do not need to relate $\sigma$ to $\rho$ in any way
beyond the fact that they both give rise to the same topology $\frak
S$.)   For $E\in\Cal E$, $V\in\Cal V$ and $n\in\Bbb N$ let $S_{EVn}$ be
the set of those $\phi\in Z^{\Cal E}$ such that {\it either}
$\phi(E)(g)=0$ for every $g\in V$ {\it or} there is an $F\in\Cal E$ such
that $F\subseteq E$, $\diam_{\tau}(F)\le 2^{-n}$ and $\phi(F)(g)>0$ for
some $g\in V$.   Then $S_{EVn}$ is the union of a closed set and an open
set, so is a G$_{\delta}$ set in $Z^{\Cal E}$ (4A2C(a-i)).
Consequently

\Centerline{$Y=\{\phi:\phi\in Z^{\Cal E},\,\phi(X)=\chi G\}
\cap\overline{\theta[X]}\cap\bigcap_{E\in\Cal E,V\in\Cal V,n\in\Bbb
N}S_{EVn}$}

\noindent is a G$_{\delta}$ subset of $Z^{\Cal E}$, being the
intersection of countably many G$_{\delta}$ sets.

\medskip

{\bf (k)} $\theta[X]\subseteq Y$.   \Prf\ Let $x\in X$.   (i) $P_X(x)=G$
so $Q_X(x)=G$ and $\theta(x)(X)=f_X(x)=\chi G$.   (ii) Of course
$\theta(x)\in\overline{\theta[X]}$.   (iii) Suppose that $E\in\Cal E$,
$V\in\Cal V$ and $n\in\Bbb N$.   If $Q_E(x)\cap V=\emptyset$ then
$\theta(x)(E)(g)=f_E(x)(g)=0$ for every $g\in V$, and
$\theta(x)(E)\in S_{EVn}$.   Otherwise, $V\cap P_E(x)$ is non-meager.
But

\Centerline{$E
=\bigcup\{F:F\in\Cal E,\,F\subseteq E,\,\diam_{\tau}(F)\le 2^{-n}\}$,}

\noindent so

\Centerline{$P_E(x)=\bigcup\{P_F(x):F\in\Cal E,\,F\subseteq E,\,
\diam_{\tau}(F)\le 2^{-n}\}$;}

\noindent because $\Cal E$ is countable, this is a countable union and
there is an $F\in\Cal E$ such that $F\subseteq E$,
$\diam_{\tau}(F)\le 2^{-n}$ and $P_F(x)\cap V$ is non-meager.   In this
case $Q_F(x)\cap V$
is non-empty and $\theta(x)(F)=f_F(x)$ is non-zero at some point of
$V$;  thus again $\theta(x)\in S_{EVn}$.   As $E$, $V$ and $n$ are
arbitrary, we have the result.\ \Qed

\medskip

{\bf (l)} (The magic bit.)   $Y\subseteq\theta[X]$.   \Prf\ Take any
$\phi\in Y$.   Choose $\sequencen{E_n}$ in $\Cal E$, $\sequencen{V_n}$
in $\Cal V$ and $\sequencen{\tilde g_n}$ in $G$ as follows.   $E_0=X$
and $V_0=G$.   Given that $\phi(E_n)$ is non-zero at some point of
$V_n$, then, because $\phi\in S_{E_nV_nn}$, there is an
$E_{n+1}\in\Cal E$ such that $E_{n+1}\subseteq E_n$,
$\diam_{\tau}(E_{n+1})\le 2^{-n}$
and $\phi(E_{n+1})$ is non-zero at some point of $V_n$;  say
$\tilde g_n\in V_n$ is such that $\phi(E_{n+1})(\tilde g_n)>0$.
Now we can
find a $V_{n+1}\in\Cal V$ such that $\tilde g_n\in V_{n+1}\subseteq V_n$
and $\diam_{\sigma}(V_{n+1})\le 2^{-n}$.   Continue.

We are supposing also that $\phi\in\overline{\theta[X]}$, so we have
a sequence $\sequence{i}{x_i}$ in $X$ such that
$\sequence{i}{\theta(x_i)}\to\phi$, that is,
$\sequence{i}{f_E(x_i)}\to\phi(E)$ for every $E\in\Cal E$.   In
particular,

\Centerline{$\lim_{i\to\infty}f_{E_{n+1}}(x_i)(\tilde
g_n)=\phi(E_{n+1})(\tilde g_n)>0$}

\noindent for every $n$.    Let
$\sequence{n}{i_n}$ be a strictly increasing sequence in $\Bbb N$ such
that $f_{E_{n+1}}(x_{i_n})(\tilde g_n)>0$ for every $n$.   Then

\Centerline{$\tilde g_n\in V_{n+1}\cap Q_{E_{n+1}}(x_{i_n})\subseteq
V_{n+1}\cap\overline{P_{E_{n+1}}(x_{i_n})}$;}

\noindent there is therefore some $g_n\in V_{n+1}\cap
P_{E_{n+1}}(x_{i_n})$, so that $g_n\action x_{i_n}\in E_{n+1}$.

$\sequencen{V_n}$ is a non-increasing sequence of sets with
$\sigma$-diameters
converging to $0$, so $\sequencen{g_n}$ is a Cauchy sequence for the
complete metric $\sigma$.   Similarly, $\sequencen{g_n\action x_{i_n}}$
is a Cauchy sequence for the complete metric $\tau$, because
$\diam_{\tau}(E_{n+1})\le 2^{-n}$.   We therefore have $g\in G$,
$y\in X$ such that $\sequencen{g_n}\to g$ for $\frak S$ and
$\sequencen{g_n\action x_{i_n}}\to y$ for $\frak T^*$.   In this case,
$\sequencen{g_n\action x_{i_n}}\to y$ for the coarser topology
$\frak T'$, while
$\sequencen{g_n^{-1}}\to g^{-1}$ for $\frak S$.   Because the action is
$(\frak S\times\frak T',\frak T')$-continuous,

\Centerline{$\sequencen{x_{i_n}}
=\sequencen{g_n^{-1}\action(g_n\action x_{i_n})}\to g^{-1}\action y$}

\noindent for $\frak T'$.   But $\theta$ is continuous for $\frak T'$,
by the definition of $\frak T'$, so

\Centerline{$\theta(g^{-1}\action y)
=\lim_{n\to\infty}\theta(x_{i_n})=\phi$,}

\noindent and $\phi\in\theta[X]$.   As $\phi$ is arbitrary,
$Y\subseteq\theta[X]$.\ \Qed

\medskip

{\bf (m)} Thus $\theta[X]=Y$ is a G$_{\delta}$ set in the compact metric
space $Z^{\Cal E}$, and is a Polish space in its induced topology
(4A2Qd).   But this means that $(X,\frak T')$, which is homeomorphic to
$\theta[X]$, is also Polish.

\medskip

{\bf (n)} I have still to check that $\frak T'$ has the same Borel sets
as $\frak T$.   But $\frak T$, $\frak T^*$ and $\frak T'$ are all Polish
topologies and $\frak T^*$ is finer than both the other two.   By 423Fb,
$\frak T^*$ has the same Borel sets as either of the others.

This completes the proof.
}%end of proof of 424H

\exercises{\leader{424X}{Basic exercises $\pmb{>}$(a)}
%\spheader 424Xa
Let $\family{i}{I}{(X_i,\Sigma_i)}$ be a countable family of standard
Borel spaces, and $(X,\Sigma)$ their direct sum, that is,
$X=\{(x,i):i\in I,\,x\in X_i\}$, $\Sigma=\{E:E\subseteq X,\,\{x:(x,i)\in
E\}\in\Sigma_i$ for every $i\}$.   Show that $(X,\Sigma)$ is a standard
Borel space.
%424B

\sqheader 424Xb Let $(X,\Sigma)$ be a standard Borel space and $\Tau$ a
countably generated $\sigma$-subalgebra of $\Sigma$.   Show that there
is an analytic Hausdorff space $Z$ such that $\Tau$ is isomorphic to the
Borel $\sigma$-algebra of $Z$.   \Hint{by 4A3I, we can suppose that $X$
is a Polish space and $\Tau$ is generated by a sequence of
open-and-closed
sets, corresponding to a continuous function from $X$ to
$\{0,1\}^{\Bbb N}$.}

\sqheader 424Xc Let $(X,\Sigma)$ be a standard Borel space and $\Tau_1$,
$\Tau_2$ two countably generated $\sigma$-subalgebras of $\Sigma$ which
separate the same points, in the sense that if $x$, $y\in X$ then there
is an $E\in\Tau_1$ such that $x\in E$ and $y\in X\setminus E$ iff there is
an $E'\in\Tau_2$ such that $x\in E'$ and $y\in X\setminus E'$.   Show that
$\Tau_1=\Tau_2$.   \Hint{424Xb, 423Fb.}   In particular, if $\Tau_1$
separates the points of $X$ then $\Tau_1=\Sigma$.
%424Xb

\spheader 424Xd Let $\frak A$ be a Dedekind $\sigma$-complete Boolean
algebra.   Show that $\frak A$ is isomorphic to the Borel
$\sigma$-algebra of an
analytic Hausdorff space iff it is isomorphic to a countably generated
$\sigma$-subalgebra of the Borel $\sigma$-algebra of $[0,1]$.
%424Xb

\sqheader 424Xe Let $U$ be a separable Banach space.   Show that its
Borel $\sigma$-algebra is generated, as $\sigma$-algebra, by the sets of
the form
$\{u:h(u)\le\alpha\}$ as $h$ runs over the dual $U^*$ and $\alpha$ runs
over $\Bbb R$.

\sqheader 424Xf Let $X$ be a compact metrizable space.   Show that the
Borel $\sigma$-algebra of $C(X)$ (the Banach space of continuous
real-valued
functions on $X$) is generated, as $\sigma$-algebra,  by the sets
$\{u:u\in C(X),\,u(x)\ge\alpha\}$ as $x$ runs over $X$ and $\alpha$ runs
over $\Bbb R$.
%424Xe

\spheader 424Xg Let $(X,\Sigma)$ be a standard Borel space, $Y$ any set,
and $\Tau$ a $\sigma$-algebra of subsets of $Y$.
Write $\Tau^*$ for the $\sigma$-algebra of subsets of $Y$ generated by
$\Cal S(\Tau)$, where $\Cal S$ is Souslin's operation.
Let $W\in\Cal S(\Sigma\tensorhat\Tau)$.
Show that $W[X]\in\Cal S(\Tau)$ and that there is a
$(\Tau^*,\Sigma)$-measurable function $f:W[X]\to X$ such that
$(f(y),y)\in W$ for every $y\in W[X]$.
\Hint{423M.}

\spheader 424Xh Let $(X,\Sigma)$ be a standard Borel space, $Y$ any set,
and $\Tau$ a countably generated $\sigma$-algebra of subsets of $Y$.
Let $f:X\to Y$ be a $(\Sigma,\Tau)$-measurable function, and write
$\Tau^*$ for the $\sigma$-algebra of subsets of $Y$ generated by
$\Cal S(\Tau)$, where $\Cal S$ is Souslin's
operation.    Show that there is a
$(\Tau^*,\Sigma)$-measurable function $g:f[X]\to X$ such that $gf$ is
the identity on $X$.   \Hint{start with the case in which $\Tau$
separates the points of $Y$, so that the graph of $f$ belongs to
$\Sigma\tensorhat\Tau$.}
%424Xg

\sqheader 424Xi Show that 424Xc and 424Xh are both false if we omit the
phrase `countably generated' from the hypotheses.   \Hint{consider (i) the
countable-cocountable algebra of $\Bbb R$ (ii) the split interval.}
%424Xc %424Xh

\spheader 424Xj\dvAnew{2007} Let $(X,\Sigma,\mu)$ be a
$\sigma$-finite measure space in which $\Sigma$
is countably generated.   Let $\Cal A$ be the set of atoms $A$ of
the Boolean algebra $\Sigma$
such that $\mu A>0$, and set $H=X\setminus\bigcup\Cal A$.   Show that the
subspace measure on $H$ is atomless.
%424?

\leader{424Y}{Further exercises (a)}
%\spheader 424Ya
Let $(X,\frak T)$ be a Polish space and $\Cal F$ the family of closed
subsets of $X$.   Let $\Sigma$ be the $\sigma$-algebra of subsets of
$\Cal F$ generated by the sets
$\Cal E_H=\{F:F\in\Cal F,\,F\cap H\ne\emptyset\}$ as $H$ runs over the
open subsets of $X$.   (i) Show
that $(\Cal F,\Sigma)$ is a standard Borel space.   \Hint{take a
complete metric $\rho$ defining the topology of $X$.   Set
$S^*=\bigcup_{n\in\Bbb
N}\BbbN^n$ and choose a family $\family{\sigma}{S^*}{U_{\sigma}}$ of
open sets in $X$ such that $U_{\emptyset}=X$,
$\diam U_{\sigma}\le 2^{-n}$ whenever $\#(\sigma)=n+1$,
$U_{\sigma}=\bigcup_{i\in\Bbb N}U_{\sigma^{\smallfrown}\fraction{i}}$ for every
$\sigma$,
$\overline{U}_{\sigma^{\smallfrown}\fraction{i}}\subseteq U_{\sigma}$ for
every $\sigma$, $i$.   Define $f:\Cal F\to\{0,1\}^{S^*}$ by setting
$f(F)(\sigma)=1$ if $F\cap U_{\sigma}\ne\emptyset$, $0$ otherwise.
Show that $Z=f[\Cal F]$ is a Borel set and that $f$ is an isomorphism
between $\Sigma$ and the Borel $\sigma$-algebra of $Z$.}   (This is the
{\bf Effros Borel structure} on $\Cal F$.)   (ii) Show that
$[X]^n\in\Sigma$ for every $n\in\Bbb N$.

\spheader 424Yb Let $(X,\Sigma)$ be a standard Borel space.   Let
$\Bbb T$ be the family of Polish topologies on $X$ for which $\Sigma$ is
the Borel $\sigma$-algebra.   Show that any sequence in $\Bbb T$ has an
upper bound in $\Bbb T$, and that any sequence with a lower bound has a
least upper bound.

\spheader 424Yc Let $(X,\Sigma)$ be a standard Borel space.   Say that
$C\subseteq X$ is {\bf coanalytic} if its complement belongs to
$\Cal S(\Sigma)$.   Show that for any such $C$ the partially ordered set
$\Sigma\cap\Cal PC$ has cofinality $1$ if $C\in\Sigma$ and cofinality
$\omega_1$ otherwise.  \Hint{423P.}
%423P

\spheader 424Yd Let $I^{\|}$ be the split interval.   Show that there is
a $\sigma$-algebra $\Sigma$ of subsets of $I^{\|}$ such that
$(I^{\|},\Sigma)$ is a standard Borel space and
$\{(x,y):x$, $y\in I^{\|}$, $x\le y\}\in\Sigma\tensorhat\Sigma$.
%+

\spheader 424Ye Let $(X,\Sigma)$ be a standard Borel space.    Show that
if $X$ is uncountable, $\Sigma$ has a countably generated
$\sigma$-subalgebra not isomorphic either to $\Sigma$ or to $\Cal PI$
for any set $I$.
%mt42bits

\spheader 424Yf Let $(X,\Sigma,\mu)$ be a $\sigma$-finite countably
separated perfect measure space (definition:  342K/451Ad).
Show that there is a standard Borel space $(Y,\Tau)$ such that 
$Y\in\Sigma$, $\Tau\subseteq\Sigma$ and $\mu$ is inner regular with respect
to $\Tau$.
}%end of exercises

\endnotes{
\Notesheader{424} In this treatise I have generally indulged my
prejudice in favour of `complete' measures.   Consequently Borel
$\sigma$-algebras, as such, have taken subordinate roles.
But important parts of the theory of Lebesgue measure, and Radon
measures on Polish spaces in general, are associated with the fact that
these are completions of measures defined on standard Borel spaces.
Moreover, such spaces provide a suitable framework for a large part of
probability theory.   Of course they become deficient in contexts where
we need to look at uncountable independent families of random variables,
and there are also difficulties with
$\sigma$-subalgebras, even countably generated ones, since these can
correspond to the Borel algebras of general analytic spaces, which will
not always be standard Borel structures (424F, 423L).   424Xf and 424Ya
suggest the ubiquity of standard Borel structures;  the former shows
that they are not always presented as
countably generated algebras, while the latter is an example in which we
have to make a special construction in order to associate a topology
with the algebra.   The theory is of course dominated by the results of
\S423, especially 423Fb and 423I.

I include 424H in this section because there is no other convenient
place for it, but I have an excuse:  the idea of `Borel measurable
action' can, in this context, be described entirely in terms of
$\sigma$-algebras, since the Borel algebra of $G\times X$ is just the
$\sigma$-algebra product of the Borel algebras of the factors (as in
424Bb).   Of course for the theorem as expressed here we do need to know
that $G$ has a Polish group structure;  but $X$ could be presented just
as a standard Borel space.   The result is a dramatic expression of the
fact that, given a standard Borel space $(X,\Sigma)$, we have a great
deal of freedom in defining a corresponding Polish topology on $X$.
}%end of notes

\frnewpage

