\frfilename{mt373.tex}
\versiondate{1.1.99/10.2.02}
\copyrightdate{1996}
     
\def\chaptername{Linear operators between function spaces}
\def\sectionname{Decreasing rearrangements}
     
\newsection{373}
     
I take a section to discuss operators in the class $\Cal T^{(0)}$ of
371F-371H and \S372 and two associated classes $\Cal T$, 
$\Cal T^{\times}$ (373A).   These turn
out to be intimately related to the idea of `decreasing
rearrangement' (373C).   In 373D-373F I give elementary properties of
decreasing rearrangements;  then in 373G-373O I show how they may be
used to characterize the set $\{Tu:T\in\Cal T\}$ for a given $u$.
The argument uses a natural topology on the set $\Cal T$ (373K).
I conclude with remarks on the possible values of $\int Tu\times v$ for
$T\in\Cal T$ (373P-373Q) and identifications between $\Cal
T^{(0)}_{\bar\mu,\bar\nu}$, $\Cal T^{(0)}_{\bar\nu,\bar\mu}$ and $\Cal
T^{\times}_{\bar\mu,\bar\nu}$ (373R-373T).
     
\vleader{60pt}{373A}{Definition} Let $(\frak A,\bar\mu)$ and 
$(\frak B,\bar\nu)$ be measure algebras.   \cmmnt{Recall that
$M^{1,\infty}(\frak A,\bar\mu)=L^1(\frak A,\bar\mu)+L^{\infty}(\frak A)$ is the set of those $u\in L^0(\frak A)$ such that $(|u|-\alpha\chi 1)^+$ is integrable for some $\alpha$, its norm $\|\,\|_{1,\infty}$ being defined by the formulae
     
$$\eqalign{\|u\|_{1,\infty}
&=\min\{\|v\|_1+\|w\|_{\infty}:v\in L^1,\,w\in L^{\infty},\,v+w=u\}\cr
&=\min\{\alpha+\int(|u|-\alpha\chi 1)^+:\alpha\ge 0\}\cr}$$
     
\noindent (369Ob).}%end of comment
     
\spheader 373Aa $\Cal T_{\bar\mu,\bar\nu}$ will be
the space of linear operators 
$T:M^{1,\infty}(\frak A,\bar\mu)\to M^{1,\infty}(\frak B,\bar\nu)$ such that $\|Tu\|_1\le\|u\|_1$ for every
$u\in L^1(\frak A,\bar\mu)$ and $\|Tu\|_{\infty}\le\|u\|_{\infty}$ for
every $u\in L^{\infty}(\frak A)$.   \cmmnt{(Compare the definition of
$\Cal T^{(0)}$ in 371F.)}
     
\spheader 373Ab If $\frak B$ is Dedekind complete\cmmnt{, so that
$M^{1,\infty}(\frak A,\bar\mu)$, being a solid linear subspace of the Dedekind complete space $L^0(\frak B)$, is Dedekind complete}, 
$\Cal T^{\times}_{\bar\mu,\bar\nu}$ will be 
$\Cal T_{\bar\mu,\bar\nu}
\cap\eurm L^{\times}(M^{1,\infty}(\frak A,\bar\mu);
\discretionary{}{}{}M^{1,\infty}(\frak A,\bar\mu))$.
     
\vleader{108pt}{373B}{Proposition} Let $(\frak A,\bar\mu)$ and 
$(\frak B,\bar\nu)$ be measure algebras.
     
(a) $\Cal T=\Cal T_{\bar\mu,\bar\nu}$ is a convex set in the unit
ball of 
$\eurm B(M^{1,\infty}(\frak A,\bar\mu);M^{1,\infty}(\frak B,\bar\nu))$.
     
(b) If $T\in\Cal T$ then $T\restr M^{1,0}(\frak A,\bar\mu)$ belongs to
$\Cal T^{(0)}_{\bar\mu,\bar\nu}$\cmmnt{, as defined in 371F}.
So if $T\in\Cal T$, $p\in\coint{1,\infty}$ and 
$u\in L^p(\frak A,\bar\mu)$ then
$Tu\in L^p(\frak B,\bar\nu)$ and $\|Tu\|_p\le\|u\|_p$.
     
(c) If $\frak B$ is Dedekind complete and $T\in\Cal T$, then 
$T\in\eurm L^{\sim}(M^{1,\infty}(\frak A,\bar\mu);M^{1,\infty}(\frak B,\bar\nu))$ and
$T_1\in\Cal T$ whenever $T_1\in\eurm
L^{\sim}(M^{1,\infty}(\frak A,\bar\mu);M^{1,\infty}(\frak B,\bar\nu))$ and $|T_1|\le|T|$;  in particular, $|T|\in\Cal T$.
     
(d) If $\pi:\frak A\to\frak B$ is a measure-preserving Boolean
homomorphism,
then we have a corresponding operator $T\in\Cal T$ defined by saying
that $T(\chi a)=\chi(\pi a)$ for every $a\in\frak A$.   If $\pi$ is
order-continuous, then so is $T$.
     
(e) If $(\frak C,\bar\lambda)$ is another measure algebra and 
$T\in\Cal T$, $S\in\Cal T_{\bar\nu,\bar\lambda}$ then 
$ST\in\Cal T_{\bar\mu,\bar\lambda}$.
     
\proof{{\bf (a)} As 371G, parts (a-i) and (a-ii) of the proof.
     
\medskip
     
{\bf (b)} If $u\in M^{1,0}_{\bar\mu}$ and $\epsilon>0$, then $u$ is
expressible as $u'+u''$ where $\|u''\|_{\infty}\le\epsilon$ and $u'\in
L^1_{\bar\mu}$.   (Set
     
\Centerline{$u''=(u^+\wedge\epsilon\chi 1)-(u^-\wedge\epsilon\chi 1)$.)}
     
\noindent So
     
\Centerline{$(|Tu|-\epsilon\chi 1)^+\le|Tu-Tu''|\in L^1_{\bar\nu}$.}
     
\noindent As $\epsilon$ is arbitrary, $Tu\in M^{1,0}_{\bar\nu}$;  as $u$
is arbitrary, $T\restr M^{1,0}_{\bar\mu}\in \Cal T^{(0)}$.   Now the
rest is a consequence of 371Gd.
     
\medskip
     
{\bf (c)} Because $M^{1,\infty}_{\bar\nu}$ is a solid
linear subspace of $L^0(\frak B)$, which is Dedekind complete because
$\frak B$ is, 
$\eurm L^{\sim}(M^{1,\infty}_{\bar\mu};M^{1,\infty}_{\bar\nu})$ 
is a Riesz space (355Ea).
     
Take any $u\ge 0$ in $M^{1,\infty}_{\bar\mu}$.   Let
$\alpha\ge 0$ be such that $(u-\alpha\chi 1)^+\in L^1_{\bar\mu}$.
Because $T\restr L^1_{\bar\mu}$ belongs to
$\eurm B(L^1_{\bar\mu};L^1_{\bar\nu})
=\eurm L^{\sim}(L^1_{\bar\mu};L^1_{\bar\nu})$ (371D),
$w_0=\sup\{Tv:v\in L^1_{\bar\mu},\,0\le v\le(u-\alpha\chi 1)^+\}$ is
defined in $L^1_{\bar\nu}$.   Now if $v\in M^{1,\infty}_{\bar\mu}$ and
$0\le v\le u$, we must have
     
\Centerline{$Tv=T(v-\alpha\chi 1)^++T(v\wedge\alpha\chi 1)
\le w_0+\alpha\chi 1\in M^{1,\infty}_{\bar\nu}$.}
     
\noindent Thus $\{Tv:0\le v\le u\}$ is bounded above in
$M^{1,\infty}_{\bar\nu}$.   As $u$ is arbitrary,
$T\in\eurm L^{\sim}(M^{1,\infty}_{\bar\mu};M^{1,\infty}_{\bar\nu})$
(355Ba).
     
Now take $T_1$ such that $|T_1|\le|T|$ in
$\eurm L^{\sim}(M^{1,\infty}_{\bar\mu};M^{1,\infty}_{\bar\nu})$.   371D
also tells us that
$\||T\restr L^1_{\bar\mu}|\|=\|T\restr L^1_{\bar\mu}\|$,
so that
     
$$\eqalign{\|T_1u\|_1
&=\||T_1u|\|_1
\le\||T_1||u|\|_1
\le\||T||u|\|_1\cr
&=\|\sup_{|v|\le|u|}Tv\|_1
\le\|\sup_{|v|\le|u|}v\|_1
=\|u\|_1\cr}$$
     
\noindent for every $u\in L^1_{\bar\mu}$ (using a formula in
355Eb for the first equality).   At the same time, if $u\in
L^{\infty}(\frak A)$, then
     
$$\eqalign{|T_1u|
&\le|T_1||u|
\le|T||u|
=\sup_{|v|\le|u|}Tv\cr
&\le\sup_{|v|\le|u|}\|Tv\|_{\infty}\chi 1
\le\sup_{|v|\le|u|}\|v\|_{\infty}\chi 1
=\|u\|_{\infty}\chi 1,\cr}$$
     
\noindent so $\|T_1u\|_{\infty}\le\|u\|_{\infty}$.   Thus $T_1\in\Cal
T$.
     
\medskip
     
{\bf (d)} By 365O and 363F, we have norm-preserving positive linear
operators $T_1:L^1_{\bar\mu}\to L^1_{\bar\nu}$ and
$T_{\infty}:L^{\infty}(\frak A)\to L^{\infty}(\frak B)$ defined by
saying that $T_1(\chi a)=\chi(\pi a)$ whenever $\bar\mu a<\infty$ and
$T_{\infty}(\chi a)=\chi(\pi a)$ for every $a\in\frak A$.   If 
$u\in S(\frak A^f)=L^1_{\bar\mu}\cap S(\frak A)$ (365F), then
$T_1u=T_{\infty}u$, because
both $T_1$ and $T_{\infty}$ are linear and they agree on 
$\{\chi a:\bar\mu a<\infty\}$.   If $u\ge 0$ in
$M^{\infty,1}_{\bar\mu}=L^1_{\bar\mu}\cap L^{\infty}(\frak A)$, there is a non-decreasing sequence $\sequencen{u_n}$
in $S(\frak A^f)$ such that $u=\sup_{n\in\Bbb N}u_n$ and
     
\Centerline{$\lim_{n\to\infty}\|u-u_n\|_1
=\lim_{n\to\infty}\|u-u_n\|_{\infty}=0$}
     
\noindent (see the proof of 369Od), so that
     
\Centerline{$T_1u=\sup_{n\in\Bbb N}T_1u_n
=\sup_{n\in\Bbb N}T_{\infty}u_n=T_{\infty}u$.}
     
\noindent Accordingly $T_1$ and $T_{\infty}$ agree on 
$L^1_{\bar\mu}\cap L^{\infty}(\frak A)$.   But this means that if 
$u\in M^{1,\infty}_{\bar\mu}$
is expressed as $v+w=v'+w'$, where $v$, $v'\in L^1_{\bar\mu}$ and $w$,
$w'\in L^{\infty}(\frak A)$, we shall have
     
\Centerline{$T_1v'+T_{\infty}w'
=T_1v+T_{\infty}w+T_1(v'-v)-T_{\infty}(w-w')
=T_1v+T_{\infty}w$,}
     
\noindent because $v'-v=w-w'\in M^{\infty,1}_{\bar\mu}$.   Accordingly
we have an operator $T:M^{1,\infty}_{\bar\mu}\to M^{1,\infty}_{\bar\nu}$
defined by setting
     
\Centerline{$T(v+w)=T_1v+T_{\infty}w$ whenever $v\in L^1_{\bar\mu}$,
$w\in L^{\infty}(\frak A)$.}
     
\noindent This formula makes it easy to check that $T$ is linear
and positive, and it clearly belongs to $\Cal T$.
     
To see that $T$ is uniquely defined, observe that $T\restr
L^1_{\bar\mu}$
and $T\restr L^{\infty}(\frak A)$ are uniquely defined by the values $T$
takes on $S(\frak A^f)$, $S(\frak A)$ respectively, because these spaces
are dense for the appropriate norms.
     
Now suppose that $\pi$ is order-continuous.   Then $T_1$ and
$T_{\infty}$ are also order-continuous (365Oa, 363Ff).   If 
$A\subseteq M^{1,\infty}_{\bar\mu}$ is non-empty and downwards-directed 
and has infimum $0$, take $u_0\in A$
and $\gamma>0$ such that $(u_0-\gamma\chi 1)^+\in L^1_{\bar\mu}$.   Set
     
\Centerline{$A_1=\{(u-\gamma\chi 1)^+:u\in A,\,u\le u_0\}$,
\quad $A_{\infty}=\{u\wedge\gamma\chi 1:u\in A\}$.}
     
\noindent Then $A_1\subseteq L^1_{\bar\mu}$ and $A_{\infty}\subseteq
L^{\infty}(\frak A)$ are both downwards-directed and have infimum $0$,
so $\inf T_1[A_1]=\inf T_{\infty}[A_{\infty}]=0$ in $L^0(\frak B)$.
But this means that $\inf(T_1[A_1]+T_{\infty}[A_{\infty}])=0$ (351Dc).
Now any $w\in
T_1[A_1]+T_{\infty}[A_{\infty}]$ is expressible as $T(u-\gamma\chi
1)^++T(u'\wedge\gamma\chi 1)$ where $u$, $u'\in A$;  because $A$ is
downwards-directed, there is a $v\in A$ such that $v\le u\wedge u'$, in
which case $Tv\le w$.   Accordingly $T[A]$ must also have infimum $0$.
As $A$ is arbitrary, $T$ is order-continuous.
     
\medskip
     
{\bf (e)} is obvious, as usual.
}%end of proof of 373B
     
\leader{373C}{Decreasing rearrangements}\cmmnt{ The following concept
is fundamental to any understanding of the class $\Cal T$.}  Let
$(\frak A,\bar\mu)$ be a measure algebra.   Write
$M^{0,\infty}_{\bar\mu}=M^{0,\infty}(\frak A,\bar\mu)$ for the set of
those $u\in L^0(\frak A)$ such that
$\bar\mu\Bvalue{|u|>\alpha}$ is finite for some $\alpha\in\Bbb R$.
\cmmnt{(See 369N for the ideology of this notation.)}   \cmmnt{It is
easy to see that} $M^{0,\infty}(\frak A,\bar\mu)$ is a solid linear
subspace of $L^0(\frak A)$.   Let $(\frak A_L,\bar\mu_L)$ be the measure
algebra of Lebesgue measure on $\coint{0,\infty}$.   For
$u\in M^{0,\infty}(\frak A,\bar\mu)$ its {\bf decreasing rearrangement} is
$u^*\in M^{0,\infty}(\frak A_L,\bar\mu_L)$, defined by setting
$u^*=g^{\ssbullet}$, where
     
\Centerline{$g(t)
=\inf\{\alpha:\alpha\ge 0,\,\bar\mu\Bvalue{|u|>\alpha}\le t\}$}
     
\noindent for every $t>0$.   \cmmnt{(This is always finite
because $\inf_{\alpha\in\Bbb R}\bar\mu\Bvalue{|u|>\alpha}=0$, by
364Aa($\beta$) and 321F.)}
     
\cmmnt{I will
maintain this usage of the symbols $\frak A_L$, $\bar\mu_L$, $u^*$ for
the rest of this section.}
     
\leader{373D}{Lemma} Let $(\frak A,\bar\mu)$ be a measure algebra.
     
(a) For any $u\in M^{0,\infty}(\frak A,\bar\mu)$, its decreasing
rearrangement $u^*$ may be defined by the formula
     
\Centerline{$\Bvalue{u^*>\alpha}
=\coint{0,\bar\mu\Bvalue{|u|>\alpha}}^{\ssbullet}$ for every 
$\alpha\ge 0$,}
     
\noindent that is,
     
\Centerline{$\bar\mu_L\Bvalue{u^*>\alpha}=\bar\mu\Bvalue{|u|>\alpha}$
for every $\alpha\ge 0$.}
     
(b) If $|u|\le|v|$ in $M^{0,\infty}(\frak A,\bar\mu)$, then 
$u^*\le v^*$;  in particular, $|u|^*=u^*$.
     
(c)(i) If $u=\sum_{i=0}^n\alpha_i\chi a_i$, where 
$a_0\Bsupseteq a_1\Bsupseteq\ldots\Bsupseteq a_n$ and $\alpha_i\ge 0$ for each $i$,
then $u^*=\sum_{i=0}^n\alpha_i\chi\coint{0,\bar\mu a_i}^{\ssbullet}$.
     
\quad(ii) If $u=\sum_{i=0}^n\alpha_i\chi a_i$ where $a_0,\ldots,a_n$ are
disjoint and $|\alpha_0|\ge|\alpha_1|\ge\ldots\ge|\alpha_n|$, then
$u^*=\sum_{i=0}^n|\alpha_i|\chi\coint{\beta_i,\beta_{i+1}}^{\ssbullet}$,
where $\beta_i=\sum_{j<i}\bar\mu a_i$ for $i\le n+1$.
     
(d) If $E\subseteq\ooint{0,\infty}$ is any Borel set, and 
$u\in M^0(\frak A,\bar\mu)$, then  
$\bar\mu_L\Bvalue{u^*\in E}=\bar\mu\Bvalue{|u|\in E}$.
     
(e) Let $h:\coint{0,\infty}\to\coint{0,\infty}$ be a non-decreasing
function such that $h(0)=0$, and write $\bar h$ for the corresponding
functions on $L^0(\frak A)^+$ and $L^0(\frak A_L)^+$\cmmnt{ (364H)}.   Then $(\bar h(u))^*=\bar h(u^*)$ whenever $u\ge 0$ in 
$M^0(\frak A,\bar\mu)$.   If $h$ is
continuous on the left, $(\bar h(u))^*=\bar h(u^*)$ whenever $u\ge 0$ in
$M^{0,\infty}(\frak A,\bar\mu)$.
     
(f) If $u\in M^{0,\infty}(\frak A,\bar\mu)$ and $\alpha\ge 0$, then
     
\Centerline{$(u^*-\alpha\chi 1)^+=((|u|-\alpha\chi 1)^+)^*$.}

\woddheader{373D}{0}{0}{0}{24pt}
     
(g) If $u\in M^{0,\infty}(\frak A,\bar\mu)$, then for any $t>0$
     
\Centerline{$\int_0^tu^*
=\inf_{\alpha\ge 0}\alpha t+\int(|u|-\alpha\chi 1)^+$.}
     
(h) If $A\subseteq (M^{0,\infty}(\frak A,\bar\mu))^+$ is non-empty and
upwards-directed and has supremum 
$u_0\in M^{0,\infty}(\frak A,\bar\mu)$, then
$u_0^*=\sup_{u\in A}u^*$.
     
\wheader{373D}{0}{0}{0}{24pt}
     
\proof{{\bf (a)} Set
     
\Centerline{$g(t)=\inf\{\alpha:\bar\mu\Bvalue{|u|>\alpha}\le t\}$}
     
\noindent as in 373C.  If $\alpha\ge 0$,
     
\Centerline{$g(t)>\alpha
\iff\bar\mu\Bvalue{|u|>\beta}>t\text{ for some }\beta>\alpha
\iff\bar\mu\Bvalue{|u|>\alpha}>t$}
     
\noindent (because
$\Bvalue{|u|>\alpha}=\sup_{\beta>\alpha}\Bvalue{|u|>\beta}$), so
     
\Centerline{$\Bvalue{u^*>\alpha}=\{t:g(t)>\alpha\}^{\ssbullet}
=\coint{0,\bar\mu\Bvalue{|u|>\alpha}}^{\ssbullet}$.}
     
\noindent Of course this formula defines $u^*$.
     
\medskip
     
{\bf (b)} This is obvious, either from the definition in 373C or from
(a) just above.
     
\medskip
     
{\bf (c)(i)} Setting $v=\sum_{i=0}^n\alpha_i\chi\coint{0,\bar\mu
a_i}^{\ssbullet}$,  we have
     
$$\eqalign{\Bvalue{v>\alpha}
&=0\text{ if }\sum_{i=0}^n\alpha_i\le\alpha,\cr
&=\coint{0,\bar\mu a_j}^{\ssbullet}
\text{ if }\sum_{i=0}^{j-1}\alpha_i\le\alpha<\sum_{i=0}^j\alpha_i,\cr
&=\coint{0,\bar\mu a_0}^{\ssbullet}
\text{ if }0\le\alpha<\alpha_0,\cr}$$
     
\noindent and in all cases is equal to
$\coint{0,\bar\mu\Bvalue{|u|>\alpha}}^{\ssbullet}$.
     
\medskip
     
\quad{\bf (ii)} A similar argument applies.   (If any $a_j$ has infinite
measure, then $a_i$ is irrelevant for $i>j$.)
     
\medskip
     
{\bf (d)} Fix $\gamma>0$ for the moment, and consider
     
\Centerline{$\Cal A=\{E:E\subseteq\ooint{\gamma,\infty}$ is a Borel set,
$\bar\mu_L\Bvalue{u^*\in E}=\bar\mu\Bvalue{|u|\in E}\}$,}
     
\Centerline{$\Cal I=\{\ooint{\alpha,\infty}:\alpha\ge\gamma\}$.}
     
\noindent Then $\Cal I\subseteq\Cal A$ (by (a)), $I\cap J\in\Cal I$ for
all $I$, $J\in\Cal I$, $E\setminus F\in\Cal A$ whenever $E$,
$F\in\Cal A$ and $F\subseteq E$ (because $u\in M^0_{\bar\mu}$, so
$\bar\mu\Bvalue{|u|\in E}<\infty$), and
$\bigcup_{n\in\Bbb N}E_n\in\Cal A$ whenever $\sequencen{E_n}$ is a
non-decreasing sequence in $\Cal A$.   So, by the Monotone Class Theorem
(136B), $\Cal A$ includes the $\sigma$-algebra of subsets of
$\ooint{\gamma,\infty}$ generated by $\Cal I$;  but this must contain
$E\cap\ooint{\gamma,\infty}$ for every Borel set $E\subseteq\Bbb R$.
     
Accordingly, for any Borel set $E\subseteq\ooint{0,\infty}$,
     
\Centerline{$\bar\mu_L\Bvalue{u^*\in E}
=\sup_{n\in\Bbb N}\bar\mu_L
  \Bvalue{u^*\in E\cap\ooint{2^{-n},\infty}\thinspace}
=\bar\mu\Bvalue{|u|\in E}$.}
     
\medskip
     
{\bf (e)} For any $\alpha>0$, $E_{\alpha}=\{t:h(t)>\alpha\}$ is a Borel
subset of
$\ooint{0,\infty}$.   If $u\in M^0_{\bar\mu}$ then, using (d) above,
     
\Centerline{$\bar\mu_L\Bvalue{\bar h(u^*)>\alpha}
=\bar\mu_L\Bvalue{u^*\in E_{\alpha}}
=\bar\mu\Bvalue{u\in E_{\alpha}}
=\bar\mu\Bvalue{\bar h(u)>\alpha}
=\bar\mu_L\Bvalue{(\bar h(u))^*>\alpha}$.}
     
\noindent As both $(\bar h(u))^*$ and $\bar h(u^*)$ are equivalence
classes of non-increasing functions, they must be equal.
     
If $h$ is continuous on the left, then
$E_{\alpha}=\ooint{\gamma,\infty}$ for some $\gamma$, so we no longer
need to use (d), and the argument
works for any $u\in (M^{0,\infty}_{\bar\mu})^+$.
     
\medskip
     
{\bf (f)} Apply (e) with $h(\beta)=\max(0,\beta-\alpha)$.
     
\medskip
     
{\bf (g)} Express $u^*$ as $g^{\ssbullet}$, where
     
\Centerline{$g(s)=\inf\{\alpha:\bar\mu\Bvalue{|u|>\alpha}\le s\}$}
     
\noindent for every $s>0$.   Because $g$ is non-increasing, it is easy
to check that, for $t>0$,
     
\Centerline{$\int_0^tg=tg(t)+\int_0^{\infty}\max(0,g(s)-g(t))ds
\le\alpha t+\int_0^{\infty}\max(0,g(s)-\alpha)ds$}
     
\noindent for every $\alpha\ge 0$;  so that
     
\Centerline{$\int_0^tu^*=\min_{\alpha\ge 0}\alpha t+\int(u^*-\alpha\chi
1)^+$.}
     
\noindent Now
     
$$\eqalign{\int(u^*-\alpha\chi 1)^+
&=\int_0^{\infty}\bar\mu_L\Bvalue{(u^*-\alpha\chi 1)^+>\beta}d\beta\cr
&=\int_0^{\infty}\bar\mu\Bvalue{(|u|-\alpha\chi 1)^+>\beta}d\beta
=\int(|u|-\alpha\chi 1)^+\cr}$$
     
\noindent for every $\alpha\ge 0$, using (f) and 365A, and
     
\Centerline{$\int_0^tu^*=\min_{\alpha\ge 0}\alpha t+\int(|u|-\alpha\chi
1)^+$.}
     
\medskip
     
{\bf (h)}
     
\Centerline{$\bar\mu\Bvalue{u_0>\alpha}
=\bar\mu(\sup_{u\in A}\Bvalue{u>\alpha})
=\sup_{u\in A}\bar\mu\Bvalue{u>\alpha}$}
     
\noindent for any $\alpha>0$, using 364L(a-ii) and 321D.   So
     
\Centerline{$\Bvalue{u_0^*>\alpha}
=\coint{0,\bar\mu\Bvalue{u_0>\alpha}}^{\ssbullet}
=\sup_{u\in A}\coint{0,\bar\mu\Bvalue{u>\alpha}}^{\ssbullet}
=\Bvalue{\sup_{u\in A}u^*>\alpha}$}
     
\noindent for every $\alpha$, and $u_0^*=\sup_{u\in A}u^*$.
}%end of proof of 373D
     
\leader{373E}{Theorem} Let $(\frak A,\bar\mu)$ be a measure algebra.
Then $\int|u\times v|\le\int u^*\times v^*$ for all $u$, 
$v\in M^{0,\infty}(\frak A,\bar\mu)$.
     
\proof{{\bf (a)} Consider first the case $u$, $v\ge 0$ in $S(\frak A)$.
Then we may express $u$, $v$ as $\sum_{i=0}^m\alpha_i\chi a_i$,
$\sum_{j=0}^n\beta_j\chi b_j$ where 
$a_0\Bsupseteq a_1\Bsupseteq\ldots\Bsupseteq a_m$, 
$b_0\Bsupseteq\ldots\Bsupseteq b_n$
in $\frak A$ and $\alpha_i$, $\beta_j\ge 0$ for all $i$, $j$ (361Ec).
Now $u^*$, $v^*$ are given by
     
\Centerline{$u^*=\sum_{i=0}^m\alpha_i\chi\coint{0,\bar\mu
a_i}^{\ssbullet}$,
\quad$v^*=\sum_{j=0}^n\beta_j\chi\coint{0,\bar\mu b_j}^{\ssbullet}$}
     
\noindent (373Dc).   So
     
$$\eqalign{\int u\times v
&=\sum_{i=0}^m\sum_{j=0}^n\alpha_i\beta_j\bar\mu(a_i\Bcap b_j)
\le\sum_{i=0}^m\sum_{j=0}^n\alpha_i\beta_j
  \min(\bar\mu a_i,\bar\mu b_j)\cr
&=\sum_{i=0}^m\sum_{j=0}^n\alpha_i\beta_j\mu_L(\coint{0,\bar\mu
  a_i}\cap\coint{0,\bar\mu b_j})
=\int u^*\times v^*.\cr}$$
     
\medskip
     
{\bf (b)} For the general case, we have non-decreasing sequences
$\sequencen{u_n}$, $\sequencen{v_n}$ in $S(\frak A)^+$ with suprema
$|u|$, $|v|$ respectively (364Jd), so that
     
\Centerline{$|u\times v|=|u|\times|v|=\sup_{n\in\Bbb N}|u|\times v_n
=\sup_{m,n\in\Bbb N}u_m\times v_n=\sup_{n\in\Bbb N}u_n\times v_n$}
     
\noindent and
     
\Centerline{$\int|u\times v|
=\int\sup_{n\in\Bbb N}u_n\times v_n
=\sup_{n\in\Bbb N}\int u_n\times v_n
\le\sup_{n\in\Bbb N}\int u^*_n\times v^*_n
\le\int u^*\times v^*$,}
     
\noindent using 373Db.
}%end of proof of 373E
     
\leader{373F}{Theorem} Let $(\frak A,\bar\mu)$ be a measure algebra, and
$u$ any member of $M^{0,\infty}(\frak A,\bar\mu)$.
     
(a) For any $p\in[1,\infty]$, $u\in L^p(\frak A,\bar\mu)$ iff 
$u^*\in L^p(\frak A_L,\bar\mu_L)$, and in this case $\|u\|_p=\|u^*\|_p$.
     
(b)(i) $u\in M^0(\frak A,\bar\mu)$ iff 
$u^*\in M^0(\frak A_L,\bar\mu_L)$;
     
\quad(ii) $u\in M^{1,\infty}(\frak A,\bar\mu)$ iff 
$u^*\in M^{1,\infty}(\frak A_L,\bar\mu_L)$, and in this case
$\|u\|_{1,\infty}=\|u^*\|_{1,\infty}$;
     
\quad(iii) $u\in M^{1,0}(\frak A,\bar\mu)$ iff 
$u^*\in M^{1,0}(\frak A_L,\bar\mu_L)$;
     
\quad(iv) $u\in M^{\infty,1}(\frak A,\bar\mu)$ iff 
$u^*\in M^{\infty,1}(\frak A_L,\bar\mu_L)$, and in this case
$\|u\|_{\infty,1}=\|u^*\|_{\infty,1}$.
     
\proof{{\bf (a)}(i) Consider first the case $p=1$.   In this case
     
\Centerline{$\int|u|=\int_0^{\infty}\bar\mu\Bvalue{|u|>\alpha}d\alpha
=\int_0^{\infty}\bar\mu_L\Bvalue{u^*>\alpha}d\alpha
=\int u^*$.}
\noindent (ii) If $1<p<\infty$, then by 373De we have
$(|u|^p)^*=(u^*)^p$, so that
     
\Centerline{$\|u\|^p_p=\int|u|^p=\int(|u|^p)^*=\int(u^*)^p=\|u^*\|_p^p$}
     
\noindent if either $\|u\|_p$ or $\|u^*\|_p$ is finite.   (iii) As for
$p=\infty$,
     
\Centerline{$\|u\|_{\infty}\le\gamma
\iff\Bvalue{|u|>\gamma}=0
\iff\Bvalue{u^*>\gamma}=0
\iff\|u^*\|_{\infty}\le\gamma$.}
     
\wheader{373F}{6}{2}{2}{36pt} %p206
     
{\bf (b)(i)}
     
$$\eqalign{u\in M^0_{\bar\mu}
&\iff\bar\mu\Bvalue{|u|>\alpha}<\infty\text{ for every }\alpha>0\cr
&\iff\bar\mu_L\Bvalue{u^*>\alpha}<\infty\text{ for every }\alpha>0
\iff u^*\in M^0_{\bar\mu_L}.\cr}$$
     
\medskip
     
\quad{\bf (ii)} For any $\alpha\ge 0$,
     
\Centerline{$\int(|u|-\alpha\chi 1)^+=\int(u^*-\alpha\chi 1)^+$}
     
\noindent as in the proof of 373Dg.   So
$\|u\|_{1,\infty}=\|u^*\|_{1,\infty}$ if either is finite,
by the formula in 369Ob.
     
\medskip
     
\quad{\bf (iii)} This follows from (i) and (ii), because
$M^{1,0}=M^0\cap M^{1,\infty}$.
     
\medskip
     
\quad{\bf (iv)} Allowing $\infty$ as a value of an integral, we have
     
$$\eqalign{\|u\|_{1,\infty}
&=\min\{\alpha+\int(|u|-\alpha\chi 1)^+:\alpha\ge 0\}\cr
&=\min\{\alpha+\int(u^*-\alpha\chi 1)^+:\alpha\ge 0\}
=\|u^*\|_{1,\infty}\cr}$$
     
\noindent by 369Ob;  in particular, $u\in M^{1,\infty}_{\bar\mu}$ iff
$u^*\in M^{1,\infty}_{\bar\mu_L}$.
}%end of proof of 373F
     
\leader{373G}{Lemma} Let $(\frak A,\bar\mu)$ and $(\frak B,\bar\nu)$ be
measure algebras.   If
     
\inset{{\it either} $u\in M^{1,\infty}(\frak A,\bar\mu)$ and 
$T\in\Cal T_{\bar\mu,\bar\nu}$}
     
\inset{{\it or} $u\in M^{1,0}(\frak A,\bar\mu)$ and 
$T\in\Cal T^{(0)}_{\bar\mu,\bar\nu}$,}
     
\noindent  then  $\int_0^t(Tu)^*\le\int_0^tu^*$ for every $t\ge 0$.
     
\proof{ Set $T_1=T\restr L^1_{\bar\mu}$, so that $\|T_1\|\le 1$ in
$\eurm B(L^1_{\bar\mu};L^1_{\bar\nu})$, and $|T_1|$ is defined in 
$\eurm B(L^1_{\bar\mu};L^1_{\bar\nu})$, also with norm at most $1$.   If
$\alpha\ge 0$, then we can express $u$ as $u_1+u_2$ where
$|u_1|\le(|u|-\alpha\chi 1)^+$ and $|u_2|\le\alpha\chi 1$.   (Let 
$w\in L^{\infty}(\frak A)$ be such that $\|w\|_{\infty}\le 1$, $u=|u|\times w$;  set $u_2=w\times(|u|\wedge\alpha\chi 1)$.)   So if
$\int(|u|-\alpha\chi 1)^+<\infty$,
     
\Centerline{$|Tu|\le|Tu_1|+|Tu_2|\le|T_1||u_1|+\alpha\chi 1$}
     
\noindent and
     
\Centerline{$\int(|Tu|-\alpha\chi
1)^+\le\int|T_1||u_1|\le\int|u_1|\le\int(|u|-\alpha\chi 1)^+$.}
     
\noindent The formula of 373Dg now tells us that
$\int_0^t(Tu)^*\le\int_0^tu^*$ for every $t$.
}%end of proof of 373G
     
\vleader{60pt}{373H}{Lemma} Let $(\frak A,\bar\mu)$ be a measure algebra, and
$\theta:\frak A^f\to\Bbb R$ an additive functional, where 
$\frak A^f=\{a:\bar\mu a<\infty\}$.
     
(a) The following are equiveridical:
     
\inset{($\alpha$) 
$\lim_{t\downarrow 0}\sup_{\bar\mu a\le t}|\theta a|
=\lim_{t\to\infty}\Bover1t\sup_{\bar\mu a\le t}|\theta a|=0$,}
     
\inset{($\beta$) there is some $u\in M^{1,0}(\frak A,\bar\mu)$
such that $\theta a =\int_au$ for every $a\in\frak A^f$,}
     
\noindent and in this case $u$ is uniquely defined.
     
(b) Now suppose that $(\frak A,\bar\mu)$ is localizable.   Then the
following are equiveridical:
     
\inset{($\alpha$) 
$\lim_{t\downarrow 0}\sup_{\bar\mu a\le t}|\theta a|=0$,\quad
$\limsup_{t\to\infty}\Bover1t\sup_{\bar\mu a\le t}|\theta a|<\infty$,}
     
\inset{($\beta$) there is some 
$u\in M^{1,\infty}(\frak A,\bar\mu)$
such that $\theta a =\int_au$ for every $a\in\frak A^f$,}
     
\noindent and again this $u$ is uniquely defined.
     
\proof{{\bf (a)(i)} Assume ($\alpha$).   For $a$, $c\in\frak A^f$, set
$\theta_c(a)=\theta(a\Bcap c)$.   Then for each $c\in\frak A^f$, there
is a unique $u_c\in L^1_{\bar\mu}$ such that
$\theta_ca=\int_au_c$ for every $a\in\frak A^f$ (365Eb).
Because $u_c$ is unique we must have $u_c=u_d\times\chi c$ whenever
$c\Bsubseteq d\in\frak A^f$.   Next, given $\alpha>0$, there is a
$t_0\ge 0$ such that $|\theta a|\le\alpha\bar\mu a$ whenever $a\in\frak
A^f$ and $\bar\mu a\ge t_0$;  so that $\bar\mu\Bvalue{u_c>\alpha}\le
t_0$ for every $c\in\frak A^f$, and $e(\alpha)=\sup_{c\in\frak
A^f}\Bvalue{u_c^+>\alpha}$ is defined in $\frak A^f$.   Of course
$e(\alpha)=\Bvalue{u_{e(1)}^+>\alpha}$ for every $\alpha\ge 1$, so
$\inf_{\alpha\in\Bbb R}e(\alpha)=0$, and
$v_1=\sup_{c\in\frak A^f}u_c^+$ is defined in $L^0=L^0(\frak A)$
(364L(a-ii) again).
Because $\Bvalue{v_1>\alpha}=e(\alpha)\in\frak A^f$ for each $\alpha>0$,
$v_1\in M^0_{\bar\mu}$.   For any $a\in\frak A^f$,
     
\Centerline{$v_1\times\chi a=\sup_{c\in\frak A^f}u_c^+\times\chi a
=u_a^+$,}
     
\noindent so $v_1\in M^{1,0}_{\bar\mu}$ and $\int_av_1=\int_au_a^+$ for every $a\in\frak A^f$.
     
Similarly, $v_2=\sup_{c\in\frak A^f}u_c^-$ is defined in $M^{1,0}_{\bar\mu}$ and
$\int_av_2=\int_au_a^-$ for every $a\in\frak A^f$.   So we can set
$u=v_1-v_2\in M^{1,0}_{\bar\mu}$ and get
     
\Centerline{$\int_au=\int_au_a=\theta a$}
     
\noindent for every $a\in\frak A^f$.   Thus ($\beta$) is true.
     
\medskip
     
\quad{\bf (ii)} Assume ($\beta$).
If $\epsilon>0$, there is a $\delta>0$ such that
$\int_a(|u|-\epsilon\chi 1)^+\le\epsilon$ whenever $\bar\mu a\le\delta$
(365Ea), so
that $|\int_au|\le\epsilon(1+\bar\mu a)$ whenever $\bar\mu a\le\delta$.
As $\epsilon$ is arbitrary, $\lim_{t\downarrow 0}\sup_{\bar\mu a\le
t}|\int_au|=0$.   Moreover, whenever $t>0$ and $\bar\mu a\le t$,
$\bover1t|\int_au|\le\epsilon+\bover1t\int(|u|-\epsilon\chi 1)^+$.
Thus
     
\Centerline{$\limsup_{t\to\infty}\bover1t\sup_{\bar\mu a\le t}|\int_a
u|\le\epsilon$.}
     
\noindent As $\epsilon$ is arbitrary, $\theta$ satisfies the conditions
in ($\alpha$).
     
\medskip
     
\quad{\bf (iii)} The uniqueness of $u$ is a consequence of 366Gd.
     
\medskip
     
{\bf (b)} The argument for (b) uses the same ideas.
     
\medskip
     
\quad{\bf (i)} Assume ($\alpha$).   Again, for each $c\in\frak A^f$, we
have $u_c\in L^1_{\bar\mu}$ such that $\theta_ca=\int_au_c$ for every $a\in\frak A^f$;  again, set 
$e(\alpha)=\sup_{c\in\frak A^f}\Bvalue{u_c^+>\alpha}$,
which is defined because $\frak A$ is supposed to be Dedekind complete.
This time, there are $t_0$, $\gamma\ge 0$ such that 
$|\theta a|\le\gamma\bar\mu a$ whenever $a\in\frak A^f$ and 
$\bar\mu a\ge t_0$;
so that $\bar\mu\Bvalue{u_c>\gamma}\le t_0$ for every $c\in\frak A^f$,
and $\bar\mu e(\gamma)<\infty$.   Accordingly
     
\Centerline{$\inf_{\alpha\ge\gamma}e(\alpha)
=\inf_{\alpha\ge\gamma}\Bvalue{u_{e(\gamma)}^+>\alpha}=0$,}
     
\noindent and once more $v_1=\sup_{c\in\frak A^f}u_c^+$ is defined in
$L^0=L^0(\frak A)$.   As before,
$v_1\times\chi a=u_a^+\in L^1_{\bar\mu}$ for any $a\in\frak A^f$,
Because $\Bvalue{v_1>\gamma}=e(\gamma)\in\frak A^f$, 
$v_1\in M^{1,\infty}_{\bar\mu}$.   Similarly, 
$v_2=\sup_{c\in\frak A^f}u_c^-$ is defined
in $M^{1,\infty}_{\bar\mu}$, with $v_2\times\chi a=u_a^-$ for every $a\in\frak A^f$.   So $u=v_1-v_2\in M^{1,\infty}_{\bar\mu}$, and
     
\Centerline{$\int_au=\int_au_a=\theta a$}
     
\noindent for every $a\in\frak A^f$.
     
\medskip
     
\quad{\bf (ii)} Assume ($\beta$).   Take $\gamma\ge 0$ such that
$\beta=\int(|u|-\gamma\chi 1)^+$ is finite.   If $\epsilon>0$, there is
a $\delta>0$ such that $\int_a(|u|-\gamma\chi 1)^+\le\epsilon$ whenever
$\bar\mu a\le\delta$, so that $|\int_au|\le\epsilon+\gamma\bar\mu a$
whenever $\bar\mu a\le\delta$.   As $\epsilon$ is arbitrary,
$\lim_{t\downarrow 0}\sup_{\bar\mu a\le t}|\int_au|=0$.   Moreover,
whenever $t>0$ and $\bar\mu a\le t$, then
$\bover1t|\int_au|\le\gamma+\bover1t\int(|u|-\epsilon\chi 1)^+$.   Thus
     
\Centerline{$\limsup_{t\to\infty}\bover1t\sup_{\bar\mu a\le t}|\int_a
u|\le\gamma<\infty$,}
     
\noindent and the function $a\mapsto\int_au$ satisfies the conditions in
($\beta$).
     
\medskip
     
\quad{\bf (iii)} $u$ is uniquely defined because $u\times\chi a$ must be
$u_a$, as defined in (i), for every $a\in\frak A^f$, and 
$(\frak A,\bar\mu)$ is semi-finite.
}%end of proof of 373H
     
\leader{373I}{Lemma} Suppose that $u$, $v$, 
$w\in M^{0,\infty}(\frak A_L,\bar\mu_L)$
are all equivalence classes of non-negative
non-increasing functions.   If
$\int_0^tu\le\int_0^tv$ for every $t\ge 0$, then 
$\int u\times w\le\int v\times w$.
     
\proof{ For $n\in\Bbb N$, $i\le 4^n$ set
$a_{ni}=\Bvalue{w>2^{-n}i}$;  set $w_n=\sum_{i=1}^{4^n}2^{-n}\chi
a_{ni}$.   Then each $a_{ni}$ is of the form $[0,t]^{\ssbullet}$, so
     
\Centerline{$\int u\times w_n=\sum_{i=1}^{4^n}2^{-n}\int_{a_{ni}}u
\le\sum_{i=1}^{4^n}2^{-n}\int_{a_{ni}}v=\int v\times w_n$.}
     
\noindent Also $\sequencen{w_n}$ is a non-decreasing sequence with
supremum $w$, so
     
\Centerline{$\int u\times w=\sup_{n\in\Bbb N}\int u\times w_n
\le\sup_{n\in\Bbb N}\int v\times w_n=\int v\times w$.}
}%end of proof of 373I
     
\leader{373J}{Corollary} Suppose that $(\frak A,\bar\mu)$ and 
$(\frak B,\bar\nu)$ are measure algebras and 
$v\in M^{0,\infty}(\frak B,\bar\nu)$.   If
     
\inset{{\it either} $u\in M^{1,0}(\frak A,\bar\mu)$ and 
$T\in\Cal T^{(0)}_{\bar\mu,\bar\nu}$}
     
\inset{{\it or} $u\in M^{1,\infty}(\frak A,\bar\mu)$ and 
$T\in\Cal T_{\bar\mu,\bar\nu}$}
     
\noindent then $\int|Tu\times v|\le\int u^*\times v^*$.
     
\proof{ Put 373E, 373G and 373I together.
}%end of proof of 373J
     
\vleader{48pt}{373K}{The very weak operator topology of $\Cal T$} Let 
$(\frak A,\bar\mu)$ and $(\frak B,\bar\nu)$ be two measure algebras.      For $u\in M^{1,\infty}(\frak A,\bar\mu)$, 
$w\in M^{\infty,1}(\frak B,\bar\nu)$ set
     
\Centerline{$\rho_{uw}(S,T)
=|\int Su\times w-\int Tu\times w|$ for all $S$,
$T\in\Cal T=\Cal T_{\bar\mu,\bar\nu}$.}
     
\noindent Then $\rho_{uw}$ is a pseudometric on $\Cal T$.   I will call
the topology generated by 
$\{\rho_{uw}:u\in M^{1,\infty}(\frak A,\bar\mu),\,
w\in M^{\infty,1}(\frak B,\bar\nu)\}$\cmmnt{ (2A3F)} the
{\bf very weak operator topology} on $\Cal T$.
     
\leader{373L}{Theorem} Let $(\frak A,\bar\mu)$ be a measure algebra and
$(\frak B,\bar\nu)$ a localizable measure algebra.   Then 
$\Cal T=\Cal T_{\bar\mu,\bar\nu}$ is compact in its very weak operator topology.
     
\proof{ Let $\Cal F$ be an ultrafilter on $\Cal T$.   If 
$u\in M^{1,\infty}_{\bar\mu}$, $w\in M^{\infty,1}_{\bar\nu}$ then
     
\Centerline{$|\int Tu\times w|\le\int u^*\times w^*<\infty$}
     
\noindent for every $T\in\Cal T$ (373J);  $\int u^*\times w^*$ is finite
because $u^*\in M^{1,\infty}$ and $w^*\in M^{\infty,1}$ (373F).
     
In particular, $\{\int Tu\times w:T\in\Cal T\}$ is bounded.
Consequently $h_u(w)=\lim_{T\to\Cal F}\int Tu\times w$ is defined in
$\Bbb R$ (2A3Se).
Because $w\mapsto\int Tu\times w$ is additive for every $T\in\Cal T$, so
is $h_u$.   Also
     
\Centerline{$|h_u(w)|\le\int u^*\times
w^*\le\|u^*\|_{1,\infty}\|w^*\|_{\infty,1}
=\|u\|_{1,\infty}\|w\|_{\infty,1}$}
     
\noindent for every $w\in M^{\infty,1}_{\bar\nu}$.
     
$|h_u(\chi b)|\le\int_0^tu^*$ whenever $b\in\frak B^f$ and $\bar\nu b\le
t$.   So
     
\Centerline{$\lim_{t\downarrow 0}\sup_{\bar\nu b\le t}|h_u(\chi b)|
\le\lim_{t\downarrow 0}\int_0^tu^*=0$,}
     
\Centerline{$\limsup_{t\to\infty}\bover1t\sup_{\bar\nu b\le t}|h_u(\chi
b)|\le\limsup_{t\to\infty}\bover1t\int_0^tu^*<\infty$.}
     
\noindent Of course $b\mapsto h_u(\chi b)$ is additive, so by 373H there
is a unique $Su\in M^{1,\infty}_{\bar\nu}$ such that $h_u(\chi
b)=\int_bSu$ for every $b\in\frak B^f$.   Since both $h_u$ and $w\mapsto
\int Su\times w$ are linear and continuous on $M^{\infty,1}_{\bar\nu}$,
and $S(\frak B^f)$ is dense in $M^{\infty,1}_{\bar\nu}$ (369Od),
     
\Centerline{$\int Su\times w=h_u(w)=\lim_{T\to\Cal F}\int Tu\times w$}
     
\noindent for every $w\in
M^{\infty,1}_{\bar\nu}$.
And this is true for every $u\in M^{1,\infty}_{\bar\mu}$.
     
For any particular $w\in M^{\infty,1}_{\bar\nu}$, all the maps
$u\mapsto \int Tu\times w$ are linear, so $u\mapsto\int Su\times w$ also
is;  that is, $S:M^{1,\infty}_{\bar\mu}\to M^{1,\infty}_{\bar\nu}$ is
linear.
     
Now $S\in\Cal T$.   \Prf\ ($\alpha$) If $u\in L^1_{\bar\mu}$ and
$b$, $c\in\frak B^f$, then
     
$$\eqalign{\int_bSu-\int_cSu
&=\lim_{T\to\Cal F}\int Tu\times(\chi b-\chi c)
\le\sup_{T\in\Cal T}\int Tu\times(\chi b-\chi c)\cr
&\le\sup_{T\in\Cal T}\|Tu\|_1\|\chi b-\chi c\|_{\infty}
\le\|u\|_1.\cr}$$
     
\noindent But, setting $e=\Bvalue{Su>0}$, we have
     
$$\eqalign{\int|Su|
&=\int_eSu-\int_{1\Bsetminus e}Su\cr
&=\sup_{b\in\frak B^f,b\Bsubseteq e}\int_bSu+\sup_{c\in\frak
B^f,c\Bsubseteq 1\Bsetminus e}(-Su)
\le\|u\|_1.\cr}$$
     
\noindent ($\beta$) If $u\in L^{\infty}(\frak A)$, then
     
\Centerline{$|\int_bSu|
\le\sup_{T\in\Cal T}|\int Tu\times\chi b|
\le\sup_{T\in\Cal T}\|Tu\|_{\infty}\bar\nu b
\le\|u\|_{\infty}\bar\nu b$}
     
\noindent for every $b\in\frak B^f$.   So
$\Bvalue{Su>\|u\|_{\infty}}=\Bvalue{-Su>\|u\|_{\infty}}=0$ and
$\|Su\|_{\infty}\le\|u\|_{\infty}$.   (Note that both parts of this
argument depend on knowing that $(\frak B,\bar\nu)$ is semi-finite, so
that we cannot be troubled by purely infinite elements of
$\frak B$.)\ \Qed
     
Of course we now have $\lim_{T\to\Cal F}\rho_{uw}(T,S)=0$ for all $u\in
M^{1,0}_{\bar\mu}$, $w\in M^{\infty,1}_{\bar\nu}$, so that
$S=\lim\Cal F$ in $\Cal T$.   As $\Cal F$ is arbitrary, $\Cal T$ is
compact (2A3R).
}%end of proof of 373L
     
\leader{373M}{Corollary} Let $(\frak A,\bar\mu)$ be a measure algebra
and $(\frak B,\bar\nu)$ a localizable measure algebra, and $u$ any
member of $M^{1,\infty}(\frak A,\bar\mu)$.   Then 
$B=\{Tu:T\in\Cal T_{\bar\mu,\bar\nu}\}$ is compact in $M^{1,\infty}(\frak B,\bar\nu)$ for the topology
$\frak T_s(M^{1,\infty}(\frak B,\bar\nu),
M^{\infty,1}(\frak B,\bar\nu))$.
     
\proof{ The point is just that the map 
$T\mapsto Tu:\Cal T_{\bar\mu,\bar\nu}\to M^{1,0}_{\bar\nu}$ is continuous for
the very weak operator topology on $\Cal T_{\bar\mu,\bar\nu}$ and
$\frak T_s(M^{1,\infty}_{\bar\nu},M^{\infty,1}_{\bar\nu})$.   So $B$
is a continuous image of a compact set, therefore compact (2A3Nb).
}%end of proof of 373M
     
\leader{373N}{Corollary} Let $(\frak A,\bar\mu)$ be a measure algebra,
$(\frak B,\bar\nu)$ a localizable measure algebra and $u$ any
member of $M^{1,\infty}(\frak A,\bar\mu)$;  set 
$B=\{Tu:T\in\Cal T_{\bar\mu,\bar\nu}\}$.   If $\sequencen{v_n}$ is any non-decreasing
sequence in $B$, then $\sup_{n\in\Bbb N}v_n$ is defined in
$M^{1,\infty}(\frak B,\bar\nu)$ and belongs to $B$.
     
\proof{ By 373M, $\sequencen{v_n}$ must have a cluster point $v\in B$
for $\frak T_s(M^{1,\infty}_{\bar\nu},M^{\infty,1}_{\bar\nu})$.
Now for any $b\in\frak B^f$, $\int_bv$ must be
a cluster point of $\sequencen{\int_nv_n}$, because $w\mapsto\int_bw$ is
continuous for
$\frak T_s(M^{1,\infty}_{\bar\nu},M^{\infty,1}_{\bar\nu})$.   But
$\sequencen{\int_bv_n}$
is a non-decreasing sequence, so its only possible cluster point is its
supremum;  thus $\int_bv=\lim_{n\to\infty}\int_bv_n$.   Consequently
$v\times\chi b$ must be the supremum of $\{v_n\times\chi b:n\in\Bbb N\}$
in $L^1$.   And this is true for every $b\in\frak B^f$;  as $(\frak
B,\bar\nu)$ is
semi-finite, $v$ is the supremum of $\sequencen{v_n}$ in $L^0(\frak B)$
and in $M^{1,\infty}_{\bar\nu}$.
}%end of proof of 373N
     
\leader{373O}{Theorem} Let $(\frak A,\bar\mu)$, $(\frak B,\bar\nu)$ be
measure algebras and $u\in M^{1,\infty}(\frak A,\bar\mu)$, 
$v\in M^{1,\infty}(\frak B,\bar\nu)$.   Then the following are equiveridical:
     
(i) there is a $T\in\Cal T_{\bar\mu,\bar\nu}$ such that $Tu=v$,
     
(ii) $\int_0^tv^*\le\int_0^tu^*$ for every $t\ge 0$.
     
\noindent In particular, given $u\in M^{1,\infty}(\frak A,\bar\mu)$,
there are $S\in\Cal T_{\bar\mu,\bar\mu_L}$, $T\in\Cal
T_{\bar\mu_L,\bar\mu}$ such that $Su=u^*$, $Tu^*=u$.
     
\proof{ (i)$\Rightarrow$(ii) is Lemma 373G.   Accordingly I shall devote
the rest of the proof to showing that (ii)$\Rightarrow$(i).
     
\medskip
     
{\bf (a)} If $(\frak A,\bar\mu)$, $(\frak B,\bar\nu)$ are measure
algebras and $u\in M^{1,\infty}_{\bar\mu}$, $v\in
M^{1,\infty}_{\bar\nu}$, I will say that $v\preccurlyeq u$ if
there is a $T\in\Cal T_{\bar\mu,\bar\nu}$ such that $Tu=v$, and that
$v\sim u$ if $v\preccurlyeq u$ and $u\preccurlyeq v$.   (Properly
speaking, I ought to write $(u,\bar\mu)\preccurlyeq (v,\bar\nu)$,
because we could in principle have two different measures on the same
algebra.   But I do not think any confusion is likely to arise in the
argument which follows.)   By 373Be, $\preccurlyeq$ is transitive and
$\sim$ is an equivalence relation.   Now we have the following facts.
     
\medskip
     
{\bf (b)} If $(\frak A,\bar\mu)$ is a measure algebra and $u_1$, $u_2\in
M^{1,\infty}_{\bar\mu}$ are such that $|u_1|\le|u_2|$, then
$u_1\preccurlyeq u_2$.   \Prf\ There is a $w\in L^{\infty}(\frak A)$
such that $u_1=w\times u_2$ and $\|w\|_{\infty}\le 1$.   Set $Tv=w\times
v$ for  for $v\in M^{1,\infty}_{\bar\mu}$;  then $T\in\Cal
T_{\bar\mu,\bar\mu}$ and $Tu_2=u_1$.\ \QeD\  So $u\sim|u|$ for every
$u\in M^{1,\infty}_{\bar\mu}$.
     
\medskip
     
{\bf (c)} If $(\frak A,\bar\mu)$ is a measure algebra and $u\ge 0$ in
$S(\frak A)$, then $u\preccurlyeq u^*$.   \Prf\ If $u=0$ this is
trivial.   Otherwise, express $u$ as $\sum_{i=0}^n\alpha_i\chi a_i$
where $a_0,\ldots,a_n$ are disjoint and non-zero and
$\alpha_0>\alpha_1\ldots>\alpha_n>0\in\Bbb R$.   If $\bar\mu a_i=\infty$
for any $i$, take $m$ to be minimal subject to $\bar\mu a_m=\infty$;
otherwise, set $m=n$.   Then
$u^*=\sum_{i=0}^m\alpha_i\chi\coint{\beta_i,\beta_{i+1}}^{\ssbullet}$,
where $\beta_0=0$, $\beta_j=\sum_{i=0}^{j-1}\bar\mu a_i$ for $1\le j\le
m+1$.
     
For $i<m$, and for $i=m$ if $\bar\mu a_m<\infty$, define
$h_i:M^{1,\infty}_{\bar\mu}\to\Bbb R$ by setting
     
\Centerline{$h_i(v)=\Bover1{\bar\mu a_i}\int_{a_i}v$}
     
\noindent for every $v\in M^{1,\infty}_{\bar\mu}$.   If $\bar\mu
a_m=\infty$, then we need a different idea to define $h_m$, as follows.
Let $I$ be $\{a:a\in\frak A,\,\bar\mu(a\cap a_m)<\infty\}$.   Then $I$
is an ideal of $\frak A$ not containing $a_m$, so there is a Boolean
homomorphism
$\pi:\frak A\to\{0,1\}$ such that $\pi a=0$ for $a\in I$ and $\pi a_m=1$
(311D).   This induces a corresponding $\|\,\|_{\infty}$-continuous
linear operator
$h:L^{\infty}(\frak A)\to L^{\infty}(\{0,1\})\cong\Bbb R$, as in 363F.
Now $h(\chi a)=0$ whenever $\bar\mu a<\infty$, and accordingly $h(v)=0$
whenever $v\in M^{\infty,1}_{\bar\mu}$, since $S(\frak A^f)$ is dense in
$M^{\infty,1}_{\bar\mu}$ for $\|\,\|_{\infty,1}$ and therefore also for
$\|\,\|_{\infty}$.   But this means
that $h$ has a unique extension to a linear functional
$h_m:M^{1,\infty}_{\bar\mu}\to\Bbb R$ such that $h_m(v)=0$ for every
$v\in L^1_{\bar\mu}$, while $h_m(\chi a_m)=1$ and
$|h(v)|\le\|v\|_{\infty}$ for every $v\in L^{\infty}(\frak A)$.
     
Having defined $h_i$ for every $i\le m$, define
$T:M^{1,\infty}_{\bar\mu}\to M^{1,\infty}_{\bar\mu_L}$ by setting
     
\Centerline{$Tv=\sum_{i=0}^mh_i(v)
\chi\coint{\beta_i,\beta_{i+1}}^{\ssbullet}$}
     
\noindent for every $v\in M^{1,\infty}_{\bar\mu}$.
     
For any $i\le m$, $v\in L^1_{\bar\mu}$,
     
\Centerline{$\int_{\beta_i}^{\beta_{i+1}}|Tv|
=|h_i(v)|\bar\mu a_i\le\int_{a_i}|v|$;}
     
\noindent summing over $i$, $\|Tv\|_1\le\|v\|_1$.   Similarly, for any
$i\le m$, $v\in L^{\infty}(\frak B)$, $|h_i(v)|\le\|v\|_{\infty}$, so
$\|Tv\|_{\infty}\le\|v\|_{\infty}$.
     
Thus $T\in\Cal T_{\bar\mu,\bar\mu_L}$.   Since $u^*=Tu$, we conclude
that $u^*\preccurlyeq u$, as claimed.\ \Qed
     
\medskip
     
{\bf (d)} If $(\frak A,\bar\mu)$ is a measure algebra and $u\ge 0$ in
$M^{1,\infty}_{\bar\mu}$, then $u^*\preccurlyeq u$.   \Prf\ Let
$\sequencen{u_n}$ be a non-decreasing sequence in $S(\frak A)$ with
$u_0\ge 0$ and $\sup_{n\in\Bbb N}u_n=u$.   Then $\sequencen{u_n^*}$ is a
non-decreasing sequence in $M^{1,\infty}_{\bar\mu_L}$ with supremum
$u^*$, by 373Db and 373Dh.   Also $u_n^*\preccurlyeq u_n\preccurlyeq u$
for every $n$, by (b) and (c) of this proof.   By 373N, $u^*\preccurlyeq
u$.\ \Qed
     
\medskip
     
{\bf (e)} If $(\frak A,\bar\mu)$ is a measure algebra and $u\ge 0$ in
$S(\frak A)$, then $u\preccurlyeq u^*$.   \Prf\ The argument is very
similar to that of (c).   Again, the result is trivial if $u=0$;
suppose that $u>0$ and define $\alpha_i$, $a_i$, $m$, $\beta_i$ as
before.   This time, set $a'_i=a_i$ for $i<m$, $a'_m=\sup_{m\le j\le
n}a_j$, $\tilde u=\sum_{i=0}^m\alpha_i\chi a'_i$;  then $u\le\tilde u$
amd
$\tilde u^*=u^*$.   Set
     
\Centerline{$h_i(v)
=\Bover1{\beta_{i+1}-\beta_i}\int_{\beta_i}^{\beta_{i+1}}v$}
     
\noindent if $i\le m$, $\beta_{i+1}<\infty$ (that is, $\bar\mu
a_i<\infty$) and $v\in M^{1,\infty}_{\bar\mu_L}$;  and if
$\bar\mu a_m=\infty$, set
     
\Centerline{$h_m(v)=\lim_{k\to\Cal F}\Bover1k\int_{0}^{k}v$}
     
\noindent for some non-principal ultrafilter $\Cal F$ on $\Bbb N$.   As
before, we have
     
\Centerline{$|h_i(v)|\bar\mu a'_i\le\int_{\beta_i}^{\beta_{i+1}}|v|$,}
     
\noindent whenever $v\in L^1_{\bar\mu_L}$, $i\le m$, while
$|h_i(v)|\le\|v\|_{\infty}$ whenever $v\in L^{\infty}(\frak A_L)$ and
$i\le m$.   So we can define $T\in\Cal T_{\bar\mu_L,\bar\mu}$ by setting
$Tv=\sum_{i=0}^mh_i(v)\chi a'_i$ for every $v\in
M^{1,\infty}_{\bar\mu_L}$, and get
     
\Centerline{$u\preccurlyeq \tilde u=Tu^*\preccurlyeq u^*$.   \Qed}
     
\medskip
     
{\bf (f)} If $(\frak A,\bar\mu)$ is a measure algebra and $u\ge 0$ in
$M^{1,\infty}_{\bar\mu}$, then $u\preccurlyeq u^*$.   \Prf\ This time I
seek to copy the ideas of (d);  there is a new obstacle to circumvent,
since $(\frak A,\bar\mu)$ might not be localizable.   Set
     
\Centerline{$\alpha_0=\inf\{\alpha:\alpha\ge
0,\,\bar\mu\Bvalue{u>\alpha}<\infty\}$, \quad$e=\Bvalue{u>\alpha_0}$.}
     
\noindent Then $e=\sup_{n\in\Bbb N}\Bvalue{u>\alpha_0+2^{-n}}$ is a
countable supremum of elements of finite measure, so the principal ideal
$\frak A_e$, with its induced measure $\bar\mu_e$, is $\sigma$-finite.
Now let $\sequencen{u_n}$ be a non-decreasing sequence in $S(\frak A)$
with $u_0\ge 0$ and $\sup_{n\in\Bbb N}u_n=u$;  set $\tilde u=u\times\chi
e$ and $\tilde u_n=u_n\times\chi e$, regarded as members of $S(\frak
A_e)$, for each $n$.   In this case
     
\Centerline{$\tilde u_n\preccurlyeq\tilde u_n^*\preccurlyeq u^*$}
\noindent for every $n$.   Because $(\frak A_e,\bar\mu_e)$ is
$\sigma$-finite, therefore localizable, 373N tells us that $\tilde
u\preccurlyeq u^*$.
     
Let $S\in\Cal T_{\bar\mu_L,\bar\mu_e}$ be such that $Su^*=\tilde u$.
As in part (e), choose a non-principal ultrafilter $\Cal F$ on $\Bbb N$
and set
     
\Centerline{$h(v)=\lim_{k\to\Cal F}\Bover1k\int_0^kv$}
     
\noindent for $v\in M^{1,\infty}_{\bar\mu_L}$.   Now define
$T:M^{1,\infty}_{\bar\mu_L}\to M^{1,\infty}_{\bar\mu}$ by setting
     
\Centerline{$Tv=Sv+h(v)\chi(1\Bsetminus e)$,}
     
\noindent here regarding $Sv$ as a member of $M^{1,\infty}_{\bar\mu}$.
(I am taking it to be obvious that $M^{1,\infty}_{\bar\mu_e}$ can be
identified with $\{w\times\chi e:w\in M^{1,\infty}_{\bar\mu}\}$.)   Then
it is easy to see that $T\in\Cal T_{\bar\mu_L,\bar\mu}$.   Also $u\le
Tu^*$, because
     
\Centerline{$h(u^*)=\inf\{\alpha:\bar\mu_L\Bvalue{u^*>\alpha}<\infty\}
=\alpha_0$,}
     
\noindent while $u\times\chi(1\Bsetminus e)\le\alpha_0\chi(1\Bsetminus
e)$.    So we get $u\preccurlyeq Tu^*\preccurlyeq u^*$.\ \Qed
     
\medskip
     
{\bf (g)} Now suppose that $u$, $v\ge 0$ in $M^{1,\infty}_{\bar\mu_L}$,
that $\int_0^tu^*\ge\int_0^tv^*$ for every $t\ge 0$, and that $v$ is of
the form $\sum_{i=1}^n\alpha_i\chi a_i$ where
$\alpha_1>\ldots>\alpha_n>0$, $a_1,\ldots,a_n\in\frak A_L$ are disjoint
and $\bar\mu_La_i<\infty$ for each $i$.
Then $v\preccurlyeq u$.   \Prf\ Induce on $n$.   If $n=0$ then $v=0$ and
the result is trivial.   For the inductive step to $n\ge 1$, if $v^*\le
u^*$ we have
     
\Centerline{$v\sim v^*\preccurlyeq u^*\sim u$,}
     
\noindent using (b), (d) and (f) above.   Otherwise, look at
$\phi(t)=\bover1t\int_0^tu^*$ for $t>0$.   We have
     
\Centerline{$\phi(t)\ge\Bover1t\int_0^tv^*=\alpha_1$}
     
\noindent for $t\le \beta=\bar\mu a_1$, while
$\lim_{t\to\infty}\phi(t)<\alpha_1$, because
$(\lim_{t\to\infty}\phi(t))\chi 1\le u^*$ and $v^*\le\alpha_1\chi 1$ and
$v^*\not\le u^*$.   Becaasue $\phi$ is continuoue, there is a
$\gamma\ge\beta$ such that
$\phi(\gamma)=\alpha_1$.   Define $T_0\in\Cal T_{\bar\mu_L,\bar\mu_L}$
by setting
     
\Centerline{$ T_0w=(\Bover1{\gamma}\int_0^{\gamma}w)
\chi\coint{0,\gamma}^{\ssbullet}
+(w\times\chi\coint{\gamma,\infty}^{\ssbullet})$}
     
\noindent for every $w\in M^{1,\infty}_{\bar\mu_L}$.   Then
$T_0u^*\preccurlyeq u^*\sim u$, and
     
\Centerline{$ T_0u^*\times\chi\coint{0,\gamma}^{\ssbullet}
=(\Bover1{\gamma}\int_0^{\gamma}u^*)\chi\coint{0,\gamma}^{\ssbullet}
=\alpha_1\chi\coint{0,\gamma}^{\ssbullet}$.}
     
We need to know that $\int_0^tT_0u^*\ge\int_0^tv^*$ for every $t$;  this
is because
     
$$\eqalign{\int_0^tT_0u^*
&=\alpha_1t\ge\int_0^tv^*\text{ whenever }t\le\gamma,\cr
&=\int_0^{\gamma}T_0u^*+\int_{\gamma}^tT_0u^*
=\int_0^tu^*\ge\int_0^tv^*\text{ whenever }t\ge\gamma.\cr}$$
     
Set
     
\Centerline{$u_1=T_0u^*\times\chi\coint{\beta,\infty}^{\ssbullet}$,
\quad$v_1=v^*\times\chi\coint{\beta,\infty}^{\ssbullet}$.}
     
\noindent Then $u_1^*$, $v_1^*$ are just translations of $T_0u^*$, $v^*$
to the left, so that
     
\Centerline{$\int_0^tu_1^*=\int_{\beta}^{\beta+t}T_0u^*
=\int_0^{\beta+t}T_0u^*-\alpha_1\beta
\ge\int_0^{\beta+t}v^*-\alpha_1\beta
=\int_{\beta}^{\beta+t}v^*=\int_0^tv_1^*$}
     
\noindent for every $t\ge 0$.   Also
$v_1=\sum_{i=2}^n\alpha_i\chi\coint{\beta_{i-1},\beta_i}^{\ssbullet}$
where $\beta_i=\sum_{j=1}^i\bar\mu a_j$ for each $j$.   So by the
inductive hypothesis, $v_1\preccurlyeq u_1$.
     
Let $S\in\Cal T_{\bar\mu_L,\bar\mu_L}$ be such that $Su_1=v_1$, and
define $T\in\Cal T_{\bar\mu_L,\bar\mu_L}$ by setting
     
\Centerline{$Tw=w\times\chi\coint{0,\beta}^{\ssbullet}
+S(w\times\chi\coint{\beta,\infty}^{\ssbullet})
\times\chi\coint{\beta,\infty}^{\ssbullet}$}
     
\noindent for every $w\in M^{1,\infty}_{\bar\mu_L,\bar\mu_L}$.   Then
$TT_0u^*=v^*$, so $v\sim v^*\preccurlyeq u^*\sim u$, as required.\ \Qed
     
\medskip
     
{\bf (h)} We are nearly home.   If $u$, $v\ge 0$ in
$M^{1,\infty}_{\bar\mu_L}$ and $\int_0^tv^*\le\int_0^tu^*$ for every
$t\ge 0$, then $v\preccurlyeq u$.   \Prf\ Let $\sequencen{v_n}$ be a
non-decreasing sequence in $S(\frak A^f_L)^+$ with supremum $v$.   Then
$v_n^*\le v^*$ for each $n$, so (g) tells us that $v_n\preccurlyeq u$
for every $n$.   By 373N, for the last time, $v\preccurlyeq u$.\ \Qed
     
\medskip
     
{\bf (i)} Finally, suppose that $(\frak A,\bar\mu)$ and $(\frak
B,\bar\nu)$ are arbitrary measure algebras and that $u\in
M^{1,\infty}_{\bar\mu}$, $v\in M^{1,\infty}_{\bar\nu}$ are such that
$\int_0^tv^*\le\int_0^tu^*$ for every $t\ge 0$.   By (b),
$v\preccurlyeq|v|$;  by (f), $|v|\preccurlyeq|v|^*$;  by 373Db,
$|v|^*=v^*$;  by (h) of this proof, $v^*\preccurlyeq u^*$;  by (d),
$u^*=|u|^*\preccurlyeq|u|$;  and by (b) again, $|u|\preccurlyeq u$.
}%end of proof of 373O
     
\leader{373P}{Theorem} Let $(\frak A,\bar\mu)$ be a measure algebra
and $(\frak B,\bar\nu)$ a semi-finite measure algebra.   Then for any
$u\in M^{1,\infty}(\frak A,\bar\mu)$ and $v\in M^0(\frak B,\bar\nu)$,
there is a $T\in\Cal T=\Cal T_{\bar\mu,\bar\nu}$ such that $\int
Tu\times v=\int u^*\times v^*$.

\proof{{\bf (a)} It is convenient to dispose immediately of some
elementary questions.
     
\medskip
     
\quad{\bf (i)} We need only find a $T\in\Cal T$ such that $\int|Tu\times
v|\ge\int u^*\times v^*$.   \Prf\ Take $v_0\in L^{\infty}(\frak B)$ such
that $|Tu\times v|=v_0\times Tu\times v$ and $\|v_0\|_{\infty}\le 1$,
and set $T_1w=v_0\times Tw$ for $w\in M^{1,\infty}_{\bar\mu}$;  then
$T_1\in\Cal T$ and
     
\Centerline{$\int T_1u\times v=\int|Tu\times v|
\ge\int u^*\times v^*\ge\int T_1u\times v$}
     
\noindent by 373J.\ \Qed
     
\medskip
     
\quad{\bf (ii)} Consequently it will be enough to consider $v\ge 0$,
since of course $\int|Tu\times v|=\int|Tu\times|v||$, while $|v|^*=v^*$.
     
\medskip
     
\quad{\bf (iii)} It will be enough to consider $u=u^*$.   \Prf\ If we
can find $T\in\Cal T_{\bar\mu_L,\bar\nu}$ such that $\int Tu^*\times
v=\int(u^*)^*\times v^*$, then we know from 373O that there is an
$S\in\Cal T_{\bar\mu,\bar\mu_L}$ such that $Su=u^*$, so that $TS\in\Cal
T$ and
     
\Centerline{$\int TSu\times v=\int(u^*)^*\times v^*=\int u^*\times v^*$.
\Qed}
     
\medskip
     
\quad{\bf (iv)} It will be enough to consider localizable $(\frak
B,\bar\nu)$.   \Prf\ Assuming that $v\ge 0$, following (ii) above, set
$e=\Bvalue{v>0}=\sup_{n\in\Bbb N}\Bvalue{v>2^{-n}}$, and let $\bar\nu_e$
be the restriction of $\bar\nu$ to the principal ideal $\frak B_e$
generated by $e$.   Then if we write $\tilde v$ for the member of
$L^0(\frak B_e)$ corresponding to $v$ (so that $\Bvalue{\tilde
v>\alpha}=\Bvalue{v>\alpha}$ for every $\alpha>0$), $\tilde v^*=v^*$.
Also $(\frak B_e,\bar\nu_e)$ is $\sigma$-finite, therefore localizable.
Now if we can find $T\in \Cal T_{\bar\mu,\bar\nu_e}$ such that $\int
Tu\times\tilde v=\int u^*\times\tilde v^*$, then $ST$ will belong to
$\Cal T_{\bar\mu,\bar\nu}$, where $S:L^0(\frak B_e)\to L^0(\frak B)$ is
the canonical embedding defined by the formula
     
$$\eqalign{\Bvalue{Sw>\alpha}&=\Bvalue{w>\alpha}
  \text{ if }\alpha\ge 0,\cr
&=\Bvalue{w>\alpha}\Bcup(1\Bsetminus e)\text{ if }\alpha<0,}$$
     
\noindent and
     
\Centerline{$\int STu\times v=\int Tu\times\tilde v=\int u^*\times
\tilde v^*=\int u^*\times v^*$.  \Qed}
     
     
\medskip
     
{\bf (b)} So let us suppose henceforth that $\bar\mu=\bar\mu_L$, $u=u^*$
is the equivalence class of a non-increasing non-negative function,
$v\ge 0$ and $(\frak B,\bar\nu)$ is localizable.
     
For $n$, $i\in\Bbb N$ set
     
\Centerline{$b_{ni}=\Bvalue{v>2^{-n}i}$,
\quad$\beta_{ni}=\bar\nu b_{ni}$,
\quad$c_{ni}=b_{ni}\Bsetminus b_{n,i+1}$,
\quad$\gamma_{ni}=\bar\nu c_{ni}=\beta_{ni}-\beta_{n,i+1}$}
     
\noindent (because $\beta_{ni}<\infty$ if $i>0$;  this is really where I
use the hypothesis that $v\in M^0$).   For $n\in\Bbb N$ set
     
\Centerline{$K_n=\{i:i\ge 1,\,\gamma_{ni}>0\}$,}
     
$$T_nw=\sum_{i\in
K_n}\bigl(\bover1{\gamma_{ni}}\int_{\beta_{n,i+1}}^{\beta_{ni}}w\bigr)
\chi c_{ni}$$
     
\noindent for $w\in M^{1,\infty}_{\bar\mu_L}$;  this is defined
in $L^0(\frak B)$ because $K_n$ is countable and $\sequence{i}{c_{ni}}$
is disjoint.   Of course $T_n:M^{1,\infty}_{\bar\mu_L}\to
L^0(\frak B)$ is linear.   If $w\in L^{\infty}(\frak A_L)$ then
     
\Centerline{$\|T_nw\|_{\infty}=\sup_{i\in
K_n}\bigl|\Bover1{\gamma_{ni}}\int_{\beta_{n,i+1}}^{\beta_{ni}}w\bigr|
\le\|w\|_{\infty}$,}
     
\noindent and if $w\in L^1_{\bar\mu_L}$ then
     
\Centerline{$\|T_nw\|_1
=\sum_{i\in K_n}\bigl|\Bover1{\gamma_{ni}}
$$\textfont3=\twelveex\int_{\beta_{n,i+1}}^{\beta_{ni}}w\bigr|\bar\nu c_{ni}
=$$\sum_{i\in K_n}\bigl|$$\textfont3=\twelveex\int_{\beta_{n,i+1}}^{\beta_{ni}}w\bigr|
\le\|w\|_1$;}
     
\noindent so $T_nw\in M^{1,\infty}_{\bar\nu}$ for every $w\in
M^{1,\infty}_{\bar\mu_L}$, and $T_n\in\Cal T$.    It will be helpful to
observe that
     
\Centerline{$\int_{c_{ni}}T_nw=\int_{\beta_{n,i+1}}^{\beta_{ni}}w$}
     
\noindent whenever $i\ge 1$, since if $i\notin K_n$ then both sides are
$0$.
     
Note next that for every $n$, $i\in\Bbb N$,
     
\Centerline{$b_{ni}=b_{n+1,2i}$,
\quad$\beta_{ni}=\beta_{n+1,2i}$,
\quad$c_{ni}=c_{n+1,2i}\Bcup c_{n+1,2i+1}$,
\quad $\gamma_{ni}=\gamma_{n+1,2i}+\gamma_{n+1,2i+1}$,}
     
\noindent so that, for $i\ge 1$,
     
\Centerline{$\int_{c_{ni}}T_nu=\int_{\beta_{n,i+1}}^{\beta_{ni}}u
=\int_{c_{ni}}T_{n+1}u$.}
     
\noindent This means that if $T$ is any cluster point of
$\sequencen{T_n}$ in $\Cal T$ for the very weak operator topology (and
such a cluster point exists, by
373L), $\int_{c_{mi}}Tu$ must be a cluster point of
$\sequencen{\int_{c_{mi}}T_nu}$, and therefore equal to
$\int_{c_{mi}}T_mu$, for every $m\in\Bbb N$, $i\ge 1$.
     
Consequently, if $m\in\Bbb N$,
     
$$\eqalignno{\int|Tu\times v|
&\ge\sum_{i=0}^{\infty}\int_{c_{mi}}|Tu|\times v
\ge\sum_{i=0}^{\infty}2^{-m}i\int_{c_{mi}}|Tu|\cr
\noalign{\noindent (because $c_{mi}\Bsubseteq\Bvalue{v>2^{-m}i}$)}
&\ge\sum_{i=1}^{\infty}2^{-m}i|\int_{c_{mi}}Tu|
=\sum_{i=1}^{\infty}2^{-m}i\int_{c_{mi}}T_mu\cr
&=\sum_{i=0}^{\infty}2^{-m}i\int_{\beta_{m,i+1}}^{\beta_{mi}}u
\ge\int u\times(v^*-2^{-m}\chi 1)^+\cr}$$
     
\noindent because
     
\Centerline{$[\beta_{m,i+1},\beta_{mi}]^{\ssbullet}
\Bsubseteq\Bvalue{v^*\le 2^{-m}(i+1)}=\Bvalue{(v^*-2^{-m}\chi 1)^+\le
2^{-m}i}$}
     
\noindent for each $i\in\Bbb N$.    But letting $m\to\infty$, we have
     
\Centerline{$\int|Tu\times v|\ge\lim_{m\to\infty}
\int u\times(v^*-2^{-m}\chi 1)^+
=\int u\times v^*$}
     
\noindent because $\sequence{m}{u\times(v^*-2^{-m}\chi 1)^+}$ is a
non-decreasing sequence with supremum $u\times v^*$.   In view of the
reductions in (a) above, this is enough to complete the proof.
}%end of proof of 373P
     
\leader{373Q}{Corollary} Let $(\frak A,\bar\mu)$ be a measure algebra,
$(\frak B,\bar\nu)$ a semi-finite measure algebra, 
$u\in M^{1,\infty}(\frak A,\bar\mu)$ and 
$v\in M^{0,\infty}(\frak B,\bar\nu)$.   Then
     
\Centerline{$\int u^*\times v^*
=\sup\{\int|Tu\times v|:T\in\Cal T_{\bar\mu,\bar\nu}\}
=\sup\{\int Tu\times v:T\in\Cal T_{\bar\mu,\bar\nu}\}
$.}
     
\proof{ There is a non-decreasing sequence $\sequencen{c_n}$ in $\frak
B$ such that $\bar\nu c_n<\infty$ for every $n$ and $v^*=\sup_{n\in\Bbb
N}(v\times\chi c_n)^*$.   \Prf\ For each rational $q>0$, we can find a
countable non-empty set $B_q\subseteq\frak B$ such that
     
\Centerline{$b\Bsubseteq\Bvalue{|v|>q}$,
$\bar\nu b<\infty$ for every $b\in B_q$,}
     
\Centerline{$\sup_{b\in B_q}\bar\nu b=\bar\nu\Bvalue{|v|>q}$}
     
\noindent (because $(\frak B,\bar\nu)$ is semi-finite).   Let
$\sequencen{b_n}$ be a sequence running over $\bigcup_{q\in\Bbb
Q,q>0}B_q$ and set $c_n=\sup_{i\le n}b_i$, $v_n=v\times\chi c_n$ for
each $n$.
Then $\sequencen{|v_n|}$ and $\sequencen{v_n^*}$ are non-decreasing and
$\sup_{n\in\Bbb N}v_n^*\le v^*$ in $L^0(\frak A_L)$.   But in fact
$\sup_{n\in\Bbb N}v_n^*=v^*$, because
     
\Centerline{$\bar\mu_L\Bvalue{v^*>q}
=\bar\mu\Bvalue{|v|>q}
=\sup_{n\in\Bbb N}\bar\mu\Bvalue{v_n>q}
=\sup_{n\in\Bbb N}\bar\mu_L\Bvalue{v_n^*>q}
=\bar\mu_L\Bvalue{\sup_{n\in\Bbb N}v_n^*>q}$}
     
\noindent for every rational $q>0$, by 373Da.\ \Qed
     
For each $n\in\Bbb N$ we have a $T_n\in\Cal T_{\bar\mu,\bar\nu}$ such
that $\int T_nu\times v_n=\int u^*\times v_n^*$ (373P).   Set
$S_nw=T_nw\times\chi c_n$ for $n\in\Bbb N$, $w\in
M^{1,\infty}_{\bar\mu}$;  then every $S_n$ belongs to $\Cal
T_{\bar\mu,\bar\nu}$, so
     
$$\eqalign{\sup\{\int Tu\times v:T\in\Cal T_{\bar\mu,\bar\mu}\}
&\ge\sup_{n\in\Bbb N}\int S_nu\times v
=\sup_{n\in\Bbb N}\int T_nu\times v_n\cr
&=\sup_{n\in\Bbb N}\int u^*\times v_n^*
=\int u^*\times v^*\cr
&\ge\sup\{\int|Tu\times v|:T\in\Cal T_{\bar\mu,\bar\mu}\}
\ge\sup\{\int Tu\times v:T\in\Cal T_{\bar\mu,\bar\mu}\}
\cr}$$
     
\noindent by 373J, as usual.
}%end of proof of 373Q
     
\leader{373R}{Order-continuous operators:  Proposition} Let 
$(\frak A,\bar\mu)$ be a measure algebra, $(\frak B,\bar\nu)$ a localizable measure algebra, and
$T_0\in\Cal T^{(0)}=\Cal T^{(0)}_{\bar\mu,\bar\nu}$.   Then
there is a $T\in\Cal T^{\times}=\Cal T^{\times}_{\bar\mu,\bar\nu}$
extending $T_0$.   If $(\frak A,\bar\mu)$ is semi-finite, $T$ is
uniquely defined.
     
\proof{{\bf (a)} Suppose first that $T_0\in\Cal T^{(0)}$ is
non-negative, regarded as a member of $\eurm
L^{\sim}(M^{1,0}_{\bar\mu};M^{1,0}_{\bar\nu})$.   In this
case $T_0$ has an extension to an
order-continuous positive linear operator
$T:M^{1,\infty}_{\bar\mu}\to L^0(\frak B)$ defined by saying that
$Tw=\sup\{T_0u:u\in M^{1,0}_{\bar\mu},\,0\le u\le w\}$ for every
$w\ge 0$ in
$M^{1,\infty}_{\bar\mu}$.   \Prf\ I use 355F.   $M^{1,0}_{\bar\mu}$ is a
solid linear subspace of $M^{1,\infty}_{\bar\mu}$.
$T_0$ is order-continuous when its codomain is taken to be
$M^{1,0}_{\bar\nu}$, as noted in 371Gb, and therefore if its codomain is
taken to
be $L^0(\frak B)$, because $M^{1,0}$ is a solid linear subspace in
$L^0$, so the embedding is order-continuous.    If
$w\ge 0$ in $M^{1,\infty}_{\bar\mu}$, let
$\gamma\ge 0$ be such that $u_1=(w-\gamma\chi 1)^+$ is integrable.   If
$u\in M^{1,0}_{\bar\mu}$ and $0\le u\le w$, then
$(u-\gamma\chi 1)^+\le u_1$, so
     
\Centerline{$T_0u=T_0(u-\gamma\chi 1)^++T_0(u\wedge\gamma\chi 1)
\le T_0u_1+\gamma\chi 1\in L^0(\frak B)$.}
     
\noindent Thus $\{T_0u:u\in M^{1,0}_{\bar\nu},\,0\le u\le w\}$ is
bounded above in $L^0(\frak B)$, for any $w\ge 0$ in
$M^{1,\infty}_{\bar\mu}$.     $L^0(\frak B)$ is Dedekind
complete, because $(\frak B,\bar\nu)$ is localizable, so
$\sup\{T_0u:0\le
u\le w\}$ is defined in $L^0(\frak B)$;  and this is true for every
$w\in(M^{1,\infty}_{\bar\mu})^+$.   Thus the conditions of 355F are
satisfied and we have the result.\ \Qed
     
\medskip
     
{\bf (b)} Now suppose that $T_0$ is any member of $\Cal T^{(0)}$.   Then
$T_0$ has an extension to a member of $\Cal T^{\times}$.   \Prf\
$|T_0|$, $T_0^+=\bover12(|T_0|+T_0)$ and $T_0^-=\bover12(|T_0|-T_0)$,
taken in $\eurm L^{\sim}(M^{1,0}_{\bar\mu};M^{1,0}_{\bar\nu})$, all
belong to $\Cal T^{(0)}$ (371G), so have  extensions $S$, $S_1$ and
$S_2$ to order-continuous positive linear operators from
$M^{1,\infty}_{\bar\mu}$ to $L^0(\frak B)$ as defined in (a).   Now for
any $w\in L^1_{\bar\mu}$,
     
\Centerline{$\|Sw\|_1=\||T_0|w\|_1\le\|w\|_1$,}
     
\noindent and for any $w\in L^{\infty}(\frak A)$,
     
\Centerline{$|Sw|\le S|w|=\sup\{|T_0|u:u\in M^{1,0}_{\bar\mu},\,0\le
u\le w\}\le\|w\|_{\infty}\chi 1$,}
     
\noindent so $\|Sw\|_{\infty}\le\|w\|_{\infty}$.   Thus $S\in\Cal T$;
similarly, $S_1$ and $S_2$ can be regarded as operators from
$M^{1,\infty}_{\bar\mu}$ to $M^{1,\infty}_{\bar\nu}$, and as such belong
to $\Cal T$.   Next, for $w\ge 0$ in $M^{1,\infty}_{\bar\mu}$,
     
$$\eqalign{S_1w+S_2w
&=\sup\{T_0^+u:u\in M^{1,0}_{\bar\mu},\,0\le u\le w\}
  +\sup\{T_0^-u:u\in M^{1,0}_{\bar\mu},\,0\le u\le w\}\cr
&=\sup\{T_0^+u+T_0^-u:u\in M^{1,0}_{\bar\mu},\,0\le u\le w\}
=Sw.\cr}$$
     
\noindent But this means that
     
\Centerline{$S=S_1+S_2\ge|S_1-S_2|$}
     
\noindent and $T=S_1-S_2\in\Cal T$, by 373Bc;  while of course $T$
extends $T_0^+-T_0^-=T_0$.   Finally, because $S_1$ and $S_2$ are
order-continuous,
$T\in\eurm L^{\times}(M^{1,\infty}_{\bar\mu};M^{1,\infty}_{\bar\nu})$,
so $T\in\Cal T^{\times}$.\ \Qed
     
\medskip
     
{\bf (c)} If $(\frak A,\bar\mu)$ is semi-finite, then
$M^{1,0}_{\bar\mu}$ is order-dense in $M^{1,\infty}_{\bar\mu}$ (because
it includes $L^1_{\bar\mu}$, which is order-dense in $L^0(\frak A)$);
so that the extension $T$ is unique, by 355Fe.
}%end of proof of 373R
     
\leader{373S}{Adjoints in $\Cal T^{(0)}$:  Theorem}  Let 
$(\frak A,\bar\mu)$ and $(\frak B,\bar\nu)$ be measure algebras, and $T$ any member of $\Cal T^{(0)}_{\bar\mu,\bar\nu}$.   Then there is a unique
operator $T'\in\Cal T^{(0)}_{\bar\nu,\bar\mu}$ such that 
$\int_aT'(\chi b)=\int_bT(\chi a)$ for every $a\in\frak A^f$, 
$b\in\frak B^f$, and now
$\int u\times T'v=\int Tu\times v$ whenever 
$u\in M^{1,0}(\frak A,\bar\mu)$,
$v\in M^{1,0}(\frak B,\bar\nu)$ are such that 
$\int u^*\times v^*<\infty$.
     
\proof{{\bf (a)} For each $v\in M^{1,0}_{\bar\nu}$ we can
define $T'v\in M^{1,0}_{\bar\mu}$ by the formula
     
\Centerline{$\int_aT'v=\int T(\chi a)\times v$}
     
\noindent for every $a\in\frak A^f$.   \Prf\ Set $\theta a=\int T(\chi
a)\times v$ for each $a\in\frak A^f$;  because $\int(\chi a)^*\times
v^*<\infty$, the integral is defined and finite (373J).   Of course
$\theta:\frak A^f\to\Bbb R$ is additive because $\chi$ is additive and
$T$, $\times$ and $\int$ are linear.   Also
     
\Centerline{$\lim_{t\downarrow 0}\sup_{\bar\mu a\le t}|\theta a|
\le\lim_{t\downarrow 0}\int_0^tv^*=0$,}
     
\Centerline{$\lim_{t\to\infty}\Bover1t\sup_{\bar\mu a\le t}|\theta a|
\le\lim_{t\to\infty}\bover1t\int_0^tv^*=0$}
     
\noindent because $v\in M^{1,0}_{\bar\nu}$, so $v^*\in
M^{1,0}_{\bar\mu_L}$.   By 373Ha, there is a unique $T'v\in
M^{1,0}_{\bar\mu}$ such
that $\int_aT'v=\theta a$ for every $a\in\frak A^f$.\ \Qed
     
\medskip
     
{\bf (b)} Because the formula uniquely determines $T'v$, we see that
$T':M^{1,0}_{\bar\nu}\to M^{1,0}_{\bar\mu}$ is linear.   Now $T'\in\Cal
T^{(0)}_{\bar\nu,\bar\mu}$.   \Prf\ (i) If $v\in L^1_{\bar\nu}$, then
(because $T'v\in M^{1,0}_{\bar\mu}$) $|T'v|=\sup_{a\in\frak
A^f}|T'v|\times\chi a$, and
     
$$\eqalign{\|T'v\|_1
&=\int|T'v|
=\sup_{a\in\frak A^f}\int_a|T'v|
=\sup_{b,c\in\frak A^f}(\int_bT'v-\int_cT'v)\cr
&=\sup_{b,c\in\frak A^f}\int T(\chi b-\chi c)\times v
\le\sup_{b,c\in\frak A^f}\int(\chi b-\chi c)^*\times v^*\cr
&=\int v^*
=\|v\|_1.\cr}$$
     
\noindent (ii) Now suppose that $v\in L^{\infty}(\frak B)\cap
M^{1,0}_{\bar\nu}$, and set $\gamma=\|v\|_{\infty}$.   \Quer\ If
$a=\Bvalue{|T'v|>\gamma}\ne 0$, then $T'v\ne 0$ so $v\ne 0$ and
$\gamma>0$ and $\bar\mu a<\infty$, because $T'v\in M^{1,0}_{\bar\mu}$.
Set $b=\Bvalue{(T'v)^+>\gamma}$, $c=\Bvalue{(T'v)^->\gamma}$;  then
     
$$\eqalign{\gamma\bar\mu a
&<\int_a|T'v|
=\int_bT'v-\int_cT'v
=\int T(\chi b-\chi c)\times v\cr
&\le\gamma\|T(\chi b-\chi c)\|_1
\le\gamma\|\chi b-\chi c\|_1
=\gamma\bar\mu a,\cr}$$
     
\noindent which is impossible.\ \BanG\
Thus $\Bvalue{|T'v|>\gamma}=0$ and
$\|T'v\|_{\infty}\le\gamma=\|v\|_{\infty}$.
     
Putting this together with (i), we see that $T'\in \Cal
T^{(0)}_{\bar\nu,\bar\mu}$.\ \Qed
     
\medskip
     
{\bf (c)} Let $|T|$ be the modulus of $T$ in
$\eurm L^{\sim}(M^{1,0}_{\bar\mu};M^{1,0}_{\bar\nu})$, so that
$|T|\in\Cal T^{(0)}_{\bar\mu,\bar\mu}$, by 371Gb.   If $u\ge 0$ in
$M^{1,0}_{\bar\mu}$, $v\ge 0$ in $M^{1,0}_{\bar\nu}$ are such that $\int
u^*\times v^*<\infty$, let $\sequencen{u_n}$ be a
non-decreasing sequence in $S(\frak A^f)^+$ with supremum $u$.   In this
case $|T|u=\sup_{n\in\Bbb N}|T|u_n$, so $\int|T|u\times v=\sup_{n\in\Bbb
N}\int|T|u_n\times v$ and
     
\Centerline{$|\int Tu\times v-\int Tu_n\times v|
\le\int|T|(u-u_n)\times v\to 0$}
     
\noindent as $n\to\infty$, because
     
\Centerline{$\int|T|u\times v\le\int u^*\times v^*<\infty$.}
     
\noindent   At the same time,
     
\Centerline{$|\int u\times T'v-\int u_n\times T'v|
\le\int(u-u_n)\times|T'v|\to 0$}
     
\noindent because $\int u\times|T'v|\le\int u^*\times v^*<\infty$.   So
     
\Centerline{$\int Tu\times v=\lim_{n\to\infty}\int Tu_n\times v
=\lim_{n\to\infty}\int u_n\times T'v=\int u\times T'v$,}
     
\noindent the middle equality being valid because each $u_n$ is a linear
combination of indicator functions.
     
Because $T$ and $T'$ are linear, it follows at once that
$\int u\times T'v=\int Tu\times v$ whenever $u\in M^{1,0}_{\bar\mu}$,
$v\in M^{1,0}_{\bar\nu}$ are such that $\int u^*\times v^*<\infty$.
     
\medskip
     
{\bf (d)} Finally, to see that $T'$ is uniquely defined by the formula
in the statement of the theorem, observe that this surely defines
$T'(\chi b)$ for every $b\in\frak B^f$, by the remarks in (a).
Consequently it defines $T'$ on $S(\frak B^f)$.   Since $S(\frak B^f)$
is order-dense in $M^{1,0}_{\bar\nu}$, and any member of
$\Cal T^{(0)}_{\bar\nu,\bar\mu}$ must belong to
$\eurm L^{\times}(M^{1,0}_{\bar\nu};M^{1,0}_{\bar\mu})$ (371Gb), the
restriction of $T'$ to $S(\frak B^f)$ determines $T'$ (355J).
}%end of proof of 373S
     
\leader{373T}{Corollary} Let $(\frak A,\bar\mu)$ and $(\frak B,\bar\nu)$
be localizable measure algebras.   Then for any 
$T\in\Cal T^{\times}_{\bar\mu,\bar\nu}$ there is a unique
$T'\in\Cal T^{\times}_{\bar\nu,\bar\mu}$ such that 
$\int u\times T'v=\int Tu\times v$ whenever 
$u\in M^{1,\infty}(\frak A,\bar\mu)$,
$v\in M^{1,\infty}(\frak B,\bar\nu)$ are such that 
$\int u^*\times v^*<\infty$.
     
\proof{ The restriction $T\restr M^{1,0}_{\bar\mu}$ belongs to
$\Cal T^{(0)}_{\bar\mu,\bar\nu}$ (373Bb), so there is a unique $S\in\Cal
T^{(0)}_{\bar\nu,\bar\mu}$ such that $\int u\times Sv=\int Tu\times v$
whenever
$u\in M^{1,0}_{\bar\mu}$, $v\in M^{1,0}_{\bar\nu}$ are
such that $\int u^*\times v^*<\infty$ (373S).   Now there is a unique
$T'\in\Cal T^{\times}_{\bar\nu,\bar\mu}$ extending $S$ (373R).   If
$u\ge 0$ in $M^{1,\infty}_{\bar\mu}$, $v\ge 0$ in
$M^{1,\infty}_{\bar\nu}$ are such that $\int u^*\times v^*<\infty$, then
$\int u\times T'v=\int Tu\times v$.  \Prf\ If $T\ge 0$, then both are
     
\Centerline{$\sup\{\int u_0\times T'v_0:u_0\in M^{1,0}_{\bar\mu},\,v\in
M^{1,0}_{\bar\nu},\,0\le u_0\le u,\,0\le v_0\le v\}$}
     
\noindent because both $T$ and $T'$ are (order-\nobreak)continuous.   In
general, we can
apply the same argument to $T^+$ and $T^-$, taken in $\eurm
L^{\sim}(M^{1,\infty}_{\bar\mu};M^{1,\infty}_{\bar\nu})$, since these
belong to $\Cal T^{\times}_{\bar\mu,\bar\nu}$, by 373B and 355H,
and we shall surely have $T'=(T^+)'-(T^-)'$.\ \QeD\
As in 373S, it follows that $\int u\times T'v=\int Tu\times v$ whenever
$u\in M^{1,\infty}_{\bar\mu}$, $v \in M^{1,\infty}_{\bar\nu}$ are such
that $\int u^*\times v^*<\infty$.
}%end of proof of 373T
     
\leader{373U}{Corollary} Let $(\frak A,\bar\mu)$ and $(\frak B,\bar\nu)$
be localizable measure algebras, and $\pi:\frak A\to\frak B$ an
order-continuous measure-preserving Boolean homomorphism.   Then the
associated map $T\in\Cal T^{\times}_{\bar\mu,\bar\nu}$\cmmnt{ (373Bd)}
has an adjoint $P\in\Cal T^{\times}_{\bar\nu,\bar\mu}$ defined by the
formula $\int_aP(\chi b)=\bar\nu(b\Bcap\pi a)$ for $a\in\frak A^f$, $b\in\frak B^f$.
     
\proof{ The adjoint $P=T'$ must have the property that
     
\Centerline{$\int_aP(\chi b)=\int \chi a\times P(\chi b)
=\int T(\chi a)\times\chi b=\int\chi(\pi a)\times\chi b
=\bar\nu(\pi a\Bcap b)$}
     
\noindent for every $a\in\frak A^f$, $b\in\frak B^f$.   To see that this
defines $P$ uniquely, let $S\in\Cal T^{\times}_{\bar\nu,\bar\mu}$ be any
other operator with the same property.   By 373Hb, $S(\chi b)=P(\chi b)$
for every $b\in\frak B^f$, so $S$ and $P$ agree on $S(\frak B^f)$.
Because both $P$ and $S$ are supposed to belong to $\eurm
L^{\times}(M^{1,\infty}_{\bar\nu};M^{1,\infty}_{\bar\mu})$, and 
$S(\frak B^f)$ is order-dense in $M^{1,\infty}_{\bar\nu}$, $S=P$, by 355J.
}%end of proof of 373U
     
\exercises{\leader{373X}{Basic exercises (a)}
%\spheader 373Xa
Let $(\frak A,\bar\mu)$ and $(\frak B,\bar\nu)$ be measure algebras, and
$\pi:\frak A\to\frak B$ a {\it ring} homomorphism such that 
$\bar\nu\pi a\le\bar\mu a$ for every $a\in\frak A$.   (i) Show that there is a
unique $T\in\Cal T_{\bar\mu,\bar\nu}$ such that $T(\chi a)=\chi(\pi a)$
for every $a\in\frak A$, and that $T$ is a Riesz homomorphism.   (ii)
Show that $T$ is (sequentially) order-continuous iff $\pi$ is.
%373B
     
\sqheader 373Xb Let $(\frak A,\bar\mu)$ and $(\frak B,\bar\mu)$ be
measure algebras, and $\phi:\Bbb R\to\Bbb R$ a convex function such that
$\phi(0)\le
0$.   Show that if $T\in\Cal T_{\bar\mu,\bar\nu}$ and $T\ge 0$, then
$\bar\phi(Tu)\le T(\bar\phi(u))$ whenever $u\in M^{1,\infty}_{\bar\mu}$
is such that $\bar\phi(u)\in M^{1,\infty}_{\bar\mu}$.   \Hint{233J,
365Rb.}
%373B
     
\spheader 373Xc  Let $(\frak A,\bar\mu)$ be a measure
algebra.    Show that if $w\in L^{\infty}(\frak A)$ and
$\|w\|_{\infty}\le 1$ then 
$u\mapsto u\times w:M^{1,\infty}_{\bar\mu}\to M^{1,\infty}_{\bar\mu}$ belongs to $\Cal T^{\times}_{\bar\mu,\bar\mu}$.
%373B
     
\spheader 373Xd Let $(\frak A,\bar\mu)$ and $(\frak B,\bar\nu)$ be
measure algebras.    Show
that if $\langle a_i\rangle_{i\in I}$, $\langle b_i\rangle_{i\in I}$ are
disjoint families in $\frak A$, $\frak B$ respectively, and $\langle
T_i\rangle_{i\in I}$ is any family in $\Cal T_{\bar\mu,\bar\nu}$, and
{\it either} $I$ is countable {\it or} $\frak B$ is Dedekind complete,
then we have an operator $T\in\Cal T_{\bar\mu,\bar\nu}$ such that
$Tu\times\chi
b_i=T_i(u\times\chi a_i)\times\chi b_i$ for every $u\in
M^{1,\infty}_{\bar\mu,\bar\nu}$, $i\in I$.
%373B
     
\sqheader 373Xe Let $I$, $J$ be sets and write $\mu=\bar\mu$,
$\nu=\bar\nu$
for counting measure on $I$, $J$ respectively.   Show that there is a
natural one-to-one correspondence between $\Cal
T^{\times}_{\bar\mu,\bar\nu}$ and the set of matrices $\langle
a_{ij}\rangle_{i\in I,j\in J}$ such that $\sum_{i\in I}|a_{ij}|\le 1$
for every $j\in J$, $\sum_{j\in J}|a_{ij}|\le 1$ for every $i\in I$.
%373B
     
\sqheader 373Xf Let $(X,\Sigma,\mu)$ and $(Y,\Tau,\nu)$ be
$\sigma$-finite measure spaces, with measure algebras $(\frak
A,\bar\mu)$ and $(\frak B,\bar\nu)$, and product measure $\lambda$ on
$X\times Y$.   Let
$h:X\times Y\to\Bbb R$ be a measurable function such that
$\int|h(x,y)|dx\le
1$ for $\nu$-almost every $y\in Y$ and $\int|h(x,y)|dy\le 1$ for
$\mu$-almost every $x\in X$.   Show that there is a corresponding
$T\in\Cal T^{\times}_{\bar\mu,\bar\nu}$ defined by writing
$T(f^{\ssbullet})=g^{\ssbullet}$ whenever $f\in\eusm L^1(\mu)+\eusm
L^{\infty}(\mu)$ and $g(y)=\int h(x,y)f(x)dx$ for almost every $y$.
%373B
     
\sqheader 373Xg Let $\mu$ be Lebesgue measure on $\Bbb R$, and 
$(\frak A,\bar\mu)$ its measure algebra.   Show that for any 
$\mu$-integrable function $h$ with $\int|h|d\mu\le 1$ we have a 
corresponding $T\in\Cal T^{\times}_{\bar\mu,\bar\mu}$ defined by setting
$T(f^{\ssbullet})=(h*f)^{\ssbullet}$ whenever 
$g\in\eusm L^1(\mu)+\eusm L^{\infty}(\mu)$, writing $h*f$ for the 
convolution of $h$ and $f$
(255E).   Explain how this may be regarded as a special case of 373Xf.
%373B
     
\sqheader 373Xh Let $(\frak A,\bar\mu)$ be a probability algebra and
$u\in L^0(\frak A)^+$;  let $\nu_u$ be its distribution (364Xd).   Show that each of $u^*$, $\nu_u$ is uniquely determined by the other.
%373C
     
\spheader 373Xi Let $(\frak A,\bar\mu)$ and $(\frak B,\bar\nu)$ be
measure algebras, and $\pi:\frak A\to\frak B$ a measure-preserving
Boolean homomorphism;  let $T:M^{1,\infty}_{\bar\mu}\to
M^{1,\infty}_{\bar\nu}$ be the corresponding operator (373Bd).   Show
that $(Tu)^*=u^*$ for every $u\in M^{1,\infty}_{\bar\mu}$.
%373C
     
\spheader 373Xj Let $(\frak A,\bar\mu)$ be a totally finite measure
algebra, and $A$ a subset of $L^1_{\bar\mu}$.   Show that the following
are equiveridical:  (i) $A$ is uniformly integrable;
(ii) $\{u^*:u\in A\}$ is uniformly integrable in $L^1_{\bar\mu_L}$;
(iii) $\lim_{t\downarrow 0}\sup_{u\in A}\int_0^tu^*=0$.
%373D
     
\spheader 373Xk Let $(\frak A,\bar\mu)$ be a measure algebra, and
$A\subseteq (M^0_{\bar\mu})^+$ a non-empty downwards-directed set.
Show that $(\inf A)^*=\inf_{u\in A}u^*$ in $L^0(\frak A_L)$.
%373D
     
\spheader 373Xl Let $(\frak A,\bar\mu)$ be a measure algebra.   Show
that $\|u\|_{1,\infty}=\int_0^1u^*$ for every
$u\in M^{1,\infty}(\frak A,\bar\mu)$.
%373F
     
\spheader 373Xm Let $(\frak A,\bar\mu)$ and $(\frak B,\bar\nu)$ be
measure algebras, and $\phi$ a Young's function
(369Xc).   Write $U_{\phi,\bar\mu}\subseteq L^0(\frak A)$,
$U_{\phi,\bar\nu}\subseteq L^0(\frak B)$ for the corresponding Orlicz
spaces.   (i) Show that if $T\in\Cal T_{\bar\mu,\bar\nu}$ and $u\in
U_{\phi,\bar\mu}$, then $Tu\in U_{\phi,\bar\nu}$ and
$\|Tu\|_{\phi}\le\|u\|_{\phi}$.  (ii) Show that $u\in U_{\phi,\bar\mu}$
iff $u^*\in U_{\phi,\bar\mu_L}$, and in this case
$\|u\|_{\phi}=\|u^*\|_{\phi}$.
%373F
     
\sqheader 373Xn Let $(\frak A,\bar\mu)$ be a measure algebra and 
$(\frak B,\bar\nu)$ a totally finite measure algebra.   Show that if 
$A\subseteq L^1_{\bar\mu}$ is uniformly integrable, then 
$\{Tu:u\in A,\,T\in\Cal T_{\bar\mu,\bar\nu}\}$ is uniformly integrable in
$L^1_{\bar\nu}$.
%373G
     
\spheader 373Xo(i) Give examples of $u$, $v\in L^1(\frak A_L)$ such
that $(u+v)^*\not\le u^*+v^*$.   (ii) Show that if $(\frak A,\bar\mu)$
is any measure algebra and $u$, $v\in M^{0,\infty}_{\bar\mu}$,
then $\int_0^t(u+v)^*\le\int_0^tu^*+v^*$ for every $t\ge 0$.
%373G
     
\spheader 373Xp Let $(\frak A,\bar\mu)$ and $(\frak B,\bar\nu)$ be two
measure algebras.   For $u\in M^{1,0}_{\bar\mu}$, $w\in
M^{\infty,1}_{\bar\nu}$ set
     
\Centerline{$\rho_{uw}(S,T)=|\int(Su-Tu)\times w|$ for all $S$,
$T\in\Cal T^{(0)}=\Cal T^{(0)}_{\bar\mu,\bar\nu}$.}
     
\noindent The topology generated
by the pseudometrics $\rho_{uw}$ is the {\bf very weak operator
topology} on $\Cal T^{(0)}$.   Show that $\Cal T^{(0)}$ is compact in
this topology.
%373L
     
\spheader 373Xq Let $(\frak A,\bar\mu)$ and $(\frak B,\bar\nu)$ be
measure algebras and let $u\in M^{1,0}_{\bar\mu}$.  (i) Show that
$B=\{Tu:T\in\Cal T^{(0)}_{\bar\mu,\bar\nu}\}$ is compact for the
topology $\frak T_s(M^{1,0}_{\bar\nu},M^{\infty,1}_{\bar\nu})$.
(ii) Show that any non-decreasing sequence in $B$ has a
supremum in $L^0(\frak B)$ which belongs to $B$.
%373M
     
\spheader 373Xr Let $(\frak A,\bar\mu)$ and $(\frak B,\bar\nu)$ be
measure algebras, and $u\in M^{1,0}_{\bar\mu}$, $v\in
M^{1,0}_{\bar\nu}$.
Show that the following are equiveridical:  (i) there is a
$T\in\Cal T^{(0)}_{\bar\mu,\bar\nu}$ such that $Tu=v$;  (ii)
$\int_0^tu^*\le\int_0^tv^*$ for every $t\ge 0$.
%373O
     
\spheader 373Xs Let $(\frak A,\bar\mu)$ and $(\frak B,\bar\nu)$ be
measure algebras.   Suppose that $u_1$, $u_2\in M^{1,\infty}_{\bar\mu}$
and $v\in M^{1,\infty}_{\bar\nu}$ are such that
$\int_0^tv^*\le\int_0^t(u_1+u_2)^*$
for every $t\ge 0$.   Show that there are $v_1$, $v_2\in
M^{1,\infty}_{\bar\nu}$ such that $v_1+v_2=v$ and
$\int_0^tv_i^*\le\int_0^tu_i^*$ for both $i$, every $t\ge 0$.
%373O
     
\sqheader 373Xt Set $g(t)=t/(t+1)$ for $t\ge 0$, and set
$v=g^{\ssbullet}$,
$u=\chi[0,1]^{\ssbullet}\in L^{\infty}(\frak A_L)$.   Show that $\int
u^*\times v^*=1>\int Tu\times v$ for every $T\in\Cal
T_{\bar\mu_L,\bar\mu_L}$.
%373P, 373Q
     
\spheader 373Xu Let $(\frak A,\bar\mu)$ and $(\frak B,\bar\nu)$ be
measure algebras, and for $T\in\Cal T^{(0)}_{\bar\mu,\bar\nu}$ define
$T'\in\Cal T^{(0)}_{\bar\nu,\bar\mu}$ as in 373S.   Show that $T''=T$.
%373S
     
\spheader 373Xv Let $(\frak A,\bar\mu)$ and $(\frak B,\bar\nu)$ be
measure algebras, and give $\Cal T^{(0)}_{\bar\mu,\bar\nu}$, $\Cal
T^{(0)}_{\bar\nu,\bar\mu}$ their very weak operator topologies (373Xp).
Show that the map $T\mapsto T':\Cal T^{(0)}_{\bar\mu,\bar\nu}\to\Cal
T^{(0)}_{\bar\nu,\bar\mu}$ is an isomorphism for the convex, order and
topological structures of the two spaces.   (By the `convex
structure' of a convex set $C$ in a linear space I mean the operation
$(x,y,t)\mapsto tx+(1-t)y:C\times C\times[0,1]\to C$.)
%373S
     
     
\leader{373Y}{Further exercises (a)}
%\spheader 373Ya
Let $(\frak A,\bar\mu)$ be the measure algebra of Lebesgue measure on
$[0,1]$.   Set $u=f^{\ssbullet}$ and $v=g^{\ssbullet}$ in $L^0(\frak
A)$, where $f(t)=t$,
$g(t)=1-2|t-\bover12|$ for $t\in[0,1]$.   Show that $u^*=v^*$, but that
there is no measure-preserving Boolean homomorphism $\pi:\frak A\to\frak
A$ such that $T_{\pi}v=u$, writing 
$T_{\pi}:L^0(\frak A)\to L^0(\frak A)$ for the operator induced by $\pi$, 
as in 364P.   \Hint{show that
$\{\Bvalue{v>\alpha}:\alpha\in\Bbb R\}$ does not $\tau$-generate $\frak
A$.}
%373O
     
\spheader 373Yb Let $(\frak A,\bar\mu)$ be a totally finite homogeneous
measure algebra of uncountable Maharam type.   Let $u$,
$v\in(M^{1,\infty}_{\bar\mu})^+$ be such that $u^*=v^*$.   Show
that there is a
measure-preserving automorphism $\pi:\frak A\to\frak A$ such that
$T_{\pi}u=v$.
%373O
     
\spheader 373Yc Let $u$, $v\in M^{1,\infty}_{\bar\mu_L}$ be such that
$u=u^*$, $v=v^*$ and
$\int_0^tv\le\int_0^tu$ for every $t\ge 0$.   Show that there is a
non-negative $T\in\Cal T_{\bar\mu_L,\bar\mu_L}$ such that $Tu=v$ and
$\int_0^tTw\le\int_0^tw$ for every $w\ge 0$ in $M^{1,\infty}$.   Show
that any such $T$ must belong to $\Cal
T^{\times}_{\bar\mu_L,\bar\mu_L}$.
%373O
     
\spheader 373Yd Let $(\frak A,\bar\mu)$ and $(\frak B,\bar\nu)$ be
measure algebras, and $u\in M^{1,\infty}_{\bar\mu}$.  (i) Suppose that
$w\in S(\frak B^f)$.   Show directly (without quoting the result of
373O, but possibly using some of the ideas of the proof) that for every
$\gamma<\int u^*\times w^*$ there is a $T\in\Cal T_{\bar\mu,\bar\nu}$
such that $\int Tu\times w\ge\gamma$.
(ii) Suppose that $(\frak B,\bar\nu)$ is localizable and that $v\in
M^{1,\infty}_{\bar\nu}\setminus\{Tu:T\in\Cal
T_{\bar\mu,\bar\nu}\}$.   Show that there is a $w\in S(\frak B^f)$ such
that $\int v\times w>\sup\{\int Tu\times w:T\in\Cal
T_{\bar\mu,\bar\nu}\}$.   \Hint{use 373M and the Hahn-Banach theorem in
the following form:  if $U$ is a linear space with the topology  $\frak
T_s(U,V)$ defined by
a linear subspace $V$ of $\eurm L(U;\Bbb R)$, $C\subseteq U$ is a
non-empty closed convex set, and $v\in U\setminus C$, then there is an
$f\in V$ such that $f(v)>\sup_{u\in C}f(u)$.}   (iii) Hence prove 373O
for localizable $(\frak B,\bar\nu)$.   (iv) Now prove 373O for general
$(\frak B,\bar\nu)$.
     
\spheader 373Ye(i) Define $v\in L^{\infty}(\frak A_L)$ as in 373Xt.
Show that there is no $T\in\Cal T^{\times}_{\bar\mu_L,\bar\mu_L}$ such
that $Tv=v^*$.   (ii) Set $h(t)=1+\max(0,\bover{\sin t}t)$ for $t>0$,
$w=h^{\ssbullet}\in L^{\infty}(\frak A_L)$.
Show that there is no $T\in\Cal T^{\times}_{\bar\mu_L,\bar\mu_L}$ such
that $Tw^*=w$.
%373Xt, 373P
     
\spheader 373Yf Let $(\frak A,\bar\mu)$ be the measure algebra of
Lebesgue
measure on $[0,1]$.   Show that $\Cal T_{\bar\mu,\bar\mu_L}=\Cal
T^{\times}_{\bar\mu,\bar\mu_L}$ can be identified, as convex ordered
space,
with $\Cal T^{\times}_{\bar\mu_L,\bar\mu}$, and that this is a proper
subset
of $\Cal T_{\bar\mu_L,\bar\mu}$.
%373T
     
\spheader 373Yg Show that the adjoint operation of 373T is not as a rule
continuous for the very weak operator topologies of $\Cal
T^{\times}_{\bar\mu,\bar\nu}$, $\Cal T^{\times}_{\bar\nu,\bar\mu}$.
%373T
}%end of exercises
     
\cmmnt{
\Notesheader{373} 373A-373B are just alternative expressions of concepts
already treated in 371F-371H.   My use of the simpler formula
$\Cal T_{\bar\mu,\bar\nu}$ symbolizes my view that $\Cal T$, rather than
$\Cal T^{(0)}$ or $\Cal T^{\times}$, is the most natural vehicle for
these ideas;  I used $\Cal T^{(0)}$ in \S\S371-372 only because that
made it
possible to give theorems which applied to all measure algebras, without
demanding localizability (compare 371Gb with 373Bc).
     
The obvious examples of operators in $\Cal T$ are those derived from
measure-preserving Boolean homomorphisms, as in 373Bd, and their
adjoints (373U).   Note that the latter include conditional expectation
operators.   In return, we find that operators in $\Cal T$ share some of
the characteristic properties of the operators derived from Boolean
homomorphisms (373Bb, 373Xb, 373Xm).
Other examples are multiplication operators (373Xc), operators obtained
by piecing others together (373Xd) and kernel operators of the type
described in 373Xe-373Xf, including convolution operators (373Xg).
(For a general theory of kernel operators, see \S376 below.)
     
Most of the section is devoted to the relationships between the classes
$\Cal T$ of operators and the `decreasing rearrangements' of 373C.
If you like, the decreasing rearrangement $u^*$ of $u$ describes the
`distribution' of $|u|$ (373Xh);  but for $u\notin M^0$ it loses some
information (373Xt, 373Ye).   It is important to be conscious that even
when $u\in L^0(\frak A_L)$, $u^*$ is not necessarily obtained by
`rearranging' the elements of the algebra $\frak A_L$ by a
measure-preserving automorphism (which would, of course, correspond to
an automorphism of the measure space $(\coint{0,\infty},\mu_L)$, by
344C).   I will treat `rearrangements' of this narrower type in the
next section;  for the moment, see 373Ya.   Apart from this, the basic
properties of decreasing rearrangements are straightforward enough
(373D-373F).   The only obscure area concerns the relationship between
$(u+v)^*$ and $u^*$, $v^*$ (see 373Xo).
     
In 373G I embark on results involving both decreasing rearrangements and
operators in $\Cal T$, leading to the characterization of the sets
$\{Tu:T\in\Cal T\}$ in 373O.   In one direction this is easy, and is the
content of 373G.   In the other direction it depends on a deeper
analysis, and the easiest method seems to be through studying the
`very weak operator topology' on $\Cal T$ (373K-373L), even though this
is an effective tool only when one of the algebras involved is
localizable (373L).   A functional analyst is likely to feel that the
method is both natural and illuminating;  but from the point of view of
a measure theorist it is not perfectly satisfactory, because it is
essentially non-constructive.   While it tells us that there are
operators $T\in\Cal T$ acting in the required ways, it gives only the
vaguest of hints concerning what they actually look like.
     
Of course the very weak operator topology is interesting in its own
right;  and see also 373Xp-373Xq.
     
The proof of 373O can be thought of as consisting of three steps.
Given that $\int_0^tv^*\le\int_0^tu^*$ for every $t$, then I set out to
show that $v$ is expressible as $T_1v^*$ (parts (c)-(d) of the proof),
that $v^*$ is expressible as $T_2u^*$ (part (g)) and that $u^*$ is
expressible as $T_3u$
(parts (e)-(f)), each $T_i$ belonging to an appropriate $\Cal T$.   In
all three steps the general case follows easily from the case of $u$,
$v\in S(\frak A)$, $S(\frak B)$.   If we are willing to use a more
sophisticated version of the
Hahn-Banach theorem than those given in 3A5A and 363R, there is
an alternative route (373Yd).   I note that the central step above, from
$u^*$ to $v^*$, can be performed with an order-continuous $T_2$ (373Yc),
but that in general neither of the other steps can (373Ye), so that we
cannot use $\Cal T^{\times}$ in place of $\Cal T$ here.
     
A companion result to 373O, in that it also shows that
$\{Tu:T\in\Cal T\}$ is large enough to reach natural bounds, is 373P;
given $u$ and
$v$, we can find $T$ such that $\int Tu\times v$ is as large as
possible.   In this form the result is valid only for $v\in M^{(0)}$
(373Xt).  But if we do not demand that the supremum should be attained,
we can deal with other $v$ (373Q).
     
We already know that every operator in $\Cal T^{(0)}$ is a difference of
order-continuous operators, just because $M^{1,0}$ has an
order-continuous norm (371Gb).   It is therefore not surprising that
members of $\Cal T^{(0)}$ can be extended to members of $\Cal T^{\times}$, 
at least when the codomain $M^{1,\infty}_{\bar\nu}$ is
Dedekind complete (373R).   It is also very natural to look for a
correspondence between $\Cal T_{\bar\mu,\bar\nu}$ and 
$\Cal T_{\bar\nu,\bar\mu}$, because if $T\in\Cal T_{\bar\mu,\bar\nu}$ 
we shall surely have an adjoint operator $(T\restr L^1_{\bar\mu})'$ from
$(L^1_{\bar\nu})^*$ to $(L^1_{\bar\mu})^*$, and we can hope that this
will correspond to some member of $\Cal T_{\bar\nu,\bar\mu}$.   But when
we come to the details, the normed-space properties of a general member
of $\Cal T$ are not enough (373Yf), and we need some kind of
order-continuity.   For members of $\Cal T^{(0)}$ this is automatically
present (373S), and now the canonical isomorphism between $\Cal T^{(0)}$
and $\Cal T^{\times}$ gives us an isomorphism between 
$\Cal T^{\times}_{\bar\mu,\bar\nu}$ and 
$\Cal T^{\times}_{\bar\nu,\bar\mu}$
when $\bar\mu$ and $\bar\nu$ are localizable (373T).
}%end of comment
     
\discrpage
     
