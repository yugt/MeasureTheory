\frfilename{mt113.tex}
\versiondate{6.4.05}
\copyrightdate{1999}

\def\chaptername{Measure spaces}
\def\sectionname{Outer measures and \Caratheodory's construction}

\newsection{113}

I introduce the most important method of constructing measures.

\leader{113A}{Outer \dvrocolon{measures}}\cmmnt{ I come now to the
third basic definition of this chapter.

\medskip

\noindent}{\bf Definition} Let $X$ be a set.   An {\bf outer measure} on
$X$ is a function $\theta:\Cal PX\to[0,\infty]$ such that

\qquad(i) $\theta \emptyset=0$,

\qquad(ii) if $A\subseteq B\subseteq X$ then $\theta A\le\theta B$,

\qquad(iii) for every sequence $\langle A_n\rangle_{n\in\Bbb N}$ of
subsets of $X$,
$\theta(\bigcup_{n\in\Bbb N}A_n)\le\sum_{n=0}^{\infty}\theta A_n$.

\cmmnt{
\leader{113B}{Remarks (a)} For comments on the use of
`$\infty$', see 112B.

\header{113Bb}{\bf (b)} Yet again, the most important outer measures
must wait until
\S\S114-115.   The idea of the `outer' measure of a set $A$ is
that it should be some kind of upper bound for the possible measure of
$A$.   If we are lucky, it may actually be the measure of $A$;  but this
is likely to be true only for sets with adequately smooth boundaries.

\header{113Bc}{\bf (c)} Putting (i) and (iii) of the definition
together, we see that
if $\theta$ is an outer measure on $X$, and $A$, $B$ are two subsets of
$X$, then $\theta(A\cup B)\le\theta A+\theta B$;  compare 112Ca and 112Cc.
}%end of comment

\leader{113C}{\Caratheodory's Method:  Theorem} Let $X$ be a set and
$\theta$ an outer measure on $X$.   Set

\Centerline{$\Sigma=\{E:E\subseteq X,\,
\theta A=\theta(A\cap E)+\theta(A\setminus E)$ for every $A\subseteq X\}$.}

\noindent Then $\Sigma$ is a $\sigma$-algebra of subsets of $X$.
Define $\mu:\Sigma\to[0,\infty]$ by writing $\mu E=\theta E$ for
$E\in\Sigma$;  then $(X,\Sigma,\mu)$ is a measure space.

\proof{{\bf (a)} The first step is to note that for any $E$,
$A\subseteq X$ we have
$\theta(A\cap E)+\theta(A\setminus E)\ge\theta A$, by 113Bc;  so that

\Centerline{$\Sigma=\{E:E\subseteq X,\,
\theta A\ge\theta(A\cap E)+\theta(A\setminus E)$
for every $A\subseteq X\}$.}


\medskip

{\bf (b)} Evidently $\emptyset\in\Sigma$, because

\Centerline{$\theta(A\cap\emptyset)+\theta(A\setminus\emptyset)
=\theta\emptyset+\theta A=\theta A$}

\noindent for every $A\subseteq X$.   If $E\in\Sigma$, then
$X\setminus E\in\Sigma$, because

\Centerline{$\theta(A\cap(X\setminus E))+\theta(A\setminus(X\setminus E))
=\theta(A\setminus E)+\theta(A\cap E)=\theta A$}

\noindent for every $A\subseteq X$.

\medskip

{\bf (c)} Now suppose that $E$, $F\in\Sigma$ and $A\subseteq X$.   Then

\ifdim\pagewidth<391pt
\startsideshiftedpicture
\hskip-20pt
\sideshiftedpicture{mt113c1}{-15pt}{90.05pt}{100.68pt}
\sideshiftedpicture{mt113c2}{-15pt}{90.05pt}{100.68pt}
\sideshiftedpicture{mt113c3}{-15pt}{90.05pt}{100.68pt}
\sideshiftedpicture{mt113c4}{-15pt}{90.05pt}{100.68pt}
\else
\startsideshiftedpicture
\hskip10pt
\sideshiftedpicture{mt113c1}{0pt}{97.05pt}{100.68pt}
\sideshiftedpicture{mt113c2}{0pt}{97.05pt}{100.68pt}
\sideshiftedpicture{mt113c3}{0pt}{97.05pt}{100.68pt}
\sideshiftedpicture{mt113c4}{0pt}{97.05pt}{100.68pt}
\fi

\ifdim\pagewidth=390pt\vskip-15pt\fi
\ifdim\pagewidth>467pt\vskip-10pt\fi

$$\eqalignno{\theta(A\cap(&E\cup F))+\theta(A\setminus(E\cup F))
    &\text{diagram (i)}\cr
&=\theta(A\cap(E\cup F)\cap E)+\theta(A\cap(E\cup F)\setminus E)
  +\theta(A\setminus(E\cup F))
    &\text{diag.\ (ii)}\cr
\noalign{\noindent (because $E\in\Sigma$ and
$A\cap(E\cup F)\subseteq X)$}
&=\theta(A\cap E)+\theta((A\setminus E)\cap F)
  +\theta((A\setminus E)\setminus F)\cr
&=\theta(A\cap E)+\theta(A\setminus E)
    &\text{diag.\ (iii)}\cr
\noalign{\noindent (because $F\in\Sigma$)}
&=\theta A
    &\text{diag.\ (iv)}\cr}$$

\noindent (again because $E\in\Sigma$).   Because $A$ is arbitrary,
$E\cup F\in\Sigma$.

\medskip

{\bf (d)} Thus $\Sigma$ is closed under simple unions and complements,
and contains $\emptyset$.
%Observe that it follows that
%$E\setminus F=X\setminus(F\cup(X\setminus E))$ belongs to $\Sigma$ whenever $E$,
%$F\in\Sigma$.
Now suppose that $\langle E_n\rangle_{n\in\Bbb N}$ is a
sequence in $\Sigma$, with $E=\bigcup_{n\in\Bbb N}E_n$.   Set

\Centerline{$G_n=\bigcup_{m\le n}E_m$;}

\noindent then $G_n\in\Sigma$ for each $n$, by induction on $n$.   Set

\Centerline{$F_0=G_0=E_0$,\quad
$F_n=G_n\setminus G_{n-1}=E_n\setminus G_{n-1}$ for $n\ge 1$;}

\noindent  then
%every $F_n$ belongs to $\Sigma$, and
$E=\bigcup_{n\in\Bbb N}F_n=\bigcup_{n\in\Bbb N}G_n$.

Take any $n\ge 1$ and any $A\subseteq X$.   Then

$$\eqalign{\theta(A\cap G_n)
&=\theta(A\cap G_n\cap G_{n-1})+\theta(A\cap G_n\setminus G_{n-1})\cr
&=\theta(A\cap G_{n-1})+\theta(A\cap F_n).\cr}$$

\noindent An induction on $n$ shows that $\theta(A\cap
G_n)=\sum_{m=0}^n\theta(A\cap F_m)$ for every $n\ge 0$.

Suppose that $A\subseteq X$.   Then $A\cap
E=\bigcup_{n\in\Bbb N}A\cap F_n$, so

$$\eqalign{\theta(A\cap E)
&\le\sum_{n=0}^{\infty}\theta(A\cap F_n)\cr
&=\lim_{n\to\infty}\sum_{m=0}^n\theta(A\cap F_m)
=\lim_{n\to\infty}\theta(A\cap G_n).}$$

\noindent On the other hand,

$$\eqalign{\theta(A\setminus E)
&=\theta(A\setminus\bigcup_{n\in\Bbb N}G_n)\cr
&\le\inf_{n\in\Bbb N}\theta(A\setminus G_n)
=\lim_{n\to\infty}\theta(A\setminus G_n),}$$

\noindent using 113A(ii) to see that
$\langle\theta(A\setminus G_n)\rangle_{n\in\Bbb N}$ is non-increasing
and that $\theta(A\setminus E)\le \theta(A\setminus G_n)$ for every $n$.
Accordingly

$$\eqalign{\theta(A\cap E)+\theta(A\setminus E)
&\le\lim_{n\to\infty}\theta(A\cap G_n)
+\lim_{n\to\infty}\theta(A\setminus G_n)\cr
&=\lim_{n\to\infty}(\theta(A\cap G_n)+\theta(A\setminus G_n))
=\theta A\cr}$$

\noindent because every $G_n$ belongs to $\Sigma$, so $\theta(A\cap
G_n)+\theta(A\setminus G_n)=\theta A$ for every $n$.   But $A$ is
arbitrary, so $E\in\Sigma$, by the remark in (a) above.

Because $\langle E_n\rangle_{n\in\Bbb N}$ is arbitrary,
condition (iii) of 111A is satisfied, and $\Sigma$ is a $\sigma$-algebra
of subsets of $X$.

\medskip

{\bf (e)} Now let us turn to $\mu$, the restriction of $\theta$ to
$\Sigma$, and
Definition 112A.   Of course $\mu\emptyset=\theta\emptyset=0$.   So let
$\langle E_n\rangle_{n\in\Bbb N}$ be any disjoint sequence in $\Sigma$.
Set $G_n=\bigcup_{m\le n}E_m$ for each $n$, as in (d), and

\Centerline{$E=\bigcup_{n\in\Bbb N}E_n=\bigcup_{n\in\Bbb N}G_n$.}

\noindent As in (d),

$$\eqalign{\mu G_{n+1}
=\theta G_{n+1}
&=\theta(G_{n+1}\cap E_{n+1})+\theta(G_{n+1}\setminus E_{n+1})\cr
&=\theta E_{n+1}+\theta G_n
=\mu E_{n+1}+\mu G_n\cr}$$

\noindent for each $n$, so $\mu G_n=\sum_{m=0}^n\mu E_m$ for every $n$.

Now

\Centerline{$\mu E
=\theta E
\le\sum_{n=0}^{\infty}\theta E_n
=\sum_{n=0}^{\infty}\mu E_n$.}

\noindent But also

\Centerline{$\mu E=\theta E\ge\theta G_n=\mu G_n=\sum_{m=0}^n\mu E_m$}

\noindent for each $n$, so $\mu E\ge\sum_{n=0}^{\infty}\mu E_n$.

Accordingly $\mu E=\sum_{n=0}^{\infty}\mu E_n$.   As
$\langle E_n\rangle_{n\in\Bbb N}$ is arbitrary, 112A(iii-$\beta$) is
satisfied and $(X,\Sigma,\mu)$ is a measure space.
}%end of proof of 113C

\leader{113D}{Remark} Note\cmmnt{ from (a) in the proof above} that in
this construction

\cmmnt{\Centerline{$\Sigma=\{E:E\subseteq X,\,\theta(A\cap
E)+\theta(A\setminus
E)\le\theta A$ for every $A\subseteq X\}$\dvro{,}{.}}

\noindent Since $\theta(A\cap E)+\theta(A\setminus E)$ is
necessarily less than or equal to $\theta A$ when $\theta A=\infty$,}

\Centerline{$\Sigma=\{E:E\subseteq X,\,\theta(A\cap E)+\theta(A\setminus
E)\le\theta A$ whenever $A\subseteq X$ and $\theta A<\infty\}$.}

\exercises{
\leader{113X}{Basic exercises $\pmb{>}$(a)}
%\spheader 113Xa
Let $X$ be a set and $\theta$ an outer measure on $X$, and let
$\mu$ be the measure on $X$ defined from $\theta$ by \Caratheodory's
method.   Show that if $\theta A=0$, then $\mu$ measures $A$, so that a
set $A\subseteq X$ is $\mu$-negligible iff $\theta A=0$, and $\mu$ is `complete' in
the sense of 112Df.

\spheader 113Xb Let $X$ be a set.
{(i)} Show that if $\theta_1$, $\theta_2$ are outer measures on $X$, so is
$\theta_1+\theta_2$, setting
$(\theta_1+\theta_2)(A)=\theta_1A+\theta_2A$ for every $A\subseteq X$.
{(ii)} Show that if
$\langle\theta_i\rangle_{i\in I}$ is any non-empty family of
outer measures on $X$, so is $\theta=\sup_{i\in I}\theta_i$, setting
$\theta A=\sup_{i\in I}\theta_iA$ for every $A\subseteq X$.   {(iii)}
Show that if $\theta_1$, $\theta_2$ are outer measures on $X$ so is
$\theta_1\wedge\theta_2$, setting

\Centerline{$(\theta_1\wedge\theta_2)(A)
=\inf\{\theta_1B+\theta_2(A\setminus B):B\subseteq A\}$}

\noindent for every $A\subseteq X$.

\sqheader 113Xc Let $X$ and $Y$ be sets, $\theta$ an outer
measure on $X$, and $f:X\to Y$ a function.   Show that the functional
$B\mapsto\theta(f^{-1}[B]):\Cal PY\to[0,\infty]$ is an outer measure on
$Y$.

\sqheader 113Xd Let $X$ be a set and $\theta$ an outer measure
on $X$;  let $Y$ be any subset of $X$.   {(i)} Show that
$\theta\restrp\Cal PY$, the restriction of $\theta$ to subsets of $Y$, is
an outer measure on $Y$.   {(ii)} Show that if $E\subseteq X$ is
measured by the measure on $X$ defined from $\theta$ by
\Caratheodory's method, then $E\cap Y$ is measured by the measure on
$Y$ defined from $\theta\restrp\Cal PY$.

\sqheader 113Xe Let $X$ and $Y$ be sets, $\theta$ an outer
measure on $Y$, and $f:X\to Y$ a function.   Show that the functional
$A\mapsto\theta(f[A]):\Cal PX\to[0,\infty]$ is an outer measure.

\spheader 113Xf Let $X$ and $Y$ be sets, $\theta$ an outer
measure on $X$, and $R\subseteq X\times Y$ a relation.   Show that
the map $B\mapsto\theta(R^{-1}[B]):\Cal PY\to[0,\infty]$ is an outer
measure on $Y$, where $R^{-1}[B]=\{x:\exists\,y\in B,\,(x,y)\in R\}$
(1A1Bc).   Explain how this is a common generalization of (c),
(d-i) and (e) above, and how it can be proved by putting them together.

\spheader 113Xg Let $X$ be a set and $\theta$ an outer measure
on $X$.   Suppose that $E\subseteq X$ is measured by the measure on
$X$ defined from $\theta$ by \Caratheodory's method.   Show that
$\theta(E\cap A)+\theta(E\cup A)=\theta E+\theta A$ for every
$A\subseteq X$.

\spheader 113Xh Let $X$ be a set and $\theta:\Cal PX\to[0,\infty]$ a functional such that $\theta\emptyset=0$, $\theta A\le\theta B$ whenever $A\subseteq B\subseteq X$, and $\theta(A\cup B)\le\theta A+\theta B$ whenever $A$, $B\subseteq X$.   Set

\Centerline{$\Sigma=\{E:E\subseteq X,\,
\theta A=\theta(A\cap E)+\theta(A\setminus E)$ for every
$A\subseteq X\}$.}

\noindent Show that $\emptyset$, $X\setminus E$ and $E\cup F$ belong to
$\Sigma$ for all $E$,
$F\in\Sigma$, so that $E\setminus F$, $E\cap F\in\Sigma$ for all $E$,
$F\in\Sigma$.   Show that $\theta(E\cup F)=\theta E+\theta F$ whenever
$E$, $F\in\Sigma$ and $E\cap F=\emptyset$.

\leader{113Y}{Further exercises (a)}
%\spheader 113Ya
Let $(X,\Sigma,\mu)$ be a measure space.   For $A\subseteq X$
set $\mu^*A=\inf\{\mu E:E\in\Sigma,\,A\subseteq E\}$.   Show that for
every $A\subseteq X$ the infimum is attained, that is, there is an
$E\in\Sigma$ such that $A\subseteq E$ and $\mu E=\mu^*A$.   Show that
$\mu^*$ is an outer measure on $X$.

\spheader 113Yb Let $(X,\Sigma,\mu)$ be a measure space and $D$
any subset of
$X$.   Show that $\Sigma_D=\{E\cap D:E\in\Sigma\}$ is a $\sigma$-algebra
of subsets of $D$.   Set $\mu_D=\mu^*\restr \Sigma_D$, the function with
domain $\Sigma_D$ such that $\mu_DB=\mu^*B$ for every $B\in\Sigma_D$,
where $\mu^*$ is defined as in (a) above;  show that
$(D,\Sigma_D,\mu_D)$ is a measure space.   ($\mu_D$ is the {\bf subspace
measure} on $D$.)

\spheader 113Yc Let $(X,\Sigma,\mu)$ be a measure space and let
$\mu^*$ be the
associated outer measure on $X$, as in 113Ya.   Let $\check\mu$ be the
measure on $X$ constructed by \Caratheodory's method from $\mu^*$, and
$\check\Sigma$ its domain.   Show that $\Sigma\subseteq\check\Sigma$ and
that $\check\mu$ extends $\mu$.

\spheader 113Yd Let $X$ be a set and
$\tau:\Cal PX\to[0,\infty]$ any function such that $\tau\emptyset=0$.
For $A\subseteq X$ set

$$\eqalign{\theta A
=\inf\{\sum_{j=0}\sp{\infty}\tau C_j:\langle C_j\rangle_{j\in\Bbb N}
\text{ is a sequence of }&\text{subsets of }X\cr
&\text{such that }
A\subseteq\bigcup_{j\in\Bbb N}C_j\}.\cr}$$

\noindent Show that $\theta$ is an outer measure on $X$.   \Hint{you
will need 111F(b-ii) or something equivalent.}

\spheader 113Ye Let $X$ be a set and $\theta_1$, $\theta_2$ two
outer measures on $X$.   Show that $\theta_1\wedge\theta_2$, as
described in 113Xb(iii), is the outer measure derived by the process of
113Yd from the functional $\tau C=\min(\theta_1C,\theta_2C)$.

\spheader 113Yf Let $X$ be a set and
$\langle\theta_i\rangle_{i\in I}$ any non-empty family of outer measures
on $X$.   Set $\tau C=\inf_{i\in I}\theta_iC$ for each 
$C\subseteq X$.   Show that the outer measure derived from $\tau$ by the
process of 113Yd is the largest outer measure $\theta$ such that
$\theta A\le\theta_iA$ whenever $A\subseteq X$ and $i\in I$.

\spheader 113Yg Let $X$ be a set and $\phi:\Cal PX\to[0,\infty]$
a functional such that

\inset{$\phi\emptyset=0$;

$\phi(A\cup B)\ge\phi A+\phi B$ for all disjoint $A$, $B\subseteq X$;

if $\sequencen{A_n}$ is a non-increasing sequence of subsets of
$X$ and $\phi A_0<\infty$ then
$\phi(\bigcap_{n\in\Bbb N}A_n)=\lim_{n\to\infty}\phi A_n$;

if $\phi A=\infty$ and $a\in\Bbb R$ there is a $B\subseteq A$ such
that $a\le\phi B<\infty$.}

\noindent Set

\Centerline{$\Sigma=\{E:E\subseteq X,\,\phi(A\cap E)+\phi(A\setminus
E)=\phi A$ for every $A\subseteq X\}$.}

\noindent Show that $(X,\Sigma,\phi\restr\Sigma)$ is a measure space.

\spheader 113Yh Let $(X,\Sigma,\mu)$ be a measure space and for
$A\subseteq X$ set
$\mu_*A=\sup\{\mu E:E\in\Sigma$, $E\subseteq A$, $\mu E<\infty\}$.
Show that $\mu_*:\Cal PX\to[0,\infty]$ satisfies the
conditions of
113Yg, and that if $\mu X<\infty$ then the measure defined from $\mu_*$
by the method of 113Yg extends $\mu$.

\spheader 113Yi Let $X$ be a set and $\Cal A$ an {\bf algebra}
of subsets of $X$, that is, a family of subsets of $X$ such that

\inset{$\emptyset\in\Cal A$,

$X\setminus E\in\Cal A$ for every $E\in\Cal A$,

$E\cup F\in\Cal A$ whenever $E$, $F\in\Cal A$.}

\noindent Let $\phi:\Cal A\to[0,\infty]$ be a function such that

\inset{$\phi\emptyset=0$,

$\phi(E\cup F)=\phi E+\phi F$ whenever $E$, $F\in\Cal A$
and $E\cap F=\emptyset$,

$\phi E=\lim_{n\to\infty}\phi E_n$ whenever
$\sequencen{E_n}$ is a non-decreasing sequence in $\Cal A$ with union
$E$.}

\noindent Show that there is a measure $\mu$ on $X$ extending $\phi$.
\Hint{set $\phi A=\infty$ for $A\in\Cal PX\setminus\Cal A$;
define $\theta$ from $\phi$ as in 113Yd, and $\mu$ from $\theta$.}

\spheader 113Yj (T.de Pauw) Let $X$ be a set, $\Tau$ a
$\sigma$-algebra of subsets of $X$, and $\theta$ an outer measure on
$X$.   Set
$\Sigma=\{E:E\in\Tau,\,\theta A=\theta(A\cap E)+\theta(A\setminus E)$
for every $A\in\Tau\}$.   Show that $\Sigma$ is
a $\sigma$-algebra of subsets of $X$ and that $\theta\restr\Sigma$ is a
measure.

\spheader 113Yk\dvAnew{2010} Let $X$, $\tau:\Cal PX\to[0,\infty]$ and
$\theta$ be as in 113Yd;  let $\mu$ be the measure defined by
\Caratheodory's method from $\theta$, and $\Sigma$ the domain of $\mu$.  
Suppose that $E\subseteq X$ is such that 
$\theta(C\cap E)+\theta(C\setminus E)\le\tau C$ whenever $C\subseteq X$
is such that $0<\tau C<\infty$.   Show that $E\in\Sigma$.
}%end of exercises

\endnotes{
\Notesheader{113} We are proceeding by the easiest stages I can devise
to the construction of a non-trivial measure space, that is, Lebesgue
measure on $\Bbb R$.   There are many constructions of Lebesgue measure,
but in my view \Caratheodory's method (113C) is the right one to begin
with, because it is the most powerful and versatile single technique for
constructing measures.   It is, of course, abstract -- it deals with
arbitrary outer measures on arbitrary sets;  but I really think that the
Lebesgue theory, intertwined as it is with the rich structure of Euclidean
space, is harder than the abstract theory of measure.
We do at least have here a serious theorem for you to get your
teeth into, mastery of which should be both satisfying and useful.   I
must say that I think it very remarkable that such a direct construction
should be effective.    Looking at the proof, it is perhaps worth while
distinguishing between the `algebraic' or `finite' parts ((a)-(c)) and
the parts involving sequences of sets ((d)-(e));   the former amount to
a proof of 113Xh.   Outer measures of various kinds appear throughout
measure theory, and I sketch a few of the relevant constructions in
113X-113Y.
}%end of notes


\discrpage

