\frfilename{mt388.tex}
\versiondate{14.1.04}
\copyrightdate{2001}
     
\def\chaptername{Automorphisms}
\def\sectionname{Dye's theorem}
     
\newsection{388}
     
I have repeatedly said that any satisfactory classification theorem for
automorphisms of measure algebras remains elusive.   There is however a
classification, at least for the Lebesgue measure algebra, of the
`orbit structures' corresponding to measure-preserving automorphisms;
in fact, they are defined by the fixed-point subalgebras, which I
described in \S333.   We have to work hard for this result,
but the ideas are instructive.
     
\leader{388A}{Orbit \dvrocolon{structures}}\cmmnt{ I %3{8}7C
said that this
section was directed to a classification of `orbit structures',
without saying what these might be.   In fact what I will do is to
classify the full subgroups generated by measure-preserving
automorphisms of the Lebesgue measure algebra.   One aspect of the
relation with `orbits' is the following.   (Cf.\ 381Qc.)
     
\medskip
     
\noindent}{\bf Proposition} Let $(X,\Sigma,\mu)$ be a
localizable countably separated
measure space\cmmnt{ (definition:  343D)}, with measure algebra
$(\frak A,\bar\mu)$.   Suppose that
$f$ and $g$ are measure space automorphisms from $X$ to itself, inducing
measure-preserving automorphisms $\pi$, $\phi$ of $\frak A$.   Then the
following are equiveridical:
     
(i) $\phi$ belongs to the full subgroup of $\Aut\frak A$ generated by
$\pi$;
     
(ii) for almost every $x\in X$, there is an $n\in\Bbb Z$ such that
$g(x)=f^n(x)$;
     
(iii) for almost every $x\in X$,
$\{g^n(x):n\in\Bbb Z\}\subseteq\{f^n(x):n\in\Bbb Z\}$.
     
\proof{{\bf(i)$\Rightarrow$(ii)} Let $\sequence{k}{H_k}$ be a sequence
in $\Sigma$ which separates the points of $X$;  we may suppose that
$H_0=X$.   By 381Ib, there is a partition of unity
$\langle a_n\rangle_{n\in\Bbb Z}$ in $\frak A$ such that
$\phi c=\pi^nc$ for every $c\Bsubseteq a_n$, $n\in\Bbb Z$.   For each
$n\in\Bbb Z$ let $E_n\in\Sigma$ be such that $E_n^{\ssbullet}=a_n$;
then $Y_0= \bigcup_{n\in \Bbb Z}E_n$ is conegligible.   The
transformation $f^n$
induces $\pi^n$, so for any $k\in\Bbb N$ and $n\in\Bbb Z$ the set
     
$$\eqalign{F_{nk}
&=\{x:f^n(x)\in E_n\cap H_k,\,g(x)\notin E_n\cap H_k\}\cr
&\qquad\qquad\qquad
\cup\{x:g(x)\in E_n\cap H_k,\,f^n(x)\notin E_n\cap H_k\}\cr}$$
     
\noindent is negligible, and
$Y=g^{-1}[Y_0]\setminus\bigcup_{n\in\Bbb Z,k\in\Bbb N}F_{nk}$ is
conegligible.   Now, for any $x\in Y$, there is
some $n$ such that $g(x)\in E_n$, so that $f^n(x)\in E_n$ and
$\{k:g(x)\in H_k\}=\{k:f^n(x)\in H_k\}$
and $g(x)=f^n(x)$.   As $Y$ is conegligible, (ii) is satisfied.
     
\medskip
     
{\bf (ii)$\Rightarrow$(iii)} For $x\in X$, set
$\Omega_x=\{f^n(x):n\in\Bbb Z\}$;
we are supposing that $A_0=\{x:g(x)\notin \Omega_x\}$ is negligible.
Set $A=\bigcup_{n\in\Bbb Z}g^{-n}[A_0]$, so that $A$ is negligible and
$g^n(x)\in X\setminus A$ for every $x\in X\setminus A$, $n\in\Bbb Z$.
     
Suppose that $x\in X\setminus A$ and $n\in\Bbb N$.   Then
$g^n(x)\in \Omega_x$.   \Prf\ Induce on $n$.   Of course
$g^0(x)=x\in \Omega_x$.   For the inductive
step to $n+1$, $g^n(x)\in\Omega_x\setminus A_0$,
so there is a $k\in\Bbb Z$ such that $g^n(x)=f^k(x)$.   At the same
time, there is an $i\in\Bbb Z$ such that $g(g^n(x))=f^i(g^n(x))$,
so that $g^{n+1}(x)=f^{i+k}(x)\in \Omega_x$.   Thus the induction
continues.\ \Qed
     
Consequently $g^{-n}(x)\in \Omega_x$ whenever $x\in X\setminus A$ and
$n\in\Bbb N$.   \Prf\ Since $g^{-n}(x)\in X\setminus A$, there is a
$k\in\Bbb Z$ such that $x=g^ng^{-n}(x)=f^kg^{-n}(x)$ and
$g^{-n}(x)=f^{-k}(x)\in\Omega_x$.\ \Qed
     
Thus $\{g^n(x):n\in\Bbb Z\}\subseteq \Omega_x$ for every $x$ in the
conegligible set $X\setminus A$.
     
\medskip
     
{\bf (iii)$\Rightarrow$(ii)} is trivial.
     
\medskip
     
{\bf (ii)$\Rightarrow$(i)} Set
     
\Centerline{$E_n=\{x:g(x)=f^n(x)\}
=X\setminus\bigcup_{k\in\Bbb N}(g^{-1}[H_k]\symmdiff f^{-n}[H_k])$,}
     
\noindent for $n\in\Bbb Z$.
Then (ii) tells us that $\bigcup_{n\in\Bbb Z}
E_n$ is conegligible, so $\bigcup_{n\in\Bbb Z}g[E_n]$ is conegligible.
But also each $E_n$ is measurable, so $g[E_n]$ also is, and we can set
$a_n=g[E_n]^{\ssbullet}$.   Now for $y\in g[E_n]$, $y=f^n(g^{-1}(y))$,
that is, $g^{-1}(y)=f^{-n}(y)$;  so $\phi a=\pi^na$ for every
$a\Bsubseteq a_n$.   Since $\sup_{n\in\Bbb Z}a_n=1$ in $\frak A$, $\phi$
belongs to the full subgroup generated by $\pi$.
}%end of proof of 388A
     
\cmmnt{\medskip
     
\noindent{\bf Remark} Of course the requirement `countably separated'
is essential here;  for other measure spaces we can have $\phi$ and
$\pi$ actually equal without $g(x)$ and $f(x)$ being related for any
particular $x$ (see 343I and 343J).
}%end of comment
     
\leader{388B}{Corollary} %3{8}7D
Under the hypotheses of 388A, $\pi$ and $\phi$
generate the same full subgroup of $\Aut\frak A$ iff
$\{f^n(x):n\in\Bbb Z\}=\{g^n(x):n\in\Bbb Z\}$ for almost every $x\in X$.
     
\leader{388C}{}\cmmnt{ Extending some ideas %3{8}7F
from 381M-381N, we have the following fact.
     
\medskip
     
\noindent}{\bf Lemma} Let $(\frak A,\bar\mu)$ be a totally finite
measure algebra, and $\pi:\frak A\to\frak A$ a measure-preserving
automorphism;  let $\frak C$ be its fixed-point subalgebra
$\{c:\pi c=c\}$.
Let $\langle d_i\rangle_{i\in I}$, $\langle e_i\rangle_{i\in I}$ be two
disjoint families in $\frak A$ such that
$\bar\mu(c\Bcap d_i)=\bar\mu(c\Bcap e_i)$ for every $i\in I$ and
$c\in\frak C$.   Then there is a $\phi\in G_{\pi}$, the full subgroup of
$\Aut\frak A$ generated by $\pi$, such that $\phi d_i=e_i$ for every
$i\in I$.
     
\proof{ Adding $d^*=1\Bsetminus\sup_{i\in I}d_i$,
$e^*=1\Bsetminus\sup_{i\in I}e_i$ to the respective families, we may
suppose that $\langle d_i\rangle_{i\in I}$, $\langle e_i\rangle_{i\in
I}$ are partitions of unity.
Define $\sequencen{a_n}$ inductively by the formula
     
\Centerline{$a_n
=\sup_{i\in I}(d_i\setminus\sup_{m<n}a_m)
  \Bcap\pi^{-n}(e_i\Bsetminus\sup_{m<n}\pi^ma_m)$.}
     
\noindent Then $a_n\Bcap d_i\Bcap a_m=0$ whenever $m<n$ and $i\in I$,
so $\sequencen{a_n}$ is disjoint.   Also
     
\Centerline{$\pi^na_n
\Bsubseteq\sup_{i\in I}e_i\Bsetminus\sup_{m<n}\pi^ma_m$}
     
\noindent for each $n$, so $\sequencen{\pi^na_n}$ is disjoint.   Note
that as $\pi^n(a_n\Bcap d_j)\Bsubseteq e_j$ for each $j$,
     
$$\eqalign{\pi^na_n\Bcap e_i
&=\sup_{j\in I}\pi^n(a_n\Bcap d_j)\Bcap e_i
=\sup_{j\in I}\pi^n(a_n\Bcap d_j)\Bcap e_j\Bcap e_i\cr
&=\pi^n(a_n\Bcap d_i)\Bcap e_i
=\pi^n(a_n\Bcap d_i)\cr}$$
     
\noindent for every $i\in I$ and $n\in\Bbb N$.
     
\Quer\ Suppose, if possible, that $a=1\Bsetminus \sup_{n\in\Bbb N}a_n$
is non-zero.   Then there is an $i\in I$ such that $a\Bcap d_i\ne 0$.
Set $c=\sup_{n\in\Bbb N}\pi^n(a\Bcap d_i)$;  then $\pi c\Bsubseteq c$
so $c\in\frak C$.   Now
     
$$\eqalign{\sum_{n=0}^{\infty}\bar\mu(c\Bcap e_i\Bcap\pi^na_n)
&=\sum_{n=0}^{\infty}\bar\mu(c\Bcap\pi^n(a_n\Bcap d_i))
=\sum_{n=0}^{\infty}\bar\mu(\pi^n(c\Bcap a_n\Bcap d_i))\cr
&=\sum_{n=0}^{\infty}\bar\mu(c\Bcap a_n\Bcap d_i)
=\bar\mu(c\Bcap d_i\Bsetminus a)
<\bar\mu(c\Bcap d_i)
=\bar\mu(c\Bcap e_i).\cr}$$
     
\noindent So $b=c\Bcap e_i\Bsetminus\sup_{n\in\Bbb N}\pi^na_n$ is
non-zero, and there is an $n\in\Bbb N$ such that
$b\Bcap\pi^n(a\Bcap d_i)$
is non-zero.   But look at $a'=\pi^{-n}(b\Bcap\pi^n(a\Bcap d_i))$.
We have $0\ne a'\Bsubseteq a\Bcap d_i$, so
$a'\Bsubseteq d_i\Bsetminus \sup_{m<n}a_m$;  while
     
\Centerline{$\pi^na'\Bsubseteq b
  \Bsubseteq e_i\Bsetminus\sup_{m<n}\pi^ma_m$.}
     
\noindent But this means that $a'\Bsubseteq a_n$, which is absurd.\
\Bang
     
This shows that $\sequencen{a_n}$ is a partition of unity in $\frak A$.
Since
     
\Centerline{$\sum_{n=0}^{\infty}\bar\mu(\pi^na_n)
=\sum_{n=0}^{\infty}\bar\mu a_n=\bar\mu 1$,}
     
\noindent $\sequencen{\pi^na_n}$ also is a partition of unity.   We can
therefore define $\phi\in G_{\pi}$ by setting $\phi d=\pi^nd$ whenever
$n\in\Bbb N$ and $d\Bsubseteq a_n$.   Now, for any $i\in I$,
     
\Centerline{$\phi d_i=\sup_{n\in\Bbb N}\phi(d_i\Bcap a_n)
=\sup_{n\in\Bbb N}\pi^n(d_i\Bcap a_n)
=\sup_{n\in\Bbb N}e_i\Bcap\pi^na_n=e_i$.}
     
\noindent So we have found a suitable $\phi$.
}%end of proof of 388C
     
\leader{388D}{von Neumann automorphisms:  Definitions (a)} %3{8}7G
Let $\frak A$ be a Boolean algebra and $\pi\in\Aut\frak A$ an
automorphism.   $\pi$ is {\bf weakly von Neumann} if there is a
sequence $\sequencen{a_n}$ in $\frak A$ such that $a_0=1$ and,
for every $n$, $a_{n+1}\Bcap\pi^{2^n}a_{n+1}=0$,
$a_{n+1}\Bcup\pi^{2^n}a_{n+1}=a_n$.
In this case, $\pi$ is {\bf von Neumann} if $\sequencen{a_n}$ can be
chosen in such a way that
$\{\pi^ma_n:m$, $n\in\Bbb N\}\,\,\tau$-generates $\frak A$, and
{\bf relatively von Neumann} if $\sequencen{a_n}$ can be chosen so that
$\{\pi^ma_n:m$, $n\in\Bbb N\}\cup\{c:\pi c=c\}\,\,\tau$-generates
$\frak A$.
     
\spheader 388Db \cmmnt{ There is another way of looking at
automorphisms
of this type which will be useful.}  If $\frak A$ is a Boolean algebra
and $\pi:\frak A\to\frak A$ an automorphism, then a {\bf dyadic cycle
system} for $\pi$ is a finite or infinite family
$\langle d_{mi}\rangle_{m\le n,i<2^m}$ or
$\langle d_{mi}\rangle_{m\in\Bbb N,i<2^m}$
such that ($\alpha$) for each
$m$, $\langle d_{mi}\rangle_{i<2^m}$ is a partition of unity such that
$\pi d_{mi}=d_{m,i+1}$ whenever $i<2^m-1$\cmmnt{ (so that
$\pi d_{m,2^m-1}$ must be $d_{m0}$)}
($\beta$) $d_{m0}=d_{m+1,0}\Bcup d_{m+1,2^m}$
for every $m<n$ (in the finite case) or
for every $m\in\Bbb N$ (in the infinite case).
An easy induction on $m$ shows that if $k\le m$ then
     
\Centerline{$d_{ki}=\sup\{d_{mj}:j<2^m,\,j\equiv i$ mod $2^k\}$}
     
\noindent for every $i<2^k$.
     
Conversely, if $d$ is such that $\langle\pi^jd\rangle_{j<2^n}$ is a
partition of unity in $\frak A$, then we can form a finite dyadic cycle
system $\langle d_{mi}\rangle_{m\le n,i<2^m}$ by setting
$d_{mi}=\sup\{\pi^jd:j<2^n,\,j\equiv i$ mod $2^m\}$ whenever $m\le n$
and $j<2^m$.
     
\spheader 388Dc Now an
automorphism $\pi:\frak A\to\frak A$ is weakly von Neumann iff
it has an infinite dyadic cycle system
$\langle d_{mi}\rangle_{m\in\Bbb N,i<2^m}$.   \cmmnt{(The $a_m$ of (a)
correspond to the $d_{m0}$ of (b);  starting from the definition in
(a), you must check first, by induction on $m$, that $\langle\pi^ia_m
\rangle_{i<2^m}$ is a partition of unity in $\frak A$.)}   $\pi$ is
von Neumann iff
it has a dyadic cycle system $\langle d_{mi}\rangle_{m\in\Bbb N,i<2^m}$
which $\tau$-generates $\frak A$.
     
\leader{388E}{Example}\cmmnt{ The %3{8}7H
following is the basic example of a
von Neumann transformation -- in a sense, the only example of a
measure-preserving von Neumann transformation.}   Let $\mu$ be the usual
measure on $X=\{0,1\}^{\Bbb N}$, $\Sigma$ its domain,  and
$(\frak A,\bar\mu)$ its measure algebra.   Define $f:X\to X$ by setting
     
$$\eqalign{f(x)(n)&=1-x(n)\text{ if }x(i)=0\text{ for every }i<n,\cr
&=x(n)\text{ otherwise}.\cr}$$
     
\noindent Then $f$ is a homeomorphism and a measure space automorphism.
\prooflet{\Prf\
(i) To see that $f$ is a homeomorphism, perhaps the easiest way is to
look at $g$, where
     
$$\eqalign{g(x)(n)&=1-x(n)\text{ if }x(i)=1\text{ for every }i<n,\cr
&=x(n)\text{ otherwise},\cr}$$
     
\noindent and check that $f$ and $g$ are both continuous and that $fg$
and $gf$ are both the identity function.
(ii) To see that $f$ is \imp, it is enough to check that
$\mu\{x:f(x)(i)=z(i)$
for every $i\le n\}=2^{-n-1}$ for every $n\in\Bbb N$,
$z\in X$ (254G).   But
     
\Centerline{$\{x:f(x)(i)=z(i)$ for every $i\le n\}
=\{x:x(i)=g(z)(i)$ for every $i\le n\}$.}
     
\noindent (iii) Similarly, $g$ is \imp, so $f$ is a
measure space automorphism.\ \Qed}%end of prooflet
     
\cmmnt{If $n\in\Bbb N$, $x\in X$ then
     
$$\eqalign{f^{2^k}(x)(n)&=1-x(n)\text{ if }n\ge k
\text{ and }x(i)=0\text{ whenever }k\le i<n,\cr
&=x(n)\text{ otherwise}.\cr}$$
     
\noindent (Induce on $k$.   For the inductive step, observe that if we
identify $X$ with $\{0,1\}\times X$ then
$f^2(\epsilon,y)=(\epsilon,f(y))$ for every $\epsilon\in\{0,1\}$ and
$y\in X$.)}%end of comment
     
Let $\pi:\frak A\to\frak A$ be the corresponding
automorphism\cmmnt{, setting $\pi E^{\ssbullet}=f^{-1}[E]^{\ssbullet}$
for $E\in\Sigma$}.
Then $\pi$ is a von Neumann automorphism.  \prooflet{\Prf\
Set $E_n=\{x:x\in X,\,x(i)=1$ for every $i<n\}$, $a_n=E_n^{\ssbullet}$.
Then $f^{-2^n}[E_{n+1}]=\{x:x(i)=1$ for $i<n$, $x(n)=0\}$, so
$a_{n+1}$ and $\pi^{2^n}a_{n+1}$ split $a_n$ for each $n$, and
$\sequencen{a_n}$ witnesses that $\pi$ is weakly von Neumann.   Next,
inducing on $n$, we find that $\{f^{-i}[E_n]:i<2^n\}$ runs over the
basic cylinder sets of the form $\{x:x(i)=z(i)$ for every $i<n\}$
determined by coordinates less than
$n$.   Since the equivalence classes of such sets $\tau$-generate
$\frak A$ (see part (a) of the proof of 331K), $\pi$ is a von Neumann
automorphism.\ \Qed}
     
$f$ is sometimes called the {\bf odometer transformation}.
\cmmnt{For another way of looking at the functions $f$ and $g$, see
445Xp in Volume 4.}
     
\leader{388F}{}\cmmnt{ We %3{8}7I
are now ready to approach the main results of this section.
     
\medskip
     
\noindent}{\bf Lemma} Let $(\frak A,\bar\mu)$ be a totally finite
measure algebra and $\pi:\frak A\to\frak A$ an
aperiodic measure-preserving
automorphism.   Let $\frak C$ be its fixed-point subalgebra.
Then for any $a\in\frak A$ there is a $b\Bsubseteq a$ such that
$\bar\mu(b\cap c)=\bover12\bar\mu(a\Bcap c)$ for every $c\in\frak C$
and $\pi_b$ is a weakly von Neumann automorphism,
writing $\pi_b$ for the induced automorphism of the principal ideal
$\frak A_b$\cmmnt{, as in 381M}.
     
\cmmnt{\medskip
     
\noindent{\bf Remark} On first reading, there is something to be said
for supposing here that $\pi$ is ergodic, that is, that
$\frak C=\{0,1\}$.
}%end of comment
     
\proof{ I should remark straight away that $\pi$ is doubly recurrent on
every $b\in\frak A$ (386A), so we have an induced automorphism
$\pi_b:\frak A_b\to\frak A_b$ for every $b\in\frak A$ (381M).
     
\medskip
     
{\bf (a)} Set $\epsilon_n=\bover12(1+2^{-n})$ for each $n\in\Bbb N$, so
that $\sequencen{\epsilon_n}$ is strictly decreasing, with
$\epsilon_0=1$
and $\lim_{n\to\infty}\epsilon_n=\bover12$.   Now there are
$\sequencen{b_n}$, $\langle d_{ni}\rangle_{n\in\Bbb N,i<2^n}$
such that, for each $n\in\Bbb N$,
     
\Centerline{$b_{n+1}\Bsubseteq b_n\Bsubseteq a$,\quad
$\bar\mu(b_n\Bcap c)=\epsilon_n\bar\mu(a\Bcap c)$ for every $c\in\frak
C$,}
     
\Centerline{$\langle d_{ni}\rangle_{i<2^n}$ is disjoint,
\quad$\sup_{i<2^n}d_{ni}
=b_n$,}
     
\Centerline{$\pi_{b_n}d_{ni}=d_{n,i+1}$ for every $i<2^n-1$,}
     
\Centerline{$b_{n+1}\Bcap d_{ni}=d_{n+1,i}\Bcup d_{n+1,i+2^n}$ for every
$i<2^n$.}
     
\noindent\Prf\ Start with $b_0=d_{00}=a$.   To construct $b_{n+1}$ and
$\langle d_{n+1,i}\rangle_{i<2^{n+1}}$, given 
$\langle d_{ni}\rangle_{i<2^n}$,
note first that (because $\pi_{b_n}$
is measure-preserving and $\pi_{b_n}(c\Bcap d)=c\Bcap\pi_{b_n}d$ for
every $d\Bsubseteq b_n$, see 381Nf)
$\bar\mu(d_{n0}\Bcap c)=\bar\mu(d_{ni}\Bcap c)$ whenever $c\in\frak C$,
$i<2^n$, so
     
\Centerline{$\bar\mu(d_{n0}\Bcap c)=2^{-n}\bar\mu(b_n\Bcap c)
=2^{-n}\epsilon_n\bar\mu(a\Bcap c)$}
     
\noindent for every $c\in\frak C$, and
     
\Centerline{$d_{n0}=b_n\Bsetminus\sup_{i<2^n-1}\pi_{b_n}d_{ni}
=\pi_{b_n}d_{n,2^n-1}=\pi^{2^n}_{b_n}d_{n0}$.}
     
\noindent   Now $\pi_{b_n}$ is aperiodic (381Ng)
so $\pi_{b_n}^{2^n}$ also is (381Bd), and there is a
$d_{n+1,0}\Bsubseteq d_{n0}$ such that
     
\Centerline{$\pi_{b_n}^{2^n}d_{n+1,0}\Bcap d_{n+1,0}=0$,
\quad$\bar\mu(d_{n+1,0}\Bcap c)=2^{-n-1}\epsilon_{n+1}\bar\mu(a\Bcap c)$
for every $c\in\frak C$}
     
\noindent (applying 386C(iii) to
$\pi^{2^n}_{b_n}\restrp\frak A_{d_{n0}}$, with
$\gamma=\epsilon_{n+1}/2\epsilon_n$).   Set
$d_{n+1,j}=\pi_{b_n}^jd_{n+1,0}$ for each
$j<2^{n+1}$.   Because
$\pi_{b_n}^{2^n}d_{n+1,0}\Bsubseteq\discretionary{}{}{}
d_{n0}\Bsetminus d_{n+1,0}$, while 
$\langle\pi_{b_n}^jd_{n0}\rangle_{j<2^n}$ is disjoint,
$\langle\pi_{b_n}^jd_{n+1,0}\rangle_{j<2^{n+1}}$ is disjoint.   Set
$b_{n+1}=\sup_{i<2^{n+1}}\pi_{b_n}^id_{n+1,0}$;  then
$b_{n+1}\Bsubseteq b_n$ and
$\bar\mu(b_{n+1}\Bcap c)=\epsilon_{n+1}\bar\mu(a\Bcap c)$ for
every $c\in\frak C$.   For $j<2^{n+1}$, $d_{n+1,j}\Bsubseteq d_{ni}$
where $i$ is either $j$ or $j-2^n$, so
$b_{n+1}\Bcap d_{ni}=d_{n+1,i}\Bcup d_{n+1,i+2^n}$.
     
For $i<2^{n+1}-1$,
     
\Centerline{$\pi_{b_n}d_{n+1,i}=d_{n+1,i+1}\Bsubseteq b_{n+1}$,}
     
\noindent so we must also have
     
\Centerline{$\pi_{b_{n+1}}d_{n+1,i}=(\pi_{b_n})_{b_{n+1}}d_{n+1,i}
=d_{n+1,i+1}$}
     
\noindent (using 381Ne).   Thus the induction continues.\ \Qed
     
\medskip
     
{\bf (b)} Set
     
\Centerline{$b=\inf_{n\in\Bbb N}b_n$,
\quad$e_{ni}=b\Bcap d_{ni}$ for $n\in\Bbb N$,
$i<2^n$.}
     
\noindent Because $\sequencen{b_n}$ is
non-increasing,
     
\Centerline{$\bar\mu(b\Bcap c)=\lim_{n\to\infty}\bar\mu(b_n\Bcap c)
=\Bover12\bar\mu(a\Bcap c)$}
     
\noindent for every $c\in\frak C$.   Next,
     
\Centerline{$e_{ni}=b\Bcap b_{n+1}\Bcap d_{ni}
=b\Bcap(d_{n+1,i}\Bcup d_{n+1,i+2^n})=e_{n+1,i}\Bcup e_{n+1,i+2^n}$}
     
\noindent whenever $i<2^n$.
     
If $m\le n$, $j<2^m$ then
     
\Centerline{$b_n\Bcap d_{mj}=\sup\{d_{ni}:i<2^n,\,i\equiv j$ mod
$2^m\}$}
     
\noindent (induce on $n$).   So
     
\Centerline{$\bar\mu(b_n\Bcap d_{mj})=2^{n-m}\bar\mu d_{n0}
=2^{-m}\epsilon_n$;}
     
\noindent taking the limit as $n\to\infty$, $\bar\mu e_{mj}
=2^{-m}\bar\mu b$.    Next,
     
$$\eqalign{\pi_{b_n}(b_n\Bcap d_{mj})
&=\sup\{d_{n,i+1}:i<2^n,\,i\equiv j\text{ mod }2^m\}\cr
&=\sup\{d_{ni}:i<2^n,\,i\equiv j+1\text{ mod }2^m\}
=b_n\Bcap d_{m,j+1},\cr}$$
     
\noindent here interpreting $d_{n,2^n}$ as $d_{n0}$, $d_{m,2^m}$ as
$d_{m0}$.   Consequently
$\pi_be_{mj}\Bsubseteq e_{m,j+1}$.
\Prf\Quer\ Otherwise, there are a non-zero $e\Bsubseteq d_{mj}\Bcap b$
and $k\ge 1$ such that $\pi^ie\Bcap b=0$ for $1\le i<k$ and
$\pi^ke\Bsubseteq b\Bsetminus d_{m,j+1}$.   Take $n\ge m$ so large that
$\bar\mu e>k\bar\mu(b_n\Bsetminus b)$, so that
     
\Centerline{$e'=e\Bsetminus\sup_{1\le i<k}\pi^{-i}(b_n\Bsetminus b)
\ne 0$;}
     
\noindent now $\pi^ie'\Bcap b_n=0$ for $1\le i<k$, while
$\pi^ke'\Bsubseteq b_n$, and
     
\Centerline{$\pi_{b_n}e'=\pi^ke'\Bsubseteq 1\Bsetminus d_{m,j+1}$.}
     
\noindent But this means that $\pi_{b_n}(b_n\Bcap d_{mj})
\notBsubseteq d_{m,j+1}$, which is impossible.\ \Bang\Qed
     
Since $\bar\mu(\pi_be_{mj})=\bar\mu e_{m,j+1}$, we must have
$\pi_be_{mj}=e_{m,j+1}$.   And this is true whenever $m\in\Bbb N$ and
$j<2^m$, if we identify $e_{m,2^m}$ with $e_{m0}$.
Thus $\langle e_{mi}\rangle_{m\in\Bbb N,i<2^m}$ is a dyadic cycle
system for $\pi_b$ and $\pi_b$ is a weakly von Neumann automorphism.
}%end of proof of 388F
     
\leader{388G}{Lemma} %3{8}7J
Let $(\frak A,\bar\mu)$ be a totally finite measure
algebra and $\pi$, $\psi$ two measure-preserving automorphisms of
$\frak A$.   Suppose that $\psi$ belongs to the full subgroup $G_{\pi}$
of $\Aut\frak A$ generated by $\pi$ and that
there is a $b\in\frak A$ such that $\sup_{n\in\Bbb Z}\psi^nb=1$
and the induced automorphisms $\psi_b$, $\pi_b$ on $\frak A_b$ are
equal.   Then $G_{\psi}=G_{\pi}$.
     
\proof{{\bf (a)} The first fact to note is that if
$0\ne b'\Bsubseteq b$, $n\in\Bbb Z$ and $\pi^nb'\Bsubseteq b$, then
there are $m\in\Bbb Z$,
$b''\Bsubseteq b'$ such that $b''\ne 0$ and $\pi^nd=\psi^md$ for every
$d\Bsubseteq b''$.   \Prf\ ($\alpha$) If $n=0$ take $b''=b'$, $m=0$.
($\beta$) Next, suppose that $n>0$.   We have
$0\ne b'\Bsubseteq b\Bcap\pi^{-n}b$, so by 381Nc there are $i$, $b_1'$
such that $1\le i\le n$, $0\ne b_1'\Bsubseteq b'$ and $\pi^nd=\pi_b^id$
for every $d\Bsubseteq b_1'$.   Now by 381Nb there are a non-zero
$b''\Bsubseteq b_1'$ and an $m\ge i$ such that $\psi_b^id=\psi^md$ for
every $d\Bsubseteq b''$;  so that $\pi^nd=\psi^md$ for every
$d\Bsubseteq b''$.   ($\gamma$) If $n<0$, then apply ($\beta$) to
$\pi^{-1}$ and $\psi^{-1}$, recalling that
$(\pi^{-1})_b=\pi_b^{-1}=\psi_b^{-1}=(\psi^{-1})_b$ (381Na).\ \Qed
     
\medskip
     
{\bf (b)} Now take any non-zero $a\in\frak A$.
Then there are $m$, $n\in\Bbb Z$ such
that $a_1=a\Bcap\psi^mb\ne 0$, $a_2=\pi a_1\Bcap\psi^nb\ne 0$.   Set
$b_1=\psi^{-m}\pi^{-1}a_2$.   Because $\psi\in G_{\pi}$, there are a
non-zero $b_2\Bsubseteq b_1$ and
a $k\in\Bbb Z$ such that $\psi^{-n}\pi\psi^md=\pi^kd$ for every
$d\Bsubseteq b_2$.   Now
     
\Centerline{$\pi^k b_2
=\psi^{-n}\pi\psi^mb_2
\Bsubseteq\psi^{-n}\pi\psi^mb_1
=\psi^{-n}a_2\Bsubseteq b$.}
     
\noindent By (a), there are a non-zero $b_3\Bsubseteq b_2$ and an
$r\in\Bbb Z$
such that $\pi^kd=\psi^rd$ for every $d\Bsubseteq b_3$.   Consider
$a'=\psi^mb_3$.   Then
     
\Centerline{$0\ne a'\Bsubseteq\psi^mb_1
=\pi^{-1}a_2\Bsubseteq a_1\Bsubseteq a$;}
     
\noindent and, for $d\Bsubseteq a'$,
$\psi^{-m}d\Bsubseteq b_3\Bsubseteq b_2$, so that
     
\Centerline{$\pi d=\psi^n(\psi^{-n}\pi\psi^m)\psi^{-m}d
=\psi^n\pi^k\psi^{-m}d
=\psi^{n+r-m}d$.}
     
\noindent As $a$ is arbitrary, this shows that $\pi\in G_{\psi}$, so
that $G_{\pi}\subseteq G_{\psi}$ and the two are equal.
}%end of proof of 388G
     
\leader{388H}{Lemma} %3{8}7K
Let $(\frak A,\bar\mu)$ be a totally finite
measure algebra, $\pi:\frak A\to\frak A$ an aperiodic measure-preserving
automorphism, and $\phi$ any member of the full subgroup $G_{\pi}$ of
$\Aut\frak A$ generated by $\pi$.   Suppose that
$\langle d_{mi}\rangle_{m\le n,i<2^m}$ is a finite dyadic cycle system
for $\phi$.   Then there is a
weakly von Neumann automorphism $\psi$, with dyadic cycle
system $\langle d'_{mi}\rangle_{m\in\Bbb N,i<2^m}$, such that
$G_{\psi}=G_{\pi}$, $\psi a=\phi a$ whenever $a\Bcap d_{n0}=0$, and
$d'_{mi}=d_{mi}$ whenever $m\le n$ and $i<2^m$.
     
\proof{ Write $\frak C$ for the closed subalgebra $\{c:\pi c=c\}$.
By 388F there is a $b\Bsubseteq d_{n0}$ such that
$\bar\mu(b\Bcap c)=\bover12\bar\mu(d_{n0}\Bcap c)$ for every
$c\in\frak C$ and $\pi_b:\frak A_b\to\frak A_b$ is a
weakly von Neumann automorphism.   Let
$\langle e_{ki}\rangle_{k\in\Bbb N,i<2^k}$ be a dyadic cycle system for
$\pi_b$.
     
If we define $\psi_1\in\Aut\frak A$ by setting
     
\Centerline{$\psi_1d=\pi_bd$ for $d\Bsubseteq b$,
\quad$\psi_1d=\pi_{1\Bsetminus b}d$ for $d\Bsubseteq 1\Bsetminus b$,}
     
\noindent then $\psi_1\in G_{\pi}$.   Next, for any $c\in\frak C$,
     
\Centerline{$\bar\mu(\phi^{-2^n+1}b\Bcap c)
=\bar\mu\phi^{-2^n+1}(b\Bcap c)
=\bar\mu(b\Bcap c)
=\Bover12\bar\mu(d_{n0}\Bcap c)
=\bar\mu((d_{n0}\Bsetminus b)\Bcap c)$}
     
\noindent because $\phi^{-2^n+1}\in G_{\pi}$, so $\phi^{-2^n+1}c=c$.
By 388C, there is a $\psi_2\in G_{\pi}$ such that
$\psi_2(d_{n0}\Bsetminus b)=\phi^{-2^n+1}b$.
Set $\psi_3=\phi^{-2^n+1}\psi_2^{-1}\phi^{-2^n+1}\psi_1$,
so that $\psi_3\in G_{\pi}$ and
     
\Centerline{$\psi_3b=\phi^{-2^n+1}\psi_2^{-1}\phi^{-2^n+1}b
=\phi^{-2^n+1}(d_{n0}\Bsetminus b)$.}
     
\noindent Thus $\psi_3b$ and $\psi_2(d_{n0}\Bsetminus b)$ are disjoint
and have union $\phi^{-2^n+1}d_{n0}=d_{n1}$ (if $n=0$, we must read
$d_{01}$ as $d_{00}=1$).   Accordingly we can define
$\psi\in G_{\pi}$ by setting

$$\eqalign{\psi d&=\psi_3d\text{ if }d\Bsubseteq b,\cr
&=\psi_2d\text{ if }d\Bsubseteq d_{n0}\Bsetminus b,\cr
&=\phi d\text{ if }d\Bcap d_{n0}=0.\cr}$$
     
Since $\psi d_{n0}=d_{n1}$, we have $\psi d_{ni}=\phi d_{ni}$ for
every $i<2^n$, and therefore $\psi^id_{m0}=d_{mi}$ whenever
$m\le n$ and $i<2^m$.   Looking at $\psi^{2^n}$, we have
     
\Centerline{$\psi^{2^n}d_{n0}=\phi^{2^n}d_{n0}=d_{n0}$,
\quad$\psi^{2^n}b=\phi^{2^n-1}\psi_3b=d_{n0}\Bsetminus b$,}
     
\noindent so that $\psi^{2^n}(d_{n0}\Bsetminus b)=b$ and
$\psi^{2^{n+1}}b=b$.   Accordingly
     
\Centerline{$\psi^{2^{n+1}}d
=\phi^{2^n-1}\psi_2\phi^{2^n-1}\psi_3d
=\psi_1d=\pi_bd$}
     
\noindent for every $d\Bsubseteq b$.   Also $\sup_{i<2^{n+1}}\psi^ib=1$,
so 388G tells us that $G_{\psi}=G_{\pi}$.
     
Now define $\sequence{m}{a_m}$ as follows.   For $m\le n$, $a_m=d_{m0}$;
for $m>n$, $a_m=e_{m-n-1,0}$.   Then for $m<n$ we have
     
\Centerline{$\psi^{2^m}a_{m+1}=\psi^{2^m}d_{m+1,0}
=\phi^{2^m}d_{m+1,0}=d_{m+1,2^m}=a_m\Bsetminus a_{m+1}$,}
     
\noindent for $m=n$ we have
     
\Centerline{$\psi^{2^n}a_{n+1}=\psi^{2^n}e_{00}
=\psi^{2^n}b=d_{n0}\Bsetminus b=a_n\Bsetminus a_{n+1}$,}
     
\noindent and for $m>n$ we have
     
$$\eqalign{\psi^{2^m}a_{m+1}
&=(\psi^{2^{n+1}})^{2^{m-n-1}}e_{m-n,0}
=(\pi_b)^{2^{m-n-1}}e_{m-n,0}\cr
&=e_{m-n,2^{m-n-1}}
=e_{m-n-1,0}\Bsetminus e_{m-n,0}
=a_m\Bsetminus a_{m+1}.\cr}$$
     
\noindent Thus $\sequence{m}{a_m}$ witnesses that $\psi$ is a weakly
von Neumann automorphism.   If $d'_{mi}=\psi^ia_m$ for $m\in\Bbb N$,
$i<2^m$ then $\langle d'_{mi}\rangle_{m\in\Bbb N,i<2^m}$ will be a
dyadic cycle system for $\psi$ and $d'_{mi}=d_{mi}$ for $m\le n$, as
required.
}%end of proof of 388H
     
\leader{388I}{Lemma} %3{8}7L
Let $(\frak A,\bar\mu)$ be a totally finite
measure algebra and $\frak C$ a closed subalgebra of $\frak A$ such that
$\frak A$ is relatively atomless over $\frak C$.   For $a\in\frak A$
write $\frak C_a=\{a\Bcap c:c\in\frak C\}$.
     
(a) Suppose that $b\in\frak A$, $w\in\frak C$ and $\delta>0$ are such
that $\bar\mu(b\Bcap c)\ge\delta\bar\mu c$ whenever $c\in\frak C$ and
$c\Bsubseteq w$.   Then there is an $e\in\frak A$ such that
$e\Bsubseteq b\Bcap w$ and $\bar\mu(e\Bcap c)=\delta\bar\mu c$ whenever
$c\in\frak C_w$.
     
(b) Suppose that $k\ge 1$ and that $(b_0,\ldots,b_r)$ is a finite
partition of unity in $\frak A$.   Then there is a partition $E$ of
unity in $\frak A$ such that
     
\Centerline{$\bar\mu(e\Bcap c)=\Bover1k\bar\mu c$ for every $e\in E$,
$c\in\frak C$,}
     
\Centerline{$\#(\{e:e\in E,\Exists i\le r,\,b_i\Bcap e\notin\frak C_e\})
\le r+1$.}
     
\proof{{\bf (a)} Set $a=b\Bcap w$ and consider the principal ideal
$\frak A_a$ generated by $\frak A$.   We know that
$(\frak A_a,\bar\mu\restrp\frak A_a)$ is a totally finite measure
algebra (322H), and that $\frak C_a$ is a closed subalgebra of
$\frak A_a$ (333Bc);  and it is easy to see that $\frak A_a$ is
relatively atomless over $\frak C_a$.
     
Let $\theta:\frak C_w\to\frak C_a$ be the Boolean homomorphism defined
by setting $\theta c=c\Bcap b$ for $c\in\frak C_w$.   If $c\in\frak C_w$
and $\theta c=0$, then $c\in\frak C$ and
$\delta\bar\mu c\le\bar\mu(c\Bcap b)=0$, so $c=0$;  thus $\theta$ is
injective;  since it is certainly surjective, it is a Boolean
isomorphism.   We can therefore define a functional
$\nu=\bar\mu\theta^{-1}:\frak C_a\to\coint{0,\infty}$, and we shall have
$\delta\nu d\le\bar\mu d$ for every $d\in\frak C_a$.   By 331B, there is
an $e\in\frak A_a$ such that $\delta\nu d=\bar\mu(d\Bcap e)$ for every
$d\in\frak C_a$, that is, $\delta\bar\mu c=\bar\mu(c\Bcap e)$ for every
$c\in\frak C_w$, as required.
     
\medskip
     
{\bf (b)(i)} Write $D$ for the set of all those $e\in\frak A$ such that
$\bar\mu(c\Bcap e)=\bover1k\bar\mu c$ for every $c\in\frak C$ and
$b_i\Bcap e\in\frak C_e$ for every $i\le r$.   Then whenever
$a\in\frak A$ and $\gamma>\bover{r+1}k$ is such that
$\mu(a\Bcap c)=\gamma\mu c$ for every $c\in\frak C$, there is an $e\in
D$ such that $e\Bsubseteq a$.   \Prf\ For $d\in\frak A$, $c\in\frak C$
set $\nu_d(c)=\bar\mu(d\Bcap c)$, so that
$\nu_d:\frak C\to\coint{0,\infty}$ is a completely additive functional.
For $i\le r$ set
$v_i=\Bvalue{\bar\mu\restrp\frak C>k\nu_{a\Bcap b_i}}$, in the notation
of 326T;  so that $v_i\in\frak C$ and
$\bar\mu c\ge k\mu(a\Bcap b_i\Bcap c)$ whenever $c\in\frak C$ and
$c\Bsubseteq v_i$, while $\bar\mu c\le k\bar\mu(a\Bcap b_i\Bcap c)$
whenever $c\in\frak C$ and $c\Bcap v_i=0$.   Setting
$v=\inf_{i\le r}v_i$, we have
     
\Centerline{$k\gamma\bar\mu v=k\bar\mu(a\Bcap v)
=\sum_{i=0}^rk\mu(a\Bcap b_i\Bcap v)\le(r+1)\bar\mu v$.}
     
\noindent Since $k\gamma>r+1$, $v=0$.   So if we now set
$w_i=(\inf_{j<i}v_j)\Bsetminus v_i$ for $i\le r$ (starting with
$w_0=1\Bsetminus v_0$), $(w_0,\ldots,w_r)$ is a partition of unity in
$\frak C$, and $\bar\mu c\le k\bar\mu(a\Bcap b_i\Bcap c)$ whenever
$c\in\frak C$ and $c\Bsubseteq w_i$.
     
By (a), we can find for each $i\le r$ an $e_i\in\frak A$ such that
$e_i\Bsubseteq a\Bcap b_i\Bcap w_i$ and
$\bar\mu(c\Bcap e_i)=\bover1k\bar\mu c$ whenever $c\in\frak C$ and
$c\Bsubseteq w_i$.   Set $e=\sup_{i\le r}e_i$, so that $e\Bsubseteq a$,
     
\Centerline{$e\Bcap b_i=e\Bcap w_i\Bcap b_i=e_i=e\Bcap w_i\in\frak C_e$}
     
\noindent for each $i$, and
     
\Centerline{$\bar\mu(c\Bcap e)
=\sum_{i=0}^r\bar\mu(c\Bcap e_i)
=\sum_{i=0}^r\bar\mu(c\Bcap w_i\Bcap e_i)
=\sum_{i=0}^r\Bover1k\bar\mu(c\Bcap w_i)
=\Bover1k\bar\mu c$}
     
\noindent for every $c\in\frak C$.   So $e$ has all the properties
required.\ \Qed
     
\medskip
     
\quad{\bf (ii)} Let $E_0\subseteq D$ be a maximal disjoint family, and
set $m=\#(E_0)$, $a=1\Bsetminus\sup E_0$.   Then
     
\Centerline{$\bar\mu(a\Bcap c)
=\bar\mu c-\sum_{e\in E_0}\bar\mu(c\Bcap e)
=(1-\Bover{m}k)\bar\mu c$}
     
\noindent for every $c\in\frak C$, while $a$ does not include any member
of $D$.   By (i), $1-\bover{m}k\le\bover{r+1}k$, that is, $k-m\le r-1$.
     
Applying (a) repeatedly, with $w=1$ and $\delta=\bover1k$, we can find
disjoint $d_0,\ldots,d_{k-m-1}\Bsubseteq a$ such that
$\bar\mu(c\Bcap d_i)=\bover1k\bar\mu c$ for every $c\in\frak C$ and
$i<k-m$.   So if we set $E=E_0\cup\{d_i:i<k-m\}$ we shall have a
partition of unity with the properties required.
}%end of proof of 388I
     
\leader{388J}{Lemma} %3{8}7M
Let $(\frak A,\bar\mu)$ be a totally finite measure algebra and
$\pi:\frak A\to\frak A$ an
aperiodic measure-preserving automorphism, with fixed-point subalgebra
$\frak C$.   Suppose
that $\phi$ is a member of the full subgroup $G_{\pi}$ of $\Aut\frak A$
generated by $\pi$ with a finite dyadic cycle system
$\langle d_{mi}\rangle_{m\le n,i<2^m}$, and that $a\in\frak A$ and
$\epsilon>0$.   Then there is a $\psi\in G_{\pi}$ such that
     
(i) $\psi$ has a dyadic cycle system
$\langle d'_{mi}\rangle_{m\le k,i<2^m}$, with $k\ge n$ and $d'_{mi}
=d_{mi}$ for $m\le n$, $i<2^m$;
     
(ii) $\psi d=\phi d$ if $d\Bcap d_{n0}=0$;
     
(iii) there is an $a'$ in the subalgebra of $\frak A$ generated by
$\frak C\cup\{d'_{ki}:i<2^k\}$ such that
$\bar\mu(a\Bsymmdiff a')\le\epsilon$.
     
\proof{{\bf (a)} Take $k\ge n$ so large that
$2^k\epsilon\ge 2^n2^{2^n}\bar\mu 1$.
Let $\frak D$ be the subalgebra of the principal ideal
$\frak A_{d_{n1}}$
generated by $\{d_{n1}\Bcap\phi^{-j}a:j<2^n\}$;  then $\frak D$ has
atoms $b_0,\ldots,b_r$ where $r<2^{2^n}$.   (If $n=0$, take
$d_{01}=d_{00}=1$.)    Applying 388I to the closed subalgebra
$\frak C_{d_{n1}}$ of $\frak A_{d_{n1}}$, we can find a partition of
unity $E$ of $\frak A_{d_{n1}}$ such that
     
\Centerline{$\bar\mu(e\Bcap c)=2^{-k}\bar\mu c$ for every $e\in E$,
$c\in\frak C$,}
     
\noindent
     
\Centerline{$E_1=\{e:e\in E$, there is some $i\le r$ such that
$b_i\Bcap e\notin\frak C_e\}$}
     
\noindent has cardinal at most $r+1\le 2^{2^n}$.
Of course $\bar\mu e=2^{-k}$ for every $e\in E$, so $\#(E)=2^{k-n}$ and
$\bar\mu(\sup E_1)\le 2^{-k}2^{2^n}\le 2^{-n}\epsilon$.   Write $e^*$
for $\sup E_1$.
     
\medskip
     
{\bf (b)} For $e\in E$ set $e'=\phi^{2^n-1}e$;  then
$\{e':e\in E\}$ is a disjoint
family, of cardinal $2^{k-n}$;  enumerate it as
$\langle v_i\rangle_{i<2^{k-n}}$.   Note that
     
\Centerline{$\sup_{i<2^{k-n}}v_i=\phi^{2^n-1}(\sup E)=d_{n0}$,}
     
\Centerline{$\bar\mu(v_i\Bcap c)
=\bar\mu(\phi^{-2^n+1}v_i\Bcap c)=2^{-k}\bar\mu c$}
     
\noindent for every $c\in\frak C$ and $i<2^{k-n}$.   There is therefore
a $\psi_1\in G_{\pi}$ such that
     
\Centerline{$\psi_1v_i=\phi^{-2^n+1}v_{i+1}$ for $i<2^{k-n}-1$,
\quad$\psi_1v_{2^{k-n}-1}=\phi^{-2^n+1}v_0$}
     
\noindent (388C).   We have
     
$$\eqalign{\psi_1d_{n0}
&=\psi_1(\sup_{i<2^{k-n}}v_i)
=\sup_{i<2^{k-n}}\psi_1v_i
=\sup_{i<2^{k-n}-1}\phi^{-2^n+1}v_{i+1}\Bcup\phi^{-2^n+1}v_0\cr
&=\sup_{i<2^{k-n}}\phi^{-2^n+1}v_i
=\phi^{-2^n+1}d_{n0}
=d_{n1}
=\phi d_{n0}.\cr}$$
     
\noindent So we may define $\psi\in G_{\pi}$ by setting
     
$$\eqalign{\psi d&=\psi_1d\text{ if }d\Bsubseteq d_{n0},\cr
&=\phi d\text{ if }d\Bcap d_{n0}=0.\cr}$$
     
\medskip
     
{\bf (c)} For each $i<2^{k-n}$,
     
\Centerline{$\psi^{2^n}v_i=\phi^{2^n-1}\psi_1v_i=v_{i+1}$}
     
\noindent (identifying $v_{2^{k-n}}$ with $v_0$).   Moreover,
$\psi^jv_i\Bsubseteq d_{nl}$ whenever $i<2^{k-n}$ and
$j\equiv l$ mod $2^n$.   So
$\langle\psi^jv_0\rangle_{j<2^k}$ is a partition of unity in $\frak A$.
What this means is that if we set
     
\Centerline{$d'_{mj}=\sup\{\psi^iv_0:i<2^k,\,i\equiv j$ mod $2^m\}$}
     
\noindent for $m\le k$, then $\langle d'_{mj}\rangle_{m\le k,j<2^m}$ is
a dyadic cycle system for $\psi$, with $d'_{mj}=d_{mj}$ if $m\le n$,
$j<2^m$.
     
\medskip
     
{\bf (d)} Let $\frak B$ be the subalgebra of $\frak A$ generated by
$\frak C\cup\{d'_{kj}:j<2^k\}$.
Recall the definition of $\{v_i:i<2^{k-n}\}$ as
$\{\phi^{2^n-1}e:e\in E\}$;  this implies that
     
\Centerline{$\{\psi v_i:i<2^{k-n}\}=\{\psi_1v_i:i<2^{k-n}\}
=\{\phi^{-2^n+1}v_i:i<2^{k-n}\}=E$,}
     
\noindent so that
     
\Centerline{$\{\psi^{j+1}v_i:i<2^{k-n}\}=\{\phi^je:e\in E\}$}
     
\noindent for $j<2^n$, and
     
\Centerline{$\frak B\supseteq\{d'_{kj}:j<2^k\}
=\{\psi^jv_i:i<2^{k-n},\,j<2^n\}
=\{\phi^je:e\in E,\,j<2^n\}$.}
     
Set $E_0=E\setminus E_1$.   For $e\in E_0$ and $i\le r$ there is a
$c_{ei}\in\frak C$ such that $e\Bcap b_i=e\Bcap c_{ei}$.   Set
     
\Centerline{$K=\{(i,j):1\le i\le r,\,j<2^n,\,
  b_i\Bsubseteq\phi^{-j}a\}$,}
     
\Centerline{$a'=\sup\{\phi^je\Bcap c_{ei}:e\in E_0,\,
  (i,j)\in K\}$.}
     
\noindent Then $a'$ is a supremum of (finitely many)
members of $\frak B$, so belongs to $\frak B$.   If $(i,j)\in K$ and
$e\in E_0$, then
     
\Centerline{$\phi^je\Bcap c_{ei}=\phi^j(e\Bcap c_{ei})
=\phi^j(e\Bcap b_i)\Bsubseteq a$,}
     
\noindent so $a'\Bsubseteq a$.   Next,
$d_{n1}\Bcap\phi^{-j}(a\Bsetminus a')\Bsubseteq e^*$ for each $j<2^n$.
\Prf\ Set
     
\Centerline{$I=\{i:i\le r,\,(i,j)\in K\}
=\{i:b_i\Bsubseteq\phi^{-j}a\}$;}
     
\noindent then $d_{n1}\Bcap\phi^{-j}a=\sup_{i\in I}b_i$.
Now, for each $i\in I$,
     
\Centerline{$b_i=\sup_{e\in E}(b_i\Bcap e)
\Bsubseteq\sup_{e\in E_0}(e\Bcap c_{ei})\Bcup e^*$,}
     
\noindent so that
     
\Centerline{$d_{n1}\Bcap\phi^{-j}a
=\sup_{i\in I}b_i\Bsubseteq\sup_{e\in E_0,i\in I}(e\Bcap c_{ei})
  \Bcup e^*
=(d_{n1}\Bcap\phi^{-j}a')\Bcup e^*$.\ \Qed}
     
\noindent But this means that
     
\Centerline{$\bar\mu(d_{n,j+1}\Bcap a\Bsetminus a')
=\bar\mu(d_{n1}\Bcap\phi^{-j}(a\Bsetminus a'))
\le\bar\mu e^*\le 2^{-n}\epsilon$}
     
\noindent for every $j<2^n$ (interpreting $d_{n,2^n}$ as $d_{n0}$, as
usual), and
     
\Centerline{$\bar\mu(a\Bsymmdiff a')
=\sum_{j=1}^{2^n}\bar\mu(d_{nj}\Bcap a\Bsetminus a')
\le\epsilon$,}
     
\noindent so that the final condition of the lemma is satisfied.
}%end of proof of 388J
     
\leader{388K}{Theorem} %3{8}7N
Let $(\frak A,\bar\mu)$ be a totally finite
measure algebra, with Maharam type $\omega$,
and $\pi:\frak A\to\frak A$ an aperiodic measure-preserving
automorphism.   Then there is a relatively
von Neumann automorphism $\phi:\frak A\to\frak A$ such that $\phi$ and
$\pi$ generate the same full subgroups of $\Aut\frak A$.
     
\proof{{\bf (a)} The idea is to construct $\phi$ as the limit of a
sequence $\sequencen{\phi_n}$ of weakly von Neumann automorphisms such
that $G_{\phi_n}=G_{\pi}$.   Each
$\phi_n$ will have a dyadic cycle system
$\langle d_{nmi}\rangle_{m\in\Bbb N,i<2^m}$;  there will be a strictly
increasing sequence $\sequencen{k_n}$ such that
     
\Centerline{$d_{n+1,m,i}=d_{n,m,i}$ whenever $m\le k_n$, $i<2^m$,}
     
\Centerline{$\phi_{n+1}a=\phi_na$ whenever $a\Bcap d_{n,k_n,0}=0$.}
     
\noindent Interpolated between the $\phi_n$ will be a second sequence
$\sequencen{\psi_n}$ in $G_{\pi}$,
with associated (finite) dyadic cycle systems
$\langle d'_{nmi}\rangle_{m\le k'_n,i<2^m}$.
     
\medskip
     
{\bf (b)} Before starting on the inductive construction we must fix on
a countable set $B\subseteq\frak A$ which $\tau$-generates $\frak A$,
and a sequence $\sequencen{b_n}$ in $B$ such that every member of $B$
recurs cofinally often in the sequence.   (For instance, take the
sequence of first members of an enumeration of $B\times\Bbb N$.)   As
usual, I write $\frak C$ for the closed subalgebra $\{c:\pi c=c\}$.
The induction begins with $\psi_0=\pi$, $k'_0=0$, $d'_{000}=1$.   Given
$\psi_n\in G_{\pi}$ and its dyadic cycle system
$\langle d'_{nmi}\rangle_{m\le k'_n,i<2^m}$, use 388H to find a weakly
von Neumann automorphism $\phi_n$, with dyadic cycle system
$\langle d_{nmi}\rangle_{m\in\Bbb N,i<2^m}$, such that
$G_{\phi_n}=G_{\pi}$, $d_{nmi}=d'_{nmi}$ for $m\le k'_n$ and $i<2^m$,
and $\phi_na=\psi_na$ whenever $a\Bcap d'_{n,k'_n,0}=0$.
     
\medskip
     
{\bf (c)} Given the weakly von Neumann automorphism
$\phi_n$, with its dyadic cycle system
$\langle d_{nmi}\rangle_{m\in\Bbb N,i<2^m}$, such that
$G_{\phi_n}=G_{\pi}$, then we have a partition of
unity $\langle e_{nj}\rangle_{j\in\Bbb Z}$ such that $\pi a=\phi_n^ja$
whenever $j\in\Bbb Z$ and $a\Bsubseteq e_{nj}$ (381I).   Take $r_n$
such that $\bar\mu\tilde e_n\le 2^{-n}$, where
$\tilde e_n=\sup_{|j|>r_n}e_{nj}$,
and $k_n>k'_n$ such that $2^{-k_n}(2r_n+1)\le 2^{-n}$.   Set
     
\Centerline{$e^*_n
=\sup_{|j|\le r_n}\phi_n^{-j}d_{n,k_n,0}$,}
     
\noindent so that $\bar\mu e^*_n\le 2^{-n+1}$.
     
Now use 388J to find a $\psi_{n+1}\in G_{\pi}$, with a dyadic cycle
system $\langle d'_{n+1,m,i}\rangle_{m\le k'_{n+1},i<2^m}$, such that
$k'_{n+1}\ge k_n$, $d'_{n+1,m,i}=d_{nmi}$ if $m\le k_n$,
$\psi_{n+1}a=\phi_na$ if
$a\Bcap d_{n,k_n,0}=0$, and there is a $b'_n$ in the algebra generated
by $\frak C\cup\{d'_{n+1,m,i}:m\le k'_{n+1},\,i<2^m\}$ such that
$\bar\mu(b_n\Bsymmdiff b'_n)\le 2^{-n}$.   Continue.
     
\medskip
     
{\bf (d)} The effect of this construction is to ensure that if $l<n$ in
$\Bbb N$ then
     
\Centerline{$d_{lmi}=d_{nmi}$ whenever $m\le k_l$, $i<2^m$,}
     
\Centerline{$\phi_na=\phi_la$ whenever $a\Bcap d_{l,k_l,0}=0$,}
     
\Centerline{$b'_l$ belongs to the subalgebra generated by
$\frak C\cup\{d_{nmi}:m\le k_n,\,i<2^m\}$,}
     
\noindent and, of course, $d_{n,k_n,0}\Bsubseteq d_{l,k_l,0}$.   Since
$\sequencen{k_n}$ is strictly increasing, $\inf_{n\in\Bbb
N}d_{n,k_n,0}=0$.   Now, for each $n\in\Bbb N$,
     
\Centerline{$d_{n,k_n,1}
=\phi_nd_{n,k_n,0}
=\phi_{n+1}d_{n,k_n,0}\Bsupseteq
\phi_{n+1}d_{n+1,k_{n+1},0}
=d_{n+1,k_{n+1},1}$,}
     
\noindent so setting
     
\Centerline{$a_0=1\Bsetminus d_{0,k_0,0}$,
\quad$a_{n+1}=
d_{n,k_n,0}\Bsetminus d_{n+1,k_{n+1},0}$ for each $n$,}
     
\noindent we have
     
\Centerline{$\phi_0a_0=1\Bsetminus d_{0,k_0,1}$,
\quad$\phi_{n+1}a_{n+1}=d_{n,k_n,1}\Bsetminus d_{n+1,k_{n+1},1}$ for
each
$n$,}
     
\noindent and $\sequencen{\phi_na_n}$ is a partition of unity.
There is therefore
a $\phi\in\Aut\frak A$ defined by setting $\phi a=\phi_na$ if
$a\Bsubseteq a_n$;  because $G_{\pi}$ is full, $\phi\in G_{\pi}$.
     
\medskip
     
{\bf (e)} If $m\le n$, then $a_m\Bcap d_{m,k_m,0}=0$, so
$\phi_na=\phi_ma=\phi a$ for every $a\Bsubseteq a_m$.   Thus
$\phi_na=\phi a$ for
every $a\Bsubseteq\sup_{m\le n}a_m=1\Bsetminus d_{n,k_n,0}$.   In
particular, $\phi d_{nmi}=d_{n,m,i+1}$ whenever $m\le k_n$, $1\le i<2^m$
(counting $d_{n,m,2^m}$ as $d_{nm0}$, as usual);  so that in fact
$\phi d_{nmi}=d_{n,m,i+1}$ whenever $m\le k_n$, $i<2^m$.
     
For each $n$, we have $d_{nmi}=d'_{n+1,m,i}=d_{n+1,m,i}$ whenever
$m\le k_n$ and $i<2^m$.   We therefore have a family
$\langle d^*_{mi} \rangle_{m\in\Bbb N,i<2^m}$ defined by saying that
$d^*_{mi}=d_{nmi}$
whenever $n\in\Bbb N$, $m\le k_n$ and $i<2^m$.   Now, for any
$m\in \Bbb N$, there is a $k_n>m$, so that
$\langle d^*_{mi}\rangle_{i<2^m}
=\langle d_{nmi}\rangle_{i<2^m}$ is a partition of unity;  and
     
\Centerline{$d^*_{mi}=d_{nmi}=d_{n,m+1,i}\Bcup d_{n,m+1,i+2^m}
=d^*_{m+1,i}\Bcup d^*_{m+1,i+2^m}$}
     
\noindent for each $i<2^m$.   Moreover,
     
\Centerline{$\phi d^*_{m,i}=\phi_n d_{nmi}=d_{n,m,i+1}
=d^*_{m,i+1}$}
     
\noindent at least for $1\le i<2^m$ (counting $d^*_{m,2^m}$ as
$d^*_{m,0}$, as usual), so that in fact $\phi d^*_{mi}=d^*_{m,i+1}$
for every $i<2^m$.   Thus $\langle d^*_{mi}\rangle_{m\in\Bbb N,i<2^m}$
is a dyadic cycle system for $\phi$, and $\phi$ is a weakly von Neumann
automorphism.
     
Writing $\frak B$ for the closed subalgebra of $\frak A$ generated by
$\frak C\Bcup\{d^*_{mi}:m\in\Bbb N,\,i<2^m\}$, then
     
$$\eqalign{\frak C\cup \{d'_{nmi}:m\le k'_n,\,i<2^m\}
&=\frak C\cup\{d_{n+1,m,i}:m\le k'_n,\,i<2^m\}\cr
&=\frak C\cup\{d^*_{mi}:m\le k'_n,\,i<2^m\}
\subseteq\frak B\cr}$$
     
\noindent for any $n\in\Bbb N$.   So $b'_n\in\frak B$ for every $n$.
If $b\in B$ and $\epsilon>0$, there is an $n\in\Bbb N$ such that
$2^{-n}\le\epsilon$ and $b_n=b$, so that $\bar\mu(b\Bsymmdiff b'_n)
\le\epsilon$;  as every $b'_n$ belongs to $\frak B$, and $\frak B$
is closed, $b\in\frak B$;  as $b$ is arbitrary, and
$B\,\,\tau$-generates $\frak A$, $\frak B=\frak A$.   Thus $\phi$ is a
relatively von Neumann automorphism.
     
\medskip
     
{\bf (f)} If $n\in\Bbb N$ and $d\Bcap e^*_n=0$, then $\phi^jd=\phi_n^jd$
and $\phi^{-j}d=\phi_n^{-j}d$ whenever $0\le j\le r_n$.   \Prf\ Induce
on
$j$.   For $j=0$ the result is trivial.   For the inductive step to
$j+1\le r_n$, note that if $d'\Bcap d_{n,k_n,1}=0$ then
$\phi_n^{-1}d'\Bcap d_{n,k_n,0}=0$, so
     
\Centerline{$\phi^{-1}d'=\phi^{-1}\phi_n(\phi_n^{-1}d')
=\phi^{-1}\phi(\phi_n^{-1}d')=\phi_n^{-1}d'$.}
     
\noindent Now we have
     
\Centerline{$\phi^{j+1}d=\phi(\phi_n^jd)
=\phi_n(\phi_n^jd)=\phi_n^{j+1}d$}
     
\noindent because
     
\Centerline{$\phi_n^jd\Bcap d_{n,k_n,0}
=\phi_n^j(d\Bcap\phi_n^{-j}d_{n,k_n,0})=0$,}
     
\noindent while
     
\Centerline{$\phi^{-j-1}d=\phi^{-1}(\phi_n^{-j}d)
=\phi_n^{-1}(\phi_n^{-j}d)=\phi_n^{-j-1}d$}
     
\noindent because
     
\Centerline{$\phi_n^{-j}d\Bcap d_{n,k_n,1}
=\phi_n^{-j}(d\Bcap\phi_n^{j+1}d_{n,k_n,0})=0$.  \Qed}
     
\noindent Thus $\phi^jd=\phi_n^jd$ whenever $|j|\le r_n$.
     
\medskip
     
{\bf (g)} Finally, $G_{\phi}=G_{\pi}$.   \Prf\ I remarked in (d) that
$\phi\in G_{\pi}$, so that $G_{\phi}\subseteq G_{\pi}$.   To see that
$\pi\in G_{\phi}$, take any non-zero $a\in\frak A$.   Because
$\bar\mu(e^*_n\Bcup\tilde e_n)\le 2^{-n+1}$ for each $n$, there is an
$n$ such that $a'=a\Bsetminus(e^*_n\Bcup\tilde e_n)\ne 0$.  Now there is
some $j\in\Bbb Z$ such that
$a''=a'\Bcap e_{nj}\ne 0$;  since $a'\Bcap\tilde e_n=0$, $|j|\le r_n$.
If $d\Bsubseteq a''$, then $\pi d=\phi_n^jd$, by the definition of
$e_{nj}$.   But also $\phi_n^jd=\phi^jd$, by (f), because
$d\Bcap e_n^*=0$.   So $\pi d=\phi^jd$ for every $d\Bsubseteq a''$.
As $a$ is arbitrary, $\pi\in G_{\phi}$ and
$G_{\pi}\subseteq G_{\phi}$.\ \Qed
     
This completes the proof.
}%end of proof of 388K
     
\leader{388L}{Theorem} %3{8}7O
Let $(\frak A_1,\bar\mu_1)$ and
$(\frak A_2, \bar\mu_2)$ be totally finite measure algebras of countable
Maharam type, and $\pi_1:\frak A_1 \to\frak A_1$,
$\pi_2:\frak A_2\to\frak A_2$ measure-preserving
automorphisms.   For each $i$, let $\frak C_i$ be the fixed-point
subalgebra of $\pi_i$ and $G_{\pi_i}$ the full subgroup of
$\Aut\frak A_i$ generated by $\pi_i$.   If
$(\frak A_1,\bar\mu_1,\frak C_1)$ and $(\frak A_2,\bar\mu_2,\frak C_2)$
are isomorphic, so are $(\frak A_1,\bar\mu_1,G_{\pi_1})$ and
$(\frak A_2,\bar\mu_2,G_{\pi_2})$.
     
\proof{{\bf (a)} It is enough to consider the case
in which $(\frak A_1,\bar\mu_1,\frak C_1)$ and $(\frak
A_2,\bar\mu_2,\frak C_2)$ are actually equal;  I therefore delete the
subscripts and speak of a structure $(\frak A,\bar\mu,\frak C)$, with
two automorphisms $\pi_1$, $\pi_2$ of $\frak A$ both with
fixed-point subalgebra $\frak C$.
     
\medskip
     
{\bf (b)} Suppose first that $\frak A$ is relatively atomless over
$\frak C$, that is, that both the $\pi_i$ are aperiodic (381P).   In
this case, 388K tells us that there are relatively von Neumann
automorphisms
$\phi_1$ and $\phi_2$ of $\frak A$ such that $G_{\pi_1}=G_{\phi_1}$ and
$G_{\pi_2}=G_{\phi_2}$.   But
$(\frak A,\bar\mu,\phi_1)$ and $(\frak A,\bar\mu,\phi_2)$ are
isomorphic.   \Prf\ Let $\langle d_{mi}\rangle_{m\in\Bbb N,i<2^m}$ and
$\langle d'_{mi}\rangle_{m\in\Bbb N,i<2^m}$ be dyadic cycle systems for
$\phi_1$, $\phi_2$ respectively such that $\frak C\cup\{d_{mi}:m\in\Bbb
N,\,i<2^m\}$ and $\frak C\cup\{d'_{mi}:m\in\Bbb N,\,i<2^m\}$ both
$\tau$-generate $\frak A$.
     
Writing $\frak B_1$, $\frak B_2$ for the subalgebras of $\frak A$
generated by $\frak C\cup\{d_{mi}:m\in\Bbb N,\,i<2^m\}$ and $\frak
C\cup\{d'_{mi}:m\in\Bbb N,\,i<2^m\}$ respectively, it is easy to see
that these algebras are isomorphic:  we just set $\theta_0c=c$ for
$c\in\frak C$, $\theta_0d_{mi}=d'_{mi}$ for $i<2^m$ to obtain a
measure-preserving isomorphism $\theta_0:\frak B_1\to\frak B_2$.
Because these are topologically dense subalgebras of $\frak A$, there
is a unique extension of $\theta_0$ to a measure-preserving
automorphism $\theta:\frak A\to\frak A$ (324O).   Next, we see that
     
\Centerline{$\theta\phi_1\theta^{-1}c=c=\phi_2c$ for every
$c\in\frak C$,}
     
\Centerline{$\theta\phi_1\theta^{-1}d'_{mi}=\theta\phi_1d_{mi}
=\theta d_{m,i+1}=d'_{m,i+1}=\phi_2d'_{mi}$}
     
\noindent for $m\in\Bbb N$, $i<2^m$ (as usual, taking $d_{m,2^m}$ to be
$d_{m0}$ and $d'_{m,2^m}$ to be $d'_{m0}$).   But this means that
$\theta\phi_1\theta^{-1}b=\phi_2b$ for every $b\in\frak B_2$, so (again
because $\frak B_2$ is dense in $\frak A$)
$\theta\phi_1\theta^{-1}=\phi_2$.   Thus $\theta$ is an isomorphism
between $(\frak A,\bar\mu,\phi_1)$ and $(\frak A,\bar\mu,\phi_2)$.
\Qed\
     
Of course $\theta$ is now also an isomorphism between
$(\frak A,\bar\mu,G_{\phi_1})=(\frak A,\bar\mu,G_{\pi_1})$ and
$(\frak A,\bar\mu,G_{\phi_2})=(\frak A,\bar\mu,G_{\pi_2})$.
     
\medskip
     
{\bf (c)} Next, consider the case in which $\pi_1$ is periodic, with
period $n$, for some $n\ge 1$.   In this case $\pi_2\in G_{\pi_1}$.
\Prf\ Let $(d_0,\ldots,d_{n-1})$
be a partition of unity in $\frak A$ such that $\pi_1d_i=d_{i+1}$ for
$i<n-1$ and $\pi_1d_{n-1}=d_0$ (382Fb).   If $d\Bsubseteq d_j$, then
$c=\sup_{i<n}\pi_1^id\in\frak C$ and $d=d_j\Bcap c$;  so any member of
$\frak A$ is of the form $\sup_{j<n}d_j\Bcap c_j$ for some family
$c_0,\ldots,c_{n-1}$ in $\frak C$.
     
If $a\in\frak A\Bsetminus\{0\}$, take $i$,
$j<n$ such that $a'=a\Bcap d_i\Bcap\pi_2^{-1}d_j\ne 0$.   Then any
$d\Bsubseteq a'$ is of the form
     
\Centerline{$d_i\Bcap c_1=\pi_2^{-1}(d_j\Bcap c_2)
=c_2\Bcap\pi_2^{-1}d_j$}
     
\noindent for some $c_1$, $c_2\in\frak C$;  setting $c=c_1\Bcap c_2$, we
have
     
\Centerline{$d=d_i\Bcap c$,
\quad$\pi_2 d=d_j\Bcap c=\pi_1^{j-i}d$.}
     
\noindent As $a$ is arbitrary, this shows that
$\pi_2\in G_{\pi_1}$.\ \Qed
     
Now $\sup_{n\in\Bbb Z}\pi_2^nd_0$
belongs to $\frak C$ and includes $d_0$, so must be $1$.   Finally,
the two induced automorphisms $(\pi_1)_{d_0}$,
$(\pi_2)_{d_0}$ on $\frak A_{d_0}$ are both the identity.   \Prf\
If $0\ne\tilde d\Bsubseteq d_0$ there are a non-zero
$d'\Bsubseteq\tilde d$ and an $m\ge 1$ such that
$(\pi_2)_{d_0}d=\pi_2^md$ for every $d\Bsubseteq d'$.   As
$\pi_2^m\in G_{\pi_1}$, there are a
non-zero $d\Bsubseteq d'$ and a $k\in\Bbb Z$ such that
$\pi_2^md=\pi_1^kd$.   Now $\pi_1^kd\Bsubseteq d_0$ so $k$ is a multiple
of $n$ and $(\pi_2)_{d_0}d=\pi_1^kd=d$.   This shows that
$\{d:(\pi_2)_{d_0}d=d\}$ is order-dense in $\frak A_{d_0}$ and must be
the whole of $\frak A_{d_0}$.   As for $\pi_1$, we have
$(\pi_1)_{d_0}d=\pi_1^nd=d$ for every $d\Bsubseteq d_0$.\ \Qed
     
So 388G tells us that $G_{\pi_1}=G_{\pi_2}$.
     
\medskip
     
{\bf (d)} For the general case, we see from 381H that there is a
partition of unity $\langle c_i\rangle_{1\le i\le\omega}$ in $\frak C$
such that $\pi_1\restrp\frak A_{c_{\omega}}$ is aperiodic and if $i$ is
finite and $c_i\ne 0$ then $\pi_1\restrp\frak A_{c_i}$ is periodic with
period $i$.   For each $i$, let $H_i$ be
$\{\phi\restrp\frak A_{c_i}:\phi\in G_{\pi_1}\}$;  then $H_i$ is a full
subgroup of $\Aut\frak A_{c_i}$, and
     
\Centerline{$G_{\pi_1}=\{\phi:\phi\in\Aut\frak A$,
$\phi\restrp\frak A_{c_i}\in H_i$ whenever $1\le i\le\omega\}$.}
     
\noindent Similarly, writing
$H'_i=\{\phi\restrp\frak A_{c_i}:\phi\in G_{\pi_2}\}$,
     
\Centerline{$G_{\pi_2}=\{\phi:\phi\in\Aut\frak A$,
$\phi\restrp\frak A_{c_i}\in H'_i$ whenever $1\le i\le\omega\}$.}
     
\noindent Note also that $H_i$, $H'_i$ are the full subgroups of
$\Aut\frak A_{c_i}$ generated by $\pi_1\restrp\frak A_{c_i}$,
$\pi_2\restrp\frak A_{c_i}$ respectively.   By (b) and (c), $H_i=H'_i$
for finite $i$, while there is a measure-preserving automorphism
$\theta:\frak A_{c_{\omega}}\to\frak A_{c_{\omega}}$ such that
$\theta H_{\omega}\theta^{-1}=H'_{\omega}$.   Now we can define a
measure-preserving automorphism $\theta_1:\frak A\to\frak A$ by setting
$\theta_1a=\theta a$ if $a\Bsubseteq c_{\omega}$, $\theta_1a=a$ if
$a\Bcap c_{\omega}=0$, and we shall have
$\theta_1G_{\pi_1}\theta_1^{-1}=G_{\pi_2}$.   Thus
$(\frak A,\bar\mu,G_{\pi_1})$ and $(\frak A,\bar\mu,G_{\pi_2})$ are
isomorphic,
as claimed.
}%end of proof of 388L
     
\exercises{\leader{388X}{Basic exercises $\pmb{>}$(a)}
%\spheader 388Xa %3{8}7Xc
Let $(\frak A,\bar\mu)$ be a totally finite measure
algebra, and $\pi:\frak A\to\frak A$ a measure-preserving automorphism.
Let us say that a {\bf pseudo-cycle} for $\pi$ is a partition of unity
$\langle a_i\rangle_{i<n}$, where $n\ge 1$, such that $\pi a_i=a_{i+1}$
for $i<n-1$ (so that $\pi a_{n-1}=a_0$).   (i) Show that if we have
pseudo-cycles $\langle a_i\rangle_{i<n}$ and $\langle b_j\rangle_{j<m}$,
where $m$ is a multiple of $n$, then we have a pseudo-cycle $\langle
c_j\rangle_{j<m}$ with $c_0\Bsubseteq a_0$, so that
$a_i=\sup\{c_j:j<m,\,j\equiv i$ mod $n\}$ for every $i<n$.   (ii) Show
that $\pi$ is weakly von Neumann iff it has a pseudo-cycle of length
$2^n$ for any $n\in\Bbb N$.
%388D
     
\spheader 388Xb %3{8}7Xd
Let $(\frak A_1,\bar\mu_1)$ and $(\frak A_2,\bar\mu_2)$
be probability algebras, and $\pi_1:\frak A_1\to\frak A_2$ and
$\pi_2:\frak A_2\to\frak A_2$ measure-preserving von Neumann
automorphisms.   Show that there is a measure-preserving Boolean
isomorphism $\theta:\frak A_1\to\frak A_2$ such that
$\pi_2=\theta\pi_2\theta^{-1}$.
%388D
     
\spheader 388Xc %3{8}7Xe
Let $(\frak A,\bar\mu)$ be an atomless probability
algebra of countable Maharam type, and $\Aut_{\bar\mu}\frak A$ the
group of measure-preserving Boolean automorphisms of $\frak A$.   Let
$\pi\in\Aut_{\bar\mu}\frak A$ be a von Neumann automorphism.   (i)
Show that for any ultrafilter $\Cal F$ on $\Bbb N$ there is
a $\phi_{\Cal F}\in\Aut_{\bar\mu}\frak A$ defined by the formula
$\phi_{\Cal F}(a)=\lim_{n\to\Cal F}\pi^na$ for every $a\in\frak A$, the
limit being taken in the measure-algebra topology.   (ii) Show that
$\{\phi_{\Cal F}:\Cal F$ is an ultrafilter on $\Bbb N\}$ is a
subgroup of $\Aut_{\bar\mu}\frak A$ homeomorphic to
$\Bbb Z_2^{\Bbb N}$.   \Hint{388E.}
%388E
     
\spheader 388Xd %3{8}7Xf
Let $\frak A$ be a Boolean algebra and
$\pi:\frak A\to\frak A$ a weakly von Neumann automorphism.   Show that
$\pi^n$ is a weakly von Neumann automorphism for every
$n\in\Bbb Z\setminus\{0\}$.   \Hint{consider $n=2$, $n=-1$, odd $n\ge 3$
separately.   The formula of 388E may be useful.}
%388E
     
\spheader 388Xe %3{8}7Xg
Let $\frak A$ be a Boolean algebra and
$\pi:\frak A\to\frak A$ a von Neumann automorphism.   (i) Show that
$\pi^2$ is not ergodic. 
(ii) Show that $\pi^2$ is relatively von
Neumann.   (iii) Show that $\pi^n$ is von Neumann for every odd
$n\in\Bbb Z$.   (iv)\dvAnew{2010} 
Show that if $\frak A$ is a probability algebra (when
endowed with a suitable measure), $\pi$ is ergodic.
%388E 388Xd 388Xb
     
\leader{388Y}{Further exercises (a)}
%\spheader 388Ya %3{8}7Yc
Let $X$ be a set, $\Sigma$ a $\sigma$-algebra of subsets
of $X$, and $\Cal I$ a $\sigma$-ideal of $\Sigma$ such that the quotient
algebra $\frak A=\Sigma/\Cal I$ is Dedekind complete and there is a
countable subset of $\Sigma$ separating the points of $X$.   Suppose
that $f$ and $g$ are automorphisms of the structure $(X,\Sigma,\Cal I)$
inducing $\pi$, $\phi\in\Aut\frak A$.   Show that the following are
equiveridical:  (i) $\phi$ belongs to the full subgroup of $\Aut\frak A$
generated by $\pi$;
(ii) $\{x:x\in X,\,f(x)\notin\{g^n(x):n\in\Bbb Z\}\}\in\Cal I$;
(iii) $\{x:x\in X,\,\{f^n(x):n\in\Bbb Z\}
\not\subseteq\{g^n(x):n\in\Bbb Z\}\}\in\Cal I$.
%388A
     
\spheader 388Yb %3{8}7Yd
Let $(\frak A,\bar\mu)$ be a totally finite measure
algebra and $\pi:\frak A\to\frak A$ a relatively von Neumann
automorphism.   Show that $\pi$ is aperiodic and has zero entropy.
%388D
     
\spheader 388Yc %3{8}7Ye
Let $(\frak A_1,\bar\mu_1)$ and $(\frak A_2,\bar\mu_2)$
be atomless probability algebras, and $(\frak A,\bar\mu)$ their
probability algebra free product.   Let $\pi_1:\frak A_1\to\frak A_1$ be
a measure-preserving von Neumann automorphism and
$\pi_2:\frak A_2\to\frak A_2$ a mixing measure-preserving automorphism.
Let $\pi:\frak A\to\frak A$ be the measure-preserving automorphism such
that $\pi(a_1\otimes a_2)=\pi_1(a_1)\otimes\pi_2(a_2)$ for all
$a_1\in\frak A_1$, $a_2\in\frak A_2$.   Show that $\pi$ is an ergodic
weakly von Neumann automorphism which is not a relatively von Neumann
automorphism.
%388E
     
\spheader 388Yd %3{8}7Yf
Let $\mu$ be Lebesgue measure on $[0,1]^2$, and
$(\frak A,\bar\mu)$ its measure algebra;  let $\frak C$ be the closed
subalgebra
of elements expressible as $(E\times[0,1])^{\ssbullet}$, where
$E\subseteq [0,1]$ is measurable.   Suppose that $\pi:\frak A\to\frak A$
is a measure-preserving automorphism such that $\frak C=\{c:\pi c=c\}$.
Show that there
is a family $\langle f_x\rangle_{x\in[0,1]}$ of ergodic measure space
automorphisms of $[0,1]$ such that $(x,y)\mapsto (x,f_x(y))$ is a
measure space automorphism of $[0,1]^2$ representing $\pi$.
     
\spheader 388Ye %3{8}7Yg
Show that the odometer transformation on
$\{0,1\}^{\Bbb N}$ is expressible as the product of
two Borel measurable measure-preserving involutions.
%388E
     
\spheader 388Yf Let $(\frak A,\bar\mu)$ be a probability algebra and
$\pi\in\Aut\frak A$ a relatively von Neumann automorphism;  let
$T=T_{\pi}:L^1_{\bar\mu}\to L^1_{\bar\mu}$ be the corresponding Riesz
homomorphism (365O).   (i) Show that $\bigcup_{n\ge 1}\{u:T^nu=u\}$ is
dense in $L^1_{\bar\mu}$.   (ii) Show that $\{T^n:n\in\Bbb Z\}$ is
relatively compact in $\eurm B(L^1_{\bar\mu};L^1_{\bar\mu})$ for the
strong operator topology.

\spheader 388Yg\dvAnew{2010} Give an example of a ccc Dedekind complete
Boolean algebra $\frak A$ and a von Neumann automorphism 
$\pi\in\Aut\frak A$ which is not ergodic.
%388Xe 388E mt38bits
}%end of exercises
     
\cmmnt{
\Notesheader{388} Dye's theorem ({\smc Dye 59}) is actually Theorem 388L
in the case in which $\pi_1$, $\pi_2$ are ergodic, that is, in which
$\frak C_1$ and $\frak C_2$ are both trivial.   I take the trouble to
give the generalized form here (a simplified version of that in {\smc
Krieger 76}) because it seems a natural target, once
we have a classification of the relevant structures
$(\frak A,\bar\mu,\frak C)$ (333R).   The essential mathematical ideas
are the same in both cases.   You can find the special case worked out
in {\smc Hajian Ito \& Kakutani 75}, from which I have taken the
argument used
here;  and you may find it useful to go through the version above, to
check what kind of simplifications arise if each $\frak C$ is taken to
be $\{0,1\}$.   Essentially the difference will be that every
`aperiodic' turns into `ergodic' (with an occasional `atomless'
thrown in) and `331B' turns into `331C'.   As far as I know,
there is no simplification available in the structure of the argument;
of course the details become a bit easier, but with the possible
exception of 388I-388J I think there is little difference.
     
Of course modifying a general argument to give a simpler proof of a
special case is a standard exercise in this kind of mathematics.   What
is much more interesting is the reverse process.   What kinds of theorem
about ergodic automorphisms will in fact be true of all automorphisms?
A variety of very powerful approaches to such questions have been
developed in the last half-century, and I hope to describe some of the
ideas in Volumes 4 and 5.   The methods used in this section are
relatively straightforward and do not require any deep theoretical
underpinning beyond Maharam's lemma 331B.   But an alternative
approach
can be found using 388Yd:  in effect (at least for the Lebesgue measure
algebra) any measure-preserving automorphism can be disintegrated into
ergodic measure space automorphisms (the fibre maps $f_x$ of 388Yd).
It is sometimes possible to guess which theorems about ergodic
transformations are `uniformisable' in the sense that they can be
applied to such a family $\langle f_x\rangle_{x\in[0,1]}$, in a
systematic way, to provide a structure which can be interpreted on the
product measure.   The details tend to be complex, which is one of the
reasons why I do not attempt to work through them here;  but such
disintegrations can be a most valuable aid to intuition.
     
In this section I use von Neumann automorphisms as an auxiliary tool:
the point is, first, that two von Neumann automorphisms are isomorphic
-- that is, the von Neumann automorphisms on a given totally finite
measure algebra $(\frak A,\bar\mu)$ (necessarily isomorphic to the
Lebesgue measure algebra, since we must have $\frak A$ atomless and
$\tau(\frak A)=\omega$) form a conjugacy class in the group
$\Aut_{\bar\mu}\frak A$ of measure-preserving automorphisms;  and next,
that for any ergodic measure-preserving automorphism $\pi$ (on an
atomless totally finite algebra of countable Maharam type) there is a
von Neumann automorphism $\phi$ such that $G_{\pi}=G_{\phi}$ (388K).
But I think they are remarkable in themselves.   A
(weakly) von Neumann automorphism has a `pseudo-cycle' (388Xa) for
every power of 2.   For some purposes, existence is all we need to know;
but in the arguments of 388H-388K we need to keep track of named
pseudo-cycles in what I call `dyadic cycle systems' (388D).
     
In this volume I have systematically preferred arguments which deal
directly with measure algebras, rather than with measure spaces.   I
believe that such arguments can have a simplicity and clarity which
repays the extra effort of dealing with more abstract structures.   But
undoubtedly it is necessary, if you are to have any hope of going
farther in the subject, to develop methods of transferring intuitions
and theorems between the two contexts.   I offer 381Xl as an example.
The description there of `induced automorphism' requires a certain
amount of manoeuvering around negligible sets, but gives a
valuably graphic description.   In the same way, 381Xf, 388A and 381Qc
provide alternative ways of looking at full subgroups.
     
There are contexts in which it is useful to know whether an element of
the full subgroup generated by $\pi$ actually belongs to the full
semigroup generated by $\pi$ (381Yb);  for instance, this happens in
388C.
}%end of comment
     
     
\frnewpage
     
