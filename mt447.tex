\frfilename{mt447.tex}
\versiondate{7.1.10}
\copyrightdate{1998}

\def\undphi{\underline{\phi}\vthsp}
\def\undpsi{\underline{\psi}\vthsp}
\def\undtheta{\underline{\theta}}
\def\BbbR{\mathchoice{\hbox{$\Bbb R$\hskip0.02em}}
  {\hbox{$\Bbb R$\hskip0.02em}}
  {\hbox{$\scriptstyle\Bbb R$\hskip0.04em}}
  {\hbox{$\scriptscriptstyle\Bbb R$\hskip0.04em}}}

\def\chaptername{Topological groups}
\def\sectionname{Translation-invariant liftings}

\newsection{447}

I devote a section to the main theorem of
{\smc Ionescu Tulcea \& Ionescu Tulcea 67}:
a group carrying Haar measures has a
translation-invariant lifting (447J).   The argument uses an inductive
construction of the same type as that used in \S341 for the ordinary
Lifting Theorem.   It depends on the structure theory for locally
compact groups described in \S446.   On the way I describe a Vitali
theorem for certain metrizable groups (447C), with a corresponding
density theorem (447D).


\leader{447A}{Liftings and lower densities} Let $X$ be a group carrying
Haar measures, $\Sigma$ its algebra of Haar measurable sets and $\frak A$
its Haar measure algebra\cmmnt{ (442H, 443A)}.

\spheader 447Aa Recall that a {\bf lifting} of $\frak A$ is either a
Boolean homomorphism $\theta:\frak A\to\Sigma$ such that
$(\theta a)^{\ssbullet}=a$ for every $a\in\frak A$,
or a Boolean homomorphism
$\phi:\Sigma\to\Sigma$ such that $E\symmdiff\phi E$ is Haar negligible
for every $E\in\Sigma$ and $\phi E=\emptyset$ whenever $E$ is Haar
negligible\cmmnt{ (341A)}.   Such a lifting $\theta$ or $\phi$ is {\bf
\lti} if $\theta((xE)^{\ssbullet})=x(\theta E^{\ssbullet})$ or
$\phi(xE)=x(\phi E)$ for every $E\in\Sigma$ and $x\in X$.
\cmmnt{(In the notation of 443C, a lifting $\theta:\frak A\to\Sigma$
is \lti\ if $\theta(x\action_la)=x\theta a$ for every $x\in X$,
$a\in\frak A$.)}

\cmmnt{The language of 341A demanded a named measure;  I spoke there
of a lifting for a measure space $(X,\Sigma,\mu)$ or a measure $\mu$.
But (as noted in 341Lh) what the concept really depends on is a triple
$(X,\Sigma,\Cal I)$, where $\Sigma$ is an algebra of subsets of $X$ and
$\Cal I$ is an ideal of $\Sigma$.   Variations in the measure which do
not affect the algebra of measurable sets or the null ideal are
irrelevant.   So, in the present context, we can speak of a
`lifting for Haar measure' without declaring which Haar measure we are
using, nor even whether it is a left or right Haar measure.}

\spheader 447Ab Now suppose that $\Sigma_0$ is a $\sigma$-subalgebra of
$\Sigma$.   In this case, a {\bf partial lower density} on $\Sigma_0$ is
a function $\undphi:\Sigma_0\to\Sigma$ such that $\undphi E=\undphi F$
whenever $E$, $F\in\Sigma_0$ and $E\symmdiff F$ is negligible,
$E\symmdiff\undphi E$ is negligible for every $E\in\Sigma_0$,
$\undphi\emptyset=\emptyset$ and
$\undphi(E\cap F)=\undphi E\cap\undphi F$ for all $E$, $F\in\Sigma_0$.
\cmmnt{(See 341C-341D.)}  As in (a), such
a function is {\bf \lti} if $xE\in\Sigma_0$ and
$\undphi(xE)=x(\undphi E)$ for every $x\in X$ and $E\in\Sigma_0$.

\leader{447B}{Lemma} Let $X$ be a group carrying Haar measures and $Y$ a
subgroup of $X$.   Write $\Sigma_Y$ for the algebra of Haar
measurable subsets $E$ of $X$ such that $EY=E$, and suppose that
$\undphi:\Sigma_Y\to\Sigma_Y$ is a \lti\ partial lower density.   Then
$G\subseteq\undphi(GY)$ for every open set $G\subseteq X$.

\proof{ Of course $GY$ is open (4A5Ed), so belongs to $\Sigma_Y$.   Let
$a\in G$ and let $U$ be an open neighbourhood of the identity in $X$
such that $U^{-1}Ua\subseteq G$.
Then $UaY$ is a non-empty open set, therefore not negligible (442Aa),
and there is an $x\in UaY\cap\undphi(UaY)$.   Express $x$ as $uay$ where
$u\in U$ and $y\in Y$;  then $ua=xy^{-1}\in Ua\cap\undphi(UaY)$, because
$\undphi(UaY)\in\Sigma_Y$.   So

\Centerline{$a=u^{-1}ua\in u^{-1}\undphi(UaY)
=\undphi(u^{-1}UaY)\subseteq\undphi(GY)$.}

\noindent As $a$ is arbitrary, $G\subseteq\undphi(GY)$.
}%end of proof of 447B

\leader{447C}{Vitali's theorem} Let $X$ be a topological group with a
left Haar measure $\mu$, and $\sequencen{V_n}$ a $B$-sequence in
$X$\cmmnt{ (definition:  446L)}.
If $A\subseteq X$ is any set and $K_x$ is an infinite subset of $\Bbb N$
for every $x\in A$, then there is a disjoint family $\Cal V$ of sets
such that $A\setminus\bigcup\Cal V$ is negligible and every member of
$\Cal V$ is of the form $xV_n$ for some $x\in A$ and $n\in K_x$.

\proof{{\bf (a)} There is surely some $r$ such that $V_r$ is totally
bounded for the bilateral uniformity on $X$ (443H);
replacing $V_i$ by $V_r$ for $i<r$ and $K_x$ by $K_x\setminus r$
for each $x$, we may suppose that $V_0$ is totally bounded.

\medskip

{\bf (b)} Choose $\sequencen{I_n}$ inductively, as follows.   Given
$I_j\subseteq X$ for $j<n$, choose a set $I_n\subseteq A$ which is
maximal subject to the conditions

\Centerline{$n\in K_x$ for every $x\in I_n$,}

\Centerline{$xV_n\cap yV_j=\emptyset$ whenever $x\in I_n$, $j<n$ and
$y\in I_j$,}

\Centerline{$xV_n\cap yV_n=\emptyset$ whenever $x$, $y\in I_n$ are
distinct.}

\noindent On completing the induction, set
$\Cal V=\{xV_n:n\in\Bbb N,\,x\in I_n\}$;  this is a disjoint family.

\medskip

{\bf (c)} \Quer\ Suppose, if possible, that $A\setminus\bigcup\Cal V$ is
not negligible.   By 415B, the subspace measure on
$A\setminus\bigcup\Cal V$ is $\tau$-additive and has a non-empty support.   Take any $a$ belonging to this support
and set $G=\interior(aV_0)$;  then $G$ is totally bounded and
$\delta=\mu^*(G\cap A\setminus\bigcup\Cal V)$ is non-zero.   Let $M$ be
such that every $V_nV_n^{-1}$ can be covered by $M$ left translates of
$V_n$, so that $\mu(V_nV_n^{-1})\le M\mu V_n$ for every $n$.   Set

\Centerline{$J_n=I_n\cap GV_0^{-1}$,
\quad$E_n=J_nV_n$,
\quad$\tilde E_n=E_nV_n^{-1}$}

\noindent for each $n$.

If $n\in\Bbb N$ and $x\in J_n$, then $xV_n\subseteq GV_0^{-1}V_0$.
Accordingly (because $\family{x}{J_n}{xV_n}$ is disjoint)

\Centerline{$\#(J_n)\mu V_n=\sum_{x\in J_n}\mu(xV_n)
\le\mu(GV_0^{-1}V_0)<\infty$}

\noindent because $GV_0^{-1}V_0$ is totally bounded (4A5Ob).   So $J_n$ is
finite and $E_n$ is closed.
Note that if $x\in I_n$ and $G\cap xV_n\ne\emptyset$, then
$x\in GV_n^{-1}\subseteq GV_0^{-1}$ and $x\in J_n$;  so $G\cap\bigcup\Cal V=G\cap\bigcup_{n\in\Bbb N}E_n$.
Also, $\sequencen{E_n}$ is a disjoint sequence of subsets of
$GV_0^{-1}V_0$;  accordingly $\sum_{n=0}^{\infty}\mu E_n$ is finite,
and there is an $m\in\Bbb N$ such that
$M\sum_{n=m}^{\infty}\mu E_n<\delta$.

Observe next that, for any $x\in X$ and $n\in\Bbb N$,

\Centerline{$\mu(xV_nV_n^{-1})=\mu(V_nV_n^{-1})\le M\mu
V_n=M\mu(xV_n)$.}

\noindent So

\Centerline{$\mu\tilde E_n\le\sum_{x\in J_n}\mu(xV_nV_n^{-1})
\le M\sum_{x\in J_n}\mu(xV_n)=M\mu E_n$}

\noindent for each $n$, and $\mu(\bigcup_{n\ge m})\tilde E_n<\delta$.

This means that $\bigcup_{n\ge m}\tilde E_n$ cannot include $A\cap
G\setminus\bigcup\Cal V$, and there is a $z$ belonging to

$$\eqalign{A\cap G\setminus(\bigcup\Cal V\cup\bigcup_{n\ge m}\tilde E_n)
&=A\cap G\setminus(\bigcup_{n\in\Bbb N}E_n\cup\bigcup_{n\ge m}\tilde
E_n)\cr
&=A\cap G\setminus(\bigcup_{n<m}E_n\cup\bigcup_{n\ge m}\tilde
E_n).\cr}$$

\noindent Now there must be a first $k\ge m$ such that $k\in K_z$ and
$zV_k\subseteq G\setminus\bigcup_{n<m}E_n$.   (This is where we use the
hypothesis that $\{V_n:n\in\Bbb N\}$ is a base of neighbourhoods of the
identity.)   Since $z\in G\setminus J_k$, $z\notin I_k$, and there are
$j\le k$, $x\in I_j$ such that $zV_k\cap xV_j\ne\emptyset$.   In this
case, $x\in GV_0^{-1}$, so $x\in J_j$.   Accordingly

\Centerline{$z\in xV_jV_k^{-1}\subseteq xV_jV_j^{-1}
\subseteq\tilde E_j$}

\noindent and $j<m$;  but this means that
$zV_k\cap xV_j\subseteq zV_k\cap E_j$ must be empty, which is
impossible.\ \Bang

\medskip

{\bf (d)} Thus $\mu(A\setminus\bigcup\Cal V)=0$, and $\Cal V$ is an
appropriate family.
}%end of proof of 447C

\leader{447D}{Theorem} Let $X$ be a topological group with a left Haar
measure $\mu$, and $\sequencen{V_n}$ a $B$-sequence in $X$.
Then for any Haar measurable set $E\subseteq X$,

$$\lim_{n\to\infty}\bover{\mu(E\cap xV_n)}{\mu V_n}=\chi E(x)$$

\noindent for almost every $x\in X$.

\proof{{\bf (a)} Let $\alpha<1$, and set

$$A=\{x:x\in E,\,\liminf_{n\to\infty}\bover{\mu(E\cap xV_n)}{\mu
V_n}<\alpha\}.$$

\noindent\Quer\ Suppose, if possible, that $A$ is not negligible.   Then
there is an open set $G$ of finite measure such that $\mu^*(G\cap
A)=\gamma>0$.   Let $\delta>0$ be such that
$\gamma>\alpha(\gamma+\delta)+\delta$.
Take a Borel set $F$ which is a measurable envelope of $G\cap A$ and a
closed set $F_1\subseteq G\setminus F$ such that $\mu
F_1\ge\mu(G\setminus F)-\delta$.
Writing $H=G\setminus F_1$, we see that $H\cap A=G\cap A$ and

\Centerline{$\mu H\le\mu^*(H\cap A)+\delta=\gamma+\delta$.}

For each $x\in H\cap A$, set

\Centerline{$K_x=\{n:xV_n\subseteq H,\,\mu(E\cap xV_n)\le\alpha\mu
V_n\}$.}

\noindent Then $K_x$ is infinite.   By Vitali's theorem in the form
447C, there is a disjoint family $\Cal V\subseteq\{xV_n:x\in H\cap
A,\,n\in K_x\}$ such that $(H\cap A)\setminus\bigcup\Cal V$ is
negligible.   Since every member of $\Cal V$ has non-zero measure, while
$\mu H$ is finite, $\Cal V$ is countable.   Now $\mu(\bigcup\Cal
V)\ge\mu^*(H\cap A)$, so $\mu(H\setminus\bigcup\Cal V)\le\delta$;  also,
because $\mu(E\cap V)\le\alpha\mu V$ for every $V\in\Cal V$, and $\Cal
V$ is disjoint,

\Centerline{$\mu(E\cap\bigcup\Cal V)
\le\alpha\mu(\bigcup\Cal V)\le\alpha\mu H$}

\noindent and

\Centerline{$\gamma=\mu^*(A\cap H)\le\mu(E\cap H)
\le\mu(E\cap\bigcup\Cal V)+\delta\le\alpha\mu H+\delta
\le\alpha(\gamma+\delta)+\delta$,}

\noindent which is impossible, by the choice of $\delta$.\ \Bang

\medskip

{\bf (b)} As $\alpha$ is arbitrary,

$$\liminf_{n\to\infty}\bover{\mu(E\cap xV_n)}{\mu V_n}=1$$

\noindent for almost every $x\in E$.   Similarly,

$$\liminf_{n\to\infty}\bover{\mu(xV_n\setminus E)}{\mu V_n}=1,
\quad\limsup_{n\to\infty}\bover{\mu(E\cap xV_n)}{\mu V_n}=0
$$

\noindent for almost every $x\in X\setminus E$, so

$$\lim_{n\to\infty}\bover{\mu(E\cap xV_n)}{\mu V_n}=\chi E(x)$$

\noindent for almost every $x\in X$.
}%end of proof of 447D

\leader{447E}{}\cmmnt{ We need to recall some results from
443P-443R.}   %443P 443Q 443R
Let $X$ be a locally compact Hausdorff topological group,
and $Y$ a closed subgroup of $X$ such that the modular function of $Y$
is the restriction to $Y$ of the modular function of $X$.   Let $\mu$ be
a left Haar measure on $X$ and $\mu_Y$ a left Haar measure on $Y$.

\spheader 447Ea Writing $C_k(X)$ for the space of continuous real-valued
functions on $X$ with compact support, and $X/Y$ for the set of left
cosets of $Y$ in $X$ with the quotient topology, we have a linear
operator $T:C_k(X)\to C_k(X/Y)$ defined by writing
$(Tf)(x^{\ssbullet})=\int_Yf(xy)\mu_Y(dy)$ whenever $x\in X$ and
$f\in C_k(X)$\cmmnt{ (443P)};  moreover,
$T[C_k(X)^+]=C_k(X/Y)^+$\cmmnt{ (443Pa)}, and we have an invariant
Radon measure
$\lambda$ on $X/Y$ such that $\int Tfd\lambda=\int fd\mu$ for every
$f\in C_k(X)$\cmmnt{ (see part (b) of the proof of 443R)}.
\cmmnt{Turning this
structure round, we see from 443Qb that} $\mu$,
$\mu_Y$ and $\lambda$ here are related
in exactly the same way as $\mu$, $\nu$ and $\lambda$ in 443Q.
If $Y$ is a normal subgroup of $X$, so that $X/Y$ is the quotient group,
$\lambda$ is a left Haar measure.   If $Y$ is compact and $\mu_Y$ is the
Haar probability measure on $Y$, then $\lambda$ is the image measure
$\mu\pi^{-1}$, where $\pi(x)=x^{\ssbullet}=xY$ for every
$x\in X$\cmmnt{ (443Qd)}.

\spheader 447Eb If $E\subseteq X$ and $EY=Y$, then $E$ is Haar
measurable iff $\tilde E=\{x^{\ssbullet}:x\in E\}$ belongs to the domain
of $\lambda$, and $E$ is Haar negligible iff $\tilde E$ is
$\lambda$-negligible\cmmnt{ (443Qc)}.
%If $E\subseteq X$ is Haar measurable, and $\{y:xy\in E\}$ is
%$\mu_Y$-negligible for $\mu$-almost every $x\in X$, then $E$ is
%$\mu$-negligible (??).

\spheader 447Ec Now suppose that $X$ is $\sigma$-compact.   Then for any
Haar measurable $E\subseteq X$, $\mu E=\int g\,d\lambda$ in
$[0,\infty]$, where $g(x^{\ssbullet})=\mu_Y(Y\cap x^{-1}E)$ is defined for
almost every $x\in X$\cmmnt{ (443Qe)}.   \cmmnt{In particular, }$E$
is Haar negligible
iff $\mu_Y(Y\cap x^{-1}E)=0$ for almost every $x\in X$.

\spheader 447Ed Again suppose that $X$ is $\sigma$-compact.   Then we
can extend the operator $T$ of part (a) to an operator from
$\eusm L^1(\mu)$ to $\eusm L^1(\lambda)$ by writing
$(Tf)(x^{\ssbullet})=\int f(xy)\mu_Y(dy)$ whenever $f\in\eusm L^1(\mu)$,
$x\in X$ and the integral is defined, and $\int Tfd\lambda=\int fd\mu$ for every
$f\in\eusm L^1(\mu)$\cmmnt{ (443Qe)}.   If $f\in\eusm L^1(\mu)$, and
we set $f_x(y)=f(xy)$
for all those $x\in X$, $y\in Y$ for which $xy\in\dom f$, then
$Q=\{x:f_x\in\eusm L^1(\mu_Y)\}$ is $\mu$-conegligible, and
$x\mapsto f_x^{\ssbullet}:Q\to L^1(\mu_Y)$ is almost
continuous\cmmnt{ (443Qf)}.

\spheader 447Ee If $X$ is $\sigma$-compact, $Y$ is compact and $\mu_Y$
is the Haar probability measure on $Y$\cmmnt{, so that $\lambda$ is
the image
measure $\mu\pi^{-1}$}, then\cmmnt{ we can apply 235G to
the formula in (d) to see that}

\Centerline{$\iint f(xy)\mu_Y(dy)\mu(dx)=\int(Tf)(x^{\ssbullet})\mu(dx)
=\int Tf\,d\lambda=\int fd\mu$}

\noindent for every\cmmnt{ $\mu$-integrable function $f$, and
therefore (because $\mu$ is $\sigma$-finite) for every} function $f$
such that
$\int fd\mu$ is defined in $[-\infty,\infty]$.   In particular,
$\mu E=\int\nu(Y\cap x^{-1}E)\mu(dx)$ for every Haar measurable set $E\subseteq X$.

\leader{447F}{Lemma} Let $X$ be a $\sigma$-compact locally compact
Hausdorff topological group and $Y$ a closed subgroup of $X$ such that
the modular function of $Y$ is the restriction to $Y$ of the modular
function of $X$.   Let $Z$ be a compact normal subgroup of $Y$ such that
the quotient group $Y/Z$ has a $B$-sequence.   Let $\Sigma_Y$ be the
$\sigma$-algebra of those Haar measurable subsets $E$ of $X$ such that
$EY=E$, and $\Sigma_Z$ the algebra of Haar measurable sets $E\subseteq X$
such that $EZ=E$.   Let $\undphi:\Sigma_Y\to\Sigma_Y$ be a \lti\
partial lower density.   Then there is a \lti\ partial lower density
$\undpsi:\Sigma_Z\to\Sigma_Z$ extending $\undphi$.

\proof{{\bf (a)} Let $\mu$ be a left Haar measure on $X$, $\mu_Y$ a left
Haar measure on $Y$ and $\mu_Z$ the Haar probability measure on $Z$;  then there
is a left Haar measure $\nu$ on $Y/Z$ such that
$\int g(y)\mu_Y(dy)=\int (Tg)(u)\nu(du)$ for every $g\in C_k(Y)$, where
$(Tg)(y^{\ssbullet})=\int g(yz)\mu_Z(dz)$ for every $y\in Y$ (447Ea).
We are supposing that
$Y/Z$ has a $B$-sequence $\sequencen{V_n}$.   It follows that there is
a sequence $\sequencen{h_n}$ in $C_k(Y)^+$ such that
(i) $\int h_n(y)\mu_Y(dy)=1$ for every $n$
(ii) whenever $F\subseteq Y$ is Haar
measurable (regarded as a subset of $Y$, that is), and $FZ=Z$, then

\Centerline{$\lim_{n\to\infty}\int\chi F(by)h_n(y)\mu_Y(dy)
=\chi F(b)$}

\noindent for $\mu_Y$-almost every $b\in Y$.

\Prf\ Since any subsequence of $\sequencen{V_n}$ is a $B$-sequence, and $Y/Z$ is locally compact, we
may suppose that every $V_n$ is compact.   For each
$n\in\Bbb N$, choose a non-negative $h'_n\in C_k(Y/Z)$ such that

\Centerline{$\int h'_nd\nu=1$,
\quad$\int|h'_n-\Bover1{\nu V_n}\chi V_n|d\nu\le 2^{-n}$.}

\noindent (This is possible by 416I, or otherwise.)   Let
$h_n\in C_k(Y)^+$ be such that $Th_n=h'_n$ (447Ea again);  then
$\int h_nd\mu_Y=\int h'_nd\nu=1$.   Now
if $F\subseteq Y$ is Haar measurable and $FZ=Z$, there is a Haar
measurable $\tilde F\subseteq Y/Z$ such that
$F=\{y:y^{\ssbullet}\in\tilde F\}$ (447Eb).   Take $b\in Y$ and
$n\in\Bbb N$.   Because

\Centerline{$\int\chi F(byz)h_n(yz)\mu_Z(dz)
=\int\chi\tilde F(b^{\ssbullet}y^{\ssbullet})h_n(yz)\mu_Z(dz)
=\chi\tilde F(b^{\ssbullet}y^{\ssbullet})h'_n(y^{\ssbullet})$}

\noindent for every $y\in Y$,

$$\eqalignno{\int\chi F(by)h_n(y)\mu_Y(dy)
&=\iint\chi F(byz)h_n(yz)\mu_Z(dz)\mu_Y(dy)\cr
\displaycause{447Ee}
&=\int\chi\tilde F(b^{\ssbullet}y^{\ssbullet})h'_n(y^{\ssbullet})\mu_Y(dy)
=\int\chi\tilde F(b^{\ssbullet}u)h'_n(u)\nu(du)\cr}$$

\noindent because $y\mapsto y^{\ssbullet}$ is \imp\ for $\mu_Y$ and $\nu$ (447Ee).   So

$$\eqalign{|\chi F(b)-\int\chi F(by)h_n(y)\mu_Y(dy)|
&=|\chi\tilde F(b^{\ssbullet})-\int\chi\tilde
F(b^{\ssbullet}u)h'_n(u)\nu(du)|\cr
&\le|\chi\tilde F(b^{\ssbullet})-\Bover1{\nu V_n}\int\chi\tilde
F(b^{\ssbullet}u)\chi V_n(u)\nu(du)|\cr
&\qquad\qquad\qquad\qquad
  +\int|h'_n(u)-\Bover1{\nu V_n}\chi V_n(u)|\nu(du)\cr
&\le|\chi\tilde F(b^{\ssbullet})
  -\Bover{\nu(\tilde F\cap b^{\ssbullet}V_n)}{\nu V_n}|
  +2^{-n}.\cr}$$

\noindent Since

\Centerline{$\{v:v\in Y/Z,\,\lim_{n\to\infty}\Bover{\nu(\tilde F\cap vV_n)}{\nu V_n}\ne\chi\tilde F(v)\}$}

\noindent is $\nu$-negligible (447D), its inverse image in $Y$ is
$\mu_Y$-negligible, so
$\chi F(b)=
\discrversionA{\break}{}
\lim_{n\to\infty}\int\chi F(by)h_n(y)\mu_Y(dy)$
for almost every $b$, as claimed.\ \Qed

\medskip

{\bf (b)} We find now that if $E\in\Sigma_Z$, then
$\lim_{n\to\infty}\int\chi E(xy)h_n(y)\mu_Y(dy)=\chi E(x)$ for
$\mu$-almost every $x\in X$.   \Prf\ Set $E_x=Y\cap x^{-1}E$ for
$x\in X$.   Because $X$ is $\sigma$-compact, we can express $E$ as the union of
a non-decreasing sequence $\sequence{k}{F^{(k)}}$ where each $F^{(k)}$
is Haar measurable and relatively compact;  set
$F_x^{(k)}=Y\cap x^{-1}F^{(k)}$ for each $x$.   In this case, for any $k\in\Bbb N$,
$Q_k=\{x:F^{(k)}_x\in\dom(\mu_Y)\}$ is conegligible, and
$x\mapsto(\chi F^{(k)}_x)^{\ssbullet}:Q_k\to L^1(\mu_Y)$ is almost
continuous (447Ed), so that
$x\mapsto\int\chi F^{(k)}_x(y)h_n(y)\mu_Y(dy):Q_k\to[0,1]$ is
measurable, for each
$n$.   But this means that, setting $Q=\bigcap_{k\in\Bbb N}Q_k$,

\Centerline{$x\mapsto\int\chi E_x(y)h_n(y)\mu_Y(dy)
=\lim_{k\to\infty}\int\chi F^{(k)}_x(y)h_n(y)\mu_Y(dy):Q\mapsto[0,1]$}

\noindent is measurable, for every $n\in\Bbb N$.   Note that if
$x\in X$ and $y\in Y$
then $F^{(k)}_{xy}=y^{-1}F^{(k)}_x$, so that $Q_kY=Q_k$ for every
$k\in\Bbb N$, and $QY=Q$.

Now consider $\tilde Q=\{x:x\in Q,\,
\lim_{n\to\infty}\int\chi E(xy)h_n(y)\mu_Y(dy)=\chi E(x)\}$.   This is a Haar measurable subset of $X$.   If $a\in Q$, then

\Centerline{$Y\cap a^{-1}\tilde Q
=\{y:y\in Y$, $\lim_{n\to\infty}\chi E(ays)h_n(s)\mu_Y(ds)=\chi E(ay)\}$}

\noindent is $\mu_Y$-conegligible, by the choice of the $h_n$ in (a)
above.   Because $Q$ is $\mu$-conegligible, $Q\setminus\tilde Q$ is $\mu$-negligible
(447Ec) and $\tilde Q$ is conegligible, as required.\ \Qed

\medskip

{\bf (c)} We are now ready for the formulae at the centre of this proof.
For any Haar measurable set $E\subseteq X$, $n\in\Bbb N$ and $\gamma<1$,
set

$$\eqalign{\psi_{n\gamma}(E)
&=\bigcup\{G\cap\undphi F:G\subseteq X\text{ is open, }F\in\Sigma_Y,\cr
&\qquad\qquad\int\chi E(xy)h_n(y)\mu_Y(dy)
  \text{ is defined and at least }\gamma
  \text{ for every }x\in G\cap F\},\cr}$$

\Centerline{$\undpsi E=\bigcap_{\gamma<1}\bigcup_{n\in\Bbb N}
  \bigcap_{m\ge n}\psi_{m\gamma}(E)$.}

\noindent The rest of the proof is devoted to checking that
$\undpsi\restr\Sigma_Z$ is a \lti\ partial lower density extending
$\undphi$.

\medskip

{\bf (d)} I had better make one remark straight away.   If
$G\subseteq X$ is open, then $G\subseteq\undphi(GY)$ (447B).
It follows that if $G\subseteq X$ is open, $F\in\Sigma_Y$ and
$G\cap\undphi F\ne\emptyset$, then $G\cap F\ne\emptyset$.   \Prf\ If
$a\in G\cap\undphi F$, then $a\in\undphi(GY)\cap\undphi F=\undphi(GY\cap
F)$, so $GY\cap F\ne\emptyset$, that is, $G\cap F=G\cap FY^{-1}$ is
non-empty.\ \QeD\  I mention this now because we need to know that the
condition

\Centerline{$\int\chi E(xy)h_n(y)\mu_Y(dy)$ is defined and at least $\gamma$ for every $x\in G\cap F$}

\noindent in the definition of $\psi_{n\gamma}(E)$ is never vacuously
satisfied if $G\cap\undphi F\ne\emptyset$.   In particular,
$\psi_{n\gamma}(\emptyset)=\emptyset$ whenever $n\in\Bbb N$ and
$0<\gamma<1$, so that $\undpsi\emptyset=\emptyset$.

\medskip

{\bf (e)} If $E\subseteq X$ is Haar measurable, $n\in\Bbb N$ and
$\epsilon>0$, then for almost every $a\in X$ there are an open set
$G\subseteq X$ and an $F\in\Sigma_Y$ such that $a\in G\cap\undphi F$ and
$\int|\chi E(xy)-\chi E(ay)|h_n(y)\mu_Y(dy)\le\epsilon$ whenever
$x\in G\cap F$.   \Prf\ Let
$\tilde\lambda$ be the invariant Radon measure on $X/Y$ derived from $\mu$
and $\mu_Y$ as in 447Ea.
Take $\delta>0$ such that
$\delta(1+2\|h_n\|_{\infty})\le\epsilon$.   Because $X$ is
$\sigma$-compact, locally compact and Hausdorff, therefore Lindel\"of
and regular (4A2Hd, 3A3Bb), there is a sequence
$\sequence{r}{f_r}$ of continuous functions from $X$ to $[0,1]$ such
that $\int|\chi E(x)-f_r(x)|\mu(dx)\le 2^{-r}$ for every $r$
(415Pb).   Set $g_r=|\chi E-f_r|$ for each $r$.
For $f\in\eusm L^1(\mu)$, define $\tilde Tf\in\eusm L^1(\tilde\lambda)$ by writing $(\tilde Tf)(x^{\ssbullet})=\int f(xy)\mu_Y(dy)$ whenever this is defined (447Ed);
we have $\int\tilde Tfd\tilde\lambda=\int fd\mu$.

Set $Q=\{x:Y\cap x^{-1}E\in\dom\mu_Y\}$, so that
$Q\in\Sigma_Y$ is conegligible (447Ec).   For each $r$, set
$\tilde F_r=\{u:u\in X/Y$, $\tilde Tg_r(u)$ is defined and at least
$\delta\}$;   then

\Centerline{$\tilde\lambda\tilde F_r
\le\Bover1{\delta}\int Tg_rd\tilde\lambda
=\Bover1{\delta}\int g_rd\mu\le 2^{-r}/\delta$.}

\noindent So $\bigcap_{r\in\Bbb N}\tilde F_r$ is $\tilde\lambda$-negligible.
Set

$$\eqalign{F_r
&=\{x:x\in X,\,x^{\ssbullet}\in\tilde F_r\}
=\{x:\tilde Tg_r(x^{\ssbullet})\ge\delta\}\cr
&=\{x:\int g_r(xy)\mu_Y(dy)\ge\delta\}
=\{x:\int|\chi E(xy)-f_r(xy)|\mu_Y(dy)\ge\delta\}\cr}$$

\noindent for each $r$;  then $F_rY=F_r$, $F_r$ is Haar measurable (447Eb) and
$\bigcap_{r\in\Bbb N}F_r$ is $\mu$-negligible (also by 447Eb).
Since $(X\setminus F_r)\symmdiff\undphi(X\setminus F_r)$ is negligible for each $r$,
$Q_1=Q\cap\bigcup_{r\in\Bbb N}\undphi(X\setminus F_r)\setminus F_r$ is
conegligible.   Note that $Q_1Y=Q_1$.

Suppose that $a\in Q_1$.   Then there is an $r\in\Bbb N$ such that

\Centerline{$a\in Q_1\cap\undphi(X\setminus F_r)\setminus F_r
=(Q_1\setminus F_r)\cap\undphi(Q_1\setminus F_r)$.}

\noindent Set $F=Q_1\setminus F_r\in\Sigma_Y$.    Consider the function
$x\mapsto\int f_r(xy)h_n(y)\mu_Y(dy)$.   We chose $h_n$ with compact
support $L\subseteq Y$ say.  If $V$ is a compact neighbourhood of $a$ in
$X$, then $f_r$ is uniformly continuous on $VL$ for the right
uniformity on
$X$ (4A5Ha, 4A2Jf).   There is therefore an open neighbourhood $U$ of
the identity of
$X$ such that $|f_r(x')-f_r(x)|\le\delta$ whenever $x$, $x'\in VL$ and
$x'x^{-1}\in U$;  of course we may suppose that $G=Ua$ is a subset of
$V$.

Take any $x\in G\cap F$.   Then if $y\in L$, we have $ay$, $xy$ both in
$VL$, while $xy(ay)^{-1}\in U$, so that $|f_r(xy)-f_r(ay)|\le\delta$.
Accordingly $|f_r(xy)-f_r(ay)|h_n(y)\le\delta h_n(y)$ for every $y\in
Y$, and

\Centerline{$\int|f_r(ay)-f_r(xy)|h_n(y)\mu_Y(dy)
\le\delta$.}

\noindent At the same time, because both $x$ and $a$ belong to
$F=Q_1\setminus F_r$,

\Centerline{$\int|\chi E(ay)-f_r(ay)|h_n(y)\mu_Y(dy)
\le\delta\|h_n\|_{\infty}$,}

\Centerline{$\int|\chi E(xy)-f_r(xy)|h_n(y)\mu_Y(dy)
\le\delta\|h_n\|_{\infty}$.}

\noindent Putting these together,

\Centerline{$\int|\chi E(ay)-\chi E(xy)|h_n(y)\mu_Y(dy)
\le\delta(1+2\|h_n\|_{\infty})\le\epsilon$.}

\noindent Thus $G$ and $F$ witness that $a$ has the property required;
as $a$ is any member of the conegligible set $Q_1$, we have the
result.\ \Qed

\medskip

{\bf (f)} If $E\in\Sigma_Z$ then
$E\symmdiff\undpsi E$ is negligible.   \Prf\ By (e), applied in turn to
every $n$ and every $\epsilon$ of the form $2^{-i}$, there is a
conegligible set $Q_1\subseteq X$ such that whenever $a\in Q_1$,
$n\in\Bbb N$ and $\epsilon>0$ there are an open set $G$ containing $a$
and an $F\in\Sigma_Y$ such that $a\in\undphi F$ and
$\int|\chi E(xy)-\chi E(ay)|h_n(y)\mu_Y(dy)\le\epsilon$ for every
$x\in G\cap F$.   By (b), there is a conegligible set $Q_2\subseteq X$
such that $\lim_{n\to\infty}\int\chi E(ay)h_n(y)\mu_Y(dy)=\chi E(a)$ for
every $a\in Q_2$.

Suppose that $a\in Q_1\cap Q_2\cap E$.   Let $\gamma<1$;  set
$\epsilon=\bover12(1-\gamma)$.   Because $a\in Q_2$, there is an
$n\in\Bbb N$ such that
$\int\chi E(ay)h_m(y)\mu_Y(dy)\ge 1-\epsilon$ for every $m\ge n$.   Take
any $m\ge n$.   Because $a\in Q_1$, there are an open set $G$ and an
$F\in\Sigma_Y$ such that $a\in G\cap\undphi F$ and
$\int|\chi E(ay)-\chi E(xy)|h_m(y)\mu_Y(dy)\le\epsilon$ whenever
$x\in G\cap F$.   But now

\Centerline{$\int\chi E(xy)h_m(y)\mu_Y(dy)\ge 1-2\epsilon=\gamma$}

\noindent for every $x\in G\cap F$, so $a\in\psi_{m\gamma}(E)$.   This
is true for every $m\ge n$;  as $\gamma$ is arbitrary, $a\in\undpsi E$.
As $a$ is arbitrary, $Q_1\cap Q_2\cap E\subseteq\undpsi E$.

Now suppose that $a\in Q_1\cap Q_2\cap\undpsi E$.   Then there is an
$n\in\Bbb N$ such that $a\in\psi_{m,3/4}(E)$ for every $m\ge n$.   There
is an $m\ge n$ such that $|\int\chi E(ay)h_m(y)\mu_Y(dy)-\chi
E(a)|\le\bover14$.   There are an open set $G_1$ and an $F_1\in\Sigma_Y$
such that $a\in G_1\cap\undphi F_1$ and
$\int\chi E(xy)h_m(y)\mu_Y(dy)\ge\bover34$ for every $x\in G_1\cap F_1$.   There are also an open set $G_2$ and an $F_2\in\Sigma_Y$ such that
$a\in G_2\cap\undphi F_2$ and
$\int|\chi E(ay)-\chi E(xy)|h_m(y)\mu_Y(dy)\le\bover14$ for every
$x\in G_2\cap F_2$.   Set
$G=G_1\cap G_2$, $F=F_1\cap F_2$;  then $a\in G\cap\undphi F$, so $G\cap
F$ is not empty ((d) above).   Take $x\in G\cap F$.   Then

\Centerline{$\int\chi E(xy)h_m(y)\mu_Y(dy)\ge\Bover34$,}

\Centerline{$\int|\chi E(ay)-\chi E(xy)|h_m(y)\mu_Y(dy)
\le\Bover14$,}

\Centerline{$|\int\chi E(ay)h_m(y)\mu_Y(dy)-\chi E(a)|
\le\Bover14$,}

\noindent so $\chi E(a)\ge\bover14$ and $a\in E$.   This shows that
$Q_1\cap Q_2\cap\undpsi E\subseteq E$.

Accordingly $E\symmdiff\undpsi E\subseteq X\setminus(Q_1\cap Q_2)$ is
negligible, as required.\ \QeD\

In particular, $\undpsi E$ is Haar measurable for every $E\in\Sigma_Z$.

\medskip

{\bf (g)} If $E\in\Sigma_Z$ then $\undpsi E\in\Sigma_Z$.   \Prf\ We have just seen that $\undpsi E$ is Haar measurable.   Take
$z\in Z$, $n\in\Bbb N$, $\gamma<1$ and $a\in\psi_{n\gamma}(E)$.   Then
there are an open $G\subseteq X$ and an $F\in\Sigma_Y$ such that $a\in
G\cap\undphi F$ and $\int\chi E(xy)h_n(y)\mu_Y(dy)\ge\gamma$ for
every $x\in G\cap F$.   Because $F$ and $\undphi F$ belong to
$\Sigma_Y$, $az\in\undphi F$.   Of course $Gz$ is an open set containing
$az$.   If $x\in Gz\cap F$, then $xz^{-1}\in G\cap F$ and
$\int\chi E(xz^{-1}y)h_n(y)\mu_Y(dy)\ge\gamma$.   But

\Centerline{$\chi E(xz^{-1}y)=\chi E(xy\cdot y^{-1}z^{-1}y)=\chi E(xy)$}

\noindent for every $y\in Y$, because $Z\normalsubgroup Y$ (so
$z'=y^{-1}z^{-1}y\in Z$) and we are supposing that $E\in\Sigma_Z$ (so
$xyz'\in E$ iff $xy\in E$).   So

\Centerline{$\int\chi E(xy)h_n(y)\mu_Y(dy)
=\int\chi E(xz^{-1}y)h_n(y)\mu_Y(dy)\ge\gamma$.}

\noindent As $x$ is arbitrary, $Gz$ and $F$ witness that
$az\in\psi_{n\gamma}E$.

This shows that, for any $n$ and $\gamma$, $az\in\psi_{n\gamma}(E)$
whenever $a\in\psi_{n\gamma}(E)$ and $z\in Z$.   It follows at once that
$az\in\undpsi E$ whenever $a\in\undpsi E$ and $z\in Z$, as claimed.\
\Qed

\medskip

{\bf (h)} For any Haar measurable $E\subseteq X$ and $c\in X$,
$\undpsi(cE)=c\undpsi E$.   \Prf\ Suppose that $n\in\Bbb N$, $\gamma<1$
and $a\in\psi_{n\gamma}(E)$.   Then there are an open set $G\subseteq X$
and an $F\in\Sigma_Y$ such that $a\in G\cap\undphi F$ and
$\int\chi E(xy)h_n(y)\mu_Y(dy)\ge\gamma$ for every $x\in G\cap F$.   Now
$cG$ is an open set containing $ca$, $cF\in\Sigma_Y$ and
$\undphi(cF)=c\undphi F$ contains $ca$, and if $x\in cG\cap cF$ we have

\Centerline{$\int\chi(cE)(xy)h_n(xy)\mu_Y(dy)
=\int\chi E(c^{-1}xy)h_n(xy)\mu_Y(dy)\ge\gamma$}

\noindent because $c^{-1}x\in G\cap F$.   But this means that $cG$, $cF$
witness that $ca\in\psi_{n\gamma}(cE)$.   Since $a$ is arbitrary,
$c\psi_{n\gamma}(E)\subseteq\psi_{n\gamma}(cE)$;  as $n$ and $\gamma$
are arbitrary, $c\undpsi E\subseteq\undpsi(cE)$.   Similarly, of course,
$c^{-1}\undpsi(cE)\subseteq\undpsi E$, so in fact
$\undpsi(cE)=c\undpsi E$, as claimed.\ \Qed

\medskip

{\bf (i)} If $E_1$, $E_2\subseteq X$ are Haar measurable and
$E_1\setminus E_2$ is $\mu$-negligible,
$\undpsi E_1\subseteq\undpsi E_2$.   \Prf\ Take $n\in\Bbb N$, $\gamma<1$
and $a\in\psi_{n\gamma}(E_1)$.   Then there are an open set
$G\subseteq X$
and an $F\in\Sigma_Y$ such that $a\in G\cap\undphi F$ and
$\int\chi E_1(xy)h_n(y)\mu_Y(dy)\ge\gamma$ for every $x\in G\cap F$.
Let $Q$ be the set of those $x\in X$ such that
$\mu_Y$ measures $Y\cap x^{-1}(E_1\setminus E_2)$ and
$\int\chi(E_1\setminus E_2)(xy)\mu_Y(dy)=0$;  then $QY=Q$ is
conegligible (447Ec).   Now $Q\cap F\in\Sigma_Y$, and
$\undphi(Q\cap F)=\undphi F$ contains $a$.   But if $x\in G\cap Q\cap F$, $\chi E_1(xy)\le\chi E_2(xy)$ for $\mu_Y$-almost every $y$.   So

\Centerline{$\int\chi E_2(xy)h_n(y)\mu_Y(dy)
\ge\int\chi E_1(xy)h_n(y)\mu_Y(dy)\ge\gamma$.}

\noindent Thus $G$ and $Q\cap F$ witness that $a\in\psi_{n\gamma}(E_2)$.
As $a$ is arbitrary, $\psi_{n\gamma}(E_1)\subseteq\psi_{n\gamma}(E_2)$;
as $n$ and $\gamma$ are arbitrary, $\undpsi E_1\subseteq\undpsi E_2$.\
\Qed

In particular, (i) $\undpsi E_1=\undpsi E_2$ whenever $E_1\symmdiff E_2$
is negligible (ii) $\undpsi E_1\subseteq\undpsi E_2$ whenever
$E_1\subseteq E_2$.

\medskip

{\bf (j)} If $E_1$, $E_2\subseteq X$ are Haar measurable,
$\undpsi(E_1\cap E_2)=\undpsi E_1\cap\undpsi E_2$.   \Prf\ By (i),
$\undpsi(E_1\cap E_2)\subseteq\undpsi E_1\cap\undpsi E_2$.   So take
$a\in\undpsi E_1\cap\undpsi E_2$.   Let $\gamma<1$.   Set
$\delta=\bover12(1+\gamma)<1$.   Then there are $n_1$, $n_2\in\Bbb N$
such that $a\in\psi_{m\delta}(E_1)$ for every $m\ge n_1$ and
$a\in\psi_{m\delta}(E_2)$ for every $m\ge n_2$.   Set $n=\max(n_1,n_2)$
and take any $m\ge n$.   Then there are open sets $G_1$,
$G_2\subseteq X$ and $F_1$, $F_2\in\Sigma_Y$ such that
$a\in G_1\cap G_2\cap\undphi F_1\cap\undphi F_2$,
$\int\chi E_1(xy)h_m(y)\mu_Y(dy)\ge\delta$ for
every $x\in G_1\cap F_1$ and $\int\chi E_2(xy)h_m(y)\mu_Y(dy)\ge\delta$ for
every $x\in G_2\cap F_2$.   Let $Q\in\Sigma_Y$ be the conegligible set
of those $x\in X$ such that $\int\chi(E_1\cap E_2)(xy)\mu_Y(dy)$
is defined.   Set $G=G_1\cap G_2$, $F=F_1\cap F_2\cap Q$;  then $G$ is
open, $F\in\Sigma_Y$ and

\Centerline{$\undphi F=\undphi(F_1\cap F_2)=\undphi F_1\cap\undphi
F_2$,}

\noindent so that $a\in G\cap\undphi F$.   Now take any $x\in G\cap F$.
We have

$$\eqalign{\int(1-\chi&(E_1\cap E_2))(xy)h_m(y)\mu_Y(dy)\cr
&\le \int(1-\chi E_1(xy))h_m(y)\mu(dy)
  +\int(1-\chi E_2(xy))h_m(y)\mu(dy)\cr
&\le 2(1-\delta)=1-\gamma,\cr}$$

\noindent and $\int\chi(E_1\cap E_2)(xy)h_m(y)\mu_Y(dy)\ge\gamma$,
because $\int h_m(y)\mu_Y(dy)=1$.

As $x$ is arbitrary, $G$ and $F$ witness that
$a\in\psi_{m\gamma}(E_1\cap E_2)$.   And this is true for every
$m\ge n$.   As $\gamma$ is arbitrary, $a\in\undpsi(E_1\cap E_2)$.   As
$a$ is arbitrary,
$\undpsi E_1\cap\undpsi E_2\subseteq\undpsi(E_1\cap E_2)$ and
the two are equal.\ \Qed

\medskip

{\bf (k)} If $E\in\Sigma_Y$, $\undpsi E=\undphi E$.   \Prf\ (i) Suppose
$a\in\undphi E$, $n\in\Bbb N$ and $\gamma<1$.   Set $F=E\cap\undphi E\in\Sigma_Y$, $G=X$.   Then $G\cap\undphi F=\undphi(E\cap\undphi E)=\undphi E$ contains $a$.
Take any $x\in F$.   Then

\Centerline{$\int\chi E(xy)h_n(y)\mu_Y(dy)
=\int h_n(y)\mu_Y(dy)=1$;}

\noindent as $x$ is arbitrary, $a\in\psi_{n\gamma}(E)$.   As $n$ and
$\gamma$ are arbitrary, $a\in\undpsi E$;  as $a$ is arbitrary,
$\undphi E\subseteq\undpsi E$.   (ii) Suppose $a\in\undpsi E$.   Then
there must
be some open $G\subseteq X$ and $F\in\Sigma_Y$ and $n\in\Bbb N$ such
that $a\in G\cap\undphi F$ and $\int\chi E(xy)h_n(y)\mu_Y(dy)>0$ for
every $x\in G\cap F$.   This surely implies that
$G\cap F\subseteq EY=E$, so that $GY\cap F\subseteq E$.   But
$a\in G\subseteq\undphi(GY)$, by 447B, so

\Centerline{$a\in\undphi(GY)\cap\undphi F=\undphi(GY\cap
F)\subseteq\undphi E$.}

\noindent This shows that $\undpsi E\subseteq\undphi E$.\ \Qed

\medskip

{\bf (l)} Thus we have assembled all the facts required to establish
that $\undpsi\restr\Sigma_Z$ is a
\lti\  partial lower density extending $\undphi$.
}%end of proof of 447F

\leader{447G}{Lemma} Let $X$ be a $\sigma$-compact locally compact
Hausdorff topological group, and $\sequencen{Y_n}$ a non-increasing
sequence of compact subgroups of $X$ with intersection $Y$.   Let
$\Sigma$ be the algebra of Haar measurable subsets of $X$;  set
$\Sigma_{Y_n}=\{E:E\in\Sigma,\,EY_n=E\}$ for each $n$, and
$\Sigma_Y=\{E:E\in\Sigma,\,EY=E\}$.   Suppose that for each $n\in\Bbb N$
we are given a \lti\ partial lower density
$\undphi_n:\Sigma_{Y_n}\to\Sigma_{Y_n}$, and that $\undphi_{n+1}$
extends $\undphi_n$ for every $n$.   Then there is a \lti\ partial lower
density $\undphi:\Sigma_Y\to\Sigma_Y$ extending every $\undphi_n$.

\proof{{\bf (a)} Fix a left Haar measure $\mu$ on $X$, and for each
$n\in\Bbb N$ let $\nu_n$ be the Haar probability measure on $Y_n$
(442Ie).   As noted in 443Sb, the modular function of $X$ must be equal
to $1$, and equal to the modular function of $Y_n$, everywhere in every
$Y_n$.

\medskip

{\bf (b)} We need to know that for any $E\in\Sigma_Y$ there is an $F$ in
the $\sigma$-algebra $\Lambda$ generated by
$\bigcup_{n\in\Bbb N}\Sigma_{Y_n}$ such that $E\symmdiff F$ is
negligible.   \Prf\ Because
$X$ is $\sigma$-compact and $\mu$ is a Radon measure (442Ac), there is a
sequence $\sequence{i}{K_i}$ of compact sets such
that $K_i\subseteq E$ for every $i$ and
$E\setminus\bigcup_{i\in\Bbb N}K_i$ is negligible.   For each
$i\in\Bbb N$, $\bigcap_{n\in\Bbb N}K_iY_n=K_iY$ (4A5Eh), so is included
in $E$.   Set $F=\bigcup_{i\in\Bbb N}\bigcap_{n\in\Bbb N}K_iY_n$;  then
$F$ belongs to the $\sigma$-algebra generated by $\bigcup_{n\in\Bbb N}\Sigma_{Y_n}$, and $K_i\subseteq F\subseteq E$ for every $i$, so
$E\symmdiff F$ is negligible.\ \Qed

\medskip

{\bf (c)} For each $E\in\Sigma_Y$, $n\in\Bbb N$ set

\Centerline{$g_{En}(x)=\nu_n(Y_n\cap x^{-1}E)$ whenever this is
defined.}

\noindent By 447Ee, $g_{En}$ is defined $\mu$-almost everywhere and is
$\Sigma$-measurable.   In fact $g_{En}$ is $\Sigma_{Y_n}$-measurable,
because
$g_{En}(xy)=g_{En}(x)$ whenever $x\in X$, $y\in Y_n$ and either is defined.
If $F\in\Sigma_{Y_n}$ then
$g_{En}(x)\chi F(x)=\nu_n(Y_n\cap x^{-1}(E\cap F))$ whenever this is
defined, which is almost everywhere;  so
$\int_Fg_{En}d\mu=\mu(E\cap F)$, by 447Ee.   If $E$,
$E'\in\Sigma_Y$ and $E\symmdiff E'$ is negligible, then
$g_{En}\eae g_{E'n}$, because $g_{E\symmdiff E',n}=0$ a.e.

It follows that $\sequencen{g_{En}}\to\chi E\,\,\mu$-a.e.\ for every
$E\in\Sigma_Y$.   \Prf\ Let $G\subseteq X$ be any non-empty relatively
compact open set, and set $U=GY_0$, so that $U$ also is a non-empty
relatively compact open set, $UY_n=U$ for every $n$ and $UY=U$.   Set
$\mu_U(F)=\Bover{\mu F}{\mu U}$ whenever $F\in\Sigma$ and $F\subseteq U$,
so that $\mu_U$ is a probability measure on $U$.

Writing $\Sigma_{Y_n}^{(U)}$, $\Sigma_Y^{(U)}$ and $\Lambda^{(U)}$ for
the subspace $\sigma$-algebras on $U$ generated by $\Sigma_{Y_n}$,
$\Sigma_Y$ and $\Lambda$, we see that if $F\in\Sigma_{Y_n}^{(U)}$ then

\Centerline{$\int_Fg_{En}d\mu_U
=\bover1{\mu U}\int_Fg_{En}d\mu
=\mu_U(E\cap F)$.}

\noindent So $g_{En}\restr U$ is a conditional expectation of
$\chi(E\cap U)$ on $\Sigma^{(U)}_{Y_n}$.   By L\'evy's martingale
theorem (275I), $\sequencen{g_{En}}$ converges almost everywhere in $U$
to a conditional expectation $g$ of $\chi(E\cap U)$ on $\Lambda^{(U)}$,
because of course $\Lambda^{(U)}$ is the $\sigma$-algebra of subsets of
$U$ generated by $\bigcup_{n\in\Bbb N}\Sigma^{(U)}_{Y_n}$.   But as
there is an $F\in\Lambda$ such that $E\symmdiff F$ is negligible, by (b)
above, $g$ must be equal to $\chi E$ almost everywhere in $U$.

Thus $g_{En}\to\chi E$ almost everywhere in $U$ and therefore almost
everywhere in $G$.   As $G$ is arbitrary, $g_{En}\to\chi E$ almost
everywhere in $X$, by 412Jb (applied to the family $\Cal K$ of subsets
of relatively compact open sets).\ \Qed

\medskip

{\bf (d)} Now we can use the method of 341G, as follows.   For
$E\in\Sigma_Y$, $k\ge 1$ and $n\in\Bbb N$ set

\Centerline{$H_{kn}(E)=\{x:x\in\dom(g_{En}),\,g_{En}(x)\ge 1-2^{-k}\}
\in\Sigma_{Y_n}$,}

\Centerline{$\tilde H_{kn}(E)=\undphi_n(H_{kn}(E))$,
\quad$\undphi E
=\bigcap_{k\ge 1}\bigcup_{n\in\Bbb N}\bigcap_{m\ge n}\tilde H_{km}(E)$.}

\noindent By the arguments of parts (e)-(i) of the proof of 341G,
$\undphi$ is a lower density on $\Sigma_Y$ extending every $\undphi_n$.

\medskip

{\bf (e)} To see that $\undphi$ is \lti, we may argue as follows.   Let
$E\in\Sigma_Y$ and $a\in X$.   Then, for any $n$,

\Centerline{$g_{aE,n}(x)=\nu_n(Y_n\cap x^{-1}aE)=g_{En}(a^{-1}x)$}

\noindent for almost every $x$, so
$aH_{kn}(E)\symmdiff H_{kn}(aE)$ is negligible, and

\Centerline{$\tilde H_{kn}(aE)
=\undphi_n(H_{kn}(aE))=\undphi_n(aH_{kn}(E))=a\undphi_n(H_{kn}(E))
=a\tilde H_{kn}(E)$,}

\noindent for every $k$.   Accordingly

%$$\eqalign{\undphi(aE)
%&=\bigcap_{k\ge 1}\bigcup_{n\in\Bbb N}\bigcap_{m\ge n}
%  \tilde H_{km}(aE)\cr
%&=a(\bigcap_{k\ge 1}\bigcup_{n\in\Bbb N}\bigcap_{m\ge n}
%  \tilde H_{km}(E))
%=a\undphi(E),\cr}$$

\Centerline{$\undphi(aE)
=\bigcap_{k\ge 1}\bigcup_{n\in\Bbb N}\bigcap_{m\ge n}
  \tilde H_{km}(aE)
=a(\bigcap_{k\ge 1}\bigcup_{n\in\Bbb N}\bigcap_{m\ge n}
  \tilde H_{km}(E))
=a\undphi(E)$,}

\noindent as required.
}%end of proof of 447G

\leader{447H}{Lemma} Let $X$ be a locally compact Hausdorff topological
group, and $\Sigma$ the algebra of Haar measurable sets in $X$.   Then
there is a \lti\ lower density $\undphi:\Sigma\to\Sigma$.

\proof{{\bf (a)} To begin with (down to the end of (c) below) let us
suppose that $X$ is $\sigma$-compact.   By 446P, there is a family
$\langle X_{\xi}\rangle_{\xi\le\kappa}$ of closed subgroups of $X$,
where $\kappa$ is an infinite cardinal, such that

\inset{$X_0$ is an open subgroup of $X$,

for every $\xi<\kappa$, $X_{\xi+1}$ is a normal subgroup of
$X_{\xi}$ and $X_{\xi}/X_{\xi+1}$ has a $B$-sequence,

for every non-zero limit ordinal $\xi\le\kappa$,
$X_{\xi}=\bigcap_{\eta<\xi}X_{\eta}$,

$X_1$ is compact,

$X_{\kappa}=\{e\}$, where $e$ is the identity of $X$.}

\noindent Note that for every $\xi\le\kappa$, the modular function
$\Delta_{\xi}$ of $X_{\xi}$ is just the restriction to $X_{\xi}$ of the
modular function $\Delta$ of $X$.   \Prf\ For $\xi=0$ this is because
$X_0$ is an open subgroup of $X$ (443Sd).   For $\xi\ge 1$,
$X_{\xi}$ is compact, so $\Delta_{\xi}$ and $\Delta\restr X_{\xi}$
are both constant with value $1$, as noted in 443Sb.\ \Qed

\medskip

{\bf (b)} For each $\xi\le\kappa$, write $\Sigma_{\xi}$ for the
$\sigma$-algebra $\{E:E\in\Sigma$, $EX_{\xi}=E\}$.    I seek to choose
inductively a family $\langle\undphi_{\xi}\rangle_{\xi\le\kappa}$ such
that each $\undphi_{\xi}:\Sigma_{\xi}\to\Sigma_{\xi}$ is a \lti\ partial
lower density, and $\undphi_{\xi}$ extends $\undphi_{\eta}$ whenever
$\eta<\xi$.

\medskip

\quad{\bf (i)} {\it Start\/} Since $X_0$ is an open subgroup of $X$,
every
member of $\Sigma_0$ is open, and we can start the induction by setting
$\undphi_0E=E$ for every $E\in\Sigma_0$.

\medskip

\quad{\bf (ii)} {\it Inductive step to a successor ordinal\/} If we have
defined $\langle\undphi_{\eta}\rangle_{\eta\le\xi}$, where $\xi<\kappa$,
then $X_{\xi+1}\normalsubgroup X_{\xi}$ is compact and $X_{\xi}/X_{\xi+1}$ has a
$B$-sequence.   So the conditions of 447F are satisfied and
$\undphi_{\xi}$ has an extension to a \lti\ partial lower density
$\undphi_{\xi+1}:\Sigma_{\xi+1}\to\Sigma_{\xi+1}$.   Of course
$\undphi_{\xi+1}$ extends $\undphi_{\eta}$ for every $\eta\le\xi$
because $\undphi_{\xi}$ does.

\medskip

\quad{\bf (iii)} {\it Inductive step to a limit ordinal of countable
cofinality\/} If we have defined
$\langle\undphi_{\eta}\rangle_{\eta<\xi}$, where $\xi\le\kappa$ is a
non-zero limit ordinal of countable cofinality, let $\sequencen{\xi_n}$
be a strictly increasing sequence with supremum $\xi$;  we may suppose
that $\xi_0\ge 1$, so that every $X_{\xi_n}$ is compact.   Then
$X_{\xi}=\bigcap_{n\in\Bbb N}X_{\xi_n}$, so by 447G there is a \lti\
partial lower density $\undphi:\Sigma_{\xi}\to\Sigma_{\xi}$ extending
every $\undphi_{\xi_n}$, and therefore extending $\undphi_{\eta}$
whenever $\eta<\xi$.

\medskip

\quad{\bf (iv)} {\it Inductive step to a limit ordinal of uncountable
cofinality\/} Suppose we have defined
$\langle\undphi_{\eta}\rangle_{\eta<\xi}$, where $\xi\le\kappa$ is a
limit ordinal of uncountable cofinality.   Then for every
$E\in\Sigma_{\xi}$ there are an $\eta<\xi$ and an $F\in\Sigma_{\eta}$
such that $E\symmdiff F$ is negligible.   \Prf\ (Cf.\ part (b) of the
proof of 447G.)   Because $X$ is $\sigma$-compact, there are
non-decreasing sequences $\sequence{i}{K_i}$, $\sequence{i}{L_i}$ of
compact subsets of $E$, $X\setminus E$ respectively such that
$E\setminus\bigcup_{i\in I}K_i$ and
$(X\setminus E)\setminus\bigcup_{i\in\Bbb N}L_i$ are negligible.   For
each $i\in\Bbb N$, $K_iX_{\xi}\cap L_i\subseteq EX_{\xi}\setminus E$ is
empty;   by 4A5Eh again, there is an $\eta_i<\xi$ such that
$K_iX_{\eta_i}\cap L_i$ is empty.
Set $\eta=\sup_{i\in\Bbb N}\eta_i$, $F=\bigcup_{i\in\Bbb N}K_iX_{\eta}$;
this works.\ \Qed

Accordingly we have a function
$\undphi_{\xi}:\Sigma_{\xi}\to\Sigma_{\xi}$ defined by writing
$\undphi_{\xi}(E)=\undphi_{\eta}(F)$ whenever $E\in\Sigma_{\xi}$,
$\eta<\xi$, $F\in\Sigma_{\eta}$ and $E\symmdiff F$ is negligible.
\Prf\ If $\eta\le\eta'<\xi$ and $F\in\Sigma_{\eta}$,
$F'\in\Sigma_{\eta'}$ are such that $E\symmdiff F$ and $E\symmdiff F'$
are both negligible, then $F\symmdiff F'$ is negligible so
$\undphi_{\eta}(F)=\undphi_{\eta'}(F)=\undphi_{\eta'}(F')$.\ \QeD\   It
is easy to check that $\undphi_{\xi}$ is a \lti\ partial lower density
(cf.\ part (A-d) of the proof of 341H), and of course it extends
$\undphi_{\eta}$ for every $\eta<\xi$.

\medskip

{\bf (c)} On completing the induction, we see that
$\Sigma_{\kappa}=\Sigma$, so that $\undphi_{\kappa}:\Sigma\to\Sigma$ is
a \lti\ lower density.

\medskip

{\bf (d)} For the general case, recall that $X$ certainly has an open
$\sigma$-compact subgroup $Y$ say (4A5El).   If $\Sigma$ is the algebra
of Haar measurable subsets of $X$, and $\Tau$ is the algebra of Haar
measurable subsets of $Y$, then $\Tau$ is just
$\Sigma\cap\Cal PY=\{E\cap Y:E\in\Sigma\}$, and the Haar negligible
subsets of $Y$ are just sets of the form $E\cap Y$ where $E$ is a Haar
negligible subset of $X$ (443F).

Let $\undpsi:\Tau\to\Tau$ be a \lti\ lower density.   For
$E\in\Sigma$ set

\Centerline{$\undphi E=\{x:x\in X,\,e\in\undpsi(Y\cap x^{-1}E)\}$,}

\noindent where $e$ is the identity of $X$.   It is easy to check that

\Centerline{$\undphi\emptyset=\emptyset$,}

\Centerline{$\undphi E=\undphi F$ if $E\symmdiff F$ is negligible,}

\Centerline{$\undphi(E\cap F)=\undphi E\cap\undphi F$ for all $E$,
$F\in\Sigma$,}

\noindent directly from the corresponding properties of $\undpsi$.   If
$E\in\Sigma$ and $a\in X$, then

\Centerline{$x\in\undphi E
\iff e\in\undpsi(Y\cap x^{-1}E)
\iff e\in\undpsi(Y\cap(ax)^{-1}aE)
\iff ax\in\undphi(aE)$,}

\noindent so $\undphi(aE)=a\undphi E$.

I have not yet checked that $E\symmdiff\undphi E$ is always negligible.
But if $E\in\Sigma$, then

\Centerline{$E\symmdiff\undphi E
=\{x:e\in\undpsi(Y\cap x^{-1}E)\symmdiff(Y\cap x^{-1}E)\}$,}

\noindent so

$$\eqalign{(E\symmdiff\undphi E)\cap Y
&=\{x:x\in Y,\,e\in\undpsi(x^{-1}(E\cap Y))\}
  \symmdiff(E\cap Y)\cr
&=\{x:x\in Y,\,e\in x^{-1}\undpsi(E\cap Y)\}\symmdiff(E\cap Y)
=\undpsi(E\cap Y)\symmdiff(E\cap Y)\cr}$$

\noindent is negligible.   Moreover, for any $a\in X$,

\Centerline{$(E\symmdiff\undphi E)\cap aY
=a((a^{-1}E\symmdiff\undphi(a^{-1}E))\cap Y)$}

\noindent because $\undphi$ is translation-invariant, so
$(E\symmdiff\undphi E)\cap aY$ is negligible.   Since $\{aY:a\in X\}$ is
an open cover of $X$, $E\symmdiff\undphi E$ is negligible (412Jb again).
In particular, $\undphi E\in\Sigma$.   So $\undphi:\Sigma\to\Sigma$ is a
\lti\ lower density, as required.
}%end of proof of 447H

\leader{447I}{Theorem}\cmmnt{ ({\smc Ionescu Tulcea \&
Ionescu Tulcea 67})}
Let $X$ be a locally compact Hausdorff topological group.   Then
it has a \lti\ lifting for its Haar measures.

\proof{ (Cf. 345B-345C.)   Write $\Sigma$ for the algebra of Haar
measurable subsets of $X$, and let $\undphi:\Sigma\to\Sigma$ be a \lti\
lower density (447H).   Let $\phi_0:\Sigma\to\Sigma$ be any lifting such
that $\phi_0E\supseteq\undphi E$ for every $E\in\Sigma$ (341Jb).   For
$E\in\Sigma$, set

\Centerline{$\phi E=\{x:e\in\phi_0(x^{-1}E)\}$,}

\noindent where $e$ is the identity of $X$.   It is easy to check that
$\phi:\Sigma\to\Cal PX$ is a Boolean homomorphism.   Also

\Centerline{$x\in\undphi E
\Longrightarrow e\in\undphi(x^{-1}E)
\Longrightarrow e\in\phi_0(x^{-1}E)
\Longrightarrow x\in\phi E$.}

\noindent So $\phi$ is a lifting (341Ib).   Finally, $\phi$ is \lti\ by
the argument used in (d) of the proof of 447H (and also in (e) of the
proof of 345B).
}%end of proof of 447I

\leader{447J}{Corollary} Let $X$ be any topological group carrying Haar
measures.   Then it has a \lti\ lifting for its left Haar measures.

\proof{ Let $\mu$ be a left Haar measure on $X$.   By 443L, we have a
locally compact Hausdorff topological group $Z$ and a continuous
homomorphism $f:X\to Z$, \imp\ for $\mu$ and an appropriate left Haar
measure $\nu$ on $Z$, such that for every $E$ in the domain $\Sigma$ of
$\mu$ there is an $F$ in the domain $\Tau$ of $\nu$ such that
$f^{-1}[F]\subseteq E$ and $E\setminus f^{-1}[F]$ is negligible.   Let
$\psi$ be a \lti\ lifting for $\nu$.
Since $F^{\ssbullet}\mapsto f^{-1}[F]^{\ssbullet}$ is an isomorphism
between the measure algebras of $\mu$ and $\nu$, we have a lifting
$\phi:\Sigma\to\Sigma$ given by saying that
$\phi E=f^{-1}[\psi F]$ whenever
$F\in\Tau$ and $E\symmdiff f^{-1}[F]$ is negligible (346D).
Now $\phi$ is \lti\ because $f$ is a group homomorphism and $\psi$ is \lti.
}%end of proof of 447J

\exercises{\leader{447X}{Basic exercises $\pmb{>}$(a)}
%\spheader 447Xa
Let $X=\Bbb R\times\{-1,1\}$, with its usual topology,
and define a multiplication on $X$ by setting
$(x,\delta)(y,\epsilon)=(x+\delta y,\delta\epsilon)$.   Show that $X$ is
a locally compact topological group.   Show that there is no lifting for
the Haar measure algebra of $X$ which is both left- and
right-translation-invariant.   \Hint{345Xc.}
%447A  %mt44bits

\spheader 447Xb Let $X$ be a topological group carrying Haar measures
which has a $B$-sequence.   Show that it has a $B$-sequence
$\sequencen{V_n}$ such that
$\sup_{n\in\Bbb N}\mu(V_nV_n^{-1})/\mu V_n$ is finite for any Haar
measure $\mu$ on $X$, whether left or right.
%447C

\spheader 447Xc Let $X$ be a topological group with a left Haar measure
$\mu$, and $\sequencen{V_n}$ a $B$-sequence for $X$.   Show that if
$f\in\eusm L^0(\mu)$ is locally integrable, then
$f(x)=\lim_{n\to\infty}\Bover1{\mu V_n}\biggerint_{xV_n}fd\mu$ for almost
every $x$.
%447C 261C

\leader{447Y}{Further exercises (a)}
%\spheader 447Ya
Describe a compact Hausdorff topological group such that
its Haar measure has no lifting which is both left- and
right-translation-invariant.
%447Xa, 447A   S^1\times\{-1,1\}

\spheader 447Yb Let $(X,\Sigma,\mu)$ be a measure space, with measure
algebra $\frak A$, and $\undtheta:\frak A\to\Sigma$ a lower density.
Show that we have a function
$q:L^{\infty}(\frak A)^+\to L^{\infty}(\Sigma)^+$ such that
$\{x:q(u)(x)>\alpha\}=\bigcup_{\beta>\alpha}\undtheta\Bvalue{u>\beta}$
for every $\alpha\ge 0$ and $u\in L^{\infty}(\frak A)^+$.   Show that
$q(u)^{\ssbullet}=u$, $q(\alpha u)=\alpha q(u)$, $q(u\wedge v)=q(u)\wedge q(v)$ and $q(\chi a)=\chi(\undtheta a)$ for every $u$,
$v\in L^{\infty}(\frak A)^+$, $\alpha\ge 0$ and $a\in\frak A$.
%447I

}%end of exercises

\endnotes{
\Notesheader{447} The structure of the proof of 447I is exactly that of
the proof of the ordinary Lifting Theorem in \S341;  the lifting is
built from a lower density which is constructed inductively on a family
of sub-$\sigma$-algebras.   To get a translation-invariant lifting it is
natural to look for a translation-invariant lower density, and a simple
trick (already used in \S345) ensures that this is indeed enough.   The
refinements we need here are dramatic but natural.   To make the
final lower density $\undphi$ (in 447H) translation-invariant, it is
clearly sensible (if we can do it) to keep all the partial lower
densities $\undphi_{\xi}$ translation-invariant.   This means that their
domains $\Sigma_{\xi}$ should be translation-invariant.   It does not
quite follow that they have to be of the form
$\Sigma_{\xi}=\{E:EX_{\xi}=X_{\xi}\}$ for closed subgroups $X_{\xi}$,
but if we look at the leading example of $\{0,1\}^I$ (345C) this also is
a reasonable thing to try first.   So now we have to consider what extra
hypotheses will be needed to make the induction work.   The inductive
step to limit ordinals of uncountable cardinality remains elementary, at
least if the $X_{\xi}$ are compact (part (b-iv) of the proof of 447H).
The inductive step to limit ordinals of countable cofinality (447G) is
harder, but can be managed with ideas already presented.   Indeed,
compared with the version in 341G, we have the advantage of a formula
for the auxiliary functions $g_{En}$, which is very helpful when we come
to translation-invariance.   We have to do something
about the fact that we are no longer working with a probability
space -- that is the point of the $\mu_U$ in part (c)
of the proof of 447G.
(Another expression of the manoeuvre here is in 369Xq.)

Where we do need a new idea is in the inductive step to a successor
ordinal.   If $\Sigma_{\xi+1}$ is to be translation-invariant, it must
be much bigger than the $\sigma$-algebra generated by
$\Sigma_{\xi}\cup\{E\}$, as discussed in 341F.   To make the step a
small one (and therefore presumably easier), we want $X_{\xi+1}$ to be a
large subgroup of $X_{\xi}$ in some sense;  as it turns out, a helpful
approach is to ask for $X_{\xi+1}$ to be a normal subgroup of $X_{\xi}$
and for $X_{\xi}/X_{\xi+1}$ to be small.   At this point we have to know
something of the structure theory of locally compact topological groups.
The right place to start is surely the theory of compact
Hausdorff groups.   Such a group $X$ actually has a continuous
decreasing chain $\langle X_{\xi}\rangle_{\xi\le\kappa}$ of closed
normal subgroups, from $X_0=X$ to $X_{\kappa}=\{e\}$, such that all the
quotients $X_{\xi}/X_{\xi+1}$ are Lie groups.   I do not define `Lie
group' here, because for our purposes it is enough to know that the
quotients have faithful finite-dimensional representations, and
therefore have `$B$-sequences' in the sense of 446L.   Having
identified this as a relevant property, it is not hard to repeat
arguments from \S221 and \S261 to prove versions of Vitali's theorem
and Lebesgue's Density Theorem in such groups (447C-447D).   This will
evidently provide translation-invariant lower densities for groups of
this special type, just as Lebesgue lower density is a
translation-invariant lower density on $\BbbR^r$ (345B).

Of course we still have to find a way of combining this construction
with a translation-invariant lower density on $\Sigma_{\xi}$ to produce
a translation-invariant lower density on $\Sigma_{\xi+1}$, and this is
what I do in 447F.   The argument I offer is essentially that of
{\smc Ionescu Tulcea \& Ionescu Tulcea 67}, \S7, and is the deepest part of
this section.

For compact groups, these ideas are all we need, and indeed the step to
a limit ordinal of countable cofinality is a little easier, since we
have a Haar probability measure on the whole group.   The next step, to
general $\sigma$-compact locally compact groups, demands much deeper
ideas from the structure theory, but from the point of view of the
present section the modifications are minor.   The subgroups $X_{\xi}$
are now not always normal subgroups of $X$, which means that we have to
be more careful in the description of the quotient spaces $X/X_{\xi}$
(they must consist of {\it left} cosets), and we have to watch the
modular functions of the $X_{\xi}$ in order to be sure that there are
invariant measures on the quotients.   An extra obstacle at the
beginning is that we may have to start the chain with a proper subgroup
$X_0$ of $X$, but since $X_0$ can be taken to be open, it is pretty
clear that this will not be serious, and in fact it gives no trouble
(part (b-i) of the proof of 447H).   For $\xi\ge 1$, the $X_{\xi}$ are
compact, so the
inductive steps to limit ordinals are nearly the same.

The step to a general locally compact Hausdorff topological group (part
(d) of the proof of 447H) is essentially elementary.   And finally I
note that the whole thing applies to general topological groups with
Haar measures (447J), for the usual reasons.   There is an implicit
challenge here:  find expressions of the arguments used in this section
which will be valid in the more general context.   The measure-theoretic
part of such a programme might be achievable, but I do not see any hope
of a workable structure theory to match that of \S446 which does not use
443L or something like it.
}%end of notes

\discrpage



