\frfilename{mt231.tex} 
\versiondate{25.8.15} 
\copyrightdate{1994} 
      
\def\chaptername{The Radon-Nikod\'ym theorem} 
\def\sectionname{Countably additive functionals} 
      
\newsection{231} 
      
I begin with an abstract description of the objects which will, in 
appropriate circumstances, correspond to the indefinite integrals of 
general integrable functions.   In this section I give those parts of 
the theory which do not involve a measure, but only a set with a 
distinguished $\sigma$-algebra of subsets.   The basic concepts are 
those of `finitely additive' and `countably additive' functional, and 
there is one substantial theorem, the `Hahn decomposition' (231E). 
      
\leader{231A}{Definition} Let $X$ be a set and $\Sigma$ an algebra of 
subsets of $X$\cmmnt{ (136E)}.   A functional $\nu:\Sigma\to\Bbb R$ is 
{\bf finitely additive}, or just {\bf additive}, 
if $\nu(E\cup F)=\nu E+\nu F$ whenever $E$, 
$F\in\Sigma$ and $E\cap F=\emptyset$. 
      
\leader{231B}{Elementary facts} Let $X$ be a set, $\Sigma$ an algebra of 
subsets of $X$, and $\nu:\Sigma\to\Bbb R$ a finitely additive 
functional. 
      
\header{231Ba}{\bf (a)} $\nu\emptyset=0$.   \prooflet{(For 
$\nu\emptyset=\nu(\emptyset\cup\emptyset)=\nu\emptyset+\nu\emptyset$.) 
} 
      
\header{231Bb}{\bf (b)} If $E_0,\ldots,E_n$ are disjoint members of 
$\Sigma$ then $\nu(\bigcup_{i\le n}E_i)=\sum_{i=0}^n\nu E_i$. 
      
\header{231Bc}{\bf (c)} If $E$, $F\in\Sigma$ and $E\subseteq F$ then 
$\nu F=\nu E + \nu(F\setminus E)$.   More generally, for any $E$, 
$F\in\Sigma$, 
      
\Centerline{$\nu F=\nu(F\cap E)+\nu(F\setminus E)$,}

\Centerline{$\nu E+\nu F
\prooflet{\mskip5mu=\nu E+\nu(F\setminus E)+\nu(F\cap E)}
=\nu(E\cup F)+\nu(E\cap F)$,}

\Centerline{$\nu E-\nu F 
\prooflet{\mskip5mu = \nu(E\setminus F)+\nu(E\cap F) 
-\nu(E\cap F)-\nu(F\setminus E)} 
=\nu(E\setminus F)-\nu(F\setminus E)$.} 
      
\leader{231C}{Definition} Let $X$ be a set and $\Sigma$ an algebra of 
subsets of $X$.   A function 
$\nu:\Sigma\to\Bbb R$ is {\bf countably additive} or 
{\bf $\sigma$-additive} if $\sum_{n=0}^{\infty}\nu E_n$ exists in $\Bbb 
R$ and is equal to $\nu(\bigcup_{n\in\Bbb N}E_n)$ for every 
disjoint sequence $\sequencen{E_n}$ in $\Sigma$ such that 
$\bigcup_{n\in\Bbb N}E_n\in\Sigma$. 
      
\cmmnt{\medskip 
      
\noindent{\bf Remark} Note that when I use the phrase `countably 
additive functional' I 
mean to exclude the possibility of $\infty$ as a value of the 
functional.   Thus a measure is a countably additive functional iff it 
is totally finite (211C). 
      
You will sometimes see the phrase `{\bf signed measure}' used to mean 
what I call a countably additive functional. 
} 
      
\leader{231D}{Elementary facts} Let $X$ be a set, $\Sigma$ a 
$\sigma$-algebra of subsets of $X$ and $\nu:\Sigma\to\Bbb R$ a countably 
additive functional. 
      
\header{231Da}{\bf (a)} $\nu$ is finitely additive.   \prooflet{\Prf\ 
(i) Setting $E_n=\emptyset$ for every $n\in\Bbb N$, 
$\sum_{n=0}^{\infty}\nu\emptyset$ must be defined in $\Bbb R$ so 
$\nu\emptyset$ must be $0$.   (ii) Now if $E$, $F\in\Sigma$ 
and $E\cap F=\emptyset$ we can set $E_0=E$, $E_1=F$, $E_n=\emptyset$ for 
$n\ge 2$ and get 
      
\Centerline{$\nu(E\cup F)=\nu(\bigcup_{n\in\Bbb N}E_n) 
=\sum_{n=0}^{\infty}\nu E_n=\nu E+\nu F$.  \Qed} 
} 
      
\header{231Db}{\bf (b)} If $\sequencen{E_n}$ is a non-decreasing 
sequence in $\Sigma$, with union $E\in\Sigma$, then 
      
\Centerline{$\nu E 
=\nu E_0+\sum_{n=0}^{\infty}\nu(E_{n+1}\setminus E_n) 
=\lim_{n\to\infty}\nu E_n$.} 
      
\header{231Dc}{\bf (c)} If $\sequencen{E_n}$ is a non-increasing 
sequence in $\Sigma$ with intersection $E\in\Sigma$, then 
      
\Centerline{$\nu E 
\prooflet{\mskip5mu =\nu E_0-\lim_{n\to\infty}\nu(E_0\setminus E_n)} 
=\lim_{n\to\infty}\nu E_n$.} 
      
\header{231Dd}{\bf (d)} If $\nuprime:\Sigma\to\Bbb R$ is another countably 
additive functional, and $c\in\Bbb R$, then $\nu+\nuprime:\Sigma\to\Bbb R$ 
and $c\nu:\Sigma\to\Bbb R$ are countably additive. 
      
\header{231De}{\bf (e)} If $H\in\Sigma$, then 
$\nu_H:\Sigma\to\Bbb R$ is countably additive, where 
$\nu_HE=\nu(E\cap H)$ for every 
$E\in\Sigma$.   \prooflet{\Prf\ If $\sequencen{E_n}$ is a disjoint 
sequence in 
$\Sigma$ with union $E\in\Sigma$ then $\sequencen{E_n\cap H}$ is 
disjoint, with union $E\cap H$, so 
      
\Centerline{$\nu_HE 
=\nu(H\cap E) 
=\nu(\bigcup_{n\in\Bbb N}(H\cap E_n)) 
=\sum_{n=0}^{\infty}\nu(H\cap E_n) 
=\sum_{n=0}^{\infty}\nu_HE_n$.  \Qed}} 
      
\cmmnt{\medskip 
      
\noindent{\bf Remark} For the time being, we shall be using the notion 
of `countably additive functional' only on $\sigma$-algebras $\Sigma$, 
in which case we can take it for granted that the unions and 
intersections above belong to $\Sigma$. 
}%end of comment 
      
\leader{231E}{}\cmmnt{ All the ideas above amount to minor 
modifications of 
ideas already needed at the very beginning of the theory of measure 
spaces.   We come now to something more substantial. 
      
\medskip 
      
\noindent}{\bf Theorem} Let $X$ be a set, $\Sigma$ a $\sigma$-algebra of 
subsets of $X$, and $\nu:\Sigma\to\Bbb R$ a countably additive 
functional.   Then 
      
(a) $\nu$ is bounded; 
      
(b) there is a set $H\in\Sigma$ such that 
      
\Centerline{$\nu F\ge 0$ whenever 
$F\in\Sigma$ and $F\subseteq H$,} 
      
\Centerline{$\nu F\le 0$ whenever $F\in\Sigma$ and 
$F\cap H=\emptyset$.} 
      
      
\proof{{\bf (a)} \Quer\ Suppose, if possible, otherwise.    For 
$E\in\Sigma$, set $M(E)=\sup\{|\nu F|:F\in\Sigma,\,F\subseteq E\}$; 
then $M(X)=\infty$.   Moreover, whenever $E_1$, $E_2$, $F\in\Sigma$ and 
$F\subseteq E_1\cup E_2$, then 
      
\Centerline{$|\nu F|=|\nu(F\cap E_1)+\nu(F\setminus E_1)| 
\le|\nu(F\cap E_1)|+|\nu(F\setminus E_1)|\le M(E_1)+M(E_2)$,} 
      
\noindent so $M(E_1\cup E_2)\le 
M(E_1)+M(E_2)$.   Choose a sequence $\sequencen{E_n}$ in $\Sigma$ as 
follows.   $E_0=X$.   Given that $M(E_n)=\infty$, where $n\in\Bbb N$, 
then surely there is an $F_n\subseteq E_n$ such that 
$|\nu F_n|\ge 1+|\nu E_n|$, in which case 
$|\nu(E_n\setminus F_n)|\ge 1$.   Now 
at least one of $M(F_n)$, $M(E_n\setminus F_n)$ is infinite;  if 
$M(F_n)=\infty$, set $E_{n+1}=F_n$;  otherwise, set 
$E_{n+1}=E_n\setminus F_n$;  in either case, note that 
$|\nu(E_n\setminus E_{n+1})|\ge 1$ and $M(E_{n+1})=\infty$, so that the 
induction will continue. 
      
On completing this induction, set $G_n=E_n\setminus E_{n+1}$ for 
$n\in\Bbb N$.   Then $\sequencen{G_n}$ is a disjoint sequence in 
$\Sigma$, so $\sum_{n=0}^{\infty}\nu G_n$ is defined in $\Bbb R$ and 
$\lim_{n\to\infty}\nu G_n=0$;  but $|\nu G_n|\ge 1$ for every $n$. 
\Bang 
      
\medskip 
      
{\bf (b)(i)} By (a), $\gamma=\sup\{\nu E:E\in\Sigma\}<\infty$.   Choose 
a sequence $\sequencen{E_n}$ in $\Sigma$ such that 
$\nu E_n\ge\gamma-2^{-n}$ for every $n\in\Bbb N$.   For 
$m\le n\in\Bbb N$, set $F_{mn}=\bigcap_{m\le i\le n}E_i$.   Then 
$\nu F_{mn}\ge\gamma-2\cdot 2^{-m}+2^{-n}$ for every $n\ge m$.   \Prf\ 
Induce on $n$.   For $n=m$, this is due to the choice of $E_m=F_{mm}$. 
For the inductive step, we have $F_{m,n+1}=F_{mn}\cap E_{n+1}$, while 
surely $\gamma\ge\nu(E_{n+1}\cup F_{mn})$, so 
      
$$\eqalignno{\gamma+\nu F_{m,n+1} 
&\ge\nu(E_{n+1}\cup F_{mn})+\nu(E_{n+1}\cap F_{mn})\cr&
=\nu E_{n+1}+\nu F_{mn}
\Displaycause{231Bc}
\ge\gamma-2^{-n-1}+\gamma-2\cdot 2^{-m}+2^{-n}
\Displaycause{by the choice of $E_{n+1}$ and the inductive 
hypothesis} 
=2\gamma-2\cdot 2^{-m}+2^{-n-1}.
\cr}$$ 
      
\noindent Subtracting $\gamma$ from both sides, 
$\nu F_{m,n+1}\ge\gamma-2\cdot 2^{-m}+2^{-n-1}$ and the induction 
proceeds. \Qed 
      
\medskip 
      
\quad{\bf (ii)} For $m\in\Bbb N$, set 
      
\Centerline{$F_m=\bigcap_{n\ge m}F_{mn}=\bigcap_{n\ge m}E_n$.} 
      
\noindent Then 
      
\Centerline{$\nu F_m 
=\lim_{n\to\infty}\nu F_{mn}\ge\gamma-2\cdot 2^{-m}$,} 
      
\noindent by 231Dc. 
Next, $\sequence{m}{F_m}$ is non-decreasing, so setting 
$H=\bigcup_{m\in\Bbb N}F_m$ we have 
      
\Centerline{$\nu H=\lim_{m\to\infty}\nu F_m\ge\gamma$;} 
      
\noindent  since $\nu H$ 
is surely less than or equal to $\gamma$, $\nu H=\gamma$. 
      
\medskip

\quad{\bf (iii)} If $F\in\Sigma$ and $F\subseteq H$, then 
      
\Centerline{$\nu H-\nu F=\nu(H\setminus 
F)\le\gamma=\nu H$,} 
      
\noindent so $\nu F\ge 0$.   If $F\in\Sigma$ and $F\cap 
H=\emptyset$ then 
      
\Centerline{$\nu H+\nu F=\nu(H\cup F)\le\gamma=\nu H$} 
      
\noindent so $\nu F\le 
0$.   This completes the proof. 
}%end of proof of 231E 
      
      
\leader{231F}{Corollary} Let $X$ be a set, $\Sigma$ a $\sigma$-algebra 
of subsets of $X$, and $\nu:\Sigma\to\Bbb R$ a countably additive 
functional.   Then $\nu$ can be expressed as the difference of two 
totally finite measures with domain $\Sigma$. 
      
\proof{ Take $H\in\Sigma$ as described in 231Eb.   Set 
$\nu_1 E=\nu(E\cap H)$, $\nu_2 E=-\nu(E\setminus H)$ for $E\in\Sigma$. 
Then, as in 231Dd-e, both $\nu_1$ and $\nu_2$ are countably 
additive functionals on $\Sigma$, and of course $\nu=\nu_1-\nu_2$.   But 
also, by the choice of $H$, both $\nu_1$ and $\nu_2$ are non-negative, 
so are totally finite measures. 
}%end of proof of 231F 
      
\medskip 
      
\noindent{\bf Remark} This is called the `{\bf Jordan decomposition}' 
of 
$\nu$.   The expression of 231Eb is a `{\bf Hahn decomposition}'. 
      
      
      
\exercises{ 
\leader{231X}{Basic exercises (a)} 
%\spheader 231Xa 
Let $\Sigma$ be the family of subsets $A$ of 
$\Bbb N$ such that one of $A$, $\Bbb N\setminus A$ is finite.   Show 
that $\Sigma$ is an algebra of subsets of $\Bbb N$.   (This is the {\bf 
finite-cofinite algebra} of subsets of $\Bbb N$;  compare 211Ra.) 
%231A 
      
\spheader 231Xb Let $X$ be a set, $\Sigma$ an algebra of subsets 
of $X$ and $\nu:\Sigma\to\Bbb R$ a finitely additive functional.  Show 
that $\nu(E\cup F\cup G)+\nu(E\cap F)+\nu(E\cap G)+\nu(F\cap G) 
=\nu E+\nu F+\nu G+\nu(E\cap F\cap G)$ for all $E$, $F$, $G\in\Sigma$. 
Generalize this result to longer sequences of sets. 
%231A 
      
\sqheader 231Xc Let $\Sigma$ be the finite-cofinite algebra of 
subsets of $\Bbb N$, as in 231Xa.   Define $\nu:\Sigma\to\Bbb Z$ by 
setting 
      
\Centerline{$\nu E 
=\lim_{n\to\infty}\bigl(\#(\{i:i\le n,\,2i\in E\}) 
  -\#(\{i:i\le n,\,2i+1\in E\})\bigr)$} 
      
\noindent for every $E\in\Sigma$.   Show that $\nu$ is well-defined and 
finitely additive and unbounded. 
%231Xa, 231A 
      
\spheader 231Xd Let $X$ be a set and $\Sigma$ an algebra of 
subsets of $X$.   (i) Show that if $\nu:\Sigma\to\Bbb R$ and 
$\nuprime:\Sigma\to\Bbb R$ are finitely additive, so are $\nu+\nuprime$ and 
$c\nu$ for 
any $c\in\Bbb R$.   (ii) Show that if $\nu:\Sigma\to\Bbb R$ is finitely 
additive and $H\in\Sigma$, then $\nu_H$ is finitely additive, where 
$\nu_H(E)=\nu(H\cap E)$ for every $E\in\Sigma$. 
%231A 
      
\spheader 231Xe Let $X$ be a set, $\Sigma$ an algebra of subsets 
of $X$ and $\nu:\Sigma\to\Bbb R$ a finitely additive functional.   Let 
$S$ be the linear space of those real-valued functions on $X$ 
expressible in the form $\sum_{i=0}^na_i\chi E_i$ where $E_i\in\Sigma$ 
for each $i$.   (i) Show that we have a linear functional 
$\dashint:S\to\Bbb R$ given by writing 
      
\Centerline{$\dashint\sum_{i=0}^na_i\chi E_i 
=\sum_{i=0}^na_i\nu E_i$} 
      
\noindent whenever $a_0,\ldots,a_n\in\Bbb R$ and 
$E_0,\ldots,E_n\in\Sigma$.   (ii) Show that if $\nu E\ge 0$ for every 
$E\in\Sigma$ then $\dashint f\ge 0$ whenever $f\in S$ and $f(x)\ge 0$ 
for every $x\in X$.   (iii) Show that if $\nu$ is bounded and 
$X\ne\emptyset$ then 
      
\Centerline{$\sup\{|\dashint f|:f\in S,\,\|f\|_{\infty}\le 1\} 
=\sup_{E,F\in\Sigma}|\nu E-\nu F|$,} 
      
\noindent writing $\|f\|_{\infty}=\sup_{x\in X}|f(x)|$. 
%231B 
      
\sqheader 231Xf Let $X$ be a set, $\Sigma$ a $\sigma$-algebra of 
subsets of $X$ and $\nu:\Sigma\to \Bbb R$ a finitely additive 
functional.   Show that the following are equiveridical: 
      
\quad (i) $\nu$ is countably additive; 
      
\quad (ii) $\lim_{n\to\infty}\nu E_n=0$ whenever $\sequencen{E_n}$ is a 
non-increasing sequence in $\Sigma$ and 
$\bigcap_{n\in\Bbb N}E_n=\emptyset$; 
      
\quad (iii) $\lim_{n\to\infty}\nu E_n=0$ whenever $\sequencen{E_n}$ is a 
sequence in $\Sigma$ and 
$\bigcap_{n\in\Bbb N}\bigcup_{m\ge n}E_m=\emptyset$; 
      
\quad (iv) $\lim_{n\to\infty}\nu E_n=\nu E$ whenever $\sequencen{E_n}$ 
is a sequence in $\Sigma$ and 
      
\Centerline{$E=\bigcap_{n\in\Bbb N}\bigcup_{m\ge n}E_m 
=\bigcup_{n\in\Bbb N}\bigcap_{m\ge n}E_m$.} 
      
\noindent({\it Hint}:  for (i)$\Rightarrow$(iv), consider non-negative 
$\nu$ first.) 
%231D 
      
\spheader 231Xg Let $X$ be a set and $\Sigma$ a $\sigma$-algebra 
of subsets of 
$X$, and let $\nu:\Sigma\to\coint{-\infty,\infty}$ be a function which 
is countably additive in the sense that $\nu\emptyset=0$ and whenever 
$\sequencen{E_n}$ is a disjoint sequence in $\Sigma$, 
$\sum_{n=0}^{\infty}\nu E_n=\lim_{n\to\infty}\sum_{i=0}^n\nu E_i$ is 
defined in $\coint{-\infty,\infty}$ and is 
equal to $\nu(\bigcup_{n\in\Bbb N}E_n)$.   Show that $\nu$ is bounded 
above and attains its upper bound (that is, there is an $H\in\Sigma$ 
such that $\nu H=\sup_{F\in\Sigma}\nu F$).   Hence, or otherwise, show 
that $\nu$ is expressible as the difference of a totally finite measure 
and a measure, both with domain $\Sigma$. 
%231F 
      
\leader{231Y}{Further exercises (a)} Let $X$ be a set, $\Sigma$ an 
algebra of subsets of $X$, and $\nu:\Sigma\to\Bbb R$ a bounded finitely 
additive 
functional.   Set 
      
\Centerline{$\nu^+E=\sup\{\nu F:F\in\Sigma,\,F\subseteq E\}$,} 
      
\Centerline{$\nu^-E=-\inf\{\nu F:F\in\Sigma,\,F\subseteq E\}$,} 
      
\Centerline{$|\nu|E=\sup\{\nu F_1-\nu 
F_2:F_1,\,F_2\in\Sigma,\,F_1,\,F_2\subseteq E\}$.} 
      
\noindent Show that $\nu^+$, $\nu^-$ and $|\nu|$ are all bounded 
finitely additive functionals on $\Sigma$ and that $\nu=\nu^+-\nu^-$, 
$|\nu|=\nu^++\nu^-$.   Show that if $\nu$ is countably additive so are 
$\nu^+$, $\nu^-$ and $|\nu|$.   ($|\nu|$ is sometimes called the {\bf 
variation} of $\nu$.) 
      
\header{231Yb}{\bf (b)} Let $X$ be a set and $\Sigma$ an algebra of 
subsets of $X$.   Let $\nu_1$, $\nu_2$ be two bounded finitely additive 
functionals defined on $\Sigma$.   Set 
      
\Centerline{$(\nu_1\vee\nu_2)(E) 
=\sup\{\nu_1 F+\nu_2(E\setminus F):F\in\Sigma,\,F\subseteq E\}$,} 
      
\Centerline{$(\nu_1\wedge\nu_2)(E) 
=\inf\{\nu_1 F+\nu_2(E\setminus F):F\in\Sigma,\,F\subseteq E\}$.} 
      
\noindent Show that $\nu_1\vee\nu_2$ and $\nu_1\wedge\nu_2$ are finitely 
additive functionals, and that 
$\nu_1+\nu_2=\nu_1\vee\nu_2+\nu_1\wedge\nu_2$. 
Show that, in the language of 231Ya, 
      
\Centerline{$\nu^+=\nu\vee 0$, 
\quad$\nu^-=(-\nu)\vee 0=-(\nu\wedge 0)$, 
\quad$|\nu|=\nu\vee(-\nu)=\nu^+\vee\nu^-=\nu^++\nu^-$,} 
      
\Centerline{$\nu_1\vee\nu_2=\nu_1+(\nu_2-\nu_1)^+$, 
\quad$\nu_1\wedge\nu_2=\nu_1-(\nu_1-\nu_2)^+$,} 
      
\noindent so that $\nu_1\vee\nu_2$ and $\nu_1\wedge\nu_2$ are countably 
additive if $\nu_1$ and $\nu_2$ are. 
      
\header{231Yc}{\bf (c)} Let $X$ be a set and $\Sigma$ an algebra of 
subsets of $X$.   Let $M$ be the set of all bounded finitely additive 
functionals 
from $\Sigma$ to $\Bbb R$.   Show that $M$ is a linear space under the 
natural definitions of addition and scalar multiplication.   Show that 
$M$ has a partial order $\le$ defined by saying that 
      
\Centerline{$\nu\le\nuprime$ iff $\nu E\le\nuprime E$ for every $E\in\Sigma$,} 
      
\noindent and that for this partial order $\nu_1\vee\nu_2$, 
$\nu_1\wedge\nu_2$, as defined in 231Yb, are $\sup\{\nu_1,\nu_2\}$, 
$\inf\{\nu_1,\nu_2\}$. 
      
\header{231Yd}{\bf (d)} Let $X$ be a set and $\Sigma$ an algebra of 
subsets of $X$.   Let $\nu_0,\ldots,\nu_n$ be bounded finitely additive 
functionals on $\Sigma$ and set 
      
\Centerline{$\check\nu 
E=\sup\{\sum_{i=0}^n\nu_iF_i:F_0,\ldots,F_n\in\Sigma,\,\bigcup_{i\le 
n}F_i=E,\,F_i\cap F_j=\emptyset$ for $i\ne j\}$,} 
      
\Centerline{$\hat\nu 
E=\inf\{\sum_{i=0}^n\nu_iF_i:F_0,\ldots,F_n\in\Sigma,\,\bigcup_{i\le 
n}F_i=E,\,F_i\cap F_j=\emptyset$ for $i\ne j\}$} 
      
\noindent for $E\in\Sigma$.   Show that $\check\nu$ and $\hat\nu$ are 
finitely additive and are, respectively, $\sup\{\nu_0,\ldots,\nu_n\}$ 
and 
$\inf\{\nu_0,\ldots,\nu_n\}$ in the partially ordered set of finitely 
additive 
functionals on $\Sigma$. 
      
\header{231Ye}{\bf (e)} Let $X$ be a set and $\Sigma$ a $\sigma$-algebra 
of subsets of $X$;  let $M$ be the partially ordered set of all bounded 
finitely additive functionals from $\Sigma$ to $\Bbb R$.   (i) Show that 
if $A\subseteq M$ is non-empty and bounded above  in $M$, then $A$ has a 
supremum $\check\nu$ in $M$, given by the formula 
      
$$\eqalign{\check\nu E 
=\sup\{\sum_{i=0}^n\nu_iF_i:\nu_0,\ldots,\nu_n\in 
A,\,F_0,&\ldots,F_n\in\Sigma,\,\bigcup_{i\le n}F_i=E,\cr 
&\qquad\quad F_i\cap F_j=\emptyset\text{ for }i\ne j\}.\cr}$$ 
      
\noindent (ii) Show that if $A\subseteq M$ is non-empty and bounded 
below in $M$ then it has an infimum $\hat\nu\in M$, given by the formula 
      
$$\eqalign{\hat\nu E 
=\inf\{\sum_{i=0}^n\nu_iF_i:\nu_0,\ldots,\nu_n\in 
A,\,F_0,&\ldots,F_n\in\Sigma,\,\bigcup_{i\le n}F_i=E,\cr 
&\qquad\quad F_i\cap F_j=\emptyset\text{ for }i\ne j\}.\cr}$$ 
      
\header{231Yf}{\bf (f)} Let $X$ be a set, $\Sigma$ an algebra of subsets 
of $X$, and $\nu:\Sigma\to\Bbb R$ a non-negative finitely additive 
functional.   For $E\in\Sigma$ set 
      
\Centerline{$\nu_{ca}(E)=\inf\{\sup_{n\in\Bbb N}\nu 
F_n:\sequencen{F_n}$ is a non-decreasing sequence in $\Sigma$ with union 
$E\}$.} 
      
\noindent Show that $\nu_{ca}$ is a countably additive functional on 
$\Sigma$ and that if $\nuprime$ is any countably additive functional with 
$\nuprime\le\nu$ then $\nuprime\le\nu_{ca}$.   Show that 
$\nu_{ca}\wedge(\nu-\nu_{ca})=0$. 
      
\header{231Yg}{\bf (g)} Let $X$ be a set, $\Sigma$ an algebra of subsets 
of $X$, and $\nu:\Sigma\to\Bbb R$ a bounded finitely additive 
functional.   Show that $\nu$ is uniquely expressible as 
$\nu_{ca}+\nu_{pfa}$, where $\nu_{ca}$ is countably 
additive, $\nu_{pfa}$ is finitely additive and if 
$0\le\nuprime\le|\nu_{pfa}|$ and $\nuprime$ is countably additive then 
$\nuprime=0$. 
      
\header{231Yh}{\bf (h)} Let $X$ be a set and $\Sigma$ an algebra of 
subsets of $X$.   Let $M$ be the linear space of bounded finitely 
additive functionals on $\Sigma$, and for $\nu\in M$ set 
$\|\nu\|=|\nu|(X)$, defining $|\nu|$ as in 231Ya.   ($\|\nu\|$ is the 
{\bf total variation} of $\nu$.)   Show that $\|\,\|$ 
is a norm on $M$ under which 
$M$ is a Banach space.   Show that the space of bounded countably 
additive functionals on $\Sigma$ is a closed linear subspace of $M$. 
      
\header{231Yi}{\bf (i)} Repeat as many as possible of the results of 
this section for complex-valued functionals. 
}%end of exercises 
      
\endnotes{ 
\Notesheader{231} The real purpose of this section has been to describe 
the Hahn decomposition of a countably additive functional (231E).   The 
leisurely exposition in 231A-231D is intended as a review of the 
most elementary properties of measures, in the slightly more general 
context of `signed measures', with those properties corresponding to 
`additivity' alone separated from those which depend on `countable 
additivity'.   In 231Xf I set out necessary and sufficient conditions 
for a finitely additive functional on a $\sigma$-algebra to be countably 
additive, designed to suggest that a finitely additive functional is 
countably additive iff it is `sequentially order-continuous' in some 
sense.   The fact that a countably additive functional can be expressed 
as the difference of non-negative countably additive functionals (231F) 
has an important counterpart in the theory of finitely additive 
functionals:  a 
finitely additive functional can be expressed as the difference of 
non-negative finitely additive functionals if (and only if) it is 
bounded (231Ya).   But I do not think that this, or the further 
properties of bounded finitely additive functionals described in 231Xe 
and 231Y, will be important to us before Volume 3. 
}%end of notes 
      
      
\discrpage 
      
