\frfilename{mt2.tex} 
\versiondate{24.2.16} 
 
\noindent{\bf Introduction to Volume 2} 
 
\medskip 
 
For this second volume I have chosen seven topics through which to 
explore the insights and challenges offered by measure theory.   Some, 
like the Radon-Nikod\'ym theorem (Chapter 23) are necessary for any 
understanding of the structure of the subject;   others, like Fourier 
analysis (Chapter 28) and the discussion of function spaces (Chapter 24) 
demonstrate the power of measure theory to attack problems in general 
real and functional analysis.   But all have applications outside 
measure theory, and all have influenced its development.   These are the 
parts of measure theory which any analyst may find himself using. 
 
Every topic is one which ideally one would wish undergraduates to have  
seen, but the length of this volume makes it plain that no ordinary  
undergraduate 
course could include very much of it.   It is directed rather at graduate  
level, where I hope it will be found adequate to support all but the most  
ambitious courses in measure theory, though it is perhaps a bit too solid  
to be suitable for direct 
use as a course text, except with careful selection of the parts to be 
covered.   If you are using it to teach yourself measure theory, I 
strongly recommend an eclectic approach, looking for particular subjects 
and theorems that seem startling or useful, and working backwards from 
them.   My other objective, of course, is to provide an account of the  
central ideas at this level in measure theory, rather fuller than can  
easily be found in one volume elsewhere.   I cannot claim that it is  
`definitive', but I do think I cover a good deal of ground in ways that 
provide a firm foundation for further study.   As in Volume 1, I usually 
do not shrink from giving `best' results, like Lindeberg's condition 
for the Central Limit Theorem (\S274), or the theory of products of 
arbitrary measure spaces (\S251).    
If I were teaching this material to students in a PhD programme 
I would rather accept a limitation in the breadth of the course than 
leave them unaware of what could be done in the areas discussed. 
 
The topics interact in complex ways -- one of the purposes of this book 
is to exhibit their relationships.   There is no canonical linear 
ordering in which they should be taken.   Nor do I think organization 
charts are very helpful, not least because it may be only two or three 
paragraphs in a section which are needed for a given chapter later on. 
I do at least try to lay the material of each section out in an order 
which makes initial segments useful by themselves.   But the order in 
which to take the chapters is to a considerable extent for you to 
choose, perhaps after a glance at their individual introductions.   I 
have done my best to pitch the exposition at much the same level 
throughout the volume, sometimes allowing gradients to steepen in the 
course of a chapter or a section, but always trying to return to 
something which anyone who has mastered Volume 1 ought to be able to 
cope with.   (Though perhaps the main theorems of Chapter 26 are harder 
work than the principal results elsewhere, and \S286 is only for the most  
determined.) 
 
I said there were seven topics, and you will see eight chapters ahead of 
you.   This is because Chapter 21 is rather different from the rest. 
It is the purest of pure measure theory, and is here only because there 
are places later in the volume where (in my view) the theorems make 
sense only in the light of some abstract concepts which are not 
particularly difficult, but are also not obvious.   However it is fair 
to say that the most important ideas of this volume do not really depend 
on the work of Chapter 21. 
 
As always, it is a puzzle to know how much prior knowledge to assume in 
this volume.   I do of course call on the results of Volume 1 of this 
treatise whenever they seem to be relevant.   I do not doubt, however, 
that there will be readers who have learnt the elementary theory from 
other sources.   Provided you can, from first principles, construct 
Lebesgue measure and prove the basic convergence theorems for integrals 
on arbitrary measure spaces, you ought to be able to embark on the 
present volume.   Perhaps it would be helpful to have in hand the 
results-only version of Volume 1, since that includes the most important 
definitions as well as statements of the theorems. 
 
There is also the question of how much material from outside measure 
theory is needed.   Chapter 21 calls for some non-trivial set theory 
(given in \S2A1), but the more advanced ideas are needed only for the 
counter-examples in \S216, and can be passed over to begin with.   The 
problems become acute in Chapter 24.   Here we need a variety of results 
from functional analysis, some of them depending on non-trivial ideas in 
general topology.   For a full understanding of this material there is 
no substitute for a course in normed spaces up to and including a study 
of weak compactness.   But I do not like to insist on such a 
preparation, because it is likely to be simultaneously too much and too 
little.   Too much, because I hardly mention linear operators at this 
stage;  too little, because I do ask for some of the theory of 
non-locally-convex spaces, which is often omitted in first courses on 
functional analysis.   At the risk, therefore, of wasting paper, I have 
written out condensed accounts of the essential facts  
(\S\S2A3-2A5).  %\S2A3 \S2A4 \S2A5 
 
\bigskip
     
\noindent{\bf Note on second printing, April 2003}
     
\medskip
     
For the second printing of this volume, I have made two substantial
corrections to inadequate proofs and a large number of minor amendments;
I am most grateful to T.D.Austin for his careful reading of the first
printing.   In addition, I have added a dozen exercises and a handful of
straightforward results which turn out to be relevant to the work of
later volumes and fit naturally here.
     
The regular process of revision of this work has led me to make a couple
of notational innovations not described explicitly in the early printings
of Volume 1.   I trust that most readers will find these immediately
comprehensible.   If, however, you find that there is a puzzling
cross-reference which you are unable to match with anything in the
version of Volume 1 which you are using, it may be worth while checking
the errata pages in {\tt
http://www.essex.ac.uk/maths/people/fremlin/mterr.htm}.
     
\bigskip

\noindent{\bf Note on hardback edition, January 2010}

\medskip

For the new (`Lulu') edition of this volume, I have eliminated a number of
further errors;  no doubt many remain.   There are many new exercises,
several new theorems
%214P, 222L Dini derivates, half of \S234, 244O, 251L, \S266, 272Q, 272W
and some corresponding rearrangements of material.
%\S234
The new results are mostly additions with little effect on the structure of
the work, but there is a short new section (\S266) on the Brunn-Minkowski
inequality.

\bigskip

\noindent{\bf Note on second printing of hardback edition, April 2016}

\medskip

There is the usual crop of small mistakes to be corrected, and assorted minor
amendments and additions.   But my principal reason for issuing a new printed
version is a major fault in the proof of Carleson's theorem, where an imprudent
move to simplify the argument of {\smc Lacey \& Thiele 00} was based on an
undergraduate error\footnote{I am most
grateful to A.Derighetti for bringing this to my attention}.   
While the blunder is conspicuous enough, a resolution
seems to require an adjustment in a definition, and is not a fair demand on
a graduate seminar, the intended readership for this material.
Furthermore, the proof was supposed to be a distinguishing feature of not
only this volume, but of the treatise as a whole.   So, with apologies to any
who retired hurt from an encounter with the original version, I present a
revision which I hope is essentially sound.

\frnewpage 
 
