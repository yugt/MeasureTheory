\frfilename{mt416.tex}
\versiondate{24.6.05}
\copyrightdate{2002}

\def\chaptername{Topologies and measures I}
\def\sectionname{Radon measure spaces}

\newsection{416}

We come now to the results for which the chapter so far has been
preparing.   The centre of topological measure theory is the theory of
`Radon' measures\cmmnt{ (411Hb)},
measures inner regular with respect to compact sets.   Most of the
section is devoted to pulling the earlier work together, and in
particular to
re-stating theorems on quasi-Radon measures in the new context.   Of
course this has to begin with a check that Radon measures are
quasi-Radon
(416A).   It follows immediately that Radon measures are (strictly)
localizable (416B).   After presenting a miscellany of elementary facts,
I turn to the constructions of \S413, which take on simpler and more
dramatic forms in this context (416J-416P).   I proceed to investigate
subspace measures (416R-416T) and some special product measures
(416U).   I end the section
with further notes on the forms which earlier theorems on Stone spaces
(416V) and compact measure spaces (416W) take when applied to Radon
measure spaces.

\leader{416A}{Proposition} A Radon measure space is quasi-Radon.

\proof{ Let $(X,\frak T,\Sigma,\mu)$ be a Radon measure space.   Because
$\frak T$ is Hausdorff, every compact set is closed, so $\mu$ is inner
regular with respect to the closed sets.   By 411E, $\mu$ is
$\tau$-additive;  by 411Gf, it is effectively locally finite.   Thus all
parts of condition (ii) of 411Ha are satisfied, and $\mu$ is a
quasi-Radon measure.
}%end of proof of 416A

\leader{416B}{Corollary} A Radon measure space is strictly localizable.

\proof{ Put 416A and 415A together.}

\leader{416C}{}\cmmnt{ In order to use the results of \S415
effectively, it will be helpful to spell out elementary conditions
ensuring that a quasi-Radon measure is Radon.

\medskip

\noindent}{\bf Proposition} Let $(X,\frak T,\Sigma,\mu)$ be a locally
finite Hausdorff quasi-Radon measure space.   Then the following are
equiveridical:

\inset{(i) $\mu$ is a Radon measure;}

\inset{(ii) whenever $E\in\Sigma$ and $\mu E>0$ there is a compact set
$K$ such that $\mu(E\cap K)>0$;}

\inset{(iii) $\sup\{K^{\ssbullet}:K\subseteq X$ is compact$\}=1$ in the
measure algebra of $\mu$.}

\noindent If $\mu$ is totally finite we can add

\inset{(iv) $\sup\{\mu K:K\subseteq X$ is compact$\}=\mu X$.}

\proof{(i)$\Rightarrow$(ii) and (ii)$\Leftrightarrow$(iii) are trivial.   For
(ii)$\Rightarrow$(i), observe that if $E\in\Sigma$ and $\mu E>0$ there
is a compact set $K\subseteq E$ such that $\mu K>0$.   \Prf\ There is a
compact set $K'$ such that $\mu(E\cap K')>0$, by hypothesis;  now there
is a closed
set $K\subseteq E\cap K'$ such that $\mu K>0$, because $\mu$ is inner
regular with respect to the closed sets, and $K$ is compact.\ \QeD\   By
412B, $\mu$ is tight.   Being a
complete, locally determined, locally finite topological measure, it is
a Radon measure.

When $\mu X<\infty$, of course, we also have (ii)$\Leftrightarrow$(iv).
}%end of proof of 416C

\leader{416D}{}\cmmnt{ Some further elementary facts are worth writing
out plainly.

\medskip

\noindent}{\bf Lemma} (a) In a Radon measure space, every compact set
has finite measure.

(b) Let $(X,\frak T,\Sigma,\mu)$ be a Radon measure space, and
$E\subseteq X$ a set such that $E\cap K\in\Sigma$ for every compact
$K\subseteq X$.   Then $E\in\Sigma$.

(c) A Radon measure is inner regular with respect to the
self-supporting compact sets.

(d) Let $X$ be a Hausdorff space and $\mu$ a tight locally finite complete
locally determined measure on $X$.   If $\mu$ measures
every compact set, $\mu$ is a Radon measure.

(e)\dvAnew{2009} 
Let $X$ be a Hausdorff space and $\familyiI{\mu_i}$ a family of Radon
measures on $X$.   Let $\mu=\sum_{i\in I}\mu_i$ be their
sum\cmmnt{ (definition: 234G\formerly{112Ya})}.  %from 2008
Suppose that $\mu$ is locally finite.   Then it is a Radon measure.

\proof{{\bf (a)} 411Ga.

\medskip

{\bf (b)} We have only to remember that $\mu$ is complete, locally
determined and tight, and apply 413F(ii).

\medskip

{\bf (c)} If $(X,\frak T,\Sigma,\mu)$ is a Radon measure space,
$E\in\Sigma$ and $\gamma<\mu E$, there is a compact set $K\subseteq E$
such that $\mu K\ge\gamma$.   By 414F, there is a self-supporting
relatively closed set $L\subseteq K$ such that $\mu L=\mu K$;  but now
of course $L$ is compact, while $L\subseteq E$ and $\mu L\ge\gamma$.

\medskip

{\bf (d)} Let $\Cal K$ be the family of compact subsets of $X$;  write
$\Sigma$ for the domain of $\mu$.   If $F\subseteq X$ is closed, then
$F\cap K\in\Cal K\subseteq\Sigma$ for every $K\in\Cal K$;  accordingly
$F\in\Sigma$, by 412Ja.   But this means that every closed set,
therefore every open set, belongs to $\Sigma$, and $\mu$ is a Radon
measure.

\medskip

{\bf (e)} Because every $\mu_i$ is a topological measure,
so is $\mu$;  because every $\mu_i$ is complete, so is $\mu$
(234Ha).   By hypothesis, $\mu$ is locally finite.
If $\mu E>0$, then there is some $i\in I$ such that $\mu_iE>0$;
now there is a
compact $K\subseteq E$ such that $0<\mu_iK\le\mu K$.   So $\mu$ is inner
regular with respect to the compact sets.

Now suppose that $E\subseteq X$ is such that $\mu$ measures $E\cap F$
whenever $\mu F<\infty$.   Then, in particular, $E\cap K$ is measured by
$\mu$, therefore measured by every $\mu_i$, whenever
$K\subseteq X$ is compact.   By 413F(ii), $\mu_i$ measures $E$ for every
$i$, so $\mu$ measures $E$.   As $E$ is
arbitrary, $\mu$ is locally determined and is a Radon measure.
}%end of proof of 416D

\cmmnt{\medskip

\noindent{\bf Remark} In (e) above, note that if $I$ is finite then
$\mu$ is necessarily locally finite.}

\leader{416E}{Specification of Radon \dvrocolon{measures}}\cmmnt{ In
415H I described some conditions which enable us to be sure that
two quasi-Radon measures on a given topological space are the same.   In
the case of Radon measures we have a similar list.   This time I include
a note on the natural ordering of Radon measures.

\medskip

\noindent}{\bf Proposition} Let $X$ be a Hausdorff space and $\mu$,
$\nu$ two Radon measures on $X$.

(a) The following are equiveridical:

\inset{(i) $\nu\le\mu$ in the sense of 234P\cmmnt{, that
is, $\nu E$ is defined and $\nu E\le\mu E$ whenever $\mu$ measures $E$};

(ii) $\mu K\le\nu K$ for every compact set $K\subseteq X$;

(iii) $\mu G\le\nu G$ for every open set $G\subseteq X$;

(iv) $\mu F\le\nu F$ for every closed set $F\subseteq X$.}

\noindent If $X$ is locally compact, we can add

\inset{(v) $\int fd\mu\le\int fd\nu$ for every non-negative continuous
function $f:X\to\Bbb R$ with compact support.}

(b) The following are equiveridical:

\inset{(i) $\mu=\nu$;

(ii) $\mu K=\nu K$ for every compact set $K\subseteq X$;

(iii) $\mu G=\nu G$ for every open set $G\subseteq X$;

(iv) $\mu F=\nu F$ for every closed set $F\subseteq X$.}

\noindent If $X$ is locally compact, we can add

\inset{(v) $\int fd\mu=\int fd\nu$ for every continuous function
$f:X\to\Bbb R$ with compact support.}

\proof{{\bf (a)(i)$\Rightarrow$(iv)$\Rightarrow$(ii)} and
{\bf (i)$\Rightarrow$(iii)} are trivial, if we recall that $\nu\le\mu$ when
$\dom\nu\supseteq\dom\mu$ and $\nu E\le\mu E$ for every $E\in\dom\mu$.

\medskip

\quad{\bf (ii)$\Rightarrow$(i)} If (ii) is true, then

\Centerline{$\mu E=\sup_{K\subseteq E\text{ is compact}}\mu K
\le\sup_{K\subseteq E\text{ is compact}}\nu K=\nu E$}

\noindent for every set $E$ measured by both $\mu$ and $\nu$.   Also
$\dom\nu\subseteq\dom\mu$.   \Prf\ Suppose that $E\in\dom\nu$ and that
$K\subseteq X$ is a compact set such that $\mu K>0$.   Then there are
compact sets $K_1\subseteq K\cap E$, $K_2\subseteq K\setminus E$ such
that

\Centerline{$\nu K_1+\nu K_2>\nu(K\cap E)+\nu(K\setminus E)-\mu K
=\nu K-\mu K$.}

\noindent So

\Centerline{$\mu(K\setminus(K_1\cup K_2))
\le\nu(K\setminus(K_1\cup K_2))<\mu K$}

\noindent and $\mu K_1+\mu K_2>0$.   This shows that
$\mu_*(K\cap E)+\mu_*(K\setminus E)>0$.   As $K$ is arbitrary,
$E\in\dom\mu$ (413F(vii)).\ \Qed

So (i) is true.

\medskip

\quad{\bf (iii)$\Rightarrow$(ii)} The point is that if $K\subseteq X$ is
compact, then
$\mu K=\inf\{\mu G:G\subseteq X$ is open, $K\subseteq G\}$.
\Prf\ Because $X=\bigcup\{\mu G:G\subseteq X$ is open, $\mu G<\infty\}$,
there is an open set $G_0$ of finite measure including $K$.   Now, for
any $\gamma>\mu K$, there is a compact set $L\subseteq G_0\setminus K$
such that $\mu L\ge\mu G_0-\gamma$, so that $\mu G\le\gamma$, where
$G=G_0\setminus L$ is an open set including $K$.\ \Qed

The same is true for $\nu$.   So, if (iii) is true,

\Centerline{$\mu K=\inf_{G\supseteq K\text{ is open}}\mu G
\le\inf_{G\supseteq K\text{ is open}}\nu G=\nu K$}

\noindent for every compact $K\subseteq X$, and (ii) is true.

\medskip

\quad{\bf (iii)$\Rightarrow$(v)} If (iii) is true and
$f:X\to\coint{0,\infty}$ is a non-negative continuous function, then

$$\eqalignno{\int fd\mu
&=\int_0^{\infty}\mu\{x:f(x)>t\}dt\cr
\displaycause{252O}
&\le\int_0^{\infty}\nu\{x:f(x)>t\}dt
=\int fd\nu.\cr}$$

\medskip

\quad{\bf (v)$\Rightarrow$(iii)} If $X$ is locally compact and (v) is
true, take any open set $G\subseteq X$, and consider
the set $A$ of continuous functions $f:X\to[0,1]$
with compact support such that $f\le\chi G$.
Then $A$ is upwards-directed and $\sup_{f\in A}f(x)=\chi G(x)$
for every $x\in X$, by 4A2G(e-i).   So

\Centerline{$\mu G=\sup_{f\in A}\int fd\mu
\le\sup_{f\in A}\int fd\nu=\nu G$}

\noindent by 414Ba.   As $G$ is arbitrary, (iii) is true.

\medskip

{\bf (b)} now follows at once, or from 415H.
}%end of proof of 416E

\leader{416F}{Proposition} Let $X$ be a Hausdorff space and $\mu$ a
Borel measure on $X$.   Then the following are equiveridical:

\quad(i) $\mu$ has an extension to a Radon measure on $X$;

\quad(ii) $\mu$ is locally finite and tight;

\quad(iii) $\mu$ is locally finite and effectively locally finite, and
$\mu G=\sup\{\mu K:K\subseteq G$ is compact$\}$ for every open set
$G\subseteq X$;

\quad(iv) $\mu$ is locally finite,
effectively locally finite and $\tau$-additive,
and $\mu G=\sup\{\mu(G\cap K):K\subseteq X$ is compact$\}$ for every
open set $G\subseteq X$.

In this case the extension is unique;  it is the c.l.d.\ version
of $\mu$.

\proof{{\bf (a)(i)$\Rightarrow$(iv)} If $\mu=\tilde\mu\restr\Cal B(X)$
where $\tilde\mu$ is a Radon measure and $\Cal B(X)$ is the Borel
$\sigma$-algebra of $X$, then of course $\mu$ is locally finite and
effectively locally finite and $\tau$-additive because $\tilde\mu$ is (see 416A)
and every
open set belongs to $\Cal B(X)$.   Also

\Centerline{$\mu G=\sup\{\mu K:K\subseteq G$ is compact$\}
\le\sup\{\mu(G\cap K):K\subseteq X$ is compact$\}\le\mu G$}

\noindent for every open set $G\subseteq X$, because $\tilde\mu$ is tight
and compact sets belong to $\Cal B(X)$.

\medskip

{\bf (b)(iv)$\Rightarrow$(iii)} Suppose that (iv) is true.
Of course $\mu$ is locally finite and effectively locally finite.
Suppose that $G\subseteq X$ is open and that $\gamma<\mu G$.
Then there is a compact $K\subseteq X$ such that $\mu(G\cap K)>\gamma$.
By 414K, the subspace measure $\mu_K$ is $\tau$-additive.
Now $K$ is a compact Hausdorff space, therefore regular.
By 414Ma there is a closed set $F\subseteq G\cap K$ such that
$\mu_KF\ge\gamma$.   Now $F$ is compact,
$F\subseteq G$ and $\mu F\ge\gamma$.
As $G$ and $\gamma$ are arbitrary, (iii) is true.

\medskip

{\bf (c)(iii)$\Rightarrow$(ii)} I have to show that if $\mu$ satisfies
the conditions of (iii) it is tight.
Let $\Cal K$ be the family of compact subsets of $X$ and
$\Cal A$ the family of subsets of $X$ which are either open or closed.
Then whenever $A\in\Cal A$, $F\in\Sigma$ and $\mu(A\cap F)>0$, there is
a $K\in\Cal K$ such that $K\subseteq A$ and $\mu(K\cap F)>0$.   \Prf\
Because $\mu$ is effectively locally finite, there is an open set $G$ of
finite measure such that $\mu(G\cap A\cap F)>0$.   ($\alpha$) If $A$ is
open, then there will be a compact set $K\subseteq G\cap A$ such that
$\mu K>\mu(G\cap A)-\mu(G\cap A\cap F)$, so that $\mu(K\cap F)>0$.
($\beta$) If $A$ is closed, then let $L\subseteq G$ be a compact set
such that $\mu L>\mu G-\mu(G\cap A\cap F)$;  then $K=L\cap A$ is compact
and $\mu(K\cap F)>0$.\ \Qed

By 412C, $\mu$ is inner regular with respect to $\Cal K$, as required.

\medskip

{\bf (d)(ii)$\Rightarrow$(i)} If $\mu$ is locally finite and tight,
let $\tilde\mu$ be the c.l.d.\ version
of $\mu$.   Then $\tilde\mu$ is complete, locally determined, locally
finite (because $\mu$ is), a topological measure (because $\mu$ is) and
tight (because $\mu$ is, using 412Ha);  so is a Radon
measure.   Every compact set has finite measure for $\mu$, so $\mu$ is
semi-finite and $\tilde\mu$ extends $\mu$ (213Hc).

\medskip

{\bf (e)} By 416Eb there can be at most one Radon measure extending
$\mu$, and we have observed in (d) above that in the present case it is
the c.l.d.\ version of $\mu$.
}%end of proof of 416F

\leader{416G}{}\cmmnt{ One of the themes of \S434 will be the
question:  on which Hausdorff spaces is every locally finite quasi-Radon
measure a Radon measure?   I do not think we are ready for a general
investigation of this, but I can give one easy special
result.

\medskip

\noindent}{\bf Proposition} Let $(X,\frak T)$ be a locally compact
Hausdorff space and $\mu$ a locally finite quasi-Radon measure on $X$.
Then $\mu$ is a Radon measure.

\proof{ $\mu$ satisfies condition (ii) of 416C.   \Prf\ Take
$E\in\dom\mu$
such that $\mu E>0$.   Let $\Cal G$ be the family of relatively compact
open subsets of $X$;  then $\Cal G$ is
upwards-directed and has union $X$.   By 414Ea, there is a $G\in\Cal G$
such that $\mu(E\cap G)>0$.   But now $\overline{G}$ is compact and
$\mu(E\cap\overline{G})>0$.\ \QeD\   By 416C, $\mu$ is a Radon measure.
}%end of proof of 416G

\leader{416H}{Corollary} Let $(X,\frak T)$ be a locally compact
Hausdorff space, and $\mu$ a locally finite, effectively locally
finite, $\tau$-additive Borel measure on $X$.   Then $\mu$ is tight and
its c.l.d.\ version
is a Radon measure, the unique Radon measure on $X$ extending $\mu$.

\proof{ By 415Cb, the c.l.d.\ version $\tilde\mu$ of $\mu$ is a
quasi-Radon measure extending $\mu$.   Because $\mu$ is locally
finite, so is $\tilde\mu$;  by 416G, $\tilde\mu$ is a Radon measure.
By 416Eb, the extension is unique.   Now

\Centerline{$\mu E=\tilde\mu E
=\sup_{K\subseteq E\text{ is compact}}\tilde\mu K
=\sup_{K\subseteq E\text{ is compact}}\mu K$}

\noindent for every Borel set $E\subseteq X$, so $\mu$ itself is tight.
}%end of proof of 416H

\leader{416I}{}\cmmnt{ While on the subject of locally compact spaces,
I mention an important generalization of a result from Chapter 24.

\medskip

\noindent}{\bf Proposition} Let $(X,\frak T,\Sigma,\mu)$ be a locally
compact Radon measure space.   Write $C_k$ for the space of continuous
real-valued functions on $X$ with compact supports.   If
$1\le p<\infty$, $f\in\eusm L^p(\mu)$ and $\epsilon>0$, there is a
$g\in C_k$ such that $\|f-g\|_p\le\epsilon$.

\proof{ By 415Pa, there is a bounded continuous function
$h_1:X\to\Bbb R$ such that $G=\{x:h_1(x)\ne 0\}$ has finite measure and
$\|f-h_1\|_p\le\bover12\epsilon$.   Let $K\subseteq G$ be a compact set
such that
$\|h_1\|_{\infty}(\mu(G\setminus K))^{1/p}\le\bover12\epsilon$, and let
$h_2\in C_k$ be such that
$\chi K\le h_2\le\chi G$ (4A2G(e-i)).   Set $g=h_1\times h_2$.   Then
$g\in C_k$ and

\Centerline{$\int|h_1-g|^p\le\int_{G\setminus K}|h_1|^p
\le\mu(G\setminus K)\|h_1\|_{\infty}^p$,}

\noindent so $\|h_1-g\|_p\le\bover12\epsilon$ and
$\|f-g\|_p\le\epsilon$, as required.
}%end of proof of 416I

\leader{416J}{}\cmmnt{ I turn now to constructions of Radon measures
based on ideas in \S413.

\medskip

\noindent}{\bf Theorem} Let $X$ be a Hausdorff space.   Let $\Cal K$ be
the family of compact subsets of $X$ and
$\phi_0:\Cal K\to\coint{0,\infty}$ a functional such that

\inset{($\alpha$) $\phi_0K
=\phi_0L+\sup\{\phi_0K':K'\in\Cal K,\,K'\subseteq K\setminus L\}$
whenever $K$, $L\in\Cal K$ and $L\subseteq K$,}

\inset{($\gamma$) for every $x\in X$ there is an open set $G$ containing
$x$ such that $\sup\{\phi_0K:K\in\Cal K,\,K\subseteq G\}$ is finite.}

\noindent Then there is a unique Radon measure on $X$ extending
$\phi_0$.

\proof{ By 413M, there is a unique complete locally determined measure
$\mu$ on $X$, extending $\phi_0$, which is inner regular with respect to
$\Cal K$.   By ($\gamma$), $\mu$ is locally finite;  by 416Dd, it is a
Radon measure.
}%end of proof of 416J

\leader{416K}{Proposition}\cmmnt{ (see {\smc Tops{\o}e 70a})}
%Theorems 2-3 p 201
Let $X$ be a Hausdorff space,
$\Tau$ a subring of
$\Cal PX$ such that $\Cal H=\{G:G\in\Tau$ is open$\}$ covers $X$, and
$\nu:\Tau\to\coint{0,\infty}$ a finitely additive functional\cmmnt{
(definition:  361B)}.   Then there is a Radon measure $\mu$ on $X$ such
that $\mu K\ge\nu K$ for every compact $K\in\Tau$ and $\mu G\le\nu G$ for
every open $G\in\Tau$.

\proof{{\bf (a)} For $H\in\Cal H$ set $\Tau_H=\Tau\cap\Cal PH$;  then
$\Tau_H$ is an algebra of subsets of $H$, and $\nu_H=\nu\restr\Tau_H$ is
additive.   By 391G, there is an additive functional
$\tilde\nu_H:\Cal PH\to\coint{0,\infty}$ extending $\nu_H$.   Let $\frak F$
be an ultrafilter on $\Cal H$ containing $\{H:H_0\subseteq H\in\Cal H\}$
for every $H_0\in\Cal H$, and $\tilde\Tau$ the ideal of subsets of $X$
generated by $\Cal H$.   If $A\in\tilde\Tau$ then there is an
$H_0\in\Cal H$ including $A$, and now

\Centerline{$\tilde\nu_H(A\cap H)=\tilde\nu_H(A\cap H_0)
\le\nu_H(H\cap H_0)=\nu(H\cap H_0)\le\nu H_0$}

\noindent for every $H\in\Cal H$, so
$\tilde\nu A=\lim_{H\to\frak F}\nu_H(A\cap H)$ is defined in
$\coint{0,\infty}$.   Note that if $A\in\Tau\cap\tilde\Tau$ then there is
an $H_0\in\Cal H$ including $A$, so that $\nu_HA=\nu A$ whenever
$H\in\Cal H$ and $H\supseteq H_0$, and $\tilde\nu A=\nu A$.
Also $\tilde\nu:\tilde\Tau\to\coint{0,\infty}$ is
additive because all the functionals $A\mapsto\tilde\nu_H(A\cap H)$ are.

\medskip

{\bf (b)} Let $\Cal K$ be the family of compact subsets of $X$.
Because $X=\bigcup\Cal H$, $\Cal K\subseteq\tilde\Tau$.
For $K\in\Cal K$, set

\Centerline{$\phi_0K=\inf\{\tilde\nu G:G\in\tilde\Tau$ is open,
$K\subseteq G\}$.}

\noindent Then $\phi_0$ satisfies the conditions of 416J.   \Prf\
($\alpha$) Take $K$, $L\in\Cal K$ such that $L\subseteq K$, and
$\epsilon>0$.   Then there are open sets $G_0$, $H_0\in\tilde\Tau$ such
that

\Centerline{$K\subseteq G_0$,
\quad$\tilde\nu G_0\le\phi_0K+\epsilon$,
\quad$L\subseteq H_0$,
\quad$\tilde\nu H_0\le\phi_0L+\epsilon$.}

\noindent(i) If $K'\in\Cal K$ is such that
$K'\subseteq K\setminus L$, there are disjoint open sets $H$,
$H'\subseteq X$ such that $L\subseteq H$ and $K'\subseteq H'$ (4A2F(h-i)).   So

\Centerline{$\phi_0L+\phi_0K'
\le\tilde\nu(G_0\cap H)+\tilde\nu(G_0\cap H')\le\tilde\nu G_0
\le\phi_0K+\epsilon$.}

\noindent (ii) In the other direction, consider $K_1=K\setminus H_0$.
Then there is an open set $H_1\in\tilde\Tau$ such that $K_1\subseteq H_1$
and $\tilde\nu H_1\le\phi_0K_1+\epsilon$, so that

\Centerline{$\phi_0K\le\tilde\nu(H_0\cup H_1)\le\tilde\nu H_0+\tilde\nu H_1
\le\phi_0L+\phi_0K'+2\epsilon$.}

\noindent As $\epsilon$ is arbitrary,

\Centerline{$\phi_0K
=\phi_0L+\sup\{\phi_0K':K'\in\Cal K$, $K'\subseteq K\setminus L\}$}

\noindent as required by 416J($\alpha$).   ($\gamma$) If $x\in X$ there is
an $H_0\in\Cal H$ containing $x$, and now

\Centerline{$\sup\{\phi_0K:K\in\Cal K$, $K\subseteq H_0\}
\le\tilde\nu H_0$}

\noindent is finite.   So the second hypothesis also is satisfied.\ \Qed

\medskip

{\bf (c)} By 416J, we have a Radon measure $\mu$ on $X$ extending $\phi_0$.
If $K\in\Tau$ is compact, then $\mu K\ge\nu K$.   \Prf\ Since
$K\in\Tau\cap\tilde\Tau$, $\tilde\nu K=\nu K$.   Now

\Centerline{$\mu K=
\phi_0K=\inf\{\tilde\nu G:G\in\tilde\Tau$ is open, $K\subseteq G\}
\ge\tilde\nu K=\nu K$.  \Qed}

\noindent If $G\in\Tau$ is open, then
$\mu G=\sup\{\mu K:K\subseteq G$ is compact$\}$.   But if $K\subseteq G$ is
compact then

\Centerline{$\mu K=\phi_0K\le\tilde\nu G=\nu G$,}

\noindent so $\mu G\le\nu G$.

Thus $\mu$ has the required properties.
}%end of proof of 416K

\leader{416L}{Proposition} Let $X$ be a regular Hausdorff space.   Let
$\Cal K$ be the family of compact subsets of $X$, and
$\phi_0:\Cal K\to\coint{0,\infty}$ a functional such that

\inset{($\alpha_1$) $\phi_0K\le\phi_0(K\cup L)\le\phi_0K+\phi_0L$ for
all $K$, $L\in\Cal K$,}

\inset{($\alpha_2$) $\phi_0(K\cup L)=\phi_0K+\phi_0L$ whenever $K$,
$L\in\Cal K$ and $K\cap L=\emptyset$,}

\inset{($\gamma$) for every $x\in X$ there is an open set $G$ containing
$X$ such that
$\sup\{\phi_0 K:K\in\Cal K,\,
  K\subseteq G\}$ is finite.}

\noindent Then there is a unique Radon measure $\mu$ on $X$ such that

\Centerline{$\mu K=\inf_{G\subseteq X\text{ is open},K\subseteq G}
  \sup_{L\subseteq G\text{ is compact}}\phi_0L$}

\noindent for every $K\in\Cal K$.

\proof{{\bf (a)} For open sets $G\subseteq X$ set

\Centerline{$\psi G=\sup_{L\in\Cal K,L\subseteq G}\phi_0L$,}

\noindent and for compact sets $K\subseteq X$ set

\Centerline{$\phi_1K
=\inf\{\psi G:G\subseteq X$ is open, $K\subseteq G\}$.}

\noindent  Evidently $\psi G\le\psi H$ whenever $G\subseteq H$.   We
need to know that $\psi(G\cup H)\le\psi G+\psi H$ for all open sets $G$,
$H\subseteq X$.   \Prf\ If $L\subseteq G\cup H$ is compact, then the
disjoint compact sets
$L\setminus G$, $L\setminus H$ can be separated by disjoint open sets
$H'$, $G'$ (4A2F(h-i) again);  now $L\setminus G'\subseteq H$,
$L\setminus H'\subseteq G$ are compact and cover $L$, so

\Centerline{$\phi_0L
\le\phi_0(L\setminus G')+\phi_0(L\setminus H')
\le\psi H+\psi G$.}

\noindent As $L$ is arbitrary, $\psi(G\cup H)\le\psi G+\psi H$.\ \Qed

Moreover, $\psi(G\cup H)=\psi G+\psi H$ if $G\cap H=\emptyset$.   \Prf\
If $K\subseteq G$, $L\subseteq H$ are compact, then

\Centerline{$\phi_0K+\phi_0L
=\phi_0(K\cup L)
\le\psi(G\cup H)$.}

\noindent As $K$ and $L$ are arbitrary, $\psi G+\psi H\le\psi(G\cup
H)$.\ \Qed

\medskip

{\bf (b)} It follows that $\phi_1K$ is finite for every compact
$K\subseteq X$.   \Prf\ Set $\Cal G=\{G:G\subseteq X$ is open,
$\psi G<\infty\}$.   Then (a) tells us that $\Cal G$ is
upwards-directed.
But also we are supposing that $\Cal G$ covers $X$, by ($\gamma$).
So if $K\subseteq X$ is compact
there is a member of $\Cal G$ including $K$ and $\phi_1K<\infty$.\ \Qed

\medskip

{\bf (c)} Now $\phi_1$ satisfies the conditions of 416J.

\medskip

\Prf\grheada\ Suppose that $K$, $L\in\Cal K$ and $L\subseteq K$.   Set
$\gamma=\sup\{\phi_1M:M\in\Cal K,\,M\subseteq K\setminus L\}$.   Take
any $\epsilon>0$.

Let $G$ be an open set such that $K\subseteq G$ and
$\psi G\le\phi_1K+\epsilon$.   If $M\in\Cal K$ and
$M\subseteq K\setminus L$,
there are disjoint open sets $U$, $V$ such that $L\subseteq U$ and
$M\subseteq V$ (4A2F(h-i) once more);  we may suppose that
$U\cup V\subseteq G$.   In this case,

$$\eqalignno{\phi_1L+\phi_1M
&\le\psi U+\psi V
=\psi(U\cup V)\cr
\noalign{\noindent (by the second part of (a) above)}
&\le\psi G
\le\phi_1K+\epsilon.\cr}$$


\noindent As $M$ is arbitrary, $\gamma\le\phi_1K-\phi_1L+\epsilon$.

On the other hand, there is an open set $H$ such that $L\subseteq H$ and
$\psi H\le\phi_1L+\epsilon$.   Set $F=K\setminus H$, so that $F$ is a
compact subset of $K\setminus L$.   Then there is an open set $V$ such
that $F\subseteq V$ and $\psi V\le\phi_1F+\epsilon$.   In this case
$K\subseteq H\cup V$, so

$$\eqalign{\phi_1K
&\le\psi(H\cup V)
\le\psi H+\psi V\cr
&\le\phi_1L+\phi_1F+2\epsilon
\le\phi_1L+\gamma+2\epsilon,\cr}$$

\noindent so $\gamma\ge\phi_1K-\phi_1L-2\epsilon$.

As $\epsilon$ is arbitrary, $\gamma=\phi_1K-\phi_1L$;  as $K$ and $L$
are arbitrary, $\phi_1$ satisfies condition ($\alpha$) of 416J.

\medskip

\quad\grheadc\ Any $x\in X$ is contained in an open set $G$ such that
$\psi G<\infty$;  but now $\sup\{\phi_1K:K\in\Cal K,\,K\subseteq
G\}\le\psi G$ is finite.   So $\phi_1$ satisfies condition ($\gamma$) of
416J.\ \Qed

\medskip

{\bf (d)} By 416J, there is a unique Radon measure on $X$ extending
$\phi_1$, as claimed.
}%end of proof of 416L

\leader{416M}{Corollary} Let $X$ be a locally compact Hausdorff space.
Let $\Cal K$ be the family of compact subsets of $X$, and
$\phi_0:\Cal K\to\coint{0,\infty}$ a functional such that

\inset{$\phi_0K\le\phi_0(K\cup L)\le\phi_0K+\phi_0L$ for all $K$,
$L\in\Cal K$,}

\inset{$\phi_0(K\cup L)=\phi_0K+\phi_0L$ whenever $K$, $L\in\Cal K$
and $K\cap L=\emptyset$.}

\noindent Then there is a unique Radon measure $\mu$ on $X$ such that

\Centerline{$\mu K
=\inf\{\phi_0K':K'\in\Cal K,\,K\subseteq\interior K'\}$}

\noindent for every $K\in\Cal K$.

\proof{ Observe that $\phi_0$ satisfies the conditions of 416L;
416L($\gamma$) is true because $X$ is locally compact.
Define $\psi$, $\phi_1$ as in the proof of 416L, and set

\Centerline{$\phi_1'K=\inf\{\phi_0K':K'\in\Cal K,\,K\subseteq\interior
K'\}$}

\noindent for every $K\in\Cal K$.   Then $\phi_1'=\phi_1$.   \Prf\ Let
$K\in\Cal K$, $\epsilon>0$.   (i) There is an open set $G\subseteq X$
such that $K\subseteq G$ and $\psi G\le\phi_1K+\epsilon$.   Now the
relatively compact open subsets with closures included in $G$ form an
upwards-directed cover of $K$, so there is a $K'\in\Cal K$ such that
$K\subseteq\interior K'$ and $K'\subseteq G$.   Accordingly

\Centerline{$\phi_1'K\le\phi_0K'\le\psi G\le\phi K+\epsilon$.}

\noindent (ii) There is an $L\in\Cal K$ such that
$K\subseteq\interior L$ and $\phi_0L\le\phi_1'K+\epsilon$, so that

\Centerline{$\phi_1K\le\psi(\interior
L)\le\phi_0L\le\phi'_1K+\epsilon$.}

\noindent (iii) As $K$ and $\epsilon$ are arbitrary,
$\phi'_1=\phi_1$.\ \Qed

Now 416L tells us that there is a unique Radon measure extending
$\phi_1$, and this is the measure we seek.
}%end of proof of 416M

\leader{416N}{}\cmmnt{ The extension theorems in the second half of
\S413 also have important applications to Radon measures.

\medskip

\noindent}{\bf Henry's theorem}\cmmnt{ ({\smc Henry 69})} Let $X$ be a
Hausdorff space and $\mu_0$
a measure on $X$ which is locally finite and tight.   Then $\mu_0$ has an
extension to a Radon measure
$\mu$ on $X$;  and the extension may be made in such a way that whenever
$\mu E<\infty$ there is an $E_0\in\Sigma_0$ such that
$\mu(E\symmdiff E_0)=0$.

\proof{ All the work has been done in \S413;  we need to check here only
that the family $\Cal K$ of compact subsets of $X$ and the measure
$\mu_0$ satisfy the hypotheses of 413O.   But ($\dagger$) and
($\ddagger$) there
are elementary, and $\mu_0^*K<\infty$ for every $K\in\Cal K$ by 411Ga.

Now take the measure $\mu$ from 413O.   It is complete, locally
determined and inner regular with respect to $\Cal K$;  also
$\Cal K\subseteq\dom\mu$.   Because $\mu_0$ is locally finite and $\mu$
extends $\mu_0$, $\mu$ is locally finite.   By 416Dd, $\mu$ is a Radon
measure.   And the construction of 413O ensures that every set of finite
measure for
$\mu$ differs from a member of $\Sigma_0$ by a $\mu$-negligible set.
}%end of proof of 416N

\leader{416O}{Theorem} Let $X$ be a Hausdorff space and $\Tau$ a
subalgebra of $\Cal PX$.   Let $\nu:\Tau\to\coint{0,\infty}$ be a
finitely additive functional such that

\Centerline{$\nu E=\sup\{\nu F:F\in\Tau,\,F\subseteq E,\,F$ is closed$\}$
for every $E\in\Tau$,}

\Centerline{$\nu X
=\sup_{K\subseteq X\text{ is compact}}\inf_{F\in\Tau,F\supseteq K}\nu F$.}

\noindent Then there is a Radon measure $\mu$ on $X$ extending $\nu$.

\proof{{\bf (a)} For $A\subseteq X$, write

\Centerline{$\nu^*A=\inf_{F\in\Tau,F\supseteq A}\nu F$.}

\noindent Let $\sequencen{K_n}$ be a sequence of compact subsets of $X$
such that $\lim_{n\to\infty}\nu^*K_n=\nu X$;  replacing $K_n$ by
$\bigcup_{i<n}K_i$ if necessary, we may suppose that $\sequencen{K_n}$
is non-decreasing and that $K_0=\emptyset$.   For each $n\in\Bbb N$, set

\Centerline{$\nuprime_nE=\nu^*(E\cap K_n)$}

\noindent for every $E\in\Tau$.   Then $\nuprime_n$ is additive.   \Prf\
(I copy from the proof of 413N.)
If $E$, $F\in\Tau$ and $E\cap F=\emptyset$,

$$\eqalignno{\nuprime_n(E\cup F)
&=\inf\{\nu G:G\in\Tau,\,K_n\cap(E\cup F)\subseteq G\}\cr
&=\inf\{\nu G:
  G\in\Tau,\,K_n\cap(E\cup F)\subseteq G\subseteq E\cup F\}\cr
&=\inf\{\nu(G\cap E)+\nu(G\cap F):
  G\in\Tau,\,K_n\cap(E\cup F)\subseteq G\subseteq E\cup F\}\cr
&=\inf\{\nu G_1+\nu G_2:
  G_1,\,G_2\in\Tau,\,K_n\cap E\subseteq G_1\subseteq E,\,
  K_n\cap F\subseteq G_2\subseteq F\}\cr
&=\inf\{\nu G_1:G_1\in\Tau,\,K_n\cap E\subseteq G_1\subseteq E\}\cr
&\qquad\qquad\qquad
  +\inf\{\nu G_2:G_2\in\Tau,\,K_n\cap F\subseteq G_2\subseteq F\}\cr
&=\nuprime_nE+\nuprime_nF.\cr}$$

\noindent As $E$ and $F$ are arbitrary, $\nuprime_n$ is additive.\ \Qed

\medskip

{\bf (b)} For each $n\in\Bbb N$, set
$\nu_nE=\nuprime_{n+1}E-\nuprime_nE$ for
every $E\in\Tau$;  then $\nu_n$ is additive.   Because
$K_{n+1}\supseteq K_n$, $\nu_n$ is non-negative.

If $E\in\Tau$ and $E\cap K_{n+1}=\emptyset$, then
$\nu_nE=\nuprime_{n+1}E=0$.   So if we set
$\Tau_n=\{E\cap K_{n+1}:E\in\Tau\}$, we have an additive functional
$\tilde\nu_n:\Tau_n\to\coint{0,\infty}$ defined by setting
$\tilde\nu_n(E\cap K_{n+1})=\nu_nE$ for every $E\in\Tau$.
Also $\tilde\nu_nH=\sup\{\tilde\nu_nK:K\in\Tau_n$, $K\subseteq H$, $K$
is compact$\}$ for every $H\in\Tau_n$.   \Prf\ Express $H$ as
$E\cap K_{n+1}$ where $E\in\Tau$.   Given $\epsilon>0$, there is a closed set
$F\in\Tau$ such that $F\subseteq E$ and $\nu F\ge\nu E-\epsilon$;  but
now $K=F\cap K_{n+1}\in\Tau_n$ is a compact subset of $H$, and

\Centerline{$\tilde\nu_n(H\setminus K)
=\nu_n(E\setminus F)
\le\nuprime_{n+1}(E\setminus F)
\le\nu(E\setminus F)
\le\epsilon$,}

\noindent so $\tilde\nu_nK\ge\tilde\nu_nH-\epsilon$.\ \Qed

\medskip

{\bf (c)} For each $n\in\Bbb N$, we have a Radon measure $\mu_n$ on
$K_{n+1}$, with domain $\Sigma_n$ say, such that
$\mu_nK_{n+1}\le\tilde\nu_nK_{n+1}$ and $\mu_nK\ge\tilde\nu_nK$ for
every compact set $K\subseteq K_{n+1}$ (416K).   Since $K_{n+1}$ is
itself compact, we must have $\mu_nK_{n+1}=\tilde\nu_nK_{n+1}$.   But
this means that $\mu_n$ extends $\tilde\nu_n$.   \Prf\ If $H\in\Tau_n$ and
$\epsilon>0$ there is a compact set $K\in\Tau_n$ such that $K\subseteq H$
and $\tilde\nu_nK\ge\tilde\nu_nH-\epsilon$, so that
$(\mu_n)_*H\ge\mu_nK\ge\tilde\nu_nH-\epsilon$;  as $\epsilon$ is
arbitrary, $(\mu_n)_*H\ge\tilde\nu_nH$.   So there is an
$F_1\in\Sigma_n$ such that $F_1\subseteq H$ and
$\mu_nF_1\ge\tilde\nu_nH$.   Similarly, there is an $F_2\in\Sigma_n$
such that $F_2\subseteq K_{n+1}\setminus H$ and
$\mu_nF_2\ge\tilde\nu_n(K_{n+1}\setminus H)$.   But in this case
$H\setminus F_1\subseteq K_{n+1}\setminus(F_1\cup F_2)$ is
$\mu_n$-negligible, because

\Centerline{$\mu_nF_1+\mu_nF_2
\ge\tilde\nu_nH+\tilde\nu_n(K_{n+1}\setminus H)
=\tilde\nu_nK_{n+1}
=\mu_nK_{n+1}$.}

\noindent So $H\setminus F_1$ and $H$ belong to $\Sigma_n$ and
$\mu_nH=\mu_nF_1=\tilde\nu_nH$.\ \Qed

\medskip

{\bf (d)} Set

\Centerline{$\Sigma=\{E:E\subseteq X,\,E\cap K_{n+1}\in\Sigma_n$ for
every $n\in\Bbb N\}$,}

\Centerline{$\mu E=\sum_{n=0}^{\infty}\mu_n(E\cap K_{n+1})$ for every
$E\in\Sigma$.}

\noindent Then $\mu$ is a Radon measure on $X$ extending $\nu$.

\medskip

\Prf\ {\bf (i)} It is easy to check that $\Sigma$ is a $\sigma$-algebra
of subsets of $X$ including $\Tau$, just because each $\Sigma_n$ is a
$\sigma$-algebra of subsets of $K_{n+1}$ including $\Tau_n$;  and that
$\mu$ is a complete measure because every $\mu_n$ is.

\medskip

\quad{\bf (ii)} If $E\in\Tau$, then

$$\eqalign{\mu E
&=\sum_{n=0}^{\infty}\mu_n(E\cap K_{n+1})
=\sum_{n=0}^{\infty}\tilde\nu_n(E\cap K_{n+1})
=\sum_{n=0}^{\infty}\nu_nE
=\lim_{n\to\infty}\sum_{i=0}^n\nu_iE\cr
&=\lim_{n\to\infty}\sum_{i=0}^n
  \nu^*(E\cap K_{i+1})-\nu^*(E\cap K_i)
=\lim_{n\to\infty}\nu^*(E\cap K_n)
\le\nu E.\cr}$$

\noindent On the other hand,

\Centerline{$\mu X=\lim_{n\to\infty}\nu^*K_{n+1}=\nu X$,}

\noindent so in fact $\mu E=\nu E$ for every $E\in\Tau$, that is, $\mu$
extends $\nu$.   In particular, $\mu$ is totally finite, therefore
locally determined and locally finite.

\medskip

\quad{\bf (iii)} If $G\subseteq X$ is open, then $G\cap
K_{n+1}\in\Sigma_n$ for every $n$, so $G\in\Sigma$;  thus $\mu$ is a
topological measure.   If $\mu E>0$, there is some $n\in\Bbb N$ such
that $\mu_n(E\cap K_{n+1})>0$;  now there is a compact set
$K\subseteq
E\cap K_{n+1}$ such that $\mu_nK>0$, so that $\mu K>0$.   This shows
that $\mu$ is tight, so is a Radon measure, as required.\ \Qed
}%end of proof of 416O

\cmmnt{\medskip

\noindent{\bf Remark} Observe that in this construction

$$\eqalign{\mu K_{n+1}
&=\sum_{i=0}^{\infty}\mu_i(K_{n+1}\cap K_{i+1})
=\sum_{i=0}^{\infty}\tilde\nu_i(K_{n+1}\cap K_{i+1})
=\sum_{i=0}^{\infty}\nu_i(K_{n+1}\cap K_{i+1})\cr
&=\sum_{i=0}^{\infty}\nu'_{i+1}(K_{n+1}\cap K_{i+1})-\nu'_i(K_{n+1}\cap K_{i+1})\cr
&=\sum_{i=0}^{\infty}\nu^*(K_{n+1}\cap K_{i+1})-\nu^*(K_{n+1}\cap K_i)\cr
&=\sum_{i=0}^n\nu^*(K_{n+1}\cap K_{i+1})-\nu^*(K_{n+1}\cap K_i)
=\nu^*K_{n+1}\cr}$$

\noindent for every $n\in\Bbb N$.
What this means is that if instead of the hypothesis

\Centerline{$\nu X
=\sup_{K\subseteq X\text{ is compact}}
   \inf_{F\in\Tau,F\supseteq K}\nu F$}

\noindent we are presented with a specified non-decreasing sequence
$\sequencen{L_n}$ of compact subsets of $X$ such that
$\nu X=\sup_{n\in\Bbb N}\nu^*L_n$,
then we can take $K_{n+1}=L_n$ in the argument above
and we shall have $\mu L_n=\nu^*L_n$ for every $n$.
}%end of comment

\vleader{72pt}{416P}{Theorem} Let $X$ be a Hausdorff space and $\mu$ a
locally finite measure on $X$ which is inner regular with respect to the
closed sets.   Then the following are equiveridical:

\inset{(i) $\mu$ has an extension to a Radon measure on $X$;

(ii) for every non-negligible measurable set $E\subseteq X$ there is a
compact set $K\subseteq E$ such that $\mu^*K>0$.}

\noindent If $\mu$ is totally finite, we can add

\inset{(iii) $\sup\{\mu^*K:K\subseteq X$ is compact$\}=\mu X$.}

\proof{ Write $\Sigma$ for the domain of $\mu$.

\medskip

{\bf (a)(i)$\Rightarrow$(ii)} If $\lambda$ is a Radon measure
extending $\mu$, and $\mu E>0$,
then $\lambda E>0$, so there is a compact set $K\subseteq E$ such that
$\lambda K>0$;  but now, because $\lambda$ is an extension of $\mu$,

\Centerline{$\mu^*K\ge\lambda^*K=\lambda K>0$.}

\medskip

{\bf (b)(ii)$\Rightarrow$(i) \& (iii)}\grheada\ Let $\Cal E$ be the family
of measurable envelopes of compact sets.   Then $\mu E<\infty$ for every
$E\in\Cal E$.   \Prf\ If $E\in\Cal E$, there is a compact set $K$ such
that $E$ is a measurable envelope of $K$.   Now $\mu E=\mu^*K$ is finite
by 411Ga, as usual.\ \Qed

Next, $\Cal E$ is closed under finite unions, by 132Ed.
The hypothesis (ii) tells us that if $\mu E>0$ then there is some
$F\in\Cal E$ such that $F\subseteq E$ and $\mu F>0$;  for there is a
compact set $K\subseteq E$ such that $\mu^*K>0$, $K$ has a measurable
envelope $F_0$, and $F=E\cap F_0$ is still a measurable envelope of $K$.
So in fact $\mu$ is inner regular with respect to $\Cal E$ (412Aa).
In particular, $\mu$ is semi-finite.

If $\gamma<\mu X$ there is an $F\in\Cal E$ such that $\mu F\ge\gamma$,
and now there is a compact set $K$ such that $F$ is a measurable
envelope of $K$, so that $\mu^*K=\mu F\ge\gamma$.   As $\gamma$ is
arbitrary, (iii) is true.

\medskip

\quad\grheadb\ Because $\mu$ is inner regular with respect to $\Cal E$,
$D=\{E^{\ssbullet}:E\in\Cal E\}$ is order-dense in the measure algebra
$(\frak A,\bar\mu)$ of $\mu$ (412N), so there is a family
$\familyiI{d_i}$ in $D$ which is a partition of unity in $\frak A$
(313K).   For each $i\in I$, take $E_i\in\Cal E$ such that
$E_i^{\ssbullet}=d_i$.   Then

$$\eqalignno{\sum_{i\in I}\mu(E\cap E_i)
&=\sum_{i\in I}\bar\mu(E^{\ssbullet}\Bcap d_i)
=\bar\mu E^{\ssbullet}\cr
\displaycause{321E}
&=\mu E\cr}$$

\noindent for every $E\in\Sigma$.

\medskip

\quad\grheadc\ For each $i\in I$, let $\mu_i$ be the subspace measure on
$E_i$.   Then there is a Radon measure $\lambda_i$ on $E_i$ extending
$\mu_i$.   \Prf\ Because $\mu$ is inner regular with respect to the
closed sets, $\mu_i$ is inner regular with respect to the relatively
closed subsets of $E_i$ (412Oa).   Also there is a compact subset
$K\subseteq E_i$ such that

\Centerline{$\mu_iE_i=\mu E_i=\mu^*K=\mu_i^*K$,}

\noindent so $\mu_i$ satisfies the conditions of 416O and has an
extension to a Radon measure.\ \Qed

\medskip

\quad\grheadd\ Define

\Centerline{$\lambda E=\sum_{i\in I}\lambda_i(E\cap E_i)$}

\noindent whenever $E\subseteq X$ is such that $\lambda_i$ measures
$E\cap E_i$ for every $i\in I$.   Then $\lambda$ is a Radon measure on
$X$ extending $\mu$.   \Prf\ It is easy to check that it is a measure,
just because every $\lambda_i$ is a measure, and it extends $\mu$ by
($\beta$) above.   If $G\subseteq X$ is open, then $G\cap E_i$ is
relatively open for every $i\in I$, so $\lambda$ measures $G$;  thus
$\lambda$ is a topological measure.   If $\lambda E=0$ and
$A\subseteq E$, then
$\lambda_i(A\cap E_i)\le\lambda(E\cap E_i)=0$ for every $i$, so
$\lambda A=0$;  thus $\lambda$ is complete.   For all distinct $i$,
$j\in I$,

\Centerline{$\lambda_i(E_i\cap E_j)=\mu_i(E_i\cap E_j)
=\mu(E_i\cap E_j)=\bar\mu(d_i\Bcap d_j)=0$,}

\noindent so $\lambda E_i=\lambda_iE_i=\mu_iE_i$ is finite.   This means
that if $E\subseteq X$ is such that $\lambda$ measures $E\cap F$
whenever
$\lambda F<\infty$, then $\lambda$ must measure $E\cap E_i$ for every
$i$, and $\lambda$ measures $E$;  thus $\lambda$ is locally determined.
If $\lambda E>0$ there are an $i\in I$ such that $\lambda_i(E\cap
E_i)>0$ and a compact $K\subseteq E\cap E_i$ such that
$0<\lambda_iK=\lambda K$;  consequently $\lambda$ is tight.   Finally, if
$x\in X$, there is an $E\in\Sigma$ such that $x\in\interior E$ and
$\lambda E=\mu E<\infty$, so $\lambda$ is locally finite.   Thus
$\lambda$ is a Radon measure.\ \Qed

So (i) is true.

\medskip

{\bf (c)} Finally, suppose that $\mu$ is totally finite and (iii) is
true.   Then we can appeal directly to 416O to see that (i) is true.
}%end of proof of 416P

\leader{416Q}{Proposition} (a) Let $X$ be a compact Hausdorff space and
$\Cal E$ the algebra of open-and-closed subsets of $X$.   Then any
non-negative finitely additive functional from $\Cal E$ to $\Bbb R$ has
an extension to a Radon measure on $X$.   If $X$ is zero-dimensional
then the extension is unique.

(b) Let $\frak A$ be a Boolean algebra, and $Z$ its Stone space.   Then
there is a one-to-one correspondence between non-negative additive
functionals $\nu$ on $\frak A$ and Radon measures $\mu$ on $Z$ given by
the formula

\Centerline{$\nu a=\mu\widehat{a}$ for every $a\in\frak A$,}

\noindent where for $a\in\frak A$ I write $\widehat{a}$ for the
corresponding open-and-closed subset of $Z$.

\proof{{\bf (a)} Let $\nu:\Cal E\to\coint{0,\infty}$ be a non-negative
additive functional.   Then $\nu$ satisfies the conditions of 416O
(because every member of $\Cal E$ is closed, while $X$ is compact), so
has an extension to a Radon measure $\mu$.   If $X$ is zero-dimensional,
$\Cal E$ is a base for the topology of $X$ closed under finite unions
and intersections, so $\mu$ is unique, by 415H(iv) or 415H(v).

\medskip

{\bf (b)} The map $a\mapsto\widehat{a}$ is a Boolean isomorphism between
$\frak A$ and the algebra $\Cal E$ of open-and-closed subsets of $Z$, so
we have a one-to-one correspondence between non-negative additive
functionals $\nu$ on $\frak A$ and non-negative additive functionals
$\nuprime$ on $\Cal E$ defined by the formula $\nuprime\widehat{a}=\nu
a$.   Now
$Z$ is compact, Hausdorff and zero-dimensional, so $\nuprime$ has a
unique
extension to a Radon measure on $Z$, by part (a).   And of course every
Radon measure $\mu$ on $Z$ gives us a non-negative additive functional
$\mu\restr\Cal E$ on $\Cal E$, corresponding to a non-negative additive
functional on $\frak A$.
}%end of proof of 416Q

\leader{416R}{Theorem} (a) Any subspace of a Radon measure space is
a quasi-Radon measure space.

(b) A measurable subspace of a Radon measure space is a Radon measure
space.

(c) If $(X,\frak T,\Sigma,\mu)$ is a Hausdorff complete locally
determined
topological measure space, and $Y\subseteq X$ is such that the subspace
measure $\mu_Y$ on $Y$ is a Radon measure, then $Y\in\Sigma$.

\proof{{\bf (a)} Put 416A and 415B together.

\medskip

{\bf (b)} Let $(X,\frak T,\Sigma,\mu)$ be a Radon measure space, and
$(E,\frak T_E,\Sigma_E,\mu_E)$ a member of $\Sigma$ with the induced
topology and measure.   Because $\mu$ is complete and locally
determined,
so is $\mu_E$ (214Ka).   Because $\frak T$ is Hausdorff, so is
$\frak T_E$ (4A2F(a-i)).   Because $\mu$ is locally finite, so is
$\mu_E$.   Because $\mu$ is tight (and a
subset of $E$ is compact for $\frak T_E$ whenever it is compact for
$\frak T$), $\mu_E$ is tight (412Oa).

\medskip

{\bf (c)} \Quer\ If $Y\notin\Sigma$, then there is a set $F\in\Sigma$
such that $\mu_*(Y\cap F)<\mu^*(Y\cap F)$ (413F(v)).   But now
$\mu^*(Y\cap
F)=\mu_Y(Y\cap F)$, so there is a compact set $K\subseteq Y\cap F$ such
that $\mu_YK>\mu_*(Y\cap F)$.   When regarded as a subset of $X$, $K$ is
still compact;  because $\frak T$ is Hausdorff, $K$ is closed, so
belongs to $\Sigma$, and

\Centerline{$\mu_*(Y\cap F)\ge\mu K=\mu_YK>\mu_*(Y\cap F)$,}

\noindent which is absurd.\ \Bang
}%end of proof of 416R

\leader{416S}{}\cmmnt{ Corresponding to 415O, we have the following.

\medskip

\noindent}{\bf Proposition} Let $(X,\frak T,\Sigma,\mu)$ be a Radon
measure space.

(a) If $\nu$ is a locally finite indefinite-integral measure over
$\mu$, it is a Radon measure.

(b) If $\nu$ is a Radon measure on $X$ and $\nu K=0$ whenever 
$K\subseteq X$ is compact and $\mu K=0$, then $\nu$ is an
indefinite-integral measure over $\mu$.

\proof{{\bf (a)} 
Because $\mu$ is complete and locally determined, so is $\nu$
(234Nb\formerly{2{}34F}).   Because $\mu$ is tight, so is $\nu$ (412Q).
So if $\nu$ is also locally finite, it is a Radon measure.

\medskip

{\bf (b)} Write $\Tau$ for the domain of $\nu$.

\medskip

\quad{\bf (i)} If $E\in\Sigma\cap\Tau$ and $\mu E=0$, then 
$\nu K=\mu K=0$ for every compact $K\subseteq E$, so $\nu E=0$.


\medskip

\quad{\bf (ii)}
$\Tau\supseteq\Sigma$.   \Prf\ If $E\in\Sigma$ and $K\subseteq X$ is
compact, there are Borel sets $F$, $F'$ such that
$F\subseteq E\cap K\subseteq F'\subseteq K$ and $\mu(F'\setminus F)=0$.
Consequently $\nu(F'\setminus F)=0$ and
$E\cap K\in\Tau$, because $\nu$ is complete.   By 416Db, $E\in\Tau$;
as $E$ is arbitrary, $\Sigma\subseteq\Tau$.\ \Qed

\medskip

\quad{\bf (iii)} If $E\in\Tau$, there is an $F\in\Sigma$ such that
$E\subseteq F$ and $\nu(F\setminus E)=0$.   \Prf\ By 416Dc and
412Ia, there is a
decomposition $\family{i}{I}{X_i}$ of $X$ for the measure $\mu$ such that
at most one of the $X_i$ is not a compact
$\mu$-self-supporting set, and that exceptional one, if any,
is $\mu$-negligible.   For each $i$, let $F_i$ be such that

\inset{----- if $X_i$ is a compact $\mu$-self-supporting set, 
then $F_i$ is a Borel subset of $X_i$,
$F_i\supseteq E\cap X_i$ and $\nu(F_i\setminus E)=0$,

----- if $X_i$ is not a compact $\mu$-self-supporting set, $F_i=X_i$.}

\noindent Then $F_i\in\Sigma$ for every $i$ so $F=\bigcup_{i\in I}F_i$
belongs to $\Sigma$.   We also have $\nu(F_i\setminus E)=0$
for every $i$, because if
$X_i$ is not compact and $\mu$-self-supporting then $\nu X_i=\mu X_i=0$.
Of course $E\subseteq F$.   If $K\subseteq X$ is
compact, there is an open set $G\supseteq X$ such that $\mu G<\infty$; 
consequently
$\{i:\mu(X_i\cap G)\ge\epsilon\}$ is finite for every $\epsilon>0$, so
$\{i:X_i\cap K\ne\emptyset\}$ is countable 
and $\nu(K\cap F\setminus E)=0$.   By 412Jb, $\nu(F\setminus E)=0$.\
\Qed

Applying the same argument to $X\setminus E$, we can get an $F'\in\Sigma$
such that $F'\subseteq E$ and $\nu(E\setminus F')=0$.   As $E$ is
arbitrary, $\nu$ is the completion of its restriction to $\Sigma$.

\medskip

\quad{\bf (iv)} Now look at the conditions of 234O.   We know that
$\mu$ is localizable and $\nu$ is semi-finite.   We saw in (ii) above that
$\Tau\supseteq\Sigma$ and in (i) that $\nu$ is zero on $\mu$-negligible
sets.   In (iii) we saw that $\nu$ is the completion
of $\nu\restr\Sigma$.   And if $\nu E>0$ there is a compact $K\subseteq E$
such that $\nu K>0$, while $\mu K<\infty$.   So 234O tells us that
$\nu$ is an indefinite-integral measure over $\mu$.
}%end of proof of 416S

\leader{416T}{}\cmmnt{ I said in the notes to \S415 that the most
important quasi-Radon measure spaces are subspaces of Radon measure
spaces.   I do not know of a useful necessary and sufficient condition,
but the following deals with completely regular spaces.

\medskip

\noindent}{\bf Proposition} Let $(X,\frak T,\Sigma,\mu)$ be a locally
finite completely regular Hausdorff quasi-Radon measure space.   Then it
is isomorphic, as topological measure space, to a subspace of a locally
compact Radon measure space.

\proof{{\bf (a)}  Write $\beta X$ for the Stone-\v{C}ech
compactification
of $X$ (4A2I);  I will take it that $X$ is actually a subspace of
$\beta X$.   Let $\Cal U$ be the set of those open subsets $U$ of
$\beta X$ such
that $\mu(U\cap X)<\infty$;  then $\Cal U$ is upwards-directed and
covers $X$, because $\mu$ is locally finite.   Set
$W=\bigcup\Cal U\supseteq X$.
Then $W$ is an open subset of $\beta X$, so is locally compact.

\medskip

{\bf (b)} Let $\Cal B(W)$ be the Borel $\sigma$-algebra of $W$.   Then
$V\cap X$ is
a Borel subset of $X$ for every $V\in\Cal B(W)$ (4A3Ca), so we have a
measure $\nu:\Cal B(W)\to[0,\infty]$ defined by setting
$\nu V=\mu(X\cap V)$ for every
$V\in\Cal B(W)$.   Now $\nu$ satisfies the conditions of 415Cb.   \Prf\
($\alpha$) If $\nu V>0$, then, because $\mu$ is effectively locally
finite, there is an open set $G\subseteq X$ such that $\mu(G\cap V)>0$
and $\mu G<\infty$.   There is an open set $U\subseteq\beta X$ such that
$U\cap X=G$, in which case $U\subseteq W$, $\nu U<\infty$ and
$\nu(U\cap V)>0$.   Thus $\nu$ is effectively locally finite.
($\beta$) If $\Cal U$ is an upwards-directed family of open subsets of
$W$, then $\{U\cap X:U\in\Cal U\}$ is an upwards-directed family of open
subsets of $X$, so

$$\eqalign{\nu(\bigcup\Cal U)
&=\mu(X\cap\bigcup\Cal U)
=\mu(\bigcup\{U\cap X:U\in\Cal U\})\cr
&=\sup_{U\in\Cal U}\mu(X\cap U)
=\sup_{U\in\Cal U}\nu U.\text{ \Qed}\cr}$$

\noindent So the c.l.d.\ version $\tilde\nu$ of $\nu$ is a quasi-Radon
measure on $W$ (415Cb).

\medskip

{\bf (c)} The construction of $W$ ensures that $\nu$ and $\tilde\nu$ are
locally finite.   By 416G, $\tilde\nu$ is a Radon measure.   So the
subspace measure $\tilde\nu_X$ is a quasi-Radon measure on $X$ (416Ra).
But $\tilde\nu_XG=\mu G$ for every open set $G\subseteq X$.
\Prf\ Note first that as $\nu$ effectively locally finite, therefore
semi-finite, $\tilde\nu$ extends $\nu$ (213Hc).   If $K\subseteq W$ is a
compact set not meeting $X$, then

\Centerline{$\tilde\nu K=\nu K=\mu(K\cap X)=0$;}

\noindent accordingly $\tilde\nu_*(W\setminus X)=0$, by 413Ee.
Now there is an open set $U\subseteq W$ such that $G=X\cap U$, and

$$\eqalignno{\tilde\nu_XG
=\tilde\nu^*G
&\le\tilde\nu U
=\nu U
=\mu(U\cap X)
=\mu G\cr
&=\tilde\nu^*(U\cap X)+\tilde\nu_*(U\setminus X)\cr
\noalign{\noindent (by 413E(c-ii), because $\tilde\nu$ is semi-finite)}
&\le\tilde\nu^*(U\cap X)+\tilde\nu_*(W\setminus X)
=\tilde\nu^*G.\text{  \Qed}\cr}$$

\noindent So 415H(iii) tells us that $\mu=\tilde\nu_X$ is the subspace
measure induced by $\nu$.
}%end of proof of 416T

\leader{416U}{Theorem} (a) If $\familyiI{(X_i,\frak T_i,\Sigma_i,\mu_i)}$
is a family of compact metrizable Radon probability spaces such that
every $\mu_i$ is strictly positive, the product measure on
$X=\prod_{i\in I}X_i$ is a completion regular Radon measure.

(b) In particular, the usual measures on $\{0,1\}^I$ and $[0,1]^I$
and $\Cal PI$ are completion regular Radon measures, for any set $I$.

\proof{{\bf (a)} By 415E,
it is a completion regular quasi-Radon probability
measure;  but $X$ is a compact Hausdorff space, so it is a Radon
measure, by 416G or otherwise.

\medskip

{\bf (b)} follows at once.
(I suppose it is obvious that by the `usual measure
on $[0,1]^I$' I mean the product measure when each copy of $[0,1]$ is
given Lebesgue measure.   Recall also that the `usual measure on
$\Cal PI$' is just the copy of the usual measure on $\{0,1\}^I$ induced
by the standard bijection $A\leftrightarrow\chi A$ (254Jb), which is a
homeomorphism (4A2Ud).
}%end of proof of 416U

\leader{416V}{Stone \dvrocolon{spaces}}\cmmnt{ The results of
415Q-415R become simpler and more striking in the present context.

\medskip

\noindent}{\bf Theorem} Let $(X,\frak T,\Sigma,\mu)$ be a Radon measure
space, and $(Z,\frak S,\Tau,\nu)$ the Stone space of its measure algebra
$(\frak A,\bar\mu)$.   For $E\in\Sigma$ let $E^*$ be the open-and-closed
set in $Z$ corresponding to the image $E^{\ssbullet}$ of $E$ in
$\frak A$.   Define $R\subseteq Z\times X$ by saying that $(z,x)\in R$
iff $x\in F$ whenever $F\subseteq X$ is closed and $z\in F^*$.

(a) $R$ is the graph of a function $f:Q\to X$, where $Q=R^{-1}[X]$.   If
we set $W=\bigcup\{K^*:K\subseteq X$ is compact$\}$, then $W\subseteq Q$
is a $\nu$-conegligible open set, and the subspace measure $\nu_W$ on
$W$ is a Radon measure.

(b) Setting $g=f\restrp W$, $g$ is continuous and
$\mu$ is the image measure $\nu_Wg^{-1}$.

(c) If $X$ is compact, $W=Q=Z$ and $\mu=\nu g^{-1}$.

\proof{{\bf (a)} By 415Ra, $R$ is the graph of a function.   If
$K\subseteq X$ is compact and $z\in K^*$, then
$\Cal F=\{F:F\subseteq X$ is closed, $z\in F^*\}$ is a family of
non-empty closed subsets of $X$, closed
under finite intersections, and containing the compact set $K$;  so it
has non-empty intersection, and there is an $x\in K$ such that
$(z,x)\in R$, that is, $z\in Q$ and $f(z)\in K$.   Thus $W\subseteq Q$.
Of course $W$ is an open set, being the union of a family of
open-and-closed sets;  but it is also conegligible, because
$\sup\{K^{\ssbullet}:K\subseteq X$ is compact$\}=1$ in $\frak A$ (412N),
so $Z\setminus W$ must be nowhere
dense, therefore negligible.   Now the subspace measure $\nu_W$ is
quasi-Radon because $\nu$ is (411P(d-iv), 415B);  but $W$ is a union of
compact open sets of
finite measure, so $\nu_W$ is locally finite and $W$ is locally compact;
by 416G, $\nu_W$ is a Radon measure.

\medskip

{\bf (b)} $g$ is continuous.   \Prf\ Let $G\subseteq X$ be an open set
and $z\in g^{-1}[G]$.   Let $K\subseteq X$ be a compact set such that
$z\in K^*$.   As remarked above, $g(z)=f(z)$ belongs to $K$.   $K$,
being a compact Hausdorff space, is regular (3A3Bb), so there is an open
set $H$ containing $g(z)$ such that $L=\overline{H\cap K}\subseteq G$.
Note that $L$ is compact, so $L^*\subseteq W$.   Now $g(z)$ does not
belong to the
closed set $X\setminus H$, so $z\notin(X\setminus H)^*$ and $z\in H^*$;
accordingly $z\in(H\cap K)^*\subseteq L^*$.   If $w\in L^*$,
$g(w)\in L\subseteq G$;  so $L^*\subseteq g^{-1}[G]$, and
$z\in\interior g^{-1}[G]$.   As $z$ is arbitrary, $g^{-1}[G]$ is open;
as $G$ is arbitrary, $g$ is continuous.\ \Qed

By 415Rb, we know that $\mu=\nu_Qf^{-1}$, where $\nu_Q$ is the
subspace measure on $Q$.   But as $\nu$ is complete and both $Q$ and $W$
are conegligible,  we have

\Centerline{$\nu_Qf^{-1}[A]=\nu f^{-1}[A]=\nu g^{-1}[A]=\nu_Wg^{-1}[A]$}

\noindent whenever $A\subseteq X$ and any of the four terms is defined,
so that $\mu=\nu_Qf^{-1}=\nu_Wg^{-1}$.

\medskip

{\bf (c)} If $X$ is compact, then $Z=X^*\subseteq W$, so $W=Q=Z$ and
$\nu g^{-1}=\nu_Wg^{-1}=\mu$.
}%end of proof of 416V

\leader{416W}{Compact measure \dvrocolon{spaces}}\cmmnt{ Recall that a
semi-finite measure space $(X,\Sigma,\mu)$ is `compact' (as a measure
space) if there is a family $\Cal K\subseteq\Sigma$ such that $\mu$ is
inner regular with respect to $\Cal K$ and $\bigcap\Cal K'\ne\emptyset$
whenever $\Cal K'\subseteq\Cal K$ has the finite intersection property
(342A);  while $(X,\Sigma,\mu)$ is `perfect' if whenever $f:X\to\Bbb R$
is measurable and $\mu E>0$, there is a compact set $K\subseteq f[E]$
such that $\mu f^{-1}[K]>0$ (342K).   In \S342 I introduced these
concepts in order to study the realization of homomorphisms between
measure algebras.   The following result is now very easy.

\medskip

\noindent}{\bf Proposition} (a) Any Radon measure space is a compact
measure space, therefore perfect.

(b) Let $(X,\frak T,\Sigma,\mu)$ be a Radon measure
space, with measure algebra $(\frak A,\bar\mu)$, and $(Y,\Tau,\nu)$ a
complete strictly localizable measure space, with measure algebra
$(\frak B,\bar\nu)$.   If $\pi:\frak A\to\frak B$ is an order-continuous
Boolean homomorphism, there is a function $f:Y\to X$ such that
$f^{-1}[E]\in\Tau$
and $f^{-1}[E]^{\ssbullet}=\pi E^{\ssbullet}$ for every $E\in\Sigma$.
If $\pi$ is measure-preserving, $f$ is \imp.

\proof{{\bf (a)} If ($X,\frak T,\Sigma,\mu)$ is a Radon measure space,
$\mu$ is inner regular with respect to the compact class consisting of
the compact subsets of $X$, so $(X,\Sigma,\mu)$ is a compact measure
space.   By 342L, it is perfect.

\medskip

{\bf (b)} Use (i)$\Rightarrow$(v) of Theorem 343B.   (Of course $f$ is
\imp\ iff $\pi$ is measure-{\vthsp}preserving.)
}%end of proof of 416W

\exercises{
\leader{416X}{Basic exercises $\pmb{>}$(a)}
%\spheader 416Xa
Let $(X,\frak T,\Sigma,\mu)$ be a Radon measure space, and $E\in\Sigma$
an atom for the measure.   Show that there is a point $x\in E$ such that
$\mu\{x\}=\mu E$.
%414G

\spheader 416Xb
Let $X$ be a topological space and $\mu$ a point-supported measure on
$X$, as described in 112Bd.   (i) Show that $\mu$ is tight, so is a
Radon measure iff
it is locally finite.   In particular, show that if $X$ has its discrete
topology then counting measure on $X$ is a Radon measure.   (ii) Show
that every purely atomic Radon measure is of this type.
%/

\sqheader 416Xc Let
$\langle(X_i,\frak T_i,\Sigma_i,\mu_i)\rangle_{i\in I}$ be a family of
Radon measure spaces, with direct sum $(X,\Sigma,\mu)$
(214L).   Give $X$ its disjoint union topology.   Show that $\mu$ is a
Radon measure.
%/

\spheader 416Xd Let $(X,\frak T,\Sigma,\mu)$ be a Radon measure space.
Show that $\alpha\mu$, defined on $\Sigma$, is a
Radon measure for any $\alpha>0$.
%416D

\spheader 416Xe Let $(X,\frak T,\Sigma,\mu)$ be a $\sigma$-finite
Radon measure space with $\mu X>0$.   Show that there is a
Radon probability measure on $X$ with the same measurable sets and the
same negligible sets as $\mu$.
%416A

\spheader 416Xf Let $(X,\frak T,\Sigma,\mu)$ be a Radon measure space.
(i) Show that $\mu$ has a decomposition $\langle X_i\rangle_{i\in I}$ in
which every $X_i$ except at most one is a self-supporting compact set,
and the exceptional one, if any, is negligible.   (ii) Show that $\mu$
has a decomposition $\familyiI{X_i}$ in which every $X_i$ is expressible
as the intersection of a closed set with an open set.   \Hint{enumerate
the open sets of finite measure as
$\langle G_{\xi}\rangle_{\xi<\kappa}$, and set
$X_{\xi}=G_{\xi}\setminus\bigcup_{\eta<\xi}G_{\eta}$.}
%416B

\spheader 416Xg Let $X$ be a Hausdorff space and $\mu$, $\nu$ two Radon
measures on $X$ such that $\nu G=\mu G$ whenever $G\subseteq X$ is open
and $\min(\mu G,\nu G)<\infty$.   Show that $\mu=\nu$.
%416E

\spheader 416Xh Explain how to prove 416H from 416F, without appealing
to \S415.
%416H

\spheader 416Xi Give a direct proof of 416I not relying on 415O.
%416I

\spheader 416Xj Let $(X,\frak T)$ be a completely regular Hausdorff
space and $\mu$ a locally finite topological measure on $X$ which is
inner regular with respect to the closed sets.   Show that

$$\eqalign{\mu K
&=\inf\{\mu G:G\supseteq K\text{ is a cozero set}\}\cr
&=\inf\{\mu F:F\supseteq K\text{ is a zero set}\}
=\inf\{\int fd\mu:\chi K\le f\in C(X)\}\cr}$$

\noindent for every compact set $K\subseteq X$.
%416I

\spheader 416Xk Let $(X,\frak T,\Sigma,\mu)$ be a locally compact Radon
measure space, and $C_k$ the space of continuous real-valued functions
on $X$ with compact supports.   Show that $\{f^{\ssbullet}:f\in C_k\}$
is dense in $L^0(\mu)$ for the topology of convergence in measure.
%416I

\spheader 416Xl Let $X$ be a completely regular Hausdorff space and
$\nu$ a locally finite Baire measure on $X$.   (i) Show that
$\nu^*K=\inf\{\nu G:G\subseteq X$ is a cozero set, $K\subseteq G\}$ for
every compact set $K\subseteq X$.   (ii) Show that there
is a Radon measure $\mu$ on $X$ such that $\mu K=\nu^*K$ for every
compact set $K\subseteq X$.   \Hint{in the language of the proof of
416L, $\phi_1=\nu^*\restr\Cal K$.}
%416L

\spheader 416Xm Let $X$ be a Hausdorff space and $\nu$ a non-negative
finitely additive functional defined on some algebra of subsets of $X$.
Show that there is a Radon measure $\mu$ on $X$ such that
$\mu X\le\nu X$ and $\mu K\ge\nu E$ whenever $E\in\Tau$, $K\subseteq X$
is compact
and $E\subseteq K$.   \Hint{start by extending $\nu$ to $\Cal PX$.}
%416K

\spheader 416Xn Let $(X,\frak T)$ be a Hausdorff space,
$\Sigma\supseteq\frak T$ a $\sigma$-algebra of subsets of $X$, and
$\nu:\Sigma\to\coint{0,\infty}$ a finitely additive functional such that
$\nu E=\sup\{\nu K:K\subseteq E$ is compact$\}$ for every $E\in\Sigma$.
Show that $\nu$ is countably additive and that its completion is a Radon
measure on $X$.
%416K

\spheader 416Xo Explain how to prove 416Rb from 416C and 415B.
%416R

\spheader 416Xp Let $X$ be a Hausdorff space, $\mu$ a complete locally
finite measure on $X$, and $Y$ a conegligible subset of $X$.   Show that
$\mu$ is a Radon measure iff the subspace measure on $Y$ is a Radon
measure.
%416R

\spheader 416Xq Let $X$ be a Hausdorff space, $Y$ a subset of $X$, and
$\nu$ a Radon measure on $Y$.   Define a measure $\mu$ on $X$ by setting
$\mu E=\nu(E\cap Y)$ whenever $\nu$ measures $E\cap Y$.   Show that if
{\it either} $Y$ is closed {\it or} $\nu$ is totally finite, $\mu$ is a
Radon measure on $X$.   (Cf.\ 418I.)
%416R

\spheader 416Xr Let $(X,\frak T,\Sigma,\mu)$ be a Radon measure space
and $\Cal E\subseteq\Sigma$ a non-empty upwards-directed family.   Set
$\nu F=\sup_{E\in\Cal E}\mu(E\cap F)$ whenever $F\subseteq X$ is such
that $\mu$ measures $E\cap F$ for every $E\in\Cal E$.   Show that $\nu$
is a Radon measure on $X$.
%416R

\spheader 416Xs Let $\sequencen{X_n}$ be a sequence of Hausdorff spaces
with product $X$;  write $\Cal B(X_n)$ for the Borel $\sigma$-algebra of
$X_n$.
Let $\Tau$ be the $\sigma$-algebra $\Tensorhat_{n\in\Bbb N}\Cal B(X_n)$
(definition:  254E).   Let $\nu:\Tau\to\coint{0,\infty}$ be a
finitely additive functional such that
$E\mapsto\nu\pi_n^{-1}[E]:\Cal B(X_n)\to\coint{0,\infty}$ is countably
additive and tight for each $n\in\Bbb N$, writing
$\pi_n(x)=x(n)$ for $x\in X$, $n\in\Bbb N$.   Show that there is
a unique Radon measure on $X$ extending $\nu$.
%416R

\spheader 416Xt Set $S_2=\bigcup_{n\in\Bbb N}\{0,1\}^n$, and let
$\phi:S_2^*\to\coint{0,\infty}$ be a functional such that
$\phi(\sigma)=\phi(\sigma^{\smallfrown}\fraction{0})
 +\phi(\sigma^{\smallfrown}\fraction{1})$
for every $\sigma\in S_2$, writing $\sigma^{\smallfrown}\fraction{0}$ and
$\sigma^{\smallfrown}\fraction{1}$ for the two members of $\{0,1\}^{n+1}$ extending
any $\sigma\in\{0,1\}^n$.   Show that there is a unique Radon measure
$\mu$ on $\{0,1\}^{\Bbb N}$ such that
$\mu\{x:x\restr\{0,\ldots,n-1\}=\sigma\}
=\phi(\sigma)$ whenever $n\in\Bbb N$, $\sigma\in\{0,1\}^{\Bbb N}$.
\Hint{use 416Xs or 416Q.}
%416Xs, 416O

\spheader 416Xu Let $(X,\frak T,\Sigma,\mu)$ be a Radon measure space.
Show that a measure $\nu$
on $X$ is an indefinite-integral measure over $\mu$ iff ($\alpha$) $\nu$
is a complete, locally determined topological measure ($\beta$) $\nu$ is
tight ($\gamma$) $\nu K=0$
whenever $K\subseteq X$ is compact and $\mu K=0$.
%416S

\spheader 416Xv Let $(X,\frak T,\Sigma,\mu)$ be a compact Hausdorff
quasi-Radon measure space.   Let $W\subseteq X$ be the union of the open
subsets of $X$ of finite measure.   Show that the subspace measure on
$W$ is a Radon measure.
%416T

\spheader 416Xw Let $(X,\frak T,\Sigma,\mu)$ be a completely regular
Radon measure space.   Show that it
is isomorphic, as topological measure space,
to a measurable subspace of a locally compact Radon measure space.
%416T

\spheader 416Xx Let $(X,\frak T,\Sigma,\mu)$ be a compact Radon measure
space and $(Z,\frak S,\Tau,\nu)$ the Stone space of its measure algebra.
For $E\in\Sigma$ let $E^*$ be the corresponding open-and-closed subset
of $Z$, as in 416V.   Show that the function described in 416V is the
unique continuous function $h:Z\to X$ such that
$\nu(E^*\symmdiff h^{-1}[E])=0$ for every $E\in\Sigma$.
%416V

\spheader 416Xy Show that the right-facing Sorgenfrey line (415Xc), with
Lebesgue measure, is a
quasi-Radon measure space which, regarded as a measure space, is
compact, but, regarded as a topological measure space, is not a Radon
measure space.
%416W

\spheader 416Xz Let $(\frak A,\bar\mu)$ be a totally finite measure
algebra and $(Z,\frak T,\Sigma,\mu)$ its Stone space.
Show that if $\nu$ is a strictly positive Radon measure on $Z$ then $\mu$
is an indefinite-integral measure over $\nu$.
%416W 416Xu

\leader{416Y}{Further exercises (a)}
%\spheader 416Ya
Let $X$ be a Hausdorff space and $\nu$ a countably additive real-valued
functional defined on a $\sigma$-algebra $\Sigma$ of subsets of $X$.
Show that the following are equiveridical:  (i)
$|\nu|:\Sigma\to\coint{0,\infty}$,
defined as in 362B, is a Radon measure on $X$;  (ii) $\nu$ is
expressible as $\mu_1-\mu_2$, where $\mu_1$, $\mu_2$ are Radon measures
on $X$ and $\Sigma=\dom\mu_1\cap\dom\mu_2$.
%/

\spheader 416Yb Let $X$ be a topological space and $\mu_0$ a semi-finite
measure on $X$ which is inner regular with respect to the family
$\Cal K_{\text{ccc}}$ of closed countably compact sets.   Show that $\mu_0$
has an extension to a complete locally determined topological measure
$\mu$ on $X$, still inner regular with respect to $\Cal K_{\text{ccc}}$;
and that the extension may be done in such a way that whenever
$\mu E<\infty$ there
is an $E_0\in\Sigma_0$ such that $\mu(E\symmdiff E_0)=0$.   \Hint{use
the argument of 416N, but with
$\Cal K=\{K:K\in\Cal K_{\text{ccc}},\,\mu_0^*K<\infty\}$.}
%416N

\spheader 416Yc Let $X$ be a topological space and $\mu_0$ a semi-finite
measure on $X$ which is inner regular with respect to the family
$\Cal K_{\text{sc}}$ of sequentially compact sets.   Show that $\mu_0$
has an extension to a complete locally determined topological measure
$\mu$ on $X$, still inner regular with respect to $\Cal K_{\text{sc}}$;
and that the extension may be done in such a way that whenever
$\mu E<\infty$
there is an $E_0\in\Sigma_0$ such that $\mu(E\symmdiff E_0)=0$.
%416Yb, 416N

\spheader 416Yd Let $X\subseteq\beta\Bbb N$ be the union of all those
open sets $G\subseteq\beta\Bbb N$ such that
$\sum_{n\in G\cap\Bbb N}\bover1{n+1}$ is finite.   For $E\subseteq X$
set $\mu E=\sum_{n\in E\cap\Bbb N}\bover1{n+1}$.   Show that $\mu$ is a
$\sigma$-finite Radon
measure on the locally compact Hausdorff space $X$.   Show that $\mu$ is
not outer regular with respect to the open sets.

\spheader 416Ye Set $S=\bigcup_{n\in\Bbb N}\BbbN^n$, and let
$\phi:S\to\coint{0,\infty}$ be a functional such that
$\phi(\sigma)=\sum_{i=0}^{\infty}\phi(\sigma^{\smallfrown}\fraction{i})$ for every
$\sigma\in S$, writing $\sigma^{\smallfrown}\fraction{i}$ for the
members of
$\BbbN^{n+1}$ extending any $\sigma\in\BbbN^n$.   Show that there is a
unique Radon measure $\mu$ on $\BbbN^{\Bbb N}$ such that
$\mu I_{\sigma}=\phi(\sigma)$ for every $\sigma\in S$, where
$I_{\sigma}=\{x:x\restr\{0,\ldots,n-1\}=\sigma\}$ for any $n\in\Bbb N$,
$\sigma\in\BbbN^{\Bbb N}$.  \Hint{set
$\theta A=\inf\{\sum_{\sigma\in R}\phi(\sigma):R\subseteq S$,
$A\subseteq\bigcup_{\sigma\in R}I_{\sigma}\}$ for every
$A\subseteq\BbbN^{\Bbb N}$, and use \Caratheodory's method.}
%416Xt, 416Xs, 416Q

\spheader 416Yf Let $(X,\frak T,\Sigma,\mu)$ be a Radon measure space.
Show that a measure $\nu$ on $X$ is an indefinite-integral measure over
$\mu$ iff (i) there is a topology $\frak S$ on $X$, including $\frak T$,
such that $\nu$ is a Radon measure with respect to $\frak S$
(ii) $\nu K=0$ whenever $K$ is a $\frak T$-compact set and $\mu K=0$.
%416S, 416Xu

\spheader 416Yg Let $\sequencen{x_n}$ enumerate a dense subset of
$X=\{0,1\}^{\frakc}$ (4A2B(e-ii)).   Let $\nu_{\frakc}$ be the usual
measure on $X$, and set
$\mu E=\bover12\nu_{\frakc} E+\sum_{n=0}^{\infty}2^{-n-2}\chi E(x_n)$ for
$E\in\dom\nu_{\frakc}$.   (i) Show that $\mu$ is a strictly positive Radon
probability measure on $X$ with Maharam type $\frak c$.   (ii) Let
$I\in[\frak c]^{\le\omega}$ be such that $x_m\restr I\ne x_n\restr I$
whenever $m\ne n$.   Set $Z=\{0,1\}^I$ and let $\pi:X\to Z$ be the
canonical map.   Show that if $f\in C(X)$ is such that
$\int f\times g\pi\,d\mu=0$ for every $g\in C(Z)$, then $f=0$.
\Hint{otherwise, take $n\in\Bbb N$ such that
$|f(x_n)|\ge\bover12\|f\|_{\infty}$, and let $g\ge 0$ be such that
$g(\pi x_n)=1$ and
$\int g\,d(\mu\pi^{-1})<3\cdot 2^{-n-3}$;  show that
$\int f\times g\pi\,d\mu>0$.}   (iii) Show that there is no orthonormal
basis for $L^2(\mu)$ in $\{f^{\ssbullet}:f\in C(X)\}$.
(See {\smc Hart \& Kunen 99}.)
%416I 416U

\spheader 416Yh\dvAnew{2010}
Let $(X,\frak T,\Sigma,\mu)$ be a Radon measure space and
$\Cal A\subseteq\Sigma$ a countable set.   Let $\frak S$ be the topology
generated by $\frak T\cup\Cal A$.   Show that $\mu$ is $\frak S$-Radon.
%out of order query

\spheader 416Yi\dvAnew{2013}
In 254Yh\Latereditions, show that if we start from a continuous
\imp\ $f:[0,1]\to[0,1]^2$, as in 134Yl, we get a continuous \imp\
surjection $g:[0,1]\to[0,1]^{\Bbb N}$.
%416U
}%end of exercises

\endnotes{\Notesheader{416} The original measures studied by Radon
({\smc Radon 1913}) were, in effect, what I call differences of Radon
measures on $\BbbR^r$, as introduced in \S256.
Successive generalizations moved first to Radon measures on general
compact Hausdorff spaces, then to locally compact Hausdorff spaces, and
finally to
arbitrary Hausdorff spaces, as presented in this section.   I ought
perhaps to remark that, following {\smc Bourbaki 65}, many authors use
the term `Radon measure' to describe a linear
functional on a space of continuous functions;  I will discuss the
relationship between such functionals and the measures of this chapter
in \S436.   For the moment, observe that by 415I a Radon measure on
a completely regular space can be determined from the integrals it
assigns to continuous functions.   It is also common for the phrase
`Radon measure'
to be used for what I would call a tight Borel measure;
you have to check each author to see whether local
finiteness is also assumed.   In my usage, a Radon measure is
necessarily
the c.l.d.\ version of a Borel measure.   The Borel measures which
correspond to Radon measures are described in 416F.

In \S256, I discussed Radon measures on $\BbbR^r$ as a preparation for
a discussion of convolutions of measures.   It should now be becoming
clear why I felt that it was impossible, in that context, to give you a
proper idea of what a Radon measure, in the modern form, `really' is.
In Euclidean space, too many concepts coincide.   As a trivial example,
the simplest definition of `local finiteness' (256Ab) is not the
right formulation in other spaces (411Fa).
Next, because every closed set is a
countable union of compact sets, there is no distinction between `inner
regular with
respect to closed sets' and `inner regular with respect to compact
sets', so one cannot get any intuition for which is important in which
arguments.
(When we come to subspace measures on non-measurable subsets, of course,
this changes;  quasi-Radon measures on subsets of Euclidean space are
important and interesting.)   Third, the fact that the c.l.d.\ product
of two Radon measures on Euclidean space is already a Radon measure
(256K) leaves us with no idea of what to do with a general product of
Radon measures.   (There are real difficulties at this point, which I
will attack
in the next section.   For the moment I offer just 416U.)   And
finally, we simply cannot represent a product of uncountably many Radon
probability measures on Euclidean spaces as a measure on Euclidean
space.

As you would expect, a very large proportion of the results of this
chapter, and many theorems from earlier volumes, were originally proved
for compact Radon measure spaces.   The theory of general totally finite
Radon measures is, in effect, the theory of measurable subspaces of
compact Radon
measure spaces, while the theory of quasi-Radon measures is pretty much
the theory of non-measurable subspaces of Radon measure spaces.   Thus
the theorem that
every quasi-Radon measure space is strictly localizable is almost a
consequence of the facts that every Radon measure space is strictly
localizable and any subspace of a strictly localizable space is strictly
localizable.

The cluster of results between 416J and 416Q form only a sample, I hope
a reasonably representative sample, of the many theorems on construction
of Radon measures from functionals on algebras or lattices of sets.
(See also 416Ye.)   The essential simplification,
compared with the theorems in \S413 and \S415, is that we do not need to
mention any $\sigma$- or
$\tau$-additivity condition of the type 413I($\beta$) or 415K($\beta$),
because we are dealing with a `compact class', the family of compact
subsets of a Hausdorff space.   We can use this even at some distance,
as in 416O (where the hypotheses do not require any non-empty compact
set to belong to the domain of the original functional).   The
particular feature of 416O which makes it difficult to prove from such
results as 413J and 413O above is that we have to retain control of the
outer measures of a sequence $\sequencen{K_n}$ of non-measurable sets.
In general this is hard to do, and is possible here principally because
the sequence is non-decreasing, so that we can make sense of the
functionals $\nu_nE=\nu^*(E\cap K_{n+1})-\nu^*(E\cap K_n)$;  compare 214P.
}%end of notes

\discrpage


