\frfilename{mt464.tex}
\versiondate{25.5.13}
\copyrightdate{1999}

\def\chaptername{Pointwise compact sets of measurable functions}
\def\sectionname{Talagrand's measure}

\newsection{464}

An obvious question arising from 463I and its corollaries is, do we
really need the
hypothesis that the measure involved is perfect?   A very remarkable
construction by M.Talagrand (464D) shows that these results are
certainly not true of all probability spaces (464E).   Investigating the
properties of this measure we are led to some surprising facts about
additive functionals on algebras $\Cal PI$ and the duals of
$\ell^{\infty}$ spaces (464M, 464R).

\leader{464A}{The usual measure on $\Cal PI$} Recall\cmmnt{ from
254J and 416U} that for any set $I$ we have a standard measure $\nu$, a
Radon measure for the usual topology on $\Cal PI$\cmmnt{, defined by
saying that $\nu\{a:a\subseteq I,\,a\cap J=c\}=2^{-\#(J)}$ whenever
$J\subseteq I$ is a finite set and $c\subseteq J$, or by copying from
the usual product measure on $\{0,1\}^I$ by means of the bijection
$a\mapsto\chi a:\Cal PI\to\{0,1\}^I$}.   \cmmnt{We shall need a couple
of simple facts about these measures.}

\spheader 464Aa If $\family{j}{J}{I_j}$ is any partition of $I$, then
$\nu$ can be identified with the product of the family
$\family{j}{J}{\nu_j}$,
where $\nu_j$ is the usual measure on $\Cal PI_j$\cmmnt{, and we
identify $\Cal PI$ with $\prod_{j\in J}\Cal PI_j$ by matching
$a\subseteq I$ with
$\family{j}{J}{a\cap I_j}$;  this is the `associative law' 254N}.   It
follows that if we have any family $\family{j}{J}{A_j}$ of subsets of
$\Cal PI$, and if for each $j$ the set $A_j$ is `determined by
coordinates in $I_j$' in the sense that, for $a\subseteq I$, $a\in A_j$
iff $a\cap I_j\in A_j$, then
$\nu^*(\bigcap_{j\in J}A_j)=\prod_{j\in J}\nu^*A_j$\cmmnt{ (use
254Lb)}.

\spheader 464Ab Similarly, if $f_1$, $f_2$ are non-negative real-valued
functions on $\Cal PI$, and if there are disjoint sets $I_1$,
$I_2\subseteq I$ such that $f_j(a)=f_j(a\cap I_j)$ for every
$a\subseteq I$ and both $j$, then the upper integral
$\overline{\int}f_1+f_2\,d\nu$ is
$\overline{\int}f_1d\nu+\overline{\int}f_2d\nu$.   \prooflet{\Prf\ We
may suppose that $I_2=I\setminus I_1$.   For
each $j$, define $g_j:\Cal PI_j\to\coint{0,\infty}$ by setting
$g_j=f_j\restrp\Cal PI_j$, so that $f_j(a)=g_j(a\cap I_j)$ for every
$a\subseteq I$.   Let $\nu_j$ be the usual measure on $\Cal PI_j$, so
that we can identify $\nu$ with the product measure $\nu_1\times\nu_2$,
if we identify $\Cal PI$ with $\Cal PI_1\times\Cal PI_2$;  that is, we
think of a subset of $I$ as a pair $(a_1,a_2)$ where $a_j\subseteq I_j$
for both $j$.

Now we have

$$\eqalignno{\overline{\int}f_1+f_2\,d\nu
&=\overline{\int}g_1\,d\nu_1
   +\overline{\int}g_2\,d\nu_2\cr
\noalign{\noindent (253K)}
&=\overline{\int}g_1\,d\nu_1\cdot\overline{\int}\chi(\Cal PI_2)d\nu_2
   +\overline{\int}\chi(\Cal PI_1)d\nu_1\cdot\overline{\int}g_2\,d\nu_2\cr
&=\overline{\int}f_1\,d\nu+\overline{\int}f_2\,d\nu\cr}$$

\noindent by 253J, because we can think of $f_1(a_1,a_2)$ as
$g_1(a_1)\cdot(\chi\Cal PI_2)(a_2)$ for all $a_1$, $a_2$.\ \Qed}

\spheader 464Ac If $A\subseteq\Cal PI$ is such that $b\in A$ whenever
$a\in A$, $b\subseteq I$ and $a\symmdiff b$ is finite, then $\nu^*A$
must be either $0$ or $1$\cmmnt{;  this is the zero-one law 254Sa,
applied to the set
$\{\chi a:a\in A\}\subseteq\{0,1\}^I$ and the usual measure on
$\{0,1\}^I$}.

\vleader{72pt}{464B}{Lemma} Let $I$ be any set, and $\nu$ the usual measure on
$\Cal PI$.

(a)(i) There is a sequence $\sequencen{m(n)}$ in $\Bbb N$ such that
$\prod_{n=0}^{\infty}1-2^{-m(n)}=\bover12$.

\quad(ii) Given such a sequence, write $X$ for
$\prod_{n\in\Bbb N}(\Cal PI)^{m(n)}$, and let $\lambda$ be the product
measure on $X$.   We have a function
$\phi:X\to\Cal PI$ defined by setting

\Centerline{$\phi(\sequencen{\langle a_{ni}\rangle_{i<m(n)}})
=\bigcup_{n\in\Bbb N}\bigcap_{i<m(n)}a_{ni}$}

\noindent whenever $\sequencen{\langle a_{ni}\rangle_{i<m(n)}}\in X$.
Now $\phi$ is \imp\ for $\lambda$ and $\nu$.

(b) The map

\Centerline{$(a,b,c)\mapsto(a\cap b)\cup(a\cap c)\cup(b\cap c):
(\Cal PI)^3\to\Cal PI$}

\noindent is \imp\ for the product measure on $(\Cal PI)^3$.

\proof{{\bf (a)(i)} Choose $m(n)$ inductively so that, for each $n$ in
turn, $m(n)$ is minimal subject to the requirement
$\prod_{k=0}^n1-2^{-m(k)}>\bover12$.

\medskip

\quad{\bf (ii)} If $t\in I$, then

\Centerline{$\{\frak x:\frak x\in X,\,t\notin\phi(\frak x)\}
=\{\sequencen{\langle a_{ni}\rangle_{i<m(n)}}:
  t\notin\bigcup_{n\in\Bbb N}\bigcap_{i<m(n)}a_{ni}\}$}

\noindent has measure

\Centerline{$\prod_{n=0}^{\infty}
\lambda_n\{\ofamily{i}{m(n)}{a_i}:t\notin\bigcap_{i<m(n)}a_i\}$,}

\noindent where $\lambda_n$ is the product measure on $(\Cal PI)^{m(n)}$
for each $n$.   But this is just

\Centerline{$\prod_{n=0}^{\infty}1-2^{-m(n)}=\Bover12$,}

\noindent by the choice of $\sequencen{m(n)}$.   Accordingly
$\lambda\{\frak x:t\in\phi(\frak x)\}=\bover12$ for every $t\in I$.
Next, if we identify $X$ with
$\Cal P(\{(t,n,i):t\in I$, $n\in\Bbb N$, $i<m(n)\})$, each set
$E_t=\{\frak x:t\in\phi(\frak x)\}$ is determined by
coordinates in $J_t=\{(t,n,i):n\in\Bbb N,\,i<m(n)\}$.   Since the sets
$J_t$ are disjoint, the sets $E_t$, for different $t$, are
stochastically independent (464Aa), so if $J\subseteq I$ is finite,

\Centerline{$\lambda\{\frak x:J\subseteq\phi(\frak x)\}
=\prod_{t\in J}\lambda E_t=2^{-\#(J)}=\nu\{a:J\subseteq a\}$.}

This shows that $\lambda\phi^{-1}[F]=\nu F$ whenever $F$ is of the form
$\{a:J\subseteq a\}$ for some finite $J\subseteq I$.   By the Monotone
Class Theorem (136C), $\lambda\phi^{-1}[F]=\nu F$ for every $F$
belonging to the $\sigma$-algebra generated by sets of this form.   But
this $\sigma$-algebra certainly contains all sets of the form
$\{a:a\cap J=c\}$ where $J\subseteq I$ is finite and $c\subseteq J$,
which are the
sets corresponding to the basic cylinder sets in the product
$\{0,1\}^I$.   By 254G, $\phi$ is \imp.

\medskip

{\bf (b)} This uses the same idea as (a-ii).   Writing $\nu^3$ for the
product measure on $(\Cal PI)^3$, then, for any $t\in I$,

$$\eqalign{\nu^3\{(a,b,&c):t\in(a\cap b)\cup(a\cap c)\cup(b\cap c)\}\cr
&=\nu^3\{(a,b,c):t\in a\cap b\}+\nu^3\{(a,b,c):t\in a\cap c\}\cr
&\qquad\qquad\qquad+\nu^3\{(a,b,c):t\in b\cap c\}
  -2\nu^3\{(a,b,c):t\in a\cap b\cap c\}\cr
&=3\cdot\Bover14-2\cdot\Bover18=\Bover12.\cr}$$

\noindent Once again, these sets are independent for different $t$, and
this is all we need to know in order to be sure that the map is \imp.
}%end of proof of 464B

\leader{464C}{Lemma} Let $I$ be any set, and let $\nu$ be the usual
measure on $\Cal PI$.

(a)\cmmnt{ (see {\smc Sierpi\'nski 45})} If $\Cal F\subseteq\Cal PI$
is any filter containing every cofinite
set, then $\nu_*\Cal F=0$ and $\nu^*\Cal F$ is either $0$ or $1$.   If
$\Cal F$ is a non-principal ultrafilter then $\nu^*\Cal F=1$.

(b)\cmmnt{ ({\smc Talagrand 80}
\footnote{The date of this paper is misleading, as there was an unusual
backlog in the journal;  in reality it preceded
{\smc Fremlin \& Talagrand 79}.})}
If $\sequencen{\Cal F_n}$ is a sequence of filters on $I$, all of
outer measure $1$, then $\bigcap_{n\in\Bbb N}\Cal F_n$ also has outer
measure $1$.

\proof{{\bf (a)} That $\nu^*\Cal F\in\{0,1\}$ is immediate from 464Ac.
If $\Cal F$ is an ultrafilter, then
$\Cal PI=\Cal F\cup\{I\setminus a:a\in\Cal F\}$;  but as
$a\mapsto I\setminus a$ is a measure space
automorphism of $(\Cal PI,\nu)$,
$\nu^*\{I\setminus a:a\in\Cal F\}=\nu^*\Cal F$, and both must be at
least $\bover12$, so $\nu^*\Cal F=1$.   Equally,
$\nu^*(\Cal PI\setminus\Cal F)=\nu^*\{I\setminus a:a\in\Cal F\}=1$, so
$\nu_*\Cal F=0$.   Returning to a general filter containing every
cofinite set, this is included in a non-principal ultrafilter, so also
has inner measure $0$.

\medskip

{\bf (b)} Let $\sequencen{m(n)}$ be a sequence in $\Bbb N$ such that
$\prod_{n=0}^{\infty}1-2^{-m(n)}=\bover12$, and let $X$, $\lambda$ and
$\phi:X\to\Cal PI$ be as in 464Ba.  Consider the set
$D=\prod_{n\in\Bbb N}\Cal F_n^{m(n)}$ as
a subset of $X$.   By 254Lb, $\lambda^*D=1$.   If we set
$\Cal F=\bigcap_{n\in\Bbb N}\Cal F_n$, then we see that whenever
$\frak x=\sequencen{\langle a_{ni}\rangle_{i<m(n)}}$ belongs to $D$,

\Centerline{$\phi(\frak x)\supseteq\bigcap_{i<m(n)}a_{ni}\in\Cal F_n$}

\noindent for every $n$, so that $\phi(\frak x)\in\Cal F$.   Thus
$D\subseteq\phi^{-1}[\Cal F]$ and

\Centerline{$\nu^*\Cal F\ge\lambda^*\phi^{-1}[\Cal F]\ge\lambda^*D=1$}

\noindent because $\phi$ is \imp\ (413Eh).

As $\sequencen{\Cal F_n}$ is arbitrary, we have the result.
}%end of proof of 464C

\leader{464D}{Construction}\cmmnt{ ({\smc Talagrand 80})} Let $I$ be
any set, and $\nu$ the usual
Radon measure on $\Cal PI$, with $\Tau$ its domain.   Let $\Sigma$ be
the set

$$\eqalign{\{E:E\subseteq\Cal PI,
&\text{ there are a set }F\in\Tau
   \text{ and a filter }\Cal F\text{ on }I\cr
&\hskip5em
 \text{ such that }\nu^*\Cal F=1
   \text{ and }E\cap\Cal F=F\cap\Cal F\}.\cr}$$

\noindent Then there is a unique extension of $\nu$ to a complete
probability measure $\mu$, with domain $\Sigma$, defined by saying that
$\mu E=\nu F$ whenever $E\in\Sigma$, $F\in\Tau$ and there is a filter
$\Cal F$ on $I$ such that $\nu^*\Cal F=1$ and $E\cap\Cal F=F\cap\Cal F$.
\prooflet{(By 464C, we can apply 417A with

\Centerline{$\Cal A=\{\Cal PI\setminus\Cal F:\Cal F$ is a filter on $I$
such that $\nu^*\Cal F=1\}$.)}
}%end of prooflet

\medskip

\noindent{\bf Definition} This measure $\mu$ is {\bf Talagrand's
measure} on $\Cal PI$.

\leader{464E}{Example} If $\mu$ is Talagrand's measure on
$X=\Cal P\Bbb N$, and $\Sigma$ its domain, then there is a set
$K\subseteq\Bbb R^X$, consisting of $\Sigma$-measurable functions and
compact for the topology $\frak T_p$ of pointwise convergence, such
that $K$ is not compact for the topology of convergence in
measure\cmmnt{, that is, there is a sequence in $K$
with no subsequence which is convergent almost everywhere, even though
every cluster point is $\Sigma$-measurable}.

\proof{ Take $K$ to be
$\{\chi\Cal F:\Cal F$ is an ultrafilter on $\Bbb N\}$.   Then $K$ is
$\frak T_p$-compact (in fact, is precisely the set of Boolean
homomorphisms from $\Cal P\Bbb N$ to $\{0,1\}$, identified with the
Stone space of $\Cal P\Bbb N$ in 311E).   By 464Ca, we see that any
non-principal ultrafilter $\Cal F$ on $\Bbb N$ belongs to $\Sigma$, and
$\mu\Cal F=1$.   On the other hand, all the principal ultrafilters
$\Cal F_n=\{a:n\in a\subseteq\Bbb N\}$ are
measured by $\nu$ and therefore by $\mu$, and form a stochastically
independent sequence of sets of measure $\bover12$.   So $K$ consists of
$\Sigma$-measurable functions;  but the sequence $\sequencen{\chi F_n}$
has no subsequence which is convergent almost everywhere, and $K$ is not
relatively compact for the topology of convergence in measure.
}%end of proof of 464E

\cmmnt{\medskip

\noindent{\bf Remark} In this example, a very large number of members of
$K$ are equal almost everywhere;  indeed, all non-principal ultrafilters
are equal a.e., and if we look at $\{f^{\ssbullet}:f\in K\}$ in
$L^0(\mu)$, it is a countable discrete set.   Given that $K$ is a set of
measurable functions homeomorphic to $\beta\Bbb N$, something like this
has to happen (see 536D\formerly{5{}36C}
in Volume 5).   Looking at this from a different angle, if
we wish to extend the usual measure $\nu$ on $\Cal P\Bbb N$ to measure
every ultrafilter, there will have to be two distinct ultrafilters
$\Cal F_1$, $\Cal F_2$ such that $\Cal F_1\symmdiff\Cal F_2$ is negligible for the extended measure.
}%end of comment

\leader{464F}{The $L$-space $\ell^{\infty}(I)^*$}\cmmnt{ For the next
step, we shall need to recall some facts from Volume 3.}
Let $I$ be any set.

\spheader 464Fa \cmmnt{The space} $\ell^{\infty}(I)$\cmmnt{ of
bounded real-valued
functions on $I$} is an $M$-space\cmmnt{ (354Ha)}, so\cmmnt{ its
dual} $\ell^{\infty}(I)^{\sim}=\ell^{\infty}(I)^*$ is an
$L$-space\cmmnt{ (356N),
that is, a Banach lattice such that $\|f+g\|=\|f\|+\|g\|$ for all
non-negative $f$, $g\in\ell^{\infty}(I)^*$}.
\dvro{We}{Since $\ell^{\infty}(I)$ can be identified with the space
$L^{\infty}(\Cal PI)$ as described in \S363 (see 363Ha), we} can
identify $\ell^{\infty}(I)^*$ with the $L$-space $M$ of bounded finitely
additive functionals on $\Cal PI$\cmmnt{ (363K)}, matching any
$f\in\ell^{\infty}(I)^*$ with
the functional $a\mapsto f(\chi a):\Cal PI\to\Bbb R$ in $M$.

\spheader 464Fb\dvro{ Write $M_{\tau}$ for the
band of completely additive functionals on $\Cal PI$.}{ In \S\S361-363 I
examined some of the bands in $M$.   The most
significant ones for our present purposes are the band $M_{\tau}$ of
completely additive functionals (362Bb) and its complement
$M_{\tau}^{\perp}$ (352P);  because $M$ is Dedekind complete, we have
$M=M_{\tau}\oplus M_{\tau}^{\perp}$ (353I).}   \cmmnt{In
fact} $M_{\tau}$ is just the set of those $\theta\in M$ such that
$\theta a=\sum_{t\in a}\theta\{t\}$ for every $a\subseteq I$, while
$M_{\tau}^{\perp}$ is the set of those $\theta\in M$ such that
$\theta\{t\}=0$ for every $t\in I$.   \prooflet{\Prf\ For any $\theta\in
M$, we can set $\alpha_t=\theta\{t\}$ for each $t\in I$;  in this
case,

\Centerline{$|\sum_{t\in J}\alpha_t|=|\theta J|\le\|\theta\|$}

\noindent for every finite set $J\subseteq I$, so
$\sum_{t\in I}|\alpha_t|$ is finite, and we have a functional $\theta_1$
defined by setting
$\theta_1a=\sum_{t\in a}\alpha_t$ for every $a\subseteq I$.   It is easy
to check that $\theta_1\in M_{\tau}$.   If we write
$\theta_2=\theta-\theta_1$ then $\theta_2\{t\}=0$ for every $t\in
I$.   So if $\phi\in M_{\tau}$ and $0\le\phi\le|\theta_2|$, we still
have $\phi\{t\}=0$ for every $t\in I$, and $\phi a=0$ for every
finite $a\subseteq I$ and $\phi=0$;  thus $\theta_2\in
M_{\tau}^{\perp}$.   Now $\theta\in M_{\tau}$ iff $\theta_2=0$, that is,
iff $\theta=\theta_1$ and $\theta a=\sum_{t\in a}\theta\{t\}$ for every
$a\subseteq I$;  while $\theta\in M_{\tau}^{\perp}$ iff $\theta_1=0$,
that is, $\theta\{t\}=0$ for every $t\in I$.\ \Qed
}%end of prooflet

Observe that if $\theta\in M_{\tau}^{\perp}$ and $a$, $b\subseteq I$ are
such that $a\symmdiff b$ is finite, then $\theta a=\theta b$\cmmnt{,
because $\theta(a\setminus b)=\theta(b\setminus a)=0$}.

\spheader 464Fc \cmmnt{It will be useful to have an elementary fact
out in the open.}   If $\theta\in M^+\setminus\{0\}$, 
then $\{a:a\subseteq I,\,\theta a=\theta I\}$ is a filter\cmmnt{;  
this is because $\{a:\theta a=0\}$ is a proper ideal in $\Cal PI$}.

\leader{464G}{}\cmmnt{ We also need a new result not exactly covered
by those in Chapters 35 and 36.

\medskip

\noindent}{\bf Lemma} Let $\frak A$ be any Boolean algebra.   Write $M$
for the $L$-space of bounded additive functionals on $\frak A$, and
$M^+$ for its positive cone\cmmnt{, the set of non-negative additive
functionals}.   Suppose that $\Delta:M^+\to\coint{0,\infty}$ is a
functional such that

\inset{($\alpha$) $\Delta$ is non-decreasing,}

\inset{($\beta$) $\Delta(\alpha\theta)=\alpha\Delta(\theta)$ whenever
$\theta\in M^+$, $\alpha\ge 0$,}

\inset{($\gamma$)
$\Delta(\theta_1+\theta_2)=\Delta(\theta_1)+\Delta(\theta_2)$ whenever
$\theta_1$, $\theta_2\in M^+$ are such that, for some $e\subseteq I$,
$\theta_1(1\setminus e)=\theta_2e=0$,}

\inset{($\delta$) $|\Delta(\theta_1)-\Delta(\theta_2)|
\le\|\theta_1-\theta_2\|$ for all $\theta_1$, $\theta_2\in M^+$.}

\noindent Then there is a non-negative $h\in M^*$ extending $\Delta$.

\proof{{\bf (a)} If $\theta_1$, $\theta_2\in M^+$ are such that
$\theta_1\wedge\theta_2=0$ in $M^+$, then
$\Delta(\theta_1+\theta_2)=\Delta(\theta_1)+\Delta(\theta_2)$.
\Prf\ Let $\epsilon>0$.   Then there is an $e\in\frak A$ such that
$\theta_1(1\Bsetminus e)+\theta_2e\le\epsilon$ (362Ba).   Set
$\theta_1'a=\theta_1(a\Bcap e)$, $\theta_2'(a)=\theta_2(a\Bsetminus e)$
for $a\in\frak A$;  then $\theta_1'$ and $\theta_2'$ belong to $M^+$ and
$\theta_1'(1\setminus e)=\theta_2'e=0$, so
$\Delta(\theta_1'+\theta_2')=\Delta(\theta_1')+\Delta(\theta_2')$, by
hypothesis ($\gamma$).   On the other hand,

\Centerline{$0\le\theta_1a-\theta_1'a=\theta(a\Bsetminus
e)\le\theta(1\Bsetminus e)\le\epsilon$}

\noindent for every $a\in\frak A$, so
$\|\theta_1-\theta_1'\|\le\epsilon$.   Similarly,
$\|\theta_2-\theta_2'\|\le\epsilon$ and
$\|(\theta_1+\theta_2)-(\theta_1'-\theta_2')\|\le 2\epsilon$.   But this
means that we can use the hypothesis ($\delta$) to see that

\Centerline{$|\Delta(\theta_1+\theta_2)
  -\Delta(\theta_1)-\Delta(\theta_2)|
\le|\Delta(\theta'_1+\theta'_2)
  -\Delta(\theta'_1)-\Delta(\theta'_2)|+4\epsilon
=4\epsilon$.}

\noindent As $\epsilon$ is arbitrary,
$\Delta(\theta_1+\theta_2)=\Delta(\theta_1)+\Delta(\theta_2)$.\ \Qed

\medskip

{\bf (b)} Now recall that $M$, being a Dedekind complete Riesz space,
can be identified with an order-dense solid linear subspace of
$L^0(\frak C)$ for some Boolean algebra $\frak C$ (368H).   Inside
$L^0(\frak C)$ we have the order-dense Riesz subspace $S(\frak C)$
(364Ja).   Write $S_1$ for $M\cap S(\frak C)$, so that $S_1$ is an
order-dense Riesz subspace of $M$ (352Nc, 353A).

If $\theta_1$, $\theta_2\in S_1^+=S_1\cap M^+$, then
$\Delta(\theta_1+\theta_2)=\Delta(\theta_1)+\Delta(\theta_2)$.   \Prf\
We can express $\theta_1$ as $\sum_{i=0}^m\alpha_i\chi c_i$, where
$c_0,\ldots,c_m$ are disjoint members of $\frak C$, and $\alpha_i\ge 0$
for each $i$ (361Ec);  adding a term $0\cdot\chi(1\Bsetminus\sup_{i\le
m}c_i)$ if necessary, we may suppose that $\sup_{i\le m}c_i=1$.
Similarly, we can express $\theta_2$ as $\sum_{j=0}^n\beta_j\chi d_j$
where $d_0,\ldots,d_n\in\frak C$ are disjoint, $\beta_j\ge 0$ for every
$j$ and $\sup_{j\le n}d_j=1$.   In this case,
$\theta_1=\sum_{i\le m,j\le n}\alpha_i\chi(c_i\Bcap d_j)$ and
$\theta_2=\sum_{i\le m,j\le n}\beta_j\chi(c_i\Bcap d_j)$.
Re-enumerating $\{c_i\Bcap d_j:i\le m,\,j\le n\}$ as $\langle e_i:i\le
k\}$ we have expressions of $\theta_1$, $\theta_2$ in the form
$\sum_{i\le k}\gamma_i\chi e_i$, $\sum_{i\le k}\delta_i\chi e_i$, while
$e_0,\ldots,e_k$ are disjoint.

Setting $\theta_1^{(r)}=\sum_{i=0}^r\gamma_i\chi e_i$ for $r\le k$, we
see that $\theta_1^{(r)}\wedge\gamma_{r+1}\chi e_{r+1}=0$ for $r<k$, so
(a) above, together with the hypothesis ($\beta$), tell us that

\Centerline{$\Delta(\theta^{(r+1)}_1)
=\Delta(\theta^{(r)}_1)+\Delta(\gamma_{r+1}\chi e_{r+1})
=\Delta(\theta^{(r)}_1)+\gamma_{r+1}\Delta(\chi e_{r+1})$}

\noindent for $r<k$.   Accordingly
$\Delta(\theta_1)=\sum_{i=0}^k\gamma_i\Delta(\chi e_i)$.   Similarly,
$\Delta(\theta_2)=\sum_{i=0}^k\delta_i\Delta(\chi e_i)$ and

\Centerline{$\Delta(\theta_1+\theta_2)
=\sum_{i=0}^k(\gamma_i+\delta_i)\Delta(\chi e_i)
=\Delta(\theta_1)+\Delta(\theta_2)$,}

\noindent as claimed.\ \Qed

\medskip

{\bf (c)} Consequently,
$\Delta(\theta_1+\theta_2)=\Delta(\theta_1)+\Delta(\theta_2)$ for all
$\theta_1$, $\theta_2\in M^+$.   \Prf\ Let $\epsilon>0$.   Because $S_1$
is order-dense in $M$, and the norm of $M$ is order-continuous (354N),
$S_1$ is norm-dense (354Ef), and there are $\theta'_1$,
$\theta'_2\in S_1^+$ such that $\|\theta_j-\theta'_j\|\le\epsilon$ for
both $j$ (354Be).   But now, just as in (a),

\Centerline{$|\Delta(\theta_1+\theta_2)
  -\Delta(\theta_1)-\Delta(\theta_2)|
\le|\Delta(\theta'_1+\theta'_2)
  -\Delta(\theta'_1)-\Delta(\theta'_2)|+4\epsilon
=4\epsilon$.}

\noindent As $\epsilon$ is arbitrary, we have the result.\ \Qed

\medskip

{\bf (d)} Now (c) and the hypothesis ($\beta$) are sufficient to ensure
that $\Delta$ has an extension to a positive linear functional (355D).
}%end of proof of 464G

\leader{464H}{}\cmmnt{ The next lemma contains the key ideas needed
for the rest of the section.

\medskip

\noindent}{\bf Lemma} Let $I$ be any set, and $M$ the $L$-space of
bounded additive functionals on $\Cal PI$;  let $\nu$ be the usual
measure on $\Cal PI$.   For $\theta\in M^+$, set

\Centerline{$
\Delta(\theta)=\overline{\intop}\theta\,d\nu$.}

(a) For every $\theta\in M^+$,
$\bover12\theta I\le\Delta(\theta)\le\theta I$.

(b) There is a non-negative $h\in M^*$ such that
$h(\theta)=\Delta(\theta)$ for every $\theta\in M^+$.

(c) If $\theta\in(M_{\tau}^{\perp})^+$, where $M_{\tau}\subseteq M$ is
the band of completely additive functionals, then
$\theta\le\Delta(\theta)\,\,\nu$-a.e., and
$\nu^*\{a:\alpha\le\theta a\le\Delta(\theta)\}=1$ for every
$\alpha<\Delta(\theta)$.

(d) Suppose that $\theta\in(M_{\tau}^{\perp})^+$ and $\beta$,
$\gamma\in[0,1]$ are such that $\theta I=1$ and
$\beta\theta'I\le\Delta(\theta')\le\gamma\theta'I$ whenever
$\theta'\le\theta$ in $M^+$.   Then, for any $\alpha<\beta$,

\quad(i) for any finite set $K\subseteq\Cal PI$, the set

\Centerline{$\{a:a\subseteq I,\,
\alpha\theta b\le\theta(a\cap b)\le\gamma\theta b$ for every $b\in K\}$}

\noindent has outer measure $1$ in $\Cal PI$;

\quad(ii) if $\alpha\ge\bover12$, the set

\Centerline{$R=\{(a,b,c):a,\,b,\,c\subseteq I,\,
\theta((a\cap b)\cup(a\cap c)\cup(b\cap c))
\ge 2\alpha^2+(1-2\alpha)\gamma^2\}$}

\noindent has outer measure $1$ in $(\Cal PI)^3$;

\quad(iii) if $\alpha\ge\bover12$, then
$2\alpha^2+(1-2\alpha)\gamma^2\le\gamma$.

(e) Any $\theta\in M^+$ can be expressed as $\theta_1+\theta_2$ where
$\Delta(\theta_1)=\bover12\theta_1I$ and $\Delta(\theta_2)=\theta_2I$.

(f) Suppose that $0\le\theta'\le\theta$ in $M$.

\quad(i) If $\Delta(\theta)=\bover12\theta I$, then
$\Delta(\theta')=\bover12\theta'I$.

\quad(ii) If $\Delta(\theta)=\theta I$, then $\Delta(\theta')=\theta'I$.

\proof{{\bf (a)} Of course

\Centerline{$\Delta(\theta)\le\biggerint\theta I\,d\nu=\theta I$.}

\noindent On the other hand, because
$a\mapsto I\setminus a:\Cal PI\to\Cal PI$ is an automorphism of the
measure space $(\Cal PI,\nu)$,
$\overline{\int}f(I\setminus a)\nu(da)=\overline{\int}f(a)\nu(da)$ for
any real-valued function $f$ (cf.\ 235Xn\Latereditions).   In particular,

\Centerline{$\Delta(\theta)
=\overline{\intop}\theta(a)\nu(da)
=\overline{\intop}\theta(I\setminus a)\nu(da)$.}

\noindent So

$$\eqalignno{2\Delta(\theta)
&=\overline{\int}\theta(a)\nu(da)
  +\overline{\int}\theta(I\setminus a)\nu(da)
\ge\overline{\int}\theta(a)+\theta(I\setminus a)\nu(da)\cr
\noalign{\noindent (133J(b-ii))}
&=\overline{\int}\theta I\,\nu(da)
=\theta I,\cr}$$

\noindent and $\Delta(\theta)\ge\bover12\theta I$.

\medskip

{\bf (b)} I use 464G with $\frak A=\Cal PI$.   Examine the conditions
($\alpha$)-($\delta$) there.

\medskip

\quad\grheada\ Of course $\Delta$ is non-decreasing (133Jc).

\medskip

\quad\grheadb\ $\Delta(\alpha\theta)=\alpha\Delta(\theta)$ for every
$\theta\in M^+$ and every $\alpha\ge 0$, by 133J(b-iii).

\medskip

\quad\grheadc\ If $\theta_1$, $\theta_2\in M^+$ then
$\Delta(\theta_1+\theta_2)\le\Delta(\theta_1)+\Delta(\theta_2)$ by
133J(b-ii), as in (a) above.
If $\theta_1$, $\theta_2\in M^+$
and $e\subseteq I$ are such that $\theta_1(I\setminus e)=\theta_2e=0$,
then $\theta_1a=\theta_1(a\cap e)$, $\theta_2(a)=\theta_2(a\setminus e)$
for every $a\subseteq I$.   So
$\Delta(\theta_1+\theta_2)=\Delta(\theta_1)+\Delta(\theta_2)$ by 464Ab.

\medskip

\quad\grheadd\ For every $a\subseteq I$,
$\theta_2a\le\theta_1a+\|\theta_2-\theta_1\|$, so
$\Delta(\theta_2)\le\Delta(\theta_1)+\|\theta_2-\theta_1\|$.
Similarly, $\Delta(\theta_1)
\le\Delta(\theta_2)+\|\theta_1-\theta_2\|$.   So
$|\Delta(\theta_1)-\Delta(\theta_2)|\le\|\theta_1-\theta_2\|$, and
$\Delta$ satisfies condition ($\delta$) of 464G.

\medskip

\quad\grheade\ Accordingly 464G tells us that $\Delta$ has an extension
to a member of $M^*$.

\medskip

{\bf (c)} If $\theta\in M_{\tau}^{\perp}$, then $\theta a=\theta b$
whenever $a\symmdiff b$ is finite (464Fb), so all the sets
$\{a:\theta a>\alpha\}$ have outer measure either $0$ or $1$, by 464Ac.
But this means that if $f$ is a measurable function
and $\theta\leae f$ and $\int f=\Delta(\theta)$, as in 133J(a-i),
$\{a:f(a)\le\Delta(\theta)\}$ has
positive measure and meets $A=\{a:\theta a>\Delta(\theta)\}$ in a
negligible set;  so $A$ cannot have full outer measure and is
negligible.

On the other hand, if $\alpha<\Delta(\theta)$, then $\theta$ cannot be
dominated a.e.\ by $\alpha\chi(\Cal PI)$, so
$\{a:\theta a>\alpha\}$ is not negligible and has outer measure $1$.
Consequently $\nu^*\{a:\alpha<\theta a\le\Delta(\theta)\}=1$. 

\medskip

{\bf (d)(i)} The point is just that for any $b\subseteq I$, the
functional $\theta_b$, defined by saying that
$\theta_b(a)=\theta(a\cap b)$ for every $a\subseteq I$, belongs to $M^+$
and is dominated by
$\theta$, so that $\beta\theta_bI\le\Delta(\theta_b)\le\gamma\theta_bI$,
and $\{a:\alpha\theta_bI\le\theta_ba\le\gamma\theta_bI\}$ has outer
measure $1$, by (c).   (If $\alpha\theta_bI=\Delta(\theta_b)$, this is
because $\theta_bI=0$, and the result is trivial;  otherwise,
$\alpha\theta_bI<\Delta(\theta_b)\le\gamma\theta_bI$.)   But this just
says that, for any $b\subseteq I$,

\Centerline{$\{a:\alpha\theta b\le\theta(a\Bcap b)\le\gamma\theta b\}$}

\noindent has outer measure $1$.

Now, given any finite set $K\subseteq\Cal PI$, let $\Cal B$ be the
subalgebra of $\Cal PI$ generated by $K$, and $b_0,\ldots,b_n$ the atoms
of $\Cal B$.   Then all the sets

\Centerline{$A_i=\{a:\alpha\theta b_i\le\theta(a\cap b_i)
\le\gamma\theta b_i\}$}

\noindent have outer measure $1$;  because each $A_i$ is determined by
coordinates in $b_i$, and $b_0,\ldots,b_n$ are disjoint,
$A=\bigcap_{i\le n}A_i$ still has outer measure $1$, by 464Aa.  But if
$a\in A$ and $b\in K$, then $b=\bigcup_{i\in J}b_i$ for some
$J\subseteq\{0,\ldots,n\}$ and

$$\eqalign{\alpha\theta b
&=\sum_{i\in J}\alpha\theta b_i
\le\sum_{i\in J}\theta(a\cap b_i)\cr
&=\theta(a\cap b)
\le\sum_{i\in J}\gamma\theta b_i
=\gamma\theta b.\cr}$$

\noindent So $\{a:\alpha\theta b\le\theta(a\cap b)\le\gamma\theta b$ for
every $b\in K\}$ includes $A$ and has outer measure $1$.

\medskip

\quad{\bf (ii)} \Quer\ Suppose, if possible, otherwise;  then
$\nu^3_*(\Cal PI\setminus R)>0$, where $\nu^3$ is the product measure on
$(\Cal PI)^3$, and there is a measurable set
$W\subseteq(\Cal PI)^3\setminus R$ such that $\nu^3W>0$.   For
$a\subseteq I$, set
$W_a=\{(b,c):(a,b,c)\in W\}$;  then $\nu^3W=\int\nu^2W_a\nu(da)$, where
$\nu^2$ is the product measure on $(\Cal PI)^2$, by Fubini's theorem
(252D).   Set $E=\{a:\nu^2W_a$ is defined and not zero$\}$;  then
$\nu E>0$.   Since $\{a:\alpha\le\theta a\le\gamma\}$ has outer measure
$1$, there is an $a\in E$ such that $\alpha\le\theta a\le\gamma$.

For $b\subseteq I$, set $W_{ab}=\{c:(b,c)\in W_a\}=\{c:(a,b,c)\in W\}$.
Then $0<\nu^2W_a=\int\nu W_{ab}\nu(db)$, so $F=\{b:\nu W_{ab}$ is
defined and not $0\}$ has non-zero measure.   But also, by (i),

\Centerline{$\{b:\theta b\ge\alpha,\,\theta(a\cap b)\le\gamma\theta
a\}$}

\noindent has outer measure $1$, so we can find a $b\in F$ such that
$\theta b\ge\alpha$ and $\theta(a\cap b)\le\gamma\theta a$.

By (i) again,

\Centerline{$\{c:\theta(c\cap(a\symmdiff b))\ge\alpha\theta(a\symmdiff
b)\}$}

\noindent has outer measure $1$, so meets $W_{ab}$;  accordingly we have
a $c\subseteq I$ such that $(a,b,c)\in W$ while $\theta(c\cap(a\symmdiff
b))\ge\alpha\theta(a\symmdiff b)$.

Now calculate, for this triple $(a,b,c)$,

$$\eqalignno{\theta((a\cap b)\cup(a\cap c)\cup(b\cap c))
&=\theta(a\cap b)+\theta(c\cap(a\symmdiff b))
\ge\theta(a\cap b)+\alpha\theta(a\symmdiff b)\cr
\noalign{\noindent (by the choice of $c$)}
&=\alpha(\theta a+\theta b)+(1-2\alpha)\theta(a\cap b)\cr
&\ge\alpha(\theta a+\alpha)+(1-2\alpha)\gamma\theta a\cr
\noalign{\noindent (by the choice of $b$, recalling that
$1-2\alpha\le 0)$}
&\ge 2\alpha^2+(1-2\alpha)\gamma^2\cr}$$

\noindent by the choice of $a$.   But this means that $(a,b,c)\in W\cap
R$, which is supposed to be impossible.\ \Bang

\medskip

\quad{\bf (iii)} Now recall that the map
$(a,b,c)\mapsto(a\cap b)\cup(a\cap c)\cup(b\cap c)$ is \imp\ (464Bb).
Since $\theta a\le\Delta(\theta)\le\gamma$ for $\nu$-almost every $a$,
we must have
$\theta((a\cap b)\cup(a\cap c)\cup(b\cap c))\le\gamma$ for
$\nu^3$-almost every $(a,b,c)$.   But as $R$ is not negligible, there
must be some $(a,b,c)\in R$ such that
$\theta((a\cap b)\cup(a\cap c)\cup(b\cap c))\le\gamma$, and
$2\alpha^2+(1-2\alpha)\gamma^2\le\gamma$.

\medskip

{\bf (e)(i)} $M^*$, being the dual of an $L$-space, is an $M$-space
(356Pb), so can be represented as $C(Z)$ for some compact Hausdorff space
$Z$ (354L).   The functional $h$ of (b) above therefore corresponds to a
function $w\in C(Z)$.   Any $\theta\in M^+$ acts on $M^*$ as a positive
linear functional, so corresponds to a Radon measure $\mu_{\theta}$ on
$Z$ (436J/436K);  we have
$\Delta(\theta)=h(\theta)=\int w\,d\mu_{\theta}$.
The inequalities $\bover12\theta I\le\Delta(\theta)\le\theta I$ become
$\bover12\mu_{\theta}Z\le\int w\,d\mu_{\theta}\le\mu_{\theta}Z$, because
the constant function $\chi Z$ corresponds to the standard order unit
of $M^*$ (356Pb again), so that

\Centerline{$\mu_{\theta}Z=\biggerint\chi Z\,d\mu_{\theta}
=\|\theta\|=\theta I$}

\noindent for every $\theta\ge 0$.   Since $0\le h(\theta)\le\|\theta\|$
for every $\theta\ge 0$, $\|w\|_{\infty}=\|h\|\le 1$ and $0\le w\le\chi
Z$.

\medskip

\quad{\bf (ii)} Now suppose that $\beta<\gamma$ and that
$G=\{z:z\in Z,\,\beta<w(z)<\gamma\}$ is non-empty.   In this case there
is a non-zero $\theta_0\in M^+$ such that
$\beta\theta'I\le\Delta(\theta')\le\gamma\theta'I$ whenever
$0\le\theta'\le\theta_0$.

\Prf\grheada\ We have a
solid linear subspace $V=\{v:v\in C(Z),\,v(z)=0$ for every $z\in G\}$ of
$C(Z)$.   Consider $U=\{\theta:\theta\in M,\,\innerprod{\theta}{v}=0$
for every $v\in V\}$, where I write $\innerprod{\,}{\,}$ for the duality
between $M$ and $C(Z)$ corresponding to the identification of $C(Z)$
with $M^*$.

\medskip

\qquad\grheadb\ If $\theta\in U\cap M^+$, then
$\beta\theta I\le\Delta(\theta)\le\gamma\theta I$.   To see this,
observe that $\int v\,d\mu_{\theta}=\innerprod{\theta}{v}=0$ for every
$v\in V$, so

$$\eqalignno{\mu_{\theta}(Z\setminus\overline{G})
&=\sup\{\mu_{\theta}K:K\subseteq Z\setminus\overline{G}\text{ is
compact}\}\cr
&=\sup\{\int v\,d\mu_{\theta}:v\in C(Z),\,
  0\le v\le\chi(Z\setminus\overline{G})\}\cr
\displaycause{because whenever
$K\subseteq Z\setminus\overline{G}$ is compact there is a $v\in C(Z)$
such that $\chi K\le v\le\chi(Z\setminus\overline{G})$, by 4A2F(h-ii)}
&=0.\cr}$$

\noindent Because $w$ is continuous,
$\beta\le w(z)\le\gamma$ for every $z\in\overline{G}$;  thus
$\beta\le w\le\gamma\,\,\mu_{\theta}$-a.e.\ and 
$\int w\,d\mu_{\theta}$ must belong to
$[\beta\mu_{\theta}Z,\gamma\mu_{\theta}Z]=[\beta\theta I,\gamma\theta I]$.

\medskip

\qquad\grheadc\ The dual $M^*=M^{\times}$ of $M$ is
perfect (356Lb), and $C(Z)$ is perfect;  moreover, $M$ is perfect
(356Pa), so the duality
$\innerprod{\,}{\,}$ identifies $M$ with $C(Z)^{\times}$.   Now
$V^{\perp}$, taken in $C(Z)$, contains any continuous function zero on
$Z\setminus G$, so is not $\{0\}$;  since $V^{\perp}$, like $C(Z)$, must
be perfect (356La), $(V^{\perp})^{\times}$ is non-trivial.   Take any
$\psi>0$ in $(V^{\perp})^{\times}$.   Being perfect, $C(Z)$ is Dedekind
complete (356K), so there is a band projection $P:C(Z)\to V^{\perp}$
(353I).   Now $\psi P$ is a positive element of $C(Z)^{\times}$
which is zero on $V$, and must correspond to a non-zero element
$\theta_0$ of $U\cap M^+$.

\medskip

\qquad\grheadd\ If $0\le\theta'\le\theta_0$ in $M$,
then, for any $v\in V$,

\Centerline{$|\innerprod{\theta'}{v}|
\le\innerprod{\theta'}{|v|}
\le\innerprod{\theta_0}{|v|}
=0$,}

\noindent because $|v|\in V$.   So $\theta'\in U$ and
$\beta\theta'I\le\Delta(\theta')\le\gamma\theta'I$, by ($\beta$).   Thus
$\theta_0$ has the required property.\ \Qed

\medskip

\quad{\bf (iii)} It follows at once that $w(z)\ge\bover12$ for every
$z\in Z$.   \Prf\Quer\ If $w(z_0)<\bover12$, then we can apply (ii) with
$\beta=-1$, $\gamma\in\ooint{w(z_0),\bover12}$ to see that there is a
non-zero $\theta\in M^+$ such that $\Delta(\theta)\le\gamma\theta
I<\bover12\theta I$, which is impossible, by (a).\ \Bang\Qed

\medskip

\quad{\bf (iv)} But we find also that $w(z)\notin\ooint{\bover12,1}$ for
any $z\in Z$.  \Prf\Quer\ If $w(z_0)=\delta\in\ooint{\bover12,1}$, then
$2\delta^2+(1-2\delta)\delta^2>\delta$ (because
$\delta(2\delta-1)(1-\delta)>0$).   We can therefore find $\alpha$,
$\beta$ and $\gamma$ such that $\bover12\le\alpha<\beta<\delta<\gamma$
and $2\alpha^2+(1-2\alpha)\gamma^2>\gamma$.   But now
$\{z:\beta<w(z)<\gamma\}$ is non-empty, so by (ii) there is a non-zero
$\theta\in M^+$ such that
$\beta\theta'I\le\Delta(\theta')\le\gamma\theta'I$ whenever
$0\le\theta'\le\theta$.   Multiplying $\theta$ by a suitable scalar if
necessary, we can arrange that $\theta I$ should be $1$.   But this is
impossible, by (d-iii).\ \Bang\Qed

\medskip

\quad{\bf (v)} Thus $w$ takes only the values $\bover12$ and $1$;  let
$H_1$ and $H_2$ be the corresponding open-and-closed subsets of $Z$.

Take $\theta\in M^+$.   For $u\in C(Z)$, set
$\phi(u)=\int u\,d\mu_{\theta}$ and
$\phi_j(u)=\int_{H_j}u\,d\mu_{\theta}$ for each
$j$.   Then each $\phi_j$ is a positive linear functional on $C(Z)$ and
$\phi_j\le\phi$.   But $\phi$ is the image of $\theta$ under the
canonical isomorphism from $M$ to $C(Z)^{\times}\cong M^{\times\times}$,
and $C(Z)^{\times}$ is solid in $C(Z)^{\sim}$ (356B), so both $\phi_1$
and $\phi_2$ belong to the image of $M$, and correspond to $\theta_1$,
$\theta_2\in M$.   For any $u\in C(Z)\cong M^*$,

\Centerline{$\innerprod{\theta_1+\theta_2}{u}
=\phi_1(u)+\phi_2(u)
=\innerprod{\theta}{u}$,}

\noindent so $\theta=\theta_1+\theta_2$.   We have

$$\eqalign{\Delta(\theta_j)=\phi_j(w)&=\int_{H_j}w\,d\mu_{\theta}\cr
&=\Bover12\mu_{\theta}(H_1)=\Bover12\theta_1I\text{ if }j=1,\cr
&=\mu_{\theta}(H_2)=\theta_2I\text{ if }j=2.\cr}$$

\noindent So we have a suitable decomposition
$\theta=\theta_1+\theta_2$.

\medskip

{\bf (f)} This is easy.   Set $\theta''=\theta-\theta'$;  then

\Centerline{$\bover12\theta'I\le\Delta(\theta')\le\theta'I$,}

\Centerline{$\bover12\theta''I\le\Delta(\theta'')\le\theta''I$}

\noindent by (a), while
$\Delta(\theta')+\Delta(\theta'')=\Delta(\theta)$ by (b), and of course
$\theta'I+\theta''I=\theta I$.   But this means that

\Centerline{$\Delta(\theta')-\bover12\theta'I
\le\Delta(\theta)-\bover12\theta I$,
\quad$\theta'I-\Delta(\theta')\le\theta I-\Delta(\theta)$,}

\noindent and the results follow.
}%end of proof of 464H

\leader{464I}{Measurable and purely non-measurable functionals} As
before, let $I$ be any set, $\nu$ the usual measure on $\Cal PI$, $\Tau$
its domain, and
$M$ the $L$-space of bounded additive functionals on $\Cal PI$.
\cmmnt{Following {\smc Fremlin \& Talagrand 79},} I say that
$\theta\in M$ is {\bf measurable} if it is $\Tau$-measurable when
regarded as a
real-valued function on $\Cal PI$, and {\bf purely
non-measurable} if $\{a:a\subseteq I,\,|\theta|(a)=|\theta|(I)\}$ has
outer measure $1$.   \cmmnt{(Of course the zero functional is both
measurable and purely non-measurable.)}

\leader{464J}{Examples}\cmmnt{ Before going farther, I had better
offer some examples of measurable and purely non-measurable
functionals.}   Let $I$, $\nu$ and $M$ be as in 464I.

\spheader 464Ja Any $\theta\in M_{\tau}$ is measurable\cmmnt{, where
$M_{\tau}$ is the space of completely additive functionals on $\Cal
PI$}.   \prooflet{\Prf\ By 464Fb, $\theta$ can be expressed as a sum of
point masses;  say $\theta a=\sum_{t\in a}\alpha_t$ for some family
$\family{t}{I}{\alpha_t}$ in $\Bbb R$.   Since $\sum_{t\in I}|\alpha_t|$
must be finite, $\{t:\alpha_t\ne 0\}$ is countable, and we can express
$\theta$ as the limit of a sequence of finite sums $\sum_{t\in
K}\alpha_t\hat t$, where $\hat t(a)=1$ if $t\in a$, $0$ otherwise.   But
of course every $\hat t$ is a measurable function, so $\sum_{t\in
K}\alpha_t\hat t$ is measurable for every finite set $K$, and $\theta$
is measurable.\ \Qed}

\spheader 464Jb\cmmnt{ For a less elementary measurable functional,
consider the following construction.}   Let $\sequencen{t_n}$ be any
sequence of distinct points in $I$.
Then\cmmnt{ $\lim_{n\to\infty}\Bover1n\#(\{i:i<n,\,t_i\in a\})
=\Bover12$
for $\nu$-almost every $a\subseteq I$.   \prooflet{\Prf\ Set $f_n(a)=1$
if $t_n\in a$, $0$ otherwise.   Then $\sequencen{f_n}$ is an independent
sequence of random variables.   By any of the versions of the Strong Law
of Large Numbers in \S273 (273D, 273H, 273I),
$\lim_{n\to\infty}\Bover1n\sum_{i=0}^{n-1}f_i=\Bover12$ a.e., which is
what was claimed.\ \QeD\ }   So} if we take any non-principal
ultrafilter $\Cal F$ on $\Bbb N$, and set
$\theta a=\lim_{n\to\Cal F}\Bover1n\#(\{i:i<n,\,t_i\in a\})$ for
$a\subseteq I$, $\theta$ will be
constant $\nu$-almost everywhere, and measurable\cmmnt{;  and it is
easy to check that $\theta$ is additive}.   \cmmnt{Note that
$\theta\{t\}=0$
for every $t$, so} $\theta\in M_{\tau}^{\perp}$\cmmnt{, by 464Fb}.

\spheader 464Jc If $\Cal F$
is any non-principal ultrafilter on $I$, and we set $\theta a=1$
for $a\in\Cal F$, $0$ otherwise, then $\theta$ is an additive functional
which is purely non-measurable\cmmnt{, by 464Ca}.

\cmmnt{For further remarks on where to look for measurable and purely
non-measurable functionals, see 464P-464Q below.}

\leader{464K}{The space $M_{\text{m}}$:  Lemma} Let $I$ be any set,
$\nu$ the usual measure on $\Cal PI$, and $M$ the $L$-space of bounded
additive functionals on $\Cal PI$.
Write $M_{\text{m}}$ for the set of measurable $\theta\in M$,
$M_{\tau}$ for the space of completely additive
functionals on $\Cal PI$ and
$\Delta(\theta)=\overline{\int}\theta\,d\nu$ for
$\theta\in M^+$\cmmnt{, as in 464H}.

(a) If $\theta\in M_{\text{m}}\cap M_{\tau}^{\perp}$ and $b\subseteq I$,
then $\theta(a\cap b)=\bover12\theta b$ for $\nu$-almost every
$a\subseteq I$.

(b) $|\theta|\in M_{\text{m}}$ for every $\theta\in M_{\text{m}}$.

(c) A functional $\theta\in M^+$ is measurable iff
$\Delta(\theta)=\bover12\theta I$.

(d) $M_{\text{m}}$ is a solid linear subspace of $M$.

\proof{{\bf (a)(i)} $\theta=\bover12\theta I\,\,\nu$-a.e.   \Prf\ For any
$\alpha\in\Bbb R$, $A_{\alpha}=\{a:\theta a<\alpha\}$ is measurable;
but also $a'\in A$ whenever $a\in A$ and $a\symmdiff a'$ is finite, by
464Fb, so $\nu A$ must be either $1$ or $0$, by 464Ac.   Setting
$\delta=\sup\{\alpha:\nu A_{\alpha}=0\}$, we see that
$\nu A_{\delta}=0$,
$\nu A_{\delta+2^{-n}}=1$ for every $n\in\Bbb N$, so that
$\theta=\delta$ a.e.   Also, because $a\mapsto I\setminus a$ is a
measure space automorphism, $\theta(I\setminus a)=\delta$ for almost
every $a$, so there is some $a$ such that
$\theta a=\theta(I\setminus  a)=\delta$, and
$\delta=\bover12\theta I$.\ \Qed

\medskip

\quad{\bf (ii)} $\theta(a\cap b)=\bover12\theta b$ for almost every $a$.
\Prf\ We know that $\theta a=\bover12\theta I$ for almost every $a$.
But $a\mapsto a\symmdiff b:\Cal PI\to\Cal PI$ is \imp, so
$\theta(a\symmdiff b)=\bover12\theta I$
for almost every $a$.   This means that $\theta a=\theta(a\symmdiff b)$
for almost every $a$, and

\Centerline{$\theta(a\cap b)
=\Bover12(\theta b+\theta a-\theta(a\symmdiff b))=\Bover12\theta b$}

\noindent for almost every $a$.\ \Qed

\medskip

{\bf (b)(i)} If $\theta\in M_{\text{m}}\cap M_{\tau}^{\perp}$, then
$\theta^+$, taken in $M$, is measurable.  \Prf\ For any
$n\in\Bbb N$ we can find $b_n\subseteq I$ such that
$\theta^-b_n+\theta^+(I\setminus b_n)\le 2^{-n}$, so that

\Centerline{$|\theta^+a-\theta(a\cap b_n)|
=|\theta^+a-\theta^+(a\cap b_n)+\theta^-(a\cap b_n)|
\le\theta^+(I\setminus b_n)+\theta^-b_n
\le 2^{-n}$}

\noindent for every $a\subseteq I$.   But as $a\mapsto\theta(a\cap b_n)$ is
constant a.e.\ for every $n$, by (a), so is $\theta^+$, and
$\theta^+$ is measurable.\ \Qed

Consequently $|\theta|=2\theta^+-\theta$ is measurable.

\medskip

\quad{\bf (ii)} Now take an arbitrary $\theta\in M_{\text{m}}$.
Because $M$ is Dedekind complete (354N, 354Ee),
$M=M_{\tau}+M_{\tau}^{\perp}$ (353I again), and we can express $\theta$ as
$\theta_1+\theta_2$ where $\theta_1\in M_{\tau}$ and
$\theta_2\in M_{\tau}^{\perp}$;
moreover, $|\theta|=|\theta_1|+|\theta_2|$ (352Fb).
Now $\theta_1$ is measurable, by 464Ja, so $\theta_2=\theta-\theta_1$ is
measurable;  as $\theta_2\in M_{\tau}^{\perp}$, (i) tells us
that $|\theta_2|$ is measurable.   On the other hand, $|\theta_1|$
belongs to $M_{\tau}$ and is measurable, so
$|\theta|=|\theta_1|+|\theta_2|$ is measurable.

Thus (b) is true.

\medskip

{\bf (c)} Let $f$ be a $\nu$-integrable function such that
$\theta\leae f$ and $\int fd\nu=\Delta(\theta)$.   Then

\Centerline{$\theta I-f(a)\le\theta I-\theta a=\theta(I\setminus a)$}

\noindent for almost every $a$, so

\Centerline{$
\theta I-\Delta(\theta)
=\int\theta I-f(a)\nu(da)
\le\underline{\intop}\theta(I\setminus a)\nu(da)
=\underline{\intop}\theta(a)\nu(da)$}

\noindent because $a\mapsto I\setminus a$ is a measure space
automorphism, as in the proof of 464Ha.   So if
$\Delta(\theta)=\bover12\theta I$ then
$\underline{\int}\theta\,d\nu=\overline{\int}\theta\,d\nu$ and $\theta$
is $\nu$-integrable (133Jd), therefore (because $\nu$ is complete)
$(\dom\nu)$-measurable.   On the other hand, if $\theta$ is measurable,
then

\Centerline{$\Delta(\theta)=\biggerint\theta\,d\nu
=\int\theta(I\setminus a)\nu(da)
=\theta I-\int\theta\,d\nu=\theta I-\Delta(\theta)$,}

\noindent so surely $\Delta(\theta)=\bover12\theta I$.

\medskip

{\bf (d)} Of course $M_{\text{m}}$ is a linear subspace.   If
$\theta_0\in M_{\text{m}}$ and $|\theta|\le|\theta_0|$, then
$|\theta_0|\in M_{\text{m}}$, by (b), so
$\Delta(|\theta_0|)=\bover12|\theta_0|(I)$, by (c).   Because
$\theta^+\le|\theta|\le|\theta_0|$, $\Delta(\theta^+)=\bover12\theta^+I$
(464H(f-i)), and $\theta^+$ is measurable, by (c) in the reverse
direction.   Similarly, $\theta^-$ is measurable, and
$\theta=\theta^+-\theta^-$ is measurable.   As $\theta$ and $\theta_0$
are arbitrary, $M_{\text{m}}$ is solid.
}%end of proof of 464K

\leader{464L}{The space $M_{\text{pnm}}$:  Lemma} Let $I$ be any set,
$\nu$ the usual measure on $\Cal PI$, and $M$ the $L$-space of bounded
additive functionals on $\Cal PI$.   This time,
write $M_{\text{pnm}}$ for the set of those members of $M$ which are
purely non-measurable\cmmnt{ in the sense of 464I}.

(a) If $\theta\in M^+$, then $\theta$ is purely non-measurable iff
$\Delta(\theta)=\theta I$.

(b) $M_{\text{pnm}}$ is a solid linear subspace of $M$.

\proof{{\bf (a)(i)} If $\theta$ is purely non-measurable, and
$f\ge\theta$ is integrable, then $\{a:f(a)\ge\theta I\}$ is a measurable
set including $\{a:\theta a=\theta I\}$, so has measure $1$, and $\int
f\ge\theta I$;  as $f$ is arbitrary, $\Delta(\theta)=\theta I$.

\medskip

\quad{\bf (ii)} If $\Delta(\theta)=\theta I$, then
$\Delta(\theta')=\theta'I$ whenever $0\le\theta'\le\theta$, by 464Hf.
But this means that $\theta'$ cannot be measurable whenever
$0<\theta'\le\theta$, by 464Kc above, so that $\theta'\notin M_{\tau}$
whenever $0<\theta'\le\theta$, by 464Ja.   Thus $\theta\in
M_{\tau}^{\perp}$.

By 464Hc, $\nu^*\{a:\alpha\le\theta
a\le\Delta(\theta)\}=1$ for every $\alpha<\Delta(\theta)$.
Let $\sequencen{m(n)}$ be a sequence in $\Bbb N$ such that
$\prod_{n=0}^{\infty}1-2^{-m(n)}=\bover12$, and define $X$, $\lambda$
and $\phi$ as in 464Ba.   Set $\eta_n=2^{-n}/m(n)>0$ for each $n$.
Consider the sets
$A_n=\{a:\theta a\ge(1-\eta_n)\theta I\}$ for each $n\in\Bbb N$.   Then
$\nu^*A_n=1$ for each $n$, and
$\lambda^*(\prod_{n\in\Bbb N}A_n^{m(n)})=1$.   Because $\phi$ is \imp,
$\nu^*(\phi[\prod_{n\in\Bbb N}A_n^{m(n)}])=1$ (413Eh again).   But if
$\frak x=\sequencen{\langle a_{ni}\rangle_{i<m(n)}}$ belongs to
$\prod_{n\in\Bbb N}A_n^{m(n)}$, then

\Centerline{$\theta(\bigcap_{i<m(n)}a_{ni})
\ge\theta I-m(n)\eta_n\theta I=(1-2^{-n})\theta I$}

\noindent for each $n$, and

\Centerline{$\theta(\phi(\frak x))
\ge\sup_{n\in\Bbb N}\theta(\bigcap_{i<m(n)}a_{ni})=\theta I$.}

\noindent Thus the filter $\{a:\theta a=\theta I\}$ includes
$\phi[\prod_{n\in\Bbb N}A_n^{m(n)}]$ and has outer measure $1$, so that
$\theta$ is purely non-measurable.

\medskip

{\bf (b)(i)} If $\theta\in M_{\text{pnm}}$ and $|\theta'|\le|\theta|$,
then $\Delta(|\theta|)=|\theta|(I)$, by the definition in 464I, so
$\Delta(|\theta'|)=|\theta'|(I)$, by 464H(f-ii), and
$\theta'\in M_{\text{pnm}}$, by (a) above.   Thus $M_{\text{pnm}}$ is
solid.

\medskip

\quad{\bf (ii)} If $\theta_1$, $\theta_2\in M_{\text{pnm}}$, then

\Centerline{$\Delta(|\theta_1|+|\theta_2|)
=\Delta(|\theta_1|)+\Delta(|\theta_2|)
=|\theta_1|(I)+|\theta_2|(I)
=(|\theta_1|+|\theta_2|)(I)$}

\noindent (using 464Hb), and $|\theta_1|+|\theta_2|\in M_{\text{pnm}}$;
as $|\theta_1+\theta_2|\le|\theta_1|+|\theta_2|$, $\theta_1+\theta_2\in
M_{\text{pnm}}$.   Thus $M_{\text{pnm}}$ is closed under addition.

\medskip

\quad{\bf (iii)} It follows from (ii) that if $\theta\in M_{\text{pnm}}$
then $n\theta\in M_{\text{pnm}}$ for every integer $n\ge 1$, and then
from (i) that $\alpha\theta\in M_{\text{pnm}}$ for every $\alpha\in \Bbb
R$;  so that $M_{\text{pnm}}$ is closed under scalar multiplication, and
is a linear subspace.
}%end of proof of 464L

\leader{464M}{Theorem}\cmmnt{ ({\smc Fremlin \& Talagrand 79})} Let
$I$ be any set.   Write $M$ for the $L$-space of bounded finitely
additive functionals on $\Cal PI$,
and $M_{\text{m}}$, $M_{\text{pnm}}$ for the spaces of measurable and
purely non-measurable functionals\cmmnt{, as in 464K-464L}.   Then
$M_{\text{m}}$ and $M_{\text{pnm}}$ are complementary bands in $M$.

\proof{{\bf (a)} We know from 464K and 464L that these are both solid
linear subspaces of $M$.   Next, $M_{\text{m}}\cap M_{\text{pnm}}=\{0\}$.
\Prf\ If $\theta$ belongs to the intersection,
then $\Delta(|\theta|)=\bover12|\theta|(I)=|\theta|(I)$, by 464Kc and
464La;  so $\theta=0$.\ \Qed

\medskip

{\bf (b)} Now recall that every element of $M^+$ is expressible in the
form $\theta_1+\theta_2$ where $\theta_1\in M_{\text{m}}^+$ and
$\theta_2\in M_{\text{pnm}}^+$;  this is 464He, using 464Kc and 464La
again.   Because $M_{\text{m}}$ and $M_{\text{pnm}}$ are linear
subspaces, with intersection $\{0\}$,
$M=M_{\text{m}}\oplus M_{\text{pnm}}$.   Now
$M_{\text{pnm}}\subseteq M_{\text{m}}^{\perp}$,
so $M_{\text{m}}+M_{\text{m}}^{\perp}=M$ and
$M_{\text{m}}=M_{\text{m}}^{\perp\perp}$ is a complemented band (352Ra);
similarly, $M_{\text{pnm}}$ is a complemented band.   Since
$(M_{\text{m}}+M_{\text{pnm}})^{\perp}=\{0\}$, $M_{\text{m}}$ and
$M_{\text{pnm}}$ are complementary bands (see 352S).
}%end of proof of 464M

\leader{464N}{Corollary}\cmmnt{ ({\smc Fremlin \& Talagrand 79})} Let
$I$ be any set, and let $\mu$ be Talagrand's measure on $\Cal PI$;
write $\Sigma$ for its domain.   Then every bounded additive functional
on $\Cal PI$ is $\Sigma$-measurable.

\proof{ Defining $M$, $M_{\text{m}}$ and $M_{\text{pnm}}$ as in 
464K-464M, %464K 464L 464M 
we see that every functional in $M_{\text{m}}$ is
$\Sigma$-measurable because it is (by definition)
$(\dom\nu)$-measurable,
where $\nu$ is the usual measure on $\Cal PI$.   If
$\theta\in M_{\text{pnm}}^+$, then $\Cal F=\{a:\theta a=\theta I\}$ is a
non-measurable filter;  but this means that $\mu\Cal F=1$, by the
construction of $\mu$, so that $\theta=\theta I\,\,\mu$-a.e.   So if
$\theta$ is any member of $M_{\text{pnm}}$, both $\theta^+$ and
$\theta^-$ are $\Sigma$-measurable, and $\theta=\theta^+-\theta^-$ also
is.
}%end of proof of 464N

\leader{464O}{Remark}\cmmnt{ Note that we have a very simple
description of the behaviour of additive functionals as seen by the
measure $\mu$.}   Since $M_{\tau}\subseteq M_{\text{m}}$, we have a
three-part band decomposition
$M=M_{\tau}\oplus(M_{\text{m}}\cap M_{\tau}^{\perp})
\oplus M_{\text{pnm}}$.

(i) Functionals in $M_{\tau}$ are $\Tau$-measurable, where $\Tau$ is the
domain of $\nu$, therefore
$\Sigma$-measurable\cmmnt{, just because they can be built up from the
functionals $a\mapsto\chi a(t)$, as in 464Fb}.

(ii) A functional\cmmnt{ in $M_{\text{m}}$ is $\Tau$-measurable, by
definition;  but a functional} $\theta$ in
$M_{\text{m}}\cap M_{\tau}^{\perp}$ is\cmmnt{ actually} constant, with
value $\bover12\theta I$, $\nu$-almost everywhere\cmmnt{, by 464Ka}.
\cmmnt{Thus the almost-constant nature of the functionals described in
464Jb is typical of measurable functionals in $M_{\tau}^{\perp}$.}

(iii) Finally, a functional $\theta\in M_{\text{pnm}}$ is equal to
$\theta I\,\,\mu$-almost everywhere\cmmnt{;  once again, this follows
from the definition of `purely non-measurable' and the construction of
$\mu$ for $\theta\ge 0$, and from the fact that $M_{\text{pnm}}$ is
solid for other $\theta$}.

(iv) Thus\cmmnt{ we see that} any
$\theta\in M_{\tau}^{\perp}
=(M_{\text{m}}\cap M_{\tau}^{\perp})\oplus M_{\text{pnm}}$ is constant
$\mu$-a.e.   \cmmnt{We also have}
$\int\theta\,d\mu=\Delta(\theta)$ for every $\theta\ge 0$
\prooflet{(look at $\theta\in M_{\text{m}}$ and $\theta\in
M_{\text{pnm}}$ separately, using 464La for the latter)}, so that if
$h\in M^*$ is the linear functional of 464Hb, then
$\int\theta\,d\mu=h(\theta)$ for every $\theta\in M$.

\leader{464P}{More on purely non-measurable functionals (a)} We can
discuss non-negative additive functionals on $\Cal PI$ in terms of the
Stone-\v Cech compactification $\beta I$ of $I$\cmmnt{, as follows}.
For any set $A\subseteq\beta I$ set
$H_A=\{a:a\subseteq I,\,A\subseteq\widehat{a}\}$, where
$\widehat{a}\subseteq\beta I$ is the
open-and-closed set corresponding to $a\subseteq I$.   If
$A\ne\emptyset$, $H_A$ is a filter on $I$.   Write $\Cal A$ for the
family of those sets $A\subseteq\beta I$ such that $\nu^*H_A=1$,
where $\nu$ is the usual measure on $\Cal PI$.   Then $\Cal A$ is a
$\sigma$-ideal.
\prooflet{\Prf\ Of course $A\in\Cal A$ whenever $A\subseteq B\in\Cal A$,
since then $H_A\supseteq H_B$.   If $\sequencen{A_n}$ is a sequence in
$\Cal A$ with union $A$, then $\nu^*(\bigcap_{n\in\Bbb N}H_{A_n})=1$, by
464C;  but $H_A\supseteq\bigcap_{n\in\Bbb N}H_{A_n}$.\ \QeD\ }   Note
that if $A\in\Cal A$ then $\overline{A}\in\Cal A$\cmmnt{, because
$H_A=H_{\overline{A}}$}.   \cmmnt{We see also that} $\{z\}\in\Cal A$
for every $z\in\beta I\setminus I$\cmmnt{ (since $H_{\{z\}}$ is a
non-principal ultrafilter, as in 464Jc)}, while $\{t\}\notin\Cal A$ for
any $t\in I$\cmmnt{ (since $H_{\{t\}}=\{a:t\in a\}$ has measure
$\bover12$)}.

\dvro{We}{Because $\beta I$ can be identified with the Stone space of
$\Cal PI$ (4A2I(b-i)), we}
have a one-to-one correspondence between non-negative additive
functionals $\theta$ on $\Cal PI$ and Radon measures $\mu_{\theta}$ on
$\beta I$, defined by
writing $\mu_{\theta}(\widehat{a})=\theta a$ whenever $a\subseteq I$ and
$\theta\in M^+$\cmmnt{ (416Qb)}.
\cmmnt{(This is not the same as the measure $\mu_{\theta}$ of part (e)
of the proof of 464H, which is on a much larger space.)}
Now suppose that $\theta$ is a non-negative additive functional on
$\Cal PI$.   Then $\Cal F_{\theta}=\{a:\theta a=\theta I\}$ is either
$\Cal PI$ or a filter on $I$.   If we set
$F_{\theta}=\bigcap\{\widehat{a}:a\in\Cal F_{\theta}\}$, then
$\Cal F_{\theta}=H_{F_{\theta}}$\cmmnt{ (4A2I(b-iii))}.
\cmmnt{Since $a\in\Cal F_{\theta}$ iff $\theta a=\theta I$,

$$\eqalign{F_{\theta}
&=\bigcap\{\widehat{a}:a\in\Cal F_{\theta}\}
=\bigcap\{\widehat{a}:\theta a=\theta I\}\cr
&=\bigcap\{\widehat{a}:\mu_{\theta}\widehat{a}=\mu_{\theta}(\beta I)\}
=\beta I\setminus\bigcup\{\widehat{a}:\mu_{\theta}\widehat{a}=0\}.\cr}$$

\noindent But this is just the support of $\mu_{\theta}$ (411N),
because $\{\widehat{a}:a\subseteq I\}$ is a base for the topology of
$\beta I$.
}%end of comment

\cmmnt{Thus we see that} $\theta\in M^+$ is purely non-measurable iff
the support\cmmnt{ $F_{\theta}$} of the measure $\mu_{\theta}$ belongs to 
$\Cal A$.   \cmmnt{If you like, $\theta$ is purely non-measurable iff the
support of $\mu_{\theta}$ is `small'.}

\spheader 464Pb\cmmnt{ Yet another corollary of 464C is the
following.}   Since $M$ is a set of real-valued functions on $\Cal PI$,
it has the corresponding topology $\frak T_p$ of pointwise convergence
as a subspace of
$\BbbR^{\Cal PI}$.   Now if $C\subseteq M_{\text{pnm}}$ is countable,
its $\frak T_p$-closure $\overline{C}$ is included in
$M_{\text{pnm}}$.   \prooflet{\Prf\ It is enough to consider the case in
which $C$ is non-empty and $0\notin C$.   For each $\theta\in C$,
$\Cal F_{|\theta|}=\{a:|\theta|(a)=|\theta|(I)\}$ is a filter with outer
measure $1$, so $\Cal F=\{a:|\theta|(a)=|\theta|(I)$ for every
$\theta\in C\}$ also has outer measure $1$, by 464Cb.   Now suppose that
$\theta_0\in\overline{C}$.   If $a\in\Cal F$, then
$|\theta|(I\setminus a)=0$ for every $\theta\in C$, that is,
$\theta b=0$ whenever $\theta\in C$ and $b\subseteq I\setminus a$
(362Ba).   But this means that
$\theta_0b=0$ whenever $b\subseteq I\setminus a$, so
$|\theta_0|(I\setminus a)=0$ and $|\theta_0|(a)=|\theta_0|(I)$.   Thus
$\Cal F_{|\theta_0|}$ includes $\Cal F$, so has outer measure $1$, and
$\theta_0$ also is purely non-measurable.\ \Qed}

\spheader 464Pc If $\theta\in M$ is such that $\theta a=0$ for every
countable set $a\subseteq I$, then $\theta\in M_{\text{pnm}}$.
\prooflet{\Prf\  $\nu$ is inner regular with respect to the family
$\Cal W$ of sets
which are determined by coordinates in countable subsets of $I$, by
254Ob.   But if $W\in\Cal W$ and $\nu W>0$, let $J\subseteq I$
be a countable set such that $W$ is determined by coordinates in $J$;
then $|\theta|(J)=0$, so if $a$ is any member of $W$ we shall have
$a\cup(I\setminus J)\in W\cap\Cal F_{|\theta|}$.   As $W$ is arbitrary,
$\nu^*\Cal F_{|\theta|}=1$ and $\theta$ is purely non-measurable.\ \Qed}

In particular, if $\theta\in M_{\sigma}\cap M_{\tau}^{\perp}$, where
$M_{\sigma}$ is the space of countably additive functionals on
$\Cal PI$\cmmnt{ (362B)}, then $\theta\in M_{\text{pnm}}$.
\cmmnt{(For `ordinary' sets $I$, $M_{\sigma}=M_{\tau}$;  see
438Xa.   But this observation is peripheral to the concerns of the
present section.)}

In the language of (a) above, we have a closed set in $\beta I$, being
$F=\beta I\setminus\bigcup\{\widehat{a}:a\in[I]^{\le\omega}\}$;  and if
$\theta$ is such that the support of $\mu_{\theta}$ is included in $F$,
then $\theta$ is purely non-measurable.

\leader{464Q}{More on measurable functionals (a)} We know that
$M_{\text{m}}$ is a band in $M$, and that it includes the band
$M_{\tau}$.   So it is natural to look at the band
$M_{\text{m}}\cap M_{\tau}^{\perp}$.

\spheader 464Qb If $\theta$ is any non-zero non-negative
functional in $M_{\text{m}}\cap M_{\tau}^{\perp}$, we can find a family
$\langle a_{\xi}\rangle_{\xi<\omega_1}$ in $\Cal PI$ which is
independent in the sense that $\theta(\bigcap_{\xi\in K}a_{\xi})
=2^{-\#(K)}\theta I$ for every non-empty finite $K\subseteq I$.
\prooflet{\Prf\ Choose
the $a_{\xi}$ inductively, observing that at the inductive step we have
to satisfy only countably many conditions of the form
$\theta(a_{\xi}\cap b)=\bover12\theta b$, where $b$ runs over the
subalgebra generated by $\{a_{\eta}:\eta<\xi\}$, and that each such
condition is satisfied $\nu$-a.e., by 464Ka;  so that $\nu$-almost
any $a$ will serve for $a_{\xi}$.\ \Qed}

In terms of the associated measure $\mu_{\theta}$ on $\beta I$, this
means that $\mu_{\theta}$ has Maharam type at least 
$\omega_1$\cmmnt{ (use 331Ja)}.
If $\theta I=1$\cmmnt{, so that $\mu_{\theta}$ is a probability
measure}, then
$\langle(\widehat{a}_{\xi})^{\ssbullet}\rangle_{\xi<\omega_1}$ is an
uncountable stochastically
independent family in the measure algebra of 
$\mu_{\theta}$\cmmnt{ (325Xf)},

Turning this round, we see that if $\lambda$ is a Radon measure on
$\beta I$, of countable Maharam type, and $\lambda I=0$, then the
corresponding functional on $\Cal PI$ is purely non-measurable.

\cmmnt{[For a stronger result in this direction, see 521S
in Volume 5.]}

\spheader 464Qc \cmmnt{Another striking property of measurable
additive functionals is the following.}   If
$\theta\in M_{\text{m}}\cap M_{\tau}^{\perp}$, and $n\in\Bbb N$, then $\theta(a_0\cap a_1\cap\ldots\cap a_n)=2^{-n-1}\theta I$ for $\nu^{n+1}$-almost every
$a_0,\ldots,a_n\subseteq I$, where $\nu^{n+1}$ is the product measure on
$(\Cal PI)^{n+1}$.   \prooflet{\Prf\ For $K\subseteq\{0,\ldots,n\}$
write $\psi_K(a_0,\ldots,a_n)=I\cap\bigcap_{i\in
K}a_i\setminus\bigcup_{i\le n,i\notin K}a_i$ for
$a_0,\ldots,a_n\subseteq I$;  for $\Cal S\subseteq\Cal P\{0,\ldots,n\}$
write $\phi_{\Cal S}(\frak x)=\bigcup_{K\in\Cal S}\psi_K(\frak x)$ for
$\frak x\in(\Cal PI)^{n+1}$.   Then, for any $t\in I$,

\Centerline{$\nu^{n+1}\{\frak x:t\in\phi_{\Cal S}(\frak x)\}
=\sum_{K\in\Cal S}\nu^{n+1}\{\frak x:t\in\phi_K(\frak x)\}
=2^{-n-1}\#(\Cal S)$.}

\noindent Moreover, for different $t$, these sets are independent.   So
if $\#(\Cal S)=2^n$, that is, if $\Cal S$ is just half of
$\Cal P\{0,\ldots,n\}$, then $\phi_{\Cal S}$ will be \imp, by the arguments in 464B.   (In fact 464Bb is the special case $n=2$,
$\Cal S=\{K:\#(K)\ge 2\}$.)

Accordingly we shall have
$\theta(\phi_{\Cal S}(\frak x))=\bover12\theta I$ for
$\nu^{n+1}$-almost every
$\frak x$ whenever $\Cal S\subseteq\Cal P\{0,\ldots,n\}$ has $2^n$ members,
as in 464Ka.   Since there are only finitely many
sets $\Cal S$, the set $E$ is $\nu^{n+1}$-conegligible, where

\Centerline{$E=\{\frak x:\frak x\in(\Cal PI)^{n+1},\,
  \theta(\phi_{\Cal S}(\frak x))=\Bover12\theta I$ whenever
  $\Cal S\subseteq\Cal P\{0,\ldots,n\}$ and $\#(\Cal S)=2^n\}$.}

\noindent But given $\frak x=(a_0,\ldots,a_n)\in E$, let
$K_1,\ldots,K_{2^{n+1}}$ be a listing of $\Cal P\{0,\ldots,n\}$ in such
an order that
$\theta(\psi_{K_i}(\frak x))\le\theta(\psi_{K_j}(\frak x))$ whenever $i\le j$, and consider $\Cal S=\{K_1,\ldots,K_{2^n}\}$; since

\Centerline{$\sum_{i=1}^{2^n}\theta(\psi_{K_i}(\frak x))
=\theta(\phi_{\Cal S}(\frak x))
=\Bover12\theta I
=\Bover12\sum_{i=0}^{2^{n+1}}\theta(\psi_{K_i}(\frak x))$,}

\noindent we must have $\theta(\psi_{K_i}(\frak x))=2^{-n-1}\theta I$
for every $i$.   In particular, $\theta(\psi_{\{0,\ldots,n\}}(\frak
x))=2^{-n-1}\theta I$, that is,
$\theta(\bigcap_{i\le n}a_i)=2^{-n-1}\theta I$.   As this is true
whenever $a_0,\ldots,a_n\in E$, we have the result.\ \Qed
}%end of prooflet

\leader{464R}{A note on $\ell^{\infty}(I)$}\cmmnt{ As already noted in
464F, we have, for any set $I$, a natural additive map
$\chi:\Cal PI\to\ell^{\infty}(I)\cong L^{\infty}(\Cal PI)$, giving rise
to an isomorphism between the $L$-space $M$ of bounded additive
functionals on
$\Cal PI$ and $\ell^{\infty}(I)^*$.}   If we write $\tilde\mu$ for the
image measure $\mu\chi^{-1}$ on $\ell^{\infty}(I)$, where $\mu$ is
Talagrand's measure on $\Cal PI$, and $\tilde\Sigma$ for the domain of
$\tilde\mu$, then\cmmnt{ every member of $\ell^{\infty}(I)^*$ is
$\tilde\Sigma$-measurable, by 464N.   Thus} $\tilde\Sigma$ includes the
cylindrical $\sigma$-algebra of $\ell^{\infty}(I)$\cmmnt{ (4A3T)}.
\cmmnt{We also have a band decomposition $\ell^{\infty}(I)^*
=\ell^{\infty}(I)^*_{\text{m}}\oplus\ell^{\infty}(I)^*_{\text{pnm}}$
corresponding to the decomposition
$M=M_{\text{m}}\oplus M_{\text{pnm}}$ (464M).}

\cmmnt{In this context,} $M_{\tau}$ corresponds to
$\ell^{\infty}(I)^{\times}$\cmmnt{ (363K, as before)}.   
\cmmnt{Since we can
identify $\ell^{\infty}(I)$ with $\ell^1(I)^*=\ell^1(I)^{\times}$
(243Xl), and $\ell^1(I)$, like any
$L$-space, is perfect, $\ell^{\infty}(I)^{\times}$ is the canonical
image of $\ell^1(I)$ in $\ell^{\infty}(I)^*$.}  \dvro{Any}{Because any
functional in $M_{\tau}^{\perp}$ is $\mu$-almost constant (464O),
any} functional in $(\ell^{\infty}(I)^{\times})^{\perp}$ will be
$\tilde\mu$-almost constant.



\exercises{\leader{464X}{Basic exercises (a)}
%\spheader 464Xa
Let $I$ be any set and $\lambda$ a Radon measure on
$\beta I$.   Show that if the support of $\lambda$ is a separable subset
of $\beta I\setminus I$, then the corresponding additive functional on
$\Cal PI$ is purely non-measurable.
%464P

\spheader 464Xb Let $I$ be any set, and $\tilde\mu$ the image of
Talagrand's measure on $\ell^{\infty}(I)$, as in 464R.   Show that
$\tilde\mu$ has a barycenter in $\ell^{\infty}(I)$ iff $I$ is finite.
%464R

\leader{464Y}{Further exercises (a)}
%\spheader 464Ya
Show that there is a sequence $\sequencen{\Cal F_n}$ of distinct
non-principal ultrafilters on $\Bbb N$ with the following property:  if
we define $h(a)=\{n:a\in\Cal F_n\}$ for $a\subseteq\Bbb N$, then
$\{h(a):a\subseteq\Bbb N\}$ is negligible for the usual measure on
$\Cal P\Bbb N$.
%464E  %mt46bits

\spheader 464Yb
Let $\mu$ be Talagrand's measure on $\Cal P\Bbb N$, and
$\lambda$ the corresponding product measure on
$X=\Cal P\Bbb N\times\Cal P\Bbb N$.   Define $\Phi:X\to\ell^{\infty}$ by
setting $\Phi(a,b)=\chi a-\chi b$ for all $a$, $b\subseteq\Bbb N$.
Show that $\Phi$ is Pettis integrable (463Ya) with indefinite Pettis
integral $\Theta$ defined by setting $(\Theta E)(n)=\int_Ef_nd\lambda$,
where $f_n(a,b)=\chi a(n)-\chi b(n)$.   Show that
$K=\{h\Phi:h\in\ell^{\infty*},\,\|h\|\le 1\}$ contains every $f_n$, and
in particular is not $\frak T_m$-compact, so the identity map from
$(K,\frak T_p)$ to $(K,\frak T_m)$ is not continuous.
%464N

\spheader 464Yc Let $I$ be any set.   Write $\pmb{c}_0(I)$ for the closed
linear subspace of $\ell^{\infty}(I)$ consisting of those $x\in\Bbb R^I$
such that $\{t:t\in I$, $|x(t)|\ge\epsilon\}$ is finite for every
$\epsilon>0$;  that is, $C_0(I)$ if $I$ is given its discrete topology
(436I).   Show that, in 464R, $M_{\tau}^{\perp}$ can be identified as
Banach lattice with $(\ell^{\infty}(I)/\pmb{c}_0(I))^*$.
%464R

\spheader 464Yd\dvAnew{2011}(i) Let $\theta:\Cal P\Bbb N\to\Bbb R$
be an additive functional which is $\Tau$-measurable in the
sense of 464I.   Show that
$\{\theta\{n\}:n\in\Bbb N\}$ is bounded.   
(ii) Let $\theta:\Cal P\Bbb N\to\Bbb R$ be an additive functional which is
universally measurable for the usual topology of $\Cal P\Bbb N$.
Show that $\theta$ is bounded.  
(iii) Let $\frak A$ be a
Dedekind $\sigma$-complete Boolean algebra and $\theta:\frak A\to\Bbb R$ an
additive functional which is universally measurable for the
order-sequential topology on $\frak A$
(definition:  393L).   Show that $\theta$ is bounded
and $\theta^+$ is universally measurable.
%464J mt46bits

\spheader 464Ye\dvAnew{2011} Show that there is a $\Tau$-measurable
finitely additive
functional $\theta:\Cal P\Bbb N\to\Bbb R$ which is not bounded.
%464Jb 464Ye out of order query
%query:  Borel measurable => completely additive?
}%end of exercises

\leader{464Z}{Problem} Let $I$ be an infinite set, and $\tilde\mu$ the
image on $\ell^{\infty}(I)$ of Talagrand's measure\cmmnt{ (464R)}.
Is $\tilde\mu$ a topological measure for the
weak topology of $\ell^{\infty}(I)$?

\endnotes{
\Notesheader{464} The central idea of this section appears in 464B:  the
algebraic structure of $\Cal PI$ leads to a variety of \imp\ functions
$\phi$ from powers $(\Cal PI)^K$ to $\Cal PI$.   The simplest of these
is the measure space automorphism $a\mapsto I\setminus a$, as used in
the proofs of 464Ca, 464Ha, 464Ka and 464Kc.   Then we have the map
$(a,b,c)\mapsto(a\cap b)\cup(a\cap c)\cup(b\cap c)$, as in 464Bb, and
the generalization of this in the argument of 464Qc;  and, most
important of all, the map
$\sequencen{\langle a_{ni}\rangle_{i<m(n)}}
\mapsto\bigcup_{n\in\Bbb N}\bigcap_{i<m(n)}a_{ni}$ of 464Ba.   In each
case we can use
probabilistic intuitions to guide us to appropriate formulae, since the
events $t\in\phi(\frak x)$ are always independent, so if they have
probability $\bover12$ for every $t\in I$, the function $\phi$ will be
\imp.   Of course this depends on the analysis of product measures in
\S254.   It means also that we must use the `ordinary' product measure
defined there;  but happily this coincides with the `Radon' product
measure of \S417 (416U).

Talagrand devised his measure when seeking an example of a pointwise
compact set of measurable functions which is not compact for the
topology of convergence in measure, as in 464E.   The remarkable fact
that it is already, in effect, a measure on the cylindrical
$\sigma$-algebra of $\ell^{\infty}$ (464R) became apparent later, and
requires a much more detailed analysis.   An alternative argument not
explicitly involving the Riesz space structure of the space $M$ of
bounded additive functionals may be found in
{\smc Fremlin \& Talagrand 79}.   The proof I give here depends on the surprising fact that, for
non-negative additive functionals, the upper integral
$\overline{\int}\,d\nu$ is additive (464Hb), even though the functionals
may be very far from being measurable.   Once we know this, we can apply
the theory of Banach lattices to investigate the corresponding linear
functional on the $L$-space $M$.   There is a further key step in 464Hd.
We there have a non-negative $\theta\in M_{\tau}^{\perp}$ such that
$\theta I=1$ and
$\beta\theta'I\le\overline{\int}\theta'd\nu\le\gamma\theta'I$ whenever
$0\le\theta'\le\theta$.   It is easy to deduce that
$\{a:\alpha\theta b\le\theta(a\Bcap b)\le\gamma\theta b$
for every $b\in K\}$ has outer
measure $1$ for every finite $K\subseteq\Cal PI$ and $\alpha<\beta$.
If $\alpha$ and
$\gamma$ are very close, this means that there will be many families
$(a,b,c)$ in $\Cal PI$ which, as measured by $\theta$, look like
independent sets of measure $\gamma$, so that $\theta((a\cap
b)\cup(a\cap c)\cup(b\cap c))\bumpeq 3\gamma^2-2\gamma^3$.   But since
we must also have $\theta((a\cap b)\cup(a\cap c)\cup(b\cap c))\le\gamma$
almost everywhere, we can get information about the possible values of
$\gamma$.

Once having noted this remarkable dichotomy between `measurable' and
`purely non-measurable' functionals, it is natural to look for other
ways in which they differ.   The position seems to be that `simple'
Radon measures on $\beta I\setminus I$ (e.g., all measures with
separable support (464Xa) or countable Maharam type (464Qb)) have to
correspond to purely non-measurable functionals.   Of course the
simplest possible measures on $\beta I$ are those concentrated on $I$,
which give rise to functionals which belong to $M_{\tau}$ and are
therefore measurable;  it is functionals in
$M_{\text{m}}\cap M_{\tau}^{\perp}$ which have to give rise to
`complicated' Radon measures.

The fact that every element of $M_{\tau}^{\perp}$ is almost constant for
Talagrand's measure leads to an interesting Pettis integrable function
(464Yb).
The suggestion that Talagrand's measure on $\ell^{\infty}$ might be a
topological measure for the weak topology (464Z) is a bold one, but no
more outrageous than the suggestion that it might measure every
continuous linear functional once seemed.   Talagrand's measure does of
course measure every Baire set for the weak topology (4A3U).
I note here that the usual measure on $\{0,1\}^I$, when transferred to
$\ell^{\infty}(I)$, is actually a Radon measure for the weak*
topology $\frak T_s(\ell^{\infty},\ell^1)$, because on $\{0,1\}^I$ this
is just the ordinary topology.
}%end of notes

\discrpage


