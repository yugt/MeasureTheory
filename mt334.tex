\frfilename{mt334.tex}
\versiondate{26.9.08}
\copyrightdate{1995}

\def\chaptername{Maharam's theorem}
\def\sectionname{Products}

\newsection{334}

I devote a short section to results on the Maharam classification of the
measure algebras of product measures, or, if you prefer, of the free
products of measure algebras.   The complete classification, even for
probability algebras, is complex\cmmnt{ (334Xc, 334Ya)}, so I content
myself
with a handful of the most useful results.   I start with upper bounds
for the Maharam type of the c.l.d.\ product of two measure spaces (334A)
and the localizable measure algebra free product of two semi-finite
measure algebras (334B), and go on to the corresponding results for
general
products of probability spaces and algebras (334C-334D).   Finally, I
show that any infinite power of a probability space is \Mth\ (334E).

In this section I will write $\tau(\mu)$ for the Maharam type of a measure
$\mu$, defined as the Maharam type of its measure algebra (331Fc).

\leader{334A}{Theorem} Let $(X,\Sigma,\mu)$ and $(Y,\Tau,\nu)$
be measure spaces, and
$\lambda$ the c.l.d.\ product measure on $X\times Y$.
Then $\tau(\lambda)\le\max(\omega,\tau(\mu),\tau(\nu))$.

\proof{ Let $\frak A$, $\frak B$, $\frak C$ be the measure algebras of
$\mu$, $\nu$ and $\lambda$, respectively.
Recall from 325A that we have order-continuous Boolean
homomorphisms $\varepsilon_1:\frak A\to\frak C$ and
$\varepsilon_2:\frak B\to\frak C$ defined by setting
$\varepsilon_1(E^{\ssbullet})=(E\times Y)^{\ssbullet}$,
$\varepsilon_2(F^{\ssbullet})=(X\times F)^{\ssbullet}$
for $E\in\Sigma$ and $F\in\Tau$.    Let $A\subseteq\frak A$,
$B\subseteq\frak B$ be $\tau$-generating sets with
$\#(A)=\tau(\frak A)=\tau(\mu)$, $\#(B)=\tau(\nu)$;  set
$C=\varepsilon_1[A]\cup\varepsilon_2[B]$.   Then $C\,\,\tau$-generates
$\frak C$.   \Prf\ Let $\frak C_1$ be the order-closed subalgebra of
$\frak C$ generated by $C$.   Because $\varepsilon_1$ is
order-continuous, $\varepsilon_1^{-1}[\frak C_1]$ is an
order-closed subalgebra of $\frak A$, and it includes $A$, so must be
the whole of $\frak A$;  thus $\varepsilon_1a\in\frak C_1$ for every
$a\in\frak A$.   Similarly, $\varepsilon_2b\in\frak C_1$ for every
$b\in\frak B$.

This means that

\Centerline{$\Lambda_1=\{W:W\in\Lambda,\,W^{\ssbullet}\in\frak C_1\}$}

\noindent contains $E\times F$ for every $E\in\Sigma$, $F\in\Tau$.
Also $\Lambda_1$ is a $\sigma$-algebra of subsets of $X\times Y$,
because $\frak C_1$ is (sequentially) order-closed in $\frak C$.   So
$\Lambda_1\supseteq\Sigma\tensorhat\Tau$ (definition:  251D).   But this
means that
if $W\in\Lambda$ there is a $V\in\Lambda_1$ such that $V\subseteq W$ and
$\lambda V=\lambda W$ (251Ib);  that is, if $c\in\frak C$ there
is a $d\in\frak C_1$ such that
$d\Bsubseteq c$ and $\bar\lambda d=\bar\lambda c$.   Thus $\frak C_1$ is
order-dense in $\frak C$, and

\Centerline{$c=\sup\{d:d\in\frak C_1,\,d\Bsubseteq c\}\in\frak C_1$}

\noindent for every $c\in\frak C$.   So $\frak C_1=\frak C$ and
$C\,\,\tau$-generates $\frak C$, as claimed.\ \Qed

Consequently

\Centerline{$\tau(\lambda)=\tau(\frak C)\le\#(C)
\le\max(\omega,\tau(\mu),\tau(\nu))$.}
}%end of proof of 334A

\leader{334B}{Corollary} Let $(\frak A,\bar\mu)$,
$(\frak B,\bar\nu)$ be
semi-finite measure algebras, with localizable measure algebra free
product $(\frak C,\bar\lambda)$\cmmnt{ (325E)}.   Then
$\tau(\frak C)\le\max(\omega,\tau(\frak A),\tau(\frak B))$.

\proof{ By the construction of part (a) of the proof of 325D, $\frak C$
can be regarded as the measure algebra of the c.l.d.\ product of the
Stone representations of $(\frak A,\bar\mu)$ and
$(\frak B,\bar\nu)$;  so the result follows at once from 334A.
}%end of proof of 334B

\vleader{36pt}{334C}{Theorem} Let
$\langle(X_i,\Sigma_i,\mu_i)\rangle_{i\in I}$ be a family of probability
spaces, with product $(X,\Lambda,\lambda)$.   Then

\Centerline{$\tau(\lambda)
\le\max(\omega,\#(I),\sup_{i\in I}\tau(\mu_i))$.}

\proof{ For $i\in I$, let $\frak A_i$ be the measure algebra of $\mu_i$;
let $\frak C$ be the measure algebra of $\lambda$.
Recall from 325I that we have order-continuous Boolean
homomorphisms $\varepsilon_i:\frak A_i\to\frak C$ corresponding to the
inverse-measure-preserving maps $x\mapsto\pi_i(x)=x(i):X\to X_i$.
For each $i\in I$, let $A_i\subseteq\frak A_i$ be a set of cardinal
$\tau(\mu_i)$ which $\tau$-generates $\frak A_i$.   Set
$C=\bigcup_{i\in I}\varepsilon_i[A_i]$.   Then $C\,\,\tau$-generates
$\frak C$.
\Prf\ Let $\frak C_1$ be the order-closed subalgebra of $\frak C$
generated by $C$.   Because $\varepsilon_i$ is
order-continuous, $\varepsilon_i^{-1}[\frak C_1]$ is an
order-closed subalgebra of $\frak A_i$, and it includes $A_i$, so must
be the whole of $\frak A_i$;  thus $\varepsilon_ia\in\frak C_1$ for every
$a\in\frak A_i$, $i\in I$.

This means that

\Centerline{$\Lambda_1=\{W:W\in\Lambda,\,W^{\ssbullet}\in\frak C_1\}$}

\noindent contains $\pi_i^{-1}[E]$ for every $E\in\Sigma_i$, $i\in I$.
Also $\Lambda_1$ is a $\sigma$-algebra of subsets of $X$, because
$\frak C_1$ is (sequentially) order-closed in $\frak C$.   So
$\Lambda_1\supseteq\Tensorhat_{i\in I}\Sigma_i$.   But this means
that if $W\in\Lambda$ there is a $V\in\Lambda_1$ such that
$V^{\ssbullet}=W^{\ssbullet}$ (254Ff), that is, that
$\frak C_1=\frak C$, and $C\,\,\tau$-generates $\frak C$, as claimed.\
\Qed

Consequently

\Centerline{$\tau(\lambda)=\tau(\frak C)\le\#(C)
\le\max(\omega,\#(I),\sup_{i\in I}\tau(\mu_i))$.}
}%end of proof of 334C

\leader{334D}{Corollary} Let
$\langle(\frak A_i,\bar\mu_i)\rangle_{i\in I}$
be a family of probability algebras, with probability algebra free
product $(\frak C,\bar\lambda)$.    Then

\Centerline{$\tau(\frak C)
\le\max(\omega,\#(I),\sup_{i\in I}\tau(\frak A_i))$.}

\proof{ See 325J-325K.}


\leader{334E}{}\cmmnt{ I come now to the question of when a
product of probability spaces is \Mth.   I give just one
result in detail, leaving others to the exercises.

\medskip

\noindent}{\bf Theorem} Let $(X,\Sigma,\mu)$ be a probability space
and $I$ an infinite set;  let $\lambda$ be the 
product measure on $X^I$.   Then $\lambda$ is \Mth.   If
$\tau(\mu)=0$ then $\tau(\lambda)=0$;   otherwise
$\tau(\lambda)=\max(\tau(\mu),\#(I))$.

\proof{{\bf (a)} Let $(\frak A,\bar\mu)$ be the measure algebra of $\mu$
and $(\frak C,\bar\lambda)$ the measure algebra of $\lambda$.
If $\tau(\mu)=0$, that is, $\frak A=\{0,1\}$, then
$\frak C=\{0,1\}$ (by 254Fe, or 325Jc, or otherwise), and in this case
is surely homogeneous, with $\tau(\frak C)=0$;  so that $\lambda$ is \Mth\
and $\tau(\lambda)=0$.   So let us suppose
hencforth that $\tau(\mu)>0$.  We have

\Centerline{$\tau(\frak C)
\le\max(\omega,\#(I),\tau(\frak A))=\max(\#(I),\tau(\frak A)$}

\noindent by 334C.

\medskip

{\bf (b)} Fix on $b\in\frak A\setminus\{0,1\}$.   For each $i\in I$, let
$\varepsilon_i:\frak A\to\frak C$ be the canonical measure-preserving
homomorphism corresponding to the \imp\ function
$x\mapsto x(i):X^I\to X$.   For each $n\in\Bbb N$, there is a set
$J\subseteq I$ of cardinal
$n$, and now the finite subalgebra of $\frak C$ generated by
$\{\varepsilon_ib:i\in J\}$ has atoms of measure at most $\delta^n$, where
$\delta=\max(\bar\mu b,1-\bar\mu b)<1$.   Consequently $\frak C$ can
have no
atom of measure greater than $\delta^n$, for any $n$, and is therefore
atomless.

\medskip

{\bf (c)} Because $I$ is infinite, there is a bijection between $I$ and
$I\times\Bbb N$;  that is, there is a partition 
$\langle J_i\rangle_{i\in I}$ of $I$ into countably infinite sets.   Now
$(X^I,\lambda)$ can be identified with the product of the family
$\langle(X^{J_i},\lambda_i)\rangle_{i\in I}$, where $\lambda_i$
is the product measure on $X^{J_i}$ (254N).   By (b), every $\lambda_i$
is atomless, so there are sets $E_i\subseteq X^{J_i}$ of measure
$\bover12$.   The sets $E'_i=\{x:x\restr J_i\in E_i\}$ are now
stochastically independent in $X$.   Accordingly we have an \imp\
function $f:X\to\{0,1\}^I$, endowed with its usual measure $\nu_I$,
defined by setting $f(x)(i)=1$ if $x\in E'_i$, $0$ otherwise, and
therefore a measure-preserving Boolean homomorphism 
$\pi:\frak B_I\to\frak C$, writing $\frak B_I$ for the measure algebra of
$\nu_I$.

Now if $c\in\frak C\setminus\{0\}$ and $\frak C_c$ is the corresponding
ideal, $b\mapsto c\Bcap\pi b:\frak B_I\to\frak C_c$ is an
order-continuous Boolean homomorphism.   It follows that
$\tau(\frak C_c)\ge\#(I)$ (331Jb).

\medskip

{\bf (d)} Again take any non-zero $c\in\frak C$.   For each $i\in I$,
set $a_i=\inf\{a:\varepsilon_ia\Bsupseteq c\}$.
Writing $\frak A_{a_i}$ for
the corresponding principal ideal of $\frak A$, we have an
order-continuous Boolean homomorphism
$\varepsilon'_i:\frak A_{a_i}\to\frak C_c$, given by the formula

\Centerline{$\varepsilon'_ia=\varepsilon_ia\Bcap c$
for every $a\in\frak A_{a_i}$.}

\noindent Now $\varepsilon'_i$ is injective, so is a Boolean isomorphism
between $\frak A_{a_i}$ and its image $\varepsilon'_i[\frak A_{a_i}]$,
which by 314F(a-i) is a closed subalgebra of $\frak C_c$.   So

\Centerline{$\tau(\frak A_{a_i})
=\tau(\varepsilon'_i[\frak A_{a_i}])\le\tau(\frak C_c)$}

\noindent by 332Tb.

For any finite $J\subseteq I$,

\Centerline{$0<\bar\lambda c\le\bar\lambda(\inf_{i\in J}\varepsilon_ia_i)
=\prod_{i\in J}\bar\lambda(\varepsilon_ia_i)=\prod_{i\in J}\bar\mu a_i$.}

\noindent So for any $\delta<1$, $\{i:\bar\mu a_i\le\delta\}$ must be
finite, and $\sup_{i\in I}\bar\mu a_i=1$.   In particular,
$\sup_{i\in I}a_i=1$ in $\frak A$.   But this means that if $\zeta$ is
any cardinal
such that the Maharam-type-$\zeta$ component $e_{\zeta}$ of $\frak A$ is
non-zero, then $e_{\zeta}\Bcap a_i\ne 0$ for some $i\in I$, so that

\Centerline{$\zeta\le\tau(\frak A_{e_{\zeta}\Bcap a_i})
\le\tau(\frak A_{a_i})\le\tau(\frak C_c)$.}

\noindent As $\zeta$ is arbitrary,
$\tau(\frak A)\le\max(\omega,\tau(\frak C_c))$ (332S).

\medskip

{\bf (e)} Putting (a)-(d) together, we have

\Centerline{$\max(\tau(\frak A),\#(I))\le\max(\omega,\tau(\frak C_c))
=\tau(\frak C_c)\le\tau(\frak C)\le\max(\tau(\frak A),\#(I))$}

\noindent for every non-zero $c\in\frak C$;  so $\frak C$ is
homogeneous, with $\tau(\frak C)=\max(\tau(\frak A),\#(I))$.   
Re-stating this in terms of $\lambda$ and $\mu$, $\lambda$ is \Mth\
and $\tau(\lambda)=\max(\tau(\mu),\#(I))$.
}%end of proof of 334E

\exercises{
\leader{334X}{Basic exercises (a)}
%spheader 334Xa
Let $(X,\Sigma,\mu)$ and
$(Y,\Tau,\nu)$ be complete locally determined measure spaces with c.l.d.\
product $(X\times Y,\Lambda,\lambda)$.   Show that
if $\nu Y>0$ then $\tau(\mu)\le\tau(\lambda)$.
%334A

\sqheader 334Xb Let $\langle(\frak A_i,\bar\mu_i)\rangle_{i\in I}$ be a
family of probability algebras, with probability algebra free product
$(\frak C,\bar\lambda)$.    Show that $\tau(\frak A_i)\le\tau(\frak C)$
for every $i$, and that

\Centerline{$\#(\{i:i\in I,\,\tau(\frak A_i)>0\})\le\tau(\frak C)$.}
%334C

\spheader 334Xc Let $(X,\Sigma,\mu)$ and $(Y,\Tau,\nu)$ be
$\sigma$-finite measure spaces, and $\lambda$ the product measure on
$X\times Y$.   Show that $\lambda$ is \Mth\ iff one of
$\mu$, $\nu$ is \Mth\ with Maharam type at least
as great as the Maharam type of the other.
%334E

\spheader 334Xd Show that the product of any family of \Mth\ probability
spaces is again \Mth.
%334E

\sqheader 334Xe Let $(X,\Sigma,\mu)$ be a probability space of
Maharam type $\kappa$, and $I$ any set of cardinal at least
$\max(\omega,\kappa)$.   Show that the product measure on
$X\times\{0,1\}^I$ is \Mth, with Maharam type $\#(I)$.
%334E

\leader{334Y}{Further exercises (a)} Let
$\langle(X_i,\Sigma_i,\mu_i)\rangle_{i\in I}$ be an infinite family
of probability spaces, with product $(X,\Lambda,\lambda)$.   Let
$\kappa_i$ be the Maharam type of $\mu_i$ for each $i$;  set
$\kappa=\max(\#(I),\sup_{i\in I}\kappa_i)$.  Show that {\it either}
$\lambda$ is \Mth, with Maharam type $\kappa$, {\it
or} there are $\kappa'<\kappa$, $X'_i\in\Sigma_i$ such that
$\sum_{i\in I}\mu_i(X_i\setminus X'_i)<\infty$, the Maharam type of
the subspace measure on $X'_i$ is at most $\kappa'$ for every $i\in I$
and $\#(\{i:\kappa_i\ne 0\})\le\kappa'$.
}%end of exercises

\cmmnt{
\Notesheader{334} The results above are all very natural ones;  I have
spelt them out partly for completeness and partly for the sake of an
application in \S346 below.
%346G uses 334E
But note the second alternative in
334Ya;  it is possible, even in an infinite product, for a kernel of
relatively small Maharam type to be preserved.

}%end of notes

\frnewpage

