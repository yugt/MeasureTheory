\frfilename{mt449.tex}
\versiondate{13.6.13}
\copyrightdate{2003}

\def\chaptername{Topological groups}
\def\sectionname{Amenable groups}

\def\length{\mathop{\text{length}}\nolimits}
\def\BbbZ{\mathchoice{\hbox{$\Bbb Z$\hskip0.02em}}
  {\hbox{$\Bbb Z$\hskip0.02em}}
  {\hbox{$\scriptstyle\Bbb Z$\hskip0.04em}}
  {\hbox{$\scriptscriptstyle\Bbb Z$\hskip0.04em}}}
\def\trs{^{\top}}

\newsection{449}

I end this chapter with a brief introduction to `amenable' topological
groups.   I start with the
definition (449A) and straightforward results assuring us that there
are many amenable groups (449C).   At a slightly deeper level we have a
condition for a group to be amenable in terms of a universal object
constructible from the group, not invoking `all compact Hausdorff
spaces' (449E).   I give some notes on amenable locally compact
groups, concentrating on a long list of properties equivalent to
amenability (449J), and a version of Tarski's theorem characterizing
amenable discrete groups (449M).   I end with Banach's theorem on extending
Lebesgue measure in one and two dimensions.

\leader{449A}{Definition} A topological group $G$ is {\bf amenable} if
whenever $X$ is a non-empty compact Hausdorff space and $\action$ is a
continuous action of $G$ on $X$, then there is a $G$-invariant
Radon probability measure on $X$.

\cmmnt{
\medskip

\noindent{\bf Warning}:  other definitions have been used, commonly
based on conditions equivalent to amenability for locally compact
Hausdorff groups, such as those listed in 449J(ii)-449J(xiv).   In
addition, many authors use the phrase `amenable group' to
mean a group which is amenable in its discrete topology.   The danger
of this to the non-specialist is that many theorems concerning amenable
discrete groups do not generalize in the ways one might expect.}

\leader{449B}{Lemma} Let $G$ be a topological group, $X$ a locally
compact Hausdorff space, and $\action$ a continuous action of $G$ on
$X$.

(a) Writing $C_0$ for the Banach space of continuous real-valued
functions on $X$ vanishing at $\infty$\cmmnt{ (436I)}, the map
$a\mapsto a^{-1}\action f:C_0\to C_0$\cmmnt{ (definition:  4A5Cc)}
is uniformly continuous for the right uniformity on $G$ and the norm
uniformity of $C_0$, for any $f\in C_0$.

(b) If $\mu$ is a $G$-invariant Radon measure on $X$ and
$1\le p<\infty$, then
$a\mapsto a^{-1}\action u:G\to L^p$\cmmnt{ (definition: 441Kc)} is
uniformly
continuous for the right uniformity on $G$ and the norm uniformity of
$L^p=L^p(\mu)$, for any $u\in L^p$.

\proof{{\bf (a)(i)} Note first that if $a\in G$ and $f\in C_0$, then
$x\mapsto a\action x:X\to X$ is a homeomorphism (4A5Bd), so
$x\mapsto f(a\action x)$ belongs to $C_0$;  but this is just the
function $a^{-1}\action f$.

\medskip

\quad{\bf (ii)} For any $\epsilon>0$ and $f\in C_0$ there is a
neighbourhood $V$ of the identity $e$ of $G$ such that
$\|f-a^{-1}\action_lf\|_{\infty}\le\epsilon$ for every $a\in V$.
\Prf\Quer\
Suppose, if possible, otherwise.   For each symmetric neighbourhood
$V$ of $e$ set

\Centerline{$Q_V
=\{(a,x):a\in V,\,x\in X,\,|f(x)|\ge\Bover{\epsilon}2,\,
 |f(x)-f(a\action x)|\ge\epsilon\}$.}

\noindent We are supposing that there are $a\in V$ and $x\in X$ such
that $|f(x)-f(a\action x)|\ge\epsilon$.   If
$|f(x)|\ge\bover12\epsilon$
then $(a,x)\in Q_V$.   Otherwise, $|f(a\action
x)|\ge\bover12\epsilon$,
$a^{-1}\in V$ and $|f(a\action x)-f(a^{-1}a\action x)|\ge\epsilon$, so
$(a^{-1},a\action x)\in Q_V$.   Thus $Q_V$ is never empty.
Since $Q_V\subseteq Q_{V'}$
whenever $V\subseteq V'$, there is an ultrafilter $\Cal F$ on
$G\times X$ such that $Q_V\in\Cal F$ for every neighbourhood $V$ of
$e$.
Setting $\pi_1(a,x)=a$ and $\pi_2(a,x)=x$ for $(a,x)\in G\times X$, we
see that the image filter $\pi_1[[\Cal F]]$ contains every
neighbourhood
of $e$, so converges to $e$, while $\pi_2[[\Cal F]]$ contains the
compact set $\{x:|f(x)|\ge\bover12\epsilon\}$, so must have a limit
$x_0$ in $X$.   So $\Cal F\to(e,x_0)$ in
$G\times X$.   Next, because the action is continuous,
$\action[[\Cal F]]\to e\action x_0=x_0$, and there must be an
$F\in\Cal F$ such that
$|f(x_0)-f(a\action x)|\le\bover13\epsilon$ for
every $(a,x)\in F$.   Also, of course, there is an $F'\in\Cal F$ such
that $|f(x_0)-f(x)|\le\bover13\epsilon$ whenever $(a,x)\in F'$.   But
now there is an $(a,x)\in Q_G\cap F\cap F'$, and we have

\Centerline{$|f(x)-f(a\action x)|\ge\epsilon$,
\quad$|f(x_0)-f(a\action x)|\le\bover13\epsilon$,
\quad$|f(x_0)-f(x)|\le\bover13\epsilon$}

\noindent simultaneously, which is impossible.\ \Bang\Qed

Now we find that if $a$, $b\in G$, $ab^{-1}\in V$ and $x\in X$, then

\Centerline{$|(a^{-1}\action f)(x)-(b^{-1}\action f)(x)|
=|f(a\action x)-f(b\action x)|
=|f(ab^{-1}\action(b\action x))-f(b\action x)|
\le\epsilon$.}

\noindent As $x$ is arbitrary,
$\|a^{-1}\action f-b^{-1}\action f\|_{\infty}\le\epsilon$;  as
$\epsilon$ is arbitrary, $a\mapsto a^{-1}\action f$ is uniformly
continuous for the right uniformity.

\medskip

{\bf (b)(i)} Suppose that $f:X\to\Bbb R$ is continuous and has compact
support $K=\overline{\{x:f(x)\ne 0\}}$.
Let $H\supseteq K$ be an open set of finite measure.
Then $V_0=\{a:a\in G,\,a\action x\in H$ for every $x\in K\}$ is a
neighbourhood of $e$.   \Prf\ If we take a continuous function $f_0$
with compact support such that $\chi K\le f_0\le\chi H$
(4A2G(e-i)), then
$V_0\supseteq\{a:\|f_0-a^{-1}\action f_0\|_{\infty}<1\}$, which
is a neighbourhood of $e$ by (a).\ \QeD\  Let $\epsilon>0$.   By (a)
again, there is a symmetric neighbourhood $V_1$ of $e$ such that
$(\|f-a^{-1}\action f\|_{\infty})^p\mu H\le\epsilon^p$ for every
$a\in V_1$;  we
may suppose that $V_1\subseteq V_0$.   If $a\in V_1$,
$f(x)=f(a\action x)=0$ for every $x\in X\setminus H$, so that

\Centerline{$\|f-a^{-1}\action f\|_p^p
=\biggerint_H|f-a^{-1}\action f|^pd\mu
\le(\|f-a^{-1}\action f\|_{\infty})^p\mu H\le\epsilon^p$.}

Now suppose that $a$, $b\in G$ and that $ab^{-1}\in V_1$.   Then

$$\eqalignno{\|a^{-1}\action f-b^{-1}\action f\|_p^p
&=\int|f(a\action x)-f(b\action x)|^p\mu(dx)\cr
&=\int|f(a\action(b^{-1}\action x))
  -f(b\action(b^{-1}\action x))|^p\mu(dx)\cr
\displaycause{because $\mu$ is $G$-invariant, see 441L}
&=\int|ba^{-1}\action f-f)|^pd\mu
\le\epsilon^p,\cr}$$

\noindent and $\|a^{-1}\action f-b^{-1}\action f\|_p\le\epsilon$.   As
$\epsilon$ is
arbitrary, $a\mapsto(a^{-1}\action f)^{\ssbullet}$ is uniformly
continuous for the right uniformity.

\medskip

\quad{\bf (ii)} In general, given $u\in L^p(\mu)$ and $\epsilon>0$,
there is an $f\in C_k(X)$ such that
$\|u-f^{\ssbullet}\|_p\le\epsilon$
(416I).   Let $V$ be a neighbourhood of $e$ such that
$\|a^{-1}\action f-a^{-1}\action f\|_{\infty}\le\epsilon$ whenever
$ab^{-1}\in V$;  then
$\|a^{-1}\action u-(a^{-1}\action f)^{\ssbullet}\|_p
=\|u-f^{\ssbullet}\|_p$ (because $\mu$ is $G$-invariant), so

\Centerline{$\|a^{-1}\action u-b^{-1}\action u\|_p
\le\|a^{-1}\action u-a^{-1}\action f^{\ssbullet}\|_p
+\|a^{-1}\action f-b^{-1}\action f\|_p
+\|b^{-1}\action f^{\ssbullet}-b^{-1}\action u\|_p
\le 3\epsilon$}

\noindent whenever $ab^{-1}\in V$ (using 441Kc).   As $\epsilon$ is
arbitrary,
$a\mapsto a^{-1}\action u$ is uniformly continuous for the right
uniformity.   This completes the proof.
}%end of proof of 449B

\leader{449C}{Theorem} (a) Let $G$ and $H$ be topological groups such
that there is a continuous surjective homomorphism from $G$ onto $H$.
If $G$ is amenable, so is $H$.

(b) Let $G$ be a topological group and suppose that there is a dense
subset $A$ of $G$ such that every finite subset of $A$ is included in
an amenable subgroup of $G$.    Then $G$ is amenable.

(c) Let $G$ be a topological group and $H$ a normal subgroup of $G$.
If $H$ and $G/H$ are both amenable, so is $G$.

(d) Let $G$ be a topological group with two amenable subgroups $H_0$
and
$H_1$ such that $H_0$ is normal and $H_0H_1=G$.   Then $G$ is
amenable.

(e) The product of any family of amenable topological groups is
amenable.

(f) Any abelian topological group is amenable.

(g) Any compact Hausdorff topological group is amenable.

\proof{{\bf (a)} Let $\phi:G\to H$ be a continuous surjective
homomorphism.   Let $X$ be a non-empty compact Hausdorff space and
$\action:H\times X\to X$ a continuous action.   For $a\in G$ and
$x\in X$, set $a\action_1x=\phi(a)\action x$.   Then $\action_1$ is a
continuous action of $G$ on $X$, so there is a $G$-invariant Radon
probability measure $\mu$ on $X$.   Because $\phi[G]=H$, $\mu$ is also
$H$-invariant;  as $X$ and $\action$ are arbitrary, $H$ is amenable.

\medskip

{\bf (b)(i)} Let $X$ be a non-empty compact Hausdorff space and
$\action$ a continuous action of $G$ on $X$.   Let $P=P_{\text{R}}(X)$
be the set of
Radon probability measures on $X$ with the narrow topology (437Jd),
so that $P$
is a compact Hausdorff space (437R(f-ii));  recall that in this context the
vague and narrow topologies coincide (437Kc).   For $a\in G$ and
$x\in X$, set
$T_a(x)=a\action x$, so that $T_a:X\to X$ is a homeomorphism.
For $a\in G$ and $\mu\in P$ write  $a\action\mu$ for the image measure
$\mu T_a^{-1}$, so that $a\action\mu\in P$ (418I);  it is easy to
check that $(a,\mu)\mapsto a\action\mu:G\times P\to P$ is an action.
The point is that it is continuous.   \Prf\ Let $f\in C(X)$, $a_0\in
G$,
$\mu_0\in P$ and $\epsilon>0$.   By 449Ba, there is a neighbourhood
$V$
of $a_0$ in $G$ such that
$\|a^{-1}\action f-a_0^{-1}\action f\|_{\infty}\le\bover12\epsilon$
for
every $a\in V$.   Next, there is a neighbourhood $W$ of $\mu_0$ in $P$
such that
$|\int a_0^{-1}\action fd\mu-\int a_0^{-1}\action fd\mu_0|
\le\bover12\epsilon$
for every $\mu\in W$.   But now, if $a\in V$ and $\mu\in W$,

$$\eqalignno{|\int fd(a\action\mu)-\int fd(a_0\action\mu_0)|
&=|\int fT_ad\mu-\int fT_{a_0}d\mu_0|\cr
\displaycause{235J}
&=|\int a^{-1}\action fd\mu-\int a_0^{-1}\action fd\mu_0|\cr
&\le|\int a^{-1}\action fd\mu-\int a_0^{-1}\action fd\mu|
  +|\int a_0^{-1}\action fd\mu-\int a_0^{-1}\action fd\mu_0|\cr
&\le\|a^{-1}\action f-a_0^{-1}\action f\|_{\infty}+\Bover12\epsilon
\le\epsilon.\cr}$$

\noindent As $\epsilon$, $a_0$ and $\mu_0$ are arbitrary,
$(a,\mu)\mapsto\int fd(a\action\mu)$ is continuous;  as $f$ is
arbitrary, $\action$ is continuous.\ \Qed

\medskip

\quad{\bf (ii)} Because the topology of $P$ is Hausdorff, it follows
that $Q_a=\{\mu:\mu\in P,\,a\action\mu=\mu\}$ is closed in $P$ for any
$a\in G$, and that  $G_{\mu}=\{a:a\in G,\,a\action\mu=\mu\}$ is closed
in $G$ for any $\mu\in P$.
Now for any finite subset $I$ of $A$ there is an amenable subgroup
$H_I$ of $G$ including $I$.   The restriction of the action to
$H_I\times X$ is a continuous action of $H_I$ on $X$, so has an
$H_I$-invariant Radon
probability measure, and
$\bigcap_{a\in I}Q_a\supseteq\bigcap_{a\in H_I}Q_a$ is non-empty.
Because $P$ is compact, there is a $\mu\in\bigcap_{a\in A}Q_a$.
Since $G_{\mu}$ includes the dense set $A$, it is the whole
of $G$, and $\mu$ is $G$-invariant.   As $X$ and $\action$ are
arbitrary, $G$ is amenable.

\medskip

{\bf (c)} Let $X$ be a compact Hausdorff space and
$\action$ a continuous action of $G$ on $X$.   Let $P$ be the space of
Radon probability measures on $X$ with its narrow topology.   Define
$a\action\mu$, for $a\in G$ and $\mu\in P$, as in (b-i) above, so that
this is a continuous action of $G$ on $P$.   Set
$Q=\{\mu:\mu\in P,\,a\action\mu=\mu$ for every $a\in H\}$;  then $Q$
is a closed subset of $P$ and, because $H$ is amenable, is non-empty,
since it is the set of $H$-invariant Radon probability measures on
$X$.   Next, $b\action\mu\in Q$ for every $\mu\in Q$ and $b\in G$.
\Prf\ If $a\in H$, then

\Centerline{$a\action(b\action\mu)=(ab)\action\mu
=(bb^{-1}ab)\action\mu=b\action((b^{-1}ab)\action\mu)=b\action\mu$,}

\noindent because $H$ is normal, so $b^{-1}ab\in H$.  As $a$ is
arbitrary, $b\action\mu\in Q$.\ \QeD\    Accordingly we have a
continuous action of $G$ on the compact Hausdorff space $Q$.

If $a\in H$ and $b\in G$, then $b\action\mu=(ba)\action\mu$ for every
$\mu\in Q$.   We therefore have a map $\varaction:G/H\times Q\to Q$
defined
by setting $b^{\ssbullet}\varaction\mu=b\action\mu$ whenever $b\in G$
and $\mu\in Q$.   It is easy to check that this is an action.
Moreover, it is continuous, because if
$W\subseteq Q$ is relatively open then $\{(b,\mu):b\action\mu\in W\}$
is open in $G\times Q$, so its image
$\{(b^{\ssbullet},\mu):b^{\ssbullet}\varaction\mu\in W\}$ is open in
$(G/H)\times Q$ (using 4A2B(f-iv)).
Because $G/H$ is amenable, there is a
$(G/H)$-invariant Radon probability measure $\lambda$ on $Q$.

Now consider the formula
$p(f)=\int_Q(\int_X f(x)\mu(dx))\lambda(d\mu)$.   If $f\in C(X)$, then
$\mu\mapsto\int_Xf(x)\mu(dx)$ is continuous for the vague topology on
$Q$, so $p(f)$ is well-defined.   Clearly $p$ is a linear functional,
$p(f)\ge 0$ if $f\ge 0$, and $p(\chi X)=1$;  so there is a Radon
probability measure $\nu$ on $X$ such that $p(f)=\int fd\nu$ for every
$f\in C(X)$ (436J/436K).   If $b\in G$, then, in the language of (b) above,

$$\eqalignno{\int fd(b\action\nu)
&=\int fT_bd\nu
=p(fT_b)
=\int_Q\bigl(\int_X fT_bd\mu\bigr)\lambda(d\mu)\cr
&=\int_Q\int_X fd(b\action\mu)\lambda(d\mu)
=\int_Q\int_X fd(b^{\ssbullet}\varaction\mu)\lambda(d\mu)
=\int_Q\int_X fd\mu\,\lambda(d\mu)\cr
\displaycause{because $\lambda$ is $G/H$-invariant}
&=\int fd\nu\cr}$$

\noindent for every $f\in C(X)$, so that $b\action\nu=\nu$.   Thus
$\nu$
is $G$-invariant.   As $X$ and $\action$ are arbitrary, $G$ is
amenable.

\medskip

{\bf (d)} The canonical map from $H_1$ to $G/H_0$ is a continuous
surjective homomorphism.   By (a), $G/H_0$ is amenable;  by (c), $G$
is
amenable.

\medskip

{\bf (e)} By (c) or (d), the product of two amenable topological
groups
is amenable, since each can be regarded as a normal subgroup of the
product.   It follows that the product of finitely
many amenable topological groups is amenable.   Now let
$\familyiI{G_i}$
be any family of amenable topological groups with product $G$.   For
finite $J\subseteq I$ let $H_J$ be the set of those $a\in G$ such that
$a(i)$ is the identity in $G_i$ for every $i\in I\setminus J$.   Then
$H_J$ is isomorphic (as topological group) to $\prod_{i\in J}G_i$, so
is
amenable.   Since $\{H_J:J\in[I]^{<\omega}\}$ is an upwards-directed
family of subgroups of $G$ with dense union, (b) tells us that $G$ is
amenable.

\medskip

{\bf (f)(i)} The first step is to observe that the group $\Bbb Z$,
with
its discrete topology, is amenable.   \Prf\ Let $X$ be a compact
Hausdorff space and $\action$ a continuous action of $\Bbb Z$ on $X$.
Set $\phi(x)=1\action x$ for $x\in X$.   Then $\phi:X\to X$ is
continuous, so by
437T there is a Radon probability measure $\mu$ on $X$ such that
$\mu$ is equal to the image measure $\mu\phi^{-1}$.   Because $\phi$
is
bijective, we see that, for $E\subseteq X$,
$E\in\dom\mu$ iff $\phi[E]\in\dom(\mu\phi^{-1})=\dom\mu$, and in this
case $\mu\phi[E]=\mu E$;  that is, $\phi^{-1}$, like $\phi$, is \imp.
Now we can induce on $n$ to see
that $\mu(\phi^n)^{-1}$ and $\mu(\phi^{-n})^{-1}$ are equal to $\mu$
for
every $n$.   Since $n\action x=\phi^n(x)$ for every $x\in X$ and
$n\in\Bbb Z$, $\mu$ is $\Bbb Z$-invariant.
As $X$ and $\action$ are arbitrary, $\Bbb Z$ is amenable.\ \Qed

\medskip

\quad{\bf (ii)} Now let $G$ be any abelian topological group.   For
each
finite set $I\subseteq G$ let $\phi_I:\BbbZ^I\to G$ be the continuous
homomorphism defined by setting $\phi_I(z)=\prod_{a\in I}a^{z(a)}$ for
$z\in\BbbZ^I$.   By (i) just above and (e), we know that $\BbbZ^I$
(with
its discrete topology) is amenable, so (a) tells us that the subgroup
$G_I=\phi_I[\BbbZ^I]$ is amenable.   But now
$\{G_I:I\in[G]^{<\omega}\}$
is an upwards-directed family of amenable subgroups of $G$ with union
$G$, so from (b) we see that $G$ is amenable.

\medskip

{\bf (g)} This is immediate from 443Ub.   (See also 449Xe.)
}%end of proof of 449C

\leader{449D}{Theorem} Let $G$ be a topological group.

(a) Write $U$ for the set of bounded real-valued functions on
$G$ which are uniformly continuous for the right uniformity of $G$.
Then $U$ is an $M$-space, and we have an action $\action_l$ of $G$ on
$U$ defined by
the formula $(a\action_lf)(y)=f(a^{-1}y)$ for $a$, $y\in G$ and
$f\in U$.

(b) Let $Z\subseteq\BbbR^U$ be the set of Riesz homomorphisms
$z:U\to\Bbb R$ such that $z(\chi G)=1$.   Then $Z$ is a compact
Hausdorff space, and we have a continuous action of $G$ on $Z$ defined
by the formula $(a\action z)(f)=z(a^{-1}\action_lf)$ for
$a\in G$, $z\in Z$ and $f\in U$.

(c) Setting $\hat a(f)=f(a)$ for $a\in G$ and $f\in U$, the map
$a\mapsto\hat a:G\to Z$ is a continuous function from $G$ onto a dense
subset of $Z$.   If $a$, $b\in G$ then
$a\action\hat b=\widehat{ab}$.

(d) Now suppose that $X$ is a compact Hausdorff space,
$(a,x)\mapsto a\action x$ is a continuous action of $G$ on $X$, and
$x_0\in X$.    Then there is a unique
continuous function $\phi:Z\to X$ such
that $\phi(\hat e)=x_0$ and $\phi(a\action z)=a\action\phi(z)$ for
every $a\in G$ and $z\in Z$.

(e) If $G$ is Hausdorff then the action of $G$ on $Z$ is faithful and
the map $a\mapsto\hat a$ is a homeomorphism between $G$ and its image
in
$Z$.

\proof{{\bf (a)} Because $U$ is a norm-closed Riesz
subspace of $C_b(G)$ containing the constant functions (4A2Jh), it is
an $M$-space.   To see that the given
formula defines an action, we need to check that $a\action_lf$
belongs to $U$ whenever $a\in G$ and $f\in U$.   Of course
$a\action_lf$
is continuous and $\|a\action_lf\|_{\infty}=\|f\|_{\infty}$ is finite.
If $\epsilon>0$ there is a neighbourhood $V$ of the identity $e$ in
$G$
such that
$|f(b)-f(c)|\le\epsilon$ whenever $b$, $c\in G$ and
$bc^{-1}\in V$;  now $a^{-1}Va$ is a neighbourhood of $e$, and if
$bc^{-1}\in aVa^{-1}$ then $(a^{-1}b)(a^{-1}c)^{-1}\in V$, so
$|(a\action_lf)(b)-(a\action_lf)(c)|
=|f(a^{-1}b)-f(a^{-1}c)|\le\epsilon$.   As $\epsilon$ is arbitrary,
$a\action_lf$ is uniformly continuous with respect to the right
uniformity.   Now $\action_l$ is an action, just as in 4A5Cc.

\medskip

{\bf (b)(i)} Because $U$ is an $M$-space with standard order unit
$\chi G$, $Z$ is a compact Hausdorff space and $U$ can be identified,
as normed Riesz space, with $C(Z)$ (354L).
For any $a\in G$, the map $f\mapsto a\action_lf:U\to U$ is a Riesz
homomorphism leaving the constant functions fixed.   So we can define
$a\action z$, for $z\in Z$, by saying that
$(a\action z)(f)=z(a^{-1}\action_lf)$ for any $f\in U$, and
$a\action z$
will belong to $Z$ for any $a\in G$ and $z\in Z$.   As usual, it is
easy to check that this formula defines an action of $G$ on $Z$.

\medskip

\quad{\bf (ii)} $(a,z)\mapsto a\action z$ is continuous.   \Prf\ Take
$a_0\in G$, $z_0\in Z$ and any neighbourhood $W$ of $a_0\action z_0$
in $Z$.   Because $U$ corresponds to the whole of $C(Z)$, and $Z$ is
completely regular, there is an $f\in U^+$ such that
$(a_0\action z_0)(f)=1$ and $z(f)=0$ for every $z\in Z\setminus W$.
Set $W_0=\{z:z(a_0^{-1}\action_lf)>\bover12\}$.
Observe that $z_0(a_0^{-1}\action_lf)=1$, so
$W_0$ is an open subset of $Z$ containing $z_0$.   Next, set
$V_0=\{a:a\in G,\,\|a^{-1}\action_lf-a_0^{-1}\action_lf\|_{\infty}
\le\bover12\}$.
There is a neighbourhood $V$ of $e$ such that
$|f(b)-f(c)|\le\bover12$ whenever $b$, $c\in G$ and
$bc^{-1}\in V$.   If $a\in Va_0$ then
$ab(a_0b)^{-1}=aa_0^{-1}\in V$ so

\Centerline{$|(a^{-1}\action_lf)(b)-(a_0^{-1}\action_lf)(b)|
=|f(ab)-f(a_0b)|\le\Bover12$}

\noindent for every $b\in G$, and $a\in V_0$.   Thus $V_0\supseteq Va_0$
is a neighbourhood of $a_0$.

Now if $a\in V_0$ and $z\in W_0$ we shall have

\Centerline{$(a\action z)(f)=z(a^{-1}\action_lf)
\ge z(a_0^{-1}\action_lf)-\Bover12>0$}

\noindent and $a\action z\in W$.
As $a_0$, $z_0$ and $W$ are arbitrary, the action of $G$ on $Z$ is
continuous.\ \Qed

\medskip

{\bf (c)} Of course $\hat a$, as defined, is a Riesz homomorphism
taking
the correct value at $\chi G$, so belongs to $Z$.   Because
$U\subseteq C(X)$, the map
$a\mapsto\hat a$ is continuous.   \Quer\ If
$\{\hat a:a\in G\}$ is not dense in $Z$, there is a non-zero $h\in
C(Z)$
such that $h(\hat a)=0$ for every $a\in G$;  but as $U$ is identified
with $C(Z)$, there is an $f\in U$ such that $z(f)=h(z)$ for every
$z\in Z$.   In this case, $f$ cannot be the zero function, but
$f(a)=\hat a(f)=h(\hat a)=0$ for every $a\in G$.\ \BanG\  Thus the
image of $G$ is dense, as claimed.

If $a$, $b\in G$ and $f\in U$ then

\Centerline{$(a\action\hat b)(f)
=\hat b(a^{-1}\action_lf)
=(a^{-1}\action_lf)(b)
=f(ab)
=\widehat{ab}(f)$,}

\noindent so $a\action\hat b=\widehat{ab}$.

\medskip

{\bf (d)} We have a Riesz homomorphism $T:C(X)\to\BbbR^G$ defined by
setting $(Tg)(a)=g(a\action x_0)$ for every $g\in C(X)$ and $a\in G$.
Now $Tg\in U$ for every $g\in C(X)$.   \Prf\
$Tg(a)=(a^{-1}\action g)(x_0)$;    since the map
$a\mapsto a^{-1}\action g$ is
uniformly continuous (449Ba), so is $Tg$.
$\|Tg\|_{\infty}\le\|g\|_{\infty}$ is finite, so $Tg\in U$.\ \Qed

Of course $T(\chi X)=\chi G$.   So if $z\in Z$, $zT:C(X)\to\Bbb R$ is
a Riesz homomorphism such that $(zT)(\chi X)=1$.   There is therefore a
unique $\phi(z)\in X$ such that $(zT)(g)=g(\phi(z))$ for every
$g\in C(X)$ (354L again).   Since the function $z\mapsto g(\phi(z))=z(Tg)$
is continuous for every $g\in C(X)$, $\phi$ is continuous.

Now suppose that $a\in G$.   Then $\phi(\hat a)=a\action x_0$.   \Prf\
If $g\in C(X)$, then

\Centerline{$g(\phi(\hat a))=\hat a(Tg)=(Tg)(a)
=g(a\action x_0)$.  \Qed}

\noindent So if $a$, $b\in G$, then

\Centerline{$\phi(a\action\hat b)
=\phi(\widehat{ab})
=(ab)\action x_0
=a\action(b\action x_0)=a\action\phi(b)$.}

\noindent Since $\{\hat b:b\in G\}$ is dense in $Z$, and all the
functions here are continuous, $\phi(a\action z)=a\action\phi(z)$ for
all $a\in G$ and $z\in Z$.

To see that $\phi$ is unique, observe that if $a\in G$ then
$\phi(\hat a)=\phi(\widehat{ae})=\phi(a\action\hat{e})$ must be
$a\action\phi(\hat e)=a\action x_0$;  since $\{\hat a:a\in G\}$ is dense in
$Z$, $X$ is Hausdorff and $\phi$ is declared to be continuous, $\phi$ is
uniquely defined.

\medskip

{\bf (e)} Now suppose that the topology of $G$ is Hausdorff.   Then it
is defined by the bounded uniformly continuous functions (4A2Ja);
the map $a\mapsto\hat a$ is therefore injective and is a homeomorphism
between $G$ and its image in $Z$.   If $a$, $b\in G$ are distinct,
then
$a\action\hat e=\hat a\ne\hat b=b\action\hat e$, so the action is
faithful.
}%end of proof of 449D

\medskip

\noindent{\bf Remark} \dvro{The}{Following {\smc Brook 70}, the} space
$Z$, together
with the canonical action of $G$ on it and the map
$a\mapsto\hat a:G\to Z$, is called the {\bf greatest ambit} of the
topological group $G$.

\leader{449E}{Corollary} Let $G$ be a topological group.   Then the
following are equiveridical:

(i) $G$ is amenable;

(ii) there is a $G$-invariant Radon probability measure on the
greatest ambit of $G$;

(iii) writing $U$ for the space of bounded real-valued functions on
$G$ which are uniformly continuous for the right uniformity, there is
a
positive linear functional $p:U\to\Bbb R$ such that $p(\chi G)=1$ and
$p(a\action_lf)=p(f)$ for every $f\in U$ and $a\in G$.

\proof{ Let $Z$ be the greatest ambit of $G$.

\medskip

{\bf (i)$\Rightarrow$(ii)} As soon as we know that $Z$ is a
compact Hausdorff space and the action of $G$ on $Z$ is continuous
(449Db), this becomes a special case of the definition of `amenable
topological group'.

\medskip

{\bf (ii)$\Rightarrow$(i)} Let $\mu$ be a $G$-invariant Radon
probability measure on $Z$.   Given any continuous action of $G$ on a
non-empty compact
Hausdorff space $X$, fix $x_0\in X$ and let $\phi:Z\to X$ be
a continuous function such that
$\phi(a\action z)=a\action\phi(z)$ for every $a\in G$ and $z\in Z$,
as in 449Dd.   Let $\nu$ be the image measure $\mu\phi^{-1}$.   Then
$\nu$ is a Radon probability measure on $X$ (418I again).
If $F\in\dom\nu$ and $a\in G$, then

\Centerline{$\nu(a\action F)=\mu\phi^{-1}[a\action F]
=\mu(a\action\phi^{-1}[F])=\mu\phi^{-1}[F]=\nu F$.}

\noindent As $a$ and $F$ are arbitrary, $\nu$ is $G$-invariant;  as
$X$
and $\action$ are arbitrary, $G$ is amenable.

\medskip

{\bf (ii)$\Leftrightarrow$(iii)} The identification of $U$ with
$C(Z)$ (see (b-i)
of the proof of 449D) means that we have a one-to-one correspondence
between Radon probability
measures $\mu$ on $Z$ and positive linear functionals $p$ on $U$ such
that $p(\chi G)=1$, given by the formula $p(f)=\int z(f)\mu(dz)$ for
$f\in U$ (436J/436K again).   Now

$$\eqalignno{\mu\text{ is }&G\text{-invariant}\cr
&\iff\int (a\action z)(f)\mu(dz)=\int z(f)\mu(dz)
  \text{ for every }f\in U,\,a\in G\cr
\displaycause{441L}
&\iff\int z(a^{-1}\action_lf)\mu(dz)=\int z(f)\mu(dz)
  \text{ for every }f\in U,\,a\in G\cr
&\iff\int z(a\action_lf)\mu(dz)=\int z(f)\mu(dz)
  \text{ for every }f\in U,\,a\in G\cr
&\iff p(a\action_lf)=p(f)
  \text{ for every }f\in U,\,a\in G.\cr}$$

\noindent So there is a $G$-invariant $\mu$, as required by (ii), iff
there is a $G$-invariant $p$ as required by (iii).
}%end of proof of 449E

\leader{449F}{Corollary} Let $G$ be a topological group.

(a) If $G$ is amenable, then

\quad(i) every open subgroup of $G$ is amenable;

\quad(ii)\dvAformerly{4{}49Xg} every dense subgroup of $G$ is amenable.

(b)\dvAnew{2010}
Suppose that for every sequence $\sequencen{V_n}$ of neighbourhoods of
the identity $e$ of $G$ there is a normal subgroup $H$ of $G$ such that
$H\subseteq\bigcap_{n\in\Bbb N}V_n$ and $G/H$ is amenable.   Then $G$ is
amenable.

\proof{ Write $U_G$ for
the set of bounded real-valued functions on $G$ which are uniformly
continuous for the right uniformity of $G$;  if $H$ is a subgroup of $G$,
let $U_H$ the set of
bounded real-valued functions on $H$ which are uniformly continuous
for the right uniformity of $H$;  and if
$H\normalsubgroup G$, let $U_{G/H}$ the set of
bounded real-valued functions on the quotient
$G/H$ which are uniformly continuous
for the right uniformity of $G/H$.

\medskip

{\bf (a)(i)}\grheada\ Let $H$ be an open subgroup of $G$.
Take a set $A\subseteq G$ meeting
each right coset of $H$ in just one point, so that each member of $G$ is
uniquely expressible as $ya$ where $y\in H$ and $a\in A$.   Define
$T:U_H\to\BbbR^G$ by setting $(Tf)(ya)=f(y)$ whenever $f\in U_H$, $y\in H$
and $a\in A$.   Then $T$ is a positive linear operator.   Also
$T[U_H]\subseteq U_G$.   \Prf\ Let
$f\in U_H$.   Of course $Tf$ is bounded.   If $\epsilon>0$, there is a
neighbourhood $W$ of the identity in $H$ such that
$|f(x)-f(y)|\le\epsilon$ whenever $x$, $y\in H$ and $xy^{-1}\in W$.
Because $H$ is open, $W$ is also a neighbourhood of the identity in
$G$.   Now suppose that $x$, $y\in G$ and $xy^{-1}\in W$.   Express $x$ as
$x_0a$ and $y$ as $y_0b$ where $x_0$, $y_0\in H$ and $a$, $b\in A$.
Then

\Centerline{$x_0ab^{-1}y_0^{-1}\in W\subseteq H$,}

\noindent so $ab^{-1}\in H$ and
$a\in Hb$ and $a=b$ and $x_0y_0^{-1}\in W$ and

\Centerline{$|(Tf)(x)-(Tf)(y)|=|f(x_0)-f(y_0)|\le\epsilon$.}

\noindent As $\epsilon$ is arbitrary, $Tf$ is uniformly continuous and
belongs to $U_G$.\ \Qed

\medskip

\qquad\grheadb\ Next, $b\action_l(Tf)=T(b\action_lf)$ whenever $f\in U_H$
and $b\in H$.   \Prf\ If $x\in G$, express it as $ya$ where $y\in H$ and
$a\in A$.   Then

\Centerline{$(b\action_lTf)(x)=(Tf)(b^{-1}x)=(Tf)(b^{-1}ya)
=f(b^{-1}y)=(b\action_lf)(y)=T(b\action_lf)(x)$.  \Qed}

\medskip

\qquad\grheadc\ By 449E, there is a positive linear functional
$p:U_G\to\Bbb R$
such that $p(\chi G)=1$ and $p(a\action_lf)=p(f)$ whenever $f\in U_G$
and $a\in G$.   Set $q(f)=p(Tf)$ for $f\in U_H$;  then $q$ is a positive
linear operator, $q(\chi H)=1$ and $q$ is $H$-invariant, by (ii).   So
by 449E in the other direction, $H$ is amenable.

\medskip

\quad{\bf (ii)} Now suppose that $H$ is a dense subgroup of $G$.
It is easy to see that the right
uniformity of $H$ is the subspace uniformity induced by the right
uniformity of $G$ (3A4D), so that $f\restr H\in U_H$ for every $f\in U_G$.
In the other direction, if $g\in U_H$,
then $g$ extends uniquely to a member
of $U_G$, by 3A4G;  write $Tg$ for the extension.   In this case,
$b\action_lTg=T(b\action_lg)$ for every $g\in U_H$ and $b\in H$.   \Prf\
$b\action_lTg$ and $T(b\action_lg)$ are continuous, and for $a\in H$,

\Centerline{$(b\action_lTg)(a)=Tg(b^{-1}a)=g(b^{-1}a)
=(b\action_lg)(a)=T(b\action_lg)(a)$;}

\noindent as $H$ is dense in $G$, $b\action_lTg=T(b\action_lg)$.\ \Qed

Now we can use the same argument as in (i-$\gamma$) above to see that $H$
is amenable.

\woddheader{449F}{4}{2}{2}{48pt}

{\bf (b)(i)} Let $\Cal H$ be the family of normal subgroups $H$ of $G$ such
that $G/H$ is amenable.

\medskip

\qquad\grheada\ For $H\in\Cal H$,
let $\pi_H:G\to G/H$ be the canonical homomorphism and
$p_H:U_{G/H}\to\Bbb R$ a
positive linear functional such that $p_H(\chi(G/H))=1$ and
$p_H(c\action_lg)=p_H(g)$ whenever $g\in U_{G/H}$ and $c\in G/H$.
Let $U'_H$ be
$\{f:f\in U_G$, $f(x)=f(y)$ whenever $x$, $y\in G$ and $xy^{-1}\in H\}$.
Then $U'_H$ is a linear subspace of $U_G$ containing $\chi G$.

If $f\in U'_H$
then there is a unique $g\in U_{G/H}$ such that $f=g\pi_H$.   \Prf\
Because $f(x)=f(y)$ whenever $\pi_Hx=\pi_Hy$, and $\pi_H$ is surjective,
there is a unique function $g:G/H\to\Bbb R$ such that $f=g\pi_H$;  because
$f$ is bounded, so is $g$.   Given $\epsilon>0$, there is an open
neighbourhood
$W$ of $e$ such that $|f(x)-f(y)|\le\epsilon$ whenever $xy^{-1}\in W$.
In this case, $\pi_H[W]$ is a neighbourhood of the identity in $G/H$
(4A5J(a-i)).   Suppose that $c_0$, $c_1\in G/H$ are such that
$c_0c_1^{-1}\in\pi_H[W]$.   Then there are $x_0$, $x_1\in G$ and $x\in W$
such that $\pi_Hx_0=c_0$, $\pi_Hx_1=c_1$ and $\pi_Hx=c_0c_1^{-1}$.
As $\pi_H(x_0x_1^{-1})=\pi_Hx$, there is a $y\in H$ such that
$yx_0x_1^{-1}=x$ belongs to $W$;  so that $\pi_H(yx_0)=c_0$ and

\Centerline{$|g_H(c_0)-g_H(c_1)|
=|f(yx_0)-f(x_1)|\le\epsilon$.}

\noindent As $\epsilon$ is arbitrary, $g\in U_{G/H}$.\ \Qed

We therefore have a functional $p'_H:U'_H\to\Bbb R$ defined by setting
$p'_H(g\pi_H)=p_H(g)$ whenever $g\in U_{G/H}$.   Of course $g\ge 0$
whenever $g\pi_H\ge 0$, so $p'_H$ is a positive linear functional, and
$p'_H(\chi G)=1$.

\medskip

\qquad\grheadb\
If $f\in U'_H$ and $a\in G$ then $a\action_lf\in U'_H$ and
$p'_H(a\action_lf)=p'_H(f)$.   \Prf\ Let $g\in U_{G/H}$ be such that
$f=g\pi_H$.   Then

\Centerline{$(a\action_lf)(x)
=f(a^{-1}x)
=g\pi_H(a^{-1}x)
=g(\pi_H(a)^{-1}\pi_H(x))
=(\pi_H(a)\action_lg)(\pi_H(x))$}

\noindent for every $x\in G$;  so $a\action_lf=(\pi_H(a)\action_lg)\pi_H$
belongs to $U'_H$, and

\Centerline{$p'_H(a\action_lf)=p_H(\pi_H(a)\action_lg)
=p_H(g)=p'_H(f)$.  \Qed}

\medskip

\quad{\bf (ii)} For any family $\Cal V$ of neighbourhoods of $e$,
set $\Cal H_{\Cal V}=\{H:H\in\Cal H$, $H\subseteq\bigcap\Cal V\}$.
Now for any $f\in U_G$ there is a countable family $\Cal V$ of
neighbourhoods of $e$ such that $f\in U'_H$ for every
$H\in\Cal H_{\Cal V}$.
\Prf\ For each $n\in\Bbb N$ choose a
neighbourhood $V_n$ of $e$ such that $|f(x)-f(y)|\le 2^{-n}$ whenever
$xy^{-1}\in V_n$, and set $\Cal V=\{V_n:n\in\Bbb N\}$.\ \Qed

\medskip

\quad{\bf (iii)} We are supposing that $\Cal H_{\Cal V}$ is non-empty for
every countable family $\Cal V$ of neighbourhoods of $e$.   There is
therefore an ultrafilter $\Cal F$ on $\Cal H$ containing $\Cal H_{\Cal V}$
for every countable $\Cal V$.   Now we see
from (ii) that for any $f\in U_G$ the set
$\{H:H\in\Cal H$, $f\in U'_H\}$ belongs to $\Cal F$, while
$|p'_H(f)|\le\|f\|_{\infty}$ for any $H$ such that $f\in U'_H$;  so we can
set $p(f)=\lim_{H\to\Cal F}p'_H(f)$ for every $f\in U_G$.    In this case,
of course, $p$ is a positive linear functional and $p(\chi G)=1$.
Also, given $f\in U_G$ and $a\in G$, then
$p'_H(f)=p'_H(a\action_lf)$ whenever $f\in U'_H$, by (i-$\beta$), so
$p(f)=p(a\action_lf)$.   Thus $p$ satisfies (iii) of 449E and $G$ is
amenable.
}%end of proof of 449F

\leader{449G}{Example} Let $F_2$ be the free group on two generators,
with its discrete topology.   Then $F_2$ is a $\sigma$-compact unimodular
locally compact
Polish group.   But it is not amenable.   \prooflet{\Prf\ Let $a$
and $b$ be the generators of $F_2$.   Then every element of $F_2$ is
uniquely expressible as a word (possibly empty) in the letters $a$,
$b$,
$a^{-1}$, $b^{-1}$ in which the letters $a$, $a^{-1}$ are never
adjacent
and the letters $b$, $b^{-1}$ are never adjacent.   Write $A$ for the
set of elements of $F_2$ for which the canonical word does not begin
with either $b$ or $b^{-1}$, and $B$ for the set of elements of $F_2$
for which
the canonical word does not begin with either $a$ or $a^{-1}$.   Then
$A\cup B=F_2$ and $A\cap B=\{e\}$.   \Quer\ Suppose, if possible, that
$F_2$ is amenable.   Every member of $\ell^{\infty}(F_2)$ is uniformly
continuous with
respect to the right uniformity.   So there is an $F_2$-invariant
positive linear functional $p:\ell^{\infty}(F_2)\to\Bbb R$ such that
$p(\chi F_2)=1$.   Let $\nu$ be the corresponding non-negative
additive
functional on $\Cal PF_2$, so that $\nu C=p(\chi C)$ for every
$C\subseteq F_2$.   For $c\in F_2$ and $C\subseteq F_2$,
$c\action_l\chi C=\chi(cC)$, so $\nu(cC)=\nu C$ for every
$C\subseteq F_2$
and $c\in F_2$.   In particular, $\nu(b^nA)=\nu A$ for every
$n\in\Bbb Z$;
but as all the $b^nA$, for $n\in\Bbb Z$, are disjoint, $\nu A=0$.
Similarly $\nu B=0$ and

\Centerline{$0=\nu(A\cup B)=\nu F_2=p(\chi F_2)=1$,}

\noindent which is absurd.\ \BanG\
Thus $F_2$ is not amenable, as claimed.\ \Qed}

\leader{449H}{}\cmmnt{ In this section so far, I have taken care to
avoid assuming that groups are locally compact.   Some of the most
interesting amenable groups are very far from being locally compact
(e.g., 449Xh).
But of course a great deal of work has been done on amenable locally
compact groups.   In particular, there is a remarkable list of equivalent
properties, some of which I will present in the next theorem.
It will be useful to have the following facts to hand.

\medskip

\noindent}{\bf Lemma} Let $G$ be a locally compact Hausdorff topological
group, and $U$ the space of bounded real-valued functions on $G$
which are uniformly continuous for the right uniformity\cmmnt{, as
in 449D-449E}.   Let $\mu$ be a left Haar measure on $G$, and $*$ the
corresponding convolution on $\eusm L^0(\mu)$\cmmnt{ (444O)}.

(a) If $h\in\eusm L^1(\mu)$ and $f\in\eusm L^{\infty}(\mu)$ then
$h*f\in U$.

(b) Let $p:U\to\Bbb R$ be a positive linear functional such that
$p(a\action_lf)=p(f)$ whenever $f\in U$ and $a\in G$.   Then
$p(h*f)=p(f)\int h\,d\mu$ for every $h\in\eusm L^1(\mu)$ and
$f\in U$.

%I suppose this works for groups carrying Haar measures?

\proof{{\bf (a)} Recall that we know from 444Rc that $h*f$ is defined
everywhere in $G$ and is continuous.   For any $x\in G$,

\Centerline{$(h*f)(x)=\int h(xy)f(y^{-1})\mu(dy)
=\int (x^{-1}\action_lh)\times\Reverse{f}$,}

\noindent where $\Reverse{f}(y)=f(y^{-1})$ whenever $y^{-1}\in\dom f$
(4A5C(c-ii)).   By 449Bb, applied to the left action of $G$ on itself,
$x\mapsto(x^{-1}\action_lh)^{\ssbullet}:G\to L^1(\mu)$ is uniformly
continuous for
the right uniformity of $G$ and the norm uniformity of $L^1(\mu)$.
Since $u\mapsto\int u\times v:L^1(\mu)\to\Bbb R$ is uniformly
continuous for every $v\in L^{\infty}(\mu)$, $x\mapsto(h*f)(x)
=\int(x^{-1}\action_lh)^{\ssbullet}\times\Reverse{f}^{\ssbullet}$ is
uniformly continuous for the right uniformity (3A4Cb).
Of course $\sup_{x\in G}|(h*f)(x)|\le\|h\|_1\|f\|_{\infty}$ is finite,
so $h*f\in U$.

\medskip

{\bf (b)} Let $\epsilon>0$.   Then there are a compact set
$K\subseteq G$ such that $\int_{G\setminus K}|h|d\mu\le\epsilon$
(412Je) and a symmetric open neighbourhood $V_0$ of $e$ such that
$|f(x)-f(y)|\le\epsilon$ whenever $xy^{-1}\in V_0$.   Let
$a_0,\ldots,a_n\in G$ be such that
$K\subseteq\bigcup_{i\le n}a_iV_0$, and set
$E_i=a_iV_0\setminus\bigcup_{j<i}a_jV_0$, $\alpha_i=\int_{E_i}h\,d\mu$
for each $i\le n$ and $F=G\setminus\bigcup_{j\le n}a_jV_0$.
If $x\in G$ and $y\in E_i$, then
$y^{-1}x(a_i^{-1}x)^{-1}=(a_i^{-1}y)^{-1}$ belongs to $V_0$, so
$|f(y^{-1}x)-f(a_i^{-1}x)|\le\epsilon$.   So, for any $x\in G$,

$$\eqalignno{\bigl|(h*f)(x)&-\sum_{i=0}^n\alpha_if(a_i^{-1}x)\bigr|\cr
&=\bigl|\int h(y)f(y^{-1}x)\mu(dy)
    -\sum_{i=0}^n\alpha_if(a_i^{-1}x)\bigr|\cr
&=\bigl|\int_Fh(y)f(y^{-1}x)\mu(dy)
  +\sum_{i=0}^n(\int_{E_i}h(y)f(y^{-1}x)\mu(dy)
     -\alpha_if(a_i^{-1}x))\bigr|\cr
&\le\|f\|_{\infty}\int_F|h|d\mu
  +\sum_{i=0}^n\bigl|\int_{E_i}h(y)
    (f(y^{-1}x)-f(a_i^{-1}x))\mu(dy)\bigr|\cr
&\le\|f\|_{\infty}\int_{X\setminus K}|h|d\mu
  +\epsilon\sum_{i=0}^n\int_{E_i}|h|d\mu
\le\epsilon(\|f\|_{\infty}+\|h\|_1).\cr}$$

\noindent Thus

\Centerline{$\|h*f-\sum_{i=0}^n\alpha_ia_i\action_lf\|_{\infty}
\le\epsilon(\|f\|_{\infty}+\|h\|_1)$.}

\noindent Since $p(a_i\action_lf)=p(f)$ for every $i$, it follows that

$$\eqalign{|p(h*f)-p(f)\int h\,d\mu|
&\le\epsilon(\|f\|_{\infty}+\|h\|_1)p(\chi G)
   +|\sum_{i=0}^n\alpha_i-\int h\,d\mu||p(f)|\cr
&\le\epsilon(\|f\|_{\infty}+\|h\|_1)p(\chi G)
   +|\int_Fh\,d\mu|\|f\|_{\infty}p(\chi G)\cr
&\le\epsilon(2\|f\|_{\infty}+\|h\|_1)p(\chi G).\cr}$$

\noindent As $\epsilon$ is arbitrary, $p(h*f)=p(f)\int h\,d\mu$, as
claimed.
}%end of proof of 449H

\leader{449I}{Notation}\dvAnew{2010}
It will save repeated explanations if I say now
that for the next two results,
given a locally compact Hausdorff group $G$,
$\Sigma_G$ will be the algebra of Haar measurable subsets of $G$ and
$\Cal N_G$ the ideal of Haar negligible subsets of $G$\cmmnt{ (443A)},
while $\Cal B_G$ will be
the Borel $\sigma$-algebra of $G$.   Recall that all three
are left- and right-translation-invariant and inversion-invariant, and
indeed autohomeomorphism-invariant, in that if $\gamma:G\to G$ is a
function of any of the types

\Centerline{$x\mapsto ax$,\quad$x\mapsto xa$,
\quad$x\mapsto x^{-1}$}

\noindent or is a group automorphism which is also a homeomorphism, and
$E\subseteq G$, then
$\gamma[E]$ belongs to $\Sigma_G$, $\Cal N_G$ or $\Cal B_G$ iff $E$
does\cmmnt{ (443Aa)}.

\woddheader{449J}{0}{0}{0}{108pt}

\leader{449J}{Theorem} Let $G$ be a locally compact Hausdorff group;
fix a left Haar measure $\mu$ on $G$.   Write $\eusm L^1$ for
$\eusm L^1(\mu)$ and $L^{\infty}$ for $L^{\infty}(\mu)$, etc.
Let $C_{k1}^+$ be the set of continuous
functions $h:G\to\coint{0,\infty}$ with compact supports such that
$\int h\,d\mu=1$, and suppose that $q\in\coint{1,\infty}$.
Then the following are equiveridical:

(i) $G$ is amenable;

(ii) there is a positive linear functional $p:C_b(G)\to\Bbb R$ such
that $p(\chi G)=1$ and $p(a\action_lf)=p(f)$ for every $f\in C_b(G)$ and
every $a\in G$;

(iii)\dvAnew{2010} there is a finitely additive functional
$\phi:\Cal B_G\to[0,1]$ such that
$\phi G=1$, $\phi(aE)=\phi E$ for every $E\in\Cal B_G$ and $a\in G$, and
$\phi E=0$ for every Haar negligible $E\in\Cal B_G$;

(iv)\dvAnew{2010}
there is a finitely additive functional $\phi:\Sigma_G\to[0,1]$ such that
$\phi G=1$, $\phi(aE)=\phi(Ea)=\phi(E^{-1})=\phi E$ for every
$E\in\Sigma_G$ and $a\in G$, and $\phi E=0$ for every $E\in\Cal N_G$;

(v)\dvAnew{2010} there is a positive
linear functional $\tilde p:L^{\infty}\to\Bbb R$ such that
$\tilde p(\chi G^{\ssbullet})=1$ and
$\tilde p(a\action_lu)=\tilde p(a\action_ru)=\tilde p(a\action_cu)
=\tilde p(\Reverse{u})=\tilde p(u)$ for
every $u\in L^{\infty}$ and every $a\in G$\cmmnt{, where $\action_l$,
$\action_r$, $\action_c$ and $\ssplrarrow$ are defined as in
443Af and 443Gc};

(vi) there is a positive
linear functional $\tilde p:L^{\infty}\to\Bbb R$ such that
$\tilde p(\chi G^{\ssbullet})=1$ and
$\tilde p(a\action_lu)=\tilde p(u)$ for
every $u\in L^{\infty}$ and every $a\in G$;

(vii) there is a positive
linear functional $\tilde p:L^{\infty}\to\Bbb R$ such that
$\tilde p(\chi G^{\ssbullet})=1$ and
$\tilde p(\nu*u)=\nu G\cdot\tilde p(u)$ for
every $u\in L^{\infty}$ and every totally finite Radon measure
$\nu$ on $G$\cmmnt{, where $\nu*u$ is defined as in 444Ma};

(viii) there is a positive linear functional
$\tilde p:L^{\infty}\to\Bbb R$ such that
$\tilde p(\chi G^{\ssbullet})=1$ and
$\tilde p(v*u)=\tilde p(u)\int v$ for
every $v\in L^1$ and $u\in L^{\infty}$;

(ix) for every finite set $J\subseteq\eusm L^1$ and
$\epsilon>0$, there is an $h\in C_{k1}^+$ such that
$\|g*h-(\int g\,d\mu)h\|_1\le\epsilon$ for every $g\in J$;

(x) for every compact set $K\subseteq G$ and $\epsilon>0$, there is
an $h\in C_{k1}^+$ such that $\|a\action_lh-h\|_1\le\epsilon$ for
every $a\in K$;

(xi)\dvAnew{2010}
for any finite set $I\subseteq G$ and $\epsilon>0$, there is a
$u\in L^q$ such that $\|u\|_q=1$ and $\|u-a\action_lu\|_q\le\epsilon$
for every $a\in I$;

(xii)\dvAnew{2010}
for any finite set $I\subseteq G$ and $\epsilon>0$, there is a
compact set $L\subseteq G$ with non-zero measure such that
$\mu(L\symmdiff aL)\le\epsilon\mu L$ for every $a\in I$;

(xiii)\dvArevised{2010}
for every compact set $K\subseteq G$ and $\epsilon>0$, there is a
symmetric compact neighbourhood $L$ of the identity $e$ in $G$
such that $\mu(L\symmdiff aL)\le\epsilon\mu L$ for every $a\in K$;

(xiv)\cmmnt{ ({\smc Emerson \& Greenleaf 67})} for every compact set
$K\subseteq G$ and $\epsilon>0$, there is a
compact set $L\subseteq G$ with non-zero measure such that
$\mu(KL)\le(1+\epsilon)\mu L$.

\proof{{\bf (a)(i)$\Rightarrow$(vii)} Write $U$ for the space
of bounded real-valued functions on $G$ which are uniformly continuous
for the right uniformity.   Then we have a positive linear functional
$p:U\to\Bbb R$ such that $p(\chi G)=1$ and $p(a\action_lf)=p(f)$ for
every $f\in U$ and $a\in G$ (449E).   Now if
$f\in\eusm L^{\infty}$, $h_1$,
$h_2\in\eusm L^1$ and $\int h_1d\mu=\int h_2d\mu$, then
$p(h_1*f)=p(h_2*f)$.   \Prf\ By 449Ha, both $h_1*f$ and $h_2*f$ belong
to $U$.   Set $h=h_1-h_2$.   By 444T, there is a neighbourhood $V$ of
$e$ such that $\|h*\nu-h\|_1\le\epsilon$ whenever
$\nu$ is a quasi-Radon measure on $G$ such that $\nu V=\nu G=1$,
defining $h*\nu$ as in 444J.   In
particular, taking $\nu$ to be the indefinite-integral measure over
$\mu$ defined from
$g=\Bover1{\mu V}\chi V$, $\|h*g-h\|_1\le\epsilon$ (using 444Pb).   Now

$$\eqalignno{|p(h_1*f)-p(h_2*f)|
&=|p(h*f)|
\le |p((h*g)*f)|+|p((h*g-h)*f)|\cr
&\le |p(h*(g*f))|+\|(h*g-h)*f\|_{\infty}\cr
\displaycause{because $*$ is associative, 444Oe}
&\le|p(g*f)\int h\,d\mu|+\|h*g-h\|_1\|f\|_{\infty}\cr
\displaycause{449Hb}
&\le\epsilon\|f\|_{\infty}.\cr}$$

\noindent As $\epsilon$ is arbitrary, $p(h_1*f)=p(h_2*f)$,
as claimed.\ \Qed

Of course $p(h*f)=0$ whenever $h\in\eusm L^1$, $f\in\eusm L^{\infty}$
and $f=0$ a.e.\ (444Ob).   We can therefore define a functional
$\tilde p:L^{\infty}\to\Bbb R$ by saying that
$\tilde p(f^{\ssbullet})=p(h*f)$ whenever $f\in\eusm L^{\infty}$,
$h\in\eusm L^1$ and $\int h\,d\mu=1$.   $\tilde p$ is positive and
linear because $p$ is.   It follows that
$p(h*f)=\tilde p(f^{\ssbullet})\int h\,d\mu$ whenever $h\in\eusm L^1$
and $f\in\eusm L^{\infty}$.   Also
$\tilde p(\chi G^{\ssbullet})=p(\chi G)=1$ because
$h*\chi G=(\int h\,d\mu)\chi G$ for every $h\in\eusm L^1$.

If $u\in L^{\infty}$ and $\nu$ is a totally finite Radon measure on
$G$,
express $u$ as $f^{\ssbullet}$ where
$f\in\eusm L^{\infty}$, so that $\nu*u=(\nu*f)^{\ssbullet}$ (444Ma).
Taking any non-negative
$h\in\eusm L^1$ such that $\int h\,d\mu=1$, we have

$$\eqalignno{h*(\nu*f)
&=h\mu*(\nu*f)\cr
\displaycause{444Pa;  here $h\mu$ is the indefinite-integral measure,
as in 444J}
&=(h\mu*\nu)*f\cr
\displaycause{444Ic}
&=(h*\nu)\mu*f\cr
\displaycause{444K}
&=(h*\nu)*f}$$

\noindent (444Pa again).   So

$$\eqalignno{\tilde p(\nu*u)
&=\tilde p((\nu*f)^{\ssbullet})
=p(h*(\nu*f))\cr
&=p((h*\nu)*f)
=\int h*\nu\,d\mu\cdot\tilde p(u)
=\nu G\cdot\tilde p(u)\cr}$$

\noindent (444K).  As $\nu$ and $u$ are arbitrary, $\tilde p$ has the
required properties.

\medskip

{\bf (b)(vii)$\Rightarrow$(vi)} Take $\tilde p$ from (vii).
If $a\in G$ and $u\in L^{\infty}$, consider the Dirac measure $\delta_a$
on $G$ concentrated at $a$.   Then $\delta_a*u=a\action_lu$.
\Prf\ Take $f\in\eusm L^{\infty}$ such that $f^{\ssbullet}=u$.
Then

\Centerline{$(\delta_a*f)(x)
=\int f(y^{-1}x)\delta_a(dy)=f(a^{-1}x)=(a\action_lf)(x)$}

\noindent whenever $a^{-1}x\in\dom f$, so $\delta_a*f=a\action_lf$ and

\Centerline{$\delta_a*u=\delta_a*f^{\ssbullet}=(\delta_a*f)^{\ssbullet}
=(a\action_lf)^{\ssbullet}=a\action_lu$.  \Qed}

\noindent Accordingly, using (vii),

\Centerline{$\tilde p(a\action_lu)
=\tilde p(\delta_a*u)=\delta_a(G)\tilde p(u)=\tilde p(u)$,}

\noindent as required by (vi).

\medskip

{\bf (c)(vi)$\Rightarrow$(v)}\grheada\ The first step is to note that
since there is a left-invariant mean there must also be a right-invariant
mean, that is, a positive linear functional
$\tilde q:L^{\infty}\to\Bbb R$ such that
$\tilde q(\chi G^{\ssbullet})=1$ and $\tilde q(a\action_ru)=\tilde q(u)$
for every $u\in L^{\infty}$ and every $a\in G$.   \Prf\ Set
$\tilde q(u)=\tilde p(\Reverse{u})$ for $u\in L^{\infty}$.
Evidently $\tilde q$ is a positive linear functional and
$\tilde q(\chi G^{\ssbullet})=1$.   By 443Gc,

\Centerline{$\tilde q(a\action_ru)
=\tilde p((a\action_ru)\ssplrarrow)
=\tilde p(a\action_l\Reverse{u})
=\tilde p(\Reverse{u})
=\tilde q(u)$}

\noindent whenever $u\in L^{\infty}$ and $a\in G$.\ \Qed

\medskip

\qquad\grheadb\ At this point, recall that $L^1$ is a Banach algebra
under convolution (444Sb), and that $L^{\infty}$ can be identified
with its normed space dual, because $\mu$ is a quasi-Radon measure,
therefore localizable (415A), and we can use 243Gb.
We therefore have an Arens
multiplication on $(L^{\infty})^*\cong(L^1)^{**}$
defined by the formulae of 4A6O.
Of course $\tilde p$ and $\tilde q$ both belong to $(L^{\infty})^*$;
write $\tilde r_0=\tilde p\frsmallcirc\tilde q$ for their Arens product.
To see that $\tilde r_0(\chi G^{\ssbullet})=1$,
note that if $u$, $v\in L^1$ then, defining
$\chi G^{\ssbullet}\frsmallcirc u$ and
$\tilde q\frsmallcirc\chi G^{\ssbullet}$ as in 4A6O, we have

\Centerline{$\int(\chi G^{\ssbullet}\frsmallcirc u)\times v
=\int\chi G^{\ssbullet}\times(u*v)
=\int u\int v$}

\noindent as noted in 444Sb;  consequently
$\chi G^{\ssbullet}\frsmallcirc u=(\int u)\chi G^{\ssbullet}$,

\Centerline{$\int(\tilde q\frsmallcirc\chi G^{\ssbullet})\times u
=\tilde q(\chi G^{\ssbullet}\frsmallcirc u)
=\tilde q((\int u)\chi G^{\ssbullet})
=\int u$}

\noindent and
$\tilde q\frsmallcirc\chi G^{\ssbullet}=\chi G^{\ssbullet}$.   Now, of
course,

\Centerline{$\tilde r_0(\chi G^{\ssbullet})
=\tilde p(\tilde q\frsmallcirc\chi G^{\ssbullet})
=\tilde p(\chi G^{\ssbullet})
=1$.}

\noindent As noted in
4A6O, $\|\tilde r_0\|\le\|\tilde p\|\|\tilde q\|=1$, so
$\tilde r_0$ must be a positive linear functional.

\medskip

\qquad\grheadc\ We find next that
$\tilde r_0(a\action_lu)
=\tilde r_0(u)$ whenever $u\in L^{\infty}$ and
$a\in G$.   \Prf\ By 443Ge, we have a bounded linear operator
$S:L^1\to L^1$ defined by setting $Sv=a^{-1}\action_lv$ for every
$v\in L^1$.    By 444Sa, $S(u*v)=(Su)*v$ for all $u$, $v\in L^1$.
Identifying $L^{\infty}$ with
$(L^1)^*$, we have the adjoint operator
$S':L^{\infty}\to L^{\infty}$ given by saying that

$$\eqalign{\int S'u\times v
&=\int u\times Sv
=\int u\times(a^{-1}\action_lv)\cr
&=\int a\action_l(u\times a^{-1}\action_lv)
=\int(a\action_lu)\times v\cr}$$

\noindent whenever $u\in L^{\infty}$ and $v\in L^1$, so that
$S'u=a\action_lu$ for every $u\in L^{\infty}$.   But this means that

\Centerline{$(S''\tilde p)(u)=\tilde p(a\action_lu)=\tilde p(u)$}

\noindent for every $u$, so that $S''\tilde p=\tilde p$.   By 4A6O(b-i),

\Centerline{$S''\tilde r_0
=S''(\tilde p\frsmallcirc\tilde q)
=(S''\tilde p)\frsmallcirc\tilde q
=\tilde p\frsmallcirc\tilde q=\tilde r_0$,}

\noindent that is, $\tilde r_0(a\action_lu)=\tilde r_0(u)$ for every
$u\in L^{\infty}$.\ \Qed

\medskip

\qquad\grheadd\ In the same way,
$\tilde r_0(a\action_ru)
=\tilde r_0(u)$ whenever $u\in L^{\infty}$ and
$a\in G$.   \Prf\ This time, define $T:L^1\to L^1$ by
setting $Tv=\Delta(a^{-1})a^{-1}\action_rv$ for every
$v\in L^1$, where $\Delta$ is the left modular function of $G$;
444Sa tells us that $T(u*v)=u*Tv$ for all $u$, $v\in L^1$.
Since $\int fd\mu=\Delta(a)\int a\action_rfd\mu$
for every $f\in\eusm L^1$ (442Kc),
$\int v=\Delta(a)\int a\action_rv$ for every $v\in L^1$, and

$$\eqalign{\int T'u\times v
&=\int u\times Tv
=\Delta(a^{-1})\int u\times(a^{-1}\action_rv)\cr
&=\int a\action_r(u\times a^{-1}\action_rv)
=\int(a\action_ru)\times v\cr}$$

\noindent whenever $u\in L^{\infty}$ and $v\in L^1$.   Thus
$T'u=a\action_ru$ for every $u\in L^{\infty}$.   But now we have

\Centerline{$(T''\tilde q)(u)=\tilde q(a\action_ru)=\tilde q(u)$}

\noindent for every $u$, so that $T''\tilde q=\tilde q$.   By 4A6O(b-ii),
$T''\tilde r_0=\tilde r_0$, that is,
$\tilde r_0(a\action_ru)=\tilde r_0(u)$ for every
$u\in L^{\infty}$.\ \Qed

\medskip

\qquad\grheade\ Thus $\tilde r_0$ is both left- and
right-invariant.   To get reversal-invariance, set

\Centerline{$\tilde r(u)=\Bover12(\tilde r_0(u)+\tilde r_0(\Reverse{u}))$}

\noindent for $u\in L^{\infty}$.   Then $\tilde r_0$ is a
positive linear functional and $\tilde r_0(\chi G^{\ssbullet})=1$.
Because

\Centerline{$a\action_l\Reverse{u}=(a\action_ru)\ssplrarrow$,
\quad$a\action_r\Reverse{u}=(a\action_lu)\ssplrarrow$,}

\noindent $u\mapsto\tilde r_0(\Reverse{u})$ and $\tilde r$ are also
both left- and right-invariant, and of course
$\tilde r(\Reverse{u}) =\tilde r(u)$ for every $u$.
Finally,

\Centerline{$\tilde r(a\action_cu)=\tilde r(a\action_l(a\action_ru))
=\tilde r(u)$}

\noindent for every $u\in L^{\infty}$ and $a\in G$, so
$\tilde r$ has all the properties required by (v).

\medskip

{\bf (d)(v)$\Rightarrow$(iv)}
Take $\tilde p$ from (v), and set
$\phi E=\tilde p(\chi E^{\ssbullet})$ for every $E\in\Sigma_G$.
Then $\phi:\Sigma_G\to[0,1]$ is additive and $\phi G=1$;  also, if
$E\in\Cal N_G$, $\chi E^{\ssbullet}=0$ in $L^{\infty}$ and
$\phi E=0$.   If $E\in\Sigma_G$ and $a\in G$,
then  $\chi(aE)=a\action_l(\chi E)$ (4A5C(c-ii)) and

\Centerline{$\phi(aE)
=\tilde p(\chi(aE)^{\ssbullet})
=\tilde p((a\action_l\chi E)^{\ssbullet})
=\tilde p(a\action_l(\chi E^{\ssbullet}))
=\tilde p(\chi E^{\ssbullet})
=\phi E$.}

\noindent Next, $\chi E^{-1}=(\chi E)\ssplrarrow$ and
$(\chi E^{-1})^{\ssbullet}=(\chi E^{\ssbullet})\ssplrarrow$, so

\Centerline{$\phi(E^{-1})
=\tilde p((\chi E^{\ssbullet})\ssplrarrow)
=\tilde p(\chi E^{\ssbullet})
=\phi E$.}

\noindent Consequently, for $E\in\Sigma_G$ and $a\in G$,

\Centerline{$\phi(Ea)=\phi((Ea)^{-1})
=\phi(a^{-1}E^{-1})=\phi(E^{-1})=\phi E$.}

\noindent Thus $\phi$ satisfies the requirements of (iv).

\medskip

{\bf (e)(iv)$\Rightarrow$(iii)} This is trivial;  we have only to take
$\phi:\Sigma_G\to[0,1]$ as in (iv)
and consider $\phi\restr\Cal B_G$.

\medskip

{\bf (f)(iii)$\Rightarrow$(ii)} Given $\phi:\Cal B_G\to[0,1]$ as in (iii),
set $p(f)=\dashint fd\phi$ for $f\in C_b(G)$, where $\dashint\,d\phi$
is as defined in 363L, that is, the unique $\|\,\|_{\infty}$-continuous
linear functional on the space
$L^{\infty}(\Cal B_G)$ of bounded Borel measurable functions from
$G$ to $\Bbb R$ such that $\dashint\chi E\,d\phi=\phi E$ for every
$E\in\Cal B_G$.   $p$ is positive because $\phi$ is non-negative (363Lc),
and $p(\chi G)=\phi G=1$.   If $a\in G$, then

\Centerline{$\dashint a\action_l\chi E\,d\phi=\dashint\chi(aE)d\phi
=\phi(aE)=\phi E=\dashint\chi E\,d\phi$}

\noindent for every $E\in\Cal B_G$;  because
$f\mapsto\dashint fd\phi$ and $f\mapsto\dashint a\action_lfd\phi$ are both
linear and $\|\,\|_{\infty}$-continuous, they agree on
$L^{\infty}(\Cal B_G)\supseteq C_b(G)$, and

\Centerline{$p(a\action_lf)=\dashint a\action_lfd\phi
=\dashint fd\phi=p(f)$}

\noindent for every $f\in C_b(G)$, as required.

\medskip

{\bf (g)(ii)$\Rightarrow$(i)} Given $p$ as in (ii), its restriction to
the space of bounded right-uniformly-continuous functions  is
positive, linear and $G$-invariant, so $G$ is amenable, by 449E.

\medskip

{\bf (h)(vii)$\Rightarrow$(viii)} Take $\tilde p$ as in (vii).   If
$g\in\eusm L^1$, $f\in\eusm L^{\infty}$ and $g\ge 0$, then

$$\eqalign{\tilde p(g*f)^{\ssbullet}
&=\tilde p(g\mu*f)^{\ssbullet}
=\tilde p(g\mu*f^{\ssbullet})\cr
&=(g\mu)(G)\tilde p(f^{\ssbullet})
=\int g\,d\mu\cdot\tilde p(f^{\ssbullet});\cr}$$

\noindent translating into terms of $L^1$ and $L^{\infty}$ as in 444Sa,
we get $\tilde p(v*u)=\int v\cdot\tilde p(u)$ for all
$u\in L^{\infty}$ and $v\in(L^1)^+$.   By linearity, the same is
true for all $v\in L^1$, as required by (viii).

\woddheader{449J}{4}{2}{2}{60pt}

{\bf (i)(viii)$\Rightarrow$(ix)} Suppose that (viii) is true.

\medskip

\quad\grheada\ Note first that if
$J\subseteq\eusm L^{\infty}$ is finite and $\epsilon>0$, then

\Centerline{$A(J,\epsilon)
=\{h:h\in C_{k1}^+,\,|\int f\times h\,d\mu-\tilde p(f^{\ssbullet})|
  \le\epsilon$ for every $f\in J\}$}

\noindent is non-empty.   \Prf\ It is enough to consider the case in which
$\chi G\in J$.   Let $\eta\in\ooint{0,\bover12}$ be such
that $\eta+5\eta\sup_{f\in J}\|f^{\ssbullet}\|_{\infty}\le\epsilon$.
Because
$\tilde p\in(L^{\infty})^*\cong(L^1)^{**}$, there is a $u_0\in L^1$ such
that $\|u_0\|_1\le 1$ and
$|\tilde p(f^{\ssbullet})-\int f^{\ssbullet}\times u_0|\le\eta$ for
every $f\in J$ (4A4If).   In particular,
$\int u_0\ge\tilde p(\chi G^{\ssbullet})-\eta=1-\eta$.   By 416I, there is
a continuous $h_0:G\to\Bbb R$ with compact support such that
$\|u_0-h_0^{\ssbullet}\|_1\le\eta$.   Now $\int h_0d\mu\ge 1-2\eta$
and $\int|h_0|d\mu\le 1+\eta$.   So if we set $h_0^+=h_0\vee 0$,
$\gamma=\int h_0^+d\mu$ and $h=\Bover1{\gamma}h_0^+$, we shall have

\Centerline{$\gamma\le 1+\eta$,
\quad$\|h_0^+-h_0\|_1=\Bover12\int|h_0|-h_0\le 2\eta$,
\quad$\|h-h_0^+\|=|\gamma-1|\le 2\eta$,}

\noindent so
$\|u_0-h^{\ssbullet}\|_1\le 5\eta$, while $h\in C_{k1}^+$.   This will mean
that

\Centerline{$|\tilde p(f^{\ssbullet})-\int f\times h\,d\mu|
\le\eta+5\eta\|f^{\ssbullet}\|_{\infty}\le\epsilon$}

\noindent for every $f\in J$.\ \QeD\

We therefore have a filter $\Cal F$ on $C_{k1}^+$
containing every $A(J,\epsilon)$, and
$\tilde p(f^{\ssbullet})=\lim_{h\to\Cal F}\int f\times h\,d\mu$ for
every $f\in\eusm L^{\infty}$.

\medskip

\quad\grheadb\ Now
$0=\lim_{h\to\Cal F}(g*h)^{\ssbullet}-(\int g\,d\mu)h^{\ssbullet}$ for
the weak topology of $L^1$, for every $g\in\eusm L^1$.   \Prf\ Set
$\gamma=\int g\,d\mu$.   Let $f\in\eusm L^{\infty}$.   Define $g'$ by
setting $g'(x)=\Delta(x^{-1})g(x^{-1})$ whenever this is defined,
where $\Delta$ is the left modular function of $G$, as before;
then $g'\in\eusm L^1$
and $\int g'd\mu=\gamma$ (442K(b-ii)).
Set $v=(g')^{\ssbullet}\in L^1$.   If $h\in C_{k1}^+$, then

$$\eqalignno{\int f\times(g*h)d\mu
&=\iint f(xy)g(x)h(y)\mu(dx)\mu(dy)\cr
\displaycause{444Od}
&=\int\bigl(\int\Delta(x^{-1})g'(x^{-1})f(xy)\mu(dx)\bigr)h(y)\mu(dy)\cr
&=\int (g'*f)(y)h(y)\mu(dy)\cr
\displaycause{444Oa}
&=\int (g'*f)\times h\,d\mu.\cr}$$

\noindent By (viii), we have

\Centerline{$\tilde p(g'*f)^{\ssbullet}
=\tilde p(v*f^{\ssbullet})
=\tilde p(f^{\ssbullet})\int v
=\gamma\tilde p(f^{\ssbullet})$,}

\noindent so we get

$$\eqalign{\lim_{h\to\Cal F}\int f\times(g*h-\gamma h)d\mu
&=\lim_{h\to\Cal F}\int (g'*f-\gamma f)\times h\,d\mu\cr
&=\tilde p(g'*f)^{\ssbullet}-\gamma\tilde p(f^{\ssbullet})
=0.\cr}$$

\noindent As $f$ is arbitrary, and $(L^1)^*$ can be identified with
$L^{\infty}$, this is all we need.\ \Qed

\medskip

\quad\grheadc\ Now take any finite set $J\subseteq\eusm L^1$ and
$\epsilon>0$.   On $(L^1)^J$ let $\frak T$ be the locally convex linear
space topology which is the product topology when each copy of $L^1$ is
given the norm topology, and $\frak S$ the corresponding weak topology.
Define $T:C_k(G)\to(L^1)^J$ by setting

\Centerline{$Th
=\bigl\langle\hbox{$(g*h)^{\ssbullet}
  -(\int g\,d\mu)h^{\ssbullet}$}\bigr\rangle_{g\in J}$}

\noindent for $h\in C_k(G)$, where $C_k(G)$ is the linear space of
continuous real-valued functions on $G$ with compact support.    Then
$T$ is linear.   Moreover, by ($\beta$), $\lim_{h\to\Cal F}Th=0$ in
$(L^1)^J$ for the product topology, if each copy of $L^1$ is given its
weak topology.   By 4A4Be, this is just $\frak S$.   In particular,
$0$ belongs to the $\frak S$-closure of $T[C_{k1}^+]$.   But $C_{k1}^+$
is convex and
$T$ is linear, so $T[C_{k1}^+]$ is convex;  by 4A4Ed, $0$ belongs to the
$\frak T$-closure of $T[C_{k1}^+]$.   There is therefore an
$h\in C_{k1}^+$ such that
$\|(g*h)^{\ssbullet}-(\int g\,d\mu)h^{\ssbullet}\|_1\le\epsilon$ for
every $g\in J$.   As $J$ and $\epsilon$ are arbitrary, (ix) is true.

\medskip

{\bf (j)(ix)$\Rightarrow$(x)} Suppose that (ix) is true and that
we are given a compact set $K\subseteq G$ and $\epsilon>0$.   Set
$\eta=\bover13\epsilon$.
Fix any $h_0\in C_{k1}^+$.   Let $V$ be a neighbourhood of $e$ such
that $\|c\action_lh_0-h_0\|_1\le\eta$ whenever $c\in V$ (443Gf).   Let
$I\subseteq G$ be a finite set such that $K\subseteq IV$.   By (ix),
there is an $h_1\in C_{k1}^+$ such that

\Centerline{$\|b\action_lh_0*h_1-h_1\|_1\le\eta$ for every $b\in I$,
\quad$\|h_0*h_1-h_1\|_1\le\eta$.}

\noindent (I omit brackets because
$(b\action_lh_0)*h_1=b\action_l(h_0*h_1)$, see 444Of.)
Set $h=h_0*h_1$.   Then $h\in C_{k1}^+$.   \Prf\ $h$ is
continuous, by
444Rc (or otherwise).   If we write $M_i$ for the support $\supp(h_i)$ of
$h_i$ for both $i$, then $h(x)=0$ for every $x\in G\setminus M_0M_1$,
so $h$ has compact support.   Of course $h\ge 0$, and
$\int h\,d\mu=\int h_0d\mu\int h_1d\mu=1$ (444Qb), so $h\in C_{k1}^+$.\
\Qed

If $a\in K$, there are $b\in I$, $c\in V$ such that $a=bc$, so that

$$\eqalignno{\|a\action_lh-h\|_1
&=\|b\action_l(c\action_lh_0*h_1)-h_0*h_1\|_1\cr
&\le\|b\action_l(c\action_lh_0-h_0)*h_1\|_1
  +\|b\action_lh_0*h_1-h_1\|_1+\|h_1-h_0*h_1\|_1\cr
&\le\|c\action_lh_0-h_0\|_1+\eta+\eta
\le 3\eta=\epsilon.\cr}$$

\noindent So this $h$ will serve.

\medskip

{\bf (k)(x)$\Rightarrow$(xiii)} Suppose that (x) is true.

\medskip

\quad\grheada\ I show first that for any compact set $K\subseteq G$ and
$\epsilon>0$ there is an $h\in C_{k1}^+$ such that

\Centerline{$\Reverse{h}=h$,
\quad$h(e)=\|h\|_{\infty}$,}

\Centerline{$\|a\action_lh-h\|_1\le\epsilon$ for every $a\in K$.}

\noindent \Prf\ By (x), there is an $h_0\in C_{k1}^+$ such that
$\|a\action_lh_0-h_0\|_1\le\epsilon$ for every $a\in K$.   Set
$\gamma=\int\Reverse{h}_0d\mu$;  because
$\Reverse{h}_0\in C_k(X)^+\setminus\{0\}$, $\gamma$ is finite and not $0$.
Try $h=h_0*\bover1{\gamma}\Reverse{h}_0$, so that
$h(x)=\bover1{\gamma}\int h_0(y)h_0(x^{-1}y)\mu(dy)$ for every $x\in G$.
By 444Rc, $h\in C_b(G)$ and

\Centerline{$\|h\|_{\infty}\le\|h_0\|_2\|\Bover1{\gamma}h_0\|_2
=\Bover1{\gamma}\int h_0^2d\mu=h(e)$.}

\noindent Because $h_0\ge 0$, $h\ge 0$;   by 444Qb, $\int h\,d\mu=1$;  and
(as in (j) above) $\supp(h)$ is included in the compact set
$\supp(h_0)\supp(\Reverse{h}_0)$, so
$h\in C_{k1}^+$.   By 444Rb, or otherwise, $h=\Reverse{h}$.

Finally, if $a\in K$, then

$$\eqalignno{\|a\action_lh-h\|_1
&=\Bover1{\gamma}\|a\action_l(h_0*\Reverse{h}_0)-h_0*\Reverse{h}_0\|_1
=\Bover1{\gamma}\|(a\action_lh_0-h_0)*\Reverse{h}_0\|_1\cr
\displaycause{444Of once more}
&\le\Bover1{\gamma}\|a\action_lh_0-h_0\|_1\|\Reverse{h}_0\|_1\cr
\displaycause{444Qb again}
&=\|a\action_lh_0-h_0\|_1
\le\epsilon,\cr}$$

\noindent as required.\ \Qed

\medskip

\quad\grheadb\ Next, for any $\epsilon$, $\delta>0$ and
any compact set $K\subseteq G$ there are an open
symmetric neighbourhood $V$ of $e$ and a closed set $F$ such
that $\mu V<\infty$, $\mu F\le\delta$ and
$\mu(aV\symmdiff V)\le\epsilon\mu V$ whenever $a\in K\setminus F$.
\Prf\ Of course it is enough to deal with the case in which $\mu K>0$.
Set $\eta=\epsilon\delta/\mu K$.   By ($\alpha$), there is an
$h\in C_{k1}^+$ such that
$h(e)=\|h\|_{\infty}$, $h=\Reverse{h}$ and
$\|a\action_lh-h\|_1<\eta$ for every $a\in K$.

Set $K_0=\supp(h)$ and $K^*=K_0\cup KK_0$, so that
$K^*\subseteq G$ is compact.   Set

$$\eqalign{Q&=\{(a,x,t):a\in K,\,x\in G,\,t\in\Bbb R,\cr
&\mskip150mu\text{either }h(x)\le t<h(a^{-1}x)
  \text{ or }h(a^{-1}x)\le t<h(x)\}.\cr}$$

\noindent Then $Q$ is a Borel subset of $G\times G\times\Bbb R$ included
in the compact set $K\times K^*\times[0,h(e)]$.   Let $\mu_L$
be Lebesgue measure on $\Bbb R$, and let $\mu\times\mu\times\mu_L$ be
the $\tau$-additive product measure on $G\times G\times\Bbb R$ (417D).
(Of course this is actually a Radon measure.)   For $t\in\Bbb R$ let
$V_t$ be the open set $\{x:h(x)>t\}$.   Now 417H tells us that

$$\eqalignno{(\mu\times\mu\times\mu_L)(Q)
&=\int_{G\times G}\mu_L\{t:(a,x,t)\in Q\}(\mu\times\mu)(d(a,x))\cr
\displaycause{where $\mu\times\mu$ is the $\tau$-additive product
measure on $G\times G$, so that we can identify
$\mu\times\mu\times\mu_L$ with $(\mu\times\mu)\times\mu_L)$, as in
417Db}
&=\int_K\int_G|h(a^{-1}x)-h(x)|\mu(dx)\mu(da)\cr
\displaycause{we can use 417H again because
$\{x:h(a^{-1}x)\ne h(x)\}\subseteq K^*$ if $a\in K$, and $\mu K^*$ is
finite}
&=\int_K\|a\action_lh-h\|_1\mu(da)
<\eta\mu K\cr
\displaycause{by the choice of $h$}
&=\eta\mu K\int h\,d\mu
=\eta\mu K(\mu\times\mu_L)\{(x,t):0\le t<h(x)\}\cr
&=\eta\mu K\int_0^{h(e)}\mu V_t\,\mu_L(dt)\cr}$$

\noindent as in 252N.   (The c.l.d.\ and $\tau$-additive product
measures on $G\times\Bbb R$ coincide, by 417T.)   On the other hand,

$$\eqalignno{(\mu\times\mu\times\mu_L)(Q)
&=\int_{K\times\Bbb R}\mu\{x:(a,x,t)\in Q\}(\mu\times\mu_L)(d(a,t))\cr
\displaycause{again, we can use 417H because $x\in K^*$ whenever
$(a,x,t)\in Q$}
&=\int_{K\times\Bbb R}\mu(V_t\symmdiff aV_t)(\mu\times\mu_L)(d(a,t))\cr
&=\int_0^{\infty}\int_K
  \mu(V_t\symmdiff aV_t)\mu(da)\mu_L(dt)\cr}$$

\noindent because $V_t=aV_t=G$ whenever $t<0$ and $a\in G$.   So there
must be some $t\in\ooint{0,h(e)}$ such that

\Centerline{$\int_K\mu(V_t\symmdiff aV_t)\mu(da)
<\eta\mu K\mu V_t=\epsilon\delta\mu V_t$.}

\noindent Set $V=V_t$ and
$F=\{a:a\in K,\,\mu(V_t\symmdiff aV_t)\ge\epsilon\mu V_t\}$;  then
$V$ is open, $F$ is closed (443C) and $\mu(V\symmdiff aV)\le\epsilon\mu V$
for every $a\in K\setminus F$;  also $0<\mu V<\infty$,
$V$ is symmetric (because $h=\Reverse{h}$) and $e\in V$
(because $t<h(e)$).\ \Qed

\medskip

\quad\grheadc\ Now let $K\subseteq G$ be a compact set and
$\epsilon>0$,
as in the statement of (xiii);  enlarging $K$ and lowering $\epsilon$ if
necessary, we may suppose that $\mu K>0$ and $\epsilon\le 1$.   Set
$K_1=K\cup KK$, so that $K_1$ is still compact.   By ($\beta$), we
have a symmetric open neighbourhood $V$ of $e$, of finite measure,
such that
$W=\{a:a\in K_1,\,\mu(aV\symmdiff V)>\bover13\epsilon\mu V\}$ has
measure less than $\bover12\mu K$.   If $a\in K$, then $W\cup a^{-1}W$
cannot cover $K$, so there is a $b\in K\setminus W$ such that
$ab\notin W$;  thus $b$ and $ab$ both belong to $K_1\setminus W$, and

\Centerline{$\mu(aV\symmdiff V)
\le\mu(aV\symmdiff abV)+\mu(abV\symmdiff V)
\le\mu(V\symmdiff bV)+\Bover13\epsilon\mu V
\le\Bover23\epsilon\mu V$.}

We can now find a compact symmetric neighbourhood $L$ of $e$, included in
$V$, with $8\mu(V\setminus L)\le\epsilon\mu L$.   In this case,
we shall have $\mu L>0$ and

$$\eqalign{\mu(aL\symmdiff L)
&\le\mu(aV\setminus aL)+\mu(aV\symmdiff V)+\mu(V\setminus L)\cr
&\le 2\mu(V\setminus L)+\Bover23\epsilon(\mu L+\mu(V\setminus L))
\le\epsilon\mu L\cr}$$

\noindent for every $a\in K$, as required.

\medskip

{\bf (l)(xiii)$\Rightarrow$(xiv)} Suppose that (xiii) is true, and that
$K\subseteq G$ is compact and $\epsilon>0$.   Enlarging $K$ if
necessary, we may
suppose that it includes a neighbourhood of $e$.   Of course we may
also suppose that $\epsilon\le 1$.

\medskip

\quad\grheada\ The first thing to note is that there is a set
$I\subseteq G$ such that $KI=G$ and
$m=\sup_{y\in G}\#(\{x:x\in I,\,y\in Kx\})$ is finite.   \Prf\ Let $V$
be an open neighbourhood of $e$ such that $VV^{-1}\subseteq K$.   Let
$I\subseteq G$ be maximal subject to $x^{-1}V\cap y^{-1}V=\emptyset$
for all distinct $x$, $y\in I$.   If $x\in G$, there must be a $y\in I$
such that $x^{-1}V\cap y^{-1}V\ne\emptyset$, so that
$x^{-1}\in y^{-1}VV^{-1}$ and $x\in VV^{-1}y\subseteq Ky\subseteq KI$;
as $x$ is arbitrary, $G\subseteq KI$.   If $y\in G$ and
$I_y=\{x:x\in I,\,y\in Kx\}$, then $I_y\subseteq K^{-1}y$, so that
$\{x^{-1}V:x\in I_y\}$ is a disjoint family of subsets of $y^{-1}KV$.
But this means that $\#(I_y)\mu V\le\mu(y^{-1}KV)=\mu(KV)$.
Accordingly $\sup_{y\in G}\#(I_y)\le\Bover{\mu(KV)}{\mu V}$ is
finite.\
\Qed

\medskip

\quad\grheadb\ Set $\gamma=\sup_{a\in K}\Delta(a)$, $K^*=KKK^{-1}$.
Let $\delta>0$ be such that

\Centerline{$\delta\gamma m<\epsilon(\mu K-\delta\gamma)$,}

\noindent and let $\eta>0$ be such that

\Centerline{$1+\Bover{\delta\gamma m}{\mu K-\delta\gamma}
\le(1+\epsilon)(1-\Bover{\eta}{\delta}\mu K^*)$.}

\noindent By (xiii), there is a non-negligible
compact set $L^*\subseteq G$ such that
$\mu(aL^*\symmdiff L^*)\le\eta\mu L^*$ for every $a\in K^*$.
Set $L=\{x:x\in L^*,\,\mu(K^*\setminus L^*x^{-1})\le\delta\}$;  note
that $L$ is closed (because $x\mapsto\mu(K^*\cap L^*x^{-1})$ is
continuous, see 443C), therefore compact.

\medskip

\quad\grheadc\ $\mu L\ge(1-\Bover{\eta}{\delta}\mu K^*)\mu L^*$.
\Prf\
Set $W=\{(x,y):x\in K^*,\,y\in L^*\setminus L,\,xy\notin L^*\}$.
Then $W$ is a relatively compact Borel subset of $G\times G$, so we may
apply Fubini's theorem (in the form 417H, as usual) to see that

$$\eqalign{\delta\mu(L^*\setminus L)
&\le\int_{L^*\setminus L}\mu(K^*\setminus L^*y^{-1})\mu(dy)
=\int\mu W^{-1}[\{y\}]\mu(dy)\cr
&=\int\mu W[\{x\}]\mu(dx)
=\int_{K^*}\mu((L^*\setminus L)\setminus x^{-1}L^*)\mu(dx)\cr
&\le\int_{K^*}\mu(L^*\setminus x^{-1}L^*)\mu(dx)
=\int_{K^*}\mu(xL^*\setminus L^*)\mu(dx)
\le\eta\mu K^*\mu L^*\cr}$$

\noindent by the choice of $L^*$.   Accordingly

\Centerline{$\mu L=\mu L^*-\mu(L^*\setminus L)
\ge(1-\Bover{\eta}{\delta}\mu K^*)\mu L^*$.  \Qed}

\noindent In particular, $\mu L>0$.

\medskip

\quad\grheadd\ Set $J=\{x:x\in I,\,L\cap Kx\ne\emptyset\}$.
For each $x\in J$, choose $z_x\in Kx\cap L$.   Then
$\Delta(z_x)\le\gamma\Delta(x)$ and
$\Delta(x)(\mu K-\gamma\delta)\le\mu(Kx\cap L^*)$ for every $x\in J$.
\Prf\ $\Delta(z_x)=\Delta(z_xx^{-1})\Delta(x)\le\gamma\Delta(x)$ because
$z_xx^{-1}\in K$.   Next, because $z_x\in L$,

\Centerline{$\mu(K^*z_x\setminus L^*)
=\Delta(z_x)\mu(K^*\setminus L^*z_x^{-1})\le\delta\Delta(z_x)$.}

\noindent Since $x\in K^{-1}z_x$, $Kx\subseteq KK^{-1}z_x\subseteq K^*z_x$,

\Centerline{$\mu(Kx\setminus L^*)
\le\mu(K^*z_x\setminus L^*)\le\delta\Delta(z_x)\le\delta\gamma\Delta(x)$}

\noindent and

\Centerline{$\mu(Kx\cap L^*)=\mu(Kx)-\mu(Kx\setminus L^*)
\ge\Delta(x)\mu K-\delta\gamma\Delta(x)
=\Delta(x)(\mu K-\delta\gamma)$.  \Qed}

\wheader{449J}{6}{2}{2}{48pt}

\quad\grheade\ Now recall that $\sum_{x\in I}\chi(Kx)\le m\chi G$, by
the definition of $m$, so that

\Centerline{$(\mu K-\delta\gamma)\sum_{x\in J}\Delta(x)
\le\sum_{x\in J}\mu(Kx\cap L^*)
\le m\mu L^*$.}

\noindent
Since $KI=G$, $L\subseteq KJ$ and

$$\eqalign{KL&\subseteq\bigcup_{x\in J}KKx
\subseteq\bigcup_{x\in J}KKK^{-1}z_x=\bigcup_{x\in J}K^*z_x,\cr
\cr
\mu(KL\setminus L^*)
&\le\sum_{x\in J}\mu(K^*z_x\setminus L^*)
=\sum_{x\in J}\Delta(z_x)\mu(K^*\setminus L^*z_x^{-1})\cr
&\le\delta\sum_{x\in J}\Delta(z_x)
\le\delta\gamma\sum_{x\in J}\Delta(x)
\le\Bover{\delta\gamma m}{\mu K-\delta\gamma}\mu L^*.\cr}$$

\noindent Accordingly

$$\eqalignno{\mu(KL)
&\le\mu L^*(1+\Bover{\delta\gamma m}{\mu K-\delta\gamma})
\le\mu L\cdot\bover{1+\bover{\delta\gamma m}
  {\mu K-\delta\gamma}}{1-\bover{\eta}{\delta}\mu K^*}\cr
\displaycause{by ($\gamma$) above}
&\le(1+\epsilon)\mu L,\cr}$$

\noindent by the choice of $\delta$ and $\eta$.   Thus we have found
an
appropriate set $L$.

\medskip

{\bf (m)(xiv)$\Rightarrow$(xii)} If $I\subseteq G$ is finite and
$\epsilon>0$,
$I\cup\{e\}$ is compact, so there is a compact set $L\subseteq G$, of
non-zero measure, such that
$\mu(IL\cup L)\le(1+\bover12\epsilon)\mu L$.   Consequently

\Centerline{$\mu(L\symmdiff aL)
=2\mu(aL\setminus L)
\le 2\mu(IL\setminus L)
\le\epsilon\mu L$}

\noindent for every $a\in I$, as required by (xii).

\medskip

{\bf (n)(xii)$\Rightarrow$(ii)} Write $\Cal L$ for the family of all
compact subsets of $G$ with non-zero measure.   For $L\in\Cal L$,
define $p_L:C_b(G)\to\Bbb R$ by setting $p_L(f)=\Bover1{\mu L}\int_Lfd\mu$
for $f\in C_b(G)$.   Of course $|p_L(f)|\le\|f\|_{\infty}$.
For finite $I\subseteq G$, $\epsilon>0$ set

\Centerline{$\Cal A(I,\epsilon)
=\{L:L\in\Cal L,\,\mu(aL\symmdiff L)\le\epsilon\mu L$ for
every $a\in I\}$.}

\noindent By (xii), no $\Cal A(I,\epsilon)$ is empty.   So we have an
ultrafilter $\Cal F$ on $\Cal L$ containing every $\Cal A(I,\epsilon)$.
Set $p(f)=\lim_{L\to\Cal F}p_L(f)$ for $f\in C_b(G)$;  then
$p:C_b(G)\to\Bbb R$ is a positive linear functional and $p(\chi G)=1$.

If $a\in G$, $f\in C_b(G)$ and $L\in\Cal A(\{a\},\epsilon)$, then

$$\eqalign{|p_L(a\action_lf)-p_L(f)|
&=\Bover1{\mu L}\bigl|\int_Lf(a^{-1}x)\mu(dx)-\int_Lf(x)\mu(dx)\bigr|\cr
&=\Bover1{\mu L}\bigl|\int_{aL}fd\mu-\int_Lfd\mu\bigr|
\le\Bover1{\mu L}\|f\|_{\infty}\mu(aL\symmdiff L)
\le\epsilon\|f\|_{\infty}.\cr}$$

\noindent
Since $\Cal F$ contains $\Cal A(\{a\},\epsilon)$ for every
$\epsilon>0$,

\Centerline{$|p(a\action_lf)-p(f)|
=\lim_{L\to\Cal F}|p_L(a\action_lf)-p_L(f)|=0$.}

\noindent As $f$ and $a$ are arbitrary, $p$ witnesses that (ii) is true.

\medskip

{\bf (o)(xii)$\Rightarrow$(xi)} Given a finite set $I\subseteq G$ and
$\epsilon>0$, (xii) tells us that there is a compact set $L\subseteq G$
of non-zero measure such
that $\mu(aL\symmdiff L)\le\epsilon^q\mu L$ for every $a\in I$.   Try
$u=\Bover1{(\mu L)^{1/q}}(\chi L)^{\ssbullet}$.   Then $\|u\|_q=1$.
If $a\in I$, then $a\action_lu=\Bover1{(\mu L)^{1/q}}\chi(aL)^{\ssbullet}$,
so

\Centerline{$\int|u-a\action_lu|^q
=\int(\Bover1{(\mu L)^{1/q}}\chi(aL\symmdiff L)^{\ssbullet})^q
=\Bover{\mu(aL\symmdiff L)}{\mu L}
\le\epsilon^q$}

\noindent and $\|u-a\action_lu\|_q\le\epsilon$, as required by (xi).

\medskip

{\bf (p)(xi)$\Rightarrow$(xii)} Now assume that (xi) is true.   Let
$I\subseteq G$ be a finite set, and $\epsilon>0$.   Set
$\delta=\Bover{\epsilon}{4+\epsilon}$, and let
$\eta>0$ be such that that $(1+\eta\#(I))^q\le 1+\delta$.

Take $u\in L^q$ such that $\|u\|_q=1$ and
$\|u-a\action_lu\|_q\le\eta$ for every $a\in I$.   Setting $v=|u|$, we
see that $a\action_lv=|a\action_lu|$, so
$|v-a\action_lv|\le|u-a\action_lu|$ and $\|v-a\action_lv\|_q\le\eta$ for
every $a\in I$, while $\|v\|_q=1$.   Let $f:G\to\coint{0,\infty}$ be
a function such
that $f^{\ssbullet}=v$ in $L^q$;  then $\|f\|_q=1$ while
$\|f-a\action_lf\|_q\le\eta$ for every $a\in I$.   Set
$g=\sup_{a\in I\cup\{e\}}a\action_lf$;  then

\Centerline{$f\le g\le f+\sum_{a\in I}(a\action_lf-f)^+$,}

\noindent so

\Centerline{$\|g\|_q\le 1+\sum_{a\in I}\|a\action_lf-f\|_q
\le 1+\eta\#(I)$,
\quad$\int g^qd\mu\le(1+\eta\#(I))^q\le 1+\delta$.}

\noindent For $t>0$, set $E_t=\{x:f(x)^q\ge t\}$, $F_t=\{t:g(x)^q\ge t\}$.
Then

\Centerline{$\int_0^{\infty}\mu E_tdt=\int f^qd\mu=1$,
\quad$\int_0^{\infty}\mu F_tdt=\int g^qd\mu\le 1+\delta$,}

\noindent where the integrals here are with respect to Lebesgue measure
(252O).   There must therefore be a $t>0$ such that $\mu E_t>0$ and
$\mu F_t\le(1+\delta)\mu E_t$.

If $a\in I$,
then $a\action_lf\le g$, so $aE_t=\{x:(a\action_lf)(x)\ge t\}$
is included in $F_t$;  also, of course, $E_t\subseteq F_t$.
We therefore have

\Centerline{$\mu(E_t\symmdiff aE_t)=2\mu(aE_t\setminus E_t)
\le 2\mu(F_t\setminus E_t)\le 2\delta\mu E_t$.}

\noindent There is no reason why $E_t$ should be compact, so it may
not itself be the $L$ we seek.   However, $\mu E_t$ is certainly finite, so
there must be a compact $L\subseteq E$ such that
$\mu L\ge(1-\delta)\mu E_t$.   In this case, $\mu L>0$ and

$$\eqalign{\mu(aL\symmdiff L)
&\le\mu(aL\symmdiff aE_t)+\mu(aE_t\symmdiff E_t)+\mu(E_t\symmdiff L)\cr
&\le 2\mu(E_t\setminus L)+2\delta\mu E_t
\le 4\delta\mu E_t
\le\Bover{4\delta}{1-\delta}\mu L
=\epsilon\mu L\cr}$$

\noindent for every $a\in I$.   So this $L$ will serve.
}%end of proof of 449J

\cmmnt{\medskip

\noindent{\bf Remark} Of course there are many variations possible in the
conditions listed above, some of which are in
449Xk-449Xm.  %449Xk 449Xl 449Xm
}

\leader{449K}{Proposition} Let $G$ be an amenable locally compact
Hausdorff group, and $H$ a subgroup of $G$.   Then $H$ is amenable.

\proof{\cmmnt{({\smc Paterson 88}, 1.12)}
{\bf (a)} For most of the proof (down to the end of (g) below), suppose
that $H$ is closed.   Let $V$ be a compact neighbourhood of the identity in
$G$.   Let $I\subseteq G$ be a maximal set such that
$VzH\cap Vz'H=\emptyset$ for all distinct $z$, $z'\in I$.
Then $V^{-1}VIH=G$.   \Prf\ If $x\in G$, there is a $z\in I$ such that
$VxH\cap VzH\ne\emptyset$, that is,

\Centerline{$x\in V^{-1}VzHH^{-1}=V^{-1}VzH\subseteq V^{-1}VIH$.  \Qed}

\medskip

{\bf (b)} If $x\in G$ then

\Centerline{$I\cap V^{-1}xH
=\{z:z\in I$, $z\in V^{-1}xH\}
=\{z:z\in I$, $x\in VzH^{-1}=VzH\}$}

\noindent has at most one element.   If $K\subseteq G$ is compact, then
there is a finite set $J\subseteq G$ such that $K\subseteq V^{-1}J$, and
now

\Centerline{$I\cap KH\subseteq\bigcup_{x\in J}I\cap V^{-1}xH$}

\noindent is finite.

\medskip

{\bf (c)} Let $h\in C_k(X)^+$ be such that $h\ge\chi(V^{-1}V)$;  write
$W$ for the support of $h$.   Set $g(x)=\sum_{z\in I}h(xz^{-1})$ for
$x\in G$.

If $K\subseteq G$ is compact, then

$$\eqalign{&\{z:z\in I,\,h(xyz^{-1})\ne 0
   \text{ for some }x\in K\text{ and }y\in H\}\cr
&\mskip100mu
\subseteq I\cap\{z:KHz^{-1}\cap W\ne\emptyset\}\cr
&\mskip100mu
=I\cap\{z:zH^{-1}K^{-1}\cap W^{-1}\ne\emptyset\}
=I\cap W^{-1}KH\cr}$$

\noindent is finite.   In particular, $\{z:z\in I$, $h(xz^{-1})\ne 0\}$
and $g(x)$ are finite for every $x\in G$.   Next, if $x_0\in G$,
then $J=\{z:h(xz^{-1})\ne 0$ for some $x\in x_0V\}$ is finite, and
$g(x)=\sum_{z\in J}h(xz^{-1})$ for $x\in x_0V$, so $g$ is continuous at
$x_0$;  as $x_0$ is arbitrary, $g\in C(G)$.   Of course $g\ge 0$ because
$h\ge 0$.

\medskip

{\bf (d)(i)} For $x\in G$, set $g_x(y)=g(x^{-1}y)$ for $y\in H$.   Then
$g_x\in C_k(H)$.   \Prf\ $g_x$ is continuous because $g$ is.   Now
$J=\{z:z\in I,\,h(x^{-1}yz^{-1})\ne 0$ for some $y\in H\}$ is finite, and

$$\eqalign{\{y:g_x(y)\ne 0\}
&\subseteq\{y:\text{ there is a }z\in I
   \text{ such that }h(x^{-1}yz^{-1})\ne 0\}
   \cr
&\subseteq\{y:\text{ there is a }z\in J
   \text{ such that }x^{-1}yz^{-1}\in W\}
=xWJ\cr}$$

\noindent which is compact.\ \Qed

\medskip

\quad{\bf (ii)} Moreover, for any $x_0\in G$ there is a compact set
$L\subseteq H$ such that
$\{x:|g_x-g_{x_0}|\le\epsilon\chi L\}$ is a neighbourhood of $x_0$ for
every $\epsilon>0$.   \Prf\

\Centerline{$J=\{z:z\in I,\,h(x^{-1}yz^{-1})\ne 0
   \text{ for some }x\in x_0V\text{ and }y\in H\}$}

\noindent is finite, by (c) in its full strength.   Let $L$ be the compact
set $H\cap x_0VWJ$.

Take any $\epsilon>0$.   If $x\in x_0V$, then
$g_x(y)=\sum_{z\in J}h(x^{-1}yz^{-1})$ for every $y\in H$, and
$g_x(y)=0$ for $y\in H\setminus L$.   Moreover, setting
$g'_x(y)=\sum_{z\in J}h(x^{-1}yz^{-1})$ for $x\in G$ and $y\in H$,
$(x,y)\mapsto g'_x(y)$ is continuous, so $x\mapsto g'_x:G\to C(H)$ is
continuous if we give $C(H)$ the topology of uniform convergence on compact
subsets of $H$ (4A2G(g-i)).   In particular, $x\mapsto g'_x\restr L$ is
continuous for the norm topology of $C(L)$, and
$U
=\{x:x\in x_0V$, $\|g'_x\restr L-g'_{x_0}\restr L\|_{\infty}\le\epsilon\}$
is a neighbourhood of $x_0$.   But if $x\in U$, then

\Centerline{$g_x(y)=g_{x_0}(y)=0$ for $y\in H\setminus L$,}

\Centerline{$|g_x(y)-g_{x_0}(y)|=|g'_x(y)-g'_{x_0}(y)|\le\epsilon$
for $y\in L$,}

\noindent so $|g_x-g_{x_0}|\le\epsilon\chi L$.\ \Qed

\medskip

{\bf (e)} Now take a left Haar measure $\nu$ on $H$.   (This is where it
really matters whether $H$ is closed.)   Define $T:C_b(H)\to\BbbR^G$ by
setting

\Centerline{$(Tf)(x)=\int g_x\times fd\nu
=\int_Hg(x^{-1}y)f(y)\nu(dy)$}

\noindent for $f\in C_b(H)$ and $x\in G$.   Then $Tf\in C(G)$ for every
$f\in C_b(H)$.   \Prf\ Given $x_0\in G$ and $\epsilon>0$, let
$L\subseteq H$ be a compact set as in (d-ii).   Let $\delta>0$ be such that
$\delta\int_L|f|d\nu\le\epsilon$.   Then

\Centerline{$\{x:|(Tf)(x)-(Tf)(x_0)|\le\epsilon\}
\supseteq\{x:|g_x-g_{x_0}|\le\delta\chi L\}$}

\noindent is a neighbourhood of $x_0$.   As $x_0$ and $\epsilon$ are
arbitrary, $Tf$ is continuous.\ \Qed

Clearly, $T:C_b(H)\to C(G)$ is a positive linear operator.   Next, if
$f\in C_b(H)$ and $b\in H$, $T(b\action_lf)=b\action_l(Tf)$.    \Prf\

$$\eqalign{T(b\action_lf)(x)
&=\int_Hg(x^{-1}y)(b\action_lf)(y)\nu(dy)
=\int_Hg(x^{-1}y)f(b^{-1}y)\nu(dy)\cr
&=\int_Hg(x^{-1}by)f(y)\nu(dy)
=(Tf)(b^{-1}x)
=(b\action_lTf)(x)\cr}$$

\noindent for every $x\in G$.\ \Qed

We need to know that $T(\chi H)(x)>0$ for every $x\in G$.   \Prf\
There is a $z\in I$ such that $x^{-1}\in V^{-1}VzH$, as remarked in (a).
Now $x^{-1}Hz^{-1}$ meets $V^{-1}V$, so there is a $y\in H$ such that

\Centerline{$1\le h(x^{-1}yz^{-1})\le g(x^{-1}y)=g_x(y)$}

\noindent and $T(\chi H)(x)=\int_Hg_xd\nu>0$
because $g_x$ is continuous and non-negative and $\nu$ is 
strictly positive.\ \Qed

\medskip

{\bf (f)} We therefore have a positive linear operator
$S:C_b(H)\to C(G)$ defined by setting $Sf=\Bover{Tf}{T(\chi H)}$ for
$f\in C_b(H)$.   Since $S(\chi H)=\chi G$,
$S[C_b(H)]\subseteq C_b(G)$;  moreover, for $f\in C_b(H)$ and
$b\in H$,

$$\eqalign{S(b\action_lf)
&=\Bover{T(b\action_lf)}{T(\chi H)}
=\Bover{T(b\action_lf)}{T(b\action_l\chi H)}\cr
&=\Bover{b\action_l(Tf)}{b\action_l(T\chi H)}
=b\action_l\Bover{Tf}{T\chi H}
=b\action_lSf.\cr}$$

\medskip

{\bf (g)} At this point, recall that by 449J(ii) there is a positive linear
functional $p:C_b(G)\to\Bbb R$ such that $p(\chi G)=1$ and
$p(a\action_lf)=p(f)$ whenever $f\in C_b(G)$ and $a\in G$.
Set $q=pS:C_b(H)\to\Bbb R$;  then
$q$ is a positive linear functional, $q(\chi H)=1$ and
$q(b\action_lf)=q(f)$ whenever $f\in C_b(H)$ and $b\in H$.   So $q$
witnesses that 449J(ii) is true for $H$, and $H$ is amenable.

\medskip

{\bf (h)} All this has been on the assumption that $H$ is closed.   But in
general $\overline{H}$ is a closed subgroup of $G$, therefore amenable by
(a)-(g) here, and $H$ is dense in $\overline{H}$, therefore amenable by
449F(a-ii).
}%end of proof of 449K

%can't we improve the argument in (b), or at least move much of it to 4A5,
%by showing that  \phi  is continuous in a more general case?

%Very likely the method of Paterson 88, 1.12, would be better still,
%if I knew what a Bruhat function was.

\leader{449L}{}\cmmnt{ If we make a further step back towards the
origin of this topic, and suppose that our group is discrete, then we
have a striking further condition to add to the lists above.   I give
this as a corollary of a general result on group actions recalling the
main theorems of \S\S395 and 448.

\medskip

\noindent}{\bf Tarski's theorem} Let $G$ be a group acting on a
non-empty set $X$.   Then the following are equiveridical:

(i) there is an additive functional
$\nu:\Cal PX\to[0,1]$ such that $\nu X=1$ and $\nu(a\action A)=\nu A$
whenever $A\subseteq X$ and $a\in G$;

(ii) there are no
$A_0,\ldots,A_n$, $a_0,\ldots,a_n$, $b_0,\ldots,b_n$ such that
$A_0,\ldots,A_n$ are subsets of $X$ covering $X$,
$a_0,\ldots,a_n,b_0,\penalty-100\ldots,b_n$ belong to $G$, and 
$a_0\action A_0$, $b_0\action A_0$, $a_1\action A_1$,
$b_1\action A_1,\ldots,b_n\action A_n$ are all disjoint.

\proof{{\bf (a)(i)$\Rightarrow$(ii)} This is elementary, for if
$\nu:\Cal PX\to[0,1]$ is a non-zero additive functional and
$A_0,\ldots,A_n$ cover $X$ and $a_0,\ldots,b_n\in G$, then

\Centerline{$\sum_{i=0}^n\nu(a_i\action A_i)
  +\sum_{i=0}^n\nu(b_i\action A_i)
=2\sum_{i=0}^n\nu A_i\ge 2\nu X>\nu X$,}

\noindent and $a_0\action A_0,\ldots,b_n\action A_n$ cannot all be
disjoint.

\medskip

{\bf (b)(ii)$\Rightarrow$(i)} Now suppose that (ii) is true.

\medskip

\quad\grheada\ Suppose that $c_0,\ldots,c_n\in G$.   Then there is a
finite set $I\subseteq X$ such that
$\#(\{c_i\action x:i\le n,\,x\in I\})<2\#(I)$.   \Prf\Quer\ Otherwise,
by the Marriage Lemma in the form 4A1H, applied to the set

\Centerline{$R=\{((x,j),c_i\action x):x\in X,\,j\in\{0,1\},\,i\le n\}
\subseteq(X\times\{0,1\})\times X$,}

\noindent there is an injective function $\phi:X\times\{0,1\}\to X$
such
that $\phi(x,j)\in\{c_i\action x:i\le n\}$ for every $x\in X$ and
$j\in\{0,1\}$.   Now set
$B_{ij}=\{x:\phi(x,0)=c_i\action x,\,\phi(x,1)=c_j\action x\}$ for $i$,
$j\le n$, so that $X=\bigcup_{i,j\le n}B_{ij}$.   Let
$A_{ij}\subseteq B_{ij}$ be such that $\langle A_{ij}\rangle_{i,j\le n}$
is a partition of $X$, and set $a_{ij}=c_i$, $b_{ij}=c_j$ for $i$,
$j\le n$;  then $a_{ij}\action A_{ij}\subseteq\phi[A_{ij}\times\{0\}]$,
$b_{ij}\action A_{ij}\subseteq\phi[A_{ij}\times\{1\}]$ are all
disjoint, which is supposed to be impossible.\ \Bang\Qed

\medskip

\quad\grheadb\ Suppose that $J\subseteq G$ is finite and $\epsilon>0$.
Then there is a non-empty finite set $I\subseteq X$ such that
$\#(I\symmdiff c\action I)\le\epsilon\#(I)$ for every $c\in J$.
\Prf\ It is enough to consider the case in which the identity $e$ of
$G$ belongs to $J$.   \Quer\ Suppose, if possible, that there is no such
set $I$.    Let $m\ge 1$ be such that $(1+\bover12\epsilon)^m\ge 2$.   Set
$K=J^m=\{c_1c_2\ldots c_m:c_1,\ldots,c_m\in J\}$.   By ($\alpha$),
there
is a finite set $I_0\subseteq X$ such that $\#(I_0^*)<2\#(I_0)$, where
$I_0^*=\{a\action x:a\in K,\,x\in I_0\}$.   Choose $c_1,\ldots,c_m$
and $I_1,\ldots,I_m$ inductively such that

\inset{\noindent given that $I_k$ is a non-empty finite subset of $X$,
where $0\le k<m$, take $c_{k+1}\in J$ such that
$\#(I_k\symmdiff c_{k+1}\action I_k)>\epsilon\#(I_k)$ and set
$I_{k+1}=I_k\cup c_{k+1}\action I_k$.}

\noindent Then $I_k\subseteq\{a\action x:a\in J^k,\,x\in I_0\}$ for
each $k\le m$, and in particular $I_m\subseteq I_0^*$.   But also

$$\eqalign{\#(I_{k+1})
&=\#(I_k)+\#((c_{k+1}\action I_k)\setminus I_k)\cr
&=\#(I_k)+\Bover12\#((c_{k+1}\action I_k)\symmdiff I_k)
\ge(1+\Bover12\epsilon)\#(I_k)\cr}$$

\noindent for every $k<m$, so

\Centerline{$\#(I^*_0)\ge\#(I_m)\ge(1+\Bover12\epsilon)^m\#(I_0)
\ge 2\#(I_0)$,}

\noindent contrary to the choice of $I_0$.\ \Bang\Qed

\medskip

\quad\grheadc\ There is therefore an ultrafilter $\Cal F$ on
$[X]^{<\omega}\setminus\{\emptyset\}$ such that

\Centerline{$\Cal A_{c\epsilon}
=\{I:I\in[X]^{<\omega}\setminus\{\emptyset\},\,
\#(I\symmdiff c\action I)\le\epsilon\#(I)\}$}

\noindent belongs to $\Cal F$ for every $c\in G$ and $\epsilon>0$.
For
$I\in[X]^{<\omega}\setminus\{\emptyset\}$ and $A\subseteq X$ set
$\nu_I(A)=\#(A\cap I)/\#(I)$, and set $\nu A=\lim_{I\to\Cal F}\nu_IA$
for every $A\subseteq X$, so that $\nu:\Cal PX\to[0,1]$ is an additive
functional and $\nu X=1$.

Now $\nu$ is $G$-invariant.   \Prf\ If $A\subseteq X$ and $c\in G$ and
$\epsilon>0$, then $\Cal A_{c^{-1},\epsilon}\in\Cal F$.   If
$I\in\Cal A_{c^{-1},\epsilon}$, then

$$\eqalign{|\nu_I(c\action A)-\nu_I(A)|
&=\Bover1{\#(I)}|\#(I\cap(c\action A))-\#(I\cap A)|\cr
&=\Bover1{\#(I)}|\#((c^{-1}\action I)\cap A)-\#(I\cap A)|\cr
&\le\Bover1{\#(I)}\#((c^{-1}\action I)\symmdiff I)
\le\epsilon.\cr}$$

\noindent As $\epsilon$ is arbitrary,
$\lim_{I\to\Cal F}\nu_I(c\action A)-\nu_I(A)=0$, and
$\nu(c\action A)=\nu A$.   As $A$ and $c$ are arbitrary, $\nu$ is
$G$-invariant.\ \Qed

So $\nu$ witnesses that (i) is true, and the proof is complete.
}%end of proof of 449L

\leader{449M}{Corollary} Let $G$ be a group with its discrete
topology.   Then the following are equiveridical:

(i) $G$ is amenable;

(ii) there are no
$A_0,\ldots,A_n$, $a_0,\ldots,a_n$, $b_0,\ldots,b_n$ such that
$G=\bigcup_{i\le n}A_i$
$a_0,\ldots,a_n,b_0,\ldots,b_n$ belong to $G$, and $a_0A_0$,
$b_0A_0$, $a_1A_1$,
$b_1A_1,\ldots,b_nA_n$ are disjoint.

\proof{ All we have to observe is that ($\alpha$) every function from
$G$ to $\Bbb R$ is uniformly continuous for the right uniformity of
$G$,
so that $G$ is amenable iff there is an invariant positive linear
functional $p:\ell^{\infty}(G)\to\Bbb R$ such that $p(\chi G)=1$
($\beta$) that a positive linear functional on $\ell^{\infty}(G)$ is
$G$-invariant iff the corresponding additive functional on $\Cal PG$
is
$G$-invariant.   So (i) of 449L is equivalent to amenability of $G$ as
defined in 449A.
}%end of proof of 449M

\leader{449N}{Theorem} Let $G$ be a group which is amenable in its
discrete topology, $X$ a set, and $\action$ an action of $G$ on $X$.
Let $\Cal E$ be a subring of $\Cal PX$ and
$\nu:\Cal E\to\coint{0,\infty}$ a finitely additive functional which
is $G$-invariant in the sense that $g\action E\in\Cal E$ and
$\nu(g\action E)=\nu E$ whenever $E\in\Cal E$ and $g\in G$.   Then
there is an extension of $\nu$ to a $G$-invariant non-negative finitely
additive functional $\tilde\nu$ defined on the ideal $\Cal I$ of
subsets of $X$ generated by $\Cal E$.

\proof{{\bf (a)} There is a non-negative finitely additive functional
$\theta:\Cal I\to\Bbb R$ extending $\nu$.   \Prf\ Let $V$ be the
linear subspace of $\ell^{\infty}(X)$ generated by $\{\chi E:E\in\Cal E\}$,
so that $V$ can be identified with the Riesz space $S(\Cal E)$ (361L).
Let $U$ be the solid linear subspace of $\ell^{\infty}(X)$ generated
by $V$.   For $u\in U$ set
$q(u)=\inf\{\dashint v\,d\nu:v\in V$, $|u|\le v\}$, where
$\dashint\,\,d\nu:S(\Cal E)\to\Bbb R$ is the positive linear functional
corresponding to $\nu:\Cal E\to\coint{0,\infty}$ as in 361F-361G.
Then $q$ is a seminorm, and $|\dashint v\,d\nu|\le q(v)$ for every
$v\in V$.   So there is a linear functional $f:U\to\Bbb R$ such that
$f(v)=\dashint v\,d\nu$ for every $v\in V$ and $|f(u)|\le q(u)$ for every
$u\in U$ (4A4D(a-i)).   Set $\theta A=f(\chi A)$ for $A\in\Cal I$.
Then $\theta:\Cal I\to\Bbb R$ is additive and extends $\nu$.   If
$A\in\Cal I$, there is an $E\in\Cal E$ including $A$.   Now we have

\Centerline{$\theta(E\setminus A)=f(\chi E-\chi A)
\le q(\chi E-\chi A)\le\dashint\chi E\,d\nu=\nu E=\theta E$,}

\noindent so $\theta A\ge 0$.   So $\theta$ is non-negative.\ \Qed

\medskip

{\bf (b)} As in the proof of 449M, we have a positive $G$-invariant
linear functional $p:\ell^{\infty}(G)\to\Bbb R$ such that $p(\chi G)=1$.
For $A\in\Cal I$, set $f_A(a)=\theta(a^{-1}\action A)$ for $a\in G$, and
$\tilde\nu A=p(f_A)$.   Then $\tilde\nu:\Cal I\to\coint{0,\infty}$ is
additive.   If $E\in\Cal E$ then $f_A(a)=\nu E$ for every $a$, so
$\tilde\nu$ extends $\nu$.   If $A\in\Cal I$ and $a$, $b\in G$, then
$f_{aA}(b)=\theta(b^{-1}aA)=f_A(a^{-1}b)$, so $f_{aA}=a\action_lf_A$
and

\Centerline{$\tilde\nu(aA)=p(f_{aA})=p(f_A)=\tilde\nu A$.}

\noindent Thus $\tilde\nu$ is $G$-invariant, as required.
}%end of proof of 449N

\leader{449O}{Corollary}\cmmnt{ ({\smc Banach 1923})} If $r=1$ or $r=2$,
there is a functional $\theta:\Cal P\BbbR^r\to[0,\infty]$ such that
(i) $\theta(A\cup B)=\theta A+\theta B$ whenever $A$,
$B\subseteq\BbbR^r$ are disjoint (ii) $\theta E$ is the Lebesgue
measure of $E$ whenever $E\subseteq\BbbR^r$ is Lebesgue measurable
(iii) $\theta(g[A])=\theta A$ whenever $A\subseteq\BbbR^r$ and
$g:\BbbR^r\to\BbbR^r$ is an isometry.

\proof{{\bf (a)} The point is that the group $G$ of all isometries of
$\BbbR^r$, with its discrete topology, is amenable.   \Prf\ Let
$G_0\subseteq G$ be the subgroup consisting of rotations about
$\tbf{0}$;  because $r\le 2$, this is abelian, therefore amenable
(449Cf).   Let $G_1\subseteq G$ be the subgroup consisting of
isometries
keeping $\tbf{0}$ fixed;  then $G_0$ is a normal subgroup of $G_1$,
and $G_1/G_0$ is abelian, so $G_1$ is amenable (449Cc).   Let
$G_2\subseteq G$ be the normal subgroup consisting of translations;
then $G_2$ is abelian, therefore amenable.   Now $G=G_1G_2$, so $G$ is
amenable (449Cd).\ \Qed

\medskip

{\bf (b)} Let $\Cal E$ be the ring of subsets of $\BbbR^r$ with finite
Lebesgue measure, and let $\nu$ be the restriction of Lebesgue measure
to $\Cal E$.   Then $\nu$ is $G$-invariant.   By 449N, there is a
$G$-invariant extension $\tilde\nu$ of $\nu$ to the ideal $\Cal I$
generated by
$\Cal E$.   Setting $\theta A=\tilde\nu A$ for $A\in\Cal I$,
$\infty$
for $A\in\Cal P\BbbR^r\setminus\Cal I$, we have a suitable functional
$\theta$.
}%end of proof of 449O

\exercises{\leader{449X}{Basic exercises $\pmb{>}$(a)}
%\sqheader 449Xa
Let $G$ be a topological group.   On $G$ define a binary operation
$\diamond$ by saying that $x\diamond y=yx$ for all $x$,
$y\in G$.   Show that $(G,\diamond)$ is a topological group isomorphic
to $G$, so is amenable iff $G$ is.
%449A

\spheader 449Xb
Show that any finite topological group is amenable.
%449C

\sqheader 449Xc Show that, for any $r\in\Bbb N$, the isometry group of
$\BbbR^r$, with the topology of pointwise convergence, is amenable.
\Hint{443Xw, 449Cd.}
%441G isometry group, 449C

\spheader 449Xd Find a locally compact Polish
group which is amenable but not unimodular.
\Hint{442Xf, 449Cd.}
%449C

\spheader 449Xe\dvAnew{2013} Prove 449Cg directly from 441C, without
mentioning Haar measure.
%449C 

\spheader 449Xf Let $G$ be a topological group and $U$ the space of
bounded real-valued functions on $G$ which are uniformly continuous
for the right uniformity of $G$.   Show that $\action_r$, as defined in
4A5Cc, gives an action of $G$ on $U$.
%449D

%find example in which it is not continuous for $\|\,\|_{\infty}$ on
%$U$?

\spheader 449Xg Let $\action$ be an action of a group $G$ on a set
$X$, and $U$ a Riesz subspace of $\ell^{\infty}(X)$, containing the
constant functions, such that $a\action f\in U$ whenever $f\in U$
and $a\in G$.
Show that the following are equiveridical:  (i) there is a
$G$-invariant
positive linear functional $p:U\to\Bbb R$ such that $p(\chi X)=1$;
(ii)
$\sup_{x\in X}\sum_{i=0}^nf_i(x)-f_i(a_i\action x)\ge 0$ whenever
$f_0,\ldots,f_n\in U$ and $a_0,\ldots,a_n\in G$.   \Hint{if (ii) is
true, let $V$ be the linear subspace generated by
$\{f-a\action f:f\in U$, $a\in G\}$ and show that
$\inf_{g\in V}\|g-\chi X\|_{\infty}=1$.}
%449E

\sqheader 449Xh Let $X$ be a set and $G$ the group of all
permutations of $X$.   (i) Give $X$ the zero-one metric,
so that $G$ is the
isometry group of $X$.   Show that $G$, with the topology of
pointwise convergence (441G), is amenable.   \Hint{for any
$I\in[X]^{<\omega}$,
$\{a:a\in G,\,a(x)=x$ for every $x\in X\setminus I\}$ is amenable.}
(ii) Show that if $X$ is infinite then $G$, with its discrete
topology, is not amenable.   \Hint{the left action of $F_2$ on itself
can be regarded as an embedding of $F_2$ in $G$.}
%449C 449Fa 449G

\spheader 449Xi Let $G$ be a Hausdorff topological group, and $\hat G$
its completion with respect to its bilateral uniformity (definition:
4A5Hb).   Show that $G$ is amenable iff $\hat G$ is.
%449Cb 449Fa

\spheader 449Xj(i) Let $G$ be the group with generators $a$, $b$ and
relations $a^2=b^3=e$ (that is, the quotient of the free group on two
generators $a$ and $b$ by the normal subgroup generated by
$\{a^2,b^3\}$).   Show that $G$, with its discrete topology, is not
amenable.
(ii) Let $G$ be the group with generators $a$, $b$ and
relations $a^2=b^2=e$.   Show that $G$, with its discrete topology, is
amenable.   \Hint{we have a function $\length:G\to\Bbb N$ such that
$\length(ab)\le\length(a)+\length(b)$ for all $a$, $b\in G$ and
$\limsup_{n\to\infty}\Bover1n\#(\{a:\length(a)\le n\})$ is finite.
See also 449Yf.}
%449Xj 449G

\spheader 449Xk Let $G$ be a locally compact Hausdorff group, and $\mu$ a
left Haar measure on $G$.   Show that $G$ is amenable iff
for every finite set $I\subseteq G$, finite set
$J\subseteq\eusm L^{\infty}(\mu)$ and $\epsilon>0$, there is an
$h\in C_{k1}(G)^+$ (definition:  449J) such that
$|\int f(ax)h(x)\mu(dx)-\int f(x)h(x)\mu(dx)|\le\epsilon$
for every $a\in I$ and $f\in J$.   \Hint{the image of the unit ball in
$L^1$ is weak* dense in the unit ball of $(L^{\infty})^*$.}
%449J

\spheader 449Xl Let $G$ be a locally compact Hausdorff group,
and $\mu$ a left Haar measure on $G$.   Show that the following are
equiveridical:
(i) $G$ is amenable;
(ii) there is a positive linear functional
$p^{\#}:L^{\infty}(\mu)\to\Bbb R$ such that
$p^{\#}(\chi G^{\ssbullet})=1$ and
$p^{\#}(a\action_ru)=p^{\#}(u)$ for every $u\in L^{\infty}(\mu)$ and
every $a\in G$;
(iii) for every finite set $I\subseteq G$, finite set
$J\subseteq\eusm L^{\infty}(\mu)$ and $\epsilon>0$, there is an
$h\in C_{k1}^+$
such that $|\int f(xa)h(x)\mu(dx)-\int f(x)h(x)\mu(dx)|\le\epsilon$
for every $a\in I$ and $f\in J$.
%449J

\spheader 449Xm\dvAnew{2010} Let $G$ be a locally compact Hausdorff
group and
$\CalBa_G$ its Baire $\sigma$-algebra.    Show that
$G$ is amenable iff there is a non-zero finitely additive
$\phi:\CalBa_G\to[0,1]$
such that $\phi(aE)=\nu E$ for every $a\in G$ and $E\in\CalBa_G$.
%449J

\spheader 449Xn A {\bf symmetric F{\o}lner sequence} in a group $G$ is a
sequence $\sequencen{L_n}$ of non-empty finite
symmetric subsets of $G$ such that
$\lim_{n\to\infty}\Bover{\#(L_n\symmdiff aL_n)}{\#(L_n)}=0$ for every
$a\in G$.   Show that a group $G$ has a symmetric F{\o}lner sequence iff
it is countable and amenable when given its discrete topology.
%449J

\sqheader 449Xo Let $G$ be a group which is amenable when given its
discrete topology.   Let $\phi:\Cal PG\to[0,1]$ be an additive
functional such that $\phi G=1$ and $\phi(aE)=\phi E$ whenever
$E\subseteq G$ and $a\in G$.   For $E\subseteq G$ set
$\psi E=\dashint\phi(Ex)\phi(dx)$.   Show that
$\psi:\Cal PG\to[0,1]$ is an additive functional, that
$\psi G=1$ and that $\psi(aE)=\psi(Ea)=\psi E$ for
every $E\subseteq G$ and $a\in G$.
%449M

\spheader 449Xp Let $X$ be a non-empty set, $G$ a group and $\action$
an action of $G$ on $X$.   Suppose that $G$ is an amenable group when
given its discrete topology.   Show that there is an additive functional
$\nu:\Cal PX\to[0,1]$ such that $\nu X=1$ and
$\nu(a\action A)=\nu A$ for every $A\subseteq X$ and every $a\in G$.
%449M

\spheader 449Xq
Let $G$ be a locally compact Hausdorff group and
$\mu$ a left Haar measure on $G$.   Show that if $G$, with its discrete
topology, is amenable, then there is a functional
$\phi:\Cal PG\to[0,\infty]$, extending $\mu$, such that
$\phi(A\cup B)=\phi A+\phi B$ whenever $A$, $B\subseteq G$
are disjoint and $\phi(xA)=\phi A$ for every $x\in G$ and
$A\subseteq G$.
%449N

\spheader 449Xr Let $X$ be a compact metrizable space, $\phi:X\to X$ a
continuous function and $\mu$ a Radon probability measure on $X$ such
that $\mu\phi^{-1}=\mu$.   Show that for $\mu$-almost every $x\in X$,
$\lim_{n\to\infty}\Bover1{n+1}\sum_{i=0}^nf(\phi^i(x))$ is defined for
every $f\in C(X)$.   \Hint{4A2Pe, 372J.}
%+  372J Ergodic Theorem %4A2Pe  $C(X)$ is separable.

\leader{449Y}{Further exercises (a)}
%\spheader 449Ya
If $S$ is a semigroup with identity $e$ and $X$ is a set, an
{\bf action} of $S$ on $X$ is a map
$(s,x)\mapsto s\action x:S\times X\to X$
such that $s\action(t\action x)=(st)\action x$ and $e\action x=x$ for
every $s$, $t\in S$ and $x\in X$.   A topological semigroup $S$ with
identity is {\bf amenable} if for every non-empty compact Hausdorff
space $X$ and every continuous action of $S$ on $X$ there is a Radon
probability measure $\mu$ on $X$ such that
$\int f(s\action x)\mu(dx)=\int f(x)\mu(dx)$ for every $s\in S$ and
$f\in C(X)$.   Show that

\quad(i) $(\Bbb N,+)$, with its discrete topology, is amenable;

\quad(ii) if $S$ is a topological semigroup and $\Cal S$ is an
upwards-directed family of amenable sub-semigroups of $S$ with dense
union in $S$, then $S$ is amenable;

\quad(iii) if $\familyiI{S_i}$ is a family of amenable topological
semigroups with product $S$ then $S$ is amenable;

\quad(iv) if $S$ is an amenable topological semigroup, $S'$ is a
topological
semigroup, and there is a continuous multiplicative surjection from
$S$ onto $S'$, then $S'$ is amenable;

\quad(v) if $S$ is an abelian topological semigroup, then it is
amenable.
%449C

\spheader 449Yb Give an example of a topological semigroup  $S$ with
identity such that $S$ is amenable
in the sense of 449Ya but $(S,\diamond)$ is not, where $a\diamond b=ba$
for $a$, $b\in S$.
%449Ya 449C 44bits

\spheader 449Yc Let $G$ be a topological group and $U$ the space of
bounded real-valued functions on $G$ which are uniformly
continuous for the right uniformity.   Let $M_{\text{qR}}^+$ be the
space of totally finite quasi-Radon measures on $G$.   (i) Show that if
$\nu\in M_{\text{qR}}^+$ then $\nu*f$ (definition:
444H) belongs to $U$ for every $f\in U$.
(ii) Show that $(\nu,f)\mapsto\nu*f:M_{\text{qR}}^+\times U\to U$ is
continuous if $M_{\text{qR}}^+$ is given its narrow topology and $U$ is
given its norm topology.   (iii) Show that if
$p:U\to\Bbb R$ is a continuous linear functional such that
$p(a\action_lf)=p(f)$ for every $f\in U$ and $a\in G$, then
$p(\nu*f)=\nu G\cdot p(f)$ for every $f\in U$ and every totally finite
quasi-Radon measure $\nu$ on $G$.
%449H

\spheader 449Yd Re-work 449J for general groups carrying Haar
measures.
%449J

\spheader 449Ye Let $G$ be a group with a symmetric F{\o}lner sequence
$\sequencen{L_n}$ (449Xn), and $\action$ an action of $G$ on a reflexive
Banach space $U$ such that $u\mapsto a\action u$ is a linear operator of
norm at most $1$ for every $a\in G$.   For $n\in\Bbb N$ set
$T_nu=\Bover1{\#(L_n)}\sum_{a\in L_n}a\action u$ for $u\in U$.
Show that
for every $u\in U$ the sequence $\sequencen{T_nu}$ is norm-convergent
to a $v\in U$ such that $a\action v=v$ for every $a\in G$.   \Hint{372A.
See also 461Yg below.}
%449Xn 372A 449J

\spheader 449Yf
Let $G$ be a locally compact Hausdorff group and
$\mu$ a left Haar measure on $G$.   Suppose that $G$ is {\bf exponentially
bounded}, that is,
$\limsup_{n\to\infty}(\mu(K^n))^{1/n}\le 1$ for every compact set
$K\subseteq G$.    Show that $G$ is amenable.
%449J(xiv)

\spheader 449Yg\dvAnew{2010} Let
$G$ be a group and $\action$ an action of $G$ on a set
$X$.   Let $\Tau$ be an algebra of subsets of $X$ such that
$g\action E\in\Tau$ for every $E\in\Tau$ and $g\in G$, and $H$ a member
of $\Tau$;   write $\Tau_H$ for $\{E:E\in\Tau$, $E\subseteq H\}$.
Let $\nu:\Tau_H\to[0,\infty]$ be
a functional which is additive in the sense that $\nu\emptyset=0$ and
$\nu(E\cup F)=\nu E+\nu F$ whenever $E$, $F\in\Tau_H$ are disjoint,
and locally $G$-invariant in the sense that $g\action E\in\Tau$ and
$\nu(g\action E)=\nu E$ whenever $E\in\Tau_H$, $g\in G$ and
$g\action E\subseteq H$.   Show that
there is an extension of $\nu$ to a $G$-invariant additive functional
$\tilde\nu:\Tau\to[0,\infty]$.
%for 449Yh

\spheader 449Yh
Let $X$ be a set, $A$ a subset of $X$, and $\action$ an
action of a group $G$ on $X$.
Show that the following are equiveridical:  (i)
there is a functional $\theta:\Cal PX\to[0,\infty]$ such that
$\theta A=1$, $\theta(B\cup C)=\theta B+\theta C$ and
$\theta(a\action B)=\theta B$ for all disjoint
$B$, $C\subseteq X$ and $a\in G$;   (ii) there are no
$A_0,\ldots,A_n$, $a_0,\ldots,a_n$, $b_0,\ldots,b_n$ such that
$A_0,\ldots,A_n$ are subsets of $G$ covering $A$,
$a_0,\ldots,b_n$ belong to $G$, and $a_0\action A_0$,
$b_0\action A_0$,
$a_1\action A_1$, $b_1\action A_1,\ldots,b_n\action A_n$ are
disjoint subsets of $A$.
%449Yg 449L mt44bits

\spheader 449Yi ({\smc Swierczkowski 58}) Let $G$ be the group of
orthogonal $3\times 3$ matrices.   Set
$S=\Matrix{\bover35&\bover45&0\\-\bover45&\bover35&0\\0&0&1}$ and
$T=\Matrix{1&0&0\\0&\bover35&\bover45\\0&-\bover45&\bover35}$.   Show
that $S$ and $T$ are free in $G$ (that is, no non-trivial product
of the form $S^{n_0}T^{n_1}S^{n_2}T^{n_3}\ldots
S^{n_{2k}}$ can be the identity), so that
$G$ is not amenable in its discrete topology.   \Hint{let $R$ be the
ring of $3\times 3$ matrices over the field $\Bbb Z_5$.   In $R$ set
$\sigma=\Matrix{3&4&0\\1&3&0\\0&0&0}$,
$\tau=\Matrix{0&0&0\\0&3&4\\0&1&3}$.   Show that
$\sigma^2=\sigma$.   Now suppose $\rho\in R$ is defined as a
non-trivial
product of the elements $\sigma$, $\tau$ and their transposes $\sigma\trs$,
$\tau\trs$ in which $\sigma$ and $\sigma\trs$ are never adjacent,
$\tau$ and $\tau\trs$ are never adjacent, and the last term is
$\sigma$.   Set $\Matrix{a\\b\\c}=\rho\Matrix{1\\0\\0}$.   Show that if
the first term in the product is $\sigma$ or $\sigma\trs$, then $c=0$
and $b\ne 0$, and otherwise $a=0$ and $b\ne 0$.}
%449O

\spheader 449Yj
Let $F_2$ be the free group on two generators $a$,
$b$.   (i) Show that there is a partition $(A,B,C,D)$ of
$F_2$ such that $aA=A\cup B\cup C$ and $bB=A\cup B\cup D$.
(ii) Let $S_2$ be the unit
sphere in $\BbbR^3$.   Show that if $S$, $T$ are the matrices of
449Yi, there is a partition $(A,B,C,D,E)$ of $S^2$ such that
$E$ is countable, $S[A]=A\cup B\cup C$ and $T[B]=A\cup B\cup D$.
(This is a version of the {\bf Hausdorff paradox}.)
(iii) Show that there is no non-zero rotation-invariant additive
functional from $\Cal PS_2$ to $[0,1]$.   (iv) Show that there is no
rotation-invariant additive
extension of Lebesgue measure to all subsets of the
unit ball $B(\tbf{0},1)$.   (See {\smc Wagon 85}.)
%449Yi 449O Wagon
}%end of exercises

\endnotes{
\Notesheader{449}
The general theory of amenable groups is outside the scope of this book.
Here I have tried only to indicate some of the specifically
measure-theoretic arguments which are used in the theory.   Primarily we
have the Riesz representation theorem, enabling us to move between
linear functionals and measures.   Since the invariant measures
considered in the definition of `amenable group' are all Radon measures
on compact Hausdorff spaces, they can equally well be thought of as
linear functionals on spaces of continuous functions.   What is striking
is that the definition in terms of continuous actions on arbitrary
compact Hausdorff spaces can be reduced to a question concerning an
invariant mean on a single space easily constructed from the group
(449E).

The first part of this section deals with general topological groups.
It is a remarkable fact that some of the most important
non-locally-compact topological groups are amenable.   For most of
these we shall have to wait until we have done `concentration of measure'
(\S\S476, 492) and can approach `extremely amenable' groups (\S493).
But there is an easy example in 449Xh which already indicates one of
the basic methods.

For a much fuller account of the theory of amenable locally compact groups,
see {\smc Paterson 88}.
Theorem 449J here is mostly taken from {\smc Greenleaf 69}, where you
will find many references to its development.   Historically the subject
was dominated by the case of discrete groups, in which combinatorial
rather than measure-theoretic formulations seem more appropriate.   In
449J, conditions (ii)-(viii) relate to invariant means of one kind or
another, strengthening that of 449E.
Note that the means of
449E(iii) and 449J(ii)-(viii) are normalized by conditions
$p(\chi G)=\tilde p(\chi G^{\ssbullet})=\phi G=1$,
while the left Haar
measure $\mu$ of 449J has a degree of freedom;  so that when they come
together in 449J(viii) the two sides of the equation
$\tilde p(g*f)^{\ssbullet}=\tilde p(f^{\ssbullet})\int g\,d\mu$ must
move together if we change $\mu$ by a scalar factor.   Of course this
happens through the hidden dependence of the convolution operation on
$\mu$.   (The convolutions in 449J(vii) do {\it not} involve $\mu$.)
Between 449J(i) and 449J(viii) there is a double step.   First
we note that a convolution $g*f$ is a kind of weighted average of left
translates of $f$, so that if we have a mean which is invariant under
translations we can hope that it will be invariant under convolutions
(449H, 449Yc).   What is more remarkable is that an invariant mean on
the space $U$ of bounded uniformly continuous functions should in the first
place give rise to a left-invariant mean on the space
$L^{\infty}$ of (equivalence classes of) bounded Haar measurable
functions (449J(vi)), and then even a two-sided-invariant mean (449J(v)).

Condition (xii) in 449J looks at a different aspect of the
phenomenon.   In effect, it amounts to saying that not only is there
an invariant mean, but that there is an invariant mean defined by the
formula $p(f)=\lim_{L\to\Cal F}\Bover1{\mu L}\int_Lf$ for a suitable
filter $\Cal F$ on the family of sets of non-zero finite measure.
This may be called a `F{\o}lner condition', following
{\smc F{\o}lner 55}.   (449Xk looks for an invariant mean of
the form $p(f)=\lim_{h\to\Cal F}\int f\times h\,d\mu$, where $\Cal F$
is a suitable filter on $\eusm L^1(\mu)$.)   In 449J(xi), the case $q=1$
is just a weaker version of condition (x), but the case $q=2$ tells us
something new.

The techniques developed in \S444 to handle Haar measures on groups
which are not locally compact can also be used in 449H-449J, using
`totally bounded for the bilateral uniformity' in place of `compact'
when appropriate (449Yd).   443L provides another route to the same
generalization.

In 449K, it is natural to ask whether the hypothesis `locally compact' is
necessary.   It certainly cannot be dropped completely;  there is an
important
amenable Polish group with a closed subgroup which is not amenable
(493Xf).

The words `right' and `left' appear repeatedly in this section, and it
is not perhaps immediately clear which of the ordinary symmetries we can
expect to find.   The fact that the operation $x\mapsto x^{-1}:G\to G$
always gives us an isomorphism between a group and the same set with the
multiplication reversed (449Xa) means that we do not have to distinguish
between `left amenable' and `right amenable' groups, at least if we
start from the definition in 449A.   In 449C also there is nothing to
break the symmetry between left and right.   In 449B and 449D-449E,
however, we must commit ourselves to the {\it left} action of the group
on the space of functions which are uniformly continuous with respect to
the {\it right} uniformity.   If we wish to change one, we must also
expect to change the other.   In the list of conditions
in 449J, some can be reflected straightforwardly (see 449Xl),
but in groups which are not unimodular there seem to be difficulties.
For semigroups we do have a difference
between `left' and `right' amenability (449Yb).

It is not surprising that in the search for invariant means we should
repeatedly use averaging and limiting processes.   The
`finitely additive integrals' $\dashint\,d\phi$, $\dashint\,d\nu$
in part (f) the proof of 449J and part (a) of the proof of 449N
are an effective way of using one
invariant additive functional $\phi$ or $\nu$ to build another.
Similarly, because we are looking only for finite additivity,
we can be optimistic about taking cluster
points of families of almost-invariant functionals, as in the proofs of
449F, 449J and 449L.

In the case of discrete groups, in which all considerations of
measurability and continuity evaporate, we have a completely different
technique available, as in 449L.   Here we can go directly from a
non-paradoxicality condition, a weaker version of conditions already
introduced in 395E and 448E,
to a F{\o}lner condition (($\beta$) in part
(b) of the proof of 449L) which easily implies amenability.   I remind
you that I still do not know how far these ideas can be applied to
other algebras than $\Cal PX$ (395Z).   The difficulty is that the
unscrupulous use of the axiom of choice in the infinitary Marriage
Lemma seems to give us no control over the nature of the sets
$A_{ij}$ described in (b-$\alpha$) of the proof of 449L;  moreover, the
structure
of the proof depends on having a suitable invariant measure (counting
measure on $X$) to begin with.   For more on amenable discrete groups
and their connexions with measure theory see {\smc Laczkovich 02}.
}%end of notes

\discrpage

