\frfilename{mt3a2.tex}
\versiondate{22.11.07}
\copyrightdate{1995}

\def\chaptername{Appendix}
\def\sectionname{Rings}

\newsection{3A2}

I give a very brief outline of the indispensable parts of the elementary
theory of (commutative) rings.   I assume that you have seen at least a
little group theory.

\leader{3A2A}{Definition} A {\bf ring} is a triple $(R,+,.)$ such that

\inset{$(R,+)$ is an abelian group;  its identity
will\cmmnt{ always} be denoted $0$ or $0_R$;}

\inset{$(R,.)$ is a semigroup, that is, $ab\in R$ for all $a$, $b\in R$
and $a(bc)=(ab)c$ for all $a$, $b$, $c\in R$;}

\inset{$a(b+c)=ab + ac$, $(a+b)c=ac+bc$ for all $a$, $b$, $c\in R$.}

\noindent A {\bf commutative ring} is one in\cmmnt{ which
multiplication is commutative, that is,} $ab=ba$ for all $a$, $b\in R$.

\leader{3A2B}{Elementary facts} Let $R$ be a ring.

\header{3A2Ba}{\bf (a)} $a0=0a=0$ for every $a\in R$.   \prooflet{\Prf\

\Centerline{$a0=a(0+0)=a0+a0$,
\quad$0a=(0+0)a=0a+0a$;}

\noindent because $(R,+)$ is a group, we may subtract $a0$ or $0a$ from
each side of the appropriate equation to see that $0=a0$, $0=0a$.\ \Qed}

\header{3A2Bb}{\bf (b)} $(-a)b=a(-b)=-(ab)$ for all $a$, $b\in R$.
\prooflet{\Prf\

\Centerline{$ab+((-a)b)=(a+(-a))b=0b=0=a0=a(b+(-b))=ab+a(-b)$;}

\noindent subtracting $ab$ from each term, we get
$(-a)b=-(ab)=a(-b)$.\ \Qed}

\leader{3A2C}{Subrings} If $R$ is a ring, a {\bf subring} of $R$ is a
set $S\subseteq R$ such that $0\in S$ and $a+b$, $ab$, $-a$ belong to
$S$ for all $a$, $b\in S$.   In this case $S$, together with the
addition and multiplication induced by those of $R$, is a ring in its
own right.

\leader{3A2D}{Homomorphisms (a)} Let $R$, $S$ be two rings.   A function
$\phi:R\to S$ is a {\bf ring homomorphism} if
$\phi(a+b)=\phi(a)+\phi(b)$ and $\phi(ab)=\phi(a)\phi(b)$ for all $a$,
$b\in R$.      The {\bf kernel} of $\phi$ is $\{a:a\in
R,\,\phi(a)=0_S\}$.

\header{3A2Db}{\bf (b)} Note that if $\phi:R\to S$ is a ring
homomorphism, then it is also a group homomorphism from $(R,+)$ to
$(S,+)$, so that $\phi(0_R)=0_S$ and
$\phi(-a)=-\phi(a)$ for every $a\in R$;   moreover, $\phi[R]$ is a
subring of $S$, and $\phi$ is injective iff its kernel is $\{0_R\}$.

\header{3A2Dc}{\bf (c)} If $R$, $S$ and $T$ are rings, and $\phi:R\to
S$, $\psi:S\to T$ are ring homomorphisms, then $\psi\phi:R\to T$ is a
ring homomorphism\prooflet{, because

\Centerline{$(\psi\phi)(a*b)=\psi(\phi(a*b))=\psi(\phi(a)*\phi(b))
=\psi(\phi(a))*\psi(\phi(b))$}

\noindent for all $a$, $b\in R$, taking $*$ to be either addition or
multiplication}.   If $\phi$ is bijective, then $\phi^{-1}:S\to R$ is a
ring homomorphism\prooflet{, because

\Centerline{$\phi^{-1}(c*d)
=\phi^{-1}(\phi(\phi^{-1}(c))*\phi(\phi^{-1}(d)))
=\phi^{-1}\phi(\phi^{-1}(c)*\phi^{-1}(d))
=\phi^{-1}(c)*\phi^{-1}(d)$}

\noindent for all $c$, $d\in S$, again taking $*$ to be either addition
or multiplication}.

\leader{3A2E}{Ideals (a)} Let $R$ be a ring.   An {\bf ideal} of $R$ is
a subring $I$ of $R$ such that
$ab\in I$ and $ba\in I$ whenever $a\in I$ and $b\in R$.
In this case we write $I\normalsubgroup R$.

\cmmnt{Note that $R$ and $\{0\}$ are always ideals of $R$.}

\header{3A2Eb}{\bf (b)} If $R$ and $S$ are rings and $\phi:R\to S$ is a
ring homomorphism, then the kernel $I$ of $\phi$ is an ideal of $R$.
\prooflet{\Prf\ (i) Because $\phi$ is a group homomorphism, $I$ is a
subgroup of $(R,+)$.   (ii) If $a\in I$, $b\in R$ then

\Centerline{$\phi(ab)=\phi(a)\phi(b)=0_S\phi(b)=0_S$,
\quad$\phi(ba)=\phi(b)\phi(a)=\phi(b)0_S=0_S$}

\noindent so $ab$, $ba\in I$.\ \Qed}

\leader{3A2F}{Quotient rings (a)} Let $R$ be a ring and $I$ an ideal of
$R$.   A {\bf coset} of $I$ is a set of the form $a+I=\{a+x:x\in I\}$
where $a\in R$.   \cmmnt{(Because $+$ is commutative, we do not need
to distinguish between `left cosets' $a+I$ and `right cosets'
$I+a$.)}   Let $R/I$ be the set of cosets of $I$ in $R$.

\header{3A2Fb}{\bf (b)} For $A$, $B\in R/I$, set

\Centerline{$A+B=\{x+y:x\in A,\,y\in B\}$,
\quad$A\cdot B=\{xy+z:x\in A,\,y\in B,\,z\in I\}$.}

\noindent Then $A+B$, $A\cdot B$ both belong to $R/I$;  moreover, if
$A=a+I$
and $B=b+I$, then $A+B=(a+b)+I$ and $A\cdot B=ab+I$.   \prooflet{\Prf\
{(i)}

$$\eqalignno{A+B
&=(a+I)+(b+I)\cr
&=\{(a+x)+(b+y):x,\,y\in I\}\cr
&=\{(a+b)+(x+y):x,\,y\in I\}\cr
\noalign{\noindent (because addition is associative and commutative)}
&\subseteq\{(a+b)+z:z\in I\}=(a+b)+I\cr
\noalign{\noindent (because $I+I\subseteq I$)}
&=\{(a+b)+(z+0):z\in I\}\cr
&\subseteq (a+I)+(b+I)=A+B\cr}$$

\noindent because $0\in I$.   {(ii)}

$$\eqalignno{A\cdot B
&=\{(a+x)(b+y)+z:x,\,y,\,z\in I\}\cr
&=\{ab+(ay+xb+z):x,\,y,\,z\in I\}\cr
&\subseteq\{ab+w:w\in I\}=ab+I\cr
\noalign{\noindent (because $ay$, $xb\in I$ for all $x$, $y\in I$, and
$I$ is closed under addition)}
&=\{(a+0)(b+0)+w:w\in I\}\cr
&\subseteq A\cdot B.\text{  \Qed}\cr}$$
}

\header{3A2Fc}{\bf (c)}\cmmnt{ It is now an elementary exercise to
check that} $(R/I,+,\cdot )$ is a ring, with zero $0+I=I$ and additive
inverses $-(a+I)=(-a)+I$.

\header{3A2Fd}{\bf (d)} Moreover, the map $a\mapsto a+I:R\to R/I$ is a
ring homomorphism.

\header{3A2Fe}{\bf (e)} Note that for $a$, $b\in R$, the following are
equiveridical:  (i) $a\in b+I$;  (ii) $b\in a+I$;  (iii)
$(a+I)\cap(b+I)\ne\emptyset$;  (iv) $a+I=b+I$;  (v) $a-b\in I$.   Thus
the cosets of $I$ are just the equivalence classes in $R$ under the
equivalence relation $a\sim b\iff a+I=b+I$;  accordingly I shall
generally write $a^{\ssbullet}$ for $a+I$\cmmnt{, if there seems no
room for
confusion}.    In particular, the kernel of the canonical map from $R$ to
$R/I$ is just $\{a:a+I=I\}=I=0^{\ssbullet}$.

\header{3A2Ff}{\bf (f)} If $R$ is commutative so is
$R/I$\prooflet{, since

\Centerline{$a^{\ssbullet}b^{\ssbullet}=(ab)^{\ssbullet}
=(ba)^{\ssbullet}=b^{\ssbullet}a^{\ssbullet}$}

\noindent for all $a$, $b\in R$}.

\leader{3A2G}{Factoring homomorphisms through quotient rings:
Proposition} Let $R$ and $S$ be rings, $I$ an ideal of $R$, and
$\phi:R\to S$ a homomorphism such that $I$ is included in the kernel of
$\phi$.   Then we have a ring homomorphism $\pi:R/I\to S$ such that
$\pi(a^{\ssbullet})=\phi(a)$ for every $a\in R$.   $\pi$ is injective
iff $I$ is precisely the kernel of $\phi$.

\proof{ If $a$, $b\in R$ and $a^{\ssbullet}=b^{\ssbullet}$ in $R/I$,
then $a-b\in I$ (3A2Fe), so $\phi(a)-\phi(b)=\phi(a-b)=0$, and
$\phi(a)=\phi(b)$.   This means that the formula offered does indeed
define a function $\pi$ from $R/I$ to $S$.   Now if $a$, $b\in R$ and
$*$ is either multiplication or addition,

\Centerline{$\pi(a^{\ssbullet}*b^{\ssbullet})
=\pi((a*b)^{\ssbullet})
=\phi(a*b)
=\phi(a)*\phi(b)
=\pi(a^{\ssbullet})*\pi(b^{\ssbullet})$.}

\noindent So $\pi$ is a ring homomorphism.

The kernel of $\pi$ is $\{a^{\ssbullet}:\phi(a) = 0\}$, which is $\{0\}$
iff $\phi(a)=0\iff a^{\ssbullet}=0\iff a\in I$.
}%end of proof of 3A2G

\vleader{72pt}{3A2H}{Product rings (a)} Let $\langle R_i\rangle_{i\in I}$ be
any family of rings.   Set $R=\prod_{i\in I}R_i$ and for $a$, $b\in R$
define $a+b$, $ab\in R$ by setting

\Centerline{$(a+b)(i)=a(i)+b(i)$,\quad $(ab)(i)=a(i)b(i)$}

\noindent for every $i\in I$.   \cmmnt{It is easy to check from the
definition in 3A2A that} $R$ is a ring;  its zero is given by the
formula

\Centerline{$0_R(i)=0_{R_i}$ for every $i\in I$,}

\noindent and its additive inverses by the formula

\Centerline{$(-a)(i)=-a(i)$ for every $i\in I$.}

\header{3A2Hb}{\bf (b)} Now let $S$ be any other ring.
Then\cmmnt{ it is easy to see that} a function $\phi:S\to R$ is a ring
homomorphism iff $s\mapsto\phi(s)(i):S\to R_i$ is a ring homomorphism
for every $i\in I$.

\header{3A2Hc}{\bf (c)}\cmmnt{ Note that} $R$ is commutative iff $R_i$
is commutative for every $i$.



\discrpage

