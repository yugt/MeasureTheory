\frfilename{mt31.tex}
\versiondate{29.10.12}
\copyrightdate{1996}

\def\chaptername{Boolean algebras}
\def\sectionname{Introduction}

\newchapter{31}

The theory of measure algebras naturally depends on certain parts of the
general theory of Boolean algebras.   In this chapter I collect those
results which will be useful later.   Since many students encounter the
formal notion of Boolean algebra for the first time in this context, I
start at the beginning;  and indeed I include in the Appendix
(\S3A2) a brief account of the necessary part of the theory of
rings, as not everyone will have had time for this bit of abstract
algebra in an undergraduate course.   But unless you find the algebraic
theory of Boolean algebras so interesting that you wish to study it for
its own sake -- in which case you should perhaps turn to {\smc Sikorski
64} or {\smc Koppelberg 89} -- I do not think it would be very sensible
to read the whole of this chapter before proceeding to the main work of
the volume in Chapter 32.   Probably \S311 is necessary to get an idea
of what a Boolean algebra looks like, and a glance at the statements of
the theorems in \S312 and 313A-313B
would be useful, but the later sections can
wait until you have need of them, on the understanding that apparently
innocent formal manipulations may depend on concepts which take some
time to master.   I hope that the cross-references
will be sufficiently well-targeted to make it possible to read this
material in parallel with its applications.

As for the actual material covered, \S311 introduces Boolean rings and
algebras, with M.H.Stone's theorem on their representation as rings and
algebras of sets.   \S312 is devoted to subalgebras, homomorphisms and
quotients, following a path parallel to the corresponding ideas in group
theory, ring theory and linear algebra.   In \S313 I come to the special
properties of Boolean algebras associated with their lattice
structures, with notions of order-preservation, order-continuity and
order-closure.   \S314 continues this with a discussion of
order-completeness, and the elaboration of the Stone representation of an
arbitrary Boolean algebra into
the Loomis-Sikorski representation of a $\sigma$-complete Boolean algebra;
this brings us to regular open algebras.   \S315 deals with `simple' and
`free' products of Boolean algebras, corresponding to `products' and
`tensor products' of linear spaces, and to projective and inductive limits
of families of Boolean algebras.   Finally, \S316 examines three special
topics:  the countable chain condition, weak distributivity and
homogeneity.

\discrpage

