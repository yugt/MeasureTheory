\frfilename{mt412.tex}
\versiondate{31.1.05}
\copyrightdate{2000}

\def\chaptername{Topologies and measures I}
\def\sectionname{Inner regularity}

\newsection{412}

As will become apparent as the chapter progresses, the concepts
introduced in \S411 are synergic;  their most interesting manifestations
are in combinations of various kinds.   Any linear account of their
properties will be more than usually like a space-filling curve.   But I
have to start somewhere, and enough results can be expressed in terms of
inner regularity, more or less by itself, to be a useful beginning.

After a handful of elementary basic facts (412A) and a list of standard
applications (412B), I give some useful sufficient conditions for inner
regularity of topological and Baire measures (412D, 412E, 412G), based
on an important
general construction (412C).   The rest of the section amounts to a
review of ideas from Volume 2 and Chapter 32 in the light of the new
concept here.   I touch on completions (412H), c.l.d.\ versions and
complete locally determined spaces (412H, 412J, 412L), strictly
localizable spaces (412I), \imp\ functions (412K, 412M), measure
algebras (412N), subspaces (412O, 412P), indefinite-integral measures
(412Q) and product measures (412R-412V),
  %412R 412S 412T 412U 412V
with a brief mention of outer regularity (412W);  most of the hard work
has already been done in Chapters 21 and 25.

\leader{412A}{}\cmmnt{ I begin by repeating a lemma from Chapter
34, with some further straightforward facts.

\medskip

\noindent}{\bf Lemma} (a) Let $(X,\Sigma,\mu)$ be a measure space and
$\Cal K$ a family of sets such that

\inset{whenever $E\in\Sigma$ and $\mu E>0$ there is a
$K\in\Cal K\cap\Sigma$ such that $K\subseteq E$ and $\mu K>0$.}

\noindent Then whenever $E\in\Sigma$ there is a countable disjoint
family $\familyiI{K_i}$ in $\Cal K\cap\Sigma$ such that $K_i\subseteq E$
for every $i$ and $\sum_{i\in I}\mu K_i=\mu E$.   If moreover

\inset{($\dagger$) $K\cup K'\in\Cal K$ whenever $K$, $K'$ are disjoint
members of $\Cal K$,}

\noindent then $\mu$ is inner regular with respect to $\Cal K$.   If
$\bigcup_{i\in I}K_i\in\Cal K$ for every countable disjoint family
$\familyiI{K_i}$ in $\Cal K$, then for every $E\in\Sigma$ there is a
$K\in\Cal K\cap\Sigma$ such that $K\subseteq E$ and $\mu K=\mu E$.

(b) Let $(X,\Sigma,\mu)$ be a measure space, $\Tau$ a
$\sigma$-subalgebra of $\Sigma$, and $\Cal K$ a family of sets.   If
$\mu$ is inner regular with respect to $\Tau$ and $\mu\restrp\Tau$ is
inner regular with respect to $\Cal K$, then $\mu$ is inner regular with
respect to $\Cal K$.

(c) Let $(X,\Sigma,\mu)$ be a semi-finite measure space and
$\sequencen{\Cal K_n}$ a sequence of families of sets such that
$\mu$ is inner regular with respect to $\Cal K_n$ and

\inset{($\ddagger$) if $\sequence{i}{K_i}$ is a
non-increasing sequence in $\Cal K_n$, then
$\bigcap_{i\in\Bbb N}K_i\in\Cal K_n$}

\noindent for every $n\in\Bbb N$.   Then $\mu$ is inner regular with
respect to $\bigcap_{n\in\Bbb N}\Cal K_n$.

\proof{{\bf (a)} This is 342B-342C.

\medskip

{\bf (b)} If $E\in\Sigma$ and $\gamma<\mu E$, there are an $F\in\Tau$
such that $F\subseteq E$ and $\mu F>\gamma$, and a $K\in\Cal K\cap\Tau$
such that $K\subseteq F$ and $\mu K\ge\gamma$.

\medskip

{\bf (c)} Suppose that $E\in\Sigma$ and that $0\le\gamma<\mu E$.
Because $\mu$ is semi-finite, there is an $F\in\Sigma$ such that
$F\subseteq E$ and $\gamma<\mu F<\infty$ (213A).   Choose
$\sequence{i}{K_i}$ inductively, as
follows.   Start with $K_0=F$.   Given that $K_i\in\Sigma$ and
$\gamma<\mu
K_i$, then let $n_i\in\Bbb N$ be such that $2^{-n_i}(i+1)$ is an odd
integer, and choose $K_{i+1}\in\Cal K_{n_i}$ such that
$K_{i+1}\subseteq K_i$ and $\mu K_{i+1}>\gamma$;
this will be possible because $\mu$ is
inner regular with respect to $\Cal K_{n_i}$.   Consider
$K=\bigcap_{i\in\Bbb N}K_i$.   Then $K\subseteq E$ and
$\mu K=\lim_{i\to\infty}\mu K_i\ge\gamma$.   But also

\Centerline{$K=\bigcap_{j\in\Bbb N}K_{2^n(2j+1)}\in\Cal K_n$}

\noindent because $\sequence{j}{K_{2^n(2j+1)}}$ is a non-increasing
sequence in $\Cal K_n$, for each $n$.   So
$K\in\bigcap_{n\in\Bbb N}\Cal K_n$.   As
$E$ and $\gamma$ are arbitrary, $\mu$ is inner regular with respect to
$\bigcap_{n\in\Bbb N}\Cal K_n$.
}%end of proof of 412A

\vleader{60pt}{412B}{Corollary} Let $(X,\Sigma,\mu)$ be a measure space and
$\frak T$ a topology on $X$.   Suppose that $\Cal K$ is

\inset{{\it either} the family of Borel subsets of $X$}

\inset{{\it or} the family of closed subsets of $X$}

\inset{{\it or} the family of compact subsets of $X$}

\inset{{\it or} the family of zero sets in $X$,}

\noindent and suppose that whenever $E\in\Sigma$ and $\mu E>0$ there is
a $K\in\Cal K\cap\Sigma$ such that $K\subseteq E$ and $\mu K>0$.   Then
$\mu$ is inner regular with respect to $\Cal K$.

\proof{ In every case, $\Cal K$ satisfies the condition ($\dagger$) of
412Aa.
}%end of proof of 412B

\leader{412C}{}\cmmnt{ The next lemma provides a particularly useful
method of proving that measures are inner regular with respect to
`well-behaved' families of sets.

\medskip

\noindent}{\bf Lemma} Let $(X,\Sigma,\mu)$ be a semi-finite measure
space, and suppose that $\Cal A\subseteq\Sigma$ is such that

\inset{$\emptyset\in\Cal A\subseteq\Sigma$,

$X\setminus A\in\Cal A$ for every $A\in\Cal A$.}

\noindent Let $\Tau$ be the $\sigma$-subalgebra of $\Sigma$ generated by 
$\Cal A$.
Let $\Cal K$ be a family of subsets of $X$ such that

\inset{($\dagger$) $K\cup K'\in\Cal K$ whenever $K$, $K'\in\Cal K$,

($\ddagger$) $\bigcap_{n\in\Bbb N}K_n\in\Cal K$ for every
sequence $\sequencen{K_n}$ in $\Cal K$,

whenever $A\in\Cal A$, $F\in\Sigma$ and $\mu(A\cap F)>0$, there
is a $K\in\Cal K\cap\Tau$ such that $K\subseteq A$ and
$\mu(K\cap F)>0$.}

\noindent Then $\mu\restrp\Tau$ is inner regular with respect to $\Cal K$.

\proof{{\bf (a)} Write $\frak A$ for the measure algebra of
$(X,\Sigma,\mu)$, and
$\Cal L=\Cal K\cap\Tau$, so that $\Cal L$ also is closed under finite
unions and countable intersections.   Set

\Centerline{$\Cal H
=\{E:E\in\Sigma,\,\sup_{L\in\Cal L,L\subseteq E}L^{\ssbullet}
    =E^{\ssbullet}\}$ in $\frak A$,}

\Centerline{$\Tau'=\{E:E\in\Cal H,\,X\setminus E\in\Cal H\}$,}

\noindent so that the last two conditions tell us that
$\Cal A\subseteq\Tau'$.

\medskip

{\bf (b)} The intersection of any sequence in $\Cal H$ belongs to
$\Cal H$.   \Prf\ Let $\sequencen{H_n}$ be a sequence in $\Cal H$ with
intersection $H$.   Write $A_n$ for
$\{L^{\ssbullet}:L\in\Cal L,\,L\subseteq H_n\}
\subseteq\frak A$ for each $n\in\Bbb N$.   Since $\mu$ is semi-finite,
$\frak A$ is \wsid\ (322F).   As $A_n$ is upwards-directed and
$\sup A_n=H_n^{\ssbullet}$ for each $n\in\Bbb N$,

$$\eqalignno{H^{\ssbullet}
&=\inf_{n\in\Bbb N}H_n^{\ssbullet}\cr
\displaycause{because $F\mapsto F^{\ssbullet}:\Sigma\to\frak A$ is
sequentially order-continuous, by 321H}
&=\inf_{n\in\Bbb N}\sup A_n
=\sup\{\inf_{n\in\Bbb N}a_n:a_n\in A_n\text{ for every }n\in\Bbb N\}\cr
\displaycause{316H(iv)}
&=\sup\{(\bigcap_{n\in\Bbb N}L_n)^{\ssbullet}:L_n\in\Cal L,
  \,L_n\subseteq H_n\text{ for every }n\in\Bbb N\}\cr
&\Bsubseteq\{L^{\ssbullet}:L\in\Cal L,\,L\subseteq H\}\cr
\displaycause{by ($\ddagger$)}
&\Bsubseteq H^{\ssbullet},\cr}$$

\noindent and $H\in\Cal H$.\ \Qed

\medskip

{\bf (c)} The union of any sequence in $\Cal H$ belongs to $\Cal H$.
\Prf\ If $\sequencen{H_n}$ is a sequence in $\Cal H$
with union $H$ then

\Centerline{$\sup_{L\in\Cal L,L\subseteq H}L^{\ssbullet}
\Bsupseteq\sup_{n\in\Bbb N}\sup_{L\in\Cal L,L\subseteq E_n}
  L^{\ssbullet}
=\sup_{n\in\Bbb N}H_n^{\ssbullet}
=H^{\ssbullet}$,}

\noindent so $H\in\Cal H$.\ \Qed

\medskip

{\bf (d)} $\Tau'$ is a $\sigma$-subalgebra of $\Sigma$.
\Prf\ (i) $\emptyset$ and $X$ belong to $\Cal A\subseteq\Cal H$, so
$\emptyset\in\Tau'$.
(ii) Obviously $X\setminus E\in\Tau'$ whenever $E\in\Tau'$.
(iii) If $\sequencen{E_n}$ is a sequence in $\Tau'$ with union $E$ then
$E\in\Cal H$, by (c);  but also
$X\setminus E=\bigcap_{n\in\Bbb N}(X\setminus E_n)$ belongs to $\Cal H$,
by (b).   So $E\in\Tau'$.\ \Qed

\medskip

{\bf (e)} Accordingly $\Tau\subseteq\Tau'$, and
$E^{\ssbullet}=\sup_{L\in\Cal L,L\subseteq E}L^{\ssbullet}$ for every
$E\in\Tau$.   It follows at once that if $E\in\Tau$ and $\mu E>0$, there
must be an $L\in\Cal L$ such that $L\subseteq E$ and $\mu L>0$;  since
($\dagger$) is true, and $\Cal L\subseteq\Tau$, we can apply 412Aa to
see that $\mu\restrp\Tau$ is inner regular with respect to $\Cal L$,
therefore with respect to $\Cal K$.
}%end of proof of 412C

\leader{412D}{}\cmmnt{ As corollaries of the last lemma I give
two-and-a-half basic theorems.

\medskip

\noindent}{\bf Theorem} Let $(X,\frak T)$ be a topological space and
$\mu$ a semi-finite Baire measure on $X$.   Then $\mu$ is inner regular
with respect to the zero sets.

\proof{ Write $\Sigma$ for the Baire $\sigma$-algebra of $X$, the domain
of $\mu$, $\Cal K$ for the family of zero sets, and $\Cal A$ for
$\Cal K\cup\{X\setminus K:K\in\Cal K\}$.   Since the union of two zero
sets is a zero set (4A2C(b-ii)), the intersection of a sequence of zero
sets is a zero set (4A2C(b-iii)), and the complement of a zero set is
the union of a sequence of zero sets (4A2C(b-vi)), the conditions of
412C are satisfied;  and as the $\sigma$-algebra generated by $\Cal A$
is just $\Sigma$, $\mu$ is inner regular with respect to $\Cal K$.
}%end of proof of 412D

\leader{412E}{Theorem} Let $(X,\frak T)$ be a perfectly normal
topological space\cmmnt{ (e.g., any metrizable space)}.   Then any
semi-finite Borel measure on $X$ is inner regular with respect to the
closed sets.

\proof{ Because the Baire and Borel $\sigma$-algebras are the same
(4A3Kb), this is a special case of 412D.
}%end of proof of 412E

\leader{412F}{Lemma} Let $(X,\Sigma,\mu)$ be a measure space and
$\frak T$ a topology on $X$ such that $\mu$ is effectively locally
finite with respect to $\frak T$.   Then

\Centerline{$\mu E=\sup\{\mu(E\cap G):G$ is a measurable open set of
finite measure$\}$}

\noindent for every $E\in\Sigma$.

\proof{ Apply 412Aa with $\Cal K$ the family of subsets of measurable
open sets of finite measure.
}%end of proof of 412F

\vleader{72pt}{412G}{Theorem} Let $(X,\Sigma,\mu)$ be a measure space
with a
topology $\frak T$ such that $\mu$ is effectively locally finite with
respect to $\frak T$ and $\Sigma$ is the $\sigma$-algebra generated by
$\frak T\cap\Sigma$.   If

\Centerline{$\mu G=\sup\{\mu F:F\in\Sigma$ is closed, $F\subseteq G\}$}

\noindent for every measurable open set $G$ of finite measure, then
$\mu$ is inner regular with respect to the closed sets.

\proof{ In
412C, take $\Cal K$ to be the family of measurable closed subsets of
$X$, and $\Cal A$ to be the family of measurable sets which are
{\it either} open {\it or} closed.   If $G\in\Sigma\cap\frak T$,
$F\in\Sigma$ and $\mu(G\cap F)>0$, then there is an open set $H$ of
finite measure such that
$\mu(H\cap G\cap F)>0$, because $\mu$ is effectively locally finite;
now there is a $K\in\Cal K$ such that $K\subseteq H\cap G$ and
$\mu K>\mu(H\cap G)-\mu(H\cap G\cap F)$, so that $\mu(K\cap F)>0$.
This is the only non-trivial item in the list of hypotheses in 412C, so
we can conclude that $\mu\restrp\Tau$ is inner regular with respect to
$\Cal K$, where $\Tau$ is the $\sigma$-algebra generated by $\Cal A$;
but of course this is just $\Sigma$.
}%end of proof of 412G

\cmmnt{\medskip

\noindent{\bf Remark} There is a similar result in 416F(iii) below.
}%end of comment

\leader{412H}{Proposition} Let $(X,\Sigma,\mu)$ be a measure space and
$\Cal K$ a family of sets.

(a) If $\mu$ is inner regular with respect to
$\Cal K$, so are its completion $\hat\mu$\cmmnt{ (212C)} and
c.l.d.\ version $\tilde\mu$\cmmnt{ (213E)}.

(b) Now suppose that

\inset{($\ddagger$) $\bigcap_{n\in\Bbb N}K_n\in\Cal K$ whenever
$\sequencen{K_n}$ is a non-increasing sequence in $\Cal K$.}

\noindent If {\it either} $\hat\mu$ is inner regular with respect to
$\Cal K$ {\it or} $\mu$ is semi-finite and $\tilde\mu$ is inner
regular with respect to $\Cal K$, then $\mu$ is inner regular with
respect to $\Cal K$.

\proof{{\bf (a)} If $F$ belongs to the domain of $\hat\mu$, then there
is an $E\in\Sigma$ such that $E\subseteq F$ and
$\hat\mu(F\setminus E)=0$.
So if $0\le\gamma<\hat\mu F=\mu E$, there is a $K\in\Cal K\cap\Sigma$
such that $K\subseteq E\subseteq F$ and $\hat\mu K=\mu K\ge\gamma$.

If $H$ belongs to the domain of $\tilde\mu$ and
$0\le\gamma<\tilde\mu H$, there is an $E\in\Sigma$ such that
$\mu E<\infty$ and $\hat\mu(E\cap H)>\gamma$ (213D).   Now there is a
$K\in\Cal K\cap\Sigma$ such that
$K\subseteq E\cap H$ and $\mu K\ge\gamma$.   As $\mu K<\infty$,
$\tilde\mu K=\mu K\ge\gamma$.

\medskip

{\bf (b)} Write $\check\mu$ for whichever of $\hat\mu$, $\tilde\mu$ is
supposed to be inner regular with respect to $\Cal K$.   Then
$\check\mu$ is inner regular with respect to $\Sigma$ (212Ca, 213Fc), so
is inner regular with respect to $\Cal K\cap\Sigma$ (412Ab).   Also
$\check\mu$ extends $\mu$ (212D, 213Hc).
Take $E\in\Sigma$ and $\gamma<\mu E=\check\mu E$.   Then there is a
$K\in\Cal K\cap\Sigma$ such that $K\subseteq E$ and
$\gamma<\check\mu K=\mu K$.   As $E$ and $\gamma$ are arbitrary, $\mu$
is inner regular with respect to $\Cal K$.
}%end of proof of 412H

\leader{412I}{Lemma} Let $(X,\Sigma,\mu)$ be a strictly localizable
measure space and $\Cal K$ a family of sets such that whenever
$E\in\Sigma$ and $\mu E>0$ there is a $K\in\Cal K\cap\Sigma$ such that
$K\subseteq E$ and $\mu K>0$.

(a) There is a decomposition $\familyiI{X_i}$ of $X$ such that at most
one $X_i$ does not belong to $\Cal K$, and that exceptional one, if any,
is negligible.

(b) There is a disjoint family $\Cal L\subseteq\Cal K\cap\Sigma$ such
that $\mu^*A=\sum_{L\in\Cal L}\mu^*(A\cap L)$ for every $A\subseteq X$.

(c) If $\mu$ is $\sigma$-finite then the family $\familyiI{X_i}$ of (a)
and the set $\Cal L$ of (b) can be taken to be countable.

\proof{{\bf (a)} Let $\family{j}{J}{E_j}$ be any decomposition of $X$.
For each $j\in J$, let $\Cal K_j$ be a maximal disjoint subset of

\Centerline{$\{K:K\in\Cal K\cap\Sigma,\,K\subseteq E_j,\,\mu K>0\}$.}

\noindent Because $\mu E_j<\infty$, $\Cal K_j$ must be countable.   Set
$E_j'=E_j\setminus\bigcup\Cal K_j$.   By the maximality of $\Cal K_j$,
$E'_j$ cannot include any non-negligible set in $\Cal K\cap\Sigma$;  but
this means that $\mu E'_j=0$.   Set $X'=\bigcup_{j\in J}E'_j$.   Then

\Centerline{$\mu X'=\sum_{j\in J}\mu(X'\cap E_j)
=\sum_{j\in J}\mu E'_j=0$.}

Note that if $j$, $j'\in J$ are distinct, $K\in\Cal K_j$ and
$K'\in\Cal K_{j'}$, then $K\cap K'=\emptyset$;  thus
$\Cal L=\bigcup_{j\in J}\Cal K_j$ is disjoint.
Let $\familyiI{X_i}$ be any indexing of $\{X'\}\cup\Cal L$.   This is a
partition (that is, disjoint cover) of $X$ into sets of finite measure.
If $E\subseteq X$ and
$E\cap X_i\in\Sigma$ for every $i\in I$, then for every $j\in J$

\Centerline{$E\cap E_j
=(E\cap X'\cap E_j)\cup\bigcup_{K\in\Cal K_j}E\cap K$}

\noindent belongs to $\Sigma$, so that $E\in\Sigma$ and

\Centerline{$\mu E=\sum_{j\in J}\mu(E\cap E_j)
=\sum_{j\in J}\sum_{K\in\Cal K_j}\mu(E\cap K)
=\sum_{i\in I}\mu(E\cap X_i)$.}

\noindent Thus $\familyiI{X_i}$ is a decomposition of $X$, and it is of
the right type because every $X_i$ but one belongs to
$\Cal L\subseteq\Cal K$.

\medskip
\wheader{412I}{0}{0}{0}{36pt}

{\bf (b)} If now $A\subseteq X$ is any set,

\Centerline{$\mu^*A=\mu_AA=\sum_{i\in I}\mu_A(A\cap X_i)
=\sum_{i\in I}\mu^*(A\cap X_i)$}

\noindent by 214Ia, writing $\mu_A$ for the subspace measure on $A$.
So we have

\Centerline{$\mu^*A=\mu^*(A\cap X')+\sum_{L\in\Cal L}\mu^*(A\cap L)
=\sum_{L\in\Cal L}\mu^*(A\cap L)$,}

\noindent while $\Cal L\subseteq\Cal K$ is disjoint.

\medskip

{\bf (c)} If $\mu$ is $\sigma$-finite we can take $J$ to be countable,
so that $I$ and $\Cal L$ will also be countable.
}%end of proof of 412I

\leader{412J}{Proposition} Let $(X,\Sigma,\mu)$ be a complete locally
determined measure space, and $\Cal K$ a family of sets such that $\mu$
is inner regular with respect to $\Cal K$.

(a) If $E\subseteq X$ is such that $E\cap K\in\Sigma$ for every
$K\in\Cal K\cap\Sigma$, then $E\in\Sigma$.

(b) If $E\subseteq X$ is such that $E\cap K$ is negligible for every
$K\in\Cal K\cap\Sigma$, then $E$ is negligible.

(c) For any $A\subseteq X$,
$\mu^*A=\sup_{K\in\Cal K\cap\Sigma}\mu^*(A\cap K)$.

(d) Let $f$ be a non-negative $[0,\infty]$-valued function defined on a
subset of $X$.   If $\int_Kf$ is defined in $[0,\infty]$ for every
$K\in\Cal K$, then $\int f$ is defined and equal to
$\sup_{K\in\Cal K}\int_Kf$.

(e) If $f$ is a $\mu$-integrable function and $\epsilon>0$, there is a
$K\in\Cal K$ such that $\int_{X\setminus K}|f|\le\epsilon$.

\cmmnt{\medskip

\noindent{\bf Remark} In (c), we must interpret $\sup\emptyset$ as $0$
if $\Cal K\cap\Sigma=\emptyset$.
}

\proof{{\bf (a)} If $F\in\Sigma$ and $\mu F<\infty$, then $E\cap
F\in\Sigma$.   \Prf\ If $\mu F=0$, this is trivial, because $\mu$ is
complete and $E\cap F$ is negligible.   Otherwise, there is a sequence
$\sequencen{K_n}$ in $\Cal K\cap\Sigma$ such that $K_n\subseteq F$ for
each $n$ and $\sup_{n\in\Bbb N}\mu K_n=\mu F$.   Now
$E\cap F\setminus\bigcup_{n\in\Bbb N}K_n$ is negligible, therefore
measurable,
while $E\cap K_n$ is measurable for every $n\in\Bbb N$, by hypothesis;
so $E\cap F$ is measurable.\ \QeD\   As $\mu$ is locally determined,
$E\in\Sigma$, as claimed.

\medskip

{\bf (b)} By (a), $E\in\Sigma$;  and because $\mu$ is inner regular with
respect to $\Cal K$, $\mu E$ must be $0$.

\medskip

{\bf (c)} Let $\mu_A$ be the subspace measure on $A$.   Because $\mu$ is
complete and locally determined, $\mu_A$ is semi-finite (214Id).   So if
$0\le\gamma<\mu^*A=\mu_AA$, there is an $H\subseteq A$ such that
$\mu_AH$ is defined, finite and greater than $\gamma$.   Let
$E\in\Sigma$ be a
measurable envelope of $H$ (132Ee), so that $\mu E=\mu^*H>\gamma$.
Then there is a $K\in\Cal K\cap\Sigma$ such that $K\subseteq E$ and
$\mu K\ge\gamma$.   In this case

\Centerline{$\mu^*(A\cap K)\ge\mu^*(H\cap K)=\mu(E\cap K)
=\mu K\ge\gamma$.}

\noindent As $\gamma$ is arbitrary,

\Centerline{$\mu^*A\le\sup_{K\in\Cal K\cap\Sigma}\mu^*(A\cap K)$;}

\noindent but the reverse inequality is trivial, so we have the result.

\medskip

{\bf (d)} Applying (b) with $E=X\setminus\dom f$, we see that $f$ is
defined almost everywhere in $X$.   Applying (a) with
$E=\{x:x\in\dom f,\,f(x)\ge\alpha\}$ for each $\alpha\in\Bbb R$, we see
that $f$ is measurable.   So $\int f$ is defined in $[0,\infty]$, and of
course $\int f\ge\sup_{K\in\Cal K}\int_Kf$.   If $\gamma<\int f$, there
is a non-negative simple function $g$ such that $g\leae f$ and
$\int g>\gamma$;   taking $E=\{x:g(x)>0\}$, there is a $K\in\Cal K$ such
that $K\subseteq E$ and
$\mu(E\setminus K)\|g\|_{\infty}\le\int g-\gamma$, so that
$\int_Kf\ge\int_Kg\ge\gamma$.   As $\gamma$ is arbitrary,
$\int f=\sup_{K\in\Cal K}\int_Kf$.

\medskip

{\bf (e)} By (d), there is a $K\in\Cal K$ such that
$\int_K|f|\ge\int|f|-\epsilon$.
}%end of proof of 412J

\cmmnt{\medskip

\noindent{\bf Remark} See also 413F below.
}%end of comment

\leader{412K}{Proposition} Let $(X,\Sigma,\mu)$ be a complete locally
determined measure space, $(Y,\Tau,\nu)$ a measure space and
$f:X\to Y$ a function.   Suppose that $\Cal K\subseteq\Tau$ is such
that

\inset{(i) $\nu$ is inner regular with respect to $\Cal K$;

(ii) $f^{-1}[K]\in\Sigma$ and $\mu f^{-1}[K]=\nu K$ for every
$K\in\Cal K$;

(iii) whenever $E\in\Sigma$ and $\mu E>0$ there is a $K\in\Cal K$
such that $\nu K<\infty$ and $\mu(E\cap f^{-1}[K])>0$.}

\noindent Then $f$ is \imp\ for $\mu$ and $\nu$.

\proof{{\bf (a)} If $F\in\Tau$, $E\in\Sigma$ and $\mu E<\infty$, then
$E\cap f^{-1}[F]\in\Sigma$.   \Prf\ Let $H_1$, $H_2\in\Sigma$ be
measurable envelopes for $E\cap f^{-1}[F]$ and $E\setminus f^{-1}[F]$
respectively.   \Quer\ If $\mu(H_1\cap H_2)>0$, there is a
$K\in\Cal K$ such that $\nu K$ is finite and $\mu(H_1\cap H_2\cap
f^{-1}[K])>0$.   Because $\nu$ is inner regular with respect to
$\Cal K$, there are $K_1$, $K_2\in\Cal K$ such that $K_1\subseteq
K\cap F$, $K_2\subseteq K\setminus F$ and

$$\eqalign{\nu K_1+\nu K_2
&>\nu(K\cap F)+\nu(K\setminus F)-\mu(H_1\cap H_2\cap f^{-1}[K]) \cr
&=\nu K-\mu(H_1\cap H_2\cap f^{-1}[K]).\cr}$$

\noindent Now

\Centerline{$\mu(H_1\cap f^{-1}[K_2])
=\mu^*(E\cap f^{-1}[F]\cap f^{-1}[K_2])=0$,}

\Centerline{$\mu(H_2\cap f^{-1}[K_1])
=\mu^*(E\cap f^{-1}[K_1]\setminus f^{-1}[F])=0$,}

\noindent so
$\mu(H_1\cap H_2\cap f^{-1}[K_1\cup K_2])=0$ and

$$\eqalign{\mu(H_1\cap H_2\cap f^{-1}[K])
&\le\mu(f^{-1}[K]\setminus f^{-1}[K_1\cup K_2])\cr
&=\mu f^{-1}[K]-\mu f^{-1}[K_1]-\mu f^{-1}[K_2]\cr
&=\nu K-\nu K_1-\nu K_2
<\mu(H_1\cap H_2\cap f^{-1}[K]),\cr}$$

\noindent which is absurd.\ \Bang

Now $(E\cap H_1)\setminus(E\cap f^{-1}[F])\subseteq H_1\cap H_2$ is
negligible, therefore measurable (because $\mu$ is complete), and
$E\cap f^{-1}[F]\in\Sigma$, as claimed.\ \Qed

\medskip

{\bf (b)} It follows (because $\mu$ is locally determined) that
$f^{-1}[F]\in\Sigma$ for every $F\in\Tau$.

\medskip

{\bf (c)} If $F\in\Tau$ and $\nu F=0$ then $\mu f^{-1}[F]=0$.
\Prf\Quer\ Otherwise, there is a $K\in\Cal K$ such that $\nu
K<\infty$ and

\Centerline{$0<\mu(f^{-1}[F]\cap f^{-1}[K])=\mu f^{-1}[F\cap K]$.}

\noindent  Let $K'\in\Cal K$ be such that $K'\subseteq K\setminus F$
and $\nu K'>\nu K-\mu f^{-1}[F\cap K]$.   Then
$f^{-1}[K']\cap f^{-1}[F\cap K]=\emptyset$, so

\Centerline{$\nu K
=\mu f^{-1}[K]
\ge\mu f^{-1}[K']+\mu f^{-1}[F\cap K]
>\nu K'+\nu K-\nu K'=\nu K$,}

\noindent which is absurd.\ \Bang\Qed

\medskip

{\bf (d)} Finally, $\mu f^{-1}[F]=\nu F$ for every $F\in\Tau$.
\Prf\ Let $\familyiI{K_i}$ be a countable disjoint family in $\Cal K$
such that $K_i\subseteq F$ for every $i$ and
$\sum_{i\in I}\nu K_i=\nu F$ (412Aa).   Set
$F'=F\setminus\bigcup_{i\in I}K_i$.
Then

\Centerline{$\mu f^{-1}[F]
=\mu f^{-1}[F']
 +\sum_{i\in I}\mu f^{-1}[K_i]
=\mu f^{-1}[F']
 +\sum_{i\in I}\nu K_i
=\mu f^{-1}[F']+\nu F$.}

\noindent If $\nu F=\infty$ then surely $\mu f^{-1}[F]=\infty=\nu F$.
Otherwise, $\nu F'=0$ so $\mu f^{-1}[F']=0$ (by (c)) and again
$\mu f^{-1}[F]=\nu F$.\ \Qed

Thus $f$ is \imp.
}%end of proof of 412K

\leader{412L}{Corollary} Let $X$ be a set and $\Cal K$ a family of
subsets of $X$.   Suppose
that $\mu$ and $\nu$ are two complete locally determined measures on $X$,
with domains including $\Cal K$, agreeing on $\Cal K$, and both inner
regular with respect to $\Cal K$.   Then they are
identical\cmmnt{ (and, in particular, have the same domain)}.

\proof{ Apply 412K with $X=Y$ and $f$ the identity function to see that
$\mu$ extends $\nu$;  similarly, $\nu$ extends $\mu$ and the two
measures are the same.
}%end of proof of 412L

\leader{412M}{Corollary} Let $(X,\Sigma,\mu)$ be a complete probability
space, $(Y,\Tau,\nu)$ a probability space and $f:X\to Y$ a function.
Suppose that whenever $F\in\Tau$ and $\nu F>0$ there is a $K\in\Tau$
such that $K\subseteq F$, $\nu K>0$, $f^{-1}[K]\in\Sigma$ and
$\mu f^{-1}[K]\ge\nu K$.   Then $f$ is \imp.

\proof{ Set
$\Cal K^*=\{K:K\in\Tau,\,f^{-1}[K]\in\Sigma,\,\mu f^{-1}[K]\ge\nu K\}$.
Then $\Cal K^*$ is closed under countable
disjoint unions and includes $\Cal K$, so for every $F\in\Tau$ there is
a $K\in\Cal K^*$ such that $K\subseteq F$ and $\nu K=\nu F$, by 412Aa.
But this means that $\mu f^{-1}[K]=\nu K$ for every $K\in\Cal K^*$.
\Prf\ There is a $K'\in\Cal K^*$ such that $K'\subseteq Y\setminus K$
and $\nu K'=1-\nu K$;  but in this case

\Centerline{$\mu f^{-1}[K']+\mu f^{-1}[K]\le 1=\nu K'+\nu K$,}

\noindent so $\mu f^{-1}[K]$ must be equal to $\nu K$.\ \QeD\  Moreover,
there is a $K^*\in\Cal K^*$ such that $\nu K^*=\nu Y=1$, so
$\mu f^{-1}[K^*]=\mu X=1$ and
$\mu(E\cap f^{-1}[K^*])>0$ whenever $\mu E>0$.   Applying 412K to
$\Cal K^*$ we have the result.
}%end of proof of 412M

\vleader{48pt}{412N}{Lemma} Let $(X,\Sigma,\mu)$ be a measure space and
$\Cal K$ a family of subsets of $X$ such that $\mu$ is inner regular
with respect to $\Cal K$.   Then

\Centerline{$E^{\ssbullet}
=\sup\{K^{\ssbullet}:K\in\Cal K\cap\Sigma,\,K\subseteq E\}$}

\noindent in the measure algebra $\frak A$ of $\mu$, for every
$E\in\Sigma$.
In particular, $\{K^{\ssbullet}:K\in\Cal K\cap\Sigma\}$ is order-dense
in $\frak A$;  and if $\Cal K$ is closed under finite unions, then
$\{K^{\ssbullet}:K\in\Cal K\cap\Sigma\}$ is topologically dense in
$\frak A$ for the measure-algebra topology.

\proof{ \Quer\ If $E^{\ssbullet}
\ne\sup\{K^{\ssbullet}:K\in\Cal K\cap\Sigma,\,K\subseteq E\}$, there is
a non-zero $a\in\frak A$ such
that $a\Bsubseteq E^{\ssbullet}\Bsetminus K^{\ssbullet}$ whenever
$K\in\Cal K\cap\Sigma$ and $K\subseteq E$.   Express $a$ as
$F^{\ssbullet}$ where
$F\subseteq E$.   Then $\mu F>0$, so there is a $K\in\Cal K\cap\Sigma$
such that $K\subseteq F$ and $\mu K>0$.   But in this case
$0\ne K^{\ssbullet}\Bsubseteq a$, while $K\subseteq E$.\ \BanG\

It follows at once that $D=\{K^{\ssbullet}:K\in\Cal K\cap\Sigma\}$ is
order-dense.   If $\Cal K$ is closed under finite unions, and
$a\in\frak A$, then $D_a=\{d:d\in D$, $d\Bsubseteq a\}$ is
upwards-directed and has supremum $a$, so
$a\in\overline{D}_a\subseteq\overline{D}$ (323D(a-ii)).

}%end of proof of 412N

\leader{412O}{Lemma} Let $(X,\Sigma,\mu)$ be a measure space and
$\Cal K$ a family of subsets of $X$ such that $\mu$ is inner regular
with respect to $\Cal K$.

(a) If $E\in\Sigma$, then the subspace measure
$\mu_E$\cmmnt{ (131B)} is inner regular with respect to $\Cal K$.

(b) Let $Y\subseteq X$ be any set such that the
subspace measure $\mu_Y$\cmmnt{ (214B)} is semi-finite.   Then
$\mu_Y$ is inner regular with respect to
$\Cal K_Y=\{K\cap Y:K\in\Cal K\}$.

\proof{{\bf (a)} This is elementary.

\medskip

{\bf (b)} Suppose that $F$ belongs to the domain $\Sigma_Y$ of
$\mu_Y$ and
$0\le\gamma<\mu_YF$.   Because $\mu_Y$ is semi-finite there is an
$F'\in\Sigma_Y$ such that $F'\subseteq F$ and $\gamma<\mu_YF'<\infty$.
Let $G\in\Sigma$ be such that $F'=G\cap Y$, and let $E\subseteq G$
be a measurable envelope for $F'$ with respect to $\mu$, so that

\Centerline{$\mu E=\mu^*F'=\mu_YF'>\gamma$.}

\noindent There is a $K\in\Cal K\cap\Sigma$ such that $K\subseteq E$ and
$\mu K\ge\gamma$, in which case $K\cap Y\in\Cal K_Y\cap\Sigma_Y$ and

\Centerline{$\mu_Y(K\cap Y)=\mu^*(K\cap Y)=\mu^*(K\cap F')
=\mu(K\cap E)=\mu K\ge\gamma$.}

\noindent As $F$ and $\gamma$ are arbitrary, $\mu_Y$ is inner regular
with respect to $\Cal K_Y$.
}%end of proof of 412O

\cmmnt{\medskip

\noindent{\bf Remark} Recall from 214Ic that if $(X,\Sigma,\mu)$ has
locally determined negligible sets (in particular, is either strictly
localizable or complete and locally
determined), then all its subspaces are semi-finite.
}%end of comment

\leader{412P}{Proposition} Let $(X,\Sigma,\mu)$ be a measure space,
$\frak T$ a topology on $X$ and $Y$ a subset of $X$;  write $\frak T_Y$
for the subspace topology of $Y$ and $\mu_Y$ for the subspace measure on
$Y$.   Suppose that {\it either} $Y\in\Sigma$ {\it or}
$\mu_Y$ is semi-finite.

(a) If $\mu$ is a topological measure, so is $\mu_Y$.

(b) If $\mu$ is inner regular with respect to the Borel sets, so is
$\mu_Y$.

(c) If $\mu$ is inner regular with respect to the closed sets, so is
$\mu_Y$.

(d) If $\mu$ is inner regular with respect to the zero sets, so is
$\mu_Y$.

(e) If $\mu$ is effectively locally finite, so is $\mu_Y$.

\proof{(a) is an immediate consequence of the definitions of `subspace
measure', `subspace topology' and `topological measure'.   The other
parts follow directly from 412O if we recall that

(i) a subset of $Y$ is Borel for $\frak T_Y$ whenever it is expressible
as $Y\cap E$ for some Borel set $E\subseteq X$ (4A3Ca);

(ii) a subset of $Y$ is closed in $Y$ whenever it is expressible as
$Y\cap F$ for some closed set $F\subseteq X$;

(iii) a subset of $Y$ is a zero set in $Y$ whenever it is expressible as
$Y\cap F$ for some zero set $F\subseteq X$ (4A2C(b-v));

(iv) $\mu$ is effectively locally finite iff it is inner regular with
respect to subsets of open sets of finite measure.
}%end of proof of 412P

\leader{412Q}{Proposition} Let $(X,\Sigma,\mu)$ be a measure space, and
$\nu$ an indefinite-integral measure over $\mu$\cmmnt{ (definition:
234J\formerly{2{}34B})}.   If $\mu$ is inner regular with respect to a family $\Cal K$ of
sets, so is $\nu$.

\proof{ Because $\mu$ and its completion $\hat\mu$ give the same
integrals, $\nu$ is an indefinite-integral measure over $\hat\mu$
(234Ke);  and
as $\hat\mu$ is still inner regular with respect to $\Cal K$ (412Ha), we
may suppose that $\mu$ itself is complete.   Let $f$ be a
Radon-Nikod\'ym derivative of $\nu$ with respect to $\mu$;  by
234Ka\formerly{2{}34C}, we
may suppose that $f:X\to\coint{0,\infty}$ is $\Sigma$-measurable.

Suppose that $F\in\dom\nu$ and that
$\gamma<\nu F$.   Set $G=\{x:f(x)>0\}$, so that $F\cap G\in\Sigma$
(234La\formerly{2{}34D}).   For $n\in\Bbb N$, set
$H_n=\{x:x\in F,\,2^{-n}\le f(x)\le 2^n\}$, so that $H_n\in\Sigma$ and

\Centerline{$\nu F
=\int f\times\chi F d\mu
=\lim_{n\to\infty}\int f\times\chi H_nd\mu$.}

\noindent Let $n\in\Bbb N$ be such that
$\int f\times\chi H_nd\mu>\gamma$.

If $\mu H_n=\infty$, there is a $K\in\Cal K$ such that $K\subseteq H_n$
and $\mu K\ge 2^n\gamma$, so that
$\nu K\ge\gamma$.   If $\mu H_n$ is finite, there is a
$K\in\Cal K$ such that
$2^n(\mu H_n-\mu K)\le\int f\times\chi H_nd\mu-\gamma$, so that
$\int f\times\chi(H_n\setminus K)d\mu+\gamma\le\int f\times\chi H_nd\mu$ and
$\nu K=\int f\times\chi K\,d\mu\ge\gamma$.
Thus in either case we have a $K\in\Cal K$ such that $K\subseteq F$ and
$\nu K\ge\gamma$;  as $F$ and $\gamma$ are arbitrary, $\nu$ is inner
regular with respect to $\Cal K$.
}%end of proof of 412Q

\leader{412R}{Lemma} Let $(X,\Sigma,\mu)$ and $(Y,\Tau,\nu)$ be measure
spaces, with c.l.d.\ product
$(X\times Y,\Lambda,\lambda)$\cmmnt{ (251F)}.   Suppose that
$\Cal K\subseteq\Cal PX$, $\Cal L\subseteq\Cal PY$ and
$\Cal M\subseteq\Cal P(X\times Y)$ are such that

\quad(i) $\mu$ is inner regular with respect to $\Cal K$;

\quad(ii) $\nu$ is inner regular with respect to $\Cal L$;

\quad(iii) $K\times L\in\Cal M$ for all $K\in\Cal K$, $L\in\Cal L$;

\quad(iv) $M\cup M'\in\Cal M$ whenever $M$, $M'\in\Cal M$;

\quad(v) $\bigcap_{n\in\Bbb N}M_n\in\Cal M$ for every sequence
$\sequencen{M_n}$ in $\Cal M$.

\noindent Then $\lambda$ is inner regular with respect to $\Cal M$.

\proof{ Write
$\Cal A=\{E\times F:E\in\Sigma\}\cup\{X\times F:F\in\Tau\}$.   Then the
$\sigma$-algebra of subsets of $X\times Y$ generated by $\Cal A$ is
$\Sigma\tensorhat\Tau$.   If
$V\in\Cal A$, $W\in\Lambda$ and $\lambda(W\cap V)>0$, there is an
$M\in\Cal M\cap(\Sigma\tensorhat\Tau)$ such that $M\subseteq W$ and
$\lambda(M\cap V)>0$.   \Prf\ Suppose that $V=E\times Y$ where
$E\in\Sigma$.   There must be $E_0\in\Sigma$ and $F_0\in\Tau$, both of
finite measure, such that $\lambda(W\cap V\cap(E_0\times F_0))>0$
(251F).   Now there are $K\in\Cal K\cap\Sigma$ and $L\in\Cal L\cap\Tau$
such that $K\subseteq E\cap E_0$, $L\subseteq F\cap F_0$ and

\Centerline{$\mu((E\cap E_0)\setminus K)\cdot\nu F_0
+\mu E_0\cdot\nu((F\cap F_0)\setminus L)
<\lambda(W\cap V\cap(E_0\times F_0))$;}

\noindent but this means that
$M=K\times L$ is included in $V$ and $\mu(W\cap M)>0$, while
$M\in\Cal M\cap(\Sigma\tensorhat\Tau)$.   Reversing the
roles of the coordinates, the same argument deals with the case in which
$V=X\times F$ for some $F\in\Tau$.\ \Qed

By 412C, $\lambda\restr\Sigma\tensorhat\Tau$ is inner regular
with respect to $\Cal M$.   But $\lambda$ is inner regular with respect
to $\Sigma\tensorhat\Tau$ (251Ib) so is also inner regular
with respect to $\Cal M$ (412Ab).
}%end of proof of 412R

\leader{412S}{Proposition} Let $(X,\Sigma,\mu)$ and $(Y,\Tau,\nu)$ be
measure spaces, with c.l.d.\ product
$(X\times Y,\Lambda,\lambda)$.   Let $\frak T$, $\frak S$ be topologies
on $X$ and $Y$ respectively, and give $X\times Y$ the product topology.

(a) If $\mu$ and $\nu$ are inner regular with respect to the closed
sets, so is $\lambda$.

(b) If $\mu$ and $\nu$ are tight\cmmnt{ (that is, inner regular with
respect to the closed compact sets)}, so is $\lambda$.

(c) If $\mu$ and $\nu$ are inner regular with respect to the zero sets,
so is $\lambda$.

(d) If $\mu$ and $\nu$ are inner regular with respect to the Borel sets,
so is $\lambda$.

(e) If $\mu$ and $\nu$ are effectively locally finite, so is $\lambda$.

\proof{ We have only to read the conditions (i)-(v) of 412R carefully
and check that they apply in each case.   (In part (e), recall that
`effectively locally finite' is the same thing as `inner regular with
respect to the subsets of open sets of finite measure'.)
}%end of proof of 412S

\leader{412T}{Lemma} Let $\langle(X_i,\Sigma_i,\mu_i)\rangle_{i\in I}$
be a family of probability spaces, with product probability space
$(X,\Lambda,\lambda)$\cmmnt{ (\S254)}.   Suppose that
$\Cal K_i\subseteq\Cal PX_i$, $\Cal M\subseteq\Cal PX$ are such that

\quad (i) $\mu_i$ is inner regular with respect to $\Cal K_i$ for each
$i\in I$;

\quad (ii) $\pi_i^{-1}[K]\in\Cal M$ for every $i\in I$ and $K\in\Cal K_i$,
writing $\pi_i(x)=x(i)$ for $x\in X$;

\quad (iii) $M\cup M'\in\Cal M$ whenever $M$, $M'\in\Cal M$;

\quad (iv) $\bigcap_{n\in\Bbb N}M_n\in\Cal M$ for every sequence
$\sequencen{M_n}$ in $\Cal M$.

\noindent Then $\lambda$ is inner regular with respect to $\Cal M$.

\proof{ (Compare 412R.)
Write $\Cal A=\{\pi_i^{-1}[E]:i\in I,\,E\in\Sigma_i\}$.   If
$V\in\Cal A$, $W\in\Lambda$ and $\lambda(W\cap V)>0$, express $V$ as
$\pi_i^{-1}[E]$, where $i\in I$ and $E\in\Sigma_i$, and take
$K\in\Cal K_i$ such that $K\subseteq E$ and
$\mu_i(E\setminus K)<\lambda(W\cap V)$;  then $M=\pi_i^{-1}[K]$ belongs
to $\Cal M\cap\Cal A$, is included in $V$, and meets $W$ in a
non-negligible set.   So the conditions of 412C are met.

It follows that $\lambda_0=\lambda\restr\Tensorhat_{i\in I}\Sigma_i$ is
inner regular with respect to $\Cal M$.   But $\lambda$ is the
completion of $\lambda_0$ (254Fd, 254Ff), so is also inner regular with
respect to $\Cal M$ (412Ha).
}%end of proof of 412T

\leader{412U}{Proposition} Let
$\langle(X_i,\Sigma_i,\mu_i)\rangle_{i\in I}$ be a family of probability
spaces, with product probability space
$(X,\Lambda,\lambda)$.   Suppose that we are given a topology
$\frak T_i$ on each $X_i$, and let $\frak T$ be the product topology on
$X$.

(a) If every $\mu_i$ is inner regular with respect to the closed sets,
so is $\lambda$.

(b) If every $\mu_i$ is inner regular with respect to the zero sets, so
is $\lambda$.

(c) If every $\mu_i$ is inner regular with respect to the Borel sets, so
is $\lambda$.

\proof{ This follows from 412T just as 412S follows from 412R.
}%end of proof of 412U

\leader{412V}{Corollary} Let
$\langle(X_i,\Sigma_i,\mu_i)\rangle_{i\in I}$ be a family of probability
spaces, with product probability space
$(X,\Lambda,\lambda)$.   Suppose that we are given a Hausdorff topology
$\frak T_i$ on each $X_i$, and let $\frak T$ be the product topology on
$X$.   Suppose that every $\mu_i$ is tight, and that $X_i$ is compact
for all but countably many $i\in I$.   Then $\lambda$ is tight.

\proof{ By 412Ua, $\lambda$ is inner regular with respect to the closed
sets.   If $W\in\Lambda$ and $\gamma<\lambda W$, let $V\subseteq W$ be a
measurable closed set such that $\lambda V>\gamma$.   Let $J$
be the set of those $i\in I$ such that $X_i$ is not compact;  we are
supposing that $J$ is countable.   Let $\family{i}{J}{\epsilon_i}$ be a
family of strictly positive real numbers such that
$\sum_{i\in J}\epsilon_j\le\lambda V-\gamma$ (4A1P).   For each
$i\in J$, let $K_i\subseteq X_i$ be a compact measurable set such that
$\mu_i(X_i\setminus K_i)\le\epsilon_i$;  and for
$i\in I\setminus J$, set $K_i=X_i$.   Then
$K=\prod_{i\in I}K_i$ is a compact measurable subset of $X$, and

\Centerline{$\lambda(X\setminus K)
\le\sum_{i\in J}\mu(X_i\setminus K_i)\le\lambda V-\gamma$,}

\noindent so $\lambda(K\cap V)\ge\gamma$;  while $K\cap V$ is a compact
measurable subset of $W$.   As $W$ and $\gamma$ are arbitrary, $\lambda$
is tight.
}%end of proof of 412V

\leader{*412W}{Outer \dvrocolon{regularity}}\cmmnt{ I have already
mentioned the complementary notion of `outer regularity' (411D).   In
this book it will not be given much prominence.   It is however a useful
tool when dealing with Lebesgue measure (see, for instance, the proof of
225K), for reasons which the next proposition will make clear.

\medskip

\noindent}{\bf Proposition} Let $(X,\Sigma,\mu)$ be a measure space and
$\frak T$ a topology on $X$.

(a) Suppose that $\mu$ is outer regular with respect to the open sets.
Then for any integrable function $f:X\to[0,\infty]$ and $\epsilon>0$,
there is a lower semi-continuous measurable function
$g:X\to[0,\infty]$ such that $f\le g$ and $\int g\le\epsilon+\int f$.

(b) Now suppose that there is a sequence of measurable open sets of
finite measure covering $X$.   Then the following are equiveridical:

\quad(i) $\mu$ is inner regular with respect to the closed sets;

\quad(ii) $\mu$ is outer regular with respect to the open sets;

\quad(iii) for any measurable set $E\subseteq X$ and $\epsilon>0$, there
are a measurable closed set $F\subseteq E$ and a measurable open set
$H\supseteq E$ such that $\mu(H\setminus F)\le\epsilon$;

\quad(iv) for every measurable function $f:X\to\coint{0,\infty}$ and
$\epsilon>0$, there is a lower semi-continuous measurable function
$g:X\to[0,\infty]$ such that $f\le g$ and $\int g-f\le\epsilon$;

\quad(v) for every measurable function $f:X\to\Bbb R$ and $\epsilon>0$,
there is a lower semi-continuous measurable function
$g:X\to\ocint{-\infty,\infty}$ such that $f\le g$ and
$\mu\{x:g(x)\ge f(x)+\epsilon\}\le\epsilon$.

\proof{{\bf (a)} Let $\eta\in\ocint{0,1}$ be such that
$\eta(7+\int fd\mu)\le\epsilon$.
For $n\in\Bbb Z$, set $E_n=\{x:(1+\eta)^n\le f(x)<(1+\eta)^{n+1}\}$, and
let $E'_n\in\Sigma$ be a measurable envelope of $E_n$;  let
$G_n\supseteq E'_n$ be a measurable open set such that
$\mu G_n\le 3^{-|n|}\eta+\mu E'_n$.   Set
$g=\sum_{n=-\infty}^{\infty}(1+\eta)^{n+1}\chi G_n$.   Then $g$ is lower
semi-continuous (4A2B(d-iii), 4A2B(d-v)), $f\le g$ and

$$\eqalign{\int gd\mu
&=\sum_{n=-\infty}^{\infty}(1+\eta)^{n+1}\mu G_n\cr
&\le(1+\eta)\sum_{n=-\infty}^{\infty}(1+\eta)^n\mu E'_n
   +\sum_{n=-\infty}^{\infty}(1+\eta)^{n+1}3^{-|n|}\eta\cr
&\le(1+\eta)\int fd\mu+7\eta
\le\int fd\mu+\epsilon,\cr}$$

\noindent as required.

\medskip

{\bf (b)} Let $\sequencen{G_n}$ be a sequence of open sets of finite
measure covering $X$;  replacing it by $\sequencen{\bigcup_{i<n}G_i}$ if
necessary, we may suppose that $\sequencen{G_n}$ is non-decreasing and
that $G_0=\emptyset$.

\medskip

\quad{\bf (i)$\Rightarrow$(iii)} Suppose that $\mu$ is inner regular
with respect to the closed sets, and that $E\in\Sigma$, $\epsilon>0$.
For each $n\in\Bbb N$ let $F_n\subseteq G_n\setminus E$ be a measurable
closed set such that $\mu F_n\ge\mu(G_n\setminus E)-2^{-n-2}\epsilon$.
Then $H=\bigcup_{n\in\Bbb N}(G_n\setminus F_n)$ is a measurable open set
including $E$ and $\mu(H\setminus E)\le\bover12\epsilon$.   Applying the
same argument to $X\setminus E$, we get a closed set $F\subseteq E$ such
that $\mu(E\setminus F)\le\bover12\epsilon$, so that
$\mu(H\setminus F)\le\epsilon$.

\medskip

\quad{\bf (ii)$\Rightarrow$(iii)} The same idea works.   Suppose that
$\mu$ is outer regular with respect to the open sets, and that
$E\in\Sigma$, $\epsilon>0$.    For each $n\in\Bbb N$, let
$H_n\supseteq G_n\cap E$ be an open set such that
$\mu(H_n\setminus E)\le 2^{-n-2}\epsilon$;  then
$H=\bigcup_{n\in\Bbb N}H_n$ is a measurable open set including $E$, and
$\mu(H\setminus E)\le\bover12\epsilon$.   Now repeat the argument on
$X\setminus E$ to find a measurable closed set $F\subseteq E$ such that
$\mu(E\setminus F)\le\bover12\epsilon$.

\medskip

\quad{\bf (iii)$\Rightarrow$(iv)} Assume (iii), and let
$f:X\to\coint{0,\infty}$ be a measurable function, $\epsilon>0$.   Set
$\eta_n=2^{-n}\epsilon/(16+4\mu G_n)$ for each $n\in\Bbb N$.   For
$k\in\Bbb N$ set
$E_k=\bigcup_{n\in\Bbb N}\{x:x\in G_n,\,k\eta_n\le f(x)<(k+1)\eta_n\}$,
and choose an open set $H_k\supseteq E_k$ such that
$\mu(H_k\setminus E_k)\le 2^{-k}$.    Set

\Centerline{$g=\sup_{k,n\in\Bbb N}(k+1)\eta_n\chi(G_n\cap H_k)$.}

\noindent Then $g:X\to[0,\infty]$ is lower semi-continuous (4A2B(d-v)
again).   Since

\Centerline{$\sup_{k,n\in\Bbb N}k\eta_n\chi(G_n\cap E_k)
\le f\le\sup_{k,n\in\Bbb N}(k+1)\eta_n\chi(G_n\cap E_k)$,}

\noindent $f\le g$ and

\Centerline{$g-f
\le\sup_{k,n\in\Bbb N}(k+1)\eta_n\chi(G_n\cap H_k\setminus E_k)
  +\sup_{k,n\in\Bbb N}\eta_n\chi(G_n\cap E_k)$}

\noindent has integral at most

\Centerline{$\sum_{k=0}^{\infty}\sum_{n=0}^{\infty}(k+1)\eta_n2^{-k}
  +\sum_{n=0}^{\infty}\eta_n\mu G_n
\le\epsilon$.}

\medskip

\quad{\bf (i)$\Rightarrow$(v)} Assume (i), and suppose that
$f:X\to\Bbb R$ is
measurable and $\epsilon>0$.   For each $n\in\Bbb N$, let
$\alpha_n\ge 0$ be such that $\mu E_n<2^{-n-1}\epsilon$, where
$E_n=\{x:x\in G_{n+1}\setminus G_n,\,f(x)\le -\alpha_n\}$.   Let
$F_n\subseteq(G_{n+1}\setminus G_n)\setminus E_n$ be a measurable closed
set such that
$\mu((G_{n+1}\setminus G_n)\setminus F_n)\le 2^{-n-2}\epsilon$.
Because $\sequencen{F_n}$ is disjoint,
$h=\sum_{n=0}^{\infty}\alpha_n\chi F_n$ is defined as a function from
$X$ to $\Bbb R$.   $\{F_n:n\in\Bbb N\}$ is locally finite, so
$\{x:h(x)\ge\alpha\}=\bigcup_{n\in\Bbb N,\alpha_n\ge\alpha}F_n$ is
closed for every $\alpha>0$ (4A2B(h-i)), and $h$ is upper
semi-continuous.   Now $f_1=f+h$ is a measurable function.   Since
(i)$\Rightarrow$(iii)$\Rightarrow$(iv), there is a measurable lower
semi-continuous function $g_1:X\to[0,\infty]$ such that
$f_1^+\le g_1$ and $\int g_1-f_1^+\le\bover12\epsilon^2$, where
$f_1^+=\max(0,f_1)$.   But if we now set
$g=g_1-h$, $g$ is lower semi-continuous, $f\le g$ and

$$\eqalign{\{x:f(x)+\epsilon\le g(x)\}
&\subseteq\{x:f_1^+(x)+\epsilon\le g_1(x)\}
  \cup\{x:f_1(x)<0\}\cr
&\subseteq\{x:f_1^+(x)+\epsilon\le g_1(x)\}
  \cup\bigcup_{n\in\Bbb N}(G_{n+1}\setminus G_n)\setminus F_n\cr}$$

\noindent has measure at most $\epsilon$, as required.

\medskip

\quad{\bf (iv)$\Rightarrow$(ii)} and {\bf (v)$\Rightarrow$(ii)} Suppose
that either (iv) or (v) is true, and that $E\in\Sigma$, $\epsilon>0$.
Then there is a measurable lower semi-continuous function
$g:X\to\ocint{0,\infty}$ such that $\chi E\le g$ and
$\mu\{x:\chi E(x)+\bover12\le g(x)\}\le\epsilon$, since this is
certainly true if $\int g-\chi E\le\bover12\epsilon$.   Set
$G=\{x:g(x)>\bover12\}$;  then $E\subseteq G$ and
$\mu(G\setminus E)\le\epsilon$.

\medskip

\quad{\bf (iii)$\Rightarrow$(i)} is trivial.   Assembling these
fragments, the proof is complete.
}%end of proof of 412W


\exercises{
\leader{412X}{Basic exercises (a)}
%\spheader 412Xa
Let $(X,\Sigma,\mu)$ be a measure space and $\frak T$ a
topology on $X$ such that $\mu$ is inner regular with respect to the
closed sets {\it and} with respect to the compact sets.   Show that $\mu$
is tight.
%412A

\spheader 412Xb Explain how 213A is a special case of 412Aa.
%412A

\spheader 412Xc Let $X$ be a set and $\Cal K$ a family of sets.
Suppose that $\mu$ and $\nu$ are two semi-finite measures on $X$
with the same domain and the same null ideal.   Show that if one is
inner regular with respect to $\Cal K$, so is the other.    \Hint{show that if
$\nu F<\infty$ then $\nu F=\sup\{\nu E:E\subseteq F$, $\mu E<\infty\}$.}
%412A

\sqheader 412Xd Let $(X,\Sigma,\mu)$ be a measure space, and $\Sigma_0$
a $\sigma$-subalgebra of $\Sigma$ such that $\mu$ is inner regular with
respect to $\Sigma_0$.   Show that if $1\le p<\infty$ then every member
of $L^p(\mu)$ is of the form $f^{\ssbullet}$ for some
$\Sigma_0$-measurable $f:X\to\Bbb R$.
%412Xb, 412A

\sqheader 412Xe Let $(X,\Sigma,\mu)$ be a semi-finite measure space and
$\Cal A\subseteq\Sigma$ an algebra of sets such that the
$\sigma$-algebra generated by $\Cal A$ is $\Sigma$.   Write $\Cal K$ for
$\{\bigcap_{n\in\Bbb N}E_n:E_n\in\Cal A$ for every $n\in\Bbb N\}$.
Show that $\mu$ is inner regular with respect to $\Cal K$.
%412C

\spheader 412Xf Let $(X,\frak T,\Sigma,\mu)$ be an effectively locally
finite Hausdorff topological measure space such that $\mu$ is inner
regular with respect to the Borel sets.   Suppose that
$\mu G=\sup\{\mu K:K\subseteq G$ is compact$\}$ for every open set
$G\subseteq X$.   Show that $\mu$ is tight.
%412G

\spheader 412Xg Let $(X,\frak T)$ be a topological space such that every
open set is an F$_{\sigma}$ set.   Show that any effectively locally
finite Borel measure on $X$ is inner regular with respect to the closed
sets.
%412G

\spheader 412Xh Let $(X,\frak T)$ be a normal topological space and
$\mu$ a topological measure on $X$ which is inner regular with respect
to the closed sets.   Show that
$\mu G=\max\{\mu H:H\subseteq G$ is a cozero set$\}$ for every open set
$G\subseteq X$.   Show that if $\mu$ is totally finite, then
$\mu F=\min\{\mu H:H\supseteq F$ is a zero set$\}$ for every closed set
$F\subseteq X$.
%412G

\spheader 412Xi Let $(X,\Sigma,\mu)$ be a complete locally determined
measure space, and suppose that $\mu$ is inner regular with respect to a
family $\Cal K$ of sets.   Let $\Sigma_0$ be the $\sigma$-algebra of
subsets of
$X$ generated by $\Cal K\cap\Sigma$.   (i) Show that $\mu$ is the
c.l.d.\ version of $\mu\restr\Sigma_0$.   \Hint{412J-412L.}   (ii) Show
that if $\mu$ is
$\sigma$-finite, it is the completion of $\mu\restr\Sigma_0$.
%412L

\sqheader 412Xj(i) Let $(X,\Sigma,\mu)$ be a $\sigma$-finite
measure space and $\Tau$ a $\sigma$-subalgebra of $\Sigma$.   Show that
if $\mu$ is inner regular
with respect to $\Tau$ then the completion of $\mu\restr\Tau$ extends
$\mu$, so that $\mu$ and $\mu\restr\Tau$ have the same negligible sets.
(ii) Show that if $\mu$ is a $\sigma$-finite topological measure which
is inner regular
with respect to the Borel sets, then every $\mu$-negligible set is
included in a $\mu$-negligible Borel set.
%412Xi 412L

\spheader 412Xk Devise a direct proof of 412L, not using 412K, by (i)
showing that $\mu^*(A\cap K)=\nu^*(A\cap K)$ whenever $A\subseteq X$ and
$K\in\Cal K$ (ii) showing that $\mu^*=\nu^*$ (iii) quoting 213C.
%412L

\spheader 412Xl Let $(X,\Sigma,\mu)$ be a complete locally determined
measure space, $Y$ a set and $f:X\to Y$ a function.   Show that the
following are equiveridical:  (i) $\mu$ is inner regular with respect to
$\{f^{-1}[B]:B\subseteq Y\}$ (ii) $f^{-1}[f[E]]\setminus E$ is
negligible for every $E\in\Sigma$.
%412M

\spheader 412Xm Let $\familyiI{(X_i,\Sigma_i,\mu_i)}$ be a family of
measure spaces, with direct sum $(X,\Sigma,\mu)$.   Suppose that for
each $i\in I$ we are given a topology $\frak T_i$ on $X_i$, and let
$\frak T$ be
the corresponding disjoint union topology on $X$.   Show that (i) $\mu$
is inner regular with respect to the closed sets iff every $\mu_i$ is
(ii) $\mu$ is inner regular with respect to the compact sets iff every
$\mu_i$ is (iii) $\mu$ is inner regular with respect to the zero sets
iff every $\mu_i$ is (iv) $\mu$ is inner regular with respect to the
Borel sets iff every $\mu_i$ is.
%412P

\spheader 412Xn Use 412L and 412Q to shorten the proof of 253I.
%412Q

\spheader 412Xo Let $\familyiI{X_i}$ be a family of sets, and
suppose that we are given, for each $i\in I$, a $\sigma$-algebra
$\Sigma_i$ of subsets of $X_i$ and a topology $\frak T_i$ on $X_i$.
Let $\frak T$ be the product topology on $X=\prod_{i\in I}X_i$, and
$\Sigma=\Tensorhat_{i\in I}\Sigma_i$.   Let $\mu$ be a totally finite
measure with domain
$\Sigma$, and set $\mu_i=\mu\pi_i^{-1}$ for each $i\in I$, where
$\pi_i(x)=x(i)$ for $i\in I$, $x\in X$.   (i) Show that $\mu$ is
inner regular with respect to the family $\Cal K$ of sets expressible as
$X\setminus\bigcup_{n\in\Bbb N}\prod_{i\in I}E_{ni}$ where
$E_{ni}\in\Sigma_i$ for every $n$, $i$ and $\{i:E_{ni}\ne X_i\}$ is
finite for each $n$.
(ii) Show that if every $\mu_i$ is inner regular with respect to the
closed sets, so is $\mu$.
(iii) Show that if every $\mu_i$ is inner regular with respect to the
zero sets, so is $\mu$.
(iv) Show that if every $\mu_i$ is inner regular with respect to the
Borel sets, so is $\mu$.
(v) Show that if every $\mu_i$ is tight, and all but countably many of
the $X_i$ are compact, then $\mu$ is tight.
%412V

\spheader 412Xp Let $(X,\Sigma,\mu)$ be a measure space and $\frak T$ a
Lindel\"of topology on $X$ such that $\mu$ is locally finite.   (i) Show
that $\mu$ is
$\sigma$-finite.   (ii) Show that $\mu$ is inner regular with respect to
the closed sets iff it is outer regular with respect to the open sets.
%412W

\spheader 412Xq Let $X$ be a topological space and $\mu$ a measure on
$X$ which is outer regular with respect to the open sets.   Show that
for any $Y\subseteq X$ the subspace measure on $Y$ is outer regular with
respect to the open sets.
%412W

\spheader 412Xr Let $X$ be a topological space and $\mu$ a measure on
$X$ which is outer regular with respect to the open sets.   Show that if
$f:X\to\Bbb R$ is integrable and $\epsilon>0$ then there is a lower
semi-continuous $g:X\to\ocint{-\infty,\infty}$ such that $f\le g$ and
$\int g-f\le\epsilon$.
%412W

\sqheader 412Xs Let $X$ be a topological space and $\mu$ a measure on $X$ which is
effectively locally finite and inner regular with respect to the closed sets.
(i) Show that if $\mu E<\infty$ and $\epsilon>0$ there is a measurable open set $G$
such that $\mu(E\symmdiff G)\le\epsilon$.   (ii) Show that if $f$ is a
non-negative integrable function and $\epsilon>0$ there is a measurable lower
semi-continuous function $g:X\to\coint{0,\infty}$ such that $\int|f-g|\le\epsilon$.
(iii) Show that if $f$ is an integrable
real-valued function there are measurable lower semi-continuous functions
$g_1$, $g_2:X\to[0,\infty]$ such that $f\eae g_1-g_2$ and
$\int g_1+g_2\le\int|f|+\epsilon$.   (iv) Now suppose that $\mu$ is $\sigma$-finite.
Show that for every measurable $f:X\to\Bbb R$ there are measurable lower
semi-continuous functions $g_1$, $g_2:X\to[0,\infty]$ such that $f\eae g_1-g_2$.
%412W

\spheader 412Xt Let $(X,\Sigma,\mu)$ be a semi-finite measure space and
$\sequencen{f_n}$ a sequence in $\eusm L^0(\mu)$ which converges almost
everywhere to $f\in\eusm L^0(\mu)$.   Show that $\mu$ is inner regular
with respect to
$\{E:\sequencen{f_n\restr E}$ is uniformly convergent$\}$.
(Cf.\ 215Yb.)
%412-

\spheader 412Xu In 216E, give $\{0,1\}^I$ its usual compact Hausdorff
topology.   Show that the measure $\mu$ described there is inner regular
with respect to the zero sets.
%412-

\spheader 412Xv Let $\familyiI{\mu_i}$ be a family of measures on a set
$X$, with sum $\mu$ (234G\formerly{1{}12Ya}).   Suppose that $\Cal K$ is a
family of sets such that every $\mu_i$ is inner regular with respect to
$\Cal K$.   Show that if {\it either} $\Cal K\subseteq\dom\mu$ {\it or}
$\bigcap_{n\in\Bbb N}K_n\in\Cal K$ for every non-increasing sequence
$\sequencen{K_n}$ in $\Cal K$, then $\mu$ is inner regular with respect to
$\Cal K$.
%412A

\spheader 412Xw Let $\mu$, $\nu$ be c.l.d.\ measures on a set $X$, both
inner regular with respect to a family 
$\Cal K\subseteq\dom\mu\cap\dom\nu$.   Suppose that 
$\mu K\le\nu K<\infty$ for every $K\in\Cal K$.   Show that $\mu\le\nu$ in
the sense of 234P.
%412L

\spheader 412Xx\dvAnew{2010}
Let $X$ be a topological space, and $\mu$ a tight topological 
measure on $X$.   Suppose that $\Cal F$ is a 
non-empty downwards-directed family of
closed compact subsets of $X$ with intersection $F_0$, and that 
$\gamma=\inf_{F\in\Cal F}\mu F$ is finite.   Show that $\mu F_0=\gamma$.
%412A

\leader{412Y}{Further exercises (a)}
%\spheader 412Ya
Let $\Cal K$ be the family of subsets of $\Bbb R$ which are homeomorphic
to the Cantor set.   Show that Lebesgue measure is inner regular with
respect to $\Cal K$.   \Hint{show that if
$F\subseteq\Bbb R\setminus\Bbb Q$ is an uncountable compact set, then
$\{x:[x-\delta,x+\delta]\cap F$ is uncountable for every $\delta>0\}$
belongs to $\Cal K$.}
%-%

\spheader 412Yb(i) Show that if $X$ is a perfectly normal space
then any semi-finite topological measure on $X$ which is inner
regular with respect to the Borel sets is inner regular
with respect to the closed sets.   (ii) Show that any subspace of a
perfectly normal space is perfectly normal.  (iii) Show
that $\omega_1$, with its order topology, is completely regular, normal
and Hausdorff, but
not perfectly normal.   (iv) Show that $[0,1]^I$ is perfectly normal iff
$I$ is countable.
%412E

\spheader 412Yc Let $(X,\Sigma,\mu)$ be a measure space, and suppose
that $\mu$ is inner regular with respect to
$\Cal K$.   Write $\Sigma^f$ for $\{E:E\in\Sigma$, $\mu E<\infty\}$.    Show that
$\{E^{\ssbullet}:E\in\Cal K\cap\Sigma^f\}$ is dense in
$\{E^{\ssbullet}:E\in\Sigma^f\}$ for the strong measure-algebra
topology.
%412N

\spheader 412Yd Let $(X,\Sigma,\mu)$ be $[0,1]$ with Lebesgue measure,
and $Y=[0,1]$ with counting measure $\nu$;  give $X$ its usual topology
and $Y$ its discrete topology, and let $\lambda$ be the c.l.d.\ product
measure on $X\times Y$.   (i) Show that $\mu$, $\nu$ and $\lambda$ are
all tight (for the appropriate
topologies) and therefore completion regular.
(ii) Let $\lambda_0$ be the primitive product measure on
$X\times Y$ (definition:  251C).   Show that $\lambda_0$ is not tight.
\Hint{252Yk.}
{\it Remark\/}:  it is undecidable in ZFC whether $\lambda_0$ is inner
regular with respect to the closed sets.
%412S %mt41bits

\spheader 412Ye Give an example of a Hausdorff topological measure space
$(X,\frak T,\Sigma,\mu)$ such that $\mu$ is complete, strictly
localizable and outer regular with respect to the open sets, but not
inner regular with respect to the closed sets.
%412W
}%end of exercises

\endnotes{\Notesheader{412} In this volume we are returning to
considerations which have been left on one side for almost the whole of
Volume 3 -- the exceptions being in Chapter 34, where I looked at
realization of homomorphisms of measure algebras by functions between
measure spaces, and was necessarily dragged into an investigation of
measure spaces which had enough points to be adequate codomains (343B).
The idea of `inner regularity' is to distinguish families $\Cal K$ of
sets which will be large enough to describe the measure entirely, but
whose members will be of recognisable types.   For an example of this
principle see 412Ya.   Of course we cannot always find a single type of
set adequate to fill a suitable family $\Cal K$, though this happens
oftener than one might expect, but it is surely easier to think about an
arbitrary zero set (for instance) than an arbitrary measurable set, and
whenever a measure is inner regular with respect to a recognisable class
it is worth knowing about it.

I have tried to use the symbols $\dagger$ and $\ddagger$ (412A, 412C)
consistently enough for them to act as a guide to some of the ideas
which will be used repeatedly in this chapter.   Note the emphasis on
disjoint unions and countable intersections;  I mentioned
similar conditions in 136Xi-136Xj.
You will recognise 412Aa as an exhaustion principle;  note that it is
enough to use disjoint unions, as in 313K.   In the examples of this
section this disjointness is not important.   Of course inner regularity
has implications for the measure algebra (412N), but it is important to
recognise that `$\mu$ is inner regular with respect to $\Cal K$' is
saying much more than `$\{K^{\ssbullet}:K\in\Cal K\}$ is order-dense in
the measure algebra';  the latter formulation tells us only that
whenever
$\mu E>0$ there is a $K\in\Cal K$ such that $K\setminus E$ is negligible
and $\mu K>0$, while the former tells us that we can take $K$ to be
actually a subset of $E$.

412D, 412E and 412G are all of great importance.   412D looks striking,
but of course the reason it works is just that the Baire
$\sigma$-algebra
is very small.   In 412E the Baire and Borel
$\sigma$-algebras coincide, so it is nothing but a special case of 412D;
but as metric spaces are particularly important it is worth having it
spelt out explicitly.   In 412D and 412E the hypothesis `semi-finite' is
sufficient, while in 412G we need `effectively locally finite';  this is
because in both 412D and 412E the open sets we are looking at are
countable unions of measurable closed sets.   There are interesting
non-metrizable spaces in
which the same thing happens (412Yb).   As you know, I am strongly
biased in favour of complete and locally determined measures, and
the Baire and Borel measures dealt with in these three results are
rarely complete;  but they can still be applied to completions and
c.l.d.\ versions of these measures, using 412Ab or 412H.

412O-412V are essentially routine.   For subspace measures, the only
problem we need to come to terms with is the fact that subspaces of
semi-finite measure spaces need not be semi-finite (216Xa).   For
product measures the point is that the c.l.d.\ product of two measure
spaces, and the product of any family of probability spaces, as I
defined them in Chapter 25, are inner regular with respect to the
$\sigma$-algebra
of sets generated by the cylinder sets.   This is not in general true of
the `primitive' product measure (412Yd), which is one of my reasons for
being prejudiced against it.   I should perhaps warn you of a trap in
the language I use here.   I say that if the factor measures are inner
regular with respect to the closed sets, so is the c.l.d.\ product
measure.   But I
do not say that all closed sets in the product are measured by the
product measure, even if closed sets in the factors are measured by
the factor measures.   So the path is open for a different product
measure to exist, still inner regular with respect to the closed sets;
and indeed I
shall be going down that path in \S417.   The uniqueness result in
412L specifically refers to complete locally determined measures defined
on all sets of the family $\Cal K$.

There is one special difficulty in 412V:  in order to ensure that there
are enough compact measurable sets in $X=\prod_{i\in I}X_i$, we need to
know that all but countably many of the $X_i$ are actually compact.
When we come to look more closely at products of Radon probability
spaces we shall need to consider this point again (417Q, 417Xq).

In fact some of the ideas of 412U-412V are not restricted to the product
measures considered there.   Other measures on the product space will
have inner regularity properties if their images on the factors, their
`marginals' in the language of probability theory, are inner regular;
see 412Xo.   I will return to this in \S454.

This section is almost exclusively concerned with {\it inner}
regularity.   The complementary notion of {\it outer} regularity is not
much use except in $\sigma$-finite spaces (415Xh), and not always then
(416Yd).   In totally finite spaces, of course, and some others, any
version of inner regularity corresponds to a version of outer
regularity, as in 412Wb(i)-(ii);  and when we have something as strong
as 412Wb(iii) available it is worth knowing about it.
}%end of notes

\discrpage

