\frfilename{mt474.tex}
\versiondate{17.11.12}
\copyrightdate{2001}

\def\varinnerprod#1#2{#1\dotproduct#2}

\def\chaptername{Geometric measure theory}
\def\sectionname{The distributional perimeter}

\newsection{474}

The next step is a dramatic excursion, defining (for appropriate sets
$E$) a perimeter {\it measure} for which a version of the Divergence
Theorem is true (474E).   I begin the section with elementary notes on
the divergence of a vector field
(474B-474C).   I then define `locally finite perimeter' (474D),
`perimeter measure' and `outward normal' (474F) and `reduced boundary'
(474G).   The definitions rely on the Riesz representation theorem, and
we have to work very hard to relate them to any geometrically natural
idea of `boundary'.   Even half-spaces (474I) demand some attention.
From Poincar\'e's inequality\cmmnt{ (473K)} we can prove isoperimetric
inequalities for perimeter measures (474L).   With some effort we can
locate the reduced boundary as a subset of the topological
boundary\cmmnt{ (474Xc)}, and obtain asymptotic inequalities on the
perimeter measures of small balls (474N).   With much more effort we can
find a geometric description of outward normal functions in terms of
`Federer exterior normals' (474R), and get a tight asymptotic
description of the perimeter measures of small balls (474S).   I end
with the Compactness Theorem for sets of bounded perimeter (474T).

%taken from {\smc Evans \& Gariepy 92}

\cmmnt{
\leader{474A}{Notation} I had better repeat some of the notation from
\S473.   $r\ge 2$ is a fixed integer.   $\mu$ is Lebesgue measure on
$\BbbR^r$, and $\beta_r=\mu B(\tbf{0},1)$ is the volume of the unit
ball.   $S_{r-1}=\partial B(\tbf{0},1)$ is the unit sphere.   $\nu$ is
normalized $(r-1)$-dimensional Hausdorff measure on $\BbbR^r$.
We shall sometimes need to look at Lebesgue measure on $\BbbR^{r-1}$,
which I will denote $\mu_{r-1}$.   As in \S473, I will use Greek letters
to represent coordinates, so that $x=(\xi_1,\ldots,\xi_r)$ for
$x\in\BbbR^r$, etc., and $\beta_r$ will be the $r$-dimensional volume of
the unit ball in $\BbbR^r$.

$\eusm D$ is the set of smooth functions $f:\BbbR^r\to\Bbb R$ with
compact support;
$\eusm D_r$ the set of smooth functions $\phi:\BbbR^r\to\BbbR^r$ with
compact support, that is, the set of functions
$\phi=(\phi_1,\ldots,\phi_r):\BbbR^r\to\BbbR^r$ such that
$\phi_i\in\eusm D$ for every $i$.

I continue to use the sequence $\sequencen{\tilde h_n}$ from 473E;
these functions all belong to $\eusm D$, are non-negative everywhere and
zero outside $B(\tbf{0},\bover1{n+1})$, are even, and have integral $1$.
}%end of comment

\leader{474B}{The divergence of a vector field (a)} For a function
$\phi$ from a subset of $\BbbR^r$ to $\BbbR^r$, write
$\diverg\phi=\sum_{i=1}^r\pd{\phi_i}{\xi_i}$, where
$\phi=(\phi_1,\ldots,\phi_r)$;  for definiteness, let us take the domain
of $\diverg\phi$ to be the set of points at which $\phi$ is
differentiable\cmmnt{ (in the strict sense of 262Fa)}.   \cmmnt{Note
that} $\diverg\phi\in\eusm D$ for every $\phi\in\eusm D_r$.
\cmmnt{We need the following elementary facts.}

\spheader 474Bb  If $f:\BbbR^r\to\Bbb R$ and $\phi:\BbbR^r\to\BbbR^r$
are functions, then
$\diverg(f\times\phi)=\varinnerprod{\phi}{\grad f}+f\times\diverg\phi$
at any point at which $f$ and $\phi$ are both differentiable.
\prooflet{(Use 473Bc;  compare 473Bd.)}

\spheader 474Bc If $\phi:\BbbR^r\to\BbbR^r$ is a Lipschitz function with
compact support, then $\int\diverg\phi\,d\mu=0$.
\prooflet{\Prf\ $\diverg\phi$ is defined almost everywhere (by
Rademacher's theorem, 262Q), measurable (473Be), bounded (473Cc) and
with compact support, so

\Centerline{$\int\diverg\phi\,d\mu
=\sum_{i=1}^r\int\Pd{\phi_i}{\xi_i}d\mu$}

\noindent is defined in $\Bbb R$.   For each $i\le r$,
Fubini's theorem tells us that we can replace integration with respect
to $\mu$ by a repeated integral, in which the inner integral is

\Centerline{$\int_{-\infty}^{\infty}
\Pd{\phi_i}{\xi_i}(\xi_1,\ldots,\xi_r)d\xi_i
=0$}

\noindent because $\phi_i(\xi_1,\ldots,\xi_r)=0$ whenever $|\xi_i|$ is
large enough.   So $\int\Pd{\phi_i}{\xi_i}d\mu$ also is zero.   Summing
over $i$, we have the result.\ \Qed}

\spheader 474Bd If $\phi:\BbbR^r\to\BbbR^r$ and $f:\BbbR^r\to\Bbb R$ are
Lipschitz functions, one of which has compact support, then
$f\times\phi$ is Lipschitz.   \prooflet{\Prf\
Take $n\in\Bbb N$ such that
$f(x)\phi(x)=\tbf{0}$ for $\|x\|>n$, and $\gamma\ge 0$ such that
$|f(x)-f(y)|\le\gamma\|x-y\|$ and $\|\phi(x)-\phi(y)\|\le\gamma\|x-y\|$ for
all $x$, $y\in\BbbR^r$, while also
$|f(x)|\le\gamma$ whenever $\|x\|\le n+1$ and
$\|\phi(x)\|\le\gamma$ whenever $\|x\|\le n+1$.   If $x$, $y\in\BbbR^r$
then

\inset{----- if $\|x\|\le n+1$ and $\|y\|\le n+1$,

\Centerline{$\|f(x)\phi(x)-f(y)\phi(y)\|
\le|f(x)|\|\phi(x)-\phi(y)\|+\|\phi(y)\||f(x)-f(y)|
\le 2\gamma^2\|x-y\|$;}

----- if $\|x\|\le n$ and $\|y\|>n+1$,

\Centerline{$|f(x)\phi(x)-f(y)\phi(y)\|
=|f(x)|\|\phi(x)\|\le\gamma^2
\le\gamma^2\|x-y\|$;}

----- if $\|x\|>n$ and $\|y\|>n$, $|f(x)\phi(x)-f(y)\phi(y)\|=0$.}

\noindent So $2\gamma^2$ is a Lipschitz constant for $f\times\phi$.\ \Qed}

It follows that

\Centerline{$\int\varinnerprod{\phi}{\grad f}\,d\mu
  +\int f\times\diverg\phi\,d\mu=0$.}

\prooflet{\noindent\Prf\ $f$ and $\phi$ and $f\times\phi$
are all differentiable almost everywhere.   So

\Centerline{$\int\varinnerprod{\phi}{\grad f}\,d\mu
  +\int f\times\diverg\phi\,d\mu
=\int\diverg(f\times\phi)d\mu
=0$}

\noindent by (b) and (c) above.\ \Qed}

\spheader 474Be If $f\in\eusm L^{\infty}(\mu)$, $g\in\eusm L^1(\mu)$ is
even\cmmnt{ (that is, $g(-x)$ is defined and equal to $g(x)$ for
every $x\in\dom g$),} and $\phi:\BbbR^r\to\BbbR^r$ is a Lipschitz
function with compact support, then
$\int(f*g)\times\diverg\phi=\int f\times\diverg(g*\phi)$\cmmnt{, where
$g*\phi=(g*\phi_1,\ldots,g*\phi_r)$}.   \prooflet{\Prf\ For each $i$,

$$\eqalignno{\int(f*g)\times\Pd{\phi_i}{\xi_i}d\mu
&=\iint f(x)g(y)\Pd{\phi_i}{\xi_i}(x+y)\mu(dy)\mu(dx)\cr
\displaycause{255G/444Od}
&=\iint f(x)g(-y)\Pd{\phi_i}{\xi_i}(x+y)\mu(dy)\mu(dx)\cr
\displaycause{because $g$ is even}
&=\int f\times(g*\Pd{\phi_i}{\xi_i})d\mu
=\int f\times\Pd{}{\xi_i}(g*\phi_i)d\mu\cr}$$

\noindent as in 473Dd.   Now take the sum over $i$ of both sides.\
\Qed}

\leader{474C}{Invariance under isometries:  Proposition} Suppose that
$T:\BbbR^r\to\BbbR^r$ is an isometry, and that $\phi$ is a function
from a subset of $\BbbR^r$ to $\BbbR^r$.   Then

\Centerline{$\diverg(T^{-1}\phi T)=(\diverg\phi)T$.}

\proof{ Set $z=T(\tbf{0})$.   By 4A4Jb, the isometry $x\mapsto T(x)-z$
is linear and preserves inner products, so there is an orthogonal matrix
$S$ such that $T(x)=z+S(x)$ for every $x\in\BbbR^r$.   Now suppose that
$x\in\BbbR^r$ is such that $(\diverg\phi)(T(x))$ is defined.   Then
$T(y)-T(x)-S(y-x)=0$ for every $y$, so $T$ is differentiable at $x$,
with derivative $S$, and $\phi T$ is differentiable at $x$, with
derivative $DS$, where $D$ is the derivative of $\phi$ at $T(x)$, by
473Bc.   Also $T^{-1}(y)=S^{-1}(y-z)$ for every $y$, so $T^{-1}$ is
differentiable at $\phi(T(x))$ with derivative $S^{-1}$, and $T^{-1}\phi
T$ is differentiable at $x$, with derivative $S^{-1}DS$.   Now if $D$ is
$\langle\delta_{ij}\rangle_{1\le i,j\le r}$ and $S$ is
$\langle\sigma_{ij}\rangle_{1\le i,j\le r}$ and $S^{-1}DS$ is
$\langle\tau_{ij}\rangle_{1\le i,j\le r}$, then $S^{-1}$ is the
transpose $\langle\sigma_{ji}\rangle_{1\le i,j\le r}$ of $S$, because
$S$ is orthogonal, so

$$\eqalign{\diverg(T^{-1}\phi T)(x)
&=\sum_{i=1}^r\tau_{ii}
=\sum_{i=1}^r\sum_{j=1}^r\sigma_{ji}\sum_{k=1}^r
  \delta_{jk}\sigma_{ki}\cr
&=\sum_{j=1}^r\sum_{k=1}^r\delta_{jk}
  \sum_{i=1}^r\sigma_{ji}\sigma_{ki}
=\sum_{j=1}^r\delta_{jj}
=\diverg\phi(T(x))\cr}$$

\noindent because $\sum_{i=1}^r\sigma_{ji}\sigma_{jk}=1$ if $j=k$ and
$0$ otherwise.   If $\diverg(T^{-1}\phi T)(x)$ is defined, then (because
$T^{-1}$ also is an isometry)

\Centerline{$(\diverg\phi)(T(x))=\diverg(TT^{-1}\phi TT^{-1})(T(x))
=\diverg(T^{-1}\phi T)(T^{-1}T(x))=\diverg(T^{-1}\phi T)(x)$.}

\noindent So the functions $\diverg(T^{-1}\phi T)$ and $(\diverg\phi)T$
are identical.
}%end of proof of 474C

\leader{474D}{Locally finite perimeter:  Definition} Let
$E\subseteq\BbbR^r$ be a Lebesgue measurable set.    Its {\bf perimeter}
$\per E$ is

\Centerline{$\sup\{|\int_E\diverg\phi\,d\mu|:
  \phi:\BbbR^r\to B(\tbf{0},1)$ is a Lipschitz function with compact
support$\}$}

\noindent (allowing $\infty$).   A set $E\subseteq\BbbR^r$
has {\bf locally finite perimeter} if it is Lebesgue measurable and

\Centerline{$\sup\{|\int_E\diverg\phi\,d\mu|:
  \phi:\BbbR^r\to\BbbR^r$ is a Lipschitz function,
  $\|\phi\|\le\chi B(\tbf{0},n)\}$}

\noindent is finite for every $n\in\Bbb N$.   Of course a Lebesgue
measurable set with finite perimeter also has locally finite perimiter.
%Cacciopoli set = finite perimeter (Tamanini & Giacomelli)

\leader{474E}{Theorem} Suppose that $E\subseteq\BbbR^r$ has locally
finite perimeter.

(i) There are a Radon measure $\lambda^{\partial}_E$ on $\BbbR^r$ and a
Borel measurable function $\psi:\BbbR^r\to S_{r-1}$ such that

\Centerline{$\int_E\diverg\phi\,d\mu
=\int\varinnerprod{\phi}{\psi}\,d\lambda^{\partial}_E$}

\noindent for every Lipschitz function $\phi:\BbbR^r\to\BbbR^r$ with
compact support.

(ii) This formula uniquely determines $\lambda^{\partial}_E$, which can
also be defined by saying that

\Centerline{$\lambda^{\partial}_E(G)
=\sup\{|\int_E\diverg\phi\,d\mu|:
  \phi:\BbbR^r\to\BbbR^r$ is Lipschitz, $\|\phi\|\le\chi G\}$}

\noindent whenever $G\subseteq\BbbR^r$ is open.

(iii) If $\hat\psi$ is another function defined
$\lambda^{\partial}_E$-a.e.\ and satisfying the formula in (i),
then $\hat\psi$ and $\psi$ are equal
$\lambda^{\partial}_E$-almost everywhere.

\proof{{\bf (a)(i)} For each $l\in\Bbb N$, set

\Centerline{$\gamma_l=\sup\{|\int_E\diverg\phi\,d\mu|:
  \phi:\BbbR^r\to\BbbR^r$ is Lipschitz,
$\|\phi\|\le\chi B(\tbf{0},l)\}$.}

\noindent If $f:\BbbR^r\to\Bbb R$ is a Lipschitz function and
$f(x)=0$ for $\|x\|\ge l$, then
$|\int_E\Pd{f}{\xi_i}d\mu|\le\gamma_l\|f\|_{\infty}$ for every $i\le r$.
\Prf\ It is enough
to consider the case $\|f\|_{\infty}=1$, since the result is trivial if
$\|f\|_{\infty}=0$, and otherwise we can look at an appropriate scalar
multiple of $f$.   In this case, set $\phi(x)=f(x)e_i$ for every $x$, where
$e_i$ is
the unit vector $(0,\ldots,0,1,0,\ldots,0)$ with a $1$ in the $i$th place.
Then $\phi$ is Lipschitz and $\|\phi\|=|f|\le\chi B(\tbf{0},l)$, so

\Centerline{$|\int_E\Pd{f}{\xi_i}d\mu|
=|\int_E\diverg\phi\,d\mu|\le\gamma_l$.\ \Qed}

\medskip

\quad{\bf (ii)}
Write $C_k$ for the space of continuous functions with compact
support from $\BbbR^r$ to $\Bbb R$.   By 473Dc and 473De,
$f*\tilde h_n\in\eusm D$ for every $f\in C_k$ and $n\in\Bbb N$.
Now the point is that

\Centerline{$L_i(f)
=\lim_{n\to\infty}\int_E\Pd{}{\xi_i}(f*\tilde h_n)d\mu$}

\noindent is defined whenever $f\in C_k$ and $i\le r$.   \Prf\ Applying
473Ed, we see that
$\|f-f*\tilde h_n\|_{\infty}\to 0$ as $n\to\infty$.   Let $l$ be such that
$f(x)=0$ for $\|x\|\ge l$.   Then
$\|(f*\tilde h_m)-(f*\tilde h_n)\|_{\infty}\to 0$ as $m$,
$n\to\infty$, while all the $f*\tilde h_m$ are zero outside
$B(\tbf{0},l+1)$ (473Dc), so that

\Centerline{$\bigl|\int_E\Pd{}{\xi_i}(f*\tilde h_m)d\mu
  -\int_E\Pd{}{\xi_i}(f*\tilde h_n)d\mu\bigr|
\le\gamma_{l+1}\|(f*\tilde h_m)-(f*\tilde h_n)\|_{\infty}
\to 0$}

\noindent as $m$, $n\to\infty$.   Thus
$\sequencen{\int_E\Pd{}{\xi_i}(f*\tilde h_n)d\mu}$ is a Cauchy sequence
and must have a limit.\ \Qed

\medskip

\quad{\bf (iii)} If $f\in C_k$ is Lipschitz and zero outside
$B(\tbf{0},l)$, then

\Centerline{$|\int_E\Pd{f}{\xi_i}d\mu
  -\int_E\Pd{}{\xi_i}(f*\tilde h_n)d\mu|
\le\gamma_{l+1}\|f-f*\tilde h_n\|_{\infty}\to 0$}

\noindent as $n\to\infty$, and
$L_i(f)=\int_E\Pd{f}{\xi_i}d\mu$.   Consequently
$|L_i(f)|\le\gamma_l\|f\|_{\infty}$.

\medskip

{\bf (b)} Because all the functionals
$f\mapsto\int_E\Pd{}{\xi_i}(f*\tilde h_n)d\mu$ are linear, $L_i$ is
linear.   Moreover, by the last remark in (a-iii), it is order-bounded when
regarded as a linear functional on the Riesz space $C_k$, so is
expressible as a difference $L_i^+-L_i^-$ of positive linear functionals
(356B).

By the Riesz Representation Theorem (436J), we have Radon measures
$\lambda_i^+$, $\lambda_i^-$ on $\BbbR^r$ such that
$L_i^+(f)=\int fd\lambda_i^+$, $L_i^-(f)=\int fd\lambda_i^-$ for every
$f\in C_k$.   Let $\hat\lambda$ be the sum
$\sum_{i=1}^r\lambda_i^++\lambda_i^-$, so
that $\hat\lambda$ is a Radon measure (416De) and every $\lambda_i^+$,
$\lambda_i^-$ is an indefinite-integral measure over $\hat\lambda$
(416Sb).

For each $i\le r$, let $g_i^+$, $g_i^-$ be Radon-Nikod\'ym derivatives
of $\lambda_i^+$, $\lambda_i^-$ with respect to $\hat\lambda$.
Adjusting them on a $\hat\lambda$-negligible set if necessary, we may
suppose that they are all bounded non-negative Borel measurable
functions from $\BbbR^r$ to $\Bbb R$.   (Recall from 256C that
$\hat\lambda$ must be the completion of its restriction to the Borel
$\sigma$-algebra.)   Set $g_i=g_i^+-g_i^-$ for each $i$.   Then

$$\eqalign{\int_E\Pd{f}{\xi_i}d\mu
&=L_i^+(f)-L_i^-(f)
=\int fd\lambda_i^+-\int fd\lambda_i^-\cr
&=\int f\times g_i^+d\hat\lambda-\int f\times g_i^-d\hat\lambda
=\int f\times g_id\hat\lambda\cr}$$

\noindent for every Lipschitz function $f$ with compact support (235K).
Set $g=\sqrt{\sum_{i=1}^rg_i^2}$.   For $i\le r$, set
$\psi_i(x)=\Bover{g_i(x)}{g(x)}$ when $g(x)\ne 0$, $\Bover1{\sqrt{r}}$
when $g(x)=0$, so that
$\psi=(\psi_1,\ldots,\psi_r):\BbbR^r\to S_{r-1}$ is Borel measurable.
Let $\lambda^{\partial}_E$ be the indefinite-integral measure over
$\hat\lambda$
defined by $g$;  then $\lambda^{\partial}_E$ is a Radon measure on
$\BbbR^r$ (256E/416Sa).

\medskip

{\bf (c)} Now take any Lipschitz function $\phi:\BbbR^r\to\BbbR^r$ with
compact support.   Express it as
$(\phi_1,\ldots,\phi_r)$ where $\phi_i:\BbbR^r\to\Bbb R$ is a Lipschitz
function with compact support for each $i$.   Then

$$\eqalignno{\int_E\diverg\phi\,d\mu
&=\sum_{i=1}^r\int_E\Pd{\phi_i}{\xi_i}d\mu
=\sum_{i=1}^rL_i(\phi_i)\cr
&=\sum_{i=1}^rL_i^+(\phi_i)-\sum_{i=1}^rL_i^-(\phi_i)
=\sum_{i=1}^r\int\phi_id\hat\lambda^+_i
  -\sum_{i=1}^r\int\phi_id\hat\lambda^-_i\cr
&=\sum_{i=1}^r\int\phi_i\times g^+_id\hat\lambda
  -\sum_{i=1}^r\int\phi_i\times g^-_id\hat\lambda\cr
\displaycause{by 235K again}
&=\sum_{i=1}^r\int\phi_i\times g_id\hat\lambda
=\sum_{i=1}^r\int\phi_i\times\psi_id\lambda^{\partial}_E\cr
\displaycause{235K once more, because $\psi_i\times g=g_i$}
&=\int\varinnerprod{\phi}{\psi}\,d\lambda^{\partial}_E.\cr}$$

\noindent So we have $\lambda^{\partial}_E$ and $\psi$ satisfying (i).

\medskip

{\bf (d)} Now suppose that $G\subseteq\BbbR^r$ is open.   If
$\phi:\BbbR^r\to\BbbR^r$ is a Lipschitz function with compact support
and $\|\phi\|\le\chi G$, then

\Centerline{$|\int_E\diverg\phi\,d\mu|
=|\int\varinnerprod{\phi}{\psi}\,d\lambda^{\partial}_E|
\le\int\|\phi\|d\lambda^{\partial}_E
\le\lambda^{\partial}_E(G)$.}

\noindent On the other hand, if $\gamma<\lambda^{\partial}_E(G)$, let
$G_0\subseteq G$ be a bounded open set such that
$\gamma<\lambda^{\partial}_E(G_0)$, and set
$\epsilon=\bover13(\lambda^{\partial}_E(G_0)-\gamma)$.   Let
$K\subseteq G_0$ be a compact set such that
$\lambda^{\partial}_E(G_0\setminus K)\le\epsilon$.   Let $\delta>0$ be
such that $\|x-y\|\ge 2\delta$ whenever $y\in K$ and
$x\in\BbbR^r\setminus G_0$, and set
$H=\{x:\inf_{y\in K}\|x-y\|<\delta\}$.
Now there are $f_1,\ldots,f_r\in C_k$ such that

\Centerline{$\sum_{i=1}^rf_i^2\le\chi H$,
\quad$\int\sum_{i=1}^r
  f_i\times\psi_id\lambda^{\partial}_E
\ge\gamma$.}

\noindent\Prf\ For each $i\le r$, we can find a sequence
$\sequence{m}{g_{mi}}$ in $C_k$ such that
$\int|g_{mi}-(\psi_i\times\chi K)|d\lambda^{\partial}_E\le 2^{-m}$ for
every $m\in\Bbb N$ (416I);
multiplying the $g_{mi}$ by a function which takes the value $1$ on $K$
and $0$ outside $H$ if necessary, we can suppose that $g_{mi}(x)=0$ for
$x\notin H$.   Set

\Centerline{$f_{mi}
=\Bover{g_{mi}}{\max(1,\sqrt{\sumop_{j=1}^rg_{mj}^2})}\in C_k$}

\noindent for each $m$ and $i$.   Now
$\lim_{m\to\infty}f_{mi}(x)=\psi_i(x)$ for every $i\le r$ whenever
$\lim_{m\to\infty}g_{mi}(x)=\psi_i(x)$ for every $i\le r$, which is the
case for $\lambda^{\partial}_E$-almost every $x\in K$.   Also
$\sum_{i=1}^rf_{mi}^2\le\chi H$ for every $m$, so
$|\sum_{i=1}^rf_{mi}\times\psi_i|\le\chi H$ for every $m$, while

\Centerline{$\lim_{m\to\infty}\sum_{i=1}^r\int_K
  f_{mi}\times\psi_id\lambda^{\partial}_E
=\sum_{i=1}^r\int_K\psi_i^2d\lambda^{\partial}_E
=\lambda^{\partial}_E(K)$.}

\noindent At the same time,

\Centerline{$|\sum_{i=1}^r\int_{\Bbb R^r\setminus K}
  f_{mi}\times\psi_id\lambda^{\partial}_E|
\le\lambda^{\partial}_E(H\setminus K)\le\epsilon$}

\noindent for every $m$, so

\Centerline{$\sum_{i=1}^r\int f_{mi}\times\psi_id\lambda^{\partial}_E
\ge\lambda^{\partial}_E(G_0)-3\epsilon=\gamma$}

\noindent for all $m$ large enough, and we may take $f_i=f_{mi}$ for
such an $m$.\ \Qed

Now, for $n\in\Bbb N$, set

\Centerline{$\phi_n=(f_1*\tilde h_n,\ldots,f_r*\tilde h_n)
\in\eusm D_r$.}

\noindent For all $n$ large enough, we shall have
$\|x-y\|\ge\Bover1{n+1}$ for every $x\in\BbbR^r\setminus G_0$ and
$y\in H$, so that $\phi_n(x)=0$ if $x\notin G_0$.   By 473Dg,

\Centerline{$\|\phi_n(x)\|
\le\sup_{y\in\Bbb R}\sqrt{\sumop_{i=1}^rf_i(y)^2}\le 1$}

\noindent for every $x$ and $n$, so that $\|\phi_n\|\le\chi G_0$ for all
$n$ large enough.   Next,
$\lim_{n\to\infty}\phi_n(x)=(f_1(x),\ldots,
\ifdim\pagewidth=390pt\penalty-100\fi
f_r(x))$ for every $x\in\BbbR^r$ (473Ed), so

\Centerline{$\int_E\diverg\phi_nd\mu
=\int\varinnerprod{\phi_n}{\psi}\,d\lambda^{\partial}_E
\to\int\sum_{i=1}^rf_i\times\psi_id\lambda^{\partial}_E
\ge\gamma$}

\noindent as $n\to\infty$, by Lebesgue's Dominated Convergence Theorem.
As $\gamma$ is arbitrary,

$$\eqalign{\lambda^{\partial}_E(G)
&\le\sup\{\int_E\diverg\phi\,d\mu:
  \phi\in\eusm D_r,\,\|\phi\|\le\chi G\}\cr
&\le\sup\{\int_E\diverg\phi\,d\mu:
  \phi\text{ is Lipschitz},\,\|\phi\|\le\chi G\}\cr}$$

\noindent and we have equality.

\medskip

{\bf (e)} Thus $\lambda^{\partial}_E$ must satisfy (ii).   By 416Eb, it
is uniquely defined.   Now suppose that $\hat\psi$ is another function
from a $\lambda^{\partial}_E$-conegligible set to $\BbbR^r$ and
satisfies (i).   Then

\Centerline{$\int\varinnerprod{\phi}{\psi}\,d\lambda^{\partial}_E
=\int\varinnerprod{\phi}{\hat\psi}\,d\lambda^{\partial}_E$}

\noindent for every Lipschitz function $\phi:\BbbR^r\to\BbbR^r$ with
compact support.   Take any $i\le r$ and any compact set
$K\subseteq\BbbR^r$.   For $m\in\Bbb N$, set
$f_m(x)=\max(0,1-2^m\inf_{y\in K}\|y-x\|)$ for $x\in\BbbR^r$, so that
$\sequence{m}{f_m}$ is a sequence of Lipschitz functions with compact
support and $\lim_{m\to\infty}f_m=\chi K$.   Set

\Centerline{$\phi_{m}=(0,\ldots,f_m,\ldots,0)$,}

\noindent where the non-zero term is in the $i$th position.   Then

$$\eqalignno{\int_K\psi_id\lambda^{\partial}_E
&=\lim_{m\to\infty}\int f_m\times\psi_id\lambda^{\partial}_E
=\lim_{m\to\infty}
 \int\varinnerprod{\phi_{m}}{\psi}\,d\lambda^{\partial}_E\cr
&=\lim_{m\to\infty}
  \int\varinnerprod{\phi_{m}}{\hat\psi}\,d\lambda^{\partial}_E
=\int_K\hat\psi_id\lambda^{\partial}_E.\cr}$$

\noindent By the Monotone Class Theorem (136C), or otherwise,
$\int_F\hat\psi_id\lambda^{\partial}_E
=\int_F\psi_id\lambda^{\partial}_E$ for every bounded Borel set $F$, so
that $\hat\psi_i=\psi_i\,\,\lambda^{\partial}_E$-a.e.;  as $i$ is
arbitrary, $\psi=\hat\psi\,\,\lambda^{\partial}_E$-a.e.   This completes
the proof.
}%end of proof of 474E

\medskip

\leader{474F}{Definitions} In the context of 474E, I will call
$\lambda^{\partial}_E$ the {\bf perimeter measure} of $E$, and if $\psi$
is a function from a $\lambda^{\partial}_E$-conegligible subset of
$\BbbR^r$ to $S_{r-1}$ which has the property in (i) of the theorem, I
will call it an {\bf outward-normal} function for $E$.

\cmmnt{The words `perimeter' and `outward normal' are intended to
suggest geometric interpretations;  much of this section and the next
will be devoted to validating this suggestion.}

Observe that if $E$ has locally finite perimeter, then
$\per E=\lambda^{\partial}_E(\BbbR^r)$.   The definitions in 474D-474E also
make it clear that if $E$, $F\subseteq\Bbb R$ are Lebesgue measurable and
$\mu(E\symmdiff F)=0$, then $\per E=\per F$ and $E$ has locally finite
perimeter iff $F$ has;  and in this case
$\lambda^{\partial}_E=\lambda^{\partial}_F$ and an outward-normal function
for $E$ is an outward-normal function for $F$.

\leader{474G}{The reduced boundary} Let $E\subseteq\BbbR^r$ be a set
with locally finite
perimeter;  let $\lambda^{\partial}_E$ be its perimeter measure and
$\psi$ an outward-normal function for $E$.   The {\bf reduced boundary}
$\partial^{\$}E$ is the set of those $y\in\BbbR^r$ such that, for some
$z\in S_{r-1}$,

\Centerline{$\lim_{\delta\downarrow 0}
\Bover1{\lambda^{\partial}_EB(y,\delta)}
\int_{B(y,\delta)}\|\psi(x)-z\|\,\lambda^{\partial}_E(dx)=0$.}

\noindent\cmmnt{(When requiring that the limit be defined, I mean to
insist that $\lambda^{\partial}_EB(y,\delta)$ should be non-zero for
every $\delta>0$, that is, that $y$ belongs to the support of
$\lambda^{\partial}_E$.  {\bf Warning!}  Some authors use the phrase
`reduced boundary' for a slightly larger set.)  }Note that, writing
$\psi=(\psi_1,\ldots,\psi_r)$ and $z=(\zeta_1,\ldots,\zeta_r)$, we must
have

\Centerline{$\zeta_i
=\lim_{\delta\downarrow 0}\Bover1{\lambda^{\partial}_EB(y,\delta)}
  \int_{B(y,\delta)}\psi_id\lambda^{\partial}_E$,}

\noindent so that $z$ is uniquely defined;  call it $\psi_E(y)$.
\cmmnt{Of course }$\partial^{\$}E$ and $\psi_E$ are determined
entirely by the set $E$\cmmnt{, because
$\lambda^{\partial}_E$ is uniquely determined and $\psi$ is determined
up to a $\lambda^{\partial}_E$-negligible set (474E)}.

\cmmnt{By Besicovitch's Density Theorem (472Db),}

\Centerline{$\lim_{\delta\downarrow 0}
  \Bover1{\lambda^{\partial}_EB(x,\delta)}
  \int_{B(x,\delta)}|\psi_i(x)-\psi_i(y)|\lambda^{\partial}_E(dx)
=0$}

\noindent for every $i\le r$, for $\lambda^{\partial}_E$-almost every
$y\in\BbbR^r$;  and for any such $y$, $\psi_E(y)$ is defined and equal
to $\psi(y)$.   \cmmnt{Thus} $\partial^{\$}E$ is
$\lambda^{\partial}_E$-conegligible and $\psi_E$ is an outward-normal
function for $E$.   I will call $\psi_E:\partial^{\$}E\to S_{r-1}$ the
{\bf canonical outward-normal function} of $E$.

Once again, we see that if $E$, $F\subseteq\BbbR^r$ are sets with locally
finite perimeter and $E\symmdiff F$ is Lebesgue negligible, then
they have the same reduced boundary and the same canonical outward-normal
function.

\leader{474H}{Invariance under isometries:  Proposition} Let
$E\subseteq\BbbR^r$ be a set with locally finite perimeter.   Let
$\lambda^{\partial}_E$ be its perimeter measure, and $\psi_E$ its canonical
outward-normal function.   If $T:\BbbR^r\to\BbbR^r$ is any
isometry, then $T[E]$ has locally finite perimeter,
$\lambda^{\partial}_{T[E]}$ is the image measure
$\lambda^{\partial}_ET^{-1}$,
the reduced boundary $\partial^{\$}T[E]$ is $T[\partial^{\$}E]$,
and the canonical
outward-normal function of $T[E]$ is $S\psi_ET^{-1}$,
where $S$ is the derivative of $T$.

\proof{{\bf (a)} As noted in 474C, the derivative of $T$ is constant,
and is an orthogonal matrix.    Suppose that $n\in\Bbb N$.
Let $\phi:\BbbR^r\to\BbbR^r$ be a Lipschitz function such that
$\|\phi\|\le\chi B(\tbf{0},n)$.   Then

$$\eqalignno{|\int_{T[E]}\diverg\phi\,d\mu|
&=|\int_E(\diverg\phi)T\,d\mu|\cr
\displaycause{263D, because $|\det S|=1$}
&=|\int_E\diverg(T^{-1}\phi T)d\mu|\cr
\displaycause{474C}
&=|\int_E\diverg(S^{-1}\phi T)d\mu|\cr
\displaycause{because $S^{-1}\phi T$ and $T^{-1}\phi T$ differ by a
constant, and must have the same derivative}
&\le\lambda^{\partial}_E(T^{-1}[B(\tbf{0},n)])\cr}$$

\noindent because $S^{-1}\phi T$ is a Lipschitz function and

\Centerline{$\|S^{-1}\phi T\|=\|\phi T\|\le\chi T^{-1}[B(\tbf{0},n)]$.}

\noindent Since $T^{-1}[B(\tbf{0},n)]$ is bounded,
$\lambda^{\partial}_E(T^{-1}[B(\tbf{0},n)])$ is finite
for every $n$, and $T[E]$ has locally finite perimeter.

\medskip

{\bf (b)} We can therefore speak of its perimeter measure
$\lambda^{\partial}_{T[E]}$.   Let $G\subseteq\BbbR^r$ be an open set.
If $\phi:\BbbR^r\to\BbbR^r$ is a Lipschitz function and
$\|\phi\|\le\chi T[G]$, then

\Centerline{$|\int_{T[E]}\diverg\phi\,d\mu|
=|\int_E\diverg(S^{-1}\phi T)d\mu|
\le\lambda^{\partial}_E(G)$}

\noindent because $S^{-1}\phi T$ is a Lipschitz function dominated by
$\chi G$.   As $\phi$ is arbitrary,
$\lambda^{\partial}_{T[E]}(T[G])\le\lambda^{\partial}_E(G)$.   Applying
the same argument in reverse, with $T^{-1}$ in the place of $T$, we see
that $\lambda^{\partial}_E(G)\le\lambda^{\partial}_{T[E]}(T[G])$, so the
two are equal.   This means that the Radon measures
$\lambda^{\partial}_{T[E]}$ and $\lambda^{\partial}_ET^{-1}$ (418I)
agree on open sets, and must be identical (416Eb again).

\medskip

{\bf (c)} Now consider $S\psi_ET^{-1}$.   Since $\psi_E$ is defined
$\lambda^{\partial}_E$-almost everywhere and takes values in $S_{r-1}$,
$\psi_E T^{-1}$ and $S\psi_E T^{-1}$ are defined
$\lambda^{\partial}_{T[E]}$-almost everywhere and take values in
$S_{r-1}$.   If $\phi:\BbbR^r\to\BbbR^r$ is a Lipschitz function with
compact support,

$$\eqalignno{\int_{T[E]}\diverg\phi\,d\mu
&=\int_E(\diverg\phi)T\,d\mu
=\int_E\diverg(T^{-1}\phi T)d\mu\cr
&=\int_E\diverg(S^{-1}\phi T)d\mu
=\int\varinnerprod{(S^{-1}\phi T)}{\psi_E}\,d\lambda^{\partial}_E\cr
&=\int\varinnerprod{(\phi T)}{(S\psi_E)}d\lambda^{\partial}_E\cr
\displaycause{because $S$ is orthogonal}
&=\int\varinnerprod{\phi}{(S\psi_E T^{-1})}
  d(\lambda^{\partial}_ET^{-1})\cr
\displaycause{235G}
&=\int\varinnerprod{\phi}{(S\psi_E T^{-1})}
  d\lambda^{\partial}_{T[E]}.\cr}$$

\noindent Accordingly $S\psi_ET^{-1}$ is an outward-normal function for
$T[E]$.    Write $\psi_{T[E]}$ for the canonical outward-normal function
of $T[E]$.

\medskip

{\bf (d)} Take $y\in\Bbb R$ and consider

$$\eqalign{\Bover1{\lambda^{\partial}_{T[E]}B(y,\delta)}
   &\int_{B(y,\delta)}\|S\psi_ET^{-1}(x)-S\psi_ET^{-1}(y)\|
   \lambda^{\partial}_{T[E]}(dx)\cr
&=\Bover1{\lambda^{\partial}_EB(T^{-1}(y),\delta)}
   \int_{B(T^{-1}(y),\delta)}\|S\psi_E(x)-S\psi_ET^{-1}(y)\|
   \lambda^{\partial}_E(dx)\cr
&=\Bover1{\lambda^{\partial}_EB(T^{-1}(y),\delta)}
   \int_{B(T^{-1}(y),\delta)}\|\psi_E(x)-\psi_ET^{-1}(y)\|
   \lambda^{\partial}_E(dx)\cr}$$

\noindent for any $\delta>0$ for which

\Centerline{$\lambda^{\partial}_{T[E]}B(y,\delta)
=\lambda^{\partial}_ET^{-1}[B(y,\delta)]
=\lambda^{\partial}_EB(T^{-1}(y),\delta)$}

\noindent is non-zero.   We see that

\Centerline{$\lim_{\delta\downarrow 0}
   \Bover1{\lambda^{\partial}_{T[E]}B(y,\delta)}
   \int_{B(y,\delta)}\|S\psi_ET^{-1}(x)-S\psi_ET^{-1}(y)\|
   \lambda^{\partial}_{T[E]}(dx)$}

\noindent is defined and equal to $0$ whenever

\Centerline{$\Bover1{\lambda^{\partial}_EB(T^{-1}(y),\delta)}
   \int_{B(T^{-1}(y),\delta)}\|\psi_E(x)-\psi_ET^{-1}(y)\|
   \lambda^{\partial}_E(dx)$}

\noindent is defined and equal to $0$, that is,
$T^{-1}(y)\in\partial^{\$}E$.   In this case,
$y\in\partial^{\$}T[E]$ and $S\psi_ET^{-1}(y)=\psi_{T[E]}(y)$.
So $\partial^{\$}T[E]\supseteq T[\partial^{\$}E]$
and $S\psi_ET^{-1}$ extends $\psi_{T[E]}$.

Applying the argument to $T^{-1}$, we see that
$S^{-1}\psi_{T[E]}T$ extends $\psi_E$, that is,
$\psi_{T[E]}$ extends $S\psi_ET^{-1}$.   So
$S\psi_ET^{-1}$ is exactly the canonical outward-normal function of
$T[E]$, and its domain $T[\partial^{\$}E]$ is $\partial^{\$}T[E]$.
}%end of proof of 474H

\leader{474I}{\dvrocolon{Half-spaces}}\cmmnt{ It will be useful, and
perhaps instructive, to check the most elementary special case.

\medskip

\noindent}{\bf Proposition} Let $H\subseteq\BbbR^r$ be a half-space
$\{x:\varinnerprod{x}{v}\le\alpha\}$, where $v\in S^{r-1}$.   Then $H$
has locally finite perimeter;  its perimeter measure
$\lambda^{\partial}_H$ is defined by saying

\Centerline{$\lambda^{\partial}_H(F)=\nu(F\cap\partial H)$}

\noindent whenever $F\subseteq\BbbR^r$ is such that $\nu$ measures
$F\cap\partial H$, and the constant function with value $v$ is an
outward-normal function for $H$.

\proof{{\bf (a)} Suppose, to begin with, that $v$ is the unit vector
$(0,\ldots,0,1)$ and that $\alpha=0$, so that $H=\{x:\xi_r\le 0\}$.   Let
$\phi:\BbbR^r\to\BbbR^r$ be a Lipschitz function with compact support.
Then for any $i<r$

\Centerline{$\int_{H}\Pd{\phi_i}{\xi_i}\mu(dx)=0$}

\noindent because we can regard this as a multiple integral in which the
inner integral is with respect to $\xi_i$ and is therefore always zero.
On the other hand, integrating with respect to the $r$th coordinate
first,

$$\eqalignno{\int_H\Pd{\phi_r}{\xi_r}\mu(dx)
&=\int_{\Bbb R^{r-1}}\int_{-\infty}^0
  \Pd{\phi_r}{\xi_r}(z,t)dt\,\mu_{r-1}(dz)\cr
&=\int_{\Bbb R^{r-1}}\phi_r(z,0)\mu_{r-1}(dz)
=\int_{\partial H}\phi_r(x)\nu(dx)\cr
\displaycause{identifying $\nu$ on $\BbbR^{r-1}\times\{0\}$ with
$\mu_{r-1}$ on $\BbbR^{r-1}$}
&=\int_{\partial H}\varinnerprod{\phi}{v}\,d\nu
=\int\varinnerprod{\phi}{v}\,d\lambda\cr}$$

\noindent where $\lambda$ is the indefinite-integral measure over $\nu$
defined by the function $\chi(\partial H)$.   Note that (by 234La)
$\lambda$ can also be regarded as $\nu_{\partial H}\iota^{-1}$, where
$\nu_{\partial H}$ is the subspace measure on $\partial H$ and
$\iota:\partial H\to\BbbR^r$ is the identity map.   Now
$\nu_{\partial H}$ can be identified with Lebesgue measure on
$\BbbR^{r-1}$, by 265B or otherwise, so in particular is a Radon
measure, and $\lambda$ also is a Radon measure, by 418I again or otherwise.

This means that $\lambda$ and the constant function with value $v$
satisfy the conditions of 474E, and must be the perimeter measure of $H$
and an outward-normal function.

\medskip

{\bf (b)} For the general case, let $S$ be an orthogonal matrix such
that $S(0,\ldots,0,1)=v$, and set $T(x)=S(x)+\alpha v$ for every $x$, so
that $H=T[\{x:\xi_r\le 0\}]$.   By 474H, the perimeter measure of $H$ is
$\lambda T^{-1}$ and the constant function with value $v$ is an
outward-normal function for $H$.   Now the Radon measure
$\lambda^{\partial}_H=\lambda T^{-1}$ is defined by saying that

\Centerline{$\lambda^{\partial}_HF
=\lambda T^{-1}[F]
=\nu(T^{-1}[F]\cap\{x:\xi_r=0\})
=\nu(F\cap T[\{x:\xi_r=0\}])
=\nu(F\cap\partial H)$}

\noindent whenever $\nu(F\cap\partial H)$ is defined, because $\nu$
(being a scalar multiple of a Hausdorff measure) must be invariant under
the isometry $T$.
}%end of proof of 474I

\leader{474J}{Lemma} Let $E\subseteq\BbbR^r$ be a set with locally
finite perimeter.   Let $\lambda^{\partial}_E$ be the perimeter measure
of $E$, and $\psi_E$ its canonical outward-normal function.   Then
$\BbbR^r\setminus E$ also has locally finite perimeter;  its perimeter
measure is $\lambda^{\partial}_E$, its reduced boundary is
$\partial^{\$}E$, and its canonical outward-normal function is
$-\psi_E$.

\proof{ Of course $\BbbR^r\setminus E$ is Lebesgue measurable.
By 474Bc,

\Centerline{$\int_{\Bbb R^r\setminus E}\diverg\phi\,d\mu
=-\int_E\diverg\phi\,d\mu
=\int\varinnerprod{\phi}{(-\psi_E)}\,d\lambda^{\partial}_E$}

\noindent for every Lipschitz function $\phi:\BbbR^r\to\BbbR^r$ with
compact support.   The uniqueness assertions in 474E tell us that
$\BbbR^r\setminus E$ has locally
finite perimeter, that its perimeter measure is $\lambda^{\partial}_E$,
and that $-\psi_E$ is an outward-normal function for
$\BbbR^r\setminus E$.   Referring to the definition of `reduced
boundary' in 474G, we see at once that
$\partial^{\$}(\BbbR^r\setminus E)=\partial^{\$}E$ and that
$\psi_{\Bbb R^r\setminus E}=-\psi_E$.
}%end of proof of 474J

\leader{474K}{Lemma} Let $E\subseteq\BbbR^r$ be a set with locally
finite perimeter;  let $\lambda^{\partial}_E$ be its perimeter measure,
and $\psi$ an outward-normal function for $E$.   Let
$\phi:\BbbR^r\to\BbbR^r$ be a Lipschitz function with compact support,
and $g\in\eusm D$ an even function.   Then

\Centerline{$\int\varinnerprod{\phi}{\grad(g*\chi E)}d\mu
  +\int\varinnerprod{(g*\phi)}{\psi}\,d\lambda^{\partial}_E
=0$.}

\proof{

$$\eqalignno{\int\varinnerprod{\phi}{\grad(g*\chi E)}d\mu
&=-\int(g*\chi E)\times\diverg\phi\,d\mu\cr
\displaycause{474Bd, using 473Dd to see that $g*\chi E$ is
Lipschitz}
&=-\int\chi E\times\diverg(g*\phi)d\mu\cr
\displaycause{474Be}
&=-\int\varinnerprod{(g*\phi)}{\psi}\,d\lambda^{\partial}_E\cr}$$

\noindent (because $g*\phi$ is smooth and has compact support, so is
Lipschitz), as required.
}%end of proof of 474K

\leader{474L}{Two isoperimetric inequalities:  Theorem} Let
$E\subseteq\BbbR^r$ be a set with locally finite perimeter, and
$\lambda^{\partial}_E$ its perimeter measure.

(a) If $E$ is bounded, then $(\mu E)^{(r-1)/r}\le\per E$.

(b) If $B\subseteq\BbbR^r$ is a closed ball, then

\Centerline{$\min(\mu(B\cap E),\mu(B\setminus E))^{(r-1)/r}
\le 2c\lambda^{\partial}_E(\interior B)$,}

\noindent where $c=2^{r+4}\sqrt r(1+2^{r+1})$.

\proof{{\bf (a)} Let $\epsilon>0$.   By 473Ef, there is an $n\in\Bbb N$
such that
$\|f-\chi E\|_{r/(r-1)}\le\epsilon$, where $f=\chi E*\tilde h_n$.   Note
that $f$ is smooth (473De again) and has compact support, because $E$ is
bounded.   Let $\eta>0$ be such that

\Centerline{$\int\|\grad f\|d\mu
\le\int\Bover{\|\grad f\|^2}{\sqrt{\eta+\|\grad f\|^2}}
  \,d\mu+\epsilon$,}

\noindent and set $\phi=\Bover{\grad f}{\sqrt{\eta+\|\grad f\|^2}}$.
Then $\phi\in\eusm D_r$ and $\|\phi(x)\|\le 1$ for every $x\in\BbbR^r$.
Now we can estimate

$$\eqalignno{\int\|\grad f\|d\mu
&\le\int\varinnerprod{\phi}{\grad f}\,d\mu+\epsilon\cr
&=-\int\varinnerprod{(\tilde h_n*\phi)}{\psi}\,d\lambda^{\partial}_E
  +\epsilon\cr
\displaycause{where $\psi$ is an outward-normal function for $E$, by
474K}
&\le\int\|\tilde h_n*\phi\|d\lambda^{\partial}_E+\epsilon
\le\per E+\epsilon\cr}$$

\noindent because $\|(\tilde h_n*\phi)(x)\|\le 1$ for every
$x\in\BbbR^r$, by 473Dg.   Accordingly

$$\eqalignno{(\mu E)^{(r-1)/r}
&=\|\chi E\|_{r/(r-1)}
\le\|f\|_{r/(r-1)}+\epsilon
\le\int\|\grad f\|d\mu+\epsilon\cr
\displaycause{473H}
&\le\per E+2\epsilon.\cr}$$

\noindent As $\epsilon$ is arbitrary, we have the result.

\medskip

{\bf (b)(i)} Set
$\alpha=\min(\mu(B\cap E),\mu(B\setminus E))$.   If $\alpha=0$, the
result is trivial;  so suppose that $\alpha>0$.   Take
any $\epsilon\in\ocint{0,\alpha}$.   Let $B_1$ be a closed ball, with
the same centre as $B$ and strictly smaller non-zero radius, such that
$\mu(B\setminus B_1)\le\epsilon$;  then
$\alpha-\epsilon\le\min(\mu(B_1\cap E),\mu(B_1\setminus E))$.   For
$f\in\eusm L^{r/(r-1)}(\mu)$ set

\Centerline{$\gamma_0(f)=\Bover1{\mu B_1}\int_{B_1}fd\mu$,
\quad$\gamma_1(f)=\|(f\times\chi B_1)-\gamma_0(f)\chi B_1\|_{r/(r-1)}$;}

\noindent then both $\gamma_0$ and $\gamma_1$ are continuous functions
on $\eusm L^{r/(r-1)}(\mu)$ if we give it its usual pseudometric
$(f,g)\mapsto\|f-g\|_{r/(r-1)}$.   Now
$\gamma_1(\chi(E\cap B))\ge\bover12(\alpha-\epsilon)^{(r-1)/r}$.   \Prf\
We have

\Centerline{$\gamma_0(\chi(E\cap B))=\Bover{\mu(B_1\cap E)}{\mu B_1}$,}

$$\eqalign{\gamma_1(\chi(E\cap B))^{r/(r-1)}
&=\int_{B_1}|\chi(E\cap B)
  -\gamma_0(\chi(E\cap B))|^{r/(r-1)}\cr
&=\mu(B_1\cap E)\bigl(1-\Bover{\mu(B_1\cap E)}{\mu B_1}\bigr)^{r/(r-1)}
   +\mu(B_1\setminus E)
     \bigl(\Bover{\mu(B_1\cap E)}{\mu B_1}\bigr)^{r/(r-1)}\cr
&=\mu(B_1\cap E)
    \bigl(\Bover{\mu(B_1\setminus E)}{\mu B_1}\bigr)^{r/(r-1)}
   +\mu(B_1\setminus E)
    \bigl(\Bover{\mu(B_1\cap E)}{\mu B_1}\bigr)^{r/(r-1)}.\cr}$$

\noindent Either $\mu(B_1\cap E)\ge\bover12\mu B_1$ or
$\mu(B_1\setminus E)\ge\bover12\mu B_1$;  suppose the former.   Then

\Centerline{$\gamma_1(\chi(E\cap B))^{r/(r-1)}
\ge\Bover1{2^{r/(r-1)}}\mu(B_1\setminus E)
\ge\Bover1{2^{r/(r-1)}}(\alpha-\epsilon)$}

\noindent and
$\gamma_1(\chi(E\cap B))\ge\Bover12(\alpha-\epsilon)^{(r-1)/r}$.
Exchanging $B_1\cap E$ and $B_1\setminus E$ we have the same result if
$\mu(B_1\cap E)\ge\bover12\mu B_1$.\ \Qed

\medskip

\quad{\bf (ii)} Express $B$ as $B(y,\delta)$ and $B_1$ as
$B(y,\delta_1)$.   Take $n_0\ge\Bover2{\delta-\delta_1}$.  Because
$\gamma_1$ is
$\|\,\|_{r/(r-1)}$-continuous, there is an $n\ge n_0$ such that
$\gamma_1(f)\ge\bover12(\alpha-\epsilon)^{(r-1)/r}-\epsilon$, where
$f=\tilde h_n*\chi(E\cap B)$ (473Ef);  as in part (a) of the proof,
$f\in\eusm D$.   Let $\eta>0$ be such that

\Centerline{$\int_{B_1}
  \Bover{\|\grad f\|^2}{\sqrt{\eta+\|\grad f\|^2}}d\mu
\ge\int_{B_1}\|\grad f\|d\mu-\epsilon$.}

\noindent Let $m\ge n_0$ be such that
$\int\varinnerprod{\phi}{\grad f}\,d\mu
\ge\int_{B_1}\|\grad f\|d\mu-2\epsilon$, where

\Centerline{$\phi=\tilde h_m
  *(\Bover{\grad f}{\sqrt{\eta+\|\grad f\|^2}}\times\chi B_1)$.}

\noindent Note that $\phi(x)=0$ if
$\|x-y\|\ge\bover12(\delta+\delta_1)$, so that $(\tilde h_n*\phi)(x)=0$
if $x\notin\interior B$.   By 473Dg, $\|\phi(x)\|\le 1$ for every $x$
and $\|(\tilde h_n*\phi)(x)\|\le 1$ for every $x$, so
$\|\tilde h_n*\phi\|\le\chi(\interior B)$.

Now we have

$$\eqalignno{\int\varinnerprod{\phi}{\grad f}\,d\mu
&=\int\varinnerprod{\phi}{\grad(\tilde h_n*\chi(E\cap B))}\,d\mu\cr
&=-\int(\tilde h_n*\chi(E\cap B))\times\diverg\phi\,d\mu\cr
\displaycause{474Bd}
&=-\int_{E\cap B}\diverg(\tilde h_n*\phi)d\mu\cr
\displaycause{474Be}
&=-\int_E\diverg(\tilde h_n*\phi)d\mu
\le\lambda^{\partial}_E(\interior B)\cr}$$

\noindent (474E).

\medskip

\quad{\bf (iii)} Accordingly

$$\eqalignno{\Bover12(\alpha-\epsilon)^{(r-1)/r}-\epsilon
&\le\gamma_1(f)
\le c\int_{B_1}\|\grad f\|d\mu\cr
\displaycause{473K}
&\le c\bigl(\int\varinnerprod{\phi}{\grad f}\,d\mu+2\epsilon\bigr)
\le c(\lambda^{\partial}_E(\interior B)+2\epsilon).\cr}$$

\noindent As $\epsilon$ is arbitrary,
$\alpha^{(r-1)/r}\le 2c\lambda^{\partial}_E(\interior B)$, as claimed.
}%end of proof of 474L

\leader{474M}{Lemma} Suppose that $E\subseteq\BbbR^r$ has locally finite
perimeter, with perimeter measure $\lambda^{\partial}_E$ and an
outward-normal function $\psi$.
Then for any $y\in\BbbR^r$ and any Lipschitz function
$\phi:\BbbR^r\to\BbbR^r$,

\Centerline{$\int_{E\cap B(y,\delta)}\diverg\phi\,d\mu
=\int_{B(y,\delta)}\varinnerprod{\phi}{\psi}\,d\lambda^{\partial}_E
+\int_{E\cap\partial B(y,\delta)}
  \varinnerprod{\phi(x)}{\Bover1{\delta}(x-y)}\,\nu(dx)$}

\noindent for almost every $\delta>0$.

\proof{{\bf (a)} For $t>0$, set

\Centerline{$w(t)
=\int_{E\cap\partial B(y,t)}\varinnerprod{\phi(x)}
  {\Bover1t(x-y)}\,\nu(dx)$}

\noindent when this is defined.   By 265G, applied to functions of the
form

$$x\mapsto
\oldcases{\varinnerprod{\phi(x)}{\Bover{x-y}{\|x-y\|}}
  &if $x\in E$ and $0<\|x-y\|\le\alpha$\cr
0&otherwise,\cr}$$

\noindent $w$ is defined almost everywhere in $\ooint{0,\infty}$ and is
measurable (for Lebesgue measure on $\Bbb R$).

Let $\delta>0$ be any point in the Lebesgue set of $w$ (223D).   Then

\Centerline{$\lim_{t\downarrow 0}\Bover1t\int_{\delta}^{\delta+t}
  |w(s)-w(\delta)|ds
\le 2\lim_{t\downarrow 0}\Bover1{2t}\int_{\delta-t}^{\delta+t}
  |w(s)-w(\delta)|
=0$.}

\noindent Let $\epsilon>0$.   Then there is an $\eta>0$ such that

\Centerline{$\Bover1{\eta}\int_{\delta}^{\delta+\eta}
  |w(s)-w(\delta)|ds
  \le\epsilon$,
\quad$\int_{B(y,\delta+\eta)\setminus B(y,\delta)}
  \|\phi\|d\lambda^{\partial}_E\le\epsilon$,}

\Centerline{$\int_{B(y,\delta+\eta)\setminus B(y,\delta)}
  \|\diverg\phi\|d\mu
\le\epsilon$.}

\medskip

{\bf (b)} Set

$$\eqalign{g(x)&=1\text{ if }\|x-y\|\le\delta,\cr
&=1-\Bover1{\eta}(\|x-y\|-\delta)
  \text{ if }\delta\le\|x-y\|\le\delta+\eta,\cr
&=0\text{ if }\|x-y\|\ge\delta+\eta.\cr}$$

\noindent Then $g$ is continuous, and $\grad g(x)=\tbf{0}$ if
$\|x-y\|<\delta$
or $\|x-y\|>\delta+\eta$;  while if $\delta<\|x-y\|<\delta+\eta$,
$\grad g(x)=-\Bover{x-y}{\eta\|x-y\|}$.   This means that

$$\eqalign{\int_E\varinnerprod{\phi}{\grad g}\,d\mu
&=-\Bover1{\eta}\int_{\delta}^{\delta+\eta}\int_{E\cap\partial B(y,t)}
  \varinnerprod{\Bover1t(x-y)}{\phi(x)}\,\nu(dx)dt\cr
&=-\Bover1{\eta}\int_{\delta}^{\delta+\eta}w(t)dt.\cr}$$

\noindent By the choice of $\eta$,

\Centerline{$|\int_E\varinnerprod{\phi}{\grad g}\,d\mu+w(\delta)|
\le\epsilon$.}

\medskip

{\bf (c)} By 474E and 474Bb we have

$$\eqalignno{\int\varinnerprod{(g\times\phi)}{\psi}
  \,d\lambda^{\partial}_E
&=\int_E\diverg(g\times\phi)d\mu\cr
\displaycause{of course $g\times\phi$ is Lipschitz, by 473Ca and 262Ba}
&=\int_E\varinnerprod{\phi}{\grad g}\,d\mu
   +\int_Eg\times\diverg\phi\,d\mu.\cr}$$

\noindent Next, by the choice of $\eta$,

\Centerline{$|\int\varinnerprod{((g\times\phi)}{\psi}\,
  d\lambda^{\partial}_E-\int_{B(y,\delta)}\varinnerprod{\phi}{\psi}
  \,d\lambda^{\partial}_E|
\le\int_{B(y,\delta+\eta)\setminus B(y,\delta)}
  \|\phi\|d\lambda^{\partial}_E
\le\epsilon$,}

\noindent while

\Centerline{$|\int_E\varinnerprod{\phi}{\grad g}\,d\mu
+\int_{E\cap\partial B(y,\delta)}
    \varinnerprod{\phi(x)}{\Bover1{\delta}(x-y)}\nu(dx)|
=|\int_E\varinnerprod{\phi}{\grad g}\,d\mu+w(\delta)|
\le\epsilon$}

\noindent and

$$\eqalign{\bigl|\int_Eg\times\diverg\phi\,d\mu
  &-\int_{E\cap B(y,\delta)}\diverg\phi\,d\mu\bigr|\cr
&\le\int_{B(y,\delta+\eta)\setminus B(y,\delta)}\|\diverg\phi\|d\mu
\le\epsilon.\cr}$$

\noindent Putting these together, we have

\Centerline{$|\int_{E\cap B(y,\delta)}\diverg\phi\,d\mu
-\int_{B(y,\delta)}\varinnerprod{\phi}{\psi}\,d\lambda^{\partial}_E
-\int_{E\cap\partial B(y,\delta)}
  \varinnerprod{\phi(x)}{\Bover1{\delta}(x-y)}\nu(dx)|
\le 3\epsilon$.}

\noindent As $\epsilon$ is arbitrary, this gives the result.
}%end of proof of 474M

\vleader{60pt}{474N}{Lemma} Let $E\subseteq\BbbR^r$ be a set with locally
finite perimeter, and $\lambda^{\partial}_E$ its perimeter measure.
Then, for any $y\in\partial^{\$}E$,

\quad(i) $\liminf_{\delta\downarrow 0}
  \Bover{\mu(B(y,\delta)\cap E)}{\delta^r}
\ge\Bover1{(3r)^r}$;

\quad(ii) $\liminf_{\delta\downarrow 0}
  \Bover{\mu(B(y,\delta)\setminus E)}{\delta^r}
\ge\Bover1{(3r)^r}$;

\quad(iii) $\liminf_{\delta\downarrow 0}
  \Bover{\lambda^{\partial}_EB(y,\delta)}{\delta^{r-1}}
\ge\Bover1{2c(3r)^{r-1}}$,

\noindent where $c=2^{r+4}\sqrt r(1+2^{r+1})$;

\quad(iv) $\limsup_{\delta\downarrow 0}
  \Bover{\lambda^{\partial}_EB(y,\delta)}{\delta^{r-1}}
\le 4\pi\beta_{r-2}$.

\proof{{\bf (a)} Let $\psi_E$ be the canonical outward-normal function
of $E$ (474G).   Take $y\in\partial^{\$}E$.   Set

\Centerline{$\Phi=\{\phi:\phi$ is a Lipschitz function from
$\BbbR^r$ to $B(\tbf{0},1)\}$.}

\noindent Because the space $L^1(\mu)$ is separable in its usual (norm)
topology (244I), so is
$\{(\diverg\phi\times\chi B(y,1))^{\ssbullet}:\phi\in\Phi\}$
(4A2P(a-iv)), and there must be
a countable set $\Phi_0\subseteq\Phi$ such that

\inset{whenever $\phi\in\Phi$ and $m\in\Bbb N$ there is a
$\hat\phi\in\Phi_0$ such that
\discrcenter{390pt}
{$\int_{B(y,1)}|\diverg\phi-\diverg\hat\phi|d\mu\le 2^{-m}$.}}

\noindent Now, for each $\phi\in\Phi_0$,

$$\eqalign{|\int_{E\cap B(y,\delta)}\diverg\phi\,d\mu|
&=|\int_{B(y,\delta)}\varinnerprod{\phi}{\psi_E}\,d\lambda^{\partial}_E
+\int_{E\cap\partial B(y,\delta)}
  \varinnerprod{\phi(x)}{\Bover1{\delta}(x-y)}\nu(dx)|\cr
&\le\lambda^{\partial}_EB(y,\delta)
+\nu(E\cap\partial B(y,\delta))\cr}$$

\noindent for almost every $\delta>0$, by 474M.   But this means that,
for almost every $\delta\in\ocint{0,1}$,

$$\eqalign{\per(E\cap B(y,\delta))
&=\sup_{\phi\in\Phi}|\int_{E\cap B(y,\delta)}\diverg\phi\,d\mu|\cr
&=\sup_{\phi\in\Phi_0}|\int_{E\cap B(y,\delta)}\diverg\phi\,d\mu|
\le\lambda^{\partial}_EB(y,\delta)
  +\nu(E\cap\partial B(y,\delta)).\cr}$$

\medskip

{\bf (b)} It follows that, for some $\delta_0>0$,

\Centerline{$\per(E\cap B(y,\delta))
\le 3\nu(E\cap\partial B(y,\delta))$}

\noindent for almost every $\delta\in\ocint{0,\delta_0}$.   \Prf\
Applying 474M with $\phi(x)=\psi_E(y)$ for every $x$, we have

\Centerline{$0
=\int_{B(y,\delta)}\varinnerprod{\psi_E(y)}{\psi_E(x)}
     \lambda^{\partial}_E(dx)
  +\int_{E\cap\partial B(y,\delta)}
    \varinnerprod{\psi_E(y)}{\Bover1{\delta}(x-y)}\nu(dx)$}

\noindent for almost every $\delta\in[0,1]$.   But by the definition of
$\psi_E(y)$,

\Centerline{$\lim_{\delta\downarrow 0}
  \Bover1{\lambda^{\partial}_EB(y,\delta)}\int_{B(y,\delta)}
     \varinnerprod{\psi_E(y)}{\psi_E}\,d\lambda^{\partial}_E
=\lim_{\delta\downarrow 0}
  \Bover1{\lambda^{\partial}_EB(y,\delta)}\int_{B(y,\delta)}
     \varinnerprod{\psi_E(y)}{\psi_E(y)}\,d\lambda^{\partial}_E
=1$.}

\noindent So there is some $\delta_0>0$ such that, for almost every
$\delta\in\ocint{0,\delta_0}$,

$$\eqalignno{\lambda^{\partial}_EB(y,\delta)
&\le 2\int_{B(y,\delta)}
  \varinnerprod{\psi_E(y)}{\psi_E}\,d\lambda^{\partial}_E\cr
&=-2\int_{E\cap\partial B(y,\delta)}
   \varinnerprod{\psi_E(y)}{\Bover1{\delta}(x-y)}\nu(dx)
\le 2\nu(E\cap\partial B(y,\delta)).&(\dagger)\cr}$$

\noindent But this means that, for almost every such $\delta$,

\Centerline{$\per(E\cap B(y,\delta))
\le\lambda^{\partial}_EB(y,\delta)
  +\nu(E\cap\partial B(y,\delta))
\le 3\nu(E\cap\partial B(y,\delta))$.   \Qed}

\medskip

{\bf (c)} Set $g(t)=\mu(E\cap B(y,t))$ for $t\ge 0$.   By 265G,
$g(t)=\int_0^{t}\nu(E\cap\partial B(y,s))ds$ for every $t$, so $g$ is
absolutely continuous on $[0,1]$ and $g'(t)=\nu(E\cap\partial B(y,t))$
for almost every $t$.   Now we can estimate

$$\eqalignno{g(t)^{(r-1)/r}
&=\mu(E\cap B(y,t))^{(r-1)/r}
\le\per(E\cap B(y,t))\cr
\displaycause{474La}
&\le 3\nu(E\cap\partial B(y,t))
=3g'(t)\cr}$$

\noindent for almost every $t\in[0,\delta_0]$.   So

\Centerline{$\Bover{d}{dt}\bigl(g(t)^{1/r}\bigr)
=\Bover1rg(t)^{(1-r)/r}g'(t)\ge\Bover1{3r}$}

\noindent for almost every $t\in[0,\delta_0]$;  since
$t\mapsto g(t)^{1/r}$ is non-decreasing, $g(t)^{1/r}\ge\Bover{t}{3r}$
(222C) and $g(t)\ge(3r)^{-r}t^r$ for every $t\in[0,\delta_0]$.

\medskip

{\bf (d)} Accordingly

\Centerline{$\liminf_{\delta\downarrow 0}
  \Bover{\mu(B(y,\delta)\cap E)}{\delta^r}
\ge\inf_{0<\delta\le\delta_0}
  \Bover{\mu(B(y,\delta)\cap E)}{\delta^r}
\ge\Bover1{(3r)^r}$.}

\noindent This proves (i).

\medskip

{\bf (e)} Because
$\lambda^{\partial}_{\Bbb R^r\setminus E}=\lambda^{\partial}_E$ and
$-\psi_E$ is the canonical outward-normal function of
$\BbbR^r\setminus E$ (474J), $y$ also belongs to
$\partial^{\$}(\BbbR^r\setminus E)$, so
the second formula of this lemma follows from the first.

\medskip

{\bf (f)} By 474Lb,

\Centerline{$\lambda^{\partial}_EB(y,\delta)
\ge\Bover1{2c}\min\bigl(\mu(B(y,\delta)\cap E),
  \mu(B(y,\delta)\setminus E)\bigr)^{(r-1)/r}$}

\noindent for every $\delta\ge 0$.   So

$$\eqalign{\liminf_{\delta\downarrow 0}
  \bover{\lambda^{\partial}_EB(y,\delta)}{\delta^{r-1}}
&\ge\Bover1{2c}\liminf_{\delta\downarrow 0}
  \min\bigl(\bover{\mu(B(y,\delta)\cap E)}{\delta^r},
  \bover{\mu(B(y,\delta)\setminus E)}{\delta^r}\bigr)^{(r-1)/r}\cr
&\ge\Bover1{2c}\min\bigl(\liminf_{\delta\downarrow 0}
  \bover{\mu(B(y,\delta)\cap E)}{\delta^r},
  \liminf_{\delta\downarrow 0}\bover{\mu(B(y,\delta)\setminus E)}{\delta^r}
  \bigr)^{(r-1)/r}\cr
&\ge\Bover1{2c}\bigl(\Bover1{(3r)^r}\bigr)^{(r-1)/r}
=\Bover1{2c(3r)^{r-1}}.\cr}$$

\noindent Thus (iii) is true.

\medskip

{\bf (g)} Returning to the inequality ($\dagger$) in the proof of (b)
above, we have a $\delta_0>0$ such that

\Centerline{$\lambda^{\partial}_EB(y,\delta)
\le 2\nu(E\cap\partial B(y,\delta))
\le 2\nu(\partial B(y,\delta))=4\pi\beta_{r-2}\delta^{r-1}$}

\noindent (265F) for almost every $\delta\in\ocint{0,\delta_0}$.   But this means that, for any $\delta\in\coint{0,\delta_0}$,

\Centerline{$\lambda^{\partial}_EB(y,\delta)
\le\inf_{t>\delta}\lambda^{\partial}_EB(y,t)
\le\inf_{t>\delta}4\pi\beta_{r-2}t^{r-1}
=4\pi\beta_{r-2}\delta^{r-1}$,}

\noindent and (iv) is true.
}%end of proof of 474N

\leader{474O}{Definition} Let $A\subseteq\BbbR^r$ be any set, and
$y\in\BbbR^r$.   A {\bf Federer exterior normal to
$A$ at $y$} is a $v\in S_{r-1}$ such that,

\Centerline{$\lim_{\delta\downarrow 0}
  \Bover{\mu^*((H\symmdiff A)\cap B(y,\delta))}{\mu B(y,\delta)}=0$,}

\noindent where $H$ is the half-space
$\{x:\varinnerprod{(x-y)}{v}\le 0\}$.

\leader{474P}{Lemma} If $A\subseteq\BbbR^r$ and $y\in\BbbR^r$, there can
be at most one Federer exterior normal to $A$ at $y$.

\proof{ Suppose that $v$, $v'\in S_{r-1}$ are two Federer exterior
normals to $E$ at $y$.   Set

\Centerline{$H=\{x:\varinnerprod{(x-y)}{v}\le 0\}$,
\quad$H'=\{x:\varinnerprod{(x-y)}{v'}\le 0\}$.}


\noindent Then

\Centerline{$\lim_{\delta\downarrow 0}
  \Bover{\mu((H\symmdiff H')\cap B(y,\delta))}{\mu B(y,\delta)}
\le\lim_{\delta\downarrow 0}
  \Bover{\mu^*((H\symmdiff A)\cap B(y,\delta))}{\mu B(y,\delta)}
 +\lim_{\delta\downarrow 0}
  \Bover{\mu^*((H'\symmdiff A)\cap B(y,\delta))}{\mu B(y,\delta)}
=0$.}

\noindent But for any $\delta>0$,

\Centerline{$(H\symmdiff H')\cap B(y,\delta)
=y+\delta((H_0\symmdiff H'_0)\cap B(\tbf{0},1))$,}

\noindent where

\Centerline{$H_0=\{x:\varinnerprod{x}{v}\le 0\}$,
\quad$H'_0=\{x:\varinnerprod{x}{v'}\le 0\}$.}

\noindent So

\Centerline{$0
=\lim_{\delta\downarrow 0}
  \Bover{\mu((H\symmdiff H')\cap B(y,\delta))}{\mu B(y,\delta)}
=\Bover{\mu((H_0\symmdiff H_0')\cap B(\tbf{0},1))}{\mu B(\tbf{0},1)}
=\Bover{\mu((H_0\symmdiff H_0')\cap B(\tbf{0},n))}{\mu B(\tbf{0},n)}$}

\noindent for every $n\ge 1$, and $\mu(H_0\symmdiff H_0')=0$.   Since
$\mu$ is strictly positive, and $H_0$ and $H_0'$ are both the closures
of their interiors, they must be identical;  and it follows that $v=v'$.
}%end of proof of 474P

\leader{474Q}{Lemma}\dvAnew{2012} Set $c'=2^{r+3}\sqrt{r-1}(1+2^r)$.
Suppose that $c^*$, $\epsilon$ and $\delta$ are such that

\Centerline{$c^*\ge 0$,
\quad$\delta>0$,\quad$0<\epsilon<\Bover1{\sqrt{2}}$,
\quad$c^*\epsilon^3<\bover14\beta_{r-1}$,
\quad$4c'\epsilon\le\bover18\beta_{r-1}$.}

\noindent Set
$V_{\delta}=\{z:z\in\BbbR^{r-1}$, $\|z\|\le\delta\}$ and
$C_{\delta}=V_{\delta}\times[-\delta,\delta]$, regarded as a cylinder
in $\BbbR^r$.   Let $f\in\eusm D$ be such that

\Centerline{$\int_{C_{\delta}}
  \|\grad_{r-1}f\|+\max(\Pd{f}{\xi_r},0)d\mu
\le c^*\epsilon^3\delta^{r-1}$,}

\noindent where $\grad_{r-1}f
=(\Pd{f}{\xi_1},\ldots,\Pd{f}{\xi_{r-1}},0)$.  Set

\Centerline{$F=\{x:x\in C_{\delta}$, $f(x)\ge\Bover34\}$,
\quad$F'=\{x:x\in C_{\delta}$, $f(x)\le\Bover14\}$.}

\noindent and for $\gamma\in\Bbb R$ set
$H_{\gamma}=\{x:x\in\BbbR^r$, $\xi_r\le\gamma\}$.   Then there is a
$\gamma\in\Bbb R$ such that

\Centerline{$\mu(F\symmdiff(H_{\gamma}\cap C_{\delta}))
\le 9\mu(C_{\delta}\setminus(F\cup F'))
   +(c^*\beta_{r-1}+16c')\epsilon\delta^r$.}

\proof{{\bf (a)} For $t\in[-\delta,\delta]$ set

\Centerline{$f_t(z)=f(z,t)$ for $z\in\BbbR^{r-1}$,}

\Centerline{$F_t=\{z:z\in V_{\delta}$, $f_t(z)\ge\Bover34\},$
\quad$F'_t=\{z:z\in V_{\delta}$, $f_t(z)\le\Bover14\}$;}

\noindent set

\Centerline{$\gamma
=\sup\bigl(\{-\delta\}\cup\{t:t\in[-\delta,\delta]$,
  $\mu_{r-1}F_t\ge\Bover34\mu_{r-1}V_{\delta}\}\bigr)$,}

\Centerline{$G
=\{t:t\in[-\delta,\delta]$,
  $\int_{V_{\delta}}\|\grad f_t\|d\mu_{r-1}\ge\epsilon^2\delta^{r-2}\}$.}

\noindent Note that $(\grad_{r-1}f)(z,t)=((\grad f_t)(z),0)$, so we
have

\Centerline{$\int_{-\delta}^{\delta}\int_{V_{\delta}}
   \|\grad f_t\|d\mu_{r-1}dt
=\int_{C_{\delta}}\|\grad_{r-1}f\|d\mu
\le c^*\epsilon^3\delta^{r-1}$}

\noindent and $\mu_1G\le c^*\epsilon\delta$, where $\mu_1$ is Lebesgue
measure on $\Bbb R$.

\medskip

{\bf (b)} If $t\in[-\delta,\delta]\setminus G$, then

\Centerline{$\min(\mu_{r-1}F'_t,\mu_{r-1}F_t)
\le 4c'\epsilon\delta^{r-1}$.}

\noindent\Prf\ If $r>2$,

$$\eqalignno{\min\bigl(\mu_{r-1}F'_t,
  \mu_{r-1}F_t)\bigr)^{(r-2)/(r-1)}
&\le 4c'\int_{V_{\delta}}\|\grad f_t\|d\mu_{r-1}\cr
\displaycause{473L}
&\le 4c'\epsilon^2\delta^{r-2}\cr}$$

\noindent because $t\notin G$, so that

$$\eqalign{\min(\mu_{r-1}F't,\mu_{r-1}F_t)
&\le (4c'\epsilon^2)^{(r-1)/(r-2)}\delta^{r-1}\cr
&\le 4c'\epsilon\delta^{r-1}\cr}$$

\noindent because $4c'\ge 1$ and $\Bover{2(r-1)}{r-2}\ge 1$ and
$\epsilon\le 1$.   If $r=2$, then

\Centerline{$\int_{V_{\delta}}\|\grad f_t\|d\mu_{r-1}
\le\epsilon^2<\Bover12$,}

\noindent so at least one of $F'_t$,
$F_t$ is empty, as noted in 473M, and
$\min(\mu_{r-1}F'_t,\mu_{r-1}F_t)=0$.\ \Qed

\medskip

{\bf (c)} If $-\delta\le s<t\le\delta$, then

\Centerline{$\int_{-\delta}^{\delta}
  \max(\Pd{f}{\xi_r}(z,\xi),0)d\xi
\ge\int_{s}^{t}\Pd{f}{\xi_r}(z,\xi)d\xi
=f(z,t)-f(z,s)\ge\Bover12$}

\noindent for every $z\in F'_s\cap F_t$.   Accordingly

\Centerline{$\Bover12\mu_{r-1}(F'_s\cap F_t)
\le\int_{C_{\delta}}\max(\Pd{f}{\xi_r},0)d\mu
\le c^*\epsilon^3\delta^{r-1}
<\Bover14\beta_{r-1}\delta^{r-1}
=\Bover14\mu_{r-1}V_{\delta}$}

\noindent and $\mu_{r-1}(F'_s\cap F_t)<\bover12\mu_{r-1}V_{\delta}$.
It follows that if $-\delta\le s<\gamma$, so that there is a $t>s$ such
that $\mu_{r-1}F_t\ge\bover34\mu_{r-1}V_{\delta}$, then
$\mu_{r-1}F'_s<\bover34\mu_{r-1}V_{\delta}$.

\medskip

{\bf (d)} Now

\Centerline{$\mu((F\symmdiff(H_{\gamma}\cap C_{\delta}))
\le 9\mu(C_{\delta}\setminus(F\cup F'))
   +\epsilon\delta^r(c^*\beta_{r-1}+16c')$.}

\noindent\Prf\ Set

\Centerline{$\tilde G
=\{t:-\delta\le t\le\delta$,
  $\mu_{r-1}(F_t\cup F'_t)
    \le\Bover78\mu_{r-1}V_{\delta}\}$,}

\Centerline{$\hat G=\{t:-\delta\le t\le\delta$,
  $\mu_{r-1}F_t
    \le 4c'\epsilon\delta^{r-1}\}$,}

\Centerline{$\hat G'=\{t:-\delta\le t\le\delta$,
  $\mu_{r-1}F'_t
    \le 4c'\epsilon\delta^{r-1}\}$.}

\noindent Then

\Centerline{$\Bover18\mu_{r-1}V_{\delta}\cdot\mu_1\tilde G
\le\mu(C_{\delta}\setminus(F\cup F'))$,}

\Centerline{$\mu(F\cap(V_{\delta}\times\hat G))
\le 8c'\epsilon\delta^r$,}

\Centerline{$\mu(F'\cap(V_{\delta}\times\hat G'))
\le 8c'\epsilon\delta^r$.}

\noindent So if we set

$$\eqalign{W
=(C_{\delta}\setminus(F\cup F'))
  &\cup(V_{\delta}
    \times(\tilde G\cup G\cup\{\gamma\}))\cr
&\cup(F\cap(V_{\delta}\times\hat G))
  \cup(F'\cap(V_{\delta}\times\hat G')),\cr}$$

\noindent we shall have

$$\eqalign{\mu W
&\le\mu(C_{\delta}\setminus(F\cup F'))
   +\mu_1\tilde G\cdot\mu_{r-1}V_{\delta}
   +\mu_1G\cdot\mu_{r-1}V_{\delta}
   +16c'\epsilon\delta^r\cr
&\le 9\mu(C_{\delta}\setminus(F\cup F'))
   +c^*\epsilon\delta\mu_{r-1}V_{\delta}
   +16c'\epsilon\delta^r\cr
&=9\mu(C_{\delta}\setminus(F\cup F'))
   +\epsilon\delta^r(c^*\beta_{r-1}+16c')\cr
}$$

\noindent (using the estimate of $\mu_1G$ in (a)).

\Quer\ Suppose, if possible, that there is a point
$(z,t)\in(F\symmdiff(H_{\gamma}\cap C_{\delta}))\setminus W$.
Since $t\notin G$, (b) tells us that

\Centerline{$\min(\mu_{r-1}F'_t,\mu_{r-1}F_t)
\le 4c'\epsilon\delta^{r-1}
\le\Bover18\mu_{r-1}V_{\delta}$.}

\noindent So
$t\in\hat G\cup\hat G'$.    Also, since $t\notin\tilde G$,
$\mu_{r-1}F_t+\mu_{r-1}F'_t\ge\Bover78\mu_{r-1}V_{\delta}$;  so
(since $t\ne\gamma$) either
$\mu_{r-1}F_t\ge\Bover34\mu_{r-1}V_{\delta}$ and
$t<\gamma$, or
$\mu_{r-1}F'_t\ge\Bover34\mu_{r-1}V_{\delta}$ and
$t>\gamma$ (by (c)).   Now

$$\eqalignno{t<\gamma
&\Longrightarrow\mu_{r-1}F_t
  \ge\Bover34\mu_{r-1}V_{\delta}\cr
&\Longrightarrow\mu_{r-1}F'_t
   \le 4c'\epsilon\delta^{r-1}\cr
&\Longrightarrow t\in\hat G'\cr
&\Longrightarrow (z,t)\notin F'\cr
\displaycause{because $(z,t)\notin F'\cap(V_{\delta}\times\hat G')$}
&\Longrightarrow (z,t)\in F\cr
\displaycause{because $(z,t)\notin C_{\delta}\setminus(F\cup F')$}
&\Longrightarrow (z,t)\in F\cap H_{\gamma},\cr}$$

\noindent which is impossible.   And similarly

$$\eqalignno{t>\gamma
&\Longrightarrow\mu_{r-1}F'
  \ge\Bover34\mu_{r-1}V_{\delta}\cr
&\Longrightarrow\mu_{r-1}F_t
   \le 4c'\epsilon\delta^{r-1}\cr
&\Longrightarrow t\in\hat G\cr
&\Longrightarrow (z,t)\notin F\cr
&\Longrightarrow (z,t)\notin F\cup H_{\gamma},\cr}$$

\noindent which is equally impossible.\ \Bang

Thus
$F\symmdiff(H_{\gamma}\cap C_{\delta})\subseteq W$
has measure at most

\Centerline{$9\mu(C_{\delta}\setminus(F\cup F'))
   +\epsilon\delta^r(c^*\beta_{r-1}+16c')$,}

\noindent as claimed.\ \Qed
}%end of proof of 474Q

\leader{474R}{Theorem} Let $E\subseteq\BbbR^r$ be a set with locally
finite perimeter, $\psi_E$
its canonical outward-normal function, and $y$
any point of its reduced boundary $\partial^{\$}E$.   Then $\psi_E(y)$
is the Federer exterior normal to $E$ at $y$.

\proof{ Write $\lambda^{\partial}_E$ for the perimeter measure of $E$, as
usual.

\medskip

{\bf (a)} To begin with (down to the end of (c-ii) below) suppose
that $y=\tbf{0}$ and that $\psi_E(y)=(0,\ldots,0,1)=v$ say.    Set

\Centerline{$c=2^{r+4}\sqrt r(1+2^{r+1})$,
\quad$c'=2^{r+3}\sqrt{r-1}(1+2^r)$,}

\Centerline{$c_1=1+\max\bigl(4\pi\beta_{r-2},
   2c(3r)^{r-1}\bigr)$,}

\noindent (counting $\beta_0$ as $1$, if $r=2$),

\Centerline{$c^*=\sqrt2(2\sqrt2)^{r-1}c_1$,
\quad$c_1^*=10+\Bover12(c^*+\Bover{16c'}{\beta_{r-1}})$.}

\noindent As in 474Q, set

\Centerline{$V_{\delta}=\{z:z\in\BbbR^{r-1}$, $\|z\|\le\delta\}$,
\quad$C_{\delta}=V_{\delta}\times[-\delta,\delta]$.
\quad$H_{\gamma}=\{x:\xi_r\le\gamma\}$}

\noindent for $\delta>0$ and $\gamma\in\Bbb R$, and
$\grad_{r-1}f=(\Pd{f}{\xi_1},\ldots,\Pd{f}{\xi_{r-1}},0)$ for
$f\in\eusm D$.

\medskip

{\bf (b)(i)} Take any $\epsilon>0$ such that

\Centerline{$\epsilon<\Bover1{\sqrt 2}$,
\quad$c^*\epsilon^3<\Bover14\beta_{r-1}$,
\quad$2^{r+1}c'\epsilon<\Bover18\beta_{r-1}$.}

\noindent Then there is a $\delta_0\in\ocint{0,1}$ such that

\Centerline{$\Bover1{\lambda^{\partial}_EB(\tbf{0},\delta)}
  \int_{B(\tbf{0},\delta)}\|\psi_E(x)-v\|\lambda^{\partial}_E(dx)
\le\epsilon^3$,}

\Centerline{$\Bover1{c_1}\delta^{r-1}
\le\lambda^{\partial}_EB(\tbf{0},\delta)
\le c_1\delta^{r-1}$}

\noindent for every $\delta\in\ocint{0,2\delta_0\sqrt2}$ (using
474N(iii) and 474N(iv) for the inequalities bounding
$\lambda^{\partial}_EB(\tbf{0},\delta)$).

\medskip

\quad{\bf (ii)} Suppose that $0<\delta\le\delta_0$.
Note first that

$$\eqalignno{\int_{C_{2\delta}}\|v-\psi_E\|d\lambda^{\partial}_E
&\le\int_{B(\tbf{0},2\delta\sqrt2)}\|v-\psi_E\|d\lambda^{\partial}_E
\le\epsilon^3\lambda^{\partial}_EB(\tbf{0},2\delta\sqrt2)\cr
&\le c_1\epsilon^3(2\delta\sqrt2)^{r-1}
=\Bover{c^*}{\sqrt2}\epsilon^3\delta^{r-1}.\cr}$$

\medskip

\quad{\bf (iii)}
$\lim_{n\to\infty}\tilde h_n*\chi E\eae\chi E$ (473Ee), so there is an
$n\ge\bover1{\delta}$ such that
$\int_{C_{\delta}}|\tilde h_n*\chi E-\chi E|d\mu
\le\bover14\epsilon\mu C_{\delta}$.   Setting

\Centerline{$f=\tilde h_n*\chi E$,
\quad$F=\{x:x\in C_{\delta}$, $f(x)\ge\bover34\}$,
\quad$F'=\{x:x\in C_{\delta}$, $f(x)\le\bover14\}$,}

\noindent we have $f\in\eusm D$ (473De once more) and

\Centerline{$\mu(C_{\delta}\Bsetminus(F\cup F'))\le\epsilon\mu C_{\delta}$,
\quad$\mu(F\symmdiff(E\cap C_{\delta}))\le\epsilon\mu C_{\delta}$.}

\medskip

\quad{\bf (iv)}

\Centerline{$\int_{C_{\delta}}
  \|\grad_{r-1}f\|+\max(\Pd{f}{\xi_r},0)d\mu
\le c^*\epsilon^3\delta^{r-1}$.}

\noindent\Prf\Quer\ Suppose, if possible, otherwise.   Note that because
$\mu C_{\delta}=2\beta_{r-1}\delta^r$,
$\lim_{\delta'\uparrow\delta}\mu C_{\delta'}=\mu C_{\delta}$, so there
is some $\delta'<\delta$ such that

\Centerline{$\int_{C_{\delta'}}
  \|\grad_{r-1}f\|+\max(\Pd{f}{\xi_r},0)d\mu
>c^*\epsilon^3\delta^{r-1}$.}

\noindent For $1\le i\le r$ and $x\in\BbbR^r$, set

$$\eqalignno{\theta_i(x)
&=\bover{\pd{f}{\xi_i}(x)}{\|\grad_{r-1}f(x)\|}
  \text{ if }i<r,\,x\in C_{\delta'}
  \text{ and }\grad_{r-1}(x)\ne 0,\cr
&=1\text{ if }i=r,\,x\in C_{\delta'}
  \text{ and }\Pd{f}{\xi_r}(x)\ge 0,\cr
&=0\text{ otherwise}.\cr}$$

\noindent Then all the $\theta_i$ are
$\mu$-integrable.   Setting $\theta=(\theta_1,\ldots,\theta_r)$,

\Centerline{$\int\varinnerprod{\theta}{\grad f}d\mu
=\int_{C_{\delta'}}
  \|\grad_{r-1}f\|+\max(\Pd{f}{\xi_r},0)d\mu
>c^*\epsilon^3\delta^{r-1}$.}

\noindent By 473Ef,
$\sequence{k}{\|\theta_i-\theta_i*\tilde h_k\|_1}\to 0$ for each $i$;
since $\grad f$ is bounded,

\Centerline{$\int
  \varinnerprod{(\tilde h_k*\theta)}{\grad f}\,d\mu
>c^*\epsilon^3\delta^{r-1}$}

\noindent for any $k$ large enough.  If we ensure also that
$\Bover1{k+1}\le\delta-\delta'$, and set $\phi=\tilde h_k*\theta$, we
shall get a function $\phi\in\eusm D$, with
$\|\phi(x)\|\le\sqrt2\chi C_{\delta}$ for every $x$ (by 473Dc and 473Dg),
such that

\Centerline{$\int\varinnerprod{\phi}{\grad f}d\mu
>c^*\epsilon^3\delta^{r-1}$.}

\noindent Moreover, referring to the definition of $*$ in 473Dd and
473Dg,

\Centerline{$\varinnerprod{(\tilde h_n*\phi)(x)}{v}
=(\tilde h_n*(\tilde h_k*\theta_r))(x)\ge 0$}

\noindent for every $x$, because $\tilde h_n$, $\tilde h_k$ and
$\theta_r$ are all non-negative.

Now

$$\eqalignno{c^*\epsilon^3\delta^{r-1}
&<\int\varinnerprod{\phi}{\grad f}d\mu
=\int\varinnerprod{\phi}{\grad(\tilde h_n*\chi E)}d\mu\cr
&=-\int\varinnerprod{(\tilde h_n*\phi)}{\psi_E}
  \,d\lambda^{\partial}_E\cr
\displaycause{474K}
&\le\int\varinnerprod{(\tilde h_n*\phi)}{(v-\psi_E)}
  \,d\lambda^{\partial}_E
\le\sqrt{2}\int_{C_{2\delta}}\|v-\psi_E\|\,d\lambda^{\partial}_E\cr
\displaycause{because $\|(\tilde h_n*\phi)(x)\|\le\sqrt2$ for every $x$,
by 473Dg again, and $(\tilde h_n*\phi)(x)=0$ if
$x\notin C_{\delta}+C_{1/(n+1)}\subseteq C_{2\delta}$}
&\le c^*\epsilon^3\delta^{r-1};\cr}$$

\noindent which is absurd.\ \Bang\Qed

\medskip

\quad{\bf (v)} By 474Q, there is a $\gamma\in\Bbb R$ such that

$$\eqalign{\mu(F\symmdiff(H_{\gamma}\cap C_{\delta}))
&\le 9\mu(C_{\delta}\setminus(F\cup F'))
   +(c^*\beta_{r-1}+16c')\epsilon\delta^r\cr
&\le 9\epsilon\mu C_{\delta}
   +\Bover1{2\beta_{r-1}}(c^*\beta_{r-1}+16c')\epsilon\mu C_{\delta}
=(c_1^*-1)\epsilon\mu C_{\delta},\cr}$$

\noindent and

\Centerline{$\mu((E\symmdiff H_{\gamma})\cap C_{\delta})
\le\mu(F\symmdiff(E\cap C_{\delta}))
    +\mu(F\symmdiff(H_{\gamma}\cap C_{\delta}))
\le c_1^*\epsilon\mu C_{\delta}$.}

\medskip

\quad{\bf (vi)} As $\epsilon$ is arbitrary, we see that

\Centerline{$\lim_{\delta\downarrow 0}\inf_{\gamma\in\Bbb R}
   \Bover1{\mu C_{\delta}}
   \mu((E\symmdiff H_{\gamma})\cap C_{\delta})=0$.}

\medskip

{\bf (c)} Again take $\epsilon\in\ocint{0,1}$.

\medskip

\quad{\bf (i)} By (b) above and 474N(i)-(ii) there is a
$\delta_1>0$ such that whenever $0<\delta\le\delta_1$ then

\Centerline{$\mu(B(\tbf{0},\delta)\cap E)
\ge\Bover1{2(3r)^r\beta_r}\mu B(\tbf{0},\delta)$,}

\Centerline{$\mu(B(\tbf{0},\delta)\setminus E)
\ge\Bover1{2(3r)^r\beta_r}\mu B(\tbf{0},\delta)$}

\noindent and there is a $\gamma\in\Bbb R$ such that

\Centerline{$\mu((E\symmdiff H_{\gamma})\cap C_{\delta})
<\min(\epsilon,\Bover{\epsilon^r}{4\beta_{r-1}(3r)^r})\mu C_{\delta}$.}

\noindent In this case, $|\gamma|\le\epsilon\delta$.
\Prf\Quer\ Suppose, if possible, that $\gamma<-\epsilon\delta$.
Then

$$\eqalign{\mu(B(\tbf{0},\epsilon\delta)\cap E)
&\le\mu(E\cap C_{\delta}\setminus H_{\gamma})\cr
&<\Bover{\epsilon^r}{4\beta_{r-1}(3r)^r}\mu C_{\delta}
=\Bover1{2\beta_r(3r)^r}\mu B(\tbf{0},\epsilon\delta)\cr}$$

\noindent which is impossible.\ \BanG\  In the same way,
\Quer\ if $\gamma>\epsilon\delta$,

$$\eqalign{\mu(B(\tbf{0},\epsilon\delta)\setminus E)
&\le\mu((C_{\delta}\setminus H_{\gamma})\setminus E)\cr
&<\Bover{\epsilon^r}{4\beta_{r-1}(3r)^r}\mu C_{\delta}
=\Bover1{2\beta_r(3r)^r}\mu B(\tbf{0},\epsilon\delta).
\text{ \Bang}\Qed\cr}$$

\medskip

\quad{\bf (ii)} It follows that

$$\eqalign{\mu((E\symmdiff H_0)\cap C_{\delta})
&\le\mu((E\symmdiff H_{\gamma})\cap C_{\delta})
   +\mu((H_{\gamma}\symmdiff H_0)\cap C_{\delta})\cr
&\le\epsilon\mu C_{\delta}+\epsilon\delta\mu_{r-1}V_{\delta}
=\Bover32\epsilon\mu C_{\delta}.\cr}$$

\noindent As $\epsilon$ is arbitrary,

\Centerline{$\lim_{\delta\downarrow 0}
  \Bover{\mu((E\symmdiff H_0)\cap C_{\delta})}{\mu C_{\delta}}
=0$,}

\noindent and

\Centerline{$\lim_{\delta\downarrow 0}
  \Bover{\mu((E\symmdiff H_0)\cap B_{\delta})}{\mu B_{\delta}}
\le\Bover{2\beta_{r-1}}{\beta_r}\lim_{\delta\downarrow 0}
  \Bover{\mu((E\symmdiff H_0)\cap C_{\delta})}{\mu C_{\delta}}
=0$.}

\medskip

{\bf (d)} Thus $v$ is a Federer exterior normal to $E$ at $\tbf{0}$ if
$\psi_E(\tbf{0})=v$.   For the general case, let $S$ be an orthogonal
matrix such that
$S\psi_E(y)=v$, and set $T(x)=S(x-y)$ for every $x$.   
The point is of course that

\Centerline{$\tbf{0}=T(y)\in T[\partial^{\$}E]=\partial^{\$}T[E]$,
\quad$v=S\psi_ET^{-1}(\tbf{0})=\psi_{T[E]}(\tbf{0})$}

\noindent (474H).   So if we set

\Centerline{$H
=\{x:\varinnerprod{(x-y)}{\psi_E(y)}\le 0\}
=\{x:\varinnerprod{T(x)}{v}\le 0\}
=T^{-1}[H_0]$,}

\noindent then

\Centerline{$\Bover{\mu((H\symmdiff E)\cap B(y,\delta))}
   {\mu B(y,\delta)}
=\Bover{\mu((H_0\symmdiff T[E])\cap B(\tbf{0},\delta))}
   {\mu B(\tbf{0},\delta)}
\to 0$}

\noindent as $\delta\downarrow 0$, and $\psi_E(y)$ is a Federer exterior
normal to $E$ at $y$, as required.
}%end of proof of 474R

\leader{474S}{Corollary} Let $E\subseteq\BbbR^r$ be a set with locally
finite perimeter, and $\lambda^{\partial}_E$ its perimeter measure.
Let $y$ be any point of the reduced boundary of $E$.   Then

\Centerline{$\lim_{\delta\downarrow 0}
\Bover{\lambda^{\partial}_EB(y,\delta)}{\beta_{r-1}\delta^{r-1}}=1$.}

\proof{{\bf (a)} Set $v=\psi_E(y)$ and
$H=\{x:\varinnerprod{(x-y)}{v}\le 0\}$, as in 474R.   Now

\Centerline{$\int_{H\cap\partial B(y,\delta)}
  \varinnerprod{v}{\Bover1{\delta}(x-y)}\nu(dx)
=-\beta_{r-1}\delta^{r-1}$}

\noindent for almost every $\delta>0$.   \Prf\ Set $\phi(x)=v$ for every
$x\in\BbbR^r$.   By 474I, $\phi$ is an outward-normal function for $H$,
so 474M tells us that, for almost every $\delta>0$,

$$\eqalignno{\int_{H\cap\partial B(y,\delta)}
  \varinnerprod{v}{\Bover1{\delta}(x-y)}\,\nu(dx)
&=\int_{H\cap B(y,\delta)}\diverg\phi\,d\mu
   -\int_{B(y,\delta)}\varinnerprod{v}{v}\,d\lambda^{\partial}_H\cr
&=-\lambda^{\partial}_HB(y,\delta)
=-\nu(B(y,\delta)\cap\partial H)\cr
\displaycause{using the identification of $\lambda^{\partial}_H$ in
474I}
&=-\beta_{r-1}\delta^{r-1}\cr}$$

\noindent(identifying $\nu$ on the hyperplane $\partial H$ with Lebesgue
measure on $\BbbR^{r-1}$, as usual).\ \Qed

\medskip

{\bf (b)} Now, given $\epsilon>0$, there is a $\delta_0>0$ such that
whenever $0<\delta\le\delta_0$ there is an $\eta$ such that
$\delta\le\eta\le\delta(1+\epsilon)$ and
$|\lambda^{\partial}_EB(y,\eta)-\beta_{r-1}\eta^{r-1}|
\le\epsilon\eta^{r-1}$.   \Prf\ Let $\zeta>0$ be such that

\Centerline{$\zeta(1+\Bover{5\pi}{r}\beta_{r-2})(1+\epsilon)^r
\le\epsilon^2$.}

\noindent By 474N(iv) and 474R and the definition of $\psi_E$, there is
a $\delta_0>0$ such that

\Centerline{$\lambda^{\partial}_EB(y,\delta)
\le 5\pi\beta_{r-2}\delta^{r-1}$,}

\Centerline{$\mu((E\symmdiff H)\cap B(y,\delta))\le\zeta\delta^r$,}

\Centerline{$\int_{B(y,\delta)}
  \|\psi_E(x)-v\|\lambda^{\partial}_E(dx)
\le\zeta\lambda^{\partial}_EB(y,\delta)$}

\noindent whenever $0<\delta\le(1+\epsilon)\delta_0$.   Take
$0<\delta\le\delta_0$.   Then, for almost every $\eta>0$, we have

\Centerline{$\int_{B(y,\eta)}
  \varinnerprod{v}{\psi_E(x)}\,\lambda^{\partial}_E(dx)
  +\int_{E\cap\partial B(y,\eta)}
    \varinnerprod{v}{\Bover1{\eta}(x-y)}\,\nu(dx)
=0$}

\noindent by 474M, applied with $\phi$ the constant function with value
$v$.   Putting this together
with (a), we see that, for almost every
$\eta\in\ocint{0,(1+\epsilon)\delta_0}$,

$$\eqalignno{|\lambda^{\partial}_EB(y,\eta)&-\beta_{r-1}\eta^{r-1}|
=|\int_{B(y,\eta)}\varinnerprod{v}{v}\,d\lambda^{\partial}_E
   -\beta_{r-1}\eta^{r-1}|\cr
&\le|\int_{B(y,\eta)}\varinnerprod{v}{(v-\psi_E)}\,d\lambda^{\partial}_E|
   +|\int_{B(y,\eta)}\varinnerprod{v}{\psi_E}\,d\lambda^{\partial}_E
       -\beta_{r-1}\eta^{r-1}|\cr
&\le\int_{B(y,\eta)}\|\psi_E-v\|d\lambda^{\partial}_E\cr
&\hskip3em
   +|\int_{B(y,\eta)}
      \varinnerprod{v}{\psi_E(x)}\,\lambda^{\partial}_E(dx)
     +\int_{H\cap\partial B(y,\eta)}
       \varinnerprod{v}{\Bover1{\eta}(x-y)}\,\nu(dx)|\cr
\displaycause{using (a) above}
&\le\zeta\lambda^{\partial}_EB(y,\eta)\cr
&\hskip3em
   +|\int_{H\cap\partial B(y,\eta)}
       \varinnerprod{v}{\Bover1{\eta}(x-y)}\,\nu(dx)
      -\int_{E\cap\partial B(y,\eta)}
       \varinnerprod{v}{\Bover1{\eta}(x-y)}\,\nu(dx)|\cr
&\le 5\pi\beta_{r-2}\zeta\eta^{r-1}
   +\nu((E\symmdiff H)\cap\partial B(y,\eta)).\cr}$$

\woddheader{474S}{0}{0}{0}{30pt}

Integrating with respect to $\eta$, we have

$$\eqalignno{\int_0^{\delta(1+\epsilon)}
  |\lambda^{\partial}_EB(y,\eta)-\beta_{r-1}\eta^{r-1}|d\eta
&\le \Bover{5\pi}{r}\beta_{r-2}\zeta\delta^r(1+\epsilon)^r
   +\mu((E\symmdiff H)\cap B(y,\delta(1+\epsilon)))\cr
\displaycause{using 265G, as usual}
&\le \Bover{5\pi}{r}\beta_{r-2}\zeta\delta^r(1+\epsilon)^r
   +\zeta\delta^r(1+\epsilon)^r
\le\epsilon^2\delta^r\cr}$$

\noindent by the choice of $\zeta$.   But this means that there must be
some $\eta\in[\delta,\delta(1+\epsilon)]$ such that

\Centerline{$|\lambda^{\partial}_EB(y,\eta)-\beta_{r-1}\eta^{r-1}|
\le\epsilon\delta^{r-1}
\le\epsilon\eta^{r-1}$. \Qed}

\medskip

{\bf (c)} Now we see that

\Centerline{$\lambda^{\partial}_EB(y,\delta)
\le\lambda^{\partial}_EB(y,\eta)
\le(\beta_{r-1}+\epsilon)\eta^{r-1}
\le(\beta_{r-1}+\epsilon)(1+\epsilon)^{r-1}\delta^{r-1}$.}

\noindent But by the same argument we have an
$\hat\eta\in[\Bover{\delta}{1+\epsilon},\delta]$ such that
$|\lambda^{\partial}_EB(y,\hat\eta)-\beta_{r-1}\hat\eta^{r-1}|
\le\epsilon\hat\eta^{r-1}$, so that

\Centerline{$\lambda^{\partial}_EB(y,\delta)
\ge\lambda^{\partial}_EB(y,\hat\eta)
\ge(\beta_{r-1}-\epsilon)\hat\eta^{r-1}
\ge(\beta_{r-1}-\epsilon)(1+\epsilon)^{1-r}\delta^{r-1}$.}

Thus, for every $\delta\in\ocint{0,\delta_0}$,

\Centerline{$(\beta_{r-1}-\epsilon)(1+\epsilon)^{1-r}\delta^{r-1}
\le\lambda^{\partial}_EB(y,\delta)
\le(\beta_{r-1}+\epsilon)(1+\epsilon)^{r-1}\delta^{r-1}$.}

\noindent As $\epsilon$ is arbitrary,

\Centerline{$\lim_{\delta\downarrow 0}
  \Bover{\lambda^{\partial}_EB(y,\delta)}{\delta^{r-1}}
=\beta_{r-1}$,}

\noindent as claimed.
}%end of proof of 474S

\leader{474T}{The Compactness Theorem} Let $\Sigma$ be the algebra of
Lebesgue measurable subsets of $\BbbR^r$, and give it the topology
$\frak T_m$ of convergence in measure defined by the pseudometrics
$\rho_H(E,F)=\mu((E\symmdiff F)\cap H)$ for measurable sets $H$ of
finite measure\cmmnt{ (cf.\ \S\S245 and 323)}.   Then

(a) $\per:\Sigma\to[0,\infty]$ is lower semi-continuous;

(b) for any $\gamma<\infty$, $\{E:E\in\Sigma$, $\per E\le\gamma\}$ is
compact.

\proof{{\bf (a)} Let $\sequencen{E_n}$ be any $\frak T_m$-convergent
sequence in $\Sigma$ with limit $E\in\Sigma$.   If
$\phi:\BbbR^r\to B(\tbf{0},1)$ is a Lipschitz function with compact
support, then $\diverg\phi$ is integrable, so
$F\mapsto\int_F\diverg\phi\,d\mu$ is truly continuous (225A), and

\Centerline{$|\int_E\diverg\phi\,d\mu|
=\lim_{n\to\infty}|\int_{E_n}\diverg\phi\,d\mu|
\le\sup_{n\in\Bbb N}\per E_n$.}

\noindent As $\phi$ is arbitrary, $\per E\le\sup_{n\in\Bbb N}\per E_n$.
This means that $\{E:\per E\le\gamma\}$ is sequentially closed,
therefore closed (4A2Ld), for any $\gamma$, and $\per$ is lower
semi-continuous.

\medskip

%\allowmorestretch{468}{
{\bf (b)} Let us say that a
`dyadic cube' is a set expressible in the form
$\prod_{1\le i\le r}\coint{2^{-n}k_i,2^{-n}(k_i+1)}$ where $n$,
$k_1,\ldots,k_r\in\Bbb Z$.   Set $\Cal A=\{E:\per E\le\gamma\}$.   
%}

\medskip

\quad{\bf (i)} For $E\in\Cal A$, $n\in\Bbb N$ and
$\epsilon\in\ocint{0,1}$ let $G(E,n,\epsilon)$ be the union of all the
dyadic cubes $D$ with side length $2^{-n}$ such that
$\epsilon\mu D\le\mu(E\cap D)\le(1-\epsilon)\mu D$.   Then
$\mu G(E,n,\epsilon)\le\Bover{c_1}{2^n\epsilon}\gamma$, where
$c_1=2^{r+5}(1+2^{r+1})(1+\sqrt r)^{r+1}$.

\Prf\ Express $G(E,n,\epsilon)$ as a disjoint union
$\bigcup_{i\in I}D_i$ where each $D_i$ is a dyadic cube of side length
$2^{-n}$ and
$\min(\mu(D_i\cap E),\mu(D_i\setminus E))\ge\epsilon\mu D_i$.   Let
$x_i$ be the centre of $D_i$ and $B_i$ the ball
$B(x_i,2^{-n-1}\sqrt r)$, so that $D_i\subseteq B_i$ and
$\mu B_i=\beta_r(\bover12\sqrt r)^r\mu D_i$.   For any $x\in\BbbR^r$,
the ball $B(x,2^{-n-1}\sqrt r)$ is included in a closed cube with side
length $2^{-n}\sqrt r$, so can contain at the very most $(1+\sqrt r)^r$
different $x_i$, because different $x_i$ differ by at least $2^{-n}$ in
some coordinate.   Turning this round,
$\sum_{i\in I}\chi B_i\le(1+\sqrt r)^r\chi(\BbbR^r)$.

Set $c=2^{r+4}\sqrt r(1+2^{r+1})$.   Then, for each $i\in I$,

$$\eqalignno{2c\lambda^{\partial}_EB_i
&\ge\min(\mu(B_i\cap E),\mu(B_i\setminus E))^{(r-1)/r}\cr
\displaycause{474Lb}
&\ge\min(\mu(D_i\cap E),\mu(D_i\setminus E))^{(r-1)/r}
\ge(\epsilon\mu D_i)^{(r-1)/r}
\ge 2^{-n(r-1)}\epsilon.\cr}$$

\noindent So

$$\eqalignno{\mu G(E,n,\epsilon)
&=2^{-nr}\#(I)
\le\Bover{2c}{2^n\epsilon}\sum_{i\in I}\lambda^{\partial}_EB_i\cr
&\le\Bover{2c(1+\sqrt r)^r}{2^n\epsilon}
  \lambda^{\partial}_E(\BbbR^r)
\le\Bover{c_1}{2^n\epsilon}\gamma.
  \text{ \Qed}\cr}$$

\medskip

\quad{\bf (ii)} Now let $\sequencen{E_n}$ be any sequence in $\Cal A$.
Then we can find a subsequence $\sequencen{E'_n}$ such that whenever
$n\in\Bbb N$, $D$ is a dyadic cube of side length $2^{-n}$ meeting
$B(\tbf{0},n)$, and $i$, $j\ge n$, then
$|\mu(D\cap E'_i)-\mu(D\cap E'_j)|\le\Bover1{(n+1)^{r+2}}\mu D$.   Now

\Centerline{$\mu((E'_n\symmdiff E'_{n+1})\cap B(\tbf{0},n))
\le\Bover{3\beta_r(n+\sqrt r)^r}{(n+1)^{r+2}}
   +2^{-n}(n+1)^{r+1}c_1\gamma$}

\noindent whenever $n\ge 1$.   \Prf\ Let $\Cal E$ be
the set of dyadic cubes of side length $2^{-n}$ meeting $B(\tbf{0},n)$;
then every member of $\Cal E$ is included in
$B(\tbf{0},n+2^{-n}\sqrt r)$, so
$\mu(\bigcup\Cal E)\le\beta_r(n+\sqrt r)^r$.   Let $\Cal E_1$ be the
collection of those dyadic cubes of side length $2^{-n}$ included in
$G(E'_n,n,\Bover1{(n+1)^{r+2}})$.   If $D\in\Cal E\setminus\Cal E_1$,
either $\mu(E'_n\cap D)\le\Bover1{(n+1)^{r+2}}\mu D$ and
$\mu(E'_{n+1}\cap D)\le\Bover2{(n+1)^{r+2}}\mu D$ and
$\mu((E'_n\symmdiff E'_{n+1})\cap D)\le\Bover3{(n+1)^{r+2}}\mu D$, or
$\mu(D\setminus E'_n)\le\Bover1{(n+1)^{r+2}}\mu D$ and
$\mu(D\setminus E'_{n+1})\le\Bover2{(n+1)^{r+2}}\mu D$ and
$\mu((E'_n\symmdiff E'_{n+1})\cap D)\le\Bover3{(n+1)^{r+2}}\mu D$.
So

$$\eqalign{\mu((E'_n\symmdiff E'_{n+1})\cap B(\tbf{0},n))
&\le\sum_{D\in\Cal E}\mu((E'_n\symmdiff E'_{n+1})\cap D)\cr
&\le\sum_{D\in\Cal E\setminus\Cal E_1}
   \mu((E'_n\symmdiff E'_{n+1})\cap D)
   +\mu(\bigcup\Cal E_1)\cr
&\le\Bover3{(n+1)^{r+2}}\mu(\bigcup\Cal E)
   +\mu G(E'_n,n,\Bover1{(n+1)^{r+2}})\cr
&\le\Bover{3\beta_r(n+\sqrt r)^r}{(n+1)^{r+2}}
   +2^{-n}(n+1)^{r+2}c_1\gamma,
\cr}$$

\noindent as claimed.\ \Qed

\medskip

\quad{\bf (iii)} This means that
$\sum_{i=0}^{\infty}\mu((E'_i\symmdiff E'_{i+1})\cap B(\tbf{0},n))$ is
finite for each $n\in\Bbb N$, so that if we set
$E=\bigcup_{i\in\Bbb N}\bigcap_{j\ge i}E'_j$, then

\Centerline{$\mu((E\symmdiff E'_i)\cap B(\tbf{0},n))
\le\sum_{j=i}^{\infty}\mu((E'_j\symmdiff E'_{j+1})\cap B(\tbf{0},n))
\to 0$}

\noindent as $i\to\infty$ for every $n\in\Bbb N$.   It follows that
$\lim_{i\to\infty}\rho_H(E,E'_i)=0$ whenever $\mu H<\infty$ (see the
proofs of 245Eb and 323Gb).   Thus we have a subsequence
$\sequence{i}{E'_i}$ of the original sequence $\sequence{i}{E_i}$ which
is convergent for the topology $\frak T_m$ of convergence in measure.
By (a), its limit belongs to $\Cal A$.   But since $\frak T_m$ is
pseudometrizable (245Eb/323Gb), this is enough to show that $A$ is
compact for $\frak T_m$ (4A2Lf).
}%end of proof of 474T

\exercises{\leader{474X}{Basic exercises (a)}
%\spheader 474Xa
Show that for any $E\subseteq\BbbR^r$ with locally
finite perimeter, its reduced boundary is a Borel set and its canonical
outward-normal function is Borel measurable.
%474G 472Xd

\sqheader 474Xb Show that if $E\subseteq\Bbb R^r$ has finite perimeter
then either $E$ or its complement has finite measure.
%474L

\spheader 474Xc(i) Show that if $E\subseteq\BbbR^r$ has locally finite
perimeter, then $\partial^{\$}E\subseteq\partial E$.
\Hint{474N(i)-(ii).}   (ii) Show that if $H\subseteq\BbbR^r$ is a
half-space, as in 474I, then $\partial^{\$}E=\partial E$.
%474N 474H

\spheader 474Xd Let $E\subseteq\BbbR^r$ be a set with locally finite
perimeter, and $y\in\partial^{\$}E$.   Show that
$\lim_{\delta\downarrow 0}
\Bover{\mu(E\cap B(y,\delta))}{\mu B(y,\delta)}=\Bover12$.
%474R

\spheader 474Xe In the proof of 474S, use 265E to show that

\Centerline{$\int_{H\cap\partial B(y,\delta)}
  \varinnerprod{v}{\Bover1{\delta}(x-y)}\nu(dx)
=-\beta_{r-1}\delta^{r-1}$}

\noindent for {\it every} $\delta>0$.
%474S


\leader{474Y}{Further exercises (a)}
%\spheader 474Ya
In 474E, explain how to interpret the pair $(\psi,\lambda^{\partial}_E)$
as a vector measure (definition: 394O\formerly{3{}93O})
$\theta_E:\Cal B\to\BbbR^r$,
where $\Cal B$ is the Borel $\sigma$-algebra of $\BbbR^r$,
in such a way that we
have $\int_E\diverg\phi\,d\mu=\int\varinnerprod{\phi}{d\theta_E}$
for Lipschitz functions $\phi$ with compact support.
}%end of exercises

\endnotes{
\Notesheader{474} When we come to the Divergence Theorem itself in the
next section, it will be nothing but a repetition of Theorem 474E with
the perimeter measure and the outward-normal function properly
identified.   The idea of the indirect approach here is to start by
defining the pair $(\psi_E,\lambda^{\partial}_E)$ as a kind of
`distributional derivative' of the set $E$.   I take the space to match
the details with the language of the rest of this treatise, but really
474E amounts to nothing more than the Riesz representation theorem;
since the functional $\phi\mapsto\int_E\diverg\phi\,d\mu$ is linear, and
we restrict attention to sets $E$ for which it is continuous in an
appropriate sense (and can therefore be extended to arbitrary continuous
functions $\phi$ with compact support), it must be representable by a
(vector) measure, as in 474Ya.   For the process to be interesting, we
have to be able to identify at least some of the appropriate sets $E$
with their perimeter measures and outward-normal functions.
Half-spaces are straightforward enough (474I), and 474R tells us what
the outward-normal functions have to be;  but for a proper description
of the family of sets with locally finite perimeter we must wait until
the next section.   I see no quick way to show from the results here
that (for instance) the union of two sets with finite perimeter again
has finite perimeter.   And I notice that I have not even shown that balls
have finite perimeters.   After 475M things should be much clearer.

I have tried to find the shortest path to the Divergence Theorem itself,
and have not attempted to give `best' results in the intermediate
material.   In particular, in the isoperimetric inequality 474La, I show
only that the measure of a set $E$ is controlled by the magnitude of its
perimeter measure.   Simple scaling arguments show that if there is any
such control, then it must be of the form
$\gamma(\mu E)^{(r-1)/r}\le\per E$;  the
identification of the best constant $\gamma$ as $r\beta_r^{1/r}$, giving
equality for balls, is the real prize, to which I shall come in 476H.
Similarly, there is a dramatic jump from the crude estimates in 474N
to the exact limits in 474Xd and 474S.
When we say that a set $E$ has a Federer exterior normal at a point $y$,
we are clearly saying that there is an `approximate' tangent plane at that
point, as measured by ordinary volume $\mu$.
474S strengthens this by saying that, when measured by the
perimeter measure, the boundary of $E$ looks like a hyperplane through $y$
with normalised $(r-1)$-dimensional measure.   In 475G below we shall
come to a partial explanation of this.

The laborious arguments of 474C and 
474H are doing no more than establish the
geometric invariance of the concepts here, which ought, one would think, to
be obvious.   The trouble is that I have given
definitions of inner product and divergence and Lebesgue measure
in terms of the standard coordinate system of $\BbbR^r$.
If these were not invariant under isometries they would be far less
interesting.   But even if we are confident that there must be a result
corresponding to 474H, I think a little thought is required to identify the
exact formulae involved in the transformation.

I leave the Compactness Theorem (474T) to the end of the section because
it is off the line I have chosen to the Divergence Theorem (though it
can be used to make the proof of 474R more transparent;
see {\smc Evans \& Gariepy 92}, 5.7.2).
I have expressed 474T in terms of the topology of convergence in measure
on the algebra of Lebesgue measurable sets.   But since the perimeter of
a measurable set $E$ is not altered if we change $E$ by a negligible
set (474F),
`perimeter' can equally well be regarded as a function defined on
the measure algebra, in which case 474T becomes a theorem about the
usual topology of the measure algebra of Lebesgue measure, as described
in \S323.
}%end of notes

\discrpage

