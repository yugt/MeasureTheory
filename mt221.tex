\frfilename{mt221.tex}
\versiondate{2.6.03}
\copyrightdate{1995}

\def\chaptername{The Fundamental Theorem of Calculus}
\def\sectionname{Vitali's theorem in $\Bbb R$}

\newsection{221}
\def\headlinesectionname{Vitali's theorem in $\eightBbb R$}

I give the first theorem of this chapter a section to itself.   It
occupies a position between measure theory and geometry (it is,
indeed, one of the fundamental results of `geometric measure
theory'), and its proof involves both the measure and the geometry of
the real line.

\leader{221A}{Vitali's theorem} Let $A$ be a bounded subset of $\Bbb R$
and $\Cal I$ a family of non-singleton closed
intervals in $\Bbb R$ such that every point of
$A$ belongs to arbitrarily short members of $\Cal I$.   Then there is a
countable set $\Cal I_0\subseteq\Cal I$ such that (i) $\Cal I_0$ is
disjoint, that is,
$I\cap I'=\emptyset$ for all distinct $I$, $I'\in\Cal I_0$ (ii)
$\mu(A\setminus\bigcup\Cal I_0)=0$, where $\mu$ is Lebesgue measure on
$\Bbb R$.

\proof{{\bf (a)} If there is a finite disjoint
set $\Cal I_0\subseteq\Cal I$
such that $A\subseteq\bigcup\Cal I_0$ (including the
possibility that $A=\Cal I_0=\emptyset$), we can stop.   So let us
suppose henceforth that there is no such $\Cal I_0$.

Let $\mu\sp*$ be Lebesgue outer
measure on $\Bbb R$.   Suppose that $|x|<M$ for every $x\in A$, and set

\Centerline{$\Cal I'=\{I:I\in\Cal I,\,I\subseteq[-M,M]\}$.}

\medskip

{\bf (b)} In this case, if $\Cal I_0$ is any finite disjoint subset of
$\Cal I'$, there is a $J\in\Cal I'$ which is disjoint from any member of
$\Cal I_0$.   \Prf\ Take $x\in A\setminus\bigcup\Cal I_0$.   Now there
is a $\delta>0$ such that
$[x-\delta,x+\delta]$ does not meet any member of $\Cal I_0$, and as
$|x|<M$ we can suppose that $[x-\delta,x+\delta]\subseteq[-M,M]$.   Let
$J$ be a member of $\Cal I$, containing $x$, and of length at most
$\delta$;  then $J\in\Cal I'$ and $J\cap\bigcup\Cal I_0=\emptyset$.\
\Qed

\medskip


{\bf (c)} We can now choose a sequence $\sequencen{\gamma_n}$ of real
numbers and a disjoint sequence $\sequencen{I_n}$ in $\Cal I'$
inductively, as follows.   Given $\langle I_j\rangle_{j<n}$ (if $n=0$,
this is the empty sequence, with no members), with $I_j\in\Cal I'$ for
each $j<n$, and $I_j\cap I_k=\emptyset$ for $j<k<n$, set

\Centerline{$\Cal J_n=\{I:I\in\Cal I',\,I\cap I_j=\emptyset$ for every
$j<n\}$.}

\noindent   We know from (b) that $\Cal J_n\ne\emptyset$.   Set

\Centerline{$\gamma_n=\sup\{\mu I:I\in\Cal J_n\}$;}

\noindent then $0<\gamma_n\le 2M$.   We may therefore choose a set
$I_n\in\Cal J_n$ such that $\mu I_n\ge{1\over 2}\gamma_n$, and this
continues the induction.

\medskip

{\bf (e)} Because the $I_n$ are disjoint Lebesgue measurable subsets of
$[-M,M]$, we have

\Centerline{$\sum_{n=0}^{\infty}\gamma_n
\le 2\sum_{n=0}^{\infty}\mu I_n\le 4M<\infty$,}

\noindent and $\lim_{n\to\infty}\gamma_n=0$.   Now define $I'_n$ to be
the closed
interval with the same midpoint as $I_n$ but five times the length, so
that it projects past each end of $I_n$ by at least $\gamma_n$.   I
claim that, for any $n$,

\Centerline{$A\subseteq\bigcup_{j<n}{I}_j\cup\bigcup_{j\ge n}I_j'$.}

\noindent\Prf\Quer\ Suppose, if possible, otherwise.   Take
any $x$ belonging to
$A\setminus(\bigcup_{j<n}{I}_j\cup\bigcup_{j\ge n}I'_j)$.   Let
$\delta>0$ be such that


\Centerline{$[x-\delta,x+\delta]
\subseteq[-M,M]\setminus\bigcup_{j<n}I_j$,}

\noindent and let $J\in\Cal I$ be such that

\Centerline{$x\in J\subseteq[x-\delta,x+\delta]$.}

\noindent   Then

\Centerline{$\mu J>0=\lim_{m\to\infty}\gamma_m$;}

\noindent  let $m$ be the least integer
greater than or equal to $n$ such that $\gamma_m<\mu J$.   In this case
$J$ cannot belong to $\Cal J_m$, so there must be some $k<m$ such that
$J\cap I_k\ne\emptyset$, because certainly $J\in\Cal I'$.   By the
choice of $\delta$, $k$ cannot be less than $n$, so $n\le k<m$, and
$\gamma_k\ge\mu J$.   In this case, the distance from $x$ to the nearest
endpoint of $I_k$ is at most $\mu J\le \gamma_k$.   But the ends of
$I'_k$ project beyond the ends of $I_k$ by at least $\gamma_k$, so $x\in
I'_k$;  which contradicts the choice of $x$. \Bang \Qed

\medskip

{\bf (f)} It follows that

\Centerline{$\mu\sp*(A\setminus\bigcup_{j<n}I_j)
\le\mu(\bigcup_{j\ge n}I'_j)\le\sum_{j=n}^{\infty}\mu I'_j
\le 5\sum_{j=n}^{\infty}\mu I_j$.}

\noindent As

\Centerline{$\sum_{j=0}^{\infty}\mu I_j\le 2M<\infty$,}

\noindent we must have

\Centerline{$\lim_{n\to\infty}\mu\sp*(A\setminus\bigcup_{j<n}I_j)=0,$}

\noindent and

\Centerline{$\mu(A\setminus\bigcup_{j\in\Bbb N}I_j)
=\mu\sp*(A\setminus\bigcup_{j\in\Bbb N}I_j)
\le\inf_{n\in\Bbb N}\mu^*(A\setminus\bigcup_{j<n}I_j)
=0$.}

Thus in this case we may set $\Cal I_0=\{I_n:n\in\Bbb N\}$ to obtain
a countable disjoint family in $\Cal I$ with
$\mu(A\setminus\bigcup\Cal I_0)=0$.
}%end of proof of 221A

\cmmnt{
\leader{221B}{Remarks (a)} I have expressed this theorem in the form 
`there is a countable set $\Cal I_0\subseteq\Cal I$ such that $\ldots$'
in an
attempt to find a concise way of expressing the three possibilities

\inset{(i) $A=\Cal I=\emptyset$, so that we must take
$\Cal I_0=\emptyset$;}

\inset{(ii) there are disjoint $I_0,\ldots,I_n\in\Cal I$ such that
$A\subseteq I_0\cup\ldots\cup I_n$, so that we can take
$\Cal I_0=\{I_0,\ldots,I_n\}$;}

\inset{(iii) there is a disjoint sequence $\sequencen{I_n}$ in $\Cal I$
such that $\mu(A\setminus\bigcup_{n\in\Bbb N}I_n)=0$, so that we can
take $\Cal I_0=\{I_n:n\in\Bbb N\}$.}

\noindent Of course many applications, like the proof of 221A itself,
will use forms of these three alternatives.

\header{221Bb}{\bf (b)} The actual theorem here, as stated, will be used
in the next section.   But quite as important as the statement of the
theorem is the principle of its proof.   The $I_n$ are chosen
`greedily', that
is, when we come to choose $I_n$ we look at the family $\Cal J_n$ of
possible intervals, given the choices $I_0,\ldots,I_{n-1}$ already made,
and choose an $I_n\in\Cal J_n$ which is `about' as big as it could be.
The supremum of the possibilities for $\mu I_n$ is $\gamma_n$;  but
since we do
not know that there is any $I\in\Cal J_n$ such that $\mu I=\gamma_n$, we
must settle for a little less.   I follow the standard formula in taking
$\mu I_n\ge\bover12\gamma_n$, but of course I could have taken
$\mu I_n\ge\bover{99}{100}\gamma_n$, or $\mu I_n\ge(1-2^{-n})\gamma_n$,
if that had helped later on.
The remarkable thing is that this works;  we can choose the $I_n$
without foresight and without considering their interrelationships (for
that matter, without examining the set $A$) beyond the minimal
requirement that $I_n\cap I_j=\emptyset$ for $j<n$, and even this
arbitrary and casual procedure yields a suitable sequence.

\header{221Bc}{\bf (c)} I have stated the theorem in terms of bounded
sets $A$ and closed intervals, which is adequate for our needs, but very
small changes in the proof suffice to deal with arbitrary
(non-singleton) intervals, and another refinement handles unbounded sets
$A$.   (See 221Ya.)
}%end of comment

\exercises{
\leader{221X}{Basic exercises (a)} Let $\alpha\in\ooint{0,1}$.
Suppose, in
part (c) of the proof of 221A, we take $\mu I_n\ge\alpha\gamma_n$ for
each $n\in\Bbb N$, rather than $\mu I_n\ge\bover12\gamma_n$.   What will
be the appropriate constant to take in place of $5$ in defining the sets
$I_j'$ of part (e)?

\leader{221Y}{Further exercises (a)}
Let $A$ be a subset of $\Bbb R$ and $\Cal I$ a family of non-singleton
intervals in $\Bbb R$ such that every point of $A$ belongs to
arbitrarily short members of $\Cal I$.   Show that there is a
countable disjoint set $\Cal I_0\subseteq\Cal I$ such that $A\setminus
\bigcup\Cal I_0$ is Lebesgue negligible.   \Hint{apply 221A to the
sets $A\cap\ooint{n,n+1}$, $\{\overline{I}:I\in\Cal
I,\,\overline{I}\subseteq\ooint{n,n+1}\}$, writing $\overline{I}$ for
the closed interval with the same endpoints as $I$.}

\header{221Yb}{\bf (b)} Let $\Cal J$ be any family of non-singleton
intervals in $\Bbb R$.   Show that $\bigcup\Cal J$ is Lebesgue
measurable.
\Hint{apply (a) to $A=\bigcup\Cal J$ and the family $\Cal I$ of
non-singleton subintervals of members of $\Cal J$.}

\header{221Yc}{\bf (c)} Let $(X,\rho)$ be a metric space, $A$ a subset
of $X$, and $\Cal I$ a family of closed balls of non-zero radius in $X$
such that every point of $A$ belongs to arbitrarily small members of
$\Cal I$.
(I say here that a set is a `closed ball of non-zero radius' if it is
expressible in the form $B(x,\delta)=\{y:\rho(y,x)\le\delta\}$ where
$x\in X$ and $\delta>0$.   Of course it is possible for such a ball to
be a singleton $\{x\}$.)   Show that either $A$ can be covered by a
finite disjoint family in $\Cal I$ or there is a disjoint sequence
$\sequencen{B(x_n,\delta_n)}$ in $\Cal I$ such that

\Centerline{$A\subseteq\bigcup_{m\le n}B(x_m,\delta_m)
  \cup\bigcup_{m>n}B(x_m,5\delta_m)$ for every $n\in\Bbb N$}

\noindent or there is a disjoint sequence $\sequencen{B(x_n,\delta_n)}$
in $\Cal I$ such that $\inf_{n\in\Bbb N}\delta_n>0$.

\spheader 221Yd Give an example of a family $\Cal I$ of open intervals
such that every point of $\Bbb R$ belongs to arbitrarily small members
of $\Cal I$, but if $\sequencen{I_n}$ is any disjoint sequence in
$\Cal I$, and for
each $n\in\Bbb N$ we write $I'_n$ for the closed interval with the same
centre as $I_n$ and ten times the length, then there is an $n$ such that
$\ooint{0,1}\not\subseteq\bigcup_{m<n}I_m\cup\bigcup_{m\ge n}I'_m$.
%mt22bits

\spheader 221Ye(i) Show that if $\Cal I$ is a {\it finite} family of
intervals in $\Bbb R$ there are $\Cal I_0$, $\Cal I_1\subseteq\Cal I$
such that $\bigcup(\Cal I_0\cup\Cal I_1)=\bigcup\Cal I$ and both
$\Cal I_0$ and $\Cal I_1$ are disjoint families.   \Hint{induce on
$\#(\Cal I)$.}   (ii) Suppose that $\Cal I$ is a family of non-singleton
intervals, of length at most $1$, covering a bounded set
$A\subseteq\Bbb R$, and that $\epsilon>0$.   Show that there is a
disjoint subfamily $\Cal I_0$ of $\Cal I$ such that
$\mu^*(A\setminus\bigcup\Cal I_0)\le\bover12\mu^*A+\epsilon$.
\Hint{replacing each member of $\Cal I$ by a slightly longer one with
rational endpoints, reduce to the case in which $\Cal I$ is countable
and thence to the case in which $\Cal I$ is finite;  now use (i).}
(iii) Use (ii) to prove Vitali's theorem.   (I learnt this argument from
J.Aldaz.)
%221A
}%end of exercises

\endnotes{
\Notesheader{221} I have headed this section `Vitali's theorem in
$\Bbb R$' because there is an $r$-dimensional version, which will appear
in Chapter 26 below.
There is an anomaly in the position of this theorem.   It is an
indispensable element of the proofs of some of the most important
theorems in measure theory;  on the other hand, the ideas involved in
its own proof
are not used elsewhere in the elementary theory.   I have therefore
myself sometimes omitted the proof when teaching this material, and
would not
reproach any student who left it to one side for the moment.   At some
stage, of course, any measure theorist must master the method, not just
for the sake of completeness, but in order to gain an intuition for
possible
variations.   I must emphasize that it is the {\it principle} of the
proof, rather than its details, which is important, because there are
innumerable
forms of `Vitali's theorem'.   (I offer some variations in the exercises
here and in \S261 below, and there are many others which are important
in more advanced work;  one will appear in \S472 in Volume 4.)   This
principle is, I suppose, that

\inset{(i) we choose the $I_n$ greedily, according to some more or less
natural criterion applicable to each $I_n$ as we come to choose it,
without attempting to look ahead;}

\inset{(ii) we prove that their sizes tend to zero, even though we
seemed to do nothing to ensure that they would (but note the shift from
$\Cal I$ to
$\Cal I'$ in part (a) of the proof of 221A, which is exactly what is
needed to make this step work);}

\inset{(iii) we check that for a suitable definition of $I_n'$,
enlarging $I_n$, we shall have
$A\subseteq\bigcup_{m<n}I_m\cup\bigcup_{m\ge n}I'_m$
for every $n$, while $\sum_{n=0}^{\infty}\mu I'_n<\infty$.}

\noindent In a way, we have to count ourselves lucky every time this
works.   The reason for studying as many variations as possible of a
technique of this kind is to learn to guess when we might be lucky.
}%end of comment

\discrpage

