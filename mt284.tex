\frfilename{mt284.tex}
\versiondate{30.8.13}
\copyrightdate{1994}

\def\rdtf{rapidly decreasing test function}
\def\Ft{Fourier transform}

\def\chaptername{Fourier analysis}
\def\sectionname{Fourier transforms II}

\newsection{284}

The basic paradox
of Fourier transforms is the fact that while for certain functions (see
283J-283K) we have $(\varhatf)\varspcheck=f$, `ordinary' integrable
functions $f$ (for instance, the indicator functions of non-trivial
intervals) give rise to non-integrable Fourier transforms $\varhatf$
for which there is no direct definition available for
$\varhatf\varcheck{\phantom{f}}$, making it a puzzle to decide in what
sense the formula
$f=\varhatf\varcheck{\phantom{f}}$ might be true.   What now seems by
far the most
natural resolution of the problem lies in declaring the Fourier
transform to be an operation on {\it distributions} rather than on {\it
functions}.   I shall not attempt to describe this theory properly
(almost any book on `Distributions' will cover the ground better than
I can possibly do here), but will try to convey the fundamental ideas,
so far as they are relevant to the questions dealt with here, in
language which will make the transition to a fuller treatment
straightforward.   At the same time, these methods make it easy to prove
strong versions of the `classical' theorems concerning Fourier
transforms.

\leader{284A}{Test functions:  Definition} Throughout this section, a
{\bf rapidly decreasing test function} or {\bf Schwartz function}
will be a function  $h:\Bbb R\to\Bbb C$ such
that $h$ is {\bf smooth}, that is, differentiable everywhere any finite
number of times, and moreover

\Centerline{$\sup_{x\in\Bbb R}|x|^k|h^{(m)}(x)|<\infty$}

\noindent for all $k$, $m\in\Bbb N$, writing $h^{(m)}$ for the $m$th
derivative of $h$.

\leader{284B}{}\cmmnt{ The following elementary facts will be useful.

\medskip

\noindent}{\bf Lemma} (a) If $g$ and $h$ are rapidly decreasing test
functions, so are $g+h$ and $ch$, for any $c\in\Bbb C$.

(b) If $h$ is a rapidly decreasing test function and $y\in\Bbb R$, then
$x\mapsto h(y-x)$ is a rapidly decreasing test function.

(c) If $h$ is any rapidly decreasing test function, then $h$ and $h^2$
are integrable.

(d) If $h$ is a \rdtf, so is its derivative $h'$.


(e) If $h$ is a \rdtf, so is the function $x\mapsto xh(x)$.

(f) For any $\epsilon>0$, the function
$x\mapsto e^{-\epsilon x^2}$ is a rapidly decreasing test function.

\proof{{\bf (a)} is trivial.

\medskip

{\bf (b)} Write $g(x)=h(y-x)$ for $x\in\Bbb R$.   Then
$g^{(m)}(x)=(-1)^mh^{(m)}(y-x)$ for every $m$, so $g$ is smooth.   For
any $k\in\Bbb N$,

\Centerline{$|x|^k\le 2^k(|y|^k+|y-x|^k)$}

\noindent for every $x$, so

$$\eqalign{\sup_{x\in\Bbb R}|x|^k|g^{(m)}(x)|
&=\sup_{x\in\Bbb R}|x|^k|h^{(m)}(y-x)|\cr
&\le 2^k|y|^k\sup_{x\in\Bbb R}|h^{(m)}(y-x)|
   +2^k\sup_{x\in\Bbb R}|y-x|^k|h^{(m)}(y-x)|\cr
&=2^k|y|^k\sup_{x\in\Bbb R}|h^{(m)}(x)|
   +2^k\sup_{x\in\Bbb R}|x|^k|h^{(m)}(x)|
<\infty.\cr}$$

\medskip

{\bf (c)} Because

\Centerline{$M=\sup_{x\in\Bbb R}|h(x)|+x^2|h(x)|$}

\noindent is finite, we have

\Centerline{$\int|h|\le\int\Bover{M}{1+x^2}dx<\infty$.}

\noindent Of course we now have $|h^2|\le M|h|$, so $h^2$ also is
integrable.

\medskip

{\bf (d)} This is immediate from the definition, as every derivative of
$h'$ is a derivative of $h$.

\medskip

{\bf (e)} Setting $g(x)=xh(x)$, $g^{(m)}(x)=xh^{(m)}(x)+mh^{(m-1)}(x)$
for $m\ge 1$, so

\Centerline{$\sup_{x\in\Bbb R}|x^kg^{(m)}(x)|
\le\sup_{x\in\Bbb R}|x^{k+1}h^{(m)}(x)|
   +m\sup_{x\in\Bbb R}|x^kh^{(m-1)}(x)|$}

\noindent is finite, for all $k\in\Bbb N$, $m\ge 1$.

\medskip

{\bf (f)} If $h(x) = e^{-\epsilon x^2}$, then for each $m\in\Bbb N$ we
have $h^{(m)}(x)=p_m(x)h(x)$, where $p_0(x)=1$ and
$p_{m+1}(x)=p'_m(x)-2\epsilon xp_m(x)$, so that $p_m$ is a polynomial.
Because $e^{\epsilon x^2}\ge\epsilon^{k+1}x^{2k+2}/(k+1)!$ for all $x$,
$k\ge 0$,

\Centerline{$\lim_{|x|\to\infty}|x|^kh(x)
=\lim_{x\to\infty}x^k/e^{\epsilon x^2}=0$}

\noindent for every $k$, and $\lim_{|x|\to\infty}p(x)h(x)=0$ for every
polynomial $p$;  consequently

\Centerline{$\lim_{|x|\to\infty}x^kh^{(m)}(x)
=\lim_{|x|\to\infty}x^kp_m(x)h(x)=0$}

\noindent for all $k$, $m$, and $h$ is a rapidly decreasing test
function.
}%end of proof of 284B


\leader{284C}{Proposition} Let $h:\Bbb R\to\Bbb C$ be a rapidly
decreasing test function.   Then $\varhat{h}:\Bbb R\to\Bbb C$ and
$\varcheck{h}:\Bbb R\to\Bbb C$ are rapidly
decreasing test functions, and $\varhat{h}\varcheck{\phantom{h}} =
\varcheck{h}\varhat{\phantom{h}} = h$.

\proof{{\bf (a)} Let $k$, $m\in\Bbb N$.   Then
$\sup_{x\in\Bbb R}(|x|^m+|x|^{m+2})|h^{(k)}(x)|<\infty$ and
$\int_{-\infty}^{\infty}|x^mh^{(k)}(x)|dx
\ifdim\pagewidth=390pt\penalty-100\fi
<\infty$.   We may therefore
use 283Ch-283Ci to see that
$y\mapsto i^{k+m}y^k\varhat{h}\vthsp^{(m)}(y)$ is the Fourier
transform of $x\mapsto x^mh^{(k)}(x)$, and therefore that
$\lim_{|y|\to\infty}y^k\varhat{h}\vthsp^{(m)}(y)=0$, by 283Cg, so that
(because $\varhat{h}\vthsp^{(m)}$ is continuous)
$\sup_{y\in\Bbb R}|y^k\varhat{h}\vthsp^{(m)}(y)|$ is finite.   As $k$
and $m$ are arbitrary, $\varhat{h}$ is a rapidly decreasing test
function.

\medskip

{\bf (b)} Since $\varcheck{h}(y)=\varhat{h}(-y)$ for every $y$, it
follows at once that $\varcheck{h}$ is a rapidly decreasing test
function.

\medskip

{\bf (c)} By 283J, it follows from (a) and (b) that
$\varhat{h}\varcheck{\phantom{h}}=\varcheck{h}\varhat{\phantom{h}}=h$.
}%end of proof of 284C

\leader{284D}{Definition} I will use the phrase {\bf tempered function}
on $\Bbb R$ to mean
a measurable complex-valued function $f$, defined almost everywhere in
$\Bbb R$, such that

\Centerline{$\int_{-\infty}^{\infty}
  \Bover1{1+|x|^k}|f(x)|dx<\infty$}

\noindent for some $k\in\Bbb N$.

\leader{284E}{}\cmmnt{ As in 284B I spell out some elementary facts.

\medskip

\noindent}{\bf Lemma} (a) If $f$ and $g$ are tempered functions, so are
$|f|$, $f+g$ and $cf$, for any $c\in\Bbb C$.

(b) If $f$ is a tempered function then it is integrable over any bounded
interval.

(c) If $f$ is a tempered function and $x\in\Bbb R$, then $t\mapsto
f(x+t)$ and $t\mapsto f(x-t)$ are both tempered functions.

\proof{{\bf (a)} is elementary;  if

\Centerline{$\int_{-\infty}^{\infty}
  \Bover1{1+|x|^j}f(x)dx<\infty$,
\quad$\int_{-\infty}^{\infty}\Bover1{1+|x|^k}g(x)dx<\infty$,}

\noindent then

\Centerline{$\int_{-\infty}^{\infty}\Bover1{1+|x|^{j+k}}
|(f+g)(x)|dx<\infty$}

\noindent because

\Centerline{$1+|x|^{j+k}\ge\max(1,|x|^{j+k})\ge\max(1,|x|^j,|x|^k)
\ge\Bover12\max(1+|x|^j,1+|x|^k)$}

\noindent for all $x$.

\medskip

{\bf (b)} If

\Centerline{$\int_{-\infty}^{\infty}\Bover1{1+|x|^k}|f(x)|dx
=M<\infty$,}

\noindent then for any $a\le b$

\Centerline{$\int_a^b|f|\le M(1+|a|^k+|b|^k)(b-a)<\infty$.}

\medskip

{\bf (c)} The idea is the same as in 284Bb.   If $k\in\Bbb N$ is such
that

\Centerline{$\int_{-\infty}^{\infty}\Bover1{1+|t|^k}|f(t)|dt
=M<\infty$,}

\noindent then we have

\Centerline{$1+|x+t|^k
\le 2^k(1+|x|^k)(1+|t|^k)$}

\noindent so that

\Centerline{$\Bover1{1+|t|^k}
\le 2^k(1+|x|^k)\Bover1{1+|x+t|^k}$}

\noindent for every $t$, and

\Centerline{$\int_{-\infty}^{\infty}\Bover{|f(x+t)|}{1+|t|^k}dt
\le 2^k(1+|x|^k)\int_{-\infty}^{\infty}
   \Bover{|f(x+t)|}{1+|x+t|^k}dt
\le 2^k(1+|x|^k)M<\infty$.}

\noindent Similarly,

\Centerline{$\int_{-\infty}^{\infty}\Bover{|f(x-t)|}{1+|t|^k}dt
\le 2^k(1+|x|^k)M<\infty$.}
}%end of proof of 284E

\leader{284F}{}\cmmnt{ Linking the two concepts, we have the
following.

\medskip

\noindent}{\bf Lemma} Let $f$ be a tempered function on $\Bbb R$ and $h$
a rapidly decreasing test function.   Then $f\times h$ is integrable.

\proof{ Of course $f\times h$ is measurable.   Let
$k\in\Bbb N$ be such that
$\int_{-\infty}^{\infty}
\bover1{1+|x|^k}|f(x)|dx<\infty$.
There is an $M$ such that $(1+|x|^k)|h(x)|\le M$ for every
$x\in\Bbb R$, so that

\Centerline{$\int_{-\infty}^{\infty}|f\times h|
\le M\int_{-\infty}^{\infty}\Bover1{1+|x|^k}|f(x)|dx<\infty$.}
}%end of proof of 284F

\leader{284G}{Lemma} Suppose that $f_1$ and $f_2$ are tempered functions
and that $\int f_1\times h=\int f_2\times h$ for every rapidly
decreasing test function $h$.   Then $f_1\eae f_2$.

\proof{{\bf (a)} Set $g=f_1-f_2$;  then $\int g\times h=0$ for every
rapidly decreasing test function $h$.   Of course $g$ is a tempered
function, so is integrable over any bounded interval.   By 222D, it will
be enough if I can show that $\int_a^bg=0$ whenever $a<b$, since then we
shall have $g=0$ a.e.\ on every bounded interval and $f_1\eae f_2$.

\medskip

{\bf (b)} Consider the function $\tilde\phi(x)=e^{-1/x}$ for $x>0$.   Then
$\tilde\phi$ is differentiable arbitrarily often everywhere in
$\ooint{0,\infty}$, $0<\tilde\phi(x)<1$ for every $x>0$, and
$\lim_{x\to\infty}\tilde\phi(x)=1$.   Moreover, writing $\tilde\phi^{(m)}$ for
the $m$th derivative of $\tilde\phi$,

\Centerline{$\lim_{x\downarrow 0}\tilde\phi^{(m)}(x)
=\lim_{x\downarrow 0}\Bover1x\tilde\phi^{(m)}(x)=0$}

\noindent for every $m\in\Bbb N$.   \Prf\ (Compare 284Bf.)   We have
$\tilde\phi^{(m)}(x)=p_m(\bover1x)\tilde\phi(x)$, where $p_0(t)=1$ and
$p_{m+1}(t)=t^2(p_m(t)-p_m'(t))$, so that $p_m$ is a polynomial for each
$m\in\Bbb N$.   Now for any $k\in\Bbb N$,

\Centerline{$0\le\limsup_{t\to\infty}t^ke^{-t}
\le\lim_{t\to\infty}\Bover{(k+1)!t^k}{t^{k+1}}=0$,}

\noindent so

\Centerline{$\lim_{x\downarrow 0}\tilde\phi^{(m)}(x)
=\lim_{t\to\infty}p_m(t)e^{-t}=0$,}

\Centerline{$\lim_{x\downarrow 0}\Bover1x\tilde\phi^{(m)}(x)
=\lim_{t\to\infty}tp_m(t)e^{-t}=0$. \Qed}

\medskip

{\bf (c)} Consequently, setting $\phi(x)=0$ for $x\le 0$, $e^{-1/x}$ for
$x>0$, $\phi$ is smooth, with $m$th derivative

\Centerline{$\phi^{(m)}(x)=0$ for $x\le 0$,
\quad$\phi^{(m)}(x)=\tilde\phi^{(m)}(x)$ for $x>0$.}

\noindent (The proof is an easy induction on $m$.)   Also
$0\le\phi(x)\le 1$ for every $x\in\Bbb R$, and
$\lim_{x\to\infty}\phi(x)=1$.

\medskip

{\bf (d)} Now take any $a<b$, and for $n\in\Bbb N$ set

\Centerline{$\phi_n(x)=\phi(n(x-a))\phi(n(b-x))$.}

\noindent Then $\phi_n$ will be smooth and  $\phi_n(x)=0$ if
$x\notin\ooint{a,b}$, so surely $\phi_n$ is a rapidly decreasing test
function, and

\Centerline{$\int_{-\infty}^{\infty}g\times\phi_n=0$.}

\noindent Next,  $0\le\phi_n(x)\le 1$ for every $x$, $n$, and if $a<x<b$
then $\lim_{n\to\infty}\phi_n(x)=1$.   So

\Centerline{$\int_a^bg=\int g\times\chi(\ooint{a,b})
=\int g\times(\lim_{n\to\infty}\phi_n)
=\lim_{n\to\infty}\int g\times\phi_n=0$,}

\noindent using Lebesgue's Dominated Convergence Theorem.   As $a$ and
$b$ are arbitrary, $g=0$ a.e., as required.
}%end of proof of 284G

\leader{284H}{Definition} Let $f$ and $g$ be tempered
functions\cmmnt{ in
the sense of 284D}.   Then I will say that $g$ {\bf represents the
Fourier transform of} $f$ if

\Centerline{$\int_{-\infty}^{\infty}g\times h
=\int_{-\infty}^{\infty}f\times\varhat{h}$}

\noindent for every rapidly decreasing test function $h$.

\leader{284I}{Remarks (a)}\cmmnt{ As usual, when shifting definitions
in this way, we have some checking to do.}
If $f$ is an integrable complex-valued function on $\Bbb R$ and $\varhatf$
is its Fourier transform, then\cmmnt{ surely $\varhatf$ is a tempered
function,
being a bounded continuous function;  and if $h$ is any rapidly
decreasing test function, then
$\int\varhatf\times h=\int f\times\varhat{h}$
by 283O.   Thus} $\varhatf$ `represents the Fourier transform of
$f$'\cmmnt{ in
the sense of 284H above}.

\header{284Ib}{\bf (b)} Note also that\cmmnt{ 284G assures us that} if
$g_1$, $g_2$ are two tempered functions both representing the Fourier
transform of $f$, then $g_1\eae g_2$\cmmnt{, since we must have

\Centerline{$\int g_1\times h=\int f\times\varhat{h}=\int g_2\times h$}

\noindent for every rapidly decreasing test function $h$.}

\spheader 284Ic It is I suppose obvious that if $f_1$, $f_2$, $g_1$ and
$g_2$ are tempered functions and $g_i$ represents the Fourier transform of
$f_i$ for both $i$, then $cg_1+g_2$ represents the Fourier transform of
$cf_1+f_2$ for every $c\in\Bbb C$.

\cmmnt{\spheader 284Id
Of course the value of this indirect approach is that we can assign
Fourier transforms, in a sense, to many more functions.   But we must
note at once that if $g$ `represents the Fourier transform of $f$'
then so will any function equal almost everywhere to $g$;  we can no
longer expect to be able to speak of `the' Fourier transform of $f$
as a function.   We could say that `the' Fourier transform of $f$ is
a functional
$\phi$ on the space of rapidly decreasing test functions, defined by
setting $\phi(h)=\int f\times\varhat{h}$;   alternatively, we could say
that `the' Fourier transform of $f$ is a member of $L_{\Bbb C}^0$, the
space of equivalence classes of almost-everywhere-defined measurable
functions (241J).
}%end of comment

\spheader 284Ie\cmmnt{ It is now natural to say that} {\bf $g$
represents the inverse Fourier transform of $f$}\cmmnt{ just when $f$
represents the Fourier transform of $g$;  that is,}  when $\int f\times
h=\int g\times\varhat{h}$ for every rapidly decreasing test function
$h$.   \cmmnt{Because
$\varhat{h}\varcheck{\phantom{h}}=\varcheck{h}\varhat{\phantom{h}}=h$
for every such $h$ (284C), this is the same thing as saying that
$\int g\times h=\int f\times\varcheck{h}$ for every rapidly decreasing test
function $h$, which is the other natural expression of what it might
mean to say that `$g$ represents the inverse Fourier transform of
$f$'.}

\spheader 284If If $f$, $g$ are tempered functions and we write
$\Reverse{g}(x)=g(-x)$ whenever this is defined, then\cmmnt{ $\Reverse{g}$ will also
be a tempered function, and we shall always have

\Centerline{$\int\Reverse{g}\times\varhat{h}
=\int g(-x)\varhat{h}(x)dx
=\int g(x)\varhat{h}(-x)dx
=\int g\times\varcheck{h}$,}

\noindent so that}

\qquad $g$ represents the Fourier transform of $f$

\cmmnt{\qquad\qquad$\iff\,\int g\times h=\int f\times\varhat{h}$ for
every test function $h$

\qquad\qquad$\iff\,\int g\times\varcheck{h}=\int
f\times\varcheck{h}\varhat{\phantom{h}}$ for every $h$

\qquad\qquad$\iff\,\int\Reverse{g}\times\varhat{h}=\int f\times h$ for every
$h$
}%end of comment

\qquad\qquad$\iff\,\Reverse{g}$ represents the inverse Fourier transform of
$f$.

\cmmnt{\noindent Combining this with (d), we get

\qquad $g$ represents the \Ft\ of $f$

%\qquad\qquad$\iff\,(\Reverse{f})\ssplrarrow=f$
%represents the inverse \Ft\ of $g$

\qquad\qquad$\iff\,\Reverse{\Reverse{f}}=f$
represents the inverse \Ft\ of $g$

\qquad\qquad$\iff\,\Reverse{f}$ represents the \Ft\ of $g$.
}%end of comment

\cmmnt{\spheader 284Ig Yet again, we ought to be conscious that a check
is called for:  if
$f$ is integrable and $\varcheckf$ is its inverse Fourier transform as
defined in 283Ab, then

\Centerline{$\int\varcheckf\times\varhat{h}
=\int f\times\varhat{h}\varcheck{\phantom{h}}
=\int f\times h$}

\noindent for every rapidly decreasing test function $h$, so
$\varcheckf$ `represents the inverse Fourier transform of $f$' in the
sense given here.
}%end of comment

\leader{284J}{Lemma} Let $f$ be any tempered function and $h$ a rapidly
decreasing test function.   Then $f*h$, defined by the formula

\Centerline{$(f*h)(y)=\int_{-\infty}^{\infty}f(t)h(y-t)dt$,}

\noindent is defined everywhere.

\proof{ Take any $y\in\Bbb R$.   By 284Bb,
$t\mapsto h(y-t)$ is a rapidly decreasing test function, so the integral
is always defined in $\Bbb C$, by 284F.
}%end of proof of 284J

\leader{284K}{Proposition} Let $f$ and $g$ be tempered functions such
that $g$ represents the Fourier transform of $f$, and $h$ a rapidly
decreasing test function.

(a) The Fourier transform of the integrable function $f\times h$ is
$\bover1{\sqrt{2\pi}}g*\varhat{h}$.

(b) The Fourier transform of the continuous function $f*h$ is
represented by the
product $\sqrt{2\pi}g\times\varhat{h}$.

\proof{{\bf (a)} Of course $f\times h$ is integrable, by 284F,
while $g*\varhat{h}$ is defined everywhere, by 284C and 284J.

Fix $y\in\Bbb R$.   Set $h_1(x)=\varhat{h}(y-x)$ for $x\in\Bbb R$;
then $h_1$ is a rapidly decreasing test function because $\varhat{h}$
is (284Bb).   Now

$$\eqalign{\varhat{h}_1(t)
&=\Bover1{\sqrt{2\pi}}\int_{-\infty}^{\infty}
  e^{-itx}\varhat{h}(y-x)dx
=\Bover1{\sqrt{2\pi}}\int_{-\infty}^{\infty}
  e^{-it(y-x)}\varhat{h}(x)dx\cr
&=\Bover{1}{\sqrt{2\pi}}e^{-ity}
   \int_{-\infty}^{\infty}e^{itx}\varhat{h}(x)dx
=e^{-ity}\varhat{h}\varcheck{\phantom{h}}(t)
=e^{-ity}h(t),\cr}$$

\noindent using 284C.   Accordingly

$$\eqalignno{(f\times h)\varsphat(y)
&=\Bover1{\sqrt{2\pi}}\int_{-\infty}^{\infty}e^{-ity}f(t)h(t)dt\cr
&=\Bover1{\sqrt{2\pi}}
  \int_{-\infty}^{\infty}f(t)\varhat{h}_1(t)dt
=\Bover1{\sqrt{2\pi}}\int_{-\infty}^{\infty}g(t)h_1(t)dt\cr
\noalign{\noindent (because $g$ represents the Fourier transform of
$f$)}
&=\Bover1{\sqrt{2\pi}}\int_{-\infty}^{\infty}g(t)\varhat{h}(y-t)dt
=\Bover1{\sqrt{2\pi}}(g*\varhat{h})(y).\cr}$$

\noindent As $y$ is arbitrary, $\Bover1{\sqrt{2\pi}}g*\varhat{h}$ is the
Fourier transform of $f\times h$.


\medskip

{\bf (b)} Write $f_1$ for the Fourier transform of $g\times\varhat{h}$,
$\Reverse{f}(x)=f(-x)$ when this is defined, and $\Reverse{h}(x)=h(-x)$ for every $x$,
so that $\Reverse{f}$ represents the Fourier transform of $g$, by 284If, and
$\Reverse{h}$ is the Fourier transform of $\varhat{h}$.   By (a), we have
$f_1=\Bover1{\sqrt{2\pi}}\Reverse{f}*\Reverse{h}$.
This means that the inverse Fourier transform of
$\sqrt{2\pi}g\times\varhat{h}$ must be
$\sqrt{2\pi}\Reverse{f}_1=(\Reverse{f}*\Reverse{h})\ssplrarrow$;
and as

$$\eqalign{(\Reverse{f}*\Reverse{h})\ssplrarrow(y)
&=(\Reverse{f}*\Reverse{h})(-y)
=\int_{-\infty}^{\infty}\Reverse{f}(t)\Reverse{h}(-y-t)dt\cr
&=\int_{-\infty}^{\infty}f(-t)h(y+t)dt
=\int_{-\infty}^{\infty}f(t)h(y-t)dt
=(f*h)(y),\cr}$$

\noindent the inverse Fourier transform of
$\sqrt{2\pi}g\times\varhat{h}$ is $f*h$ (which is therefore continuous),
and $\sqrt{2\pi}g\times\varhat{h}$ must represent the Fourier transform
of $f*h$.
}%end of proof of 284K

\cmmnt{\medskip

\noindent{\bf Remark} Compare 283M.   It is typical of the theory of
Fourier transforms that we have formulae valid in a wide variety of
contexts, each requiring a different interpretation and a different
proof.
}%end of comment

\leader{284L}{}\cmmnt{ We are now ready for a result corresponding to
282H.   I use a different method, or at least a different arrangement of
the ideas, through the following fact, which is important in other ways.

\medskip

\noindent}{\bf Proposition}  Let $f$ be a tempered function.
Writing $\psi_{\sigma}(x) =\bover1{\sigma\sqrt{2\pi}}e^{-x^2/2\sigma^2}$
for $x\in\Bbb R$ and $\sigma>0$, then

\Centerline{$\lim_{\sigma\downarrow 0}(f*\psi_{\sigma})(x)=c$}

\noindent whenever $x\in\Bbb R$ and $c\in\Bbb C$ are such that

\Centerline{$\lim_{\delta\downarrow 0}
\Bover1{\delta}\int_{0}^{\delta}|f(x+t)+f(x-t)-2c|dt=0$.}

\proof{{\bf (a)} By 284Bf, every $\psi_{\sigma}$ is a rapidly
decreasing test function, so that $f*\psi_{\sigma}$ is defined
everywhere, by 284J.   We need to know that
$\int_{-\infty}^{\infty}\psi_{\sigma}=1$;  this is because (substituting
$u=x/\sigma$)

\Centerline{$\int_{-\infty}^{\infty}\psi_{\sigma}
=\Bover1{\sqrt{2\pi}}\int_{-\infty}^{\infty}e^{-u^2/2}du=1$,}

\noindent by 263G.   The argument now follows the lines of 282H.   Set

\Centerline{$\phi(t)=|f(x+t)+f(x-t)-2c|$}

\noindent when this is defined, which is almost everywhere, and
$\Phi(t)=\int_0^t\phi$, defined for all $t\ge 0$ because $f$ is
integrable over every bounded interval (284Eb).   We have

$$\eqalignno{|(f*\psi_{\sigma})(x)-c|
&=|\int_{-\infty}^{\infty}f(x-t)\psi_{\sigma}(t)dt
-c\int_{-\infty}^{\infty}\psi_{\sigma}(t)dt|\cr
&=|\int_{-\infty}^0(f(x-t)-c)\psi_{\sigma}(t)dt
+\int_{0}^{\infty}(f(x-t)-c)\psi_{\sigma}(t)dt|\cr
&=|\int_{0}^{\infty}(f(x+t)-c)\psi_{\sigma}(t)dt
+\int_{0}^{\infty}(f(x-t)-c)\psi_{\sigma}(t)dt|\cr
\noalign{\noindent (because $\psi_{\sigma}$ is an even function)}
&=|\int_{0}^{\infty}(f(x+t)+f(x-t)-2c)\psi_{\sigma}(t)dt|\cr
&\le\int_{0}^{\infty}|f(x+t)+f(x-t)-2c|\psi_{\sigma}(t)dt
=\int_0^{\infty}\phi\times\psi_{\sigma}.\cr}$$

\medskip

{\bf (b)} I should explain why this last integral is finite.   Because
$f$ is a tempered function, so are the functions $t\mapsto f(x+t)$,
$t\mapsto f(x-t)$ (284Ec);  of course constant functions are tempered,
so $t\mapsto\phi(t)=|f(x+t)+f(x-t)-2c|$ is tempered, and because
$\psi_{\sigma}$ is a rapidly decreasing test function we may
apply 284F
to see that the product is integrable.

\medskip

{\bf (c)} Let $\epsilon>0$.   By hypothesis, $\lim_{t\downarrow
0}\Phi(t)/t=0$;
let $\delta>0$ be such that $\Phi(t)\le\epsilon t$ for
every $t\in[0,\delta]$.   Take any $\sigma\in\ocint{0,\delta}$.   I
break the
integral $\int_0^{\infty}\phi\times\psi_{\sigma}$ up into three parts.

\medskip

\quad{\bf (i)} For the integral from $0$ to $\sigma$, we have

$$\int_0^{\sigma}\phi\times\psi_{\sigma}
\le\int_0^{\sigma}\Bover{1}{\sigma\sqrt{2\pi}}\phi
=\Bover1{\sigma\sqrt{2\pi}}\Phi(\sigma)
\le\Bover{\epsilon\sigma}{\sigma\sqrt{2\pi}}\le\epsilon,$$

\noindent because $\psi_{\sigma}(t)\le\bover1{\sigma\sqrt{2\pi}}$ for
every $t$.

\medskip

\quad{\bf (ii)} For the integral from $\sigma$ to $\delta$, we have

$$\eqalignno{\int_{\sigma}^{\delta}\phi\times\psi_{\sigma}
&\le\Bover1{\sigma\sqrt{2\pi}}\int_{\sigma}^{\delta}
  \phi(t)\Bover{2\sigma^2}{t^2}dt\cr
\noalign{\noindent (because $e^{-t^2/2\sigma^2}
=1/e^{t^2/2\sigma^2}\le
1/(t^2/2\sigma^2)=2\sigma^2/t^2$ for every $t\ne 0$)}
&=\sigma\sqrt{\Bover2{\pi}}\int_{\sigma}^{\delta}
  \Bover{\phi(t)}{t^2}dt
=\sigma\sqrt{\Bover2{\pi}}\bigl(\Bover{\Phi(\delta)}{\delta^2}
   -\Bover{\Phi(\sigma)}{\sigma^2}
   +\int_{\sigma}^{\delta}\Bover{2\Phi(t)}{t^3}dt\bigr)\cr
\noalign{\noindent (integrating by parts -- see 225F)}
&\le\sigma\bigl(\Bover{\epsilon}{\delta}
   +\int_{\sigma}^{\delta}\Bover{2\epsilon}{t^2}dt\bigr)\cr
\noalign{\noindent (because $\Phi(t)\le\epsilon t$ for $0\le
t\le\delta$ and $\sqrt{2/\pi}\le 1$)}
&\le\sigma\bigl(\Bover{\epsilon}{\delta}
   +\Bover{2\epsilon}{\sigma}\bigr)
\le 3\epsilon.\cr}$$

\medskip

\quad{\bf (iii)} For the integral from $\delta$ to $\infty$, we have

$$\eqalign{\int_{\delta}^{\infty}\phi\times\psi_{\sigma}
&=\bover1{\sqrt{2\pi}}\int_{\delta}^{\infty}
  \phi(t)\bover{e^{-t^2/2\sigma^2}}{\sigma}dt.\cr}$$

\noindent Now for any $t\ge\delta$,

\Centerline{$\sigma\mapsto\Bover1{\sigma}
e^{-t^2/2\sigma^2}:\ocint{0,\delta}\to\Bbb R$}

\noindent is monotonically increasing, because its derivative

\Centerline{$\Bover{d}{d\sigma}\Bover1{\sigma}e^{-t^2/2\sigma^2}
=\Bover1{\sigma^2}\bigl(\Bover{t^2}{\sigma^2}-1\bigr)
e^{-t^2/2\sigma^2}$}

\noindent is positive, and

\Centerline{$\lim_{\sigma\downarrow 0}\Bover1{\sigma}
e^{-t^2/2\sigma^2}=
\lim_{a\to\infty}ae^{-a^2t^2/2}=0$.}

\noindent So we may apply Lebesgue's Dominated Convergence Theorem to
see that

$$\lim_{n\to\infty}\int_{\delta}^{\infty}
   \phi(t)\bover{e^{-t^2/2\sigma_n^2}}{\sigma_n}dt=0$$

\noindent whenever $\sequencen{\sigma_n}$ is a sequence in
$\ocint{0,\delta}$ converging to $0$, so that

$$\lim_{\sigma\downarrow 0}\int_{\delta}^{\infty}
   \phi(t)\bover{e^{-t^2/2\sigma^2}}{\sigma}dt=0.$$

\noindent There must therefore
be a $\sigma_0\in\ocint{0,\delta}$ such that

\Centerline{$\int_{\delta}^{\infty}\phi\times\psi_{\sigma}
\le\epsilon$}

\noindent for every $\sigma\le \sigma_0$.

\medskip

\quad{\bf (iv)} Putting these together, we see that

\Centerline{$|(f*\psi_{\sigma})(x)-c|
\le\int_0^{\infty}\phi\times\psi_{\sigma}
\le\epsilon+3\epsilon+\epsilon=5\epsilon$}

\noindent whenever $0<\sigma\le\sigma_0$.   As $\epsilon$ is arbitrary,
$\lim_{\sigma\downarrow 0}(f*\psi_{\sigma})(x)=c$, as claimed.
}%end of proof of 284L

\vleader{60pt}{284M}{Theorem} Let $f$ and $g$ be tempered functions such that
$g$ represents the Fourier transform of $f$.   Then

(a)(i) $g(y)
=\lim_{\epsilon\downarrow 0}\Bover1{\sqrt{2\pi}}\int_{-\infty}^{\infty}
e^{-iyx}e^{-\epsilon x^2}f(x)dx$ for almost every $y\in\Bbb R$.

\quad(ii) If $y\in\Bbb R$ is such that $a=\lim_{t\in\dom g,t\uparrow
y}g(t)$ and $b=\lim_{t\in\dom g,t\downarrow y}g(t)$ are both defined in
$\Bbb C$, then

\Centerline{$\lim_{\epsilon\downarrow 0}\Bover1{\sqrt{2\pi}}
\int_{-\infty}^{\infty}
e^{-iyx}e^{-\epsilon x^2}f(x)dx=\Bover12(a+b)$.}

(b)(i) $f(x)
=\lim_{\epsilon\downarrow 0}\Bover1{\sqrt{2\pi}}\int_{-\infty}^{\infty}
e^{ixy}e^{-\epsilon y^2}g(y)dy$ for almost every $x\in\Bbb R$.

\quad(ii) If $x\in\Bbb R$ is such that $a=\lim_{t\in\dom f,t\uparrow
x}f(t)$ and $b=\lim_{t\in\dom f,t\downarrow x}f(t)$ are both defined in
$\Bbb C$, then

\Centerline{$\lim_{\epsilon\downarrow 0}\Bover1{\sqrt{2\pi}}
\int_{-\infty}^{\infty}
e^{ixy}e^{-\epsilon y^2}g(y)dy=\Bover12(a+b)$.}

\proof{{\bf (a)(i)}  By 223D,

\Centerline{$\lim_{\delta\downarrow 0}\Bover1{2\delta}
\int_{-\delta}^{\delta}|g(y+t)-g(y)|dt=0$}

\noindent for almost every $y\in\Bbb R$, because $g$ is integrable over
any bounded interval.   Fix any such $y$.   Set
$\phi(t)=|g(y+t)+g(y-t)-2g(y)|$ whenever this is defined.   Then, as in
the proof of 282Ia,

\Centerline{$\int_0^{\delta}\phi
\le\int_{-\delta}^{\delta}|g(y+t)-g(y)|dt$,}

\noindent so $\lim_{\delta\downarrow
0}\bover1{\delta}\int_0^{\delta}\phi=0$.   Consequently, by 284L,

\Centerline{$g(y)=\lim_{\sigma\to\infty}(g*\psi_{1/\sigma})(y)$.}

\noindent We know from 283N that the Fourier transform of
$\psi_{\sigma}$ is $\bover1{\sigma}\psi_{1/\sigma}$ for any $\sigma>0$.
Accordingly, by 284K, $g*\psi_{1/\sigma}$ is the Fourier transform of
$\sigma\sqrt{2\pi} f\times\psi_{\sigma}$, that is,

\Centerline{$(g*\psi_{1/\sigma})(y)
=\int_{-\infty}^{\infty}e^{-iyx}\sigma\psi_{\sigma}(x)f(x)dx$.}

\noindent So

$$\eqalign{g(y)
&=\lim_{\sigma\to\infty}
   \int_{-\infty}^{\infty}e^{-iyx}\sigma\psi_{\sigma}(x)f(x)dx\cr
&=\lim_{\sigma\to\infty}\Bover1{\sqrt{2\pi}}\int_{-\infty}^{\infty}
   e^{-iyx}e^{-x^2/2\sigma^2}f(x)dx\cr
&=\lim_{\epsilon\downarrow 0}\Bover1{\sqrt{2\pi}}\int_{-\infty}^{\infty}
   e^{-iyx}e^{-\epsilon x^2}f(x)dx.\cr}$$

\noindent And this is true for almost every $y$.

\medskip

\quad{\bf (ii)} Again, setting $c=\bover12(a+b)$,
$\phi(t)=|g(y+t)+g(y-t)-2c|$ whenever this is defined, we have
$\lim_{t\in\dom\phi,t\downarrow 0}\phi(t)\penalty-100=0$, so of course
$\lim_{\delta\downarrow 0}\bover{1}{\delta}\int_0^{\delta}\phi=0$, and

$$c
=\lim_{\sigma\to\infty}(g*\psi_{1/\sigma})(y)
=\lim_{\epsilon\downarrow 0}\Bover1{\sqrt{2\pi}}\int_{-\infty}^{\infty}
   e^{-iyx}e^{-\epsilon x^2}f(x)dx$$

\noindent as before.


\medskip

{\bf (b)} This can be shown by similar arguments;  or it may be actually
deduced from (a), by observing that $x\mapsto\Reverse{f}(x)=f(-x)$ represents
the Fourier transform of $g$ (see 284Ie), and applying (a) to $g$ and
$\Reverse{f}$.
}%end of proof of 284M

\leader{284N}{$L^2$ \dvrocolon{spaces}}\cmmnt{ We are now ready for
results corresponding to 282J-282K.

\medskip

\noindent}{\bf Lemma} Let $\eusm L_{\Bbb C}^2$ be the space of
square-integrable
complex-valued functions on $\Bbb R$, and $\eusm S$ the space of rapidly
decreasing test functions.   Then for every $f\in\eusm L_{\Bbb C}^2$ and
$\epsilon>0$ there is an $h\in\eusm S$ such that $\|f-h\|_2\le\epsilon$.

\proof{ Set $\phi(x)=e^{-1/x}$ for $x>0$, zero for $x\le
0$;  recall from the proof of 284G that $\phi$ is smooth.   For any
$a<b$, the functions

\Centerline{$x\mapsto\phi_n(x)=\phi(n(x-a))\phi(n(b-x))$}

\noindent provide a sequence of test functions converging to
$\chi\ooint{a,b}$ from below, so (as in 284G)

\Centerline{$\inf_{h\in\eusm S}\,\|\chi\ooint{a,b}-h\|^2_2
\le\lim_{n\to\infty}\int_a^b|1-\phi_n|^2
=0$.}

\noindent Because $\eusm S$ is a linear space (284Ba), it follows that
for every step-function $g$ with bounded
support and every $\epsilon>0$ there is an $h\in\eusm S$ such that
$\|g-h\|_2\le\bover12\epsilon$.   But we know from 244H/244Pb
that for every
$f\in\eusm L_{\Bbb C}^2$ and $\epsilon>0$ there is  a step-function $g$
with bounded
support such that $\|f-g\|_2\le\bover12\epsilon$;  so there must be an
$h\in\eusm S$ such that

\Centerline{$\|f-h\|_2\le\|f-g\|_2+\|g-h\|_2\le \epsilon$.}

\noindent As $f$ and $\epsilon$ are arbitrary, we have the result.
}%end of proof of 284N

\leader{284O}{Theorem} (a) Let $f$ be any complex-valued function which
is square-integrable over $\Bbb R$.   Then $f$ is a tempered function
and its Fourier transform is represented by another square-integrable
function $g$, and $\|g\|_2=\|f\|_2$.

(b) If $f_1$ and $f_2$ are complex-valued functions, square-integrable
over $\Bbb R$, with Fourier transforms represented by functions $g_1$, $g_2$, then

\Centerline{$\int_{-\infty}^{\infty}f_1\times\bar f_2
=\int_{-\infty}^{\infty}g_1\times\bar g_2$.}

(c) If $f_1$ and $f_2$ are complex-valued functions, square-integrable
over $\Bbb R$, with Fourier transforms represented by functions $g_1$, $g_2$, then the integrable function $f_1\times f_2$ has Fourier transform $\bover1{\sqrt{2\pi}}g_1*g_2$.

(d) If $f_1$ and $f_2$ are complex-valued functions, square-integrable
over $\Bbb R$, with Fourier transforms represented by functions $g_1$, $g_2$, then $\sqrt{2\pi}g_1\times g_2$ represents the Fourier transform of the continuous function $f_1*f_2$.

\proof{{\bf (a)(i)} Consider first the case in which $f$ is a
rapidly decreasing test function and $g$ is its Fourier transform;  we
know that $g$ is also a \rdtf, and that $f$ is the inverse \Ft\ of $g$
(284C).   Now the complex conjugate $\overline{g}$ of $g$ is given by
the formula

\Centerline{$\overline{g}(y)
=\overline{\Bover1{\sqrt{2\pi}}
  \intop\nolimits_{-\infty}^{\infty}e^{-iyx}f(x)dx}
=\Bover1{\sqrt{2\pi}}\int_{-\infty}^{\infty}e^{iyx}\overline{f}(x)dx$,}

\noindent so that $\overline{g}$ is the inverse Fourier transform of
$\overline{f}$.   Accordingly

\Centerline{$\int f\times\overline{f}
=\int\varcheck{g}\times\overline{f}
=\int g\times\varcheck{\overline{f}}
=\int g\times\overline{g}$,}

\noindent using 283O for the middle equality.

\medskip

\quad{\bf (ii)}  Now suppose that $f\in\eusm L_{\Bbb C}^2$.   I said
that $f$ is a tempered function;  this is simply because

\Centerline{$\int_{-\infty}^{\infty}
\bigl(\Bover{1}{1+|x|}\bigr)^2dx<\infty$,}

\noindent so

\Centerline{$\int_{-\infty}^{\infty}
\Bover{|f(x)|}{1+|x|}dx<\infty$}

\noindent (244Eb).   By 284N, there is a sequence $\sequencen{f_n}$
of rapidly decreasing test functions such that
$\lim_{n\to\infty}\|f-f_n\|_2=0$.   By (i),

\Centerline{$\lim_{m,n\to\infty}\|\varhatf_m-\varhatf_n\|_2
=\lim_{m,n\to\infty}\|f_m-f_n\|_2=0$,}

\noindent and the sequence $\sequencen{{\varhatf}\vthsp_n^{\ssbullet}}$
of equivalence classes is a Cauchy sequence in $L_{\Bbb C}^2$.
Because $L_{\Bbb C}^2$ is complete (244G/244Pb),
$\sequencen{{\varhatf}\vthsp_n^{\ssbullet}}$ has a limit in
$L_{\Bbb C}^2$, which is
representable as $g^{\ssbullet}$ for some $g\in\eusm L_{\Bbb C}^2$.
Like $f$, $g$ must be a tempered function.   Of course

\Centerline{$\|g\|_2=\lim_{n\to\infty}\|\varhatf_n\|_2
=\lim_{n\to\infty}\|f_n\|_2=\|f\|_2$.}

\noindent Now if $h$ is any rapidly decreasing test function,
$h$ and $\varhat{h}$ are square-integrable (284Bc, 284C), so we shall have

\Centerline{$\int g\times h
=\lim_{n\to\infty}\int\varhatf_n\times h
=\lim_{n\to\infty}\int f_n\times\varhat{h}
=\int f\times\varhat{h}$.}

\noindent So $g$ represents the Fourier transform of $f$.

\medskip

{\bf (b)} By 284Ib, any functions representing the Fourier transforms of
$f_1$ and $f_2$ must be equal almost everywhere to square-integrable
functions, and therefore square-integrable, with the right norms.   It
follows as in 282K (part (d) of the proof) that if $g_1$, $g_2$
represent the Fourier
transforms of $f_1$, $f_2$, so that $ag_1+bg_2$ represents the Fourier
transform of $af_1+bf_2$ and $\|ag_1+bg_2\|_2=\|af_1+bf_2\|_2$ for all
$a$, $b\in\Bbb C$, we must have

\Centerline{$\int f_1\times\overline{f}_2=(f_1|f_2)
=(g_1|g_2)=\int g_1\times\overline{g}_2$.}

\medskip

{\bf (c)} Of course $f_1\times f_2$ is integrable because it is the
product of two square-integrable functions (244E/244Pb).

\medskip

\quad{\bf (i)} Let $y\in\Bbb R$ and set $f(x)=\overline{f_2(x)}e^{iyx}$
for $x\in\Bbb R$.   Then $f\in\eusm L_{\Bbb C}^2$.   We need to know
that the Fourier transform of $f$ is represented by $g$, where
$g(u)=\overline{g_2(y-u)}$.   \Prf\ Let $h$ be a rapidly decreasing test
function.   Then

$$\eqalign{\int g\times h
&=\int\overline{g_2(y-u)}h(u)du
=\int \overline{g_2(u)}h(y-u)du\cr
&=\overline{\int g_2\times h_1}
=\overline{\int f_2\times \varhat{h}_1},\cr}$$

\noindent where $h_1(u)=\overline{h(y-u)}$.   To compute $\varhat{h}_1$,
we have

$$\eqalign{\varhat{h}_1(v)
&=\Bover1{\sqrt{2\pi}}\int_{-\infty}^{\infty}e^{-ivu}h_1(u)du
=\Bover1{\sqrt{2\pi}}\int_{-\infty}^{\infty}
  e^{-ivu}\overline{h(y-u)}du\cr
&=\overline{\Bover1{\sqrt{2\pi}}\int_{-\infty}^{\infty}
  e^{ivu}h(y-u)du}
=\overline{\Bover1{\sqrt{2\pi}}\int_{-\infty}^{\infty}
  e^{iv(y-u)}h(u)du}
=\overline{e^{ivy}\varhat{h}(v)}.\cr}$$

\noindent So

\Centerline{$\int g\times h
=\overline{\intop f_2\times\varhat{h}_1}
=\int\overline{f_2(v)\varhat{h}_1(v)}dv
=\int\overline{f_2(v)}e^{ivy}\varhat{h}(v)dv
=\int f\times\varhat{h}$;}

\noindent as $h$ is arbitrary, $g$ represents the Fourier transform of
$f$. \Qed

\medskip

\quad{\bf (ii)} We now have

$$\eqalignno{(f_1\times f_2)\varsphat(y)
&=\Bover1{\sqrt{2\pi}}
  \int_{-\infty}^{\infty}e^{-iyx}f_1(x)f_2(x)dx\cr
&=\Bover1{\sqrt{2\pi}}
  \int_{-\infty}^{\infty}f_1\times\bar f
=\Bover1{\sqrt{2\pi}}
  \int_{-\infty}^{\infty}g_1\times\bar g\cr
\noalign{\noindent (using part (b))}
&=\Bover1{\sqrt{2\pi}}\int_{-\infty}^{\infty}g_1(u)g_2(y-u)du
=\Bover1{\sqrt{2\pi}}(g_1*g_2)(y).\cr}$$

\noindent As $y$ is arbitrary,
$(f_1\times f_2)\varsphat=\bover1{\sqrt{2\pi}}g_1*g_2$, as claimed.

\medskip

{\bf (d)} By (c), the Fourier transform of $\sqrt{2\pi}g_1\times g_2$ is
$\Reverse{f}_1*\Reverse{f}_2$, writing $\Reverse{f}_1(x)=f_1(-x)$,
so that $\Reverse{f}_1$ represents
the Fourier transform of $g_1$.   So the inverse Fourier transform of
$\sqrt{2\pi}g_1\times g_2$ is $(\Reverse{f}_1*\Reverse{f}_2)\ssplrarrow$.
But, just as in the
proof of 284Kb, $(\Reverse{f}_1*\Reverse{f}_2)\ssplrarrow=f_1*f_2$,
so $f_1*f_2$ is the
inverse Fourier transform of $\sqrt{2\pi}g_1\times g_2$, and
$\sqrt{2\pi}g_1\times g_2$ represents the Fourier transform of
$f_1*f_2$, as claimed.   Also $f_1*f_2$, being the Fourier transform of
an integrable function, is continuous (283Cf;  see also 255K).
}%end of proof of 284O

\leader{284P}{Corollary} Writing $L_{\Bbb C}^2$ for the Hilbert space of
equivalence classes of square-integrable complex-valued functions on
$\Bbb R$, we have a linear isometry $T:L_{\Bbb C}^2\to L_{\Bbb C}^2$
given by saying that
$T(f^{\ssbullet})=g^{\ssbullet}$ whenever $f$, $g\in\eusm L_{\Bbb C}^2$
and $g$ represents the Fourier transform of $f$.

\cmmnt{
\leader{284Q}{Remarks (a)} 284P corresponds, of course, to 282K, where
the similar isometry between $\ell_{\Bbb C}^2(\Bbb Z)$ and
$L_{\Bbb C}^2(\ocint{-\pi,\pi})$
%terms reversed for sake of line break
is described.   In that case there was a marked asymmetry which is
absent from the present situation;  because the relevant measure on
$\Bbb Z$, counting measure, gives non-zero mass to every point, members
of $\ell_{\Bbb C}^2$ are true functions, and it is not surprising that
we have a
straightforward formula for $S(f^{\ssbullet})\in\ell_{\Bbb C}^2$ for
every $f\in\eusm L_{\Bbb C}^2(\ocint{-\pi,\pi})$.   The difficulty of
describing
$S^{-1}:\ell_{\Bbb C}^2(\Bbb Z)\to L_{\Bbb C}^2(\ocint{-\pi,\pi})$ is
very similar to the difficulty of describing
$T:L_{\Bbb C}^2(\Bbb R)\to L_{\Bbb C}^2(\Bbb R)$ and its inverse.
284Yg and 286U-286V show just how close this similarity is.

\header{284Qb}{\bf (b)} I have spelt out parts (c) and (d) of 284O in
detail, perhaps in unnecessary detail, because they give me an
opportunity to insist on the difference between
`$\sqrt{2\pi}g_1\times g_2$ {\it represents} the Fourier transform of
$f_1*f_2$' and
`$\bover1{\sqrt{2\pi}}g_1*g_2$ {\it is} the Fourier transform of
$f_1\times f_2$'.   The actual functions $g_1$ and $g_2$ are not
well-defined by the hypothesis that they represent the Fourier
transforms of $f_1$ and $f_2$, though their equivalence classes
$g_1^{\ssbullet}$, $g_2^{\ssbullet}\in L_{\Bbb C}^2$ are.   So the
product
$g_1\times g_2$ is also not uniquely defined as a function, though its
equivalence class
$(g_1\times g_2)^{\ssbullet}=g_1^{\ssbullet}\times g_2^{\ssbullet}$ is
well-defined as a member of $L_{\Bbb C}^1$.   However the
continuous function $g_1*g_2$ is unaffected by changes to $g_1$ and
$g_2$ on negligible sets, so is well defined as a function;  and since
$f_1\times f_2$ is integrable, and has a true Fourier transform, it is
to be expected that $(f_1\times f_2)\varsphat$ should be exactly equal
to $\bover1{\sqrt{2\pi}}g_1*g_2$.

This distinction between `being' a Fourier transform and `representing' a
Fourier transform echoes a question which arose in 233D concerning
conditional expectations.   I spoke there of `a' conditional expectation
on $\Tau$ of a function $f$
as being `a $\mu\restr\Tau$-integrable function $g$ such that
$\int_Fg\,d\mu=\int_Ffd\mu$ for every $F\in\Tau$';  the point being that
any $\mu\restr\Tau$-integrable function equal almost everywhere to $g$
would equally be a conditional expectation of $f$.   Here we see that if
$g$ represents the Fourier transform of $f$ then any function almost
everywhere equal to $g$ will also represent the Fourier transform of $f$.
In 242J I suggested resolving this complication by regarding conditional
expectation as a map between $L^1$ spaces rather than between $\eusm L^1$
spaces.   Here, similarly, we could think of the Fourier transform
considered in 284H as being a linear operator
defined on a certain subspace of $L^0(\mu)$.

In the case of conditional expectations, I think that there are solid
reasons for taking the operators on $L^1$ spaces as the real embodiment of
the idea;  I will expand on these in Chapter 36 of the next volume.   In
the case of Fourier transforms, I do not think the arguments have the same
force.   In 284R below, and in \S285, we
shall see that there are important cases in which we want to talk about
Fourier transforms which cannot be represented by members of $L^0$, so
that this would still be only a half-way house.

\spheader 284Qc Of course 284Oc-284Od also exhibit a
characteristic
feature of arguments involving Fourier transforms, the extension by
continuity of relations valid for test functions.

\spheader 284Qd 284Oa is a version of {\bf Plancherel's
theorem}.   The formula $\|f\|_2=\|\varhatf\|_2$ is {\bf Parseval's
identity}.
}%end of comment

\cmmnt{
\leader{284R}{Dirac's delta function} Consider the tempered
function $\chi\Bbb R$ with constant value $1$.   In what sense, if any, can
we assign a Fourier transform to $\chi\Bbb R$?

If we examine $\int\chi\Bbb R\times\varhat{h}$, as suggested in 284H, we
get

\Centerline{$\int_{-\infty}^{\infty}\chi\Bbb R\times\varhat{h}
=\int_{-\infty}^{\infty}\varhat{h}
=\sqrt{2\pi}\varhat{h}\varcheck{\phantom{h}}(0)
=\sqrt{2\pi}h(0)$}

\noindent for every rapidly decreasing test function $h$.   Of course
there is no {\it function} $g$ such that
$\int g\times h=\sqrt{2\pi}h(0)$ for every rapidly decreasing test
function $h$, since
(using the arguments of 284G) we should have to have
$\int_{a}^bg=\sqrt{2\pi}$ whenever $a<0<b$, so that the indefinite
integral of $g$ could not be continuous at $0$.   However there is a
{\it measure} on $\Bbb R$ with exactly the right property, the Dirac
measure $\delta_0$ concentrated at $0$;  this is a Radon probability
measure (257Xa), and $\int h\,d\delta_0=h(0)$ for
every function $h$ defined at $0$.   So we shall have

\Centerline{$\int_{-\infty}^{\infty}\chi\Bbb R\times\varhat{h}
=\sqrt{2\pi}\int h\,d\delta_0$}

\noindent for every rapidly decreasing test function $h$, and we can
reasonably say that the measure $\nu=\sqrt{2\pi}\delta_0$ `represents
the Fourier transform of $\chi\Bbb R$'.

We note with pleasure at this point that

\Centerline{$\Bover1{\sqrt{2\pi}}\int e^{ixy}\nu(dy)=1$}

\noindent for every $x\in\Bbb R$, so that $\chi\Bbb R$ can be called the
inverse Fourier transform of $\nu$.

If we look at the formulae of Theorem 284M, we get ideas consistent with
this pairing of $\chi\Bbb R$ with $\nu$.   We have

\Centerline{
$\Bover1{\sqrt{2\pi}}\int_{-\infty}^{\infty}
  e^{-iyx}e^{-\epsilon x^2}\chi\Bbb R(x)dx
=\Bover1{\sqrt{2\pi}}\int_{-\infty}^{\infty}e^{-iyx}e^{-\epsilon x^2}dx
=\Bover1{\sqrt{2\epsilon}}e^{-y^2/4\epsilon}$}

\noindent for every $y\in\Bbb R$, using 283N with
$\sigma=1/\sqrt{2\epsilon}$.   So

\Centerline{$\lim_{\epsilon\downarrow 0}\Bover1{\sqrt{2\pi}}
\int_{-\infty}^{\infty}e^{-iyx}e^{-\epsilon x^2}\chi\Bbb R(x)dx=0$}

\noindent for every $y\ne 0$, and the Fourier transform of $\chi\Bbb R$
should be
zero everywhere except at $0$.   On the other hand, the functions
$y\mapsto \Bover1{\sqrt{2\epsilon}}e^{-y^2/4\epsilon}$ all have integral
$\sqrt{2\pi}$, concentrated more and more closely about $0$ as
$\epsilon$ decreases to $0$, so also point us directly to $\nu$, the
measure which gives mass $\sqrt{2\pi}$ to $0$.

Thus allowing measures, as well as functions, enables us to extend the
notion of Fourier transform.   Of course we can go very much farther
than this.   If $h$ is any rapidly decreasing test function, then (because
$\varcheck{h}\varhat{\phantom{h}}=h$)

\Centerline{$\int_{-\infty}^{\infty}x\varhat{h}(x)dx
=-i\sqrt{2\pi}h'(0)$,}

\noindent so that the identity function $x\mapsto x$ can be assigned, as
a Fourier transform, the operator $h\mapsto -i\sqrt{2\pi}h'(0)$.

At this point we are entering the true theory of (Schwartzian)
distributions or `generalized functions', and I had better stop.
The `Dirac delta function' is most naturally regarded as the measure
$\delta_0$ above;  alternatively, as
$\Bover1{\sqrt{2\pi}}\varhat{\chi\Bbb R}$.
}%end of comment

\exercises{\leader{284W}{The multidimensional case} As in \S283, I give
exercises
designed to point the way to the $r$-dimensional generalization.

\header{284Wa}{\bf (a)} A {\bf rapidly decreasing test function} on
$\BbbR^r$ is a function $h:\BbbR^r\to\Bbb C$ such that (i) $h$ is {\bf
smooth}, that is, all repeated partial derivatives

\Centerline{$\Bover{\partial^m h}{\partial
\xi_{j_1}\ldots\partial\xi_{j_m}}$}

\noindent are defined and continuous everywhere in $\BbbR^r$ (ii)

\Centerline{$\sup_{x\in\BbbR^r}\|x\|^k|h(x)|<\infty$,
\quad $\sup_{x\in\Bbb
R^r}\|x\|^k|\Bover{\partial^m h}{\partial
\xi_{j_1}\ldots\partial\xi_{j_m}}(x)|<\infty$}

\noindent for every $k\in\Bbb N$, $j_1,\ldots,j_m\le r$.   A {\bf
tempered function} on $\BbbR^r$ is a measurable complex-valued function
$f$, defined almost everywhere in $\BbbR^r$, such that, for some
$k\in\Bbb N$,

\Centerline{$\int_{\Bbb R^r}\Bover1{1+\|x\|^k}|f(x)|dx<\infty$.}

\noindent Show that if $f$ is a tempered function on $\BbbR^r$ and $h$
is a rapidly decreasing test function on $\BbbR^r$ then $f\times h$ is
integrable.

\header{284Wb}{\bf (b)} Show that if $h$ is a rapidly decreasing test
function on $\BbbR^r$ so is $\varhat{h}$, and that in this case
$\varhat{h}\varcheck{\phantom{h}}=h$.

\header{284Wc}{\bf (c)} Show that if $f$ is a tempered function on
$\Bbb R^r$ and $\int f\times h=0$ for every rapidly decreasing test
function $h$ on $\BbbR^r$, then $f=0$ a.e.

\header{284Wd}{\bf (d)} If $f$ and $g$ are tempered functions on
$\Bbb R^r$, I say that {\bf  $g$ represents the Fourier transform of $f$}
if $\int g\times h=\int f\times\varhat{h}$ for every rapidly decreasing
test function $h$ on $\BbbR^r$.   Show that if $f$ is integrable then
$\varhatf$ represents the Fourier transform of $f$ in this sense.

\header{284We}{\bf (e)} Let $f$ be any tempered function on $\BbbR^r$.
Writing $\psi_{\sigma}(x)
=\bover1{(\sigma\sqrt{2\pi})^r}e^{-x\dotproduct x/2\sigma^2}$
for $x\in\BbbR^r$, show that
$\lim_{\sigma\downarrow 0}(f*\psi_{\sigma})(x)=c$ whenever
$x\in\BbbR^r$, $c\in\Bbb C$ are such that
$\lim_{\delta\downarrow 0}\bover1{\delta^r}
\int_{B(x,\delta)}|f(t)-c|dt=0$, writing
$B(x,\delta)=\{t:\|t-x\|\le\delta\}$.

\header{284Wf}{\bf (f)} Let $f$ and $g$ be tempered functions on
$\BbbR^r$ such that $g$ represents the Fourier transform of $f$, and $h$
a rapidly decreasing test function.   Show that (i) the Fourier
transform of $f\times h$ is $\bover1{(\sqrt{2\pi})^r}g*\varhat{h}$ (ii)
$(\sqrt{2\pi})^rg\times\varhat{h}$ represents the Fourier transform of
$f*h$.

\header{284Wg}{\bf (g)} Let $f$ and $g$ be tempered functions on
$\BbbR^r$ such that $g$ represents the Fourier transform of $f$.   Show
that

\Centerline{$g(y)
=\lim_{\epsilon\downarrow 0}\Bover1{(\sqrt{2\pi})^r}
  \int_{\Bbb R^r}
  e^{-iy\dotproduct x}e^{-\epsilon x\dotproduct x}f(x)dx$}

\noindent for almost every $y\in\BbbR^r$.

\header{284Wh}{\bf (h)} Show that for any square-integrable
complex-valued function $f$ on $\BbbR^r$ and any $\epsilon>0$ there is
a \rdtf\ $h$ such that $\|f-h\|_2\le\epsilon$.

\header{284Wi}{\bf (i)} Let $\eusm L_{\Bbb C}^2$ be the space of
square-integrable complex-valued functions on $\BbbR^r$.   Show that

\quad (i) for every $f\in\eusm L_{\Bbb C}^2$ there is a
$g\in\eusm L_{\Bbb C}^2$ which
represents the Fourier transform of $f$, and in this case
$\|g\|_2=\|f\|_2$;

\quad (ii) if $g_1$, $g_2\in\eusm L_{\Bbb C}^2$ represent the Fourier
transforms of $f_1$, $f_2\in\eusm L_{\Bbb C}^2$, then
$\bover1{(\sqrt{2\pi})^r}g_1*g_2$ is
the Fourier transform of $f_1\times f_2$, and
$(\sqrt{2\pi})^rg_1\times g_2$ represents the Fourier transform of
$f_1*f_2$.

\spheader 284Wj Let $T$ be an invertible real $r\times r$ matrix, regarded
as a linear operator from $\BbbR^r$ to itself.   (i) Show that
$\varhatf=|\det T|(fT)\varsphat T\trs$ for every integrable
complex-valued function $f$ on $\BbbR^r$.   (ii) Show that $hT$
is a \rdtf\ for
every \rdtf\ $h$.   (iii) Show that if $f$, $g$ are a tempered functions
and $g$ represents the Fourier transform of $f$, then
$\Bover1{|\det T|}g(T\trs)^{-1}$ represents the Fourier transform of $fT$;
so that if $T$ is orthogonal, then $gT$ represents the
Fourier transform of $fT$.
%mt28bits
}%end of exercises 284W

\exercises{
\leader{284X}{Basic exercises (a)} Show that if $g$ and $h$ are rapidly
decreasing test functions, so is $g\times h$.
%284A

\spheader 284Xb Show that there are non-zero continuous integrable
functions $f$, $g:\Bbb R\to\Bbb C$ such that $f*g=0$ everywhere.
\Hint{take them to be \Ft s of suitable test functions.}
%284C

\spheader 284Xc Suppose that $f:\Bbb R\to\Bbb C$ is a
differentiable function such that its derivative $f'$ is a tempered
function and, for some $k\in\Bbb N$,

\Centerline{$\lim_{x\to\infty}x^{-k}f(x)
=\lim_{x\to-\infty}x^{-k}f(x)=0$.}

\noindent   (i) Show that $\int f\times h'=-\int f'\times h$ for every
rapidly decreasing test function $h$.   (ii) Show that if $g$ is a
tempered function representing the Fourier transform of $f$, then
$y\mapsto iyg(y)$ represents the Fourier transform of $f'$.
%284H

\spheader 284Xd\dvAnew{2013}
For a tempered function $f$ and $\alpha\in\Bbb R$, set

\Centerline{$(S_{\alpha}f)(x)=f(x+\alpha)$,
\quad$(M_{\alpha}f)(x)=e^{i\alpha x}f(x)$,
\quad$(D_{\alpha}f)(x)=f(\alpha x)$}

\noindent whenever these are defined.   (i) Show that $S_{\alpha}f$,
$M_{\alpha}f$ and (if $\alpha\ne 0$) $D_{\alpha}f$ are tempered functions.
(ii) Show that if $g$ is a tempered function which represents the Fourier
transform of $f$, then $M_{-\alpha}g$ represents the Fourier transform of
$S_{\alpha}f$, $S_{-\alpha}g$ represents the Fourier transform of
$M_{\alpha}f$, $\bar{\Reverse{g}}=\Reverse{\bar g}$ represents the 
Fourier transform of
$\bar f$, and if $\alpha\ne 0$ then $\Bover1{|\alpha|}D_{1/\alpha}g$
represents the Fourier transform of $D_{\alpha}f$.
%284I

\spheader 284Xe Show that if $h$ is a rapidly decreasing test
function and $f$ is any measurable complex-valued function, defined
almost everywhere in $\Bbb R$, such that
$\int_{-\infty}^{\infty}|x|^k|f(x)|dx<\infty$ for every $k\in\Bbb N$,
then the convolution $f*h$ is a rapidly decreasing test function.
\Hint{show that the Fourier transform of $f*h$ is a test
function.}
%284J

\sqheader 284Xf Let $f$ be a tempered function such that
$\lim_{a\to\infty}\int_{-a}^af$ exists in $\Bbb C$.   Show that
this limit is also equal to
$\lim_{\epsilon\downarrow 0}\int_{-\infty}^{\infty}
e^{-\epsilon x^2}f(x)dx$.
\Hint{set $g(x)=f(x)+f(-x)$.   Use 224J to show that if $0\le a\le b$
then $|\int_a^bg(x)e^{-\epsilon x^2}dx|\le\sup_{c\in[a,b]}|\int_a^cg|$,
so that $\lim_{a\to\infty}\int_0^ag(x)e^{-\epsilon x^2}dx$ exists
uniformly in $\epsilon$, while
$\lim_{\epsilon\downarrow 0}\int_0^ag(x)e^{-\epsilon x^2}dx
=\int_0^ag$ for every $a\ge 0$.}
%284L

\sqheader 284Xg Let $f$ and $g$ be tempered functions on
$\Bbb R$ such that $g$ represents the
Fourier transform of $f$.   Show that

\Centerline{$g(y)
=\lim_{a\to\infty}\Bover1{\sqrt{2\pi}}\int_{-a}^ae^{-iyx}f(x)dx$}

\noindent at almost all points $y$ for which the limit exists.
\Hint{284Xf, 284M.}
%284M, 284Xf, 284L

\sqheader 284Xh Let $f$ be an integrable complex-valued function
on $\Bbb R$ such that $\varhatf$ also is integrable.   Show that
$\varhatf\varcheck{\phantom{f}}=f$ at any point at which $f$ is
continuous.
%284M

\spheader 284Xi Show that for every $p\in\coint{1,\infty}$,
$f\in\eusm L_{\Bbb C}^p$ and $\epsilon>0$ there is a rapidly decreasing
test function $h$ such that $\|f-h\|_p\le\epsilon$.
%284N

\sqheader 284Xj Let $f$ and $g$ be square-integrable complex-valued
functions on $\Bbb R$ such that $g$ represents the Fourier transform of
$f$.   Show that

\Centerline{$\int_c^df=\Bover{i}{\sqrt{2\pi}}
\int_{-\infty}^{\infty}\Bover{e^{icy}-e^{idy}}{y}g(y)dy$}

\noindent whenever $c<d$ in $\Bbb R$.
%284O

\spheader 284Xk Let $f$ be a measurable complex-valued function,
defined almost everywhere in $\Bbb R$, such that $\int|f|^p<\infty$,
where $1<p\le 2$.   Show that $f$ is a tempered function and that there
is a tempered function $g$ representing the Fourier transform of $f$.
\Hint{express $f$ as $f_1+f_2$, where $f_1$ is integrable and
$f_2$ is square-integrable.}  ({\bf Remark} Defining $\|f\|_p$,
$\|g\|_q$ as in 244D, where $q=p/(p-1)$, we have
$\|g\|_q\le(2\pi)^{(p-2)/2p}\|f\|_p$;  see {\smc Zygmund 59}, XVI.3.2.)
%284O

\spheader 284Xl Let $f$, $g$ be square-integrable
complex-valued functions on $\Bbb R$ such that $g$ represents the
Fourier transform of $f$.

\quad{(i)} Show that

\Centerline{$\Bover1{\sqrt{2\pi}}\int_{-a}^ae^{ixy}g(y)dy
=\Bover1{\pi}\int_{-\infty}^{\infty}\Bover{\sin at}{t}f(x-t)dt$}

\noindent whenever $x\in\Bbb R$ and $a>0$.   \Hint{find the
inverse Fourier transform of $y\mapsto e^{-ixy}\chi[-a,a](y)$, and use
284Ob.}

\quad{(ii)} Show that if $f(x)=0$ for $x\in\ooint{c,d}$ then

\Centerline{$\Bover1{\sqrt{2\pi}}\lim_{a\to\infty}\int_{-a}^a
e^{ixy}g(y)dy=0$}

\noindent for $x\in\ooint{c,d}$.

\quad{(iii)} Show that if $f$ is differentiable at $x\in\Bbb R$,
then

\Centerline{$\Bover1{\sqrt{2\pi}}\lim_{a\to\infty}\int_{-a}^a
e^{ixy}g(y)dy=f(x)$.}

\quad{(iv)} Show that if $f$ has bounded variation over some
interval properly containing $x$, then

\Centerline{$\Bover1{\sqrt{2\pi}}\lim_{a\to\infty}\int_{-a}^a
e^{ixy}g(y)dy=\Bover12(\lim_{t\in\dom f,t\uparrow x}f(t)+\lim_{t\in\dom
f,t\downarrow x}f(t))$.}
%284O

\spheader 284Xm Let $f$ be an integrable complex function on $\Bbb R$.
Show that if $\varhatf$ is square-integrable, so is $f$.
%284O

\spheader 284Xn Let $f_1$, $f_2$ be square-integrable complex-valued
functions on $\Bbb R$ with \Ft s represented by $g_1$, $g_2$.   Show
that
$\int_{-\infty}^{\infty}f_1(t)f_2(-t)dt
=\int_{-\infty}^{\infty}g_1(t)g_2(t)dt$.
%284O

\spheader 284Xo Suppose $x\in\Bbb R$.
Write $\delta_x$ for Dirac measure on $\Bbb R$ concentrated
at $x$.   Describe a sense in which $\sqrt{2\pi}\delta_x$ can
be regarded as
the Fourier transform of the function $t\mapsto e^{ixt}$.
%284R

\spheader 284Xp For any tempered function $f$ and $x\in\Bbb R$,
let $\delta_x$ be the Dirac measure on $\Bbb R$ concentrated at $x$,
and set

\Centerline{$(\delta_x*f)(u)=\int f(u-t)\delta_x(dt)=f(u-x)$}

\noindent for every $u$ for which $u-x\in\dom f$ (cf.\ 257Xe).   If $g$
represents
the Fourier transform of $f$, find a corresponding representation of the
Fourier transform of $\delta_x*f$, and relate it to the product of $g$
with the Fourier transform of $\delta_x$.
%284R, 284Xo

\spheader 284Xq(i) Show that

\Centerline{$\lim_{\delta\downarrow 0,a\to\infty}
\bigl(\int_{-a}^{-\delta}\Bover1x e^{-iyx}dx
+\int_{\delta}^a\Bover1x e^{-iyx}dx\bigr)
=-\pi i\sgn y$}

\noindent for every $y\in\Bbb R$, writing $\sgn y=y/|y|$ if $y\ne 0$ and
$\sgn 0=0$.   \Hint{283Da.}

\quad{(ii)} Show that

\Centerline{$\lim_{c\to\infty}\Bover1c\int_0^c
  \int_{-a}^ae^{ixy}\sgn y\,dy\,da
=\Bover{2i}x$}

\noindent for every $x\ne 0$.

\quad{(iii)} Show that for any rapidly decreasing test function $h$,

$$\eqalign{\int_0^{\infty}\Bover1x(\varhat{h}(x)-\varhat{h}(-x))dx
&=\lim_{\delta\downarrow 0,a\to\infty}\bigl(\int_{-a}^{-\delta}
\Bover1x\varhat{h}(x)dx
+\int_{\delta}^{a}\Bover1x\varhat{h}(x)dx\bigr)\cr
&=-\Bover{i\pi}{\sqrt{2\pi}}\int_{-\infty}^{\infty}h(y)\sgn
y\,dy.\cr}$$

\quad{(iv)} Show that for any rapidly decreasing test function $h$,

\Centerline{$\Bover{i\pi}{\sqrt{2\pi}}\int_{-\infty}^{\infty}
  \varhat{h}(x)\sgn x\,dx
=\int_0^{\infty}\Bover1y(h(y)-h(-y))dy$.}
%284notes

\spheader 284Xr Let $\sequencen{h_n}$ be a sequence of rapidly
decreasing test functions such that
$\phi(f)=\lim_{n\to\infty}\int_{-\infty}^{\infty}h_n\times f$ is defined
for every rapidly decreasing test function $f$.   Show that
$\lim_{n\to\infty}\int_{-\infty}^{\infty}h'_n\times f$,
$\lim_{n\to\infty}\int_{-\infty}^{\infty}\varhat{h}_n\times f$ and
$\lim_{n\to\infty}\int_{-\infty}^{\infty}(h_n*g)\times f$ are defined
for all rapidly decreasing test functions $f$ and $g$, and are zero if
$\phi$ is identically zero.   \Hint{255G will help with the last.}
%284+

\leader{284Y}{Further exercises (a)}
%\spheader 284Ya
Let $f$ be an integrable complex-valued function
on $\ocint{-\pi,\pi}$, and $\tilde f$ its periodic extension, as in
282Ae.   Show that $\tilde f$ is a tempered function.   Show that for
any rapidly decreasing test function $h$,
$\int\tilde f\times\varhat{h}
=\sqrt{2\pi}\sum_{k=-\infty}^{\infty}c_kh(k)$, where
$\sequence{k}{c_k}$ is the sequence of Fourier coefficients of $f$.
({\it Hint\/}:  begin with the case $f(x)=e^{inx}$.   Next show that

\Centerline{$M=\sum_{k=-\infty}^{\infty}|h(k)|
+\sum_{k=-\infty}^{\infty}
\sup_{x\in[(2k-1)\pi,(2k+1)\pi]}|\varhat{h}(x)|<\infty$,}

\noindent and that

\Centerline{$|\int\tilde f\times\varhat{h}
-\sqrt{2\pi}\sum_{k=-\infty}^{\infty}c_kh(k)|
\le M\|f\|_1$.}

\noindent Finally apply 282Ib.)
%284C

\header{284Yb}{\bf (b)} Let $f$ be a complex-valued function, defined
almost everywhere in $\Bbb R$, such that $f\times h$ is integrable for
every rapidly decreasing test function $h$.   Show that $f$ is tempered.
%284F

\spheader 284Yc Let $f$ and $g$ be tempered functions on $\Bbb
R$ such that $g$ represents the Fourier transform of $f$.   Show that

$$\int_c^df
=\Bover{i}{\sqrt{2\pi}}\lim_{\sigma\to\infty}\int_{-\infty}^{\infty}
\Bover{e^{icy}-e^{idy}}{y}e^{-y^2/2\sigma^2}g(y)dy$$

\noindent whenever $c\le d$ in $\Bbb R$.   \Hint{set
$\theta=\chi[c,d]$.   Show that both sides are
$\lim_{\sigma\to\infty}\int f\times(\theta*\psi_{1/\sigma})$, defining
$\psi_{\sigma}$ as in 283N and 284L.}
%284H

\spheader 284Yd Show that if $g:\Bbb R\to\Bbb R$ is an odd
function of bounded variation such that
$\int_1^{\infty}\bover1x g(x)dx=\infty$, then $g$ does not represent the
Fourier transform of any tempered function.   \Hint{283Ye, 284Yc.}
%284Yc, 284H

\ifdim\pagewidth>467pt\fontdimen3\tenrm=1.84pt
  \fontdimen4\tenrm=1.22pt\fi
\spheader 284Ye
Let $\eusm S$ be the space of
rapidly decreasing test functions.   For $k$, $m\in\Bbb N$ set
$\tau_{km}(h)=\sup_{x\in\Bbb R}|x|^k|h^{(m)}(x)|$ for every
$h\in\eusm S$, writing $h^{(m)}$ for the $m$th derivative of $h$ as
usual.   (i) Show
that each $\tau_{km}$ is a seminorm and that $\eusm S$ is
complete and separable for
the metrizable linear space topology $\frak T$ they define.
(ii) Show that $h\mapsto\varhat{h}:\eusm S\to\eusm S$
is continuous for $\frak T$.   (iii) Show that if $f$ is any
tempered function, then $h\mapsto\int f\times h$ is
$\frak T$-continuous.   (iv) Show that if $f$ is an integrable
function such that $\int|x^kf(x)|dx<\infty$ for every $k\in\Bbb N$, then
$h\mapsto f*h:\eusm S\to\eusm S$ is $\frak T$-continuous.
%284J
\fontdimen3\tenrm=1.67pt\fontdimen4\tenrm=1.22pt

\spheader 284Yf Show that if $f$ is a tempered function on $\Bbb R$ and

\Centerline{$\gamma
=\lim_{c\to\infty}\Bover1c\int_{0}^c\int_{-a}^af(x)dxda$}

\noindent is defined in $\Bbb C$, then $\gamma$ is also

\Centerline{$\lim_{\epsilon\downarrow 0}\int_{-\infty}^{\infty}
f(x)e^{-\epsilon|x|}dx$.}
%284Xf, 284L

\spheader 284Yg Let $f$, $g$ be square-integrable
complex-valued functions on $\Bbb R$ such that $g$ represents the
Fourier transform of $f$.   Suppose that $m\in\Bbb Z$ and that
$(2m-1)\pi<x<(2m+1)\pi$.   Set $\tilde f(t)=f(t+2m\pi)$ for those
$t\in\ocint{-\pi,\pi}$ such that $t+2m\pi\in\dom f$.   Let
$\langle c_k\rangle_{k\in\Bbb Z}$ be the sequence of Fourier
coefficients of $\tilde f$.   Show that

\Centerline{$\Bover1{\sqrt{2\pi}}\lim_{a\to\infty}\int_{-a}^a
e^{ixy}g(y)dy
=\lim_{n\to\infty}\sum_{k=-n}^nc_ke^{ikx}$}

\noindent in the sense that if one limit exists in $\Bbb C$ so does the
other, and they are then equal.   \Hint{284Xl(i), 282Da.}
%284O, 284Xl

\spheader 284Yh Show that if $f$ is integrable over $\Bbb R$ and there
is some $M\ge 0$ such that $f(x)=\varhatf(x)=0$ for $|x|\ge M$, then
$f=0$ a.e.   \Hint{reduce to the case $M=\pi$.   Looking at the Fourier
series of $f\restr\ocint{-\pi,\pi}$, show that $f$ is expressible in the
form $f(x)=\sum_{k=-m}^mc_ke^{ikx}$ for almost every
$x\in\ocint{-\pi,\pi}$.   Now compute $\varhatf(2n+\bover12)$ for large
$n$.}

\spheader 284Yi Let $\nu$ be a Radon measure on $\Bbb R$ 
which is `tempered' in the sense that
$\int_{-\infty}^{\infty}\Bover1{1+|x|^k}\nu(dx)$ is finite for some
$k\in\Bbb N$.   (i) Show that every rapidly decreasing test function is
$\nu$-integrable.   (ii) Show that if $\nu$ has bounded support
(definition:  256Xf), and $h$ is a rapidly decreasing test function,
then $\nu*h$ is a rapidly decreasing test function, where
$(\nu*h)(x)=\int_{-\infty}^{\infty}h(x-y)\nu(dy)$ for $x\in\Bbb R$.
(iii) Show that there is a sequence $\sequencen{h_n}$ of rapidly
decreasing test functions such that
$\lim_{n\to\infty}\int_{-\infty}^{\infty}h_n\times f
=\int_{-\infty}^{\infty}fd\nu$ for every rapidly decreasing test
function $f$.
%284Xr

\spheader 284Yj Let $\phi:\eusm S\to\Bbb R$ be a functional defined by
the formula of 284Xr.   Show that $\phi$ is continuous for the topology
of 284Ye.
({\it Note\/}:  it helps to know a little more about metrizable linear
topological spaces than is covered in \S2A5.)
%284Xr

}%end of exercises

\cmmnt{
\Notesheader{284} Yet again I must warn you that the
material above gives a very restricted view of the subject.   I have
tried to indicate how the theory of Fourier transforms of `good'
functions -- here taken to be the rapidly decreasing test
functions -- may be extended, through a kind of duality,
to a very much wider class
of functions, the `tempered functions'.   Evidently, writing $\eusm S$
for the linear space of rapidly decreasing test functions, we can seek
to investigate a Fourier transform of any linear functional
$\phi:\eusm S\to\Bbb C$, writing $\varhat\phi(h)=\phi(\varhat{h})$ for
any $h\in\eusm S$.   (It is actually commoner at this point to restrict
attention to functionals $\phi$ which are continuous for the standard
topology on $\eusm S$, described in 284Ye;  these are called 
{\bf tempered (Schwartzian) distributions}.)   
By 284F-284G, we can identify some of these
functionals with equivalence classes of tempered functions, and then set
out to investigate those tempered functions whose Fourier transforms can
again be represented by tempered functions.

I suppose the structure of the theory of Fourier transforms is best laid
out through the formulae involved.   Our aim is to set up pairs
$(f,g)=(f,\varhatf)=(\varcheck{g},g)$ in such a way that we have

\quad {\it Inversion\/}:  $\varhat{h}\varcheck{\phantom{h}}
=\varcheck{h}\varhat{\phantom{h}}=h$;   %284If

\quad {\it Reversal\/}:  $\varcheck{h}(y)=\varhat{h}(-y)$;  %284If

\quad {\it Linearity\/}:
$(h_1+h_2)\varsphat=\varhat{h}_1+\varhat{h}_2$,
\quad$(ch)\varsphat=c\varhat{h}$;  %284Ic

\quad {\it Differentiation\/}: $(h')\varsphat(y)=iy\varhat{h}(y)$;

\quad {\it Shift\/}:  if $h_1(x)=h(x+c)$ then
$\varhat{h}_1(y)=e^{iyc}\varhat{h}(y)$;  %284Xd

\quad {\it Modulation\/}:  if $h_1(x)=e^{icx}h(x)$ then
$\varhat{h}_1(y)=\varhat{h}(y-c)$;  %284Xd

\quad {\it Symmetry\/}:  if $h_1(x)=h(-x)$ then $\varhat{h}_1(y)
=\varhat{h}(-y)$;  %284Xd

\quad {\it Complex Conjugate\/}:
$(\overline{h})\varsphat(y)=\overline{\varhat{h}(-y)}$;  %284Xd

\quad {\it Dilation\/}:  if $h_1(x)=h(cx)$, where $c>0$, then
$\varhat{h}_1(y)=\Bover1c\varhat{h}(\Bover{y}{c})$;  %284Xd

\quad {\it Convolution\/}:
$(h_1*h_2)\varsphat=\sqrt{2\pi}\varhat{h}_1\times\varhat{h}_2$, %284Kb
\quad $(h_1\times h_2)\varsphat
=\Bover1{\sqrt{2\pi}}\varhat{h}_1*\varhat{h}_2$; %284Ka

\quad {\it Duality\/}: $\int_{-\infty}^{\infty}h_1\times\varhat{h}_2
=\int_{-\infty}^{\infty}\varhat{h}_1\times h_2$;

\quad {\it Parseval\/}:  $\int_{-\infty}^{\infty}h_1\times\overline{h_2}
=\int_{-\infty}^{\infty}\varhat{h}_1\times\overline{\varhat{h}_2}$; %284Ob

\noindent and, of course,

\Centerline{$\varhat{h}(y)
=\Bover1{\sqrt{2\pi}}\int_{-\infty}^{\infty}e^{-iyx}h(x)dx$,} %284Xg

\Centerline{$\int_c^d\varhat{h}(y)dy
=\Bover{i}{\sqrt{2\pi}}\int_{-\infty}^{\infty}
\Bover{e^{-icy}-e^{-idy}}{y}h(y)dy$.}  %284Xj

\noindent (I have used the letter $h$ in the list above to suggest what
is in fact the case, that all the formulae here are valid for rapidly
decreasing test functions.)   On top of all this, it is often important
that the operation $h\mapsto\varhat{h}$ should be continuous in some
sense.

The challenge of the `pure' theory of Fourier transforms is to find
the widest possible variety of objects $h$ for which the formulae above
will be valid, subject to appropriate interpretations of $\varsphat$,
$*$ and $\int_{-\infty}^{\infty}$.
I must of course remark here that from the very beginnings, the subject
has been enriched by its applications in other parts of mathematics, the
physical sciences and the social sciences, and that again and again
these have suggested further possible pairs $(f,\varhatf)$, making new
demands on our power to interpret the rules we seek to follow.   Even
the theory of distributions does not seem to give a full canonical
account of what can be done.   First, there are great difficulties in
interpreting the `product' of two arbitrary distributions, making
several of the formulae above problematic;  and second, it is not
obvious that only one kind of distribution need be considered.   In this
section I have looked at just one space of `test functions', the
space $\eusm S$ of rapidly decreasing test functions;  but at least two
others are significant, the space $\eusm D$ of smooth functions with
bounded support and the space $\eusm Z$ of Fourier transforms of
functions in
$\eusm D$.   The advantage of starting with $\eusm S$ is that it gives a
symmetric theory, since $\varhat{h}\in\eusm S$ for every $h\in\eusm S$;
but it is easy to find objects (e.g., the function $x\mapsto e^{x^2}$,
or the function $x\mapsto 1/|x|$) which cannot be interpreted as
functionals on $\eusm S$, so that their Fourier transforms must be
investigated by other methods, if at all.   In 284Xq I sketch some of
the arguments which can be used to justify the assertion that the
Fourier transform of the function $x\mapsto 1/x$ is, or can be
represented by, the function
$y\mapsto -i\sqrt{\bover{\pi}2}\sgn y$;  the general principle in this
case being that we approach both $0$ and $\infty$ symmetrically.
For a variety of such matching pairs, established by arguments based on
the idea in 284Xr, see {\smc Lighthill 59}, chap.\ 3.

Accordingly it seems that, after two centuries,
% Fourier "Analytic Theory of Heat" 1822
%    1807 M�moire sur la propagation de la chaleur dans les corps solides
% Wikipedia:  "When Fourier submitted a later competition essay in 1811,
% the committee (which included Lagrange, Laplace, Malus and Legendre,
% among others) concluded: ...the manner in which the author arrives at
% these equations is not exempt of difficulties and...his analysis to
% integrate them still leaves something to be desired on the score of
% generality and even rigour [citation needed]"
we must still
proceed by carefully examining particular classes of function, and
checking appropriate interpretations of the formulae.   In the work
above I have repeatedly used the concepts

\Centerline{$\lim_{a\to\infty}\int_{-a}^af$,
\quad$\lim_{\epsilon\downarrow 0}\int_{-\infty}^{\infty}
  e^{-\epsilon x^2}f(x)dx$}

\noindent as alternative interpretations of $\int_{-\infty}^{\infty}f$.
(Of course they are closely related;  see 284Xf.)   The reasons for
using the particular kernel $e^{-\epsilon x^2}$ are that it belongs to
$\eusm S$, it is an even function,
its Fourier transform is calculable and easy to manipulate, and it is
associated with the normal probability density function
$\bover1{\sigma\sqrt{2\pi}}e^{-x^2/2\sigma^2}$, so that any
miscellaneous facts we gather have a chance of being valuable elsewhere.
But there are applications in which alternative kernels are more
manageable -- e.g., $e^{-\epsilon|x|}$ (283Xq, 283Yc, 284Yf).

One of the guiding principles here is that purely formal manipulations,
along the lines of those in the list above, and (especially) changes in
the order of integration, with other exchanges of limit, again and again
give rise to formulae which, suitably interpreted, are valid.   First
courses in analysis are often inhibitory;  students are taught to
distrust any manipulation which they cannot justify.   To my own eye,
the delight of this topic lies chiefly in the variety of the arguments
demanded by a rigorous approach, the ground constantly shifting with the
context;  but there is no doubt that cheerful sanguinity is often the
best guide to the manipulations which it will be right to try to
justify.

This being a book on measure theory, I am of course particularly
interested in the possibility of a measure appearing as a Fourier
transform.   This is what happens if we seek the Fourier transform of
the constant function $\chi\Bbb R$ (284R).   More generally, any periodic
tempered function $f$ with period $2\pi$ can be assigned a Fourier
transform which is a `signed measure' (for our present purposes, a
complex linear combination of measures) concentrated on $\Bbb Z$, the
mass at each $k\in\Bbb Z$ being determined by the corresponding Fourier
coefficient of $f\restr\ocint{-\pi,\pi}$ (284Xo, 284Ya).   In the next
section I will go farther in this direction, with particular reference
to probability distributions on $\BbbR^r$.   But the reason why {\it
positive} measures have not forced themselves on our attention so far is
that we do not expect to get a positive function as a Fourier transform
unless some very special conditions are satisfied, as in 283Yc.

As in \S282, I have used the Hilbert space structure of $L_{\Bbb C}^2$
as the basis of the discussion of Fourier transforms of functions in
$\eusm L_{\Bbb C}^2$ (284O-284P).   But as with Fourier series,
Carleson's theorem (286U) provides a more direct description.

In 284Wj, I offer a calculation based on the change-of-variable formula in
263A to present a multidimensional
version of Reversal and Dilation.   But what I am really trying to do is
to show that Fourier transforms on
$\BbbR^r$ are based on the geometry of the
Euclidean inner product, not on the Cartesian coordinate system.
}%end of notes

\discrpage

