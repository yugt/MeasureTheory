\frfilename{mt223.tex}
\versiondate{9.9.04}
\copyrightdate{1995}
     
\def\chaptername{The Fundamental Theorem of Calculus}
\def\sectionname{Lebesgue's density theorems}
\def\undpsi{\underline{\psi}}
     
\newsection{223}
     
I now turn to a group of results which may be thought of as corollaries
of Theorem 222E, but which also have a vigorous life of their own,
including the possibility of significant generalizations which will be
treated in Chapter 26.   The idea is that any measurable function $f$ on
$\Bbb R$ is almost everywhere `continuous' in a variety of very weak
senses;  for almost every $x$, the value $f(x)$ is determined by the
behaviour of $f$ near $x$, in the sense that $f(y)\bumpeq f(x)$ for
`most' $y$ near $x$.    I should perhaps say that while I recommend this
work as a preparation for Chapter 26, and I also rely on it in Chapter
28, I shall not refer to it again in the present chapter, so that
readers in a hurry to characterize indefinite integrals may proceed
directly to \S224.
     
\leader{223A}{Lebesgue's Density Theorem:  integral form} Let  $I$ be an
interval in $\Bbb R$, and let $f$ be a real-valued function which is
integrable over $I$.   Then
     
$$\eqalign{f(x)
&=\lim_{h\downarrow 0}\Bover1{h}\int_{x}^{x+h}f
=\lim_{h\downarrow 0}\Bover1h\int_{x-h}^xf
=\lim_{h\downarrow 0}\Bover1{2h}\int_{x-h}^{x+h}f\cr}$$
     
\noindent for almost every $x\in I$.
     
     
\proof{ Setting $F(x)=\int_{I\cap\ooint{-\infty,x}}f$, we
know from 222G that
     
$$\eqalign{f(x)=F'(x)
&=\lim_{h\downarrow 0}\Bover1{h}(F(x+h)-F(x))
   =\lim_{h\downarrow 0}\Bover1{h}\int_{x}^{x+h}f\cr
&=\lim_{h\downarrow 0}\Bover1h(F(x)-F(x-h))
   =\lim_{h\downarrow 0}\Bover1h\int_{x-h}^xf\cr
&=\lim_{h\downarrow 0}\Bover1{2h}(F(x+h)-F(x-h))
   =\lim_{h\downarrow 0}\Bover1{2h}\int_{x-h}^{x+h}f\cr}$$
     
\noindent for almost every $x\in I$.
}%end of proof of 223A
     
\leader{223B}{Corollary} Let $E\subseteq\Bbb R$ be a measurable set.
Then
     
\Centerline{$\lim_{h\downarrow 0}\Bover1{2h}\mu(E\cap[x-h,x+h])=1$ for
almost every $x\in E$,}
     
\Centerline{$\lim_{h\downarrow 0}\Bover1{2h}\mu(E\cap[x-h,x+h])=0$ for
almost every $x\in \Bbb R\setminus E$.}
     
     
\proof{ Take $n\in \Bbb N$.   Applying 223A to
$f=\chi(E\cap[-n,n])$, we see that
     
\Centerline{$\lim_{h\downarrow 0}\Bover1{2h}\int_{x-h}^{x+h}f
=\lim_{h\downarrow 0}\Bover1{2h}\mu(E\cap[x-h,x+h])$}
     
\noindent whenever $x\in\ooint{-n,n}$ and either limit exists, so that
     
\Centerline{$\lim_{h\downarrow 0}\Bover1{2h}\mu(E\cap[x-h,x+h])=1$ for
almost every $x\in E\cap[-n,n]$,}
     
\Centerline{$\lim_{h\downarrow 0}\Bover1{2h}\mu(E\cap[x-h,x+h])=0$ for
almost every $x\in [-n,n]\setminus E$.}
     
\noindent As $n$ is arbitrary, we have the result.
}%end of proof of 223B
     
\cmmnt{\medskip
     
\noindent{\bf Remark} For a measurable set $E\subseteq\Bbb R$, a point
$x$ such that $\lim_{h\downarrow 0}\Bover1{2h}\mu(E\cap[x-h,x+h])=1$ is
sometimes called a {\bf density point} of $E$.
}%end of comment
     
\leader{223C}{Corollary} Let $f$ be a measurable real-valued function
defined almost everywhere in $\Bbb R$.   Then for almost every
$x\in\Bbb R$,
     
\Centerline{$\lim_{h\downarrow 0}\Bover1{2h}\mu\{y:y\in\dom f,\,|y-x|\le
h,\,|f(y)-f(x)|\le\epsilon\}=1$,}
     
\Centerline{$\lim_{h\downarrow 0}\Bover1{2h}\mu\{y:y\in\dom f,\,|y-x|\le
h,\,|f(y)-f(x)|\ge\epsilon\}=0$}
     
\noindent for every $\epsilon>0$.
     
\proof{ For $q$, $q'\in\Bbb Q$, set
     
\Centerline{$D_{qq'}=\{x:x\in\dom f,\,q\le f(x)<q'\}$,}
     
\noindent so that $D_{qq'}$ is measurable,
     
\Centerline{$C_{qq'}=\{x:x\in D_{qq'},\,\lim_{h\downarrow 0}
\Bover1{2h}\mu(D_{qq'}\cap [x-h,x+h])=1\}$,}
     
     
\noindent so that $D_{qq'}\setminus C_{qq'}$ is negligible, by 223B;
now
set
     
\Centerline{$C=\dom f\setminus\bigcup_{q,q'\in\Bbb Q}(D_{qq'}\setminus
C_{qq'})$,}
     
\noindent so that $C$ is conegligible.   If $x\in C$ and
$\epsilon>0$, then there are $q$, $q'\in \Bbb Q$ such that
$f(x)-\epsilon\le q\le f(x)<q'\le f(x)+\epsilon$, so that $x$ belongs to
$D_{qq'}$ and therefore to $C_{qq'}$, and now
     
$$\eqalign{\liminf_{h\downarrow 0}\Bover1{2h}
&\mu\{y:y\in \dom f\cap [x-h,x+h],\,|f(y)-f(x)|\le\epsilon\}\cr
&\ge\liminf_{h\downarrow 0}
\Bover1{2h}\mu(D_{qq'}\cap [x-h,x+h])
=1,\cr}$$
     
\noindent so
     
\Centerline{$\lim_{h\downarrow 0}
\Bover1{2h}\mu\{y:y\in \dom f\cap [x-h,x+h],\,|f(y)-f(x)|\le\epsilon\}
=1$.}
     
It follows at once that
     
\Centerline{$\lim_{h\downarrow 0}
\Bover1{2h}\mu\{y:y\in \dom f\cap [x-h,x+h],\,|f(y)-f(x)|>\epsilon\}
=0$}
     
\noindent for almost every $x$;  but since $\epsilon$ is arbitrary, this
is also true of $\bover12\epsilon$, so in fact
     
\Centerline{$\lim_{h\downarrow 0}
\Bover1{2h}\mu\{y:y\in \dom f\cap [x-h,x+h],\,|f(y)-f(x)|\ge\epsilon\}
=0$}
     
\noindent for almost every $x$.
}%end of proof of 223C
     
\leader{223D}{Theorem} Let $I$ be an interval in $\Bbb R$, and let $f$
be a real-valued function which is integrable over $I$.   Then
     
\Centerline{$\lim_{h\downarrow 0}\Bover1{2h}
\int_{x-h}^{x+h}|f(y)-f(x)|dy=0$}
     
\noindent for almost every $x\in I$.
     
\proof{{\bf (a)} Suppose first that $I$ is a bounded open
interval $\ooint{a,b}$.   For each $q\in\Bbb Q$, set $g_q(x)=|f(x)-q|$
for $x\in I\cap\dom f$;  then $g$ is integrable over $I$, and
     
\Centerline{$\lim_{h\downarrow 0}\Bover1{2h}
\int_{x-h}^{x+h}g_q=g_q(x)$}
     
\noindent for almost every $x\in I$, by 223A.   Setting
     
\Centerline{$E_q=\{x:x\in I\cap\dom f,\,\lim_{h\downarrow 0}
\Bover1{2h}\int_{x-h}^{x+h}g_q=g_q(x)\}$,}
     
\noindent we have $I\setminus E_q$ negligible, so $I\setminus E$ is
negligible, where $E=\bigcap_{q\in\Bbb Q}E_q$.   Now
     
\Centerline{$\lim_{h\downarrow 0}\Bover1{2h}
\int_{x-h}^{x+h}|f(y)-f(x)|dy=0$}
     
\noindent for every $x\in E$.   \Prf\ Take $x\in E$ and $\epsilon>0$.
Then there is a $q\in\Bbb Q$ such that $|f(x)-q|\le\epsilon$, so that
     
\Centerline{$|f(y)-f(x)|\le|f(y)-q|+\epsilon=g_q(y)+\epsilon$}
     
\noindent for every $y\in I\cap\dom f$, and

$$\eqalign{\limsup_{h\downarrow 0}\Bover1{2h}
\int_{x-h}^{x+h}|f(y)-f(x)|dy
&\le\limsup_{h\downarrow 0}\Bover1{2h}
\int_{x-h}^{x+h}g_q(y)+\epsilon\,dy\cr
&=\epsilon+g_q(x)
\le 2\epsilon.\cr}$$
     
\noindent As $\epsilon$ is arbitrary,
     
\Centerline{$\lim_{h\downarrow 0}\Bover1{2h}
\int_{x-h}^{x+h}|f(y)-f(x)|dy=0$,}
     
\noindent as required.\ \Qed
     
\medskip
     
{\bf (b)}  If $I$ is an unbounded open interval, apply (a) to the
intervals $I_n=I\cap\ooint{-n,n}$ to see that the limit is zero almost
everywhere in every $I_n$, and therefore on $I$.   If $I$ is an
arbitrary interval, note that it differs by at most two points from an
open interval, and that since we are looking only for something to
happen almost everywhere we can ignore these points.
}%end of proof of 223D
     
\medskip
     
\noindent{\bf Remark} The set
     
\Centerline{$\{x:x\in\dom f,\,\lim_{h\downarrow 0}\Bover1{2h}
\int_{x-h}^{x+h}|f(y)-f(x)|dy=0\}$}
     
\noindent is sometimes called the {\bf Lebesgue set} of $f$.
     
\leader{223E}{Complex-valued functions}\cmmnt{ I have expressed
the results above in terms of real-valued functions, this being the most
natural vehicle for the ideas.   However there are applications of great
importance in which the functions involved are complex-valued, so I
spell out the relevant statements here.   In all cases the proof is
elementary, being nothing more than applying the corresponding result
(223A, 223C or 223D) to the real and imaginary parts of the function
$f$.}
     
\header{223Ea}{\bf (a)} Let  $I$ be an interval in $\Bbb R$, and let $f$
be a complex-valued function which is integrable over $I$.   Then
     
$$\eqalign{f(x)
=\lim_{h\downarrow 0}\Bover1{h}\int_{x}^{x+h}f
=\lim_{h\downarrow 0}\Bover1h\int_{x-h}^xf
=\lim_{h\downarrow 0}\Bover1{2h}\int_{x-h}^{x+h}f\cr}$$
     
\noindent for almost every $x\in I$.
     
\header{223Eb}{\bf (b)} Let $f$ be a measurable complex-valued function
defined almost everywhere in $\Bbb R$.   Then for almost every
$x\in\Bbb R$,
     
\Centerline{$\lim_{h\downarrow 0}\Bover1{2h}\mu\{y:y\in\dom f,\,|y-x|\le
h,\,|f(y)-f(x)|\ge\epsilon\}=0$}
     
\noindent for every $\epsilon>0$.
     
\header{223Ec}{\bf (c)} Let $I$ be an interval in $\Bbb R$, and let $f$
be a complex-valued function which is integrable over $I$.   Then
     
\Centerline{$\lim_{h\downarrow 0}\Bover1{2h}
\int_{x-h}^{x+h}|f(y)-f(x)|dy=0$}
     
\noindent for almost every $x\in I$.
     
\exercises{
\leader{223X}{Basic exercises $\pmb{>}$(a)} Let $E\subseteq[0,1]$ be a 
measurable set for which there is an $\alpha>0$ such that
$\mu(E\cap[a,b])\ge\alpha(b-a)$ whenever $0\le a\le b\le 1$.
Show that $\mu E=1$.
%223B     
     
\spheader 223Xb Let $A\subseteq\Bbb R$ be any set.   Show that
$\lim_{h\downarrow 0}\Bover1{2h}\mu^*(A\cap[x-h,x+h])=1$ for
almost every $x\in A$.  \Hint{apply 223B to a measurable envelope
$E$ of $A$.}
%223B
     
\spheader 223Xc Let $E$, $F\subseteq\Bbb R$ be measurable sets, and
$x\in\Bbb R$ a point which is a density point of both.   Show that $x$
is a density point of $E\cap F$.
%223B
     
\spheader 223Xd Let $E\subseteq\Bbb R$ be a non-negligible measurable
set.   Show that for any $n\in\Bbb N$ there is a $\delta>0$ such that
$\bigcap_{i\le n}E+x_i$ is non-empty whenever $x_0,\ldots,x_n\in\Bbb R$
are such that $|x_i-x_j|\le\delta$ for all $i$, $j\le n$.   \Hint{find a
non-trivial interval $I$ such that $\mu(E\cap I)>\bover{n}{n+1}\mu I$.}
%223B
     
\spheader 223Xe Let $f$ be any real-valued function defined almost
everywhere in $\Bbb R$.   Show that $\lim_{h\downarrow
0}\Bover1{2h}\mu^*\{y:y\in\dom f,\,|y-x|\le
h,\,|f(y)-f(x)|\le\epsilon\}=1$ for almost every $x\in\Bbb R$.
\Hint{use the argument of 223C, but with 223Xb in place of 223B.}
%223C
     
\sqheader 223Xf Let $I$ be an interval in $\Bbb R$, and let $f$
be a real-valued function which is integrable over $I$.   Show that
$\lim_{h\downarrow 0}\bover1{h}
\int_{x}^{x+h}|f(y)-f(x)|dy=0$ for almost every $x\in I$.
%223D
     
\spheader 223Xg Let $E$, $F\subseteq\Bbb R$ be
measurable sets, and suppose that $F$ is bounded and of non-zero
measure.   Let $x\in\Bbb R$ be such that
$\lim_{h\downarrow 0}\Bover1{2h}\mu(E\cap[x-h,x+h])=1$.   Show that
$\lim_{h\downarrow 0}\Bover{\mu(E\cap(x+hF))}{h\,\mu F}=1$.
({\it Hint:\/} it helps to know that $\mu(hF)=h\mu F$
(134Ya, 263A).   Show that if $F\subseteq[-M,M]$, then
     
\Centerline{$\Bover1{2hM}\mu(E\cap[x-hM,x+hM])
\le 1-\Bover{\mu F}{2M}
  \bigl(1-\Bover{\mu(E\cap(x+hF))}{h\,\mu F}\bigr)$.)}
     
\noindent (Compare 223Ya.)
%223B
     
\spheader 223Xh Let $f$ be a real-valued function which is integrable
over $\Bbb R$, and $E$ be the Lebesgue set of $f$.   Show that
$\lim_{h\downarrow 0}\bover1{2h}\int_{x-h}^{x+h}|f(t)-c|dt=|f(x)-c|$ 
for every $x\in E$ and $c\in\Bbb R$.
%223D
     
\spheader 223Xi Let $f$ be an integrable real-valued function defined
almost everywhere in $\Bbb R$.   Let $x\in\dom f$ be such that
$\lim_{n\to\infty}\Bover{n}2\int^{x+1/n}_{x-1/n}|f(y)-f(x)|=0$.   Show
that $x$ belongs to the Lebesgue set of $f$.
%223D
     
\spheader 223Xj Let $f$ be an integrable real-valued function defined
almost everywhere in $\Bbb R$, and $x$ any point of the Lebesgue set of
$f$.   Show
that for every $\epsilon>0$ there is a $\delta>0$ such that whenever $I$
is a non-trivial interval and $x\in I\subseteq[x-\delta,x+\delta]$, then
$|f(x)-\Bover1{\mu I}\int_If|\le\epsilon$.
%223D
     
\leader{223Y}{Further exercises (a)} Let $E$, $F\subseteq\Bbb R$ be
measurable sets, and suppose that $0<\mu F<\infty$.   Let $x\in\Bbb R$
be such that $\lim_{h\downarrow 0}\Bover1{2h}\mu(E\cap[x-h,x+h])=1$.
Show that
     
\Centerline{$\lim_{h\downarrow 0}\Bover{\mu(E\cap(x+hF))}{h\,\mu F}=1$.}
     
\noindent \Hint{apply 223Xg to sets of the form $F\cap[-M,M]$.}
%223Xg 223B
     
\header{223Yb}{\bf (b)} Let $\frak T$ be the family of measurable sets
$G\subseteq\Bbb R$ such that every point of $G$ is a density point of
$G$.   {(i)} Show that $\frak T$ is a topology on $\Bbb R$.   ({\it
Hint\/}: take $\Cal G\subseteq\frak T$.   By 215B(iv) there is a
countable
$\Cal G_0\subseteq\Cal G$ such that $\mu(G\setminus\bigcup\Cal G_0)=0$
for every $G\in\Cal G$.   Show that
     
\Centerline{$\bigcup\Cal G\subseteq\{x:\limsup_{h\downarrow 0}
  \Bover1{2h}\mu(\bigcup\Cal G_0\cap[x-h,x+h])>0\}$,}
     
\noindent so that $\mu(\bigcup\Cal G\setminus\bigcup\Cal G_0)=0$.)   {(ii)} Show that a function
$f:\Bbb R\to\Bbb R$ is measurable iff it is $\frak T$-continuous at
almost every $x\in\Bbb R$.   ($\frak T$ is the {\bf density topology} on
$\Bbb R$.   See 414P in Volume 4.)
%223B
     
\spheader 223Yc Show that if $f:[a,b]\to\Bbb R$ is bounded and
continuous for the density topology on $\Bbb R$, then
$f(x)=\bover{d}{dx}\int_a^xf$ for every $x\in\ooint{a,b}$.
%223Yb 223B
     
\spheader 223Yd Show that a function $f:\Bbb R\to\Bbb R$ is continuous
for the density topology at $x\in\Bbb R$ iff 
$\lim_{h\downarrow 0}\bover1{2h}
  \mu^*\{y:y\in[x-h,x+h],\,|f(y)-f(x)|\ge\epsilon\}=0$ for
every $\epsilon>0$.
%223Yb 223B
     
\spheader 223Ye A set $A\subseteq\Bbb R$ is {\bf porous} at a point
$x\in\Bbb R$ if $\limsup_{y\to x}\Bover{\rho(y,A)}{|y-x|}>0$, where
$\rho(y,A)=\inf_{a\in A}|y-a|$.   (Take $\rho(y,\emptyset)=\infty$.)
Show that if $A$ is porous at every $x\in A$ then $A$ is negligible.
%223B
     
\spheader 223Yf For a measurable set $E\subseteq\Bbb R$ write
$\intstar E$ for the set of its density points.   Show that if $E$, 
$F\subseteq\Bbb R$ are measurable then (i)
$\intstar(E\cap F)=\intstar E\cap\intstar F$ 
(ii) $\intstar E\subseteq\intstar F$
iff $\mu(E\setminus F)=0$ (iii) $\mu(E\symmdiff\intstar E)=0$ (iv)
$\intstar(\intstar E)=\intstar E$ (v) for
every compact set $K\subseteq\intstar E$ there is a compact set
$L\subseteq K\cup E$ such that $K\subseteq\intstar L$.
%223B
     
\spheader 223Yg Let $f$ be an integrable real-valued function defined
almost everywhere in $\Bbb R$, and $x$ any point of the Lebesgue set of
$f$.   Show that for every $\epsilon>0$ there is a $\delta>0$ such that
$|f(x)\int g-\int f\times g|\le\epsilon\int g$ whenever
$g:\Bbb R\to\coint{0,\infty}$ is such that $g$ is non-decreasing on
$\ocint{-\infty,x}$, non-increasing on $\coint{x,\infty}$ and zero
outside $[x-\delta,x+\delta]$.   \Hint{express $g$ as a limit almost
everywhere of functions of the form
$\Bover{g(x)}{n+1}\sum_{i=0}^n\chi\ooint{a_i,b_i}$, where
$x-\delta\le a_0\le\ldots\le a_n\le x\le b_n\le\ldots\le b_0
\le x+\delta$.}
%223D

\spheader 223Yh For each integrable real-valued function $f$ defined almost
everywhere in $\Bbb R$, let $E_f$ be the Lebesgue set of $f$.   Show that
$E_f\cap E_g\subseteq E_{f+g}$, $E_f\subseteq E_{|f|}$ for all integrable
$f$, $g$.
%223D

\spheader 223Yi\dvAnew{2014} 
Let $E\subseteq\Bbb R$ be a non-negligible measurable set. 
Show that $0$ belongs to the interior of $E-E=\{x-y:x$, $y\in E\}$.
%223B
}%end of exercises
     
\endnotes{
\Notesheader{223} The results of this section can be
thought of as saying that a measurable function is in some sense 
`almost continuous';  indeed, 223Yb is an attempt to make this notion
precise.   For an integrable function we have stronger results, of which
the furthest-reaching seems to be 223D/223Ec.
     
There are $r$-dimensional versions of all these theorems, using balls
centered on $x$ in place of intervals $[x-h,x+h]$;  I give these in
261C-261E.   %261C 261D 261E
A new idea is needed for the $r$-dimensional version of
Lebesgue's density theorem (261C), but the rest of the
generalization is straightforward.   A less natural, and less important,
extension, also in \S261, involves functions defined on
non-measurable sets (compare 223Xb-223Xe).
}%end of comment
     
\discrpage

