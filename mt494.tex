\frfilename{mt494.tex}
\versiondate{17.5.13}

\copyrightdate{2009}

\def\glitem#1{{\it #1}\enskip}
\def\mag{\mathop{\text{mag}}\nolimits}

\def\chaptername{Further topics}
\def\sectionname{Groups of measure-preserving automorphisms}

\newsection{494\dvAnew{2009}}

I return to the study of automorphism groups of measure algebras,
as in Chapter 38 of Volume 3, but this time with the
intention of exploring possible topological group structures.
Two topologies in particular have attracted interest,
the `weak' and `uniform' topologies
(494A).   After a brief account of their basic properties
(494B-494C) %494B 494C
I begin work on the four main theorems.   The first is the Halmos-Rokhlin
theorem that if $(\frak A,\bar\mu)$ is the Lebesgue
probability algebra the set of weakly mixing measure-preserving
automorphisms of $\frak A$ which are not mixing is comeager for the weak
topology on $\AmuA$ (494E).   This depends on a striking characterization
of weakly mixing automorphisms of a probability algebra
in terms of eigenvectors of the corresponding operators on
the complex Hilbert space $L^2_{\Bbb C}(\frak A,\bar\mu)$ 
(494D\cmmnt{, 494Xj(i)}).   
It turns out that there is an elegant example of a weakly
mixing automorphism which is not mixing which can be described in terms of
a Gaussian distribution of the kind introduced in \S456, so I give it here
(494F).

We need a couple of preliminary results on fixed-point
subalgebras (494G-494H) before approaching the other three theorems.
If $(\frak A,\bar\mu)$ is an atomless
probability algebra, then $\AmuA$ is extremely amenable under its weak
topology (494L);  if $\AmuA$ is given its uniform topology, then every
group homomorphism from $\AmuA$ to a Polish group is continuous (494O);
finally, there is no strictly increasing sequence of subgroups with union
$\AmuA$ (494Q).   All these results have wide-ranging extensions to
full subgroups of $\AmuA$ subject to certain restrictions on the
fixed-point subalgebras.

\cmmnt{The work of this section will rely heavily on concepts and results
from
Volume 3 which have hardly been mentioned so far in the present volume.
I hope that the cross-references, and the brief remarks in 494Ac-494Ad,
will be adequate.}

\leader{494A}{Definitions}\cmmnt{ ({\smc Halmos 56})}
Let $(\frak A,\bar\mu)$ be a measure algebra,
and $\AmuA$ the group of measure-preserving automorphisms of
$\frak A$\cmmnt{ (see \S383)}.
Write $\frak A^f$ for $\{c:c\in\frak A$, $\bar\mu c<\infty\}$.

\spheader 494Aa I will say that the
{\bf weak topology} on $\AmuA$ is that generated by the pseudometrics
$(\pi,\phi)\mapsto\bar\mu(\pi c\Bsymmdiff\phi c)$
as $c$ runs over $\frak A^f$.
%Tornquist 11:  "weak topology"

\spheader 494Ab I will say that the {\bf uniform topology} on $\AmuA$
is that generated by the pseudometrics

\Centerline{$(\pi,\phi)
\mapsto\sup_{a\in\frak A}\bar\mu(c\Bcap(\pi a\Bsymmdiff\phi a))$}

\noindent as $c$ runs over $\frak A^f$.

\cmmnt{\spheader 494Ac I recall some notation from
Volume 3.   For any Boolean algebra $\frak A$ and $a\in\frak A$,
$\frak A_a$ will be the principal ideal
of $\frak A$ generated by $a$ (312D).
I will generally use the symbol $\iota$
for the identity in the automorphism group $\Aut\frak A$ of $\frak A$.
If $\pi\in\Aut\frak A$ and $a\in\frak A$, $a$ supports $\pi$ if $\pi d=d$
whenever $d\Bcap a=0$;  the support

\Centerline{$\supp\pi=\sup\{a\Bsymmdiff\pi a:a\in\frak A\}$}

\noindent of $\pi$ is the smallest member of $\frak A$ supporting $\pi$,
if this is defined (381Bb, 381Ei).   A subgroup $G$
of $\Aut\frak A$ is `full' if $\phi\in G$ whenever $\phi\in\Aut\frak A$
and there are $\familyiI{a_i}$, $\familyiI{\pi_i}$ such that
$\familyiI{a_i}$ is a partition of unity in $\frak A$ and $\pi_i\in G$ and
$\phi d=\pi_id$ whenever $i\in I$ and $d\Bsubseteq a_i$ (381Be).

If $a$, $b\in\frak A\setminus\{0\}$ are disjoint
and $\pi\in\Aut\frak A$ is such that
$\pi a=b$, then $\cycle{a\,_{\pi}\,b}\in\Aut\frak A$ will be the exchanging
involution defined by saying that

$$\eqalign{\cycle{a\,_{\pi}\,b}(d)
&=\pi d\text{ if }d\Bsubseteq a,\cr
&=\pi^{-1}d\text{ if }d\Bsubseteq b,\cr
&=d\text{ if }d\Bsubseteq 1\Bsetminus(a\Bcup b)\cr}$$

\noindent (381R).

\spheader 494Ad In addition, I will repeatedly use the following ideas.
Suppose that $(\frak A,\bar\mu)$ is a probability algebra (322Aa),
$\frak C$ is a closed subalgebra of $\frak A$ (323I), and
$L^{\infty}(\frak C)$ the $M$-space defined in \S363.   Then for each
$a\in\frak A$ we have a conditional expectation
$u_a\in L^{\infty}(\frak C)$ of $\chi a$ on $\frak C$, so that
$\int_cu_a=\bar\mu(a\Bcap c)$ for every $c\in\frak C$ (365R).

If $\frak A$ is relatively atomless over $\frak C$ (331A), $a\in\frak A$,
and $v\in L^{\infty}(\frak C)$ is such that $0\le v\le u_a$, there is a
$b\in\frak A$ such that $b\Bsubseteq a$
and $v=u_b$ (apply Maharam's lemma 331B
to the functional $c\mapsto\int_cv:\frak C\to[0,1]$).
Elaborating on this, we see that if $\sequencen{v_n}$ is a sequence in
$L^{\infty}(\frak C)^+$ and $\sum_{i=0}^nv_i\le u_a$ for every $n$,
there are disjoint $b_0,\ldots\Bsubseteq a$ such that $v_i=u_{b_i}$ for
every $i$ (choose the $b_i$ inductively).}

\leader{494B}{Proposition}
Let $(\frak A,\bar\mu)$ be a measure algebra, and give $\AmuA$ 
its weak topology.

(a) $\AmuA$ is a topological group.

(b) $(\pi,a)\mapsto\pi a:\AmuA\times\frak A\to\frak A$ is continuous for
the weak topology on $\AmuA$ and the measure-algebra topology on
$\frak A$.

(c) If $(\frak A,\bar\mu)$ is semi-finite\cmmnt{ (definition:  322Ad)},
$\AmuA$ is Hausdorff.

(d) If $(\frak A,\bar\mu)$ is localizable\cmmnt{ (definition:  322Ae)},
$\AmuA$ is complete under its bilateral uniformity.

(e)\dvAformerly{4{}93Xe}
If $(\frak A,\bar\mu)$ is $\sigma$-finite\cmmnt{ (definition:  322Ac)}
and $\frak A$ has countable
Maharam type\cmmnt{ (definition:  331Fa)}, then $\AmuA$ is a Polish group.

\proof{{\bf (a)}  (Compare 441G.)
Set $\rho_c(\pi,\phi)=\bar\mu(\pi c\Bsymmdiff\phi c)$ for
$\pi$, $\phi\in\AmuA$ and $c\in\frak A^f$;  it is elementary to check that
$\rho_c$ is always a pseudometric, so 494Aa is a proper definition of a
topology.   If $\pi$, $\phi\in\AmuA$ and $c\in\frak A^f$, then for any
$\pi'$, $\phi'\in\AmuA$ we have

$$\eqalign{\rho_c(\pi'\phi',\pi\phi)
&=\bar\mu(\pi'\phi'c\Bsymmdiff\pi\phi c)
\le\bar\mu(\pi'\phi'c\Bsymmdiff\pi'\phi c)
  +\bar\mu(\pi'\phi c\Bsymmdiff\pi\phi c)\cr
&=\bar\mu(\phi'c\Bsymmdiff\phi c)
  +\rho_{\phi c}(\pi',\pi)
=\rho_c(\phi',\phi)+\rho_{\phi c}(\pi',\pi);\cr}$$

\noindent as $c$ is arbitrary, $(\pi',\phi')\mapsto\pi'\phi'$ is continuous
at $(\pi,\phi)$;  thus multiplication is continuous.
If $\phi\in\AmuA$ and $c\in\frak A^f$, then for any $\pi\in\AmuA$

$$\eqalign{\rho_c(\pi^{-1},\phi^{-1})
&=\bar\mu(\pi^{-1}c\Bsymmdiff\phi^{-1}c)
=\bar\mu(c\Bsymmdiff\pi\phi^{-1}c)\cr
&=\bar\mu(\phi\phi^{-1}c\Bsymmdiff\pi\phi^{-1}c)
=\rho_{\phi^{-1}c}(\pi,\phi);\cr}$$

\noindent as $c$ is arbitrary, $\pi\mapsto\pi^{-1}$ is continuous at
$\phi$;  thus inversion is continuous and $\AmuA$ is a topological
group.

\medskip

{\bf (b)} Suppose that $\phi\in\AmuA$, $b\in\frak A$ and that $V$ is a
neighbourhood of $\phi b$ in $\frak A$.   Then there are  $c\in\frak A^f$ and
$\epsilon>0$ such that $V$ includes
$\{d:\bar\mu(c\Bcap(d\Bsymmdiff\phi b))\le 4\epsilon\}$.   In this case,
because inversion in $\AmuA$ is continuous,

\Centerline{$U=\{\pi:\pi\in\AmuA$,
$\bar\mu(\pi^{-1}c\Bsymmdiff\phi^{-1}c)\le\epsilon$,
$\bar\mu(\pi(\phi^{-1}c\Bcap b)
   \Bsymmdiff\phi(\phi^{-1}c\Bcap b))\le\epsilon\}$,}

\Centerline{$V'=\{a:a\in\frak A$,
$\bar\mu(\phi^{-1}c\Bcap(a\Bsymmdiff b))\le\epsilon\}$}

\noindent are neighbourhoods of $\phi$, $b$ respectively.   If
$\pi\in U$ and $a\in V'$, then

$$\eqalign{\bar\mu(c\Bcap(\pi a\Bsymmdiff\phi b))
&\le\bar\mu(c\Bcap(\pi a\Bsymmdiff\pi b))
    +\bar\mu(c\Bcap(\pi b\Bsymmdiff\phi b))\cr
&=\bar\mu(\pi^{-1}c\Bcap(a\Bsymmdiff b))
    +\bar\mu((c\Bcap\pi b)\Bsymmdiff(c\Bcap\phi b))\cr
&\le\bar\mu(\pi^{-1}c\Bsymmdiff\phi^{-1}c)
    +\bar\mu(\phi^{-1}c\Bcap(a\Bsymmdiff b))
    +\bar\mu(\pi(\pi^{-1}c\Bcap b)\Bsymmdiff\phi(\phi^{-1}c\Bcap b))\cr
&\le\epsilon+\epsilon
    +\bar\mu(\pi(\pi^{-1}c\Bcap b)\Bsymmdiff\pi(\phi^{-1}c\Bcap b))\cr
&\mskip100mu
    +\bar\mu(\pi(\phi^{-1}c\Bcap b)\Bsymmdiff\phi(\phi^{-1}c\Bcap b))\cr
&\le 2\epsilon
    +\bar\mu((\pi^{-1}c\Bcap b)\Bsymmdiff(\phi^{-1}c\Bcap b))
    +\epsilon\cr
&\le 3\epsilon
    +\bar\mu(\pi^{-1}c\Bsymmdiff\phi^{-1}c)
\le 4\epsilon,\cr}$$

\noindent and $\pi a\in V$.   As $V$, $\phi$ and $b$ are arbitrary,
$(\pi,a)\mapsto\pi a$ is continuous.

\medskip

{\bf (c)} Because $(\frak A,\bar\mu)$ is semi-finite, the measure-algebra
topology on $\frak A$ is Hausdorff (323Ga), so the product topology on
$\frak A^{\frak A}$ is Hausdorff.   Now
$\pi\mapsto\family{a}{\frak A}{\pi a}:\AmuA\to\frak A^{\frak A}$ is
injective, and by (b) it is continuous, so the topology of $\AmuA$ must be
Hausdorff.

\medskip

{\bf (d)(i)} For $c\in\frak A^f$ and $\pi\in\AmuA$, set
$\theta_c(\pi)=\pi c$;  then $\theta_c:\AmuA\to\frak A^f$
is uniformly continuous for the bilateral uniformity of $\AmuA$ and the
measure metric $\rho$ of $\frak A^f$ (323Ad).   \Prf\ We have
$\rho(d,d')=\bar\mu(d\Bsymmdiff d')$ for $d$, $d'\in\frak A^f$.
Let $\epsilon>0$;  then $U=\{\pi:\rho_c(\pi,\iota)\le\epsilon\}$ is a
neighbourhood of $\iota$ in $\AmuA$,
and $W=\{(\pi,\phi):\phi^{-1}\pi\in U\}$ belongs
to the bilateral uniformity.   If $(\pi,\phi)\in W$, then

\Centerline{$\rho(\theta_c(\pi),\theta_c(\phi))
=\bar\mu(\pi c\Bsymmdiff\phi c)
=\bar\mu(\phi^{-1}\pi c\Bsymmdiff c)
=\rho_c(\phi^{-1}\pi,\iota)\le\epsilon$;}

\noindent as $\epsilon$ is arbitrary, $\theta_c$ is uniformly continuous.\
\Qed

\medskip

\quad{\bf (ii)} Let $\Cal F$ be a filter on $\AmuA$ which is Cauchy for the
bilateral uniformity on $\AmuA$.   If $c\in\frak A^f$, the image filter
$\theta_c[[\Cal F]]$ is Cauchy for the measure metric on
$\frak A^f$ (4A2Ji).   Because
$\frak A^f$ is complete under its measure metric
(323Mc), $\theta_c[[\Cal F]]$
converges to $\psi_0c$ say for the measure metric.

If $c$, $d\in\frak A^f$ and $*$ is either of the
Boolean operations $\Bcap$, $\Bsymmdiff$, then

$$\eqalignno{\psi_0(c*d)
&=\lim_{\pi\to\Cal F}\pi(c*d)
=\lim_{\pi\to\Cal F}\pi c*\pi d
=\lim_{\pi\to\Cal F}\pi c*\lim_{\pi\to\Cal F}\pi d\cr
\displaycause{because $*$ is continuous for the measure metric,
see 323Ma}
&=\psi_0c*\psi_0d.\cr}$$

\noindent So $\psi_0:\frak A^f\to\frak A^f$ is a
ring homomorphism.   Next, if $c\in\frak A^f$, then

\Centerline{$\bar\mu\psi_0c
=\bar\mu(\lim_{\pi\to\Cal F}\pi c)
=\lim_{\pi\to\Cal F}\bar\mu\pi c
=\bar\mu c$}

\noindent because $\bar\mu:\frak A^f\to\coint{0,\infty}$ is continuous
(323Mb).

Now recall that $\pi\mapsto\pi^{-1}:\AmuA\to\AmuA$ is uniformly continuous
for the bilateral uniformity (4A5Hc).   So if we set
$\theta'_c(\pi)=\pi^{-1}c$ for $\pi\in\AmuA$, we can apply the argument
just above to $\theta'$ to find a ring homomorphism
$\psi_0':\frak A^f\to\frak A^f$ such that
$\psi_0'c=\lim_{\pi\to\Cal F}\pi^{-1}c$
for every $c\in\frak A^f$.   To relate $\psi_0$
and $\psi_0'$, we can argue as follows.   Given $c\in\frak A^f$,

$$\eqalign{\bar\mu(c\Bsymmdiff\psi_0\psi'_0c)
&=\bar\mu(c\Bsymmdiff\lim_{\phi\to\Cal F}\phi\psi'_0c)
=\lim_{\phi\to\Cal F}\bar\mu(c\Bsymmdiff\phi\psi'_0c)
=\lim_{\phi\to\Cal F}\bar\mu(\phi^{-1}c\Bsymmdiff\psi'_0c)\cr
&=\bar\mu(\lim_{\phi\to\Cal F}(\phi^{-1}c\Bsymmdiff\psi'_0c))
=\bar\mu((\lim_{\phi\to\Cal F}\phi^{-1}c)\Bsymmdiff\psi'_0c)
=\bar\mu(\psi'_0c\Bsymmdiff\psi'_0c)
=0;\cr}$$

\noindent as $c$ is arbitrary, $\psi_0\psi'_0$ is the identity on
$\frak A^f$.   Similarly, $\psi_0'\psi_0$ is the
identity on $\frak A^f$.   Thus $\psi_0$, $\psi'_0$ are the two halves of a
measure-preserving ring isomorphism of $\frak A^f$.

If we give $\frak A$ its measure-algebra uniformity (323Ab), then
$\psi_0$ is uniformly continuous for the induced uniformity on $\frak A^f$.
\Prf\ If $c$, $d_1$, $d_2\in\frak A^f$, then

\Centerline{$\bar\mu(c\Bcap(\psi_0d_1\Bsymmdiff\psi_0d_2))
=\bar\mu(\psi_0^{-1}c\Bcap(d_1\Bsymmdiff d_2))$.\ \Qed}

\noindent Since $\frak A^f$ is dense in $\frak A$ for the measure-algebra
topology on $\frak A$ (323Bb), and $\frak A$ is complete
for the measure-algebra uniformity (323Gc), there is a unique
extension of $\psi_0$ to a uniformly continuous function
$\psi:\frak A\to\frak A$ (3A4G).   Since the Boolean operations
$\Bsymmdiff$, $\Bcap$ on $\frak A$ are continuous for the measure-algebra
topology (323Ba), $\psi$ is a ring homomorphism.   Similarly, we have a
unique continuous $\psi':\frak A\to\frak A$ extending $\psi'_0$;  since
$\psi\psi'$ and $\psi'\psi$ are continuous functions agreeing with the
identity operator $\iota$ on $\frak A^f$, they are both $\iota$, and
$\psi\in\Aut\frak A$.   To see that $\psi$ is measure-preserving, note
just that if $a\in\frak A$ then

$$\eqalignno{\bar\mu\psi a
&=\sup\{\bar\mu c:c\in\frak A^f,\,c\Bsubseteq\psi a\}
=\sup\{\bar\mu\psi_0c:c\in\frak A^f,\,\psi_0c\Bsubseteq\psi a\}\cr
\displaycause{because $\psi_0$ is a permutation of $\frak A^f$}
&=\sup\{\bar\mu c:c\in\frak A^f,\,\psi c\Bsubseteq\psi a\}
=\sup\{\bar\mu c:c\in\frak A^f,\,c\Bsubseteq a\}
=\bar\mu a.\cr}$$

\noindent Thus $\psi\in\AmuA$.

Finally, $\Cal F\to\psi$.   \Prf\ If $c\in\frak A^f$ and $\epsilon>0$,
there is an $F\in\Cal F$ such that
$\bar\mu(\pi c\Bsymmdiff\phi c)\le\epsilon$ whenever $\pi$, $\phi\in F$.
We have

\Centerline{$\bar\mu(\pi c\Bsymmdiff\psi c)
=\bar\mu(\pi c\Bsymmdiff\lim_{\phi\to\Cal F}\phi c)
=\lim_{\phi\to\Cal F}\bar\mu(\pi c\Bsymmdiff\phi c)
\le\epsilon$}

\noindent for every $\pi\in F$.   As $c$ and $\epsilon$ are arbitrary,
$\Cal F\to\psi$ for the weak topology on $\AmuA$.\ \QeD\   As $\Cal F$ is
arbitrary, the bilateral uniformity is complete.

\medskip

{\bf (e)(i)} The point is that $\frak A^f$ is separable for the measure
metric.   \Prf\ Because $\frak A$ has countable Maharam type, there is a
countable subalgebra $\frak B$ of $\frak A$ which $\tau$-generates
$\frak A$;  by 323J, $\frak B$ is dense in $\frak A$ for the measure
algebra topology.   Next, there is a non-decreasing sequence
$\sequencen{c_n}$ in $\frak A$ with supremum $1$.   Set
$D=\{b\Bcap c_n:b\in\frak B$, $n\in\Bbb N\}$.   Then $D$ is a countable
subset of $\frak A^f$.   If $c\in\frak A^f$ and $\epsilon>0$, there are an
$n\in\Bbb N$ such that $\bar\mu(c\Bsetminus c_n)\le\epsilon$, and a
$b\in\frak B$ such that $\bar\mu(c_n\Bcap(c\Bsymmdiff b))\le\epsilon$.
Now $d=b\Bcap c_n$ belongs to $D$, and

$$\eqalign{\bar\mu(c\Bsymmdiff d)
&\le\bar\mu(c\Bsymmdiff(c\Bcap c_n))
   +\bar\mu((c\Bcap c_n)\Bsymmdiff(b\Bcap c_n))\cr
&=\bar\mu(c\Bsetminus c_n)
   +\bar\mu(c_n\Bcap(c\Bsymmdiff b))
\le 2\epsilon.\cr}$$

\noindent As $c$ and $\epsilon$ are arbitrary, $D$ is dense in $\frak A^f$
and $\frak A^f$ is separable.\ \Qed

\medskip

%regular Hausdorff second-countable > separable & metrizable 4{}A2Pb
%countable subbase > second-countable 4{}A2Oa

\quad{\bf (ii)} Let $D$ be a countable dense subset of $\frak A^f$, and
$\Cal U$ the family of sets of the form

\Centerline{$\{\pi:\pi\in\AmuA$, $\bar\mu(d\Bsymmdiff\pi d')<2^{-n}\}$}

\noindent where $d$, $d'\in D$ and $n\in\Bbb N$.   All these sets are open
for the weak topology.
\Prf\ If $U=\{\pi:\bar\mu(d\Bsymmdiff\pi d')<2^{-n}\}$ and $\phi\in U$,
set $\eta=\bover13(2^{-n}-\bar\mu(d\Bsymmdiff\phi d'))$.   Then
$V=\{\pi:\bar\mu(d\Bcap(\pi d'\Bsymmdiff\phi d'))\le\eta\}$ is a
neighbourhood of $\phi$.   If $\pi\in V$, then

\Centerline{$\bar\mu(d\Bsymmdiff\pi d')
\le\bar\mu(d\Bsymmdiff\phi d')+\bar\mu(\phi d'\Bsymmdiff\pi d')
<2^{-n}$}

\noindent and $\pi\in U$.   Thus $\phi\in\interior U$;  as $\phi$ is
arbitrary, $U$ is open.\ \Qed

\medskip

\quad{\bf (iii)} In fact $\Cal U$ is a subbase for the weak topology on
$\AmuA$.   \Prf\ If $W\subseteq\AmuA$ is open and $\phi\in W$, there are
$c_0,\ldots,c_n\in\frak A^f$ and $k\in\Bbb N$
such that $W$ includes
$\{\pi:\bar\mu(\pi c_i\Bsymmdiff\phi c_i)\le 2^{-k}$ for every
$i\le n\}$.   Let $d_0,\ldots,d_n,d_0',\ldots,d'_n\in D$ be such that
$\bar\mu(d_i\Bsymmdiff c_i)<2^{-k-2}$,
$\bar\mu(d'_i\Bsymmdiff\phi c_i)<2^{-k-2}$ for each $i\le n$.   Set
$U_i=\{\pi:\bar\mu(d'_i\Bsymmdiff\pi d_i)<2^{-k-1}\}$;  then
$U_i\in\Cal U$ and $\phi\in U_i$ for each $i\le n$, because

\Centerline{$\bar\mu(d'_i\Bsymmdiff\phi d_i)
\le\bar\mu(d'_i\Bsymmdiff\phi c_i)+\bar\mu(\phi c_i\Bsymmdiff\phi d_i)
=\bar\mu(d'_i\Bsymmdiff\phi c_i)+\bar\mu(c_i\Bsymmdiff d_i)
<2^{-k-1}$.}

\noindent If $\pi\in U_i$, then

\Centerline{$\bar\mu(\pi c_i\Bsymmdiff\phi c_i)
\le\bar\mu(\pi c_i\Bsymmdiff\pi d_i)+\bar\mu(\pi d_i\Bsymmdiff d'_i)
   +\bar\mu(d'_i\Bsymmdiff\phi c_i)
\le 2^{-k}$,}

\noindent so $\bigcap_{i\le n}U_i\subseteq W$.   As $W$ and $\phi$ are
arbitrary, $\Cal U$ is a subbase for the topology.\ \Qed

\medskip

\quad{\bf (iv)} Since $\Cal U$ is countable, the weak topology is
second-countable (4A2Oa).   Since the weak topology is a group
topology, it is regular (4A5Ha, or otherwise);  by (c) above it is
Hausdorff;  so by 4A2Pb it is separable and metrizable.   Accordingly the
bilateral uniformity is metrizable (4A5Q(v));  by (d) above, $\AmuA$ is
complete under the bilateral uniformity, so its topology is Polish.
}%end of proof of 494B

\leader{494C}{Proposition}
Let $(\frak A,\bar\mu)$ be a measure algebra, and give
$\AmuA$ its uniform topology.

(a) $\AmuA$ is a topological group.

(b) For $c\in\frak A^f$ and $\epsilon>0$, set

\Centerline{$U(c,\epsilon)=\{\pi:\pi\in\AmuA$, $\pi$ is supported by an
$a\in\frak A$ such that $\bar\mu(c\Bcap a)\le\epsilon\}$.}

\noindent Then $\{U(c,\epsilon):c\in\frak A^f$, $\epsilon>0\}$ is a base of
neighbourhoods of $\iota$.

(c) The set of
periodic measure-preserving automorphisms of $\frak A$ with supports of
finite measure is dense in $\AmuA$.

(d) The weak topology on $\AmuA$ is coarser than the uniform
topology.

(e) If $(\frak A,\bar\mu)$ is semi-finite, $\AmuA$ is Hausdorff.

(f) If $(\frak A,\bar\mu)$ is localizable,
$\AmuA$ is complete under its bilateral uniformity.

(g) If $(\frak A,\bar\mu)$ is semi-finite and
$G$ is a full subgroup of $\AmuA$,
then $G$ is closed.

(h) If $(\frak A,\bar\mu)$ is $\sigma$-finite, then $\AmuA$ is
metrizable.

(i) Suppose that $(\frak A,\bar\mu)$ is $\sigma$-finite and $\frak A$ has
countable Maharam type.   If $D\subseteq\AmuA$ is countable, then the full
subgroup $G$ of $\AmuA$ generated by $D$, with its induced topology,
is a Polish group.

\proof{{\bf (a)} For $\pi$, $\phi\in\AmuA$ and $c\in\frak A^f$, set
$\rho'_c(\pi,\phi)
=\sup_{a\in\frak A}\bar\mu(c\Bcap(\pi a\Bsymmdiff\phi a))$;  as in part (a)
of the proof of 494B, it is
elementary that every $\rho'_c$ is a pseudometric, so the uniform topology
$\frak T_u$ is properly defined.
If $\pi$, $\phi$, $\pi'$, $\phi'\in\AmuA$,
$c\in\frak A^f$ and $a\in\frak A$, then

$$\eqalign{\bar\mu(c\Bcap(\pi'\phi'a\Bsymmdiff\pi\phi a))
&\le\bar\mu(c\Bcap(\pi'\phi'a\Bsymmdiff\pi\phi'a))
  +\bar\mu(c\Bcap(\pi\phi'a\Bsymmdiff\pi\phi a))\cr
&\le\rho'_c(\pi',\pi)
  +\bar\mu(\pi^{-1}c\Bcap(\phi'a\Bsymmdiff\phi a))\cr
&\le\rho'_c(\pi',\pi)
  +\rho'_{\pi^{-1}c}(\phi',\phi);\cr}$$

\noindent as $a$ is arbitrary,
$\rho'_c(\pi'\phi',\pi\phi)
\le\rho'_c(\pi',\pi)+\rho'_{\pi^{-1}c}(\phi',\phi)$;  as $c$ is arbitrary,
$(\pi',\phi')\mapsto\pi'\phi'$ is continuous at $(\pi,\phi)$;  thus
multiplication is continuous.   If $\pi$, $\phi\in\AmuA$, $c\in\frak A^f$
and $a\in\frak A$, then

\Centerline{$\bar\mu(c\Bcap(\pi^{-1}a\Bsymmdiff\phi^{-1}a))
=\bar\mu(\phi c\Bcap(\phi\pi^{-1}a\Bsymmdiff\pi\pi^{-1}a)
\le\rho'_{\phi c}(\phi,\pi)$;}

\noindent thus $\rho'_c(\pi^{-1},\phi^{-1})\le\rho'_{\phi c}(\pi,\phi)$,
$\pi\mapsto\pi^{-1}$ is continuous at $\phi$, and inversion is continuous.
So once more we have a topological group.

\medskip

{\bf (b)(i)} If $c\in\frak A^f$, $\pi\in\AmuA$ and $\epsilon>0$ are such
that $\rho'_c(\pi,\iota)\le\bover13\epsilon$, then $\pi\in U(c,\epsilon)$.
\Prf\ Consider
$A=\{a:a\in\frak A_c$, $a\Bcap\pi a=0\}$.   Then

\Centerline{$\bar\mu a\le\bar\mu(c\Bcap(\pi a\Bsymmdiff a))
\le\rho'_c(\pi,\iota)\le\Bover13\epsilon$}

\noindent for
every $a\in A$.   If $B\subseteq A$ is upwards-directed, then
$b^*=\sup B$ is defined in $\frak A$, and
$\bar\mu b^*=\sup_{b\in B}\bar\mu b$ (321C).   Now
$\pi b^*=\sup_{b\in B}\pi b$, so
$b^*\Bcap\pi b^*=\sup_{b\in B}b\Bcap\pi b=0$, and $b^*\in A$.
By Zorn's Lemma, $A$ has a maximal element $a^*$.
Suppose that $d\in\frak A_c$ is
disjoint from $\pi^{-1}a^*\Bcup a^*\Bcup\pi a^*$.   Then
$a^*\Bcup(d\Bsetminus\pi d)\in A$;  by the maximality of $a^*$,
$d\Bsubseteq\pi d$ and $d=\pi d$ (because $\bar\mu d=\bar\mu\pi d<\infty$).
Thus
$(1\Bsetminus c)\Bcup(\pi^{-1}a^*\Bcup a^*\Bcup\pi a^*)$ supports $\pi$
and witnesses that $\pi\in U(c,\epsilon)$.\ \Qed

So every $U(c,\epsilon)$ is a $\frak T_u$-neighbourhood of $\iota$.

\medskip

\quad{\bf (ii)} Conversely, if $c\in\frak A^f$, $\epsilon>0$ and
$\pi\in U(c,\epsilon)$, then $\rho'_c(\pi,\iota)\le\epsilon$.   \Prf\
Let $d\in\frak A$ be such that $\pi$ is supported by $d$ and
$\bar\mu(c\Bcap d)\le\epsilon$.   Then, for any $a\in\frak A$,
$a\Bsymmdiff\pi a\Bsubseteq d$, so
$\bar\mu(c\Bcap(a\Bsymmdiff\pi a))\le\epsilon$;  which is what we need to
know.\ \Qed

So $\{U(c,\epsilon):c\in\frak A^f$, $\epsilon>0\}$ is a base of
neighbourhoods of $\iota$ for $\frak T_u$.

\medskip

{\bf (c)} Take a non-empty open subset $U$ of $\AmuA$ and
$\phi\in U$.

\medskip

\quad{\bf (i)} By (b), there are a $c\in\frak A^f$ and an $\epsilon>0$
such that $U(c,3\epsilon)U(c,3\epsilon)\subseteq U^{-1}\phi$.   Now there
is a $\psi\in\AmuA$ such that $\psi^{-1}\phi\in U(c,3\epsilon)$
and $\psi$ is supported by $e=c\Bcup\phi c$.
\Prf\ By 332L, applied to $\frak A_e$ and $\phi\restrp\frak A_c$,
there is a measure-preserving automorphism $\psi_0:\frak A_e\to\frak A_e$
agreeing with $\phi$ on $\frak A_c$;  now set

\Centerline{$\psi a=\psi_0(a\Bcap e)\Bcup(a\Bsetminus e)$}

\noindent for every $a\in\frak A$ to get $\psi\in\AmuA$ agreeing with
$\phi$ on $\frak A_c$ and supported by $e$.   As
$\psi^{-1}\phi a=a$ for $a\Bsubseteq c$, $\psi^{-1}\phi$ is supported by
$1\Bsetminus c$ and belongs to $U(c,3\epsilon)$.\ \Qed

\medskip

\quad{\bf (ii)} By 381H, applied to $\psi\restrp\frak A_e$,
there is a partition
$\langle c_m\rangle_{1\le m\le\omega}$ of unity in $\frak A_e$ such that
$\psi c_m\Bsubseteq c_m$ for every $m$, $\psi\restrp\frak A_{c_m}$ is
periodic with period $m$ for every $m\in\Bbb N\setminus\{0\}$, and
$\psi\restrp\frak A_{c_{\omega}}$ is aperiodic.   Of course $\psi c_m=c_m$
for every $m$, just because $\bar\mu\psi c_m=\bar\mu c_m$.
Let $n\ge 1$ be such that $\bar\mu c_{\omega}\le n!\epsilon$ and
$\bar\mu(\sup_{n<m<\omega}c_m)\le\epsilon$.
By the Halmos-Rokhlin-Kakutani lemma (386C), applied to
$\psi\restrp\frak A_{c_{\omega}}$, there is a $b\Bsubseteq c_{\omega}$
such that $b$, $\psi b,\ldots,\psi^{n!-1}b$ are disjoint and
$\bar\mu(c_{\omega}\Bsetminus\sup_{i<n!}\psi^ib)\le\epsilon$.
Note that $\bar\mu b$ is also at most $\epsilon$.

Set

\Centerline{$d=\sup_{n<m<\omega}c_m
  \Bcup(c_{\omega}\Bsetminus\sup_{i<n!}\psi^ib)$,
\quad$d'=d\Bcup\psi^{n!-1}b$,}

\noindent and let $\pi:\frak A\to\frak A$ be the measure-preserving Boolean
automorphism such that

$$\eqalign{\pi a
&=\psi a\text{ if }1\le m\le n\text{ and }a\Bsubseteq c_m,\cr
&=\psi a\text{ if }0\le i\le n!-2\text{ and }a\Bsubseteq\psi^ib,\cr
&=\psi^{-n!+1}a\text{ if }a\Bsubseteq\psi^{n!-1}b,\cr
&=a\text{ if }a\Bsubseteq d\Bcup(1\Bsetminus e).\cr}$$

\noindent Then

$$\eqalign{\pi^{n!}a
&=\psi^{n!}a=a\text{ if }1\le m\le n\text{ and }a\Bsubseteq c_m,\cr
&=\psi^{n!}\psi^{-n!}a=a
   \text{ if }0\le i<n!\text{ and }a\Bsubseteq\psi^ib,\cr
&=a\text{ if }a\Bsubseteq d\cup(1\Bsetminus e),\cr}$$

\noindent so $\pi^{n!}=\iota$ and $\pi$ is periodic.
Since $\pi$ is supported by $e$, and $\frak A_e$ is Dedekind complete,
$\pi$ has a support of finite measure.   On the other hand
$\pi a=\psi a$ whenever $a\Bcap d'=0$, so
$\pi^{-1}\psi$ is supported by $d'$ and
belongs to $U(c,\bar\mu d')\subseteq U(c,3\epsilon)$.

Now

\Centerline{$\pi^{-1}\phi
=\pi^{-1}\psi\psi^{-1}\phi
\in U(c,3\epsilon)U(c,3\epsilon)
\subseteq U^{-1}\phi$,
\quad$\pi\in U$;}

\noindent as $U$ is arbitrary,
the set of periodic automorphisms with supports of finite measure
is dense in $\AmuA$.

\medskip

{\bf (d)}
Let $V$ be a neighbourhood of the identity $\iota$ for the weak topology
$\frak T_w$ on $\AmuA$.   Then there are $c_0,\ldots,c_k\in\frak A^f$
and $\epsilon_0,\ldots,\epsilon_k>0$ such that

\Centerline{$V\supseteq\{\pi:
  \bar\mu(c_i\Bsymmdiff\pi c_i)\le\epsilon_i$ for every $i\le k\}$.}

\noindent Set $c=\sup_{i\le k}c_i$,
$\epsilon=\bover12\min_{i\le k}\epsilon_i$.   If $\pi\in U(c,\epsilon)$ as
defined in (b), there is an $a\in\frak A$, supporting $\pi$, such that
$\bar\mu(c\Bcap a)\le\epsilon$.   In this case, for each $i\le k$,
$c_i\Bsetminus\pi c_i\Bsubseteq c\Bcap a$, so

\Centerline{$\bar\mu(c_i\Bsymmdiff\pi c_i)
=2\bar\mu(c_i\Bsetminus\pi c_i)
\le 2\bar\mu(c\Bcap a)
\le\epsilon_i$.}

\noindent Thus $V\supseteq U(c,\epsilon)$ and $V$ is a neighbourhood of
$\iota$ for $\frak T_u$.   As $\AmuA$ is a topological group
under either topology, it follows that
$\frak T_u$ is finer than $\frak T_w$ (4A5Fb).

\medskip

{\bf (e)} Because $(\frak A,\bar\mu)$ is semi-finite,
the weak topology is Hausdorff (494Bc), so the uniform topology,
being finer, must also be Hausdorff.

\medskip

{\bf (f)} Let $\Cal F$ be a Cauchy filter for the
$\frak T_u$-bilateral uniformity on $\AmuA$.   Because the identity map
from $(\AmuA,\frak T_u)$ to $(\AmuA,\frak T_w)$ is continuous ((d) above),
it is uniformly continuous for the corresponding bilateral uniformities
(4A5Hd), and $\Cal F$ is Cauchy for the $\frak T_w$-bilateral
uniformity (4A2Ji).   It follows that $\Cal F$ has a $\frak T_w$-limit
$\psi$ say (494Bd), in which case $\psi a$ is the limit
$\lim_{\pi\to\Cal F}\pi a$, for the measure-algebra topology of $\frak A$,
for every $a\in\frak A$ (494Bb).
But $\psi$ is also the $\frak T_u$-limit of $\Cal F$.
\Prf\ Suppose that $c\in\frak A^f$ and $\epsilon>0$.
Set $V(c,\epsilon)=\{\pi:\rho'_c(\pi,\iota)\le\epsilon\}$, where
$\rho'_c$ is defined as (a) above.   Then
$V(c,\epsilon)$ is a $\frak T_u$-neighbourhood of
$\iota$, so $\{(\pi,\phi):\phi\pi^{-1}\in V(c,\epsilon)\}$ belongs to the
$\frak T_u$-bilateral uniformity, and there is an $F\in\Cal F$ such that
$\phi\pi^{-1}\in V(c,\epsilon)$ whenever $\pi$, $\phi\in F$.

Now if $\phi\in F$ and $a\in\frak A$,

$$\eqalignno{\bar\mu(c\Bcap(\phi a\Bsymmdiff\psi a))
&=\bar\mu(c\Bcap(\phi a\Bsymmdiff\lim_{\pi\to\Cal F}\pi a))
=\lim_{\pi\to\Cal F}\bar\mu(c\Bcap(\phi a\Bsymmdiff\pi a))\cr
\displaycause{because $b\mapsto\bar\mu(c\Bcap(\phi a\Bsymmdiff b))$ is
continuous}
&=\lim_{\pi\to\Cal F}\bar\mu(c\Bcap(\phi\pi^{-1}\pi a\Bsymmdiff\pi a))
\le\sup_{\pi\in F,b\in\frak A}\bar\mu(c\Bcap(\phi\pi^{-1}b\Bsymmdiff b))
\le\epsilon.\cr}$$

\noindent Thus $\rho'_c(\phi,\psi)\le\epsilon$ for every $\phi\in F$.
As $c$ and $\epsilon$ are arbitrary, $\Cal F$ is $\frak T_u$-convergent to
$\psi$.\ \Qed

As $\Cal F$ is arbitrary, $\AmuA$ is complete for the $\frak T_u$-bilateral
uniformity.

\medskip

{\bf (g)(i)} Suppose that $\phi$ belongs to the closure of $G$ in $\AmuA$.
Let $B$ be the set of those $b\in\frak A$ for which there is a $\pi\in G$
such that $\pi$ and $\phi$ agree on the principal ideal $\frak A_b$.
Then $B$ is order-dense
in $\frak A$.   \Prf\ Suppose that $a\in\frak A\setminus\{0\}$.
Because $(\frak A,\bar\mu)$ is semi-finite, there is a non-zero
$c\in\frak A^f$ such that $c\Bsubseteq a$.   Take
$\epsilon\in\ooint{0,\bar\mu c}$.   Then there is a $\pi\in G$ such that
$\pi^{-1}\phi\in U(c,\epsilon)$.   Let $d\in\frak A$ be such that $d$
supports $\pi^{-1}\phi$ and $\bar\mu(c\Bcap d)\le\epsilon$.   Set
$b=c\Bsetminus d$.   If $b'\Bsubseteq b$, then
$\pi^{-1}\phi b'=b'$, that is, $\phi b'=\pi b'$;  so $\pi$ and $\phi$ agree
on $\frak A_b$ and $b\in B$, while $0\ne b\Bsubseteq a$.\ \Qed

\medskip

\quad{\bf (ii)} There is therefore a partition $\familyiI{b_i}$ of unity
consisting of members of $B$.   For each $i\in I$ take $\pi_i\in G$ such
that $\pi_i$ and $\phi$ agree on $\frak A_{b_i}$;  because $G$ is full,
$\familyiI{(b_i,\pi_i)}$ witnesses that $\phi\in G$.   As $\phi$ is
arbitrary, $G$ is closed.

\medskip

{\bf (h)} Let $\sequencen{c_n}$ be a non-decreasing sequence in $\frak A^f$
with supremum $1$.   Then $\{U(c_n,2^{-n}):n\in\Bbb N\}$ is a base of
neighbourhoods of $\iota$.   \Prf\ If $c\in\frak A^f$
and $\epsilon>0$, there is an $n\in\Bbb N$ such that
$\bar\mu(c\Bsetminus c_n)+2^{-n}\le\epsilon$.   If $\pi\in U(c_n,2^{-n})$,
there is an $a\in\frak A$, supporting $\pi$, such that
$\bar\mu(c_n\Bcap a)\le 2^{-n}$;  in which case
$\bar\mu(c\Bcap a)\le\epsilon$ and $\pi\in U(c,\epsilon)$.   Thus we have
found an $n$ such that $U(c_n,2^{-n})\subseteq U(c,\epsilon)$.\ \Qed

By 4A5Q, $\AmuA$ is metrizable.

\medskip

{\bf (i)}\grheada\ By (h), $\AmuA$ and therefore $G$ are metrizable;  the
bilateral uniformity of $\AmuA$ is therefore metrizable (4A5Q(v)).
By (f), $\AmuA$ is complete under its bilateral uniformity;  by
(g), $G$ is closed, so is complete under the induced uniformity.   So
there is a metric on $G$, inducing its topology, under which $G$ is
complete, and all I have to show is that $G$ is separable.

\medskip

\quad\grheadb\ Since the subgroup of $\AmuA$ generated by $D$ is again
countable, we may suppose that $D$ is itself a subgroup of $\AmuA$.
Let $\sequencen{c_n}$ be a non-decreasing sequence in $\frak A^f$
with supremum $1$, and $\frak B$ a countable subalgebra of $\frak A$,
which $\tau$-generates $\frak A$;  by 323J again,
$\frak B$ is dense in $\frak A$ for the measure-algebra topology.   For
$m$, $n\in\Bbb N$, $\pi_0,\ldots,\pi_m\in D$ and
$b_0,\ldots,b_m\in\frak B$, write
$E(m,n,\pi_0,\ldots,\pi_m,b_0,\ldots,b_m)$ for

\Centerline{$\{\pi:\pi\in G$,
$\sum_{i=0}^m\bar\mu(c_n\Bcap b_i\Bcap\supp(\pi^{-1}\pi_i))\le 2^{-n}\}$.}

\noindent (The supports are defined because $\frak A$ is Dedekind complete;
see 381F.)   Let $D'\subseteq G$ be a countable set such that
$D'\cap E(m,n,\pi_0,\ldots,\pi_m,b_0,\ldots,b_m)$ is non-empty whenever
$m$, $n\in\Bbb N$, $\pi_0,\ldots,\pi_m\in D$ and
$b_0,\ldots,b_m\in\frak B$ are
such that $E(m,n,\pi_0,\ldots,\pi_m,b_0,\ldots,b_m)$ is non-empty.

Suppose that $\pi\in G$, $c\in\frak A^f$ and $\epsilon>0$.
Let $n\in\Bbb N$ be such that
$\bar\mu(c\Bsetminus c_n)+2^{-n+2}<\epsilon$.   We have a family
$\family{j}{J}{(a_j,\pi_j)}$ such that
$\family{j}{J}{a_j}$ is a partition
of unity in $\frak A$ consisting of elements of finite measure, and, for
each $j\in J$, $\pi_j\in D$ and $\pi$ agrees with $\pi_j$ on
$\frak A_{a_j}$ (381Ia),
that is, $a_j\Bcap\supp(\pi^{-1}\pi_j)=0$.
Let $j_0,\ldots,j_m\in J$ be such that
$\bar\mu(c_n\Bsetminus\sup_{i\le m}a_{j_i})\le 2^{-n}$;  for each
$i\le m$, let $b_i\in\frak B$ be such that
$\bar\mu(c_n\Bcap(b_i\Bsymmdiff a_{j_i}))\le\Bover{2^{-n}}{m+1}$.  In
this case, $\pi\in E(m,n,\pi_{j_0},\ldots,\pi_{j_m},b_0,\ldots,b_m)$, so
there is a
$\tilde\pi\in D'\cap E(m,n,\pi_{j_0},\ldots,\pi_{j_m},b_0,\ldots,b_m)$.
Consider $d=\supp(\pi^{-1}\tilde\pi)$.   If we set
$d_i=\supp(\pi^{-1}\pi_{j_i})\Bcup\supp(\tilde\pi^{-1}\pi_{j_i})$, then
$\pi$ and $\tilde\pi$ both agree with $\pi_{j_i}$ on $1\Bsetminus d_i$,
so $d\Bsubseteq d_i$.   Now

$$\eqalign{\bar\mu(c_n\Bcap d)
&\le\bar\mu(c_n\Bsetminus\sup_{i\le m}b_i)
   +\sum_{i=0}^m\bar\mu(c_n\Bcap b_i\Bcap d)\cr
&\le\bar\mu(c_n\Bsetminus\sup_{i\le m}a_{j_i})
   +\sum_{i=0}^m\bar\mu(a_{j_i}\Bsetminus b_i)
   +\sum_{i=0}^m\bar\mu(c_n\Bcap b_i\Bcap d_i)\cr
&\le 2^{-n}+2^{-n}
   +\sum_{i=0}^m\bar\mu(c_n\Bcap b_i\Bcap\supp(\pi^{-1}\pi_{j_i}))
   +\sum_{i=0}^m\bar\mu(c_n\Bcap b_i\Bcap\supp(\tilde\pi^{-1}\pi_{j_i}))\cr
&\le 4\cdot 2^{-n}=2^{-n+2},\cr}$$

\noindent and $\bar\mu(c\Bcap d)<\epsilon$.
But this means that $\pi^{-1}\tilde\pi\in U(c,\epsilon)$;  as $c$,
$\epsilon$ and $\pi$ are arbitrary, $D'$ is dense in $G$ and $G$ is
separable.
}%end of proof of 494C

\leader{494D}{Lemma}\dvAnew{2011}
Let $(\frak A,\bar\mu)$ be a probability algebra and
$\phi\in\Aut_{\bar\mu}\frak A$.   Let
$T=T_{\phi}:L^2_{\Bbb C}\to L^2_{\Bbb C}$ be
the corresponding operator on the complex Hilbert space
$L^2_{\Bbb C}=L^2_{\Bbb C}(\frak A,\bar\mu)$\cmmnt{ (366M)}.
Then the following are equiveridical:

\inset{($\alpha$) $\phi$ is weakly
mixing\cmmnt{ (definition:  372Ob)};

($\beta$) $\inf_{k\in\Bbb N}|\innerprod{T^kw}{w}|<1$ whenever
$w\in L^2_{\Bbb C}$, $\|w\|_2=1$ and $\int w=0$;

($\gamma$) $\inf_{k\in\Bbb N}|\innerprod{T^kw}{w}|=0$ whenever
$w\in L^2_{\Bbb C}$, $\|w\|_2=1$ and $\int w=0$.}

\proof{{\bf (a)}
Regarding $\Bbb Z$, with addition and its discrete
topology, as a topological group, its dual group is the circle group
$S^1=\{z:z\in\Bbb C$, $|z|=1\}$ with multiplication and its usual topology
(445Bb-445Bc);  the duality being given by the functional
$(k,z)\mapsto z^k:\Bbb Z\times S^1\to S^1$.
For $u\in L^2_{\Bbb C}$, define $h_u:\Bbb Z\to\Bbb C$ by
setting $h_u(k)=\innerprod{T^ku}{u}$ for $k\in\Bbb Z$.   Then $h_u$ is
positive definite in the sense of 445L.   \Prf\ If
$\zeta_0,\ldots,\zeta_n\in\Bbb C$ and $m_0,\ldots,m_n\in\Bbb Z$, then

$$\eqalignno{\sum_{j,k=0}^n\zeta_j\bar\zeta_kh_u(m_j-m_k)
&=\sum_{j,k=0}^n\zeta_j\bar\zeta_k\innerprod{T^{m_j-m_k}u}{u}
=\sum_{j,k=0}^n\zeta_j\bar\zeta_k\innerprod{T^{m_j}u}{T^{m_k}u}\cr
\displaycause{366Me}
&=\innerprod{\sum_{j=0}^n\zeta_jT^{m_j}u}
  {\sum_{k=0}^n\zeta_kT^{m_k}u}
\ge 0.  \text{ \Qed}\cr}$$

\noindent By Bochner's theorem (445N),
there is a Radon probability measure $\nu_u$ on $S^1$
such that

\Centerline{$\int z^k\nu_u(dz)=h_u(k)=\innerprod{T^ku}{u}$}

\noindent for every $k\in\Bbb Z$.   Note that

\Centerline{$\nu_u(S^1)=\int z^0d\nu_u=\innerprod{u}{u}=\|u\|_2^2$.}

\medskip

{\bf (b)(i)} Let $P\subseteq C(S^1;\Bbb C)$ be the set of
functions which are expressible in the form

\Centerline{$p(z)=\sum_{k\in\Bbb Z}\zeta_kz^k$ for every $z\in S^1$}

\noindent where $\zeta_k\in\Bbb C$ for every $k\in\Bbb Z$ and
$\{k:\zeta_k\ne 0\}$ is finite.    Then $P$ is a linear subspace of
the complex Banach space $C(S^1;\Bbb C)$, closed under multiplication.
Also, if $p\in P$, then $\bar p\in P$, where
$\bar p(z)=\overline{p(z)}$ for every $z\in S^1$.
\Prf\ If $p(z)=\sum_{k\in\Bbb Z}\zeta_kz^k$, then

\Centerline{$\bar p(z)=\sum_{k\in\Bbb Z}\bar\zeta_kz^{-k}
=\sum_{k\in\Bbb Z}\bar\zeta_{-k}z^k$}

\noindent for every $z\in S^1$.\ \QeD\
Of course $P$ contains the constant function $z\mapsto z^0$ and the
identity function $z\mapsto z$, so by the Stone-Weierstrass theorem (281G)
$P$ is $\|\,\|_{\infty}$-dense in $C(S^1;\Bbb C)$.

\medskip

\quad{\bf (ii)} For any $p\in P$ the coefficients of the
corresponding expression
$p(z)=\sum_{k\in\Bbb Z}\zeta_kz^k$ are uniquely defined, since

\Centerline{$\zeta_k=\Bover1{2\pi}\int_{-\pi}^{\pi}e^{-ikt}p(e^{it})dt$}

\noindent for every $k$.   So we can define $u_p$, for every
$u\in L^2_{\Bbb C}$, by saying that $u_p=\sum_{k\in\Bbb Z}\zeta_kT^ku$.
Now we have

\Centerline{$\innerprod{u_p}{u}
=\sum_{k\in\Bbb Z}\zeta_k\innerprod{T^ku}{u}
=\sum_{k\in\Bbb Z}\zeta_k\int z^k\nu_u(dz)
=\int p\,d\nu_u$.}

\noindent We also see that

$$\eqalign{\int u_p
&=\innerprod{u_p}{\chi 1}
=\sum_{k\in\Bbb Z}\zeta_k\innerprod{T^ku}{\chi 1}\cr
&=\sum_{k\in\Bbb Z}\zeta_k\innerprod{u}{T^{-k}\chi 1}
=\sum_{k\in\Bbb Z}\zeta_k\innerprod{u}{\chi 1}
=p(1)\int u.\cr}$$

It is elementary to check that
if $p\in P$ and $q(z)=zp(z)$ for every $z\in S^1$, then $u_q=Tu_p$.
Note also that $p\mapsto u_p:P\to L^2_{\Bbb C}$ is linear.

\medskip

\quad{\bf (iii)} For any $p\in P$ and $u\in L^2_{\Bbb C}$,
$\|u_p\|_2\le\|u\|_2\|p\|_{\infty}$.   \Prf\
If $p(z)=\sum_{k\in\Bbb Z}\zeta_kz^k$ for $z\in S^1$, set

\Centerline{$q(z)=p(z)\overline{p(z)}
=\sum_{j,k\in\Bbb Z}\zeta_j\bar\zeta_kz^{j-k}$}

\noindent for $z\in S^1$.   Then

$$\eqalign{\|u_p\|_2^2
&=\innerprod{\sum_{j\in\Bbb Z}\zeta_jT^ju}{\sum_{k\in\Bbb Z}\zeta_kT^ku}
=\sum_{j,k\in\Bbb Z}\zeta_j\bar\zeta_k\innerprod{T^ju}{T^ku}
=\sum_{j,k\in\Bbb Z}\zeta_j\bar\zeta_k\innerprod{T^{j-k}u}{u}\cr
&=\innerprod{u_q}{u}
=\int q\,d\nu_u
\le\|q\|_{\infty}\nu_u(S^1)
=\|p\|_{\infty}^2\|u\|_2^2.  \text{ \Qed}}$$

\medskip

{\bf (c) Case 1} Suppose that $\nu_u\{z\}=0$ whenever $u\in L^2_{\Bbb C}$,
$z\in S^1$ and $\int u=0$.

\medskip

\quad{\bf (i)} If $u\in L^2_{\Bbb C}$ and $\int u=0$, then
$\lim_{n\to\infty}\Bover1{n+1}\sum_{k=0}^n|\innerprod{T^ku}{u}|^2=0$.
\Prf\ For any $n\in\Bbb N$,

$$\eqalign{\Bover1{n+1}\sum_{k=0}^n|\innerprod{T^ku}{u}|^2
&=\Bover1{n+1}\sum_{k=0}^n\innerprod{T^ku}{u}\innerprod{u}{T^ku}
=\Bover1{n+1}\sum_{k=0}^n\innerprod{T^ku}{u}\innerprod{T^{-k}u}{u}\cr
&=\Bover1{n+1}\sum_{k=0}^n\int z^k\nu_u(dz)\int w^{-k}\nu_u(dw)\cr
&=\Bover1{n+1}\sum_{k=0}^n\int z^kw^{-k}\nu_u^2(d(z,w))\cr}$$

\noindent where $\nu_u^2$ is the product measure on $(S^1)^2$.   But observe
that

\Centerline{$|\Bover1{n+1}\sum_{k=0}^nz^kw^{-k}|\le 1$}

\noindent for all $z$, $w\in S^1$, while for $w\ne z$ we have

\Centerline{$\Bover1{n+1}\sum_{k=0}^nz^kw^{-k}
=\Bover{1-(w^{-1}z)^{n+1}}{(n+1)(1-w^{-1}z)}\to 0$.}

\noindent Since

\Centerline{$\nu_u^2\{(w,z):w=z\}=\int\nu_u\{z\}\nu_u(dz)=0$,}

\noindent Lebesgue's Dominated Convergence Theorem tells us that

\Centerline{
$\lim_{n\to\infty}\Bover1{n+1}\sum_{k=0}^{n}|\innerprod{T^ku}{u}|^2
=\lim_{n\to\infty}\Bover1{n+1}\sum_{k=0}^n\int z^kw^{-k}\nu_u^2(d(z,w))
=0$.\ \Qed}

\medskip

\quad{\bf (ii)} Write $\Cal F_d$ for the asymptotic density filter on
$\Bbb N$ (491S).   If $u\in L^2_{\Bbb C}$ and $\int u=0$, then
$\lim_{k\to\Cal F_d}|\innerprod{T^ku}{u}|^2=0$, by (i) above and 491Sb.
It follows at once that $\lim_{k\to\Cal F_d}\innerprod{T^ku}{u}=0$.

In fact $\lim_{k\to\Cal F_d}\innerprod{T^ku}{v}=0$ whenever
$\int u=\int v=0$.   \Prf\ We have

$$\eqalignno{\lim_{k\to\Cal F_d}\innerprod{T^ku}{v}+\innerprod{T^kv}{u}
&=\lim_{k\to\Cal F_d}\innerprod{T^k(u+v)}{u+v}
   -\innerprod{T^ku}{v}-\innerprod{T^kv}{v}
=0,&(*)\cr}$$

\noindent and similarly

\Centerline{$\lim_{k\to\Cal F_d}i\innerprod{T^ku}{v}-i\innerprod{T^kv}{u}
=\lim_{k\to\Cal F_d}\innerprod{T^k(iu)}{v}+\innerprod{T^kv}{iu}
=0$,}

\noindent so
$\lim_{k\to\Cal F_d}\innerprod{T^ku}{v}-\innerprod{T^kv}{u}=0$;
adding this to (*), $\lim_{k\to\Cal F}\innerprod{T^ku}{v}=0$.\ \Qed

\medskip

\quad{\bf (iii)} Now take any $a$, $b\in\frak A$ and set
$u=\chi a-(\bar\mu a)\chi 1$, $v=\chi b-(\bar\mu b)\chi 1$.
In this case, $\int u=\int v=0$ and

$$\eqalign{\innerprod{T^ku}{v}
&=\innerprod{\chi(\phi^ka)-(\bar\mu a)\chi 1}{\chi b-(\bar\mu b)\chi 1}\cr
&=\bar\mu(b\Bcap\phi^ka)-\bar\mu a\cdot\bar\mu b
   -\bar\mu(\phi^ka)\cdot\bar\mu b+\bar\mu a\cdot\bar\mu b
=\bar\mu(b\Bcap\phi^ka)-\bar\mu a\cdot\bar\mu b\cr}$$

\noindent for every $k$, so
$\lim_{k\to\Cal F_d}\bar\mu(b\Bcap\phi^ka)-\bar\mu a\cdot\bar\mu b=0$ and

\Centerline{$\lim_{n\to\infty}\Bover1{n+1}\sum_{k=0}^n
  |\bar\mu(b\Bcap\phi^ka)-\bar\mu a\cdot\bar\mu b|=0$,}

\noindent by 491Sb in the other direction.
As $a$ and $b$ are arbitrary, $\phi$ is weakly mixing.

\medskip

{\bf (d) Case 2} Suppose there are $u\in L^2_{\Bbb C}$ and
$t\in\ocint{-\pi,\pi}$ such that $\int u=0$ and $\nu_u\{e^{it}\}>0$.

\medskip

\quad{\bf (i)} For $n\in\Bbb N$, set
set $f_n(z)=\max(0,1-2^n|z-e^{it}|)$ for $z\in S^1$.   Then

$$\eqalign{|zf_n(z)-e^{it}f_n(z)|
&\le 2^{-n}\text{ if }|z-e^{it}|\le 2^{-n},\cr
&=0\text{ for other }z\in S^1.\cr}$$

\noindent Because $P$ is
$\|\,\|_{\infty}$-dense in $C(S^1;\Bbb C)$, there is a $p_n\in P$
such that $\|p_n-f_n\|_{\infty}\le 2^{-n}$, in which case

\Centerline{$|zp_n(z)-e^{it}p_n(z)|\le 3\cdot 2^{-n}$}

\noindent for every $z\in S^1$.   Set $q_n(z)=zp_n(z)$ for $z\in S^1$;
then

$$\eqalignno{\|Tu_{p_n}-e^{-it}u_{p_n}\|_2
&=\|u_{q_n}-e^{-it}u_{p_n}\|_2
\le\|u\|_2\|q_n-e^{-it}p_n\|_{\infty}\cr
\displaycause{by (b-iii)}
&\le 3\cdot 2^{-n}\|u\|_2,\cr}$$

\noindent while

\Centerline{$\|u_{p_n}\|_2\le\|u\|_2\|p_n\|_{\infty}\le 2\|u\|_2$.}

\medskip

\quad{\bf (ii)}
Let $\Cal F$ be any non-principal ultrafilter on $\Bbb N$.
Then $v=\lim_{n\to\Cal F}u_{p_n}$ is defined for the weak topology of the
complex Hilbert space $L^2_{\Bbb C}$ (4A4Ka).   Also

\Centerline{$\innerprod{v}{u}=\lim_{n\to\Cal F}\innerprod{u_{p_n}}{u}
=\lim_{n\to\infty}\int p_nd\nu_u
=\lim_{n\to\infty}\int f_nd\nu_u
=\nu_u\{e^{it}\}>0$}

\noindent so $v\ne 0$.   But we also have

\Centerline{$\int v=\innerprod{v}{\chi 1}
=\lim_{n\to\Cal F}\innerprod{u_{p_n}}{\chi 1}
=\lim_{n\to\Cal F}p_n(1)\int u=0$,}

\noindent and, taking limits in the weak topology on $L^2_{\Bbb C}$,

$$\eqalignno{Tv
&=\lim_{n\to\Cal F}Tu_{p_n}\cr
\displaycause{because $T$ is continuous for the weak topology,
see 4A4Bd}
&=\lim_{n\to\Cal F}e^{it}u_{p_n}
=e^{it}v.\cr}$$

\noindent Set $w=\Bover1{\|v\|_2}v$;  then $\|w\|_2=1$, $\int w=0$,

\Centerline{$\inf_{k\in\Bbb N}|\innerprod{T^kw}{w}|
=\inf_{k\in\Bbb N}|e^{ikt}\innerprod{w}{w}|
=1$}

\noindent and ($\beta$) is false.

\medskip

{\bf (e)} Putting (c) and (d) together, we see that either ($\alpha$) is
true or ($\beta$) is false, that is, that ($\beta$) implies ($\alpha$).

\medskip

{\bf (f)} On the other hand, ($\alpha$) implies ($\gamma$).
\Prf\ Suppose that $\phi$ is weakly mixing.   Then

\Centerline{$\lim_{n\to\infty}\Bover1{n+1}\sum_{k=0}^n
  |\bar\mu(b\Bcap\phi^ka)-\bar\mu a\cdot\bar\mu b|=0$}

\noindent for all $a$, $b\in\frak A$;  by 491Sb again,

\Centerline{$\lim_{k\to\Cal F_d}
\bar\mu(b\Bcap\phi^ka)-\bar\mu a\cdot\bar\mu b=0$,}

\noindent that is,

\Centerline{$\lim_{k\to\Cal F_d}\innerprod{T^k\chi a}{\chi b}
=\innerprod{\chi a}{\chi 1}\cdot\innerprod{\chi 1}{\chi b}$,}

\noindent whenever
$a$, $b\in\frak A$.   Because $\innerprod{\,}{\,\,}$ is sesquilinear,

\Centerline{$\lim_{k\to\Cal F_d}\innerprod{T^ku}{v}
=\innerprod{u}{\chi 1}\cdot\innerprod{\chi 1}{v}$}

\noindent whenever $u$, $v$ belong to
$S_{\Bbb C}=S_{\Bbb C}(\frak A)$, the complex
linear span of $\{\chi a:a\in\frak A\}$.
Because $S_{\Bbb C}$ is norm-dense in $L^2_{\Bbb C}$ (366Mb),
and $\{T^ku:k\in\Bbb N\}$ is norm-bounded, we shall have

\Centerline{$\lim_{k\to\Cal F_d}\innerprod{u}{T^{-k}v}
=\lim_{k\to\Cal F_d}\innerprod{T^ku}{v}
=\innerprod{u}{\chi 1}\cdot\innerprod{\chi 1}{v}$}

\noindent whenever $u\in S_{\Bbb C}$ and $v\in L^2_{\Bbb C}$;
now $\{T^{-k}v:k\in\Bbb N\}$ is norm-bounded, so

\Centerline{$\lim_{k\to\Cal F_d}\innerprod{T^ku}{v}
=\lim_{k\to\Cal F_d}\innerprod{u}{T^{-k}v}
=\innerprod{u}{\chi 1}\cdot\innerprod{\chi 1}{v}$}

\noindent for all $u$, $v\in L^2_{\Bbb C}$.   In particular, if $\|w\|_2=1$
and $\int w=0$,

\Centerline{$\inf_{k\in\Bbb N}|\innerprod{T^kw}{w}|
\le\lim_{k\to\Cal F_d}|\innerprod{T^kw}{w}|
=|\innerprod{w}{\chi 1}|^2=0$,}

\noindent as required.\ \Qed

\medskip

{\bf (g)} Since ($\gamma$) obviously implies ($\beta$), the three
conditions are indeed equiveridical.
}%end of proof of 494D

\leader{494E}{Theorem}\dvAnew{2011}\cmmnt{ ({\smc Halmos 44},
{\smc Rokhlin 48})} Let $(\frak A,\bar\mu)$ be a probability algebra,
and give $\AmuA$ its weak topology.

(a) If $\frak A\ne\{0,1\}$,
the set of mixing measure-preserving Boolean automorphisms is meager in
$\AmuA$.

(b) If $\frak A$ is atomless and homogeneous, the set of
two-sided Bernouilli shifts on $(\frak A,\bar\mu)$\cmmnt{ (definition:
385Qb)} is dense in $\AmuA$.

(c) If $\frak A$ has countable Maharam type, the set of weakly mixing
measure-preserving Boolean automorphisms is a G$_{\delta}$ subset of
$\AmuA$.

(d) If $\frak A$ is atomless and has countable Maharam type, the set
of weakly mixing measure-preserving Boolean automorphisms which are not
mixing is comeager in $\AmuA$, and is not empty.

\proof{{\bf (a)} Take $a\in\frak A\setminus\{0,1\}$.   Let $\delta>0$ be
such that $\bar\mu a>\delta+(\bar\mu a)^2$, and consider

\Centerline{$F_n=\{\pi:\pi\in\AmuA$,
$\bar\mu(a\Bcap\pi^ka)\le\delta+(\bar\mu a)^2$ for every $k\ge n\}$.}

\noindent Because $\pi\mapsto\bar\mu(a\Bcap\pi^ka)$ is continuous for every
$k$ (494Bb), every $F_n$ is closed.   Because $F_n$ cannot contain any
periodic automorphism, $(\AmuA)\setminus F_n$ is dense for the uniform
topology on $\AmuA$ (494Cc) and therefore for the weak topology (494Cd).
Accordingly $\bigcup_{n\in\Bbb N}F_n$ is meager;  and every mixing
measure-preserving automorphism belongs to $\bigcup_{n\in\Bbb N}F_n$.

\medskip

{\bf (b)} Suppose that $\phi\in\AmuA$, $A\subseteq\frak A$ is finite
and $\epsilon>0$.

\medskip

\quad{\bf (i)} By 494Cc, there is a periodic $\psi\in\AmuA$ such that
$\bar\mu(\phi a\Bsymmdiff\psi a)\le\epsilon$ for every $a\in\frak A$.
Let $\frak B$ be the subalgebra of $\frak A$ generated by
$\{\psi^ka:k\in\Bbb Z$, $a\in A\}$;  then $\frak B$ is finite (because
$\{\psi^k:k\in\Bbb Z\}$ is finite).   Let $B$ be the set of
atoms of $\frak B$.   Since $\psi[\frak B]=\frak B$, $\psi[B]=B$ and
$\psi\restr B$ is a permutation of $B$.   Let $B_0\subseteq B$ be such that
$B_0$ meets each orbit of $\psi\restr B$ in just one point;  enumerate
$B_0$ as $\ofamily{j}{n}{b_j}$.

Let $r\ge 1$ be such that $\#(B)+1\le\epsilon r$.   For each $j<n$,
let $m_j$ be the size of the orbit of $\psi\restr B$
containing $b_j$, and $p_j=\lceil r\bar\mu b_j\rceil-1$;  set
$M=\sum_{j=0}^{n-1}m_jp_j$.   Because $\frak A$
is atomless, we can find a disjoint family
$\ofamily{l}{p_j}{c_{jl}}$ in $\frak A_{b_j}$
such that $\bar\mu c_{jl}=\bover1r$ for every $l<p_j$.
Because
$\langle\psi^kb_j\rangle_{j<n,k<m_j}$ is disjoint, so is
$\langle\psi^kc_{jl}\rangle_{j<n,l<p_j,k<m_j}$.   Set

\Centerline{$C=\{\psi^kc_{jl}:j<n$, $l<p_j$, $k<m_j\}$,
\quad$c=\sup C$;}

\noindent then

$$\eqalign{\bar\mu c
&=\Bover{M}r
=\Bover1r\sum_{j=0}^{n-1}p_jm_j
\ge\Bover1r\sum_{j=0}^{n-1}m_j(r\bar\mu b_j-1)\cr
&=1-\Bover1r\sum_{j=0}^{n-1}m_j
=1-\Bover{\#(B)}r
\ge 1-\epsilon.\cr}$$

\noindent We shall need to know later that

\Centerline{$\Bover{M}r
=\Bover1r\sum_{j=0}^{n-1}p_jm_j
<\Bover1r\sum_{j-0}^{n-1}rm_j\bar\mu b_j=1$.}

\medskip

\quad{\bf (ii)}
Let $f:C\to C$ be the cyclic permutation defined by setting

$$\eqalign{f(\psi^kc_{jl})
&=\psi^{k+1}c_{jl}\text{ if }j<n,\,l<p_j,\,k\le m_j-2,\cr
&=c_{j,l+1}\text{ if }j<n,\,l\le p_j-2,\,k=m_j-1,\cr
&=c_{j+1,0}\text{ if }j\le n-2,\,l=p_j-1,\,k=m_j-1,\cr
&=c_{00}\text{ if }j=n-1,\,l=p_j-1,\,k=m_j-1.\cr}$$

\noindent Set

\Centerline{$C'=\{c:c\in C$, $f(c)$ and $\psi(c)$ are included in different
members of $B\}$.}

\noindent Then $\#(C')\le n$.   \Prf\ If $c\in C'$, express it as
$\psi^kc_{jl}$ where $j<n$, $l<p_j$ and $k<m_j$.   We surely have
$f(c)\ne\psi c$, so $k$ must be $m_j-1$.   In this case,

\Centerline{$\psi c=\psi^{m_j}c_{jl}\Bsubseteq\psi^{m_j}b_j=b_j$,}

\noindent so $f(c)\notBsubseteq b_j$ and $l$ must be $p_j-1$.
Thus $c=\psi^{m_j-1}c_{j,p_j-1}$ for some $j<n$, and there are only $n$
objects of this form.\ \Qed

\medskip

\quad{\bf (iii)} We know that there is a two-sided Bernouilli shift $\pi_0$
on $(\frak A,\bar\mu)$ (385Sb).   Now $\pi_0$ is mixing
(385Se), therefore ergodic (372Qb) and aperiodic (386D).
We know that $\bover{M}r<1$, so by 386C again
there is a $d_0\in\frak A$ such that
$d_0,\pi_0d_0,\ldots,\pi_0^{M-1}d_0$ are disjoint and
$\bar\mu d_0=\bover1r$.
Because $\bar\mu f^i(c_{00})=\bar\mu\pi_0^id_0=\bover1r$ for every $i<M$
and $\frak A$ is homogeneous,
there is a $\theta\in\AmuA$ such that $\theta(\pi_0^id_0)=f^i(c_{00})$ for
every $i<M$.   Set $\pi=\theta\pi_0\theta^{-1}$;  then $\pi$
is a two-sided Bernouilli shift (385Sg).   Now

\Centerline{$\pi f^i(c_{00})
=\theta\pi_0\theta^{-1}f^i(c_{00})
=\theta\pi_0\pi_0^id_0
=f^{i+1}(c_{00})$}

\noindent whenever $i\le M-2$.   So

$$\eqalign{C''
&=\{c:c\in C,\,\pi c\text{ and }\psi(c)\text{ are included in
different members of }B\}\cr
&\subseteq C'\cup\{f^{M-1}(c_{00})\}\cr}$$

\noindent has at most $n+1$ members.

Because $B$ is disjoint,
$e=\sup_{b\in B}\pi b\Bsymmdiff\psi b$ is disjoint from

\Centerline{$\sup_{b\in B}\pi b\Bcap\psi b\Bsupseteq\sup(C\setminus C'')$}

\noindent and has measure at most

\Centerline{$\bar\mu(\sup C'')+\bar\mu(1\Bsetminus c)
\le\Bover{n+1}r+\epsilon
\le 2\epsilon$.}

\noindent If $a\in A$, then $a$ is the supremum of the members of $B$ it
includes, so $\pi a\Bsymmdiff\psi a\Bsubseteq e$ and

\Centerline{$\bar\mu(\pi a\Bsymmdiff\phi a)
\le\bar\mu(\pi a\Bsymmdiff\psi a)+\bar\mu(\psi a\Bsymmdiff\phi a)
\le 3\epsilon$.}

\medskip

\quad{\bf (iv)} Thus, given $\phi\in\AmuA$, $A\in[\frak A]^{<\omega}$
and $\epsilon>0$, we can find a two-sided Bernouilli shift $\pi$ such that
$\bar\mu(\pi a\Bsymmdiff\phi a)\le 3\epsilon$ for every $a\in A$;  as
$\phi$, $A$ and $\epsilon$ are arbitrary, the two-sided Bernouilli shifts
are dense in $\AmuA$.

\medskip

{\bf (c)(i)} The point is that
$L^2_{\Bbb C}=L^2_{\Bbb C}(\frak A,\bar\mu)$ is separable in its norm
topology.   \Prf\ By 331O, there is a countable set
$A\subseteq\frak A$ which is dense for the measure-algebra topology of
$\frak A$.   Let $C$ be a countable dense subset of $\Bbb C$, and

\Centerline{$D
=\{\sum_{j=0}^n\zeta_j\chi a_j:\zeta_0,\ldots,\zeta_n\in C$,
$a_0,\ldots,a_n\in A\}$,}

\noindent so that $D$ is a countable subset of $L^2_{\Bbb C}$.
Because the function

\Centerline{$(\zeta_0,\ldots,\zeta_n,a_0,\ldots,a_n)
\mapsto\sum_{j=0}^n\zeta_j\chi a_j:\Bbb C^{n+1}\times\frak A^{n+1}
\to L^2_{\Bbb C}$}

\noindent is continuous for each $n$, $\overline{D}$ contains
$\sum_{j=0}^n\zeta_j\chi a_j$ whenever
$\zeta_0,\ldots,\zeta_n\in\Bbb C$ and $a_0,\ldots,a_n\in\frak A$,
that is, $S_{\Bbb C}=S_{\Bbb C}(\frak A)\subseteq\overline{D}$.   But
$S_{\Bbb C}$ is norm-dense in $L^2_{\Bbb C}$, so
$D$ also is dense and $L^2_{\Bbb C}$ is separable.\ \Qed

\medskip

\quad{\bf (ii)} For $\pi\in\AmuA$, let
$T_{\pi}:L^2_{\Bbb C}\to L^2_{\Bbb C}$ be the corresponding linear
operator, as in 494D.   We need to know that the function
$\pi\mapsto T_{\pi}v:\AmuA\to L^2_{\Bbb C}$ is continuous for every
$v\in L^2$.   \Prf\ It is elementary to check that
$a\mapsto\chi a:\frak A\to L^2_{\Bbb C}$ is continuous for the
measure-algebra topology on $\frak A$, so
$(\pi,a)\mapsto T_{\pi}\chi a=\chi\pi a$ is continuous (494Ba-494Bb), and
$\pi\mapsto T_{\pi}\chi a$ is continuous, for every $a\in\frak A$.
Because addition and scalar multiplication are continuous on
$L^2_{\Bbb C}$, $\pi\mapsto T_{\pi}v$ is continuous for every
$v\in S_{\Bbb C}$.   Now if $v$ is any member of $L^2_{\Bbb C}$,
$\phi\in\AmuA$ and $\epsilon>0$, there is a $v'\in S_{\Bbb C}$ such that
$\|v-v'\|_2\le\epsilon$, in which case

\Centerline{$\{\pi:\|T_{\pi}v-T_{\phi}v\|_2\le 3\epsilon\}
\supseteq\{\pi:\|T_{\pi}v'-T_{\phi}v'\|\le\epsilon\}$}

\noindent is a neighbourhood of $\phi$.   Thus $\pi\mapsto T_{\pi}v$ is
continuous for arbitrary $v\in L^2_{\Bbb C}$.\ \Qed

\medskip

\quad{\bf (iii)} It follows from (i) that the set
$V=\{v:v\in L^2_{\Bbb C}$, $\|v\|_2=1$, $\int v=0\}$ is separable
(4A2P(a-iv)).   Let $D'$ be a countable dense subset of $V$.
For $v\in D'$, set

\Centerline{$F_v
=\{\pi:|\innerprod{T_{\pi}^kv}{v}|\ge\Bover12$ for every $k\in\Bbb N\}$.}

\noindent Since the maps

\Centerline{$\pi\mapsto\pi^k\mapsto T_{\pi^k}v=T_{\pi}^kv
\mapsto\innerprod{T_{\pi}^kv}{v}$}

\noindent are all continuous (494Ba and (ii) just above), $F_v$ is closed.
Consider $E=\AmuA\setminus\bigcup_{v\in D'}F_v$.
If $\pi\in\AmuA$ is weakly mixing, then ($\alpha$)$\Rightarrow$($\gamma$)
of
494D tells us that $\pi\in E$.   On the other hand, if $\pi\in\AmuA$ is not
weakly mixing, ($\beta$)$\Rightarrow$($\alpha$) of 494D tells us that
there is
a $w\in V$ such that $\inf_{k\in\Bbb N}|\innerprod{T_{\pi}^kw}{w}|\ge 1$.
Let $v\in D'$ be such that $\|v-w\|_2\le\Bover14$.   Then, for any
$k\in\Bbb N$,

$$\eqalign{|\innerprod{T_{\pi}^kv}{v}|
&\ge|\innerprod{T_{\pi}^kw}{v}|-\|T_{\pi}^kw-T_{\pi}^kv\|_2\|v\|_2
\ge|\innerprod{T_{\pi}^kw}{v}|-\Bover14\cr
&\ge|\innerprod{T_{\pi}^kw}{w}|-\|T_{\pi}^kw\|_2\|v-w\|_2-\Bover14
\ge\Bover12.\cr}$$

\noindent So $\pi\in F_v\subseteq(\AmuA)\setminus E$.
Thus the set of weakly mixing automorphisms is
precisely $E$, and is a G$_{\delta}$ set.

\medskip

{\bf (d)} We know that every
two-sided Bernouilli shift is weakly mixing (385Se, 372Qb), so the set
$E$ of
weakly mixing automorphisms is dense, by (b) here, and G$_{\delta}$, by
(c), therefore comeager.   By (a), the set $E'$ of weakly mixing
automorphisms which are not mixing is also comeager.
By 494Be, $\AmuA$ is a Polish space, so $E'$ is non-empty.
}%end of proof of 494E

\leader{494F}{}\dvAnew{2011}\cmmnt{ 494Ed tells us that `many'
automorphisms of the
Lebesgue probability algebra are weakly mixing but not mixing.   It is
another matter to give an explicit description of one.   Bare-handed
constructions (e.g., {\smc Chacon 69}) demand ingenuity and determination.
I prefer to show you an example taken from {\smc Tao l08}, Lecture 12,
Exercises 5 and 8,
although it will take some pages in the style of this book,
as it gives practice in using ideas already presented.

\medskip

\noindent}{\bf Example} (a) There is a Radon
probability measure $\nu$ on $\Bbb R$, zero on singletons, such that

\Centerline{$\int\cos(2\pi\cdot 3^jt)\,\nu(dt)=\int\cos 2\pi t\,\nu(dt)>0$}

\noindent for every $j\in\Bbb N$.

(b) Set $\sigma_{jk}=\int\cos 2\pi(k-j)t\,\nu(dt)$ for $j$, $k\in\Bbb Z$.
Then there is a centered Gaussian distribution $\mu$ on $X=\BbbR^{\Bbb Z}$
with covariance matrix $\langle\sigma_{jk}\rangle_{j,k\in\Bbb Z}$.

(c) Let $S:X\to X$ be the shift operator defined by saying that
$(Sx)(j)=x(j+1)$ for $x\in X$ and $j\in\Bbb Z$.   Then $S$ is an
automorphism of $(X,\mu)$.

(d) Let $(\frak A,\bar\mu)$ be the measure
algebra of $\mu$ and $\phi\in\AmuA$ the automorphism represented by
$S$.   Then $\phi$ is not mixing.

(e) $\phi$ is weakly mixing.

\proof{{\bf (a)(i)} Let $\tilde\nu$ be the usual measure on
$\Cal P\Bbb N$ (254Jb, 464A).   Define $h:\Cal P\Bbb N\to\Bbb R$ by setting

\Centerline{$h(I)=\Bover23\sum_{j\in I}3^{-j}$}

\noindent for $I\subseteq\Bbb N$.   Then $h$ is continuous,
so the image measure $\nu=\tilde\nu h^{-1}$ is a Radon probability
measure on $\Bbb R$ (418I).
Also $h$ is injective, so $\nu$, like $\tilde\nu$,
is zero on singletons.

\medskip

\quad{\bf (ii)} The function $t\mapsto\fraction{3t}=3t-\lfloor 3t\rfloor$
is \imp\ for $\nu$.   \Prf\ Set $\psi_0(I)=\{j:j+1\in I\}$
for $I\subseteq\Bbb N$,
$\psi_1(t)=\fraction{3t}$ for $t\in\Bbb R$.   Then
$\psi_0:\Cal P\Bbb N\to\Cal P\Bbb N$ is \imp\ for $\tilde\nu$,
because

\Centerline{$\tilde\nu\{I:\psi_0(I)\cap J=K\}
=\tilde\nu\{I:I\cap(J+1)=K+1\}=2^{-\#(J)}$}

\noindent whenever $K\subseteq J\in[\Bbb N]^{<\omega}$.   Next,
for any  $I\in\Cal P\Bbb N\setminus\{\Bbb N,\Bbb N\setminus\{0\}\}$,

\Centerline{$\psi_1(h(I))
=\fraction{2\sumop_{j\in I}3^{-j}}
=2\sum_{j\in I\setminus\{0\}}3^{-j}
=2\sum_{j+1\in I}3^{-j-1}
=h(\psi_0(I))$.}

\noindent So $\psi_1h\eae h\psi_0$, and

\Centerline{$\nu\psi_1^{-1}=\tilde\nu h^{-1}\psi_1^{-1}
=\tilde\nu\psi_0^{-1}h^{-1}=\tilde\nu h^{-1}=\nu$.  \Qed}

Similarly, if we set

$$\eqalign{\theta(t)
&=\Bover13-t\text{ if }0\le t\le\Bover13,\cr
&=\Bover53-t\text{ if }\Bover23\le t\le 1,\cr
&=t\text{ otherwise},\cr}$$

\noindent then $\theta h(I)=h(I\symmdiff(\Bbb N\setminus\{0\}))$ for every
$I\subseteq\Bbb N$, and $\nu\theta^{-1}=\nu$.

\medskip

\quad{\bf (iii)} Consequently, for any $m\in\Bbb N$,

\Centerline{$\int\cos(2\pi\cdot 3mt)\,\nu(dt)
=\int\cos(2\pi m\fraction{3t})\,\nu(dt)
=\int\cos 2\pi mt\,\nu(dt)$}

\noindent (235G).   Inducing on $j$, we see that

\Centerline{$\int\cos(2\pi\cdot 3^jt)\,\nu(dt)=\int\cos 2\pi t\,\nu(dt)$}

\noindent for every $j\in\Bbb N$.

\medskip

\quad{\bf (iv)} As for $\int\cos 2\pi t\,\nu(dt)$, this is
equal to $\int\cos 2\pi\theta(t)\,\nu(dt)$.   Now

$$\eqalign{\cos 2\pi t+\cos 2\pi\theta(t)
&=\cos 2\pi t+\cos 2\pi(\Bover13-t)\cr
&\mskip100mu
=2\cos\Bover{\pi}3\cos 2\pi(t-\Bover16)>0
   \text{ if }0\le t\le\Bover13,\cr
&=\cos 2\pi t+\cos 2\pi(\Bover53-t)\cr
&\mskip100mu
=2\cos\Bover{5\pi}3\cos 2\pi(t-\Bover56)>0
   \text{ if }\Bover23\le t\le 1;\cr}$$

\noindent but $h[\Cal P\Bbb N]\subseteq[0,\bover13]\cup[\bover23,1]$,
so $[0,\bover13]\cup[\bover23,1]$ is $\nu$-conegligible, and
$\cos 2\pi t+\cos 2\pi\theta(t)>0$ for $\nu$-almost every $t$.
Accordingly

\Centerline{$\int\cos 2\pi t\,\nu(dt)
=\Bover12\int\cos 2\pi t+\cos 2\pi\theta(t)\nu(dt)
>0$.}

\medskip

{\bf (b)} $\sigma_{jk}=\sigma_{kj}$ for all
$j$, $k\in\Bbb Z$.   If $J\subseteq\Bbb Z$ is finite
and $\family{j}{J}{\gamma_j}\in\BbbR^J$, then

$$\eqalignno{\sum_{j,k\in J}\gamma_j\gamma_k\sigma_{jk}
&=\sum_{j,k\in J}\gamma_j\gamma_k\int\cos 2\pi(k-j)t\,\nu(dt)\cr
&=\sum_{j,k\in J}\gamma_j\gamma_k\int\cos 2\pi kt\cos 2\pi jt
   +\sin 2\pi kt\sin 2\pi jt\,\nu(dt)\cr
&=\int\sum_{j,k\in J}\gamma_j\gamma_k\cos 2\pi kt\cos2\pi jt\,\nu(dt)\cr
&\mskip100mu
   +\int\sum_{j,k\in J}\gamma_j\gamma_k\sin 2\pi kt\sin 2\pi jt\,\nu(dt)\cr
&=\int\sum_{j\in J}\gamma_j\cos 2\pi jt\,
     \sum_{k\in J}\gamma_k\cos 2\pi kt\,\nu(dt)\cr
&\mskip100mu
   +\int\sum_{j\in J}\gamma_j\sin 2\pi jt\,
      \sum_{k\in J}\gamma_k\sin 2\pi kt\,\nu(dt)\cr
&\ge 0.\cr}$$

\noindent By 456C(iv), we have a Gaussian distribution of the
right kind.

\medskip

{\bf (c)} Of course $S$ is linear, and $\Bbb Z$ is countable,
so the image measure $\mu S^{-1}$ is a centered Gaussian distribution
(456Ba).   Since

$$\eqalign{\int x(j)x(k)(\mu S^{-1})(dx)
&=\int (Sx)(j)(Sx)(k)\mu(dx)\cr
&=\int x(j+1)x(k+1)\mu(dx)
=\sigma_{j+1,k+1}
=\sigma_{jk}\cr}$$

\noindent for all $j$, $k\in\Bbb Z$, $\mu S^{-1}$ and $\mu$ have the same
covariance matrix, and are equal (456Bb).
Thus the bijection $S$ is an automorphism of $(X,\mu)$.

\medskip

{\bf (d)} Write $L^2$ for $L^2(\frak A,\bar\mu)$, and
$T_{\phi}:L^2\to L^2$ for the linear operator associated
with the automorphism $\phi$.   For $k\in\Bbb Z$, set $f_k(x)=x(k)$ for
$x\in X$ and $u_k=f_k^{\ssbullet}\in L^2$.
Then $f_kS=f_{k+1}$ so $T_{\phi}u_k=u_{k+1}$, by 364Qd.
Consider

$$\eqalign{\innerprod{T_{\phi}^{3^j}u_0}{u_0}
&=\int u_{3^j}\times u_0
=\int x(3^j)x(0)\mu(dx)\cr
&=\sigma_{3^j,0}
=\int\cos(2\pi\cdot 3^jt)\nu(dt)
=\int\cos 2\pi t\,\nu(dt)
\ne 0,}$$

\noindent for every $j$, while

\Centerline{$\int u_0=\int x(0)\mu(dx)=0$.}

\noindent By 372Q(a-iv), $\pi$ is not mixing.

\medskip

{\bf (e)(i)} $\lim_{n\to\infty}\Bover1{n+1}
\sum_{k=0}^n|\int e^{2\pi ikt}\nu(dt)|^2=0$.   \Prf\ For any $n\in\Bbb N$,

$$\eqalign{\Bover1{n+1}\sum_{k=0}^n|\int e^{2\pi ikt}\nu(dt)|^2
&=\Bover1{n+1}\sum_{k=0}^n
   \int e^{2\pi iks}\nu(ds)\int e^{-2\pi ikt}\nu(dt)\cr
&=\int\Bover1{n+1}\sum_{k=0}^n e^{2\pi ik(s-t)}\nu^2(d(s,t))\cr}$$

\noindent where $\nu^2$ is the product measure on $\BbbR^2$.   Now, for any
$s$, $t\in\Bbb R$,
$|\Bover1{n+1}\sum_{k=0}^n e^{2\pi ik(s-t)}|\le 1$ for every $n$, while
if $s-t$ is not an integer,

\Centerline{$\Bover1{n+1}\sum_{k=0}^n e^{2\pi ik(s-t)}
=\Bover{1-\exp(2\pi i(n+1)(s-t))}{(n+1)(1-\exp(2\pi i(s-t)))}
\to 0$}

\noindent as $n\to\infty$.   As $\nu$ is zero on singletons,

\Centerline{$\nu^2\{(s,t):s-t\in\Bbb Z\}
=\int\nu\{s:s\in t+\Bbb Z\}\nu(ds)=0$.}

\noindent So

\Centerline{$\lim_{n\to\infty}
  \Bover1{n+1}\sum_{k=0}^n|\int e^{2\pi ikt}\nu(dt)|^2
=\lim_{n\to\infty}
  \int\Bover1{n+1}\sum_{k=0}^n e^{2\pi ik(s-t)}\nu^2(d(s,t))
=0$}

\noindent by Lebesgue's dominated convergence theorem.\ \Qed

Consequently, as in (c-ii) of the proof of 494D,

$$\eqalign{0
&=\lim_{k\to\Cal F_d}|\int e^{2\pi ikt}\nu(dt)|^2
=\lim_{k\to\Cal F_d}\int e^{2\pi ikt}\nu(dt)\cr
&=\lim_{k\to\Cal F_d}\Real\int e^{2\pi ikt}\nu(dt)
=\lim_{k\to\Cal F_d}\int \cos(2\pi kt)\nu(dt).\cr}$$

\noindent By 491Sc,

\Centerline{$\lim_{k\to\Cal F_d}\sigma_{jk}
=\lim_{k\to\Cal F_d}\sigma_{j,j+k}
=\lim_{k\to\Cal F_d}\int\cos(2\pi kt)\nu(dt)
=0$.}

\medskip

\quad{\bf (ii)} Suppose that $f$, $g:X\to\Bbb R$ are functions such
that, for some finite $J\subseteq\Bbb Z$, there are continuous bounded
functions $f_0$, $g_0:\BbbR^J\to\Bbb R$ such that
$f(x)=f_0(x\restr J)$ and $g(x)=g_0(x\restr J)$ for every
$x\in\BbbR^X$.
Then $\lim_{n\to\Cal F_d}\int fS^n\times g\,d\mu=\int fd\mu\int g\,d\mu$.

\Prf\ For any $n\in\Bbb N$, define $ R_n:X\to\BbbR^{J\times\{0,1\}}$ by
setting

\Centerline{$ R_n(x)(j,0)=x(j)$,
\quad$ R_n(x)(j,1)=x(j+n)$}

\noindent for $x\in X$ and $j\in J$;  then $ R_n$ is linear, so
$\mu R_n^{-1}$ is a centered Gaussian distribution on
$\BbbR^{J\times\{0,1\}}$.
The covariance matrix $\sigma^{(n)}$ of $\mu R_n^{-1}$ is given by

$$\eqalign{\sigma^{(n)}_{(j,\epsilon),(k,\epsilon')}
&=\int z(j,\epsilon)z(k,\epsilon')\mu R_n^{-1}(dz)
=\int ( R_nx)(j,\epsilon)( R_nx)(k,\epsilon')\mu(dx)\cr
&=\int x(j)x(k)\mu(dx)=\sigma_{jk}\text{ if }\epsilon=\epsilon'=0,\cr
&=\int x(j)x(k+n)\mu(dx)=\sigma_{j,k+n}
   \text{ if }\epsilon=0,\,\epsilon'=1,\cr
&=\int x(j+n)x(k)\mu(dx)=\sigma_{j+n,k}=\sigma_{k,j+n}
   \text{ if }\epsilon=1,\,\epsilon'=0,\cr
&=\int x(j+n)x(j+n)\mu(dx)=\sigma_{j+n,k+n}
=\sigma_{jk}\text{ if }\epsilon=\epsilon'=1\cr}$$

\noindent for all $j$, $k\in J$.   So

$$\eqalign{\lim_{n\to\Cal F_d}\sigma^{(n)}_{(j,\epsilon),(k,\epsilon')}
&=\sigma_{jk}\text{ if }\epsilon=\epsilon',\cr
&=0\text{ if }\epsilon\ne\epsilon'.\cr}$$

\noindent Let $\tilde\mu$ be the centered Gaussian distribution
$\tilde\mu$ on $\BbbR^{J\times\{0,1\}}$ with covariance matrix $\tau$ where

$$\eqalign{\tau_{(j,\epsilon),(k,\epsilon')}
&=\sigma_{jk}\text{ if }\epsilon=\epsilon',\cr
&=0\text{ if }\epsilon\ne\epsilon'\cr}$$

\noindent for any $j$, $k\in J$.   By 456Q, there is such a distribution
and $\tilde\mu=\lim_{n\to\Cal F_d}\mu R_n^{-1}$ for the narrow topology.

Next observe that, for $x\in X$ and $z\in\BbbR^{J\times\{0,1\}}$,

$$\eqalign{ R_n(x)=z
&\Longrightarrow x(j+n)=z(j,1)\text{ for every }j\in J\cr
&\Longleftrightarrow (S^nx)(j)=z(j,1)\text{ for every }j\in J\cr
&\Longrightarrow f(S^nx)=f'_0(z),\cr
 R_n(x)=z
&\Longrightarrow x(j)=z(j,0)\text{ for every }r<m\cr
&\Longrightarrow g(x)=g'_0(z),\cr}$$

\noindent where we set

\Centerline{$f'_0(z)=f_0(\family{j}{J}{z(j,1)})$,
\quad$g'_0(z)=g_0(\family{j}{J}{z(j,0)}$}

\noindent for $z\in\BbbR^{J\times\{0,1\}}$.
So $fS^n=f'_0 R_n$, $g=g'_0 R_n$,

\Centerline{$\int fS^n\times g\,d\mu
=\int(f'_0 R_n)\times(g'_0 R_n)d\mu
=\int f'_0\times g'_0d(\mu R_n^{-1})$}

\noindent for every $n$, and

\Centerline{$\lim_{n\to\Cal F_d}\int fS^n\times g\,d\mu
=\int f'_0\times g'_0d\tilde\mu$}

\noindent because $f'_0\times g'_0$ is a bounded continuous function
(437Mb).

Since $\tau_{(j,0),(k,1)}=0$ whenever $j$, $k\in J$,
the $\sigma$-algebras $\Sigma_0$, $\Sigma_1$ generated by
coordinates in $J\times\{0\}$, $J\times\{1\}$ respectively are
$\tilde\mu$-independent (456Eb).
Since $f'_0$ is $\Sigma_0$-measurable and
$g'_0$ is $\Sigma_1$-measurable,

$$\eqalignno{\lim_{n\to\Cal F_d}\int fS^n\times g\,d\mu
&=\int f'_0\times g'_0\,d\tilde\mu
=\int f'_0d\tilde\mu\int g'_0d\tilde\mu\cr
\displaycause{272D, 272R}
&=\lim_{n\to\Cal F_d}\int f'_0d(\mu R_n^{-1})
   \cdot\lim_{n\to\Cal F_d}\int g'_0d(\mu R_n^{-1})\cr
&=\lim_{n\to\Cal F_d}\int f'_0 R_nd\mu
   \int g'_0 R_nd\mu\cr
&=\lim_{n\to\Cal F_d}\int fS^nd\mu\int g\,d\mu\cr
&=\lim_{n\to\Cal F_d}\int fd\mu\int g\,d\mu
=\int fd\mu\cdot\int g\,d\mu,\cr}$$

\noindent as required.\ \Qed

\medskip

\quad{\bf (iii)} If $F$, $F'\subseteq X$ are compact, then
$\lim_{n\to\Cal F_d}\mu(S^{-n}[F]\cap F')=\mu F\cdot\mu F'$.   \Prf\
Let $\epsilon>0$.
For $k\in\Bbb N$, set $J_k=\{j:j\in\Bbb Z$, $|j|\le k\}$ and
$F_k=\{x\restr J_k:x\in F\}$.
Set $f^{(0)}_k(z)=\max(0,1-2^k\rho_k(z,F_k))$ for $z\in\Bbb R^{J_k}$, where
$\rho_k$ is Euclidean distance in $\BbbR^{J_k}$, and
$f_k(x)=f^{(0)}_k(x\restr J_k)$ for $x\in X$.   Then
$\sequence{k}{f_k(x)}\to\chi F(x)$ for every $x\in X$.   So there is a
$k\in\Bbb N$ such that $\int|f_k-\chi F|d\mu\le\epsilon$.   Set $f=f_k$;
then $f$ is a continuous function from $X$ to $[0,1]$,
$\int|f-\chi F|\le\epsilon$, and $f$ factors
through the continuous function $f^{(0)}_k:\BbbR^{J_k}\to[0,1]$.

Similarly, there is a continuous function $g:X\to[0,1]$ such that
$\int|g-\chi F'|d\mu\le\epsilon$ and $g$ factors through a continuous
function on $\BbbR^{J_l}$ for some $l$.   Setting $J=J_k\cup J_l$,
we see that $f$ and $g$ satisfy the conditions of (ii) and

\Centerline{$\lim_{n\to\Cal F_d}\int fS^n\times g\,d\mu
=\int fd\mu\int g\,d\mu$.}

\noindent But for every $n\in\Bbb N$,

$$\eqalign{|\mu(S^{-n}[F]\cap F')-\int fS^n\times g|
&\le\int|fS^n-\chi S^{-n}[F]|+|g-\chi F'|d\mu\cr
&=\int|f-\chi F|+|g-\chi F'|d\mu
\le 2\epsilon,\cr}$$

\Centerline{$|\mu F\cdot\mu F'-\int fd\mu\int g\,d\mu|
\le\int|f-\chi F|+|g-\chi F'|d\mu
\le 2\epsilon$,}

\noindent so

$$\eqalign{&\limsup_{n\to\Cal F_d}
   |\mu(S^{-n}[F]\cap F')-\mu F\cdot\mu F'|\cr
&\mskip100mu
\le 4\epsilon+
\lim_{n\to\Cal F_d}|\int fS^n\times g\,d\mu-\int fd\mu\int g\,d\mu|
=4\epsilon.\cr}$$

\noindent As $\epsilon$ is arbitrary,
$\lim_{n\to\Cal F_d}\mu(S^{-n}[F]\cap F')=\mu F\cdot\mu F'$.\ \Qed

\medskip

\quad{\bf (iv)} Now suppose that $a$, $b\in\frak A$ and $\epsilon>0$.
Because $\mu$ is a Radon measure (454J(iii)), there are compact sets
$F_0$, $F_1\subseteq X$ such that
$\bar\mu(a\Bsymmdiff F_0^{\ssbullet})+\bar\mu(b\Bsymmdiff F_1^{\ssbullet})
\le\epsilon$.   Now, for any $n\in\Bbb N$,

$$\eqalign{|\bar\mu(\phi^na\Bcap b)-\mu(S^{-n}[F_0]\cap F_1)|
&=|\bar\mu(\phi^n a\Bcap b)
   -\bar\mu(\phi^nF_0^{\ssbullet}\Bcap F_1^{\ssbullet})|\cr
&\le\bar\mu(\phi^n a\Bsymmdiff\phi^nF_0^{\ssbullet})
  +\bar\mu(b\Bsymmdiff F_1^{\ssbullet})\cr
&=\bar\mu(a\Bsymmdiff F_0^{\ssbullet})
  +\bar\mu(b\Bsymmdiff F_1^{\ssbullet})
\le\epsilon,\cr}$$

\Centerline{$|\bar\mu a\cdot\bar\mu b-\mu F_0\cdot\mu F_1|
\le|\bar\mu a-\mu F_0|+|\bar\mu b-\mu F_1|
\le\epsilon$.}

\noindent So

$$\eqalign{&\limsup_{n\to\Cal F_d}|\bar\mu(\phi^na\Bcap b)
   -\bar\mu a\cdot\bar\mu b|\cr
&\mskip100mu
\le 2\epsilon
  +\lim_{n\to\Cal F_d}|\mu(S^{-n}[F_0]\cap F_1)-\mu F_0\cdot\mu F_1|
=2\epsilon\cr}$$

\noindent by (iii).
As $\epsilon$, $a$ and $b$ are arbitrary, $\phi$ is weakly
mixing (using 491Sb once more).
}%end of proof of 494R

\cmmnt{\medskip

\noindent{\bf Remark} Of course the measure $\nu$ of part (a) is
Cantor measure (256Hc, 256Xk).}

\leader{494G}{Proposition} Let $(\frak A,\bar\mu)$ be a
measure algebra and $G$ a full subgroup of $\AmuA$, with fixed-point
subalgebra $\frak C$\cmmnt{ (definition:  395Ga)}.

(a) If $a\in\frak A^f$ and $\pi\in G$, there is a
$\phi\in G$, supported by
$a\Bcup\pi a$, such that $\phi d=\pi d$ for every $d\Bsubseteq a$.

(b) If $(\frak A,\bar\mu)$ is localizable
and $a$, $b\in\frak A^f$, then the following are equiveridical:

\quad(i) there is a $\pi\in G$ such that $\pi a\Bsubseteq b$;

\quad(ii) $\bar\mu(a\Bcap c)\le\bar\mu(b\Bcap c)$ for every $c\in\frak C$.

(c) If $(\frak A,\bar\mu)$ is localizable
and $a$, $b\in\frak A^f$, then the following are equiveridical:

\quad(i) there is a $\pi\in G$ such that $\pi a=b$;

\quad(ii) $\bar\mu(a\Bcap c)=\bar\mu(b\Bcap c)$ for every $c\in\frak C$.

(d) If $(\frak A,\bar\mu)$ is totally finite\cmmnt{ (definition:  322Ab)}
and $\familyiI{a_i}$,
$\familyiI{b_i}$ are disjoint families in $\frak A$ such that
$\bar\mu(a_i\Bcap c)=\bar\mu(b_i\Bcap c)$ for every $i\in I$ and
$c\in\frak C$, there is a $\pi\in G$ such that $\pi a_i=b_i$ for every
$i\in I$.

(e) If $(\frak A,\bar\mu)$ is localizable and
$H=\{\pi:\pi\in\AmuA$, $\pi c=c$ for every $c\in\frak C\}$,
then $H$ is the closure of $G$ for the weak topology of $\AmuA$.

\proof{{\bf (a)} Let $\familyiI{(a_i,n_i,b_i)}$ be a maximal family such
that

----- $\familyiI{a_i}$ is a disjoint family in
$\frak A_{\pi a\Bsetminus a}\setminus\{0\}$,

----- $\familyiI{b_i}$ is a disjoint family in
$\frak A_{a\Bsetminus\pi a}$;

----- for every $i\in I$, $n_i\in\Bbb Z$ and $\pi^{n_i}a_i=b_i$.

\noindent Because $\bar\mu a<\infty$, $I$ is countable.   Set

\Centerline{$a'=(\pi a\Bsetminus a)\Bsetminus\sup_{i\in I}a_i$,
\quad$b'=(a\Bsetminus\pi a)\Bsetminus\sup_{i\in I}b_i$;}

\noindent then

\Centerline{$\bar\mu a'
=\bar\mu\pi a-\bar\mu(a\Bcap\pi a)-\sum_{i\in I}\bar\mu a_i
=\bar\mu a-\bar\mu(a\Bcap\pi a)-\sum_{i\in I}\bar\mu b_i
=\bar\mu b'$.}

\Quer\ If $a'\ne 0$, set $c=\sup_{n\in\Bbb Z}\pi^na'$.   Then $\pi c=c$, so

\Centerline{$\bar\mu(c\Bcap b_i)
=\bar\mu(c\Bcap\pi^{n_i}a_i)
=\bar\mu(\pi^{n_i}(c\Bcap a_i))
=\bar\mu(c\Bcap a_i)$}

\noindent for every $i\in I$, and

$$\eqalign{\bar\mu(c\Bcap b')
&=\bar\mu(c\Bcap a\Bsetminus\pi a)-\sum_{i\in I}\bar\mu(c\Bcap b_i)
=\bar\mu(c\Bcap a)-\bar\mu(c\Bcap a\Bcap\pi a)
  -\sum_{i\in I}\bar\mu(c\Bcap b_i)\cr
&=\bar\mu(c\Bcap\pi a)-\bar\mu(c\Bcap a\Bcap\pi a)
  -\sum_{i\in I}\bar\mu(c\Bcap a_i)
=\bar\mu(c\Bcap\pi a\Bsetminus a)-\sum_{i\in I}\bar\mu(c\Bcap a_i)\cr
&=\bar\mu(c\Bcap a')
=\bar\mu a'
>0,\cr}$$

\noindent and $c\Bcap b'\ne 0$.   There is therefore an
$n\in\Bbb Z$ such that $\pi^na'\Bcap b'\ne 0$.   But now, setting
$d=a'\Bcap\pi^{-n}b'$, $d\ne 0$ and we ought to have added
$(d,n,\pi^nd)$ to $\familyiI{(a_i,n_i,b_i)}$.\ \Bang

Thus $\sup_{i\in I}a_i=\pi a\Bsetminus a$ and
$\sup_{i\in I}b_i=a\Bsetminus\pi a$.
Now we can define $\phi\in\Aut\frak A$ by the formula

$$\eqalign{\phi d
&=\pi d\text{ if }d\Bsubseteq a,\cr
&=\pi^{n_i}d\text{ if }i\in I\text{ and }d\Bsubseteq a_i,\cr
&=d\text{ if }d\Bcap(a\Bcup\pi a)=0\cr}$$

\noindent (381C, because $I$ is countable and $\frak A$ is Dedekind
$\sigma$-complete).
Because $G$ is full, $\phi\in G$;  $\phi$ is supported by
$a\Bcup\pi a$, and $\phi$ agrees with $\pi$ on $\frak A_a$, as required.

\medskip

{\bf (b)(i)$\Rightarrow$(ii)} If $\pi a\Bsubseteq b$ and $c\in\frak C$,
then

\Centerline{$\bar\mu(a\Bcap c)
=\bar\mu\pi(a\Bcap c)
=\bar\mu(\pi a\Bcap\pi c)
\le\bar\mu(b\Bcap c)$.}

\medskip

\quad{\bf(ii)$\Rightarrow$(i)} Now suppose that
$\bar\mu(a\Bcap c)\le\bar\mu(b\Bcap c)$ for every $c\in\frak C$.

\medskip

\qquad\grheada\ Consider first the case in which $a\Bcap b=0$.
Let $\familyiI{(a_i,\pi_i,b_i)}$ be a maximal family
such that

----- $\familyiI{a_i}$ is a disjoint family in $\frak A_a\setminus\{0\}$,

----- $\familyiI{b_i}$ is a disjoint family in $\frak A_b$,

----- for every $i\in I$, $\pi_i\in G$ and $\pi_ia_i=b_i$.

\noindent Set $a'=\sup_{i\in I}a_i$,
$b'=\sup_{i\in I}b_i$ and

\Centerline{$c=\upr(b\Bsetminus b',\frak C)
=\sup_{\pi\in G}\pi(b\Bsetminus b')\in\frak C$}

\noindent (395G, because $\frak A$ is Dedekind complete).

$a\Bcap c=a'\Bcap c$.   \Prf\Quer\ Otherwise,
$a\Bsetminus a'$ meets $c$, so there is a $\pi\in G$ such that
$(a\Bsetminus a')\Bcap\pi(b\Bsetminus b')\ne 0$, in which case we ought to
have added

\Centerline{$((a\Bsetminus a')\Bcap\pi(b\Bsetminus b'),\pi^{-1},
\pi^{-1}(a\Bsetminus a')\Bcap b\Bsetminus b')$}

\noindent to our family
$\familyiI{(a_i,\pi_i,b_i)}$.\ \Bang\Qed

Now note that $1\Bsetminus c\in\frak C$, so

$$\eqalign{\bar\mu(a\Bsetminus c)
&\le\bar\mu(b\Bsetminus c)
=\bar\mu(b'\Bsetminus c)
=\sum_{i\in I}\bar\mu(b_i\Bsetminus c)\cr
&=\sum_{i\in I}\bar\mu\pi_i(a_i\Bsetminus c)
=\sum_{i\in I}\bar\mu(a_i\Bsetminus c)
=\bar\mu(a'\Bsetminus c),\cr}$$

\noindent so $a\Bsetminus c=a'\Bsetminus c$ and $a=a'$.

Accordingly we can define $\pi\in\Aut\frak A$ by setting

$$\eqalign{\pi d
&=\pi_id\text{ if }i\in I\text{ and }d\Bsubseteq a_i,\cr
&=\pi_i^{-1}d\text{ if }i\in I\text{ and }d\Bsubseteq b_i,\cr
&=d\text{ if }d\Bsubseteq 1\Bsetminus(a\Bcup b')\cr}$$

\noindent(381C again).   Because $G$ is full, $\pi\in G$, and

\Centerline{$\pi a=\pi(\sup_{i\in I}a_i)
=\sup_{i\in I}\pi a_i=\sup_{i\in I}b_i=b'\Bsubseteq b$.}

\medskip

\qquad\grheadb\ For the general case, we have

\Centerline{$\bar\mu(c\Bcap a\Bsetminus b)
=\bar\mu(c\Bcap a)-\bar\mu(c\Bcap a\Bcap b)
\le\bar\mu(c\Bcap b)-\bar\mu(c\Bcap a\Bcap b)
=\bar\mu(c\Bcap b\Bsetminus a)$}

\noindent for every $c\in\frak C$,
so ($\alpha$) tells us that there is a $\pi_0\in G$ such that
$\pi_0(a\Bsetminus b)\Bsubseteq b\setminus a$.   Now if we set
$\pi=\cycle{a\Bsetminus b\,_{\pi_0}\,\pi_0(a\Bsetminus b)}$,
$\pi\in G$ (because $G$ is full, see 381Sd), and $\pi a\Bsubseteq b$,
as required.

\medskip

{\bf (c)} If $\pi\in G$ and $\pi a=b$, then
$\pi a\Bsubseteq b$ and $\pi^{-1}b\Bsubseteq a$, so (b) tells us that
$\bar\mu(a\Bcap c)=\bar\mu(b\Bcap c)$ for every $c\in\frak C$.
If $\bar\mu(a\Bcap c)=\bar\mu(b\Bcap c)$ for every $c\in\frak C$,
then (b) tells us that there is a $\pi\in G$ such that
$\pi a\Bsubseteq b$;  but as $\bar\mu\pi a=\bar\mu a=\bar\mu b$,
we have $\pi a=b$.

\medskip

{\bf (d)} Let $j$ be any object not belonging to $I$ and set
$a_j=1\Bsetminus\sup_{i\in I}a_i$, $b_j=1\Bsetminus\sup_{i\in I}b_i$.
Then

\Centerline{$\bar\mu(a_j\Bcap c)
=\bar\mu c-\sum_{i\in I}\bar\mu(a_i\Bcap c)
=\bar\mu c-\sum_{i\in I}\bar\mu(b_i\Bcap c)
=\bar\mu(b_j\Bcap c)$}

\noindent for every $c\in\frak C$.   Set $J=I\cup\{j\}$.
By (c), there is for each
$i\in J$ a $\pi_i\in G$ such that $\pi_ia_i=b_i$.
Now $\family{i}{J}{a_i}$ and
$\family{i}{J}{b_i}$ are partitions of unity in $\frak A$, so there is a
$\pi\in\Aut\frak A$ such that $\pi d=\pi_id$ whenever $i\in J$ and
$d\Bsubseteq a_j$;  because $G$ is full, $\pi\in G$, and has the property
we seek.

\medskip

{\bf (e)(i)} If $a\in\frak A$, then
$U=\{\pi:\pi\in\AmuA$, $\pi a\notBsubseteq a\}$ is
open for the weak topology.   \Prf\ The functions

\Centerline{$\pi\mapsto\pi a:\AmuA\to\frak A$,
\quad$b\mapsto b\Bsetminus a:\frak A\to\frak A$}

\noindent are continuous (494Bb and 323Ba), and
$c\mapsto\bar\mu c:\frak A\to[0,\infty]$ is lower semi-continuous
(323Cb, because $\frak A$ is semi-finite), so
$\pi\mapsto\bar\mu(\pi a\Bsetminus a)$ is lower semi-continuous
(4A2B(d-ii)) and $U=\{\pi:\bar\mu(\pi a\Bsetminus a)>0\}$ is open.\ \Qed

Consequently, $\{\pi:\pi c\Bsubseteq c$ for every $c\in\frak C\}$ is
closed.   But if $\pi c\Bsubseteq c$ for every $c\in\frak C$, then
$\pi c=c$ for every $c\in\frak C$.  So $H$ is closed.   Of course
$H$ includes $G$, so $\overline{G}\subseteq H$.

\medskip

\quad{\bf (ii)}
Suppose that $\pi\in H$ and that $U$ is an open neighbourhood of $\pi$.
Then there are $a_0,\ldots,a_n\in\frak A^f$ and
$\delta>0$ such that $U$ includes
$\{\phi:\phi\in\AmuA$, $\bar\mu(\pi a_i\Bsymmdiff\phi a_i)\le\delta$
for every $i\le n\}$.
Set $e=\sup_{i\le n}a_i$;
let $\frak B$ be the finite subalgebra of $\frak A_e$
generated by
$\{e\Bcap a_i:i\le n\}$, and $B$ the set of its
atoms\cmmnt{ (definition:  316K)}.
If $b\in B$, then
$\bar\mu(\pi b\Bcap c)=\bar\mu(b\Bcap c)$
for every $c\in\frak C$, so there is a $\phi_b\in G$ such that
$\phi_bb=\pi b$, by (c) above.
Equally, there is a $\phi\in G$ such that $\phi e=\pi e$.   Now we can
define $\psi\in\AmuA$ by saying that

$$\eqalign{\psi d
&=\phi_bd\text{ if }b\in B\text{ and }d\Bsubseteq b,\cr
&=\phi d\text{ if }d\Bsubseteq 1\Bsetminus e;\cr}$$

\noindent as usual, $\psi\in G$, while $\psi b=\pi b$ for every $b\in B$.
But this means that $\psi a_i=\pi a_i$ for every $i\le n$, so
$\psi\in G\cap U$.   As $U$ is arbitrary, $\pi\in\overline{G}$;
as $\pi$ is arbitrary, $G$ is dense in $H$ and $H=\overline{G}$.
}%end of proof of 494G

\leader{494H}{Proposition} Let $\frak A$ be a Boolean algebra, $G$ a
full subgroup of $\Aut\frak A$, and $a\in\frak A$.   Set
$G_a=\{\pi:\pi\in G$, $\pi$ is supported by $a\}$,
$H_a=\{\pi\restrp\frak A_a:\pi\in G_a\}$.

(a) $G_a$ is a full subgroup of $\Aut\frak A$ and
$H_a$ is a full subgroup of $\Aut\frak A_a$, for every $a\in\frak A$.

(b) Suppose that $\frak A$ is Dedekind complete, and that the fixed-point
subalgebra of $G$ is $\frak C$.
Then the fixed-point subalgebra of $H_a$ is $\{a\Bcap c:c\in\frak C\}$.

\proof{{\bf (a)(i)} By 381Eb and 381Eh, $G_a$
is a subgroup of $G$,
and $\pi\mapsto\pi\restrp\frak A_a$ is a group homomorphism from $G_a$ to
$\Aut\frak A_a$, so its image $H_a$ is a subgroup of $\Aut\frak A_a$.

\medskip

\quad{\bf (ii)} Suppose that $\phi\in\Aut\frak A$ and that
$\familyiI{(a_i,\pi_i)}$ is a family in
$\frak A\times G_a$ such that $\familyiI{a_i}$ is a partition of unity in
$\frak A$ and $\pi_id=\phi d$ whenever
$i\in I$ and $d\Bsubseteq a_i$.   Then $\phi\in G$, because $G$ is full;
and

\Centerline{$\phi d
=\sup_{i\in I}\pi_i(d\Bcap a_i)=\sup_{i\in I}d\Bcap a_i=d$}

\noindent whenever $d\Bcap a=0$, so $\phi$ is supported by $a$ and belongs
to $G_a$.

\medskip

\quad{\bf (iii)} Suppose that $\phi\in\Aut\frak A_a$ and that
$\familyiI{(a_i,\pi_i)}$ is a family in
$\frak A_a\times H_a$ such that $\familyiI{a_i}$ is a partition of unity in
$\frak A_a$ and $\pi_id=\phi d$ whenever
$i\in I$ and $d\Bsubseteq a_i$.   For each $i\in I$, there is a
$\pi'_i\in G_a$ such that $\pi_i=\pi'_a\restrp\frak A_a$.
Take $j\notin I$ and set
$J=I\cup\{j\}$, $a_j=1\Bsetminus a$, $\pi'_j=\iota$;  define
$\psi\in\Aut\frak A$ by setting $\psi d=\phi d$ for $d\Bsubseteq a$,
$d$ for $d\Bsubseteq 1\Bsetminus a$.   Then
$\family{j}{J}{a_j}$ is a partition of unity in $\frak A$ and
$\psi d=\pi'_jd$ whenever $j\in J$ and $d\Bsubseteq a_j$, so
$\psi\in G$.   Also $\psi$ is supported by $a$,
so $\phi=\psi\restrp\frak A_a$ belongs to $H_a$.
As $\phi$ and $\familyiI{(a_i,\pi_i)}$ are arbitrary, $H_a$ is full.

\medskip

{\bf (b)(i)} If $c\in\frak C$, then
$\pi(a\Bcap c)=\pi a\Bcap\pi c=a\Bcap c$ whenever $\pi\in G$ and $\pi a=a$,
so $a\Bcap c$ belongs to the fixed-point subalgebra of $H_a$.

\medskip

\quad{\bf (ii)} In the other direction, take any $b$ in the fixed-point
subalgebra of $H_a$.   Set $c=\upr(b,\frak C)=\sup_{\pi\in G}\pi b$
(395G once more).
Of course $b\Bsubseteq a\Bcap c$.   \Quer\ If $b\ne a\Bcap c$, set
$e=a\Bcap c\Bsetminus b$.   Then there is a $\pi\in G$ such that
$e_1=e\Bcap\pi b\ne 0$;  set $e_2=\pi^{-1}e_1\Bsubseteq b$ and
$\phi=\cycle{e_2\,_{\pi}\,e_1}$.
Then $\phi\in G$ (381Sd again) and $\phi$ is supported by
$e_1\Bcup e_2\Bsubseteq a$, so
$\phi\restrp\frak A_a\in H_a$;  but $\phi b\ne b$, so this is impossible.\
\BanG\  Thus $b$ is expressed as the intersection of $a$ with a member of
$\frak C$, as required.
}%end of proof of 494H

\leader{494I}{}\dvAformerly{4{}93G}\cmmnt{ I take the proof of the next
theorem in a series of lemmas, the first being the leading special case.

\medskip

\noindent}{\bf Lemma}\cmmnt{ ({\smc Giordano \& Pestov 92})}
Let $(\frak A,\bar\mu)$ be an atomless homogeneous probability
algebra\cmmnt{ (definitions:  316Kb, 316N)}.
Then $\AmuA$, with its weak topology, is extremely amenable.

\proof{ I seek to apply 493C.

\medskip

{\bf (a)} Take $\epsilon>0$, a neighbourhood $V$ of the identity in
$\AmuA$,
a finite set $I\subseteq\AmuA$ and a finite family $\Cal A$ of zero sets in
$\AmuA$.
Let $\delta>0$ and $K\in[\frak A]^{<\omega}$ be such that $\pi\in V$
whenever $\pi\in\AmuA$ and $\bar\mu(a\Bsymmdiff\pi a)\le\delta$ for every
$a\in K$.   Let $C$ be the set of atoms of the finite subalgebra $\frak C$
of $\frak A$ generated by $K$, and $D$ the set of atoms of the subalgebra
$\frak D$ generated by $K\cup\bigcup_{\pi\in I}\pi[K]$;
set $k=\#(C)$ and
$k'=\#(D)$.   Let $m\in\Bbb N$ be so large that $2kk'\le m\delta$ and
$(m\delta-1)^2\ge 64m\ln\Bover1{\epsilon}$;
set $r=\lfloor m\delta\rfloor$, so
that $\exp(-\Bover{r^2}{64m})\le\epsilon$.

\medskip

{\bf (b)} For each $d\in D$ let $E_d$ be a maximal disjoint family in
$\frak A_d$ such that $\bar\mu e=\bover1m$ for every $e\in E_d$;  let $E$
be a partition of unity in $\frak A$, including
$\bigcup_{d\in D}E_d$, such that $\bar\mu e=\bover1m$ for every $e\in E$.
Let $H$ be the group of permutations of $E$.   Then we have a group
homomorphism $\theta:H\to\AmuA$ such that $\theta(\psi)\restr E=\psi$ for
every $\psi\in H$.   \Prf\ Fix $e_0\in E$.   Then for each
$e\in E$ there is a measure-preserving isomorphism
$\phi_e:\frak A_{e_0}\to\frak A_e$, because $\frak A$ is homogeneous
(331I).   For $\psi\in H_E$, $E$ and
$\family{e}{E}{\psi e}$ are partitions of unity in $\frak A$, so we can
define $\theta(\psi)\in\AmuA$ by the formula

\Centerline{$\theta(\psi)(a)=\phi_{\psi e}\phi_e^{-1}a$
whenever $a\Bsubseteq e\in E$.}

\noindent It is easy to see that $\theta(\psi)(e)=\psi e$ for every
$e\in E$.   If $\psi$, $\psi'\in H_E$, then

$$\eqalign{\theta(\psi\psi')(a)
&=\phi_{\psi\psi'e}\phi_e^{-1}a\cr
&=\phi_{\psi\psi'e}\phi_{\psi'e}^{-1}\phi_{\psi'e}\phi_e^{-1}a
=\theta(\psi)\theta(\psi')(a)\cr}$$

\noindent whenever $a\Bsubseteq e\in E$;  so $\theta$ is a group
homomorphism.\ \Qed

Write $G$ for $\theta[H]$, so that $G$ is a subgroup of $\AmuA$.

\medskip

{\bf (c)} $I\subseteq GV^{-1}$.   \Prf\ Take $\pi\in I$.   For $c\in C$,
set

\Centerline{$E'_c=\bigcup\{E_d:d\in D$, $d\Bsubseteq c\}$,
\quad$E''_c=\bigcup\{E_d:d\in D$, $d\Bsubseteq\pi c\}$.}

\noindent Since $\sup E_d\subseteq d$ and
$\bar\mu(d\Bsetminus\sup E_d)\le\bover1m$ for every $d\in D$,
$\sup E'_c\Bsubseteq c$ and $\bar\mu(c\Bsetminus\sup E'_c)\le\Bover{k'}m$;
so $m\bar\mu c-k'\le\#(E'_c)\le m\bar\mu c$.
Similarly, $\sup E''_c\Bsubseteq\pi c$ and

\Centerline{$m\bar\mu c-k'=m\bar\mu\pi c-k'\le\#(E''_c)\le m\bar\mu\pi c
=m\bar\mu c$.}

\noindent
Let $\tilde E'_c\subseteq E'_c$, $\tilde E''_c\subseteq E''_c$ be sets of
size $\min(\#(E'_c),\#(E''_c))\ge m\bar\mu c-k'$.
Setting $c'=\sup\tilde E'_c$ and $c''=\sup\tilde E''_c$ we have

\Centerline{$c'\Bsubseteq c$,
\quad$\bar\mu(c\Bsetminus c')
=\Bover1m(m\bar\mu c-\#(\tilde E'_c))\le\Bover{k'}m$,}

\noindent and similarly $c''\Bsubseteq\pi c$ and
$\bar\mu(\pi c\Bsetminus c'')\le\Bover{k'}m$.

Because $\family{c}{C}{\tilde E'_c}$ and $\family{c}{C}{\tilde E''_c}$ are
both disjoint, there is a $\psi\in H$ such that
$\psi[\tilde E'_c]=\tilde E''_c$ for every $c\in C$.   Set
$\phi=\theta(\psi)$;  then $\phi\in G$ and $\phi c'=c''$ for every
$c\in C$.   Now this means that

$$\eqalign{\bar\mu(c\Bsymmdiff\pi^{-1}\phi c)
&=\bar\mu(\pi c\Bsymmdiff\phi c)
\le\bar\mu(\pi c\Bsymmdiff c'')+\bar\mu(c''\Bsymmdiff\phi c)\cr
&=\bar\mu(\pi c\Bsymmdiff c'')+\bar\mu(c'\Bsymmdiff c)
\le\Bover{2k'}m\cr}$$

\noindent for every $c\in C$.   Consequently

\Centerline{$\bar\mu(a\Bsymmdiff\pi^{-1}\phi a)\le\Bover{2kk'}m\le\delta$}

\noindent for every $a\in K$, and $\pi^{-1}\phi\in V$.   Accordingly
$\pi\in\phi V^{-1}\subseteq GV^{-1}$;  as $\pi$ is arbitrary,
$I\subseteq GV^{-1}$.\ \Qed

\medskip

{\bf (d)} I am ready to introduce the functional $\nu$ demanded by the
hypotheses of 493C.   Let $\lambda$ be the Haar probability measure on the
finite group $H$, and $\nu$ the image measure $\lambda\theta^{-1}$,
regarded as a measure on $\AmuA$.   If $\pi\in I$, then (c) tells us that
there is a $\psi\in H$ such that
$\phi=\theta(\psi)$ belongs to $\pi V$.   In this case, for any
$A\in\Cal A$,

\Centerline{$\nu(\phi A)=\lambda\theta^{-1}[\phi A]
=\lambda(\psi\theta^{-1}[A])=\lambda\theta^{-1}[A]=\nu A$.}

\noindent So $\nu$ satisfies condition (ii) of 493C.

\medskip

{\bf (e)} As for condition (i) of 493C, consider
$W=\{\psi:\psi\in H$, $\#(\{e:e\in E$, $\psi e\ne e\})\le r\}$.
Then $\theta[W]\subseteq V$.   \Prf\ If $\psi\in W$, then
$\theta(\psi)(d)=d$ whenever $d\Bsubseteq e\in E$ and $\psi e=e$.   So
$\theta(\psi)$ is supported by $b=\sup\{e:e\in E$, $\psi e\ne e\}$.   Now
$\bar\mu b\le\bover{r}m\le\delta$.   So
$\bar\mu(a\Bsymmdiff\phi a)\le\bar\mu b\le\delta$ for every $a\in\frak A$,
and $\phi\in V$.\ \Qed

Now suppose that $F\subseteq\AmuA$ and $\nu F\ge\bover12$.   Then

$$\eqalignno{\nu(VF)
&=\lambda\theta^{-1}[VF]
\ge\lambda(W\theta^{-1}[F])\cr
\displaycause{because $\theta[W]\subseteq V$}
&\ge 1-\exp(-\Bover{r^2}{64m})\cr
\displaycause{by 492I, because $\lambda\theta^{-1}[F]=\nu F\ge\bover12$}
&\ge 1-\epsilon.\cr}$$

\noindent So $\nu$ satisfies the first condition in 493C.

\medskip

{\bf (f)} As $\epsilon$, $V$, $I$ and $\Cal A$ are arbitrary, $\AmuA$ is
extremely amenable.
}%end of proof of 494I

\vleader{48pt}{494J}{Lemma} Let $(\frak C,\bar\lambda)$ be a totally finite
measure algebra, $(\frak B,\bar\nu)$ a probability
algebra, and $(\frak A,\bar\mu)$ the localizable measure algebra
free product $(\frak C,\bar\lambda)\tensorhat(\frak B,\bar\nu)$\cmmnt{
(325E)}.
Give $\AmuA$ its weak topology, and let $G$ be the subgroup
$\{\pi:\pi\in\AmuA$, $\pi(c\otimes 1)=c\otimes 1$ for every
$c\in\frak C\}$.   Suppose that $\frak B$ is either finite, with all its
atoms of the same measure, or
homogeneous.   Then $G$ is amenable, and if either $\frak B$
is homogeneous or $\frak C$ is atomless, $G$ is extremely amenable.

\proof{{\bf (a)} Let $\Cal E$ be the family of finite partitions of unity
in $\frak C$ not containing $\{0\}$.   Then for any $E\in\Cal E$
we have a function
$\theta_E:(\Aut_{\bar\nu}\frak B)^E\to G$ defined by saying that

\Centerline{$\theta_E(\pmb{\phi})(c\otimes b)
=\sup_{e\in E}(c\Bcap e)\otimes\phi_eb$}

\noindent whenever
$\pmb{\phi}=\family{e}{E}{\phi_e}\in(\Aut_{\bar\nu}\frak B)^E$,
$c\in\frak C$ and $b\in\frak B$.
\Prf\ For each $e\in E$, the defining universal mapping theorem
325Da tells us that there is a unique measure-preserving Boolean
homomorphism $\psi_e:\frak A\to\frak A$ such that
$\psi_e(c\otimes 1)=c\otimes 1$ and $\psi_e(1\otimes b)=1\otimes\phi_eb$
for all $b\in\frak B$ and $c\in\frak C$.   To see that $\psi_e$ is
surjective, note that $\psi_e[\frak A]$ must be a closed subalgebra
including $\frak C\otimes\frak B$, which is dense (324Kb, 325D(c-i)).   So
$\psi_e\in\AmuA$.   Now
$\family{e}{E}{e\otimes 1}$ is a partition of unity in $\frak A$, and
$\psi_e(e\otimes 1)=e\otimes 1$ for every $e$, so we have a $\pi\in\AmuA$
defined by saying that $\pi a=\sup_{e\in E}\psi_e(a\Bcap e)$ for every
$a\in\frak A$.   Because $G$ is full, $\pi\in G$.  So we can set
$\theta_E(\pmb{\phi})=\pi$.   Of course $\pi$ is the only automorphism
satisfying the given formula for $\theta_E(\pmb{\phi})$.\ \Qed

\medskip

{\bf (b)(i)} It is easy to check that if $E\in\Cal E$ then
$\theta_E$ is a group homomorphism
from $(\Aut_{\bar\nu}\frak B)^E$ to $G$;
write $G_E$ for its set of values.   Because $0\notin E$, $\theta_E$ is
injective, and $G_E$ is a subgroup of $G$ isomorphic to the group
$(\Aut_{\bar\nu}\frak B)^E$.   Give $\Aut_{\bar\nu}\frak B$ its weak
topology, $(\Aut_{\bar\nu}\frak B)^E$ the product topology and $G$ the
topology induced by the weak topology of $\AmuA$.

\medskip

\qquad{\bf (ii)}  $\theta_E$ is continuous.
\Prf\ If $U$ is a neighbourhood of the identity in $G_E$, there are
$a_0,\ldots,a_n\in\frak A$ and $\epsilon>0$ such that $U$ includes
$\{\pi:\pi\in G_E$, $\bar\mu(a_i\Bsymmdiff\pi a_i)\le 3\epsilon$ for
every $i\le n\}$.   For each $i\le n$, there is an
$a'_i\in\frak C\otimes\frak B$ such that
$\bar\mu(a_i\Bsymmdiff a'_i)\le\epsilon$.  Let $\frak B_0$ be a
finite subalgebra of $\frak B$ such that $a'_i\in\frak C\otimes\frak B_0$
for every $i\le n$.   Let $\delta>0$ be such that
$\delta\bar\lambda 1\le\epsilon$.
Then there is a neighbourhood $V$ of the identity in
$\Aut_{\bar\nu}\frak B$ such that
$\bar\nu(b\Bsymmdiff\phi b)\le\delta$ whenever $\phi\in V$ and
$b\in\frak B_0$.   If now $\pmb{\phi}=\family{e}{E}{\phi_e}$ belongs to
$V^E$, then for each $i\le n$ we can express $a'_i$ as
$\sup_{j\le m_i}c_{ij}\otimes b_{ij}$ where
$\langle c_{ij}\rangle_{j\le m_i}$ is a partition of unity in $\frak C$ and
$b_{ij}\in\frak B_0$ for every $j\le m_i$ (315Oa).   So

$$\eqalignno{\bar\mu(a'_i\Bsymmdiff\theta_E(\pmb{\phi})a'_i)
&\le\sum_{\Atop{j\le m_i}{e\in E}}\bar\mu(((c_{ij}\Bcap e)\otimes b_{ij})
    \Bsymmdiff\theta_E(\pmb{\phi})((c_{ij}\Bcap e)\otimes b_{ij}))\cr
\displaycause{because
$\langle(c_{ij}\cap e)\otimes b_{ij}\rangle_{j\le m_i,e\in E}$ is a
disjoint family with supremum $a'_i$}
&=\sum_{\Atop{j\le m_i}{e\in E}}\bar\mu(((c_{ij}\Bcap e)\otimes b_{ij})
    \Bsymmdiff((c_{ij}\Bcap e)\otimes\phi_e b_{ij}))\cr
&=\sum_{\Atop{j\le m_i}{e\in E}}\bar\mu((c_{ij}\Bcap e)
   \otimes(b_{ij}\Bsymmdiff\phi_e b_{ij}))\cr
&=\sum_{\Atop{j\le m_i}{e\in E}}\bar\lambda(c_{ij}\Bcap e)
  \cdot\bar\nu(b_{ij}\Bsymmdiff\phi_e b_{ij})
\le\delta\sum_{\Atop{j\le m_i}{e\in E}}\bar\lambda(c_{ij}\Bcap e)
\le\epsilon,\cr}$$

\noindent and

\Centerline{$\bar\mu(a_i\Bsymmdiff\theta_E(\pmb{\phi})a_i)
\le\bar\mu(a_i\Bsymmdiff a'_i)
  +\bar\mu(a'_i\Bsymmdiff\theta_E(\pmb{\phi})a'_i)
  +\bar\mu(\theta_E(\pmb{\phi})a'_i\Bsymmdiff\theta_E(\pmb{\phi})a_i)
\le 3\epsilon$.}

\noindent This is true for every $i\le n$, so $\theta_E(\pmb{\phi})\in U$
whenever $\pmb{\phi}\in V^E$.   As $U$ is arbitrary, $\theta_E$
is continuous.\ \Qed

\medskip

\qquad{\bf (iii)} $\theta_E^{-1}$ is continuous.
\Prf\ Let $V$ be a neighbourhood of the identity in
$\Aut_{\bar\nu}\frak B$.   Then there are $\epsilon>0$ and
$b_0,\ldots,b_n\in\frak B$ such that $\phi\in V$ whenever
$\phi\in\Aut_{\bar\nu}\frak B$ and
$\bar\nu(b_i\Bsymmdiff\phi b_i)\le\epsilon$ for every $i\le n$.   Let
$\delta>0$ be such that $\delta\le\epsilon\bar\lambda e$ for every
$e\in E$, and let $U$ be

\Centerline{$\{\pi:\pi\in G_E$,
$\bar\mu((e\otimes b_i)\Bsymmdiff\pi(e\otimes b_i))\le\delta$ whenever
$e\in E$ and $i\le n\}$.}

\noindent Then $U$ is a neighbourhood of the identity in $G_E$.   If
$\pmb{\phi}=\family{e}{E}{\phi_e}\in(\Aut_{\bar\nu}\frak B)^E$ is such that
$\theta_E(\pmb{\phi})\in U$, then for every $e\in E$ and $i\le n$ we have

\Centerline{$\bar\nu(b_i\Bsymmdiff\phi_eb_i)
=\Bover1{\bar\lambda e}
  \bar\mu((e\otimes b_i)\Bsymmdiff\theta_E(\pmb{\phi})(e\otimes b_i))
\le\Bover{\delta}{\bar\lambda e}
\le\epsilon$,}

\noindent so $\pmb{\phi}\in V^E$.   As $V$ is arbitrary, $\theta_E^{-1}$ is
continuous.\ \Qed

\medskip

\qquad{\bf (iv)} Putting these together, $\theta_E$ is a topological group
isomorphism.

\medskip

{\bf (c)} The next step is to show that $\bigcup_{E\in\Cal E}G_E$ is dense
in $G$.

\medskip

\quad{\bf (i)} Note first that there is an upwards-directed
family $\Bbb D$ of finite
subalgebras $\frak D$ of $\frak B$ such that if $\frak D\in\Bbb D$ then
every atom of $\frak D$ has the same measure, and $\bigcup\Bbb D$ is dense
in $\frak B$ (for the measure-algebra topology of $\frak B$).
\Prf\ If $\frak B$ is finite, with all its atoms of the same measure,
this is trivial;  take $\Bbb D=\{\frak B\}$.
Otherwise, because $\frak B$ is homogeneous, $(\frak B,\bar\nu)$ must be
isomorphic to the measure algebra $(\frak B_{\kappa},\bar\nu_{\kappa})$
of the usual measure on $\{0,1\}^{\kappa}$ for some infinite
cardinal $\kappa$, and we can take $\Bbb D$ to be the family of subalgebras
determined by finite subsets of $\kappa$.\ \Qed

\medskip

\quad{\bf (ii)} Suppose that $\pi\in G$,
$a_0,\ldots,a_n\in\frak A$ and $\epsilon>0$.
Let $\frak A_0$ be the
subalgebra of $\frak A$ generated by $a_0,\ldots,a_n$ and $A$ the set of
its atoms;  let $\eta>0$ be such that $12\eta\#(A)\le\epsilon$.
Consider subalgebras of
$\frak A$ of the form $\frak C_0\otimes\frak D$ where $\frak C_0$ is a
finite subalgebra of $\frak C$ and $\frak D\in\Bbb D$.   This is an
upwards-directed family of subalgebras, and the closure of its union
includes $c\otimes b$ whenever $c\in\frak C$ and $b\in\frak B$,
so is the whole of $\frak A$.   There must therefore be a finite subalgebra
$\frak C_0$ of $\frak C$, a $\frak D\in\Bbb D$, and
$a'$, $a''\in\frak C_0\otimes\frak D$, for each $a\in A$, such that
$\bar\mu(a\Bsymmdiff a')\le\eta$ and
$\bar\mu(\pi a\Bsymmdiff a'')\le\eta$ for every $a\in A$.
Note that this implies that

\Centerline{$|\bar\mu a'-\bar\mu a''|
\le|\bar\mu a'-\bar\mu a|+|\bar\mu a-\bar\mu a''|
\le 2\eta$}

\noindent for every $a\in A$.

\medskip

\quad{\bf (iii)}
Let $E\in\Cal E$ be the set of atoms of $\frak C_0$,
$D$ the set of atoms of
$\frak D$, and $\gamma$ the common measure of the members of $D$.   For
$e\in E$ and $a\in A$, set

\Centerline{$D_{ea}'=\{d:d\in D$, $\bar\mu((e\otimes d)\Bcap a)
>\Bover12\bar\mu(e\otimes d)\}$,
\quad$b_{ea}'=\sup D_{ea}'$,}

\Centerline{$D_{ea}''=\{d:d\in D$, $\bar\mu((e\otimes d)\Bcap\pi a)
>\Bover12\bar\mu(e\otimes d)\}$,
\quad$b_{ea}''=\sup D_{ea}''$.}

\noindent Note that as $A$ is disjoint,
$\family{a}{A}{D_{ea}'}$ is disjoint, for each $e$;  and similarly
$\family{a}{A}{D_{ea}''}$ is disjoint for each $e$, because
$\family{a}{A}{\pi a}$ is disjoint.   Next, for $a\in A$ and $e\in E$,
set $D_{ea}=\{d:d\in D$, $e\otimes d\Bsubseteq a'\}$.   Then, for each
$a\in A$,

$$\eqalign{\bar\mu(a'\Bsymmdiff\sup_{e\in E}e\otimes b_{ea}')
&=\sum_{\Atop{e\in E}{d\in D'_{ea}\setminus D_{ea}}}\bar\mu(e\otimes d)
   +\sum_{\Atop{e\in E}{d\in D_{ea}\setminus D'_{ea}}}
      \bar\mu(e\otimes d)\cr
&\le\sum_{\Atop{e\in E}{d\in D'_{ea}\setminus D_{ea}}}
       2\bar\mu((e\otimes d)\Bcap a)
   +\sum_{\Atop{e\in E}{d\in D_{ea}\setminus D'_{ea}}}
      2\bar\mu((e\otimes d)\Bsetminus a)\cr
&\le\sum_{\Atop{e\in E}{d\in D\setminus D_{ea}}}
      2\bar\mu((e\otimes d)\Bcap a)
   +\sum_{\Atop{e\in E}{d\in D_{ea}}}2\bar\mu((e\otimes d)\Bsetminus a)\cr
&=2\bar\mu(a\Bsetminus a')+2\bar\mu(a'\Bsetminus a)
=2\bar\mu(a\Bsymmdiff a')
\le 2\eta,\cr}$$

\noindent and
$\bar\mu(a\Bsymmdiff\sup_{e\in E}e\otimes b_{ea}')\le 3\eta$.
Similarly, passing through $a''$ in place of $a'$, we see that
$\bar\mu(\pi a\Bsymmdiff\sup_{e\in E}e\otimes b_{ea}'')\le 3\eta$.

Consequently, for any $a\in A$,

$$\eqalignno{
\sum_{e\in E}\bar\lambda e\cdot\gamma|\#(D_{ea}')-\#(D_{ea}'')|
&=\sum_{e\in E}|\bar\mu(e\otimes b_{ea}')-\bar\mu(e\otimes b_{ea}'')|\cr
&\le\sum_{e\in E}
  \bar\mu((e\otimes b_{ea}')\Bsymmdiff((e\otimes 1)\Bcap a))\cr
&\mskip100mu
  +|\bar\mu((e\otimes 1)\Bcap a)-\bar\mu((e\otimes 1)\Bcap\pi a)|\cr
&\mskip100mu
  +\bar\mu(((e\otimes 1)\Bcap\pi a)\Bsymmdiff(e\otimes b_{ea}''))\cr
&=\sum_{e\in E}
 \bar\mu((e\otimes b_{ea}')\Bsymmdiff((e\otimes 1)\Bcap a))\cr
&\mskip100mu
  +\bar\mu(((e\otimes 1)\Bcap\pi a)\Bsymmdiff(e\otimes b_{ea}''))\cr
\displaycause{because $\pi\in G$, so
$(e\otimes 1)\Bcap\pi a=\pi((e\otimes 1)\Bcap a)$ for every $e$}
&=\bar\mu(a\Bsymmdiff\sup_{e\in E}e\otimes b_{ea}')
     +\bar\mu(\pi a\Bsymmdiff\sup_{e\in E}e\otimes b_{ea}'')
\le 6\eta.\cr}$$

\medskip

\quad{\bf (iv)} Fix $e\in E$ for the moment.    For each
$a\in A$, take $\tilde D_{ea}'\subseteq D_{ea}'$,
$\tilde D_{ea}''\subseteq D_{ea}''$ such that
$\#(\tilde D_{ea}')=\#(\tilde D_{ea}'')=\min(\#(D_{ea}'),\#(D_{ea}''))$.
As $\family{a}{A}{\tilde D_{ea}'}$ and $\family{a}{A}{\tilde D_{ea}''}$ are
always disjoint families, there is a permutation $\psi_e:D\to D$ such that
$\psi_e[\tilde D_{ea}']=\tilde D_{ea}''$ for every $a\in A$.   Because
$(\frak B,\bar\nu)$ is homogeneous, there is a
$\phi_e\in\Aut_{\bar\nu}\frak B$ such that $\phi_ed=\psi_ed$ for every
$d\in D$.

\medskip

\quad{\bf (v)}
This gives us a family $\pmb{\phi}=\family{e}{E}{\phi_e}$.   Consider
$\theta_E(\pmb{\phi})$.   For each $a\in A$,

$$\eqalign{\bar\mu(\pi a\Bsymmdiff\theta_E(\pmb{\phi})(a))
&\le\bar\mu(\pi a\Bsymmdiff\sup_{e\in E}e\otimes b_{ea}'')
   +\bar\mu((\sup_{e\in E}e\otimes b_{ea}'')
      \Bsymmdiff\theta_E(\pmb{\phi})(\sup_{e\in E}e\otimes b_{ea}'))\cr
&\mskip220mu
   +\bar\mu(\theta_E(\pmb{\phi})(\sup_{e\in E}e\otimes b_{ea}')
              \Bsymmdiff\theta_E(\pmb{\phi})(a))\cr
&\le 3\eta
   +\sum_{e\in E}\bar\mu((e\otimes b_{ea}'')
                   \Bsymmdiff(e\otimes\phi_eb_{ea}'))
   +\bar\mu((\sup_{e\in E}e\otimes b_{ea}')\Bsymmdiff a)\cr
&\le 3\eta
   +\sum_{e\in E}\bar\lambda e
      \cdot\gamma\#(D_{ea}''\symmdiff\psi_e[D_{ea}'])
   +3\eta\cr
&\le 6\eta+\sum_{e\in E}\bar\lambda e
  \cdot\gamma(\#(D_{ea}''\setminus\tilde D_{ea}'')
         +\#(D_{ea}'\setminus\tilde D_{ea}'))\cr
&=6\eta+\sum_{e\in E}\bar\lambda e
  \cdot\gamma|\#(D_{ea}'')-\#(D_{ea}')|
\le 12\eta.\cr}$$

\medskip

\quad{\bf (vi)} Now, for each $i\le n$, set
$A_i=\{a:a\in A$, $a\Bsubseteq a_i\}$;  then

\Centerline{$\bar\mu(\pi a_i\Bsymmdiff\theta_E(\pmb{\phi})(a_i))
\le\sum_{a\in A_i}\bar\mu(\pi a\Bsymmdiff\theta_E(\pmb{\phi})(a))
\le 12\eta\#(A)\le\epsilon$,}

\noindent while $\theta_E{\pmb{\phi}}\in G_E$.
As $\phi$, $a_0,\ldots,a_n$ and $\epsilon$ are arbitrary,
$\bigcup_{E\in\Cal E}G_E$ is dense in $G$.

\medskip

\quad{\bf (vii)} Note also that if $E$, $E'\in\Cal E$, there is an
$F\in\Cal E$ such that $G_F\supseteq G_E\cup G_{E'}$.   \Prf\ Set
$F=\{e\Bcap e':e\in E$, $e'\in E'\}\setminus\{0\}$.   If
$\pmb{\phi}=\family{e}{E}{\phi_e}\in(\Aut_{\bar\nu}\frak B)^E$, define
$\family{f}{F}{\psi_f}\in(\Aut_{\bar\nu}\frak B)^F$ by saying that
$\psi_f=\phi_e$ whenever $f\in F$, $e\in E$ and $f\Bsubseteq e$.
Then it is easy to check that
$\theta_F(\family{f}{F}{\psi_f})=\theta_E(\pmb{\phi})$.   So
$G_F\supseteq G_E$;  similarly, $G_F\supseteq G_{E'}$.\ \Qed

So $\{G_E:E\in\Cal E\}$ is an upwards-directed family of subgroups of $G$
with dense union in $G$.

\medskip

{\bf (d)} At this point, we start looking at the rest of the hypotheses.

\medskip

\quad{\bf (i)} Suppose that $\frak B$ is atomless.   Then 494I tells us
that $\Aut_{\bar\nu}\frak B$ is extremely amenable.   So all the products
$(\Aut_{\bar\nu}\frak B)^E$ are extremely amenable (493Bd), all the $G_E$
are extremely amenable, and $G$ is extremely amenable by (c) and 493Bb.

\medskip

\quad{\bf (ii)} Suppose that $\frak B$ is finite.   Then
$\Aut_{\bar\nu}\frak B$ is finite, therefore amenable (449Cg);
all the products $(\Aut_{\bar\nu}\frak B)^E$ are amenable (449Ce),
and $G$ is amenable (449Cb).

\medskip

{\bf (e)} I have still to finish the case in which $\frak C$ is atomless
and $\frak B$ is finite.   If $\frak B=\{0\}$ then of course $G=\{\iota\}$
is extremely amenable, so we may take it that $\bar\lambda 1>0$.

\medskip

\quad{\bf (i)} Take
$\epsilon>0$, a neighbourhood $V$ of the identity in $G$, a finite set
$I\subseteq G$ and a finite family $\Cal A$ of zero sets in $G$.
Let $V_1$ be a neighbourhood of the identity in $G$ such that
$V_1^2\subseteq V^{-1}$.   By (c), there is an $E'\in\Cal E$ such that
$I\subseteq G_{E'}V_1$.   Set $k=\#(E')$.
$V_1$ is a neighbourhood of the identity for
the uniform topology on $G$ (494Cd), so there is a $\delta>0$ such that
$\pi\in V_1$ whenever $\pi\in G$ and the support of
$\pi$ has measure at most $\delta$ (494Cb).   Let $m$ be so large that
$m\delta\ge k\bar\lambda 1$ and
$(\Bover{m\delta}{\bar\lambda 1}-1)^2\ge m\ln(\Bover{2}{\epsilon})$;  set
$r=\lfloor\Bover{m\delta}{\bar\lambda 1}\rfloor$, so that
$2\exp(-\Bover{r^2}m)\le\epsilon$.

\medskip

\quad{\bf (ii)} For each $e\in E'$ let $D_e$ be a maximal disjoint set of
elements of measure $\bover1m\bar\lambda 1$ in $\frak C_e$;
let $E\supseteq\bigcup_{e\in E'}D_e$ be a maximal disjoint
set of elements of measure $\bover1m\bar\lambda 1$ in $\frak C$.   Note that
$c=1\Bsetminus\sup_{e\in E'}\sup D_e$ has
measure at most $\bover{k}m\bar\lambda 1\le\delta$.
Consequently $G_{E'}\subseteq G_EV_1$.   \Prf\
If $\pi'\in G_{E'}$, express $\theta_{E'}^{-1}(\pi')$ as
$\family{e}{E'}{\phi'_e}$.   Let
$\family{e}{E}{\phi_e}\in(\Aut_{\bar\nu}\frak B)^E$
be such that
$\phi_e=\phi'_{e'}$ whenever $e'\in E'$ and $e\in D_{e'}$, and set
$\pi=\theta_E(\family{e}{E}{\phi_e})$.   Then
$\pi a=\pi'a$ for every $a\Bsubseteq(1\Bsetminus c)\otimes 1$,
so $\pi^{-1}\pi'$ is
supported by $c\otimes 1$ and belongs to $V_1$.
Thus $\pi'\in\pi V_1$;  as $\pi'$
is arbitrary, $G_{E'}\subseteq G_EV_1$.\ \QeD\  It follows that
$I\subseteq G_EV_1^2\subseteq G_EV^{-1}$.

\medskip

\quad{\bf (iii)} Set $H=\Aut_{\bar\nu}\frak B$, and let $\lambda_0$ be the
Haar probability measure on $H$, that is, the uniform probability measure.
Let $\lambda$ be the product measure on $H^E$, so that $\lambda$ is the
Haar probability measure on $H^E$.
Let $\nu$ be the image measure $\lambda\theta_E^{-1}$ on
$G$.   If $\pi\in I$, then $G_E$ meets $\pi V$, so there is a
$\pmb{\phi}\in H^E$ such that $\theta_E(\pmb{\phi})\in\pi V$;  now

\Centerline{$\nu(\theta_E(\pmb{\phi})F)
=\lambda\theta_E^{-1}[\theta_E(\pmb{\phi})F]
=\lambda(\pmb{\phi}\theta_E^{-1}[F])
=\lambda\theta_E^{-1}[F]
=\nu F$}

\noindent for every $F\subseteq G$, and in particular for every
$F\in\Cal A$.   Thus $\nu$ satisfies condition (ii) of 493C.

\medskip

\quad{\bf (iv)} Set

\Centerline{$U
=\{\family{e}{E}{\phi_e}:\phi_e\in\Aut_{\bar\nu}\frak B$ for every $e\in
E$, $\#(\{e:\phi_e$ is not the identity$\})\le r\}$.}

\noindent Then $\theta_E[U]\subseteq V$.   \Prf\ If
$\pmb{\phi}=\family{e}{E}{\phi_e}$ belongs to $U$, then
$b=\sup\{e:e\in E$, $\phi_e$ is not the identity$\}$ has measure at most
$\bover{r}{m}\bar\lambda 1\le\delta$, while $b$ supports
$\theta_E(\pmb{\phi})$.   So
$\bar\lambda(a\Bsymmdiff\theta_E(\pmb{\phi})(a))\le\delta$ for every
$a\in\frak A$, and $\theta_E(\pmb{\phi})\in V$.\ \Qed

Let $\rho$ be the normalized Hamming metric on $H^E$ (492D).
If $\pmb{\phi}=\family{e}{E}{\phi_e}$ and
$\pmb{\psi}=\family{e}{E}{\psi_e}$ belong to $H^E$ and
$\rho(\pmb{\phi},\pmb{\psi})\le\bover{r}{m}$, then
$\{e:\phi_e\psi^{-1}_e$ is not the identity$\}$ has at most $r$ members,
and $\pmb{\phi}\pmb{\psi}^{-1}\in U$, that is, $\pmb{\phi}\in U\pmb{\psi}$.
So if $W\subseteq H^E$ is such that $\lambda W\ge\bover12$,

$$\eqalignno{\lambda(UW)
&\ge\lambda\{\pmb{\phi}:\rho(\pmb{\phi},W)\le\Bover{r}m\}
\ge 1-2\exp(-m(\Bover{r}m)^2)\cr
\displaycause{492Ea}
&\ge 1-\epsilon\cr}$$

\noindent by the choice of $m$ and $r$.   Transferring this to $G$,
remembering that $\theta_E:H^E\to G$ is an injective homomorphism, we get

\Centerline{$\nu(VF)
=\lambda\theta_E^{-1}[VF]
\ge\lambda(U\theta_E^{-1}[F])
\ge 1-\epsilon$}

\noindent whenever $F\subseteq G$ and $\nu F\ge\bover12$.   So $\nu$
satisfies the first condition of 493C.

\medskip

\quad{\bf (v)} As $\epsilon$, $V$, $I$ and $\Cal A$ were arbitrary, 493C
tells us that $G$ is extremely amenable.   This completes the proof.
}%end of proof of 494J

\leader{494K}{Lemma} Let $(\frak A,\bar\mu)$ be a totally finite measure
algebra, and give $\AmuA$ its weak topology.   Let $G$  be a subgroup
of $\AmuA$ with fixed-point subalgebra $\frak C$, and suppose that
$G=\{\pi:\pi\in\AmuA$, $\pi c=c$ for every $c\in\frak C\}$.
Then $G$ is amenable, and if every atom of $\frak A$ belongs to $\frak C$,
then $G$ is extremely amenable.

\proof{{\bf (a)} We need the structure theorems of \S333;  the final one
333R is the best adapted to our purposes here.   I repeat some of the
special notation used in that theorem.
For $n\in\Bbb N$, set $\frak B_n=\Cal P(n+1)$ and let $\bar\nu_n$ be the
uniform probability measure on $n+1$, so that $\frak B_n$ has $n+1$ atoms
of the same measure;  for an infinite cardinal $\kappa$, let
$(\frak B_{\kappa},\bar\nu_{\kappa})$ be the measure algebra of the usual
measure on $\{0,1\}^{\kappa}$.   Then
333R tells us that there are a
partition of unity $\langle c_i\rangle_{i\in I}$ in $\frak C$, where $I$
is a countable set of cardinals, and a measure-preserving isomorphism
$\theta:\frak A\to\frak A'=\prod_{i\in I}\frak C_{c_i}\tensorhat\frak B_i$
such that
$\theta c=\familyiI{(c\Bcap c_i)\otimes 1}$ for every $c\in\frak C$.
In particular, for any $i\in I$,

$$\eqalign{(\theta c_i)(j)
&=c_i\otimes 1\text{ if }j=i,\cr
&=0\text{ otherwise},\cr}$$

\noindent that is, $\theta[\frak A_{c_i}]$ is just
the principal ideal of $\frak A'$ corresponding to the factor
$\frak C_{c_i}\tensorhat\frak B_i$.   Thus we have an
isomorphism
$\theta_i:\frak A_{c_i}\to\frak C_{c_i}\tensorhat\frak B_i$
such that $\theta_ic=c\otimes 1$ for every $c\in\frak C_{c_i}$.

\medskip

{\bf (b)} For each $i\in I$, set
$H_i=\{\pi\restrp\frak A_{c_i}:\pi\in G\}$.   Because $\pi c_i=c_i$ for
every $\pi\in G$, $H_i$ is a subgroup of $\Aut\frak A_{c_i}$, and
$\pi\mapsto\pi\restrp\frak A_{c_i}$ is a group homomorphism from $G$ to
$H_i$.   Set $\Theta(\pi)=\familyiI{\pi\restrp\frak A_{c_i}}$ for
$\pi\in G$.   Then $\Theta:G\to\prod_{i\in I}H_i$ is a group homomorphism.
Because $\familyiI{c_i}$ is a partition of unity in $\frak A$,
$\Theta$ is injective.   In the other
direction, suppose that $\pmb{\phi}=\familyiI{\phi_i}$ is such that every
$\phi_i$ is a measure-preserving automorphism of $\frak A_{c_i}$ and
$\phi_ic=c$ for every $c\in\frak C_{c_i}$.   Then we have a $\pi\in\AmuA$
such that $\pi a=\phi_ia$ whenever $i\in I$ and $a\Bsubseteq c_i$;  it is
easy to check that $\pi\in G$ and now $\Theta(\pi)=\pmb{\phi}$.
Thus

\Centerline{$H_i=\{\phi:\phi\in\Aut\frak A_{c_i}$ is measure-preserving,
$\phi c=c$ for every $c\in\frak C_{c_i}\}$}

\noindent for each $i$, and $\Theta$ is
a group isomorphism between $G$ and $\prod_{i\in I}H_i$.

\medskip

{\bf (c)} As in part (b) of the proof of 494J, the next step is to confirm
that $\Theta$ is a homeomorphism for
the weak topologies.   The argument is very similar.

\medskip

\quad{\bf (i)} If $U$ is a neighbourhood of the identity in $G$,
then there are a finite set $K\subseteq A$ and an $\epsilon>0$ such that
$U$ includes $\{\pi:\pi\in G$, $\bar\mu(a\Bsymmdiff\pi a)\le2\epsilon$
for every $a\in K\}$.   Let $J\subseteq I$ be a finite set such that
$\sum_{i\in I\setminus J}\bar\mu c_i\le\epsilon$, and set

$$\eqalign{V
&=\{\familyiI{\phi_i}:\phi_i\in H_i\text{ for every }i\in I,\,\cr
&\mskip50mu
\bar\mu((a\Bcap c_i)\Bsymmdiff\phi_i(a\Bcap c_i))
\le\Bover{\epsilon}{1+\#(J)}\text{ for every }i\in J\text{ and }a\in K\}.
\cr}$$

\noindent Then $V$ is a neighbourhood of the identity in
$\prod_{i\in I}H_i$.
If $\pmb{\phi}=\familyiI{\phi_i}$ belongs to $V$, and
$\pi=\Theta^{-1}(\pmb{\phi})$, then, for
$a\in K$,

$$\eqalign{\bar\mu(a\Bsymmdiff\pi a)
&=\sum_{i\in I}\bar\mu((a\Bcap c_i)\Bsymmdiff\pi(a\Bcap c_i))
=\sum_{i\in I}\bar\mu((a\Bcap c_i)\Bsymmdiff\phi_i(a\Bcap c_i))\cr
&\le\sum_{i\in J}\bar\mu((a\Bcap c_i)\Bsymmdiff\phi_i(a\Bcap c_i))
   +\sum_{i\in I\setminus J}\bar\mu c_i
\le\Bover{\epsilon\#(J)}{\#(J)+1}+\epsilon
\le 2\epsilon,\cr}$$

\noindent and $\pi\in U$.   As $U$ is arbitrary, $\Theta^{-1}$ is
continuous.

\medskip

\quad{\bf (ii)} If $V$ is a neighbourhood of the identity in
$\prod_{i\in I}H_i$, then there are a finite $J\subseteq I$, finite sets
$K_j\subseteq\frak A_{c_j}$ for $j\in J$, and an $\epsilon>0$ such that
$\pmb{\phi}=\familyiI{\phi_i}$ belongs to $V$ if
$\pmb{\phi}\in\prod_{i\in I}H_i$ and
$\bar\mu(a\Bsymmdiff\phi_ja)\le\epsilon$
whenever $j\in J$ and $a\in K_j$.   In this case.

\Centerline{$U=\{\pi:\pi\in G$, $\bar\mu(a\Bsymmdiff\pi a)\le\epsilon$
whenever $a\in\bigcup_{j\in J}K_j\}$}

\noindent is a neighbourhood of the identity in $G$, and $\Theta(\pi)\in V$
whenever $\pi\in U$.   As $V$ is arbitrary, $\Theta$ is continuous.

\medskip

{\bf (d)} Now observe that under the isomorphism $\theta_i$ the
group $H_i$ corresponds to the group of measure-preserving automorphisms of
$\frak C_{c_i}\tensorhat\frak B_i$ fixing $c\otimes 1$ for every
$c\in\frak C_{c_i}$.   By 494J, $H_i$ is amenable.
By (b)-(c) and 449Ce, $G$ is amenable.

\medskip

{\bf (e)} Finally, suppose that every atom of $\frak A$ belongs to
$\frak C$, and look more closely at the algebras
$\frak C_{c_i}\tensorhat\frak B_i$ and the groups $H_i$.   If
$i\in I$ is an infinite cardinal, then $\frak B_i$ is homogeneous and 494J
tells us that $H_i$ is
extremely amenable.   If $0\in I$, then $\frak B_0=\{0,1\}$ and
$\frak C_{c_0}\tensorhat\frak B_0$ is isomorphic to $\frak C_{c_0}$;  in
this case, $H_0$ consists of the identity alone, and is surely extremely
amenable.   If $i\in I$ is finite and not $0$, then $\frak B_i$ is finite;
and also $\frak C_{c_i}$ is atomless.   \Prf\Quer\ If $c\in\frak C_{c_i}$
is an atom, take an atom $b$ of $\frak B_i$;  then $c\otimes b$ is an atom
of $\frak C_{c_i}\tensorhat\frak B_i$, and $\theta^{-1}(c\otimes b)$ is an
atom of $\frak A$ not belonging to $\frak C$.\ \Bang\QeD\  So in this case
again, 494J tells us that $H_i$ is extremely amenable.   Thus $G$ is
isomorphic to a product of extremely amenable groups and is extremely
amenable (493Bd).
}%end of proof of 494K

\leader{494L}{Theorem}\discrversionA{\footnote{Expands and replaces the
former 4{}93H.}}{} Let $(\frak A,\bar\mu)$ be a measure algebra and $G$
a full subgroup of $\AmuA$, with the topology induced by the weak
topology of $\AmuA$.   Then $G$ is amenable.
If every atom of $\frak A$ with finite measure
belongs to the fixed-point subalgebra of $G$, then $G$ is
extremely amenable.

%given that G has many involutions, do we also have to suppose that it's
%full?

\proof{{\bf (a)} To begin with, suppose that $(\frak A,\bar\mu)$ is totally
finite.   Let $\frak C$ be the fixed-point subalgebra of $G$, and
$G'\supseteq G$ the subgroup $\{\pi:\pi\in\AmuA$, $\pi c=c$ for every
$c\in\frak C\}$.   Then $G$ is dense in $G'$, by 494Ge.
$\frak C$ is of course the fixed-point subalgebra of $G'$, so $G'$ is
amenable (494K) and $G$ is amenable (449F(a-ii)).
If every atom of $\frak A$ belongs to $\frak C$, then $G'$ and $G$ are
extremely amenable, by 494K and 493Bf.

\medskip

{\bf (b)} Now for the general case.

\medskip

\quad{\bf (i)} For each $a\in\frak A^f$, set

\Centerline{$G_a=\{\pi:\pi\in G$, $\pi$ is supported by $a\}$,
\quad$H_a=\{\pi\restrp\frak A_a:\pi\in G_a\}$.}

\noindent Then $H_a$ is a full subgroup of
$\Aut_{\bar\mu\restrp\frak A_a}\frak A_a$ (494Ha), and is isomorphic
to $G_a$;  moreover, the isomorphism is a homeomorphism for the weak
topologies.   \Prf\ Set $\theta(\pi)=\pi\restrp\frak A_a$ for
$\pi\in G_a$.   ($\alpha$) If $V$ is a neighbourhood of the identity in
$H_a$, let $\delta>0$ and $K\in[\frak A_a]^{<\omega}$ be such that

\Centerline{$V
\supseteq\{\phi:\phi\in H_a$, $\bar\mu(b\Bsymmdiff\phi b)\le\delta$
for every $b\in K\}$;}

\noindent then

\Centerline{$U
=\{\pi:\pi\in G_a$, $\bar\mu(b\Bsymmdiff\pi b)\le\delta$
for every $b\in K\}$}

\noindent is a neighbourhood of the identity in $G_a$, and
$\theta(\pi)\in V$
for every $\pi\in U$.   So $\theta$ is continuous.   ($\beta$) If
$U$ is a neighbourhood of the identity in $G_a$, let
$\delta>0$ and $K\in[\frak A]^{<\omega}$ be such that

\Centerline{$U
\supseteq\{\pi:\pi\in G_a$, $\bar\mu(b\Bsymmdiff\pi b)\le\delta$
for every $b\in K\}$;}

\noindent then

\Centerline{$V
=\{\phi:\phi\in G_a$,
  $\bar\mu((b\Bcap a)\Bsymmdiff\phi(b\Bcap a))\le\delta$
  for every $b\in K\}$}

\noindent is a neighbourhood of the identity in $G_a$, and
$\theta^{-1}(\phi)\in U$ for every $\phi\in V$, because

\Centerline{$\theta^{-1}(\phi)(b)=\phi(b\Bcap a)\Bcup(b\Bsetminus a)$,
\quad$b\Bsymmdiff\theta^{-1}(\phi)(b)=(b\cap a)\Bsymmdiff\phi(b\Bcap a)$}

\noindent for every $\phi\in G_a$ and $b\in\frak A$.    Thus $\theta^{-1}$
is continuous.\ \Qed

By (a), $H_a$, and therefore $G_a$, is amenable.

\medskip

\quad{\bf (ii)} $H=\bigcup_{a\in\frak A^f}G_a$ is dense in $G$.   \Prf\
If $\pi\in G$, $a_0,\ldots,a_n\in\frak A^f$ and $\epsilon>0$, set
$a=\sup_{i\le n}a_i$ and $b=a\Bcup\pi a$.   Then there is a
$\phi\in G$ such that $\phi$ agrees with $\pi$ on $\frak A_a$ and
$\phi$ is supported by $b$ (494Ga).   In this case, $\phi\in G_b$ and
$\bar\mu(\pi a_i\Bsymmdiff\phi a_i)=0\le\epsilon$ for every $i\le n$.\ \Qed

Since $\family{a}{\frak A^f}{G_a}$ is upwards-directed, and every $G_a$ is
amenable, $G$ is amenable (449Cb).

\medskip

\quad{\bf (iii)} If every atom of $\frak A$ of finite measure is fixed
under the action of $G$, then every atom of $\frak A_a$ is fixed under the
action of $H_a$, for every $a\in\frak A^f$.   So every $H_a$ and
every $G_a$ is extremely amenable, and $G$ is extremely amenable, by
493Bb.
}%end of proof of 494L

\leader{494M}{Lemma} Let $\frak A$ be a Boolean algebra, $G$ a full
subgroup of $\Aut\frak A$, and $V\subseteq G$ a symmetric set.
Let $\sim_G$ be the orbit equivalence relation on $\frak A$
induced by the action of $G$, so that $a\sim_Gb$ iff there is a $\phi\in G$
such that $\phi a=b$.   Suppose that $a\in\frak A$ and
$\pi$, $\pi'\in G$ are such that

\inset{$\pi=\cycle{b\,_{\pi}\,c}$ and $\pi'=\cycle{b'\,_{\pi'}\,c'}$ are
exchanging involutions supported by $a$,

$b\sim_Gb'$ and $a\Bsetminus(b\Bcup c)\sim_Ga\Bsetminus(b'\Bcup c')$,

$\pi\in V$,

whenever $\phi\in G$ is supported by $a$ there is a $\psi\in V$ agreeing
with $\phi$ on $\frak A_a$.}

\noindent Then $\pi'\in V^3$.

\proof{{\bf (a)} There is a $\phi\in G$, supported by $a$, such that
$\phi\pi'=\pi\phi$.   \Prf\ We know that there are $\phi_0$, $\phi_1\in G$
such that $\phi_0(a\Bsetminus(b'\Bcup c'))=a\Bsetminus(b\Bcup c)$ and
$\phi_1b'=b$.   Set $\phi_2=\pi\phi_1\pi'$;  then $\phi_2\in G$,
$\pi\phi_2=\phi_1\pi'$, $\phi_2\pi'=\pi\phi_1$ and $\phi_2c'=c$.   Because
$(a\Bsetminus(b'\Bcup c'),b',c',1\Bsetminus a)$ and
$(a\Bsetminus(b\Bcup c),b,c,1\Bsetminus a)$ are partitions of unity in
$\frak A$, there is a $\phi\in\Aut\frak A$ such that

$$\eqalign{\phi d
&=\phi_0d\text{ if }d\Bsubseteq a\Bsetminus(b'\Bcup c'),\cr
&=\phi_1d\text{ if }d\Bsubseteq b',\cr
&=\phi_2d\text{ if }d\Bsubseteq c',\cr
&=d\text{ if }d\Bsubseteq 1\Bsetminus a\cr}$$

\noindent (381C once more);  because $G$ is full, $\phi\in G$.   Of course
$\phi$ is supported by $a$.   Now

$$\eqalignno{\phi\pi'd
&=\phi d=\pi\phi d
\text{ if }d\Bsubseteq a\Bsetminus(b'\Bcup c')\cr
\noalign{\noindent(because $\phi d=\phi_0d$ is disjoint from $b\Bcup c$),}
&=\phi_2\pi'd=\pi\phi_1d=\pi\phi d
\text{ if }d\Bsubseteq b',\cr
&=\phi_1\pi'd=\pi\phi_2d=\pi\phi d
\text{ if }d\Bsubseteq c',\cr
&=\phi d=d=\pi\phi d
\text{ if }d\Bsubseteq 1\Bsetminus a.\cr}$$

\noindent So $\phi\pi'=\pi\phi$.\ \Qed

\medskip

{\bf (b)} By our hypothesis, there is a $\psi\in V$ agreeing with $\phi$ on
$\frak A_a$.   In this case,

$$\eqalign{\psi\pi'd
&=\phi\pi'd=\pi\phi d=\pi\psi d\text{ if }d\Bsubseteq a,\cr
&=\psi d=\pi\psi d\text{ if }d\Bsubseteq 1\Bsetminus a.\cr}$$

\noindent So $\psi\pi'=\pi\psi$ and

\Centerline{$\pi'=\psi^{-1}\pi\psi\in V^3$}

\noindent because $V$ is symmetric and $\pi$, $\psi\in V$.
}%end of proof of 494M

\leader{494N}{Lemma}
Let $(\frak A,\bar\mu)$ be a probability algebra and
$G\subseteq\AmuA$ a full subgroup with fixed-point subalgebra $\frak C$.
Suppose that $\frak A$ is relatively atomless over $\frak C$.
For $a\in\frak A$, let $u_a\in L^{\infty}(\frak C)$ be the conditional
expectation of $\chi a$ on $\frak C$, and
let $G_a$ be $\{\pi:\pi\in G$ is supported by $a\}$.   Suppose that
$a\Bsubseteq e$ in $\frak A$ and $V\subseteq G$ are such that

\inset{$V$ is symmetric\cmmnt{, that is, $V=V^{-1}$},

for every $\phi\in G_e$ there is a $\psi\in V$ such that
$\phi$ and $\psi$ agree on $\frak A_e$,

there is an involution in $V$ with support $a$,

$u_a\le\bover23 u_e$.}

\noindent Then $G_a
\subseteq V^{18}\cmmnt{\mskip5mu=\{\pi_1\ldots\pi_{18}:
   \pi_1,\ldots,\pi_{18}\in V\}}$.

\proof{{\bf (a)(i)} Note that 494Gb, in the language of conditional
expectations, tells us that if $b$, $c\in\frak A$ then
$b\sim_Gc$ in the notation of 494M iff $u_b=u_c$.
Similarly, 494Gd tells us that if
$\familyiI{b_i}$ and $\familyiI{c_i}$ are disjoint families in $\frak A$
and $u_{b_i}=u_{c_i}$ for every $i\in I$, there is a $\pi\in G$ such that
$\pi b_i=c_i$ for every $i$.

\medskip

\quad{\bf (ii)} It follows that every non-zero $b\in\frak A$ is the
support of an involution in $G$.   \Prf\ Because $\frak A$ is relatively
atomless over $\frak C$, there is a $c\Bsubseteq b$ such
that $u_c=\bover12u_b$ (494Ad);  now there is a $\phi\in G$ such that
$\phi c=b\Bsetminus c$, and
$\pi=\cycle{c\,_{\phi}\,b\Bsetminus c}$ is an
involution, belonging to $G$ (381Sd once more), with support $b$.\ \Qed

\medskip

{\bf (b)} If $\pi'\in G$ is an involution with
support $a'\Bsubseteq e$ and $u_{a'}=u_a$, then
$\pi'\in V^3$.   \Prf\ Let $\pi_0\in V$ be an involution with support
$a$.   Because $\frak A$ is Dedekind complete, $\pi_0$ is an exchanging
involution (382Fa);  express it as $\cycle{b_0\,_{\pi_0}\,c_0}$ and
$\pi'$ as $\cycle{b'\,_{\pi'}\,c'}$.   Because $\pi b_0=c_0$,
$u_{b_0}=u_{c_0}$, while $u_{b_0}+u_{c_0}=u_{b_0\Bcup c_0}=u_a$;
so $u_{b_0}=\bover12u_a$.
Similarly, $u_{b'}=\bover12u_{a'}=u_{b_0}$.   On the other hand,

\Centerline{$u_{e\Bsetminus a}=u_e-u_a=u_e-u_{a'}=u_{e\Bsetminus a'}$}

\noindent and $e\Bsetminus a\sim_Ge\Bsetminus a'$.   So the conditions of
494M are satisfied and $\pi'\in V^3$.\ \Qed

\medskip

{\bf (c)} Now suppose that $\pi$ is any involution in $G$
with support included in $a$.   Then $\pi\in V^6$.
\Prf\ Let $b$, $c$ be such that
$\pi=\cycle{b\,_{\pi}\,c}$.   Once again,

\Centerline{$u_b=u_c\le\Bover12u_a$,
\quad$u_{e\Bsetminus a}=u_e-u_a\ge\Bover12u_a$,}

\noindent so we can find $d\Bsubseteq e\Bsetminus a$ such that
$u_d=u_c$, while there is also a $b_1\Bsubseteq b$ such that
$u_{b_1}=\bover12u_b$.   Set

\Centerline{$c_1=\pi b_1$,
\quad$b_2=b\Bsetminus b_1$,
\quad$c_2=\pi b_2$,
\quad$\pi_1=\cycle{b_1\,_{\pi}\,c_1}$,
\quad$\pi_2=\cycle{b_2\,_{\pi}\,c_2}$;}

\noindent then $\pi_1$ and $\pi_2$ are involutions, with supports
$b_1\Bcup c_1$ and $b_2\Bcup c_2$ respectively, belonging to $G$
and such that $\pi=\pi_1\pi_2$.
Next, (a-iii) tells us that there are involutions $\pi_3$, $\pi_4\in G$
with supports $d$ and $a\Bsetminus(b\Bcup c)$ (if $a=b\Bcup c$, set
$\pi_4=\iota$).
Since $\pi_1$, $\pi_2$, $\pi_3$ and $\pi_4$ have disjoint supports,
they commute (381Ef).   Consequently
$\pi_1\pi_3\pi_4$, $\pi_2\pi_3\pi_4$ are involutions, belonging to $G$,
with supports
$a_1=b_1\Bcup c_1\Bcup(a\Bsetminus(b\Bcup c))\Bcup d$,
$a_2=b_2\Bcup c_2\Bcup(a\Bsetminus(b\Bcup c))\Bcup d$ respectively.
But now observe that

\Centerline{$u_{a_1}
=u_{b_1}+u_{c_1}+u_a-u_b-u_c+u_d
=u_b+u_a-u_b=u_a$,}

\noindent and similarly $u_{a_2}=u_a$.   By (b), both
$\pi_1\pi_3\pi_4$ and $\pi_2\pi_3\pi_4$ belong to $V^3$.
But this means that

\Centerline{$\pi=\pi_1\pi_2
=\pi_1\pi_2\pi_3^2\pi_4^2=\pi_1\pi_3\pi_4\pi_2\pi_3\pi_4$}

\noindent belongs to $V^6$, as claimed.\ \Qed

\medskip

{\bf (d)} By 382N, every member of $G_a$ is
expressible as the product of at most three involutions belonging to
$G_a$, so belongs to $V^{18}$.
}%end of proof of 494N

\leader{494O}{Theorem}\cmmnt{ ({\smc Kittrell \& Tsankov 09})}
Suppose that $(\frak A,\bar\mu)$ is an atomless probability algebra and
$G\subseteq\AmuA$ is a full ergodic subgroup\cmmnt{ (definition:
395Ge)}, with the topology induced by the uniform topology of $\AmuA$.

(a) If $V\subseteq G$ is symmetric
and $G$ can be covered by countably many left translates of $V$ in $G$,
then $V^{38}=\{\pi_1\pi_2\ldots\pi_{38}:\pi_1,\ldots,\pi_{38}\in V\}$
is a neighbourhood of the identity in $G$.

(b) If $H$ is a topological group such that for every neighbourhood
$W$ of the identity in $H$ there is a countable set $D\subseteq H$ such
that $H=DW$,
and $\theta:G\to H$ is a group homomorphism, then $\theta$ is continuous.

\proof{{\bf (a)(i)} Let $\sequencen{\psi_n}$ be a sequence in $G$ such that
$G=\bigcup_{n\in\Bbb N}\psi_nV$.   It may help if I note straight away that
$\iota\in V^2$.   \Prf\ There is an $n\in\Bbb N$ such that
$\iota\in\psi_nV$, that is, $\psi_n^{-1}\in V$;  as $V$ is symmetric,
$\psi_n\in V$ and $\iota=\psi_n\psi_n^{-1}$ belongs to $V^2$.\ \Qed

\medskip

\quad{\bf (ii)} As before, set
$G_a=\{\pi:\pi\in G$, $\pi$ is supported by  $a\}$ for
$a\in\frak A$.   Now there is a non-zero $e\in\frak A$ such that for every
$\pi\in G_e$ there is a $\phi\in V^2$ agreeing with $\pi$ on
$\frak A_e$.   \Prf\ Because $\frak A$ is atomless, there is a disjoint
sequence $\sequencen{b_n}$ in $\frak A\setminus\{0\}$.   \Quer\ Suppose, if
possible, that for every $n\in\Bbb N$ there is a
$\pi_n\in G_{b_n}$ such that there is no
$\phi\in V^2$ agreeing with $\pi_n$ on $\frak A_{b_n}$.
If $n\in\Bbb N$, then
$V^2=(\psi_nV)^{-1}\psi_nV$ and $\pi_n=\iota^{-1}\pi_n$,
so there must be a $\pi'_n\in G_{b_n}$, either $\iota$ or $\pi_n$,
not agreeing with
$\phi$ on $\frak A_{b_n}$ for any $\phi\in\psi_nV$.   Define
$\psi\in\AmuA$ by the formula

$$\eqalign{\psi d
&=\pi'_nd\text{ if }n\in\Bbb N\text{ and }d\Bsubseteq b_n,\cr
&=d\text{ if }d\Bcap\sup_{n\in\Bbb N}b_n=0.\cr}$$

\noindent Because $G$ is full, $\psi\in G$ and
there is an $m\in\Bbb N$ such that
$\psi\in\psi_mV$.   But now $\pi'_m$ agrees with $\psi$ on
$\frak A_{b_m}$, contrary to the choice of $\pi'_m$.\ \BanG\  So one of the
$b_n$ will serve for $e$.\ \Qed

\medskip

\quad{\bf (iii)} There is an involution $\pi\in V^2$, supported by $e$,
such that $\bar\mu(\supp\pi)\le\bover23\bar\mu e$.   \Prf\ Take disjoint
$b$, $b'\Bsubseteq e$ such that $\bar\mu b=\bar\mu b'=\bover12\bar\mu e$.
Because $G$ is full and ergodic, there is a $\phi\in G$ such that
$\phi b=b'$.   (By 395Gf, the fixed-point subalgebra of $G$
is $\{0,1\}$, so we can apply 494Gb.)
For every $d\in\frak A_b$, set $\phi_d=\cycle{d\,_{\phi}\,\phi d}$.
Because $G$ is full, $\phi_d\in G$.   Observe that

\Centerline{$\phi_c\phi_d
=\phi_{c\Bsetminus d}\phi_{c\Bcap d}\phi_{c\Bcap d}\phi_{d\Bsetminus c}
=\phi_{c\Bsetminus d}\phi_{d\Bsetminus c}
=\phi_{c\Bsymmdiff d}$}

\noindent for all $c$, $d\Bsubseteq b$.   Set
$A_n=\{d:d\in\frak A_b$, $\phi_d\in\psi_nV\}$ for each $n\in\Bbb N$.
Since $\frak A_b$ is complete under its measure metric, there is an
$n\in\Bbb N$ such that $A_n$ is non-meager;  because $\frak A$ is atomless,
$\frak A_b$ has no isolated points;  so there are $d_0$, $d_1\in A_n$
such that $0<\bar\mu(d_0\Bsymmdiff d_1)\le\bover13\bar\mu e$.   Set
$d=d_0\Bsymmdiff d_1$.   Then

\Centerline{$\phi_d
=\phi_{d_0}\phi_{d_1}=\phi_{d_0}^{-1}\phi_{d_1}
\in V^{-1}\psi_n^{-1}\psi_nV=V^2$,}

\noindent and we can take $\phi_d$ for $\pi$.\ \Qed

\medskip

\quad{\bf (iv)} Taking $a=\supp\pi$ in (iv), $a$ and $e$ satisfy the
conditions of 494N with respect to $V^2$ and $\frak C=\{0,1\}$, so
$G_a\Bsubseteq(V^2)^{18}=V^{36}$.

\medskip

\quad{\bf (v)}  Finally, there is a $\delta>0$ such that, in the language
of 494C,
$G\cap U(1,\delta)\Bsubseteq V^{38}$.   \Prf\Quer\ Otherwise, we can find
for each $n\in\Bbb N$ a
$\pi_n\in G\cap U(1,2^{-n-1}\bar\mu a)\setminus V^{38}$.   Set
$\pi'_n=\psi_n\pi_n\psi_n^{-1}$, $b_n=\supp\pi'_n$;  then
$\bar\mu b_n=\bar\mu(\supp\pi_n)\le 2^{-n-1}\bar\mu a$ for each $n$
(381Gd).   So
$b=\sup_{n\in\Bbb N}b_n$ has measure at most $\bar\mu a$, and there is a
$\phi\in G$ such that $\phi b\Bsubseteq a$.
In this case,
there is an $n\in\Bbb N$ such that $\phi^{-1}\in\psi_nV$, that is,
$\phi\psi_n\in V^{-1}=V$.   Now
$\pi=\phi\psi_n\pi_n\psi_n^{-1}\phi^{-1}$ has support
$\phi b_n\Bsubseteq a$, so belongs to $V^{36}$.   But this means that
$\pi_n=\psi_n^{-1}\phi^{-1}\pi\phi\psi_n$ belongs to $V^{38}$, contrary to
the choice of $\pi_n$.\ \Bang\Qed

So $V^{38}$ is a neighbourhood of $\iota$ in $G$, as claimed.

\medskip

{\bf (b)} Let $W$ be a neighbourhood of the identity in $H$.   Then there
is a symmetric neighbourhood $W_1$ of the identity in $H$ such that
$W_1^{38}\subseteq W$.   Set $V=\theta^{-1}[W_1]$.   Let $W_2$ be a
neighbourhood of the identity in $H$ such that
$W_2^{-1}W_2\subseteq W_1$, and
$\sequencen{y_n}$ a sequence in $H$ such that
$H=\bigcup_{n\in\Bbb N}y_nW_2$.
For each $n\in\Bbb N$, choose $\psi_n\in G$ such that
$\theta(\psi_n)\in y_nW_2$ whenever $\theta[G]$ meets $y_nW_2$.
If $\pi\in G$, there is an $n\in\Bbb N$ such that
$\theta(\pi)\in y_nW_2$;   in this case, $\theta(\psi_n)\in y_nW_2$, so

\Centerline{$\theta(\psi_n^{-1}\pi)\in W_2^{-1}y_n^{-1}y_nW_2
\subseteq W_1$}

\noindent and
$\psi_n^{-1}\pi\in V$.   Thus $\pi\in\psi_nV$;  as $\pi$ is arbitrary,
$G=\bigcup_{n\in\Bbb N}\psi_nV$.   By (a), $V^{38}$ is a neighbourhood
of $\iota$;  but
$V^{38}\subseteq\theta^{-1}[W_1^{38}]\subseteq\theta^{-1}[W]$, so
$\theta^{-1}[W]$ is a neighbourhood of $\iota$.   As $W$ is
arbitrary, $\theta$ is continuous (4A5Fa).
}%end of proof of 494O

\leader{494P}{Remark} Note that if a topological group $H$ is either
Lindel\"of or ccc, it satisfies the condition of (b) above.
\prooflet{\Prf\
Let $W$ be an open neighbourhood of the identity in $H$.
($\alpha$) If $H$ is Lindel\"of,
the result follows immediately from the fact that $\{yW:y\in H\}$
is an open cover of $H$.   ($\beta$) If $H$ is ccc, let $W_1$ be an open
neighbourhood of the identity such that $W_1W_1^{-1}\subseteq W$, and
$D\subseteq H$ a maximal set such that $\family{y}{D}{yW_1}$ is disjoint.
Then $D$ is countable.
If $x\in H$, there is a $y\in D$ such that $xW_1\cap yW_1\ne\emptyset$,
that is, $x\in yW_1W_1^{-1}\subseteq yW$;  thus $H=DW$.\ \Qed}
\cmmnt{  See also 494Yh.}

%for Lindel\"of but not ccc, or the other way round, see mt49bits

\leader{494Q}{}\cmmnt{ Some of the same ideas lead to an interesting
group-theoretic property of the automorphism groups here.

\medskip

\noindent}{\bf Theorem}\cmmnt{ (see {\smc Droste Holland \& Ulbrich 08})}
Let $(\frak A,\bar\mu)$ be a probability algebra and $G$ a full
subgroup of $\AmuA$ such that
$\frak A$ is relatively atomless over the fixed-point subalgebra $\frak C$
of $G$.   Let $\sequencen{V_n}$ be a non-decreasing
sequence of subsets of $G$ such that $V_n^2\subseteq V_{n+1}$ for
every $n$ and $G=\bigcup_{n\in\Bbb N}V_n$.   Then there is an $n\in\Bbb N$
such that $G=V_n$.

\proof{{\bf (a)} For the time being (down to the end of (e) below), suppose
that every $V_n$ is symmetric.
As in 494N, for each $a\in\frak A$ write $u_a$ for the
conditional expectation of $\chi a$ on $\frak C$, and set
$G_a=\{\pi:\pi\in G$, $\pi$ is supported by $a\}$.

\medskip

{\bf (b)} There are an $\alpha_1>0$, an $a_1\in\frak A$ and an
$n_0\in\Bbb N$ such that $u_{a_1}=\alpha_1\chi 1$ and
for every $\pi\in G_{a_1}$ there is a $\phi\in V_{n_0}$ agreeing with
$\pi$ on $\frak A_{a_1}$.   \Prf\Quer\ Otherwise, because $\frak A$ is
relatively atomless, we can choose inductively a disjoint sequence
$\sequencen{b_n}$ in $\frak A$ such that $u_{b_n}=2^{-n-1}\chi 1$ for each
$n$ (use 494Ad).   For each $n\in\Bbb N$ there must be a $\pi_n\in G_{b_n}$
such that there is no $\phi\in V_n$ agreeing with $\pi_n$ on
$\frak A_{b_n}$.   Because $G$ is full,
there is a $\phi\in G$ agreeing with $\pi_n$ on $\frak A_{b_n}$ for every
$n$.   But now $\phi\notin\bigcup_{n\in\Bbb N}V_n=G$.\ \Bang\Qed

\medskip

{\bf (c)} There is an
$a_0\Bsubseteq a_1$ such that $u_{a_0}=\bover23\alpha_1\chi 1$
and there is an involution $\pi\in G$ with support $a_0$.
\Prf\ Take disjoint $a$, $a'\Bsubseteq a_1$ such that
$u_a=u_{a'}=\bover13\alpha_1\chi 1$ (494Ad again).
Set $a_0=a\Bcup a'$, so that
$u_{a_0}=\bover23\alpha_1\chi 1$.   There is a $\phi\in G$ such
that $\phi a=a'$, and $\pi=\cycle{a\,_{\phi}\,a'}$ is an involution in $G$
with support $a_0$.\ \QeD\
Let $n_1\ge n_0$ be such that $\pi\in V_{n_1}$.

\medskip

{\bf (d)} By 494N, $G_{a_0}\subseteq V_{n_1}^{18}\subseteq V_{n_1+5}$.
Taking $k\ge\Bover{3}{\alpha_1}$, 494Ad once more gives us a disjoint
family $\ofamily{i}{k}{d_i}$ in $\frak A$ such that
$u_{d_i}=\bover1k\chi 1$ for every $i<k$;  since
$\sum_{i=0}^{k-1}\bar\mu d_i=1$, $\ofamily{i}{k}{d_i}$ is a partition of
unity, while $u_{d_i}\le\bover13\alpha_1\chi 1$ for every $i$.
For $i$, $j<k$, let
$\phi_{ij}\in G$ be such that $\phi_{ij}(d_i\Bcup d_j)\Bsubseteq a_0$
(494Gc).
Let $n_2\ge n_1+5$ be such that $\phi_{ij}\in V_{n_2}$ for all $i$, $j<k$.
Then any involution in $G$ belongs to $V_{n_2}^{3k^2}$.   \Prf\ Let
$\pi\in G$ be an involution;  by 382Fa again, we can
express it as $\cycle{e\,_{\pi}\,e'}$.
For $i$, $j<k$, set $e_{ij}=e\Bcap d_i\Bcap\pi d_j$,
$e'_{ij}=\pi e_{ij}=e'\Bcap\pi d_i\Bcap d_j$;  set
$\pi_{ij}=\cycle{e_{ij}\,_{\pi}\,e'_{ij}}$.
In this case, because all the $e_{ij}$ and $e'_{ij}$ are disjoint,
$\langle\pi_{ij}\rangle_{i,j<k}$ is a commuting family, and we can talk of
$\prod_{i,j<k}\pi_{ij}$, which of course is equal to $\pi$.   Now, for each
$i$, $j<k$,

\Centerline{$\phi_{ij}\pi_{ij}\phi_{ij}^{-1}
=\cycle{\phi_{ij}e_{ij}\,_{\phi_{ij}\pi\phi_{ij}^{-1}}\,\phi_{ij}e'_{ij}}$}

\noindent (381Sb) belongs to
$G_{a_0}\subseteq V_{n_1+5}\subseteq V_{n_2}$.   So

\Centerline{$\pi_{ij}
=\phi_{ij}^{-1}\phi_{ij}\pi_{ij}\phi_{ij}^{-1}\phi_{ij}$}

\noindent belongs to $V_{n_2}^3$ and $\pi=\prod_{i,j<k}\pi_{ij}$ belongs to
$V_{n_2}^{3k^2}$. \Qed

\medskip

{\bf (e)} Since, by 382N again, every member of $G$ is expressible as
a product of at most three involutions belonging to $G$,
$G\subseteq V_{n_2}^{9k^2}\subseteq V_n$, where
$n=n_2+\lceil\log_2(9k^2)\rceil$.

\medskip

{\bf (f)} This completes the proof on the assumption that every $V_n$ is
symmetric.   For the general case, set $W_n=V_n\cap V_n^{-1}$ for every
$n$.   Then $\sequencen{W_n}$ is a non-decreasing sequence of symmetric
sets with union $G$, and

\Centerline{$W_n^2\subseteq V_n^2\cap V_n^{-2}=V_n^2\cap(V_n^2)^{-1}
\subseteq V_{n+1}\cap V_{n+1}^{-1}=W_{n+1}$}

\noindent for every $n$, so there is an $n\in\Bbb N$ such that
$G=W_n=V_n$.
}%end of proof of 494Q

\leader{494R}{}\cmmnt{ There are many alternative versions of 494Q;
see, for instance, 494Xm.   Rather than attempt a portmanteau
result to cover them all, I give one which can be applied to the measure
algebra of Lebesgue measure on $\Bbb R$ and
indicates some of the new techniques required.

\medskip

\noindent}{\bf Theorem} Let $(\frak A,\bar\mu)$ be an atomless
localizable measure algebra, and $G$ a full ergodic
subgroup of $\AmuA$.   Let $\sequencen{V_n}$ be a
non-decreasing sequence of subsets of $G$, covering $G$, with
$V_n^2\subseteq V_{n+1}$ for every $n$.
Then there is an $n\in\Bbb N$ such that $G=V_n$.

\proof{{\bf (a)} My aim is to mimic the proof of 494Q.   We have a
simplification because $G$ is ergodic, but 494M will be applied in a
different way.   As before, it will be enough to consider the case in which
every $V_n$ is symmetric;  as before, I will write
$G_a$ for $\{\pi:\pi\in G$, $\pi$ is supported by $a\}$.

Because $G$ is ergodic, $\frak A$ must be quasi-homogeneous (374G);  as it
is also atomless, there is an infinite cardinal $\kappa$ such
that $\frak A_a$ is homogeneous, with Maharam type $\kappa$, for every
$a\in\frak A^f\setminus\{0\}$ (374H).
If $(\frak A,\bar\mu)$ is totally finite,
then the result is immediate from 494Q, normalizing
the measure if necessary.   So I will assume that $(\frak A,\bar\mu)$ is
not totally finite.   In this case, the orbits of $G$ can be described
in terms of `magnitude' (332Ga).   If $a\in\frak A^f$, $\mag a=\bar\mu a$;
otherwise, $\mag a$ is the cellularity of $\frak A_a$, and there will be
a disjoint family in $\frak A_a$ of this cardinality (332F).   Set
$\lambda=\mag 1\ge\omega$;  then whenever $a\in\frak A$ and
$\mag a=\lambda$, there is a partition of unity $\sequencen{a_n}$ in
$\frak A_a$ such that $\mag a_n=\lambda$ for every $n$.

\medskip

{\bf (b)} The key fact, corresponding to 494Gd, is as follows:  if
$\familyiI{a_i}$ and $\familyiI{b_i}$ are partitions of unity in
$\frak A$ such that $\mag a_i=\mag b_i$ for every $i\in I$, then there is a
$\phi\in G$ such that $\phi a_i=b_i$ for every $i\in I$.

\medskip

\Prf{\bf (i)} Consider first the case in which all the $a_i$, $b_i$ have
finite measure.   In this case, let $\family{j}{J}{(c_j,\pi_j,d_j)}$ be a
maximal family such that

\inset{----- $\family{j}{J}{c_j}$
is a disjoint family in $\frak A^f\setminus\{0\}$,

----- $\family{j}{J}{d_j}$ is a disjoint family in $\frak A$,

----- for every $j\in J$, $\pi_j\in G$, $\pi_jc_j=d_j$ and there is an
$i\in I$ such that $c_j\Bsubseteq a_i$ and $d_j\Bsubseteq b_i$.}

\noindent Set $a=1\Bsetminus\sup_{j\in J}c_j$ and
$b=1\Bsetminus\sup_{j\in J}d_j$.   \Quer\ If $a\ne 0$, there is an
$i\in I$ such that $a\Bcap a_i\ne 0$.   In this case,

$$\eqalign{\sum_{j\in J}\bar\mu(b_i\Bcap d_j)
&=\sum_{j\in J,d_j\Bsubseteq b_i}\bar\mu d_j
=\sum_{j\in J,d_j\Bsubseteq b_i}\bar\mu\pi_j^{-1}d_j\cr
&=\sum_{j\in J,c_j\Bsubseteq a_i}\bar\mu c_j
<\bar\mu a_i
=\bar\mu b_i,\cr}$$

\noindent and $b\Bcap b_i\ne 0$.   Because $G$ is ergodic, there is a
$\pi\in G$ such that $\pi(a\Bcap a_i)\Bcap(b\Bcap b_i)\ne 0$.
Setting $d=a\Bcap a_i\Bcap\pi^{-1}(b\Bcap b_i)$, we ought to have added
$(d,\pi,\pi d)$ to $\family{j}{J}{(c_j,\pi_j,d_j)}$.\ \Bang

Thus $a=0$;  similarly, $b=0$ and $\family{j}{J}{c_j}$,
$\family{j}{J}{d_j}$ are partitions of unity in $\frak A$.   Because
$\frak A$ is Dedekind complete, there is a $\phi\in\Aut\frak A$ such that
$\phi d=\pi_jd$ whenever $j\in J$ and $d\Bsubseteq c_j$, and now
$\phi\in G$ and $\phi a_i=b_i$ for every $i\in I$.

\medskip

\quad{\bf (ii)} For the general case, refine the partitions
$\familyiI{a_i}$ and $\familyiI{b_i}$ as follows.   For each $i\in I$, if
$\bar\mu a_i=\bar\mu b_i$ is finite, take $\lambda_i=1$, $c_{i0}=a_i$
and
$d_{i0}=b_i$;  otherwise, take $\lambda_i=\mag a_i=\mag b_i$, and let
$\ofamily{\xi}{\lambda_i}{c_{i\xi}}$, $\ofamily{\xi}{\lambda_i}{d_{i\xi}}$
be partitions of unity in $\frak A_{a_i}$, $\frak A_{b_i}$ respectively
with $\bar\mu c_{i\xi}=\bar\mu d_{i\xi}=1$ for every $\xi<\lambda_i$
(332I).   Now (i) tells us that there is a $\phi\in G$ such that
$\phi c_{i\xi}=d_{i\xi}$ whenever $i\in I$ and $\xi<\lambda_i$, in which
case $\phi a_i$ will be equal to $b_i$ for every $i\in I$.\ \Qed

\medskip

{\bf (c)} There are an $a_1\in\frak A$ and an $n_0\in\Bbb N$
such that $\mag a_1=\mag(1\Bsetminus a_1)=\lambda$ and whenever
$\pi\in G_{a_1}$ there is a
$\phi\in V_{n_0}$ agreeing with $\pi$ on $\frak A_{a_1}$.
\Prf\Quer\ Otherwise, let
$\sequencen{b_n}$ be a partition of unity in $\frak A$ such that
$\mag b_n=\lambda$ for every $n$.   For each $n\in\Bbb N$ there must be a
$\pi_n\in G_{b_n}$
such that there is no $\phi\in V_n$ agreeing with $\pi_n$ on
$\frak A_{b_n}$.   Now there is a
$\phi\in G$ agreeing with $\pi_n$ on $\frak A_{b_n}$ for every
$n$.   But in this case $\phi\notin\bigcup_{n\in\Bbb N}V_n=G$.\ \Bang\Qed

\medskip

{\bf (d)} There is an
$a_0\Bsubseteq a_1$ such that $\mag a_0=\mag(a_1\Bsetminus a_0)=\lambda$
and there is an involution $\pi_0\in G$ with support $a_0$.
\Prf\ Take disjoint $a$, $a'$, $a''\Bsubseteq a_1$ all of magnitude
$\lambda$;  by (b), there is a $\phi\in G$ such
that $\phi a=a'$, and $\pi_0=\cycle{a\,_{\phi}\,a'}$ is an involution
with support $a_0=a\Bcup a'$ of magnitude $\lambda$, while
$a_1\Bsetminus a_0\Bsupseteq a''$ also has magnitude $\lambda$.\ \QeD\
Let $n_1\ge n_0$ be such that $\pi_0\in V_{n_1}$.

\medskip

{\bf (e)} If $\pi\in G$ is an involution with
support $b_1\Bsubseteq a_1$ and
$\mag b_1=\mag(a_1\Bsetminus b_1)=\lambda$, then
$\pi\in V_{n_1}^3$.   \Prf\ Express $\pi_0$ and $\pi$ as
$\cycle{a\,_{\pi_0}\,a'}$ and $\cycle{b\,_{\pi}\,b'}$ respectively, and set
$\tilde a=a_1\Bsetminus(a\Bcup a')$, $\tilde b=a_1\Bsetminus b_1$;
then $a$, $a'$, $b$, $b'$, $\tilde a$ and $\tilde b$ must all have
magnitude $\lambda$.   By (b), there is a $\phi_0\in G$ such that

\Centerline{$\phi_0b=a$,
\quad$\phi_0b'=a'$,
\quad$\phi_0\tilde b=\tilde a$,
\quad$\phi_0(1\Bsetminus a_1)=1\Bsetminus a_1$.}

\noindent In particular, $a\sim_Gb$ and $\tilde a\sim_G\tilde b$, in the
language of 494M, and (c) tells us that the final hypothesis of 494M is
satisfied;  so $\pi\in V_{n_1}^3$.\ \Qed

\medskip

{\bf (f)} Now suppose that $\pi$ is any involution in $G_{a_0}$.
Then $\pi\in V_{n_1}^6$.   \Prf\ Let $b$, $b'$ be such that
$\pi=\cycle{b\,_{\pi}\,b'}$.   Next take
disjoint $c$, $c'\Bsubseteq a_1\Bsetminus a_0$ such that
$\mag c=\mag c'=\mag(a_1\Bsetminus(a_0\Bcup c\Bcup c'))=\lambda$.
Then there there is an involution $\pi'\in G$ exchanging $c$ and $c'$,
and $\pi'$, $\pi\pi'$ are both involutions in $G_{a_1}$ satisfying the
conditions of (e).   So both belong to $V_{n_1}^3$ and
$\pi=\pi\pi'\pi'\in V_{n_1}^6$.\ \Qed

\medskip

{\bf (g)} Set $d_0=a_0$, $d_1=a_1\Bsetminus a_0$ and $d_2=1\Bsetminus a_1$.
Then $d_0$, $d_1$ and $d_2$ all have magnitude $\lambda$, so for all $i$,
$j<3$ there is a $\phi_{ij}\in G$ such that
$\phi_{ij}(d_i\Bcup d_j)=d_0$.
Let $n_2\ge n_1+3$ be such that $\phi_{ij}\in V_{n_2}$ for all $i$, $j<3$.
Then any involution in $G$ belongs to $V_{n_2}^{27}$.   \Prf\ Let
$\pi\in G$ be an involution;  express it as $\cycle{e\,_{\pi}\,e'}$.
For $i$, $j<3$, set $e_{ij}=e\Bcap d_i\Bcap\pi d_j$,
$e'_{ij}=\pi e_{ij}=e'\Bcap\pi d_i\Bcap d_j$;  set
$\pi_{ij}=\cycle{e_{ij}\,_{\pi}\,e'_{ij}}$.
In this case, because all the $e_{ij}$ and $e'_{ij}$ are disjoint,
$\langle\pi_{ij}\rangle_{i,j<3}$ is a commuting family, and we can talk of
$\prod_{i,j<3}\pi_{ij}$, which of course is equal to $\pi$.   Now, for each
pair $i$, $j<3$,

\Centerline{$\phi_{ij}\pi_{ij}\phi_{ij}^{-1}
=\cycle{\phi_{ij}e_{ij}\,_{\phi_{ij}\pi\phi_{ij}^{-1}}\,\phi_{ij}e'_{ij}}$}

\noindent is an involution in $G_{a_0}$, so belongs to
$V_{n_1}^6\subseteq V_{n_1+3}\subseteq V_{n_2}$.   So

\Centerline{$\pi_{ij}
=\phi_{ij}^{-1}\phi_{ij}\pi_{ij}\phi_{ij}^{-1}\phi_{ij}$}

\noindent belongs to $V_{n_2}^3$ and $\pi=\prod_{i,j<3}\pi_{ij}$ belongs to
$V_{n_2}^{27}$. \Qed

\medskip

{\bf (h)} Since every member of $G$ is expressible as
a product of at most three involutions in $G$ (382N once more),
$G=V_{n_2}^{81}=V_{n_2+7}$.
}%end of proof of 494R

\exercises{\leader{494X}{Basic exercises (a)}
%\spheader 494Xa
Let $(\frak A,\bar\mu)$ be a semi-finite measure algebra.
Show that the natural action of $\AmuA$ on $\frak A^f$ identifies $\AmuA$
with a subgroup of the isometry group $G$ of $\frak A^f$ when $\frak A^f$
is given its measure metric, and that the weak topology on $\AmuA$
corresponds to the topology of pointwise convergence
on $G$  as described in 441G.
%494A

\spheader 494Xb Let $(\frak A,\bar\mu)$ be a semi-finite measure algebra.
Show
that the following are equiveridical:  (i) $\frak A$ is purely atomic and
has at most finitely many atoms of any fixed measure;
(ii) $\AmuA$ is locally compact in its weak topology;
(iii) $\AmuA$ is compact in its uniform topology;  (iv) $\AmuA$ has a Haar
measure for its weak topology.
%494A

\spheader 494Xc
Let $(\frak A,\bar\mu)$ be a measure algebra.   Show that the weak topology
on $\AmuA$ is that induced by the product topology on $\frak A^{\frak A}$
if $\frak A$ is given its measure-algebra topology.
%494B
% \bar\mu(\pi c\Bsymmdiff\phi c)
%  \le 2\bar\mu(\phi c\Bsymmdiff(\pi c\Bsymmdiff\phi c))

\spheader 494Xd Let $(\frak A,\bar\mu)$ be a semi-finite measure algebra,
$G$ a subgroup of $\AmuA$ and $\overline{G}$ its closure for the weak
topology on $\AmuA$.   Show that $G$ is
ergodic iff $\overline{G}$ is ergodic.
%494B

\spheader 494Xe
Let $I$ be a set, $\nu_I$ the usual measure on
$\{0,1\}^I$, and $(\frak B_I,\bar\nu_I)$ its measure algebra.
Let $\Psi$ be the group of
measure space automorphisms $g$ of $\{0,1\}^I$ for which there is a
countable set $J\subseteq I$ such that for every $x\in\{0,1\}^I$ there is a
finite set $K\subseteq J$ such that $g(x)(i)=x(i)$ for every
$i\in I\setminus K$.   For $g\in\Psi$, let
$\pi_g\in\Aut_{\bar\nu_I}\frak B_I$ be the corresponding automorphism
defined by saying that
$\pi_g(E^{\ssbullet})=g^{-1}[E]^{\ssbullet}$ whenever $\nu_I$ measures $E$.
(i) Show that
$G=\{\pi_g:g\in\Psi\}$ is a full subgroup of $\Aut_{\bar\nu_I}\frak B_I$.
(ii) Show that $G$ is ergodic and
dense in $\Aut_{\bar\nu_I}\frak B_I$ for the weak topology on
$\Aut_{\bar\nu_I}\frak B_I$.
%3{}83Xl 494B

\spheader 494Xf\dvAformerly{4{}93Yd}
Let $(\frak A,\bar\mu)$ be the measure algebra of
Lebesgue measure on $[0,1]$, and give $\AmuA$ its weak topology.   (i)
Show that the entropy function $h$ of 385M is Borel measurable.
\Hint{385Xr.}   (ii) Show that the set of ergodic measure-preserving
automorphisms is a dense G$_{\delta}$ set.
\Hint{let $D\subseteq\frak A$ be a countable dense set.   Show that
$\pi\in\AmuA$ is ergodic iff
$\inf_{n\in\Bbb N}\|\Bover1{n+1}
\sum_{i=0}^n\chi(\pi^id)-\bar\mu d\cdot\chi 1\|_1=0$ for every
$d\in D$.}
%494B

\spheader 494Xg Let $(\frak A,\bar\mu)$ be an atomless semi-finite
measure algebra.
(i) Show that $\AmuA$ is metrizable under its weak topology iff
$(\frak A,\bar\mu)$ is $\sigma$-finite and has countable Maharam type.
%iff $\{\iota\}$ is G$_{\delta}$
(ii) Show that $\AmuA$ is metrizable under its uniform topology iff
$(\frak A,\bar\mu)$ is $\sigma$-finite.
%iff $\{\iota\}$ is G$_{\delta}$
%494Bc 494Cg

\spheader 494Xh Let $(\frak A,\bar\mu)$ be a localizable measure algebra.
Show that $\supp:\AmuA\to\frak A$ is continuous for the uniform topology on
$\AmuA$ and the measure-algebra topology on $\frak A$.
%494C

\spheader 494Xi Let $(\frak A,\bar\mu)$ be an atomless homogeneous
probability algebra.   Show that there is a weakly mixing
measure-preserving automorphism of $\frak A$ which is not mixing.
\Hint{372Yj}.
%494E

\spheader 494Xj Let $(\frak A,\bar\mu)$ be a probability algebra,
$\pi\in\AmuA$ and $T$ the corresponding operator on
$L^2_{\Bbb C}=L^2_{\Bbb C}(\frak A,\bar\mu)$.   (i)
Show that $\pi$ is not ergodic iff there is a non-zero
$v\in L^2_{\Bbb C}$ such that $\int v=0$ and $Tv=v$.   (ii) Show that
$\pi$ is not weakly mixing iff there is a non-zero $v\in L^2_{\Bbb C}$
such that $\int v=0$ and $Tv$ is a multiple of $v$.   (iii)
Let $(\frak A\tensorhat\frak A,\bar\lambda)$ be the probability algebra
free product of $(\frak A,\bar\mu)$ with itself (definition:  325K), and
$\tilde\pi\in\Aut_{\bar\lambda}(\frak A\tensorhat\frak A)$ the automorphism
such that $\tilde\pi(a\otimes b)=\pi a\otimes\pi b$ for all $a$,
$b\in\frak A$.   Show that $\pi$ is weakly mixing iff $\tilde\pi$ is
ergodic iff $\tilde\pi$ is weakly mixing.
\Hint{consider $T_{\tilde\pi}(v\otimes\bar v)$.}
%494D %494Xi

\spheader 494Xk Let $(\frak A,\bar\mu)$ be a measure algebra.   For
$\pi\in\AmuA$ let $T_{\pi}$ be the corresponding operator on
$L^2_{\Bbb C}=L^2_{\Bbb C}(\frak A,\bar\mu)$.   Show that
$(\pi,v)\mapsto T_{\pi}v:\AmuA\times L^2_{\Bbb C}\to L^2_{\Bbb C}$ is
continuous if $\AmuA$ is given its weak topology and $L^2_{\Bbb C}$ its
norm topology.
%494E

\spheader 494Xl\dvAformerly{4{}93Yb}
Let $(\frak A,\bar\mu)$ be an atomless probability
algebra;  give $\frak A$ its measure metric.   Show that the
isometry group of $\frak A$, with its topology of pointwise convergence,
is extremely amenable.   \Hint{every
isometry of $\frak A$ is of the form $a\mapsto c\Bsymmdiff\pi a$, where
$c\in\frak A$ and $\pi\in\AmuA$;  now use 493Bc.}
%494L

\spheader 494Xm Let $\frak A$ be a homogeneous Dedekind
complete Boolean algebra,
and $\sequencen{V_n}$ a non-decreasing sequence of subsets of
$\Aut\frak A$, covering $\Aut\frak A$, with $V_n^2\subseteq V_{n+1}$ for
every $n$.   Show that there is an $n\in\Bbb N$ such that
$\Aut\frak A=V_n$.
%494R

\leader{494Y}{Further exercises (a)}
%\spheader 494Ya
Let $(\frak A,\bar\mu)$ be a semi-finite measure algebra.
For $\pi\in\AmuA$, let
$T_{\pi}:L^0(\frak A)\to L^0(\frak A)$ be the Riesz space automorphism
such that $T_{\pi}(\chi a)=\chi(\pi a)$ for every $a\in\frak A$
(364P).   Take any $p\in\coint{1,\infty}$ and write $L^p_{\bar\mu}$ for
$L^p(\frak A,\bar\mu)$ as defined in 366A.   Set
$G_p=\{T_{\pi}\restr L^p_{\bar\mu}:\pi\in\AmuA\}$.
(i) Show that
$\pi\mapsto T_{\pi}\restr L^p_{\bar\mu}$ is a topological group isomorphism
between $\AmuA$, with its weak topology, and $G_p$, with the strong
operator topology from
$\eurm B(L^p_{\bar\mu};L^p_{\bar\mu})$ (3A5I).
(ii) Show that $G_p$ is closed in $\eurm B(L^p_{\bar\mu};L^p_{\bar\mu})$.
(iiI) Show that if $(\frak A,\bar\mu)$ is totally finite, then
$\pi\mapsto T_{\pi}\restr L^1_{\bar\mu}$ is a topological group isomorphism
between $\AmuA$, with its uniform topology, and $G_1$ with
the topology of uniform convergence on weakly compact subsets of
$L^1_{\bar\mu}$.
%494A 494Xk useful but function spaces off-topic a bit

\spheader 494Yb Let $\frak A$ be any Boolean algebra.   For
$I\subseteq\frak A$, set
$U_I=\{\pi:\pi\in\Aut\frak A$, $\pi a=a$ for every $a\in I\}$.
(i) Show that
$\{U_I:I\in[\frak A]^{<\omega}\}$ is a base of neighbourhoods of the
identity for a Hausdorff topology on $\Aut\frak A$ under which
$\Aut\frak A$ is a topological group.   (ii) Show that if $\frak A$ is
countable then $\Aut\frak A$, with this topology, is a Polish group.
%494A

\spheader 494Yc Let $(\frak A,\bar\mu)$ be a
semi-finite measure algebra.   Show
that $\AmuA$, with its weak topology, is weakly $\alpha$-favourable.
%494B

\spheader 494Yd Let $\bar\mu$ be counting measure on $\Bbb N$.   (i) Show
that if we identify $\Aut_{\bar\mu}\Cal P\Bbb N$ with the set of
permutations on $\Bbb N$, the weak topology of $\Aut_{\bar\mu}\Cal P\Bbb N$
is the
topology induced by the usual topology of $\NN$.   (ii) Show that there is
a comeager conjugacy class in $\Aut_{\bar\mu}\Cal P\Bbb N$.
%494B 

\spheader 494Ye ({\smc Rosendal 09})
Let $(\frak A,\bar\mu)$ be the measure algebra of Lebesgue
measure on $[0,1]$, and give $\AmuA$ its weak topology.
Let $\Cal V$ be a countable base of
open neighbourhoods of $\iota$ in $\AmuA$.
(i) Show that if $I\subseteq\Bbb N$ is infinite and $V\in\Cal V$, then
$\{\pi:\pi\in\AmuA$, $\pi^n=\iota$ for some $n\in I\}$ is dense in $\AmuA$,
and that $B(I,V)=\{\pi:\pi^n\in V$ for some $n\in I\}$ is dense and open.
(ii) Show that if $I\subseteq\Bbb N$ is infinite then
$C(I)=\bigcap_{V\in\Cal V}B(I,V)$ is comeager, and is a union of
conjugacy classes.   (iii) Show that
$\bigcap\{C(I):I\in[\Bbb N]^{\omega}\}=\{\iota\}$.   (iv) Show that every
conjugacy class in $\AmuA$ is meager.
%494Yd 494B

\spheader 494Yf Let $(\frak A,\bar\mu)$ be a measure algebra.   Suppose
that $\pi\in\AmuA$ is aperiodic.   Show that the set of conjugates of $\pi$
in $\AmuA$ is dense for the weak topology on $\AmuA$.
%494E mt49bits

\spheader 494Yg Let $(\frak A,\bar\mu)$ be a probability algebra.   Show
that the set of weakly mixing automorphisms, with the subspace topology
inherited from the weak topology of $\AmuA$, is weakly $\alpha$-favourable.
%494E mt49bits

\spheader 494Yh Let $G$ be a Hausdorff
topological group.   Show that the following
are equiveridical:  (i) for every neighbourhood $V$ of the identity in $G$
there is a countable set $D\subseteq G$ such that $G=DV$;  (ii)
there is a family $\familyiI{H_i}$ of Polish groups such that
$G$ is isomorphic, as topological group, to a subgroup of
$\prod_{i\in I}H_i$.
%494O mt49bits

\spheader 494Yi Let $(\frak A,\bar\mu)$ be an atomless
$\sigma$-finite measure algebra, and $G$ a full ergodic subgroup of
$\AmuA$.   Let $V\subseteq\AmuA$ be a symmetric set such that
countably many left translates of $V$ cover $\AmuA$.   Show that $V^{228}$
is a neighbourhood of $\iota$ for the uniform topology on $\AmuA$.
%494O mt49bits

\spheader 494Yj Let $(\frak A,\bar\mu)$ be a purely atomic probability
algebra with two atoms of measure $2^{-n-2}$ for each $n\in\Bbb N$;  give
$\AmuA$ its uniform topology.   (i) Show that $\AmuA\cong\Bbb Z_2^{\Bbb N}$
is compact, therefore not extremely amenable, and can be regarded as a
linear space over the field $\Bbb Z_2$.
(ii) Show that there is a strictly increasing sequence of
subgroups of $\AmuA$ with union $\AmuA$.
(iii) Show that there is a subgroup $V$ of $\AmuA$, not open,
such that $\AmuA$ is covered
by countably many translates of $V$.   (iv) Show that there is a
discontinuous homomorphism from $\AmuA$ to a Polish group.
%494Q $\Bbb Z_2^{\Bbb N}$ has a Hamel basis

\spheader 494Yk Let $\frak A$ be a Dedekind $\sigma$-complete Boolean
algebra.   Show that it is not the union of a strictly increasing sequence
of subalgebras.
%494Q


}%end of exercises

\leader{494Z}{Problems} For $k\in\Bbb N$, say that a topological
group $G$ is {\bf $k$-Steinhaus} if whenever $V\subseteq G$ is a symmetric
set, containing the identity, such that countably many left translates of
$V$ cover $G$, then $V^k$
is a neighbourhood of the identity.   For your favourite groups, determine
the smallest $k$, if any, for which they are $k$-Steinhaus.
\cmmnt{(See {\smc Rosendal \& Solecki 07}.)}

\endnotes{
\Notesheader{494}
In 494B-494C %494B 494C
I run through properties of the weak and uniform topologies
of $\AmuA$ in parallel.   The effect is to emphasize their similarities,
but they are of course very different -- for instance, consider 494Xg, or
the contrast between 494Cg and 494Ge.   Both have expressions in terms of
standard topologies on spaces of linear operators (494Ya), and the weak
topology corresponds to the pointwise topology of an isometry group
(494Xa).   There are other more or less
natural topologies which can be considered (e.g., that of 494Yb), but at
present the two examined in this section seem to be the most important.
I spell out 494Be and 494Ci to show that the groups
here provide interesting examples of Polish groups with striking
properties.

The formulation of 494D is specifically designed for the application in
the proof of 494E(b-ii);
the version in 494Xj(ii) is much closer to the real strength of the idea,
and takes us directly to one of the important reasons for being interested
in weakly mixing automorphisms in 494Xj(iii).   The proof of 494D
through Bochner's theorem saves space here, but fails to signal
the concept of `spectral resolution' of a unitary operator on a Hilbert
space ({\smc Riesz \& Sz.-Nagy 55}, \S109),
which is an important tool in understanding
operators $T_{\pi}$ and hence automorphisms $\pi$.

While 494H and 494G are of some interest in themselves, their function here
is to prepare the way to 494L, 494O and 494Q.   The first
belongs to the series in \S493;  like the results in that section, it
depends on concentration-of-measure theorems, quoted in part (e) of the
proof of 494I and again in part (e) of the proof of 494J.   In addition,
for the generalization from ergodic full groups to arbitrary full groups,
we need the structure theory for closed subalgebras developed in
\S333.

494O and 494Q-494R break new ground.   The former, following
{\smc Kittrell \& Tsankov 09}, examines a curious
phenomenon identified by {\smc Rosendal \& Solecki 07} in the course of
a search
for automatic-continuity results.   We cannot dispense entirely with the
hypotheses that $\frak A$ should be atomless and $G$ ergodic (494Yj),
though perhaps they can be relaxed.
%what if \frak A is prob alg relatively atomless over \frak C?
Many examples are now known of
$k$-Steinhaus groups (494O, 494Yi), but as far as I am aware there are no
non-trivial cases in which the critical value of $k$ has been determined
(494Z).   The automatic-continuity corollary in 494Ob is really a result
about homomorphisms into Polish groups (see 494Yh), but applies in many
other cases (494P).

The phenomenon of 494Q, which we might call a (negative) `algebraic
cofinality' result, has attracted attention with regard to many
algebraic structures, starting with {\smc Bergman 06}.   Apart from the
variations of 494Q in 494R and 494Xm, there is a simple example
in 494Yk.   494Yj again indicates one of the limits of the result.
}%end of notes

\leaveitout{
{\smc Kechris p09}, I.4.1:  if $(\frak A,\bar\mu)$ is the Lebesgue
probability algebra and $G$, $H\subseteq\AmuA$ are ergodic full subgroups
which are isomorphic as groups, then $G$ and $H$ are conjugate in $\AmuA$
(?).
}

\discrpage

\leaveitout{
\leader{494yL}{Theorem}\cmmnt{ ({\smc Kittrell \& Tsankov 09})}
Let $(\frak A,\bar\mu)$ be a probability algebra and
$G\subseteq\AmuA$ a full subgroup with fixed-point subalgebra $\frak C$.
Suppose that $\frak A$ is relatively atomless over $\frak C$.
Give $G$ the topology induced by the uniform topology of $\AmuA$.

(a) If $V\subseteq G$ is symmetric
and $G$ can be covered by countably many left translates of $V$ in $G$,
then $V^{38}=\{\pi_1\pi_2\ldots\pi_{38}:\pi_1,\ldots,\pi_{38}\in V\}$
is a neighbourhood of the identity in $G$.

(b) If $H$ is a topological group such that for every neighbourhood
$W$ of the identity in $H$ there is a countable set $D\subseteq H$ such
that $H=DW$,
and $\theta:G\to H$ is a group homomorphism, then $\theta$ is continuous.

\proof{{\bf (a)(i)} Let $\sequencen{\psi_n}$ be a sequence in $G$ such that
$G=\bigcup_{n\in\Bbb N}\psi_nV$.   It may help if I note straight away that
$\iota\in V^2$.   \Prf\ There is an $n\in\Bbb N$ such that
$\iota\in\psi_nV$, that is, $\psi_n^{-1}\in V$;  as $V$ is symmetric,
$\psi_n\in V$ and $\iota=\psi_n\psi_n^{-1}$ belongs to $V^2$.\ \Qed

\medskip

\quad{\bf (ii)} For $a\in\frak A$, set
$G_a=\{\pi:\pi\in G$ is supported by $a\}$, and let $u_a$ be the
conditional expectation of $\chi a$ on $\frak C$.
There is a non-zero $e\in\frak A$ such that $u_e$ is a multiple of $\chi 1$
and for every
$\pi\in G_e$ there is a $\phi\in V^2$ agreeing with $\pi$ on
$\frak A_e$.   \Prf\ Because $\frak A$ is relatively atomless over
$\frak C$, we can choose inductively a disjoint
sequence $\sequencen{b_n}$ in $\frak A\setminus\{0\}$ such that
$u_{b_n}=2^{-n-1}\chi 1$ for every $n$ (use 494Ad, as usual);  then
$\sum_{n=0}^{\infty}\bar\mu b_n=1$, so $\sup_{n\in\Bbb N}b_n=1$ in
$\frak A$.
\Quer\ Suppose, if possible, that for every $n\in\Bbb N$ there is a
$\pi_n\in G_{b_n}$ such that there is no
$\phi\in V^2$ agreeing with $\pi_n$ on $\frak A_{b_n}$.
If $n\in\Bbb N$, then
$V^2=(\psi_nV)^{-1}\psi_nV$ and $\pi_n=\iota^{-1}\pi_n$,
so there must be a $\pi'_n\in G_{b_n}$, either $\iota$ or $\pi_n$,
not agreeing with
$\phi$ on $\frak A_{b_n}$ for any $\phi\in\psi_nV$.   Define
$\psi\in\AmuA$ by the formula

\Centerline{$\psi d
=\pi'_nd\text{ if }n\in\Bbb N\text{ and }d\Bsubseteq b_n$.}

\noindent Because $G$ is full, $\psi\in G$ and
there is an $m\in\Bbb N$ such that
$\psi\in\psi_mV$.   But now $\pi'_m$ agrees with $\psi$ on
$\frak A_{b_m}$, contrary to the choice of $\pi'_m$.\ \BanG\  So one of the
$b_n$ will serve for $e$.\ \Qed

\medskip

\quad{\bf (iii)} There is an involution $\pi\in V^2$, supported by $e$,
such that $u_{\supp\pi}\le\bover23u_e$.   \Prf\ Take disjoint
$b$, $b'\Bsubseteq e$ such that $u_b=u_{b'}=\bover13u_e$.
Because $G$ is full, there is a $\phi\in G$ such that
$\phi b=b'$ (494Gb).   For $d\Bsubseteq b$, set
$\phi_d=\cycle{d\,_{\phi}\,\phi d}$, so that $\phi_d$ is either $\iota$ or
an involution belonging to $G$.   Since $\frak A_b$ is surely uncountable
(for instance, there is for every $\alpha\in[0,1]$ a $d\in\frak A_b$ such
that $u_d=\alpha u_b$), there must be an $n\in\Bbb N$ and distinct $d$,
$d'\in\frak A_b$ such that $\phi_d$ and $\phi_{d'}$ both belong to
$\psi_nV$.   In this case

\Centerline{$\pi=\phi_{d\Bsymmdiff d'}=\phi_d^{-1}\phi_{d'}$}

\noindent is an involution belonging to $(\psi_nV)^{-1}\psi_nV=V^2$,
while $\supp\pi\Bsubseteq u_b\Bcup u_{b'}$ so
$u_{\supp\pi}\le\bover23u_e$.\ \Qed

\medskip

\quad{\bf (iv)} Taking $a=\supp\pi$ in (iii), $a$ and $e$ satisfy the
conditions of 494N with respect to $V^2$, so
$G_c\Bsubseteq(V^2)^{18}=V^{36}$.

\medskip

\quad{\bf (v)}  Finally, there is a $\delta>0$ such that, in the language
of 494Cb,
$G\cap U(1,\delta)\Bsubseteq V^{38}$.   \Prf\Quer\ Otherwise, we can find
for each $n\in\Bbb N$ a
$\pi_n\in G\cap U(1,2^{-n-1}\bar\mu a)\setminus V^{38}$.   Set
$\pi'_n=\psi_n\pi_n\psi_n^{-1}$, $b_n=\supp\pi'_n$;  then
$\bar\mu b_n=\bar\mu(\supp\pi_n)$ for each $n$, by 381Gd again.   So
$b'=\sup_{n\in\Bbb N}b_n$ has measure at most $\bar\mu c$, and there is a
$\phi\in G$ such that $\phi b'\Bsubseteq c$.   In this case,
there is an $n\in\Bbb N$ such that $\phi^{-1}\in\psi_nV$, that is,
$\phi\psi_n\in V^{-1}=V$.   Now
$\pi=\phi\psi_n\pi_n\psi_n^{-1}\phi^{-1}$ has support
$\phi b_n\Bsubseteq c$, so belongs to $V^{36}$.   But this means that
$\pi_n=\psi_n^{-1}\phi^{-1}\pi\phi\psi_n$ belongs to $V^{38}$, contrary to
the choice of $\pi_n$.\ \Bang\Qed

So $V^{38}$ is a neighbourhood of $\iota$ in $G$, as claimed.

\medskip

{\bf (b)} Let $W$ be a neighbourhood of $e$ in $H$.   Then there
is a symmetric neighbourhood $W_1$ of the identity in $H$ such that
$W_1^{38}\subseteq W$.   Set $V=\theta^{-1}[W_1]$.   Let $W_2$ be a
neighbourhood of the identity in $H$ such that
$W_2^{-1}W_2\subseteq W_1$, and
$\sequencen{y_n}$ a sequence in $H$ such that
$H=\bigcup_{n\in\Bbb N}y_nW_2$.
For each $n\in\Bbb N$, choose $\psi_n\in G$ such that
$\theta(\psi_n)\in y_nW_2$ whenever $\theta[G]$ meets $y_nW_2$.
If $\pi\in G$, there is an $n\in\Bbb N$ such that
$\theta(\pi)\in y_nW_2$;   in this case, $\theta(\psi_n)\in y_nW_2$, so

\Centerline{$\theta(\psi_n^{-1}\pi)\in W_2^{-1}y_n^{-1}y_nW_2
\subseteq W_1$}

\noindent and
$\psi_n^{-1}\pi\in V$.   Thus $\pi\in\psi_nV$;  as $\pi$ is arbitrary,
$G=\bigcup_{n\in\Bbb N}\psi_nV$.   By (b), $V^{38}$ is a neighbourhood
of $\iota$;  but
$V^{38}\subseteq\theta^{-1}[W_1^{38}]\subseteq\theta^{-1}[W]$, so
$\theta^{-1}[W]$ is a neighbourhood of $\iota$.   As $W$ is
arbitrary, $\theta$ is continuous (4A5Fa again).
}%end of proof of 494O
}

