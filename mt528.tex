\frfilename{mt528.tex}
\versiondate{10.2.11/11.2.11}
\copyrightdate{2007}

\def\chaptername{Cardinal functions of measure theory}
\def\sectionname{Amoeba algebras}

\def\AM{\mathop{\text{AM}}}
%\def\checked{$\surd$}

\newsection{528}

In the course of investigating the principal consequences of Martin's
axiom, {\smc Martin \& Solovay 70} introduced the partially ordered set
of open subsets of $\Bbb R$ with measure strictly less than $\gamma$,
for $\gamma>0$ (528O).   Elementary extensions of this idea lead us to a
very interesting class of partially ordered sets, which I study here in
terms of their regular open algebras, the `amoeba algebras' (528A).   Of
course the most important ones are those associated with Lebesgue
measure, and these are closely related to `localization posets'
(528I), themselves intimately connected with the localization relations of
522K.   In the second half of the section I look at the cardinal functions
of these algebras, of which the most interesting seems to be Maharam type
(528V).

As elsewhere in this chapter, I will write
$(\frak B_{\kappa},\bar\nu_{\kappa})$ for the measure algebra of the
usual measure on $\{0,1\}^{\kappa}$.   In any measure algebra
$(\frak A,\bar\mu)$
I will write $\frak A^f=\{a:a\in\frak A$, $\bar\mu a<\infty\}$.

\leader{528A}{Amoeba algebras}
Let $(\frak A,\bar\mu)$ be a measure algebra.

\spheader 528Aa If $0<\gamma\le\bar\mu 1$, the {\bf amoeba algebra}
$\AM(\frak A,\bar\mu,\gamma)$ is the regular open algebra
$\RO^{\uparrow}(P)$ where $P=\{a:a\in\frak A$, $\bar\mu a<\gamma\}$,
ordered by $\Bsubseteq$.

\spheader 528Ab
The {\bf variable-measure amoeba algebra}
$\AM^*(\frak A,\bar\mu)$\cmmnt{ ({\smc Truss 88})} is the regular open
algebra $\RO^{\uparrow}(P')$ where

\Centerline{$P'=\{(a,\alpha):a\in\frak A$,
$\alpha\in\ocint{\bar\mu a,\bar\mu 1}\}$,}

\noindent ordered by saying that

\Centerline{$(a,\alpha)\le(b,\beta)$ if $a\Bsubseteq b$ and
$\beta\le\alpha$.}

\leader{528B}{}\dvAnew{2011}\cmmnt{ It may help to have the following
simple facts set out straight away.

\wheader{528B}{4}{2}{2}{72pt}

\noindent}{\bf Lemma} Let $(\frak A,\bar\mu)$ be a measure algebra
and $0<\gamma\le\bar\mu 1$.   Set
$P=\{a:a\in\frak A$, $\bar\mu a<\gamma\}$.

(a) Two elements $a$, $b\in P$ are compatible upwards in $P$ iff
$\bar\mu(a\Bcup b)<\gamma$.

(b) Suppose that $(\frak A,\bar\mu)$ is semi-finite and atomless.

\quad(i) $P$ is separative upwards, so
$\coint{a,\infty}\in\RO^{\uparrow}(P)$ for every $a\in P$.

\quad(ii) If $A\subseteq P$ is non-empty, then the infimum
$\inf_{a\in A}\coint{a,\infty}$ is empty unless $\sup A$ is defined in
$\frak A$ and belongs to $P$, and in this case
$\inf_{a\in A}\coint{a,\infty}=\coint{\sup A,\infty}$.

\proof{{\bf (a)} $\coint{a,\infty}\cap\coint{b,\infty}
=\{c:a\Bcup b\Bsubseteq c\in P\}$ is non-empty iff $a\Bcup b\in P$.

\medskip

{\bf (b)(i)} Let $a$, $b\in P$ be
such that $a\notBsubseteq b$.   If $\bar\mu(a\Bcup b)\ge\gamma$ then $a$
and $b$ are already incompatible upwards.   Otherwise,
$\bar\mu(1\Bsetminus(a\Bcup b))\ge\gamma-\bar\mu(a\Bcup b)$.   Because
$(\frak A,\bar\mu)$ is atomless and semi-finite, there is a
$d\Bsubseteq 1\Bsetminus(a\Bcup b)$ such that
$\bar\mu d=\gamma-\bar\mu(a\Bcup b)$.   Set $c=b\Bcup d$.  Then

\Centerline{$\bar\mu c=\gamma-\bar\mu(a\Bsetminus b)<\gamma
=\bar\mu(a\Bcup c)$,}

\noindent so
$c\in\coint{b,\infty}\subseteq P$, while $a$ and $c$ are incompatible
upwards in $P$.   As $a$ and $b$ are arbitrary, $P$ is separative upwards.

By 514Me, it follows that $\coint{a,\infty}$ is a regular up-open set for
every $a\in P$.

\medskip

\quad{\bf (ii)} This is a re-phrasing of 514Mf.
}%end of proof of 528B

\leader{528C}{Proposition}\dvAformerly{5{}28B}
Suppose that $(X,\Sigma,\mu)$ is a measure
space, $(\frak A,\bar\mu)$ its measure algebra and $0<\gamma\le\mu X$.
If $\Cal E\subseteq\Sigma$ is any family such that $\mu$ is outer
regular with respect to $\Cal E$, then $\AM(\frak A,\bar\mu,\gamma)$ is
isomorphic to $\RO^{\uparrow}(\{E:E\in\Cal E$, $\mu E<\gamma\})$.

\proof{ Set $P=\{a:a\in\frak A$, $\bar\mu a<\gamma\}$,
$Q=\{E:E\in\Cal E$, $\mu E<\gamma\}$.   Because $\mu$ is outer regular
with respect to $\Cal E$, the map
$G\mapsto G^{\ssbullet}:Q\to\frak A$ maps $Q$ onto a cofinal subset $P'$
of $P$.   Moreover, two elements $E_0$ and $E_1$ of $Q$ are compatible
upwards in $Q$ iff $\mu(E_0\cup E_1)<\gamma$ iff $E_0^{\ssbullet}$
and $E_1^{\ssbullet}$ are compatible upwards in $P$.   By 514R,
$\RO^{\uparrow}(P)$ and $\RO^{\uparrow}(Q)$ are isomorphic.
}%end of proof of 528C

\leader{528D}{Proposition} (a)\cmmnt{ ({\smc Truss 88})} Let
$(\frak A,\bar\mu)$ be an atomless homogeneous probability algebra.
Then the amoeba algebras $\AM(\frak A,\bar\mu,\gamma)$ and
$\AM(\frak A,\bar\mu,\gamma')$ are isomorphic for all $\gamma$,
$\gamma'\in\ooint{0,1}$.

(b)\dvAformerly{5{}28C}
Let $(\frak A,\bar\mu)$ be a
non-totally-finite atomless quasi-homogeneous measure algebra\cmmnt{
(definition:  374G)}.   Then all the amoeba algebras
$\AM(\frak A,\bar\mu,\gamma)$, for $\gamma>0$, are isomorphic.

\proof{{\bf (a)(i)} Set $P=\{a:a\in\frak A$, $\bar\mu a<\gamma\}$, and let
$\kappa$ be the Maharam type of $\frak A$.   Then the
upwards cellularity of $P$ is at most $\kappa$.   \Prf\Quer\ Otherwise,
there is an up-antichain $A\subseteq P$ with cardinality $\kappa^+$.
Let $\epsilon>0$ be such that
$A'=\{a:a\in A$, $\bar\mu a\le\gamma-\epsilon\}$ has cardinal
$\kappa^+$.   Because the topological density of $\frak A$ is $\kappa$
(521Ea), there must be distinct $a$, $a'\in A'$ such that
$\bar\mu(a\Bsymmdiff a')<\epsilon$;  but in this case
$\bar\mu(a\Bcup a')<\gamma$, so that $a\Bcup a'$ is an upper bound for
$\{a,a'\}$ in $P$.\ \Bang\Qed

\medskip

\quad{\bf (ii)} If
$1-\sqrt{1-\gamma}\le\alpha<\gamma$ and $D$ is a countable subset of
$\ooint{\alpha,\gamma}$ such that $\sup D=\gamma$, then there is a
maximal up-antichain
$\family{(t,\xi)}{D\times\kappa}{a_{t\xi}}$ in $P$ such that
$\bar\mu a_{t\xi}=t$ for every $t\in D$, $\xi<\kappa$.   \Prf\ Start
with a stochastically independent family
$\family{(t,\xi)}{D\times\kappa}{c_{t\xi}}$ of elements of $\frak A$
with $\bar\mu c_{t\xi}=t$ for all $t\in D$, $\xi<\kappa$.   Because
$\alpha\ge 1-\sqrt{1-\gamma}$,
$A=\family{(t,\xi)}{D\times\kappa}{c_{t\xi}}$ is an up-antichain in
$P$.   Next, because $\sup D=\gamma$,
$Q=\{a:a\in P$, $\bar\mu a\in D\}$ is cofinal with $P$.   So there is a
maximal up-antichain $A'\supseteq A$ such that $A'\subseteq Q$ (513Aa).
Now (because $c^{\uparrow}(P)\le\kappa$) $\{a:a\in A'$, $\bar\mu a=t\}$
has cardinal $\kappa$ for every $t\in D$, so we can enumerate $A'$ as
$\family{(t,\xi)}{D\times\kappa}{a_{t\kappa}}$ in $P$ where
$\bar\mu a_{t\xi}=t$ for every $t\in D$ and $\xi<\kappa$.\ \Qed

\medskip

\wheader{528D}{0}{0}{0}{36pt}
\quad{\bf (iii)} There are $\alpha$, $\alpha'\in\ooint{0,1}$ such that

\Centerline{$1-\sqrt{1-\gamma}\le\alpha<\gamma$,
\quad$1-\sqrt{1-\gamma'}\le\alpha'<\gamma'$,
\quad$\Bover{\gamma-\alpha}{1-\alpha}
=\Bover{\gamma'-\alpha'}{1-\alpha'}$.}

\noindent\Prf\ We need consider only the case $\gamma\le\gamma'$.   Set

\Centerline{$\beta=\Bover1{\sqrt{1-\gamma}}-1$,
\quad$\alpha=\gamma-\beta(1-\gamma)$,
\quad$\alpha'=\gamma'-\beta(1-\gamma')$.}

\noindent Then
$\Bover{\gamma-\alpha}{1-\gamma}=\beta
=\Bover{\gamma'-\alpha'}{1-\gamma'}$.   Of course $\alpha\le\gamma$ and
$\alpha'\le\gamma'$.   On the other side,
$\alpha=1-\sqrt{1-\gamma}$, while $\beta\le\Bover1{\sqrt{1-\gamma'}}-1$
so $\alpha'\ge 1-\sqrt{1-\gamma'}$.\ \Qed

\medskip

\quad{\bf (iv)} If $a\in P$,
$\{b:a\Bsubseteq b\in P\}$ is isomorphic, as partially ordered set, to
$\{b:b\in\frak A$, $\bar\mu b<\Bover{\gamma-\bar\mu a}{1-\bar\mu a}\}$.
\Prf\ The principal ideal $\frak A_{1\Bsetminus a}$ generated by
$1\Bsetminus a$ is isomorphic, up to a scalar multiple of the measure,
to $\frak A$, and $\{b:a\Bsubseteq b\in P\}$ is isomorphic, as partially
ordered set, to $\{b:b\Bsubseteq 1\Bsetminus a$,
$\bar\mu b<\gamma-\bar\mu a\}$.\ \Qed

\medskip

\quad{\bf (v)} For each $n\in\Bbb N$, set
$\alpha_n=\gamma-2^{-n}(\gamma-\alpha)$,
$\alpha'_n=\gamma'-2^{-n}(\gamma'-\alpha')$;  then

\Centerline{$\Bover{1-\alpha'_n}{\gamma'-\alpha_n'}
=1+\Bover{1-\gamma'}{\gamma'-\alpha_n'}
=1+2^n\Bover{1-\gamma'}{\gamma'-\alpha'}
=\Bover{1-\alpha_n}{\gamma-\alpha_n}$,
\quad$\Bover{\gamma'-\alpha_n'}{1-\alpha_n'}
=\Bover{\gamma-\alpha_n}{1-\alpha_n}$}

\noindent for every $n\in\Bbb N$.   Set
$P'=\{a:a\in\frak A$, $\bar\mu a<\gamma'\}$.    By (b), we have a
maximal up-antichain $\family{(n,\xi)}{\Bbb N\times\xi}{a_{n\xi}}$ in
$P$ such that $\bar\mu a_{n\xi}=\alpha_n$ for all $n\in\Bbb N$ and
$\xi<\kappa$;  similarly, there is a maximal up-antichain
$\family{(n,\xi)}{\Bbb N\times\xi}{a'_{n\xi}}$ in $P'$ such that
$\bar\mu a'_{n\xi}=\alpha'_n$ for all $n\in\Bbb N$ and $\xi<\kappa$.
Now, for each $n\in\Bbb N$ and $\xi<\kappa$, $\coint{a_{n\xi},\infty}$,
taken in $P$, is isomorphic, as partially ordered set, to
$[a'_{n\xi},\infty[$, taken in $P'$, by (d).   So

$$\eqalignno{\RO^{\uparrow}(P)
&\cong\prod_{n\in\Bbb N,\xi<\kappa}
  \RO^{\uparrow}(\coint{a_{n\xi},\infty})\cr
\displaycause{514Nf}
&\cong\prod_{n\in\Bbb N,\xi<\kappa}
  \RO^{\uparrow}(\coint{a'_{n\xi},\infty})
\cong\RO^{\uparrow}(P').\cr}$$

\medskip

{\bf (b)} Suppose that $\beta$, $\gamma>0$.   As in Lemma 332I, we have a
partition $D$ of unity in $\frak A$ such that $\bar\mu a=\beta$ for
every $a\in D$.   Similarly, we have a partition $D'$ of unity such that
$\bar\mu a=\gamma$ for every $a\in D'$.   By 332E,
$\#(D)=\#(D')=c(\frak A)$.   Let $h:D\to D'$ be a bijection.   If $d\in D$,
the principal ideals $\frak A_d$, $\frak A_{h(d)}$ have the same
Maharam type, because $(\frak A,\bar\mu)$ is quasi-homogeneous (374H),
and are therefore isomorphic as measure algebras, up to a
scalar factor of the measure;  let $\pi_d:\frak A_d\to\frak A_{h(d)}$ be
a Boolean isomorphism such that
$\bar\mu(\pi_da)=\Bover{\gamma}{\beta}\bar\mu a$ for every
$a\Bsubseteq d$.   Now we have a function $\pi:\frak A^f\to\frak A^f$
defined by saying that $\pi a=\sup_{d\in D}\pi_d(a\Bcap d)$ whenever
$\bar\mu a<\infty$, and $\pi$ is a Boolean ring automorphism such that
$\bar\mu\pi a=\Bover{\gamma}{\beta}\bar\mu a$ for every $a\in\frak A^f$.
But now $\pi$ includes an isomorphism between the partially ordered sets
$\{a:\bar\mu a<\beta\}$ and $\{a:\bar\mu a<\gamma\}$, so their regular
open algebras $\AM(\frak A,\bar\mu,\beta)$ and
$\AM(\frak A,\bar\mu,\gamma)$ are isomorphic.
}%end of proof of 528D

\leader{528E}{Lemma}\dvAnew{2011}
Let $(\frak A,\bar\mu)$ be an atomless semi-finite
measure algebra.   Then there is a family
$\family{\alpha}{[0,\bar\mu 1]}{c_{\alpha}}$ in $\frak A$ such that
$c_{\alpha}\Bsubseteq c_{\beta}$ and $\bar\mu c_{\alpha}=\alpha$
whenever $0\le\alpha\le\beta\le\bar\mu 1$,
and $\alpha\mapsto c_{\alpha}$ is continuous for
the measure-algebra topology of $\frak A$.

\proof{ Because $(\frak A,\bar\mu)$ is semi-finite, there is a
non-decreasing sequence $\sequencen{e_n}$ in $\frak A^f$ such that
$\sup_{n\in\Bbb N}\bar\mu e_n=\bar\mu 1$, starting from
$e_0=0$;  set
$e=\sup_{n\in\Bbb N}e_n$, so that $\bar\mu e=\bar\mu 1$.   Then
$(\frak A_e,\bar\mu\restrp\frak A_e)$ is $\sigma$-finite and atomless.
Let $(\frak C,\bar\lambda)$ be the
measure algebra of Lebesgue measure on $\coint{0,\bar\mu 1}$.
For each $n\in\Bbb N$ set
$e'_n=e_{n+1}\Bsetminus e_n$ and
$d_n=\coint{\bar\mu e_n,\bar\mu e_{n+1}}^{\ssbullet}\in\frak C$.

Because $\frak A$ is atomless, 332P tells us that
there is for each $n\in\Bbb N$ a measure-preserving Boolean
homomorphism $\pi_n$
from the principal ideal $\frak C_{d_n}$ to a principal ideal
of $\frak A_{e'_n}$, which must be $\frak A_{e'_n}$ itself because
$\bar\mu e'_n=\bar\lambda d_n$;  by 324Kb, $\pi_n$ is order-continuous.
Assembling these, we have an order-continuous
measure-preserving Boolean homomorphism $\pi:\frak C\to\frak A_e$ defined
by setting $\pi d=\sup_{n\in\Bbb N}\pi_n(d\Bcap d_n)$ for every
$d\in\frak C$.   Now set $c_{\alpha}=\pi\coint{0,\alpha}^{\ssbullet}$ for
$\alpha\le\bar\mu 1$.   Because $\pi$ is continuous for the measure-algebra
topologies of $\frak C$ and $\frak A_e$ (324Fc), or otherwise,
$\alpha\mapsto c_{\alpha}$ is continuous.
}%end of proof of 528E

\leader{528F}{Proposition} Let $(\frak A,\bar\mu)$ be a semi-finite
measure algebra, and $\gamma\in\ooint{0,\infty}$.

(a) Suppose that $e\in\frak A$ and $\bar\mu e\ge\gamma$.
If $\frak A_e$ is atomless, then
$\AM(\frak A_e,\bar\mu\restrp\frak A_e,\gamma)$ can be regularly
embedded in $\AM(\frak A,\bar\mu,\gamma)$.

(b)\dvAnew{2011}
Suppose that $\frak A$ is atomless, and that $\gamma<\bar\mu 1$.
Let $\sequence{k}{e_k}$ be a non-decreasing sequence in $\frak A$
with supremum $1$, and
suppose that $\bar\mu e_k\ge\gamma$ for every $k\in\Bbb N$.   Then
we have a sequence $\sequence{k}{\pi_k}$ such that
$\pi_k:\AM(\frak A_{e_k},\bar\mu\restrp\frak A_{e_k},\gamma)\to
\AM(\frak A,\bar\mu,\gamma)$ is a regular embedding for every
$k\in\Bbb N$, and
$\bigcup_{k\in\Bbb N}
\pi_k[\AM(\frak A_{e_k},\bar\mu\restrp\frak A_{e_k},\gamma)]
\,\,\tau$-generates $\AM(\frak A,\bar\mu,\gamma)$.

(c)\dvAformerly{5{}28E} Now suppose that $(\frak A,\bar\mu)$ is atomless
and quasi-homogeneous, and that $\gamma<\bar\mu 1$.   Then
$\AM(\frak A,\bar\mu,\gamma)$ can be regularly
embedded in $\AM^*(\frak A,\bar\mu)$.

\proof{{\bf (a)} Set $P=\{a:a\in\frak A$, $\bar\mu a<\gamma\}$ and
$Q=P\cap\frak A_e$.   By 528E, we have a continuous
order-preserving function
$\alpha\mapsto c_{\alpha}:[0,\bar\mu e]\to\frak A_e$ such that
$\bar\mu c_{\alpha}=\alpha$ for each $\alpha$.   If $a\in\frak A$,
then the function $\beta\mapsto\bar\mu(c_{\beta}\Bsetminus a)$
is a continuous non-decreasing function from $[0,\bar\mu e]$ onto
$[0,\bar\mu(c_{\bar\mu e}\Bsetminus a)]$, and
we can set
$\delta(a,\alpha)=\min\{\beta:\bar\mu(c_{\beta}\Bsetminus a)=\alpha\}$
whenever $0\le\alpha\le\bar\mu(c_{\bar\mu e}\Bsetminus a)$.
In this case,

\Centerline{$\bar\mu((a\Bcap e)\Bcup c_{\delta(a,\alpha)})
=\bar\mu(a\Bcap e)+\bar\mu(c_{\delta(a,\alpha)}\Bsetminus a)
=\alpha+\bar\mu(a\Bcap e)$.}

\noindent Note that $\delta(a,\alpha)\le\delta(a',\alpha')$ whenever
$a\Bsubseteq a'$ and
$\alpha\le\alpha'\le\bar\mu(c_{\bar\mu 1}\Bsetminus a')$.

If $a\in P$, then

\Centerline{$\bar\mu(c_{\bar\mu e}\Bsetminus a)
=\bar\mu c_{\bar\mu e}-\bar\mu(a\Bcap c_{\bar\mu e})
\ge\bar\mu e-\bar\mu(a\Bcap e)
\ge\bar\mu a-\bar\mu(a\Bcap e)
=\bar\mu(a\Bsetminus e)$.}

\noindent So $\delta(a,\bar\mu(a\Bsetminus e))$ is defined, and we have a
function $f$ given by the formula

\Centerline{$f(a)=(a\Bcap e)\Bcup c_{\delta(a,\bar\mu(a\Bsetminus e))}$}

\noindent for $a\in P$.
In this case $\bar\mu f(a)=\bar\mu a$, so $f(a)\in Q$, for each $a$,
and $f$, like $\delta$, is order-preserving.   Of course $f(a)=a$ for
$a\in Q$.

If $a\in P$, $b\in Q$ and $f(a)\Bsubseteq b$, there is an $a'\in P$ such
that $a\Bsubseteq a'$ and $b=f(a')$.   \Prf\
Set $a'=a\Bcup(b\Bsetminus f(a))$.   Then

\Centerline{$\bar\mu a'
=\bar\mu a+\bar\mu(b\Bsetminus f(a))
=\bar\mu f(a)+\bar\mu(b\Bsetminus f(a))
=\bar\mu b<\gamma$,}

\noindent so $a'\in P$.   Also
$b\Bsubseteq f(a)\Bcup(a'\Bcap e)\Bsubseteq f(a')$;  as
$\bar\mu b=\bar\mu a'=\bar\mu f(a')$, $b=f(a')$.\ \QeD\   So if
$Q_0\subseteq Q$ is cofinal with $Q$, $f^{-1}[Q_0]$ will be cofinal with
$P$ (as in the proof of 514P), and we have an order-continuous Boolean
homomorphism $\pi:\RO^{\uparrow}(Q)\to\RO^{\uparrow}(P)$
defined by setting $\pi H=\interior\overline{f^{-1}[H]}$ for every
$H\in\RO^{\uparrow}(Q)$.   Finally, $f[P]=f[Q]=Q$.   So $\pi$ is injective
and is a regular embedding of
$\AM(\frak A_e,\bar\mu\restrp\frak A_e,\gamma)=\RO^{\uparrow}(Q)$ into
$\AM(\frak A,\bar\mu,\gamma)=\RO^{\uparrow}(P)$.

\medskip

{\bf (b)(i)}
For each $k\in\Bbb N$, set $Q_k=P\cap\frak A_{e_k}$ and
choose functions $f_k:P\to Q_k$
and $\pi_k:\RO^{\uparrow}(Q_k)\to\RO^{\uparrow}(P)$ as in (a)
above.   If we write
$\coint{c,\infty}=\{a:c\Bsubseteq a\in P\}$ for every $c\in P$, then
$\frak A_{e_k}\cap\coint{c,\infty}=\{b:c\Bsubseteq b\in Q_k\}$ for
$k\in\Bbb N$ and $c\in Q_k$;  in this case,
$\frak A_{e_k}\cap\coint{c,\infty}\in\RO^{\uparrow}(Q_k)$,
by 528B(b-i).

\medskip

\quad{\bf (ii)} Let $\frak G$ be the order-closed subalgebra of
$\RO^{\uparrow}(P)$ generated by
$\bigcup_{k\in\Bbb N}\pi_k[\RO^{\uparrow}(Q_k)]$.   If $a\in P$, there is a
non-empty $G\in\frak G$ included in
$\coint{a,\infty}\in\RO^{\uparrow}(P)$.   \Prf\ Because
$a\Bsubseteq\sup_{k\in\Bbb N}e_k$ and $\sequence{k}{e_k}$ is
non-decreasing, there is an infinite $I\subseteq\Bbb N$ such that
$\sum_{k\in I}\bar\mu(a\Bsetminus e_k)<\gamma-\bar\mu a$.
Set $b=\sup_{k\in I}f_k(a)$.   Then

\Centerline{$\bar\mu b
\le\bar\mu a+\sum_{k\in I}\bar\mu(a\Bsetminus f_k(a))
\le\bar\mu a+\sum_{k\in I}\bar\mu(a\Bsetminus e_k)
<\gamma$}

\noindent because $f_k(a)\Bsupseteq a\Bcap e_k$ for every $k$, by the
construction in (a).   Thus $b\in P$.   Also

\Centerline{$\bar\mu(a\Bsetminus b)
\le\inf_{k\in\Bbb N}\bar\mu(a\Bsetminus f_k(a))=0$,}

\noindent so $a\Bsubseteq b$.

Set

\Centerline{$V_k
=\frak A_{e_k}\cap\coint{f_k(a),\infty}\in\RO^{\uparrow}(Q_k)$}

\noindent for every $k$.   Then
$\pi_kV_k=\interior\overline{f_k^{-1}[V_k]}$ belongs to
$\frak G$ for each $k$, and
$G=\inf_{k\in\Bbb N}\pi_kV_k=\bigcap_{k\in\Bbb N}\pi_kV_k$
(514M(d-ii)) belongs to $\frak G$.
Because every $f_k$ is order-preserving,
$f_k(b')\Bsupseteq f_k(a)$ and $f_k(b')\in V_k$ for every
$b'\Bsupseteq b$;  thus
$b\in\interior f_k^{-1}[V_k]$ for every $k$, and $b\in G$.   This shows
that $G\ne\emptyset$.

\Quer\ Suppose, if possible, that $G\not\subseteq\coint{a,\infty}$.
Then there is a $c\in G$ such that $a\Bsetminus c\ne 0$.   If
$\bar\mu(c\Bcup a)>\gamma$, set $c'=c$.   Otherwise, let
$\delta>0$ be such that

\Centerline{$\delta+\bar\mu c<\gamma<\delta+\bar\mu(c\Bcup a)
<\bar\mu 1$.}

\noindent Because $(\frak A,\bar\mu)$ is atomless and semi-finite,
there is a $d\Bsubseteq 1\Bsetminus(c\Bcup a)$ such that
$\bar\mu d=\delta$.   Set $c'=c\Bcup d$;  then $c\Bsubseteq c'\in P$ so
$c'\in G$, while $\bar\mu(c'\Bcup a)>\gamma$, as in the previous case.

Because $I$ is infinite, $\sup_{k\in I}e_k=1$ and
there is a $k\in I$
such that $\bar\mu((c'\Bcup a)\Bcap e_k)\ge\gamma$.   In this case,
$c'\in\pi_kV_k\subseteq\overline{f_k^{-1}[V_k]}$, so
$\coint{c',\infty}$ meets $f_k^{-1}[V_k]$ and there is a
$c''\Bsupseteq c'$ such that $c''\in P$ and $f_k(c'')\in V_k$, that is,
$f_k(c'')\Bsupseteq f_k(a)$.   Now, however,

\Centerline{$f_k(c'')\Bsupseteq(c'\Bcap e_k)\Bcup(f_k(a)\Bcap e_k)
\Bsupseteq(c'\Bcup a)\Bcap e_k$}

\noindent has measure at least $\gamma$, and cannot belong to
$Q_k$.\ \Bang\Qed


\medskip

\quad{\bf (iii)} Since $\{\coint{a,\infty}:a\in P\}$ is a base for the
topology of $P$, it is a $\pi$-base for $\RO^{\uparrow}(P)$, and
$\frak G$ includes a $\pi$-base  for $\RO^{\uparrow}(P)$.   But this means
that every member of $\RO^{\uparrow}(P)$ is the supremum of the members of
$\frak G$ it includes, and belongs to $\frak G$.   Thus
$\frak G=\RO^{\uparrow}(P)$, as claimed.

\medskip

{\bf (c)(i)} This time, let
$\family{\alpha}{[0,\bar\mu 1]}{c_{\alpha}}$ be a family in $\frak A$ such
that $c_{\alpha}\Bsubseteq c_{\beta}$ and $\bar\mu c_{\alpha}=\alpha$
whenever $0\le\alpha\le\beta\le\bar\mu 1$.
Set $P=\{(a,\alpha):a\in\frak A$, $\alpha\in\ocint{\bar\mu a,\bar\mu 1}\}$.
Let $\sequencen{\gamma_n}$ be a strictly increasing sequence with
supremum $\bar\mu 1$ and $\gamma_0=0$.   For each $n\in\Bbb N$, set
$P_n=\{(a,\alpha):\gamma_n\le\bar\mu a<\alpha\le\gamma_{n+1}\}$ and
$Q_n=\{a:a\in\frak A$, $\bar\mu a<\gamma_{n+1}\}$, so that $P_n$ is an
up-open set in $P$.
Note that $\bigcup_{n\in\Bbb N}P_n$ is dense in $P$ for the up-topology,
since if $(a,\alpha)\in P$ then $(a,\min(\alpha,\gamma_{n+1}))\in P_n$
where $\gamma_n\le\bar\mu a<\gamma_{n+1}$.   Also

\Centerline{$\RO^{\uparrow}(Q_n)
=\AM(\frak A,\bar\mu,\gamma_{n+1})\cong\AM(\frak A,\bar\mu,\gamma)$.}

\noindent\Prf\ If $\bar\mu 1=\infty$, this is 528Db.   If
$(\frak A,\bar\mu)$ is totally finite, then $\frak A$ is homogeneous, so
we can apply 528Da to an appropriate multiple of the measure $\bar\mu$.\
\Qed

For $a\in\frak A^f$, the function
$\alpha\mapsto\bar\mu(c_{\alpha}\Bsetminus a):[0,\bar\mu 1]\to[0,\infty]$
is continuous and non-decreasing, and

\Centerline{$\bar\mu(c_{\bar\mu 1}\Bsetminus a)
\ge\bar\mu c_{\bar\mu 1}-\bar\mu a
=\bar\mu 1-\bar\mu a
=\bar\mu(1\Bsetminus a)$.}

\noindent So we can define $\delta(a,\alpha)$, for
$a\in\frak A^f$ and $0\le\alpha\le\bar\mu(1\Bsetminus a)$, by saying that

\Centerline{$\delta(a,\alpha)
=\min\{\beta:\bar\mu(c_{\beta}\Bsetminus a)=\alpha\}
=\min\{\beta:\bar\mu(a\Bcup c_{\beta})=\bar\mu a+\alpha\}$.}

\noindent As in (a), $\delta(a,\alpha)\le\delta(a',\alpha')$ whenever
$a\Bsubseteq a'$ and $\alpha\le\alpha'$.    For $(a,\alpha)\in P_n$, set

\Centerline{$f_n(a,\alpha)=a\Bcup c_{\delta(a,\gamma_{n+1}-\alpha)}$,}

\noindent so that
$\bar\mu f_n(a,\alpha)=\bar\mu a+\gamma_{n+1}-\alpha<\gamma_{n+1}$ and
$f_n(a,\alpha)\in Q_n$.   Of course $f_n(a,\gamma_{n+1})=a$ if
$(a,\gamma_{n+1})\in P_n$, that is, if $a\in Q_n$ and
$\bar\mu a\ge\gamma_n$.

\medskip

\quad{\bf (ii)}\grheada\ $f_n:P_n\to Q_n$ is order-preserving.   \Prf\ If
$(a,\alpha)\le(a',\alpha')$ in $P_n$, then
$\delta(a,\gamma_{n+1}-\alpha)\le\delta(a',\gamma_{n+1}-\alpha')$, so
$f_n(a,\alpha)\Bsubseteq f_n(a',\alpha')$.\ \Qed

\medskip

\qquad\grheadb\ If $p\in P_n$, $b\in Q_n$ and $f_n(p)\Bsubseteq b$,
there is a
$p'\in P_n$ such that $p\le p'$ and $b\Bsubseteq f_n(p')$.   \Prf\
Express $p$ as $(a,\alpha)$.   Consider
$a'=a\Bcup(b\Bsetminus f_n(p))$.   Then

\Centerline{$\bar\mu a'=\bar\mu a+\bar\mu b-\bar\mu f_n(p)
=\bar\mu a+\bar\mu b-\bar\mu a-\gamma_{n+1}+\alpha<\alpha$,}

\noindent so $(a',\alpha)\in P$.   Of course
$(a,\alpha)\le(a',\alpha)$, so $p'=(a',\alpha)\in P_n$.   Also
$f_n(p')\Bsupseteq f_n(p)$ and

\Centerline{$f_n(p')\Bsupseteq a'\Bsupseteq b\Bsetminus f_n(p)$,}

\noindent so $b\Bsubseteq f_n(p')$.\ \Qed

\medskip

\qquad\grheadc\ $f_n[P_n]$ is cofinal with $Q_n$.   \Prf\ If
$b\in Q_n$, take $b'\in\frak A$ such that $b\Bsubseteq b'$ and
$\gamma_n\le\bar\mu b'<\gamma_{n+1}$.   Then $(b',\gamma_{n+1})\in P_n$
and

\Centerline{$b\Bsubseteq b'=f_n(b',\gamma_{n+1})\in f_n[P_n]$.  \Qed}

\medskip

\quad{\bf (iii)} By 514P, $\RO^{\uparrow}(Q_n)$ can be regularly embedded
in
$\RO^{\uparrow}(P_n)$.   Now $\AM(\frak A,\bar\mu,\gamma)$ is isomorphic
to $\RO^{\uparrow}(Q_n)$, so there is an injective
order-continuous Boolean homomorphism
$\pi_n:\AM(\frak A,\bar\mu,\gamma)\to\RO^{\uparrow}(P_n)$.
Putting these together, we have an injective order-continuous Boolean
homomorphism $\pi:\AM(\frak A,\bar\mu,\gamma)
\to\prod_{n\in\Bbb N}\RO^{\uparrow}(P_n)$ defined by setting
$\pi u=\sequencen{\pi_n(u)}$ for $u\in\AM(\frak A,\bar\mu,\gamma)$.
On the other hand, since $\sequencen{P_n}$ is a disjoint sequence of
up-open subsets of $P$ with dense union,

\Centerline{$\prod_{n\in\Bbb N}\RO^{\uparrow}(P_n)
\cong\RO^{\uparrow}(P)=\AM^*(\frak A,\bar\mu)$}

\noindent by 315H.   So we have a regular embedding of
$\AM(\frak A,\bar\mu,\gamma)$ into $\AM^*(\frak A,\bar\mu)$, as claimed.
}%end of proof of 528F

\leader{528G}{Proposition} Let $(\frak A,\bar\mu)$ be a
measure algebra, and $\frak C$ a $\sigma$-subalgebra of $\frak A$
such that $\sup(\frak C\cap\frak A^f)=1$ in $\frak A$.   Then
$\AM^*(\frak C,\bar\mu\restrp\frak C)$ can be regularly embedded in
$\AM^*(\frak A,\bar\mu)$.

\proof{{\bf (a)} For each $a\in\frak A^f$ we have a `conditional
expectation' $u_a\in L^1(\frak C)$ defined by saying that
$\int_cu_a=\bar\mu(a\Bcap c)$ for every $c\in\frak C^f$.   (Apply 
365O\formerly{3{}65P} to
the identity map from $\frak C^f$ to $\frak A^f$.)   Note that as
the supremum of $\frak C^f$ in $\frak A$ is $1$,

\Centerline{$\int u_a=\sup_{c\in\frak C^f}\int_cu_a
=\sup_{c\in\frak C^f}\bar\mu(a\Bcap c)=\bar\mu a$.}

\noindent Also, of course,
$0\le\bar\mu(a\Bcap c)\le\bar\mu c$ for every $c\in\frak C^f$, so
$0\le u_a\le\chi 1$ in $L^{\infty}(\frak C)$.   Next, let
$u_a^*$ be the decreasing rearrangement of $u_a$, that is, the element of
$L^{\infty}(\frak A_L)$ (where $\frak A_L$ is the measure algebra of
Lebesgue measure on $\coint{0,\infty}$) such that
$\Bvalue{u^*>\alpha}=\coint{0,\bar\mu\Bvalue{u>\alpha}}^{\ssbullet}$
for every $\alpha\ge 0$ (373D).

\medskip

{\bf (b)} Set

\Centerline{$P=\{(a,\alpha):a\in\frak A$,
$\alpha\in\ocint{\bar\mu a,\bar\mu 1}\}$,
\quad$Q=\{(c,\alpha):c\in\frak C$,
$\alpha\in\ocint{\bar\mu c,\bar\mu 1}\}$.}

\noindent
Define a function $f$ on $P$ by saying that $f(a,\alpha)=(c,\beta)$ if

\Centerline{$c=\Bvalue{u_a=1}=\max\{d:d\in\frak C$, $d\Bsubseteq a\}$,}

\Centerline{$\beta
=\max\{\beta':\beta'\ge 0$,
  $\beta'+\int_{\beta'}^{\infty}u^*_a\le\alpha\}$.}

\noindent Note that $\beta>\bar\mu c$ because

\Centerline{$\bar\mu c+\int_{\bar\mu c}^{\infty}u^*_a
=\int u^*_a=\int u_a<\alpha$,}

\noindent using 373Fa for the equality in the middle, while
$\beta\le\alpha\le\bar\mu 1$;  so $(c,\beta)$ belongs to $Q$.

\medskip

{\bf (c)(i)} If $p\le p'$ in $P$,
then $f(p)\le f(p')$ in $Q$.   \Prf\ Express $p$, $p'$, $f(p)$ and $f(p')$
as $(a,\alpha)$, $(a',\alpha')$, $(c,\beta)$, $(c',\beta')$ respectively.
Then $c\Bsubseteq a\Bsubseteq a'$ so $c\Bsubseteq c'$.   Next,
$\chi a\le\chi a'$ so $u_a\le u_{a'}$ and $u^*_a\le u^*_{a'}$
(373Db);  accordingly

\Centerline{$\alpha'\le\alpha=\beta+\int_{\beta}^{\infty}u^*_a
\le\beta+\int_{\beta}^{\infty}u^*_{a'}$}

\noindent and $\beta'\le\beta$.\ \Qed

\medskip

\quad{\bf (ii)}
If $p\in P$, $q\in Q$ and $f(p)\le q$, then there is a $p'\ge p$ such that
$f(p')\ge q$.   \Prf\ Express $p$, $f(p)$ and $q$ as $(a,\alpha)$,
$(c,\beta)$ and $(d,\gamma)$ respectively.   Set $a'=a\Bcup d$.   Then

$$\eqalignno{\bar\mu a'
&=\bar\mu a+\bar\mu d-\bar\mu(a\Bcap d)
=\int u_a+\bar\mu d-\int_du_a
\ge\int u_a^*+\bar\mu d-\int_0^{\bar\mu d}u_a^*\cr
\displaycause{apply 373E with $v=\chi d$}
&=\bar\mu d+\int_{\bar\mu d}^{\infty}u_a^*
<\beta+\int_{\beta}^{\infty}u_a^*\cr
\displaycause{because $\Bvalue{u_a^*=1}=[0,\bar\mu c]^{\ssbullet}$
and $\bar\mu c\le\bar\mu d<\gamma\le\beta$}
&=\alpha.\cr}$$

\noindent So $(a',\alpha)\in P$.   Next, computing the integrals
$\int_bu_a\vee\chi d$ for $b$ belonging to $\frak C^f$
and either included in $d$ or
disjoint from it, we see that $u_{a'}=u_a\vee\chi d$ so that
$\Bvalue{u_{a'}=1}=\Bvalue{u_a=1}\Bcup d=d$.   Accordingly

\Centerline{$\bar\mu a'=\int u_{a'}=\int u^*_{a'}
=\bar\mu d+\int_{\bar\mu d}^{\infty}u^*_{a'}
<\gamma+\int_{\gamma}^{\infty}u^*_{a'}$}

\noindent (as noted above for $u_a$, we have
$\Bvalue{u_{a'}^*=1}=[0,\bar\mu d]^{\ssbullet}$), and if we set
$\alpha'=\min(\alpha,\gamma+\int_{\gamma}^{\infty}u^*_{a'})$ then
$p'=(a',\alpha')\ge p$ and $f(p')\ge q$, as required.\ \Qed

\medskip

\quad{\bf (iii)}
Since $P$ and $Q$ have a common least element $(0,\bar\mu 1)$ which is
invariant under $f$, $f$ satisfies the second condition of 514P and
$\AM^*(\frak C,\bar\mu\restrp\frak C)=\RO^{\uparrow}(Q)$ is
regularly embedded in $\AM^*(\frak A,\bar\mu)=\RO^{\uparrow}(P)$.
}%end of proof of 528G

\leader{528H}{Proposition} Let $(\frak A,\bar\mu)$ be a
semi-finite measure algebra, not $\{0\}$,
and let $\kappa\ge\max(\omega,\tau(\frak A),c(\frak A))$ be a cardinal.
Then $\AM^*(\frak A,\bar\mu)$ can be regularly embedded in
$\AM(\frak B_{\kappa},\bar\nu_{\kappa},\bover12)$.

\proof{{\bf (a)} To begin with (down to the end of (g) below), assume that
$\frak A$ is atomless.
Let $(\frak A^{\Bbb N},\bar\mu_{\infty})$ be the simple
product of a sequence of copies of $(\frak A,\bar\mu)$
(322L), so that
$\bar\mu_{\infty}\pmb{a}=\sum_{n=0}^{\infty}\mu a_n$ if
$\pmb{a}=\sequencen{a_n}\in\frak A^{\Bbb N}$.   Note that as $\frak A$
is certainly infinite, $\tau(\frak A^{\Bbb N})=\tau(\frak A)$ and
$c(\frak A^{\Bbb N})=c(\frak A)$ (514Ef).   By 526D, there is a function
$\theta:\frak A^{\Bbb N}\to\frak B_{\kappa}$ such that

\inset{$\theta(\sup A)=\sup\theta[A]$ for every non-empty
$A\subseteq\frak A^{\Bbb N}$ with a supremum in $\frak A$,

$\bar\nu_{\kappa}\theta(\pmb{a})=1-\exp(-\bar\mu_{\infty}\pmb{a})$
for every $\pmb{a}\in\frak A^{\Bbb N}$,

whenever $\familyiI{\pmb{a}^{(i)}}$ is a disjoint family in
$\frak A^{\Bbb N}$ and $\frak C_i$ is the closed subalgebra of
$\frak B_{\kappa}$
generated by $\{\theta(\pmb{a}):\pmb{a}\Bsubseteq\pmb{a}^{(i)}\}$ for
each $i$, then $\familyiI{\frak C_i}$ is stochastically independent.
}

\medskip


{\bf (b)} For $b\in\frak B_{\kappa}$, set
$g(b)=\sup\{\pmb{a}:\pmb{a}\in\frak A^{\Bbb N}$,
$\theta(\pmb{a})\Bsubseteq b\}$.

\medskip

\quad{\bf (i)} It is immediate from its definition that
$g:\frak B_{\kappa}\to\frak A^{\Bbb N}$ is order-preserving.

\medskip

\quad{\bf (ii)} Because $\theta$ is
supremum-preserving, $\theta(g(b))\Bsubseteq b$ for every $b\in\frak B_{\kappa}$.

\wheader{528H}{6}{2}{2}{36pt}

\quad{\bf (iii)} If $b\in\frak B_{\kappa}\setminus\{1\}$ then

\Centerline{$1-\bar\nu_{\kappa} b\le 1-\bar\nu_{\kappa}\theta(g(b))
=\exp(-\bar\mu_{\infty}g(b))$,}

\noindent so $\bar\mu_{\infty}g(b)\le-\ln(1-\bar\nu_{\kappa} b)$ is finite.

\medskip

\quad{\bf (iv)} $\pmb{a}\subseteq g(\theta(\pmb{a}))$ for every
$\pmb{a}\in\frak A^{\Bbb N}$;  and if
$\pmb{a}\in\frak A^{\Bbb N}$ has finite measure then
$g(\theta(\pmb{a}))=\pmb{a}$, because if $\pmb{a}'\notBsubseteq\pmb{a}$
then
$\bar\nu_{\kappa}\theta(\pmb{a}\Bcup\pmb{a}')
  >\bar\nu_{\kappa}\theta(\pmb{a})$.

\medskip

\quad{\bf (v)} If
$b\in\frak B_{\kappa}$ and $\epsilon>0$, there is a $\delta>0$ such
that $\bar\mu_{\infty}(g(b')\Bsetminus g(b))\le\epsilon$ whenever
$\bar\nu_{\kappa}(b'\Bsetminus b)\le\delta$.   \Prf\Quer\
Otherwise, $g(b)\ne 1_{\frak A^{\Bbb N}}$ so $b\ne 1_{\frak B_{\kappa}}$
and we can find a sequence $\sequencen{b_n}$ in $\frak B_{\kappa}$
such that
$\bar\nu_{\kappa}(b_n\Bsetminus b)\le 2^{-n-2}(1-\bar\nu_{\kappa}b)$ and
$\bar\mu_{\infty}(g(b_n)\Bsetminus g(b))\ge\epsilon$ for every
$n\in\Bbb N$.   For each $n$, set
$b^*_n=b\Bcup\sup_{m\ge n}b_m$;  then

\Centerline{$\bar\mu_{\infty}g(b^*_n)
=\bar\mu_{\infty}g(b)+\bar\mu_{\infty}(g(b^*_n)\Bsetminus g(b))
\ge\bar\mu_{\infty}g(b)+\bar\mu_{\infty}(g(b_n)\Bsetminus g(b))
\ge\epsilon+\bar\mu_{\infty}g(b)$.}

\noindent Note that $\bar\nu_{\kappa}b^*_0<1$ so $g(b^*_0)$ has finite
measure.

The sequences $\sequencen{b^*_n}$,
$\sequencen{g(b^*_n)}$ and $\sequencen{\theta(g(b^*_n))}$
are all non-increasing.   Set $\pmb{a}=\inf_{n\in\Bbb N}g(b^*_n)$, so that

\Centerline{$\theta(\pmb{a})
\Bsubseteq\inf_{n\in\Bbb N}\theta(g(b^*_n))
\Bsubseteq\inf_{n\in\Bbb N}b^*_n=b$}

\noindent because $\bar\nu_{\kappa}(b^*_n\Bsetminus b)\le 2^{-n-1}$ for
every $n$.   It follows that $\pmb{a}\Bsubseteq g(b)$.   At the same time,

\Centerline{$\bar\mu_{\infty}\pmb{a}
=\lim_{n\to\infty}\bar\mu_{\infty}g(b^*_n)
>\bar\mu_{\infty}g(b)$,}

\noindent which is impossible.\ \Bang\Qed

\medskip

{\bf (c)} Define $\psi:\frak A^{\Bbb N}\to\frak A$ by setting
$\psi(\pmb{a})=\sup_{n\in\Bbb N}a_n$ whenever
$\pmb{a}=\sequencen{a_n}\in\frak A^{\Bbb N}$.

\medskip

\quad{\bf (i)} $\psi$ is supremum-preserving and $\psi(0)=0$.

\medskip

\quad{\bf (ii)} If $\pmb{a}$, $\pmb{a}'\in\frak A^{\Bbb N}$ then

\Centerline{$\bar\mu(\psi(\pmb{a})\Bsymmdiff\psi(\pmb{a}'))
\le\bar\mu_{\infty}(\pmb{a}\Bsymmdiff\pmb{a}')$
\quad$\bar\mu(\psi(\pmb{a})\Bsetminus\psi(\pmb{a}'))
\le\bar\mu_{\infty}(\pmb{a}\Bsetminus\pmb{a}')$.}

\medskip

\quad{\bf (iii)} Now if $b\in\frak B_{\kappa}$, $a\in\frak A$,
$a\Bsupseteq\psi(g(b))$ and
$\bar\mu a<\alpha\in\Bbb R$, there is a $b'\Bsupseteq b$ such that
$a\Bsubseteq\psi(g(b'))$, $\bar\mu\psi(g(b'))<\alpha$ and
$\bar\nu_{\kappa}(b'\setminus b)\le 1-\exp(-\bar\mu(a\Bsetminus\psi(g(b)))$.

\Prf\ Take $\alpha'$ such that $\bar\mu a<\alpha'<\alpha$.
By (b-v), there is a $\delta>0$ such that
$\bar\mu_{\infty}(g(b')\Bsetminus g(b))\le\alpha'-\bar\mu a$ whenever
$\bar\nu_{\kappa}(b'\Bsetminus b)\le\delta$.
Set $\pmb{a}^{(n)}=\sequence{i}{a_{ni}}$ for each
$n\in\Bbb N$, where $a_{ni}=a\setminus\psi(g(b))$ if $i=n$, $0$
otherwise.   For each $n\in\Bbb N$, let $\frak C_n$ be the closed
subalgebra of $\frak B_{\kappa}$ generated by

\Centerline{$\{\theta(\pmb{a}):\pmb{a}\in\frak A^{\Bbb N}$,
$\pmb{a}\Bcap\pmb{a}^{(m)}=0$ for every $m\ge n\}$,}

\noindent and let $T_n:L^1(\frak B_{\kappa},\bar\nu_{\kappa})
\to L^1(\frak B_{\kappa},\bar\nu_{\kappa})$ be
the corresponding conditional-expectation operator (365Q\formerly{3{}65R}).
Then $\sequencen{\frak C_n}$ is non-decreasing;  also
$\bar\nu_{\kappa}\theta(\pmb{a}^{(n)}\Bcap c)
=\bar\nu_{\kappa}\theta(\pmb{a}^{(n)})\cdot\bar\nu_{\kappa} c$ for every $n\in\Bbb N$ and
$c\in\frak C_n$, by the final clause of (a).   By L\'evy's martingale
theorem (275I, 367Jb), $\sequencen{T_n(\chi b)}$ is
$\|\,\|_1$-convergent.   We can therefore find an $n\in\Bbb N$ such that
$\|T_n(\chi b)-T_{n+1}(\chi b)\|_1\le\delta\exp(-\bar\mu a)$.
Set $b'=b\Bcup\theta(\pmb{a}^{(n)})$.   Then
$g(b')\Bsupseteq g(b)\Bcup\pmb{a}^{(n)}$, so
$\psi(g(b'))\Bsupseteq a_{nn}\Bcup\psi(g(b))=a$.  Also

\Centerline{$\bar\nu_{\kappa}(b'\Bsetminus b)
\le\bar\nu_{\kappa}\theta(\pmb{a}^{(n)})
=1-\exp(-\bar\mu_{\infty}\pmb{a}^{(n)})
=1-\exp(-\bar\mu(a\Bsetminus\psi(g(b)))$.}

\Quer\ If $\bar\mu\psi(g(b'))\ge\alpha$, set
$\pmb{e}=g(b')\Bsetminus(g(b)\Bcup\sup_{m\in\Bbb N}\pmb{a}^{(m)})$.
Since

\Centerline{$\psi(g(b)\Bcup\sup_{m\in\Bbb N}\pmb{a}^{(m)})
=\psi(g(b))\Bcup a=a$,}

\noindent $\psi(\pmb{e})\Bsupseteq\psi(g(b'))\Bsetminus a$  and

\Centerline{$\bar\mu_{\infty}\pmb{e}\ge\alpha-\bar\mu a
>\alpha'-\bar\mu a$;}

\noindent as
$\pmb{e}\Bsubseteq g(\theta(\pmb{e}))$ and $\pmb{e}\Bcap g(b)=0$,
$\bar\nu_{\kappa}(\theta(\pmb{e})\Bsetminus b)>\delta$.
On the other hand,

$$\eqalignno{(1-\bar\nu_{\kappa}\theta(\pmb{a}^{(n)}))
  \bar\nu_{\kappa}(b\Bcap\theta(\pmb{e}))
&=(1-\bar\nu_{\kappa}\theta(\pmb{a}^{(n)}))
  \int_{\theta(\pmb{e})}T_n(\chi b)\cr
\displaycause{because $\pmb{e}\Bcap\pmb{a}^{(m)}=0$ for every $m$, so
$\theta(\pmb{e})\in\frak C_n$}
&=\int\chi(1\Bsetminus\theta(\pmb{a}^{(n)}))\cdot
   \int T_n(\chi b)\times\chi\theta(\pmb{e})\cr
&=\int\chi(1\Bsetminus\theta(\pmb{a}^{(n)}))
  \times T_n(\chi b)\times\chi\theta(\pmb{e})\cr
\displaycause{because
$T_n(\chi b)\times\chi\theta(\pmb{e})\in L^0(\frak C_n)$ and
$\chi(1\Bsetminus\theta(\pmb{a}^{(n)}))$ are stochastically independent}
&=\int_{\theta(\pmb{e})\Bsetminus\theta(\pmb{a}^{(n)})}T_n(\chi b),\cr
\cr
(1-\bar\nu_{\kappa}\theta(\pmb{a}^{(n)}))\bar\nu_{\kappa}(\theta(\pmb{e}))
&=\bar\nu_{\kappa}(\theta(\pmb{e})\Bsetminus\theta(\pmb{a}^{(n)})
=\bar\nu_{\kappa}(b\Bcap\theta(\pmb{e})\Bsetminus\theta(\pmb{a}^{(n)})\cr
\displaycause{because
$\theta(\pmb{e})\Bsubseteq\theta(g(b'))\Bsubseteq b'
=b\Bcup\theta(\pmb{a}^{(n)})$}
&=\int_{\theta(\pmb{e})\Bsetminus\theta(\pmb{a}^{(n)})}
  T_{n+1}(\chi b)\cr}$$

\noindent because $\theta(\pmb{e})$ and $\theta(\pmb{a}^{(n)})$ both
belong to $\frak C_{n+1}$.   So

$$\eqalign{\delta\exp(-\bar\mu a)
&=\delta\exp(-\bar\mu_{\infty}\pmb{a}^{(n)})
<\bar\nu_{\kappa}(\theta(\pmb{e})\Bsetminus b)
  \exp(-\bar\mu_{\infty}\pmb{a}^{(n)})\cr
&=(1-\bar\nu_{\kappa}\theta(\pmb{a}^{(n)}))
  \bar\nu_{\kappa}(\theta(\pmb{e})\Bsetminus b))\cr
&=\int_{\theta(\pmb{e})\Bsetminus\theta(\pmb{a}^{(n)})}
  T_{n+1}(\chi b)-T_n(\chi b)\cr
&\le\|T_n(\chi b)-T_{n+1}(\chi b)\|_1
\le\delta\exp(-\bar\mu a),\cr}$$

\noindent which is impossible.\ \Bang

So $\bar\mu\psi(g(b'))<\alpha$, as required.\ \Qed

\medskip

{\bf (d)} Fix $c\in\frak B_{\kappa}$ with measure $\bover12$;
then the principal ideal of
$\frak B_{\kappa}$ generated by $c$ is isomorphic to
$\frak B_{\kappa}$ with the measure halved.   We therefore have a Boolean
isomorphism $\pi:\frak B_{\kappa}\to(\frak B_{\kappa})_c$ such that
$\bar\nu_{\kappa}\pi b=\bover12\bar\nu_{\kappa}b$ for every
$b\in\frak B_{\kappa}$.   Set
$h(b)=\psi(g(\pi^{-1}(b\Bcap c)))$ for $b\in\frak B_{\kappa}$.   Then
$h:\frak B_{\kappa}\to\frak A$ is order-preserving and $h(b)=h(b\Bcap c)$
for
every $b\in\frak B_{\kappa}$.   Translating the results of (b) and (c), we see
that

\inset{if $b\in\frak B_{\kappa}$ and $\epsilon>0$, there is a
$\delta>0$ such that $\bar\mu(h(b')\Bsetminus h(b))\le\epsilon$ whenever
$\bar\nu_{\kappa}(b'\Bsetminus b)\le\delta$,

if $b\in\frak B_{\kappa}$, $a\in\frak A$, $a\Bsupseteq h(b)$ and
$\bar\mu a<\alpha\in\Bbb R$, there is a $b'\Bsupseteq b$ such that
$a\Bsubseteq h(b')$, $\bar\mu h(b')<\alpha$ and
$\bar\nu_{\kappa}(b'\Bsetminus b)
\le\bover12(1-\exp(-\bar\mu(a\Bsetminus h(b)))$.}

\noindent  Note also that

$$\eqalign{\bar\mu h(b)
&=\bar\mu\psi(g(\pi^{-1}(b\Bcap c)))
\le\bar\mu_{\infty}g(\pi^{-1}(b\Bcap c))\cr
&\le-\ln(1-\bar\nu_{\kappa}\pi^{-1}(b\Bcap c))
\le-\ln(1-2\bar\nu_{\kappa}(b\Bcap c))\cr}$$

\noindent if we take $\ln(0)$ to be $-\infty$.

\medskip

{\bf (e)(i)} Set $\gamma_0=\bover12(1-\exp(-\bar\mu 1))$, interpreting
$\exp(-\infty)$ as $0$, so that $0<\gamma_0\le\bover12$.
Let $P$ be the partially ordered set
$\{(a,\alpha):a\in\frak A$, $\alpha\in\ocint{\bar\mu a,\bar\mu 1}\}$
and $Q$ the partially ordered set
$\{b:b\in\frak B_{\kappa}$, $\bar\nu_{\kappa} b<\gamma_0\}$, so that
$\AM^*(\frak A,\bar\mu)=\RO^{\uparrow}(P)$ and
$\AM(\frak B_{\kappa},\bar\nu_{\kappa},\gamma_0)=\RO^{\uparrow}(Q)$.
For $b\in Q$, set
$\alpha_b=\sup\{\bar\mu h(b'):b\Bsubseteq b'\in Q\}$.
Then $\alpha_b>\bar\mu(h(b))$.   \Prf\ We have

\Centerline{$\bar\mu h(b)\le-\ln(1-2\bar\nu_{\kappa}(b\Bcap c))
<-\ln(1-2\gamma_0)=\bar\mu 1$,}

\noindent so $h(b)\ne 1$.   Because $\frak A$ is atomless,
there is an $a\in\frak A$,
disjoint from $h(b)$, such that $0<\bar\mu a<-\ln(2\bar\nu_{\kappa}b)$.
Set $\pmb{a}=\sequencen{a_n}$ where $a_0=a$, $a_n=0$ for $n\ge 1$.   Then

\Centerline{$\bar\nu_{\kappa}\theta(\pmb{a})
=1-\exp(-\bar\mu a)<1-2\bar\nu_{\kappa}b$,}

\noindent so $b'=b\Bcup\pi\theta(\pmb{a})\in Q$, while
$h(b')\Bsupseteq h(b)\Bcup a\Bsupset h(b)$.\ \Qed

\medskip

\quad{\bf (ii)} If $b\in Q$ and $\bar\mu h(b)<\alpha$, there is
a $b_1\in Q$ such that $b\Bsubseteq b_1$, $h(b_1)=h(b)$ and
$\alpha_{b'}\le\alpha$.   \Prf\ Let $\delta>0$ be such that
$\bar\mu(h(b')\Bsetminus h(b))\le\alpha-\bar\mu h(b)$ whenever
$\bar\nu_{\kappa}(b'\Bsetminus b)\le\delta$.   Because
$\gamma_0\le\bover12$, there is a $b_1\in Q$ such that
$b\Bsubseteq b_1$, $b\Bcap c=b_1\Bcap c$ and
$\bar\mu b_1\ge\gamma_0-\delta$.   Then $h(b_1)=h(b)$.   If $b'\in Q$
and $b'\Bsupseteq b_1$, then
$\bar\nu_{\kappa}(b'\Bcap c\Bsetminus b)\le\delta$, so

\Centerline{$\bar\mu(h(b')\Bsetminus h(b))
=\bar\mu(h(b'\Bcap c)\Bsetminus h(b))\le\alpha-\bar\mu h(b)$}

\noindent and $\bar\mu h(b')\le\alpha$;  thus $\alpha_{b_2}\le\alpha$.\
\Qed

\medskip

{\bf (f)} By (e-i), we can define $f:Q\to P$ by setting
$f(b)=(h(b),\alpha_b)$ for $b\in Q$.

\medskip

\quad{\bf (i)} $f$ is order-preserving because $h$ is.

\medskip

\quad{\bf (ii)} If $P_1\subseteq P$ is up-open and cofinal with $P$,
$f^{-1}[P_1]$ is cofinal with $Q$.   \Prf\ Take any $b\in Q$.   Set

\Centerline{$\alpha
=\min(\alpha_b,\bar\mu h(b)-\ln(1-2\gamma_0+2\bar\nu_{\kappa}b))
>\bar\mu h(b)$,}

\noindent so that $f(b)\le(h(b),\alpha)$ in $P$.   Then
there is an $(a,\beta)\in P_1$ such that $(h(b),\alpha)\le(a,\beta)$,
that is, $h(b)\Bsubseteq a$ and $\beta\le\alpha$.   In this case, there
is a $b_1\in\frak B_{\kappa}$ such that
$b_1\Bsupseteq b$, $h(b_1)\Bsupseteq a$, $\bar\mu h(b_1)<\beta$ and

\Centerline{$\bar\nu_{\kappa}(b_1\Bsetminus b)
\le\Bover12(1-\exp(-\bar\mu(a\Bsetminus h(b))))
<\gamma_0-\bar\nu b$}

\noindent because $\bar\mu(a\Bsetminus h(b))
<-\ln(1-2\gamma_0+2\bar\nu_{\kappa}b)$.   So
$b_1\in Q$.   By (e-ii), there is a $b_2\in Q$ such that
$b_2\Bsupseteq b_1$, $h(b_2)=h(b_1)$ and $\alpha_{b_2}\le\beta$.
Now $b\Bsubseteq b_2$, while $f(b_2)=(h(b_1),\alpha_{b_2})\ge(a,\beta)$.
As $P_1$ is up-open, $f(b_2)\in P_1$;  as $b$ is
arbitrary, $f^{-1}[P_1]$ is cofinal with $Q$.\ \Qed

\medskip

\quad{\bf (iii)} $f[Q]$ is cofinal with $P$.
\Prf\ Take $(a,\alpha)\in P$.
Set $\pmb{a}=\sequencen{a_n}\in\frak A^{\Bbb N}$ where $a_0=a$ and
$a_n=0$ for $n\ge 1$, and set $b=\pi\theta(\pmb{a})$.   Note that

\Centerline{$\bar\nu_{\kappa}b=\Bover12(1-\exp(-\bar\mu a))<\gamma_0$,}

\noindent so $b\in Q$.   Because
$\bar\mu_{\infty}\pmb{a}<\infty$, $g(\pi^{-1}b)=\pmb{a}$ and $h(b)=a$.
By (e-ii) again, we can now find a
$b_1\Bsupseteq b$ in $Q$ such that $h(b_1)=h(b)$ and
$\alpha_{b_1}\le\alpha$.   So $f(b_1)\ge(a,\alpha)$.   As $(a,\alpha)$
is arbitrary, $f[Q]$ is cofinal.\ \Qed

\medskip

{\bf (g)} By 514O, $\AM^*(\frak A,\bar\mu)=\RO^{\uparrow}(P)$ can be
regularly embedded in

\Centerline{$\RO^{\uparrow}(Q)
=\AM(\frak B_{\kappa},\bar\nu_{\kappa},\gamma_0)
\cong\AM(\frak B_{\kappa},\nu_{\kappa},\bover12)$}

\noindent by 528Da.

\medskip

{\bf (h)} All this has been done on the assumption that $\frak A$ is
atomless, as required in (e).   For the general case, consider the
localizable measure algebra free product
$(\frak C,\bar\lambda)$ of $(\frak A,\bar\mu)$ and
$(\frak B_{\omega},\bar\nu_{\omega})$ (325E).   By 521Qa, we have

\Centerline{$\max(\omega,c(\frak C),\tau(\frak C))
\le\max(\omega,c(\frak A),c(\frak B_{\omega}),\tau(\frak A),
  \tau(\frak B_{\omega}))
\le\kappa$.}

\noindent Also $\frak C$ is atomless because $\frak B_{\omega}$ is
isomorphic to a closed subalgebra of $\frak C$ (325Dd) and is atomless.
%316Xi
By (a)-(g), $\AM^*(\frak C,\bar\lambda)$ can be regularly embedded in
$\AM(\frak B_{\omega},\bar\nu_{\omega},\bover12)$.   Now consider the
canonical embedding $\varepsilon_1:\frak A\to\frak C$.
This is order-continuous and measure-preserving (325Da), so identifies the
Dedekind $\sigma$-complete Boolean algebra $\frak A$ with a
$\sigma$-subalgebra of $\frak C$;  also $\frak A^f$ has supremum $1$ both
in $\frak A$ and $\frak C$.   By 528G, $\AM^*(\frak A,\bar\mu)$ can be
regularly embedded in $\AM^*(\frak C,\bar\lambda)$ and therefore in
$\AM(\frak B_{\omega},\bar\nu_{\omega},\bover12)$.
}%end of proof of 528H

\leader{528I}{Definition} For any set $I$, the
{\bf $I$-localization poset} is the set

\Centerline{$\Cal S_I^{\infty}
=\{p:p\subseteq\Bbb N\times I$, $\#(p[\{n\}])\le 2^n$
for every $n$, $\sup_{n\in\Bbb N}\#(p[\{n\}])$ is finite$\}$,}

\noindent ordered by $\subseteq$.   For $p\in\Cal S_I^{\infty}$ set
$\|p\|=\max_{n\in\Bbb N}\#(p[\{n\}])$.    I will write $\Cal S^{\infty}$
for the $\Bbb N$-localization poset $\Cal S^{\infty}_{\Bbb N}$\cmmnt{,
already introduced in the proof of 522T}.

\leader{528J}{Proposition} Let $\kappa$ be an infinite cardinal,
$\Cal S_{\kappa}^{\infty}$ the $\kappa$-localization poset,
and $(\frak A,\bar\mu)$ a
semi-finite measure algebra, not $\{0\}$, with
$\kappa\ge\max(\omega,c(\frak A),\tau(\frak A))$.
Then the variable-measure amoeba algebra
$\AM^*(\frak A,\bar\mu)$ can be regularly embedded in
$\RO^{\uparrow}(\Cal S_{\kappa}^{\infty})$.

\proof{{\bf (a)} To begin with (down to the end of (d) below), suppose that
$\frak A$ is atomless.   Let $P$ be the partially ordered set
$\{(a,\alpha):a\in\frak A$, $\alpha\in\ocint{\bar\mu a,\mu 1}\}$, so that
$\AM^*(\frak A,\bar\mu)=\RO^{\uparrow}(P)$.   Give
$\frak A^f$ its measure metric, so that its topological density is at
most $\kappa$ (521Eb).
Set $\gamma_0=\bover12\bar\mu 1$ and for $n\ge 1$ set
$\gamma_n=2^{-2n-1}\bar\mu 1$ if $\bar\mu 1<\infty$, $4^{-n}$ otherwise.
For each $n$, let $D_n$ be a dense subset of
$\{a:a\in\frak A^f$, $\bar\mu a\le\gamma_n\}$, containing $0$, with cardinal at
most $\kappa$, and let
$\ofamily{\xi}{\kappa}{d_{n\xi}}$ be a family running over $D_n$ with
cofinal repetitions.

\medskip

{\bf (b)} If $p\in\Cal S_{\kappa}^{\infty}$ set

\Centerline{$a_p=\sup_{(n,\xi)\in p}d_{n\xi}$,
\quad$\alpha_p
=\bar\mu a_p+\sum_{n=0}^{\infty}(2^n-\#(p[\{n\}]))\gamma_n$.}

\noindent Then

\Centerline{$\bar\mu a_p<\alpha_p
\le\sum_{n=0}^{\infty}2^n\gamma_n=\bar\mu 1$,}

\noindent so we can define $f:\Cal S_{\kappa}^{\infty}\to P$ by setting
$f(p)=(a_p,\alpha_p)$.    $f$ is order-preserving, because if
$p\subseteq p'$ in $\Cal S_{\kappa}^{\infty}$ then

$$\eqalign{\alpha_{p'}
&=\bar\mu a_{p'}+\sum_{n=0}^{\infty}(2^n-\#(p'[\{n\}]))\gamma_n\cr
&\le\bar\mu a_p+\sum_{(n,\xi)\in p'\setminus p}\bar\mu d_{n\xi}
  +\sum_{n=0}^{\infty}(2^n-\#(p'[\{n\}]))\gamma_n\cr
&\le\bar\mu a_p
  +\sum_{n=0}^{\infty}\#(p'[\{n\}]\setminus p[\{n\}])\gamma_n
  +\sum_{n=0}^{\infty}(2^n-\#(p'[\{n\}]))\gamma_n
=\alpha_p.\cr}$$

\medskip

{\bf (c)} Suppose that $p\in\Cal S_{\kappa}^{\infty}$ and
$f(p)\le(a,\alpha)\in P$.   Take $\alpha'\in\ooint{\bar\mu a,\alpha}$.
For $n\in\Bbb N$, set $k_n=2^n-\#(p[\{n\}])$.   Then $a_p\Bsubseteq a$ and

\Centerline{$\bar\mu a
<\alpha\le\alpha_p=\bar\mu a_p+\sum_{n=0}^{\infty}k_n\gamma_n$.}

\noindent So there is an $r\in\Bbb N$ such that

\Centerline{$\bar\mu a
<\bar\mu a_p+\sum_{n=0}^{\infty}\gamma_n\min(r,k_n)$;}

\noindent take $r$ so large that, in addition,
$\sum_{n=r+1}^{\infty}2^n\gamma_n\le\alpha-\alpha'$.

For each $n$, set $k'_n=\min(r,k_n)$ and
$C_n=\{\sup D:D\in[D_n]^{\le k'_n}\}$.
Then (because $\frak A$ is atomless) $C_n$ is dense in
$\{c:c\in\frak A^f$, $\bar\mu c\le k'_n\gamma_n\}$.   We can therefore
choose $\sequencen{c_n}$ inductively in such a way that

\Centerline{$c_n\in C_n$,
\quad$\bar\mu(a\Bcup\sup_{m<n}c_m)<\alpha'$,}

\Centerline{$\bar\mu(a\Bsetminus(a_p\Bcup\sup_{m<n}c_m))
<\sum_{m=n}^{\infty}k'_m\gamma_m$}

\noindent for every $n\in\Bbb N$.   \Prf\
For the inductive step to $n\ge 0$, set
$b=a\Bsetminus(a_p\Bcup\sup_{m<n}c_m)$.   Take $b'\Bsubseteq b$ such that
$\bar\mu b'=\min(k'_n\gamma_n,\bar\mu b)$, so that

\Centerline{$\bar\mu(a\Bcup\sup_{m<n}c_m\Bcup b')
=\bar\mu(a\Bcup\sup_{m<n}c_m)<\alpha'$,}

\Centerline{$\bar\mu(b\setminus b')<\sum_{m=n+1}^{\infty}k'_m\gamma_m$.}

\noindent Let $c_n\in C_n$ be such that

\Centerline{$\alpha'
>\bar\mu(a\Bcup\sup_{m<n}c_m\Bcup b')+\bar\mu(c_n\Bsetminus b')
\ge\bar\mu(a\Bcup\sup_{m\le n}c_m)$,}

\Centerline{$\sum_{m=n+1}^{\infty}k'_m\gamma_m
>\bar\mu(b\setminus b')+\bar\mu(b'\Bsetminus c_n)
=\bar\mu(a\Bsetminus(a_p\Bcup\sup_{m\le n}c_m))$}

\noindent and the induction proceeds.\ \Qed

For each $n$, we can find a set $D'_n\subseteq D_n$, of size $k'_n$,
such that $c_n=\sup D'_n$.   Because $\ofamily{\xi}{\kappa}{d_{n\xi}}$
runs over $D_n$ with cofinal repetitions, we can find a set
$I_n\subseteq\kappa\setminus p[\{n\}]$ such that $\#(I_n)=k'_n$ and
$c_n=\sup_{\xi\in I_n}d_{n\xi}$.   Set
$q=p\cup\{(n,\xi):n\in\Bbb N$, $\xi\in I_n\}$.   Then

\Centerline{$\#(q[\{n\}])\le\#(p[\{n\}])+k'_n\le\min(2^n,\|p\|+r)$}

\noindent for every $n$, so $q\in\Cal S_{\kappa}^{\infty}$ and
$p\subseteq q$.    Now

\Centerline{$a_q=a_p\Bcup\sup_{n\in\Bbb N,\xi\in I_n}d_{n\xi}
=a_p\Bcup\sup_{n\in\Bbb N}c_n\Bsupseteq a$}

\noindent because

\Centerline{$\bar\mu(a\Bsetminus a_q)
\le\inf_{n\in\Bbb N}\bar\mu(a\Bsetminus(a_p\Bcup\sup_{m<n}c_m))
\le\inf_{n\in\Bbb N}\sum_{m=n}^{\infty}2^m\gamma_m
=0$.}

\noindent Also

\Centerline{$\bar\mu a_q
=\sup_{n\in\Bbb N}\bar\mu(a_p\Bcup\sup_{m<n}c_m)
\le\alpha'<\alpha$,}

\noindent while $\#(q[\{n\}])=\#(p[\{n\}])+k_n=2^n$ whenever $n\le r$,
so

\Centerline{$\alpha_q
=\bar\mu a_q+\sum_{n=r+1}^{\infty}(2^n-\#(q[\{n\}]))\gamma_n
\le\alpha'+\sum_{n=r+1}^{\infty}2^n\gamma_n
\le\alpha$.}

\medskip

{\bf (d)} What (c) shows is that if $p\in\Cal S_{\kappa}^{\infty}$ and
$f(p)\le(a,\alpha)$ in $P$, then there is a $q\supseteq p$ in
$\Cal S_{\kappa}^{\infty}$ such that $(a,\alpha)\le f(q)$.
Next, $\Cal S^{\infty}_{\kappa}$ has a least element $\emptyset$, and
$f(\emptyset)=(0,\bar\mu 1)$ is the least element of $P$.   So 514P tells
us that
$\RO^{\uparrow}(P)=\AM^*(\frak A,\bar\mu)$ can be regularly embedded in
$\RO^{\uparrow}(\Cal S_{\kappa}^{\infty})$.

\medskip

{\bf (e)} As for the general case, we can use the same trick as in
part (h) of the proof of 528H.   Let $(\frak C,\bar\lambda)$ be the
localizable measure algebra free product of $(\frak A,\bar\mu)$ and
$(\frak B_{\omega},\bar\nu_{\omega})$;  as before, $\frak C$ is atomless,
$\max(\omega,c(\frak C),\tau(\frak C))\le\kappa$ and $(\frak A,\bar\mu)$
is embedded in $(\frak C,\bar\lambda)$ as a $\sigma$-subalgebra with
sufficient elements of finite measure.   So
$\AM^*(\frak A,\bar\mu)$ is regularly embedded in
$\AM^*(\frak C,\bar\lambda)$ and in
$\RO^{\uparrow}(\Cal S_{\kappa}^{\infty})$.
}%end of proof of 528J

\leader{528K}{Theorem}\cmmnt{ ({\smc Truss 88})} Let
$(\frak A,\bar\mu)$ be an atomless $\sigma$-finite measure algebra in which
every non-zero principal ideal has Maharam type $\kappa$,
and $0<\gamma<\bar\mu 1$.   Then each of the algebras

\Centerline{$\AM(\frak A,\bar\mu,\gamma)$,
\quad$\AM^*(\frak A,\bar\mu)$,
\quad$\AM(\frak B_{\kappa},\bar\nu_{\kappa},\bover12)$}

\noindent can be regularly embedded in the other two, and all three can
be regularly embedded in $\RO^{\uparrow}(\Cal S_{\kappa}^{\infty})$.

\proof{ By 528H,
$\AM^*(\frak A,\bar\mu)$ can be regularly embedded in
$\AM(\frak B_{\kappa},\bar\nu_{\kappa},\bover12)$.   Take any
$e\in\frak A$ such that $\gamma<\bar\mu e<\infty$.   Then the principal
ideal $(\frak A_e,\bar\mu\restrp\frak A_e)$ is isomorphic, up to a
scalar multiple of the measure, to
$(\frak B_{\kappa},\bar\nu_{\kappa})$, so

$$\eqalignno{\AM(\frak B_{\kappa},\bar\nu_{\kappa},\textstyle{\bover12})
&\cong\AM(\frak B_{\kappa},\bar\nu_{\kappa},
  \Bover{\gamma}{\bar\mu e})\cr
\displaycause{528Da}
&\cong\AM(\frak A_e,\bar\mu\restrp\frak A_e,\gamma)\cr}$$

\noindent can be regularly embedded in $\AM(\frak A,\bar\mu,\gamma)$
(528Fa).   By 528Fc,
$\AM(\frak A,\bar\mu,\gamma)$ can be regularly embedded in
$\AM^*(\frak A,\bar\mu)$.   Finally, by 528J, $\AM^*(\frak A,\bar\mu)$
can be regularly embedded in $\RO^{\uparrow}(\Cal S_{\kappa}^{\infty})$.
Because regular embeddability is transitive (313N), these facts are
enough to prove the theorem.
}%end of proof of 528K

\leader{528L}{}\cmmnt{ It is possible without great effort to
calculate many of the cardinal functions of these algebras.

\medskip

\noindent}{\bf Lemma}
$\frak m(\AM(\frak B_{\omega},\bar\nu_{\omega},\bover12))\le\add\Cal N$,
where $\Cal N$ is the null ideal of Lebesgue measure on $\Bbb R$.

\proof{ Set $P=\{a:a\in\frak B_{\omega}$, $\bar\nu_{\omega}a<\bover12\}$.
Then $\wdistr(\frak B_{\omega})\ge\frak m^{\uparrow}(P)$.   \Prf\ Take a
family $\ofamily{\xi}{\kappa}{B_{\xi}}$ of maximal antichains in
$\frak B_{\omega}$, where
$\kappa<\frak m^{\uparrow}(P)$.   Let $C\subseteq\frak B_{\omega}$ be a
maximal disjoint set such that $\{b:b\in B_{\xi}$, $b\Bcap c\ne 0\}$ is
finite for every $\xi<\kappa$ and $c\in C$.   \Quer\ Suppose, if
possible, that $c_0=1\Bsetminus\sup C$ is not $0$.   Take $a_0\in P$
such that $\bar\nu_{\omega}(a_0\Bcup c_0)>\bover12$.
(If $\bar\nu_{\omega}c_0>\bover12$, take $a_0=0$;
otherwise, take $a_0\Bsubseteq 1\Bsetminus c_0$ such that
$\bover12-\bar\nu_{\omega}c_0<\bar\nu_{\omega}a_0<\bover12$.)
For each $\xi<\kappa$, set

\Centerline{$Q_{\xi}
=\{a:a\in P$, $\{b:b\in B_{\xi}$, $b\notBsubseteq a\}$ is finite$\}$;}

\noindent then $Q_{\xi}$ is cofinal with $P$.   There is therefore an
upwards-directed $R\subseteq P$ such that $a_0\in R$ and $R$ meets every
$Q_{\xi}$.   Set $e=\sup R$;  then $\bar\nu_{\omega}e\le\bover12$ so
$c_1=c_0\setminus e=(a_0\Bcup c_0)\setminus e$ is non-zero.

If $\xi<\kappa$, there is an $a\in R\cap Q_{\xi}$, so that

\Centerline{$\{b:b\in B_{\xi}$, $b\Bcap c_1\ne 0\}
\subseteq\{b:b\in B_{\xi}$, $b\notBsubseteq a\}$}

\noindent is finite.   But this means that we ought to have added $c_1$
to $C$.\ \Bang

Thus $C$ is a maximal antichain.   As $\ofamily{\xi}{\kappa}{B_{\xi}}$
is arbitrary, $\wdistr(\frak A)\ge\frak m^{\uparrow}(P)$.\ \Qed

Now 524Mb tells us that $\wdistr(\frak B_{\omega})=\add\Cal N$, so
$\frak m^{\uparrow}(P)\le\add\Cal N$.   Finally, by 517Db,

\Centerline{$\frak m(\AM(\frak A,\bar\mu,\gamma))=\frak m^{\uparrow}(P)
\le\add\Cal N$,}

\noindent as claimed.
}%end of proof of 528L

\leader{528M}{Lemma}
$\frak m^{\uparrow}(\Cal S^{\infty})\ge\add\Cal N$.

\proof{{\bf (a)} Recall the definition of the supported relations
$(\NN,\subseteq^*,\Cal S^{(\alpha)}$) from 522L, where
$\Cal S^{(\alpha)}=\{S:S\subseteq\Bbb N\times\Bbb N$,
$\#(S[\{n\}])\le\alpha(n)$ for every $n\in\Bbb N\}$ for $\alpha\in\NN$.
Putting 522L, %indep of \alpha
522M %OK for usual \alpha
and 512Db %\add(C,S,D)\le\add(A,R,B)
together, we have
$\add(\NN,\subseteq^*,\Cal S^{(\alpha)})=\add\Cal N$ whenever
$\lim_{n\to\infty}\alpha(n)=\infty$.

\medskip

{\bf (b)} The core of the argument is the following fact.   Suppose that
$Q\subseteq\Cal S^{\infty}$ is cofinal and up-open, $n\in\Bbb N$ and
$\sigma\in[\Bbb N\times\Bbb N]^{<\omega}$.   Let
$G\subseteq\Bbb N\times\Bbb N$ be a set with finite vertical sections.
Then there is a $k\in\Bbb N$ such that whenever
$\sigma\subseteq p\in\Cal S^{\infty}$, $p\subseteq\sigma\cup G$ and
$\|p\|\le n$,
there is a $q\in Q$ such that $p\subseteq q$ and $\|q\|\le k$.

\Prf\Quer\ Suppose, if possible, otherwise.   Then for each $j\in\Bbb N$
we can find $p_j\in\Cal S^{\infty}$ such that
$\sigma\subseteq p_j\subseteq\sigma\cup G$, $\|p_j\|\le n$ and $\|q\|>j$
whenever $p\subseteq q\in Q$.   Let $p$ be a cluster point of
$\sequence{j}{p_j}$ in $\Cal P(\sigma\cup G)$.   Then
$\#(p[\{i\}])\le\sup_{j\in\Bbb N}\#(p_j[\{i\}])\le\min(2^i,n)$ for every
$i$, so $p\in\Cal S^{\infty}$.  Because $Q$ is cofinal with $\Cal
S^{\infty}$, there is a $q\in Q$
such that $p\subseteq q$.   Set $k=n+\|q\|$.   Then
$(\sigma\cup G)\cap(k\times\Bbb N)$ is finite, so there is an $i\ge k$
such that $p_i\cap(k\times\Bbb N)=p\cap(k\times\Bbb N)\subseteq q$.
Set $q'=p_i\cup q$.   Then

$$\eqalign{\#(q'[\{j\}])
&=\#(q[\{j\}])\le\min(\|q\|,2^j)\text{ if }j<k,\cr
&\le\|p_i\|+\|q\|\le k\le 2^j\text{ otherwise}.\cr}$$

\noindent So $q'\in\Cal S^{\infty}$ and $\|q'\|\le k\le i$;  because $Q$
is up-open in $\Cal S^{\infty}$, $q'\in Q$, while $p_i\subseteq q'$.
But we
chose $p_i$ so that this could not happen.\ \Bang\Qed

\medskip

{\bf (c)} We need to know that $\Cal S^{\infty}$ is upwards-ccc.   \Prf\
For any $n\in\Bbb N$, finite $\sigma\subseteq\Bbb N\times\Bbb N$ the set
$\{p:p\in\Cal S^{\infty}$, $\|p\|\le 2^{n-1}$,
$p\cap(n\times\Bbb N)=\sigma\}$ is upwards-linked.\ \Qed

\medskip

{\bf (d)} Now let $\ofamily{\xi}{\kappa}{Q_{\xi}}$ be any family of
cofinal subsets of $\Cal S^{\infty}$, where $\kappa<\add\Cal N$, and
$p_0\in\Cal S^{\infty}$.   For each $\xi<\kappa$ let $A_{\xi}\subseteq
Q_{\xi}$ be a
maximal up-antichain;  by (c), $A_{\xi}$ is countable.   Set
$Q'_{\xi}=\bigcup\{\coint{q,\infty}:q\in A_{\xi}\}$, so that $Q'_{\xi}$
is an up-open cofinal subset of $\Cal S^{\infty}$.   Set
$A=\{p_0\}\cup\bigcup_{\xi<\kappa}A_{\xi}$.   For $q\in A$, let
$F_q\subseteq\NN$ be a finite set such that (identifying each member of
$F_q$ with its graph) $q\subseteq\bigcup F_q$;
set $F=\bigcup_{q\in A}F_q$, so that

\Centerline{$\#(F)\le\max(\omega,\kappa)<\add\Cal N\le\frak b$}

\noindent (522B).   Let $g_0\in\NN$ be a strictly increasing function
such that $\{i:f(i)>g_0(i)\}$ is finite for every $f\in F$, and also
$p_0[\{i\}]\subseteq g_0(i)$ for every $i$.   Set
$G=\{(i,j):i\in\Bbb N$, $j<g_0(i)\}$, so that
$G\subseteq\Bbb N\times\Bbb N$ has finite vertical sections.
Observe that if $q\in A$ then $q\setminus G$ is finite.

For each $\xi<\kappa$, $n\in\Bbb N$ and finite
$\sigma\subseteq\Bbb N\times\Bbb N$, let $k(\xi,\sigma,n)\in\Bbb N$ be
such that whenever $p\in\Cal S^{\infty}$ and
$\sigma\subseteq p\subseteq\sigma\cup G$
then there is a $q\in Q'_{\xi}$ such that $p\subseteq q$ and
$\|q\|\le k(\xi,\sigma,n)$;  such a $k$ exists by (b) above.   Set
$k_{\xi}(n)=\sup\{k(\xi,\sigma,n):\sigma\subseteq n\times g_0(n)\}$.
Again because $\kappa<\frak b$, there is a $g_1\in\NN$ such that
$\{n:k_{\xi}(n)>g_1(n)\}$ is finite for every $\xi<\kappa$.   Let
$\alpha\in\NN$ be a non-decreasing function
such that $\lim_{n\to\infty}\alpha(n)=\infty$ and

\Centerline{$\alpha(2g_1(n))\le n$,
\quad$\alpha(n)+\#(p_0[\{n\}])\le 2^n$,
\quad$2\alpha(n)\le n$}

\noindent for every $n$.

Because $\add(\NN,\subseteq^*,\Cal S^{(\alpha)})=\add\Cal N$, there is an
$S_0\in\Cal S^{(\alpha)}$ such that $f\subseteq^*S_0$
for every $f\in F$, so that $q\setminus S_0$ is finite for every
$q\in A$.   Replacing $S_0$ by $S_0\cap G$ if necessary, we may suppose
that $S_0\subseteq G$.

\medskip

{\bf (e)} Let $\Cal S$ be the family of subsets $S$ of $\Bbb N\times\Bbb N$
such that $\#(S[\{n\}])\le 2^n$ for every $n$, as in 522K.
Note that $p_0\cup S_0\in\Cal S$,
because $\alpha(n)+\#(p_0(n))\le 2^n$ for every $n$.
Let $C$ be the family of finite subsets $\sigma$ of
$\Bbb N\times\Bbb N$ such that $\sigma\cup S_0\in\Cal S$.
For each $\xi<\kappa$, set

\Centerline{$D_{\xi}
=\{\sigma:\sigma\in C$, $\Exists q\in A_{\xi}$,
  $q\subseteq\sigma\cup S_0\}$.}

\noindent Then $D_{\xi}$ is cofinal with $C$.   \Prf\ Let $\sigma\in C$.
Let $n_0$ be so large that $g_1(2^{n_0})\ge k_{\xi}(2^{n_0})$ and
$\sigma\subseteq n_0\times g_0(n_0)$.   Set $m=2g_1(2^{n_0})$,
$p=\sigma\cup(S_0\cap(m\times\Bbb N))\in\Cal S^{\infty}$.   Then
$\sigma\subseteq p\subseteq\sigma\cup G$ and
$\|p\|\le\max(2^{n_0},\alpha(m))=2^{n_0}$, so there is a $q\in Q'_{\xi}$
such that $p\subseteq q$ and

\Centerline{$\|q\|\le k(\xi,\sigma,2^{n_0})\le k_{\xi}(2^{n_0})
\le g_1(2^{n_0})=\Bover{m}{2}$.}

\noindent Let $q'\in A_{\xi}$ be such that $q'\subseteq q$.   Let
$m'\ge\max(m,n_0)$ be such that $q'\subseteq(m'\times\Bbb N)\cup S_0$,
and set $\tau=q\cap(m'\times\Bbb N)$, so that $\sigma\subseteq\tau$.

For $n<m$, we have

\Centerline{$S_0[\{n\}]\subseteq p[\{n\}]\subseteq q[\{n\}]
=\tau[\{n\}]$,}

\noindent so $(\tau\cup S_0)[\{n\}]=q[\{n\}]$ has at most $2^n$ members.
For $m\le n<m'$, we have

\Centerline{$\#((\tau\cup S_0)[\{n\}])\le\#(q[\{n\}])+\#(S_0[\{n\}])
\le\|q\|+\alpha(n)\le\Bover{m}2+\Bover{n}2\le 2^n$,}

\noindent while for $n\ge m'$ we have

\Centerline{$\#((\tau\cup S_0)[\{n\}])
=\#(S_0[\{n\}])\le\alpha(n)\le 2^n$.}

\noindent So $\tau\cup S_0\in\Cal S$ and $\tau\in C$.   Since

\Centerline{$q'\subseteq(q'\cap(m'\times\Bbb N))\cup S_0
\subseteq(q\cap(m'\times\Bbb N))\cup S_0=\tau\cup S_0$,}

\noindent $\tau\in D_{\xi}$.   As $\sigma$ is arbitrary, $D_{\xi}$ is
cofinal with $C$.\ \Qed

\medskip

{\bf (f)} Because $p_0\cup S_0\in\Cal S$,
$\sigma_0=p_0\setminus S_0$ belongs to $C$.
Because $\kappa<\add\Cal N\le\frakmctbl\le\frak m^{\uparrow}(C)$, there
is an upwards-directed set $E\subseteq C$ meeting every $D_{\xi}$ and
containing $\sigma_0$.   Set $S_1=S_0\cup\bigcup E$.   Then, because $E$
is upwards-directed,

\Centerline{$\#(S_1[\{n\}])
=\sup_{\sigma\in E}\#((\sigma\cup S_0)[\{n\}])\le 2^n$}

\noindent for every $n$, and $S_1\in\Cal S$.   Set
$R=\{p:p\in\Cal S^{\infty}$, $p\subseteq S_1\}$;  then
$R\subseteq\Cal S^{\infty}$ is
upwards-directed (in fact, closed under $\cup$), and $p_0\in R$ because
$\sigma_0\in E$.   Now $R$ meets $Q_{\xi}$ for every $\xi<\kappa$.
\Prf\ There is a
$\sigma\in D_{\xi}\cap E$.   But this means that there is a
$q\in A_{\xi}$ such that
$q\subseteq\sigma\cup S_0\subseteq S_1$ and $q\in R\cap Q_{\xi}$.\ \Qed

As $p_0$ and $\ofamily{\xi}{\kappa}{Q_{\xi}}$ are arbitrary,
$\frak m^{\uparrow}(\Cal S^{\infty})\ge\add\Cal N$.
}%end of proof of 528M

\leader{528N}{Theorem}\cmmnt{ ({\smc Brendle 00}, 2.3.10;
{\smc Judah \& Repick\'y 95})}
Let $(\frak A,\bar\mu)$ be an atomless $\sigma$-finite measure algebra
with countable Maharam type, and $0<\gamma<\bar\mu 1$.   Then the
algebras $\AM(\frak A,\bar\mu,\gamma)$ and $\AM^*(\frak A,\bar\mu)$
and the $\Bbb N$-localization poset $\Cal S^{\infty}$ (active upwards)
all have Martin numbers equal to $\add\Cal N$.

\proof{ By 517Ia and 528K, with 517Db again,

$$\eqalign{\frak m(\AM(\frak A,\bar\mu,\gamma))
&=\frak m(\text{AM}^*(\frak A,\bar\mu))
=\frak m(\AM(\frak B_{\omega},\bar\nu_{\omega},\textstyle{\bover12}))\cr
&\ge\frak m(\RO^{\uparrow}(\Cal S^{\infty}))
=\frak m^{\uparrow}(\Cal S^{\infty}).\cr}$$

\noindent As

\Centerline{$\frak m(\AM(\frak B_{\omega},\bar\nu_{\omega},\bover12))
\le\add\Cal N\le\frak m^{\uparrow}(\Cal S^{\infty})$}

\noindent (528L, 528M), all these are equal to $\add\Cal N$.
}%end of proof of 528N

\leader{528O}{Corollary} Let $\gamma>0$.   Let $\Cal G$ be the
partially ordered set

\Centerline{$\{G:G\subseteq\Bbb R$ is open, $\mu_LG<\gamma\}$,}

\noindent where $\mu_L$ is Lebesgue measure.   Then
$\frak m^{\uparrow}(\Cal G)=\add\Cal N$.

\proof{ Put 528C and 528N together.
}%end of proof of 528O

\vleader{72pt}{528P}{Proposition}
Let $(\frak A,\bar\mu)$ be an atomless
semi-finite measure algebra, and $0<\gamma<\bar\mu 1$.

(a) For any integer $m\ge 2$,

\Centerline{$c(\AM(\frak A,\bar\mu,\gamma))
=\link_m(\AM(\frak A,\bar\mu,\gamma))=\max(c(\frak A),\tau(\frak A))$.}

(b) $d(\AM(\frak A,\bar\mu,\gamma))
=\pi(\AM(\frak A,\bar\mu,\gamma))
=\max(\cff[c(\frak A)]^{\le\omega},\pi(\frak A))$.

%what about \tau  \wdistr
% I think \tau(\AM)  related to  \link\frak A  but do we have  = ?

\proof{ Set $P=\{a:a\in\frak A$, $\bar\mu a<\gamma\}$, so that
$\AM(\frak A,\bar\mu,\gamma)=\RO^{\uparrow}(P)$.

\medskip

{\bf (a)} Set $\kappa_0=\max(c(\frak A),\tau(\frak A))$,
$\kappa_1=\link_m(\RO^{\uparrow}(P))=\link_m^{\uparrow}(P)$
and $\kappa_2=c(\RO^{\uparrow}(P))=c^{\uparrow}(P)$
(514N).

\medskip

\quad{\bf (i)} The topological density of $\frak A^f$ for its
measure metric is $\kappa_0$ (521Eb), so $P$ has a metrically dense
subset $D$ of size at most $\kappa_0$.   For $d\in D$, set

\Centerline{$U_d
=\{a:a\in P$, $\bar\mu(a\Bsetminus d)<\Bover1m(\gamma-\bar\mu d)\}$.}

\noindent Then $U_d$ is upwards-$m$-linked in $P$.   Also, if $a\in P$,
there is a $d\in D$ such that
$\bar\mu(a\Bsymmdiff d)<\bover1{m+1}(\gamma-\bar\mu a)$, and now
$a\in U_d$.   So $P$ is $\kappa_0$-$m$-linked upwards and
$\kappa_1\le\kappa_0$.

\medskip

\quad{\bf (ii)} By 511Hb or 511Ia, $\kappa_2\le\kappa_1$.

\medskip

\quad{\bf (iii)} We need to check that $\kappa_2$ is infinite.   \Prf\
Take $a\in\frak A$ such that $\bar\mu a=\gamma$.   For any $n\ge 1$, we
can find disjoint $a_0,\ldots,a_n\Bsubseteq a$ all of measure
$\bover1{n+1}\gamma$;  now $\langle a\Bsetminus a_i\rangle_{i\le n}$ is
an up-antichain in $P$.   So $\kappa_2=c^{\uparrow}(P)\ge n+1$;  and
this is true for every $n$.\ \Qed

Now if $(\frak A,\bar\mu)$ is totally finite, then
$c(\frak A)=\omega\le\kappa_2$.
Otherwise, there is a partition $D$ of unity in $\frak A$ such that
$\bar\mu d=\bover12\gamma$ for every $d\in D$;  now $D$ is an
up-antichain in $P$ and $\kappa_2\ge\#(D)=c(\frak A)$.   So we see that
in all cases $\kappa_2\ge c(\frak A)$.

\medskip

\quad{\bf (iv)} If $e\in\frak A^f$ and the principal ideal $\frak A_e$
is homogeneous, then $\tau(\frak A_e)\le\kappa_2$.   \Prf\Quer\
Otherwise,
set $\alpha=\bar\mu e$, $\kappa=\tau(\frak A_e)$.   Because
$\bar\mu 1>\gamma$, there is a $d\Bsubseteq 1\Bsetminus e$ such that
$\gamma<\bar\mu(e\Bcup d)<\gamma+\bar\mu e$, that is,
$0<\gamma-\bar\mu d<\alpha$.   Set
$\beta=\sqrt{1-\Bover{\gamma-\bar\mu d}{\alpha}}$.   Because
$\frak A_e$ is isomorphic, up to a scalar multiple of the measure, to
the measure algebra of the usual measure on $[0,1]^{\kappa}$, there is a
family $\ofamily{\xi}{\kappa}{c_{\xi}}$ in $\frak A_e$ such that

\Centerline{$\bar\mu c_{\xi}=\beta\alpha$,
\quad$\bar\mu(c_{\xi}\Bcap c_{\eta})=\beta^2\alpha$}

\noindent whenever $\xi$, $\eta<\kappa$ are distinct.   Set
$b_{\xi}=d\Bcup(e\Bsetminus c_{\xi})$ for $\xi<\kappa$.   Then

\Centerline{$\bar\mu(b_{\xi}\Bcup b_{\eta})
=\bar\mu d+\alpha-\beta^2\alpha=\gamma$,}

\Centerline{$\bar\mu b_{\xi}=\bar\mu d+\alpha-\beta\alpha<\gamma$}

\noindent for all distinct $\xi$, $\eta<\kappa$.   So
$\ofamily{\xi}{\kappa}{b_{\xi}}$ is an up-antichain in $P$ and
witnesses that $\kappa_2\ge\kappa$.\ \Bang\Qed

\medskip

\quad{\bf (v)} Let $E$ be a partition of unity in $\frak A$ such that
$0<\bar\mu e<\infty$ and $\frak A_e$ is homogeneous for every $e\in E$.
For $e\in E$, let $A_e\Bsubseteq\frak A_e$ be a set of size at most
$\kappa_2$ which $\tau$-generates $\frak A_e$.   Then
$A=\bigcup_{e\in E}A_e\,\,\tau$-generates $\frak A$, so that

\Centerline{$\kappa_0
=\max(c(\frak A),\tau(\frak A))\le\max(c(\frak A),\#(A))
\le\max(c(\frak A),\kappa_2)=\kappa_2$,}

\noindent and the three cardinals must be equal.

\medskip

{\bf (b)} Set $\kappa_3=\max(\pi(\frak A),\cff[c(\frak A)]^{\le\omega})$,
$\kappa_4=\pi(\AM(\frak A,\bar\mu,\gamma))$ and
$\kappa_5=d(\AM(\frak A,\bar\mu,\gamma))$.

\medskip

\quad{\bf (i)} By 323Mc, $\frak A^f$ is complete in its measure metric.
By 323Ma, $\Bcup:\frak A^f\times\frak A^f\to\frak A^f$ is uniformly
continuous for the measure metric, and $P$ is an open subset of
$\frak A^f$, while the topological density of $\frak A^f$ is $\kappa_0$.
By 524C, $(P,\Bsubseteqshort^{\strprime},[P]^{<\omega})
\prGT(\ell^1(\kappa_0),\le,\ell^1(\kappa_0))$, where
$\Bsubseteqshort^{\strprime}$ is defined as in 512F.   It follows that

$$\eqalignno{\kappa_4
&=\pi(\RO^{\uparrow}(P))
=\cf P\cr
\displaycause{514Nb}
&=\cov(P,\Bsubseteqshort,P)
\le\max(\omega,\cov(P,\Bsubseteqshort^{\strprime},[P]^{\le\omega}))\cr
\displaycause{512Gf}
&\le\max(\omega,\cov(P,\Bsubseteqshort^{\strprime},[P]^{<\omega}))\cr
\displaycause{512Gb}
&\le\max(\omega,\cf\ell^1(\kappa_0))\cr
\displaycause{512Da}
&=\cf\ell^1(\kappa_0)
=\cf\Cal N_{\kappa_0}\cr
\displaycause{where $\Cal N_{\kappa_0}$ is the null ideal of the
usual measure on $\{0,1\}^{\kappa_0}$, as in 524I}
&=\max(\cf\Cal N,\cff[\kappa_0]^{\le\omega})\cr
\displaycause{523N}
&=\max(\cf\Cal N,\cff[\tau(\frak A)]^{\le\omega},
  \cff[c(\frak A)]^{\le\omega})\cr
&=\max(\cf\Cal N,\cff[\tau(\frak A)]^{\le\omega},c(\frak A),
  \cff[c(\frak A)]^{\le\omega})
=\max(\pi(\frak A),\cff[c(\frak A)]^{\le\omega})\cr
\displaycause{524Mc}
&=\kappa_3.\cr}$$

\medskip

\quad{\bf (ii)} By 514Nd, $d^{\uparrow}(P)=\kappa_5$.   Let
$\ofamily{\xi}{\kappa_5}{B_{\xi}}$ be a family of
upwards-centered sets covering $P$.   For each $\xi$,
$b_{\xi}=\sup B_{\xi}$ is defined in $\frak A$ (counting $\sup\emptyset$
as $0$ if necessary), and

\Centerline{$\bar\mu b_{\xi}=\sup_{I\in[B_{\xi}]^{<\omega}}\bar\mu(\sup I)
\le\gamma$.}

\noindent
Set $D=\{b_{\xi}\Bsetminus b_{\eta}:\xi$, $\eta<\kappa_5\}$.   Then $D$
is order-dense in $\frak A$.   \Prf\ If $a\in\frak A\setminus\{0\}$,
take $a'\Bsubseteq a$ such that
$0<\bar\mu a'<\gamma$.   Then $a'\in P$, so there is some $\xi<\kappa_5$
such that $a'\in B_{\xi}$ and
$a'\Bsubseteq b_{\xi}$.   Next, let $c\Bsubseteq 1\Bsetminus b_{\xi}$ be
such that

\Centerline{$\gamma-\bar\mu b_{\xi}<\bar\mu c
<\gamma-\bar\mu b_{\xi}+\bar\mu a'$.}

\noindent Then $c\Bcup(b_{\xi}\Bsetminus a')\in P$, so there is an
$\eta<\kappa_5$ such that
$c\Bcup(b_{\xi}\Bsetminus a')\Bsubseteq b_{\eta}$.   Now
$d=b_{\xi}\Bsetminus b_{\eta}\Bsubseteq a'$;  as
$\bar\mu(b_{\xi}\Bcup c)>\gamma\ge\bar\mu b_{\eta}$,
$b_{\xi}\notBsubseteq b_{\eta}$ and $d\ne 0$.   Of course $d\in D$ and
$d\Bsubseteq a$;  as $a$ is arbitrary, $D$ is order-dense.\ \Qed

Accordingly $\pi(\frak A)\le\#(D)\le\kappa_5$.   At the same time,
$\cff[c(\frak A)]^{\le\omega}\le\kappa_5$.  \Prf\ There is a disjoint set
$E\subseteq\frak A\setminus\{0\}$ of size $c(\frak A)$ (332F).   For
each $\xi<\kappa_5$, let $I_{\xi}$ be the countable set
$\{e:e\in E$, $e\Bcap b_{\xi}\ne 0\}$.
If $J\subseteq E$ is countable, let $\family{e}{J}{\epsilon_e}$ be a
strictly positive family of real numbers with sum less than $\gamma$.
For each $e\in J$ let $a_e\Bsubseteq e$ be such that
$0<\bar\mu a_e\le\epsilon_e$, and set $a=\sup_{e\in J}a_e$.   Then $a\in
P$ so there is a $\xi<\kappa_5$ such that $a\Bsubseteq b_{\xi}$ and
$J\subseteq I_{\xi}$.   As $J$ is arbitrary, $\{I_{\xi}:\xi<\kappa_5\}$
is cofinal with $[E]^{\le\omega}$, and

\Centerline{$\cff[c(\frak A)]^{\le\omega}
=\cff[E]^{\le\omega}\le\kappa_5$.\ \Qed}

Putting these together, we see that $\kappa_3\le\kappa_5$.

\medskip

\quad{\bf (iii)} By 514Da, $\kappa_5\le\kappa_4$, so the three cardinals
are equal.
}%end of proof of 528P

\leader{528Q}{Proposition} Let $\Cal S^{\infty}$ be the
$\Bbb N$-localization poset.

(a)
$\pi(\RO^{\uparrow}(\Cal S^{\infty}))=\cf\Cal S^{\infty}=\frak c$.

(b) For every $m\ge 2$,

\Centerline{$c(\RO^{\uparrow}(\Cal S^{\infty}))
=c^{\uparrow}(\Cal S^{\infty})
=\link_m(\RO^{\uparrow}(\Cal S^{\infty}))
=\link_m^{\uparrow}(\Cal S^{\infty})=\omega$.}

(c) $d(\RO^{\uparrow}(\Cal S^{\infty}))
=d^{\uparrow}(\Cal S^{\infty})=\cf\Cal N$.

%what about  \wdistr, \tau

\proof{{\bf (a)} If $p$, $q\in\Cal S^{\infty}$ and $p\not\subseteq q$,
take $(n,i)\in p\setminus q$;   then there is a $q'\in\Cal S^{\infty}$
such that $q'\supseteq q$, $\#(q'[\{n\}])=2^n$ and $(n,i)\notin q'$, in
which case $p$ and $q'$ are incompatible upwards in $\Cal S^{\infty}$.
So $\Cal S^{\infty}$ is separative upwards and 514Nb tells us that
$\pi(\RO^{\uparrow}(\Cal S^{\infty}))=\cf\Cal S^{\infty}$.

Next, there is an almost-disjoint family
$\ofamily{\xi}{\frakc}{h_{\xi}}$ in $\NN$ (5A1Mc).   Identifying each
$h_{\xi}$ with its graph, we can regard them as members of
$\Cal S^{\infty}$;  and any member of $\Cal S^{\infty}$ includes only
finitely many of them.   So $\cf\Cal S^{\infty}\ge\frak c$.   On the
other hand, of course,
$\cf\Cal S^{\infty}\le\#(\Cal S^{\infty})=\frak c$.   So
$\pi(\RO^{\uparrow}(\Cal S^{\infty}))=\cf\Cal S^{\infty}=\frak c$.

\medskip

{\bf (b)} If $m\ge 2$, let $Q$ be the countable set of pairs $(I,r)$
where $r\in\Bbb N$ and
$I\in[\Bbb N\times\Bbb N]^{<\omega}$, and for $(I,r)\in Q$ set

\Centerline{$A_{Ir}
=\{p:p\in\Cal S^{\infty}$, $p\cap(r\times\Bbb N)=I$,
$\|p\|\le\Bover{2^r}{m}\}$.}

\noindent Then $\bigcup_{i<m}p_i\in\Cal S^{\infty}$ for any family
$\ofamily{i}{m}{p_i}$ in $A_{Ir}$, that is, $A_{Ir}$ is
upwards-$m$-linked in $\Cal S^{\infty}$.   Also
$\bigcup_{(I,r)\in Q}A_{Ir}=\Cal S^{\infty}$, so
$\link_m^{\uparrow}(\Cal S^{\infty})\le\omega$.
Of course $c^{\uparrow}(\Cal S^{\infty})$ is infinite, and since
$c^{\uparrow}(\Cal S^{\infty})\le\link_m^{\uparrow}(\Cal S^{\infty})$
(511Hb again), both must be $\omega$.   Now 514N tells us that

\Centerline{$c(\RO^{\uparrow}(\Cal S^{\infty}))
=\link_m(\RO^{\uparrow}(\Cal S^{\infty}))=\omega$.}

\medskip

{\bf (c)} Consider the localization relation $(\NN,\subseteq^*,\Cal S)$
of 522K.   We know from 522M and 512Da that

\Centerline{$\cov(\NN,\subseteq^*,\Cal S)
=\cov(\Cal N,\subseteq,\Cal N)=\cf\Cal N$.}

\medskip

\quad{\bf (i)} Let $\Cal A\subseteq\Cal S$ be a set of cardinality
$\cf\Cal N$ such that for every $f\in\NN$ there is an $S\in\Cal A$ such
that $f\subseteq^*S$.   Let $\Cal A^*$ be

\Centerline{$\{S:S\in\Cal S$, $S\setminus\bigcup\Cal A'$ is finite for
some finite $\Cal A'\subseteq\Cal A\}$;}

\noindent then every member of
$\Cal S^{\infty}$ is included in some member of $\Cal A^*$.   But if
$S\in\Cal A^*$ then $\{p:p\in\Cal S^{\infty}$, $p\subseteq S\}$ is
upwards-directed.   So

\Centerline{$\duparrow(\Cal S^{\infty})\le\#(\Cal A^*)\le\cf\Cal N$.}

\medskip

\quad{\bf (ii)} Now let $\Cal Q$ be a family of upwards-centered subsets
of $\Cal S^{\infty}$ covering $\Cal S^{\infty}$.   For each
$Q\in\Cal Q$, $S_Q=\bigcup Q$ belongs to $\Cal S$.   Also every
$f\in\NN$ belongs to $\Cal S^{\infty}$ so is covered by some $S_Q$.   So
$S_Q$ witnesses that
$\cf\Cal N=\cov(\NN,\subseteq^*,\Cal S)\le\#(\Cal Q)$;  as $\Cal Q$ is
arbitrary, $\cf\Cal N\le\duparrow(\Cal S^{\infty})$.

\medskip

\quad{\bf (iii)} 514Nd tells us that

\Centerline{$d(\RO^{\uparrow}(\Cal S^{\infty}))
=\duparrow(\Cal S^{\infty})$,}

\noindent so we have equality throughout.
}%end of proof of 528Q

\leader{528R}{Theorem} Let $\kappa$ be any cardinal, and
$\Cal S^{\infty}_{\kappa}$ the $\kappa$-localization poset.
Then $\RO^{\uparrow}(\Cal S^{\infty}_{\kappa})$ has countable Maharam
type.

\proof{{\bf (a)} If $\kappa$ is finite then
$\cf\Cal S^{\infty}_{\kappa}$ is finite
and the result is trivial.   So let us suppose from now on that $\kappa$ is
infinite.

\medskip

{\bf (b)} $\Cal S^{\infty}_{\kappa}$ is separative upwards.
\Prf\ If $p$, $q\in\Cal S^{\infty}_{\kappa}$ and $p\not\subseteq q$, take
$(n,\xi)\in p\setminus q$.   Let $J\subseteq\kappa\setminus p[\{n\}]$ be a set
of size $2^n-\#(q[\{n\}]$, and set $q'=q\cup(\{n\}\times J)$;
then $q\subseteq q'\in\Cal S^{\infty}$ and $p$, $q'$ are incompatible upwards in
$\Cal S^{\infty}_{\kappa}$.\ \Qed

Accordingly $\coint{p,\infty}\in\RO^{\uparrow}(\Cal S^{\infty}_{\kappa})$
for every $p\in\Cal S^{\infty}_{\kappa}$ (514Me).

\medskip

{\bf (c)} For $n\in\Bbb N$, $m<2^n$ and $\xi<\kappa$, set

\Centerline{$G_{mn\xi}
=\sup\{\coint{p,\infty}:p\in\Cal S^{\infty}_{\kappa},\,
\#(p[\{n\}])=2^n,\,(n,\xi)\in p\text{ and }\#(p[\{n\}]\cap\xi)=m\}$,}

\noindent the supremum being taken in
$\RO^{\uparrow}(\Cal S^{\infty}_{\kappa})$.

\medskip

{\bf (d)} If $n\in\Bbb N$, $m<2^n$ and $\xi<\eta<\kappa$ then
$G_{mn\xi}\cap G_{mn\eta}=\emptyset$.   \Prf\ If
$p$, $q\in\Cal S^{\infty}_{\kappa}$,
$\#(p[\{n\}])=\#(q[\{n\}])=2^n$, $(n,\xi)\in p$, $(n,\eta)\in q$ and
$\#(p[\{n\}]\cap\xi)=\#(q[\{n\}])\cap\eta)=m$ then $p[\{n\}]\ne q[\{n\}]$,
$\#(p[\{n\}]\cup q[\{n\}])>2^n$ and $\coint{p,\infty}\cap\coint{q,\infty}$ is
empty.\ \Qed

\medskip

{\bf (e)} If $\xi<\kappa$ then
$\bigcup\{G_{mn\xi}:n\in\Bbb N$, $m<2^n\}$ is dense in
$\Cal S^{\infty}_{\kappa}$.   \Prf\ If $p\in\Cal S^{\infty}_{\kappa}$, take
$n\in\Bbb N$ such that $\#(p[\{n\}])<2^n$;  then there is a
$q\in S^{\infty}_{\kappa}$ such that $p\subseteq q$, $\xi\in q[\{n\}]$ and
$\#(q[\{n\}])=2^n$.   Set $m=\#(q[\{n\}]\cap\xi)$;  then
$\coint{q,\infty}\subseteq\coint{p,\infty}\cap G_{mn\xi}$.\ \Qed

Thus $\sup\{G_{mn\xi}:n\in\Bbb N$, $m<2^n\}=1$ in
$\RO^{\uparrow}(\Cal S^{\infty})$ whenever $\xi<\kappa$.

\medskip

{\bf (f)}
Let $\frak G$ be the order-closed subalgebra of
$\RO^{\uparrow}(\Cal S^{\infty}_{\kappa})$ generated by
$\{G_{mn\xi}:n\in\Bbb N$, $m<2^n$, $\xi<\kappa\}$.
For $n\in\Bbb N$ and $\xi<\kappa$ set
$H_{n\xi}=\coint{\{(n,\xi)\},\infty}$;
then $H_{n\xi}=\sup_{m<2^n}G_{mn\xi}$.
\Prf\ Certainly $G_{mn\xi}\subseteq H_{n\xi}$ whenever $m<2^n$.
If $\{(n,\xi)\}\subseteq p\in\Cal S^{\infty}_{\kappa}$, let
$q\in\Cal S^{\infty}_{\kappa}$ be such that $p\subseteq q$ and
$\#(q[\{n\}])=2^n$;  set $m=\#(q[\{n\}]\cap\xi)$;  then
$\coint{q,\infty}\subseteq H_{n\xi}\cap G_{mn\xi}$.   Thus
$\bigcup_{m<2^n}G_{mn\xi}$ is dense in $H_{n\xi}$ and
$H_{n\xi}=\sup_{m<2^n}G_{mn\xi}\in\frak G$.\ \QeD\  Consequently
$H_{n\xi}\in\frak G$.

\medskip

{\bf (g)} If $p\in\Cal S^{\infty}_{\kappa}$ then
$\coint{p,\infty}=\inf_{(n,\xi)\in p}H_{n\xi}$ belongs to $\frak G$,
by 514Me.
So $\frak G$ includes an order-dense subset of
$\RO^{\uparrow}(\Cal S^{\infty}_{\kappa})$ and must be the whole of
$\RO^{\uparrow}(\Cal S^{\infty}_{\kappa})$;  that is,
$\RO^{\uparrow}(\Cal S^{\infty}_{\kappa})$ is $\tau$-generated by
$\{G_{mn\xi}:n\in\Bbb N$, $m<2^n$, $\xi<\kappa\}$.
With (iv) and (v), we see
that the conditions of 514F are satisified with $J=\kappa$ and
$I=\{(m,n):n\in\Bbb N$, $m<2^n\}$, so that

\Centerline{$\tau(\RO^{\uparrow}(\Cal S^{\infty}_{\kappa}))
\le\max(\omega,\#(I))=\omega$.}
}%end of proof of 528R

\leader{528S}{}\cmmnt{ The calculation of Maharam types of
amoeba algebras seems to be a good deal harder.
However it leads through an investigation of the
structure of measure algebras, which is one of the things this book is
about, so I take the space to give one of the main theorems.   It depends
on a special property of the standard generating families in algebras
$\frak B_{\kappa}$.

\medskip

\noindent}{\bf Definition}\dvAnew{2011}
Let $(\frak A,\bar\mu)$ be a measure algebra.
I will say that a {\bf well-spread basis} for $\frak A$ is a
non-decreasing sequence $\sequencen{D_n}$ of
subsets of $\frak A$ such that

\inset{(i) setting $D=\bigcup_{n\in\Bbb N}D_n$,
$\#(D)\le\max(\omega,c(\frak A),\tau(\frak A))$;

(ii) if $a\in\frak A$, $\gamma\in\Bbb R$ and $\bar\mu a<\gamma$, there is a set
$D\subseteq\bigcup_{n\in\Bbb N}D_n$ such that
$a\Bsubseteq\sup D$ and $\bar\mu(\sup D)<\gamma$;

(iii) if $n\in\Bbb N$ and $\sequence{i}{d_i}$ is a sequence in $D_n$
such that $\bar\mu(\sup_{i\in\Bbb N}d_i)<\infty$, there
is an infinite set $J\subseteq\Bbb N$ such that $d=\sup_{i\in J}d_i$
belongs to $D_n$;

(iv) whenever $n\in\Bbb N$, $a\in\frak A$ and
$\bar\mu a\le\gamma'<\gamma<\bar\mu 1$, there is a $b\in\frak A$ such that
$a\Bsubseteq b$ and $\gamma'\le\bar\mu b<\gamma$ and
$\bar\mu(b\Bcup d)\ge\gamma$ whenever $d\in D_n$ and
$d\notBsubseteq a$.}

\leader{528T}{Lemma}\dvAnew{2011}
(a) Let $\kappa$ be an infinite cardinal, and
$\ofamily{\xi}{\kappa}{e_{\xi}}$ the standard generating family in
$\frak B_{\kappa}$.
For $n\in\Bbb N$ let $C_n$ be the set of elements of
$\frak B_{\kappa}$ expressible as
$\inf_{\xi\in I}e_{\xi}\Bcap\inf_{\xi\in J}(1\Bsetminus e_{\xi})$
where $I$,
$J\subseteq\kappa$ are disjoint and $\#(I\cup J)\le n$.   Then
$\sequencen{C_n}$ is a well-spread basis for
$(\frak B_{\kappa},\bar\nu_{\kappa})$.   Moreover,

\doubleinset{(*) for each $n\ge 1$, there is a set $C'_n\subseteq C_n$,
with cardinal $\kappa$,
such that $\bar\nu_{\kappa} c=2^{-n}$ for every $c\in C'_n$, and whenever
$a\in\frak B_{\kappa}\setminus\{1\}$ and $I\subseteq C'_n$ is infinite,
there is a $c\in I$ such that $c'\notBsubseteq a\Bcup c$ whenever
$c\Bsubset c'\in C_n$.}

(b) Let $(\frak A,\bar\mu)$ be a measure algebra and $e\in\frak A$.
If $\sequencen{C_n}$ is a well-spread basis for
$(\frak A_e,\bar\mu\restrp\frak A_e)$ and
$\sequencen{D_n}$ is a well-spread basis for
$(\frak A_{1\Bsetminus e},\bar\mu\restrp\frak A_{1\Bsetminus e})$, then
$\sequencen{C_n\cup D_n}$
is a well-spread basis for $(\frak A,\bar\mu)$.

\proof{{\bf (a)(i)} $\sequencen{C_n}$ satisfies (i) of
Definition 528S just because $\tau(\frak B_{\kappa})=\#(C_n)=\kappa$ for
$n\ge 1$, while $C_0=\{1\}$.

\medskip

\quad{\bf (ii)} For $J\subseteq\kappa$, let $\frak C_J$ be the
order-closed subalgebra of $\frak B_{\kappa}$ generated by
$\{e_{\xi}:\xi\in J\}$;  recall that for every $a\in\frak B_{\kappa}$ there
is a countable set $\supp a\subseteq\kappa$ such that
$a\in\frak C_J$ iff $J\supseteq\supp a$ (254Rd/325Mb).
Of course $\#(\supp c)\le n$ whenever $n\in\Bbb N$ and $c\in C_n$.

Suppose that $a\in\frak B_{\kappa}$ and $\gamma>\bar\nu_{\kappa}a$.
Then for each $k\in\Bbb N$ we can find an $a_k\in\frak B_{\kappa}$, with
finite support, such that
$\bar\nu_{\kappa}(a\Bsymmdiff a_k)\le 2^{-k-2}(\gamma-\bar\nu_{\kappa}a)$
(254Fe/325Jc).   Set $b=\sup_{k\in\Bbb N}a_k$;  then

\Centerline{$\bar\nu_{\kappa}b
\le\bar\nu_{\kappa}a+\sum_{k=0}^{\infty}\bar\nu_{\kappa}(a_k\Bsetminus a)
<\gamma$,}

\Centerline{$\bar\nu_{\kappa}(a\Bsetminus b)
\le\inf_{k\in\Bbb N}\bar\nu_{\kappa}(a\Bsetminus a_k)=0$,}

\noindent so $a\Bsubseteq b$.
If $k\in\Bbb N$ and $\#(\supp a_k)=n_k$, then
$a_k=\sup\{c:c\in C_{n_k}$, $c\Bsubseteq a_k\}$, so
$b=\sup\{c:c\in\bigcup_{n\in\Bbb N}C_n$, $c\Bsubseteq b\}$.   Thus
528S(ii) is satisfied.

\medskip

\quad{\bf (iii)} If $n\in\Bbb N$ and $\sequence{i}{c_i}$ is a sequence
in $C_n$, there is an infinite $I\subseteq\Bbb N$ such
that $\family{i}{I}{\supp c_i}$ is a $\Delta$-system with root $K$ say
(5A1Ic).
For $i\in I$, express $c_i$ as $c'_i\Bcap c''_i$ where $c'_i\in\frak C_K$
and $c''_i\in\frak C_{(\supp c_i)\setminus K}$;  as $\frak C_K$ is finite,
there is
a $c$ such that $J=\{i:c'_i=c\}$ is infinite.   Now $c\in C_n$, and if
$m\in\Bbb N$ then

\Centerline{$\sup_{i\in J\setminus m}c_i
=c\Bcap\sup_{i\in J\setminus m}c''_i
=c$}

\noindent because $\family{i}{J\setminus m}{c''_i}$ is a stochastically
independent family of elements of $\frak B_{\kappa}$
all of measure at least $2^{-n}$, so has supremum $1$.
In particular, 528S(iii) is satisfied.

\medskip

\quad{\bf (iv)} Suppose that $n\in\Bbb N$ and $a\in\frak B_{\kappa}$.
Then there
is a $\delta>0$ such that $\bar\nu_{\kappa}(c\Bsetminus a)\ge\delta$ whenever
$c\in C_n$ and $c\notBsubseteq a$.  \Prf\Quer\
Otherwise, there is a sequence $\sequence{i}{c_i}$
in $C_n$ such that $0<\bar\nu_{\kappa}(c_i\Bsetminus a)\le 2^{-i}$ for every
$i\in\Bbb N$.   By (iii) just above, there is an infinite set
$J\subseteq\Bbb N$ such that
$c_j\Bsubseteq\sup_{i\in J\setminus m}c_i$ for every $j\in J$.
Set $j_0=\min J$, and let $m$ be such that
$2^{-m+1}<\bar\nu_{\kappa}(c_{j_0}\Bsetminus a)$;  then

\Centerline{$2^{-m+1}<\bar\nu_{\kappa}(\sup_{j\in J\setminus m}c_j\Bsetminus a)
\le\sum_{j=m}^{\infty}\bar\nu_{\kappa}(c_j\Bsetminus a)\le 2^{-m+1}$,}

\noindent which is absurd.\ \Bang\Qed

\medskip

\quad{\bf (v)} Suppose that $n\in\Bbb N$, $a\in\frak B_{\kappa}$ and
$\bar\nu_{\kappa} a\le\gamma'<\gamma<1$.   Pick
$\delta>0$, $r>n$, $k^*\in\Bbb N$, $\epsilon>0$ and $a'\in\frak B_{\kappa}$
such that

\Centerline{$\bar\nu_{\kappa}(c\Bsetminus a)\ge\delta$
whenever $c\in C_n$ and $c\notBsubseteq a$,}

\Centerline{$2^{-r}\le\gamma-\gamma'$,
\quad$(2^{-n}-2^{-r})^n\delta\ge 2^{-r+2}$,}

\Centerline{$(1-2^{-r})^{k^*}\le 1-\gamma$,}

\Centerline{$\epsilon\le\Bover12\delta$,
\quad$\epsilon\le 2^{-r}(1-2^{-r})^{k^*}$,}

\Centerline{$\supp a'$ is finite,
\quad$\bar\nu_{\kappa}(a\Bsymmdiff a')\le\epsilon$.}

Let $\sequence{i}{K_i}$ be a
disjoint sequence in $[\kappa\setminus\supp a']^r$, and set
$c_i=\inf_{\xi\in K_i}e_{\xi}$ for each $i\in\Bbb N$.
Then $\sup_{i\in\Bbb N}c_i=1$, so there is a first $k$ such that
$\bar\nu_{\kappa}(a\Bcup\sup_{i\le k}c_i)\ge\gamma$;  set $b_1=\sup_{i<k}c_i$
and $b=a\Bcup b_1$.
Surely $a\Bsubseteq b$ and $\bar\nu_{\kappa} b<\gamma$;  also

\Centerline{$(1-2^{-r})^{k^*}\le 1-\gamma\le 1-\bar\nu_{\kappa} b_1=(1-2^{-r})^k$}

\noindent so $k\le k^*$.   Moreover,

\Centerline{$\gamma-\bar\nu_{\kappa} b
\le\bar\nu_{\kappa}(b\Bcup c_k)-\bar\nu_{\kappa} b
\le\bar\nu_{\kappa}(c_k\Bsetminus b_1)=2^{-r}(1-2^{-r})^k
\le 2^{-r}\le\gamma-\gamma'$,}

\noindent so that, in particular, $\bar\nu_{\kappa} b\ge\gamma'$.

If $c\in C_n$ and $c\notBsubseteq a$ then
$\bar\nu_{\kappa}(c\Bsetminus a)\ge\delta$ so
$\bar\nu_{\kappa}(c\Bsetminus a')\ge\delta-\epsilon$.
Express $c$ as $\inf_{i\le k}c'_i$
where $\supp c'_i\subseteq K_i$ for $i<k$ and
$\supp c'_k\subseteq\kappa\setminus\bigcup_{i<k}K_i$.   Set
$J=\{i:i<k$, $c'_i\ne 1\}$;  then $\#(J)\le n$.   Now

$$\eqalignno{\bar\nu_{\kappa}(c\Bsetminus(a'\Bcup b_1))
&=\bar\nu_{\kappa}((c'_k\Bsetminus a')
     \Bcap\inf_{i\in J}(c'_i\Bsetminus c_i)
     \Bcap\inf_{i\in k\setminus J}(1\Bsetminus c_i))\cr
&=\bar\nu_{\kappa}(c'_k\Bsetminus a')
   \cdot\prod_{i\in J}\bar\nu_{\kappa}(c'_i\Bsetminus c_i)
   \cdot\prod_{i\in k\setminus J}\bar\nu_{\kappa}(1\Bsetminus c_i)\cr
\displaycause{because $\supp(c'_k\Bsetminus a')
\subseteq\supp c'_k\cup\supp a'
\subseteq\kappa\setminus\bigcup_{i<k}K_i$, so we are taking an
infimum of stochastically independent elements of $\frak B_{\kappa}$}
&\ge(\delta-\epsilon)\cdot\prod_{i\in J}(2^{-n}-2^{-r})
   \cdot\prod_{i\in k\setminus J}(1-2^{-r})\cr
\displaycause{of course every $c'_i$ belongs to $C_n$}
&\ge\Bover12(2^{-n}-2^{-r})^n(1-2^{-r})^k\delta
\ge 2^{-r+1}(1-2^{-r})^k\cr
&\ge 2^{-r}(1-2^{-r})^k+2^{-r}(1-2^{-r})^{k^*}
\ge\gamma-\bar\nu_{\kappa} b+\epsilon,\cr}$$

\noindent and

\Centerline{$\bar\nu_{\kappa}(c\Bsetminus b)\ge\gamma-\bar\nu_{\kappa} b$,}

\noindent so $\bar\nu_{\kappa}(b\Bcup c)\ge\gamma$.

As $n$, $a$, $\gamma'$ and $\gamma$ are arbitrary, $\sequencen{C_n}$
satisfies 528S(iv) and is a well-spread basis.

\medskip

\quad{\bf (vi)} As for (*), given $n\ge 1$,
take a disjoint family $\ofamily{\xi}{\kappa}{K_{\xi}}$
in $[\kappa]^n$, and set $c_{\xi}=\inf_{\eta\in K_{\xi}}e_{\eta}$ for
$\xi<\kappa$, $C'_n=\{c_{\xi}:\xi<\kappa\}$.   If $I\subseteq\kappa$ is
infinite and $\bar\nu_{\kappa} a<1$, take $\delta>0$ such that
$\delta<2^{-n}(1-\bar\nu_{\kappa} a-\delta)$, and
$a'\in\frak B_{\kappa}$ such that
$\supp a'$ is finite and $\bar\nu_{\kappa}(a\Bsymmdiff a')\le\delta$.
Then there is a
$\xi\in I$ such that $K_{\xi}\cap\supp a_{\xi}=\emptyset$.
\Quer\
If $c\in C_n$ is such that $c_{\xi}\Bsubset c\Bsubseteq a\Bcup c_{\xi}$,
there must be a $d\in\frak B_{\kappa}$, with support $K_{\xi}$, included in
$c\Bsetminus c_{\xi}$.   But now $d\Bsubseteq a$ and
$\supp d\Bcap\supp a'=\emptyset$, so

\Centerline{$2^{-n}(1-\bar\nu_{\kappa} a-\delta)
\le 2^{-n}(1-\bar\nu_{\kappa} a')
=\bar\nu_{\kappa}(d\Bsetminus a')
\le\delta+\bar\nu_{\kappa}(d\Bsetminus a)=\delta$.  \Bang}

\woddheader{528T}{4}{2}{2}{60pt}

{\bf (b)(i)} We have

$$\eqalign{\#(\bigcup_{n\in\Bbb N}C_n\cup D_n)
&\le\max(\omega,\#(\bigcup_{n\in\Bbb N}C_n),\#(\bigcup_{n\in\Bbb N}D_n))
  \cr
&\le\max(\omega,c(\frak A_e),c(\frak A_{1\Bsetminus e}),
  \tau(\frak A_e),\tau(\frak A_{1\Bsetminus e}))
=\max(\omega,c(\frak A),\tau(\frak A))\cr}$$

\noindent by 514E.   %514Ed 514Ef

\medskip

\quad{\bf (ii)} Suppose that $a\in\frak A$ and $\bar\mu a<\gamma$.
Then there are $\gamma_1$, $\gamma_2$ such that
$\bar\mu(a\Bcap e)<\gamma_1$, $\bar\mu(a\Bsetminus e)<\gamma_2$ and
$\gamma_1+\gamma_2\le\gamma$.   Let $C\subseteq\bigcup_{n\in\Bbb N}C_n$,
$D\subseteq\bigcup_{n\in\Bbb N}D_n$ be such that
$a\Bcap e\Bsubseteq\sup C$, $a\Bsetminus e\Bsubseteq\sup D$,
$\bar\mu(\sup C)<\gamma_1$ and $\bar\mu(\sup D)<\gamma_2$.   Then
$C\cup D\subseteq\bigcup_{n\in\Bbb N}C_n\cup D_n$,
$a\Bsubseteq\sup(C\cup D)$ and $\bar\mu(\sup(C\cup D))<\gamma$.

\medskip

\quad{\bf (iii)} Suppose that $n\in\Bbb N$ and
$\sequence{i}{c_i}$ is a sequence in $C_n\cup D_n$ such that
$\bar\mu(\sup_{i\in\Bbb N}c_i)<\infty$.   Then
there is an infinite $J\subseteq\Bbb N$ such that either
$c_i\in C_n$ for every $i\in J$, or $c_i\in D_n$ for every $i\in I$.
In either case, there is an infinite $I\subseteq J$ such that
$\sup_{i\in I}c_i$ belongs to $C_n\cup D_n$.

\medskip

\quad{\bf (iv)} Thus $\sequencen{C_n\cup D_n}$ satisfies (i)-(iii) of
Definition 528S.   As for 528S(iv), suppose that
$n\in\Bbb N$, $a\in\frak A$ and
$\bar\mu a\le\gamma'<\gamma<\bar\mu 1$.   We need to find a
$b\in\frak A$ such that $a\Bsubseteq b$ and

\Centerline{$\gamma'\le\bar\mu b<\gamma
\le\bar\mu(b\Bcup c)$}

\noindent whenever $c\in C_n\cup D_n$ and $c\notBsubseteq a$.

\medskip

\qquad{\bf case 1} If $e\Bsubseteq a$, then $\bar\mu e$ is finite and

\Centerline{$\bar\mu(a\Bsetminus e)\le\gamma'-\bar\mu e
<\gamma-\bar\mu e<\bar\mu(1\Bsetminus e)$.}

\noindent So there is a $b_2\in\frak A_{1\Bsetminus e}$ such that
$a\Bsetminus e\Bsubseteq b_2$ and

\Centerline{$\gamma'-\bar\mu e\le\bar\mu b_2
<\gamma-\bar\mu e\le\bar\mu(b_2\Bcup d)$}

\noindent whenever $d\in D_n$ and $d\notBsubseteq a\Bsetminus e$;  that is,

\Centerline{$\gamma'\le\bar\mu(e\Bcup b_2)
<\gamma\le\bar\mu(e\Bcup b_2\Bcup d)$}

\noindent whenever $d\in D_n$ and $d\notBsubseteq a$.   Since
$c\Bsubseteq a$ for every $c\in C_n$, we have
$\bar\mu(e\Bcup b_2\Bcup c)\ge\gamma$ whenever $c\in C_n\cup D_n$ and
$c\notBsubseteq a$, and can take $b=e\Bcup b_2$.

\medskip

\qquad{\bf case 2} Similarly, if $a\Bsupseteq 1\Bsetminus e$, we can take
$b=(1\Bsetminus e)\Bcup b_1$ for a suitable $b_1\Bsubseteq e$.

\medskip

\qquad{\bf case 3} If neither $e$ nor $1\Bsetminus e$ is included in $a$,
we have

\Centerline{$\max(\bar\mu(a\Bcap e),\gamma-\bar\mu(1\Bsetminus e))
<\min(\bar\mu e,\gamma-\bar\mu(a\Bsetminus e))$,}

\noindent so we can find $\gamma'_1$, $\gamma_1$ such that

\Centerline{$\max(\bar\mu(a\Bcap e),\gamma-\bar\mu(1\Bsetminus e))
<\gamma'_1<\gamma_1<\min(\bar\mu e,\gamma-\bar\mu(a\Bsetminus e))$}

\noindent and $\gamma_1-\gamma'_1<\gamma-\gamma'$.
Let $b_1\in\frak A_e$ be such that $a\Bcap e\Bsubseteq b_1$ and

\Centerline{$\gamma'_1\le\bar\mu b_1<\gamma_1
\le\bar\mu(b_1\Bcup c)$}

\noindent whenever $c\in C_n$ and $c\notBsubseteq a\Bcap e$.   Set
$\gamma'_2=\gamma-\gamma_1$ and $\gamma_2=\gamma-\bar\mu b_1$, so that

\Centerline{$\bar\mu(a\Bsetminus e)<\gamma'_2<\gamma_2
\le\gamma-\gamma'_1<\bar\mu(1\Bsetminus e)$.}

\noindent Let
$b_2\in\frak A_{1\Bsetminus e}$ be such that $a\Bsetminus e\Bsubseteq b_2$
and

\Centerline{$\gamma'_2\le\bar\mu b_2<\gamma_2
\le\bar\mu(b_2\Bcup d)$}

\noindent whenever $d\in D_n$ and $d\notBsubseteq a\Bsetminus e$.

Try $b=b_1\Bcup b_2$.   Then $a\Bsubseteq b$ and
$\bar\mu b=\bar\mu b_1+\bar\mu b_2$ belongs to

\Centerline{$\coint{\bar\mu b_1+\gamma'_2,\bar\mu b_1+\gamma_2}
\subseteq\coint{\gamma'_1+\gamma-\gamma_1,\gamma}
\subseteq\coint{\gamma',\gamma}$.}

\noindent If $c\in C_n$ and $c\notBsubseteq a$, then
$c\notBsubseteq a\Bcap e$, so

\Centerline{$\bar\mu(b\Bcup c)
=\bar\mu(b_1\Bcup c)+\bar\mu b_2
\ge\gamma_1+\gamma'_2
=\gamma$;}

\noindent while if $d\in D_n$ and $d\notBsubseteq a$, then

\Centerline{$\bar\mu(b\Bcup d)
=\bar\mu b_1+\bar\mu(b_2\Bcup d)
\ge\mu b_1+\gamma_2
=\gamma$.}

\noindent So in this case also we have found a suitable $b$.
}%end of proof of \zz1F

\leader{528U}{Lemma}\dvAnew{2011}
Let $(\frak A,\bar\mu)$ be an atomless semi-finite
measure algebra and $0<\gamma<\bar\mu 1$.   Let $E$, $\epsilon$,
$\preccurlyeq$ and $\Cal F$ be such that

\inset{$E$ is a partition of unity in $\frak A$ such that $\frak A_e$ is
homogeneous and $0<\epsilon\le\bar\mu e<\infty$ for every $e\in E$;

$\preccurlyeq$ is a well-ordering of $E$ such that
$\tau(\frak A_e)\le\tau(\frak A_{e'})$ whenever $e\preccurlyeq e'$ in $E$;

$\Cal F$ is a partition of $E$ such that each member of $\Cal F$ is either
a singleton or a countable set with no $\preccurlyeq$-greatest member.}

\noindent Let $P_0$ be

\Centerline{$\{a:a\in\frak A$, $\bar\mu a<\gamma$,
$\gamma\le\bar\mu(a\Bcup e)$ whenever $\{e\}\in\Cal F\}$,}

\noindent ordered by $\Bsubseteq$.   Then $\RO^{\uparrow}(P_0)$ has
countable Maharam type.

\proof{{\bf (a)(i)}
For every $e\in E$, $(\frak A_e,\bar\mu\restrp\frak A_e)$
is a non-zero atomless homogeneous totally finite measure algebra, so is
isomorphic, up to a scalar multiple of the measure, to
$(\frak B_{\kappa},\bar\nu_{\kappa})$ for some infinite cardinal $\kappa$
(331L).   So we can copy the well-spread basis for
$(\frak B_{\kappa},\bar\nu_{\kappa})$ described in 528Ta into a
well-spread basis $\sequencen{D_{en}}$ for
$(\frak A_e,\bar\mu\restrp\frak A_e)$ such that

\inset{$\#(\bigcup_{n\in\Bbb N}D_{en})=\tau(\frak A_e)$,

$\bar\mu d\ge 2^{-n}\bar\mu e$ whenever $n\in\Bbb N$ and $d\in D_{en}$,

$D_{e0}=\{e\}$,

for each $n\ge 1$ there is a set $D'_{en}\subseteq D_{en}$, with cardinal
$\tau(\frak A_e)$, such that $\bar\mu d=2^{-n}\bar\mu e$ for every
$d\in D'_{en}$, and whenever $a\in\frak A_e\setminus\{e\}$ and
$I\subseteq D'_{en}$ is infinite, there is a $d\in I$ such that
$d'\notBsubseteq a\Bcup d$ whenever $d'\in D_{en}$ and $d'\Bsupset d$,

$(\bigcup_{n\in\Bbb N}D_{en})\setminus(\bigcup_{n\ge 1}D'_{en})$ has
cardinal $\tau(\frak A_e)$.}

\noindent (The last item is not mentioned in 528T, but is clearly
achievable by thinning the sets $D'_{en}$ appropriately, besides being
automatic if we use the construction in (a-vi) of the proof of 528T.)
Note that $\langle D'_{en}\rangle_{n\ge 1}$
is a disjoint sequence of subsets of
$\frak A_e$ for each $e$, so $\langle D'_{en}\rangle_{e\in E,n\ge 1}$
is disjoint.

\medskip

\quad{\bf (ii)} For $e\in F\in\Cal F$, set

\Centerline{$D_e
=\bigcup_{n\in\Bbb N}D_{en}\setminus\bigcup_{n\ge 1}D'_{en}$,
\quad$D^*_e
=\bigcup_{e'\in F,e'\preccurlyeq e}D_e$.}

\noindent Because $F$ is countable and
$\tau(\frak A_{e'})\le\tau(\frak A_e)$ whenever $e'\preccurlyeq e$,
$\#(D^*_e)=\tau(\frak A_e)=\#(D'_{en})$ for every $n\ge 1$.
We therefore have a partition
$\family{d}{D^*_e}{I_{ed}}$ of $\bigcup_{n\ge 1}D'_{en}$ into countably
infinite sets such that $I_{ed}\cap D'_{en}$ is infinite whenever
$d\in D^*_e$ and $n\ge 1$.

Let $\theta$ be a limit ordinal such that the set $\Omega$ of limit
ordinals less than $\theta$ has cardinal
$\#(\bigcup_{e\in E}D_e)$.   (Of course we can take $\theta$ to be either
an uncountable cardinal or the ordinal product $\omega\cdot\omega$ or $0$.)
Again because every member of
$\Cal F$ is countable, we have an enumeration
$\ofamily{\xi}{\theta}{d_{\xi}}$ of $\bigcup_{e\in E,n\in\Bbb N}D_{en}$
such that whenever $\xi\in\Omega$ then there are $F\in\Cal F$ and $e\in F$
such that

\Centerline{$d_{\xi}\in D_e$,
\quad$\{d_{\xi+i}:i\ge 1\}
=\bigcup_{e'\in F,e'\succcurlyeq e}I_{e'd_{\xi}}$.}

\noindent This will mean that whenever $\xi\in\Omega$ and $F\in\Cal F$,
$e\in F$ are such that $d_{\xi}\in\frak A_e$, then
$\{i:d_{\xi+i}\in D'_{e'n}\}$ is infinite whenever $e'\in F$,
$e\preccurlyeq e'$ and $n\in\Bbb N$.

\medskip

{\bf (b)(i)} Setting $P=\{a:a\in\frak A$, $\bar\mu a<\gamma\}$,
$P_0\in\RO^{\uparrow}(P)$.   \Prf\ Evidently $P_0$ is up-open.   If
$a\in P\setminus P_0$, that is, there is some $e$ such that
$\{e\}\in\Cal F$ and $\bar\mu(a\Bcup e)<\gamma$, set $b=a\Bcup e$;  then
$a\Bsubseteq b\in P$, while $\bar\mu(b'\Bcup e)=\bar\mu b'<\gamma$ whenever
$b'\in\coint{b,\infty}$, so $\coint{b,\infty}$ does not meet $P_0$.
Accordingly $\coint{a,\infty}\not\subseteq\overline{P}_0$ and
$a\notin\interior\overline{P}_0$.   As $a$ is arbitrary,
$P_0=\interior\overline{P}_0\in\RO^{\uparrow}(P)$.\ \Qed

It follows that $\RO^{\uparrow}(P_0)$ is the principal ideal of
$\RO^{\uparrow}(P)$ generated by $P_0$ (314R(b-ii)).
Moreover, for $a\in P_0$, $\coint{a,\infty}$ is the same
whether taken in $P$ or $P_0$, and belongs to $\RO^{\uparrow}(P)$ by
528B(b-i).

\medskip

\quad{\bf (ii)} For $a\in P_0$ and $n\in\Bbb N$, set
$A_n(a)=\{d:d\in\bigcup_{e\in E}D_{en}$, $d\Bsubseteq a\}$.   Then any
sequence in $A_n(a)$ has a subsequence with an upper bound in $A_n(a)$.
\Prf\ Set $L=\{e:e\in E$, $\bar\mu(a\Bcap e)\ge 2^{-n}\epsilon\}$;  then
$L$ is finite.   If $e\in E\setminus L$ and $d\in D_{en}$, then
$d\Bsubseteq e$ and

\Centerline{$\bar\mu d\ge 2^{-n}\bar\mu e\ge 2^{-n}\epsilon
>\bar\mu(a\Bcap e)\ge\bar\mu(a\Bcap d)$,}

\noindent so
$d\notBsubseteq a$.   Thus $A_n(a)\subseteq\bigcup_{e\in L}D_{en}$.
It follows that if $\sequence{i}{c_i}$ is any sequence in $A_n(a)$, there
is an $e\in L$ such that $J=\{i:c_i\in D_{en}\}$ is infinite.   Now there
is an infinite $I\subseteq J$ such that $c=\sup_{i\in I}c_i$ belongs to
$D_{en}$.   In this case, $c\Bsubseteq a$ so $c\in A_n(a)$ is an upper
bound of $\{c_i:i\in I\}$.\ \Qed

It follows that $A_n(a)$ has only finitely many maximal elements, and any
non-decreasing sequence in $A_n(a)$ has an upper bound in $A_n(a)$.
Consequently,
every member of $A_n(a)$ is included in a maximal element of $A_n(a)$.
\Prf\Quer\ Otherwise, we should be able to
find a strictly increasing family
$\ofamily{\xi}{\omega_1}{c_{\xi}}$ in $A_n(a)$;   but now there must be a
$\xi<\omega_1$ such that $\bar\mu c_{\xi}=\bar\mu c_{\xi+1}<\gamma$ and
$c_{\xi}=c_{\xi+1}$.\ \Bang\Qed

Set $E_n(a)=\{\xi:d_{\xi}$ is a maximal element of $A_n(a)\}$, so that
$E_n(a)$ is a finite subset of $\theta$.

\medskip

\quad{\bf (iii)} For $n\in\Bbb N$, set

\Centerline{$Q_n=\{b:b\in P_0$, $A_n(b)=A_n(b')$ whenever
$b\Bsubseteq b'\in P_0\}$.}

\noindent Then whenever $a\in P_0$ and $n\in\Bbb N$ there is a
$b\in Q_n$ such that $a\Bsubseteq b$ and $A_n(a)=A_n(b)$.
\Prf\ Let $L$ be a finite subset of $E$ including
$\{e:\bar\mu(a\Bcap e)\ge 2^{-n-1}\epsilon\}$ and such that
$\bar\mu(\sup L)>\gamma$.   Then
$\sequence{m}{\bigcup_{e\in L}D_{em}}$ is a well-spread basis for
$(\frak A_{\sup L},\bar\mu\restrp\frak A_{\sup L})$.   (Induce on $\#(L)$,
using 528Tb for the inductive step.)   Since

\Centerline{$\bar\mu(a\Bcap\sup L)
<\gamma-\bar\mu(a\Bsetminus\sup L)
<\bar\mu(\sup L)$,}

\noindent there is a a $b_0\in\frak A_{\sup L}$, including $a\Bcap\sup L$,
such that

\Centerline{$\gamma-\bar\mu(a\Bsetminus\sup L)-2^{-n-1}\epsilon
\le\bar\mu b_0
<\gamma-\bar\mu(a\Bsetminus\sup L)
\le\bar\mu(b_0\Bcup d)$}

\noindent whenever $d\in\bigcup_{e\in L}D_{en}$ and $d\notBsubseteq a$.
Set $b=b_0\Bcup a$.   Then
$\bar\mu b=\bar\mu b_0+\bar\mu(a\Bsetminus\sup L)<\gamma$, so
$b\in P_0$.   If $b\Bsubseteq b'\in P_0$ and
$d\in\bigcup_{e\in E}D_{en}\setminus A_n(a)$, then either $e\in L$ and

\Centerline{$\bar\mu(b'\Bcup d)
\ge\bar\mu(b\Bcup d)+\bar\mu(a\Bsetminus\sup L)
\ge\gamma>\bar\mu b'$,}

\noindent or $e\notin L$,

\Centerline{$\bar\mu(d\Bsetminus a)\ge\bar\mu d-\bar\mu(a\Bcap e)
\ge 2^{-n}\bar\mu e-2^{-n-1}\epsilon\ge 2^{-n-1}\epsilon$}

\noindent and

\Centerline{$\bar\mu(b'\Bcup d)
\ge\bar\mu b_0+\bar\mu(a\Bsetminus\sup L)+2^{-n-1}\epsilon\ge\gamma
>\bar\mu b'$;}

\noindent in either case $d\notBsubseteq b'$.
Thus $A_n(b')=A_n(a)=A_n(b)$ whenever
$b\Bsubseteq b'\in P_0$, and $b\in Q_n$.\ \Qed

\medskip

{\bf (c)(i)} For $m$, $n$, $i\in\Bbb N$ and $\xi\in\Omega$, set

\Centerline{$Q_{nmi\xi}=\{b:b\in Q_n$, $\xi+i\in E_n(b)$,
$\#(E_n(b)\cap\xi)=m\}$,}

\Centerline{$G_{nmi\xi}=\sup\{\coint{b,\infty}:b\in Q_{nmi\xi}\}
\in\RO^{\uparrow}(P_0)$.}

\medskip

\quad{\bf (ii)} For any $m$, $n$, $i\in\Bbb N$,
$\family{\xi}{\Omega}{G_{nmi\xi}}$ is disjoint.   \Prf\ Suppose that
$\xi<\eta$ in $\Omega$.   If $a\in Q_{nmi\xi}$ and $b\in Q_{nmi\eta}$,
we see that $\xi+i<\eta$, $\xi+i\in E_n(a)$ and

\Centerline{$\#(E_n(b)\cap\eta)=m
=\#(E_n(a)\cap\xi)<\#(E_n(a)\cap\eta)$.}

\noindent So $E_n(a)\ne E_n(b)$ and $A_n(a)\ne A_n(b)$.   But both $a$ and
$b$ are supposed to belong to $Q_n$, so $\coint{a,\infty}$ must be disjoint
from $\coint{b,\infty}$.
As $b$ is arbitrary, $\coint{a,\infty}\cap G_{nmi\eta}=\emptyset$;
as $a$ is arbitrary, $G_{nmi\xi}\cap G_{nmi\eta}=\emptyset$.\ \Qed

\medskip

\quad{\bf (iii)} For any $\xi\in\Omega$ and $a\in P_0$, there are $m$, $n$,
$i\in\Bbb N$ and $b\in Q_{nmi\xi}$ such that $a\Bsubseteq b$.
\Prf\ Let $e\in E$ be such that $d_{\xi}\Bsubseteq e$;  let $F$ be the
member of $\Cal F$ containing $e$.   If $F=\{e\}$, then
$\bar\mu(a\Bcup e)\ge\gamma>\bar\mu a$;  set $e_0=e$, so that
$e_0\in F$, $e_0\succcurlyeq e$ and $a\Bcap e_0\ne e_0$.   Otherwise,
there are infinitely many members of $F$ greater than $e$ for the ordering
$\preccurlyeq$, because $F$ has no greatest member, so
$\bar\mu(\sup_{e'\in F,e'\succcurlyeq e}e')=\infty$, and there must be
an $e_0\in F$ such that $e_0\succcurlyeq e$ and $a\Bcap e_0\ne e_0$.

Let $n\in\Bbb N$ be such that
$2^{-n}\bar\mu e_0<\min(\gamma-\bar\mu a,\bar\mu(e_0\Bsetminus a))$.
Then $\{d_{\xi+i}:i\in\Bbb N\}$ meets $D'_{e_0n}$ in an infinite set, and
there is an $i\in\Bbb N$ such that
$d_{\xi+i}\in D'_{e_0n}$, $\bar\mu d_{\xi+i}=2^{-n}\bar\mu e_0$,
and $d\notBsubseteq(a\Bcap e_0)\Bcup d_{\xi+i}$ whenever $d\in D_{e_0n}$
and $d\Bsupset d_{\xi+i}$.   Set $a'=a\Bcup d_{\xi+i}$;  then
$d_{\xi+i}$ is a maximal member of $A_n(a')$.   Let $b\in Q_n$ be such that
$a'\Bsubseteq b$ and $A_n(b)=A_n(a')$.   Then $\xi+i\in E_n(b)$.
Set $m=\#(E_n(b)\cap\xi)$.   Then $b\in Q_{nmi\xi}$ and
$a\Bsubseteq b$.\ \Qed

Accordingly $b\in\coint{a,\infty}\cap G_{nmi\xi}$.
As $a$ is arbitrary, $\bigcup_{m,n,i\in\Bbb N}G_{nmi\xi}$ is dense in $P_0$
and $\sup_{m,n,i\in\Bbb N}G_{nmi\xi}=P_0$ in $\RO^{\uparrow}(P_0)$.

\medskip

{\bf (d)(i)} Let $\frak G$ be the order-closed subalgebra of
$\RO^{\uparrow}(P_0)$ generated by
$\{G_{nmi\xi}:m$, $n$, $i\in\Bbb N$, $\xi\in\Omega\}$.
By (c-ii) and (c-iii), the conditions of 514F
are satisfied, and $\frak G$ has countable Maharam type.

\medskip

\quad{\bf (ii)} If $d\in P_0\cap\bigcup_{e\in E,n\in\Bbb N}D_{en}$ then
$\coint{d,\infty}\in\frak G$.   \Prf\ Set

\Centerline{$H
=\sup\{G_{nmi\xi}:m$, $n$, $i\in\Bbb N$, $\xi\in\Omega$ and
$G_{nmi\xi}\subseteq\coint{d,\infty}\}\in\RO^{\uparrow}(P_0)$.}

\noindent Then $H\in\frak G$ and $H\Bsubseteq\coint{d,\infty}$.
Suppose that $a\in P_0$ and $a\Bsupseteq d$.
Let $n\in\Bbb N$ be such that $d\in\bigcup_{e\in E}D_{en}$.
Then there is a $b\in Q_n$
such that $a\Bsubseteq b$.   In this case, $d\in A_n(b)$ so there is a
maximal $d'\in A_n(b)$ including $d$;  let $\xi\in\Omega$,
$i\in\Bbb N$ be such that
$d'=d_{\xi+i}$, and set $m=\#(E_n(b)\cap\xi)$.   Then $b\in Q_{nmi\xi}$.
On the other hand, for any
$b'\in Q_{nmi\xi}$, $d\Bsubseteq d_{\xi+i}\Bsubseteq b'$, so
$\coint{b',\infty}\subseteq\coint{d,\infty}$;  as
$b'$ is arbitrary, $G_{nmi\xi}\subseteq\coint{d,\infty}$ and
$G_{nmi\xi}\subseteq H$.   Accordingly $b\in H\cap\coint{a,\infty}$.
As $a$ is arbitrary, $H$ is dense in $\coint{d,\infty}$ and must be the
whole of $\coint{d,\infty}$;  thus we have $\coint{d,\infty}=H\in\frak G$.\
\Qed

\medskip

\quad{\bf (iii)} If $a\in P_0$ there is a $b\in P_0$ such that
$a\Bsubseteq b$ and $\coint{b,\infty}\in\frak G$.   \Prf\ Let $E_0$ be a
countable subset of $E$ such that $a\Bsubseteq\sup E_0$ and
$\bar\mu(\sup E_0)>\gamma$.
Set $L=\{e:e\in E_0$, $a\Bsupseteq e\}$.   Then $E_0\setminus L$ is
non-empty, and

\Centerline{$\sum_{e\in E_0\setminus L}\bar\mu(a\Bcap e)
=\bar\mu a-\bar\mu(\sup L)<\gamma-\bar\mu(\sup L)$.}

\noindent We therefore have a family
$\family{e}{E_0\setminus L}{\gamma_e}$ such that
$\bar\mu(a\Bcap e)<\gamma_e\le\bar\mu e$ for every $e\in E_0\setminus L$
and $\sum_{e\in E_0\setminus L}\gamma_e<\gamma-\bar\mu(\sup L)$.
For each $e\in E_0$ there is a $B_e\subseteq\bigcup_{n\in\Bbb N}D_{en}$
such that $a\Bcap e\Bsubseteq\sup B_e$ and $\bar\mu(\sup B_e)\le\gamma_e$,
by 528S(ii).   Set

\Centerline{$B=L\cup\bigcup_{e\in E_0\setminus L}B_e
\subseteq\bigcup_{e\in E,n\in\Bbb N}D_{en}$}

\noindent and $b=\sup B$.   Then $a\Bsubseteq b$ and

\Centerline{$\bar\mu b
=\bar\mu(\sup L)+\sum_{e\in E_0\setminus L}\bar\mu(\sup B_e)
\le\bar\mu(\sup L)+\sum_{e\in E_0\setminus L}\gamma_e
<\gamma$,}

\noindent so $b\in P_0$.   On the other hand,

\Centerline{$\coint{b,\infty}=\bigcap_{d\in B}\coint{d,\infty}
=\inf_{d\in B}\coint{d,\infty}\in\frak G$,}

\noindent as required.\ \Qed

\medskip

\quad{\bf (iv)} As $a$ is arbitrary, $\frak G$ includes a $\pi$-base for
the Boolean algebra $\RO^{\uparrow}(P_0)$ and must be the whole of
$\RO^{\uparrow}(P_0)$.   Accordingly

\Centerline{$\tau(\RO^{\uparrow}(P_0))=\tau(\frak G)\le\omega$.}

\noindent This completes the proof.
}%end of proof of 528U

\leader{528V}{Theorem}\dvAnew{2011;  extends former 5{}28Rb}
Let $(\frak A,\bar\mu)$ be an atomless semi-finite
measure algebra and $0<\gamma<\bar\mu 1$.   Then
$\AM(\frak A,\bar\mu,\gamma)$ has countable Maharam type.

\proof{ Throughout the proof, $P$ will stand for
$\{a:a\in\frak A$, $\bar\mu a<\gamma\}$.

\medskip

{\bf (a)} Suppose that there are a partition $E$ of unity in
$\frak A$ and an $\epsilon>0$ such that $\frak A_e$ is homogeneous and
$\epsilon\le\bar\mu e<\infty$ for every $e\in E$.

\medskip

\quad{\bf (i)}  Let $\preccurlyeq$ be a well-ordering of $E$ such that
$\tau(\frak A_e)\le\tau(\frak A_{e'})$ whenever $e\preccurlyeq e'$ in $E$.
Let $\Cal F_0$ be a maximal disjoint
family of subsets of $E$ of order type
$\omega$ in the ordering induced by $\preccurlyeq$.
Then $M=E\setminus\bigcup\Cal F_0$ must be finite;  set
$\Cal F=\Cal F_0\cup\{\{e\}:e\in M\}$.

\medskip

\quad{\bf (ii)} For $L\subseteq M$, set

\Centerline{$P_L
=\{a:a\in P$, $a\Bsupseteq\sup L$, $\bar\mu(a\Bcup e)\ge\gamma$ for
$e\in M\setminus L\}$.}

\noindent Then $\langle P_L\rangle_{L\subseteq M}$ is a disjoint family of
open subsets of $P$.   Also $\bigcup_{L\subseteq M}P_L$ is dense in $P$.
\Prf\ If $a\in P$, let $L\subseteq M$ be a maximal set such that
$\bar\mu(a\Bcup\sup L)<\gamma$, and set $b=a\Bcup\sup L$;  then
$a\Bsubseteq b\in P_L$.\ \QeD\  So
$\RO^{\uparrow}(P)$ is isomorphic to the simple product
$\prod_{L\subseteq M}\RO^{\uparrow}(P_L)$ (315H again).

\medskip

\quad{\bf (iii)}
If $L\subseteq M$, then $\RO^{\uparrow}(P_L)$ has countable
Maharam type.   \Prf\ If $P_L=\emptyset$ this is trivial.   Otherwise there
is an $a\in P_L$ and $\bar\mu(\sup L)\le\bar\mu a<\gamma$.
Consider $\frak A'=\frak A_{1\Bsetminus\sup L}$,
$\gamma'=\gamma-\bar\mu(\sup L)$, $E'=E\setminus L$,
$\Cal F'=\Cal F\setminus\{\{e\}:e\in L\}$ and
$\preccurlyeq'\mskip5mu=\mskip5mu\preccurlyeq\mskip-3mu
  \cap\mskip2mu(E'\times E')$.
Then $(\frak A',\bar\mu\restrp\frak A')$,
$\gamma'$, $E'$, $\epsilon$, $\preccurlyeq'$ and $\Cal F'$ satisfy the
conditions of 528U.   Setting

\Centerline{$Q_0
=\{c:c\in\frak A'$,
$\bar\mu c<\gamma'\le\bar\mu(c\Bcup e)$ for every $e\in M\setminus L\}$,}

\noindent $\RO^{\uparrow}(Q_0)$ has countable Maharam type, by 528U.
But the map $c\mapsto c\Bcup\sup L$ is an order-isomorphism between
$Q_0$ and $P_L$, so $\RO^{\uparrow}(P_L)$ has countable Maharam type.\ \Qed

\medskip

\quad{\bf (iv)} Thus $\AM(\frak A,\bar\mu,\gamma)=\RO^{\uparrow}(P)$
is isomorphic to the product of finitely many Boolean algebras
with countable Maharam type, and has countable Maharam type (514Ef).

\medskip

{\bf (b)} Now suppose that
$(\frak A,\bar\mu)$ is localizable.

\medskip

\quad{\bf (i)} In this case, let $E$ be a partition
of unity in $\frak A$ such that $\frak A_e$ is homogeneous and
$0<\bar\mu e<\infty$ for every $e\in E$.   Let $\epsilon>0$ be such that
$\sum_{e\in E,\bar\mu e\ge\epsilon}\bar\mu e>\gamma$.
For each $k\in\Bbb N$, set

\Centerline{$E_k=\{e:e\in E$, $\bar\mu e\ge 2^{-k}\epsilon\}$,
\quad$e^*_k=\sup E_k$.}

\noindent By (a),
$\AM(\frak A_{e^*_k},\bar\mu\restrp\frak A_{e^*_k},\gamma)$
has countable Maharam type for every $k$.

\medskip

\quad{\bf (ii)} Now 528Fb tells us that we have a sequence
$\sequence{k}{\pi_k}$ such that
$\pi_k$ is a regular embedding of
$\AM(\frak A_{e^*_k},\bar\mu\restrp\frak A_{e^*_k},\gamma)$ into
$\AM(\frak A,\bar\mu,\gamma)$ for each $k$, and
$\bigcup_{k\in\Bbb N}
\pi_k[\AM(\frak A_{e^*_k},\bar\mu\restrp\frak A_{e^*_k},\gamma)]
\,\,\tau$-generates $\AM(\frak A,\bar\mu,\gamma)$.
So $\AM(\frak A,\bar\mu,\gamma)$ has countable Maharam type.
\Prf\ For each $k$, we have a countable $\tau$-generating set
$D_k\subseteq\AM(\frak A_{e^*_k},\bar\mu\restrp\frak A_{e^*_k},\gamma)$.
Let $\frak G$ be the
order-closed subalgebra of $\AM(\frak A,\bar\mu,\gamma)$ generated by
$D=\bigcup_{k\in\Bbb N}\pi_k[D_k]$.   For each $k\in\Bbb N$,
$\pi_k^{-1}[\frak G]$ is an order-closed subalgebra of
$\AM(\frak A_{e^*_k},\bar\mu\restrp\frak A_{e^*_k},\gamma)$ including
$D_k$, so is the whole of
$\AM(\frak A_{e^*_k},\bar\mu\restrp\frak A_{e^*_k},\gamma)$, that is,
$\pi_k[\AM(\frak A_{e^*_k},\bar\mu\restrp\frak A_{e^*_k},\gamma)]
\subseteq\frak G$.   Since
$\bigcup_{k\in\Bbb N}
\pi_k[\AM(\frak A_{e^*_k},\bar\mu\restrp\frak A_{e^*_k},\gamma)]
\,\,\tau$-generates $\AM(\frak A,\bar\mu,\gamma)$,
$\frak G=\AM(\frak A,\bar\mu,\gamma)$ and
$\tau(\AM(\frak A,\bar\mu,\gamma))\le\#(D)\le\omega$.\ \Qed

\medskip

{\bf (c)} Thus we have the result when $(\frak A,\bar\mu)$ is localizable.
For the general case of atomless semi-finite $(\frak A,\bar\mu)$,
let $(\widehat{\frak A},\tilde\mu)$ be the
localization of $(\frak A,\bar\mu)$ (322Q).
Since the embedding $\frak A\embedsinto\widehat{\frak A}$ identifies
$\frak A^f$ with $\widehat{\frak A}^f$ (322P),
$\{a:a\in\widehat{\frak A}$, $\tilde\mu a<\gamma\}$
can be identified with $P$, and the regular open algebras
$\AM(\frak A,\bar\mu,\gamma)$ and
$\AM(\widehat{\frak A},\tilde\mu,\gamma)$ are isomorphic.   Again because
$\frak A^f$ and $\widehat{\frak A}^f$ are isomorphic, $\widehat{\frak A}$
is atomless.   By (b), the common Maharam type of
$\AM(\frak A,\bar\mu,\gamma)$ and
$\AM(\widehat{\frak A},\tilde\mu,\gamma)$ is countable.
}%end of proof of 528V

\exercises{\leader{528X}{Basic exercises (a)}
%\spheader 528Xa
Suppose that $(X,\Sigma,\mu)$ is a measure
space and $(\frak A,\bar\mu)$ its measure algebra.   Let
$\Cal E\subseteq\Sigma$ be a family such that $\mu$ is outer
regular with respect to $\Cal E$, and $P$ the set
$\{(E,\alpha):E\in\Cal E$, $\mu E<\alpha\le\mu X\}$, ordered by saying that
$(E,\alpha)\le(F,\beta)$ if $E\subseteq F$ and $\beta\le\alpha$.   Show that
$\RO^{\uparrow}(P)$ is isomorphic to $\AM^*(\frak A,\bar\mu)$.
%528C out of order query

\spheader 528Xb
 Let $(\frak A,\bar\mu)$ be an atomless
quasi-homogeneous semi-finite measure
algebra.   Show that $\AM(\frak A,\bar\mu,\gamma)$ is homogeneous whenever
$0<\gamma<\bar\mu 1$.   \Hint{first check that
$\frak A\cong\frak A_{1\Bsetminus a}$ whenever $a\in\frak A$ and
$0<\bar\mu a<\bar\mu 1$.}
%528D

\spheader 528Xc(i)
Let $(\frak A,\bar\mu)$ be a totally finite measure algebra.
Show that $\AM(\frak A,\bar\mu,\bar\mu 1)$ is isomorphic to $\frak A$.
(ii) Let $(\frak A,\bar\mu)$ be an atomless measure algebra and
$e\in\frak A$ a non-zero element of finite measure.   Show that the
principal ideal $\frak A_e$ can be regularly embedded in
$\AM(\frak A,\bar\mu,\bar\mu e)$.
%528Fa

\spheader 528Xd\dvArevised{2011}
Show that if $(\frak A,\bar\mu)$ is a probability algebra,
$0<\gamma\le 1$ and $\kappa\ge\max(\omega,\tau(\frak A))$ then
$\AM(\frak A,\bar\mu,\gamma)$ can be regularly embedded in
$\AM(\frak B_{\kappa},\bar\nu_{\kappa},\gamma)$.
%528H 528Xb mt52bits

\spheader 528Xe Let $(X,\frak T,\Sigma,\mu)$ be a quasi-Radon measure space
and $(\frak A,\bar\mu)$ its amoeba algebra.   Show that if $0<\gamma<\mu X$
then the additivity of
$\mu$ is not a precaliber of $\AM(\frak A,\bar\mu,\gamma)$.
%528L out of order query

\spheader 528Xf
Let $(\frak A,\bar\mu)$ be an atomless
$\sigma$-finite measure algebra
and $0<\gamma<\bar\mu 1$.   Show that
$\frak m(\AM(\frak A,\bar\mu,\gamma))=\wdistr(\frak A)$.
%528L for \le 517If & 528N for \ge;  mt52bits
%528N

\spheader 528Xg Let $(\frak A,\bar\mu)$ be an atomless
semi-finite measure algebra.   (i) Show that

\Centerline{$c(\AM^*(\frak A,\bar\mu))
=\link_m(\AM^*(\frak A,\bar\mu))=\max(c(\frak A),\tau(\frak A))$}

\noindent for any integer $m\ge 2$.   (ii) Show that

\Centerline{$d(\AM^*(\frak A,\bar\mu))
=\pi(\AM^*(\frak A,\bar\mu))
=\max(\cff[c(\frak A)]^{\le\omega},\pi(\frak A))$.}
%528P

\spheader 528Xh\dvAnew{2011}
Show that for any cardinal $\kappa$ there is a
probability algebra $(\frak A,\bar\mu)$ such that
$\AM(\frak A,\bar\mu,\bover12)$ has Maharam type $\kappa$.
%528V 528Xc atom of measure $\bover12$

\leader{528Y}{Further exercises (a)}
%\spheader 528Ya
Let $(\frak A,\bar\mu)$ be an atomless quasi-homogeneous
semi-finite measure algebra.   Show that
$\AM^*(\frak A,\bar\mu)$ is homogeneous.
%528Xb 528D

\spheader 528Yb\dvAnew{2011} Let $(\frak A,\bar\mu)$ be an atomless
totally finite
measure algebra, and suppose that $\AM(\frak A,\bar\mu,\gamma)$ can be
regularly embedded in $\AM^*(\frak A,\bar\mu)$ for every
$\gamma\in\ooint{0,\bar\mu 1}$.   Show that $\frak A$ is homogeneous.
%528F out of order query mt52bits

\spheader 528Yc\dvAnew{2011} Show that $\frak B_{\omega_1}$ cannot
be regularly embedded in
$\AM(\frak B_{\omega},\bar\nu_{\omega},\bover12)$.
%528K mt52bits

\spheader 528Yd
 Let $(\frak A,\bar\mu)$ be an atomless
probability algebra and $\gamma\in\ooint{0,1}$.   Show that
$\AM(\frak A,\bar\mu,\gamma)$ is not \wsid.
%528L

\spheader 528Ye Let $\kappa$ be an infinite cardinal.   Show that
(i) $\pi(\RO^{\uparrow}(\Cal S_{\kappa}^{\infty}))
=\cf\Cal S_{\kappa}^{\infty}$
is the cardinal power $\kappa^{\omega}$;   (ii) for every $m\ge 2$,

\Centerline{$c(\RO^{\uparrow}(\Cal S_{\kappa}^{\infty}))
=c^{\uparrow}(\Cal S_{\kappa}^{\infty})
=\link_m(\RO^{\uparrow}(\Cal S_{\kappa}^{\infty}))
=\link^{\uparrow}_m(\Cal S_{\kappa}^{\infty})=\kappa$;}

\noindent (iii) $d(\RO^{\uparrow}(\Cal S_{\kappa}^{\infty}))
=\duparrow(\Cal S_{\kappa}^{\infty})
=\max(\cf\Cal N,\cff[\kappa]^{\le\omega})$.
%528Q

\spheader 528Yf\dvArevised{2011}
Let $(\frak A,\bar\mu)$ be a purely atomic semi-finite
measure algebra of magnitude
at most $\frak c$, and $0<\gamma<\bar\mu 1$.   Show that
$\AM(\frak A,\bar\mu,\gamma)$ has countable Maharam type.
%528R

\spheader 528Yg\dvAnew{2011} Let $(\frak A,\bar\mu)$ be an atomless
semi-finite measure algebra and $0<\gamma<\bar\mu 1$.   Set
$\kappa=\max(\omega,c(\frak A),\tau(\frak A))$ and
$P=\{a:a\in\frak A$, $\bar\mu a<\gamma\}$;  let $\Bbb P$ be the forcing
notion $(P,\Bsubseteqshort,0,\uparrow)$ (see 5A3A).   Show that
$\VVdash_{\Bbb P}\check\kappa<\omega_1$.
%528V

\spheader 528Yh\dvAnew{2011}
Show that if $(\frak A,\bar\mu)$ is a measure
algebra with at most $\frak c$ atoms, then
$\tau(\AM^*(\frak A,\bar\mu))\le\omega$.
%528V

\leaveitout{\spheader 5{}28Yc For $\alpha\in\NN$ and any set $I$
let $(\Cal S^{\infty}_I)^{(\alpha)}$
be the partially ordered set

\Centerline{$\{p:p\subseteq\Bbb N\times I$,
$\#(p[\{n\}])\le\alpha(n)$ for every $n$,
$\sup_{n\in\Bbb N}\#(p[\{n\}])<\infty\}$.}

\noindent Show that $\RO^{\uparrow}(\Cal S^{\infty}_I)^{(\alpha)}$ can be
regularly embedded in
$\RO^{\uparrow}(\Cal S^{\infty}_I)^{(\beta)}$ whenever $I$ is infinite and
$\alpha$, $\beta$ are both unbounded.

This certainly is wrong.   But what if both $\alpha$ and $\beta$ tend to
$\infty$?
}
}%end of exercises

\leader{528Z}{Problems (a)}
%\spheader 528Za
Let $(\frak A_L,\bar\mu_L)$ be the measure
algebra of Lebesgue measure on $\Bbb R$.   Is the amoeba algebra
$\AM(\frak A_L,\bar\mu_L,1)$ isomorphic to the amoeba algebra
$\AM(\frak B_{\omega},\bar\nu_{\omega},\bover12)$?
%528K

\spheader 528Zb Let $(\frak A,\bar\mu)$ be a probability algebra, $\frak B$
a closed subalgebra of $\frak A$, and $0<\gamma<1$.   Is it
necessarily true that
$\AM(\frak B,\bar\mu\restrp\frak B,\gamma)$ can be regularly embedded in
$\AM(\frak A,\bar\mu,\gamma)$?   \cmmnt{(See 528Xd and 528G.)}
%528G 528K

%Can the category algebra \frak G_{\omega} be regularly embedded in
% \AM(\frak B_{\omega},\bar\nu_{\omega},\bover12) ?

\endnotes{
\Notesheader{528} The ideas of
528A-528K %528A 528D 528E 528F 528H 528J 528K
are based on {\smc Truss 88}.   The original amoeba algebras of
{\smc Martin \& Solovay 70}, used in their proof that
$\add\Cal N\ge\frak m$ (528L), are closest to 528C.   For some more
about the amoeba algebras derived from Lebesgue measure, see
{\smc Bartoszy\'nski \& Judah 95}, \S3.4.
In this section I have been willing
to assume that the measure algebras involved are atomless;  amoeba
algebras are surely still interesting for other measure algebras, but
the new questions seem to be combinatoric rather than measure-theoretic.
It seems still to be unknown whether the algebras
$\AM(\frak A_L,\bar\mu_L,1)$ and
$\AM(\frak B_{\omega},\bar\nu_{\omega},\bover12)$ are actually
isomorphic, rather than just mutually embeddable (528K, 528Za).

If we think of the partially ordered sets of 528A and 528I as forcing
notions, we can study them in terms of the forcing universes they
lead to.   This is associated with the
prominence of `regular embeddings' in this section.
I will not attempt to use such methods here, but I mention
them because results such as 528Yg have been part of the impulse for
studying amoeba algebras, and led naturally to 528Ya, 528R and 528U-528V.
}%end of notes

\discrpage

