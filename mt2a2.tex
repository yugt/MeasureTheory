\frfilename{mt2a2.tex} 
\versiondate{30.11.09} 
\copyrightdate{2000} 
 
\def\chaptername{Appendix} 
\def\sectionname{The topology of Euclidean space} 
 
\newsection{2A2} 
 
In the appendix to Volume 1\cmmnt{ (\S1A2)} I 
discussed open and closed sets in $\BbbR^r$;  the 
chief aim there was to support the idea of `Borel set', which is 
vital in the theory of Lebesgue measure, but of course they are also 
fundamental to the study of continuous functions, and indeed to all 
aspects of real analysis. 
I give here a very brief introduction to the further elementary facts 
about closed and compact sets and continuous functions which we need for 
this volume.   Much of this material can be derived from the 
generalizations in \S2A3, but nevertheless I sketch the proofs, since 
for the greater 
part of the volume (most of the exceptions are in Chapter 24) Euclidean 
space is sufficient for our needs. 
 
\leader{2A2A}{Closures:  Definition} For any $r\ge 1$ and any 
$A\subseteq\BbbR^r$, the {\bf closure} of $A$, $\overline{A}$, is the 
intersection of all the closed subsets of $\BbbR^r$ including $A$. 
This is\cmmnt{ itself closed (being the intersection of a non-empty 
family of closed sets, see 1A2Fd), so is} the smallest closed set 
including $A$.   \cmmnt{In particular, }$A$ is closed iff $\overline{A}=A$. 
 
\leader{2A2B}{Lemma} Let $A\subseteq\BbbR^r$ be any set.   Then for 
$x\in\BbbR^r$ the following are equiveridical: 
 
(i) $x\in\overline{A}$\cmmnt{, the closure of $A$}; 
 
(ii) $B(x,\delta)\cap A\ne\emptyset$ for every $\delta>0$, where 
$B(x,\delta)=\{y:\|y-x\|\le\delta\}$; 
 
(iii) there is a sequence $\sequencen{x_n}$ in $A$ such that 
$\lim_{n\to\infty}\|x_n-x\|=0$. 
 
\proof{{\bf (a)(i)$\Rightarrow$(ii)} Suppose that $x\in\overline{A}$ and 
$\delta>0$.   Then $U(x,\delta)=\{y:\|y-x\|<\delta\}$ is an open set 
(1A2D), so $F=\BbbR^r\setminus U(x,\delta)$ is closed, while  
$x\notin F$.   Now 
 
\Centerline{$x\in\overline{A}\setminus F 
\Longrightarrow\overline{A}\not\subseteq F 
\Longrightarrow A\not\subseteq F 
\Longrightarrow A\cap U(x,\delta)\ne\emptyset 
\Longrightarrow A\cap B(x,\delta)\ne\emptyset$.} 
 
\noindent As $\delta$ is arbitrary, (ii) is true. 
 
\medskip 
 
{\bf (b)(ii)$\Rightarrow$(iii)} If (ii) is true, then for each $n\in\Bbb 
N$ we can find an $x_n\in A$ such that $\|x_n-x\|\le 2^{-n}$, and now 
$\lim_{n\to\infty}\|x_n-x\|=0$. 
 
\medskip 
 
{\bf (c)(iii)$\Rightarrow$(i)} Assume (iii).   \Quer\ Suppose, if 
possible, that $x\notin\overline{A}$.   Then $x$ belongs to the open set 
$\BbbR^r\setminus\overline{A}$ and there is a $\delta>0$ such that 
$U(x,\delta)\subseteq\BbbR^r\setminus\overline{A}$.   But now there 
is an $n$ such that $\|x_n-x\|<\delta$, in which case $x_n\in 
U(x,\delta)\cap A\subseteq U(x,\delta)\cap\overline{A}$.   \Bang 
}%end of proof of 2A2B 
 
\leader{2A2C}{Continuous functions (a)} \cmmnt{I begin with a 
characterization of continuous functions 
in terms of open sets.}   If $r$, $s\ge 1$, $D\subseteq\BbbR^r$ and 
$\phi:D\to\BbbR^s$ is a function, we say that $\phi$ is {\bf 
continuous} if for every $x\in D$ and $\epsilon>0$ there is a $\delta>0$ 
such that $\|\phi(y)-\phi(x)\|\le\epsilon$ whenever $y\in D$ and 
$\|y-x\|\le\delta$.   \cmmnt{Now }$\phi$ is continuous 
iff for every open set 
$G\subseteq\BbbR^s$ there is an open set $H\subseteq\BbbR^r$ such that 
$\phi^{-1}[G]=D\cap H$. 
 
\prooflet{\Prf\ {\bf (i)} Suppose that $\phi$ is continuous and that 
$G\subseteq\BbbR^s$ is open.   Set 
 
\Centerline{$H=\bigcup\{U:U\subseteq\BbbR^r\text{ is open},\,\phi[U\cap 
D]\subseteq G\}$.} 
 
\noindent Then $H$ is a union of open sets, therefore open (1A2Bd), and 
$H\cap D\subseteq\phi^{-1}[G]$.   If $x\in\phi^{-1}[G]$, then 
$\phi(x)\in G$, so there is an $\epsilon>0$ such that 
$U(\phi(x),\epsilon)\subseteq G$;  now there is a $\delta>0$ such 
that $\|\phi(y)-\phi(x)\|\le\bover12\epsilon$ whenever $y\in D$ and 
$\|y-x\|\le\delta$, so that 
 
\Centerline{$\phi[U(x,\delta)\cap D]
\subseteq U(\phi(x),\epsilon)\subseteq G$} 
 
\noindent and 
 
\Centerline{$x\in U(x,\delta)\subseteq H$.} 
 
\noindent As $x$ is arbitrary, $\phi^{-1}[G]=H\cap D$.   As $G$ is 
arbitrary, $\phi$ satisfies the condition. 
 
\quad{\bf (ii)} Now suppose that $\phi$ satisfies the condition.   Take 
$x\in D$ and $\epsilon>0$.   Then $U(\phi(x),\epsilon)$ is open, 
so 
there is an open $H\subseteq\BbbR^r$ such that 
$H\cap D=\phi^{-1}[U(\phi(x),\epsilon)]$;  we see that $x\in H$, so 
there is a $\delta>0$ such that $U(x,\delta)\subseteq H$;  now if 
$y\in 
D$ and $\|y-x\|\le\bover12\delta$ then $y\in D\cap H$, $\phi(y)\in 
U(\phi(x),\epsilon)$ and $\|\phi(y)-\phi(x)\|\le\epsilon$.   As $x$ and 
$\epsilon$ are arbitrary, $\phi$ is continuous.   \Qed} 
 
\header{2A2Cb}{\bf (b)} Using the $\epsilon$-$\delta$ definition of 
continuity, it is 
easy to see that a function $\phi$ from a subset $D$ of $\BbbR^r$ to 
$\BbbR^s$ is continuous iff all its components $\phi_i$ are continuous, 
writing $\phi(x)=(\phi_1(x),\ldots,\phi_s(x))$ for $x\in D$. 
\prooflet{\Prf\ 
{\bf (i)} If $\phi$ is continuous, $i\le s$, $x\in D$ and $\epsilon>0$, 
then there is a $\delta>0$ such that 
 
\Centerline{$|\phi_i(y)-\phi_i(x)|\le\|\phi(y)-\phi(x)\|\le\epsilon$} 
 
\noindent whenever $y\in D$ and $\|y-x\|\le\delta$.   {\bf (ii)} If 
every $\phi_i$ is continuous, $x\in D$ and $\epsilon>0$, then there are 
$\delta_i>0$ such that $|\phi_i(y)-\phi_i(x)|\le\epsilon/\sqrt{s}$ 
whenever $y\in D$ and $\|y-x\|\le\delta_i$;  setting  
$\delta=\min_{1\le i\le r}\delta_i>0$, we have  
$\|\phi(y)-\phi(x)\|\le\epsilon$ whenever $y\in D$ and  
$\|y-x\|\le\delta$.  \Qed} 

\spheader 2A2Cc\dvAnew{2015}\cmmnt{ At one or two points, 
we shall encounter the 
following strengthening of the notion of `continuous function'.}    
If $r$, $s\ge 1$, $D\subseteq\BbbR^r$ and 
$\phi:D\to\BbbR^s$ is a function, we say that $\phi$ is 
{\bf uniformly continuous} if for every $\epsilon>0$ there is a $\delta>0$ 
such that $\|\phi(y)-\phi(x)\|\le\epsilon$ whenever $x$, $y\in D$ and 
$\|y-x\|\le\delta$.    A uniformly continuous function is\cmmnt{ of course}
continuous.

\leader{2A2D}{Compactness in $\BbbR^r$: Definition} A subset $F$ of 
$\BbbR^r$ is called {\bf compact} if whenever 
$\Cal G$ is a family of open sets covering $F$ then there is a finite 
subset $\Cal G_0$ of $\Cal G$ still covering $F$. 
 
\leader{2A2E}{Elementary properties of compact sets} Take any $r\ge 1$, and
subsets $D$, $F$, $G$ and $K$ of $\BbbR^r$. 
 
\header{2A2Ea}{\bf (a)} If $K$ is compact and 
$F$ is closed, then $K\cap F$ is 
compact.   \prooflet{\Prf\ Let $\Cal G$ be an open cover of $F\cap K$. 
Then $\Cal G\cup\{\BbbR^r\setminus F\}$ is an open cover of $K$, so has a finite subcover $\Cal G_0$ say.   Now  
$\Cal G_0\setminus\{\BbbR^r\setminus F\}$ is a finite subset of $\Cal G$ covering $K\cap F$.   As $\Cal G$ is arbitrary, $K\cap F$ is compact. 
\Qed} 
 
\header{2A2Eb}{\bf (b)} If $s\ge 1$, $\phi:D\to\BbbR^s$ is a 
continuous function, $K$ is compact and $K\subseteq D$, then $\phi[K]$ is 
compact. 
\prooflet{\Prf\ Let $\Cal V$ be an open cover of $\phi[K]$.   Let 
$\Cal H$ be 
 
\Centerline{$\{H:H\subseteq\BbbR^r\text{ is open},\,\exists\enskip 
V\in\Cal V,\,\phi^{-1}[V]=D\cap H\}$.} 
 
\noindent If $x\in K$, then $\phi(x)\in\phi[K]$ so there is a 
$V\in\Cal V$ such that $\phi(x)\in V$;  now there is an $H\in\Cal H$ such 
that $D\cap H\phi^{-1}[V]$ contains $x$ (2A2Ca);  as $x$ is arbitrary, 
$K\subseteq\bigcup\Cal H$.   Let $\Cal H_0$ be a finite subset of $H$ 
covering $K$.   For each 
$H\in\Cal H_0$, let $V_H\in\Cal V$ be such that 
$\phi^{-1}[V_H]=D\cap H$;  then $\{V_H:H\in\Cal H_0\}$ is a finite subset 
of $\Cal V$ covering 
$\phi[K]$.   As $\Cal V$ is arbitrary, $\phi[K]$ is compact.   \Qed} 
 
\spheader 2A2Ec If $K$ is compact, it is closed. 
\prooflet{\Prf\ Write $H=\BbbR^r\setminus K$.   Take any $x\in H$. 
Then $G_n=\BbbR^r\setminus B(x,2^{-n})$ is open for every $n\in\Bbb N$ 
(1A2G).   Also 
 
\Centerline{$\bigcup_{n\in\Bbb N}G_n=\{y:y\in\BbbR^r,\,\|y-x\|>0\}
=\Bbb R^r\setminus\{x\}\supseteq K$.} 
 
\noindent So there is some finite set 
$\Cal G_0\subseteq\{G_n:n\in\Bbb N\}$ which covers $K$.   
There must be an $n$ such that $\Cal G_0\subseteq\{G_i:i\le n\}$, so that 
 
\Centerline{$K\subseteq\bigcup\Cal G_0\subseteq\bigcup_{i\le n}G_i=G_n$,} 
 
\noindent and $B(x,2^{-n})\subseteq H$.   As $x$ is arbitrary, $H$ is 
open and $K$ is closed.  \Qed} 
 
\spheader 2A2Ed If $K$ is compact and $G$ is open and $K\subseteq G$, 
then there is a $\delta>0$ such that $K+B(\tbf{0},\delta)\subseteq G$. 
\prooflet{\Prf\ If $K=\emptyset$, this is trivial, as then 
 
\Centerline{$K+B(\tbf{0},1)=\{x+y:x\in K,\,y\in B(\tbf{0},1)\}=\emptyset$.} 
 
\noindent  Otherwise, set 
 
\Centerline{$\Cal G
=\{U(x,\delta):x\in \Bbb R^r,\,\delta>0,\,U(x,2\delta)\subseteq G\}$.} 
 
\noindent Then $\Cal G$ is a family of open sets and 
$\bigcup\Cal G=G$ (because $G$ is open), 
so $\Cal G$ is an open cover of $K$ and has 
a finite subcover $\Cal G_0$.   Express $\Cal G_0$ as 
$\{U(x_0,\delta_0),\ldots,U(x_n,\delta_n)\}$ where 
$U(x_i,2\delta_i)\subseteq G$ for each $i$.   Set 
$\delta=\min_{i\le n}\delta_i>0$.   If $x\in K$ and 
$y\in B(\tbf{0},\delta)$, then there 
is an $i\le n$ such that $x\in U(x_i,\delta_i)$;  now 
 
\Centerline{$\|(x+y)-x_i\|\le\|x-x_i\|+\|y\|<\delta_i+\delta
\le 2\delta_i$,} 
 
\noindent so $x+y\in U(x_i,2\delta_i)\subseteq G$.   As $x$ and $y$ 
are arbitrary, $K+B(\tbf{0},\delta)\subseteq G$.   \Qed} 
 
\cmmnt{ 
\medskip 
 
\noindent{\bf Remark} This result is a simple form of the {\bf Lebesgue 
covering lemma}. 
} 
 
\leader{2A2F}{}\cmmnt{ The value of the concept of `compactness' is 
greatly 
increased by the fact that there is an effective characterization of the 
compact subsets of $\BbbR^r$. 
 
\medskip 
 
\noindent}{\bf Theorem} For any $r\ge 1$, a subset $K$ of $\BbbR^r$ is 
compact iff it is closed and bounded. 
 
\proof{{\bf (a)} Suppose that $K$ is compact.   By 2A2Ec, it is closed. 
To see that it is bounded, consider $\Cal G=\{U(\tbf{0},n):n\in\Bbb 
N\}$.   $\Cal 
G$ consists entirely of open sets, and $\bigcup\Cal G=\Bbb 
R^r\supseteq K$, so there is a finite $\Cal G_0\subseteq\Cal G$ covering 
$K$. There must be an $n$ such that $\Cal G_0\subseteq\{G_i:i\le n\}$, 
so that 
 
\Centerline{$K\subseteq\bigcup\Cal G_0\subseteq\bigcup_{i\le 
n}U(\tbf{0},i)=U(\tbf{0},n)$,} 
 
\noindent and $K$ is bounded. 
 
\medskip 
 
{\bf (b)} Thus we are left with the converse;  I have to show that a 
closed bounded set is compact.   The main part of the argument is a 
proof by induction on $r$ that the closed interval $[-\tbf{n},\tbf{n}]$ 
is compact for all $n\in\Bbb N$, writing $\tbf{n}=(n,\ldots,n)\in\Bbb 
R^r$. 
 
\medskip 
 
\quad{\bf (i)} If $r=1$ and $n\in\Bbb N$ and $\Cal G$ is a family of 
open sets in $\Bbb R$ covering $[-n,n]$, set 
 
\Centerline{$A=\{x:x\in[-n,n],\text{ there is a finite }\Cal 
G_0\subseteq\Cal G\text{ such that }[-n,x]\subseteq\bigcup\Cal G_0\}$.} 
 
\noindent Then 
 $-n\in A$, because if $-n\in G\in\Cal G$ then 
$[-n,-n]\subseteq\bigcup\{G\}$, and $A$ is bounded above by $n$, so 
$c=\sup A$ exists and belongs to $[-n,n]$. 
 
Next, $c\in[-n,n]\subseteq\bigcup\Cal G$, so there is a $G\in\Cal G$ 
containing $c$.   Let $\delta>0$ be such that $U(c,\delta)\subseteq 
G$. 
There is an $x\in A$ such that $x\ge c-\delta$.   Let $\Cal G_0$ be a 
finite subset of $\Cal G$ covering $[-n,x]$.  Then $\Cal G_1=\Cal 
G_0\cup\{G\}$ is a finite subset of $\Cal G$ covering $[-n,c+{1\over 
2}\delta]$. 
But $c+{1\over2}\delta\notin A$ so $c+{1\over2}\delta>n$ and $\Cal G_1$ 
is a finite subset of $\Cal G$ covering $[-n,n]$.   As $\Cal G$ is 
arbitrary, $[-n,n]$ is compact and the induction starts. 
 
\medskip 
 
\quad{\bf (ii)} For the inductive step to $r+1$, regard the closed 
interval $F=[-\tbf{n},\tbf{n}]$, taken in $\BbbR^{r+1}$, as the product 
of the closed interval $E=[-\tbf{n},\tbf{n}]$, taken in $\BbbR^r$, with 
the closed interval $[-n,n]\subseteq\Bbb R$;  by the inductive 
hypothesis, both $E$ and $[-n,n]$ are compact.   Let $\Cal G$ be a 
family of open subsets of $\BbbR^{r+1}$ covering $F$.   Write $\Cal H$ 
for the family of open subsets $H$ of $\BbbR^r$ such that 
$H\times[-n,n]$ is covered by a finite subfamily of $\Cal G$.   Then 
$E\subseteq\bigcup\Cal H$.   \Prf\ Take $x\in E$.   Set 
 
\Centerline{$\Cal U_x=\{U:U\subseteq\Bbb R\text{ is 
open},\,\exists\enskip G\in\Cal G,\text{ open }H\subseteq\BbbR^r,\, 
x\in H\text{ and }H\times U\subseteq G\}$.} 
 
\noindent Then $\Cal U_x$ is a family of open subsets of $\Bbb R$.   If 
$\xi\in[-n,n]$, there is a $G\in\Cal G$ containing $(x,\xi)$;  there is 
a $\delta>0$ such that $U((x,\xi),\delta)\subseteq G$;  now 
$U(x,\bover12\delta)$ and $U(\xi,\bover12\delta)$ are open sets in 
$\BbbR^r$, $\Bbb R$ respectively and 
 
\Centerline{$U(x,\bover12\delta)\times U(\xi,\bover12\delta)
\subseteq U((x,\xi),\delta)\subseteq G$,} 
 
\noindent so $U(\xi,\bover12\delta)\in\Cal U_x$.   As $\xi$ is 
arbitrary, $\Cal U_x$ is an open cover of $[-n,n]$ in $\Bbb R$.   By 
(i), it has a finite subcover $U_0,\ldots,U_k$ say.   For each $j\le k$ 
we can find $H_j$, $G_j$ such that $H_j$ is an open subset of $\BbbR^r$ 
containing $x$ and $H_j\times U_j\subseteq G_j\in\Cal G$.   Now set 
$H=\bigcap_{j\le k}H_j$.  This is an open subset of $\BbbR^r$ 
containing $x$, and $H\times[-n,n]\subseteq\bigcup_{j\le n}G_j$ is 
covered by a finite subfamily of $\Cal G$.   So $x\in H\in\Cal H$.   As 
$x$ is arbitrary, $\Cal H$ covers $E$.   \Qed 
 
\medskip 
 
\quad{\bf (iii)} Now the inductive hypothesis tells us that $E$ is 
compact, so there is a finite subfamily $\Cal H_0$ of $\Cal H$ covering 
$E$.   For each $H\in\Cal H_0$ let $\Cal G_H$ be a finite subfamily of 
$\Cal G$ covering $H\times[-n,n]$.   Then $\bigcup_{H\in\Cal H_0}\Cal 
G_H$ is a finite subfamily of $\Cal G$ covering $E\times[-n,n]=F$.   As 
$\Cal G$ is arbitrary, $F$ is compact and the induction proceeds. 
 
\medskip 
 
\quad{\bf (iv)} Thus the interval $[-\tbf{n},\tbf{n}]$ is compact in 
$\BbbR^r$ for every $r$, $n$.   Now suppose that $K$ is a closed 
bounded set in $\BbbR^r$.   Then there is an $n\in\Bbb N$ such that 
$K\subseteq[-\tbf{n},\tbf{n}]$, that is, $K=K\cap[-\tbf{n},\tbf{n}]$. 
As $K$ is closed and $[-\tbf{n},\tbf{n}]$ is compact, $K$ is compact, by 
2A2Ea. 
 
This completes the proof. 
}%end of proof of 2A2F 
 
\leader{2A2G}{Corollary} If $\phi:D\to\Bbb R$ is continuous, where 
$D\subseteq\BbbR^r$, and $K\subseteq D$ is a non-empty compact set, 
then $\phi$ is bounded and attains its bounds on $K$. 
 
\proof{ By 2A2Eb, $\phi[K]$ is compact; by 2A2F it is closed and 
bounded.   To say that $\phi[K]$ is bounded is just to say that $\phi$ 
is bounded on $K$.   Because $\phi[K]$ is a non-empty bounded set, it 
has an infimum $a$ and a supremum $b$;  now both belong to 
$\overline{\phi[K]}$ (by the criterion 2A2B(ii), or otherwise);  because 
$\phi[K]$ is closed, both belong to $\phi[K]$, that is, $\phi$ attains 
its bounds. 
}%end of proof of 2A2G 
 
\leader{2A2H}{Lim sup and lim inf revisited} In \S1A3 I briefly 
discussed $\limsup_{n\to\infty}a_n$, $\liminf_{n\to\infty}a_n$ for real 
sequences $\sequencen{a_n}$.   In this volume we need the notion of 
$\limsup_{\delta\downarrow 0}f(\delta)$,  
$\liminf_{\delta\downarrow 0}f(\delta)$ for real functions $f$.   I say 
that $\limsup_{\delta\downarrow 0}f(\delta)=u\in[-\infty,\infty]$ if 
(i) for every 
$v>u$ there is an $\eta>0$ such that $f(\delta)$ is defined and less than 
or equal to $v$ for every $\delta\in\ocint{0,\eta}$ 
(ii) for every $v<u$ and $\eta>0$ there is a $\delta\in\ocint{0,\eta}$ 
such that $f(\delta)$ is 
defined and greater than or equal to $v$.   Similarly, 
$\liminf_{\delta\downarrow 0}f(\delta)=u\in[-\infty,\infty]$ if (i) for 
every $v<u$ there is an $\eta>0$ such that $f(\delta)$ is defined and  
greater than 
or equal to $v$ for every $\delta\in\ocint{0,\eta}$ (ii) for every 
$v>u$ and $\eta>0$ there is an $\delta\in\ocint{0,\eta}$ such that  
$f(\delta)$ is defined and less than or equal to $v$. 
 
\leader{2A2I}{}\cmmnt{ In the one-dimensional case, we have a particularly simple description of the open sets. 
 
\medskip 
 
\noindent}{\bf Proposition} If $G\subseteq\Bbb R$ is any open set, it is 
expressible as the union of a countable disjoint family of open 
intervals. 
 
\proof{ For $x$, $y\in G$ write $x\sim y$ if either $x\le y$ and 
$[x,y]\subseteq G$ or $y\le x$ and $[y,x]\subseteq G$.   It is easy to 
check that $\sim$ is an equivalence relation on $G$.   Let $\Cal C$ be 
the set of equivalence classes under $\sim$.   Then $\Cal C$ is a 
partition of $G$.   Now every $C\in\Cal C$ is an open interval. 
\Prf\ Set $a=\inf C$, $b=\sup C$ (allowing $a=-\infty$ and/or $b=\infty$ 
if $C$ is unbounded).   If $a<x<b$, there are $y$, $z\in C$ such that 
$y\le x\le z$, so that $[y,x]\subseteq[y,z]\subseteq G$ and $y\sim x$ 
and $x\in C$;  thus $\ooint{a,b}\subseteq C$.   If $x\in C$, there is an 
open interval $I$ containing $x$ and included in $G$;  since $x\sim y$ 
for every $y\in I$, $I\subseteq C$;  so 
 
\Centerline{$a\le\inf I<x<\sup I\le b$} 
 
\noindent and $x\in\ooint{a,b}$.   Thus $C=\ooint{a,b}$ is an open 
interval.\ \Qed 
 
To see that $\Cal C$ is countable, observe that every member of $\Cal C$ 
contains a member of $\Bbb Q$, so that we have a surjective function 
from a subset of $\Bbb Q$ onto $\Cal C$, and $\Cal C$ is countable (1A1E). 
}%end of proof of 2A2I 
 
\discrpage 
 
