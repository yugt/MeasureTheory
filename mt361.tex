\frfilename{mt361.tex}
\versiondate{6.2.08}
\copyrightdate{1995}

\def\chaptername{Function spaces}
\def\sectionname{$S$}

\newsection{361}

This is the fundamental Riesz space associated with a Boolean ring
$\frak A$.   When $\frak A$ is a ring of sets, $S(\frak A)$ can be
regarded as the linear space of `simple functions' generated by the
indicator functions of members of $\frak A$ (361L).   Its most
important property is the universal mapping theorem 361F, which
establishes a one-to-one correspondence between (finitely) additive
functions on $\frak A$ (361B-361C) and linear operators on $S(\frak A)$.
Simple universal mapping theorems of this type can be interesting, but
do not by themselves lead to new insights;  what makes this one
important is the fact that $S(\frak A)$ has a canonical Riesz space
structure, norm and multiplication (361E).   From this we can deduce
universal mapping theorems for many other classes of function (361G,
361H, 361I, 361Xb).
\cmmnt{(Particularly important are countably additive and
completely additive real-valued functionals, which will be dealt with in
the next section.)}   While the exact construction of
$S(\frak A)$\cmmnt{ (and the associated map from $\frak A$ to
$S(\frak A)$)} can be
varied (361D, 361L, 361M, 361Ya), its structure is uniquely defined, so
homomorphisms between Boolean rings correspond to maps between their
$S(\,)$-spaces (361J), and (when $\frak A$ is a
Boolean algebra) $\frak A$ can be recovered from the
Riesz space $S(\frak A)$ as the algebra of its projection bands (361K).

\leader{361A}{Boolean rings} \cmmnt{In this section I speak of Boolean
{\it rings} rather than {\it algebras};   there are ideas in \S365 below
which are more naturally expressed in terms of the ring of elements of
finite measure in a measure algebra than in terms of the whole algebra.
I should perhaps therefore recall some of the ideas of \S311, which is
the last time when Boolean rings without identity were mentioned, and
set out some simple facts.

\spheader 361Aa
Any Boolean ring $\frak A$ can be represented as the ring of compact
open subsets of its Stone space $Z$, which is
a zero-dimensional locally compact Hausdorff space
(311I);  $Z$ is just the set of surjective ring homomorphisms from
$\frak A$ onto $\Bbb Z_2$ (311E).

\medskip

}{\bf (b)} If $\frak A$ and $\frak B$ are Boolean rings and
$\pi:\frak A\to\frak B$ is a function, then the following are
equiveridical:  (i) $\pi$ is a ring homomorphism;
(ii) $\pi(a\Bsetminus b)=\pi a\Bsetminus\pi b$ for all $a$,
$b\in\frak A$;   (iii) $\pi$ is a lattice homomorphism and $\pi 0=0$.
\prooflet{\Prf\ See 312H.   To prove
(ii)$\Rightarrow$(iii), observe that if $a$, $b\in\frak A$ then

\Centerline{$\pi(a\Bcap b)=\pi a\Bsetminus\pi(a\Bsetminus b)
=\pi a\Bsetminus(\pi a\Bsetminus\pi b) =\pi a\Bcap\pi b$,}

\Centerline{$\pi a=\pi((a\Bcup b)\cap a)
=\pi(a\Bcup b)\cap\pi a\Bsubseteq\pi(a\Bcup b)$,}

\Centerline{$\pi(b\Bsetminus a)=\pi((a\Bcup b)\Bsetminus a)
=\pi(a\Bcup b)\Bsetminus\pi a$,}

\Centerline{$\pi(a\Bcup b)=\pi a\Bcup\pi(b\Bsetminus a)
=\pi a\Bcup(\pi b\Bsetminus\pi a)=\pi a\Bcup\pi b$.  \Qed}
}


\spheader 361Ac If $\frak A$ and $\frak B$ are Boolean rings and
$\pi:\frak A\to\frak B$ is a ring homomorphism, then $\pi$ is
order-continuous iff $\inf\pi[A]=0$ whenever $A\subseteq\frak A$ is
non-empty and downwards-directed and $\inf A=0$ in $\frak A$;  while
$\pi$ is sequentially order-continuous iff $\inf_{n\in\Bbb N}\pi a_n=0$
whenever $\sequencen{a_n}$ is a non-increasing sequence in $\frak A$
with infimum $0$.   \prooflet{(See 313L.)}

\spheader 361Ad \cmmnt{ The following will be a particularly important
type of Boolean ring for us.}   If $(\frak A,\bar\mu)$ is a measure
algebra, then the ideal $\frak A^f=\{a:a\in\frak A,\,\bar\mu a<\infty\}$
is a Boolean ring in its own right.   Now suppose that
$(\frak B,\bar\nu)$ is another measure algebra and
$\frak B^f\subseteq\frak B$
the corresponding ring of elements of finite measure.   We can say that
a ring homomorphism $\pi:\frak A^f\to\frak B^f$ is
{\bf measure-preserving} if $\bar\nu\pi a=\bar\mu a$
for every $a\in\frak A^f$.   In this case $\pi$ is order-continuous.
\prooflet{\Prf\ If $A\subseteq\frak A^f$ is
non-empty, downwards-directed and has infimum $0$, then
$\inf_{a\in A}\bar\mu a=0$, by 321F;  but this means that
$\inf_{a\in A}\bar\nu\pi a=0$, and $\inf\pi[A]=0$ in $\frak B^f$.\ \Qed}

\leader{361B}{Definition} Let $\frak A$ be a Boolean ring and $U$
a linear space.   A function $\nu:\frak A\to U$ is
{\bf finitely additive}, or just {\bf additive}, if
$\nu(a\Bcup b)=\nu a+\nu b$ whenever $a$, $b\in\frak A$ and
$a\Bcap b=0$.

\leader{361C}{Elementary facts}\cmmnt{ We have the following immediate
consequences of this definition, corresponding to 326B and 313L.}   Let
$\frak A$ be a Boolean ring, $U$ a linear space and
$\nu:\frak A\to U$ an additive function.

\spheader 361Ca $\nu 0=0$\prooflet{ (because $\nu 0=\nu 0+\nu 0$)}.

\spheader 361Cb If $a_0,\ldots,a_m$ are disjoint in $\frak A$, then
$\nu(\sup_{j\le m}a_j)=\sum_{j=0}^m\nu a_j$.   \prooflet{(Induce on $m$.)}

\spheader 361Cc If $\frak B$ is another Boolean ring and
$\pi:\frak B\to\frak A$ is a ring homomorphism, then
$\nu\pi:\frak B\to U$ is additive.   In particular, if $\frak B$ is a
subring of $\frak A$, then
$\nu\restrp\frak{B}:\frak B\to U$ is  additive.

\spheader 361Cd If $V$ is another linear space and $T:U\to V$ is a
linear operator, then $T\nu:\frak A\to V$ is additive.

\spheader 361Ce If $U$ is a partially ordered linear space, then $\nu$
is order-preserving iff it is non-negative, that is, $\nu a\ge 0$ for
every $a\in\frak A$.   \prooflet{\Prf\ ($\alpha$) If $\nu$ is
order-preserving, then of course $0=\nu 0\le\nu a$ for every
$a\in\frak A$.   ($\beta$) If $\nu$ is non-negative, and $a\Bsubseteq b$ in $\frak A$, then

\Centerline{$\nu a\le\nu a+\nu(b\Bsetminus a)=\nu b$.  \Qed}
}%end of prooflet

\spheader 361Cf If $U$ is a partially ordered linear space and $\nu$ is
non-negative, then (i) $\nu$ is order-continuous iff $\inf\nu[A]=0$
whenever $A\subseteq\frak A$ is a non-empty downwards-directed set with
infimum $0$ (ii) $\nu$ is sequentially order-continuous iff
$\inf_{n\in\Bbb N}\nu a_n=0$ whenever $\sequencen{a_n}$ is a
non-increasing sequence in $\frak A$ with infimum $0$.

\medskip

\prooflet{\Prf{\bf (i)}
If $\nu$ is order-continuous, then of course $\inf\nu[A]=\nu 0=0$
whenever $A\subseteq\frak A$ is a non-empty downwards-directed set with
infimum $0$.   If $\nu$ satisfies the condition, and
$A\subseteq\frak A$ is a non-empty upwards-directed set with supremum $c$, then
$\{c\Bsetminus a:a\in\frak A\}$ is downwards-directed with infimum $0$
(313Aa), so that

$$\eqalignno{\sup_{a\in A}\nu a
&=\sup_{a\in A}\nu c-\nu(c\Bsetminus a)
=\nu c-\inf_{a\in A}\nu(c\Bsetminus a)\cr
\noalign{\noindent (by 351Db)}
&=\nu c.\cr}$$

\noindent Similarly, if $A\subseteq\frak A$ is a non-empty
downwards-directed set with infimum $c$, then

\Centerline{$\inf_{a\in A}\nu a
=\inf_{a\in A}\nu c+\nu(a\Bsetminus c)
=\nu c+\inf_{a\in A}\nu(a\Bsetminus c)
=\nu c$.}

\noindent Putting these together, $\nu$ is order-continuous.

\medskip

\quad{\bf (ii)} If $\nu$ is sequentially order-continuous, then of course
$\inf_{n\in\Bbb N}\nu a_n=\nu 0=0$ whenever $\sequencen{a_n}$ is a
non-increasing
sequence in $\frak A$ with infimum $0$.   If $\nu$ satisfies the
condition, and $\sequencen{a_n}$ is a non-decreasing sequence in
$\frak A$ with supremum $c$, then $\sequencen{c\Bsetminus a_n}$ is
non-increasing and has infimum $0$, so that

\Centerline{$\sup_{n\in\Bbb N}\nu a_n
=\sup_{n\in\Bbb N}\nu c-\nu(c\Bsetminus a_n)
=\nu c-\inf_{n\in\Bbb N}\nu(c\Bsetminus a_n)
=\nu c$.}

\noindent Similarly, if $\sequencen{a_n}$ is a non-increasing sequence
in $\frak A$ with infimum $c$, then $\sequencen{a_n\Bsetminus c}$ is
non-increasing and has infimum $0$, so that

\Centerline{$\inf_{n\in\Bbb N}\nu a_n
=\inf_{n\in\Bbb N}\nu c+\nu(c\Bsetminus a_n)
=\nu c+\inf_{n\in\Bbb N}\nu(c\Bsetminus a_n)
=\nu c$.}

\noindent Thus $\nu$ is sequentially order-continuous.
\Qed}

\leader{361D}{Construction} Let $\frak A$ be a Boolean ring, and $Z$
its Stone space.   For $a\in\frak A$ write $\chi a$ for the
indicator function of the open-and-compact subset $\widehat{a}$ of
$Z$ corresponding to $a$.    Note that $\chi a=0$ iff $a=0$.
Let $S(\frak A)$ be the linear subspace of
$\Bbb R^Z$ generated by $\{\chi a:a\in\frak A\}$.   \cmmnt{Because
$\chi a$ is
a bounded function for every $a$,} $S(\frak A)$ is a subspace of the
$M$-space $\ell^{\infty}(Z)$ of all bounded real-valued functions on
$Z$\cmmnt{ (354Ha)},
and $\|\,\|_{\infty}$ is a norm on $S(\frak A)$.   \cmmnt{Because
$\chi a\times\chi b=\chi(a\Bcap b)$ for all $a$, $b\in \frak A$ (writing
$\times$ for pointwise multiplication of functions, as in 281B),}
$S(\frak A)$ is closed under $\times$.

\leader{361E}{}\cmmnt{ I give a portmanteau proposition running
through the elementary, mostly algebraic, properties of $S(\frak A)$.

\medskip

\noindent}{\bf Proposition} Let $\frak A$ be a Boolean ring, with
Stone space $Z$.   Write $S$ for $S(\frak A)$.

(a) If $a_0,\ldots,a_n\in\frak A$,  there are disjoint $b_0,\ldots,b_m$
such that each $a_i$ is expressible as the supremum of some of the
$b_j$.

(b) If $u\in S$, it is expressible in the
form $\sum_{j=0}^m\beta_j\chi b_j$ where $b_0,\ldots,b_m$ are disjoint
members of $\frak A$ and $\beta_j\in\Bbb R$ for each $j$.   If all the
$b_j$ are non-zero then $\|u\|_{\infty}=\sup_{j\le m}|\beta_j|$.

(c) If $u\in S$ is non-negative, it is expressible in the
form $\sum_{j=0}^m\beta_j\chi b_j$ where $b_0,\ldots,b_m$ are disjoint
members of $\frak A$ and $\beta_j\ge 0$ for each $j$, and simultaneously
in the form $\sum_{j=0}^m\gamma_j\chi c_j$ where
$c_0\Bsupseteq c_1\Bsupseteq\ldots\Bsupseteq c_m$ and $\gamma_j\ge 0$ for every $j$.

(d) If $u=\sum_{j=0}^m\beta_j\chi b_j$ where $b_0,\ldots,b_m$ are
disjoint members of $\frak A$ and $\beta_j\in\Bbb R$ for each $j$, then
$|u|=\sum_{j=0}^m|\beta_j|\chi b_j\in S$.

(e) $S$ is a Riesz subspace of $\Bbb R^Z$;  in its own right, it is an
Archimedean Riesz space.   If $\frak A$ is a Boolean algebra, then $S$
has an order unit $\chi 1$ and
$\|u\|_{\infty}=\min\{\alpha:\alpha\ge 0,\,|u|\le\alpha\chi 1\}$ for every $u\in S$.

(f) The map $\chi:\frak A\to S$ is injective, additive, non-negative, a
lattice homomorphism and order-continuous.

(g) Suppose that $u\ge 0$ in $S$ and $\delta\ge 0$ in $\Bbb R$.   Then

\Centerline{$\Bvalue{u>\delta}
=\max\{a:a\in\frak A,\,(\delta+\eta)\chi a\le u$ for some $\eta>0\}$}

\noindent is defined in $\frak A$, and

\Centerline{$\delta\chi\Bvalue{u>\delta}\le u
\le\delta\chi\Bvalue{u>0}\vee\|u\|_{\infty}\Bvalue{u>\delta}$.}

\noindent In particular, $u\le\|u\|_{\infty}\chi\Bvalue{u>0}$ and there
is an $\eta>0$ such that $\eta\chi\Bvalue{u>0}\le u$.   If $u$, $v\ge 0$
in $S$ then $u\wedge v=0$ iff $\Bvalue{u>0}\Bcap\Bvalue{v>0}=0$.

(h) Under $\times$, $S$ is an $f$-algebra\cmmnt{ (352W)} and a
commutative normed algebra\cmmnt{ (2A4J)}.

(i) For any $u\in S$, $u\ge 0$ iff $u=v\times v$ for some $v\in S$.

\proof{ Write $\widehat{a}$ for the
open-and-compact subset of $Z$ corresponding to $a\in\frak A$.

\medskip

{\bf (a)} Induce on $n$.   If $n=0$ take $m=0$, $b_0=a_0$.    For the
inductive step to $n\ge 1$, take disjoint $b_0,\ldots,b_m$ such that
$a_i$ is the supremum of some of the $b_j$ for each $i<n$;  now replace
$b_0,\ldots,b_m$ with
$b_0\Bcap a_n,\ldots,b_m\Bcap a_n,b_0\Bsetminus a_n,\ldots,
b_m\Bsetminus a_n,\discretionary{}{}{}a_n\Bsetminus\sup_{j\le m}b_j$
to obtain a suitable string for $a_0,\ldots,a_n$.

\medskip

{\bf (b)} If $u=0$ set $m=0$, $b_0=0$, $\beta_0=0$.   Otherwise, express
$u$ as $\sum_{i=0}^n\alpha_i\chi a_i$ where $a_0,\ldots,a_n\in\frak A$
and $\alpha_0,\ldots,\alpha_n$ are real numbers.   Let $b_0,\ldots,b_m$
be disjoint and such that every $a_i$ is expressible
as the supremum of some of the $b_j$.   Set $\gamma_{ij}=1$ if
$b_j\subseteq a_i$, $0$
otherwise, so that, because the $b_j$ are disjoint,
$\chi a_i=\sum_{j=0}^m\gamma_{ij}\chi b_j$ for each $i$.   Then

\Centerline{$u=\sum_{i=0}^n\alpha_i\chi a_i
=\sum_{i=0}^n\sum_{j=0}^m\alpha_i\gamma_{ij}\chi b_j
=\sum_{j=0}^m\beta_j\chi b_j$,}

\noindent setting $\beta_j=\sum_{i=0}^n\alpha_i\gamma_{ij}$ for each
$j\le m$.

The expression for $\|u\|_{\infty}$ is now obvious.

\medskip

{\bf (c)(i)} If $u\ge 0$ in (b), we must have $\beta_j=u(z)\ge 0$
whenever $z\in\widehat{b}_j$, so that $\beta_j\ge 0$ whenever
$b_j\ne 0$;
consequently $u=\sum_{j=0}^m|\beta_j|\chi b_j$ is in the required form.

\medskip

\quad{\bf (ii)} If we suppose that every $\beta_j$ is non-negative, and
rearrange the terms of the
sum so that $\beta_0\le\ldots\le\beta_m$, then we may set
$\gamma_0=\beta_0$, $\gamma_j=\beta_j-\beta_{j-1}$ for $1\le j\le m$,
$c_j=\sup_{i\ge j}b_i$ to get

\Centerline{$\sum_{j=0}^m\gamma_j\chi
c_j=\sum_{j=0}^m\sum_{i=j}^m\gamma_j\chi b_i
=\sum_{i=0}^m\sum_{j=0}^i\gamma_j\chi b_i
=\sum_{i=0}^m\beta_i\chi b_i=u$.}

\medskip

{\bf (d)} is trivial, because $\widehat{b}_0,\ldots,\widehat{b}_n$ are
disjoint.

\medskip

{\bf (e)} By (d), $|u|\in S$ for every $u\in S$, so $S$ is a Riesz
subspace of $\Bbb R^Z$, and in itself is an Archimedean Riesz space.
If $\frak A$ is a Boolean algebra, then $\chi 1$, the constant function
with value $1$, belongs to $S$, and is an order unit of $S$;  while

\Centerline{$\|u\|_{\infty}
=\min\{\alpha:\alpha\ge 0,\,|u(z)|\le\alpha\Forall z\in Z\}
=\min\{\alpha:\alpha\ge 0,\,|u|\le\alpha\chi 1\}$}

\noindent for every $u\in S$.

\medskip

{\bf (f)} $\chi$ is injective because $\widehat{a}\ne\widehat{b}$
whenever $a\ne b$.   $\chi$ is additive because
$\widehat{a}\cap\widehat{b}=\emptyset$
whenever $a\Bcap b=0$.   Of course $\chi$ is non-negative.
It is a lattice
homomorphism because $a\mapsto\widehat{a}:\frak A\to\Cal PZ$ and
$E\mapsto\chi E:\Cal PZ\to\Bbb R^Z$ are.   To see that $\nu$
is order-continuous, take a
non-empty downwards-directed $A\subseteq\frak A$ with infimum $0$.
\Quer\  Suppose, if possible, that $\{\chi a:a\in A\}$ does not have
infimum $0$ in $S$.   Then there is a $u>0$ in $S$ such that
$u\le\chi a$ for every $a\in\frak A$.  Now $u$ can be expressed as
$\sum_{j=0}^m\beta_j\chi b_j$ where $b_0,\ldots,b_m$ are disjoint.   There
must be some $z_0\in Z$ such that $u(z_0)>0$;  take $j$ such that
$z_0\in\widehat{b}_j$, so that $b_j\ne 0$ and $\beta_j=u(z_0)>0$.   But
now, for any $z\in\widehat{b}_j$, $a\in A$,

\Centerline{$(\chi a)(z)\ge u(z)=\beta_j>0$}

\noindent and $z\in\widehat{a}$.   As $z$ is arbitrary,
$\widehat{b}_j\subseteq\widehat{a}$
and $b_j\Bsubseteq a$;  as $a$ is arbitrary,  $b_j$ is a non-zero lower
bound for $A$ in $\frak A$.\ \BanG\   So $\inf\chi[A]=0$ in $S$.   As
$A$ is arbitrary, $\chi$ is order-continuous, by the criterion of
361C(f-i).

\medskip

{\bf (g)} Express $u$ as $\sum_{j=0}^m\beta_j\chi b_j$ where
$b_0,\ldots,b_m$ are disjoint and every $\beta_j\ge 0$.   Then given
$\delta\ge 0$, $\eta>0$ and $a\in\frak A$ we have
$(\delta+\eta)\chi a\le u$ iff
$a\Bsubseteq\sup\{b_j:j\le m,\,\beta_j\ge\delta+\eta\}$.   So
$\Bvalue{u>\delta}=\sup\{b_j:j\le m,\,\beta_j>\delta\}$.   Writing
$c=\Bvalue{u>\delta}$, $d=\Bvalue{u>0}=\sup\{b_j:\beta_j>0\}$, we have

$$\eqalign{u(z)&\le\|u\|_{\infty}\text{ if }z\in\widehat{c},\cr
&\le\delta\text{ if }z\in\widehat{d}\setminus\widehat{c},\cr
&=0\text{ if }z\in Z\setminus\widehat{d}.\cr}$$

\noindent So

\Centerline{$\delta\chi c\le u\le\|u\|_{\infty}\chi c\vee\delta\chi d$,}

\noindent as claimed.   Taking $\delta=0$ we get
$u\le\|u\|_{\infty}\chi d$.   Set

\Centerline{$\eta=\min(\{1\}\cup\{\beta_j:j\le m$, $\beta_j>0\})$;}

\noindent then $\eta>0$ and $\eta\chi d\le u$.

If $u$, $v\in S^+$ take $\eta$, $\eta'>0$ such that

\Centerline{$\eta\chi\Bvalue{u>0}\le u$,
\quad$\eta'\chi\Bvalue{v>0}\le v$.}

\noindent Then

\Centerline{$\min(\eta,\eta')\chi(\Bvalue{u>0}\Bcap\Bvalue{v>0})
\le u\wedge v
\le\max(\|u\|_{\infty},\|v\|_{\infty})
\chi(\Bvalue{u>0}\Bcap\Bvalue{v>0})$.}

\noindent So

\Centerline{$u\wedge v=0
\Longrightarrow\Bvalue{u>0}\Bcap\Bvalue{v>0}=0
\Longrightarrow u\wedge v=0$.}

\medskip

{\bf (h)} $S$ is a commutative $f$-algebra and normed algebra just
because it is a Riesz subspace of the $f$-algebra and commutative
normed algebra $\ell^{\infty}(Z)$ and is closed under multiplication.

\medskip

{\bf (i)} If $u=\sum_{j=0}^m\beta_j\chi b_j$ where $b_0,\ldots,b_m$ are
disjoint and $\beta_j\ge 0$ for every $j$, then $u=v\times v$ where
$v=\sum_{j=0}^m\sqrt{\beta_j}\chi b_j$.
}% end of proof of 361E

\leader{361F}{}\cmmnt{ I now turn to the universal mapping
theorems which really define the construction.

\medskip

\noindent}{\bf Theorem} Let $\frak A$ be a Boolean ring, and $U$ any
linear space.   Then there is a one-to-one correspondence
between additive functions $\nu:\frak A\to U$ and linear operators
$T:S(\frak A)\to U$, given by the formula $\nu=T\chi$.

\proof{{\bf (a)} The core of the proof is the following observation.
Let $\nu:\frak A\to U$ be additive.   If $a_0,\ldots,a_n\in\frak A$ and
$\alpha_0,\ldots,\alpha_n\in\Bbb R$ are such that
$\sum_{i=0}^n\alpha_i\chi a_i=0$ in $S=S(\frak A)$, then
$\sum_{i=0}^n\alpha_i\nu a_i=0$ in $U$.   \Prf\ By 361Ea, we can find
disjoint $b_0,\ldots,b_m$ such that each $a_i$ is the supremum of some
of the $b_j$;  set $\gamma_{ij}=1$ if $b_j\Bsubseteq a_i$, $0$
otherwise, so that $\chi a_i=\sum_{j=0}^m\gamma_{ij}\chi b_j$ and
$\nu a_i=\sum_{j=0}^m\gamma_{ij}\nu b_j$ for each $i$.   Set
$\beta_j=\sum_{i=0}^n\alpha_i\gamma_{ij}$ for each $j$, so that

\Centerline{$0=\sum_{i=0}^n\alpha_i\chi a_i
=\sum_{j=0}^m\beta_j\chi b_j$.}

\noindent Now $\beta_j\nu b_j=0$ in $U$ for each $j$, because either
$b_j=0$ and $\nu b_j=0$, or there is some $z\in\widehat{b}_j$ so that
$\beta_j$ must be $0$.   Accordingly

\Centerline{$0=\sum_{j=0}^m\beta_j\nu b_j
=\sum_{j=0}^m\sum_{i=0}^n\alpha_i\gamma_{ij}\nu b_j
=\sum_{i=0}^n\alpha_i\nu a_i$.  \Qed}

\medskip

{\bf (b)} It follows that if $u\in S$ is expressible simultaneously as
$\sum_{i=0}^n\alpha_i\chi a_i=\sum_{j=0}^m\beta_j\chi b_j$, then

\Centerline{$\sum_{i=0}^n\alpha_i\chi a_i+\sum_{j=0}^m(-\beta_j)\chi
b_j=0$ in $S$,}

\noindent so that

\Centerline{$\sum_{i=0}^n\alpha_i\nu a_i
+\sum_{j=0}^m(-\beta_j)\nu b_j=0$ in $U$,}

\noindent and

\Centerline{$\sum_{i=0}^n\alpha_i\nu a_i
=\sum_{j=0}^m\beta_j\nu b_j$.}

\noindent We can therefore define $T:S\to U$ by setting

\Centerline{$T(\sum_{i=0}^n\alpha_i\chi a_i)=\sum_{i=0}^n\alpha_i\nu
a_i$}

\noindent whenever $a_0,\ldots, a_n\in\frak A$ and
$\alpha_0,\ldots,\alpha_n\in\Bbb R$.

\medskip

{\bf (c)} It is now elementary to check that $T$ is linear, and that
$T\chi a=\nu a$ for every $a\in\frak A$.   Of course this last condition
uniquely defines $T$, because $\{\chi a:a\in\frak A\}$ spans the linear
space $S$.
}%end of proof of 361F

\leader{361G}{Theorem} Let $\frak A$ be a Boolean ring, and $U$ a
partially ordered linear space.   Let $\nu:\frak A\to U$ be an additive
function, and $T:S(\frak A)\to U$ the corresponding linear operator.

(a) $\nu$ is non-negative iff $T$ is positive.

(b) In this case,

\quad (i) if $T$ is order-continuous or sequentially order-continuous,
so is $\nu$;

\quad (ii) if $U$ is Archimedean and $\nu$ is order-continuous or
sequentially order-continuous, so is $T$.

(c) If $U$ is a Riesz space, then the following are equiveridical:

\quad (i) $T$ is a Riesz homomorphism;

\quad (ii) $\nu
a\wedge\nu b=0$ in $U$ whenever $a\Bcap b=0$ in $\frak A$;

\quad (iii) $\nu$ is a lattice homomorphism.

\proof{ Write $S$ for $S(\frak A)$.

\medskip

{\bf (a)} If $T$ is positive, then surely $\nu a=T\chi a\ge 0$ for every
$a\in\frak A$, so $\nu=T\chi$ is non-negative.
If $\nu$ is non-negative, and $u\ge 0$ in $S$, then $u$ is expressible
as $\sum_{j=0}^m\beta_j\chi b_j$ where $b_0,\ldots,b_m\in\frak A$ and
$\beta_j\ge 0$ for every $j$ (361Ec), so that

\Centerline{$Tu=\sum_{j=0}^m\beta_j\nu b_j\ge 0$.}

\noindent Thus $T$ is positive.

\medskip

{\bf (b)(i)} If $T$ is order-continuous (resp.\ sequentially
order-continuous) then $\nu=T\chi$ is the composition of two
order-continuous (resp.\ sequentially order-continuous) functions
(361Ef), so must be order-continuous (resp.\ sequentially
order-continuous).

\medskip

\quad{\bf (ii)} Assume now that $U$ is Archimedean.

\medskip

\qquad\grheada\ Suppose that $\nu$ is
order-continuous and that $A\subseteq S$ is non-empty,
downwards-directed and has infimum $0$.   Fix $u_0\in A$, set
$\alpha=\|u\|_{\infty}$ and $a_0=\Bvalue{u>0}$ (in the language of
361Eg).   If $\alpha=0$ then of course $\inf_{u\in A}Tu=Tu_0=0$.
Otherwise, take any $w\in U$ such that $w\not\le 0$.   Then there is some $\delta>0$ such
that $w\not\le\delta\nu a_0$, because $U$ is Archimedean.   Set
$A'=\{u:u\in A,\,u\le u_0\}$;
because $A$ is downwards-directed, $A'$ has the same lower bounds as
$A$, and $\inf A'=0$, while $A'$ is still downwards-directed.  For
$u\in A'$ set $c_u=\Bvalue{u>\delta}$, so that

\Centerline{$\delta\chi c_u\le u
\le\alpha\chi c_u+\delta\chi\Bvalue{u>0}
\le\alpha\chi c_u+\delta\chi a_0$}

\noindent (361Eg).   If $u$, $v\in A'$ and $u\le v$, then
$c_u\Bsubseteq c_v$, so $C=\{c_u:u\in A'\}$ is downwards-directed;  but if $c$ is any
lower bound for $C$ in $\frak A$, $\delta\chi c$ is a lower bound for
$A'$ in $S$, so is zero, and $c=0$ in $\frak A$.   Thus $\inf C=0$ in
$\frak A$, and $\inf_{u\in A'}\nu c_u=0$ in $U$.   But this means, in
particular, that $\bover1{\alpha}(w-\delta\nu a_0)$ is not a lower bound
for $\nu[C]$, and there is some $u\in A'$ such that
$\bover1{\alpha}(w-\delta\nu a_0)\not\le\nu c_u$, that is,
$w-\delta\nu a_0\not\le\alpha\nu c_u$, that is,
$w\not\le\delta\nu a_0+\alpha\nu c_u$.
As $u\le\alpha\chi c_u+\delta\chi a_0$,

\Centerline{$Tu\le T(\alpha\chi c_u+\delta\chi a_0)=\alpha\nu
c_u+\delta\nu a_0$,}

\noindent and $w\not\le Tu$.   Since $w$ is arbitrary, this means that
$0=\inf T[A]$;  as $A$ is arbitrary, $T$ is order-continuous.

\medskip

\qquad\grheadb\ The argument for sequential order-continuity is
essentially the same.   Suppose that $\nu$ is sequentially
order-continuous and that $\sequencen{u_n}$ is a non-increasing sequence
in $S$ with infimum $0$.   Again set $\alpha=\|u_0\|$,
$a_0=\Bvalue{u_0>0}$;  again we may suppose that $\alpha>0$;  again take
any $w\in U$ such that $w\not\le 0$.   As before, there is some
$\delta>0$ such that $w\not\le\delta\nu a_0$.   For $n\in\Bbb N$ set
$c_n=\Bvalue{u_n>\delta}$, so that

\Centerline{$\delta\chi c_n\le u_n\le\alpha\chi c_n+\delta\chi a_0$.}

\noindent   The sequence $\sequencen{c_n}$ is non-increasing because
$\sequencen{u_n}$ is,  and if $c\Bsubseteq c_n$ for every $n$, then
$\delta\chi c\le u_n$ for every $n$, so is zero, and $c=0$ in $\frak A$.
Thus $\inf_{n\in\Bbb N}c_n=0$ in $\frak A$, and
$\inf_{n\in\Bbb N}\nu c_n=0$ in $U$, because $\nu$ is sequentially order-continuous.
Replacing $A'$, $C$ in the argument above by $\{u_n:n\in\Bbb N\}$,
$\{c_n:n\in\Bbb N\}$ we find an $n$ such that $w\not\le Tu_n$.   Since
$w$ is arbitrary, this means that
$0=\inf_{n\in\Bbb N}Tu_n$;  as $\sequencen{u_n}$ is arbitrary, $T$ is
sequentially order-continuous.

\medskip

{\bf (c)(i)$\Rightarrow$(iii)} If $T:S(\frak A)\to U$ is a Riesz
homomorphism, and
$\nu=T\chi$, then surely $\nu$ is a lattice homomorphism because $T$ and
$\chi$ are.

\medskip

\quad{\bf (iii)$\Rightarrow$(ii)} is trivial.

\medskip

\quad{\bf (ii)$\Rightarrow$(i)} If $\nu a\wedge\nu b=0$ whenever $a\Bcap
b=0$, then for any $u\in S(\frak A)$ we
have an expression of $u$ as $\sum_{j=0}^m\beta_j\chi b_j$, where
$b_0,\ldots,b_m\in\frak A$ are disjoint.   Now

\Centerline{$|Tu|=|\sum_{j=0}^m\beta_j\nu b_j|
=\sum_{j=0}^m|\beta_j|\nu b_j=T(\sum_{j=0}^m|\beta_j|\chi b_j)
=T(|u|)$}

\noindent by 352Fb and 361Ed.   As $u$ is arbitrary, $T$ is a
Riesz homomorphism (352G).
}%end of proof of 361G

\leader{361H}{Theorem} Let $\frak A$ be a Boolean ring and $U$ a
Dedekind complete Riesz space.   Suppose that $\nu:\frak A\to U$ is an
additive function and $T:S=S(\frak A)\to U$ is the corresponding linear
operator.   Then $T\in\eurm L^{\sim}=\eurm L^{\sim}(S;U)$ iff $\{\nu
b:b\Bsubseteq a\}$ is order-bounded in $U$ for every $a\in\frak A$, and
in this case $|T|\in\eurm L^{\sim}$ corresponds to $|\nu|:\frak A\to U$,
defined by setting

$$\eqalign{|\nu|(a)
&=\sup\{\sum_{j=0}^n|\nu a_i|:a_0,\ldots,a_n\Bsubseteq a
\text{ are disjoint}\}\cr
&=\sup\{\nu b-\nu(a\Bsetminus b):b\Bsubseteq a\}\cr}$$

\noindent for every $a\in\frak A$.

\proof{{\bf (a)} Suppose that $T\in\eurm L^{\sim}$ and $a\in\frak A$.
Then for any $b\Bsubseteq a$, we have $\chi b\le\chi a$ so

\Centerline{$|\nu b|=|T\chi b|\le|T|(\chi a)$.}

\noindent Accordingly $\{\nu b:b\Bsubseteq a\}$ is order-bounded in $U$.

\medskip

{\bf (b)} Now suppose that $\{\nu b:b\Bsubseteq a\}$ is order-bounded in
$U$ for every $a\in\frak A$.   Then for any $a\in\frak A$ we can define
$w_a=\sup\{|\nu b|:b\Bsubseteq a\}$;  in this case,
$\nu b-\nu(a\Bsetminus b)\le 2w_a$ whenever $b\Bsubseteq a$, so
$\theta a=\sup_{b\Bsubseteq a}\nu b-\nu(a\Bsetminus b)$ is defined
in $U$.   Considering $b=a$, $b=0$ we see that $\theta a\ge|\nu a|$.
Next, $\theta:\frak A\to U$ is additive.   \Prf\ Take $a_1$,
$a_2\in\frak A$ such that $a_1\Bcap a_2=0$;  set $a_0=a_1\Bcup a_2$.
For each $j\le 2$ set

\Centerline{$A_j=\{\nu(a_j\Bcap b)-\nu(a_j\Bsetminus b):b\in\frak
A\}\subseteq U$.}

\noindent   Then $A_0\subseteq A_1+A_2$, because

\Centerline{$\nu(a_0\Bcap b)-\nu(a_0\Bsetminus b)
=\nu(a_1\Bcap b)-\nu(a_1\Bsetminus b)
  +\nu(a_2\Bcap b)-\nu(a_2\Bsetminus b)$}

\noindent for every $b\in\frak A$.   But also $A_1+A_2\subseteq A_0$,
because if $b_1$, $b_2\in\frak A$ then

\Centerline{$\nu(a_1\Bcap b_1)-\nu(a_1\Bsetminus b_1)
+\nu(a_2\Bcap b_2)-\nu(a_2\Bsetminus b_2)
=\nu(a_0\Bcap b)-\nu(a_0\Bsetminus b)
$}

\noindent where $b=(a_1\Bcap b_1)\Bcup(a_2\Bcap b_2)$.   So
$A_0=A_1+A_2$, and

\Centerline{$\theta a_0=\sup A_0=\sup A_1+\sup A_2
=\theta a_1+\theta a_2$}

\noindent (351Dc).\ \Qed

We therefore have a corresponding positive operator $T_1:S\to U$ such
that $\theta=T_1\chi$.   But we also see that
$\theta a=\sup\{\sum_{i=0}^n|\nu a_i|:a_0,\ldots,a_n\Bsubseteq a$ are disjoint$\}$ for every $a\in\frak A$.   \Prf\ If $a_0,\ldots,a_n$ are
disjoint and included in $a$, then

\Centerline{$\sum_{i=0}^n|\nu a_i|\le\sum_{i=0}^n\theta a_i
=\theta(\sup_{i\le n}a_i)\le\theta a$.}

\noindent On the other hand,

\Centerline{$\theta a\le\sup_{b\Bsubseteq a}|\nu b|+|\nu(a\Bsetminus b)|
\le\sup\{\sum_{i=0}^n|\nu a_i|:a_0,\ldots,a_n\Bsubseteq a$
  are disjoint$\}$.  \Qed}

It follows that $T\in\eurm L^{\sim}$.   \Prf\ Take any $u\ge 0$ in $S$.
Set $a=\Bvalue{u>0}$ (361Eg) and $\alpha=\|u\|_{\infty}$.   If $0<|v|\le
u$, then $v$ is expressible as $\sum_{i=0}^n\alpha_i\chi a_i$ where
$a_0,\ldots,a_n$ are disjoint and no $\alpha_i$ nor $a_i$ is zero.
Since $|v|\le\alpha\chi a$, we must have $|\alpha_i|\le\alpha$,
$a_i\Bsubseteq a$ for each $i$.   So

\Centerline{$|Tv|=|\sum_{i=0}^n\alpha_i\nu a_i|
\le\sum_{i=0}^n|\alpha_i||\nu a_i|
\le\alpha\sum_{i=0}^n|\nu a_i|
\le\alpha\theta a$.}

\noindent Thus $\{|Tv|:|v|\le u\}$ is bounded above by $\alpha\theta a$.
As $u$ is arbitrary, $T\in\eurm L^{\sim}$.
\Qed\

\medskip

{\bf (c)} Thus $T\in\eurm L^{\sim}$ iff $\nu$ is order-bounded on the
sets $\{b:b\Bsubseteq a\}$, and in this case the two formulae offered
for $|\nu|$ are consistent and make $|\nu|=\theta$.    Finally,
$\theta=|T|\chi$.   \Prf\ Take $a\in\frak A$.   If
$a_0,\ldots,a_n\Bsubseteq a$ are disjoint, then

\Centerline{$\sum_{i=0}^n|\nu a_i|=\sum_{i=0}^n|T\chi a_i|
\le\sum_{i=0}^n|T|(\chi a_i)\le|T|(\chi a)$;}

\noindent so $\theta a\le|T|(\chi a)$.   On the other hand, the argument
at the end of (b) above shows that $|T|(\chi a)\le\theta a$ for every
$a$.   Thus $|T|(\chi a)=\theta a$ for every
$a\in\frak A$, as required.\ \Qed
}%end of proof of 361H

\leader{361I}{Theorem} Let $\frak A$ be a Boolean ring, $U$ a normed
space and $\nu:\frak A\to U$
an additive function.   Give $S=S(\frak A)$ its norm $\|\,\|_{\infty}$,
and let $T:S\to U$ be the linear operator corresponding to $\nu$.   Then
$T$ is a bounded linear operator iff $\{\nu a:a\in\frak A\}$ is bounded,
and in this case $\|T\|=\sup_{a,b\in\frak A}\|\nu a-\nu b\|$.

\proof{{\bf (a)} If $T$ is bounded, then

\Centerline{$\|\nu a-\nu b\|=\|T(\chi a-\chi b)\|\le\|T\|\|\chi a-\chi
b\|_{\infty}\le
\|T\|$}

\noindent for every $a\in\frak A$, so $\nu$ is bounded and
$\sup_{a,b\in\frak A}\|\nu a-\nu b\|\le\|T\|$.

\medskip

{\bf (b)(i)} For the converse, we need a refinement of an idea in 361Ec.
If $u\in S$ and $u\ge 0$ and $\|u\|_{\infty}\le 1$, then $u$ is
expressible as $\sum_{i=0}^m\gamma_i\chi c_i$ where $\gamma_i\ge 0$ and
$\sum_{i=0}^m\gamma_i=1$.   \Prf\ If $u=0$, take $n=0$, $c_0=0$,
$\gamma_0=1$.   Otherwise, start from an expression
$u=\sum_{j=0}^n\gamma_j\chi c_j$ where
$c_0\Bsupseteq\ldots\Bsupseteq c_n$ and every $\gamma_j$ is
non-negative, as in 361Ec.   We may suppose
that $c_n\ne 0$, in which case

\Centerline{$\sum_{j=0}^n\gamma_j=u(z)\le 1$}

\noindent for every $z\in\widehat{c}_n\subseteq Z$, the Stone space of
$\frak A$.   Set $m=n+1$, $c_m=0$ and $\gamma_m=1-\sum_{j=0}^n\gamma_j$ to get the required form.\ \Qed

\medskip

\quad{\bf (ii)} The next fact we need is an elementary property of real
numbers:  if $\gamma_0,\ldots,\gamma_m$,
$\gamma'_0,\ldots,\gamma'_n\ge 0$
and $\sum_{i=0}^m\gamma_i=\sum_{j=0}^n\gamma'_j$, then there are
$\delta_{ij}\ge 0$ such that $\gamma_i=\sum_{j=0}^n\delta_{ij}$ for
every $i\le m$ and $\gamma'_j=\sum_{i=0}^m\delta_{ij}$ for every
$j\le n$.   \Prf\ This is just the case $U=\Bbb R$ of 352Fd.\ \Qed

\medskip

\quad{\bf (iii)} Now suppose that $\nu$ is bounded;  set
$\alpha_0=\sup_{a\in\frak A}\|\nu a\|<\infty$.   Then

\Centerline{$\alpha=\sup_{a,b\in\frak A}\|\nu a-\nu b\|\le 2\alpha_0$}

\noindent is also finite.
If $u\in S$ and $\|u\|_{\infty}\le 1$, then we can express $u$ as
$u^+-u^-$ where $u^+$, $u^-$ are
non-negative and also of norm at most $1$.   By (i), we can express
these as

\Centerline{$u^+=\sum_{i=0}^m\gamma_i\chi c_i$,
\quad $u^-=\sum_{j=0}^n\gamma_j'\chi c_j'$}

\noindent where all the $\gamma_i$, $\gamma_j'$ are non-negative and
$\sum_{i=0}^m\gamma_i=\sum_{j=0}^n\gamma'_j=1$.   Take
$\langle\delta_{ij}\rangle_{i\le m,j\le n}$ from (ii).   Set
$c_{ij}=c_i$, $c'_{ij}=c'_j$ for all $i$, $j$, so that

\Centerline{$u^+=\sum_{i=0}^m\sum_{j=0}^n\delta_{ij}\chi c_{ij}$,
\quad$u^-=\sum_{i=0}^m\sum_{j=0}^n\delta_{ij}\chi c'_{ij}$,}

\Centerline{$u=\sum_{i=0}^m\sum_{j=0}^n\delta_{ij}(\chi c_{ij}-\chi
c'_{ij})$,}

\Centerline{$Tu=\sum_{i=0}^m\sum_{j=0}^n\delta_{ij}(\nu c_{ij}-\nu
c'_{ij})$,}

\Centerline{$\|Tu\|
\le\sum_{i=0}^m\sum_{j=0}^n
\delta_{ij}\|\nu c_{ij}-\nu c'_{ij}\|
\le\sum_{i=0}^m\sum_{j=0}^n
\delta_{ij}\alpha=\alpha$.}

\noindent As $u$ is arbitrary, $T$ is a bounded linear operator and
$\|T\|\le\alpha$, as required.
}%end of proof of 361I

\leader{361J}{}\cmmnt{ The last few paragraphs describe the properties
of $S(\frak A)$ in terms of universal mapping theorems.   The next
theorem looks at the construction as a functor which converts Boolean
algebras into Riesz spaces and ring homomorphisms into Riesz
homomorphisms.

\medskip

\noindent}{\bf Theorem} Let $\frak A$ and $\frak B$ be Boolean rings
and $\pi:\frak A\to\frak B$ a ring homomorphism.

(a) We have a Riesz homomorphism $T_{\pi}:S(\frak A)\to S(\frak B)$
given by the formula

\Centerline{$T_{\pi}(\chi a)=\chi(\pi a)$ for every $a\in\frak A$.}

\noindent For any $u\in S(\frak A)$, $\|T_{\pi}u\|_{\infty}
=\min\{\|u'\|_{\infty}:u'\in S(\frak A),\,T_{\pi}u'=T_{\pi}u\}$;
\cmmnt{in particular,} $\|T_{\pi}u\|_{\infty}\le\|u\|_{\infty}$.
\cmmnt{Moreover,} $T_{\pi}(u\times u')=T_{\pi}u\times T_{\pi}u'$
for all $u$, $u'\in S(\frak A)$.

(b) $T_{\pi}$ is surjective iff $\pi$ is surjective, and in this case
$\|v\|_{\infty}=\min\{\|u\|_{\infty}:u\in S(\frak A),\,T_{\pi}u=v\}$ for
every
$v\in S(\frak B)$.

(c) The kernel of $T_{\pi}$ is just the set of those $u\in S(\frak A)$
such that $\pi\Bvalue{|u|>0}=0$, defining $\Bvalue{\ldots>\ldots}$ as in
361Eg.

(d) $T_{\pi}$ is injective iff $\pi$ is injective, and in this case
$\|T_{\pi}u\|_{\infty}=\|u\|_{\infty}$ for every $u\in S(\frak A)$.

(e) $T_{\pi}$ is order-continuous iff $\pi$ is order-continuous.

(f) $T_{\pi}$ is sequentially order-continuous iff $\pi$ is sequentially
order-continuous.

(g) If $\frak C$ is another Boolean ring and $\phi:\frak B\to\frak C$
is another ring homomorphism, then
$T_{\phi\pi}=T_{\phi}T_{\pi}:S(\frak A)\to S(\frak C)$.

\proof{{\bf (a)} The map $\chi\pi:\frak A\to S(\frak B)$ is additive
(361Cc), so corresponds to a linear operator $T=T_{\pi}:S(\frak A)\to S(\frak
B)$, by 361F.   $\chi$ and $\pi$ are both lattice homomorphisms, so
$\chi\pi$ also is, and $T$ is a Riesz homomorphism (361Gc).   If
$u=\sum_{i=0}^n\alpha_i\chi a_i$,
where $a_0,\ldots,a_n$ are disjoint, then look at $I=\{i:i\le n,\,\pi
a_i\ne 0\}$.   We have

\Centerline{$Tu=\sum_{i=0}^n\alpha_i\chi(\pi a_i)=\sum_{i\in
I}\alpha_i\chi(\pi a_i)$}

\noindent and $\pi a_0,\ldots,\pi a_n$ are disjoint, so that

\Centerline{$\|Tu\|_{\infty}=\sup_{i\in I}|\alpha_i|=\|u'\|_{\infty}
\le\sup_{a_i\ne 0}|\alpha_i|\le\|u\|_{\infty}$,}

\noindent where $u'=\sum_{i\in I}\alpha_i\chi a_i$, so that $Tu'=Tu$.
If $a$, $a'\in\frak A$, then

\Centerline{$T(\chi a\times\chi a')=T\chi(a\Bcap a')
=\chi\pi(a\Bcap a')=\chi\pi a\times\chi\pi a'
=T\chi a\times T\chi a'$,}

\noindent so $T$ is multiplicative.


\medskip

{\bf (b)} If $\pi$ is surjective, then $T[S(\frak A)]$ must be the
linear span of

\Centerline{$\{T(\chi a):a\in\frak A\}=\{\chi(\pi a):a\in\frak
A\}=\{\chi b:b\in\frak B\}$,}

\noindent so is the whole of $S(\frak B)$.   If $T$ is surjective,
and $b\in\frak B$, then there must be a $u\in\frak A$ such that $Tu=\chi
b$.   We can express $u$ as $\sum_{i=0}^n\alpha_i\chi a_i$ where
$a_0,\ldots,a_n$ are disjoint;  now

\Centerline{$\chi b=Tu=\sum_{i=0}^n\alpha_i\chi(\pi a_i)$,}

\noindent and $\pi a_0,\ldots,\pi a_n$ are disjoint in $\frak B$, so we
must have

\Centerline{$b=\sup_{i\in I}\pi a_i=\pi(\sup_{i\in I}a_i)\in\pi[\frak
A]$,}

\noindent where $I=\{i:\alpha_i=1\}$.   As $b$ is arbitrary, $\pi$ is
surjective.   Of course the formula for $\|v\|_{\infty}$ is a
consequence of the formula for $\|Tu\|_{\infty}$ in (a).

\medskip

{\bf (c)(i)} If $\pi\Bvalue{|u|>0}=0$ then
$|u|\le\alpha\chi a$, where $\alpha=\|u\|_{\infty}$,
and $a=\Bvalue{|u|>0}$, so

\Centerline{$|Tu|=T|u|\le\alpha T(\chi
a)=\alpha\chi(\pi a)=0$,}

\noindent and $Tu=0$.

\medskip

\quad{\bf (ii)} If $u\in S(\frak A)$ and
$Tu=0$, express $u$ as $\sum_{i=0}^n\alpha_i\chi a_i$ where
$a_0,\ldots,a_n$ are disjoint and every $\alpha_i$ is non-zero (361Eb).
In this case

\Centerline{$0=|Tu|=T|u|=\sum_{i=0}^n|\alpha_i|\chi(\pi
a_i)$,}

\noindent so $\pi a_i=0$ for every $i$, and

\Centerline{$\pi\Bvalue{|u|>0}=\pi(\sup_{i\le n}a_i)=\sup_{i\le n}\pi
a_i=0$.}

\medskip

{\bf (d)} If $T$ is injective and $a\in\frak A\setminus\{0\}$,
then $\chi(\pi a)=T(\chi a)\ne 0$, so $\pi a\ne 0$;  as $a$ is
arbitrary, $\pi$ is injective.   If $\pi$ is injective then
$\pi\Bvalue{|u|>0}\ne 0$ for every non-zero $u\in S(\frak A)$, so $T$ is
injective, by (c).
In this case the formula in (a) shows that $T$ is norm-preserving.

\medskip

{\bf (e)(i)}  If $T$ is order-continuous and $A\subseteq\frak A$
is a non-empty downwards-directed set with infimum $0$ in $\frak A$, let
$b$ be any lower bound for $\pi[A]$ in $\frak B$.   Then

\Centerline{$\chi b\le\chi(\pi a)=T(\chi a)$}

\noindent for any $a\in A$.   But $T\chi$ is order-continuous, by
361Ef, so $\inf_{a\in A}T(\chi a)=0$, and $b$ must be $0$.   As
$b$ is arbitrary, $\inf_{a\in A}\pi a=0$;  as  $A$ is arbitrary, $\pi$
is order-continuous.

\medskip

\quad{\bf (ii)}  If $\pi$ is order-continuous, so is
$\chi\pi:\frak A\to S(\frak B)$, using 361Ef again;  but now by
361G(b-ii) $T$ must be order-continuous.

\medskip

{\bf (f)(i)}  If $T$ is sequentially order-continuous, and
$\sequencen{a_n}$ is a non-increasing sequence in $\frak A$ with infimum
$0$, let $b$ be any lower bound for $\{\pi a_n:n\in\Bbb N\}$ in
$\frak B$.   Then

\Centerline{$\chi b\le\chi(\pi a_n)=T(\chi a_n)$}

\noindent for any $a\in A$.   But $T\chi$ is sequentially
order-continuous so $\inf_{n\in\Bbb N}T(\chi a_n)=0$,
and $b$ must be $0$.   As $b$ is arbitrary, $\inf_{n\in\Bbb N}\pi
a_n=0$;  as  $A$ is arbitrary, $\pi$ is sequentially order-continuous.

\medskip

\quad{\bf (ii)}  If $\pi$ is sequentially order-continuous, so is
$\chi\pi:\frak A\to S(\frak B)$;  but now $T$ must be sequentially
order-continuous.

\medskip

{\bf (g)} We need only check that

\Centerline{$T_{\phi\pi}(\chi a)=\chi(\phi(\pi a))
=T_{\phi}(\chi(\pi a))=T_{\phi}T_{\pi}(\chi a)$}

\noindent for every $a\in\frak A$.
}%end of proof of 361J

\leader{361K}{Proposition} Let $\frak A$ be a Boolean algebra.   For
$a\in\frak A$ write $V_a$ for the solid linear subspace of $S(\frak A)$
generated by $\chi a$.   Then $a\mapsto V_a$ is a Boolean isomorphism
between $\frak A$ and the algebra of projection bands in $S(\frak A)$.

\proof{ Write $S$ for $S(\frak A)$.

\medskip

{\bf (a)} The point is that, for any $a\in\frak A$,

\quad(i) $|u|\wedge|v|=0$ whenever $u\in V_a$, $v\in V_{1\Bsetminus
a}$,

\quad(ii) $V_a+V_{1\Bsetminus a}=S$.

\noindent\Prf\ (i) is just because
$\chi a\wedge\chi(1\Bsetminus a)=0$.   As for (ii), if $w\in S$ then

\Centerline{$w=(w\times\chi a)+(w\times\chi(1\Bsetminus a))
\in V_a+V_{1\Bsetminus a}$.  \Qed}

\medskip

{\bf (b)} Accordingly $V_a+V_a^{\perp}\supseteq V_a+V_{1\Bsetminus a}=S$
and $V_a$ is a projection band (352R).   Next, any projection band
$U\subseteq S$ is of the
form $V_a$.   \Prf\ We know that $\chi 1=u+v$ where $u\in U$,
$v\in U^{\perp}$.   Since $|u|\wedge|v|=0$, $u$ and $v$ must be the
indicator functions of complementary subsets of $Z$, the Stone
space of $\frak A$.   But $\{z:u(z)\ne 0\}=\{z:u(z)\ge 1\}$ must be of
the form $\widehat{a}$, where $a=\Bvalue{u>0}$, in which case
$u=\chi a$ and $v=\chi(1\Bsetminus a)$.  Accordingly $U\supseteq V_a$
and $U^{\perp}\supseteq V_{1\Bsetminus a}$.   But this means that $U$
must be $V_a$ precisely.\ \Qed

\medskip

{\bf (c)} Thus $a\mapsto V_a$ is a surjective function from $\frak A$
onto the algebra of projection bands in $S$.   Now

\Centerline{$a\Bsubseteq b\iff \chi a\in V_b\iff V_a\subseteq V_b$,}

\noindent so $a\mapsto V_a$ is order-preserving and bijective.   By
312M it is a Boolean isomorphism.
}%end of proof of 361K


\leader{361L}{Proposition} Let $X$ be a set, and $\Sigma$ a ring of
subsets of $X$, that is, a subring of the Boolean ring $\Cal PX$.   Then
$S(\Sigma)$ can be identified, as ordered linear space,
with the linear subspace of $\ell^{\infty}(X)$ generated by the
indicator
functions of members of $\Sigma$, which is a Riesz subspace of
$\ell^{\infty}(X)$.   The norm of $S(\Sigma)$ corresponds to the uniform
norm on $\ell^{\infty}(X)$, and its multiplication to pointwise
multiplication of functions.

\proof{ Let $Z$ be the Stone space of $\Sigma$, and for $E\in\Sigma$
write $\chi E$ for the indicator function of $E$ as a subset of
$X$, $\hat\chi E$ for the indicator function of the
open-and-compact subset of $Z$ corresponding to $E$.   Of course
$\chi:\Sigma\to\ell^{\infty}(X)$
is additive, so by 361F there is a linear operator $T:S\to\ell^{\infty}(X)$, writing $S$ for $S(\Sigma)$,
such that $T(\hat\chi E)=\chi E$ for every $E\in\Sigma$.

If $u\in S$, $Tu\ge 0$ iff $u\ge 0$.   \Prf\ Express $u$ as
$\sum_{j=0}^m\beta_j\hat\chi E_j$ where $E_0,\ldots,E_m$ are disjoint.
Then $Tu=\sum_{j=0}^m\beta_j\chi E_j$, so


\Centerline{$u\ge 0
\iff \beta_j\ge 0\text{ whenever }E_j\ne\emptyset
\iff Tu\ge 0$.  \Qed}

\noindent But this means ($\alpha$) that

\Centerline{$Tu=0\iff Tu\ge 0 \,\,\&\,\, T(-u)\ge 0
\iff u\ge 0\,\,\&\,-u\ge 0\iff u=0$,}

\noindent so that $T$ is injective and is a linear space isomorphism
between $S$ and its image $\eusm S$, which must be the linear space
spanned by $\{\chi E:E\in\Sigma\}$ ($\beta$) that $T$ is an
order-isomorphism between $S$ and $\eusm S$.

Because $\chi E\wedge\chi F=0$ whenever $E$, $F\in\Sigma$ and
$E\cap F=\emptyset$, $T$ is a Riesz homomorphism and $\eusm S$ is a Riesz subspace of $\ell^{\infty}(X)$ (361Gc).   Now

\Centerline{$\|u\|_{\infty}
=\inf\{\alpha:|u|\le\alpha\hat\chi X\}
=\inf\{\alpha:|Tu|\le\alpha\chi X\}
=\|Tu\|_{\infty}$}

\noindent for every $u\in S$.   Finally,

\Centerline{$T(\hat\chi E\times\hat\chi F)
=T(\hat\chi(E\cap F))
=\chi(E\cap F)
=T(\hat\chi E)\times T(\hat\chi F)$}

\noindent for all $E$, $F\in\Sigma$, so $\eusm S$ is closed under
pointwise multiplication and the multiplications of $S$, $\eusm S$ are
identified by $T$.
}%end of proof of 361L

\leader{361M}{Proposition} Let $X$ be a set, $\Sigma$ a ring of
subsets of $X$, and $\Cal I$ an ideal of $\Sigma$;  write $\frak A$ for
the quotient ring $\Sigma/\Cal I$.   Let $\eusm S$ be the
linear span of $\{\chi E:E\in\Sigma\}$ in $\Bbb R^X$, and write

\Centerline{$V=\{f:f\in \eusm S,\,\{x:f(x)\ne 0\}\in\Cal I\}$.}

\noindent\cmmnt{ Then }$V$ is a solid linear subspace of $\eusm S$.
\cmmnt{Now} $S(\frak A)$ becomes identified with the quotient Riesz space
$\eusm S/V$, if for every $E\in\Sigma$ we identify
$\chi(E^{\ssbullet})\in
S(\frak A)$ with $(\chi E)^{\ssbullet}\in \eusm S/V$.   If we give
$\eusm S$ its
uniform norm inherited from $\ell^{\infty}(X)$, $V$ is a closed linear
subspace of $\eusm S$, and the quotient norm on $\eusm S/V$ corresponds
to the norm
of $S(\frak A)$:

\Centerline{$\|f^{\ssbullet}\|=\min\{\alpha:\{x:|f(x)|>\alpha\}\in\Cal
I\}$.}

\noindent If we write $\times$ for pointwise multiplication on $\eusm
S$, then
$V$ is an ideal of the ring $(\eusm S,+,\times)$, and the multiplication
induced on $\eusm S/V$ corresponds to the multiplication of $S(\frak
A)$.

\proof{ Use 361J and 361L.   We can identify $\eusm S$ with $S(\Sigma)$.
Now the canonical ring homomorphism $E\mapsto E^{\ssbullet}$ corresponds
to a surjective Riesz homomorphism $T$ from $S(\Sigma)$ to
$S(\frak A)$ which takes $\chi E$ to $\chi(E^{\ssbullet})$.   For
$f\in\eusm S$, $\Bvalue{|f|>0}$ is just $\{x:f(x)\ne 0\}$, so   the
kernel of $T$ is just the set of those $f\in \eusm S$ such that
$\{x:f(x)\ne 0\}\in\Cal I$, which is $V$.   So

\Centerline{$S(\frak A)=T[\eusm S]\cong \eusm S/V$.}

As noted in 361Ja, $T(f\times g)=Tf\times Tg$ for all $f$, $g\in \eusm
S$, so
the multiplications of $\eusm S/V$ and $S(\frak A)$ match.   As for the
norms, the norm of $S(\frak A)$ corresponds to the norm of $\eusm S/V$
by the formulae in 361Ja or 361Jb.   To see that $V$ is closed in
$\eusm S$, we need note only that if
$f\in\overline{V}$ then

\Centerline{$\|Tf\|_{\infty}=\inf_{g\in V}\|f+g\|_{\infty}=\inf_{g\in
V}\|f-g\|_{\infty}=0$,}

\noindent so that $Tf=0$ and $f\in V$.
To check the formula for $\|f^{\ssbullet}\|$,
take any $f\in\eusm S$.   Express it as $\sum_{i=0}^n\alpha_i\chi E_i$
where $E_0,\ldots,E_n\in\Sigma$ are disjoint.   Set
$I=\{i:E_i\notin\Cal I\}$;  then

\Centerline{$\|Tf\|_{\infty}=\max_{i\in
I}|\alpha_i|=\min\{\alpha:\{x:|f(x)|>\alpha\}\in\Cal I\}$.}
}%end of proof of 361M

\exercises{\leader{361X}{Basic exercises (a)}
%\spheader 361Xa
Let $\frak A$ be a Boolean ring and $U$ a linear space.   Show that a
function $\nu:\frak A\to U$ is additive iff $\nu 0=0$ and
$\nu(a\Bcup  b)+\nu(a\Bcap b)=\nu a+\nu b$ for all $a$, $b\in\frak A$.
%361C

\sqheader 361Xb
Let $U$ be an {\bf algebra over} $\Bbb R$, that is, a
real linear space endowed with a multiplication $\times$ such that
$(U,+,\times)$ is a ring and $\alpha(w\times z)=(\alpha w)\times
z=w\times(\alpha z)$ for all $w$, $z\in U$ and all $\alpha\in\Bbb R$.
Let $\frak A$ be a Boolean ring, $\nu:\frak A\to U$ an additive
function and $T:S(\frak A)\to U$ the corresponding linear operator.
Show that $T$ is multiplicative iff $\nu(a\Bcap b)=\nu a\times\nu b$ for
all $a$, $b\in\frak A$.
%361G

\sqheader 361Xc Let $\frak A$ be a Boolean ring, and $U$ a Dedekind
complete Riesz space.   Suppose that $\nu:\frak A\to U$ is an additive
function such that the corresponding linear operator $T:S(\frak A)\to U$
belongs to $\eurm L^{\sim}=\eurm L^{\sim}(S(\frak A);U)$.   Show that
$T^+\in\eurm L^{\sim}$ corresponds to $\nu^+:\frak A\to U$, where
$\nu^+a=\sup_{b\Bsubseteq a}\nu b$ for every $a\in\frak A$.
%361H

\spheader 361Xd Let $\frak A$ and $\frak
B$ be Boolean algebras.   Show that there is a natural one-to-one
correspondence between Boolean homomorphisms $\pi:\frak A\to\frak B$ and
Riesz homomorphisms $T:S(\frak A)\to S(\frak B)$ such that $T(\chi
1_{\frak A})=\chi 1_{\frak B}$, given by setting $T(\chi a)=\chi(\pi a)$
for every $a\in\frak A$.
%361J

\spheader 361Xe Let $\frak A$, $\frak B$ be Boolean rings and
$T:S(\frak A)\to S(\frak B)$ a linear operator such that $T(u\times
v)=Tu\times Tv$ for all $u$, $v\in S(\frak A)$.    Show that there is a
ring homomorphism $\pi:\frak A\to\frak B$ such that $T(\chi a)=\chi(\pi
a)$ for every $a\in\frak A$.
%361J

\spheader 361Xf Let $\frak A$ and $\frak B$ be Boolean rings.   Show
that any isomorphism of the algebras $S(\frak A)$ and $S(\frak B)$
(using the word `algebra' in the sense of 361Xb) must be a Riesz space
isomorphism, and therefore corresponds to an isomorphism between $\frak
A$ and $\frak B$.
%361G, 361Xb, 361Xe 361J

\spheader 361Xg Let $\frak A$, $\frak B$ be Boolean algebras and
$T:S(\frak A)\to S(\frak B)$ a Riesz homomorphism.    Show that there
are a ring homomorphism $\pi:\frak A\to\frak B$ and a non-negative
$v\in S(\frak B)$ such that $T(\chi a)=v\times\chi(\pi a)$ for every
$a\in\frak A$.
%361J

\spheader 361Xh Let $\frak A$ be a Boolean algebra,
$\pi:\frak A\to\frak A$ a Boolean homomorphism and
$T_{\pi}:S(\frak A)\to S(\frak A)$ the associated Riesz homomorphism.
Let $\frak C$ be the fixed-point subalgebra of $\pi$ (312K).
Show that $S(\frak C)$ may be identified with the linear subspace of
$S(\frak A)$ generated by $\{\chi c:c\in\frak C\}$, and that this is
$\{u:u\in S(\frak A)$, $T_{\pi}u=u\}$.
%361J

\spheader 361Xi Let $\frak A$ be a Boolean ring.   Show that for any
$u\in S(\frak A)$ the solid linear subspace of $S(\frak A)$ generated by
$u$ is a projection band in $S(\frak A)$.   Show that the set of such
bands is an ideal in the algebra of all projection bands, and is
isomorphic to $\frak A$.
%361K

\sqheader 361Xj Let $X$ be a set and $\Sigma$ a $\sigma$-algebra of subsets
of $X$.    Show that the linear span $\eusm S$ in $\Bbb R^X$ of
$\{\chi E:E\in\Sigma\}$ is just the set of $\Sigma$-measurable functions $f:X\to\Bbb R$ which take only finitely many values.
%361L

\spheader 361Xk For any Boolean ring $\frak A$, we may define its
`complex $S$-space' $S_{\Bbb C}(\frak A)$ as the linear span in
$\Bbb C^Z$ of the indicator functions of open-and-compact subsets of
the Stone space $Z$ of $\frak A$.   State and prove results corresponding
to 361Eb, 361Ed, 361Eh, 361F, 361L and 361M.
%361M

\leader{361Y}{Further exercises (a)}
%\spheader 361Ya
Let $\frak A$ be a Boolean ring.   For $a\in\frak A$ let
$e_a\in\BbbR^{\frak A}$ be the function such that $e_a(a)=1$, $e_a(b)=0$
for $b\in\frak A\setminus\{a\}$;  let $V$ be the linear subspace of
$\BbbR^{\frak A}$ generated by $\{e_a:a\in\frak A\}$.
Let $W\subseteq V$ be the linear subspace
spanned by members of $V$ of the form $e_{a\Bcup b}-e_a-e_b$ where $a$,
$b\in\frak A$ are disjoint.   Define $\chi':\frak A\to V/W$ by taking
$\chi'a=e_a^{\ssbullet}$ to be the image in $V/W$ of $e_a\in V$.
Show, without using
the axiom of choice, that the pair $(V/W,\chi')$ has the universal
mapping property of $(S(\frak A),\chi)$ as described in 361F and that
$V/W$ has a Riesz space structure, a norm and a multiplicative structure
as described in 361D-361E.  Prove results corresponding to 361E-361M.
  %361E 361F 361G 361H 361I 361J 361K 361L 361M
%361D

\spheader 361Yb Let $\familyiI{\frak A_i}$ be a non-empty family
of Boolean algebras, with free product $\frak A$;  write
$\varepsilon_i:\frak A_i\to\frak A$ for the canonical maps, and

\Centerline{$C=\{\inf_{j\in J}\varepsilon_j(a_j):J\subseteq I$ is
finite, $a_j\in\frak A_j$ for every $j\in J\}$.}

\noindent Suppose that $U$ is a linear space and $\theta:C\to U$ is such
that

\Centerline{$\theta c
=\theta(c\Bcap\varepsilon_i(a))
  +\theta(c\Bcap\varepsilon_i(1\Bsetminus a))$}

\noindent whenever $c\in C$, $i\in I$ and $a\in\frak A_i$.
Show that there is a unique additive function $\nu:\frak A\to U$ extending
$\theta$.   \Hint{326E.}
%361F

\spheader 361Yc Let $\frak A$ be a Boolean ring and $U$ a Dedekind
complete Riesz space.   Let
$A\subseteq\eurm L^{\sim}=\eurm L^{\sim}(S(\frak A);U)$ be a non-empty set.   Suppose that
$\tilde T=\sup A$ is defined in $\eurm L^{\sim}$, and that $\tilde\nu=\tilde T\chi$.   Show that for any $a\in\frak A$,

\Centerline{$\tilde\nu a
=\sup\{\sum_{i=0}^nT_i(\chi a_i):T_0,\ldots,T_n\in A,
\,a_0,\ldots,a_n\Bsubseteq a$ are disjoint, $\sup_{i\le n}a_i=a\}$.}
%361H

\spheader 361Yd Let $\frak A$ be a Boolean algebra.   Show that the
algebra of all bands of $S(\frak A)$ can be identified with the Dedekind
completion of $\frak A$ (314U).
%361K

\spheader 361Ye Let $\frak A$ be a Boolean ring, and $U$ a complex
normed space.   Let $\nu:\frak A\to U$ be an additive function and
$T:S_{\Bbb C}(\frak A)\to U$ the corresponding linear operator (cf.\ 361Xk).   Show
that (giving $S_{\Bbb C}(\frak A)$ its usual norm $\|\,\|_{\infty}$)

\Centerline{$\|T\|
=\sup\{\|\sum_{j=0}^n\zeta_j\nu a_j\|:
 a_0,\ldots,a_n\in\frak A$ are disjoint, $|\zeta_j|=1$ for every $j\}$}

\noindent if either is finite.
%361M, 361Xk

\spheader 361Yf Let $U$ be a Riesz space.   Show that it is isomorphic
to $S(\frak A)$, for some Boolean algebra $\frak A$, iff it has an order
unit and every solid linear subspace of $U$ is a projection band.
%361M
}%end of exercises

\cmmnt{\Notesheader{361} The space
$S(\frak A)$ corresponds of course to the idea of `simple function'
which belongs to the very beginnings of the theory of integration
(122A).   All that 361D is trying to do is to set up a logically sound
description of this obvious concept which can be derived from the
Boolean ring $\frak A$ itself.   To my eye, there is a defect in the
construction there.   It relies on the axiom of choice, since it uses
the Stone space;  but none of the elementary properties of $S(\frak A)$
have anything to do with the axiom of choice.   In 361Ya I offer an
alternative construction which is in a formal sense more `elementary'.
If you work through the suggestion there you will find, however, that
the technical details become significantly more complicated, and would
be intolerable were it not for the intuition provided by the Stone space
construction.   Of course this intuition is chiefly valuable in the
finitistic arguments used in 361E, 361F and 361I;  and for these
arguments we really need the Stone representation only for finite
Boolean rings, which does not depend on the axiom of choice.

It is quite true that in most of this volume (and in most of this
chapter) I use the axiom of choice without scruple and without comment.
I mention it here only because I find myself using arguments dependent
on choice to prove theorems of a type to which the axiom cannot be
relevant.

The linear space structure of $S(\frak A)$, together with the map
$\chi$, are uniquely determined by the first universal mapping theorem
here, 361F.   This result says nothing about the order structure, which
needs the further refinement in 361Ga.   What is striking is that the
partial order defined by 361Ga is actually a lattice ordering, so that
we can have a universal mapping theorem for functions to Riesz spaces,
as in 361Gc and 361Ja.   Moreover, the same ordering provides a happy
abundance of results concerning order-continuous functions (361Gb,
361Je-361Jf).   When the codomain is a Dedekind complete Riesz space, so
that we have a Riesz space $\eurm L^{\sim}(S;U)$, and a corresponding
modulus function $T\mapsto|T|$ for linear operators, there are
reasonably natural formulae for $|T|\chi$ in terms of $T\chi$ (361H);
see also 361Xc and 361Yc.
The multiplicative structure of $S(\frak A)$ is defined
by 361Xb, and its norm by 361I.

The Boolean ring $\frak A$ cannot be recovered from the linear space
structure of $S(\frak A)$ alone (since this tells us only the
cardinality of $\frak A$), but if we add either the ordering or the
multiplication of $S(\frak A)$ then $\frak A$ is easy to identify (361K,
361Xf).

The most important Boolean algebras of measure theory arise either as
algebras of sets or as their quotients, so it is a welcome fact that in
such cases the spaces $S(\frak A)$ have straightforward representations
in terms of the construction of $\frak A$ (361L-361M).

In Chapter 24 I offered a paragraph in each section to sketch a version
of the theory based on the field of complex numbers rather than the
field of real numbers.   This was because so many of the most important
applications of these ideas involve complex numbers, even though (in my
view) the ideas themselves are most clearly and characteristically
expressed in terms of real numbers.   In the present chapter we are one
step farther away from these applications, and I therefore relegate
complex numbers to the exercises, as in 361Xk and 361Ye.

}%end of comment

\discrpage

